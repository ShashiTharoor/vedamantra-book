\sect{पञ्चमः प्रश्नः}\setcounter{anuvakam}{0}
\dnsub{तैत्तिरीयसंहितायां प्रथमकाण्डे पञ्चमः प्रश्नः}
%1.5.1.0
%1.5.1.1
दे॒वा॒सु॒राः संय॑त्ता आस॒न् ते दे॒वा वि॑ज॒यमु॑प॒यन्तो॒\-ऽग्नौ वा॒मं वसु॒ सं न्य॑दधते॒दमु॑ नो भविष्यति॒ यदि॑ नो जे॒ष्यन्तीति॒ तद॒ग्निर्न्य॑कामयत॒ तेनापा᳚क्राम॒त् तद्दे॒वा वि॒जित्या॑व॒रुरु॑थ्समाना॒ अन्वा॑य॒न् तद॑स्य॒ सह॒सा\-ऽदि॑थ्सन्त॒ सो॑\-ऽरोदी॒द्यदरो॑दी॒त् तद्रु॒द्रस्य॑ रुद्र॒त्वं यदश्र्वशी॑यत॒ तद्~(१)

%1.5.1.2
र॑ज॒तꣳ हिर॑ण्यमभव॒त् तस्मा᳚द्रज॒तꣳ हिर॑ण्यमदक्षि॒ण्य\-म॑श्रु॒जꣳ हि यो ब॒र्॒\mbox{}हिषि॒ ददा॑ति पु॒रा\-ऽस्य॑ संवथ्स॒राद्गृ॒हे रु॑दन्ति॒ तस्मा᳚द्ब॒र्॒\mbox{}हिषि॒ न देय॒ꣳ॒ सो᳚\-ऽग्निर॑ब्रवीद्भा॒ग्य॑सा॒न्यथ॑ व इ॒दमिति॑ पुनरा॒धेयं॑ ते॒ केव॑ल॒मित्य॑ब्रुवन्नृ॒ध्नव॒त् खलु॒ स इत्य॑ब्रवी॒द्यो म॑द्देव॒त्य॑म॒ग्निमा॒दधा॑ता॒ इति॒ तं पू॒षा\-ऽ\-ऽध॑त्त॒ तेन॑~(२)

%1.5.1.3
पू॒षा\-ऽ\-ऽर्ध्नो॒त् तस्मा᳚त् पौ॒ष्णाः प॒शव॑ उच्यन्ते॒ तं त्वष्टा\-ऽ\-ऽध॑त्त॒ तेन॒ त्वष्टा᳚\-ऽ\-ऽर्ध्नो॒त् तस्मा᳚त् त्वा॒ष्ट्राः प॒शव॑ उच्यन्ते॒ तं मनु॒रा\-ऽध॑त्त॒ तेन॒ मनु॑रार्ध्नो॒त् तस्मा᳚न्मान॒व्यः॑ प्र॒जा उ॑च्यन्ते॒ तं धा॒ता\-ऽ\-ऽध॑त्त॒ तेन॑ धा॒ता\-ऽ\-ऽर्ध्नो᳚थ्संवथ्स॒रो वै धा॒ता तस्मा᳚थ्संवथ्स॒रं प्र॒जाः प॒शवो\-ऽनु॒ प्र जा॑यन्ते॒ य ए॒वं पु॑नरा॒धेय॒स्यर्द्धिं॒ वेद॒-~(३)

%1.5.1.4
र्ध्नोत्ये॒व यो᳚\-ऽस्यै॒वं ब॒न्धुतां॒ वेद॒ बन्धु॑मान् भवति भाग॒धेयं॒ वा अ॒ग्निराहि॑त इ॒च्छमा॑नः प्र॒जां प॒शून् यज॑मान॒स्योप॑ दोद्रावो॒द्वास्य॒ पुन॒रा द॑धीत भाग॒धेये॑नै॒वैन॒ꣳ॒ सम॑र्धय॒त्यथो॒ शान्ति॑रे॒वास्यै॒षा पुन॑र्वस्वो॒रा द॑धीतै॒तद्वै पु॑नरा॒धेय॑स्य॒ नक्ष॑त्रं॒ यत्पुन॑र्वसू॒ स्वाया॑मे॒वैनं॑ दे॒वता॑यामा॒धाय॑ ब्रह्मवर्च॒सी भ॑वति द॒र्भैरा द॑धा॒त्यया॑तयामत्वाय द॒र्भैरा द॑धात्य॒द्भ्य ए॒वैन॒मोष॑धीभ्यो\-ऽव॒रुध्या\-ऽ\-ऽध॑त्ते॒ पञ्च॑कपालः पुरो॒डाशो॑ भवति॒ पञ्च॒ वा ऋ॒तव॑ ऋ॒तुभ्य॑ ए॒वैन॑मव॒रुध्या\-ऽ\-ऽध॑त्ते॥~(४)

%1.5.2.0
{\anuvakamend[{अशी॑यत॒ तत् तेन॒ वेद॑ द॒र्भैः पञ्च॑विꣳशतिश्च}]}%~(१)

%1.5.2.1
परा॒ वा ए॒ष य॒ज्ञं प॒शून् व॑पति॒ यो᳚\-ऽग्निमु॑द्वा॒सय॑ते॒ पञ्च॑कपालः पुरो॒डाशो॑ भवति॒ पाङ्क्तो॑ य॒ज्ञः पाङ्क्ताः᳚ प॒शवो॑ य॒ज्ञमे॒व प॒शूनव॑ रुन्धे वीर॒हा वा ए॒ष दे॒वानां॒ यो᳚\-ऽग्निमु॑द्वा॒सय॑ते॒ न वा ए॒तस्य॑ ब्राह्म॒णा ऋ॑ता॒यवः॑ पु॒रा\-ऽन्न॑मक्षन् प॒ङ्क्त्यो॑ याज्यानुवा॒क्या॑ भवन्ति॒ पाङ्क्तो॑ य॒ज्ञः पाङ्क्तः॒ पुरु॑षो दे॒वाने॒व वी॒रं नि॑रव॒दाया॒ग्निं पुन॒रा~(५)

%1.5.2.2
ध॑त्ते श॒ताक्ष॑रा भवन्ति श॒तायुः॒ पुरु॑षः श॒तेन्द्रि॑य॒ आयु॑ष्ये॒वेन्द्रि॒ये प्रति॑ तिष्ठति॒ यद्वा अ॒ग्निराहि॑तो॒ नर्ध्यते॒ ज्यायो॑ भाग॒धेयं॑ निका॒मय॑मानो॒ यदा᳚ग्ने॒यꣳ सर्वं॒ भव॑ति॒ सैवास्यर्धिः॒ सं वा ए॒तस्य॑ गृ॒हे वाक् सृ॑ज्यते॒ यो᳚\-ऽग्निमु॑द्वा॒सय॑ते॒ स वाच॒ꣳ॒ सꣳसृ॑ष्टां॒ यज॑मान ईश्व॒रो\-ऽनु॒ परा॑भवितो॒र्विभ॑क्तयो भवन्ति वा॒चो विधृ॑त्यै॒ यज॑मान॒स्याप॑राभावाय॒~(६)

%1.5.2.3
विभ॑क्तिं करोति॒ ब्रह्मै॒व तद॑करुपा॒ꣳ॒शु य॑जति॒ यथा॑ वा॒मं वसु॑ विविदा॒नो गूह॑ति ता॒दृगे॒व तद॒ग्निं प्रति॑ स्विष्ट॒कृतं॒ निरा॑ह॒ यथा॑ वा॒मं वसु॑ विविदा॒नः प्र॑का॒शं जिग॑मिषति ता॒दृगे॒व तद्विभ॑क्तिमु॒क्त्वा प्र॑या॒जेन॒ वष॑ट्करोत्या॒यत॑नादे॒व नैति॒ यज॑मानो॒ वै पु॑रो॒डाशः॑ प॒शव॑ ए॒ते आहु॑ती॒ यद॒भितः॑ पुरो॒डाश॑मे॒ते आहु॑ती~(७)

%1.5.2.4
जु॒होति॒ यज॑मानमे॒वोभ॒यतः॑ प॒शुभिः॒ परि॑ गृह्णाति कृ॒तय॑जुः॒ सम्भृ॑तसम्भार॒ इत्या॑हु॒र्न स॒म्भृत्याः᳚ सम्भा॒रा न यजुः॑ कर्त॒व्य॑मित्यथो॒ खलु॑ स॒म्भृत्या॑ ए॒व स॑म्भा॒राः क॑र्त॒व्यं॑ यजु॑र्य॒ज्ञस्य॒ समृ॑द्ध्यै पुनर्निष्कृ॒तो रथो॒ दक्षि॑णा पुनरुथ्स्यू॒तं वासः॑ पुनरुथ्सृ॒ष्टो॑\-ऽन॒ड्वान् पु॑नरा॒धेय॑स्य॒ समृ॑द्ध्यै स॒प्त ते॑ अग्ने स॒मिधः॑ स॒प्त जि॒ह्वा इत्य॑ग्निहो॒त्रं जु॑होति॒ यत्र॑यत्रै॒वास्य॒ न्य॑क्तं॒ तत॑~(८)

%1.5.2.5
ए॒वैन॒मव॑ रुन्धे वीर॒हा वा ए॒ष दे॒वानां॒ यो᳚\-ऽग्निमु॑द्वा॒सय॑ते॒ तस्य॒ वरु॑ण ए॒वर्ण॒यादा᳚ग्निवारु॒णमेका॑\-दश\-कपाल॒मनु॒ निर्व॑पे॒द्यं चै॒व हन्ति॒ यश्चा᳚स्यर्ण॒यात्तौ भा॑ग॒धेये॑न प्रीणाति॒ ना\-ऽ\-ऽर्ति॒मार्च्छ॑ति॒ यज॑मानः॥~(९)

%1.5.3.0
{\anuvakamend[{आ\-ऽप॑राभावाय पुरो॒डाश॑मे॒ते आहु॑ती॒ ततः॒ षट्त्रिꣳ॑शच्च}]}%~(२)

%1.5.3.1
भूमि॑र्भू॒म्ना द्यौर्व॑रि॒णा\-ऽन्तरि॑क्षं महि॒त्वा। उ॒पस्थे॑ ते देव्यदिते॒\-ऽग्निम॑न्ना॒दम॒न्नाद्या॒या\-ऽ\-ऽद॑धे॥ आ\-ऽयं गौः पृश्ञि॑रक्रमी॒दस॑नन्मा॒तरं॒ पुनः॑। पि॒तरं॑ च प्र॒यन्थ्सुवः॑॥ त्रि॒ꣳ॒शद्धाम॒ वि रा॑जति॒ वाक्प॑त॒ङ्गाय॑ शिश्रिये। प्रत्य॑स्य वह॒ द्युभिः॑॥ अ॒स्य प्रा॒णाद॑पान॒त्य॑न्तश्च॑रति रोच॒ना। व्य॑ख्यन्महि॒षः सुवः॑॥ यत् त्वा᳚~(१०)

%1.5.3.2
क्रु॒द्धः प॑रो॒वप॑ म॒न्युना॒ यदव॑र्त्या। सु॒कल्प॑मग्ने॒ तत् तव॒ पुन॒स्त्वोद्दी॑पयामसि॥ यत् ते॑ म॒न्युप॑रोप्तस्य पृथि॒वीमनु॑ दध्व॒से। आ॒दि॒त्या विश्वे॒ तद्दे॒वा वस॑वश्च स॒माभ॑रन्न्॥ मनो॒ ज्योति॑र्जुषता॒माज्यं॒ विच्छि॑न्नं य॒ज्ञꣳ समि॒मं द॑धातु। बृह॒स्पति॑स्तनुतामि॒मं नो॒ विश्वे॑ दे॒वा इ॒ह मा॑दयन्ताम्॥ स॒प्त ते॑ अग्ने स॒मिधः॑ स॒प्त जि॒ह्वाः स॒प्त~(११)

%1.5.3.3
ऋष॑यः स॒प्त धाम॑ प्रि॒याणि॑। स॒प्त होत्राः᳚ सप्त॒धा त्वा॑ यजन्ति स॒प्त योनी॒रा पृ॑णस्वा घृ॒तेन॑॥ पुन॑रू॒र्जा नि व॑र्तस्व॒ पुन॑रग्न इ॒षा\-ऽ\-ऽयु॑षा। पुन॑र्नः पाहि वि॒श्वतः॑॥ स॒ह र॒य्या नि व॑र्त॒स्वाग्ने॒ पिन्व॑स्व॒ धार॑या। वि॒श्वफ्स्नि॑या वि॒श्वत॒स्परि॑॥ लेकः॒ सले॑कः सु॒लेक॒स्ते न॑ आदि॒त्या आज्यं॑ जुषा॒णा वि॑यन्तु॒ केतः॒ सके॑तः सु॒केत॒स्ते न॑ आदि॒त्या आज्यं॑ जुषा॒णा वि॑यन्तु॒ विव॑स्वा॒ꣳ॒ अदि॑ति॒र्देव॑जूति॒स्ते न॑ आदि॒त्या आज्यं॑ जुषा॒णा वि॑यन्तु॥~(१२)

%1.5.4.0
{\anuvakamend[{त्वा॒ जि॒ह्वाः स॒प्त सु॒केत॒स्ते न॒स्त्रयो॑दश च}]}%~(३)

%1.5.4.1
भूमि॑र्भू॒म्ना द्यौर्व॑रि॒णेत्या॑हा॒\-ऽ\-ऽशिषै॒वैन॒मा ध॑त्ते स॒र्पा वै जीर्य॑न्तो\-ऽमन्यन्त॒ स ए॒तं क॑स॒र्णीरः॑ काद्रवे॒यो मन्त्र॑मपश्य॒त् ततो॒ वै ते जी॒र्णास्त॒नूरपा᳚घ्नत सर्परा॒ज्ञिया॑ ऋ॒ग्भिर्गार्\mbox{}ह॑पत्य॒मा द॑धाति पुनर्न॒वमे॒वैन॑म॒जरं॑ कृ॒त्वा\-ऽ\-ऽध॒त्ते\-ऽथो॑ पू॒तमे॒व पृ॑थि॒वीम॒न्नाद्यं॒ नोपा॑नम॒थ्सैतं~(१३)

%1.5.4.2
मन्त्र॑मपश्य॒त् ततो॒ वै ताम॒न्नाद्य॒मुपा॑\-नम॒द्यथ्स॑र्परा॒ज्ञिया॑ ऋ॒ग्भिर्गार्\mbox{}ह॑पत्यमा॒दधा᳚त्य॒न्नाद्य॒स्या\-व॑रुद्ध्या॒ अथो॑ अ॒स्यामे॒वैनं॒ प्रति॑ष्ठित॒मा ध॑त्ते॒ यत्त्वा᳚ क्रु॒द्धः प॑रो॒वपेत्या॒हाप॑ह्नुत ए॒वास्मै॒ तत् पुन॒स्त्वोद्दी॑पयाम॒सीत्या॑ह॒ समि॑न्ध ए॒वैनं॒ यत्ते॑ म॒न्युप॑रोप्त॒स्येत्या॑ह दे॒वता॑भिरे॒-~(१४)

%1.5.4.3
वैन॒ꣳ॒ सं भ॑रति॒ वि वा ए॒तस्य॑ य॒ज्ञश्छि॑द्यते॒ यो᳚\-ऽग्निमु॑द्वा॒सय॑ते॒ बृह॒स्पति॑वत्य॒र्चोप॑ तिष्ठते॒ ब्रह्म॒ वै दे॒वानां॒ बृह॒स्पति॒र्ब्रह्म॑णै॒व य॒ज्ञꣳ सं द॑धाति॒ विच्छि॑न्नं य॒ज्ञꣳ समि॒मं द॑धा॒त्वित्या॑ह॒ सन्त॑त्यै॒ विश्वे॑ दे॒वा इ॒ह मा॑दयन्ता॒मित्या॑ह स॒न्तत्यै॒व य॒ज्ञं दे॒वेभ्यो\-ऽनु॑ दिशति स॒प्त ते॑ अग्ने स॒मिधः॑ स॒प्त जि॒ह्वा~-~(१५)

%1.5.4.4
इत्या॑ह स॒प्तस॑प्त॒ वै स॑प्त॒धा\-ऽग्नेः प्रि॒यास्त॒नुव॒स्ता ए॒वाव॑ रुन्धे॒ पुन॑रू॒र्जा स॒ह र॒य्येत्य॒भितः॑ पुरो॒डाश॒माहु॑ती जुहोति॒ यज॑मानमे॒वोर्जा च॑ र॒य्या चो॑भ॒यतः॒ परि॑ गृह्णात्यादि॒त्या वा अ॒स्माल्लो॒काद॒मुं लो॒कमा॑य॒न्ते॑\-ऽमुष्मिँ॑ल्लो॒के व्य॑तृष्य॒न्त इ॒मं लो॒कं पुन॑रभ्य॒\-वेत्या॒ग्नि\-मा॒धायै॒तान् होमा॑नजुहवु॒स्त आ᳚र्ध्नुव॒न् ते सु॑व॒र्गं लो॒कमा॑य॒न्॒ यः प॑रा॒चीनं॑ पुनरा॒धेया॑द॒ग्निमा॒दधी॑त॒ स ए॒तान् होमा᳚ञ्जुहुया॒द्यामे॒वा\-ऽ\-ऽदि॒त्या ऋद्धि॒मार्ध्नु॑व॒न् तामे॒वर्ध्नो॑ति॥~(१६)

%1.5.5.0
{\anuvakamend[{ए॒तमे॒व जि॒ह्वा ए॒तान् पञ्च॑विꣳशतिश्च}]}%~(४)

%1.5.5.1
उ॒प॒प्र॒यन्तो॑ अध्व॒रं मन्त्रं॑ वोचेमा॒ग्नये᳚। आ॒रे अ॒स्मे च॑ शृण्व॒ते॥ अ॒स्य प्र॒त्नामनु॒ द्युतꣳ॑ शु॒क्रं दु॑दुह्रे॒ अह्र॑यः। पयः॑ सहस्र॒सामृषिम्᳚॥ अ॒ग्निर्मू॒र्धा दि॒वः क॒कुत् पतिः॑ पृथि॒व्या अ॒यम्। अ॒पाꣳ रेताꣳ॑सि जिन्वति॥ अ॒यमि॒ह प्र॑थ॒मो धा॑यि धा॒तृभि॒र्\mbox{}होता॒ यजि॑ष्ठो अध्व॒रेष्वीड्यः॑। यमप्न॑वानो॒ भृग॑वो विरुरु॒चुर्वने॑षु चि॒त्रं वि॒भुवं॑ वि॒शेवि॑शे॥ उ॒भा वा॑मिन्द्राग्नी आहु॒वध्या॑~(१७)

%1.5.5.2
उ॒भा राध॑सः स॒ह मा॑द॒यध्यै᳚। उ॒भा दा॒तारा॑वि॒षाꣳ र॑यी॒णामु॒भा वाज॑स्य सा॒तये॑ हुवे वाम्॥ अ॒यं ते॒ योनि॑र्\mbox{}ऋ॒त्वियो॒ यतो॑ जा॒तो अरो॑चथाः। तं जा॒नन्न॑ग्न॒ आ रो॒हाथा॑ नो वर्धया र॒यिम्॥ अग्न॒ आयूꣳ॑षि पवस॒ आ सु॒वोर्ज॒मिषं॑ च नः। आ॒रे बा॑धस्व दु॒च्छुना᳚म्॥ अग्ने॒ पव॑स्व॒ स्वपा॑ अ॒स्मे वर्चः॑ सु॒वीर्यम्᳚। दध॒त्पोषꣳ॑ र॒यिं~(१८)

%1.5.5.3
मयि॑॥ अग्ने॑ पावक रो॒चिषा॑ म॒न्द्रया॑ देव जि॒ह्वया᳚। आ दे॒वान् व॑क्षि॒ यक्षि॑ च॥ स नः॑ पावक दीदि॒वो\-ऽग्ने॑ दे॒वाꣳ इ॒हा\-ऽ\-ऽव॑ह। उप॑ य॒ज्ञꣳ ह॒विश्च॑ नः॥ अ॒ग्निः शुचि॑व्रततमः॒ शुचि॒र्विप्रः॒ शुचिः॑ क॒विः। शुची॑ रोचत॒ आहु॑तः॥ उद॑ग्ने॒ शुच॑य॒स्तव॑ शु॒क्रा भ्राज॑न्त ईरते। तव॒ ज्योतीꣴ॑ष्य॒र्चयः॑॥ आ॒यु॒र्दा अ॑ग्ने॒\-ऽस्यायु॑र्मे~(१९)

%1.5.5.4
देहि वर्चो॒दा अ॑ग्ने\-ऽसि॒ वर्चो॑ मे देहि तनू॒पा अ॑ग्ने\-ऽसि त॒नुवं॑ मे पा॒ह्यग्ने॒ यन्मे॑ त॒नुवा॑ ऊ॒नं तन्म॒ आ पृ॑ण॒ चित्रा॑वसो स्व॒स्ति ते॑ पा॒रम॑शी॒येन्धा॑नास्त्वा श॒तꣳ हिमा᳚ द्यु॒मन्तः॒ समि॑धीमहि॒ वय॑स्वन्तो वय॒स्कृतं॒ यश॑स्वन्तो यश॒स्कृतꣳ॑ सु॒वीरा॑सो॒ अदा᳚भ्यम्। अग्ने॑ सपत्न॒दम्भ॑नं॒ वर्\mbox{}षि॑ष्ठे॒ अधि॒ नाके᳚॥ सं त्वम॑ग्ने॒ सूर्य॑स्य॒ वर्च॑सा\-ऽगथाः॒ समृषी॑णाꣴ स्तु॒तेन॒ सं प्रि॒येण॒ धाम्ना᳚। त्वम॑ग्ने॒ सूर्य॑वर्चा असि॒ सं मामायु॑षा॒ वर्च॑सा प्र॒जया॑ सृज॥~(२०)

%1.5.6.0
{\anuvakamend[{आ॒हु॒वध्यै॑ र॒यिं मे॒ वर्च॑सा स॒प्तद॑श च}]}%~(५)

%1.5.6.1
सं प॑श्यामि प्र॒जा अ॒हमिड॑प्रजसो मान॒वीः। सर्वा॑ भवन्तु नो गृ॒हे॥ अम्भः॒ स्थाम्भो॑ वो भक्षीय॒ महः॑ स्थ॒ महो॑ वो भक्षीय॒ सहः॑ स्थ॒ सहो॑ वो भक्षी॒योर्जः॒ स्थोर्जं॑ वो भक्षीय॒ रेव॑ती॒ रम॑ध्वम॒स्मिँल्लो॒के᳚\-ऽस्मिन् गो॒ष्ठे᳚\-ऽस्मिन् क्षये॒\-ऽस्मिन् योना॑वि॒हैव स्ते॒तो मा\-ऽप॑ गात ब॒ह्वीर्मे॑ भूयास्त~(२१)

%1.5.6.2
सꣳहि॒तासि॑ विश्वरू॒पीरा मो॒र्जा वि॒शा\-ऽ\-ऽगौ॑प॒त्येना\-ऽ\-ऽरा॒यस्पोषे॑ण सहस्रपो॒षं वः॑ पुष्यासं॒ मयि॑ वो॒ रायः॑ श्रयन्ताम्॥ उप॑ त्वा\-ऽग्ने दि॒वेदि॑वे॒ दोषा॑वस्तर्धि॒या व॒यम्। नमो॒ भर॑न्त॒ एम॑सि। राज॑न्तमध्व॒राणां᳚ गो॒पामृ॒तस्य॒ दीदि॑विम्। वर्ध॑मान॒ꣴ॒ स्वे दमे᳚॥ स नः॑ पि॒तेव॑ सू॒नवे\-ऽग्ने॑ सूपाय॒नो भ॑व। सच॑स्वा नः स्व॒स्तये᳚॥ अग्ने॒~(२२)

%1.5.6.3
त्वं नो॒ अन्त॑मः। उ॒त त्रा॒ता शि॒वो भ॑व वरू॒थ्यः॑॥ तं त्वा॑ शोचिष्ठ दीदिवः। सु॒म्नाय॑ नू॒नमी॑महे॒ सखि॑भ्यः॥ वसु॑र॒ग्निर्वसु॑श्रवाः। अच्छा॑ नक्षि द्यु॒मत्त॑मो र॒यिं दाः᳚॥ ऊ॒र्जा वः॑ पश्याम्यू॒र्जा मा॑ पश्यत रा॒यस्पोषे॑ण वः पश्यामि रा॒यस्पोषे॑ण मा पश्य॒तेडाः᳚ स्थ मधु॒कृतः॑ स्यो॒ना मा\-ऽ\-ऽवि॑श॒तेरा॒ मदः॑। स॒ह॒स्र॒पो॒षं वः॑ पुष्यासं॒~(२३)

%1.5.6.4
मयि॑ वो॒ रायः॑ श्रयन्ताम्॥ तथ्स॑वि॒तुर्वरे᳚ण्यं॒ भर्गो॑ दे॒वस्य॑ धीमहि। धियो॒ यो नः॑ प्रचो॒दया᳚त्॥ सो॒मान॒ꣴ॒ स्वर॑णं कृणु॒हि ब्र॑ह्मणस्पते। क॒क्षीव॑न्तं॒ य औ॑शि॒जम्॥ क॒दा च॒न स्त॒रीर॑सि॒ नेन्द्र॑ सश्चसि दा॒शुषे᳚। उपो॒पेन्नु म॑घव॒न् भूय॒ इन्नु ते॒ दानं॑ दे॒वस्य॑ पृच्यते॥ परि॑ त्वाऽग्ने॒ पुरं॑ व॒यं विप्रꣳ॑ सहस्य धीमहि। धृ॒षद्व॑र्णं दि॒वेदि॑वे भे॒त्तारं॑ भङ्गु॒राव॑तः॥ अग्ने॑ गृहपते सुगृहप॒तिर॒हं त्वया॑ गृ॒हप॑तिना भूयासꣳ सुगृहप॒तिर्मया॒ त्वं गृ॒हप॑तिना भूयाः श॒तꣳ हिमा॒स्तामा॒शिष॒मा शा॑से॒ तन्त॑वे॒ ज्योति॑ष्मतीं॒ तामा॒शिष॒मा शा॑से॒\-ऽमुष्मै॒ ज्योति॑ष्मतीम्॥~(२४)

%1.5.7.0
{\anuvakamend[{भू॒या॒स्त॒ स्व॒स्तये\-ऽग्ने॑ पुष्यासं धृ॒षद्व॑र्ण॒मेका॒न्न\-त्रि॒ꣳ॒शच्च॑}]}%~(६)

%1.5.7.1
अय॑ज्ञो॒ वा ए॒ष यो॑\-ऽसा॒मोप॑प्र॒यन्तो॑ अध्व॒रमित्या॑ह॒ स्तोम॑मे॒वास्मै॑ युन॒क्त्युपेत्या॑ह प्र॒जा वै प॒शव॒ उपे॒मं लो॒कं प्र॒जामे॒व प॒शूनि॒मं लो॒कमुपै᳚त्य॒स्य प्र॒त्नामनु॒द्युत॒मित्या॑ह सुव॒र्गो वै लो॒कः प्र॒त्नः सु॑व॒र्गमे॒व लो॒कꣳ स॒मारो॑हत्य॒ग्निर्मू॒र्धा दि॒वः क॒कुदित्या॑ह मू॒र्धान॑-~(२५)

%1.5.7.2
मे॒वैनꣳ॑ समा॒नानां᳚ करो॒त्यथो॑ देवलो॒कादे॒व म॑नुष्यलो॒के प्रति॑तिष्ठत्य॒यमि॒ह प्र॑थ॒मो धा॑यि धा॒तृभि॒रित्या॑ह॒ मुख्य॑मे॒वैनं॑ करोत्यु॒भा वा॑मिन्द्राग्नी आहु॒वध्या॒ इत्या॒हौजो॒ बल॑मे॒वाव॑ रुन्धे॒\-ऽयं ते॒ योनि॑र्\mbox{}ऋ॒त्विय॒ इत्या॑ह प॒शवो॒ वै र॒यिः प॒शूने॒वाव॑ रुन्धे ष॒ड्भिरुप॑ तिष्ठते॒ षड्वा~(२६)

%1.5.7.3
ऋ॒तव॑ ऋ॒तुष्वे॒व प्रति॑ तिष्ठति ष॒ड्भिरुत्त॑राभि॒रुप॑ तिष्ठते॒ द्वाद॑श॒ सं प॑द्यन्ते॒ द्वाद॑श॒ मासाः᳚ संवथ्स॒रः सं॑वथ्स॒र ए॒व प्रति॑ तिष्ठति॒ यथा॒ वै पुरु॒षो\-ऽश्वो॒ गौर्जीर्य॑त्ये॒वम॒ग्निराहि॑तो जीर्यति संवथ्स॒रस्य॑ प॒रस्ता॑दाग्निपावमा॒नीभि॒रुप॑ तिष्ठते पुनर्न॒वमे॒वैन॑म॒जरं॑ करो॒त्यथो॑ पु॒नात्ये॒वोप॑ तिष्ठते॒ योग॑ ए॒वास्यै॒ष उप॑ तिष्ठते॒~(२७)

%1.5.7.4
दम॑ ए॒वास्यै॒ष उप॑ तिष्ठते या॒च्ञैवास्यै॒षोप॑ तिष्ठते॒ यथा॒ पापी॑या॒ञ्छ्रेय॑स आ॒हृत्य॑ नम॒स्यति॑ ता॒दृगे॒व तदा॑यु॒र्दा अ॑ग्ने॒\-ऽस्यायु॑र्मे दे॒हीत्या॑हा\-ऽ\-ऽयु॒र्दा ह्ये॑ष व॑र्चो॒दा अ॑ग्ने\-ऽसि॒ वर्चो॑ मे दे॒हीत्या॑ह वर्चो॒दा ह्ये॑ष त॑नू॒पा अ॑ग्ने\-ऽसि त॒नुवं॑ मे पा॒हीत्या॑ह~(२८)

%1.5.7.5
तनू॒पा ह्ये॑षो\-ऽग्ने॒ यन्मे॑ त॒नुवा॑ ऊ॒नं तन्म॒ आ पृ॒णेत्या॑ह॒ यन्मे᳚ प्र॒जायै॑ पशू॒नामू॒नं तन्म॒ आ पू॑र॒येति॒ वावैतदा॑ह॒ चित्रा॑वसो स्व॒स्ति ते॑ पा॒रम॑शी॒येत्या॑ह॒ रात्रि॒र्वै चि॒त्राव॑सु॒रव्यु॑ट्यै॒ वा ए॒तस्यै॑ पु॒रा ब्रा᳚ह्म॒णा अ॑भैषु॒र्व्यु॑ष्टिमे॒वाव॑ रुन्ध॒ इन्धा॑नास्त्वा श॒तꣳ~(२९)

%1.5.7.6
हिमा॒ इत्या॑ह श॒तायुः॒ पुरु॑षः श॒तेन्द्रि॑य॒ आयु॑ष्ये॒वेन्द्रि॒ये प्रति॑ तिष्ठत्ये॒षा वै सू॒र्मी कर्ण॑कावत्ये॒तया॑ ह स्म॒ वै दे॒वा असु॑राणाꣳ शतत॒र्॒\mbox{}हाꣴ स्तृꣳ॑हन्ति॒ यदे॒तया॑ स॒मिध॑मा॒दधा॑ति॒ वज्र॑मे॒वैतच्छ॑त॒घ्नीं यज॑मानो॒ भ्रातृ॑व्याय॒ प्रह॑रति॒ स्तृत्या॒ अछ॑म्बट्कार॒ꣳ॒ सं त्वम॑ग्ने॒ सूर्य॑स्य॒ वर्च॑सा गथा॒ इत्या॑है॒तत्त्वमसी॒दम॒हं भू॑यास॒मिति॒ वावैतदा॑ह॒ त्वम॑ग्ने॒ सूर्य॑वर्चा अ॒सीत्या॑हा॒\-ऽ\-ऽशिष॑मे॒वैतामा शा᳚स्ते॥~(३०)

%1.5.8.0
{\anuvakamend[{मू॒र्धानं॒ वै तिष्ठ॑त आह श॒तम॒हꣳ षोड॑श च}]}%~(७)

%1.5.8.1
सं प॑श्यामि प्र॒जा अ॒हमित्या॑ह॒ याव॑न्त ए॒व ग्रा॒म्याः प॒शव॒स्ताने॒वाव॑ रु॒न्धे\-ऽम्भः॒ स्थाम्भो॑ वो भक्षी॒येत्या॒हाम्भो॒ ह्ये॑ता महः॑ स्थ॒ महो॑ वो भक्षी॒येत्या॑ह॒ महो॒ ह्ये॑ताः सहः॑ स्थ॒ सहो॑ वो भक्षी॒येत्या॑ह॒ सहो॒ ह्ये॑ता ऊर्ज॒स्थोर्जं॑ वो भक्षी॒ये-~(३१)

%1.5.8.2
त्या॒होर्जो॒ ह्ये॑ता रेव॑ती॒ रम॑ध्व॒मित्या॑ह प॒शवो॒ वै रे॒वतीः᳚ प॒शूने॒वात्मन् र॑मयत इ॒हैव स्ते॒तो मा\-ऽप॑ गा॒तेत्या॑ह ध्रु॒वा ए॒वैना॒ अन॑पगाः कुरुत इष्टक॒चिद्वा अ॒न्यो᳚\-ऽग्निः प॑शु॒चिद॒न्यः सꣳ॑हि॒तासि॑ विश्वरू॒पीरिति॑ व॒थ्सम॒भि मृ॑श॒त्युपै॒वैनं॑ धत्ते पशु॒चित॑मेनं कुरुते॒ प्र~(३२)

%1.5.8.3
वा ए॒षो᳚\-ऽस्माल्लो॒काच्च्य॑वते॒ य आ॑हव॒नीय॑मुप॒तिष्ठ॑ते॒ गार्\mbox{}ह॑पत्य॒मुप॑ तिष्ठते॒\-ऽस्मिन्ने॒व लो॒के प्रति॑ तिष्ठ॒त्यथो॒ गार्\mbox{}ह॑पत्यायै॒व नि ह्नु॑ते गाय॒त्रीभि॒रुप॑ तिष्ठते॒ तेजो॒ वै गा॑य॒त्री तेज॑ ए॒वात्मन् ध॒त्ते\-ऽथो॒ यदे॒तं तृ॒चम॒न्वाह॒ सन्त॑त्यै॒ गार्\mbox{}ह॑पत्यं॒ वा अनु॑ द्वि॒पादो॑ वी॒राः प्रजा॑यन्ते॒ य ए॒वं वि॒द्वान् द्वि॒पदा॑भि॒र्गार्\mbox{}ह॑पत्यमुप॒तिष्ठ॑त॒~-~(३३)

%1.5.8.4
आ\-ऽस्य॑ वी॒रो जा॑यत ऊ॒र्जा वः॑ पश्याम्यू॒र्जा मा॑ पश्य॒तेत्या॑हा॒\-ऽ\-ऽशिष॑मे॒वैतामा शा᳚स्ते॒ तथ्स॑वि॒तुर्वरे᳚ण्य॒मित्या॑ह॒ प्रसू᳚त्यै सो॒मान॒ꣴ॒ स्वर॑ण॒मित्या॑ह सोमपी॒थमे॒वाव॑ रुन्धे कृणु॒हि ब्र॑ह्मणस्पत॒ इत्या॑ह ब्रह्मवर्च॒समे॒वाव॑ रुन्धे क॒दा च॒न स्त॒रीर॒सीत्या॑ह॒ न स्त॒रीꣳ रात्रिं॑ वसति॒~(३४)

%1.5.8.5
य ए॒वं वि॒द्वान॒ग्निमु॑प॒तिष्ठ॑ते॒ परि॑ त्वाऽग्ने॒ पुरं॑ व॒यमित्या॑ह परि॒धिमे॒वैतं परि॑ दधा॒त्यस्क॑न्दा॒याग्ने॑ गृहपत॒ इत्या॑ह यथाय॒जुरे॒वैतच्छ॒तꣳ हिमा॒ इत्या॑ह श॒तं त्वा॑ हेम॒न्तानि॑न्धिषी॒येति॒ वावैतदा॑ह पु॒त्रस्य॒ नाम॑ गृह्णात्यन्ना॒दमे॒वैनं॑ करोति॒ तामा॒शिष॒मा शा॑से॒ तन्त॑वे॒ ज्योति॑ष्मती॒मिति॑ ब्रूया॒द्यस्य॑ पु॒त्रो\-ऽजा॑तः॒ स्यात्ते॑ज॒स्व्ये॑वास्य॑ ब्रह्मवर्च॒सी पु॒त्रो जा॑यते॒ तामा॒शिष॒मा शा॑से॒\-ऽमुष्मै॒ ज्योति॑ष्मती॒मिति॑ ब्रूया॒द्यस्य॑ पु॒त्रो जा॒तः स्यात् तेज॑ ए॒वास्मि॑न् ब्रह्मवर्च॒सं द॑धाति॥~(३५)

%1.5.9.0
{\anuvakamend[{ऊर्जं॑ वो भक्षी॒येति॒ प्र गार्\mbox{}ह॑पत्यमुप॒तिष्ठ॑ते वसति॒ ज्योति॑ष्मती॒मेका॒न्न\-त्रि॒ꣳ॒शच्च॑}]}%~(८)

%1.5.9.1
अ॒ग्नि॒हो॒त्रं जु॑होति॒ यदे॒व किं च॒ यज॑मानस्य॒ स्वं तस्यै॒व तद्रेतः॑ सिञ्चति प्र॒जन॑ने प्र॒जन॑न॒ꣳ॒ हि वा अ॒ग्निरथौष॑धी॒रन्त॑गता दहति॒ तास्ततो॒ भूय॑सीः॒ प्रजा॑यन्ते॒ यथ्सा॒यं जु॒होति॒ रेत॑ ए॒व तथ्सि॑ञ्चति॒ प्रैव प्रा॑त॒स्तने॑न जनयति॒ तद्रेतः॑ सि॒क्तं न त्वष्ट्रा\-ऽवि॑कृतं॒ प्रजा॑यते याव॒च्छो वै रेत॑सः सि॒क्तस्य॒~(३६)

%1.5.9.2
त्वष्टा॑ रू॒पाणि॑ विक॒रोति॑ ताव॒च्छो वै तत्प्रजा॑यत ए॒ष वै दैव्य॒स्त्वष्टा॒ यो यज॑ते ब॒ह्वीभि॒रुप॑ तिष्ठते॒ रेत॑स ए॒व सि॒क्तस्य॑ बहु॒शो रू॒पाणि॒ वि क॑रोति॒ स प्रैव जा॑यते॒ श्वःश्वो॒ भूया᳚न् भवति॒ य ए॒वं वि॒द्वान॒ग्निमु॑प॒तिष्ठ॒ते\-ऽह॑र्दे॒वाना॒\-मासी॒द्रात्रि॒रसु॑राणां॒ ते\-ऽसु॑रा॒ यद्दे॒वानां᳚ वि॒त्तं वेद्य॒मासी॒त्तेन॑ स॒ह~(३७)

%1.5.9.3
रात्रिं॒ प्रावि॑श॒न् ते दे॒वा ही॒ना अ॑मन्यन्त॒ ते॑\-ऽपश्यन्नाग्ने॒यी रात्रि॑राग्ने॒याः प॒शव॑ इ॒ममे॒वाग्निꣴ स्त॑वाम॒ स नः॑ स्तु॒तः प॒शून् पुन॑र्दास्य॒तीति॒ ते᳚\-ऽग्निम॑स्तुव॒न्थ्स ए᳚भ्यः स्तु॒तो रात्रि॑या॒ अध्यह॑र॒भि प॒शून्निरा᳚र्ज॒त् ते दे॒वाः प॒शून् वि॒त्वा कामाꣳ॑ अकुर्वत॒ य ए॒वं वि॒द्वान॒ग्निमु॑प॒तिष्ठ॑ते पशु॒मान् भ॑वत्या-~(३८)

%1.5.9.4
दि॒त्यो वा अ॒स्माल्लो॒काद॒मुं लो॒कमै॒थ्सो॑\-ऽमुं लो॒कं ग॒त्वा पुन॑रि॒मं लो॒कम॒भ्य॑ध्याय॒थ्स इ॒मं लो॒कमा॒गत्य॑ मृ॒त्योर॑बिभेन्मृ॒त्युसं॑युत इव॒ ह्य॑यं लो॒कः सो॑\-ऽमन्यते॒ममे॒वाग्निꣴ स्त॑वानि॒ स मा᳚ स्तु॒तः सु॑व॒र्गं लो॒कं ग॑मयिष्य॒तीति॒ सो᳚\-ऽग्निम॑स्तौ॒थ्स ए॑नꣴ स्तु॒तः सु॑व॒र्गं लो॒कम॑गमय॒द्य~-~(३९)

%1.5.9.5
ए॒वं वि॒द्वान॒ग्निमु॑प॒तिष्ठ॑ते सुव॒र्गमे॒व लो॒कमे॑ति॒ सर्व॒मायु॑रेत्य॒भि वा ए॒षो᳚\-ऽग्नी आ रो॑हति॒ य ए॑नावुप॒तिष्ठ॑ते॒ यथा॒ खलु॒ वै श्रेया॑न॒भ्यारू॑ढः का॒मय॑ते॒ तथा॑ करोति॒ नक्त॒मुप॑ तिष्ठते॒ न प्रा॒तः सꣳ हि नक्तं॑ व्र॒तानि॑ सृ॒ज्यन्ते॑ स॒ह श्रेयाꣴ॑श्च॒ पापी॑याꣴश्चासाते॒ ज्योति॒र्वा अ॒ग्निस्तमो॒ रात्रि॒र्यन्-~(४०)

%1.5.9.6
नक्त॑मुप॒तिष्ठ॑ते॒ ज्योति॑षै॒व तम॑स्तरत्युप॒स्थेयो॒\-ऽग्नी(३)र्नोप॒\-स्थेया(३) इत्या॑हुर्मनु॒ष्या॑येन्न्वै यो\-ऽह॑रहरा॒हृत्या\-थै॑नं॒ याच॑ति॒ स इन्न्वै तमुपा᳚र्च्छ॒त्यथ॒ को दे॒वानह॑रहर्याचिष्य॒तीति॒ तस्मा॒न्नोप॒स्थेयो\-ऽथो॒ खल्वा॑हुरा॒शिषे॒ वै कं यज॑मानो यजत॒ इत्ये॒षा खलु॒ वा~(४१)

%1.5.9.7
आहि॑ताग्नेरा॒शीर्यद॒ग्नि\-मु॑प॒तिष्ठ॑ते॒ तस्मा॑दुप॒स्थेयः॑ प्र॒जा\-प॑तिः प॒शून॑सृजत॒ ते सृ॒ष्टा अ॑होरा॒त्रे प्रावि॑श॒न् ताञ्छन्दो॑\-भि॒रन्व॑\-विन्द॒द्यच्छन्दो॑भि\-रुप॒तिष्ठ॑ते॒ स्वमे॒व तदन्वि॑च्छति॒ न तत्र॑ जा॒म्य॑स्तीत्या॑हु॒र्यो\-ऽह॑रहरुप॒तिष्ठ॑त॒ इति॒ यो वा अ॒ग्निं प्र॒त्यङ्ङु॑प॒तिष्ठ॑ते॒ प्रत्ये॑नमोषति॒ यः परा॒ङ्॒ विष्व॑ङ् प्र॒जया॑ प॒शुभि॑रेति॒ कवा॑तिर्यङ्ङि॒वोप॑ तिष्ठेत॒ नैनं॑ प्र॒त्योष॑ति॒ न विष्व॑ङ् प्र॒जया॑ प॒शुभि॑रेति॥~(४२)

%1.5.10.0
{\anuvakamend[{सि॒क्तस्य॑ स॒ह भ॑वति॒ यो यत्खलु॒ वै प॒शुभि॒स्त्रयो॑दश च}]}%~(९)

%1.5.10.1
मम॒ नाम॑ प्रथ॒मं जा॑तवेदः पि॒ता मा॒ता च॑ दधतु॒र्यदग्रे᳚। तत्त्वं बि॑भृहि॒ पुन॒रा मदैतो॒स्तवा॒हं नाम॑ बिभराण्यग्ने॥ मम॒ नाम॒ तव॑ च जातवेदो॒ वास॑सी इव वि॒वसा॑नौ॒ ये चरा॑वः। आयु॑षे॒ त्वं जी॒वसे॑ व॒यं य॑थाय॒थं वि परि॑ दधावहै॒ पुन॒स्ते॥ नमो॒\-ऽग्नये\-ऽप्र॑तिविद्धाय॒ नमो\-ऽना॑धृष्टाय॒ नमः॑ स॒म्राजे᳚। अषा॑ढो~(४३)

%1.5.10.2
अ॒ग्निर्बृ॒हद्व॑या विश्व॒जिथ्सह॑न्त्यः॒ श्रेष्ठो॑ गन्ध॒र्वः॥ त्वत्पि॑तारो अग्ने दे॒वास्त्वामा॑\-हुतय॒स्त्वद्वि॑वाचनाः। सं मामायु॑षा॒ सं गौ॑प॒त्येन॒ सुहि॑ते मा धाः॥ अ॒यम॒ग्निः श्रेष्ठ॑तमो॒\-ऽयं भग॑वत्तमो॒\-ऽयꣳ स॑हस्र॒सात॑मः। अ॒स्मा अ॑स्तु सु॒वीर्यम्᳚॥ मनो॒ ज्योति॑र्जुषता॒माज्यं॒ विच्छि॑न्नं य॒ज्ञꣳ समि॒मं द॑धातु। या इ॒ष्टा उ॒षसो॑ नि॒म्रुच॑श्च॒ ताः सं द॑धामि ह॒विषा॑ घृ॒तेन॑॥ पय॑स्वती॒रोष॑धयः॒~(४४)

%1.5.10.3
पय॑स्वद्वी॒रुधां॒ पयः॑। अ॒पां पय॑सो॒ यत्पय॒स्तेन॒ मामि॑न्द्र॒ सꣳ सृ॑ज॥ अग्ने᳚ व्रतपते व्र॒तं च॑रिष्यामि॒ तच्छ॑केयं॒ तन्मे॑ राध्यताम्॥ अ॒ग्निꣳ होता॑रमि॒ह तꣳ हु॑वे दे॒वान् य॒ज्ञिया॑नि॒ह यान् हवा॑महे॥ आ य॑न्तु दे॒वाः सु॑मन॒स्यमा॑ना वि॒यन्तु॑ दे॒वा ह॒विषो॑ मे अ॒स्य॥ कस्त्वा॑ युनक्ति॒ स त्वा॑ युनक्तु॒ यानि॑ घ॒र्मे क॒पाला᳚न्युपचि॒न्वन्ति॑~(४५)

%1.5.10.4
वे॒धसः॑। पू॒ष्णस्तान्यपि॑ व्र॒त इ॑न्द्रवा॒यू विमु॑ञ्चताम्॥ अभि॑न्नो घ॒र्मो जी॒रदा॑नु॒र्यत॒ आत्त॒स्तद॑ग॒न् पुनः॑। इ॒ध्मो वेदिः॑ परि॒धय॑श्च॒ सर्वे॑ य॒ज्ञस्या\-ऽ\-ऽयु॒रनु॒ सं च॑रन्ति॥ त्रय॑स्त्रिꣳश॒त्तन्त॑वो॒ ये वि॑तत्नि॒रे य इ॒मं य॒ज्ञꣴ स्व॒धया॒ दद॑न्ते॒ तेषां᳚ छि॒न्नं प्रत्ये॒तद्द॑धामि॒ स्वाहा॑ घ॒र्मो दे॒वाꣳ अप्ये॑तु॥~(४६)

%1.5.11.0
{\anuvakamend[{अषा॑ढ॒ ओष॑धय उपचि॒न्वन्ति॒ पञ्च॑चत्वारिꣳशच्च}]}%॥10॥

%1.5.11.1
वै॒श्वा॒न॒रो न॑ ऊ॒त्या\-ऽ\-ऽप्र या॑तु परा॒वतः॑। अ॒ग्निरु॒क्थेन॒ वाह॑सा॥ ऋ॒तावा॑नं वैश्वान॒रमृ॒तस्य॒ ज्योति॑ष॒स्पतिम्᳚। अज॑स्रं घ॒र्ममी॑महे॥ वै॒श्वा॒न॒रस्य॑ द॒ꣳ॒सना᳚भ्यो बृ॒हदरि॑णा॒देकः॑ स्वप॒स्य॑या क॒विः। उ॒भा पि॒तरा॑ म॒हय॑न्नजायता॒ग्निर्द्यावा॑\-पृथि॒वी भूरि॑रेतसा॥ पृ॒ष्टो दि॒वि पृ॒ष्टो अ॒ग्निः पृ॑थि॒व्यां पृ॒ष्टो विश्वा॒ ओष॑धी॒रा वि॑वेश। वै॒श्वा॒न॒रः सह॑सा पृ॒ष्टो अ॒ग्निः स नो॒ दिवा॒ स~(४७)

%1.5.11.2
रि॒षः पा॑तु॒ नक्तम्᳚॥ जा॒तो यद॑ग्ने॒ भुव॑ना॒ व्यख्यः॑ प॒शुं न गो॒पा इर्यः॒ परि॑ज्मा। वैश्वा॑नर॒ ब्रह्म॑णे विन्द गा॒तुं यू॒यं पा॑त स्व॒स्तिभिः॒ सदा॑ नः॥ त्वम॑ग्ने शो॒चिषा॒ शोशु॑चान॒ आ रोद॑सी अपृणा॒ जाय॑मानः। त्वं दे॒वाꣳ अ॒भिश॑स्तेरमुञ्चो॒ वैश्वा॑नर जातवेदो महि॒त्वा॥ अ॒स्माक॑मग्ने म॒घव॑थ्सु धार॒याना॑मि क्ष॒त्रम॒जरꣳ॑ सु॒वीर्यम्᳚। व॒यं ज॑येम श॒तिनꣳ॑ सह॒स्रिणं॒ वैश्वा॑नर॒~(४८)

%1.5.11.3
वाज॑मग्ने॒ तवो॒तिभिः॑॥ वै॒श्वा॒न॒रस्य॑ सुम॒तौ स्या॑म॒ राजा॒ हिकं॒ भुव॑नानामभि॒श्रीः। इ॒तो जा॒तो विश्व॑मि॒दं वि च॑ष्टे वैश्वान॒रो य॑तते॒ सूर्ये॑ण॥ अव॑ ते॒ हेडो॑ वरुण॒ नमो॑\-भि॒रव॑ य॒ज्ञेभि॑रीमहे ह॒विर्भिः॑। क्षय॑न्न॒स्मभ्य॑मसुर प्रचेतो॒ राज॒न्नेनाꣳ॑सि शिश्रथः कृ॒तानि॑॥ उदु॑त्त॒मं व॑रुण॒ पाश॑\-म॒स्मद\-वा॑ध॒मं वि म॑ध्य॒मꣴ श्र॑थाय। अथा॑ व॒यमा॑दित्य~(४९)

%1.5.11.4
व्र॒ते तवाना॑गसो॒ अदि॑तये स्याम॥ द॒धि॒क्राव्ण्णो॑ अकारिषं जि॒ष्णोरश्व॑स्य वा॒जिनः॑॥ सु॒र॒भि नो॒ मुखा॑ कर॒त् प्र ण॒ आयूꣳ॑षि तारिषत्॥ आ द॑धि॒क्राः शव॑सा॒ पञ्च॑ कृ॒ष्टीः सूर्य॑ इव॒ ज्योति॑षा॒\-ऽपस्त॑तान। स॒ह॒स्र॒साः श॑त॒सा वा॒ज्यर्वा॑ पृ॒णक्तु॒ मध्वा॒ समि॒मा वचाꣳ॑सि॥ अ॒ग्निर्मू॒र्धा भुवः॑। मरु॑तो॒ यद्ध॑ वो दि॒वः सु॑म्ना॒यन्तो॒ हवा॑महे। आ तू न॒~(५०)

%1.5.11.5
उप॑ गन्तन॥ या वः॒ शर्म॑ शशमा॒नाय॒ सन्ति॑ त्रि॒धातू॑नि दा॒शुषे॑ यच्छ॒ताधि॑। अ॒स्मभ्यं॒ तानि॑ मरुतो॒ वि य॑न्त र॒यिं नो॑ धत्त वृषणः सु॒वीरम्᳚॥ अदि॑तिर्न उरुष्य॒त्वदि॑तिः॒ शर्म॑ यच्छतु। अदि॑तिः पा॒त्वꣳह॑सः॥ म॒हीमू॒ षु मा॒तरꣳ॑ सुव्र॒ता\-ना॑\-मृ॒तस्य॒ पत्नी॒मव॑से हुवेम। तु॒वि॒क्ष॒त्रा\-म॒जर॑न्ती\-मुरू॒चीꣳ सु॒शर्मा॑ण॒मदि॑तिꣳ सु॒प्रणी॑तिम्॥ सु॒त्रामा॑णं पृथि॒वीं द्याम॑ने॒हसꣳ॑ सु॒शर्मा॑ण॒मदि॑तिꣳ सु॒प्रणी॑तिम्। दैवीं॒ नावꣴ॑ स्वरि॒त्रा\-मना॑\-गस॒मस्र॑वन्ती॒मा रु॑हेमा स्व॒स्तये᳚॥ इ॒माꣳ सु नाव॒मा\-ऽरु॑हꣳ श॒तारि॑त्राꣳ श॒तस्फ्या᳚म्। अच्छि॑द्रां पारयि॒ष्णुम्॥~(५१)

{\anuvakamend[{दिवा॒ स स॑ह॒स्रिणं॒ वैश्वा॑नरा\-ऽ\-ऽदित्य॒ तू नो॑\-ऽने॒हसꣳ॑ सु॒शर्मा॑ण॒मेका॒न्न\-विꣳ॑श॒तिश्च॑}]}%॥11॥
%%% END PRASHNA
