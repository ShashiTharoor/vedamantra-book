\chapt{काण्डम् २}
\sect{प्रथमः प्रश्नः}\setcounter{anuvakam}{0}
\dnsub{तैत्तिरीयसंहितायां द्वितीयकाण्डे प्रथमः प्रश्नः}
%2.1.1.0
%2.1.1.1
वा॒य॒व्यꣴ॑ श्वे॒तमा\-ल॑भेत॒ भूति॑कामो वा॒युर्वै क्षेपि॑ष्ठा दे॒वता॑ वा॒युमे॒व स्वेन॑ भाग॒धेये॒नोप॑ धावति॒ स ए॒वैनं॒ भूतिं॑ गमयति॒ भव॑त्ये॒वाति॑क्षिप्रा दे॒वतेत्या॑हुः॒ सैन॑मीश्व॒रा प्र॒दह॒ इत्ये॒तमे॒व सन्तं॑ वा॒यवे॑ नि॒युत्व॑त॒ आल॑भेत नि॒युद्वा अ॑स्य॒ धृति॑र्धृ॒त ए॒व भूति॒मुपै॒त्यप्र॑दाहाय॒ भव॑त्ये॒व~(१)

%2.1.1.2
वा॒यवे॑ नि॒युत्व॑त॒ आ\-ल॑भेत॒ ग्राम॑कामो वा॒युर्वा इ॒माः प्र॒जा न॑स्यो॒ता ने॑नीयते वा॒युमे॒व नि॒युत्व॑न्त॒ꣴ॒ स्वेन॑ भाग॒धेये॒नोप॑ धावति॒ स ए॒वास्मै᳚ प्र॒जा न॑स्यो॒ता निय॑च्छति ग्रा॒म्ये॑व भ॑वति नि॒युत्व॑ते भवति ध्रु॒वा ए॒वास्मा॒ अन॑पगाः करोति वा॒यवे॑ नि॒युत्व॑त॒ आल॑भेत प्र॒जाका॑मः प्रा॒णो वै वा॒युर॑पा॒नो नि॒युत्प्रा॑णापा॒नौ खलु॒ वा ए॒तस्य॑ प्र॒जाया॒~(२)

%2.1.1.3
अप॑क्रामतो॒ यो\-ऽलं॑ प्र॒जायै॒ सन्प्र॒जां न वि॒न्दते॑ वा॒युमे॒व नि॒युत्व॑न्त॒ꣴ॒ स्वेन॑ भाग॒धेये॒नोप॑ धावति॒ स ए॒वास्मै᳚ प्राणापा॒ना\-भ्यां᳚ प्र॒जां प्रज॑नयति वि॒न्दते᳚ प्र॒जां वा॒यवे॑ नि॒युत्व॑त॒ आ\-ल॑भेत॒ ज्योगा॑मयावी प्रा॒णो वै वा॒युर॑पा॒नो नि॒युत् प्रा॑णापा॒नौ खलु॒ वा ए॒तस्मा॒दप॑क्रामतो॒ यस्य॒ ज्योगा॒मय॑ति वा॒युमे॒व नि॒युत्व॑न्त॒ꣴ॒ स्वेन॑ भाग॒धेये॒नोप॑~(३)

%2.1.1.4
धावति॒ स ए॒वास्मि॑न्प्राणापा॒नौ द॑धात्यु॒त यदी॒तासु॒र्भव॑ति॒ जीव॑त्ये॒व प्र॒जा\-प॑ति॒र्वा इ॒दमेक॑ आसी॒थ्सो॑\-ऽकामयत प्र॒जाः प॒शून्थ्सृ॑जे॒येति॒ स आ॒त्मनो॑ व॒पामुद॑क्खिद॒त्ताम॒ग्नौ प्रागृ॑ह्णा॒त्ततो॒\-ऽजस्तू॑प॒रः सम॑भव॒त्तꣴ स्वायै॑ दे॒वता॑या॒ आ\-ऽल॑भत॒ ततो॒ वै स प्र॒जाः प॒शून॑सृजत॒ यः प्र॒जाका॑मः~(४)

%2.1.1.5
प॒शुका॑मः॒ स्याथ्स ए॒तं प्रा॑जाप॒त्यम॒जं तू॑प॒रमाल॑भेत प्र॒जा\-प॑तिमे॒व स्वेन॑ भाग॒धेये॒नोप॑ धावति॒ स ए॒वास्मै᳚ प्र॒जां प॒शून्प्रज॑नयति॒ यच्छ्म॑श्रु॒णस्तत्पुरु॑षाणाꣳ रू॒पं यत्तू॑प॒रस्तदश्वा॑नां॒ यद॒न्यतो॑द॒न्तद्गवां॒ यदव्या॑ इव श॒फास्तदवी॑नां॒ यद॒ज\-स्तद॒जाना॑मे॒ताव॑न्तो॒ वै ग्रा॒म्याः प॒शव॒स्तान्~(५)

%2.1.1.6
रू॒पेणै॒वाव॑\-रुन्धे सोमापौ॒ष्णं त्रै॒तमाल॑भेत प॒शुका॑मो॒ द्वौ वा अ॒जायै॒ स्तनौ॒ नानै॒व द्वाव॒भिजाये॑ते॒ ऊर्जं॒ पुष्टिं॑ तृ॒तीयः॑ सोमापू॒षणा॑वे॒व स्वेन॑ भाग॒धेये॒नोप॑ धावति॒ तावे॒वास्मै॑ प॒शून्प्रज॑नयतः॒ सोमो॒ वै रे॑तो॒धाः पू॒षा प॑शू॒नां प्र॑जनयि॒ता सोम॑ ए॒वास्मै॒ रेतो॒ दधा॑ति पू॒षा प॒शून्प्रज॑नय॒त्यौदु॑म्बरो॒ यूपो॑ भव॒त्यूर्ग्वा उ॑दु॒म्बर॒ ऊर्क्प॒शव॑ ऊ॒र्जैवास्मा॒ ऊर्जं॑ प॒शूनव॑\-रुन्धे॥~(६)

%2.1.2.0
{\anuvakamend[{भव॑त्ये॒व प्र॒जाया॑ आ॒मय॑ति वा॒युमे॒व नि॒युत्व॑न्त॒ꣴ॒ स्वेन॑ भाग॒धेये॒नोप॑ प्र॒जाका॑म॒स्तान् यूप॒स्त्रयो॑दश च।}]}

%2.1.2.1
प्र॒जा\-प॑तिः प्र॒जा अ॑सृजत॒ ता अ॑स्माथ्सृ॒ष्टाः परा॑चीराय॒न्ता वरु॑णमगच्छ॒न्ता अन्वै॒त्ताः पुन॑रयाचत॒ ता अ॑स्मै॒ न पुन॑रददा॒थ्सो᳚\-ऽब्रवी॒द्वरं॑ वृणी॒ष्वाथ॑ मे॒ पुन॑र्दे॒हीति॒ तासां॒ वर॒मा\-ऽल॑भत॒ स कृ॒ष्ण एक॑शितिपादभव॒द्यो वरु॑णगृहीतः॒ स्याथ्स ए॒तं वा॑रु॒णं कृ॒ष्णमेक॑शितिपाद॒मा\-ल॑भेत॒ वरु॑ण-~(७)

%2.1.2.2
मे॒व स्वेन॑ भाग॒धेये॒नोप॑ धावति॒ स ए॒वैनं॑ वरुणपा॒शान्मु॑ञ्चति कृ॒ष्ण एक॑शितिपाद्भवति वारु॒णो ह्ये॑ष दे॒वत॑या॒ समृ॑द्ध्यै॒ सुव॑र्भानुरासु॒रः सूर्यं॒ तम॑सा\-ऽविद्ध्य॒त्तस्मै॑ दे॒वाः प्राय॑श्चित्तिमैच्छ॒न्तस्य॒ यत्प्र॑थ॒मं तमो॒\-ऽपाघ्न॒न्थ्सा कृ॒ष्णा\-ऽवि॑रभव॒द्यद्द्वि॒तीय॒ꣳ॒ सा फल्गु॑नी॒ यत्तृ॒तीय॒ꣳ॒ सा ब॑ल॒क्षी यद॑द्ध्य॒स्थाद॒पाकृ॑न्त॒न्थ्सा\-ऽवि॑\-ऽर्व॒शा~(८)

%2.1.2.3
सम॑भव॒त्ते दे॒वा अ॑ब्रुवन्देवप॒शुर्वा अ॒यꣳ सम॑भू॒त्कस्मा॑ इ॒ममाल॑फ्स्यामह॒ इत्यथ॒ वै तर्ह्यल्पा॑ पृथि॒व्यासी॒दजा॑ता॒ ओष॑धय॒स्तामविं॑ व॒शामा॑दि॒त्येभ्यः॒ कामा॒या\-ऽल॑भन्त॒ ततो॒ वा अप्र॑थत पृथि॒व्यजा॑य॒न्तौष॑धयो॒ यः का॒मये॑त॒ प्रथे॑य प॒शुभिः॒ प्र प्र॒जया॑ जाये॒येति॒ स ए॒तामविं॑ व॒शामा॑दि॒त्येभ्यः॒ कामा॒-~(९)

%2.1.2.4
याऽऽ ल॑भेताऽऽदि॒त्याने॒व काम॒ꣴ॒ स्वेन॑ भाग॒धेये॒नोप॑ धावति॒ त ए॒वैनं॑ प्र॒थय॑न्ति प॒शुभिः॒ प्र प्र॒जया॑ जनयन्त्य॒सावा॑दि॒त्यो न व्य॑रोचत॒ तस्मै॑ दे॒वाः प्राय॑श्चित्तिमैच्छ॒न्तस्मा॑ ए॒ता म॒ल्॒\mbox{}हा आल॑\-ऽभन्ताऽऽग्ने॒यीं कृ॑ष्णग्री॒वीꣳ सꣳ॑हि॒तामै॒न्द्रीꣴ श्वे॒तां बा॑र्\mbox{}हस्प॒त्यां ताभि॑रे॒वास्मि॒न्रुच॑मदधु॒र्यो ब्र॑ह्मवर्च॒सका॑मः॒ स्यात्तस्मा॑ ए॒ता म॒ल्॒\mbox{}हा आल॑भे-~(१०)

%2.1.2.5
ताऽऽग्ने॒यीं कृ॑ष्णग्री॒वीꣳ सꣳ॑हि॒तामै॒न्द्रीꣴ श्वे॒तां बा॑र्\mbox{}हस्प॒त्या\-मे॒ता ए॒व दे॒वताः॒ स्वेन॑ भाग॒धेये॒नोप॑ धावति॒ ता ए॒वास्मि॑न्ब्रह्मवर्च॒सं द॑धति ब्रह्मवर्च॒स्ये॑व भ॑वति व॒सन्ता᳚ प्रा॒तरा᳚ग्ने॒यीं कृ॑ष्णग्री॒वीमाल॑भेत ग्री॒ष्मे म॒ध्यन्दि॑ने सꣳहि॒तामै॒न्द्रीꣳ श॒रद्य॑परा॒ह्णे श्वे॒तां बा॑र्\mbox{}हस्प॒त्यां त्रीणि॒ वा आ॑दि॒त्यस्य॒ तेजाꣳ॑सि व॒सन्ता᳚ प्रा॒तर्ग्री॒ष्मे म॒ध्यन्दि॑ने श॒रद्य॑परा॒ह्णे याव॑न्त्ये॒व तेजाꣳ॑सि॒ तान्ये॒-~(११)

%2.1.2.6
वाव॑ रुन्धे संवथ्स॒रं प॒र्याल॑भ्यन्ते संवथ्स॒रो वै ब्र॑ह्मवर्च॒सस्य॑ प्रदा॒ता सं॑वथ्स॒र ए॒वास्मै᳚ ब्रह्मवर्च॒सं प्रय॑च्छति ब्रह्मवर्च॒स्ये॑व भ॑वति ग॒र्भिण॑यो भवन्तीन्द्रि॒यं वै गर्भ॑ इन्द्रि॒यमे॒वास्मि॑न्दधति सारस्व॒तीं मे॒षीमा\-ल॑भेत॒ य ई᳚श्व॒रो वा॒चो वदि॑तोः॒ सन्वाचं॒ न वदे॒द्वाग्वै सर॑स्वती॒ सर॑स्वतीमे॒व स्वेन॑ भाग॒धेये॒नोप॑ धावति॒ सैवास्मि॒न्~(१२)

%2.1.2.7
वाचं॑ दधाति प्रवदि॒ता वा॒चो भ॑व॒त्यप॑न्नदती भवति॒ तस्मा᳚न्मनु॒ष्याः᳚ सर्वां॒ वाचं॑ वदन्त्याग्ने॒यं कृ॒ष्णग्री॑व॒मा ल॑भेत सौ॒म्यं ब॒भ्रुं ज्योगा॑मयाव्य॒ग्निं वा ए॒तस्य॒ शरी॑रं गच्छति॒ सोम॒ꣳ॒ रसो॒ यस्य॒ ज्योगा॒मय॑त्य॒ग्नेरे॒वास्य॒ शरी॑रं निष्क्री॒णाति॒ सोमा॒द्रस॑मु॒त यदी॒तासु॒र्भव॑ति॒ जीव॑त्ये॒व सौ॒म्यं ब॒भ्रुमा\-ल॑भेताऽऽग्ने॒यं कृ॒ष्णग्री॑वं प्र॒जाका॑मः॒ सोमो॒~(१३)

%2.1.2.8
वै रे॑तो॒धा अ॒ग्निः प्र॒जानां᳚ प्रजनयि॒ता सोम॑ ए॒वास्मै॒ रेतो॒ दधा᳚त्य॒ग्निः प्र॒जां प्रज॑नयति वि॒न्दते᳚ प्र॒जामा᳚ग्ने॒यं कृ॒ष्ण\-ग्री॑व॒माल॑भेत सौ॒म्यं ब॒भ्रुं यो ब्रा᳚ह्म॒णो वि॒द्याम॒नूच्य॒ न वि॒रोचे॑त॒ यदा᳚ग्ने॒यो भव॑ति॒ तेज॑ ए॒वास्मि॒न्तेन॑ दधाति॒ यथ्सौ॒म्यो ब्र॑ह्मवर्च॒सं तेन॑ कृ॒ष्णग्री॑व आग्ने॒यो भ॑वति॒ तम॑ ए॒वास्मा॒दप॑हन्ति श्वे॒तो भ॑वति॒~(१४)

%2.1.2.9
रुच॑मे॒वास्मि॑न्दधाति ब॒भ्रुः सौ॒म्यो भ॑वति ब्रह्मवर्च॒स\-मे॒वास्मि॒न्त्विषिं॑ दधात्याग्ने॒यं कृ॒ष्णग्री॑व॒माल॑भेत सौ॒म्यं ब॒भ्रुमा᳚ग्ने॒यं कृ॒ष्णग्री॑वं पुरो॒धाया॒ꣴ॒ स्पर्ध॑मान आग्ने॒यो वै ब्रा᳚ह्म॒णः सौ॒म्यो रा॑ज॒न्यो॑\-ऽभितः॑ सौ॒म्यमा᳚ग्ने॒यौ भ॑वत॒स्तेज॑सै॒व ब्रह्म॑णोभ॒यतो॑ रा॒ष्ट्रं परि॑\-गृह्णात्येक॒धा स॒मावृ॑ङ्क्ते पु॒र ए॑नं दधते॥~(१५)

%2.1.3.0
{\anuvakamend[{ल॒भे॒त॒ वरु॑णं व॒शैतामविं॑ व॒शामा॑दि॒त्येभ्यः॒ कामा॑य म॒ल्॒\mbox{}हा आ\-ल॑भेत॒ तान्ये॒व सैवास्मि॒न्थ्सोमः॑ श्वे॒तो भ॑वति॒ त्रिच॑त्वारिꣳशच्च।~(२)।}]}

%2.1.3.1
दे॒वा॒सु॒रा ए॒षु लो॒केष्व॑स्पर्धन्त॒ स ए॒तं विष्णु॑र्वाम॒नम॑\-पश्य॒त्तꣴ स्वायै॑ दे॒वता॑या॒ आ\-ऽल॑भत॒ ततो॒ वै स इ॒माँल्लो॒कान॒भ्य॑\-जयद्वैष्ण॒वं वा॑म॒नमा\-ल॑भेत॒ स्पर्ध॑मानो॒ विष्णु॑रे॒व भू॒त्वेमाँल्लो॒कान॒भिज॑यति॒ विष॑म॒ आ\-ल॑भेत॒ विष॑मा इव॒ हीमे लो॒काः समृ॑द्ध्या॒ इन्द्रा॑य मन्यु॒मते॒ मन॑स्वते ल॒लामं॑ प्राशृ॒ङ्गमाल॑भेत सङ्ग्रा॒मे~(१६)

%2.1.3.2
सं य॑त्त इन्द्रि॒येण॒ वै म॒न्युना॒ मन॑सा सङ्ग्रा॒मं ज॑य॒तीन्द्र॑मे॒व म॑न्यु॒मन्तं॒ मन॑स्वन्त॒ꣴ॒ स्वेन॑ भाग॒धेये॒नोप॑ धावति॒ स ए॒वास्मि॑न्निन्द्रि॒यं म॒न्युं मनो॑ दधाति॒ जय॑ति॒ तꣳ स॑ङ्ग्रा॒ममिन्द्रा॑य म॒रुत्व॑ते पृश्ञिस॒क्थमा\-ल॑भेत॒ ग्राम॑काम॒ इन्द्र॑मे॒व म॒रुत्व॑न्त॒ꣴ॒ स्वेन॑ भाग॒धेये॒नोप॑ धावति॒ स ए॒वास्मै॑ सजा॒तान्प्रय॑च्छति ग्रा॒म्ये॑व भ॑वति॒ यदृ॑ष॒भस्ते-~(१७)

%2.1.3.3
नै॒न्द्रो यत्पृश्ञि॒स्तेन॑ मारु॒तः समृ॑द्ध्यै प॒श्चात्पृ॑श्ञिस॒क्थो भ॑वति पश्चादन्ववसा॒यिनी॑मे॒वास्मै॒ विशं॑ करोति सौ॒म्यं ब॒भ्रुमाल॑भे॒तान्न॑कामः सौ॒म्यं वा अन्न॒ꣳ॒ सोम॑मे॒व स्वेन॑ भाग॒धेये॒नोप॑ धावति॒ स ए॒वास्मा॒ अन्नं॒ प्रय॑च्छत्यन्ना॒द ए॒व भ॑वति ब॒भ्रुर्भ॑वत्ये॒तद्वा अन्न॑स्य रू॒पꣳ समृ॑द्ध्यै सौ॒म्यं ब॒भ्रुमा\-ल॑भेत॒ यमलꣳ॑~(१८)

%2.1.3.4
रा॒ज्याय॒ सन्तꣳ॑ रा॒ज्यं नोप॒नमे᳚थ्सौ॒म्यं वै रा॒ज्यꣳ सोम॑मे॒व स्वेन॑ भाग॒धेये॒नोप॑ धावति॒ स ए॒वास्मै॑ रा॒ज्यं प्रय॑च्छ॒त्युपै॑नꣳ रा॒ज्यं न॑मति ब॒भ्रुर्भ॑वत्ये॒तद्वै सोम॑स्य रू॒पꣳ समृ॑द्ध्या॒ इन्द्रा॑य वृत्र॒तुरे॑ ल॒लामं॑ प्राशृ॒ङ्गमाल॑भेत ग॒तश्रीः᳚ प्रति॒ष्ठाका॑मः पा॒प्मान॑मे॒व वृ॒त्रं ती॒र्त्वा प्र॑ति॒ष्ठां ग॑च्छ॒तीन्द्रा॑याभिमाति॒घ्ने ल॒लामं॑ प्राशृ॒ङ्गमा-~(१९)

%2.1.3.5
\-ल॑भेत॒ यः पा॒प्मना॑ गृही॒तः स्यात्पा॒प्मा वा अ॒भि\-मा॑ति॒रिन्द्र॑\-मे॒वा\-भि॑मा\-ति॒हन॒ꣴ॒ स्वेन॑ भाग॒धेये॒नोप॑ धावति॒ स ए॒वास्मा᳚त्पा॒प्मान॑म॒भिमा॑तिं॒ प्रणु॑दत॒ इन्द्रा॑य व॒ज्रिणे॑ ल॒लामं॑ प्राशृ॒ङ्गमा\-ल॑भेत॒ यमलꣳ॑ रा॒ज्याय॒ सन्तꣳ॑ रा॒ज्यं नोप॒नमे॒दिन्द्र॑मे॒व व॒ज्रिण॒ꣴ॒ स्वेन॑ भाग॒धेये॒नोप॑ धावति॒ स ए॒वास्मै॒ वज्रं॒ प्रय॑च्छति॒ स ए॑नं॒ वज्रो॒ भूत्या॑ इन्ध॒ उपै॑नꣳ रा॒ज्यं न॑मति ल॒लामः॑ प्राशृ॒ङ्गो भ॑वत्ये॒तद्वै वज्र॑स्य रू॒पꣳ समृ॑द्ध्यै॥~(२०)

%2.1.4.0
{\anuvakamend[{स॒ङ्ग्रा॒मे तेनाल॑मभिमाति॒घ्ने ल॒लामं॑ प्राशृ॒ङ्गमैनं॒ पञ्च॑दश च।3।}]}

%2.1.4.1
अ॒सावा॑दि॒त्यो न व्य॑रोचत॒ तस्मै॑ दे॒वाः प्राय॑श्चित्ति\-मैच्छ॒न्तस्मा॑ ए॒तान्दश॑र्\mbox{}षभा॒मा\-ऽल॑भन्त॒ तयै॒वास्मि॒न्रुच॑मदधु॒र्यो ब्र॑ह्मवर्च॒सका॑मः॒ स्यात्तस्मा॑ ए॒तान्दश॑र्\mbox{}षभा॒मा\-ऽल॑भेता॒मुमे॒वा\-दि॒त्यꣴ स्वेन॑ भाग॒धेये॒नोप॑ धावति॒ स ए॒वास्मि॑न्ब्रह्म\-वर्च॒सं द॑धाति ब्रह्मवर्च॒स्ये॑व भ॑वति व॒सन्ता᳚ प्रा॒तस्त्रील्लँ॒लामा॒ना\-ल॑भेत ग्री॒ष्मे म॒ध्यन्दि॑ने॒~(२१)

%2.1.4.2
त्रीञ्छि॑तिपृ॒ष्ठाञ्छ॒रद्य॑परा॒ह्णे त्रीञ्छि॑ति॒वारा॒न्त्रीणि॒ वा आ॑दि॒त्यस्य॒ तेजाꣳ॑सि व॒सन्ता᳚ प्रा॒तर्ग्री॒ष्मे म॒ध्यन्दि॑ने श॒रद्य॑परा॒ह्णे याव॑न्त्ये॒व तेजाꣳ॑सि॒ तान्ये॒वाव॑\-रुन्धे॒ त्रय॑स्त्रय॒ आल॑भ्यन्ते\-ऽभिपू॒र्वमे॒वास्मि॒न्तेजो॑ दधाति संवथ्स॒रं प॒र्याल॑भ्यन्ते संवथ्स॒रो वै ब्र॑ह्मवर्च॒सस्य॑ प्रदा॒ता सं॑वथ्स॒र ए॒वास्मै᳚ ब्रह्मवर्च॒सं प्रय॑च्छति ब्रह्मवर्च॒स्ये॑व भ॑वति संवथ्स॒रस्य॑ प॒रस्ता᳚त्प्राजाप॒त्यं कद्रु॒-~(२२)

%2.1.4.3
माल॑भेत प्र॒जा\-प॑तिः॒ सर्वा॑ दे॒वता॑ दे॒वता᳚स्वे॒व प्रति॑तिष्ठति॒ यदि॑ बिभी॒याद्दु॒श्चर्मा॑ भविष्या॒मीति॑ सोमापौ॒ष्णꣴ श्या॒ममाल॑भेत सौ॒म्यो वै दे॒वत॑या॒ पुरु॑षः पौ॒ष्णाः प॒शवः॒ स्वयै॒वास्मै॑ दे॒वत॑या प॒शुभि॒स्त्वचं॑ करोति॒ न दु॒श्चर्मा॑ भवति दे॒वाश्च॒ वै य॒मश्चा॒स्मिँल्लो॒के᳚\-ऽस्पर्धन्त॒ स य॒मो दे॒वाना॑मिन्द्रि॒यं वी॒र्य॑मयुवत॒ तद्य॒मस्य॑~(२३)

%2.1.4.4
यम॒त्वं ते दे॒वा अ॑मन्यन्त य॒मो वा इ॒दम॑भू॒द्यद्व॒यꣴ स्म इति॒ ते प्र॒जा\-प॑ति॒मुपा॑धाव॒न्थ्स ए॒तौ प्र॒जा\-प॑तिरा॒त्मन॑ उक्षव॒शौ निर॑मिमीत॒ ते दे॒वा वै᳚ष्णावरु॒णीं व॒शामा\-ऽल॑भन्तै॒न्द्रमु॒क्षाण॒न्तं वरु॑णेनै॒व ग्रा॑हयि॒त्वा विष्णु॑ना य॒ज्ञेन॒ प्राणु॑दन्तै॒न्द्रेणै॒वास्ये᳚न्द्रि॒यम॑वृञ्जत॒ यो भ्रातृ॑व्यवा॒न्थ्स्याथ्स स्पर्ध॑मानो वैष्णावरु॒णीं~(२४)

%2.1.4.5
व॒शामाल॑भेतै॒न्द्रमु॒क्षाणं॒ वरु॑णेनै॒व भ्रातृ॑व्यं ग्राहयि॒त्वा विष्णु॑ना य॒ज्ञेन॒ प्रणु॑दत ऐ॒न्द्रेणै॒वास्ये᳚न्द्रि॒यं वृ॑ङ्क्ते॒ भव॑त्या॒त्मना॒ परा᳚स्य॒ भ्रातृ॑व्यो भव॒तीन्द्रो॑ वृ॒त्रम॑ह॒न्तं वृ॒त्रो ह॒तः षो॑ड॒शभि॑र्भो॒गैर॑सिना॒त्तस्य॑ वृ॒त्रस्य॑ शीर्\mbox{}ष॒तो गाव॒ उदा॑य॒न्ता वै॑दे॒ह्यो॑\-ऽभव॒न्तासा॑मृष॒भो ज॒घने\-ऽनूदै॒त्तमिन्द्रो॑-~(२५)

%2.1.4.6
ऽचाय॒थ्सो॑\-ऽमन्यत॒ यो वा इ॒ममा॒लभे॑त॒ मुच्ये॑ता॒स्मात्पा॒\-प्मन॒ इति॒ स आ᳚ग्ने॒यं कृ॒ष्णग्री॑व॒माल॑भतै॒न्द्रमृ॑ष॒भं तस्या॒ग्निरे॒व स्वेन॑ भाग॒धेये॒नोप॑सृतः षोडश॒धा वृ॒त्रस्य॑ भो॒गानप्य॑दह\-दै॒न्द्रेणे᳚न्द्रि॒य\-मा॒त्मन्न॑धत्त॒ यः पा॒प्मना॑ गृही॒तः स्याथ्स आ᳚ग्ने॒यं कृ॒ष्णग्री॑व॒माल॑भेतै॒न्द्रमृ॑ष॒भम॒ग्निरे॒वास्य॒ स्वेन॑ भाग॒धेये॒नोप॑सृतः~(२६)

%2.1.4.7
पा॒प्मान॒मपि॑ दहत्यै॒न्द्रेणे᳚न्द्रि॒यमा॒त्मन्ध॑त्ते॒ मुच्य॑ते पा॒प्मनो॒ भव॑त्ये॒व द्या॑वापृथि॒व्यां᳚ धे॒नुमा\-ल॑भेत॒ ज्योग॑परुद्धो॒\-ऽनयो॒र्॒\mbox{}हि वा ए॒षो\-ऽप्र॑तिष्ठि॒तो\-ऽथै॒ष ज्योगप॑रुद्धो॒ द्यावा॑पृथि॒वी ए॒व स्वेन॑ भाग॒धेये॒नोप॑ धावति॒ ते ए॒वैनं॑ प्रति॒ष्ठां ग॑मयतः॒ प्रत्ये॒व ति॑ष्ठति पर्या॒रिणी॑ भवति पर्या॒रीव॒ ह्ये॑तस्य॑ रा॒ष्ट्रं यो ज्योग॑परुद्धः॒ समृ॑द्ध्यै वाय॒व्यं॑~(२७)

%2.1.4.8
व॒थ्समा ल॑भेत वा॒युर्वा अ॒नयो᳚र्व॒थ्स इ॒मे वा ए॒तस्मै॑ लो॒का अप॑शुष्का॒ विडप॑शु॒ष्का\-ऽथै॒ष ज्योगप॑रुद्धो वा॒युमे॒व स्वेन॑ भाग॒धेये॒नोप॑ धावति॒ स ए॒वास्मा॑ इ॒माँल्लो॒कान् विशं॒ प्रदा॑पयति॒ प्रास्मा॑ इ॒मे लो॒काः स्नु॑वन्ति भुञ्ज॒त्ये॑नं॒ विडुप॑तिष्ठते॥~(२८)

%2.1.5.0
{\anuvakamend[{म॒ध्यन्दि॑ने॒ कद्रुं॑ य॒मस्य॒ स्पर्ध॑मानो वैष्णावरु॒णीन्तमिन्द्रो᳚\-ऽस्य॒ स्वेन॑ भाग॒धेये॒नोप॑सृतो वाय॒व्यं॑ द्विच॑त्वारिꣳशच्च।~(४)।}]}

%2.1.5.1
इन्द्रो॑ व॒लस्य॒ बिल॒मपौ᳚र्णो॒थ्स य उ॑त्त॒मः प॒शुरासी॒त्तं पृ॒ष्ठं प्रति॑ स॒ङ्गृह्योद॑क्खिद॒त्तꣳ स॒हस्रं॑ प॒शवो\-ऽनूदा॑य॒न्थ्स उ॑न्न॒तो॑\-ऽभव॒द्यः प॒शुका॑मः॒ स्याथ्स ए॒तमै॒न्द्रमु॑न्न॒तमाल॑भे॒तेन्द्र॑मे॒व स्वेन॑ भाग॒धेये॒नोप॑ धावति॒ स ए॒वास्मै॑ प॒शून्प्रय॑च्छति पशु॒माने॒व भ॑वत्युन्न॒तो~(२९)

%2.1.5.2
भ॑वति साह॒स्री वा ए॒षा ल॒क्ष्मी यदु॑न्न॒तो ल॒क्ष्मियै॒व प॒शूनव॑\-रुन्धे य॒दा स॒हस्रं॑ प॒शून्प्रा᳚प्नु॒यादथ॑ वैष्ण॒वं वा॑म॒नमा ल॑भेतै॒तस्मि॒न्वै तथ्स॒हस्र॒मद्ध्य॑तिष्ठ॒त्तस्मा॑दे॒ष वा॑म॒नः समी॑षितः प॒शुभ्य॑ ए॒व प्रजा॑तेभ्यः प्रति॒ष्ठां द॑धाति॒ को॑\-ऽर्\mbox{}हति स॒हस्रं॑ प॒शून्प्राप्तु॒मित्या॑हुरहोरा॒त्राण्ये॒व स॒हस्रꣳ॑ स॒म्पाद्याल॑भेत प॒शवो॒~(३०)

%2.1.5.3
वा अ॑होरा॒त्राणि॑ प॒शूने॒व प्रजा॑तान्प्रति॒ष्ठां ग॑मय॒त्योष॑धीभ्यो वे॒हत॒माल॑भेत प्र॒जाका॑म॒ ओष॑धयो॒ वा ए॒तं प्र॒जायै॒ परि॑बाधन्ते॒ यो\-ऽलं॑ प्र॒जायै॒ सन्प्र॒जां न वि॒न्दत॒ ओष॑धयः॒ खलु॒ वा ए॒तस्यै॒ सूतु॒मपि॑ घ्नन्ति॒ या वे॒हद्भव॒त्योष॑धीरे॒व स्वेन॑ भाग॒धेये॒नोप॑ धावति॒ ता ए॒वास्मै॒ स्वाद्योनेः᳚ प्र॒जां प्रज॑नयन्ति वि॒न्दते᳚~(३१)

%2.1.5.4
प्र॒जामापो॒ वा ओष॑ध॒यो\-ऽस॒त्पुरु॑ष॒ आप॑ ए॒वास्मा॒ अस॑तः॒ सद्द॑दति॒ तस्मा॑दाहु॒र्यश्चै॒वं वेद॒ यश्च॒ नाप॒स्त्वावास॑त॒ सद्द॑द॒तीत्यै॒न्द्रीꣳ सू॒तव॑शा॒मा\-ल॑भेत॒ भूति॑का॒मो\-ऽजा॑तो॒ वा ए॒ष यो\-ऽलं॒ भूत्यै॒ सन्भूतिं॒ न प्रा॒प्नोतीन्द्रं॒ खलु॒ वा ए॒षा सू॒त्वा व॒शा\-ऽभ॑व॒-~(३२)

%2.1.5.5
दिन्द्र॑मे॒व स्वेन॑ भाग॒धेये॒नोप॑ धावति॒ स ए॒वैनं॒ भूतिं॑ गमयति॒ भव॑त्ये॒व यꣳ सू॒त्वा व॒शा स्यात्तमै॒न्द्रमे॒वाल॑भेतै॒तद्वाव तदि॑न्द्रि॒यꣳ सा॒क्षादे॒वेन्द्रि॒यमव॑रुन्ध ऐन्द्रा॒ग्नं पु॑नरुथ्सृ॒ष्टमा\-ल॑भेत॒ य आ तृ॒तीया॒त्पुरु॑षा॒थ्सोमं॒ न पिबे॒द्विच्छि॑न्नो॒ वा ए॒तस्य॑ सोमपी॒थो यो ब्रा᳚ह्म॒णः सन्ना~(३३)

%2.1.5.6
तृ॒तीया॒त्पुरु॑षा॒थ्सोमं॒ न पिब॑तीन्द्रा॒ग्नी ए॒व स्वेन॑ भाग॒धेये॒नोप॑ धावति॒ तावे॒वास्मै॑ सोमपी॒थं प्रय॑च्छत॒ उपै॑नꣳ सोमपी॒थो न॑मति॒ यदै॒न्द्रो भव॑तीन्द्रि॒यं वै सो॑मपी॒थ इ॑न्द्रि॒यमे॒व सो॑मपी॒थमव॑\-रुन्धे॒ यदा᳚ग्ने॒यो भव॑त्याग्ने॒यो वै ब्रा᳚ह्म॒णः स्वामे॒व दे॒वता॒मनु॒ सन्त॑नोति पुनरुथ्सृ॒ष्टो भ॑वति पुनरुथ्सृ॒ष्ट इ॑व॒ ह्ये॑तस्य॑~(३४)

%2.1.5.7
सोमपी॒थः समृ॑द्ध्यै ब्राह्मणस्प॒त्यं तू॑प॒रमाल॑भेताभि॒\-चर॒न्ब्रह्म॑ण॒स्पति॑मे॒व स्वेन॑ भाग॒धेये॒नोप॑ धावति॒ तस्मा॑ ए॒वैन॒मा वृ॑श्चति ता॒जगार्ति॒मार्च्छ॑ति तूप॒रो भ॑वति क्षु॒रप॑वि॒र्वा ए॒षा ल॒क्ष्मी यत्तू॑प॒रः समृ॑द्ध्यै॒ स्फ्यो यूपो॑ भवति॒ वज्रो॒ वै स्फ्यो वज्र॑मे॒वास्मै॒ प्रह॑रति शर॒मयं॑ ब॒र्॒\mbox{}हिः शृ॒णात्ये॒वैनं॒ वैभी॑दक इ॒ध्मो भि॒नत्त्ये॒वैनम्᳚॥~(३५)

%2.1.6.0
{\anuvakamend[{भ॒व॒त्यु॒न्न॒तः प॒शवो॑ जनयन्ति वि॒न्दते॑\-ऽभव॒थ्सन्नैतस्ये॒द्ध्मस्त्रीणि॑ च}]}%॥~(५)॥

%2.1.6.1
बा॒र्॒\mbox{}ह॒स्प॒त्यꣳ शि॑तिपृ॒ष्ठमा\-ल॑भेत॒ ग्राम॑कामो॒ यः का॒मये॑त पृ॒ष्ठꣳ स॑मा॒नानाꣴ॑ स्या॒मिति॒ बृह॒स्पति॑मे॒व स्वेन॑ भाग॒धेये॒नोप॑ धावति॒ स ए॒वैनं॑ पृ॒ष्ठꣳ स॑मा॒नानां᳚ करोति ग्रा॒म्ये॑व भ॑वति शितिपृ॒ष्ठो भ॑वति बार्\mbox{}हस्प॒त्यो ह्ये॑ष दे॒वत॑या॒ समृ॑द्ध्यै पौ॒ष्णꣴ श्या॒ममाल॑भे॒तान्न॑का॒मो\-ऽन्नं॒ वै पू॒षा पू॒षण॑मे॒व स्वेन॑ भाग॒धेये॒नोप॑ धावति॒ स ए॒वास्मा॒~(३६)

%2.1.6.2
अन्नं॒ प्रय॑च्छत्यन्ना॒द ए॒व भ॑वति श्या॒मो भ॑वत्ये॒तद्वा अन्न॑स्य रू॒पꣳ समृ॑द्ध्यै मारु॒तं पृश्ञि॒माल॑भे॒तान्न॑का॒मो\-ऽन्नं॒ वै म॒रुतो॑ म॒रुत॑ ए॒व स्वेन॑ भाग॒धेये॒नोप॑ धावति॒ त ए॒वास्मा॒ अन्नं॒ प्रय॑च्छन्त्यन्ना॒द ए॒व भ॑वति॒ पृश्ञि॑र्भवत्ये॒तद्वा अन्न॑स्य रू॒पꣳ समृ॑द्ध्या ऐ॒न्द्रम॑रु॒णमाल॑भेतेन्द्रि॒यका॑म॒ इन्द्र॑मे॒व~(३७)

%2.1.6.3
स्वेन॑ भाग॒धेये॒नोप॑ धावति॒ स ए॒वास्मि॑न्निन्द्रि॒यं द॑धातीन्द्रिया॒\-व्ये॑व भ॑वत्यरु॒णो भ्रूमा᳚न्भवत्ये॒तद्वा इन्द्र॑स्य रू॒पꣳ समृ॑द्ध्यै सावि॒त्र\-मु॑प\-द्ध्व॒स्तमाल॑भेत स॒निका॑मः सवि॒ता वै प्र॑स॒वाना॑मीशे सवि॒तार॑मे॒व स्वेन॑ भाग॒धेये॒नोप॑ धावति॒ स ए॒वास्मै॑ स॒निं प्रसु॑वति॒ दान॑कामा अस्मै प्र॒जा भ॑वन्त्युपद्ध्व॒स्तो भ॑वति सावि॒त्रो ह्ये॑ष~(३८)

%2.1.6.4
दे॒वत॑या॒ समृ॑द्ध्यै वैश्वदे॒वं ब॑हुरू॒पमाल॑भे॒तान्न॑कामो वैश्वदे॒वं वा अन्नं॒ विश्वा॑ने॒व दे॒वान्थ्स्वेन॑ भाग॒धेये॒नोप॑ धावति॒ त ए॒वास्मा॒ अन्नं॒ प्रय॑च्छन्त्यन्ना॒द ए॒व भ॑वति बहुरू॒पो भ॑वति बहुरू॒पꣴ ह्यन्न॒ꣳ॒ समृ॑द्ध्यै वैश्वदे॒वं ब॑हुरू॒पमा\-ल॑भेत॒ ग्राम॑कामो वैश्वदे॒वा वै स॑जा॒ता विश्वा॑ने॒व दे॒वान्थ्स्वेन॑ भाग॒धेये॒नोप॑ धावति॒ त ए॒वास्मै॑~(३९)

%2.1.6.5
सजा॒तान्प्रय॑च्छन्ति ग्रा॒म्ये॑व भ॑वति बहुरू॒पो भ॑वति बहुदेव॒त्यो᳚(१॒) ह्ये॑ष समृ॑द्ध्यै प्राजाप॒त्यं तू॑प॒रमा\-ल॑भेत॒ यस्याना᳚ज्ञातमिव॒ ज्योगा॒मये᳚त्प्राजाप॒त्यो वै पुरु॑षः प्र॒जा\-प॑तिः॒ खलु॒ वै तस्य॑ वेद॒ यस्याना᳚ज्ञातमिव॒ ज्योगा॒मय॑ति प्र॒जा\-प॑तिमे॒व स्वेन॑ भाग॒धेये॒नोप॑ धावति॒ स ए॒वैनं॒ तस्मा॒थ्स्रामा᳚न्मुञ्चति तूप॒रो भ॑वति प्राजाप॒त्यो ह्ये॑ष दे॒वत॑या॒ समृ॑द्ध्यै॥~(४०)

%2.1.7.0
{\anuvakamend[{अ॒स्मा॒ इन्द्र॑मे॒वैष स॑जा॒ता विश्वा॑ने॒व दे॒वान्थ्स्वेन॑ भाग॒धेये॒नोप॑ धावति॒ त ए॒वास्मै᳚ प्राजाप॒त्यो हि त्रीणि॑ च}]}%॥~(६)॥

%2.1.7.1
व॒ष॒ट्का॒रो वै गा॑यत्रि॒यै शिरो᳚\-ऽच्छिन॒त्तस्यै॒ रसः॒ परा॑\-ऽपत॒त्तं बृह॒स्पति॒रुपा॑गृह्णा॒थ्सा शि॑तिपृ॒ष्ठा व॒शा\-ऽभ॑व॒द्यो द्वि॒तीयः॑ प॒राप॑त॒त्तं मि॒त्रावरु॑णा॒वुपा॑गृह्णीता॒ꣳ॒ सा द्वि॑रू॒पा व॒शा\-ऽभ॑व॒द्यस्तृ॒तीयः॑ प॒राप॑त॒त्तं विश्वे॑ दे॒वा उपा॑गृह्ण॒न्थ्सा ब॑हुरू॒पा व॒शा\-ऽभ॑व॒द्यश्च॑तु॒र्थः प॒राप॑त॒थ्स पृ॑थि॒वीं प्रावि॑श॒त्तं बृह॒स्पति॑र॒भ्य॑-~(४१)

%2.1.7.2
गृह्णा॒दस्त्वे॒वायं भोगा॒येति॒ स उ॑क्षव॒शः सम॑भव॒द्यल्लोहि॑तं प॒राप॑त॒त्तद्रु॒द्र उपा॑गृह्णा॒थ्सा रौ॒द्री रोहि॑णी व॒शा\-ऽभ॑व\-द्बार्\mbox{}हस्प॒त्याꣳ शि॑तिपृ॒ष्ठामाल॑भेत ब्रह्मवर्च॒सका॑मो॒ बृह॒स्पति॑मे॒व स्वेन॑ भाग॒धेये॒नोप॑ धावति॒ स ए॒वास्मि॑न्ब्रह्म\-वर्च॒सं द॑धाति ब्रह्मवर्च॒स्ये॑व भ॑वति॒ छन्द॑सां॒ वा ए॒ष रसो॒ यद्व॒शा रस॑ इव॒ खलु॒~(४२)

%2.1.7.3
वै ब्र॑ह्मवर्च॒सं छन्द॑सामे॒व रसे॑न॒ रसं॑ ब्रह्मवर्च॒समव॑\-रुन्धे मैत्रावरु॒णीं द्वि॑रू॒पामा\-ल॑भेत॒ वृष्टि॑कामो मै॒त्रं वा अह॑र्वारु॒णी रात्रि॑रहोरा॒त्राभ्यां॒ खलु॒ वै प॒र्जन्यो॑ वर्\mbox{}षति मि॒त्रावरु॑णावे॒व स्वेन॑ भाग॒धेये॒नोप॑ धावति॒ तावे॒वास्मा॑ अहोरा॒त्रा\-भ्यां᳚ प॒र्जन्यं॑ वर्\mbox{}षयत॒श्छन्द॑सां॒ वा ए॒ष रसो॒ यद्व॒शा रस॑ इव॒ खलु॒ वै वृष्टि॒श्छन्द॑सामे॒व रसे॑न॒~(४३)

%2.1.7.4
रसं॒ वृष्टि॒मव॑\-रुन्धे मैत्रावरु॒णीं द्वि॑रू॒पामाल॑भेत प्र॒जाका॑मो मै॒त्रं वा अह॑र्वारु॒णी रात्रि॑रहोरा॒त्राभ्यां॒ खलु॒ वै प्र॒जाः प्रजा॑यन्ते मि॒त्रावरु॑णावे॒व स्वेन॑ भाग॒धेये॒नोप॑ धावति॒ तावे॒वास्मा॑ अहोरा॒त्रा\-भ्यां᳚ प्र॒जां प्रज॑नयत॒श्छन्द॑सां॒ वा ए॒ष रसो॒ यद्व॒शा रस॑ इव॒ खलु॒ वै प्र॒जा छन्द॑सामे॒व रसे॑न॒ रसं॑ प्र॒जामव॑~(४४)

%2.1.7.5
रुन्धे वैश्वदे॒वीं ब॑हुरू॒पामाल॑भे॒तान्न॑कामो वैश्वदे॒वं वा अन्नं॒ विश्वा॑ने॒व दे॒वान्थ्स्वेन॑ भाग॒धेये॒नोप॑ धावति॒ त ए॒वास्मा॒ अन्नं॒ प्रय॑च्छन्त्यन्ना॒द ए॒व भ॑वति॒ छन्द॑सां॒ वा ए॒ष रसो॒ यद्व॒शा रस॑ इव॒ खलु॒ वा अन्नं॒ छन्द॑सामे॒व रसे॑न॒ रस॒मन्न॒मव॑\-रुन्धे वैश्वदे॒वीं ब॑हुरू॒पामा\-ल॑भेत॒ ग्राम॑कामो वैश्वदे॒वा वै~(४५)

%2.1.7.6
स॑जा॒ता विश्वा॑ने॒व दे॒वान्थ्स्वेन॑ भाग॒धेये॒नोप॑ धावति॒ त ए॒वास्मै॑ सजा॒तान्प्रय॑च्छन्ति ग्रा॒म्ये॑व भ॑वति॒ छन्द॑सां॒ वा ए॒ष रसो॒ यद्व॒शा रस॑ इव॒ खलु॒ वै स॑जा॒ताश्छन्द॑सामे॒व रसे॑न॒ रसꣳ॑ सजा॒तानव॑\-रुन्धे बार्\mbox{}हस्प॒त्यमु॑क्षव॒शमाल॑भेत ब्रह्मवर्च॒सका॑मो॒ बृह॒स्पति॑मे॒व स्वेन॑ भाग॒धेये॒नोप॑ धावति॒ स ए॒वास्मि॑न्ब्रह्मवर्च॒सं~(४६)

%2.1.7.7
द॑धाति ब्रह्मवर्च॒स्ये॑व भ॑वति॒ वशं॒ वा ए॒ष च॑रति॒ यदु॒क्षा वश॑ इव॒ खलु॒ वै ब्र॑ह्मवर्च॒सं वशे॑नै॒व वशं॑ ब्रह्मवर्च॒समव॑\-रुन्धे रौ॒द्रीꣳ रोहि॑णी॒माल॑भेताभि॒चर॑न्रु॒द्रमे॒व स्वेन॑ भाग॒धेये॒नोप॑ धावति॒ तस्मा॑ ए॒वैन॒मावृ॑श्चति ता॒जगार्ति॒मार्च्छ॑ति॒ रोहि॑णी भवति रौ॒द्री ह्ये॑षा दे॒वत॑या॒ समृ॑द्ध्यै॒ स्फ्यो यूपो॑ भवति॒ वज्रो॒ वै स्फ्यो वज्र॑मे॒वास्मै॒ प्रह॑रति शर॒मयं॑ ब॒र्॒\mbox{}हिः शृ॒णात्ये॒वैनं॒ वैभी॑दक इ॒ध्मो भि॒नत्त्ये॒वैनम्᳚॥~(४७)

%2.1.8.0
{\anuvakamend[{अ॒भि खलु॒ वृष्टि॒श्छन्द॑सामे॒व रसे॑न प्र॒जामव॑ वैश्वदे॒वा वै ब्र॑ह्मवर्च॒सं यूप॒ एका॒न्नविꣳ॑श॒तिश्च॑।~(७)।}]}

%2.1.8.1
अ॒सावा॑दि॒त्यो न व्य॑रोचत॒ तस्मै॑ दे॒वाः प्राय॑श्चित्ति\-मैच्छ॒न्तस्मा॑ ए॒ताꣳ सौ॒रीꣴ श्वे॒तां व॒शामा\-ऽल॑भन्त॒ तयै॒वास्मि॒न्रुच॑मदधु॒र्यो ब्र॑ह्मवर्च॒सका॑मः॒ स्यात्तस्मा॑ ए॒ताꣳ सौ॒रीꣴ श्वे॒तां व॒शामाल॑भेता॒मुमे॒\-वा\-ऽऽ\-दि॒त्यꣴ स्वेन॑ भाग॒धेये॒नोप॑ धावति॒ स ए॒वास्मि॑न्ब्रह्मवर्च॒सं द॑धाति ब्रह्मवर्च॒स्ये॑व भ॑वति बै॒ल्॒\mbox{}वो यूपो॑ भवत्य॒सौ~(४८)

%2.1.8.2
वा आ॑दि॒त्यो यतो\-ऽजा॑यत॒ ततो॑ बि॒ल्व॑ उद॑तिष्ठ॒थ्सयो᳚न्ये॒व ब्र॑ह्मवर्च॒समव॑\-रुन्धे ब्राह्मणस्प॒त्यां ब॑भ्रुक॒र्णीमा ल॑भेताभि॒\-चर॑न्वारु॒णं दश॑\-कपालं पु॒रस्ता॒न्निर्व॑पे॒द्वरु॑णेनै॒व भ्रातृ॑व्यं ग्राहयि॒त्वा ब्रह्म॑णा स्तृणुते बभ्रुक॒र्णी भ॑वत्ये॒तद्वै ब्रह्म॑णो रू॒पꣳ समृ॑द्ध्यै॒ स्फ्यो यूपो॑ भवति॒ वज्रो॒ वै स्फ्यो वज्र॑मे॒वास्मै॒ प्रह॑रति शर॒मयं॑ ब॒र्॒\mbox{}हिः शृ॒णा-~(४९)

%2.1.8.3
त्ये॒वैनं॒ वैभी॑दक इ॒ध्मो भि॒नत्त्ये॒वैनं॑ वैष्ण॒वं वा॑म॒नमा\-ल॑भेत॒ यं य॒ज्ञो नोप॒नमे॒द्विष्णु॒र्वै य॒ज्ञो विष्णु॑मे॒व स्वेन॑ भाग॒धेये॒नोप॑ धावति॒ स ए॒वास्मै॑ य॒ज्ञं प्रय॑च्छ॒त्युपै॑नं य॒ज्ञो न॑मति वाम॒नो भ॑वति वैष्ण॒वो ह्ये॑ष दे॒वत॑या॒ समृ॑द्ध्यै त्वा॒ष्ट्रं व॑ड॒बमाल॑भेत प॒शुका॑म॒स्त्वष्टा॒ वै प॑शू॒नां मि॑थु॒नानां᳚~(५०)

%2.1.8.4
प्रजनयि॒ता त्वष्टा॑रमे॒व स्वेन॑ भाग॒धेये॒नोप॑ धावति॒ स ए॒वास्मै॑ प॒शून्मि॑थु॒नान्प्रज॑नयति प्र॒जा हि वा ए॒तस्मि॑न्प॒शवः॒ प्रवि॑ष्टा॒ अथै॒ष पुमा॒न्थ्सन्व॑ड॒बः सा॒क्षादे॒व प्र॒जां प॒शूनव॑\-रुन्धे मै॒त्रꣴ श्वे॒तमाल॑भेत सङ्ग्रा॒मे सं य॑त्ते सम॒यका॑मो मि॒त्रमे॒व स्वेन॑ भाग॒धेये॒नोप॑ धावति॒ स ए॒वैनं॑ मि॒त्रेण॒ सन्न॑यति~(५१)

%2.1.8.5
विशा॒लो भ॑वति॒ व्यव॑साययत्ये॒वैनं॑ प्राजाप॒त्यं कृ॒ष्णमा\-ल॑भेत॒ वृष्टि॑कामः प्र॒जा\-प॑ति॒र्वै वृष्ट्या॑ ईशे प्र॒जा\-प॑तिमे॒व स्वेन॑ भाग॒धेये॒नोप॑ धावति॒ स ए॒वास्मै॑ प॒र्जन्यं॑ वर्\mbox{}षयति कृ॒ष्णो भ॑वत्ये॒तद्वै वृष्ट्यै॑ रू॒पꣳ रू॒पेणै॒व वृष्टि॒मव॑\-रुन्धे श॒बलो॑ भवति वि॒द्युत॑मे॒वास्मै॑ जनयि॒त्वा व॑र्\mbox{}षयत्यवाशृ॒ङ्गो भ॑वति॒ वृष्टि॑मे॒वास्मै॒ निय॑च्छति॥~(५२)

%2.1.9.0
{\anuvakamend[शृ॒णाति॑ मिथु॒नाना᳚न्नयति यच्छति॥]}

%2.1.9.1
वरु॑णꣳ सुषुवा॒णम॒न्नाद्य॒न्नोपा॑नम॒थ्स ए॒तां वा॑रु॒णीं कृ॒ष्णां व॒शाम॑पश्य॒त्ताꣴ स्वायै॑ दे॒वता॑या॒ आ\-ऽल॑भत॒ ततो॒ वै तम॒न्नाद्य॒मुपा॑नम॒द्यमल॑म॒न्नाद्या॑य॒ सन्त॑म॒न्नाद्य॒न्नोप॒नमे॒थ्स ए॒तां वा॑रु॒णीं कृ॒ष्णां व॒शामा\-ल॑भेत॒ वरु॑णमे॒व स्वेन॑ भाग॒धेये॒नोप॑ धावति॒ स ए॒वास्मा॒ अन्नं॒ प्रय॑च्छत्यन्ना॒द~(५३)

%2.1.9.2
ए॒व भ॑वति कृ॒ष्णा भ॑वति वारु॒णी ह्ये॑षा दे॒वत॑या॒ समृ॑द्ध्यै मै॒त्रꣴ श्वे॒तमाल॑भेत वारु॒णं कृ॒ष्णम॒पां चौष॑धीनां च स॒न्धावन्न॑कामो मै॒त्रीर्वा ओष॑धयो वारु॒णीरापो॒\-ऽपां च॒ खलु॒ वा ओष॑धीनां च॒ रस॒मुप॑जीवामो मि॒त्रावरु॑णावे॒व स्वेन॑ भाग॒धेये॒नोप॑ धावति॒ तावे॒वास्मा॒ अन्नं॒ प्रय॑च्छतो\-ऽन्ना॒द ए॒व भ॑व-~(५४)

%2.1.9.3
त्य॒पां चौष॑धीनां च स॒न्धावाल॑भत उ॒भय॒स्याव॑रुद्ध्यै॒ विशा॑खो॒ यूपो॑ भवति॒ द्वे ह्ये॑ते दे॒वते॒ समृ॑द्ध्यै मै॒त्रꣴ श्वे॒तमा ल॑भेत वारु॒णं कृ॒ष्णं ज्योगा॑मयावी॒ यन्मै॒त्रो भव॑ति मि॒त्रेणै॒वास्मै॒ वरु॑णꣳ शमयति॒ यद्वा॑रु॒णः सा॒क्षादे॒वैनं॑ वरुण\-पा॒शान्मु॑ञ्चत्यु॒त यदी॒तासु॒र्भव॑ति॒ जीव॑त्ये॒व दे॒वा वै पुष्टिं॒ नावि॑न्द॒-~(५५)

%2.1.9.4
न्तां मि॑थु॒ने॑\-ऽपश्य॒न्तस्यां॒ न सम॑राधय॒न्ताव॒श्विना॑\-वब्रूतामा॒वयो॒र्वा ए॒षा मैतस्यां᳚ वदद्ध्व॒मिति॒ सा\-ऽश्विनो॑रे॒वाभ॑व॒द्यः पुष्टि॑कामः॒ स्याथ्स ए॒तामा᳚श्वि॒नीं य॒मीं व॒शामाल॑भेता॒श्विना॑वे॒व स्वेन॑ भाग॒धेये॒नोप॑ धावति॒ तावे॒वास्मि॒न्पुष्टिं॑ धत्तः॒ पुष्य॑ति प्र॒जया॑ प॒शुभिः॑॥~(५६)

%2.1.10.0
{\anuvakamend[{अ॒न्ना॒दो᳚\-ऽन्ना॒द ए॒व भ॑वत्यविन्द॒न्पञ्च॑चत्वारिꣳशच्च}]}

%2.1.10.1
आ॒श्वि॒नं धू॒म्रल॑लाम॒मा\-ल॑भेत॒ यो दुर्ब्रा᳚ह्मणः॒ सोमं॒ पिपा॑सेद॒श्विनौ॒ वै दे॒वाना॒मसो॑मपावास्तां॒ तौ प॒श्चा सो॑मपी॒थं प्राप्नु॑ताम॒श्विना॑वे॒तस्य॑ दे॒वता॒ यो दुर्ब्रा᳚ह्मणः॒ सोमं॒ पिपा॑सत्य॒श्विना॑वे॒व स्वेन॑ भाग॒धेये॒नोप॑ धावति॒ तावे॒वास्मै॑ सोमपी॒थं प्रय॑च्छत॒ उपै॑नꣳ सोमपी॒थो न॑मति॒ यद्धू॒म्रो भव॑ति धूम्रि॒माण॑मे॒वास्मा॒दप॑हन्ति ल॒लामो॑~(५७)

%2.1.10.2
भवति मुख॒त ए॒वास्मि॒न्तेजो॑ दधाति वाय॒व्यं॑ गोमृ॒गमा\-ल॑भेत॒ यमज॑घ्निवाꣳसमभि॒शꣳसे॑यु॒रपू॑ता॒ वा ए॒तं वागृ॑च्छति॒ यमज॑घ्निवाꣳसमभि॒शꣳस॑न्ति॒ नैष ग्रा॒म्यः प॒शुर्नाऽर॒ण्यो यद्गो॑मृ॒गो नेवै॒ष ग्रामे॒ नार॑ण्ये॒ यमज॑घ्निवाꣳसमभि॒शꣳस॑न्ति वा॒युर्वै दे॒वानां᳚ प॒वित्रं॑ वा॒युमे॒व स्वेन॑ भाग॒धेये॒नोप॑ धावति॒ स ए॒वै-~(५८)

%2.1.10.3
नं॑ पवयति॒ परा॑ची॒ वा ए॒तस्मै᳚ व्यु॒च्छन्ती॒ व्यु॑च्छति॒ तमः॑ पा॒प्मानं॒ प्रवि॑शति॒ यस्या᳚श्वि॒ने श॒स्यमा॑ने॒ सूर्यो॒ नाविर्भव॑ति सौ॒र्यं ब॑हुरू॒पमाल॑भेता॒मुमे॒वाऽऽदि॒त्यꣴ स्वेन॑ भाग॒धेये॒नोप॑ धावति॒ स ए॒वास्मा॒त्तमः॑ पा॒प्मान॒मप॑हन्ति प्र॒तीच्य॑स्मै व्यु॒च्छन्ती॒ व्यु॑च्छ॒त्यप॒ तमः॑ पा॒प्मानꣳ॑ हते॥~(५९)

%2.1.11.0
{\anuvakamend[{ल॒लामः॒ स ए॒व षट्च॑त्वारिꣳशच्च}]}%॥10॥

%2.1.11.1
इन्द्रं॑ वो वि॒श्वत॒स्परीन्द्रं॒ नरो॒ मरु॑तो॒ यद्ध॑ वो दि॒वो या वः॒ शर्म॑। भरे॒ष्विन्द्रꣳ॑ सु॒हवꣳ॑ हवामहे\-ऽꣳहो॒मुचꣳ॑ सु॒कृतं॒ दैव्यं॒ जनम्᳚। अ॒ग्निं मि॒त्रं वरु॑णꣳ सा॒तये॒ भगं॒ द्यावा॑पृथि॒वी म॒रुतः॑ स्व॒स्तये᳚। म॒मत्तु॑ नः॒ परि॑ज्मा वस॒र्॒\mbox{}हा म॒मत्तु॒ वातो॑ अ॒पां वृष॑ण्वान्। शि॒शी॒तमि॑न्द्रापर्वता यु॒वन्न॒स्तन्नो॒ विश्वे॑ वरिवस्यन्तु दे॒वाः। प्रि॒या वो॒ नाम॑~(६०)

%2.1.11.2
हुवे तु॒राणा᳚म्। आयत्तृ॒पन्म॑रुतो वावशा॒नाः। श्रि॒यसे॒ कं भा॒नुभिः॒ सम्मि॑मिक्षिरे॒ ते र॒श्मिभि॒स्त ऋक्व॑भिः सुखा॒दयः॑। ते वाशी॑मन्त इ॒ष्मिणो॒ अभी॑रवो वि॒द्रे प्रि॒यस्य॒ मारु॑तस्य॒ धाम्नः॑। अ॒ग्निः प्र॑थ॒मो वसु॑भिर्नो अव्या॒थ्सोमो॑ रु॒द्रेभि॑र॒भिर॑क्षतु॒ त्मना᳚। इन्द्रो॑ म॒रुद्भि॑र्\mbox{}ऋतु॒धा कृ॑णोत्वादि॒त्यैर्नो॒ वरु॑णः॒ सꣳशि॑शातु। सन्नो॑ दे॒वो वसु॑भिर॒ग्निः सꣳ~(६१)

%2.1.11.3
सोम॑स्त॒नूभी॑ रु॒द्रिया॑भिः। समिन्द्रो॑ म॒रुद्भि॑र्य॒ज्ञियैः॒ समा॑दि॒त्यैर्नो॒ वरु॑णो अजिज्ञिपत्। यथा॑\-ऽ\-ऽदि॒त्या वसु॑भिः सम्बभू॒वुर्म॒रुद्भी॑ रु॒द्राः स॒मजा॑नता॒भि। ए॒वा त्रि॑णाम॒न्नहृ॑णीयमाना॒ विश्वे॑ दे॒वाः सम॑नसो भवन्तु। कुत्रा॑चि॒द्यस्य॒ समृ॑तौ र॒ण्वा नरो॑ नृ॒षद॑ने। अर्\mbox{}ह॑न्तश्चि॒द्यमि॑न्ध॒ते स॑ञ्ज॒नय॑न्ति ज॒न्तवः॑। सं यदि॒षो वना॑महे॒ सꣳ ह॒व्या मानु॑षाणाम्। उ॒त द्यु॒म्नस्य॒ शव॑स~(६२)

%2.1.11.4
ऋ॒तस्य॑ र॒श्मिमाद॑दे। य॒ज्ञो दे॒वानां॒ प्रत्ये॑ति सु॒म्नमादि॑त्यासो॒ भव॑ता मृड॒यन्तः॑। आवो॒\-ऽर्वाची॑ सुम॒तिर्व॑वृत्या\-द॒ꣳ॒होश्चि॒द्या व॑रिवो॒वित्त॒रा\-ऽस॑त्। शुचि॑र॒पः सू॒यव॑सा॒ अद॑ब्ध॒ उप॑क्षेति वृ॒द्धव॑याः सु॒वीरः॑। नकि॒ष्टं घ्न॒न्त्यन्ति॑तो॒ न दू॒राद्य आ॑दि॒त्यानां॒ भव॑ति॒ प्रणी॑तौ। धा॒रय॑न्त आदि॒त्यासो॒ जग॒थ्स्था दे॒वा विश्व॑स्य॒ भुव॑नस्य गो॒पाः। दी॒र्घाधि॑यो॒ रक्ष॑माणा~(६३)

%2.1.11.5
असु॒र्य॑मृ॒तावा॑न॒श्चय॑माना ऋ॒णानि॑। ति॒स्रो भूमी᳚र्धारय॒न्त्रीꣳ रु॒त द्यून्त्रीणि॑ व्र॒ता वि॒दथे॑ अ॒न्तरे॑षाम्। ऋ॒तेना॑ऽऽदित्या॒ महि॑ वो महि॒त्वं तद॑र्यमन्वरुण मित्र॒ चारु॑। त्यां नु क्ष॒त्रिया॒ꣳ॒ अव॑ आदि॒त्यान् या॑चिषामहे। सु॒मृ॒डी॒काꣳ अ॒भिष्ट॑ये। न द॑क्षि॒णा विचि॑किते॒ न स॒व्या न प्रा॒चीन॑मादित्या॒ नोत प॒श्चा। पा॒क्या॑चिद्वसवो धी॒र्या॑चि-~(६४)

%2.1.11.6
द्यु॒ष्मानी॑तो॒ अभ॑यं॒ ज्योति॑रश्याम्। आ॒दि॒त्याना॒मव॑सा॒ नूत॑नेन सक्षी॒महि॒ शर्म॑णा॒ शन्त॑मेन। अ॒ना॒गा॒स्त्वे अ॑दिति॒त्वे तु॒रास॑ इ॒मं य॒ज्ञं द॑धतु॒ श्रोष॑माणाः। इ॒मं मे॑ वरुण श्रुधी॒ हव॑म॒द्या च॑ मृडय। त्वाम॑व॒स्युराच॑के। तत्त्वा॑ यामि॒ ब्रह्म॑णा॒ वन्द॑मान॒स्तदाऽऽशा᳚स्ते॒ यज॑मानो ह॒विर्भिः॑। अहे॑डमानो वरुणे॒ह बो॒द्ध्युरु॑शꣳस॒ मा न॒ आयुः॒ प्रमो॑षीः॥~(६५)

%2.2.0.0

{\anuvakamend[{नामा॒ग्निः सꣳ शव॑सो॒ रक्ष॑माणा धी॒र्या॑चि॒देका॒न्नप॑ञ्चा॒शच्च॑}]}%॥11॥

{\prashnaend[{वा॒य॒व्यं॑ प्र॒जा\-प॑ति॒स्ता वरु॑णं देवासु॒रा ए॒ष्व॑सावा॑दि॒त्यो दश॑र्\mbox{}षभा॒मिन्द्रो॑ व॒लस्य॑ बार्\mbox{}हस्प॒त्यं व॑षट्का॒रो॑\-ऽसौ सौ॒रीं वरु॑णमाश्वि॒नमिन्द्रं॑ वो॒ नर॒ एका॑\-दश॥११॥ वा॒य॒व्य॑माग्ने॒यीं कृ॑ष्णग्री॒वीम॒सावा॑दि॒त्यो वा अ॑होरा॒त्राणि॑ वषट्का॒रः प्र॑जनयि॒ता हु॑वे तु॒राणां॒ पञ्च॑षष्टिः॥६५॥ वा॒य॒व्यं॑ प्रमो॑षीः॥}]}

%%% END PRASHNA

\sect{द्वितीयः प्रश्नः}\setcounter{anuvakam}{0}
\dnsub{तैत्तिरीयसंहितायां द्वितीयकाण्डे द्वितीयः प्रश्नः}
%2.2.1.0
%2.2.1.1
प्र॒जा\-प॑तिः प्र॒जा अ॑सृजत॒ ताः सृ॒ष्टा इ॑न्द्रा॒ग्नी अपा॑गूहता॒ꣳ॒ सो॑\-ऽचायत्प्र॒जा\-प॑तिरिन्द्रा॒ग्नी वै मे᳚ प्र॒जा अपा॑घुक्षता॒मिति॒ स ए॒तमै᳚न्द्रा॒ग्नमेका॑\-दश\-कपालमपश्य॒त्तन्निर॑वप॒त्ताव॑स्मै प्र॒जाः प्रासा॑धयतामिन्द्रा॒ग्नी वा ए॒तस्य॑ प्र॒जामप॑गूहतो॒ यो\-ऽलं॑ प्र॒जायै॒ सन्प्र॒जां न वि॒न्दत॑ ऐन्द्रा॒ग्नमेका॑\-दश\-कपालं॒ निर्व॑पेत्प्र॒जाका॑म इन्द्रा॒ग्नी~(१)

%2.2.1.2
ए॒व स्वेन॑ भाग॒धेये॒नोप॑ धावति॒ तावे॒वास्मै᳚ प्र॒जां प्रसा॑धयतो वि॒न्दते᳚ प्र॒जामै᳚न्द्रा॒ग्नमेका॑\-दश\-कपालं॒ निर्व॑पे॒थ्स्पर्ध॑मानः॒ क्षेत्रे॑ वा सजा॒तेषु॑ वेन्द्रा॒ग्नी ए॒व स्वेन॑ भाग॒धेये॒नोप॑ धावति॒ ताभ्या॑मे॒वेन्द्रि॒यं वी॒र्यं॑ भ्रातृ॑व्यस्य वृङ्क्ते॒ वि पा॒प्मना॒ भ्रातृ॑व्येण जय॒ते\-ऽप॒ वा ए॒तस्मा॑दिन्द्रि॒यं वी॒र्यं॑ क्रामति॒ यः स॑ङ्ग्रा॒ममु॑पप्र॒यात्यै᳚न्द्रा॒ग्नमेका॑\-दश\-कपालं॒ निर्-~(२)

%2.2.1.3
व॑पेथ्सङ्ग्रा॒ममु॑पप्रया॒स्यन्नि॑न्द्रा॒ग्नी ए॒व स्वेन॑ भाग॒धेये॒नोप॑ धावति॒ तावे॒वास्मि॑न्निन्द्रि॒यं वी॒र्यं॑ धत्तः स॒हेन्द्रि॒येण॑ वी॒र्ये॑णोप॒प्रया॑ति॒ जय॑ति॒ तꣳ स॑ङ्ग्रा॒मं वि वा ए॒ष इ॑न्द्रि॒येण॑ वी॒र्ये॑णर्द्ध्यते॒ यः स॑ङ्ग्रा॒मं जय॑त्यैन्द्रा॒ग्नमेका॑\-दश\-कपालं॒ निर्व॑पेथ्सङ्ग्रा॒मं जि॒त्वेन्द्रा॒ग्नी ए॒व स्वेन॑ भाग॒धेये॒नोप॑ धावति॒ तावे॒वास्मि॑न्निन्द्रि॒यं वी॒र्यं॑~(३)

%2.2.1.4
धत्तो॒ नेन्द्रि॒येण॑ वी॒र्ये॑ण॒ व्यृ॑द्ध्य॒ते\-ऽप॒ वा ए॒तस्मा॑दिन्द्रि॒यं वी॒र्यं॑ क्रामति॒ य एति॑ ज॒नता॑मैन्द्रा॒ग्नमेका॑\-दश\-कपालं॒ निर्व॑पेज्ज॒नता॑\-मे॒ष्यन्नि॑न्द्रा॒ग्नी ए॒व स्वेन॑ भाग॒धेये॒नोप॑ धावति॒ तावे॒वास्मि॑न्निन्द्रि॒यं वी॒र्यं॑ धत्तः स॒हेन्द्रि॒येण॑ वी॒र्ये॑ण ज॒नता॑मेति पौ॒ष्णं च॒रुमनु॒निर्व॑पेत्पू॒षा वा इ॑न्द्रि॒यस्य॑ वी॒र्य॑स्यानुप्रदा॒ता पू॒षण॑मे॒व~(४)

%2.2.1.5
स्वेन॑ भाग॒धेये॒नोप॑ धावति॒ स ए॒वास्मा॑ इन्द्रि॒यं वी॒र्य॑मनु॒ प्रय॑च्छति क्षैत्रप॒त्यं च॒रुं निर्व॑पेज्ज॒नता॑मा॒गत्ये॒यं वै क्षेत्र॑स्य॒ पति॑र॒स्यामे॒व प्रति॑तिष्ठत्यैन्द्रा॒ग्नमेका॑\-दश\-कपालमु॒परि॑ष्टा॒\-न्निर्व॑पेद॒स्यामे॒व प्र॑ति॒ष्ठाये᳚न्द्रि॒यं वी॒र्य॑मु॒परि॑ष्टादा॒त्मन्ध॑त्ते॥~(५)

%2.2.2.0
{\anuvakamend[{प्र॒जाका॑म इन्द्रा॒ग्नी उ॑पप्र॒यात्यै᳚न्द्रा॒ग्नमेका॑\-दश\-कपालं॒ निर्वी॒र्यं॑ पू॒षण॑मे॒वैका॒न्नच॑त्वारि॒ꣳ॒शच्च॑}]}

%2.2.2.1
अ॒ग्नये॑ पथि॒कृते॑ पुरो॒डाश॑\-म॒ष्टा\-क॑पालं॒ निर्व॑पे॒द्यो द॑र्\mbox{}शपूर्णमास\-या॒जी सन्न॑मावा॒स्यां᳚ वा पौर्णमा॒सीं वा॑\-ऽतिपा॒दये᳚त्प॒थो वा ए॒षो\-ऽद्ध्यप॑थेनैति॒ यो द॑र्\mbox{}शपूर्णमासया॒जी सन्न॑मावा॒स्यां᳚ वा पौर्णमा॒सीं वा॑\-ऽतिपा॒दय॑त्य॒ग्निमे॒व प॑थि॒कृत॒ꣴ॒ स्वेन॑ भाग॒धेये॒नोप॑ धावति॒ स ए॒वैन॒मप॑था॒त्पन्था॒मपि॑ नयत्यन॒ड्वान्दक्षि॑णा व॒ही ह्ये॑ष समृ॑द्ध्या अ॒ग्नये᳚ व्र॒तप॑तये~(६)

%2.2.2.2
पुरो॒डाश॑\-म॒ष्टा\-क॑पालं॒ निर्व॑पे॒द्य आहि॑ताग्निः॒ सन्न॑व्र॒त्यमि॑व॒ चरे॑द॒ग्निमे॒व व्र॒तप॑ति॒ꣴ॒ स्वेन॑ भाग॒धेये॒नोप॑ धावति॒ स ए॒वैनं॑ व्र॒तमाल॑म्भयति॒ व्रत्यो॑ भवत्य॒ग्नये॑ रक्षो॒घ्ने पु॑रो॒डाश॑म॒ष्टा\-क॑पालं॒ निर्व॑पे॒द्यꣳ रक्षाꣳ॑सि॒ सचे॑रन्न॒ग्निमे॒व र॑क्षो॒हण॒ꣴ॒ स्वेन॑ भाग॒धेये॒नोप॑ धावति॒ स ए॒वास्मा॒द्रक्षा॒ꣴ॒स्यप॑\-हन्ति॒ निशि॑तायां॒ निर्व॑पे॒-~(७)

%2.2.2.3
न्निशि॑ताया॒ꣳ॒ हि रक्षाꣳ॑सि प्रे॒रते॑ स॒म्प्रेर्णा᳚न्ये॒वैना॑नि हन्ति॒ परि॑श्रिते याजये॒द्रक्ष॑सा॒मन॑न्ववचाराय रक्षो॒घ्नी या᳚ज्यानुवा॒क्ये॑ भवतो॒ रक्ष॑सा॒ꣴ॒ स्तृत्या॑ अ॒ग्नये॑ रु॒द्रव॑ते पुरो॒डाश॑\-म॒ष्टा\-क॑पालं॒ निर्व॑पेदभि॒चर॑न्ने॒षा वा अ॑स्य घो॒रा त॒नूर्यद्रु॒द्रस्तस्मा॑ ए॒वैन॒मावृ॑श्चति ता॒जगार्ति॒मार्च्छ॑त्य॒ग्नये॑ सुरभि॒मते॑ पुरो॒डाश॑\-म॒ष्टा\-क॑पालं॒ निर्व॑पे॒द्यस्य॒ गावो॑ वा॒ पुरु॑षा~-~(८)

%2.2.2.4
वा प्र॒मीये॑र॒न्॒ यो वा॑ बिभी॒यादे॒षा वा अ॑स्य भेष॒ज्या॑ त॒नूर्यथ्सु॑रभि॒मती॒ तयै॒वास्मै॑ भेष॒जं क॑रोति सुरभि॒मते॑ भवति पूतीग॒न्धस्याप॑हत्या अ॒ग्नये॒ क्षाम॑वते पुरो॒डाश॑\-म॒ष्टा\-क॑पालं॒ निर्व॑पेथ्सङ्ग्रा॒मे सं य॑त्ते भाग॒धेये॑नै॒वैनꣳ॑ शमयि॒त्वा परा॑न॒भि निर्दि॑शति॒ यमव॑रेषां॒ विद्ध्य॑न्ति॒ जीव॑ति॒ स यं परे॑षां॒ प्र स मी॑यते॒ जय॑ति॒ तꣳ स॑ङ्ग्रा॒म-~(९)

%2.2.2.5
म॒भि वा ए॒ष ए॒तानु॑च्यति॒ येषां᳚ पूर्वाप॒रा अ॒न्वञ्चः॑ प्र॒मीय॑न्ते पुरुषाहु॒तिर्ह्य॑स्य प्रि॒यत॑मा॒\-ऽग्नये॒ क्षाम॑वते पुरो॒डाश॑\-म॒ष्टा\-क॑पालं॒ निर्व॑पेद्भाग॒धेये॑नै॒वैनꣳ॑ शमयति॒ नैषां᳚ पु॒रा\-ऽ\-ऽयु॒षो\-ऽप॑रः॒ प्रमी॑यते॒\-ऽभि वा ए॒ष ए॒तस्य॑ गृ॒हानु॑च्यति॒ यस्य॑ गृ॒हान्दह॑त्य॒ग्नये॒ क्षाम॑वते पुरो॒डाश॑\-म॒ष्टा\-क॑पालं॒ निर्व॑पेद्भाग॒धेये॑नै॒वैनꣳ॑ शमयति॒ नास्याप॑रं गृ॒हान्द॑हति॥~(१०)

%2.2.3.0
{\anuvakamend[{व्र॒तप॑तये॒ निशि॑ताया॒न्निर्व॑पे॒त्पुरु॑षाः सङ्ग्रा॒मन्न च॒त्वारि॑ च}]}%~(२)

%2.2.3.1
अ॒ग्नये॒ कामा॑य पुरो॒डाश॑\-म॒ष्टा\-क॑पालं॒ निर्व॑पे॒द्यं कामो॒ नोप॒नमे॑द॒ग्निमे॒व काम॒ꣴ॒ स्वेन॑ भाग॒धेये॒नोप॑ धावति॒ स ए॒वैनं॒ कामे॑न॒ सम॑र्द्धय॒त्युपै॑नं॒ कामो॑ नमत्य॒ग्नये॒ यवि॑ष्ठाय पुरो॒डाश॑\-म॒ष्टा\-क॑पालं॒ निर्व॑पे॒थ्स्पर्ध॑मानः॒ क्षेत्रे॑ वा सजा॒तेषु॑ वा॒\-ऽग्निमे॒व यवि॑ष्ठ॒ꣴ॒ स्वेन॑ भाग॒धेये॒नोप॑ धावति॒ तेनै॒वेन्द्रि॒यं वी॒र्यं॑ भ्रातृ॑व्यस्य~(११)

%2.2.3.2
युवते॒ वि पा॒प्मना॒ भ्रातृ॑व्येण जयते॒\-ऽग्नये॒ यवि॑ष्ठाय पुरो॒डाश॑\-म॒ष्टा\-क॑पालं॒ निर्व॑पेदभिच॒र्यमा॑णो॒\-ऽग्निमे॒व यवि॑ष्ठ॒ꣴ॒ स्वेन॑ भाग॒धेये॒नोप॑ धावति॒ स ए॒वास्मा॒द्रक्षाꣳ॑सि यवयति॒ नैन॑मभि॒चर᳚न्थ्स्तृणुते॒\-ऽग्नय॒ आयु॑ष्मते पुरो॒डाश॑\-म॒ष्टा\-क॑पालं॒ निर्व॑पे॒द्यः का॒मये॑त॒ सर्व॒मायु॑रिया॒मित्य॒ग्निमे॒वाऽऽयु॑ष्मन्त॒ꣴ॒ स्वेन॑ भाग॒धेये॒नोप॑ धावति॒ स ए॒वास्मि॒-~(१२)

%2.2.3.3
न्नायु॑र्दधाति॒ सर्व॒मायु॑रेत्य॒ग्नये॑ जा॒तवे॑दसे पुरो॒डाश॑\-म॒ष्टा\-क॑पालं॒ निर्व॑पे॒द्भूति॑कामो॒\-ऽग्निमे॒व जा॒तवे॑दस॒ꣴ॒ स्वेन॑ भाग॒धेये॒नोप॑ धावति॒ स ए॒वैनं॒ भूतिं॑ गमयति॒ भव॑त्ये॒वाग्नये॒ रुक्म॑ते पुरो॒डाश॑\-म॒ष्टा\-क॑पालं॒ निर्व॑पे॒द्रुक्का॑मो॒\-ऽग्निमे॒व रुक्म॑न्त॒ꣴ॒ स्वेन॑ भाग॒धेये॒नोप॑ धावति॒ स ए॒वास्मि॒न्रुचं॑ दधाति॒ रोच॑त ए॒वाग्नये॒ तेज॑स्वते पुरो॒डाश॑-~(१३)

%2.2.3.4
म॒ष्टा\-क॑पालं॒ निर्व॑पे॒त्तेज॑स्कामो॒\-ऽग्निमे॒व तेज॑स्वन्त॒ꣴ॒ स्वेन॑ भाग॒धेये॒नोप॑ धावति॒ स ए॒वास्मि॒न्तेजो॑ दधाति तेज॒स्व्ये॑व भ॑वत्य॒ग्नये॑ साह॒न्त्याय॑ पुरो॒डाश॑\-म॒ष्टा\-क॑पालं॒ निर्व॑पे॒थ्सीक्ष॑माणो॒\-ऽग्निमे॒व सा॑ह॒न्त्यꣴ स्वेन॑ भाग॒धेये॒नोप॑ धावति॒ तेनै॒व स॑हते॒ यꣳ सीक्ष॑ते॥~(१४)

%2.2.4.0
{\anuvakamend[{भ्रातृ॑व्यस्यास्मि॒न्तेज॑स्वते पुरो॒डाश॑\-म॒ष्टात्रिꣳ॑शच्च}]}%~(३)

%2.2.4.1
अ॒ग्नये\-ऽन्न॑वते पुरो॒डाश॑\-म॒ष्टा\-क॑पालं॒ निर्व॑पे॒द्यः का॒मये॒तान्न॑\-वान्थ्स्या॒मित्य॒ग्नि\-मे॒वान्न॑वन्त॒ꣴ॒ स्वेन॑ भाग॒धेये॒नोप॑ धावति॒ स ए॒वैन॒मन्न॑वन्तं करो॒त्यन्न॑वाने॒व भ॑वत्य॒ग्नये᳚\-ऽन्ना॒दाय॑ पुरो॒डाश॑\-म॒ष्टाक॑पालं॒ निर्व॑पे॒द्यः का॒मये॑तान्ना॒दः स्या॒मित्य॒ग्निमे॒वान्ना॒दꣴ स्वेन॑ भाग॒धेये॒नोप॑ धावति॒ स ए॒वैन॑मन्ना॒दं क॑रोत्यन्ना॒द~-~(१५)

%2.2.4.2
ए॒व भ॑वत्य॒ग्नये\-ऽन्न॑पतये पुरो॒डाश॑\-म॒ष्टा\-क॑पालं॒ निर्व॑पे॒द्यः का॒मये॒तान्न॑पतिः स्या॒मित्य॒ग्निमे॒वान्न॑पति॒ꣴ॒ स्वेन॑ भाग॒धेये॒नोप॑ धावति॒ स ए॒वैन॒मन्न॑पतिं करो॒त्यन्न॑पतिरे॒व भ॑वत्य॒ग्नये॒ पव॑मानाय पुरो॒डाश॑\-म॒ष्टा\-क॑पालं॒ निर्व॑पेद॒ग्नये॑ पाव॒काया॒ग्नये॒ शुच॑ये॒ ज्योगा॑मयावी॒ यद॒ग्नये॒ पव॑मानाय नि॒र्वप॑ति प्रा॒णमे॒वास्मि॒न्तेन॑ दधाति॒ यद॒ग्नये॑~(१६)

%2.2.4.3
पाव॒काय॒ वाच॑मे॒वास्मि॒न्तेन॑ दधाति॒ यद॒ग्नये॒ शुच॑य॒ आयु॑रे॒वास्मि॒न्तेन॑ दधात्यु॒त यदी॒तासु॒र्भव॑ति॒ जीव॑त्ये॒वैतामे॒व निर्व॑पे॒च्चक्षु॑ष्कामो॒ यद॒ग्नये॒ पव॑मानाय नि॒र्वप॑ति प्रा॒णमे॒वा\-स्मि॒न्तेन॑ दधाति॒ यद॒ग्नये॑ पाव॒काय॒ वाच॑मे॒वास्मि॒न्तेन॑ दधाति॒ यद॒ग्नये॒ शुच॑ये॒ चक्षु॑रे॒वास्मि॒न्तेन॑ दधा-~(१७)

%2.2.4.4
त्यु॒त यद्य॒न्धो भव॑ति॒ प्रैव प॑श्यत्य॒ग्नये॑ पु॒त्रव॑ते पुरो॒डाश॑\-म॒ष्टा\-क॑पालं॒ निर्व॑पे॒दिन्द्रा॑य पु॒त्रिणे॑ पुरो॒डाश॒मेका॑\-दश\-कपालं प्र॒जाका॑मो़॒\-ऽग्निरे॒वास्मै᳚ प्र॒जां प्र॑ज॒नय॑ति वृ॒द्धामिन्द्रः॒ प्रय॑च्छत्य॒ग्नये॒ रस॑वते\-ऽजक्षी॒रे च॒रुं निर्व॑पे॒द्यः का॒मये॑त॒ रस॑वान्थ्स्या॒मित्य॒ग्निमे॒व रस॑वन्त॒ꣴ॒ स्वेन॑ भाग॒धेये॒नोप॑ धावति॒ स ए॒वैन॒ꣳ॒ रस॑वन्तं करोति॒~(१८)

%2.2.4.5
रस॑वाने॒व भ॑वत्यजक्षी॒रे भ॑वत्याग्ने॒यी वा ए॒षा यद॒जा सा॒क्षादे॒व रस॒मव॑\-रुन्धे॒\-ऽग्नये॒ वसु॑मते पुरो॒डाश॑\-म॒ष्टा\-क॑पालं॒ निर्व॑पे॒द्यः का॒मये॑त॒ वसु॑मान्थ्स्या॒मित्य॒ग्निमे॒व वसु॑मन्त॒ꣴ॒ स्वेन॑ भाग॒धेये॒नोप॑ धावति॒ स ए॒वैनं॒ वसु॑मन्तं करोति॒ वसु॑माने॒व भ॑वत्य॒ग्नये॑ वाज॒सृते॑ पुरो॒डाश॑\-म॒ष्टा\-क॑पालं॒ निर्व॑पेथ्सङ्ग्रा॒मे सं य॑त्ते॒ वाजं॒~(१९)

%2.2.4.6
वा ए॒ष सि॑सीर्\mbox{}षति॒ यः स॑ङ्ग्रा॒मं जिगी॑षत्य॒ग्निः खलु॒ वै दे॒वानां᳚ वाज॒सृद॒ग्निमे॒व वा॑ज॒सृत॒ꣴ॒ स्वेन॑ भाग॒धेये॒नोप॑ धावति॒ धाव॑ति॒ वाज॒ꣳ॒ हन्ति॑ वृ॒त्रं जय॑ति॒ तꣳ स॑ङ्ग्रा॒ममथो॑ अ॒ग्निरि॑व॒ न प्र॑ति॒धृषे॑ भवत्य॒ग्नये᳚\-ऽग्नि॒वते॑ पुरो॒डाश॑\-म॒ष्टा\-क॑पालं॒ निर्व॑पे॒द्यस्या॒ग्नाव॒ग्निम॑भ्यु॒द्धरे॑यु॒र्निर्दि॑ष्टभागो॒ वा ए॒तयो॑र॒न्यो\-ऽनि॑र्दिष्टभागो॒\-ऽन्यस्तौ स॒म्भव॑न्तौ॒ यज॑मान-~(२०)

%2.2.4.7
म॒भिसम्भ॑वतः॒ स ई᳚श्व॒र आर्ति॒मार्तो॒र्यद॒ग्नये᳚\-ऽग्नि॒वते॑ नि॒र्वप॑ति भाग॒धेये॑नै॒वैनौ॑ शमयति॒ नार्ति॒मार्छ॑ति॒ यज॑मानो॒\-ऽग्नये॒ ज्योति॑ष्मते पुरो॒डाश॑\-म॒ष्टा\-क॑पालं॒ निर्व॑पे॒द्यस्या॒ग्निरुद्धृ॒तो\-ऽहु॑ते\-ऽग्निहो॒त्र उ॒द्वाये॒दप॑र आ॒दीप्या॑नू॒द्धृत्य॒ इत्या॑हु॒स्तत्तथा॒ न का॒र्यं॑ यद्भा॑ग॒धेय॑म॒भि पूर्व॑ उद्ध्रि॒यते॒ किमप॑रो॒\-ऽभ्यु-~(२१)

%2.2.4.8
द्ध्रि॑ये॒तेति॒ तान्ये॒वाव॒क्षाणा॑नि सन्नि॒धाय॑ मन्थेदि॒तः प्र॑थ॒मं ज॑ज्ञे अ॒ग्निः स्वाद्योने॒रधि॑ जा॒तवे॑दाः। स गा॑यत्रि॒या त्रि॒ष्टुभा॒ जग॑त्या दे॒वेभ्यो॑ ह॒व्यं व॑हतु प्रजा॒नन्निति॒ छन्दो॑भिरे॒वैन॒ꣴ॒ स्वाद्योनेः॒ प्रज॑नयत्ये॒ष वाव सो᳚\-ऽग्निरित्या॑हु॒र्ज्योति॒स्त्वा अ॑स्य॒ परा॑पतित॒मिति॒ यद॒ग्नये॒ ज्योति॑ष्मते नि॒र्वप॑ति॒ यदे॒वास्य॒ ज्योतिः॒ परा॑पतितं॒ तदे॒\-वाव॑\-रुन्धे॥~(२२)

%2.2.5.0
{\anuvakamend[{क॒रो॒त्य॒न्ना॒दो द॑धाति॒ यद॒ग्नये॒ शुच॑ये॒ चक्षु॑रे॒वास्मि॒न्तेन॑ दधाति करोति॒ वाजं॒ यज॑मान॒मुदे॒वास्य॒ षट्च॑}]}%~(४)

%2.2.5.1
वै॒श्वा॒न॒रं द्वाद॑श\-कपालं॒ निर्व॑पेद्वारु॒णं च॒रुं द॑धि॒क्राव्ण्णे॑ च॒रुम॑भिश॒स्यमा॑नो॒ यद्वै᳚श्वान॒रो द्वाद॑श\-कपालो॒ भव॑ति संवथ्स॒रो वा अ॒ग्निर्वै᳚श्वान॒रः सं॑वथ्स॒रेणै॒वैनꣴ॑ स्वदय॒त्यप॑ पा॒पं वर्णꣳ॑ हते वारु॒णेनै॒वैनं॑ वरुणपा॒शान्मु॑ञ्चति दधि॒क्राव्ण्णा॑ पुनाति॒ हिर॑ण्यं॒ दक्षि॑णा प॒वित्रं॒ वै हिर॑ण्यं पु॒नात्ये॒वैन॑मा॒द्य॑म॒स्यान्नं॑ भवत्ये॒तामे॒व निर्व॑पेत्प्र॒जाका॑मः संवथ्स॒रो~(२३)

%2.2.5.2
वा ए॒तस्याशा᳚न्तो॒ योनिं॑ प्र॒जायै॑ पशू॒नां निर्द॑हति॒ यो\-ऽलं॑ प्र॒जायै॒ सन्प्र॒जां न वि॒न्दते॒ यद्वै᳚श्वान॒रो द्वाद॑श\-कपालो॒ भव॑ति संवथ्स॒रो वा अ॒ग्निर्वै᳚श्वान॒रः सं॑वथ्स॒रमे॒व भा॑ग॒धेये॑न शमयति॒ सो᳚\-ऽस्मै शा॒न्तः स्वाद्योनेः᳚ प्र॒जां प्रज॑नयति वारु॒णेनै॒वैनं॑ वरुणपा॒शान्मु॑ञ्चति दधि॒क्राव्ण्णा॑ पुनाति॒ हिर॑ण्यं॒ दक्षि॑णा प॒वित्रं॒ वै हिर॑ण्यं पु॒नात्ये॒वैनं॑~(२४)

%2.2.5.3
वि॒न्दते᳚ प्र॒जां वै᳚श्वान॒रं द्वाद॑श\-कपालं॒ निर्व॑पेत्पु॒त्रे जा॒ते यद॒ष्टाक॑पालो॒ भव॑ति गायत्रि॒यैवैनं॑ ब्रह्मवर्च॒सेन॑ पुनाति॒ यन्नव॑कपालस्त्रि॒वृतै॒वास्मि॒न्तेजो॑ दधाति॒ यद्दश॑\-कपालो वि॒राजै॒वास्मि॑न्न॒न्नाद्यं॑ दधाति॒ यदेका॑\-दश\-कपालस्त्रि॒ष्टुभै॒वा\-स्मि॑न्निन्द्रि॒यं द॑धाति॒ यद्द्वाद॑श\-कपालो॒ जग॑त्यै॒वास्मि॑न्प॒शून्द॑\-धाति॒ यस्मि॑ञ्जा॒त ए॒तामिष्टिं॑ नि॒र्वप॑ति पू॒त~-~(२५)

%2.2.5.4
ए॒व ते॑ज॒स्व्य॑न्ना॒द इ॑न्द्रिया॒वी प॑शु॒मान्भ॑व॒त्यव॒ वा ए॒ष सु॑व॒र्गाल्लो॒काच्छि॑द्यते॒ यो द॑र्\mbox{}शपूर्णमासया॒जी सन्न॑मावा॒स्यां᳚ वा पौर्णमा॒सीं वा॑\-ऽतिपा॒दय॑ति सुव॒र्गाय॒ हि लो॒काय॑ दर्\mbox{}श\-पूर्ण\-मा॒सावि॒ज्येते॑ वैश्वान॒रं द्वाद॑श\-कपालं॒ निर्व॑पेदमावा॒स्यां᳚ वा पौर्णमा॒सीं वा॑\-ऽति॒पाद्य॑ संवथ्स॒रो वा अ॒ग्निर्वै᳚श्वान॒रः सं॑वथ्स॒रमे॒व प्री॑णा॒त्यथो॑ संवथ्स॒रमे॒वास्मा॒ उप॑दधाति सुव॒र्गस्य॑ लो॒कस्य॒ सम॑ष्ट्या॒~(२६)

%2.2.5.5
अथो॑ दे॒वता॑ ए॒वान्वा॒रभ्य॑ सुव॒र्गं लो॒कमे॑ति वीर॒हा वा ए॒ष दे॒वानां॒ यो᳚\-ऽग्निमु॑द्वा॒सय॑ते॒ न वा ए॒तस्य॑ ब्राह्म॒णा ऋ॑ता॒यवः॑ पु॒रा\-ऽन्न॑मक्षन्नाग्ने॒यम॒ष्टा\-क॑पालं॒ निर्व॑पेद्वैश्वान॒रं द्वाद॑श\-कपालम॒ग्नि\-मु॑द्वासयि॒ष्यन् यद॒ष्टाक॑पालो॒ भव॑त्य॒ष्टाक्ष॑रा गाय॒त्री गा॑य॒त्रो᳚\-ऽग्निर्यावा॑ने॒वाग्निस्तस्मा॑ आति॒थ्यं क॑रो॒त्यथो॒ यथा॒ जनं॑ य॒ते॑\-ऽव॒सं क॒रोति॑ ता॒दृ-~(२७)

%2.2.5.6
गे॒व तद्द्वाद॑श\-कपालो वैश्वान॒रो भ॑वति॒ द्वाद॑श॒ मासाः᳚ संवथ्स॒रः सं॑वथ्स॒रः खलु॒ वा अ॒ग्नेर्योनिः॒ स्वामे॒वैनं॒ योनिं॑ गमयत्या॒द्य॑म॒स्यान्नं॑ भवति वैश्वान॒रं द्वाद॑श\-कपालं॒ निर्व॑पेन्मारु॒तꣳ स॒प्तक॑पालं॒ ग्राम॑काम आहव॒नीये॑ वैश्वान॒रमधि॑श्रयति॒ गार्\mbox{}ह॑पत्ये मारु॒तं पा॑पवस्य॒सस्य॒ विधृ॑त्यै॒ द्वाद॑श\-कपालो वैश्वान॒रो भ॑वति॒ द्वाद॑श॒ मासाः᳚ संवथ्स॒रः सं॑वथ्स॒रेणै॒वास्मै॑ सजा॒ताꣴश्च्या॑वयति मारु॒तो भ॑वति~(२८)

%2.2.5.7
म॒रुतो॒ वै दे॒वानां॒ विशो॑ देववि॒शेनै॒वास्मै॑ मनुष्यवि॒शमव॑\-रुन्धे स॒प्तक॑पालो भवति स॒प्तग॑णा॒ वै म॒रुतो॑ गण॒श ए॒वास्मै॑ सजा॒तानव॑\-रुन्धे\-ऽनू॒च्यमा॑न॒ आसा॑दयति॒ विश॑मे॒वास्मा॒ अनु॑वर्त्मानं करोति॥~(२९)

%2.2.6.0
{\anuvakamend[{प्र॒जाका॑मः संवथ्स॒रः पु॒नात्ये॒वैनं॑ पू॒तः सम॑ष्ट्यै ता॒दृङ्मा॑रु॒तो भ॑व॒त्येका॒न्नत्रि॒ꣳ॒शच्च॑}]}%~(५)

%2.2.6.1
आ॒दि॒त्यं च॒रुं निर्व॑पेथ्सङ्ग्रा॒ममु॑पप्रया॒स्यन्नि॒यं वा अदि॑तिर॒स्या\-मे॒व पूर्वे॒ प्रति॑तिष्ठन्ति वैश्वान॒रं द्वाद॑श\-कपालं॒ निर्व॑पेदा॒यतनं॑ ग॒त्वा सं॑वथ्स॒रो वा अ॒ग्निर्वै᳚श्वान॒रः सं॑वथ्स॒रः खलु॒ वै दे॒वाना॑मा॒यत॑नमे॒तस्मा॒द्वा आ॒यत॑नाद्दे॒वा असु॑रानजय॒न्॒ यद्वै᳚श्वान॒रं द्वाद॑श\-कपालं नि॒र्वप॑ति दे॒वाना॑\-मे॒वा\-ऽऽ\-यत॑ने यतते॒ जय॑ति॒ तꣳ स॑ङ्ग्रा॒ममे॒तस्मि॒न्वा ए॒तौ मृ॑जाते॒~(३०)

%2.2.6.2
यो वि॑द्विषा॒णयो॒रन्न॒मत्ति॑ वैश्वान॒रं द्वाद॑श\-कपालं॒ निर्व॑पेद्विद्विषा॒\-णयो॒रन्नं॑ ज॒ग्ध्वा सं॑वथ्स॒रो वा अ॒ग्निर्वै᳚श्वान॒रः सं॑वथ्स॒रस्व॑दितमे॒वात्ति॒ नास्मि॑न्मृजाते संवथ्स॒राय॒ वा ए॒तौ सम॑माते॒ यौ स॑म॒माते॒ तयो॒र्यः पूर्वो॑\-ऽभि॒द्रुह्य॑ति॒ तं वरु॑णो गृह्णाति वैश्वान॒रं द्वाद॑श\-कपालं॒ निर्व॑पेथ्सममा॒नयोः॒ पूर्वो॑\-ऽभि॒द्रुह्य॑ संवथ्स॒रो वा अ॒ग्निर्वै᳚श्वान॒रः सं॑वथ्स॒रमे॒वाऽऽप्त्वा नि॑र्वरु॒णं~(३१)

%2.2.6.3
प॒रस्ता॑द॒भिद्रु॑ह्यति॒ नैनं॒ वरु॑णो गृह्णात्या॒व्यं॑ वा ए॒ष प्रति॑गृह्णाति॒ यो\-ऽविं॑ प्रतिगृ॒ह्णाति॑ वैश्वान॒रं द्वाद॑श\-कपालं॒ निर्व॑पे॒दविं॑ प्रति॒गृह्य॑ संवथ्स॒रो वा अ॒ग्निर्वै᳚श्वान॒रः सं॑वथ्स॒रस्व॑दितामे॒व प्रति॑गृह्णाति॒ नाव्यं॑ प्रति॑\-गृह्णात्या॒त्मनो॒ वा ए॒ष मात्रा॑माप्नोति॒ य उ॑भ॒याद॑त्प्रतिगृ॒ह्णात्यश्वं॑ वा॒ पुरु॑षं वा वैश्वान॒रं द्वाद॑श\-कपालं॒ निर्व॑पेदुभ॒याद॑त्~(३२)

%2.2.6.4
प्रति॒गृह्य॑ संवथ्स॒रो वा अ॒ग्निर्वै᳚श्वान॒रः सं॑वथ्स॒रस्व॑दितमे॒व प्रति॑गृह्णाति॒ नाऽऽत्मनो॒ मात्रा॑माप्नोति वैश्वान॒रं द्वाद॑श\-कपालं॒ निर्व॑पेथ्स॒निमे॒ष्यन्थ्सं॑वथ्स॒रो वा अ॒ग्निर्वै᳚श्वान॒रो य॒दा खलु॒ वै सं॑वथ्स॒रं ज॒नता॑यां॒ चर॒त्यथ॒ स ध॑ना॒र्घो भ॑वति॒ यद्वै᳚श्वान॒रं द्वाद॑श\-कपालं नि॒र्वप॑ति संवथ्स॒रसा॑तामे॒व स॒निम॒भि प्रच्य॑वते॒ दान॑कामा अस्मै प्र॒जा भ॑वन्ति॒ यो वै सं॑वथ्स॒रं~(३३)

%2.2.6.5
प्र॒युज्य॒ न वि॑मु॒ञ्चत्य॑प्रतिष्ठा॒नो वै स भ॑वत्ये॒तमे॒व वै᳚श्वान॒रं पुन॑रा॒गत्य॒ निर्व॑पे॒द्यमे॒व प्र॑यु॒ङ्क्ते तं भा॑ग॒धेये॑न॒ विमु॑ञ्चति॒ प्रति॑ष्ठित्यै॒ यया॒ रज्ज्वो᳚त्त॒मां गामा॒जेत्तां भ्रातृ॑व्याय॒ प्रहि॑णुया॒न्निर्\mbox{}ऋ॑तिमे॒वास्मै॒ प्रहि॑णोति॥~(३४)

%2.2.7.0
{\anuvakamend[{नि॒र्व॒रु॒णं व॑पेदुभ॒याद॒द्यो वै सं॑वथ्स॒रꣳ षट्त्रिꣳ॑शच्च।}]}

%2.2.7.1
ऐ॒न्द्रं च॒रुं निर्व॑पेत्प॒शुका॑म ऐ॒न्द्रा वै प॒शव॒ इन्द्र॑मे॒व स्वेन॑ भाग॒धेये॒नोप॑ धावति॒ स ए॒वास्मै॑ प॒शून्प्रय॑च्छति पशु॒माने॒व भ॑वति च॒रुर्भ॑वति॒ स्वादे॒वास्मै॒ योनेः᳚ प॒शून्प्रज॑नय॒तीन्द्रा॑येन्द्रि॒याव॑ते पुरो॒डाश॒मेका॑\-दश\-कपालं॒ निर्व॑पेत्प॒शुका॑म इन्द्रि॒यं वै प॒शव॒ इन्द्र॑मे॒वेन्द्रि॒याव॑न्त॒ꣴ॒ स्वेन॑ भाग॒धेये॒नोप॑ धावति॒ स~-~(३५)

%2.2.7.2
ए॒वास्मा॑ इन्द्रि॒यं प॒शून्प्रय॑च्छति पशु॒माने॒व भ॑व॒तीन्द्रा॑य घ॒र्मव॑ते पुरो॒डाश॒मेका॑\-दश\-कपालं॒ निर्व॑पेद्ब्रह्मवर्च॒सका॑मो ब्रह्मवर्च॒सं वै घ॒र्म इन्द्र॑मे॒व घ॒र्मव॑न्त॒ꣴ॒ स्वेन॑ भाग॒धेये॒नोप॑ धावति॒ स ए॒वास्मि॑न्ब्रह्मवर्च॒सं द॑धाति ब्रह्मवर्च॒स्ये॑व भ॑व॒तीन्द्रा॑या॒र्कव॑ते पुरो॒डाश॒मेका॑\-दश\-कपालं॒ निर्व॑पे॒दन्न॑कामो॒\-ऽर्को वै दे॒वाना॒मन्न॒मिन्द्र॑मे॒वार्कव॑न्त॒ꣴ॒ स्वेन॑ भाग॒धेये॒-~(३६)

%2.2.7.3
नोप॑धावति॒ स ए॒वास्मा॒ अन्नं॒ प्रय॑च्छत्यन्ना॒द ए॒व भ॑व॒तीन्द्रा॑य घ॒र्मव॑ते पुरो॒डाश॒मेका॑\-दश\-कपालं॒ निर्व॑पे॒दिन्द्रा॑येन्द्रि॒याव॑त॒ इन्द्रा॑\-या॒र्कव॑ते॒ भूति॑कामो॒ यदिन्द्रा॑य घ॒र्मव॑ते नि॒र्वप॑ति॒ शिर॑ ए॒वास्य॒ तेन॑ करोति॒ यदिन्द्रा॑येन्द्रि॒याव॑त आ॒त्मान॑मे॒वास्य॒ तेन॑ करोति॒ यदिन्द्रा॑या॒र्कव॑ते भू॒त ए॒वान्नाद्ये॒ प्रति॑तिष्ठति॒ भव॑त्ये॒वेन्द्रा॑या-~(३७)

%2.2.7.4
ꣳहो॒मुचे॑ पुरो॒डाश॒मेका॑\-दश\-कपालं॒ निर्व॑पे॒द्यः पा॒प्मना॑ गृही॒तः स्यात्पा॒प्मा वा अꣳह॒ इन्द्र॑मे॒वाꣳहो॒मुच॒ꣴ॒ स्वेन॑ भाग॒धेये॒नोप॑ धावति॒ स ए॒वैनं॑ पा॒प्मनो\-ऽꣳह॑सो मुञ्च॒तीन्द्रा॑य वैमृ॒धाय॑ पुरो॒डाश॒मेका॑\-दश\-कपालं॒ निर्व॑पे॒द्यं मृधो॒\-ऽभि प्र॒वेपे॑रन्रा॒ष्ट्राणि॑ वा॒\-ऽभिस॑मि॒युरिन्द्र॑मे॒व वै॑मृ॒धꣴ स्वेन॑ भाग॒धेये॒नोप॑ धावति॒ स ए॒वास्मा॒न्मृधो~(३८)

%2.2.7.5
ऽप॑ह॒न्तीन्द्रा॑य त्रा॒त्रे पु॑रो॒डाश॒मेका॑\-दश\-कपालं॒ निर्व॑पेद्ब॒द्धो वा॒ परि॑यत्तो॒ वेन्द्र॑मे॒व त्रा॒तार॒ꣴ॒ स्वेन॑ भाग॒धेये॒नोप॑ धावति॒ स ए॒वैनं॑ त्रायत॒ इन्द्रा॑यार्काश्वमे॒धव॑ते पुरो॒डाश॒मेका॑\-दश\-कपालं॒ निर्व॑पे॒द्यं म॑हाय॒ज्ञो नोप॒नमे॑दे॒ते वै म॑हाय॒ज्ञस्यान्त्ये॑ त॒नू यद॑र्काश्वमे॒धाविन्द्र॑मे॒वार्का᳚श्वमे॒धव॑न्त॒ꣴ॒ स्वेन॑ भाग॒धेये॒नोप॑ धावति॒ स ए॒वास्मा॑ अन्त॒तो म॑हाय॒ज्ञं च्या॑वय॒त्युपै॑नं महाय॒ज्ञो न॑मति॥~(३९)

%2.2.8.0
{\anuvakamend[{इ॒न्द्रि॒याव॑न्त॒ꣴ॒ स्वेन॑ भाग॒धेये॒नोप॑ धावति॒ सो᳚\-ऽर्कव॑न्त॒ꣴ॒ स्वेन॑ भाग॒धेये॑नै॒वेन्द्रा॑यास्मा॒न्मृधो᳚\-ऽस्मै स॒प्त च॑}]}

%2.2.8.1
इन्द्रा॒यान्वृ॑जवे पुरो॒डाश॒मेका॑\-दश\-कपालं॒ निर्व॑पे॒द्ग्राम॑काम॒ इन्द्र॑मे॒वान्वृ॑जु॒ꣴ॒ स्वेन॑ भाग॒धेये॒नोप॑ धावति॒ स ए॒वास्मै॑ सजा॒ताननु॑कान्करोति ग्रा॒म्ये॑व भ॑वतीन्द्रा॒ण्यै च॒रुं निर्व॑पे॒द्यस्य॒ सेना\-ऽसꣳ॑शितेव॒ स्यादि॑न्द्रा॒णी वै सेना॑यै दे॒वते᳚न्द्रा॒णीमे॒व स्वेन॑ भाग॒धेये॒नोप॑ धावति॒ सैवास्य॒ सेना॒ꣳ॒ सꣴश्य॑ति॒ बल्ब॑जा॒नपी॒-~(४०)

%2.2.8.2
द्ध्मे सन्न॑ह्ये॒द्गौर्यत्राधि॑ष्कन्ना॒ न्यमे॑ह॒त्ततो॒ बल्ब॑जा॒ उद॑तिष्ठ॒न्गवा॑\-मे॒वैनं॑ न्या॒यम॑पि॒नीय॒ गा वे॑दय॒तीन्द्रा॑य मन्यु॒मते॒ मन॑स्वते पुरो॒डाश॒मेका॑\-दश\-कपालं॒ निर्व॑पेथ्सङ्ग्रा॒मे सं य॑त्त इन्द्रि॒येण॒ वै म॒न्युना॒ मन॑सा सङ्ग्रा॒मं ज॑य॒तीन्द्र॑मे॒व म॑न्यु॒मन्तं॒ मन॑स्वन्त॒ꣴ॒ स्वेन॑ भाग॒धेये॒नोप॑ धावति॒ स ए॒वास्मि॑न्निन्द्रि॒यं म॒न्युं मनो॑ दधाति॒ जय॑ति॒ तꣳ~(४१)

%2.2.8.3
स॑ङ्ग्रा॒ममे॒तामे॒व निर्व॑पे॒द्यो ह॒तम॑नाः स्व॒यं पा॑प इव॒ स्यादे॒तानि॒ हि वा ए॒तस्मा॒दप॑क्रान्ता॒न्यथै॒ष ह॒तम॑नाः स्व॒यं पा॑प॒ इन्द्र॑मे॒व म॑न्यु॒मन्तं॒ मन॑स्वन्त॒ꣴ॒ स्वेन॑ भाग॒धेये॒नोप॑ धावति॒ स ए॒वास्मि॑न्निन्द्रि॒यं म॒न्युं मनो॑ दधाति॒ न ह॒तम॑नाः स्व॒यं पा॑पो भव॒तीन्द्रा॑य दा॒त्रे पु॑रो॒डाश॒मेका॑\-दश\-कपालं॒ निर्व॑पे॒द्यः का॒मये॑त॒ दान॑कामा मे प्र॒जाः स्यु॒-~(४२)

%2.2.8.4
रितीन्द्र॑मे॒व दा॒तार॒ꣴ॒ स्वेन॑ भाग॒धेये॒नोप॑ धावति॒ स ए॒वास्मै॒ दान॑कामाः प्र॒जाः क॑रोति॒ दान॑कामा अस्मै प्र॒जा भ॑व॒न्तीन्द्रा॑य प्रदा॒त्रे पु॑रो॒डाश॒मेका॑\-दश\-कपालं॒ निर्व॑पे॒द्यस्मै॒ प्रत्त॑मिव॒ सन्न प्र॑दी॒येतेन्द्र॑मे॒व प्र॑दा॒तार॒ꣴ॒ स्वेन॑ भाग॒धेये॒नोप॑ धावति॒ स ए॒वास्मै॒ प्रदा॑पय॒तीन्द्रा॑य सु॒त्राम्णे॑ पुरो॒डाश॒मेका॑\-दश\-कपालं॒ निर्व॑पे॒दप॑रुद्धो वा-~(४३)

%2.2.8.5
ऽपरु॒द्ध्यमा॑नो॒ वेन्द्र॑मे॒व सु॒त्रामा॑ण॒ꣴ॒ स्वेन॑ भाग॒धेये॒नोप॑ धावति॒ स ए॒वैनं॑ त्रायते\-ऽनपरु॒द्ध्यो भ॑व॒तीन्द्रो॒ वै स॒दृङ् दे॒वता॑भिरासी॒थ्स न व्या॒वृत॑मगच्छ॒थ्स प्र॒जा\-प॑ति॒मुपा॑\-धाव॒त्तस्मा॑ ए॒तमै॒न्द्रमेका॑\-दश\-कपालं॒ निर॑वप॒त्तेनै॒वास्मि॑न्निन्द्रि॒य\-म॑दधा॒च्छक्व॑री याज्यानुवा॒क्ये॑ अकरो॒द्वज्रो॒ वै शक्व॑री॒ स ए॑नं॒ वज्रो॒ भूत्या॑ ऐन्ध॒~(४४)

%2.2.8.6
सो॑\-ऽभव॒थ्सो॑\-ऽबिभेद्भू॒तः प्र मा॑ धक्ष्य॒तीति॒ स प्र॒जा\-प॑तिं॒ पुन॒रुपा॑धाव॒थ्स प्र॒जा\-प॑तिः॒ शक्व॑र्या॒ अधि॑ रे॒वतीं॒ निर॑मिमीत॒ शान्त्या॒ अप्र॑दाहाय॒ यो\-ऽलꣴ॑ श्रि॒यै सन्थ्स॒दृङ्ख्स॑मा॒नैः स्यात्तस्मा॑ ए॒तमै॒न्द्रमेका॑\-दश\-कपालं॒ निर्व॑पे॒दिन्द्र॑मे॒व स्वेन॑ भाग॒धेये॒नोप॑ धावति॒ स ए॒वास्मि॑न्निन्द्रि॒यं द॑धाति रे॒वती॑ पुरोनुवा॒क्या॑ भवति॒ शान्त्या॒ अप्र॑दाहाय॒ शक्व॑री या॒ज्या॑ वज्रो॒ वै शक्व॑री॒ स ए॑नं॒ वज्रो॒ भूत्या॑ इन्धे॒ भव॑त्ये॒व॥~(४५)

%2.2.9.0
{\anuvakamend[{अपि॒ तꣴ स्यु॑र्वैन्ध भवति॒ चतु॑र्दश च}]}%~(७)

%2.2.9.1
आ॒ग्ना॒वै॒ष्ण॒वमेका॑दशकपालं॒ निर्व॑पेदभि॒चर॒न्थ्सर॑स्व॒त्याज्य॑\-भागा॒ स्याद्बा॑र्\mbox{}हस्प॒त्यश्च॒रुर्यदा᳚ग्नावैष्ण॒व एका॑\-दश\-कपालो॒ भव॑त्य॒ग्निः सर्वा॑ दे॒वता॒ विष्णु॑र्य॒ज्ञो दे॒वता॑भिश्चै॒वैनं॑ य॒ज्ञेन॑ चा॒भिच॑रति॒ सर॑स्व॒त्याज्य॑भागा भवति॒ वाग्वै सर॑स्वती वा॒चैवैन॑म॒भिच॑रति बार्\mbox{}हस्प॒त्यश्च॒रुर्भ॑वति॒ ब्रह्म॒ वै दे॒वानां॒ बृह॒स्पति॒र्ब्रह्म॑णै॒वैन॑म॒भिच॑रति॒~(४६)

%2.2.9.2
प्रति॒ वै प॒रस्ता॑दभि॒चर॑न्तम॒भिच॑रन्ति॒ द्वेद्वे॑ पुरोनुवा॒क्ये॑ कुर्या॒दति॒प्रयु॑क्त्या ए॒तयै॒व य॑जेताभिच॒र्यमा॑णो दे॒वता॑भिरे॒व दे॒वताः᳚ प्रति॒चर॑ति य॒ज्ञेन॑ य॒ज्ञं वा॒चा वाचं॒ ब्रह्म॑णा॒ ब्रह्म॒ स दे॒वता᳚श्चै॒व य॒ज्ञं च॑ मद्ध्य॒तो व्यव॑सर्पति॒ तस्य॒ न कुत॑श्च॒नोपा᳚व्या॒धो भ॑वति॒ नैन॑मभि॒चर᳚न्थ्स्तृणुत आग्नावैष्ण॒वमेका॑\-दश\-कपालं॒ निर्व॑पे॒द्यं य॒ज्ञो नो-~(४७)

%2.2.9.3
प॒नमे॑द॒ग्निः सर्वा॑ दे॒वता॒ विष्णु॑र्य॒ज्ञो᳚\-ऽग्निं चै॒व विष्णुं॑ च॒ स्वेन॑ भाग॒धेये॒नोप॑ धावति॒ तावे॒वास्मै॑ य॒ज्ञं प्रय॑च्छत॒ उपै॑नं य॒ज्ञो न॑मत्याग्नावैष्ण॒वं घृ॒ते च॒रुं निर्व॑पे॒च्चक्षु॑ष्कामो॒\-ऽग्नेर्वै चक्षु॑षा मनु॒ष्या॑ वि प॑श्यन्ति य॒ज्ञस्य॑ दे॒वा अ॒ग्निं चै॒व विष्णुं॑ च॒ स्वेन॑ भाग॒धेये॒नोप॑ धावति॒ तावे॒वा-~(४८)

%2.2.9.4
स्मि॒ञ्चक्षु॑र्धत्त॒श्चक्षु॑ष्माने॒व भ॑वति धे॒न्वै वा ए॒तद्रेतो॒ यदाज्य॑मन॒डुह॑स्तण्डु॒ला मि॑थु॒नादे॒वास्मै॒ चक्षुः॒ प्रज॑नयति घृ॒ते भ॑वति॒ तेजो॒ वै घृ॒तं तेज॒श्चक्षु॒स्तेज॑सै॒वास्मै॒ तेज॒श्चक्षु॒रव॑रुन्ध इन्द्रि॒यं वै वी॒र्यं॑ वृङ्क्ते॒ भ्रातृ॑व्यो॒ यज॑मा॒नो\-ऽय॑जमानस्याद्ध्व॒रक॑ल्पां॒ प्रति॒ निर्व॑पे॒द्भ्रातृ॑व्ये॒ यज॑माने॒ नास्ये᳚न्द्रि॒यं~(४९)

%2.2.9.5
वी॒र्यं॑ वृङ्क्ते पु॒रा वा॒चः प्रव॑दितो॒र्निर्व॑पे॒द्याव॑त्ये॒व वाक्तामप्रो॑दितां॒ भ्रातृ॑व्यस्य वृङ्क्ते॒ ताम॑स्य॒ वाचं॑ प्र॒वद॑न्तीम॒न्या वाचो\-ऽनु॒ प्रव॑दन्ति॒ ता इ॑न्द्रि॒यं वी॒र्यं॑ यज॑माने दधत्याग्ना\-वैष्ण॒व\-म॒ष्टा\-क॑पालं॒ निर्व॑पेत्प्रातः सव॒नस्या॑का॒ले सर॑स्व॒त्याज्य॑भागा॒ स्याद्बा॑र्\mbox{}हस्प॒त्यश्च॒रुर्यद॒ष्टाक॑पालो॒ भव॑त्य॒ष्टाक्ष॑रा गाय॒त्री गा॑य॒त्रं प्रा॑तः सव॒नं प्रा॑तः सव॒नमे॒व तेना᳚ऽऽप्नो-~(५०)

%2.2.9.6
त्याग्नावैष्ण॒वमेका॑\-दश\-कपालं॒ निर्व॑पे॒न्माद्ध्य॑न्दिनस्य॒ सव॑नस्या\-का॒ले सर॑स्व॒त्याज्य॑भागा॒ स्याद्बा॑र्\mbox{}हस्प॒त्यश्च॒रुर्यदेका॑\-दश\-कपालो॒ भव॒त्येका॑\-दशाक्षरा त्रि॒ष्टुप्त्रैष्टु॑भं॒ माद्ध्य॑न्दिन॒ꣳ॒ सव॑नं॒ माद्ध्य॑न्दिनमे॒व सव॑नं॒ तेना᳚ऽऽप्नोत्याग्नावैष्ण॒वं द्वाद॑श\-कपालं॒ निर्व॑पेत्तृतीय\-सव॒नस्या॑\-का॒ले सर॑स्व॒त्याज्य॑भागा॒ स्याद्बा॑र्\mbox{}ह\-स्प॒त्यश्च॒रुर्यद्द्वाद॑श\-कपालो॒ भव॑ति॒ द्वाद॑शाक्षरा॒ जग॑ती॒ जाग॑तं तृतीयसव॒नं तृ॑तीयसव॒नमे॒व तेना᳚ऽऽप्नोति दे॒वता॑भिरे॒व दे॒वताः᳚~(५१)

%2.2.9.7
प्रति॒चर॑ति य॒ज्ञेन॑ य॒ज्ञं वा॒चा वाचं॒ ब्रह्म॑णा॒ ब्रह्म॑ क॒पालै॑रे॒व छन्दाꣴ॑स्या॒प्नोति॑ पुरो॒डाशैः॒ सव॑नानि मैत्रावरु॒णमेक॑कपालं॒ निर्व॑पेद्व॒शायै॑ का॒ले यैवासौ भ्रातृ॑व्यस्य व॒शा\-ऽनू॑ब॒न्ध्या॑ सो ए॒वैषैतस्यैक॑कपालो भवति॒ नहि क॒पालैः᳚ प॒शुमर्\mbox{}ह॒त्याप्तुम्᳚॥~(५२)

%2.2.10.0
{\anuvakamend[{ब्रह्म॑णै॒वैन॑म॒भिच॑रति य॒ज्ञो न तावे॒वास्ये᳚न्द्रि॒यमा᳚प्नोति दे॒वताः᳚ स॒प्तत्रिꣳ॑शच्च।}]}

%2.2.10.1
अ॒सावा॑दि॒त्यो न व्य॑रोचत॒ तस्मै॑ दे॒वाः प्राय॑श्चित्ति\-मैच्छ॒न्तस्मा॑ ए॒तꣳ सो॑मारौ॒द्रं च॒रुं निर॑वप॒न्तेनै॒वास्मि॒न्रुच॑मद\-धु॒र्यो ब्र॑ह्मवर्च॒सका॑मः॒ स्यात्तस्मा॑ ए॒तꣳ सो॑मारौ॒द्रं च॒रुं निर्व॑पे॒थ्सोमं॑ चै॒व रु॒द्रं च॒ स्वेन॑ भाग॒धेये॒नोप॑ धावति॒ तावे॒वास्मि॑न्ब्रह्मवर्च॒सन्ध॑त्तो ब्रह्मवर्च॒स्ये॑व भ॑वति तिष्यापूर्णमा॒से निर्व॑पेद्रु॒द्रो-~(५३)

%2.2.10.2
वै ति॒ष्यः॑ सोमः॑ पू॒र्णमा॑सः सा॒क्षादे॒व ब्र॑ह्मवर्च॒समव॑\-रुन्धे॒ परि॑श्रिते याजयति ब्रह्मवर्च॒सस्य॒ परि॑गृहीत्यै श्वे॒तायै᳚ श्वे॒तव॑थ्सायै दु॒ग्धं म॑थि॒तमाज्यं॑ भव॒त्याज्यं॒ प्रोक्ष॑ण॒माज्ये॑न मार्जयन्ते॒ याव॑दे॒व ब्र॑ह्मवर्च॒सं तथ्सर्वं॑ करो॒त्यति॑ ब्रह्मवर्च॒सं क्रि॑यत॒ इत्या॑हुरीश्व॒रो दु॒श्चर्मा॒ भवि॑तो॒रिति॑ मान॒वी ऋचौ॑ धा॒य्ये॑ कुर्या॒द्यद्वै किं च॒ मनु॒रव॑द॒त्तद्भे॑ष॒जं~(५४)

%2.2.10.3
भे॑ष॒जमे॒वास्मै॑ करोति॒ यदि॑ बिभी॒याद्दु॒श्चर्मा॑ भविष्या॒मीति॑ सोमापौ॒ष्णं च॒रुं निर्व॑पेथ्सौ॒म्यो वै दे॒वत॑या॒ पुरु॑षः पौ॒ष्णाः प॒शवः॒ स्वयै॒वास्मै॑ दे॒वत॑या प॒शुभि॒स्त्वचं॑ करोति॒ न दु॒श्चर्मा॑ भवति सोमारौ॒द्रं च॒रुं निर्व॑पेत्प्र॒जाका॑मः॒ सोमो॒ वै रे॑तो॒धा अ॒ग्निः प्र॒जानां᳚ प्रजनयि॒ता सोम॑ ए॒वास्मै॒ रेतो॒ दधा᳚त्य॒ग्निः प्र॒जां प्रज॑नयति वि॒न्दते᳚~(५५)

%2.2.10.4
प्र॒जाꣳ सो॑मारौ॒द्रं च॒रुं निर्व॑पेदभि॒चर᳚न्थ्सौ॒म्यो वै दे॒वत॑या॒ पुरु॑ष ए॒ष रु॒द्रो यद॒ग्निः स्वाया॑ ए॒वैनं॑ दे॒वता॑यै नि॒ष्क्रीय॑ रु॒द्रायापि॑ दधाति ता॒जगार्ति॒मार्च्छ॑ति सोमारौ॒द्रं च॒रुं निर्व॑पे॒ज्ज्योगा॑मयावी॒ सोमं॒ वा ए॒तस्य॒ रसो॑ गच्छत्य॒ग्निꣳ शरी॑रं॒ यस्य॒ ज्योगा॒मय॑ति॒ सोमा॑दे॒वास्य॒ रसं॑ निष्क्री॒णात्य॒ग्नेः शरी॑रमु॒त यदी॒-~(५६)

%2.2.10.5
तासु॒र्भव॑ति॒ जीव॑त्ये॒व सो॑मारु॒द्रयो॒र्वा ए॒तं ग्र॑सि॒तꣳ होता॒ निष्खि॑दति॒ स ई᳚श्व॒र आर्ति॒मार्तो॑रन॒ड्वान् होत्रा॒ देयो॒ वह्नि॒र्वा अ॑न॒ड्वान् वह्नि॒र्॒\mbox{}होता॒ वह्नि॑नै॒व वह्नि॑मा॒त्मानꣴ॑ स्पृणोति सोमारौ॒द्रं च॒रुं निर्व॑पे॒द्यः का॒मये॑त॒ स्वे᳚\-ऽस्मा आ॒यत॑ने॒ भ्रातृ॑व्यं जनयेय॒मिति॒ वेदिं॑ परि॒गृह्या॒र्द्धमु॑द्ध॒न्याद॒र्द्धं नार्द्धं ब॒र्॒\mbox{}हिषः॑ स्तृणी॒याद॒र्द्धं नार्द्धमि॒द्ध्मस्या᳚भ्याद॒द्ध्याद॒र्द्धं न स्व ए॒वास्मा॑ आ॒यत॑ने॒ भ्रातृ॑व्यं जनयति॥~(५७)

%2.2.11.0
{\anuvakamend[{रु॒द्रो भे॑ष॒जं वि॒न्दते॒ यदि॑ स्तृणी॒याद॒र्द्धं द्वाद॑श च}]}

%2.2.11.1
ऐ॒न्द्रमेका॑\-दश\-कपालं॒ निर्व॑पेन्मारु॒तꣳ स॒प्तक॑पालं॒ ग्राम॑काम॒ इन्द्रं॑ चै॒व म॒रुत॑श्च॒ स्वेन॑ भाग॒धेये॒नोप॑ धावति॒ त ए॒वास्मै॑ सजा॒तान्प्रय॑च्छन्ति ग्रा॒म्ये॑व भ॑वत्याहव॒नीय॑ ऐ॒न्द्रमधि॑श्रयति॒ गार्\mbox{}ह॑पत्ये मारु॒तं पा॑पवस्य॒सस्य॒ विधृ॑त्यै स॒प्तक॑पालो मारु॒तो भ॑वति स॒प्तग॑णा॒ वै म॒रुतो॑ गण॒श ए॒वास्मै॑ सजा॒तानव॑\-रुन्धे\-ऽनू॒च्यमा॑न॒ आसा॑दयति॒ विश॑मे॒वा-~(५८)

%2.2.11.2
स्मा॒ अनु॑वर्त्मानं करोत्ये॒तामे॒व निर्व॑पे॒द्यः का॒मये॑त क्ष॒त्राय॑ च वि॒शे च॑ स॒मदं॑ दद्ध्या॒मित्यै॒न्द्रस्या॑व॒द्यन्ब्रू॑या॒दिन्द्रा॒यानु॑ ब्रू॒हीत्या॒श्राव्य॑ ब्रूयान्म॒रुतो॑ य॒जेति॑ मारु॒तस्या॑व॒द्यन्ब्रू॑यान्म॒रुद्भ्यो\-ऽनु॑ब्रू॒हीत्या॒श्राव्य॑ ब्रूया॒दिन्द्रं॑ य॒जेति॒ स्व ए॒वैभ्यो॑ भाग॒धेये॑ स॒मदं॑ दधाति वितृꣳहा॒णास्ति॑ष्ठन्त्ये॒तामे॒व~(५९)

%2.2.11.3
निर्व॑पे॒द्यः का॒मये॑त॒ कल्पे॑र॒न्निति॑ यथादेव॒तम॑व॒दाय॑ यथादेव॒तं य॑जेद्भाग॒धेये॑नै॒वैनान्॑ यथाय॒थं क॑ल्पयति॒ कल्प॑न्त ए॒वैन्द्रमेका॑\-दश\-कपालं॒ निर्व॑पेद्वैश्वदे॒वं द्वाद॑श\-कपालं॒ ग्राम॑काम॒ इन्द्रं॑ चै॒व विश्वाꣴ॑श्च दे॒वान्थ्स्वेन॑ भाग॒धेये॒नोप॑ धावति॒ त ए॒वास्मै॑ सजा॒तान्प्रय॑च्छन्ति ग्रा॒म्ये॑व भ॑वत्यै॒न्द्रस्या॑व॒दाय॑ वैश्वदे॒वस्याव॑द्ये॒दथै॒न्द्रः स्यो॒-~(६०)

%2.2.11.4
परि॑ष्टादिन्द्रि॒येणै॒वास्मा॑ उभ॒यतः॑ सजा॒तान्परि॑\-गृह्णात्युपाधा॒य्य॑\-पूर्वयं॒ वासो॒ दक्षि॑णा सजा॒ताना॒मुप॑हित्यै॒ पृश्ञि॑यै दु॒ग्धे प्रैय॑ङ्गवं च॒रुं निर्व॑पेन्म॒रुद्भ्यो॒ ग्राम॑कामः॒ पृश्ञि॑यै॒ वै पय॑सो म॒रुतो॑ जा॒ताः पृश्ञि॑यै प्रि॒यङ्ग॑वो मारु॒ताः खलु॒ वै दे॒वत॑या सजा॒ता म॒रुत॑ ए॒व स्वेन॑ भाग॒धेये॒नोप॑ धावति॒ त ए॒वास्मै॑ सजा॒तान्प्रय॑च्छन्ति ग्रा॒म्ये॑व भ॑वति प्रि॒यव॑ती याज्यानुवा॒क्ये॑~(६१)

%2.2.11.5
भवतः प्रि॒यमे॒वैनꣳ॑ समा॒नानां᳚ करोति द्वि॒पदा॑ पुरोनुवा॒क्या॑ भवति द्वि॒पद॑ ए॒वाव॑\-रुन्धे॒ चतु॑ष्पदा या॒ज्या॑ चतु॑ष्पद ए॒व प॒शूनव॑\-रुन्धे देवासु॒राः सं य॑त्ता आस॒न्ते दे॒वा मि॒थो विप्रि॑या आस॒न्ते \-ऽन्यो᳚न्यस्मै॒ ज्यैष्ठ्या॒याति॑ष्ठमानाश्चतु॒र्धा व्य॑क्रामन्न॒ग्निर्वसु॑भिः॒ सोमो॑ रु॒द्रैरिन्द्रो॑ म॒रुद्भि॒र्वरु॑ण आदि॒त्यैः स इन्द्रः॑ प्र॒जा\-प॑ति॒मुपा॑धाव॒त्तमे॒-~(६२)

%2.2.11.6
तया॑ सं॒ज्ञान्या॑\-ऽयाजयद॒ग्नये॒ वसु॑मते पुरो॒डाश॑\-म॒ष्टा\-क॑पालं॒ निर॑वप॒थ्सोमा॑य रु॒द्रव॑ते च॒रुमिन्द्रा॑य म॒रुत्व॑ते पुरो॒डाश॒मेका॑\-दश\-कपालं॒ वरु॑णाया\-ऽ\-ऽदि॒त्यव॑ते च॒रुं ततो॒ वा इन्द्रं॑ दे॒वा ज्यैष्ठ्या॑या॒भि सम॑जानत॒ यः स॑मा॒नैर्मि॒थो विप्रि॑यः॒ स्यात्तमे॒तया॑ सं॒ज्ञान्या॑ याजयेद॒ग्नये॒ वसु॑मते पुरो॒डाश॑\-म॒ष्टा\-क॑पालं॒ निर्व॑पे॒थ्सोमा॑य रु॒द्रव॑ते च॒रुमिन्द्रा॑य म॒रुत्व॑ते पुरो॒डाश॒मेका॑\-दश\-कपालं॒ वरु॑णाया\-ऽ\-ऽदि॒त्यव॑ते च॒रुमिन्द्र॑मे॒वैनं॑ भू॒तं ज्यैष्ठ्या॑य समा॒ना अ॒भिसञ्जा॑नते॒ वसि॑ष्ठः समा॒नानां᳚ भवति॥~(६३)

%2.2.12.0
{\anuvakamend[{विश॑मे॒व ति॑ष्ठन्त्ये॒तामे॒वाथै॒न्द्रस्य॑ याज्यानुवा॒क्ये॑ तं वरु॑णाय॒ चतु॑र्दश च}]}

%2.2.12.0

%2.2.12.1
हि॒र॒ण्य॒ग॒र्भ आपो॑ ह॒ यत्प्रजा॑पते। स वे॑द पु॒त्रः पि॒तर॒ꣳ॒ स मा॒तर॒ꣳ॒ स सू॒नुर्भु॑व॒थ्स भु॑व॒त्पुन॑र्मघः। स द्यामौर्णो॑द॒न्तरि॑क्ष॒ꣳ॒ स सुवः॒ स विश्वा॒ भुवो॑ अभव॒थ्स आ\-ऽभ॑वत्। उदु॒त्यं चि॒त्रम्। सप्र॑त्न॒वन्नवी॑य॒सा\-ऽग्ने᳚ द्यु॒म्नेन॑ सं॒ यता᳚। बृ॒हत्त॑तन्थ भा॒नुना᳚। निकाव्या॑ वे॒धसः॒ शश्व॑तस्क॒र्॒\mbox{}हस्ते॒ दधा॑नो॒~-~(६४)

%2.2.12.2
नर्या॑ पु॒रूणि॑। अ॒ग्निर्भु॑वद्रयि॒पती॑ रयी॒णाꣳ स॒त्रा च॑क्रा॒णो अ॒मृता॑नि॒ विश्वा᳚। हिर॑ण्यपाणिमू॒तये॑ सवि॒तार॒मुप॑ ह्वये। स चेत्ता॑ दे॒वता॑ प॒दम्। वा॒मम॒द्य स॑वितर्वा॒ममु॒ श्वो दि॒वेदि॑वे वा॒मम॒स्मभ्यꣳ॑ सावीः। वा॒मस्य॒ हि क्षय॑स्य देव॒ भूरे॑र॒या धि॒या वा॑म॒भाजः॑ स्याम। बडि॒त्था पर्व॑तानां खि॒द्रं बि॑भर्\mbox{}षि पृथिवि। प्र या भू॑मि प्रवत्वति म॒ह्ना जि॒नोषि॑~(६५)

%2.2.12.3
महिनि। स्तोमा॑सस्त्वा विचारिणि॒ प्रति॑ष्टोभन्त्य॒क्तुभिः॑। प्र या वाजं॒ न हेष॑न्तं पे॒रुमस्य॑स्यर्जुनि। ऋ॒दू॒दरे॑ण॒ सख्या॑ सचेय॒ यो मा॒ न रिष्ये᳚द्धर्यश्व पी॒तः। अ॒यं यः सोमो॒ न्यधा᳚य्य॒स्मे तस्मा॒ इन्द्रं॑ प्र॒तिर॑मे॒म्यच्छ॑। आपा᳚न्तमन्युस्तृ॒पल॑प्रभर्मा॒ धुनिः॒ शिमी॑वा॒ञ्छरु॑माꣳ ऋजी॒षी। सोमो॒ विश्वा᳚न्यत॒सा वना॑नि॒ नार्वागिन्द्रं॑ प्रति॒माना॑नि देभुः। प्र~(६६)

%2.2.12.4
सु॑वा॒नः सोम॑ ऋत॒युश्चि॑के॒तेन्द्रा॑य॒ ब्रह्म॑ ज॒मद॑ग्नि॒रर्चन्न्॑। वृषा॑ य॒न्तासि॒ शव॑सस्तु॒रस्या॒न्तर्य॑च्छ गृण॒ते ध॒र्त्रं दृꣳ॑ह। स॒बाध॑स्ते॒ मदं॑ च शुष्म॒यं च॒ ब्रह्म॒ नरो᳚ ब्रह्म॒कृतः॑ सपर्यन्न्। अ॒र्को वा॒ यत्तु॒रते॒ सोम॑चक्षा॒स्तत्रेदिन्द्रो॑ दधते पृ॒थ्सु तु॒र्याम्। वष॑ट्ते विष्णवा॒स आ कृ॑णोमि॒ तन्मे॑ जुषस्व शिपिविष्ट ह॒व्यम्।~(६७)

%2.2.12.5
वर्ध॑न्तु त्वा सुष्टु॒तयो॒ गिरो॑ मे यू॒यं पा॑त स्व॒स्तिभिः॒ सदा॑ नः। प्र तत्ते॑ अ॒द्य शि॑पिविष्ट॒ नामा॒र्यः शꣳ॑सामि व॒युना॑नि वि॒द्वान्। तं त्वा॑ गृणामि त॒वस॒मत॑वीया॒न्क्षय॑न्तम॒स्य रज॑सः परा॒के। किमित्ते॑ विष्णो परि॒चक्ष्यं॑ भू॒त्प्रयद्व॑व॒क्षे शि॑पिवि॒ष्टो अ॑स्मि। मा वर्पो॑ अ॒स्मदप॑गूह ए॒तद्यद॒न्यरू॑पः समि॒थे ब॒भूथ॑।~(६८)

%2.2.12.6
अग्ने॒ दा दा॒शुषे॑ र॒यिं वी॒रव॑न्तं॒ परी॑णसम्। शि॒शी॒हि नः॑ सूनु॒मतः॑। दा नो॑ अग्ने श॒तिनो॒ दाः स॑ह॒स्रिणो॑ दु॒रो न वाज॒ꣴ॒ श्रुत्या॒ अपा॑वृधि। प्राची॒ द्यावा॑पृथि॒वी ब्रह्म॑णा कृधि॒ सुव॒र्ण शु॒क्रमु॒षसो॒ विदि॑द्युतुः। अ॒ग्निर्दा॒ द्रवि॑णं वी॒रपे॑शा अ॒ग्निर्\mbox{}ऋषिं॒ यः स॒हस्रा॑ स॒नोति॑। अ॒ग्निर्दि॒वि ह॒व्यमात॑ताना॒ग्नेर्धामा॑नि॒ विभृ॑ता पुरु॒त्रा। मा~(६९)

%2.2.12.7
नो॑ मर्द्धी॒रा तू भ॑र। घृ॒तं न पू॒तं त॒नूर॑रे॒पाः शुचि॒ हिर॑ण्यम्। तत्ते॑ रु॒क्मो न रो॑चत स्वधावः। उ॒भे सु॑श्चन्द्र स॒र्पिषो॒ दर्वी᳚ श्रीणीष आ॒सनि॑। उ॒तो न॒ उत्पु॑पूर्या उ॒क्थेषु॑ शवसस्पत॒ इषꣴ॑ स्तो॒तृभ्य॒ आ भ॑र। वायो॑ श॒तꣳ हरी॑णां यु॒वस्व॒ पोष्या॑णाम्। उ॒त वा॑ ते सह॒स्रिणो॒ रथ॒ आ या॑तु॒ पाज॑सा। प्र याभि॒र्-~(७०)

%2.2.12.8
यासि॑ दा॒श्वाꣳस॒मच्छा॑ नि॒युद्भि॑र्वायवि॒ष्टये॑ दुरो॒णे। नि नो॑ र॒यिꣳ सु॒भोज॑सं युवे॒ह नि वी॒रव॒द्गव्य॒मश्वि॑यं च॒ राधः॑। रे॒वती᳚र्नः सध॒माद॒ इन्द्रे॑ सन्तु तु॒विवा॑जाः। क्षु॒मन्तो॒ याभि॒र्मदे॑म। रे॒वाꣳ इद्रे॒वतः॑ स्तो॒ता स्यात्त्वाव॑तो म॒घोनः॑। प्रेदु॑ हरिवः श्रु॒तस्य॑॥~(७१)

{\anuvakamend[जि॒नोषि॑ देभुः॒ प्र ह॒व्यं ब॒भूथ॒ मा याभि॑श्चत्वारि॒ꣳ॒शच्च॑]}
%2.3.0.0

{\prashnaend[{प्र॒जा\-प॑ति॒स्ताः सृ॒ष्टा अ॒ग्नये॑ पथि॒कृते॒\-ऽग्नये॒ कामा॑या॒ग्नये\-ऽन्न॑वते वैश्वान॒रमा॑दि॒त्यं च॒रुमै॒न्द्रं च॒रुमिन्द्रा॒यान्वृ॑जव आग्नावैष्ण॒वम॒सौ सो॑मारौ॒द्रमै॒न्द्रमेका॑\-दश\-कपालꣳ हिरण्यग॒र्भो द्वाद॑श॥~(१२) प्र॒जा\-प॑तिर॒ग्नये॒ कामा॑या॒भि सम्भ॑वतो॒ यो वि॑द्विषा॒णयो॑रि॒द्ध्मे सन्न॑ह्येदाग्नावैष्ण॒वमु॒परि॑ष्टा॒द्यासि॑ दा॒श्वाꣳस॒मेक॑सप्ततिः॥~(७१) प्र॒जा\-प॑तिः॒ प्रेदु॑ हरिवः श्रु॒तस्य॑॥}]}

%%% END PRASHNA
\sect{तृतीयः प्रश्नः}\setcounter{anuvakam}{0}
\dnsub{तैत्तिरीयसंहितायां द्वितीयकाण्डे तृतीयः प्रश्नः}
%2.3.1.0
%2.3.1.1
आ॒दि॒त्येभ्यो॒ भुव॑द्वद्भ्यश्च॒रुं निर्व॑पे॒द्भूति॑काम आदि॒त्या वा ए॒तं भूत्यै॒ प्रति॑ नुदन्ते॒ यो\-ऽलं॒ भूत्यै॒ सन्भूतिं॒ न प्रा॒प्नोत्या॑दि॒त्याने॒व भुव॑द्वतः॒ स्वेन॑ भाग॒धेये॒नोप॑ धावति॒ त ए॒वैनं॒ भूतिं॑ गमयन्ति॒ भव॑त्ये॒वाऽऽदि॒त्येभ्यो॑ धा॒रय॑द्वद्भ्यश्च॒रुं निर्व॑पे॒दप॑रुद्धो वा\-ऽपरु॒ध्यमा॑नो वा\-ऽ\-ऽदि॒त्या वा अ॑परो॒द्धार॑ आदि॒त्या अ॑वगमयि॒तार॑ आदि॒त्याने॒व धा॒रय॑द्वतः॒~(१)

%2.3.1.2
स्वेन॑ भाग॒धेये॒नोप॑ धावति॒ त ए॒वैनं॑ वि॒शि दा᳚ध्रत्यनपरु॒ध्यो भ॑व॒त्यदि॒ते\-ऽनु॑ मन्य॒स्वेत्य॑परु॒ध्यमा॑नो\-ऽस्य प॒दमा द॑दीते॒यं वा अदि॑तिरि॒यमे॒वास्मै॑ रा॒ज्यमनु॑ मन्यते स॒त्याशीरित्या॑ह स॒त्यामे॒वाऽऽशिषं॑ कुरुत इ॒ह मन॒ इत्या॑ह प्र॒जा ए॒वास्मै॒ सम॑नसः करो॒त्युप॒ प्रेत॑ मरुत~-~(२)

%2.3.1.3
सुदानव ए॒ना वि॒श्पति॑ना॒भ्य॑मुꣳ राजा॑न॒मित्या॑ह मारु॒ती वै विड्ज्ये॒ष्ठो वि॒श्पति॑र्वि॒शैवैनꣳ॑ रा॒ष्ट्रेण॒ सम॑र्धयति॒ यः प॒रस्ता᳚द्ग्राम्यवा॒दी स्यात्तस्य॑ गृ॒हाद्व्री॒हीना ह॑रेच्छु॒क्लाꣴश्च॑ कृ॒ष्णाꣴश्च॒ वि चि॑नुया॒द्ये शु॒क्लाः स्युस्तमा॑दि॒त्यं च॒रुं निर्व॑पेदादि॒त्या वै दे॒वत॑या॒ विड्विश॑मे॒वाव॑ \mbox{गच्छ॒-~(३)}

%2.3.1.4
त्यव॑गतास्य॒ विडन॑वगतꣳ रा॒ष्ट्रमित्या॑हु॒र्ये कृ॒ष्णाः स्युस्तं वा॑रु॒णं च॒रुं निर्व॑पेद्वारु॒णं वै रा॒ष्ट्रमु॒भे ए॒व विशं॑ च रा॒ष्ट्रं चाव॑ गच्छति॒ यदि॒ नाव॒गच्छे॑दि॒मम॒हमा॑दि॒त्येभ्यो॑ भा॒गं निर्व॑पा॒म्यामुष्मा॑द॒मुष्यै॑ वि॒शो\-ऽव॑गन्तो॒रिति॒ निर्व॑पेदादि॒त्या ए॒वैनं॑ भाग॒धेयं॑ प्रे॒फ्सन्तो॒ विश॒मव॑~(४)

%2.3.1.5
गमयन्ति॒ यदि॒ नाव॒गच्छे॒दाश्व॑त्थान्म॒यूखा᳚न्थ्स॒प्त म॑ध्यमे॒षाया॒मुप॑ हन्या\-दि॒द\-म॒हमा॑\-दि॒त्यान्ब॑ध्ना॒म्यामुष्मा॑\-द॒मुष्यै॑ वि॒शो\-ऽव॑गन्तो॒रित्या॑\-दि॒त्या ए॒वैनं॑ ब॒द्धवी॑रा॒ विश॒मव॑ गमयन्ति॒ यदि॒ नाव॒गच्छे॑दे॒तमे॒वा\-ऽऽदि॒त्यं च॒रुं निर्व॑पेदि॒ध्मे\-ऽपि॑ म॒यूखा॒न्थ्सं न॑ह्येदनपरु॒ध्यमे॒वाव॑ गच्छ॒त्याश्व॑त्था भवन्ति म॒रुतां॒ वा ए॒तदोजो॒ यद॑श्व॒त्थ ओज॑सै॒व विश॒मव॑ गच्छति स॒प्त भ॑वन्ति स॒प्तग॑णा॒ वै म॒रुतो॑ गण॒श ए॒व विश॒मव॑ गच्छति॥~(५)

%2.3.2.0
{\anuvakamend[{धा॒रय॑द्वतो मरुतो गच्छति॒ विश॒मवै॒तद॒ष्टाद॑श च}]}%~(१)

%2.3.2.1
दे॒वा वै मृ॒त्योर॑बिभयु॒स्ते प्र॒जा\-प॑ति॒मुपा॑धाव॒न्तेभ्य॑ ए॒तां प्रा॑जाप॒त्याꣳ श॒तकृ॑ष्णलां॒ निर॑वप॒त्तयै॒वैष्व॒मृत॑मदधा॒द्यो मृ॒त्योर्बि॑भी॒यात्तस्मा॑ ए॒तां प्रा॑जाप॒त्याꣳ श॒तकृ॑ष्णलां॒ निर्व॑पेत्प्र॒जा\-प॑तिमे॒व स्वेन॑ भाग॒धेये॒नोप॑ धावति॒ स ए॒वास्मि॒न्नायु॑र्दधाति॒ सर्व॒मायु॑रेति श॒तकृ॑ष्णला भवति श॒तायुः॒ पुरु॑षः श॒तेन्द्रि॑य॒ आयु॑ष्ये॒वेन्द्रि॒ये~(६)

%2.3.2.2
प्रति॑ तिष्ठति घृ॒ते भ॑व॒त्यायु॒र्वै घृ॒तम॒मृत॒ꣳ॒ हिर॑ण्य॒मायु॑श्चै॒वा\-स्मा॑ अ॒मृतं॑ च स॒मीची॑ दधाति च॒त्वारि॑चत्वारि कृ॒ष्णला॒न्यव॑ द्यति चतुरव॒त्तस्याऽऽप्त्या॑ एक॒धा ब्र॒ह्मण॒ उप॑ हरत्येक॒धैव यज॑मान॒ आयु॑र्दधात्य॒सावा॑दि॒त्यो न व्य॑रोचत॒ तस्मै॑ दे॒वाः प्राय॑श्चित्ति\-मैच्छ॒न्तस्मा॑ ए॒तꣳ सौ॒र्यं च॒रुं निर॑वप॒न्तेनै॒वास्मि॒-~(७)

%2.3.2.3
न्रुच॑मदधु॒र्यो ब्र॑ह्मवर्च॒सका॑मः॒ स्यात्तस्मा॑ ए॒तꣳ सौ॒र्यं च॒रुं निर्व॑पेद॒मुमे॒वाऽऽदि॒त्यꣴ स्वेन॑ भाग॒धेये॒नोप॑ धावति॒ स ए॒वास्मि॑न्ब्रह्मवर्च॒सं द॑धाति ब्रह्मवर्च॒स्ये॑व भ॑वत्युभ॒यतो॑ रु॒क्मौ भ॑वत उभ॒यत॑ ए॒वास्मि॒न्रुचं॑ दधाति प्रया॒जेप्र॑याजे कृ॒ष्णलं॑ जुहोति दि॒ग्भ्य ए॒वास्मै᳚ ब्रह्मवर्च॒समव॑ रुन्ध आग्ने॒यम॒ष्टा\-क॑पालं॒ निर्व॑पेथ्सावि॒त्रं द्वाद॑श\-कपालं॒ भूम्यै॑~(८)

%2.3.2.4
च॒रुं यः का॒मये॑त॒ हिर॑ण्यं विन्देय॒ हिर॑ण्यं॒ मोप॑ नमे॒दिति॒ यदा᳚ग्ने॒यो भव॑त्याग्ने॒यं वै हिर॑ण्यं॒ यस्यै॒व हिर॑ण्यं॒ तेनै॒वैन॑द्विन्दते सावि॒त्रो भ॑वति सवि॒तृप्र॑सूत ए॒वैन॑द्विन्दते॒ भूम्यै॑ च॒रुर्भ॑वत्य॒स्यामे॒वैन॑द्विन्दत॒ उपै॑न॒ꣳ॒ हिर॑ण्यं नमति॒ वि वा ए॒ष इ॑न्द्रि॒येण॑ वी॒र्ये॑णर्ध्यते॒ यो हिर॑ण्यं वि॒न्दत॑ ए॒ता-~(९)

%2.3.2.5
मे॒व निर्व॑पे॒द्धिर॑ण्यं वि॒त्त्वा नेन्द्रि॒येण॑ वी॒र्ये॑ण॒ व्यृ॑ध्यत ए॒तामे॒व निर्व॑पे॒द्यस्य॒ हिर॑ण्यं॒ नश्ये॒द्यदा᳚ग्ने॒यो भव॑त्याग्ने॒यं वै हिर॑ण्यं॒ यस्यै॒व हिर॑ण्यं॒ तेनै॒वैन॑द्विन्दति सावि॒त्रो भ॑वति सवि॒तृप्र॑सूत ए॒वैन॑द्विन्दति॒ भूम्यै॑ च॒रुर्भ॑वत्य॒स्यां वा ए॒तन्न॑श्यति॒ यन्नश्य॑त्य॒स्यामे॒वैन॑द्विन्द॒ती\-न्द्र॒-~(१०)

%2.3.2.6
स्त्वष्टुः॒ सोम॑मभी॒षहा॑पिब॒थ्स विष्व॒ङ्व्या᳚र्च्छ॒थ्स इ॑न्द्रि॒येण॑ सोमपी॒थेन॒ व्या᳚र्ध्यत॒ स यदू॒र्ध्वमु॒दव॑मी॒त्ते श्या॒माका॑ अभव॒न्थ्स प्र॒जा\-प॑ति॒मुपा॑धाव॒त्तस्मा॑ ए॒तꣳ सो॑मे॒न्द्रꣴ श्या॑मा॒कं च॒रुं निर॑वप॒त्तेनै॒वास्मि॑न्निन्द्रि॒यꣳ सो॑मपी॒थम॑दधा॒द्वि वा ए॒ष इ॑न्द्रि॒येण॑ सोमपी॒थेन॑र्ध्यते॒ यः सोमं॒ वमि॑ति॒ यः सो॑मवा॒मी स्यात्तस्मा॑-~(११)

%2.3.2.7
ए॒तꣳ सो॑मे॒न्द्रꣴ श्या॑मा॒कं च॒रुं निर्व॑पे॒थ्सोमं॑ चै॒वेन्द्रं॑ च॒ स्वेन॑ भाग॒धेये॒नोप॑ धावति॒ तावे॒वास्मि॑न्निन्द्रि॒यꣳ सो॑मपी॒थं ध॑त्तो॒ नेन्द्रि॒येण॑ सोमपी॒थेन॒ व्यृ॑ध्यते॒ यथ्सौ॒म्यो भव॑ति सोमपी॒थमे॒वाव॑ रुन्धे॒ यदै॒न्द्रो भव॑तीन्द्रि॒यं वै सो॑मपी॒थ इ॑न्द्रि॒यमे॒व सो॑मपी॒थमव॑ रुन्धे श्यामा॒को भ॑वत्ये॒ष वाव स सोमः॑~(१२)

%2.3.2.8
सा॒क्षादे॒व सो॑मपी॒थमव॑ रुन्धे॒\-ऽग्नये॑ दा॒त्रे पु॑रो॒डाश॑म॒ष्टा\-क॑पालं॒ निर्व॑पे॒दिन्द्रा॑य प्रदा॒त्रे पु॑रो॒डाश॒मेका॑\-दश\-कपालं प॒शुका॑मो॒\-ऽग्निरे॒वास्मै॑ प॒शून्प्र॑ज॒नय॑ति वृ॒द्धानिन्द्रः॒ प्र य॑च्छति॒ दधि॒ मधु॑ घृ॒तमापो॑ धा॒ना भ॑वन्त्ये॒तद्वै प॑शू॒नाꣳ रू॒पꣳ रू॒पेणै॒व प॒शूनव॑ रुन्धे पञ्चगृही॒तं भ॑वति॒ पाङ्क्ता॒ हि प॒शवो॑ बहुरू॒पं भ॑वति बहुरू॒पा हि प॒शवः॒~(१३)

%2.3.2.9
समृ॑द्ध्यै प्राजाप॒त्यं भ॑वति प्राजाप॒त्या वै प॒शवः॑ प्र॒जा\-प॑तिरे॒वास्मै॑ प॒शून्प्र ज॑नयत्या॒त्मा वै पुरु॑षस्य॒ मधु॒ यन्मध्व॒ग्नौ जु॒होत्या॒त्मान॑मे॒व तद्यज॑मानो॒\-ऽग्नौ प्र द॑धाति प॒ङ्क्त्यौ॑ याज्यानुवा॒क्ये॑ भवतः॒ पाङ्क्तः॒ पुरु॑षः॒ पाङ्क्ताः᳚ प॒शव॑ आ॒त्मान॑मे॒व मृ॒त्योर्नि॒ष्क्रीय॑ प॒शूनव॑ रुन्धे॥~(१४)

%2.3.3.0
{\anuvakamend[{इ॒न्द्रि॒ये᳚\-ऽस्मि॒न्भूम्या॑ ए॒तामिन्द्रः॒ स्यात्तस्मै॒ सोमो॑ बहुरू॒पा हि प॒शव॒ एक॑चत्वारिꣳशच्च}]}%~(२)

%2.3.3.1
दे॒वा वै स॒त्रमा॑स॒तर्द्धि॑परिमितं॒ यश॑स्कामा॒स्तेषा॒ꣳ॒ सोम॒ꣳ॒ राजा॑नं॒ यश॑ आर्च्छ॒थ्स गि॒रिमुदै॒त्तम॒ग्निरनूदै॒त्ताव॒ग्नी\-षोमौ॒ सम॑भवतां॒ ताविन्द्रो॑ य॒ज्ञवि॑भ्र॒ष्टो\-ऽनु॒ परै॒त्ताव॑ब्रवीद्या॒जय॑तं॒ मेति॒ तस्मा॑ ए॒तामिष्टिं॒ निर॑वपता\-माग्ने॒य\-म॒ष्टाक॑पाल\-मै॒न्द्रमेका॑\-दश\-कपालꣳ सौ॒म्यं च॒रुं तयै॒वास्मि॒न्तेज॑~-~(१५)

%2.3.3.2
इन्द्रि॒यं ब्र॑ह्मवर्च॒सम॑धत्तां॒ यो य॒ज्ञवि॑भ्रष्टः॒ स्यात्तस्मा॑ ए॒तामिष्टिं॒ निर्व॑पेदाग्ने॒यम॒ष्टाक॑पालमै॒न्द्रमेका॑\-दश\-कपालꣳ सौ॒म्यं च॒रुं यदा᳚ग्ने॒यो भव॑ति॒ तेज॑ ए॒वास्मि॒न्तेन॑ दधाति॒ यदै॒न्द्रो भव॑तीन्द्रि॒यमे॒वास्मि॒न्तेन॑ दधाति॒ यथ्सौ॒म्यो ब्र॑ह्मवर्च॒सं तेना᳚ऽऽग्ने॒यस्य॑ च सौ॒म्यस्य॑ चै॒न्द्रे स॒माश्ले॑षये॒त्तेज॑श्चै॒वास्मि॑न्ब्रह्मवर्च॒सं च॑ स॒मीची॑~(१६)

%2.3.3.3
दधात्यग्नीषो॒मीय॒मेका॑\-दश\-कपालं॒ निर्व॑पे॒द्यं कामो॒ नोप॒नमे॑दा\-ग्ने॒यो वै ब्रा᳚ह्म॒णः स सोमं॑ पिबति॒ स्वामे॒व दे॒वता॒ꣴ॒ स्वेन॑ भाग॒धेये॒नोप॑ धावति॒ सैवैनं॒ कामे॑न॒ सम॑र्धय॒त्युपै॑नं॒ कामो॑ नमत्यग्नीषो॒मीय॑म॒ष्टा\-क॑पालं॒ निर्व॑पेद्ब्रह्मवर्च॒सका॑मो॒\-ऽग्नी\-षोमा॑वे॒व स्वेन॑ भाग॒धेये॒नोप॑ धावति॒ तावे॒वास्मि॑न्ब्रह्मवर्च॒सं ध॑त्तो ब्रह्मवर्च॒स्ये॑व~(१७)

%2.3.3.4
भ॑वति यद॒ष्टाक॑पाल॒स्तेना᳚ऽऽग्ने॒यो यच्छ्या॑मा॒कस्तेन॑ सौ॒म्यः समृ॑द्ध्यै॒ सोमा॑य वा॒जिने᳚ श्यामा॒कं च॒रुं निर्व॑पे॒द्यः क्लैव्या᳚द्बिभी॒याद्रेतो॒ हि वा ए॒तस्मा॒द्वाजि॑नमप॒क्राम॒त्यथै॒ष क्लैब्या᳚द्बिभाय॒ सोम॑मे॒व वा॒जिन॒ꣴ॒ स्वेन॑ भाग॒धेये॒नोप॑ धावति॒ स ए॒वास्मि॒न्रेतो॒ वाजि॑नं दधाति॒ न क्ली॒बो भ॑वति ब्राह्मणस्प॒त्यमेका॑\-दश\-कपालं॒ निर्व॑पे॒द्ग्राम॑कामो॒~-~(१८)

%2.3.3.5
ब्रह्म॑ण॒स्पति॑मे॒व स्वेन॑ भाग॒धेये॒नोप॑ धावति॒ स ए॒वास्मै॑ सजा॒तान्प्र य॑च्छति ग्रा॒म्ये॑व भ॑वति ग॒णव॑ती याज्यानुवा॒क्ये॑ भवतः सजा॒तैरे॒वैनं॑ ग॒णव॑न्तं करोत्ये॒तामे॒व निर्व॑पे॒द्यः का॒मये॑त॒ ब्रह्म॒न्विशं॒ वि ना॑शयेय॒मिति॑ मारु॒ती या᳚ज्यानुवा॒क्ये॑ कुर्या॒द्ब्रह्म॑न्ने॒व विशं॒ वि ना॑शयति॥~(१९)

%2.3.4.0
{\anuvakamend[{तेजः॑ स॒मीची᳚ ब्रह्मवर्च॒स्ये॑व ग्राम॑काम॒स्त्रिच॑त्वारिꣳशच्च}]}%~(३)

%2.3.4.1
अ॒र्य॒म्णे च॒रुं निर्व॑पेथ्सुव॒र्गका॑मो॒\-ऽसौ वा आ॑दि॒त्यो᳚\-ऽर्य॒मा\-ऽर्य॒मण॑मे॒व स्वेन॑ भाग॒धेये॒नोप॑ धावति॒ स ए॒वैनꣳ॑ सुव॒र्गं लो॒कं ग॑मयत्यर्य॒म्णे च॒रुं निर्व॑पे॒द्यः का॒मये॑त॒ दान॑कामा मे प्र॒जाः स्यु॒रित्य॒सौ वा आ॑दि॒त्यो᳚\-ऽर्य॒मा यः खलु॒ वै ददा॑ति॒ सो᳚\-ऽर्य॒मा\-ऽर्य॒मण॑मे॒व स्वेन॑ भाग॒धेये॒नोप॑ धावति॒ स ए॒वा-~(२०)

%2.3.4.2
स्मै॒ दान॑कामाः प्र॒जाः क॑रोति॒ दान॑कामा अस्मै प्र॒जा भ॑वन्त्यर्य॒म्णे च॒रुं निर्व॑पे॒द्यः का॒मये॑त स्व॒स्ति ज॒नता॑मिया॒मित्य॒सौ वा आ॑दि॒त्यो᳚\-ऽर्य॒मा\-ऽर्य॒मण॑मे॒व स्वेन॑ भाग॒धेये॒नोप॑ धावति॒ स ए॒वैनं॒ तद्ग॑मयति॒ यत्र॒ जिग॑मिष॒तीन्द्रो॒ वै दे॒वाना॑मानुजाव॒र आ॑सी॒थ्स प्र॒जा\-प॑ति॒मुपा॑धाव॒त्तस्मा॑ ए॒तमै॒न्द्रमा॑नुषू॒कमेका॑\-दश\-कपालं॒ नि-~(२१)

%2.3.4.3
र॑वप॒त्तेनै॒वैन॒मग्रं॑ दे॒वता॑नां॒ पर्य॑णयद्बु॒ध्नव॑ती॒ अग्र॑वती याज्यानुवा॒क्ये॑ अकरोद्बु॒ध्नादे॒वैन॒मग्रं॒ पर्य॑णय॒द्यो रा॑ज॒न्य॑ आनुजाव॒रः स्यात्तस्मा॑ ए॒तमै॒न्द्रमा॑नुषू॒कमेका॑\-दश\-कपालं॒ निर्व॑पे॒दिन्द्र॑मे॒व स्वेन॑ भाग॒धेये॒नोप॑ धावति॒ स ए॒वैन॒मग्रꣳ॑ समा॒नानां॒ परि॑ णयति बु॒ध्नव॑ती॒ अग्र॑वती याज्यानुवा॒क्ये॑ भवतो बु॒ध्नादे॒वैन॒मग्रं॒-~(२२)

%2.3.4.4
परि॑ णयत्यानुषू॒को भ॑वत्ये॒षा ह्ये॑तस्य॑ दे॒वता॒ य आ॑नुजाव॒रः समृ॑द्ध्यै॒ यो ब्रा᳚ह्म॒ण आ॑नुजाव॒रः स्यात्तस्मा॑ ए॒तं बा॑र्\mbox{}हस्प॒त्यमा॑नुषू॒कं च॒रुं निर्व॑पे॒द्बृह॒स्पति॑मे॒व स्वेन॑ भाग॒धेये॒नोप॑ धावति॒ स ए॒वैन॒मग्रꣳ॑ समा॒नानां॒ परि॑ णयति बु॒ध्नव॑ती॒ अग्र॑वती याज्यानुवा॒क्ये॑ भवतो बु॒ध्नादे॒वैन॒मग्रं॒ परि॑ णयत्यानुषू॒को भ॑वत्ये॒षा ह्ये॑तस्य॑ दे॒वता॒ य आ॑नुजाव॒रः समृ॑द्ध्यै॥~(२३)

%2.3.5.0
{\anuvakamend[{ए॒व निरग्र॑मे॒तस्य॑ च॒त्वारि॑ च}]}%~(४)

%2.3.5.1
प्र॒जाप॑ते॒स्त्रय॑स्त्रिꣳशद्दुहि॒तर॑ आस॒न्ताः सोमा॑य॒ राज्ञे॑\-ऽददा॒त्तासाꣳ॑ रोहि॒णीमुपै॒त्ता ईर्ष्य॑न्तीः॒ पुन॑रगच्छ॒न्ता अन्वै॒त्ताः पुन॑रयाचत॒ ता अ॑स्मै॒ न पुन॑रददा॒थ्सो᳚\-ऽब्रवीदृ॒तम॑मीष्व॒ यथा॑ समाव॒च्छ उ॑पै॒ष्याम्यथ॑ ते॒ पुन॑र्दास्या॒मीति॒ स ऋ॒तमा॑मी॒त्ता अ॑स्मै॒ पुन॑रददा॒त्तासाꣳ॑ रोहि॒णीमे॒वोपै॒-~(२४)

%2.3.5.2
त्तं यक्ष्म॑ आर्च्छ॒द्राजा॑नं॒ यक्ष्म॑ आर॒दिति॒ तद्रा॑जय॒क्ष्मस्य॒ जन्म॒ यत्पापी॑या॒नभ॑व॒त्तत्पा॑पय॒क्ष्मस्य॒ यज्जा॒याभ्यो\-ऽवि॑न्द॒त्तज्जा॒येन्य॑स्य॒ य ए॒वमे॒तेषां॒ यक्ष्मा॑णां॒ जन्म॒ वेद॒ नैन॑मे॒ते यक्ष्मा॑ विन्दन्ति॒ स ए॒ता ए॒व न॑म॒स्यन्नुपा॑धाव॒त्ता अ॑ब्रुव॒न्वरं॑ वृणामहै समाव॒च्छ ए॒व न॒ उपा॑य॒ इति॒ तस्मा॑ ए॒त-~(२५)

%2.3.5.3
मा॑दि॒त्यं च॒रुं निर॑वप॒न्तेनै॒वैनं॑ पा॒पाथ्स्रामा॑दमुञ्च॒न्॒ यः पा॑पय॒क्ष्मगृ॑हीतः॒ स्यात्तस्मा॑ ए॒तमा॑दि॒त्यं च॒रुं निर्व॑पेदादि॒त्याने॒व स्वेन॑ भाग॒धेये॒नोप॑ धावति॒ त ए॒वैनं॑ पा॒पाथ्स्रामा᳚न्मुञ्चन्त्यमावा॒स्या॑यां॒ निर्व॑पेद॒मुमे॒वैन॑मा॒प्याय॑मान॒मन्वा प्या॑ययति॒ नवो॑नवो भवति॒ जाय॑\-मान॒ इति॑ पुरोनुवा॒क्या॑ भव॒त्यायु॑रे॒वास्मि॒न्तया॑ दधाति॒ यमा॑\-दि॒त्या अ॒ꣳ॒शुमा᳚प्या॒यय॒न्तीति॑ या॒ज्यैवैन॑मे॒तया᳚ प्याययति॥~(२६)

%2.3.6.0
{\anuvakamend[{ए॒वोपै॒तम॑स्मि॒न्त्रयो॑दश च}]}%~(५)

%2.3.6.1
प्र॒जा\-प॑तिर्दे॒वेभ्यो॒\-ऽन्नाद्यं॒ व्यादि॑श॒थ्सो᳚\-ऽब्रवी॒द्यदि॒माँल्लो॒का\-न॒भ्य॑ति॒\-रिच्या॑तै॒ तन्ममा॑ऽस॒दिति॒ तदि॒माँल्लो॒का\-न॒भ्यत्य॑रिच्य॒तेन्द्र॒ꣳ॒ राजा॑न॒\-मिन्द्र॑मधिरा॒जमिन्द्रꣴ॑ स्व॒राजा॑नं॒ ततो॒ वै स इ॒माँल्लो॒काꣴस्त्रे॒धा\-दु॑ह॒त्तत्त्रि॒धातो᳚स्त्रिधातु॒त्वं यं का॒मये॑ताऽन्ना॒दः स्या॒दिति॒ तस्मा॑ ए॒तं त्रि॒धातुं॒ निर्व॑पे॒दिन्द्रा॑य॒ राज्ञे॑ पुरो॒डाश॒-~(२७)

%2.3.6.2
मेका॑\-दश\-कपाल॒मिन्द्रा॑याऽधिरा॒जायेन्द्रा॑य स्व॒राज्ञे॒\-ऽयं वा इन्द्रो॒ राजा॒ऽयमिन्द्रो॑\-ऽधिरा॒जो॑\-ऽसाविन्द्रः॑ स्व॒राडि॒माने॒व लो॒कान्थ्स्वेन॑ भाग॒धेये॒नोप॑ धावति॒ त ए॒वास्मा॒ अन्नं॒ प्र य॑च्छन्त्यन्ना॒द ए॒व भ॑वति॒ यथा॑ व॒थ्सेन॒ प्रत्तां॒ गां दु॒ह ए॒वमे॒वेमाँल्लो॒कान्प्रत्ता॒न्काम॑म॒न्नाद्यं॑ दुह उत्ता॒नेषु॑ क॒पाले॒ष्वधि॑ श्रय॒त्यया॑तयामत्वाय॒ त्रयः॑ पुरो॒डाशा॑ भवन्ति॒ त्रय॑ इ॒मे लो॒का ए॒षां लो॒काना॒माप्त्या॒ उत्त॑रउत्तरो॒ ज्याया᳚न्भवत्ये॒वमि॑व॒ हीमे लो॒काः समृ॑द्ध्यै॒ सर्वे॑षामभिग॒मय॒न्नव॑ द्य॒त्यछ॑म्बट्कारं व्य॒त्यास॒मन्वा॒हाऽनि॑र्दाहाय॥~(२८)

%2.3.7.0
{\anuvakamend[{पु॒रो॒डाश॒न्त्रयः॒ षड्विꣳ॑शतिश्च}]}%~(६)

%2.3.7.1
दे॒वा॒सु॒राः संय॑त्ता आस॒न्तां दे॒वानसु॑रा अजय॒न्ते दे॒वाः प॑राजिग्या॒ना असु॑राणां॒ वैश्य॒मुपा॑य॒न्तेभ्य॑ इन्द्रि॒यं वी॒र्य॑मपा᳚\-क्राम॒त्तदिन्द्रो॑\-ऽचाय॒त्तदन्वपा᳚\-क्राम॒त्तद॑व॒रुधं॒ नाश॑क्नो॒त्तद॑स्मा\-दभ्य॒र्धो॑\-ऽचर॒थ्स प्र॒जा\-प॑ति॒मुपा॑धाव॒त्तमे॒तया॒ सर्व॑पृष्ठया\-ऽयाजय॒त्त\-यै॒वास्मि॑न्निन्द्रि॒यं वी॒र्य॑मदधा॒द्य इ॑न्द्रि॒यका॑मो~-~(२९)

%2.3.7.2
वी॒र्य॑कामः॒ स्यात्तमे॒तया॒ सर्व॑पृष्ठया याजयेदे॒ता ए॒व दे॒वताः॒ स्वेन॑ भाग॒धेये॒नोप॑ धावति॒ ता ए॒वास्मि॑न्निन्द्रि॒यं वी॒र्यं॑ दधति॒ यदिन्द्रा॑य॒ राथ॑न्तराय नि॒र्वप॑ति॒ यदे॒वाग्नेस्तेज॒स्तदे॒वाव॑ रुन्धे॒ यदिन्द्रा॑य॒ बार्\mbox{}ह॑ताय॒ यदे॒वेन्द्र॑स्य॒ तेज॒स्तदे॒वाव॑ रुन्धे॒ यदिन्द्रा॑य वैरू॒पाय॒ यदे॒व स॑वि॒तुस्तेज॒स्त-~(३०)

%2.3.7.3
दे॒वाव॑ रुन्धे॒ यदिन्द्रा॑य वैरा॒जाय॒ यदे॒व धा॒तुस्तेज॒स्तदे॒वाव॑ रुन्धे॒ यदिन्द्रा॑य शाक्व॒राय॒ यदे॒व म॒रुतां॒ तेज॒स्तदे॒वाव॑ रुन्धे॒ यदिन्द्रा॑य रैव॒ताय॒ यदे॒व बृह॒स्पते॒स्तेज॒स्तदे॒वाव॑ रुन्ध ए॒ताव॑न्ति॒ वै तेजाꣳ॑सि॒ तान्ये॒वाव॑ रुन्ध उत्ता॒नेषु॑ क॒पाले॒ष्वधि॑ श्रय॒त्यया॑तयामत्वाय॒ द्वाद॑श\-कपालः पुरो॒डाशो॑~-~(३१)

%2.3.7.4
भवति वैश्वदेव॒त्वाय॑ सम॒न्तं प॒र्यव॑द्यति सम॒न्तमे॒वेन्द्रि॒यं वी॒र्यं॑ यज॑माने दधाति व्य॒त्यास॒मन्वा॒हानि॑र्दाहा॒याश्व॑ ऋष॒भो वृ॒ष्णिर्ब॒स्तः सा दक्षि॑णा वृष॒त्वायै॒तयै॒व य॑जेताभिश॒स्यमा॑न ए॒ताश्चेद्वा अ॑स्य दे॒वता॒ अन्न॑म॒दन्त्य॒दन्त्यु॑वे॒वास्य॑ मनु॒ष्याः᳚॥~(३२)

%2.3.8.0
{\anuvakamend[{इ॒न्द्रि॒यका॑मः सवि॒तुस्तेज॒स्तत्पु॑रो॒डाशो॒\-ऽष्टात्रिꣳ॑शच्च}]}%~(७)

%2.3.8.1
रज॑नो॒ वै कौ॑णे॒यः क्र॑तु॒जितं॒ जान॑किं चक्षु॒र्वन्य॑मया॒त्तस्मा॑ ए॒तामिष्टिं॒ निर॑वपद॒ग्नये॒ भ्राज॑स्वते पुरो॒डाश॑\-म॒ष्टाक॑पालꣳ सौ॒र्यं च॒रुम॒ग्नये॒ भ्राज॑स्वते पुरो॒डाश॑\-म॒ष्टाक॑पालं॒ तयै॒वास्मि॒ञ्चक्षु॑\-र\-दधा॒द्यश्चक्षु॑ष्कामः॒ स्यात्तस्मा॑ ए॒तामिष्टिं॒ निर्व॑पेद॒ग्नये॒ भ्राज॑स्वते पुरो॒डाश॑\-म॒ष्टाक॑पालꣳ सौ॒र्यं च॒रुम॒ग्नये॒ भ्राज॑स्वते पुरो॒डाश॑\-म॒ष्टाक॑पालम॒ग्नेर्वै चक्षु॑षा मनु॒ष्या॑ वि~(३३)

%2.3.8.2
प॑श्यन्ति॒ सूर्य॑स्य दे॒वा अ॒ग्निं चै॒व सूर्यं॑ च॒ स्वेन॑ भाग॒धेये॒नोप॑ धावति॒ तावे॒वास्मि॒ञ्चक्षु॑र्धत्त॒श्चक्षु॑ष्माने॒व भ॑वति॒ यदा᳚ग्ने॒यौ भव॑त॒श्चक्षु॑षी ए॒वास्मि॒न्तत्प्रति॑ दधाति॒ यथ्सौ॒र्यो नासि॑कां॒ तेना॒भितः॑ सौ॒र्यमा᳚ग्ने॒यौ भ॑वत॒स्तस्मा॑द॒भितो॒ नासि॑कां॒ चक्षु॑षी॒ तस्मा॒न्नासि॑कया॒ चक्षु॑षी॒ विधृ॑ते समा॒नी या᳚ज्यानुवा॒क्ये॑ भवतः समा॒नꣳ हि चक्षुः॒ समृ॑द्ध्या॒ उदु॒ त्यं जा॒तवे॑दसꣳ स॒प्त त्वा॑ ह॒रितो॒ रथे॑ चि॒त्रं दे॒वाना॒मुद॑गा॒दनी॑क॒मिति॒ पिण्डा॒न्प्र य॑च्छति॒ चक्षु॑रे॒वास्मै॒ प्र य॑च्छति॒ यदे॒व तस्य॒ तत्॥~(३४)

%2.3.9.0
{\anuvakamend[{वि ह्य॑ष्टाविꣳ॑शतिश्च}]}%~(८)

%2.3.9.1
ध्रु॒वो॑\-ऽसि ध्रु॒वो॑\-ऽहꣳ स॑जा॒तेषु॑ भूयासं॒ धीर॒श्चेत्ता॑ वसु॒विद्ध्रु॒वो॑\-ऽसि ध्रु॒वो॑\-ऽहꣳ स॑जा॒तेषु॑ भूयासमु॒ग्रश्चेत्ता॑ वसु॒विद्ध्रु॒वो॑\-ऽसि ध्रु॒वो॑\-ऽहꣳ स॑जा॒तेषु॑ भूयासमभि॒भूश्चेत्ता॑ वसु॒विदाम॑नम॒स्याम॑नस्य देवा॒ ये स॑जा॒ताः कु॑मा॒राः सम॑नस॒स्तान॒हं का॑मये हृ॒दा ते मां का॑मयन्ताꣳ हृ॒दा तान्म॒ आम॑नसः कृधि॒ स्वाहाऽऽम॑नम॒-~(३५)

%2.3.9.2
स्याम॑नस्य देवा॒ याः स्त्रियः॒ सम॑नस॒स्ता अ॒हं का॑मये हृ॒दा ता मां का॑मयन्ताꣳ हृ॒दा ता म॒ आम॑नसः कृधि॒ स्वाहा॑ वैश्वदे॒वीꣳ सा᳚ङ्ग्रह॒णीं निर्व॑पे॒द्ग्राम॑कामो वैश्वदे॒वा वै स॑जा॒ता विश्वा॑ने॒व दे॒वान्थ्स्वेन॑ भाग॒धेये॒नोप॑ धावति॒ त ए॒वास्मै॑ सजा॒तान्प्र य॑च्छन्ति ग्रा॒म्ये॑व भ॑वति साङ्ग्रह॒णी भ॑वति मनो॒ग्रह॑णं॒ वै स॒ङ्ग्रह॑णं॒ मन॑ ए॒व स॑जा॒तानां᳚~(३६)

%2.3.9.3
गृह्णाति ध्रु॒वो॑\-ऽसि ध्रु॒वो॑\-ऽहꣳ स॑जा॒तेषु॑ भूयास॒मिति॑ परि॒धीन्परि॑ दधात्या॒शिष॑मे॒वैतामा शा॒स्ते\-ऽथो॑ ए॒तदे॒व सर्वꣳ॑ सजा॒तेष्वधि॑ भवति॒ यस्यै॒वं वि॒दुष॑ ए॒ते प॑रि॒धयः॑ परिधी॒यन्त॒ आम॑नम॒स्याम॑नस्य देवा॒ इति॑ ति॒स्र आहु॑तीर्जुहोत्ये॒ताव॑न्तो॒ वै स॑जा॒ता ये म॒हान्तो॒ ये क्षु॑ल्ल॒का याः स्त्रिय॒स्ताने॒वाव॑ रुन्धे॒ त ए॑न॒मव॑रुद्धा॒ उप॑ तिष्ठन्ते॥~(३७)

%2.3.10.0
{\anuvakamend[{स्वाहाम॑नमसि सजा॒तानाꣳ॑ रुन्धे॒ पञ्च॑ च}]}%~(९)

%2.3.10.1
यन्नव॒मैत्तन्नव॑नीतमभव॒द्यदस॑र्प॒त्तथ्स॒र्पिर॑भव॒द्यदध्रि॑यत॒ तद्-घृ॒तम॑\-भव\-द॒श्विनोः᳚ प्रा॒णो॑\-ऽसि॒ तस्य॑ ते दत्तां॒ ययोः᳚ प्रा॒णो\-ऽसि॒ स्वाहेन्द्र॑स्य प्रा॒णो॑\-ऽसि॒ तस्य॑ ते ददातु॒ यस्य॑ प्रा॒णो\-ऽसि॒ स्वाहा॑ मि॒त्रावरु॑णयोः प्रा॒णो॑\-ऽसि॒ तस्य॑ ते दत्तां॒ ययोः᳚ प्रा॒णो\-ऽसि॒ स्वाहा॒ विश्वे॑षां दे॒वानां᳚ प्रा॒णो॑\-ऽसि॒~(३८)

%2.3.10.2
तस्य॑ ते ददतु॒ येषां᳚ प्रा॒णो\-ऽसि॒ स्वाहा॑ घृ॒तस्य॒ धारा॑म॒मृत॑स्य॒ पन्था॒मिन्द्रे॑ण द॒त्तां प्रय॑तां म॒रुद्भिः॑। तत्त्वा॒ विष्णुः॒ पर्य॑पश्य॒त्तत्त्वेडा॒ गव्यैर॑यत्। पा॒व॒मा॒नेन॑ त्वा॒ स्तोमे॑न गाय॒त्रस्य॑ वर्त॒न्योपा॒ꣳ॒शोर्वी॒र्ये॑ण दे॒वस्त्वा॑ सवि॒तोथ्सृ॑जतु जी॒वात॑वे जीवन॒स्यायै॑ बृहद्रथन्त॒रयो᳚स्त्वा॒ स्तोमे॑न त्रि॒ष्टुभो॑ वर्त॒न्या शु॒क्रस्य॑ वी॒र्ये॑ण दे॒वस्त्वा॑ सवि॒तो-~(३९)

%2.3.10.3
थ्सृ॑जतु जी॒वात॑वे जीवन॒स्याया॑ अ॒ग्नेस्त्वा॒ मात्र॑या॒ जग॑त्यै वर्त॒न्याग्र॑य॒णस्य॑ वी॒र्ये॑ण दे॒वस्त्वा॑ सवि॒तोथ्सृ॑जतु जी॒वात॑वे जीवन॒स्याया॑ इ॒मम॑ग्न॒ आयु॑षे॒ वर्च॑से कृधि प्रि॒यꣳ रेतो॑ वरुण सोम राजन्न्। मा॒तेवा᳚स्मा अदिते॒ शर्म॑ यच्छ॒ विश्वे॑ देवा॒ जर॑दष्टि॒र्यथास॑त्। अ॒ग्निरायु॑ष्मा॒न्थ्स वन॒स्पति॑भि॒रायु॑ष्मा॒न्तेन॒ त्वायु॒षाऽऽयु॑ष्मन्तं करोमि॒ सोम॒ आयु॑ष्मा॒न्थ्स ओष॑धीभिर्य॒ज्ञ आयु॑ष्मा॒न्थ्स दक्षि॑णाभि॒र्ब्रह्माऽऽयु॑ष्म॒त्तद्ब्रा᳚ह्म॒णैरायु॑ष्मद्दे॒वा आयु॑ष्मन्त॒स्ते॑\-ऽमृते॑न पि॒तर॒ आयु॑ष्मन्त॒स्ते स्व॒धयाऽऽयु॑ष्मन्त॒स्तेन॒ त्वायु॒षाऽऽयु॑ष्मन्तं करोमि॥~(४०)

%2.3.11.0
{\anuvakamend[{विश्वे॑षां दे॒वानां᳚ प्रा॒णो॑\-ऽसि त्रि॒ष्टुभो॑ वर्त॒न्या शु॒क्रस्य॑ वी॒र्ये॑ण दे॒वस्त्वा॑ सवि॒तोथ्सोम॒ आयु॑ष्मा॒न्पञ्च॑विꣳशतिश्च}]}%॥10॥

%2.3.11.1
अ॒ग्निं वा ए॒तस्य॒ शरी॑रं गच्छति॒ सोम॒ꣳ॒ रसो॒ वरु॑ण एनं वरुणपा॒शेन॑ गृह्णाति॒ सर॑स्वतीं॒ वाग॒ग्नाविष्णू॑ आ॒त्मा यस्य॒ ज्योगा॒मय॑ति॒ यो ज्योगा॑मयावी॒ स्याद्यो वा॑ का॒मये॑त॒ सर्व॒मायु॑रिया॒मिति॒ तस्मा॑ ए॒तामिष्टिं॒ निर्व॑पेदाग्ने॒यम॒ष्टाक॑पालꣳ सौ॒म्यं च॒रुं वा॑रु॒णं दश॑\-कपालꣳ सारस्व॒तं च॒रुमा᳚ग्नावैष्ण॒वमेका॑\-दश\-कपालम॒ग्नेरे॒वास्य॒ शरी॑रं निष्क्री॒णाति॒ सोमा॒द्रसं॑~(४१)

%2.3.11.2
वारु॒णेनै॒वैनं॑ वरुणपा॒शान्मु॑ञ्चति सारस्व॒तेन॒ वाचं॑ दधात्य॒ग्निः सर्वा॑ दे॒वता॒ विष्णु॑र्य॒ज्ञो दे॒वता॑भिश्चै॒वैनं॑ य॒ज्ञेन॑ च भिषज्यत्यु॒त यदी॒तासु॒र्भव॑ति॒ जीव॑त्ये॒व यन्नव॒मैत्तन्नव॑नीतमभव॒दित्याज्य॒मवे᳚क्षते रू॒पमे॒वास्यै॒तन्म॑हि॒मानं॒ व्याच॑ष्टे॒\-ऽश्विनोः᳚ प्रा॒णो॑\-ऽसीत्या॑हा॒श्विनौ॒ वै दे॒वानां᳚~(४२)

%2.3.11.3
भि॒षजौ॒ ताभ्या॑मे॒वास्मै॑ भेष॒जं क॑रो॒तीन्द्र॑स्य प्रा॒णो॑\-ऽसीत्या॑हेन्द्रि॒यमे॒वास्मि॑न्ने॒तेन॑ दधाति मि॒त्रावरु॑णयोः प्रा॒णो॑\-ऽसीत्या॑ह प्राणापा॒नावे॒वास्मि॑न्ने॒तेन॑ दधाति॒ विश्वे॑षां दे॒वानां᳚ प्रा॒णो॑\-ऽसीत्या॑ह वी॒र्य॑मे॒वास्मि॑न्ने॒तेन॑ दधाति घृ॒तस्य॒ धारा॑म॒मृत॑स्य॒ पन्था॒मित्या॑ह यथाय॒जुरे॒वैतत्पा॑वमा॒नेन॑ त्वा॒ स्तोमे॒नेत्या॑-~(४३)

%2.3.11.4
ह प्रा॒णमे॒वास्मि॑न्ने॒तेन॑ दधाति बृहद्रथन्त॒रयो᳚स्त्वा॒ स्तोमे॒नेत्या॒हौज॑ ए॒वास्मि॑न्ने॒तेन॑ दधात्य॒ग्नेस्त्वा॒ मात्र॒येत्या॑\-हा॒ऽ॒ऽ॒\-त्मान॑\-मे॒वास्मि॑न्ने॒तेन॑ दधात्यृ॒त्विजः॒ पर्या॑हु॒र्याव॑न्त ए॒वर्त्विज॒स्त ए॑नं भिषज्यन्ति ब्र॒ह्मणो॒ हस्त॑मन्वा॒रभ्य॒ पर्या॑हुरेक॒धैव यज॑मान॒ आयु॑र्दधति॒ यदे॒व तस्य॒ तद्धिर॑ण्याद्-~(४४)

%2.3.11.5
घृ॒तं निष्पि॑ब॒त्यायु॒र्वै घृ॒तम॒मृत॒ꣳ॒ हिर॑ण्यम॒मृता॑दे॒वायु॒र्निष्पि॑बति श॒तमा॑नं भवति श॒तायुः॒ पुरु॑षः श॒तेन्द्रि॑य॒ आयु॑ष्ये॒वेन्द्रि॒ये प्रति॑ तिष्ठ॒त्यथो॒ खलु॒ याव॑तीः॒ समा॑ ए॒ष्यन्मन्ये॑त॒ ताव॑न्मानꣴ स्या॒थ्समृ॑द्ध्या इ॒मम॑ग्न॒ आयु॑षे॒ वर्च॑से कृ॒धीत्या॒हायु॑रे॒वास्मि॒न्वर्चो॑ दधाति॒ विश्वे॑ देवा॒ जर॑दष्टि॒र्यथास॒दित्या॑ह॒ जर॑दष्टिमे॒वैनं॑ करोत्य॒ग्निरायु॑ष्मा॒निति॒ हस्तं॑ गृह्णात्ये॒ते वै दे॒वा आयु॑ष्मन्त॒स्त ए॒वास्मि॒न्नायु॑र्दधति॒ सर्व॒मायु॑रेति॥~(४५)

%2.3.12.0
{\anuvakamend[{रसं॑ दे॒वाना॒ꣴ॒ स्तोमे॒नेति॒ हिर॑ण्या॒दस॒दिति॒ द्वाविꣳ॑शतिश्च}]}%॥11॥

%2.3.12.1
प्र॒जा\-प॑ति॒र्वरु॑णा॒याश्व॑मनय॒थ्स स्वां दे॒वता॑मार्च्छ॒थ्स पर्य॑दीर्यत॒ स ए॒तं वा॑रु॒णं चतु॑ष्कपालमपश्य॒त्तं निर॑वप॒त्ततो॒ वै स व॑रुणपा॒शाद॑मुच्यत॒ वरु॑णो॒ वा ए॒तं गृ॑ह्णाति॒ यो\-ऽश्वं॑ प्रतिगृ॒ह्णाति॒ याव॒तो\-ऽश्वा᳚न्प्रतिगृह्णी॒यात्ताव॑तो वारु॒णाञ्चतु॑ष्कपाला॒न्निर्व॑पे॒द्वरु॑णमे॒व स्वेन॑ भाग॒धेये॒नोप॑ धावति॒ स ए॒वैनं॑ वरुणपा॒शान्मु॑ञ्चति॒~(४६)

%2.3.12.2
चतु॑ष्कपाला भवन्ति॒ चतु॑ष्पा॒द्ध्यश्वः॒ समृ॑द्ध्या॒ एक॒मति॑रिक्तं॒ निर्व॑पे॒द्यमे॒व प्र॑तिग्रा॒ही भव॑ति॒ यं वा॒ नाध्येति॒ तस्मा॑दे॒व व॑रुणपा॒शान्मु॑च्यते॒ यद्यप॑रं प्रतिग्रा॒ही स्याथ्सौ॒र्यमेक॑कपाल॒मनु॒ निर्व॑पेद॒मुमे॒वाऽऽदि॒त्यमु॑च्चा॒रं कु॑रुते॒\-ऽपो॑\-ऽवभृ॒थमवै᳚त्य॒फ्सु वै वरु॑णः सा॒क्षादे॒व वरु॑ण॒मव॑ यजते\-ऽपोन॒प्त्रीयं॑ च॒रुं पुन॒रेत्य॒ निर्व॑पेद॒फ्सुयो॑नि॒र्वा अश्वः॒ स्वामे॒वैनं॒ योनिं॑ गमयति॒ स ए॑नꣳ शा॒न्त उप॑ तिष्ठते॥~(४७)

%2.3.13.0
{\anuvakamend[{मु॒ञ्च॒ति॒ च॒रुꣳ स॒प्तद॑श च}]}%॥12॥

%2.3.13.1
या वा॑मिन्द्रावरुणा यत॒व्या॑ त॒नूस्तये॒ममꣳह॑सो मुञ्चतं॒ या वा॑मिन्द्रावरुणा सह॒स्या॑ रक्ष॒स्या॑ तेज॒स्या॑ त॒नूस्तये॒ममꣳह॑सो मुञ्चतं॒ यो वा॑मिन्द्रावरुणाव॒ग्नौ स्राम॒स्तं वा॑मे॒तेनाव॑ यजे॒ यो वा॑मिन्द्रावरुणा द्वि॒पाथ्सु॑ प॒शुषु॒ चतु॑ष्पाथ्सु गो॒ष्ठे गृ॒हेष्व॒फ्स्वोष॑धीषु॒ वन॒स्पति॑षु॒ स्राम॒स्तं वा॑मे॒तेनाव॑ यज॒ इन्द्रो॒ वा ए॒तस्ये᳚-~(४८)

%2.3.13.2
न्द्रि॒येणाप॑ क्रामति॒ वरु॑ण एनं वरुणपा॒शेन॑ गृह्णाति॒ यः पा॒प्मना॑ गृही॒तो भव॑ति॒ यः पा॒प्मना॑ गृही॒तः स्यात्तस्मा॑ ए॒तामै᳚न्द्रावरु॒णीं प॑य॒स्यां᳚ निर्व॑पे॒दिन्द्र॑ ए॒वास्मि॑न्निन्द्रि॒यं द॑धाति॒ वरु॑ण एनं वरुणपा॒शान्मु॑ञ्चति पय॒स्या॑ भवति॒ पयो॒ हि वा ए॒तस्मा॑दप॒क्राम॒त्यथै॒ष पा॒प्मना॑ गृही॒तो यत्प॑य॒स्या॑ भव॑ति॒ पय॑ ए॒वास्मि॒न्तया॑ दधाति पय॒स्या॑यां~(४९)

%2.3.13.3
पुरो॒डाश॒मव॑ दधात्यात्म॒न्वन्त॑मे॒वैनं॑ करो॒त्यथो॑ आ॒यत॑नवन्तमे॒व च॑तु॒र्धा व्यू॑हति दि॒क्ष्वे॑व प्रति॑ तिष्ठति॒ पुनः॒ समू॑हति दि॒ग्भ्य ए॒वास्मै॑ भेष॒जं क॑रोति स॒मूह्याव॑ द्यति॒ यथावि॑द्धं निष्कृ॒न्तति॑ ता॒दृगे॒व तद्यो वा॑मिन्द्रावरुणाव॒ग्नौ स्राम॒स्तं वा॑मे॒तेनाव॑ यज॒ इत्या॑ह॒ दुरि॑ष्ट्या ए॒वैनं॑ पाति॒ यो वा॑मिन्द्रावरुणा द्वि॒पाथ्सु॑ प॒शुषु॒ स्राम॒स्तं वा॑मे॒तेनाव॑ यज॒ इत्या॑है॒ताव॑ती॒र्वा आप॒ ओष॑धयो॒ वन॒स्पत॑यः प्र॒जाः प॒शव॑ उपजीव॒नीया॒स्ता ए॒वास्मै॑ वरुणपा॒शान्मु॑ञ्चति॥~(५०)

%2.3.14.0
{\anuvakamend[{ए॒तस्य॑ पय॒स्या॑यां पाति॒ षड्विꣳ॑शतिश्च}]}%॥13॥

%2.3.14.1
स प्र॑त्न॒वन्नि काव्येन्द्रं॑ वो वि॒श्वत॒स्परीन्द्रं॒ नरः॑। त्वं नः॑ सोम वि॒श्वतो॒ रक्षा॑ राजन्नघाय॒तः। न रि॑ष्ये॒त्त्वाव॑तः॒ सखा᳚। या ते॒ धामा॑नि दि॒वि या पृ॑थि॒व्यां या पर्व॑ते॒ष्वोष॑धीष्व॒फ्सु। तेभि॑र्नो॒ विश्वैः᳚ सु॒मना॒ अहे॑ड॒न्राज᳚न्थ्सोम॒ प्रति॑ ह॒व्या गृ॑भाय। अग्नी॑षोमा॒ सवे॑दसा॒ सहू॑ती वनतं॒ गिरः॑। सं दे॑व॒त्रा ब॑भूवथुः। यु॒व-~(५१)

%2.3.14.2
मे॒तानि॑ दि॒वि रो॑च॒नान्य॒ग्निश्च॑ सोम॒ सक्र॑तू अधत्तम्। यु॒वꣳ सिन्धूꣳ॑ र॒भिश॑स्तेरव॒द्यादग्नी॑षोमा॒वमु॑ञ्चतं गृभी॒तान्। अग्नी॑षोमावि॒मꣳ सु मे॑ शृणु॒तं वृ॑षणा॒ हवम्᳚। प्रति॑ सू॒क्तानि॑ हर्यतं॒ भव॑तं दा॒शुषे॒ मयः॑। आन्यं दि॒वो मा॑त॒रिश्वा॑ जभा॒राम॑थ्नाद॒न्यं परि॑ श्ये॒नो अद्रेः᳚। अग्नी॑षोमा॒ ब्रह्म॑णा वावृधा॒नोरुं य॒ज्ञाय॑ चक्रथुरु लो॒कम्। अग्नी॑षोमा ह॒विषः॒ प्रस्थि॑तस्य वी॒तꣳ~(५२)

%2.3.14.3
हर्य॑तं वृषणा जु॒षेथा᳚म्। सु॒शर्मा॑णा॒ स्वव॑सा॒ हि भू॒तमथा॑ धत्तं॒ यज॑मानाय॒ शं योः। आ प्या॑यस्व॒ सं ते᳚। ग॒णानां᳚ त्वा ग॒णप॑तिꣳ हवामहे क॒विं क॑वी॒नामु॑प॒मश्र॑वस्तमम्। ज्ये॒ष्ठ॒राजं॒ ब्रह्म॑णां ब्रह्मणस्पत॒ आ नः॑ शृ॒ण्वन्नू॒तिभिः॑ सीद॒ साद॑नम्। स इज्जने॑न॒ स वि॒शा स जन्म॑ना॒ स पु॒त्रैर्वाजं॑ भरते॒ धना॒ नृभिः॑। दे॒वानां॒ यः पि॒तर॑मा॒विवा॑सति~(५३)

%2.3.14.4
श्र॒द्धाम॑ना ह॒विषा॒ ब्रह्म॑ण॒स्पतिम्᳚। स सु॒ष्टुभा॒ स ऋक्व॑ता ग॒णेन॑ व॒लꣳ रु॑रोज फलि॒गꣳ रवे॑ण। बृह॒स्पति॑रु॒स्रिया॑ हव्य॒सूदः॒ कनि॑क्रद॒द्वाव॑शती॒रुदा॑जत्। मरु॑तो॒ यद्ध॑ वो दि॒वो या वः॒ शर्म॑। अ॒र्य॒मा या॑ति वृष॒भस्तुवि॑ष्मान्दा॒ता वसू॑नां पुरुहू॒तो अर्\mbox{}हन्न्॑। स॒ह॒स्रा॒क्षो गो᳚त्र॒भिद्वज्र॑बाहुर॒स्मासु॑ दे॒वो द्रवि॑णं दधातु। ये ते᳚\-ऽर्यमन्ब॒हवो॑ देव॒यानाः॒ पन्था॑नो~(५४)

%2.3.14.5
राजन्दि॒व आ॒चर॑न्ति। तेभि॑र्नो देव॒ महि॒ शर्म॑ यच्छ॒ शं न॑ एधि द्वि॒पदे॒ शं चतु॑ष्पदे। बु॒ध्नादग्र॒मङ्गि॑रोभिर्गृणा॒नो वि पर्व॑तस्य दृꣳहि॒तान्यै॑रत्। रु॒जद्रोधाꣳ॑सि कृ॒त्रिमा᳚ण्येषा॒ꣳ॒ सोम॑स्य॒ ता मद॒ इन्द्र॑श्चकार। बु॒ध्नादग्रे॑ण॒ वि मि॑माय॒ मानै॒र्वज्रे॑ण॒ खान्य॑तृणन्न॒दीना᳚म्। वृथा॑सृजत्प॒थिभि॑र्दीर्घया॒थैः सोम॑स्य॒ ता मद॒ इन्द्र॑श्चकार।~(५५)

%2.3.14.6
प्र यो ज॒ज्ञे वि॒द्वाꣳ अ॒स्य बन्धुं॒ विश्वा॑नि दे॒वो जनि॑मा विवक्ति। ब्रह्म॒ ब्रह्म॑ण॒ उज्ज॑भार॒ मध्या᳚न्नी॒चादु॒च्चा स्व॒धया॒ऽभि प्र त॑स्थौ। म॒हान्म॒ही अ॑स्तभाय॒द्वि जा॒तो द्याꣳ सद्म॒ पार्थि॑वं च॒ रजः॑। स बु॒ध्नादा᳚ष्ट ज॒नुषा॒भ्यग्रं॒ बृह॒स्पति॑र्दे॒वता॒ यस्य॑ स॒म्राट्। बु॒ध्नाद्यो अग्र॑म॒भ्यर्त्योज॑सा॒ बृह॒स्पति॒मा वि॑वासन्ति दे॒वाः। भि॒नद्व॒लं वि पुरो॑ दर्दरीति॒ कनि॑क्रद॒थ्सुव॑र॒पो जि॑गाय॥~(५६)

%2.3.0.0
{\prashnaend[{आ॒दि॒त्येभ्यो॑ दे॒वा वै मृ॒त्योर्दे॒वा वै स॒त्रम॑र्य॒म्णे प्र॒जाप॑ते॒स्त्रय॑स्त्रिꣳशत्प्र॒जा\-प॑तिर्दे॒वेभ्यो॒\-ऽन्नाद्य॑न्देवासु॒रास्तान्रज॑नो द्ध्रु॒वो॑\-ऽसि॒ यन्नव॑म॒ग्निं वै प्र॒जा\-प॑ति॒र्वरु॑णाय॒ या वा॑मिन्द्रावरुणा॒ सप्र॑त्न॒वच्चतु॑र्दश॥14॥ आ॒दि॒त्येभ्य॒स्त्वष्टु॑रस्मै॒ दान॑कामा ए॒वाव॑\-रुन्धे॒\-ऽग्निं वै सप्र॑त्न॒वथ्षट्प॑ञ्चा॒शत्॥56॥ आ॒दि॒त्येभ्यः॒ सुव॑र॒पो जि॑गाय॥}]}

%2.4.0.0
[{\anuvakamend[{यु॒वं वी॒तमा॒ विवा॑सति॒ पन्था॑नो दीर्घया॒थैः सोम॑स्य॒ ता मद॒ इन्द्र॑श्चकार दे॒वा नव॑ च}]}]%॥14॥

%%% END PRASHNA

\sect{चतुर्थः प्रश्नः}\setcounter{anuvakam}{0}
\dnsub{तैत्तिरीयसंहितायां द्वितीयकाण्डे चतुर्थः प्रश्नः}
%2.4.1.0
%2.4.1.1
दे॒वा म॑नु॒ष्याः᳚ पि॒तर॒स्ते᳚\-ऽन्यत॑ आस॒न्नसु॑रा॒ रक्षाꣳ॑सि पिशा॒चास्ते᳚\-ऽन्यत॒स्तेषां᳚ दे॒वाना॑मु॒त यदल्पं॒ लोहि॑त॒मकु॑र्व॒न्तद्रक्षाꣳ॑सि॒ रात्री॑भिरसुभ्न॒न्तान्थ्सु॒ब्धान्मृ॒तान॒भि व्यौ᳚च्छ॒त्ते दे॒वा अ॑विदु॒र्यो वै नो॒\-ऽयं म्रि॒यते॒ रक्षाꣳ॑सि॒ वा इ॒मं घ्न॒न्तीति॒ ते रक्षा॒ꣴ॒स्युपा॑मन्त्रयन्त॒ तान्य॑ब्रुव॒न्वरं॑ वृणामहै॒ य-~(१)

%2.4.1.2
दसु॑रा॒ञ्जया॑म॒ तन्नः॑ स॒हास॒दिति॒ ततो॒ वै दे॒वा असु॑रानजय॒न्ते\-ऽसु॑राञ्जि॒त्वा रक्षा॒ꣴ॒स्यपा॑नुदन्त॒ तानि॒ रक्षा॒ꣴ॒स्यनृ॑तमक॒र्तेति॑ सम॒न्तं दे॒वान्पर्य॑विश॒न्ते दे॒वा अ॒ग्नाव॑नाथन्त॒ ते᳚\-ऽग्नये॒ प्रव॑ते पुरो॒डाश॑\-म॒ष्टा\-क॑पालं॒ निर॑वपन्न॒ग्नये॑ विबा॒धव॑ते॒\-ऽग्नये॒ प्रती॑कवते॒ यद॒ग्नये॒ प्रव॑ते नि॒रव॑प॒न्॒ यान्ये॒व पु॒रस्ता॒द्रक्षा॒ꣴ॒स्या-~(२)

%2.4.1.3
स॒न्तानि॒ तेन॒ प्राणु॑दन्त॒ यद॒ग्नये॑ विबा॒धव॑ते॒ यान्ये॒वाभितो॒ रक्षा॒ꣴ॒स्यास॒न्तानि॒ तेन॒ व्य॑बाधन्त॒ यद॒ग्नये॒ प्रती॑कवते॒ यान्ये॒व प॒श्चाद्रक्षा॒ꣴ॒स्यास॒न्तानि॒ तेनापा॑नुदन्त॒ ततो॑ दे॒वा अभ॑व॒न्परासु॑रा॒ यो भ्रातृ॑व्यवा॒न्थ्स्याथ्स स्पर्ध॑मान ए॒तयेष्ट्या॑ यजेता॒ग्नये॒ प्रव॑ते पुरो॒डाश॑\-म॒ष्टा\-क॑पालं॒ निर्व॑पेद॒ग्नये॑ विबा॒धव॑ते॒-~(३)

%2.4.1.4
ऽग्नये॒ प्रती॑कवते॒ यद॒ग्नये॒ प्रव॑ते नि॒र्वप॑ति॒ य ए॒वास्मा॒\-च्छ्रेया॒न्भ्रातृ॑व्य॒स्तं तेन॒ प्र णु॑दते॒ यद॒ग्नये॑ विबा॒धव॑ते॒ य ए॒वैने॑न स॒दृङ्तं तेन॒ वि बा॑धते॒ यद॒ग्नये॒ प्रती॑कवते॒ य ए॒वास्मा॒त्पापी॑या॒न्तं तेनाप॑ नुदते॒ प्र श्रेयाꣳ॑सं॒ भ्रातृ॑व्यं नुद॒ते\-ऽति॑ स॒दृशं॑ क्रामति॒ नैनं॒ पापी॑यानाप्नोति॒ य ए॒वं वि॒द्वाने॒तयेष्ट्या॒ यज॑ते॥~(४)

%2.4.2.0
{\anuvakamend[{वृ॒णा॒म॒है॒ यत्पु॒रस्ता॒द्रक्षाꣳ॑सि वपेद॒ग्नये॑ विबा॒धव॑त ए॒वं च॒त्वारि॑ च}]}%~(१)

%2.4.2.1
दे॒वा॒सु॒राः संय॑त्ता आस॒न्ते दे॒वा अ॑ब्रुव॒न्॒ यो नो॑ वी॒र्या॑वत्तम॒स्तमनु॑ स॒मार॑भामहा॒ इति॒ त इन्द्र॑मब्रुव॒न्त्वं वै नो॑ वी॒र्या॑वत्तमो\-ऽसि॒ त्वामनु॑ स॒मार॑भामहा॒ इति॒ सो᳚\-ऽब्रवीत्ति॒स्रो म॑ इ॒मास्त॒नुवो॑ वी॒र्या॑वती॒स्ताः प्री॑णी॒ताथासु॑रान॒भि भ॑विष्य॒थेति॒ ता वै ब्रू॒हीत्य॑ब्रुवन्नि॒यमꣳ॑हो॒मुगि॒यं वि॑मृ॒धेयमि॑न्द्रि॒याव॒ती-~(५)

%2.4.2.2
त्य॑ब्रवी॒त्त इन्द्रा॑याꣳहो॒मुचे॑ पुरो॒डाश॒मेका॑\-दश\-कपालं॒ निर॑वप॒न्निन्द्रा॑य वैमृ॒धायेन्द्रा॑येन्द्रि॒याव॑ते॒ यदिन्द्रा॑याꣳहो॒मुचे॑ नि॒रव॑प॒न्नꣳह॑स ए॒व तेना॑मुच्यन्त॒ यदिन्द्रा॑य वैमृ॒धाय॒ मृध॑ ए॒व तेनापा᳚घ्नत॒ यदिन्द्रा॑येन्द्रि॒याव॑त इन्द्रि॒यमे॒व तेना॒ऽऽत्मन्न॑दधत॒ त्रय॑स्त्रिꣳशत्कपालं पुरो॒डाशं॒ निर॑वप॒न्त्रय॑स्त्रिꣳश॒द्वै दे॒वता॒स्ता इन्द्र॑ आ॒त्मन्ननु॑ स॒मार॑म्भयत॒ भूत्यै॒~(६)

%2.4.2.3
तां वाव दे॒वा विजि॑तिमुत्त॒मामसु॑रै॒र्व्य॑जयन्त॒ यो भ्रातृ॑व्यवा॒न्थ्स्याथ्स स्पर्ध॑मान ए॒तयेष्ट्या॑ यजे॒तेन्द्रा॑याꣳहो॒मुचे॑ पुरो॒डाश॒मेका॑\-दश\-कपालं॒ निर्व॑पे॒दिन्द्रा॑य वैमृ॒धायेन्द्रा॑येन्द्रि॒याव॒ते\-ऽꣳह॑सा॒ वा ए॒ष गृ॑ही॒तो यस्मा॒च्छ्रेया॒न्भ्रातृ॑व्यो॒ यदिन्द्रा॑याꣳहो॒मुचे॑ नि॒र्वप॒त्यꣳह॑स ए॒व तेन॑ मुच्यते मृ॒धा वा ए॒षो॑\-ऽभिष॑ण्णो॒ यस्मा᳚थ्समा॒नेष्व॒न्यः श्रेया॑नु॒ता~-~(७)

%2.4.2.4
ऽभ्रा॑तृव्यो॒ यदिन्द्रा॑य वैमृ॒धाय॒ मृध॑ ए॒व तेनाप॑ हते॒ यदिन्द्रा॑येन्द्रि॒याव॑त इन्द्रि॒यमे॒व तेना॒ऽऽत्मन्ध॑त्ते॒ त्रय॑स्त्रिꣳशत्कपालं पुरो॒डाशं॒ निर्व॑पति॒ त्रय॑स्त्रिꣳश॒द्वै दे॒वता॒स्ता ए॒व यज॑मान आ॒त्मन्ननु॑ स॒मार॑म्भयते॒ भूत्यै॒ सा वा ए॒षा विजि॑ति॒र्नामेष्टि॒र्य ए॒वं वि॒द्वाने॒तयेष्ट्या॒ यज॑त उत्त॒मामे॒व विजि॑तिं॒ भ्रातृ॑व्येण॒ वि ज॑यते॥~(८)

%2.4.3.0
{\anuvakamend[{इ॒न्द्रि॒याव॑ती॒ भूत्या॑ उ॒तैका॒न्नप॑ञ्चा॒शच्च॑}]}%~(२)

%2.4.3.1
दे॒वा॒सु॒राः संय॑त्ता आस॒न्तेषां᳚ गाय॒त्र्योजो॒ बल॑मिन्द्रि॒यं वी॒र्यं॑ प्र॒जां प॒शून्थ्स॒ङ्गृह्या॒दाया॑प॒क्रम्या॑तिष्ठ॒त् ते॑\-ऽमन्यन्त यत॒रान् वा इ॒यमु॑पाव॒र्थ्स्यति॒ त इ॒दं भ॑विष्य॒न्तीति॒ तां व्य॑ह्वयन्त॒ विश्व॑कर्म॒न्निति॑ दे॒वा दाभीत्यसु॑राः॒ सा नान्य॑त॒राꣴश्च॒ नोपाव॑र्तत॒ ते दे॒वा ए॒तद्यजु॑रपश्य॒न्नोजो॑\-ऽसि॒ सहो॑\-ऽसि॒ बल॑मसि॒~(९)

%2.4.3.2
भ्राजो॑\-ऽसि दे॒वानां॒ धाम॒ नामा॑सि॒ विश्व॑मसि वि॒श्वायुः॒ सर्व॑मसि स॒र्वायु॑रभि॒भूरिति॒ वाव दे॒वा असु॑राणा॒मोजो॒ बल॑मिन्द्रि॒यं वी॒र्यं॑ प्र॒जां प॒शून॑वृञ्जत॒ यद्गा॑य॒त्र्य॑प॒क्रम्याति॑ष्ठ॒त् तस्मा॑दे॒तां गा॑य॒त्रीतीष्टि॑माहुः संवथ्स॒रो वै गा॑य॒त्री सं॑वथ्स॒रो वै तद॑प॒क्रम्या॑तिष्ठ॒द्यदे॒तया॑ दे॒वा असु॑राणा॒मोजो॒ बल॑मिन्द्रि॒यं वी॒र्यं॑~(१०)

%2.4.3.3
प्र॒जां प॒शूनवृ॑ञ्जत॒ तस्मा॑दे॒ताꣳ सं॑व॒र्ग इतीष्टि॑माहु॒र्यो भ्रातृ॑व्यवा॒न्थ्स्याथ्स स्पर्ध॑मान ए॒तयेष्ट्या॑ यजेता॒ग्नये॑ संव॒र्गाय॑ पुरो॒डाश॑\-म॒ष्टा\-क॑पालं॒ निर्व॑पे॒त्तꣳ शृ॒तमास॑न्नमे॒तेन॒ यजु॑षा॒ऽभि मृ॑शे॒दोज॑ ए॒व बल॑मिन्द्रि॒यं वी॒र्यं॑ प्र॒जां प॒शून्भ्रातृ॑व्यस्य वृङ्क्ते॒ भव॑त्या॒त्मना॒ परा᳚स्य॒ भ्रातृ॑व्यो भवति॥~(११)

%2.4.4.0
{\anuvakamend[{बल॑मस्ये॒तया॑ दे॒वा असु॑राणा॒मोजो॒ बल॑मिन्द्रि॒यं वी॒र्यं॑ पञ्च॑चत्वारिꣳशच्च}]}%~(३)

%2.4.4.1
प्र॒जा\-प॑तिः प्र॒जा अ॑सृजत॒ ता अ॑स्माथ्सृ॒ष्टाः परा॑चीराय॒न्ता यत्राव॑स॒न्ततो॑ ग॒र्मुदुद॑तिष्ठ॒त् ता बृह॒स्पति॑श्चा॒न्ववै॑ता॒ꣳ॒ सो᳚\-ऽब्रवी॒द्बृह॒स्पति॑र॒नया᳚ त्वा॒ प्र ति॑ष्ठा॒न्यथ॑ त्वा प्र॒जा उ॒पाव॑र्थ्स्य॒न्तीति॒ तं प्राति॑ष्ठ॒त् ततो॒ वै प्र॒जा\-प॑तिं प्र॒जा उ॒पाव॑र्तन्त॒ यः प्र॒जाका॑मः॒ स्यात् तस्मा॑ ए॒तं प्रा॑जाप॒त्यं गा᳚र्मु॒तं च॒रुं निर्व॑पेत्प्र॒जा\-प॑ति-~(१२)

%2.4.4.2
मे॒व स्वेन॑ भाग॒धेये॒नोप॑ धावति॒ स ए॒वास्मै᳚ प्र॒जां प्र ज॑नयति प्र॒जा\-प॑तिः प॒शून॑सृजत॒ ते᳚\-ऽस्माथ्सृ॒ष्टाः परा᳚ञ्च आय॒न्ते यत्राव॑स॒न्ततो॑ ग॒र्मुदुद॑तिष्ठ॒त् तान्पू॒षा चा॒न्ववै॑ता॒ꣳ॒ सो᳚\-ऽब्रवीत्पू॒षाऽनया॑ मा॒ प्र ति॒ष्ठाथ॑ त्वा प॒शव॑ उ॒पाव॑र्थ्स्य॒न्तीति॒ मां प्र ति॒ष्ठेति॒ सोमो᳚\-ऽब्रवी॒न्मम॒ \mbox{वा~-~(१३)}

%2.4.4.3
अ॑कृष्टप॒च्यमित्यु॒भौ वां॒ प्र ति॑ष्ठा॒नीत्य॑ब्रवी॒त्तौ प्राति॑ष्ठ॒त् ततो॒ वै प्र॒जा\-प॑तिं प॒शव॑ उ॒पाव॑र्तन्त॒ यः प॒शुका॑मः॒ स्यात् तस्मा॑ ए॒तꣳ सो॑मापौ॒ष्णं गा᳚र्मु॒तं च॒रुं निर्व॑पेथ्सोमापू॒षणा॑वे॒व स्वेन॑ भाग॒धेये॒नोप॑ धावति॒ तावे॒वास्मै॑ प॒शून्प्र ज॑नयतः॒ सोमो॒ वै रे॑तो॒धाः पू॒षा प॑शू॒नां प्र॑जनयि॒ता सोम॑ ए॒वास्मै॒ रेतो॒ दधा॑ति पू॒षा प॒शून्प्र ज॑नयति॥~(१४)

%2.4.5.0
{\anuvakamend[{व॒पे॒त्प्र॒जा\-प॑तिं॒ वै दधा॑ति पू॒षा त्रीणि॑ च}]}%~(४)

%2.4.5.1
अग्ने॒ गोभि॑र्न॒ आ ग॒हीन्दो॑ पु॒ष्ट्या जु॑षस्व नः। इन्द्रो॑ ध॒र्ता गृ॒हेषु॑ नः॥ स॒वि॒ता यः स॑ह॒स्रियः॒ स नो॑ गृ॒हेषु॑ रारणत्। आ पू॒षा ए॒त्वा वसु॑॥ धा॒ता द॑दातु नो र॒यिमीशा॑नो॒ जग॑त॒स्पतिः॑। स नः॑ पू॒र्णेन॑ वावनत्॥ त्वष्टा॒ यो वृ॑ष॒भो वृषा॒ स नो॑ गृ॒हेषु॑ रारणत्। स॒हस्रे॑णा॒युते॑न च॥ येन॑ दे॒वा अ॒मृतं॑~(१५)

%2.4.5.2
दी॒र्घꣴ श्रवो॑ दि॒व्यैर॑यन्त। राय॑स्पोष॒ त्वम॒स्मभ्यं॒ गवां᳚ कु॒ल्मिं जी॒वस॒ आ यु॑वस्व। अ॒ग्निर्गृ॒हप॑तिः॒ सोमो॑ विश्व॒वनिः॑ सवि॒ता सु॑\-मे॒धाः स्वाहा᳚। अग्ने॑ गृहपते॒ यस्ते॒ घृत्यो॑ भा॒गस्तेन॒ सह॒ ओज॑ आ॒\-क्रम॑\-मा\-णाय धेहि॒ श्रैष्ठ्या᳚त्प॒थो मा यो॑षं मू॒र्धा भू॑यास॒ꣴ॒ स्वाहा᳚॥~(१६)

%2.4.6.0
{\anuvakamend[{अ॒मृत॑म॒ष्टात्रिꣳ॑शच्च}]}%~(५)

%2.4.6.1
चि॒त्रया॑ यजेत प॒शुका॑म इ॒यं वै चि॒त्रा यद्वा अ॒स्यां विश्वं॑ भू॒तमधि॑ प्र॒जाय॑ते॒ तेने॒यं चि॒त्रा य ए॒वं वि॒द्वाꣴश्चि॒त्रया॑ प॒शुका॑मो॒ यज॑ते॒ प्र प्र॒जया॑ प॒शुभि॑र्मिथु॒नैर्जा॑यते॒ प्रैवाऽऽग्ने॒येन॑ वापयति॒ रेतः॑ सौ॒म्येन॑ दधाति॒ रेत॑ ए॒व हि॒तं त्वष्टा॑ रू॒पाणि॒ वि क॑रोति सारस्व॒तौ भ॑वत ए॒तद्वै दैव्यं॑ मिथु॒नं दैव्य॑मे॒वास्मै॑~(१७)

%2.4.6.2
मिथु॒नं म॑ध्य॒तो द॑धाति॒ पुष्ट्यै᳚ प्र॒जन॑नाय सिनीवा॒ल्यै च॒रुर्भ॑वति॒ वाग्वै सि॑नीवा॒ली पुष्टिः॒ खलु॒ वै वाक्पुष्टि॑मे॒व वाच॒मुपै᳚त्यै॒न्द्र उ॑त्त॒मो भ॑वति॒ तेनै॒व तन्मि॑थु॒नꣳ स॒प्तैतानि॑ ह॒वीꣳषि॑ भवन्ति स॒प्त ग्रा॒म्याः प॒शवः॑ स॒प्तार॒ण्याः स॒प्त छन्दाꣴ॑स्यु॒भय॒स्याव॑रुद्ध्या॒ अथै॒ता आहु॑तीर्जुहोत्ये॒ते वै दे॒वाः पुष्टि॑पतय॒स्त ए॒वास्मि॒न्पुष्टिं॑ दधति॒ पुष्य॑ति प्र॒जया॑ प॒शुभि॒रथो॒ यदे॒ता आहु॑तीर्जु॒होति॒ प्रति॑ष्ठित्यै॥~(१८)

%2.4.7.0
{\anuvakamend[{अ॒स्मै॒ त ए॒व द्वाद॑श च}]}%~(६)

%2.4.7.1
मा॒रु॒तम॑सि म॒रुता॒मोजो॒\-ऽपां धारां᳚ भिन्द्धि र॒मय॑त मरुतः श्ये॒न\-मा॒यिनं॒ मनो॑जवसं॒ वृष॑णꣳ सुवृ॒क्तिम्। येन॒ शर्ध॑ उ॒ग्र\-मव॑\-सृष्ट॒मेति॒ तद॑श्विना॒ परि॑ धत्तꣴ स्व॒स्ति। पु॒रो॒\-वा॒तो वर्\mbox{}ष॑ञ्जि॒न्वरा॒\-वृथ्\-स्वाहा॑ वा॒ता\-व॒द्वर्\mbox{}ष॑न्नु॒ग्ररा॒वृथ्\-स्वाहा᳚ स्त॒न\-य॒न्वर्\mbox{}ष॑न्भी॒म\-रा॒\-वृथ्\-स्वाहा॑\-ऽन\-श॒न्य॑व॒\-स्फूर्ज॑न्दि॒द्युद्वर्\mbox{}ष॑न्त्वे॒ष\-रा॒\-वृथ्\-स्वाहा॑\-ऽतिरा॒त्रं वर्\mbox{}ष॑न्पू॒र्तिरा॒वृथ्-~(१९)

%2.4.7.2
स्वाहा॑ ब॒हु हा॒यम॑वृषा॒दिति॑ श्रु॒तरा॒वृथ्\-स्वाहा॒\-ऽऽतप॑ति॒ वर्\mbox{}ष॑न्वि॒राडा॒वृथ्\-स्वाहा॑\-व॒स्फूर्ज॑न्दि॒द्युद्वर्\mbox{}ष॑न्भू॒तरा॒वृथ्\-स्वाहा॒ मान्दा॒ वाशाः॒ शुन्ध्यू॒रजि॑राः। ज्योति॑ष्मती॒स्तम॑स्वरी॒रुन्द॑तीः॒ सुफे॑नाः। मित्र॑भृतः॒ क्षत्र॑भृतः॒ सुरा᳚ष्ट्रा इ॒ह मा॑\-ऽवत। वृष्णो॒ अश्व॑स्य स॒न्दान॑मसि॒ वृष्ट्यै॒ त्वोप॑ नह्यामि॥~(२०)

%2.4.8.0
{\anuvakamend[{पू॒र्तिरा॒वृद्द्विच॑त्वारिꣳशच्च}]}%~(७)

%2.4.8.1
देवा॑ वसव्या॒ अग्ने॑ सोम सूर्य। देवाः᳚ शर्मण्या॒ मित्रा॑वरुणार्यमन्न्। देवाः᳚ सपीत॒यो\-ऽपां᳚ नपादाशुहेमन्न्। उ॒द्नो द॑त्तो\-ऽद॒धिं भि॑न्त दि॒वः प॒र्जन्या॑द॒न्तरि॑क्षात्पृथि॒व्यास्ततो॑ नो॒ वृष्ट्या॑\-ऽवत। दिवा॑ चि॒त्तमः॑ कृण्वन्ति प॒र्जन्ये॑नोदवा॒हेन॑। पृ॒थि॒वीं यद्व्यु॒न्दन्ति॑। आ यं नरः॑ सु॒दान॑वो ददा॒शुषे॑ दि॒वः कोश॒मचु॑च्यवुः। वि प॒र्जन्याः᳚ सृजन्ति॒ रोद॑सी॒ अनु॒ धन्व॑ना यन्ति~(२१)

%2.4.8.2
वृ॒ष्टयः॑। उदी॑रयथा मरुतः समुद्र॒तो यू॒यं वृ॒ष्टिं व॑र्\mbox{}षयथा पुरीषिणः। न वो॑ दस्रा॒ उप॑ दस्यन्ति धे॒नवः॒ शुभं॑ या॒तामनु॒ रथा॑ अवृथ्सत। सृ॒जा वृ॒ष्टिं दि॒व आद्भिः स॑मु॒द्रं पृ॑ण। अ॒ब्जा अ॑सि प्रथम॒जा बल॑मसि समु॒द्रियम्᳚। उन्न॑म्भय पृथि॒वीं भि॒न्द्धीदं दि॒व्यं नभः॑। उ॒द्नो दि॒व्यस्य॑ नो दे॒हीशा॑नो॒ वि सृ॑जा॒ दृतिम्᳚। ये दे॒वा दि॒विभा॑गा॒ ये᳚\-ऽन्तरि॑क्षभागा॒ ये पृ॑थि॒विभा॑गाः। त इ॒मं य॒ज्ञम॑वन्तु॒ त इ॒दं क्षेत्र॒मा वि॑शन्तु॒ त इ॒दं क्षेत्र॒मनु॒ वि वि॑शन्तु॥~(२२)

%2.4.9.0
{\anuvakamend[{य॒न्ति॒ दे॒वा विꣳ॑शति॒श्च॑}]}%~(८)

%2.4.9.1
मा॒रु॒तम॑सि म॒रुता॒मोज॒ इति॑ कृ॒ष्णं वासः॑ कृ॒ष्णतू॑षं॒ परि॑ धत्त ए॒तद्वै वृष्ट्यै॑ रू॒पꣳ सरू॑प ए॒व भू॒त्वा प॒र्जन्यं॑ वर्\mbox{}षयति र॒मय॑त मरुतः श्ये॒नमा॒यिन॒मिति॑ पश्चाद्वा॒तं प्रति॑ मीवति पुरोवा॒तमे॒व ज॑नयति व॒र्॒\mbox{}षस्याव॑रुद्ध्यै वातना॒मानि॑ जुहोति वा॒युर्वै वृष्ट्या॑ ईशे वा॒युमे॒व स्वेन॑ भाग॒धेये॒नोप॑ धावति॒ स ए॒वास्मै॑ प॒र्जन्यं॑ वर्\mbox{}षयत्य॒ष्टौ~(२३)

%2.4.9.2
जु॑होति॒ चत॑स्रो॒ वै दिश॒श्चत॑स्रो\-ऽवान्तरदि॒शा दि॒ग्भ्य ए॒व वृष्टि॒ꣳ॒ सम्प्र च्या॑वयति कृष्णाजि॒ने सं यौ॑ति ह॒विरे॒वाक॑रन्तर्वे॒दि सं यौ॒त्यव॑रुद्ध्यै॒ यती॑नाम॒द्यमा॑नानाꣳ शी॒र्॒\mbox{}षाणि॒ परा॑पत॒न्ते ख॒र्जूरा॑ अभव॒न्तेषा॒ꣳ॒ रस॑ ऊ॒र्ध्वो॑\-ऽपत॒त् तानि॑ क॒रीरा᳚ण्यभवन्थ्सौ॒म्यानि॒ वै क॒रीरा॑णि सौ॒म्या खलु॒ वा आहु॑तिर्दि॒वो वृष्टिं॑ च्यावयति॒ यत्क॒रीरा॑णि॒ भव॑न्ति~(२४)

%2.4.9.3
सौ॒म्ययै॒वाऽऽहु॑त्या दि॒वो वृष्टि॒मव॑ रुन्धे॒ मधु॑षा॒ सं यौ᳚त्य॒पां वा ए॒ष ओष॑धीना॒ꣳ॒ रसो॒ यन्मध्व॒द्भ्य ए॒वौष॑धीभ्यो वर्\mbox{}ष॒त्यथो॑ अ॒द्भ्य ए॒वौष॑धीभ्यो॒ वृष्टिं॒ नि न॑यति॒ मान्दा॒ वाशा॒ इति॒ सं यौ॑ति नाम॒धेयै॑रे॒वैना॒ अच्छै॒त्यथो॒ यथा᳚ ब्रू॒यादसा॒वेहीत्ये॒वमे॒वैना॑ नाम॒धेयै॒रा~(२५)

%2.4.9.4
च्या॑वयति॒ वृष्णो॒ अश्व॑स्य स॒न्दान॑मसि॒ वृष्ट्यै॒ त्वोप॑ नह्या॒मीत्या॑ह॒ वृषा॒ वा अश्वो॒ वृषा॑ प॒र्जन्यः॑ कृ॒ष्ण इ॑व॒ खलु॒ वै भू॒त्वा व॑र्\mbox{}षति रू॒पेणै॒वैन॒ꣳ॒ सम॑र्धयति व॒र्॒\mbox{}षस्याव॑रुद्ध्यै॥~(२६)

%2.4.10.0
{\anuvakamend[{अ॒ष्टौ भव॑न्ति नाम॒धेयै॒रैका॒न्नत्रि॒ꣳ॒शच्च॑}]}%~(९)

%2.4.10.1
देवा॑ वसव्या॒ देवाः᳚ शर्मण्या॒ देवाः᳚ सपीतय॒ इत्या ब॑ध्नाति दे॒वता॑भिरे॒वान्व॒हं वृष्टि॑मिच्छति॒ यदि॒ वर्\mbox{}षे॒त् ताव॑त्ये॒व हो॑त॒व्यं॑ यदि॒ न वर्\mbox{}षे॒च्छ्वो भू॒ते ह॒विर्निर्व॑पेदहोरा॒त्रे वै मि॒त्रावरु॑णावहोरा॒त्राभ्यां॒ खलु॒ वै प॒र्जन्यो॑ वर्\mbox{}षति॒ नक्तं॑ वा॒ हि दिवा॑ वा॒ वर्\mbox{}ष॑ति मि॒त्रावरु॑णावे॒व स्वेन॑ भाग॒धेये॒नोप॑ धावति॒ तावे॒वास्मा॑~-~(२७)

%2.4.10.2
अहोरा॒त्रा\-भ्यां᳚ प॒र्जन्यं॑ वर्\mbox{}षयतो॒\-ऽग्नये॑ धाम॒च्छदे॑ पुरो॒डाश॑\-म॒ष्टा\-क॑पालं॒ निर्व॑पेन्मारु॒तꣳ स॒प्तक॑पालꣳ सौ॒र्यमेक॑कपालम॒ग्निर्वा इ॒तो वृष्टि॒मुदी॑रयति म॒रुतः॑ सृ॒ष्टां न॑यन्ति य॒दा खलु॒ वा अ॒सावा॑दि॒त्यो न्य॑ङ्र॒श्मिभिः॑ पर्या॒वर्त॒ते\-ऽथ॑ वर्\mbox{}षति धाम॒च्छदि॑व॒ खलु॒ वै भू॒त्वा व॑र्\mbox{}षत्ये॒ता वै दे॒वता॒ वृष्ट्या॑ ईशते॒ ता ए॒व स्वेन॑ भाग॒धेये॒नोप॑ धावति॒ ता~-~(२८)

%2.4.10.3
ए॒वास्मै॑ प॒र्जन्यं॑ वर्\mbox{}षयन्त्यु॒ताव॑र्\mbox{}षिष्य॒न्वर्\mbox{}ष॑त्ये॒व सृ॒जा वृ॒ष्टिं दि॒व आद्भिः स॑मु॒द्रं पृ॒णेत्या॑हे॒माश्चै॒वामूश्चा॒पः सम॑र्धय॒त्यथो॑ आ॒भिरे॒वामूरच्छै᳚त्य॒ब्जा अ॑सि प्रथम॒जा बल॑मसि समु॒द्रिय॒मित्या॑ह यथाय॒जुरे॒वैतदुन्न॑म्भय पृथि॒वीमिति॑ वर्\mbox{}षा॒ह्वां जु॑होत्ये॒षा वा ओष॑धीनां वृष्टि॒वनि॒स्तयै॒व वृष्टि॒मा च्या॑वयति॒ ये दे॒वा दि॒विभा॑गा॒ इति॑ कृष्णाजि॒नमव॑ धूनोती॒म ए॒वास्मै॑ लो॒काः प्री॒ता अ॒भीष्टा॑ भवन्ति॥~(२९)

%2.4.11.0
{\anuvakamend[{अ॒स्मै॒ धा॒व॒ति॒ ता वा एक॑विꣳशतिश्च}]}%॥10॥

%2.4.11.1
सर्वा॑णि॒ छन्दाꣴ॑स्ये॒तस्या॒मिष्ट्या॑म॒नूच्या॒नीत्या॑हुस्त्रि॒ष्टुभो॒ वा ए॒तद्वी॒र्यं॑ यत्क॒कुदु॒ष्णिहा॒ जग॑त्यै॒ यदु॑ष्णिहक॒कुभा॑व॒न्वाह॒ तेनै॒व सर्वा॑णि॒ छन्दा॒ꣴ॒स्यव॑ रुन्धे गाय॒त्री वा ए॒षा यदु॒ष्णिहा॒ यानि॑ च॒त्वार्यध्य॒क्षरा॑णि॒ चतु॑ष्पाद ए॒व ते प॒शवो॒ यथा॑ पुरो॒डाशे॑ पुरो॒डाशो\-ऽध्ये॒वमे॒व तद्यदृ॒च्यध्य॒क्षरा॑णि॒ यज्जग॑त्या~(३०)

%2.4.11.2
परिद॒ध्यादन्तं॑ य॒ज्ञं ग॑मयेत् त्रि॒ष्टुभा॒ परि॑ दधातीन्द्रि॒यं वै वी॒र्यं॑ त्रि॒ष्टुगि॑न्द्रि॒य ए॒व वी॒र्ये॑ य॒ज्ञं प्रति॑\-ष्ठापयति॒ नान्तं॑ गमय॒त्यग्ने॒ त्री ते॒ वाजि॑ना॒ त्री ष॒धस्थेति॒ त्रिव॑त्या॒ परि॑ दधाति सरूप॒त्वाय॒ सर्वो॒ वा ए॒ष य॒ज्ञो यत् त्रै॑धात॒वीयं॒ कामा॑यकामाय॒ प्र यु॑ज्यते॒ सर्वे᳚भ्यो॒ हि कामे᳚भ्यो य॒ज्ञः प्र॑यु॒ज्यते᳚ त्रैधात॒वीये॑न यजेताभि॒चर॒न्थ्सर्वो॒ \mbox{वा~-~(३१)}

%2.4.11.3
ए॒ष य॒ज्ञो यत् त्रै॑धात॒वीय॒ꣳ॒ सर्वे॑णै॒वैनं॑ य॒ज्ञेना॒भि च॑रति स्तृणु॒त ए॒वैन॑मे॒तयै॒व य॑जेताभिच॒र्यमा॑णः॒ सर्वो॒ वा ए॒ष य॒ज्ञो यत् त्रै॑धात॒वीय॒ꣳ॒ सर्वे॑णै॒व य॒ज्ञेन॑ यजते॒ नैन॑मभि॒चर᳚न्थ्स्तृणुत ए॒तयै॒व य॑जेत स॒हस्रे॑ण य॒क्ष्यमा॑णः॒ प्रजा॑तमे॒वैन॑द्ददात्ये॒तयै॒व य॑जेत स॒हस्रे॑णेजा॒नो\-ऽन्तं॒ वा ए॒ष प॑शू॒नां ग॑च्छति॒~(३२)

%2.4.11.4
यः स॒हस्रे॑ण॒ यज॑ते प्र॒जा\-प॑तिः॒ खलु॒ वै प॒शून॑सृजत॒ ताꣴ स्त्रै॑धात॒वीये॑नै॒वासृ॑जत॒ य ए॒वं वि॒द्वाꣴस्त्रै॑धात॒वीये॑न प॒शुका॑मो॒ यज॑ते॒ यस्मा॑दे॒व योनेः᳚ प्र॒जा\-प॑तिः प॒शूनसृ॑जत॒ तस्मा॑दे॒वैना᳚न्थ्सृजत॒ उपै॑न॒मुत्त॑रꣳ स॒हस्रं॑ नमति दे॒वता᳚भ्यो॒ वा ए॒ष आ वृ॑श्च्यते॒ यो य॒क्ष्य इत्यु॒क्त्वा न यज॑ते त्रैधात॒वीये॑न यजेत॒ सर्वो॒ वा ए॒ष य॒ज्ञो~-~(३३)

%2.4.11.5
यत् त्रै॑धात॒वीय॒ꣳ॒ सर्वे॑णै॒व य॒ज्ञेन॑ यजते॒ न दे॒वता᳚भ्य॒ आ वृ॑श्च्यते॒ द्वाद॑श\-कपालः पुरो॒डाशो॑ भवति॒ ते त्रय॒श्चतु॑ष्कपालास्त्रिः षमृद्ध॒त्वाय॒ त्रयः॑ पुरो॒डाशा॑ भवन्ति॒ त्रय॑ इ॒मे लो॒का ए॒षां लो॒काना॒माप्त्या॒ उत्त॑रउत्तरो॒ ज्याया᳚न्भवत्ये॒वमि॑व॒ हीमे लो॒का य॑व॒मयो॒ मध्य॑ ए॒तद्वा अ॒न्तरि॑क्षस्य रू॒पꣳ समृ॑द्ध्यै॒ सर्वे॑षामभिग॒मय॒न्नव॑ द्य॒त्यछ॑म्बट्कार॒ꣳ॒ हिर॑ण्यं ददाति॒ तेज॑ ए॒वा-~(३४)

%2.4.11.6
व॑ रुन्धे ता॒र्प्यं द॑दाति प॒शूने॒वाव॑ रुन्धे धे॒नुं द॑दात्या॒शिष॑ ए॒वाव॑ रुन्धे॒ साम्नो॒ वा ए॒ष वर्णो॒ यद्धिर॑ण्यं॒ यजु॑षां ता॒र्प्यमु॑क्थाम॒दानां᳚ धे॒नुरे॒ताने॒व सर्वा॒न् वर्णा॒नव॑ रुन्धे॥~(३५)

%2.4.12.0
{\anuvakamend[{जग॑त्या\-ऽभि॒चर॒न्थ्सर्वो॒ वै ग॑च्छति य॒ज्ञस्तेज॑ ए॒व त्रि॒ꣳ॒शच्च॑}]}%॥11॥

%2.4.12.1
त्वष्टा॑ ह॒तपु॑त्रो॒ वीन्द्र॒ꣳ॒ सोम॒माह॑र॒त् तस्मि॒न्निन्द्र॑ उपह॒वमै᳚च्छत॒ तं नोपा᳚ह्वयत पु॒त्रं मे॑\-ऽवधी॒रिति॒ स य॑ज्ञवेश॒सं कृ॒त्वा प्रा॒सहा॒ सोम॑मपिब॒त् तस्य॒ यद॒त्यशि॑ष्यत॒ तत्त्वष्टा॑हव॒नीय॒मुप॒ प्राव॑र्तय॒थ्\-स्वाहेन्द्र॑शत्रुर्वर्ध॒स्वेति॒ स याव॑दू॒र्ध्वः प॑रा॒विध्य॑ति॒ ताव॑ति स्व॒यमे॒व व्य॑रमत॒ यदि॑ वा॒ ताव॑त्प्रव॒ण-~(३६)

%2.4.12.2
मासी॒द्यदि॑ वा॒ ताव॒दध्य॒ग्नेरासी॒थ्स स॒म्भव॑न्न॒ग्नी\-षोमा॑व॒भि सम॑भव॒थ्स इ॑षुमा॒त्रमि॑षुमात्रं॒ विष्व॑ङ्ङवर्धत॒ स इ॒माँल्लो॒कान॑वृणो॒द्य\-दि॒माँल्लो॒कान\-वृ॑णो॒त् तद्वृ॒त्रस्य॑ वृत्र॒त्वं तस्मा॒दिन्द्रो॑\-ऽबिभे॒दपि॒ त्वष्टा॒ तस्मै॒ त्वष्टा॒ वज्र॑मसिञ्च॒त् तपो॒ वै स वज्र॑ आसी॒त् तमुद्य॑न्तुं॒ नाश॑क्नो॒दथ॒ वै तर्\mbox{}हि॒ विष्णु॑-~(३७)

%2.4.12.3
र॒न्या दे॒वता॑सी॒थ्सो᳚\-ऽब्रवी॒द्विष्ण॒वेही॒दमा ह॑रिष्यावो॒ येना॒यमि॒दमिति॒ स विष्णु॑स्त्रे॒धाऽऽत्मानं॒ वि न्य॑धत्त पृथि॒व्यां तृती॑यम॒न्तरि॑क्षे॒ तृती॑यं दि॒वि तृती॑यमभिपर्याव॒र्ताद्ध्यबि॑भे॒द्यत्पृ॑थि॒व्यां तृती॑य॒मासी॒त् तेनेन्द्रो॒ वज्र॒मुद॑यच्छ॒द्विष्ण्व॑नुस्थितः॒ सो᳚\-ऽब्रवी॒न्मा मे॒ प्र हा॒रस्ति॒ वा इ॒दं~(३८)

%2.4.12.4
मयि॑ वी॒र्यं॑ तत्ते॒ प्र दा᳚स्या॒मीति॒ तद॑स्मै॒ प्राय॑च्छ॒त् तत् प्रत्य॑\-गृह्णा॒दधा॒ मेति॒ तद्विष्ण॒वेति॒ प्राय॑च्छ॒त् तद्विष्णुः॒ प्रत्य॑\-गृह्णाद॒स्मास्विन्द्र॑ इन्द्रि॒यं द॑धा॒त्विति॒ यद॒न्तरि॑क्षे॒ तृती॑य॒मासी॒त् तेनेन्द्रो॒ वज्र॒मुद॑यच्छ॒द्विष्ण्व॑नुस्थितः॒ सो᳚\-ऽब्रवी॒न्मा मे॒ प्र हा॒रस्ति॒ वा इ॒दं~(३९)

%2.4.12.5
मयि॑ वी॒र्यं॑ तत्ते॒ प्र दा᳚स्या॒मीति॒ तद॑स्मै॒ प्राय॑च्छ॒त् तत्प्रत्य॑\-गृह्णा॒द् द्विर्मा॑धा॒ इति॒ तद्विष्ण॒वेति॒ प्राय॑च्छ॒त् तद्विष्णुः॒ प्रत्य॑\-गृह्णाद॒स्मास्विन्द्र॑ इन्द्रि॒यं द॑धा॒त्विति॒ यद्दि॒वि तृती॑य॒मासी॒त् तेनेन्द्रो॒ वज्र॒मुद॑यच्छ॒द्विष्ण्व॑नुस्थितः॒ सो᳚\-ऽब्रवी॒न्मा मे॒ प्र हा॒र्येना॒ह-~(४०)

%2.4.12.6
मि॒दमस्मि॒ तत्ते॒ प्र दा᳚स्या॒मीति॒ त्वी~(३) इत्य॑ब्रवीथ्स॒न्धान्तु सं द॑धावहै॒ त्वामे॒व प्र वि॑शा॒नीति॒ यन्मां प्र॑वि॒शेः किं मा॑ भुञ्ज्या॒ इत्य॑ब्रवी॒त् त्वामे॒वेन्धी॑य॒ तव॒ भोगा॑य॒ त्वां प्र वि॑शेय॒मित्य॑ब्रवी॒त्तं वृ॒त्रः प्रावि॑शदु॒दरं॒ वै वृ॒त्रः क्षुत्खलु॒ वै म॑नु॒ष्य॑स्य॒ भ्रातृ॑व्यो॒ य~-~(४१)

%2.4.12.7
ए॒वं वेद॒ हन्ति॒ क्षुधं॒ भ्रातृ॑व्यं॒ तद॑स्मै॒ प्राय॑च्छ॒त् तत्प्रत्य॑\-गृह्णा॒त् त्रिर्मा॑धा॒ इति॒ तद्विष्ण॒वेति॒ प्राय॑च्छ॒त् तद्विष्णुः॒ प्रत्य॑\-गृह्णाद॒स्मास्विन्द्र॑ इन्द्रि॒यं द॑धा॒त्विति॒ यत् त्रिः प्राय॑च्छ॒त् त्रिः प्र॒त्यगृ॑ह्णा॒त् तत् त्रि॒धातो᳚स्त्रिधातु॒त्वं यद्विष्णु॑र॒न्वति॑ष्ठत॒ विष्ण॒वेति॒ प्राय॑च्छ॒त् तस्मा॑दैन्द्रा\-वैष्ण॒वꣳ ह॒विर्भ॑वति॒ यद्वा इ॒दं किं च॒ तद॑स्मै॒ तत्प्राय॑च्छ॒दृचः॒ सामा॑नि॒ यजूꣳ॑षि स॒हस्रं॒ वा अ॑स्मै॒ तत्प्राय॑च्छ॒त् तस्मा᳚थ्स॒हस्र॑द\-क्षिणम्॥~(४२)

%2.4.13.0
{\anuvakamend[{प्र॒व॒णं विष्णु॒र्वा इ॒दमि॒दम॒हं यो भ॑व॒त्येक॑विꣳशतिश्च}]}%॥12॥

%2.4.13.1
दे॒वा वै रा॑ज॒न्या᳚ज्जाय॑मानादबिभयु॒स्तम॒न्तरे॒व सन्तं॒ दाम्नापौ᳚म्भ॒न्थ्स वा ए॒षो\-ऽपो᳚ब्धो जायते॒ यद्रा॑ज॒न्यो॑ यद्वा ए॒षो\-ऽन॑पोब्धो॒ जाये॑त वृ॒त्रान्घ्नꣴश्च॑रे॒द्यं का॒मये॑त राज॒न्य॑मन॑पोब्धो जायेत वृ॒त्रान्घ्नꣴश्च॑रे॒दिति॒ तस्मा॑ ए॒तमै᳚न्द्राबार्\mbox{}हस्प॒त्यं च॒रुं निर्व॑पेदै॒न्द्रो वै रा॑ज॒न्यो᳚ ब्रह्म॒ बृह॒स्पति॒र्ब्रह्म॑णै॒वैनं॒ दाम्नो॒\-ऽपोम्भ॑नान्मुञ्चति हिर॒ण्मयं॒ दाम॒ दक्षि॑णा सा॒क्षादे॒वैनं॒ दाम्नो॒\-ऽपोम्भ॑नान्मुञ्चति॥~(४३)

%2.4.14.0
{\anuvakamend[{ए॒नं॒ द्वाद॑श च}]}%॥13॥

%2.4.14.1
नवो॑नवो भवति॒ जाय॑मा॒नो\-ऽह्नां᳚ के॒तुरु॒षसा॑मे॒त्यग्रे᳚। भा॒गं दे॒वे\-भ्यो॒ वि द॑धात्या॒यन्प्र च॒न्द्रमा᳚स्तिरति दी॒र्घमायुः॑। यमा॑दि॒त्या अ॒ꣳ॒शुमा᳚\-प्या॒यय॑न्ति॒ यमक्षि॑त॒मक्षि॑तयः॒ पिब॑न्ति। तेन॑ नो॒ राजा॒ वरु॑णो॒ बृह॒स्पति॒रा प्या॑ययन्तु॒ भुव॑नस्य गो॒पाः। प्राच्यां᳚ दि॒शि त्वमि॑न्द्रासि॒ राजो॒तोदी᳚च्यां वृत्रहन्वृत्र॒हाऽसि॑। यत्र॒ यन्ति॑ स्रो॒त्यास्त-~(४४)

%2.4.14.2
ज्जि॒तं ते॑ दक्षिण॒तो वृ॑ष॒भ ए॑धि॒ हव्यः॑। इन्द्रो॑ जयाति॒ न परा॑ जयाता अधिरा॒जो राज॑सु राजयाति। विश्वा॒ हि भू॒याः पृत॑ना अभि॒ष्टीरु॑प॒सद्यो॑ नम॒स्यो॑ यथास॑त्। अ॒स्येदे॒व प्र रि॑रिचे महि॒त्वं दि॒वः पृ॑थि॒व्याः पर्य॒न्तरि॑क्षात्। स्व॒राडिन्द्रो॒ दम॒ आ वि॒श्वगू᳚र्तः स्व॒रिरम॑त्रो ववक्षे॒ रणा॑य। अ॒भि त्वा॑ शूर नोनु॒मो\-ऽदु॑ग्धा इव धे॒नवः॑। ईशा॑न-~(४५)

%2.4.14.3
म॒स्य जग॑तः सुव॒र्दृश॒मीशा॑नमिन्द्र त॒स्थुषः॑। त्वामिद्धि हवा॑महे सा॒ता वाज॑स्य का॒रवः॑। त्वां वृ॒त्रेष्वि॑न्द्र॒ सत्प॑तिं॒ नर॒स्त्वां काष्ठा॒स्वर्व॑तः। यद्द्याव॑ इन्द्र ते श॒तꣳ श॒तं भूमी॑रु॒त स्युः। न त्वा॑ वज्रिन्थ्स॒हस्र॒ꣳ॒ सूर्या॒ अनु॒ न जा॒तम॑ष्ट॒ रोद॑सी। पिबा॒ सोम॑मिन्द्र॒ मन्द॑तु त्वा॒ यं ते॑ सु॒षाव॑ हर्य॒श्वाद्रिः॑।~(४६)

%2.4.14.4
सो॒तुर्बा॒हुभ्या॒ꣳ॒ सुय॑तो॒ नार्वा᳚। रे॒वती᳚र्नः सध॒माद॒ इन्द्रे॑ सन्तु तु॒विवा॑जाः। क्षु॒मन्तो॒ याभि॒र्मदे॑म। उद॑ग्ने॒ शुच॑य॒स्तव॒ वि ज्योति॒षोदु॒ त्यं जा॒तवे॑दसꣳ स॒प्त त्वा॑ ह॒रितो॒ रथे॒ वह॑न्ति देव सूर्य। शो॒चिष्के॑शं विचक्षण। चि॒त्रं दे॒वाना॒मुद॑गा॒दनी॑कं॒ चक्षु॑र्मि॒त्रस्य॒ वरु॑णस्या॒ग्नेः। आ\-ऽप्रा॒ द्यावा॑पृथि॒वी अ॒न्तरि॑क्ष॒ꣳ॒ सूर्य॑ आ॒त्मा जग॑तस्त॒स्थुष॑-~(४७)

%2.4.14.5
श्च॒। विश्वे॑ दे॒वा ऋ॑ता॒वृध॑ ऋ॒तुभि॑र्\mbox{}हवन॒श्रुतः॑। जु॒षन्तां॒ युज्यं॒ पयः॑। विश्वे॑ देवाः शृणु॒तेमꣳ हवं॑ मे॒ ये अ॒न्तरि॑क्षे॒ य उप॒ द्यवि॒ ष्ठ। ये अ॑ग्निजि॒ह्वा उ॒त वा॒ यज॑त्रा आ॒सद्या॒स्मिन्ब॒र्॒\mbox{}हिषि॑ मादयध्वम्॥~(४८)
{\anuvakamend[{तदीशा॑न॒मद्रि॑स्त॒स्थुष॑स्त्रि॒ꣳ॒शच्च॑}]}%॥14॥

{\prashnaend[दे॒वा म॑नु॒ष्या॑ देवासु॒रा अ॑ब्रुवन्देवासु॒रास्तेषा᳚ङ्गाय॒त्री प्र॒जा\-प॑ति॒स्ता यत्राग्ने॒ गोभि॑श्चि॒त्रया॑ मारु॒तन्देवा॑ वसव्या॒ अग्ने॑ मारु॒तमिति॒ देवा॑ वसव्या॒ देवाः᳚ शर्मण्या॒स्त्वष्टा॑ ह॒तपु॑त्रो दे॒वा वै रा॑ज॒न्या᳚न्नवो॑नव॒श्चतु॑र्दश॥१४॥ दे॒वा म॑नु॒ष्याः᳚ प्र॒जां प॒शून्देवा॑ वसव्याः परिद॒ध्यादि॒दमस्म्य॒ष्टाच॑त्वारिꣳशत्॥४८॥ दे॒वा म॑नु॒ष्या॑ मादयध्वम्॥]}
%%% END PRASHNA

\sect{पञ्चमः प्रश्नः}\setcounter{anuvakam}{0}
\dnsub{तैत्तिरीयसंहितायां द्वितीयकाण्डे पञ्चमः प्रश्नः}
%2.5.1.0
%2.5.1.1
वि॒श्वरू॑पो॒ वै त्वा॒ष्ट्रः पु॒रोहि॑तो दे॒वाना॑मासीथ्स्व॒स्रीयो\-ऽसु॑राणां॒ तस्य॒ त्रीणि॑ शी॒र्॒\mbox{}षाण्या॑सन्थ्सोम॒पानꣳ॑ सुरा॒पान॑म॒न्नाद॑न॒ꣳ॒ स प्र॒त्यक्षं॑ दे॒वेभ्यो॑ भा॒गम॑वदत्प॒रोक्ष॒मसु॑रेभ्यः॒ सर्व॑स्मै॒ वै प्र॒त्यक्षं॑ भा॒गं व॑दन्ति॒ यस्मा॑ ए॒व प॒रोक्षं॒ वद॑न्ति॒ तस्य॑ भा॒ग उ॑दि॒तस्तस्मा॒दिन्द्रो॑\-ऽबिभेदी॒दृङ् वै रा॒ष्ट्रं वि प॒र्याव॑र्तय॒तीति॒ तस्य॒ वज्र॑मा॒दाय॑ शी॒र्॒\mbox{}षाण्य॑च्छिन॒द्यथ्सो॑म॒पान॒-~(१)

%2.5.1.2
मासी॒थ्स क॒पिञ्ज॑लो\-ऽभव॒द्यथ्सु॑रा॒पान॒ꣳ॒ स क॑ल॒विङ्को॒ यद॒न्नाद॑न॒ꣳ॒ स ति॑त्ति॒रिस्तस्या᳚ञ्ज॒लिना᳚ ब्रह्मह॒त्यामुपा॑गृह्णा॒त्ताꣳ सं॑वथ्स॒रम॑बिभ॒स्तं भू॒तान्य॒भ्य॑क्रोश॒न्ब्रह्म॑ह॒न्निति॒ स पृ॑थि॒वीमुपा॑सीद\-द॒स्यै ब्र॑ह्मह॒त्यायै॒ तृती॑यं॒ प्रति॑ गृहा॒णेति॒ साऽब्र॑वी॒द्वरं॑ वृणै खा॒तात्प॑रा\-भवि॒ष्यन्ती॑ मन्ये॒ ततो॒ मा परा॑ भूव॒मिति॑ पु॒रा ते॑~(२)

%2.5.1.3
संवथ्स॒रादपि॑ रोहा॒दित्य॑ब्रवी॒त्तस्मा᳚त्पु॒रा सं॑वथ्स॒रात्पृ॑थि॒व्यै खा॒तमपि॑ रोहति॒ वारे॑वृत॒ꣴ॒ ह्य॑स्यै॒ तृती॑यं ब्रह्मह॒त्यायै॒ प्रत्य॑\-गृह्णा॒त् तथ्स्वकृ॑त॒मिरि॑णमभव॒त् तस्मा॒दाहि॑ताग्निः श्र॒द्धादे॑वः॒ स्वकृ॑त॒ इरि॑णे॒ नाव॑ स्येद्ब्रह्मह॒त्यायै॒ ह्ये॑ष वर्णः॒ स वन॒स्पती॒नुपा॑सीदद॒स्यै ब्र॑ह्मह॒त्यायै॒ तृती॑यं॒ प्रति॑ गृह्णी॒तेति॒ ते᳚\-ऽब्रुव॒न्वरं॑ वृणामहै वृ॒क्णात्~(३)

%2.5.1.4
प॑राभवि॒ष्यन्तो॑ मन्यामहे॒ ततो॒ मा परा॑ भू॒मेत्या॒व्रश्च॑नाद्वो॒ भूयाꣳ॑स॒ उत्ति॑ष्ठा॒नित्य॑ब्रवी॒त् तस्मा॑दा॒व्रश्च॑नाद्वृ॒क्षाणां॒ भूयाꣳ॑स॒ उत्ति॑ष्ठन्ति॒ वारे॑वृत॒ꣴ॒ ह्ये॑षां॒ तृती॑यं ब्रह्मह॒त्यायै॒ प्रत्य॑\-गृह्ण॒न्थ्स नि॑र्या॒सो॑\-ऽभव॒त् तस्मा᳚न्निर्या॒सस्य॒ नाश्यं॑ ब्रह्मह॒त्यायै॒ ह्ये॑ष वर्णो\-ऽथो॒ खलु॒ य ए॒व लोहि॑तो॒ यो वा॒\-ऽ\-ऽव्रश्च॑नान्नि॒र्येष॑ति॒ तस्य॒ नाऽऽश्यं॑~(४)

%2.5.1.5
काम॑म॒न्यस्य॒ स स्त्री॑षꣳसा॒दमुपा॑सीदद॒स्यै ब्र॑ह्मह॒त्यायै॒ तृती॑यं॒ प्रति॑ गृह्णी॒तेति॒ ता अ॑ब्रुव॒न्वरं॑ वृणामहा॒ ऋत्वि॑यात्प्र॒जां वि॑न्दामहै॒ काम॒मा विज॑नितोः॒ सम्भ॑वा॒मेति॒ तस्मा॒दृत्वि॑या॒थ्स्त्रियः॑ प्र॒जां वि॑न्दन्ते॒ काम॒मा विज॑नितोः॒ सम्भ॑वन्ति॒ वारे॑वृत॒ꣴ॒ ह्या॑सां॒ तृती॑यं ब्रह्मह॒त्यायै॒ प्रत्य॑\-गृह्ण॒न्थ्सा मल॑वद्वासा अभव॒त् तस्मा॒न्मल॑वद्वाससा॒ न सं व॑देत॒~(५)

%2.5.1.6
न स॒हाऽऽसी॑त॒ नास्या॒ अन्न॑मद्याद्ब्रह्मह॒त्यायै॒ ह्ये॑षा वर्णं॑ प्रति॒मुच्याऽऽस्ते\-ऽथो॒ खल्वा॑हुर॒भ्यञ्ज॑नं॒ वाव स्त्रि॒या अन्न॑म॒भ्यञ्ज॑नमे॒व न प्र॑ति॒गृह्यं॒ काम॑म॒न्यदिति॒ यां मल॑वद्वाससꣳ स॒म्भव॑न्ति॒ यस्ततो॒ जाय॑ते॒ सो॑\-ऽभिश॒स्तो यामर॑ण्ये॒ तस्यै᳚ स्ते॒नो यां परा॑चीं॒ तस्यै᳚ ह्रीतमु॒ख्य॑पग॒ल्भो या स्नाति॒ तस्या॑ अ॒फ्सु मारु॑को॒ या-~(६)

%2.5.1.7
ऽभ्य॒ङ्क्ते तस्यै॑ दु॒श्चर्मा॒ या प्र॑लि॒खते॒ तस्यै॑ खल॒तिर॑पमा॒री या\-ऽ\-ऽङ्क्ते तस्यै॑ का॒णो या द॒तो धाव॑ते॒ तस्यै᳚ श्या॒वद॒न्॒ या न॒खानि॑ निकृ॒न्तते॒ तस्यै॑ कुन॒खी या कृ॒णत्ति॒ तस्यै᳚ क्ली॒बो या रज्जुꣳ॑ सृ॒जति॒ तस्या॑ उ॒द्बन्धु॑को॒ या प॒र्णेन॒ पिब॑ति॒ तस्या॑ उ॒न्मादु॑को॒ या ख॒र्वेण॒ पिब॑ति॒ तस्यै॑ ख॒र्वस्ति॒स्रो रात्री᳚र्व्र॒तं च॑रेदञ्ज॒लिना॑ वा॒ पिबे॒दख॑र्वेण वा॒ पात्रे॑ण प्र॒जायै॑ गोपी॒थाय॑॥~(७)

%2.5.2.0
{\anuvakamend[{यथ्सो॑म॒पान॑न्ते वृ॒क्णात् तस्य॒ नाश्यं॑ वदेत॒ मारु॑को॒ या\-ऽख॑र्वेण वा॒ त्रीणि॑ च}]}%~(१)

%2.5.2.1
त्वष्टा॑ ह॒तपु॑त्रो॒ वीन्द्र॒ꣳ॒ सोम॒माह॑र॒त् तस्मि॒न्निन्द्र॑ उपह॒वमै᳚च्छत॒ तं नोपा᳚ह्वयत पु॒त्रं मे॑\-ऽवधी॒रिति॒ स य॑ज्ञवेश॒सं कृ॒त्वा प्रा॒सहा॒ सोम॑मपिब॒त् तस्य॒ यद॒त्यशि॑ष्यत॒ तत् त्वष्टा॑हव॒नीय॒मुप॒ प्राव॑र्तय॒थ्\-स्वाहेन्द्र॑शत्रुर्वर्ध॒स्वेति॒ यदव॑र्तय॒त् तद्वृ॒त्रस्य॑ वृत्र॒त्वं यदब्र॑वी॒थ्\-स्वाहेन्द्र॑शत्रुर्वर्ध॒स्वेति॒ तस्मा॑द॒स्ये-~(८)

%2.5.2.2
न्द्रः॒ शत्रु॑रभव॒थ्स स॒म्भव॑न्न॒ग्नी\-षोमा॑व॒भि सम॑भव॒थ्स इ॑षुमा॒त्रमि॑षुमात्रं॒ विष्व॑ङ्ङवर्धत॒ स इ॒माँल्लो॒कान॑वृणो॒द् यदि॒माँल्लो॒का\-नवृ॑णो॒त् तद्वृ॒त्रस्य॑ वृत्र॒त्वं तस्मा॒दिन्द्रो॑\-ऽबिभे॒थ्स प्र॒जा\-प॑ति॒मुपा॑\-धाव॒च्छत्रु॑र्मे\-ऽज॒नीति॒ तस्मै॒ वज्रꣳ॑ सि॒क्त्वा प्राय॑च्छदे॒तेन॑ ज॒हीति॒ तेना॒भ्या॑यत॒ ताव॑ब्रूताम॒ग्नी\-षोमौ॒ मा~(९)

%2.5.2.3
प्र हा॑रा॒वम॒न्तः स्व॒ इति॒ मम॒ वै यु॒वꣴ स्थ॒ इत्य॑ब्रवी॒न्माम॒भ्ये\-त॒मिति॒ तौ भा॑ग॒धेय॑मैच्छेतां॒ ताभ्या॑मे॒त\-म॑ग्नीषो॒मीय॒\-मेका॑\-दश\-कपालं पू॒र्णमा॑से॒ प्राय॑च्छ॒त् ताव॑ब्रूताम॒भि सन्द॑ष्टौ॒ वै स्वो॒ न श॑क्नुव॒ ऐतु॒मिति॒ स इन्द्र॑ आ॒त्मनः॑ शीतरू॒राव॑जनय॒त् तच्छी॑तरू॒रयो॒र्जन्म॒ य ए॒वꣳ शी॑तरू॒रयो॒र्जन्म॒ वेद॒~(१०)


%2.5.2.4
नैनꣳ॑ शीतरू॒रौ ह॑त॒स्ताभ्या॑मेनम॒भ्य॑नय॒त् तस्मा᳚ज्जञ्ज॒भ्यमा॑\-नाद॒ग्नी\-षोमौ॒ निर॑क्रामतां प्राणापा॒नौ वा ए॑नं॒ तद॑जहितां प्रा॒णो वै दक्षो॑\-ऽपा॒नः क्रतु॒स्तस्मा᳚ज्जञ्ज॒भ्यमा॑नो ब्रूया॒न्मयि॑ दक्षक्र॒तू इति॑ प्राणापा॒नावे॒वात्मन्ध॑त्ते॒ सर्व॒मायु॑रेति॒ स दे॒वता॑ वृ॒त्रान्नि॒र्॒\mbox{}हूय॒ वार्त्र॑घ्नꣳ ह॒विः पू॒र्णमा॑से॒ निर॑वप॒द् घ्नन्ति॒ वा ए॑नं पू॒र्णमा॑स॒ आ-~(११)

%2.5.2.5
ऽमा॑वा॒स्या॑यां प्याययन्ति॒ तस्मा॒द्वार्त्र॑घ्नी पू॒र्णमा॒से\-ऽनू᳚च्येते॒ वृध॑न्वती अमावा॒स्या॑यां॒ तथ्स॒ꣴ॒स्थाप्य॒ वार्त्र॑घ्नꣳ ह॒विर्वज्र॑मा॒दाय॒ पुन॑र॒भ्या॑यत॒ ते अ॑ब्रूतां॒ द्यावा॑पृथि॒वी मा प्र हा॑रा॒वयो॒र्वै श्रि॒त इति॒ ते अ॑ब्रूतां॒ वरं॑ वृणावहै॒ नक्ष॑त्रविहिता॒\-ऽहमसा॒नीत्य॒साव॑ब्रवीच्चि॒त्रवि॑हिता॒\-ऽहमिती॒यं तस्मा॒न्नक्ष॑त्रविहिता॒\-ऽसौ चि॒त्रवि॑हिते॒यं य ए॒वं द्यावा॑पृथि॒व्योर्-~(१२)

%2.5.2.6
वरं॒ वेदैनं॒ वरो॑ गच्छति॒ स आ॒भ्यामे॒व प्रसू॑त॒ इन्द्रो॑ वृ॒त्रम॑ह॒न्ते दे॒वा वृ॒त्रꣳ ह॒त्वा\-ऽग्नी\-षोमा॑वब्रुवन् ह॒व्यं नो॑ वहत॒मिति॒ ताव॑ब्रूता॒मप॑तेजसौ॒ वै त्यौ वृ॒त्रे वै त्ययो॒स्तेज॒ इति॒ ते᳚\-ऽब्रुव॒न्क इ॒दमच्छै॒तीति॒ गौरित्य॑ब्रुव॒न्गौर्वाव सर्व॑स्य मि॒त्रमिति॒ सा\-ऽब्र॑वी॒द्~(१३)

%2.5.2.7
वरं॑ वृणै॒ मय्ये॒व स॒तोभये॑न भुनजाध्वा॒ इति॒ तद्गौराह॑र॒त् तस्मा॒द्गवि॑ स॒तोभये॑न भुञ्जत ए॒तद्वा अ॒ग्नेस्तेजो॒ यद् घृ॒तमे॒तथ्सोम॑स्य॒ यत्पयो॒ य ए॒वम॒ग्नी\-षोम॑यो॒स्तेजो॒ वेद॑ तेज॒स्व्ये॑व भ॑वति ब्रह्मवा॒दिनो॑ वदन्ति किं देव॒त्यं॑ पौर्णमा॒समिति॑ प्राजाप॒त्यमिति॑ ब्रूया॒त् तेनेन्द्रं॑ ज्ये॒ष्ठं पु॒त्रं नि॒रवा॑सायय॒दिति॒ तस्मा᳚ज्ज्ये॒ष्ठं पु॒त्रं धने॑न नि॒रव॑साययन्ति॥~(१४)

%2.5.3.0
{\anuvakamend[{अ॒स्य॒ मा वेदा द्यावा॑पृथि॒व्योर॑ब्रवी॒दिति॒ तस्मा᳚च्च॒त्वारि॑ च}]}%~(२)

%2.5.3.1
इन्द्रं॑ वृ॒त्रं ज॑घ्नि॒वाꣳस॒म्मृधो॒\-ऽभि प्रावे॑पन्त॒ स ए॒तं वै॑मृ॒धं पू॒णमा॑से\-ऽनुनिर्वा॒प्य॑मपश्य॒त्तं निर॑वप॒त् तेन॒ वै स मृधो\-ऽपा॑हत॒ यद्वै॑मृ॒धः पू॒र्णमा॑से\-ऽनुनिर्वा॒प्यो॑ भव॑ति॒ मृध॑ ए॒व तेन॒ यज॑मा॒नो\-ऽप॑ हत॒ इन्द्रो॑ वृ॒त्रꣳ ह॒त्वा दे॒वता॑भिश्चेन्द्रि॒येण॑ च॒ व्या᳚र्ध्यत॒ स ए॒तमा᳚ग्ने॒यम॒ष्टाक॑पालममावा॒स्या॑यामपश्यदै॒न्द्रं दधि॒~(१५)

%2.5.3.2
तन्निर॑वप॒त्तेन॒ वै स दे॒वता᳚श्चेन्द्रि॒यं चावा॑रुन्ध॒ यदा᳚ग्ने॒यो᳚\-ऽष्टाक॑पालो\-ऽमावा॒स्या॑यां॒ भव॑त्यै॒न्द्रं दधि॑ दे॒वता᳚श्चै॒व तेने᳚न्द्रि॒यं च॒ यज॑मा॒नो\-ऽव॑ रुन्ध॒ इन्द्र॑स्य वृ॒त्रं ज॒घ्नुष॑ इन्द्रि॒यं वी॒र्यं॑ पृथि॒वीमनु॒ व्या᳚र्च्छ॒त् तदोष॑धयो वी॒रुधो॑\-ऽभव॒न्थ्स प्र॒जा\-प॑ति॒मुपा॑धावद्वृ॒त्रं मे॑ ज॒घ्नुष॑ इन्द्रि॒यं वी॒र्यं॑~(१६)

%2.5.3.3
पृथि॒वीमनु॒ व्या॑र॒त् तदोष॑धयो वी॒रुधो॑\-ऽभूव॒न्निति॒ स प्र॒जा\-प॑तिः प॒शून॑ब्रवीदे॒तद॑स्मै॒ सं न॑य॒तेति॒ तत्प॒शव॒ ओष॑धी॒भ्यो\-ऽध्या॒त्मन्थ्सम॑नय॒न्तत्प्रत्य॑दुह॒न्॒ यथ्स॒मन॑य॒न्तथ्सा᳚न्ना॒य्यस्य॑ सान्नाय्य॒त्वं यत्प्र॒त्यदु॑ह॒न्तत्प्र॑ति॒धुषः॑ प्रतिधु॒क्त्वꣳ सम॑नैषुः॒ प्रत्य॑धुक्ष॒न्न तु मयि॑ श्रयत॒ इत्य॑ब्रवीदे॒तद॑स्मै~(१७)

%2.5.3.4
शृ॒तं कु॑रु॒तेत्य॑ब्रवी॒त् तद॑स्मै शृ॒तम॑कुर्वन्निन्द्रि॒यं वावास्मि॑न्वी॒र्यं॑ तद॑श्रय॒न्तच्छृ॒तस्य॑ शृत॒त्वꣳ सम॑नैषुः॒ प्रत्य॑धुक्षञ्छृ॒तम॑क्र॒न्न तु मा॑ धिनो॒तीत्य॑ब्रवीदे॒तद॑स्मै॒ दधि॑ कुरु॒तेत्य॑ब्रवी॒त् तद॑स्मै॒ दध्य॑कुर्व॒न्तदे॑नमधिनो॒त् तद्द॒ध्नो द॑धि॒त्वं ब्र॑ह्मवा॒दिनो॑ वदन्ति द॒ध्नः पूर्व॑स्याव॒देयं॒~(१८)

%2.5.3.5
दधि॒ हि पूर्वं॑ क्रि॒यत॒ इत्यना॑दृत्य॒ तच्छृ॒तस्यै॒व पूर्व॒स्याऽव॑ द्येदिन्द्रि॒यमे॒वास्मि॑न्वी॒र्यꣴ॑ श्रि॒त्वा द॒ध्नोपरि॑ष्टाद्धिनोति यथापू॒र्वमुपै॑ति॒ यत्पू॒तीकै᳚र्वा पर्णव॒ल्कैर्वा॑त॒ञ्च्याथ्सौ॒म्यं तद्यत्क्व॑लै राक्ष॒सं तद्यत् त॑ण्डु॒लैर्वै᳚श्वदे॒वं तद्यदा॒तञ्च॑नेन मानु॒षं तद्यद्द॒ध्ना तथ्सेन्द्रं॑ द॒ध्ना त॑नक्ति~(१९)

%2.5.3.6
सेन्द्र॒त्वाया᳚ग्निहोत्रोच्छेष॒णम॒भ्यात॑नक्ति य॒ज्ञस्य॒ सन्त॑त्या॒ इन्द्रो॑ वृ॒त्रꣳ ह॒त्वा परां᳚ परा॒वत॑मगच्छ॒दपा॑राध॒मिति॒ मन्य॑मान॒स्तं दे॒वताः॒ प्रैष॑मैच्छ॒न्थ्सो᳚\-ऽब्रवीत्प्र॒जा\-प॑ति॒र्यः प्र॑थ॒मो॑\-ऽनुवि॒न्दति॒ तस्य॑ प्रथ॒मं भा॑ग॒धेय॒मिति॒ तं पि॒तरो\-ऽन्व॑विन्द॒न्तस्मा᳚त्पि॒तृभ्यः॑ पूर्वे॒द्युः क्रि॑यते॒ सो॑\-ऽमावा॒स्यां᳚ प्रत्याग॑च्छ॒त् तं दे॒वा अ॒भि सम॑गच्छन्ता॒ऽमा वै नो॒~-~(२०)

%2.5.3.7
ऽद्य वसु॑ वस॒तीतीन्द्रो॒ हि दे॒वानां॒ वसु॒ तद॑मावा॒स्या॑या अमावास्य॒त्वं ब्र॑ह्मवा॒दिनो॑ वदन्ति किं देव॒त्यꣳ॑ सान्ना॒य्यमिति॑ वैश्वदे॒वमिति॑ ब्रूया॒द्विश्वे॒ हि तद्दे॒वा भा॑ग॒धेय॑म॒भि स॒मग॑च्छ॒न्तेत्यथो॒ खल्वै॒न्द्रमित्ये॒व ब्रू॑या॒दिन्द्रं॒ वाव ते तद्भि॑ष॒ज्यन्तो॒\-ऽभि सम॑गच्छ॒न्तेति॑॥~(२१)

%2.5.4.0
{\anuvakamend[{दधि॑ मे ज॒घ्नुष॑ इन्द्रि॒यं वी॒र्य॑मित्य॑ब्रवीदे॒तद॑स्मा अव॒देय॑न्तनक्ति नो॒ द्विच॑त्वारिꣳशच्च}]}%~(३)

%2.5.4.1
ब्र॒ह्म॒वा॒दिनो॑ वदन्ति॒ स त्वै द॑र्\mbox{}शपूर्णमा॒सौ य॑जेत॒ य ए॑नौ॒ सेन्द्रौ॒ यजे॒तेति॑ वैमृ॒धः पू॒र्णमा॑से\-ऽनुनिर्वा॒प्यो॑ भवति॒ तेन॑ पू॒र्णमा॑सः॒ सेन्द्र॑ ऐ॒न्द्रं दध्य॑मावा॒स्या॑यां॒ तेना॑मावा॒स्या॑ सेन्द्रा॒ य ए॒वं वि॒द्वान्द॑र्\mbox{}शपूर्णमा॒सौ यज॑ते॒ सेन्द्रा॑वे॒वैनौ॑ यजते॒ श्वःश्वो᳚\-ऽस्मा ईजा॒नाय॒ वसी॑यो भवति दे॒वा वै यद्य॒ज्ञे\-ऽकु॑र्वत॒ तदसु॑रा अकुर्वत॒ ते दे॒वा ए॒ता-~(२२)

%2.5.4.2
मिष्टि॑मपश्यन्नाग्नावैष्ण॒वमेका॑\-दश\-कपाल॒ꣳ॒ सर॑स्वत्यै च॒रुꣳ सर॑स्वते च॒रुं तां पौ᳚र्णमा॒सꣳ स॒ꣴ॒स्थाप्यानु॒ निर॑वप॒न्ततो॑ दे॒वा अभ॑व॒न्परासु॑रा॒ यो भ्रातृ॑व्यवा॒न्थ्स्याथ्स पौ᳚र्णमा॒सꣳ स॒ꣴ॒स्थाप्यै॒तामिष्टि॒मनु॒ निर्व॑पेत्पौर्णमा॒सेनै॒व वज्रं॒ भ्रातृ॑व्याय प्र॒हृत्या᳚ऽऽग्नावैष्ण॒वेन॑ दे॒वता᳚श्च य॒ज्ञं च॒ भ्रातृ॑व्यस्य वृङ्क्ते मिथु॒नान्प॒शून्थ्सा॑\-रस्व॒ताभ्यां॒ याव॑दे॒वास्यास्ति॒ तथ्~(२३)

%2.5.4.3
सर्वं॑ वृङ्क्ते पौर्णमा॒सीमे॒व य॑जेत॒ भ्रातृ॑व्यवा॒न्नामा॑वा॒स्याꣳ॑ ह॒त्वा भ्रातृ॑व्यं॒ ना प्या॑ययति साकं प्रस्था॒यीये॑न यजेत प॒शुका॑मो॒ यस्मै॒ वा अल्पे॑ना॒ऽऽहर॑न्ति॒ नाऽऽत्मना॒ तृप्य॑ति॒ नान्यस्मै॑ ददाति॒ यस्मै॑ मह॒ता तृप्य॑त्या॒त्मना॒ ददा᳚त्य॒न्यस्मै॑ मह॒ता पू॒र्णꣳ हो॑त॒व्यं॑ तृ॒प्त ए॒वैन॒मिन्द्रः॑ प्र॒जया॑ प॒शुभि॑स्तर्पयति दारुपा॒त्रेण॑ जुहोति॒ न हि मृ॒न्मय॒माहु॑तिमान॒श औदु॑म्बरं~(२४)

%2.5.4.4
भव॒त्यूर्ग्वा उ॑दु॒म्बर॒ ऊर्क्प॒शव॑ ऊ॒र्जैवास्मा॒ ऊर्जं॑ प॒शूनव॑ रुन्धे॒ नाग॑तश्रीर्महे॒न्द्रं य॑जेत॒ त्रयो॒ वै ग॒तश्रि॑यः शुश्रु॒वान्ग्रा॑म॒णी रा॑ज॒न्य॑स्तेषां᳚ महे॒न्द्रो दे॒वता॒ यो वै स्वां दे॒वता॑मति॒यज॑ते॒ प्र स्वायै॑ दे॒वता॑यै च्यवते॒ न परां॒ प्राप्नो॑ति॒ पापी॑यान्भवति संवथ्स॒रमिन्द्रं॑ यजेत संवथ्स॒रꣳ हि व्र॒तं नाति॒ स्वै-~(२५)

%2.5.4.5
वैनं॑ दे॒वते॒ज्यमा॑ना॒ भूत्या॑ इन्द्धे॒ वसी॑यान्भवति संवथ्स॒रस्य॑ प॒रस्ता॑\-द॒ग्नये᳚ व्र॒तप॑तये पुरो॒डाश॑\-म॒ष्टा\-क॑पालं॒ निर्व॑पेथ्संवथ्स॒रमे॒वैनं॑ वृ॒त्रं ज॑घ्नि॒वाꣳस॑म॒ग्निर्व्र॒तप॑तिर्व्र॒तमा ल॑म्भयति॒ ततो\-ऽधि॒ कामं॑ यजेत॥~(२६)

%2.5.5.0
{\anuvakamend[{ए॒तान्तदौदु॑म्बर॒ꣴ॒ स्वा त्रि॒ꣳ॒शच्च॑}]}%~(४)

%2.5.5.1
नासो॑मयाजी॒ सं न॑ये॒दना॑गतं॒ वा ए॒तस्य॒ पयो॒ यो\-ऽसो॑मयाजी॒ यदसो॑मयाजी स॒न्नये᳚त्परिमो॒ष ए॒व सो\-ऽनृ॑तं करो॒त्यथो॒ परै॒व सि॑च्यते सोमया॒ज्ये॑व सं न॑ये॒त्पयो॒ वै सोमः॒ पयः॑ सान्ना॒य्यं पय॑सै॒व पय॑ आ॒त्मन्ध॑त्ते॒ वि वा ए॒तं प्र॒जया॑ प॒शुभि॑रर्धयति व॒र्धय॑त्यस्य॒ भ्रातृ॑व्यं॒ यस्य॑ ह॒विर्निरु॑प्तं पु॒रस्ता᳚च्च॒न्द्रमा॑~-~(२७)

%2.5.5.2
अ॒भ्यु॑देति॑ त्रे॒धा त॑ण्डु॒लान् वि भ॑जे॒द्ये म॑ध्य॒माः स्युस्तान॒ग्नये॑ दा॒त्रे पु॑रो॒डाश॑म॒ष्टाक॑पालं कुर्या॒द्ये स्थवि॑ष्ठा॒स्तानिन्द्रा॑य प्रदा॒त्रे द॒धꣴश्च॒रुं ये\-ऽणि॑ष्ठा॒स्तान् विष्ण॑वे शिपिवि॒ष्टाय॑ शृ॒ते च॒रुम॒ग्निरे॒वास्मै᳚ प्र॒जां प्र॑ज॒नय॑ति वृ॒द्धामिन्द्रः॒ प्र य॑च्छति य॒ज्ञो वै विष्णुः॑ प॒शवः॒ शिपि॑र्य॒ज्ञ ए॒व प॒शुषु॒ प्रति॑ तिष्ठति॒ न द्वे~(२८)

%2.5.5.3
य॑जेत॒ यत्पूर्व॑या सम्प्र॒ति यजे॒तोत्त॑रया छ॒म्बट्कु॑र्या॒द्यदुत्त॑रया सम्प्र॒ति यजे॑त॒ पूर्व॑या छ॒म्बट्कु॑र्या॒न्नेष्टि॒र्भव॑ति॒ न य॒ज्ञस्तदनु॑ ह्रीतमु॒ख्य॑पग॒ल्भो जा॑यत॒ एका॑मे॒व य॑जेत प्रग॒ल्भो᳚\-ऽस्य जाय॒ते\-ऽना॑दृत्य॒ तद्द्वे ए॒व य॑जेत यज्ञमु॒खमे॒व पूर्व॑या॒ऽऽलभ॑ते॒ यज॑त॒ उत्त॑रया दे॒वता॑ ए॒व पूर्व॑या\-ऽवरु॒न्ध इ॑न्द्रि॒यमुत्त॑रया देवलो॒कमे॒व~(२९)

%2.5.5.4
पूर्व॑याऽभि॒जय॑ति मनुष्यलो॒कमुत्त॑रया॒ भूय॑सो यज्ञक्र॒तूनुपै᳚त्ये॒षा वै सु॒मना॒ नामेष्टि॒र्यम॒द्येजा॒नं प॒श्चाच्च॒न्द्रमा॑ अ॒भ्यु॑देत्य॒स्मिन्ने॒वास्मै॑ लो॒के\-ऽर्धु॑कं भवति दाक्षायणय॒ज्ञेन॑ सुव॒र्गका॑मो यजेत पू॒र्णमा॑से॒ सं न॑येन्मैत्रावरु॒ण्या\-ऽ\-ऽ\-मिक्ष॑याऽमावा॒स्या॑यां यजेत पू॒र्णमा॑से॒ वै दे॒वानाꣳ॑ सु॒तस्तेषा॑मे॒तम॑र्धमा॒सं प्रसु॑त॒स्तेषां᳚ मैत्रावरु॒णी व॒शाऽमा॑वा॒स्या॑यामनूब॒न्ध्या॑ यत्~(३०)

%2.5.5.5
पू᳚र्वे॒द्युर्यज॑ते॒ वेदि॑मे॒व तत्क॑रोति॒ यद्व॒थ्सान॑पाक॒रोति॑ सदोहविर्धा॒ने ए॒व सम्मि॑नोति॒ यद्यज॑ते दे॒वैरे॒व सु॒त्याꣳ सम्पा॑दयति॒ स ए॒तम॑र्धमा॒सꣳ स॑ध॒मादं॑ दे॒वैः सोमं॑ पिबति॒ यन्मै᳚त्रावरु॒ण्या\-ऽ\-ऽ\-मिक्ष॑याऽमावा॒स्या॑यां यज॑ते॒ यैवासौ दे॒वानां᳚ व॒शाऽनू॑ब॒न्ध्या॑ सो ए॒वैषैतस्य॑ सा॒क्षाद्वा ए॒ष दे॒वान॒भ्यारो॑हति॒ य ए॑षां य॒ज्ञ-~(३१)

%2.5.5.6
म॑भ्या॒रोह॑ति॒ यथा॒ खलु॒ वै श्रेया॑न॒भ्यारू॑ढः का॒मय॑ते॒ तथा॑ करोति॒ यद्य॑व॒विध्य॑ति॒ पापी॑यान्भवति॒ यदि॒ नाव॒विध्य॑ति स॒दृङ् व्या॒वृत्का॑म ए॒तेन॑ य॒ज्ञेन॑ यजेत क्षु॒रप॑वि॒र्॒\mbox{}ह्ये॑ष य॒ज्ञस्ता॒जक्पुण्यो॑ वा॒ भव॑ति॒ प्र वा॑ मीयते॒ तस्यै॒तद्व्र॒तं नानृ॑तं वदे॒न्न मा॒ꣳ॒सम॑श्ञीया॒न्न स्त्रिय॒\-मु\-पे॑या॒न्नास्य॒ पल्पू॑लनेन॒ वासः॑ पल्पूलयेयुरे॒तद्धि दे॒वाः सर्वं॒ न कु॒र्वन्ति॑॥~(३२)

%2.5.6.0
{\anuvakamend[{च॒न्द्रमा॒ द्वे दे॑वलो॒कमे॒व यद्य॒ज्ञं प॑ल्पूलयेयुः॒ षट्च॑}]}%~(५)

%2.5.6.1
ए॒ष वै दे॑वर॒थो यद्द॑र्\mbox{}श\-पूर्ण\-मा॒सौ यो द॑र्\mbox{}शपूर्णमा॒सावि॒ष्ट्वा सोमे॑न॒ यज॑ते॒ रथ॑स्पष्ट ए॒वाव॒साने॒ वरे॑ दे॒वाना॒मव॑ स्यत्ये॒तानि॒ वा अङ्गा॒परूꣳ॑षि संवथ्स॒रस्य॒ यद्द॑र्\mbox{}श\-पूर्ण\-मा॒सौ य ए॒वं वि॒द्वान्द॑र्\mbox{}शपूर्णमा॒सौ यज॒ते\-ऽङ्गा॒परूꣴ॑ष्ये॒व सं॑वथ्स॒रस्य॒ प्रति॑ दधात्ये॒ते वै सं॑वथ्स॒रस्य॒ चक्षु॑षी॒ यद्द॑र्\mbox{}श\-पूर्ण\-मा॒सौ य ए॒वं वि॒द्वान्द॑र्\mbox{}शपूर्णमा॒सौ यज॑ते॒ ताभ्या॑मे॒व सु॑व॒र्गं लो॒कमनु॑ \mbox{पश्य-~(३३)}

%2.5.6.2
त्ये॒षा वै दे॒वानां॒ विक्रा᳚न्ति॒र्यद्द॑र्\mbox{}श\-पूर्ण\-मा॒सौ य ए॒वं वि॒द्वान्द॑र्\mbox{}शपूर्णमा॒सौ यज॑ते दे॒वाना॑मे॒व विक्रा᳚न्ति॒मनु॒ वि क्र॑मत ए॒ष वै दे॑व॒यानः॒ पन्था॒ यद्द॑र्\mbox{}श\-पूर्ण\-मा॒सौ य ए॒वं वि॒द्वान्द॑र्\mbox{}शपूर्णमा॒सौ यज॑ते॒ य ए॒व दे॑व॒यानः॒ पन्था॒स्तꣳ स॒मारो॑हत्ये॒तौ वै दे॒वाना॒ꣳ॒ हरी॒ यद्द॑र्\mbox{}श\-पूर्ण\-मा॒सौ य ए॒वं वि॒द्वान्द॑र्\mbox{}शपूर्णमा॒सौ यज॑ते॒ यावे॒व दे॒वाना॒ꣳ॒ हरी॒ ताभ्या॑-~(३४)

%2.5.6.3
मे॒वैभ्यो॑ ह॒व्यं व॑हत्ये॒तद्वै दे॒वाना॑मा॒स्यं॑ यद्द॑र्\mbox{}श\-पूर्ण\-मा॒सौ य ए॒वं वि॒द्वान्द॑र्\mbox{}शपूर्णमा॒सौ यज॑ते सा॒क्षादे॒व दे॒वाना॑मा॒स्ये॑ जुहोत्ये॒ष वै ह॑विर्धा॒नी यो द॑र्\mbox{}शपूर्णमासया॒जी सा॒यं प्रा॑तरग्निहो॒त्रं जु॑होति॒ यज॑ते दर्\mbox{}श\-पूर्ण\-मा॒सावह॑रहर्\mbox{}हविर्धा॒निनाꣳ॑ सु॒तो य ए॒वं वि॒द्वान्द॑र्\mbox{}शपूर्णमा॒सौ यज॑ते हविर्धा॒न्य॑स्मीति॒ सर्व॑मे॒वास्य॑ बर्\mbox{}हि॒ष्यं॑ द॒त्तं भ॑वति दे॒वा वा अह॑र्-~(३५)

%2.5.6.4
य॒ज्ञियं॒ नावि॑न्द॒न्ते द॑र्\mbox{}शपूर्णमा॒साव॑पुन॒न्तौ वा ए॒तौ पू॒तौ मेध्यौ॒ यद्द॑र्\mbox{}श\-पूर्ण\-मा॒सौ य ए॒वं वि॒द्वान्द॑र्\mbox{}शपूर्णमा॒सौ यज॑ते पू॒तावे॒वैनौ॒ मेध्यौ॑ यजते॒ नामा॑वा॒स्या॑यां च पौर्णमा॒स्यां च॒ स्त्रिय॒मुपे॑या॒द्यदु॑पे॒यान्निरि॑न्द्रियः स्या॒थ्सोम॑स्य॒ वै राज्ञो᳚\-ऽर्धमा॒सस्य॒ रात्र॑यः॒ पत्न॑य आस॒न्तासा॑ममावा॒स्यां᳚ च पौर्णमा॒सीं च॒ नोपै॒त्~(३६)

%2.5.6.5
ते ए॑नम॒भि सम॑नह्येतां॒ तं यक्ष्म॑ आर्च्छ॒द्राजा॑नं॒ यक्ष्म॑ आर॒दिति॒ तद्रा॑जय॒क्ष्मस्य॒ जन्म॒ यत्पापी॑या॒नभ॑व॒त् त\-त्पा॑पय॒क्ष्मस्य॒ यज्जा॒याभ्या॒मवि॑न्द॒त् तज्जा॒येन्य॑स्य॒ य ए॒वमे॒तेषां॒ यक्ष्मा॑णां॒ जन्म॒ वेद॒ नैन॑मे॒ते यक्ष्मा॑ विन्दन्ति॒ स ए॒ते ए॒व न॑म॒स्यन्नुपा॑धाव॒त्ते अ॑ब्रूतां॒ वरं॑ वृणावहा आ॒वं दे॒वानां᳚ भाग॒धे अ॑सावा॒-~(३७)

%2.5.6.6
ऽऽवदधि॑ दे॒वा इ॑ज्यान्ता॒ इति॒ तस्मा᳚थ्स॒दृशी॑ना॒ꣳ॒ रात्री॑णाममावा॒स्या॑यां च पौर्णमा॒स्यां च॑ दे॒वा इ॑ज्यन्त ए॒ते हि दे॒वानां᳚ भाग॒धे भा॑ग॒धा अ॑स्मै मनु॒ष्या॑ भवन्ति॒ य ए॒वं वेद॑ भू॒तानि॒ क्षुध॑मघ्नन्थ्स॒द्यो म॑नु॒ष्या॑ अर्धमा॒से दे॒वा मा॒सि पि॒तरः॑ संवथ्स॒रे वन॒स्पत॑य॒स्तस्मा॒दह॑रहर्मनु॒ष्या॑ अश॑नमिच्छन्ते\-ऽर्धमा॒से दे॒वा इ॑ज्यन्ते मा॒सि पि॒तृभ्यः॑ क्रियते संवथ्स॒रे वन॒स्पत॑यः॒ फलं॑ गृह्णन्ति॒ य ए॒वं वेद॒ हन्ति॒ क्षुधं॒ भ्रातृ॑व्यम्॥~(३८)

%2.5.7.0
{\anuvakamend[{प॒श्य॒ति॒ ताभ्या॒मह॑रैदसाव॒ फलꣳ॑ स॒प्त च॑}]}%~(६)

%2.5.7.1
दे॒वा वै नर्चि न यजु॑ष्यश्रयन्त॒ ते साम॑न्ने॒वाश्र॑यन्त॒ हिं क॑रोति॒ सामै॒वाक॒र्॒\mbox{}हिं क॑रोति॒ यत्रै॒व दे॒वा अश्र॑यन्त॒ तत॑ ए॒वैना॒न्प्र यु॑ङ्क्ते॒ हिं क॑रोति वा॒च ए॒वैष योगो॒ हिं क॑रोति प्र॒जा ए॒व तद्यज॑मानः सृजते॒ त्रिः प्र॑थ॒मामन्वा॑ह॒ त्रिरु॑त्त॒मां य॒ज्ञस्यै॒व तद्ब॒र्॒\mbox{}सं~(३९)

%2.5.7.2
न॑ह्य॒त्यप्र॑स्रꣳसाय॒ सन्त॑त॒मन्वा॑ह प्रा॒णाना॑म॒न्नाद्य॑स्य॒ सन्त॑त्या॒ अथो॒ रक्ष॑सा॒मप॑हत्यै॒ राथ॑न्तरीं प्रथ॒मामन्वा॑ह॒ राथ॑न्तरो॒ वा अ॒यं लो॒क इ॒ममे॒व लो॒कम॒भि ज॑यति॒ त्रिर्वि गृ॑ह्णाति॒ त्रय॑ इ॒मे लो॒का इ॒माने॒व लो॒कान॒भि ज॑यति॒ बार्\mbox{}ह॑तीमुत्त॒मामन्वा॑ह॒ बार्\mbox{}ह॑तो॒ वा अ॒सौ लो॒को॑\-ऽमुमे॒व लो॒कम॒भि ज॑यति॒ प्र वो॒~-~(४०)

%2.5.7.3
वाजा॒ इत्यनि॑रुक्तां प्राजाप॒त्यामन्वा॑ह य॒ज्ञो वै प्र॒जा\-प॑तिर्य॒ज्ञमे॒व प्र॒जा\-प॑ति॒मा र॑भते॒ प्र वो॒ वाजा॒ इत्यन्वा॒हान्नं॒ वै वाजो\-ऽन्न॑मे॒वाव॑ रुन्धे॒ प्र वो॒ वाजा॒ इत्यन्वा॑ह॒ तस्मा᳚त्प्रा॒चीन॒ꣳ॒ रेतो॑ धीय॒ते\-ऽग्न॒ आ या॑हि वी॒तय॒ इत्या॑ह॒ तस्मा᳚त्प्र॒तीचीः᳚ प्र॒जा जा॑यन्ते॒ प्र वो॒ वाजा॒~-~(४१)

%2.5.7.4
इत्यन्वा॑ह॒ मासा॒ वै वाजा॑ अर्धमा॒सा अ॒भिद्य॑वो दे॒वा ह॒विष्म॑न्तो॒ गौर्घृ॒ताची॑ य॒ज्ञो दे॒वाञ्जि॑गाति॒ यज॑मानः सुम्न॒युरि॒दम॑सी॒दम॒सीत्ये॒व य॒ज्ञस्य॑ प्रि॒यं धामाव॑ रुन्धे॒ यं का॒मये॑त॒ सर्व॒मायु॑रिया॒दिति॒ प्र वो॒ वाजा॒ इति॒ तस्या॒नूच्याग्न॒ आ या॑हि वी॒तय॒ इति॒ सन्त॑त॒मुत्त॑रमर्ध॒र्चमा ल॑भेत~(४२)

%2.5.7.5
प्रा॒णेनै॒वास्या॑पा॒नं दा॑धार॒ सर्व॒मायु॑रेति॒ यो वा अ॑र॒त्निꣳ सा॑मिधे॒नीनां॒ वेदा॑र॒त्नावे॒व भ्रातृ॑व्यं कुरुते\-ऽर्ध॒र्चौ सं द॑धात्ये॒ष वा अ॑र॒त्निः सा॑मिधे॒नीनां॒ य ए॒वं वेदा॑र॒त्नावे॒व भ्रातृ॑व्यं कुरुत॒ ऋषेर्॑\mbox{}ऋषे॒र्वा ए॒ता निर्मि॑ता॒ यथ्सा॑मिधे॒न्य॑स्ता यदसं॑युक्ताः॒ स्युः प्र॒जया॑ प॒शुभि॒र्यज॑मानस्य॒ वि ति॑ष्ठेरन्नर्ध॒र्चौ सन्द॑धाति॒ सं यु॑नक्त्ये॒वैना॒स्ता अ॑स्मै॒ संयु॑क्ता॒ अव॑रुद्धाः॒ सर्वा॑मा॒शिषं॑ दुह्रे॥~(४३)


%2.5.8.0
{\anuvakamend[{ब॒र्॒\mbox{}सं वो॑ जायन्ते॒ प्र वो॒ वाजा॑ लभेत दधाति॒ सन्दश॑ च}]}%~(७)

%2.5.8.1
अय॑ज्ञो॒ वा ए॒ष यो॑\-ऽसा॒मा\-ऽग्न॒ आ या॑हि वी॒तय॒ इत्या॑ह रथन्त॒रस्यै॒ष वर्ण॒स्तं त्वा॑ स॒मिद्भि॑रङ्गिर॒ इत्या॑ह वामदे॒व्यस्यै॒ष वर्णो॑ बृ॒हद॑ग्ने सु॒वीर्य॒मित्या॑ह बृह॒त ए॒ष वर्णो॒ यदे॒तं तृ॒चम॒न्वाह॑ य॒ज्ञमे॒व तथ्साम॑न्वन्तं करोत्य॒ग्निर॒मुष्मिँ॑ल्लो॒क आसी॑दादि॒त्यो᳚\-ऽस्मिन्तावि॒मौ लो॒कावशा᳚न्ता-~(४४)

%2.5.8.2
वास्तां॒ ते दे॒वा अ॑ब्रुव॒न्नेते॒मौ वि पर्यू॑हा॒मेत्यग्न॒ आ या॑हि वी॒तय॒ इत्य॒स्मिँल्लो॒के᳚\-ऽग्निम॑दधुर्बृ॒हद॑ग्ने सु॒वीर्य॒मित्य॒मुष्मिँ॑ल्लो॒क आ॑दि॒त्यं ततो॒ वा इ॒मौ लो॒काव॑शाम्यतां॒ यदे॒वम॒न्वाहा॒नयो᳚र्लो॒कयोः॒ शान्त्यै॒ शाम्य॑तो\-ऽस्मा इ॒मौ लो॒कौ य ए॒वं वेद॒ पञ्च॑दश सामिधे॒नीरन्वा॑ह॒ पञ्च॑दश॒~(४५)

%2.5.8.3
वा अ॑र्धमा॒सस्य॒ रात्र॑यो\-ऽर्धमास॒शः सं॑वथ्स॒र आ᳚प्यते॒ तासां॒ त्रीणि॑ च श॒तानि॑ ष॒ष्टिश्चा॒क्षरा॑णि॒ ताव॑तीः संवथ्स॒रस्य॒ रात्र॑यो\-ऽक्षर॒श ए॒व सं॑वथ्स॒रमा᳚प्नोति नृ॒मेध॑श्च॒ परु॑च्छेपश्च ब्रह्म॒वाद्य॑मवदेताम॒स्मिन्दारा॑वा॒र्द्रे᳚\-ऽग्निं ज॑नयाव यत॒रो नौ॒ ब्रह्मी॑या॒निति॑ नृ॒मेधो॒\-ऽभ्य॑वद॒थ्स धू॒मम॑जनय॒त्परु॑च्छेपो॒\-ऽभ्य॑वद॒थ्सो᳚\-ऽग्निम॑जनय॒दृष॒ इत्य॑ब्रवी॒द्-~(४६)

%2.5.8.4
यथ्स॒माव॑द्वि॒द्व क॒था त्वम॒ग्निमजी॑जनो॒ नाहमिति॑ सामिधे॒नीना॑मे॒वाहं वर्णं॑ वे॒देत्य॑ब्रवी॒द्यद् घृ॒तव॑त्प॒दम॑नू॒च्यते॒ स आ॑सां॒ वर्ण॒स्तं त्वा॑ स॒मिद्भि॑रङ्गिर॒ इत्या॑ह सामिधे॒नीष्वे॒व तज्ज्योति॑र्जनयति॒ स्त्रिय॒स्तेन॒ यदृचः॒ स्त्रिय॒स्तेन॒ यद्गा॑य॒त्रियः॒ स्त्रिय॒स्तेन॒ यथ्सा॑मिधे॒न्यो॑ वृष॑ण्वती॒मन्वा॑ह॒~(४७)

%2.5.8.5
तेन॒ पुꣴस्व॑ती॒स्तेन॒ सेन्द्रा॒स्तेन॑ मिथु॒ना अ॒ग्निर्दे॒वानां᳚ दू॒त आसी॑दु॒शना॑ का॒व्यो\-ऽसु॑राणां॒ तौ प्र॒जा\-प॑तिं प्र॒श्ञमै॑ता॒ꣳ॒ स प्र॒जा\-प॑तिर॒ग्निं दू॒तं वृ॑णीमह॒ इत्य॒भि प॒र्याव॑र्तत॒ ततो॑ दे॒वा अभ॑व॒न्परासु॑रा॒ यस्यै॒वं वि॒दुषो॒\-ऽग्निं दू॒तं वृ॑णीमह॒ इत्य॒न्वाह॒ भव॑त्या॒त्मना॒ परा᳚स्य॒ भ्रातृ॑व्यो भवत्यध्व॒रव॑ती॒मन्वा॑ह॒ भ्रातृ॑व्यमे॒वैतया᳚~(४८)

%2.5.8.6
ध्वरति शो॒चिष्के॑श॒स्तमी॑मह॒ इत्या॑ह प॒वित्र॑मे॒वैतद्यज॑\-मान\-मे॒वै\-तया॑ पवयति॒ समि॑द्धो अग्न आहु॒तेत्या॑ह परि॒धिमे॒वैतं परि॑ दधा॒\-त्य\-स्क॑न्दाय॒ यदत॑ ऊ॒र्ध्वम॑भ्याद॒ध्याद्यथा॑ बहिःपरि॒धि स्कन्द॑ति ता॒दृगे॒व तत्त्रयो॒ वा अ॒ग्नयो॑ हव्य॒वाह॑नो दे॒वानां᳚ कव्य॒वाह॑नः पितृ॒णाꣳ स॒हर॑क्षा॒ असु॑राणां॒ त ए॒तर्\mbox{}ह्याशꣳ॑सन्ते॒ मां व॑रिष्यते॒ मा-~(४९)

%2.5.8.7
मिति॑ वृणी॒ध्वꣳ ह॑व्य॒वाह॑न॒मित्या॑ह॒ य ए॒व दे॒वानां॒ तं वृ॑णीत आर्\mbox{}षे॒यं वृ॑णीते॒ बन्धो॑रे॒व नैत्यथो॒ सन्त॑त्यै प॒रस्ता॑द॒र्वाचो॑ वृणीते॒ तस्मा᳚त्प॒रस्ता॑द॒र्वाञ्चो॑ मनु॒ष्या᳚न्पि॒तरो\-ऽनु॒ प्र पि॑पते॥~(५०)

%2.5.9.0
{\anuvakamend[{अशा᳚न्तावाह॒ पञ्च॑दशाब्रवी॒दन्वा॑है॒तया॑ वरिष्यते॒ मामेका॒न्नत्रि॒ꣳ॒शच्च॑}]}%~(८)

%2.5.9.1
अग्ने॑ म॒हाꣳ अ॒सीत्या॑ह म॒हान् ह्ये॑ष यद॒ग्निर्ब्रा᳚ह्म॒णेत्या॑ह ब्राह्म॒णो ह्ये॑ष भा॑र॒तेत्या॑है॒ष हि दे॒वेभ्यो॑ ह॒व्यं भर॑ति दे॒वेद्ध॒ इत्या॑ह दे॒वा ह्ये॑तमैन्ध॑त॒ मन्वि॑द्ध॒ इत्या॑ह॒ मनु॒र्ह्ये॑तमुत्त॑रो दे॒वेभ्य॒ ऐन्द्धर्\mbox{}षि॑ष्टुत॒ इत्या॒हर्\mbox{}ष॑यो॒ ह्ये॑तमस्तु॑व॒न्विप्रा॑नुमदित॒ इत्या॑ह॒~(५१)

%2.5.9.2
विप्रा॒ ह्ये॑ते यच्छु॑श्रु॒वाꣳसः॑ कविश॒स्त इत्या॑ह क॒वयो॒ ह्ये॑ते यच्छु॑श्रु॒वाꣳसो॒ ब्रह्म॑सꣳशित॒ इत्या॑ह॒ ब्रह्म॑सꣳशितो॒ ह्ये॑ष घृ॒ताह॑वन॒ इत्या॑ह घृताहु॒तिर्\mbox{}ह्य॑स्य प्रि॒यत॑मा प्र॒णीर्य॒ज्ञाना॒मित्या॑ह प्र॒णीर्\mbox{}ह्ये॑ष य॒ज्ञानाꣳ॑ र॒थीर॑ध्व॒राणा॒मित्या॑है॒ष हि दे॑वर॒थो॑\-ऽतूर्तो॒ होतेत्या॑ह॒ न ह्ये॑तं कश्च॒न~(५२)

%2.5.9.3
तर॑ति॒ तूर्णि॑र्\mbox{}हव्य॒वाडित्या॑ह॒ सर्व॒ꣴ॒ ह्ये॑ष तर॒त्यास्पात्रं॑ जु॒हूर्दे॒वा\-ना॒\-मि\-त्या॑ह जु॒हूर्\mbox{}ह्ये॑ष दे॒वानां᳚ चम॒सो दे॑व॒पान॒ इत्या॑ह चम॒सो ह्ये॑ष दे॑व॒पानो॒\-ऽराꣳ इ॑वाग्ने ने॒मिर्दे॒वाꣴस्त्वं प॑रि॒भूर॒सीत्या॑ह दे॒वान् ह्ये॑ष प॑रि॒\-भूर्यद्ब्रू॒\-यादा व॑ह दे॒वान्दे॑वय॒ते यज॑माना॒येति॒ भ्रातृ॑व्यमस्मै~(५३)

%2.5.9.4
जनये॒दा व॑ह दे॒वान् यज॑माना॒येत्या॑ह॒ यज॑मानमे॒वैतेन॑ वर्ध\-य\-त्य॒\-ग्नि\-म॑ग्न॒ आ व॑ह॒ सोम॒मा व॒हेत्या॑ह दे॒वता॑ ए॒व तद्य॑थापू॒र्वमुप॑ ह्वयत॒ आ चा᳚ग्ने दे॒वान् वह॑ सु॒यजा॑ च यज जातवेद॒ इत्या॑हा॒ग्निमे॒व तथ्सꣴ श्य॑ति॒ सो᳚\-ऽस्य॒ सꣳशि॑तो दे॒वेभ्यो॑ ह॒व्यं व॑हत्य॒ग्निर्\mbox{}होते-~(५४)

%2.5.9.5
त्या॑हा॒ग्निर्वै दे॒वाना॒ꣳ॒ होता॒ य ए॒व दे॒वाना॒ꣳ॒ होता॒ तं वृ॑णीते॒ स्मो व॒यमित्या॑हा॒ऽ॒ऽ॒त्मान॑मे॒व स॒त्त्वं ग॑मयति सा॒धु ते॑ यजमान दे॒वतेत्या॑हा॒ऽ॒ऽ॒शिष॑मे॒वैतामा शा᳚स्ते॒ यद्ब्रू॒याद्यो᳚\-ऽग्निꣳ होता॑र॒मवृ॑था॒ इत्य॒ग्निनो॑भ॒यतो॒ यज॑मानं॒ परि॑ गृह्णीयात् प्र॒मायु॑कः स्याद्यजमानदेव॒त्या॑ वै जु॒हूर्भ्रा॑तृव्यदेव॒त्यो॑प॒भृद्-~(५५)

%2.5.9.6
यद्द्वे इ॑व ब्रू॒याद्भ्रातृ॑व्यमस्मै जनयेद् घृ॒तव॑तीमध्वर्यो॒ स्रुच॒मास्य॒\-स्वे\-त्या॑ह॒ यज॑मानमे॒वैतेन॑ वर्धयति देवा॒युव॒मित्या॑ह दे॒वान् ह्ये॑षा\-व॑ति वि॒श्व\-वा॑रा॒मित्या॑ह॒ विश्व॒ꣴ॒ ह्ये॑षाव॒तीडा॑महै दे॒वाꣳ ई॒डेन्या᳚न्नम॒स्याम॑ नम॒स्यान्॑ यजा॑म य॒ज्ञिया॒नित्या॑ह मनु॒ष्या॑ वा ई॒डेन्याः᳚ पि॒तरो॑ नम॒स्या॑ दे॒वा य॒ज्ञिया॑ दे॒वता॑ ए॒व तद्य॑थाभा॒गं य॑जति॥~(५६)

%2.5.10.0
{\anuvakamend[{विप्रा॑नुमदित॒ इत्या॑ह च॒नास्मै॒ होतो॑प॒भृद्दे॒वता॑ ए॒व त्रीणि॑ च}]}%~(९)

%2.5.10.1
त्रीꣴ स्तृ॒चाननु॑ ब्रूयाद्राज॒न्य॑स्य॒ त्रयो॒ वा अ॒न्ये रा॑ज॒न्या᳚त्पुरु॑षा ब्राह्म॒णो वैश्यः॑ शू॒द्रस्ताने॒वास्मा॒ अनु॑कान्करोति॒ पञ्च॑द॒शानु॑ ब्रूयाद् राज॒न्य॑स्य पञ्चद॒शो वै रा॑ज॒न्यः॑ स्व ए॒वैन॒ꣴ॒ स्तोमे॒ प्रति॑\-ष्ठापयति त्रि॒ष्टुभा॒ परि॑ दध्यादिन्द्रि॒यं वै त्रि॒ष्टुगि॑न्द्रि॒यका॑मः॒ खलु॒ वै रा॑ज॒न्यो॑ यजते त्रि॒ष्टुभै॒वास्मा॑ इन्द्रि॒यं परि॑ गृह्णाति॒ यदि॑ का॒मये॑त~(५७)

%2.5.10.2
ब्रह्मवर्च॒सम॒स्त्विति॑ गायत्रि॒या परि॑ दध्याद्ब्रह्मवर्च॒सं वै गा॑य॒त्री ब्र॑ह्मवर्च॒समे॒व भ॑वति स॒प्तद॒शानु॑ ब्रूया॒द्वैश्य॑स्य सप्तद॒शो वै वैश्यः॒ स्व ए॒वैन॒ꣴ॒ स्तोमे॒ प्रति॑\-ष्ठापयति॒ जग॑त्या॒ परि॑ दध्या॒ज्जाग॑ता॒ वै प॒शवः॑ प॒शुका॑मः॒ खलु॒ वै वैश्यो॑ यजते॒ जग॑त्यै॒वास्मै॑ प॒शून्परि॑ गृह्णा॒त्येक॑विꣳशति॒मनु॑ ब्रूयात्प्रति॒ष्ठाका॑मस्यैकवि॒ꣳ॒शः स्तोमा॑नां प्रति॒ष्ठा प्रति॑ष्ठित्यै॒~(५८)

%2.5.10.3
चतु॑र्विꣳशति॒मनु॑ ब्रूयाद्ब्रह्मवर्च॒सका॑मस्य॒ चतु॑र्विꣳशत्यक्षरा गाय॒त्री गा॑य॒त्री ब्र॑ह्मवर्च॒सं गा॑यत्रि॒यैवास्मै᳚ ब्रह्मवर्च॒समव॑ रुन्धे त्रि॒ꣳ॒शत॒\-मनु॑ ब्रूया॒दन्न॑कामस्य त्रि॒ꣳ॒शद॑क्षरा वि॒राडन्नं॑ वि॒राड्वि॒रा\-जै॒\-वा\-स्मा॑ अ॒न्नाद्\-य॒\-मव॑ रुन्धे॒ द्वात्रिꣳ॑शत॒\-मनु॑\-ब्रूयात्प्र\-ति॒ष्ठा\-का॑मस्य॒ द्वात्रिꣳ॑शद\-क्षरा\-नु॒ष्टु॑ग\-नु॒ष्टुप्छन्द॑सां प्रति॒ष्ठा प्रति॑ष्ठित्यै॒ षट्त्रिꣳ॑शत॒मनु॑ ब्रूयात्प॒शुका॑मस्य॒ षट्त्रिꣳ॑शदक्षरा बृह॒ती बार्\mbox{}ह॑ताः प॒शवो॑ बृह॒त्यैवास्मै॑ प॒शू-~(५९)

%2.5.10.4
नव॑ रुन्धे॒ चतु॑श्चत्वारिꣳशत॒मनु॑ ब्रूयादिन्द्रि॒यका॑मस्य॒ चतु॑श्चत्वारिꣳशदक्षरा त्रि॒ष्टुगि॑न्द्रि॒यं त्रि॒ष्टुप्त्रि॒ष्टुभै॒वास्मा॑ इन्द्रि॒यमव॑ रुन्धे॒\-ऽष्टाच॑त्वारिꣳशत॒मनु॑ ब्रूयात्प॒शुका॑मस्या॒ष्टाच॑त्वारिꣳशदक्षरा॒ जग॑ती॒ जाग॑ताः प॒शवो॒ जग॑त्यै॒वास्मै॑ प॒शूनव॑ रुन्धे॒ सर्वा॑णि॒ छन्दा॒ꣴ॒स्यनु॑ ब्रूयाद्बहुया॒जिनः॒ सर्वा॑णि॒ वा ए॒तस्य॒ छन्दा॒ꣴ॒स्यव॑रुद्धानि॒ यो ब॑हुया॒ज्यप॑रिमित॒मनु॑ ब्रूया॒दप॑रिमित॒स्याव॑रुद्ध्यै॥~(६०)

%2.5.11.0
{\anuvakamend[{का॒मये॑त॒ प्रति॑ष्ठित्यै प॒शून्थ्स॒प्तच॑त्वारिꣳशच्च}]}%॥10॥

%2.5.11.1
निवी॑तं मनु॒ष्या॑णां प्राचीनावी॒तं पि॑तृ॒णामुप॑वीतं दे॒वाना॒मुप॑ व्ययते देवल॒क्ष्ममे॒व तत्कु॑रुते॒ तिष्ठ॒न्नन्वा॑ह॒ तिष्ठ॒न्॒ ह्याश्रु॑ततरं॒ वद॑ति॒ तिष्ठ॒न्नन्वा॑ह सुव॒र्गस्य॑ लो॒कस्या॒भिजि॑त्या॒ आसी॑नो यजत्य॒स्मिन्ने॒व लो॒के प्रति॑ तिष्ठति॒ यत्क्रौ॒ञ्चम॒न्वाहा॑ऽऽसु॒रं तद्यन्म॒न्द्रं मा॑नु॒षं तद्यद॑न्त॒रा तथ्सदे॑वमन्त॒रानूच्यꣳ॑ सदेव॒त्वाय॑ वि॒द्वाꣳसो॒ वै~(६१)

%2.5.11.2
पु॒रा होता॑रो\-ऽभूव॒न्तस्मा॒द्विधृ॑ता॒ अध्वा॒नो\-ऽभू॑व॒न्न पन्था॑नः॒ सम॑रुक्षन्नन्तर्वे॒द्य॑न्यः पादो॒ भव॑ति बहिर्वे॒द्य॑न्यो\-ऽथान्वा॒हाध्व॑नां॒ विधृ॑त्यै प॒थामसꣳ॑रोहा॒याथो॑ भू॒तं चै॒व भ॑वि॒ष्यच्चाव॑ रु॒न्धे\-ऽथो॒ परि॑मितं चै॒वाप॑रिमितं॒ चाव॑ रु॒न्धे\-ऽथो᳚ ग्रा॒म्याꣴश्चै॒व प॒शूना॑र॒ण्याꣴश्चाव॑ रु॒न्धे\-ऽथो॑~(६२)

%2.5.11.3
देवलो॒कं चै॒व म॑नुष्यलो॒कं चा॒भि ज॑यति दे॒वा वै सा॑मिधे॒नीर॒नूच्य॑ य॒ज्ञं नान्व॑पश्य॒न्थ्स प्र॒जा\-प॑तिस्तू॒ष्णीमा॑घा॒र\-माघा॑र\-य॒त् ततो॒ वै दे॒वा य॒ज्ञमन्व॑पश्य॒न्॒ यत् तू॒ष्णीमा॑\-घा॒रमा॑\-घा॒रय॑ति य॒ज्ञस्यानु॑\-ख्यात्या॒ अथो॑ सामिधे॒नीरे॒वाभ्य॑न॒क्त्यलू᳚क्षो भवति॒ य ए॒वं वेदाथो॑ त॒र्पय॑त्ये॒वैना॒स्तृप्य॑ति प्र॒जया॑ प॒शुभि॒र्~(६३)

%2.5.11.4
य ए॒वं वेद॒ यदेक॑याघा॒रये॒देकां᳚ प्रीणीया॒द्यद्द्वाभ्यां॒ द्वे प्री॑णीया॒द्यत् ति॒सृभि॒रति॒ तद्रे॑चये॒न्मन॒सा घा॑रयति॒ मन॑सा॒ ह्यना᳚प्तमा॒प्यते॑ ति॒र्यञ्च॒मा घा॑रय॒त्यछ॑म्बट्कारं॒ वाक्च॒ मन॑श्चार्तीयेताम॒हं दे॒वेभ्यो॑ ह॒व्यं व॑हा॒मीति॒ वाग॑ब्रवीद॒हं दे॒वेभ्य॒ इति॒ मन॒स्तौ प्र॒जा\-प॑तिं प्र॒श्ञमै॑ता॒ꣳ॒ सो᳚\-ऽब्रवीत्~(६४)

%2.5.11.5
प्र॒जा\-प॑तिर्दू॒तीरे॒व त्वं मन॑सो\-ऽसि॒ यद्धि मन॑सा॒ ध्याय॑ति॒ तद्वा॒चा वद॒तीति॒ तत्खलु॒ तुभ्यं॒ न वा॒चा जु॑हव॒न्नित्य॑ब्रवी॒त् तस्मा॒न्मन॑सा प्र॒जाप॑तये जुह्वति॒ मन॑ इव॒ हि प्र॒जा\-प॑तिः प्र॒जाप॑ते॒राप्त्यै॑ परि॒धीन्थ्सम्मा᳚र्ष्टि पु॒नात्ये॒वैना॒न्त्रिर्म॑ध्य॒मं त्रयो॒ वै प्रा॒णाः प्रा॒णाने॒वाभि ज॑यति॒ त्रिर्द॑क्षिणा॒र्ध्यं॑ त्रय॑~-~(६५)

%2.5.11.6
इ॒मे लो॒का इ॒माने॒व लो॒कान॒भि ज॑यति॒ त्रिरु॑त्तरा॒र्ध्यं॑ त्रयो॒ वै दे॑व॒यानाः॒ पन्था॑न॒स्ताने॒वाभि ज॑यति॒ त्रिरुप॑ वाजयति॒ त्रयो॒ वै दे॑वलो॒का दे॑वलो॒काने॒वाभि ज॑यति॒ द्वाद॑श॒ सम्प॑द्यन्ते॒ द्वाद॑श॒ मासाः᳚ संवथ्स॒रः सं॑वथ्स॒रमे॒व प्री॑णा॒त्यथो॑ संवथ्स॒रमे॒वास्मा॒ उप॑ दधाति सुव॒र्गस्य॑ लो॒कस्य॒ सम॑ष्ट्या आघा॒रमा घा॑रयति ति॒र इ॑व॒~(६६)

%2.5.11.7
वै सु॑व॒र्गो लो॒कः सु॑व॒र्गमे॒वास्मै॑ लो॒कं प्र रो॑चयत्यृ॒जुमा घा॑रयत्यृ॒जुरि॑व॒ हि प्रा॒णः सन्त॑त॒मा घा॑रयति प्रा॒णाना॑म॒न्नाद्य॑स्य॒ सन्त॑त्या॒ अथो॒ रक्ष॑सा॒मप॑हत्यै॒ यं का॒मये॑त प्र॒मायु॑कः स्या॒दिति॑ जि॒ह्मं तस्या घा॑रयेत्प्रा॒णमे॒वास्मा᳚ज्जि॒ह्मं न॑यति ता॒जक्प्र मी॑यते॒ शिरो॒ वा ए॒तद्य॒ज्ञस्य॒ यदा॑घा॒र आ॒त्मा ध्रु॒वा-~(६७)

%2.5.11.8
ऽऽघा॒रमा॒घार्य॑ ध्रु॒वाꣳ सम॑नक्त्या॒त्मन्ने॒व य॒ज्ञस्य॒ शिरः॒ प्रति॑ दधा\-त्य॒\-ग्निर्दे॒वानां᳚ दू॒त आसी॒द्दैव्यो\-ऽसु॑राणां॒ तौ प्र॒जा\-प॑तिं प्र॒श्ञमै॑ता॒ꣳ॒ स प्र॒जा\-प॑तिर्ब्राह्म॒णम॑ब्रवीदे॒तद्वि ब्रू॒हीत्या श्रा॑व॒येती॒दं दे॑वाः शृणु॒तेति॒ वाव तद॑ब्रवीद॒ग्निर्दे॒वो होतेति॒ य ए॒व दे॒वानां॒ तम॑वृणीत॒ ततो॑ \mbox{दे॒वा-~(६८)}

%2.5.11.9
अभ॑व॒न्परा॑सुरा॒ यस्यै॒वं वि॒दुषः॑ प्रव॒रं प्र॑वृ॒णते॒ भव॑त्या॒त्मना॒ परा᳚स्य॒ भ्रातृ॑व्यो भवति॒ यद्ब्रा᳚ह्म॒णश्चाब्रा᳚ह्मणश्च प्र॒श्ञमे॒यातां᳚ ब्राह्म॒णायाधि॑ ब्रूया॒द्यद्ब्रा᳚ह्म॒णाया॒ध्याहा॒ऽऽत्मने\-ऽध्या॑ह॒ यद्ब्रा᳚ह्म॒णं प॒राहा॒ऽऽत्मानं॒ परा॑ह॒ तस्मा᳚द्ब्राह्म॒णो न प॒रोच्यः॑॥~(६९)

%2.5.12.0
{\anuvakamend[{वा आ॑र॒ण्याꣴश्चाव॑ रु॒न्धे\-ऽथो॑ प॒शुभिः॒ सो᳚\-ऽब्रवीद्दक्षिणा॒र्ध्य॑न्त्रय॑ इव ध्रु॒वा दे॒वाश्च॑त्वारि॒ꣳ॒शच्च॑}]}%॥11॥

%2.5.12.1
आयु॑ष्ट आयु॒र्दा अ॑ग्न॒ आ प्या॑यस्व॒ सं ते\-ऽव॑ ते॒ हेड॒ उदु॑त्त॒मं प्र णो॑ दे॒व्या नो॑ दि॒वो\-ऽग्ना॑विष्णू॒ अग्ना॑विष्णू इ॒मं मे॑ वरुण॒ तत्त्वा॑ या॒म्युदु॒ त्यं चि॒त्रम्। अ॒पां नपा॒दा ह्यस्था॑दु॒पस्थं॑ जि॒ह्माना॑मू॒र्ध्वो वि॒द्युतं॒ वसा॑नः। तस्य॒ ज्येष्ठं॑ महि॒मानं॒ वह॑न्ती॒र्॒\mbox{}हिर॑ण्यवर्णाः॒ परि॑ यन्ति य॒ह्वीः। स-~(७०)

%2.5.12.2
म॒न्या यन्त्युप॑ यन्त्य॒न्याः स॑मा॒नमू॒र्वं न॒द्यः॑ पृणन्ति। तमू॒ शुचि॒ꣳ॒ शुच॑यो दीदि॒वाꣳस॑म॒पां नपा॑तं॒ परि॑ तस्थु॒रापः॑। तमस्मे॑रा युव॒तयो॒ युवा॑नं मर्मृ॒ज्यमा॑नाः॒ परि॑ य॒न्त्यापः॑। स शु॒क्रेण॒ शिक्व॑ना रे॒वद॒ग्निर्दी॒दाया॑नि॒ध्मो घृ॒तनि॑र्णिग॒फ्सु। इन्द्रा॒वरु॑णयोर॒हꣳ स॒म्राजो॒रव॒ आ वृ॑णे। ता नो॑ मृडात ई॒दृशे᳚। इन्द्रा॑वरुणा यु॒वम॑ध्व॒राय॑ नो~(७१)

%2.5.12.3
वि॒शे जना॑य॒ महि॒ शर्म॑ यच्छतम्। दी॒र्घप्र॑यज्यु॒मति॒ यो व॑नु॒ष्यति॑ व॒यं ज॑येम॒ पृत॑नासु दू॒ढ्यः॑। आ नो॑ मित्रावरुणा॒ प्र बा॒हवा᳚। त्वं नो॑ अग्ने॒ वरु॑णस्य वि॒द्वान् दे॒वस्य॒ हेडो\-ऽव॑ यासिसीष्ठाः। यजि॑ष्ठो॒ वह्नि॑तमः॒ शोशु॑चानो॒ विश्वा॒ द्वेषाꣳ॑सि॒ प्र मु॑मुग्ध्य॒स्मत्। स त्वं नो॑ अग्ने\-ऽव॒मो भ॑वो॒ती नेदि॑ष्ठो अ॒स्या उ॒षसो॒ व्यु॑ष्टौ। अव॑ यक्ष्व नो॒ वरु॑ण॒ꣳ॒~(७२)

%2.5.12.4
ररा॑णो वी॒हि मृ॑डी॒कꣳ सु॒हवो॑ न एधि। प्रप्रा॒यम॒ग्निर्भ॑र॒तस्य॑ शृण्वे॒ वि यथ्सूर्यो॒ न रोच॑ते बृ॒हद्भाः। अ॒भि यः पू॒रुं पृत॑नासु त॒स्थौ दी॒दाय॒ दैव्यो॒ अति॑थिः शि॒वो नः॑। प्र ते॑ यक्षि॒ प्र त॑ इयर्मि॒ मन्म॒ भुवो॒ यथा॒ वन्द्यो॑ नो॒ हवे॑षु। धन्व॑न्निव प्र॒पा अ॑सि॒ त्वम॑ग्न इय॒क्षवे॑ पू॒रवे᳚ प्रत्न राजन्न्।~(७३)

%2.5.12.5
वि पाज॑सा॒ वि ज्योति॑षा। स त्वम॑ग्ने॒ प्रती॑केन॒ प्रत्यो॑ष यातुधा॒न्यः॑। उ॒रु॒क्षये॑षु॒ दीद्य॑त्। तꣳ सु॒प्रती॑कꣳ सु॒दृश॒ꣴ॒ स्वञ्च॒मवि॑द्वाꣳसो वि॒दुष्ट॑रꣳ सपेम। स य॑क्ष॒द्विश्वा॑ व॒युना॑नि वि॒द्वान्प्र ह॒व्यम॒ग्निर॒मृते॑षु वोचत्। अ॒ꣳ॒हो॒मुचे॑ वि॒वेष॒ यन्मा॒ वि न॑ इ॒न्द्रेन्द्र॑ क्ष॒त्रमि॑न्द्रि॒याणि॑ शतक्र॒तो\-ऽनु॑ ते दायि॥~(७४)
{\anuvakamend[{य॒ह्वीः सम॑ध्व॒राय॑ नो॒ वरु॑णꣳ राज॒ꣴ॒ श्चतु॑श्चत्वारिꣳशच्च}]}%॥12॥

%2.5.0.0
{\prashnaend[{{वि॒श्वरू॑प॒स्त्वष्टेन्द्रं॑ वृ॒त्रम्ब्र॑ह्मवा॒दिनः॒ स त्वै नासो॑मयाज्ये॒ष वै दे॑वर॒थो दे॒वा वै नर्चि नाय॒ज्ञो\-ऽग्ने॑ म॒हान्त्रीन्निवी॑त॒मायु॑ष्टे॒ द्वाद॑श॥१२॥} वि॒श्वरू॑पो॒ नैनꣳ॑ शीतरू॒राव॒द्य वसु॑ पूर्वे॒द्युर्वाजा॒ इत्यग्ने॑ म॒हान्निवी॑तम॒न्या यन्ति॒ चतुः॑सप्ततिः॥७४॥ वि॒श्वरू॒पो\-ऽनु॑ ते दायि॥}]}
%%% END PRASHNA

\sect{षष्ठमः प्रश्नः}\setcounter{anuvakam}{0}
\dnsub{तैत्तिरीयसंहितायां द्वितीयकाण्डे षष्ठमः प्रश्नः}
%2.6.1.0
%2.6.1.1
स॒मिधो॑ यजति वस॒न्तमे॒वर्तू॒नामव॑ रुन्धे॒ तनू॒नपा॑तं यजति ग्री॒ष्ममे॒वाव॑ रुन्ध इ॒डो य॑जति व॒र्॒\mbox{}षा ए॒वाव॑ रुन्धे ब॒र्॒\mbox{}हिर्य॑जति श॒रद॑मे॒वाव॑ रुन्धे स्वाहाका॒रं य॑जति हेम॒न्तमे॒वाव॑ रुन्धे॒ तस्मा॒थ्\-स्वाहा॑कृता॒ हेम॑न्प॒शवो\-ऽव॑ सीदन्ति स॒मिधो॑ यजत्यु॒षस॑ ए॒व दे॒वता॑ना॒मव॑ रुन्धे॒ तनू॒नपा॑तं यजति य॒ज्ञमे॒वाव॑ रुन्ध~-~(१)

%2.6.1.2
इ॒डो य॑जति प॒शूने॒वाव॑ रुन्धे ब॒र्॒\mbox{}हिर्य॑जति प्र॒जामे॒वाव॑ रुन्धे स॒मान॑यत उप॒भृत॒स्तेजो॒ वा आज्यं॑ प्र॒जा ब॒र्॒\mbox{}हिः प्र॒जास्वे॒व तेजो॑ दधाति स्वाहाका॒रं य॑जति॒ वाच॑मे॒वाव॑ रुन्धे॒ दश॒ सम्प॑द्यन्ते॒ दशा᳚क्षरा वि॒राड्वि॒राजै॒वान्नाद्य॒मव॑ रुन्धे स॒मिधो॑ यजत्य॒स्मिन्ने॒व लो॒के प्रति॑ तिष्ठति॒ तनू॒नपा॑तं यजति~(२)

%2.6.1.3
य॒ज्ञ ए॒वान्तरि॑क्षे॒ प्रति॑ तिष्ठती॒डो य॑जति प॒शुष्वे॒व प्रति॑ तिष्ठति ब॒र्॒\mbox{}हिर्य॑जति॒ य ए॒व दे॑व॒यानाः॒ पन्था॑न॒स्तेष्वे॒व प्रति॑ तिष्ठति स्वाहाका॒रं य॑जति सुव॒र्ग ए॒व लो॒के प्रति॑ तिष्ठत्ये॒ताव॑न्तो॒ वै दे॑व\-लो॒का\-स्तेष्वे॒व य॑थापू॒र्वं प्रति॑ तिष्ठति देवासु॒रा ए॒षु लो॒केष्व॑स्पर्धन्त॒ ते दे॒वाः प्र॑या॒जैरे॒भ्यो लो॒केभ्यो\-ऽसु॑रा॒न्प्राणु॑दन्त॒ तत्प्र॑या॒जानां᳚~(३)

%2.6.1.4
प्रयाज॒त्वं यस्यै॒वं वि॒दुषः॑ प्रया॒जा इ॒ज्यन्ते॒ प्रैभ्यो लो॒केभ्यो॒ भ्रातृ॑व्यान्नुदते\-ऽभि॒क्रामं॑ जुहोत्य॒भिजि॑त्यै॒ यो वै प्र॑या॒जानां᳚ मिथु॒नं वेद॒ प्र प्र॒जया॑ प॒शुभि॑र्मिथु॒नैर्जा॑यते स॒मिधो॑ ब॒ह्वीरि॑व यजति॒ तनू॒नपा॑त॒मेक॑मिव मिथु॒नं तदि॒डो ब॒ह्वीरि॑व यजति ब॒र्॒\mbox{}हिरेक॑मिव मिथु॒नं तदे॒तद्वै प्र॑या॒जानां᳚ मिथु॒नं य ए॒वं वेद॒ प्र~(४)

%2.6.1.5
प्र॒जया॑ प॒शुभि॑र्मिथु॒नैर्जा॑यते दे॒वानां॒ वा अनि॑ष्टा दे॒वता॒ आस॒\-न्नथा\-सु॑रा य॒ज्ञम॑\-जिघाꣳस॒न्ते दे॒वा गा॑य॒त्रीं व्यौ॑ह॒न् पञ्चा॒क्षरा॑णि प्रा॒ची\-ना॑नि॒ त्रीणि॑ प्रती॒चीना॑नि॒ ततो॒ वर्म॑ य॒ज्ञायाभ॑व॒द्वर्म॒ यज॑मानाय॒ यत्प्र॑या\-जा\-नू\-या॒जा इ॒ज्यन्ते॒ वर्मै॒व तद्य॒ज्ञाय॑ क्रियते॒ वर्म॒ यज॑मानाय॒ भ्रातृ॑व्या\-भिभूत्यै॒ तस्मा॒द्वरू॑थं पु॒रस्ता॒द्वर्\mbox{}षी॑यः प॒श्चाद्ध्रसी॑यो दे॒वा वै पु॒रा रक्षो᳚भ्य॒~-~(५)

%2.6.1.6
इति॑ स्वाहा\-का॒रेण॑ प्रया॒जेषु॑ य॒ज्ञꣳ स॒ꣴ॒स्थाप्य॑मपश्य॒न्तꣴ स्वा॑हा\-का॒रेण॑ प्रया॒जेषु॒ सम॑स्थापय॒न्वि वा ए॒तद्य॒ज्ञं छि॑न्दन्ति॒ यथ्\-स्वा॑हा\-का॒रेण॑ प्रया॒जेषु॑ सꣴस्था॒पय॑न्ति प्रया॒जानि॒ष्ट्वा ह॒वीꣴष्य॒भि घा॑रयति य॒ज्ञस्य॒ सन्त॑त्या॒ अथो॑ ह॒विरे॒वाक॒रथो॑ यथापू॒र्वमुपै॑ति पि॒ता वै प्र॑या॒जाः प्र॒जाऽनू॑या॒जा यत्प्र॑या॒जानि॒ष्ट्वा ह॒वीꣴष्य॑भिघा॒रय॑ति पि॒तैव तत्पु॒त्रेण॒ साधा॑रणं~(६)

%2.6.1.7
कुरुते॒ तस्मा॑दाहु॒र्यश्चै॒वं वेद॒ यश्च॒ न क॒था पु॒त्रस्य॒ केव॑लं क॒था साधा॑रणं पि॒तुरित्यस्क॑न्नमे॒व तद्यत्प्र॑या॒जेष्वि॒ष्टेषु॒ स्कन्द॑ति गाय॒त्र्ये॑व तेन॒ गर्भं॑ धत्ते॒ सा प्र॒जां प॒शून् यज॑मानाय॒ प्र ज॑नयति॥~(७)

%2.6.2.0
{\anuvakamend[{य॒ज॒ति॒ य॒ज्ञमेवा॒व॑\-रुन्धे॒ तनू॒नपा॑तं यजति प्रया॒जाना॑मे॒वं वेद॒ प्र रक्षो᳚भ्यः॒ साधा॑रणं॒ पञ्च॑त्रिꣳशच्च}]}%~(१)

%2.6.2.1
चक्षु॑षी॒ वा ए॒ते य॒ज्ञस्य॒ यदाज्य॑भागौ॒ यदाज्य॑भागौ॒ यज॑ति॒ चक्षु॑षी ए॒व तद्य॒ज्ञस्य॒ प्रति॑ दधाति पूर्वा॒र्धे जु॑होति॒ तस्मा᳚त्पूर्वा॒र्धे चक्षु॑षी प्र॒बाहु॑ग्जुहोति॒ तस्मा᳚त्प्र॒बाहु॒क्चक्षु॑षी देवलो॒कं वा अ॒ग्निना॒ यज॑मा॒नो\-ऽनु॑ पश्यति पितृलो॒कꣳ सोमे॑नोत्तरा॒र्धे᳚\-ऽग्नये॑ जुहोति दक्षिणा॒र्धे सोमा॑यै॒वमि॑व॒ हीमौ लो॒काव॒नयो᳚र्लो॒कयो॒रनु॑ख्यात्यै॒ राजा॑नौ॒ वा ए॒तौ दे॒वता॑नां॒~(८)

%2.6.2.2
यद॒ग्नी\-षोमा॑वन्त॒रा दे॒वता॑ इज्येते दे॒वता॑नां॒ विधृ॑त्यै॒ तस्मा॒द्राज्ञा॑ मनु॒ष्या॑ विधृ॑ता ब्रह्मवा॒दिनो॑ वदन्ति॒ किं तद्य॒ज्ञे यज॑मानः कुरुते॒ येना॒न्यतो॑दतश्च प॒शून्दा॒धारो॑भ॒यतो॑दत॒श्चेत्यृच॑म॒नूच्याज्य॑भागस्य जुषा॒णेन॑ यजति॒ तेना॒न्यतो॑दतो दाधा॒रर्च॑म॒नूच्य॑ ह॒विष॑ ऋ॒चा य॑जति॒ तेनो॑भ॒यतो॑दतो दाधार मूर्ध॒न्वती॑ पुरोनुवा॒क्या॑ भवति मू॒र्धान॑मे॒वैनꣳ॑ समा॒नानां᳚ करोति~(९)

%2.6.2.3
नि॒युत्व॑त्या यजति॒ भ्रातृ॑व्यस्यै॒व प॒शून्नि यु॑वते के॒शिनꣳ॑ ह दा॒र्भ्यं के॒शी सात्य॑कामिरुवाच स॒प्तप॑दां ते॒ शक्व॑री॒ꣴ॒ श्वो य॒ज्ञे प्र॑यो॒क्तासे॒ यस्यै॑ वी॒र्ये॑ण॒ प्र जा॒तान्भ्रातृ॑व्यान्नु॒दते॒ प्रति॑ जनि॒ष्यमा॑णा॒न्॒ यस्यै॑ वी॒र्ये॑णो॒भयो᳚र्लो॒कयो॒र्ज्योति॑र्ध॒त्ते यस्यै॑ वी॒र्ये॑ण पूर्वा॒र्धेना॑न॒ड्वान्भु॒नक्ति॑ जघना॒र्धेन॑ धे॒नुरिति॑ पु॒रस्ता᳚ल्लक्ष्मा पुरोनुवा॒क्या॑ भवति जा॒ताने॒व भ्रातृ॑व्या॒न्प्र णु॑दत उ॒परि॑ष्टाल्लक्ष्मा~(१०)

%2.6.2.4
या॒ज्या॑ जनि॒ष्यमा॑णाने॒व प्रति॑ नुदते पु॒रस्ता᳚ल्लक्ष्मा पुरोनुवा॒क्या॑ भवत्य॒स्मिन्ने॒व लो॒के ज्योति॑र्धत्त उ॒परि॑ष्टाल्लक्ष्मा या॒ज्या॑मुष्मि॑न्ने॒व लो॒के ज्योति॑र्धत्ते॒ ज्योति॑ष्मन्तावस्मा इ॒मौ लो॒कौ भ॑वतो॒ य ए॒वं वेद॑ पु॒रस्ता᳚ल्लक्ष्मा पुरोनुवा॒क्या॑ भवति॒ तस्मा᳚त्पूर्वा॒र्धेना॑न॒ड्वान्भु॑नक्त्यु॒परि॑ष्टाल्लक्ष्मा या॒ज्या॑ तस्मा᳚ज्जघना॒र्धेन॑ धे॒नुर्य ए॒वं वेद॑ भु॒ङ्क्त ए॑नमे॒तौ वज्र॒ आज्यं॒ वज्र॒ आज्य॑भागौ॒~(११)

%2.6.2.5
वज्रो॑ वषट्का॒रस्त्रि॒वृत॑मे॒व वज्रꣳ॑ स॒म्भृत्य॒ भ्रातृ॑व्याय॒ प्र ह॑र॒त्यछ॑म्बट्कारमप॒गूर्य॒ वष॑ट्करोति॒ स्तृत्यै॑ गाय॒त्री पु॑रोनुवा॒क्या॑ भवति त्रि॒ष्टुग्या॒ज्या᳚ ब्रह्म॑न्ने॒व क्ष॒त्रम॒न्वार॑म्भयति॒ तस्मा᳚द्ब्राह्म॒णो मुख्यो॒ मुख्यो॑ भवति॒ य ए॒वं वेद॒ प्रैवैनं॑ पुरोनुवा॒क्य॑याऽऽह॒ प्र ण॑यति या॒ज्य॑या ग॒मय॑ति वषट्का॒रेणैवैनं॑ पुरोनुवा॒क्य॑या दत्ते॒ प्र य॑च्छति या॒ज्य॑या॒ प्रति॑~(१२)

%2.6.2.6
वषट्का॒रेण॑ स्थापयति त्रि॒पदा॑ पुरोनुवा॒क्या॑ भवति॒ त्रय॑ इ॒मे लो॒का ए॒ष्वे॑व लो॒केषु॒ प्रति॑ तिष्ठति॒ चतु॑ष्पदा या॒ज्या॑ चतु॑ष्पद ए॒व प॒शूनव॑ रुन्धे द्व्यक्ष॒रो व॑षट्का॒रो द्वि॒पाद्यज॑मानः प॒शुष्वे॒वोपरि॑ष्टा॒त्प्रति॑ तिष्ठति गाय॒त्री पु॑रोनुवा॒क्या॑ भवति त्रि॒ष्टुग्या॒ज्यै॑षा वै स॒प्तप॑दा॒ शक्व॑री॒ यद्वा ए॒तया॑ दे॒वा अशि॑क्ष॒न्तद॑शक्नुव॒न्॒ य ए॒वं वेद॑ श॒क्नोत्ये॒व यच्छिक्ष॑ति॥~(१३)

%2.6.3.0
{\anuvakamend[{दे॒वता॑नाङ्करोत्यु॒परि॑ष्टाल्ल॒क्ष्मा\-ऽ\-ऽज्य॑भागौ॒ प्रति॑ श॒क्नोत्ये॒व द्वे च॑}]}%~(२)

%2.6.3.1
प्र॒जा\-प॑तिर्दे॒वेभ्यो॑ य॒ज्ञान्व्यादि॑श॒थ्स आ॒त्मन्नाज्य॑\-मधत्त॒ तं दे॒वा अ॑ब्रुवन्ने॒ष वाव य॒ज्ञो यदाज्य॒मप्ये॒व नोऽत्रा॒स्त्विति॒ सो᳚\-ऽब्रवी॒द्यजान्॑ व॒ आज्य॑\-भागा॒वुप॑ स्तृणान॒भि घा॑रया॒निति॒ तस्मा॒द्\-यज॒न्त्याज्य॑\-भागा॒वुप॑ स्तृणन्त्य॒भि घा॑रयन्ति ब्रह्मवा॒दिनो॑ वदन्ति॒ कस्मा᳚थ्स॒त्याद्या॒तया॑\-मान्य॒न्यानि॑ ह॒वीꣴष्यया॑त\-याम॒माज्य॒\-मिति॑ प्राजाप॒त्य-~(१४)

%2.6.3.2
मिति॑ ब्रूया॒दया॑तयामा॒ हि दे॒वानां᳚ प्र॒जा\-प॑ति॒रिति॒ छन्दाꣳ॑सि दे॒वेभ्यो\-ऽपा᳚क्राम॒न्न वो॑\-ऽभा॒गानि॑ ह॒व्यं व॑क्ष्याम॒ इति॒ तेभ्य॑ ए॒तच्च॑तुरव॒त् तम॑धारयन्पुरोनुवा॒क्या॑यै या॒ज्या॑यै दे॒वता॑यै वषट्का॒राय॒ यच्च॑तुरव॒त्तं जु॒होति॒ छन्दाꣴ॑स्ये॒व तत्प्री॑णाति॒ तान्य॑स्य प्री॒तानि॑ दे॒वेभ्यो॑ ह॒व्यं व॑ह॒न्त्यङ्गि॑रसो॒ वा इ॒त उ॑त्त॒माः सु॑व॒र्गं लो॒कमा॑य॒न्तदृष॑यो यज्ञवा॒स्त्व॑भ्य॒वाय॒न्ते॑-~(१५)

%2.6.3.3
ऽपश्यन्पुरो॒डाशं॑ कू॒र्मं भू॒तꣳ सर्प॑न्तं॒ तम॑ब्रुव॒न्निन्द्रा॑य ध्रियस्व॒ बृह॒स्पत॑ये ध्रियस्व॒ विश्वे᳚भ्यो दे॒वेभ्यो᳚ ध्रिय॒स्वेति॒ स नाध्रि॑यत॒ तम॑ब्रुवन्न॒ग्नये᳚ ध्रिय॒स्वेति॒ सो᳚\-ऽग्नये᳚\-ऽध्रियत॒ यदा᳚ग्ने॒यो᳚\-ऽष्टाक॑पालो\-ऽमावा॒स्या॑यां च पौर्णमा॒स्यां चा᳚च्यु॒तो भव॑ति सुव॒र्गस्य॑ लो॒कस्या॒भिजि॑त्यै॒ तम॑ब्रुवन्क॒थाहा᳚स्था॒ इत्यनु॑पाक्तो\-ऽभूव॒मित्य॑ब्रवी॒द्यथाक्षो\-ऽनु॑पाक्तो॒~-~(१६)

%2.6.3.4
ऽवार्च्छ॑त्ये॒वमवा॑र॒मित्यु॒परि॑ष्टाद॒भ्यज्या॒धस्ता॒दुपा॑नक्ति सुव॒र्गस्य॑ लो॒कस्य॒ सम॑ष्ट्यै॒ सर्वा॑णि क॒पाला᳚न्य॒भि प्र॑थयति॒ ताव॑तः पुरो॒\-डाशा॑\-न॒मुष्मिँ॑ल्लो॒के॑\-ऽभि ज॑यति॒ यो विद॑ग्धः॒ स नैर्॑\mbox{}ऋ॒तो यो\-ऽशृ॑तः॒ स रौ॒द्रो यः शृ॒तः स सदे॑व॒स्तस्मा॒दवि॑दहता शृत॒ङ्कृत्यः॑ सदेव॒त्वाय॒ भस्म॑ना॒ऽभि वा॑सयति॒ तस्मा᳚न्मा॒ꣳ॒सेनास्थि॑ छ॒न्नं वे॒देना॒भि वा॑सयति॒ तस्मा॒त्~(१७)

%2.6.3.5
केशैः॒ शिर॑श्छ॒न्नं प्रच्यु॑तं॒ वा ए॒तद॒स्माल्लो॒कादग॑तं देवलो॒कं यच्छृ॒तꣳ ह॒विरन॑भिघारितमभि॒घार्योद्वा॑सयति देव॒त्रैवैन॑द्गमयति॒ यद्येकं॑ क॒पालं॒ नश्ये॒देको॒ मासः॑ संवथ्स॒रस्यान॑वेतः॒ स्यादथ॒ यज॑मानः॒ प्र मी॑येत॒ यद्द्वे नश्ये॑तां॒ द्वौ मासौ॑ संवथ्स॒रस्यान॑वेतौ॒ स्याता॒मथ॒ यज॑मानः॒ प्र मी॑येत स॒ङ्ख्यायोद्वा॑सयति॒ यज॑मानस्य~(१८)

%2.6.3.6
गोपी॒थाय॒ यदि॒ नश्ये॑दाश्वि॒नं द्वि॑कपा॒लं निर्व॑पेद् द्यावा\-पृथि॒व्य॑मेक॑कपालम॒श्विनौ॒ वै दे॒वानां᳚ भि॒षजौ॒ ताभ्या॑मे॒वास्मै॑ भेष॒जं क॑रोति द्यावा\-पृथि॒व्य॑ एक॑कपालो भवत्य॒नयो॒र्वा ए॒तन्न॑श्यति॒ यन्नश्य॑त्य॒नयो॑रे॒वैन॑द्विन्दति॒ प्रति॑ष्ठित्यै॥~(१९)

%2.6.4.0
{\anuvakamend[{प्रा॒जा॒प॒त्यन्ते\-ऽक्षो\-ऽनु॑पाक्तो वे॒देना॒ऽभि वा॑सयति॒ तस्मा॒द्यज॑मानस्य॒ द्वात्रिꣳ॑शच्च}]}%~(३)

%2.6.4.1
दे॒वस्य॑ त्वा सवि॒तुः प्र॑स॒व इति॒ स्फ्यमा द॑त्ते॒ प्रसू᳚त्या अ॒श्विनो᳚र्बा॒हुभ्या॒मित्या॑हा॒श्विनौ॒ हि दे॒वाना॑मध्व॒र्यू आस्तां᳚ पू॒ष्णो हस्ता᳚भ्या॒मित्या॑ह॒ यत्यै॑ श॒तभृ॑ष्टिरसि वानस्प॒त्यो द्वि॑ष॒तो व॒ध इत्या॑ह॒ वज्र॑मे॒व तथ्सꣴ श्य॑ति॒ भ्रातृ॑व्याय प्रहरि॒ष्यन्थ्स्त॑म्बय॒जुर्\mbox{}ह॑रत्ये॒ताव॑ती॒ वै पृ॑थि॒वी याव॑ती॒ वेदि॒स्तस्या॑ ए॒ताव॑त ए॒व भ्रातृ॑व्यं॒ निर्भ॑जति॒~(२०)

%2.6.4.2
तस्मा॒न्नाभा॒गं निर्भ॑जन्ति॒ त्रिर्\mbox{}ह॑रति॒ त्रय॑ इ॒मे लो॒का ए॒भ्य ए॒वैनं॑ लो॒केभ्यो॒ निर्भ॑जति तू॒ष्णीं च॑तु॒र्थꣳ ह॑र॒त्यप॑रिमितादे॒वैनं॒ निर्भ॑ज॒त्युद्ध॑न्ति॒ यदे॒वास्या॑ अमे॒ध्यं तदप॑ ह॒न्त्युद्ध॑न्ति॒ तस्मा॒दोष॑धयः॒ परा॑ भवन्ति॒ मूलं॑ छिनत्ति॒ भ्रातृ॑व्यस्यै॒व मूलं॑ छिनत्ति पितृदेव॒त्याति॑खा॒तेय॑तीं खनति प्र॒जा\-प॑तिना~(२१)

%2.6.4.3
यज्ञमु॒खेन॒ सम्मि॑ता॒मा प्र॑ति॒ष्ठायै॑ खनति॒ यज॑मानमे॒व प्र॑ति॒ष्ठां ग॑मयति दक्षिण॒तो वर्\mbox{}षी॑यसीं करोति देव॒यज॑नस्यै॒व रू॒पम॑कः॒ पुरी॑षवतीं करोति प्र॒जा वै प॒शवः॒ पुरी॑षं प्र॒जयै॒वैनं॑ प॒शुभिः॒ पुरी॑षवन्तं करो॒त्युत्त॑रं परिग्रा॒हं परि॑ गृह्णात्ये॒ताव॑ती॒ वै पृ॑थि॒वी याव॑ती॒ वेदि॒स्तस्या॑ ए॒ताव॑त ए॒व भ्रातृ॑व्यं नि॒र्भज्या॒ऽऽत्मन॒ उत्त॑रं परिग्रा॒हं परि॑ गृह्णाति क्रू॒रमि॑व॒ वा~-~(२२)

%2.6.4.4
ए॒तत्क॑रोति॒ यद्वेदिं॑ क॒रोति॒ धा अ॑सि स्व॒धा अ॒सीति॑ योयुप्यते॒ शान्त्यै॒ प्रोक्ष॑णी॒रा सा॑दय॒त्यापो॒ वै र॑क्षो॒घ्नी रक्ष॑सा॒मप॑हत्यै॒ स्फ्यस्य॒ वर्त्म᳚न्थ्सादयति य॒ज्ञस्य॒ सन्त॑त्यै॒ यं द्वि॒ष्यात्तं ध्या॑येच्छु॒चैवैन॑मर्पयति॥~(२३)

%2.6.5.0
{\anuvakamend[{भ॒ज॒ति॒ प्र॒जा\-प॑तिनेव॒ वै त्रय॑स्त्रिꣳशच्च}]}%~(४)

%2.6.5.1
ब्र॒ह्म॒वा॒दिनो॑ वदन्त्य॒द्भिर्\mbox{}ह॒वीꣳषि॒ प्रौक्षीः॒ केना॒ऽप इति॒ ब्रह्म॒णेति॑ ब्रूयाद॒द्भिर्\mbox{}ह्ये॑व ह॒वीꣳषि॑ प्रो॒क्षति॒ ब्रह्म॑णा॒ऽप इ॒ध्माब॒र्॒\mbox{}हिः प्रोक्ष॑ति॒ मेध्य॑मे॒वैन॑त्करोति॒ वेदिं॒ प्रोक्ष॑त्यृ॒क्षा वा ए॒षा\-ऽलो॒मका॑\-ऽमे॒ध्या यद्वेदि॒र्मेध्या॑मे॒वैनां᳚ करोति दि॒वे त्वा॒\-ऽ\-न्तरि॑क्षाय त्वा पृथि॒व्यै त्वेति॑ ब॒र्॒\mbox{}हिरा॒साद्य॒ प्रो-~(२४)

%2.6.5.2
क्ष॑त्ये॒भ्य ए॒वैन॑ल्लो॒केभ्यः॒ प्रोक्ष॑ति क्रू॒रमि॑व॒ वा ए॒तत्क॑रोति॒ यत्खन॑त्य॒पो नि न॑यति॒ शान्त्यै॑ पु॒रस्ता᳚त्प्रस्त॒रं गृ॑ह्णाति॒ मुख्य॑मे॒वैनं॑ करो॒तीय॑न्तं गृह्णाति प्र॒जा\-प॑तिना यज्ञमु॒खेन॒ सम्मि॑तं ब॒र्॒\mbox{}हिः स्तृ॑णाति प्र॒जा वै ब॒र्॒\mbox{}हिः पृ॑थि॒वी वेदिः॑ प्र॒जा ए॒व पृ॑थि॒व्यां प्रति॑\-ष्ठा\-पय॒त्यन॑ति\-दृश्ञꣴ स्तृणाति प्र॒जयै॒वैनं॑ प॒शुभि॒रन॑तिदृश्ञं करो॒-~(२५)

%2.6.5.3
त्युत्त॑रं ब॒र्॒\mbox{}हिषः॑ प्रस्त॒रꣳ सा॑दयति प्र॒जा वै ब॒र्॒\mbox{}हिर्यज॑मानः प्रस्त॒रो यज॑मानमे॒वाय॑जमाना॒दुत्त॑रं करोति॒ तस्मा॒द्यज॑मा॒नो\-ऽय॑जमाना॒दुत्त॑रो॒\-ऽन्तर्द॑धाति॒ व्यावृ॑त्त्या अ॒नक्ति॑ ह॒विष्कृ॑तमे॒वैनꣳ॑ सुव॒र्गं लो॒कं ग॑मयति त्रे॒धान॑क्ति॒ त्रय॑ इ॒मे लो॒का ए॒भ्य ए॒वैनं॑ लो॒केभ्यो॑\-ऽनक्ति॒ न प्रति॑ शृणाति॒ यत्प्र॑तिशृणी॒यादनू᳚र्ध्वं भावुकं॒ यज॑मानस्य स्यादु॒परी॑व॒ प्र ह॑र-~(२६)

%2.6.5.4
त्यु॒परी॑व॒ हि सु॑व॒र्गो लो॒को नि य॑च्छति॒ वृष्टि॑मे॒वास्मै॒ नि य॑च्छति॒ नात्य॑ग्रं॒ प्र ह॑रे॒द्यदत्य॑ग्रं प्र॒हरे॑दत्यासा॒रिण्य॑ध्व॒र्यो\-र्नाशु॑का स्या॒न्न पु॒रस्ता॒त्प्रत्य॑स्ये॒द्यत्पु॒रस्ता᳚त्प्र॒त्यस्ये᳚थ्सुव॒र्गाल्लो॒काद्यज॑मानं॒ प्रति॑ नुदे॒त्प्राञ्चं॒ प्र ह॑रति॒ यज॑मानमे॒व सु॑व॒र्गं लो॒कं ग॑मयति॒ न विष्व॑ञ्चं॒ वि यु॑या॒द्यद्विष्व॑ञ्चं वियु॒याथ्~(२७)

%2.6.5.5
स्त्र्य॑स्य जायेतो॒र्ध्वमुद्यौ᳚त्यू॒र्ध्वमि॑व॒ हि पु॒ꣳ॒सः पुमा॑ने॒वास्य॑ जायते॒ यथ्स्फ्येन॑ वोपवे॒षेण॑ वा योयु॒प्येत॒ स्तृति॑रे॒वास्य॒ सा हस्ते॑न योयुप्यते॒ यज॑मानस्य गोपी॒थाय॑ ब्रह्मवा॒दिनो॑ वदन्ति॒ किं य॒ज्ञस्य॒ यज॑मान॒ इति॑ प्रस्त॒र इति॒ तस्य॒ क्व॑ सुव॒र्गो लो॒क इत्या॑हव॒नीय॒ इति॑ ब्रूया॒द्यत्प्र॑स्त॒रमा॑हव॒नीये᳚ प्र॒हर॑ति॒ यज॑मानमे॒व~(२८)

%2.6.5.6
सु॑व॒र्गं लो॒कं ग॑मयति॒ वि वा ए॒तद्यज॑मानो लिशते॒ यत्प्र॑स्त॒रं यो॑यु॒प्यन्ते॑ ब॒र्॒\mbox{}हिरनु॒ प्रह॑रति॒ शान्त्या॑ अनारम्भ॒ण इ॑व॒ वा ए॒तर्\mbox{}ह्य॑ध्व॒र्युः स ई᳚श्व॒रो वे॑प॒नो भवि॑तोर्ध्रु॒वा\-ऽसीती॒माम॒भि मृ॑शती॒यं वै ध्रु॒वा\-ऽस्यामे॒व प्रति॑ तिष्ठति॒ न वे॑प॒नो भ॑व॒त्यगा(३)न॑ग्नी॒दित्या॑ह॒ यद्ब्रू॒यादग॑न्न॒ग्निरित्य॒ग्नाव॒ग्निं ग॑मये॒न्निर्यज॑मानꣳ सुव॒र्गाल्लो॒काद्भ॑जे॒द\-ग॒न्नित्ये॒व ब्रू॑या॒द्यज॑मानमे॒व सु॑व॒र्गं लो॒कं ग॑मयति॥~(२९)

%2.6.6.0
{\anuvakamend[{आ॒साद्य॒ प्रान॑तिदृश्ञं करोति हरति वियु॒याद्यज॑मानमे॒वाग्निरिति॑ स॒प्तद॑श च}]}%~(५)

%2.6.6.1
अ॒ग्नेस्त्रयो॒ ज्यायाꣳ॑सो॒ भ्रात॑र आस॒न्ते दे॒वेभ्यो॑ ह॒व्यं वह॑न्तः॒ प्रामी॑यन्त॒ सो᳚\-ऽग्निर॑बिभेदि॒त्थं वाव स्य आर्ति॒मारि॑ष्य॒तीति॒ स निला॑यत॒ सो॑\-ऽपः प्रावि॑श॒त्तं दे॒वताः॒ प्रैष॑मैच्छ॒न्तं मथ्स्यः॒ प्राब्र॑वी॒त्तम॑शपद्धि॒याधि॑या त्वा वध्यासु॒र्यो मा॒ प्रावो॑च॒ इति॒ तस्मा॒न्मथ्स्यं॑ धि॒याधि॑या घ्नन्ति श॒प्तो~-~(३०)

%2.6.6.2
हि तमन्व॑विन्द॒न्तम॑ब्रुव॒न्नुप॑ न॒ आ व॑र्तस्व ह॒व्यं नो॑ व॒हेति॒ सो᳚\-ऽब्रवी॒द्वरं॑ वृणै॒ यदे॒व गृ॑ही॒तस्याहु॑तस्य बहिःपरि॒धि स्कन्दा॒त्तन्मे॒ भ्रातृ॑णां भाग॒धेय॑मस॒दिति॒ तस्मा॒द्यद् गृ॑ही॒तस्याहु॑तस्य बहिःपरि॒धि स्कन्द॑ति॒ तेषां॒ तद्भा॑ग॒धेयं॒ ताने॒व तेन॑ प्रीणाति परि॒धीन्परि॑ दधाति॒ रक्ष॑सा॒मप॑हत्यै॒ सꣴ स्प॑र्\mbox{}शयति॒~(३१)

%2.6.6.3
रक्ष॑सा॒मन॑न्ववचाराय॒ न पु॒रस्ता॒त्परि॑ दधात्यादि॒त्यो ह्ये॑वोद्यन्पु॒रस्ता॒द्\-रक्षाꣴ॑स्यप॒हन्त्यू॒र्ध्वे स॒मिधा॒वा द॑धात्यु॒परि॑ष्टादे॒व रक्षा॒ꣴ॒स्यप॑ हन्ति॒ यजु॑षा॒ऽन्यां तू॒ष्णीम॒न्यां मि॑थुन॒त्वाय॒ द्वे आ द॑धाति द्वि॒पाद्यज॑मानः॒ प्रति॑ष्ठित्यै ब्रह्मवा॒दिनो॑ वदन्ति॒ स त्वै य॑जेत॒ यो य॒ज्ञस्याऽऽर्त्या॒ वसी॑या॒न्थ्स्यादिति॒ भूप॑तये॒ स्वाहा॒ भुव॑नपतये॒ स्वाहा॑ भू॒तानां॒~(३२)

%2.6.6.4
पत॑ये॒ स्वाहेति॑ स्क॒न्नमनु॑ मन्त्रयेत य॒ज्ञस्यै॒व तदार्त्या॒ यज॑मानो॒ वसी॑यान्भवति॒ भूय॑सी॒र्॒\mbox{}हि दे॒वताः᳚ प्री॒णाति॑ जा॒मि वा ए॒तद्य॒ज्ञस्य॑ क्रियते॒ यद॒न्वञ्चौ॑ पुरो॒डाशा॑वुपाꣳशुया॒जम॑न्त॒रा य॑ज॒त्यजा॑मित्वा॒याथो॑ मिथुन॒त्वाया॒ग्निर॒मुष्मिँ॑ल्लो॒क आसी᳚द्य॒मो᳚\-ऽस्मिन्ते दे॒वा अ॑ब्रुव॒न्नेते॒मौ वि पर्यू॑हा॒मेत्य॒न्नाद्ये॑न दे॒वा अ॒ग्नि-~(३३)

%2.6.6.5
मु॒पाम॑न्त्रयन्त रा॒ज्येन॑ पि॒तरो॑ य॒मं तस्मा॑द॒ग्निर्दे॒वाना॑मन्ना॒दो य॒मः पि॑तृ॒णाꣳ राजा॒ य ए॒वं वेद॒ प्र रा॒ज्यम॒न्नाद्य॑माप्नोति॒ तस्मा॑ ए॒तद्भा॑ग॒धेयं॒ प्राय॑च्छ॒न्॒ यद॒ग्नये᳚ स्विष्ट॒कृते॑\-ऽव॒द्यन्ति॒ यद॒ग्नये᳚ स्विष्ट॒कृते॑\-ऽव॒द्यति॑ भाग॒धेये॑नै॒व तद्रु॒द्रꣳ सम॑र्धयति स॒कृथ्स॑कृ॒दव॑ द्यति स॒कृदि॑व॒ हि रु॒द्र उ॑त्तरा॒र्धादव॑ द्यत्ये॒षा वै रु॒द्रस्य॒~(३४)

%2.6.6.6
दिख्स्वाया॑मे॒व दि॒शि रु॒द्रं नि॒रव॑दयते॒ द्विर॒भि घा॑रयति चतुरव॒त्तस्याऽऽप्त्यै॑ प॒शवो॒ वै पूर्वा॒ आहु॑तय ए॒ष रु॒द्रो यद॒ग्निर्यत्पूर्वा॒ आहु॑तीर॒भि जु॑हु॒याद्रु॒द्राय॑ प॒शूनपि॑ दध्यादप॒शुर्यज॑मानः स्यादति॒हाय॒ पूर्वा॒ आहु॑तीर्जुहोति पशू॒नां गो॑पी॒थाय॑॥~(३५)

%2.6.7.0
{\anuvakamend[{श॒प्तः स्प॑र्\mbox{}शयति भू॒ताना॑म॒ग्निꣳ रु॒द्रस्य॑ स॒प्तत्रिꣳ॑शच्च}]}%~(६)

%2.6.7.1
मनुः॑ पृथि॒व्या य॒ज्ञिय॑मैच्छ॒थ्स घृ॒तं निषि॑क्तमविन्द॒थ्सो᳚\-ऽब्रवी॒त्को᳚\-ऽस्येश्व॒रो य॒ज्ञे\-ऽपि॒ कर्तो॒रिति॒ ताव॑ब्रूताम्मि॒त्रावरु॑णौ॒ गोरे॒वावमी᳚श्व॒रौ कर्तोः᳚ स्व॒ इति॒ तौ ततो॒ गाꣳ समै॑रयता॒ꣳ॒ सा यत्र॑यत्र॒ न्यक्रा॑म॒त्ततो॑ घृ॒तम॑पीड्यत॒ तस्मा᳚द् घृ॒तप॑द्युच्यते॒ तद॑स्यै॒ जन्मोप॑हूतꣳ रथन्त॒रꣳ स॒ह पृ॑थि॒व्येत्या॑ह~(३६)

%2.6.7.2
इ॒यं वै र॑थन्त॒रमि॒मामे॒व स॒हान्नाद्ये॒नोप॑ ह्वयत॒ उप॑हूतं वामदे॒व्यꣳ स॒हान्तरि॑क्षे॒णेत्या॑ह प॒शवो॒ वै वा॑मदे॒व्यं प॒शूने॒व स॒हान्तरि॑क्षे॒णोप॑ ह्वयत॒ उप॑हूतम्बृ॒हथ्स॒ह दि॒वेत्या॑है॒रं वै बृ॒हदिरा॑मे॒व स॒ह दि॒वोप॑ ह्वयत॒ उप॑हूताः स॒प्त होत्रा॒ इत्या॑ह॒ होत्रा॑ ए॒वोप॑ ह्वयत॒ उप॑हूता धे॒नुः~(३७)

%2.6.7.3
स॒हर्\mbox{}ष॒भेत्या॑ह मिथु॒नमे॒वोप॑ ह्वयत॒ उप॑हूतो भ॒क्षः सखेत्या॑ह सोमपी॒थमे॒वोप॑ ह्वयत॒ उप॑हू॒ताँ~(४) हो इत्या॑हा॒ऽ॒ऽ॒त्मान॑मे॒वोप॑ ह्वयत आ॒त्मा ह्युप॑हूतानां॒ वसि॑ष्ठ॒ इडा॒मुप॑ ह्वयते प॒शवो॒ वा इडा॑ प॒शूने॒वोप॑ ह्वयते च॒तुरुप॑ ह्वयते॒ चतु॑ष्पादो॒ हि प॒शवो॑ मान॒वीत्या॑ह॒ मनु॒र्॒\mbox{}ह्ये॑ताम्~(३८)

%2.6.7.4
अग्रे\-ऽप॑श्यद् घृ॒तप॒दीत्या॑ह॒ यदे॒वास्यै॑ प॒दाद् घृ॒तमपी᳚ड्यत॒ तस्मा॑दे॒वमा॑ह मैत्रावरु॒णीत्या॑ह मि॒त्रावरु॑णौ॒ ह्ये॑नाꣳ स॒मैर॑यतां॒ ब्रह्म॑ दे॒वकृ॑त॒मुप॑हूत॒मित्या॑ह॒ ब्रह्मै॒वोप॑ ह्वयते॒ दैव्या॑ अध्व॒र्यव॒ उप॑हूता॒ उप॑हूता मनु॒ष्या॑ इत्या॑ह देवमनु॒ष्याने॒वोप॑ ह्वयते॒ य इ॒मं य॒ज्ञमवा॒न्॒ ये य॒ज्ञप॑तिं॒ वर्धा॒नित्या॑ह~(३९)

%2.6.7.5
य॒ज्ञाय॑ चै॒व यज॑मानाय चा॒शिष॒मा शा᳚स्त॒ उप॑हूते॒ द्यावा॑पृथि॒वी इत्या॑ह॒ द्यावा॑पृथि॒वी ए॒वोप॑ ह्वयते पूर्व॒जे ऋ॒ताव॑री॒ इत्या॑ह पूर्व॒जे ह्ये॑ते ऋ॒ताव॑री दे॒वी दे॒वपु॑त्रे॒ इत्या॑ह दे॒वी ह्ये॑ते दे॒वपु॑त्रे॒ उप॑हूतो॒\-ऽयं यज॑मान॒ इत्या॑ह॒ यज॑मानमे॒वोप॑ ह्वयत॒ उत्त॑रस्यां देवय॒ज्याया॒मुप॑हूतो॒ भूय॑सि हवि॒ष्कर॑ण॒ उप॑हूतो दि॒व्ये धाम॒न्नुप॑हूतः~(४०)

%2.6.7.6
इत्या॑ह प्र॒जा वा उत्त॑रा देवय॒ज्या प॒शवो॒ भूयो॑ हवि॒ष्कर॑णꣳ सुव॒र्गो लो॒को दि॒व्यं धामे॒दम॑सी॒दम॒सीत्ये॒व य॒ज्ञस्य॑ प्रि॒यं धामोप॑ ह्वयते॒ विश्व॑मस्य प्रि॒यमुप॑हूत॒मित्या॒हाछ॑म्बट्कारमे॒वोप॑ ह्वयते॥~(४१)

%2.6.8.0
{\anuvakamend[{आ॒ह॒ धे॒नुरे॒तां वर्धा॒नित्या॑ह॒ धाम॒न्नुप॑हूत॒श्चतु॑स्त्रिꣳशच्च}]}%~(७)

%2.6.8.1
प॒शवो॒ वा इडा᳚ स्व॒यमा द॑त्ते॒ काम॑मे॒वात्मना॑ पशू॒नामा द॑त्ते॒ न ह्य॑न्यः काम॑म्पशू॒नां प्र॒यच्छ॑ति वा॒चस्पत॑ये त्वा हु॒तं प्राश्ञा॒मीत्या॑ह॒ वाच॑मे॒व भा॑ग॒धेये॑न प्रीणाति॒ सद॑स॒स्पत॑ये त्वा हु॒तं प्राश्ञा॒मीत्या॑ह स्व॒गाकृ॑त्यै चतुरव॒त्तं भ॑वति ह॒विर्वै च॑तुरव॒त्तम्प॒शव॑श्चतुरव॒त्तं यद्धोता᳚ प्राश्ञी॒याद्धोता᳚~(४२)

%2.6.8.2
आर्ति॒मार्च्छे॒द्यद॒ग्नौ जु॑हु॒याद्रु॒द्राय॑ प॒शूनपि॑ दध्यादप॒शुर्यज॑मानः स्याद्वा॒चस्पत॑ये त्वा हु॒तं प्राश्ञा॒मीत्या॑ह प॒रोक्ष॑मे॒वैन॑ज्जुहोति॒ सद॑स॒स्पत॑ये त्वा हु॒तं प्राश्ञा॒मीत्या॑ह स्व॒गाकृ॑त्यै॒ प्राश्ञ॑न्ति ती॒र्थ ए॒व प्राश्ञ॑न्ति॒ दक्षि॑णां ददाति ती॒र्थ ए॒व दक्षि॑णां ददाति॒ वि वा ए॒तद्य॒ज्ञम्~(४३)

%2.6.8.3
छि॒न्द॒न्ति॒ यन्म॑ध्य॒तः प्रा॒श्ञन्त्य॒द्भिर्मा᳚र्जयन्त॒ आपो॒ वै सर्वा॑ दे॒वता॑ दे॒वता॑भिरे॒व य॒ज्ञꣳ सं त॑न्वन्ति दे॒वा वै य॒ज्ञाद्रु॒द्रम॒न्तरा॑य॒न्थ्स य॒ज्ञम॑विध्य॒त्तं दे॒वा अ॒भि सम॑गच्छन्त॒ कल्प॑तां न इ॒दमिति॒ ते᳚\-ऽब्रुव॒न्थ्स्वि॑ष्टं॒ वै न॑ इ॒दं भ॑विष्यति॒ यदि॒मꣳ रा॑धयि॒ष्याम॒ इति॒ तथ्स्वि॑ष्ट॒कृतः॑ स्विष्टकृ॒त्त्वन्तस्यावि॑द्धं॒ निः~(४४)

%2.6.8.4
अ॒कृ॒न्त॒न्॒ यवे॑न॒ सम्मि॑तं॒ तस्मा᳚द्यवमा॒त्रमव॑ द्ये॒द्यज्ज्यायो॑\-ऽव॒द्येद्रो॒पये॒त्तद्य॒ज्ञस्य॒ यदुप॑ च स्तृणी॒याद॒भि च॑ घा॒रये॑दुभयतः सꣴश्वा॒यि कु॑र्यादव॒दाया॒भि घा॑रयति॒ द्विः सम्प॑द्यते द्वि॒पाद्यज॑मानः॒ प्रति॑ष्ठित्यै॒ यत्ति॑र॒श्चीन॑मति॒हरे॒दन॑भिविद्धं य॒ज्ञस्या॒भि वि॑ध्ये॒दग्रे॑ण॒ परि॑ हरति ती॒र्थेनै॒व परि॑ हरति॒ तत्पू॒ष्णे पर्य॑हर॒न्तत्~(४५)

%2.6.8.5
पू॒षा प्राश्य॑ द॒तो॑\-ऽरुण॒त्तस्मा᳚त्पू॒षा प्र॑पि॒ष्टभा॑गो\-ऽद॒न्तको॒ हि तं दे॒वा अ॑ब्रुव॒न्वि वा अ॒यमा᳚र्ध्यप्राशित्रि॒यो वा अ॒यम॑भू॒दिति॒ तद्बृह॒स्पत॑ये॒ पर्य॑हर॒न्थ्सो॑\-ऽबिभे॒द्बृह॒स्पति॑रि॒त्थं वाव स्य आर्ति॒मारि॑ष्य॒तीति॒ स ए॒तम्मन्त्र॑मपश्य॒थ्सूर्य॑स्य त्वा॒ चक्षु॑षा॒ प्रति॑ पश्या॒मीत्य॑ब्रवी॒न्न हि सूर्य॑स्य॒ चक्षुः॑~(४६)

%2.6.8.6
किं च॒न हि॒नस्ति॒ सो॑\-ऽबिभेत्प्रतिगृ॒ह्णन्तं॑ मा हिꣳसिष्य॒तीति॑ दे॒वस्य॑ त्वा सवि॒तुः प्र॑स॒वे᳚\-ऽश्विनो᳚र्बा॒हु\-भ्यां᳚ पू॒ष्णो हस्ता᳚भ्यां॒ प्रति॑ गृह्णा॒मीत्य॑ब्रवीथ्सवि॒तृप्र॑सूत ए॒वैन॒द्ब्रह्म॑णा दे॒वता॑भिः॒ प्रत्य॑\-गृह्णा॒थ्सो॑\-ऽबिभेत्प्रा॒श्ञन्तं॑ मा हिꣳसिष्य॒तीत्य॒ग्नेस्त्वा॒स्ये॑न॒ प्राश्ञा॒मीत्य॑ब्रवी॒न्न ह्य॑ग्नेरा॒स्यं॑ किं च॒न हि॒नस्ति॒ सो॑\-ऽबिभेत्~(४७)

%2.6.8.7
प्राशि॑तं मा हिꣳसिष्य॒तीति॑ ब्राह्म॒णस्यो॒दरे॒णेत्य॑ब्रवी॒न्न हि ब्रा᳚ह्म॒णस्यो॒दरं॒ किं च॒न हि॒नस्ति॒ बृह॒स्पते॒र्ब्रह्म॒णेति॒ स हि ब्रह्मि॒ष्ठो\-ऽप॒ वा ए॒तस्मा᳚त्प्रा॒णाः क्रा॑मन्ति॒ यः प्रा॑शि॒त्रं प्रा॒श्ञात्य॒द्भिर्मा᳚र्जयि॒त्वा प्रा॒णान्थ्सम्मृ॑शते॒\-ऽमृतं॒ वै प्रा॒णा अ॒मृत॒मापः॑ प्रा॒णाने॒व य॑थास्था॒नमुप॑ ह्वयते॥~(४८)

%2.6.9.0
{\anuvakamend[{प्रा॒श्ञी॒याद्धोता॑ य॒ज्ञं निर॑हर॒न्तच्चक्षु॑रा॒स्य॑ङ्किं च॒न हि॒नस्ति॒ सो॑\-ऽबिभे॒च्चतु॑श्चत्वारिꣳशच्च}]}%~(८)

%2.6.9.1
अ॒ग्नीध॒ आ द॑धात्य॒ग्निमु॑खाने॒वर्तून्प्री॑णाति स॒मिध॒मा द॑धा॒त्युत्त॑रासा॒माहु॑तीनां॒ प्रति॑ष्ठित्या॒ अथो॑ स॒मिद्व॑त्ये॒व जु॑होति परि॒धीन्थ्सम्मा᳚र्ष्टि पु॒नात्ये॒वैना᳚न्थ्स॒कृथ्स॑कृ॒थ्सम्मा᳚र्ष्टि॒ परा॑ङिव॒ ह्ये॑तर्\mbox{}हि॑ य॒ज्ञश्च॒तुः सम्प॑द्यते॒ चतु॑ष्पादः प॒शवः॑ प॒शूने॒वाव॑ रुन्धे॒ ब्रह्म॒न्प्र स्था᳚स्याम॒ इत्या॒हात्र॒ वा ए॒तर्\mbox{}हि॑ य॒ज्ञः श्रि॒तः~(४९)

%2.6.9.2
यत्र॑ ब्र॒ह्मा यत्रै॒व य॒ज्ञः श्रि॒तस्तत॑ ए॒वैन॒मा र॑भते॒ यद्धस्ते॑न प्र॒मीवे᳚द्वेप॒नः स्या॒द्यच्छी॒र्ष्णा शी॑र्\mbox{}षक्ति॒मान्थ्स्या॒द्यत्तू॒ष्णीमासी॒तासं॑ प्रत्तो य॒ज्ञः स्या॒त्प्र ति॒ष्ठेत्ये॒व ब्रू॑याद्वा॒चि वै य॒ज्ञः श्रि॒तो यत्रै॒व य॒ज्ञः श्रि॒तस्तत॑ ए॒वैन॒ꣳ॒ सम्प्र य॑च्छति॒ देव॑ सवितरे॒तत्ते॒ प्र~(५०)

%2.6.9.3
आ॒हेत्या॑ह॒ प्रसू᳚त्यै॒ बृह॒स्पति॑र्ब्र॒ह्मेत्या॑ह॒ स हि ब्रह्मि॑ष्ठः॒ स य॒ज्ञम्पा॑हि॒ स य॒ज्ञप॑तिम्पाहि॒ स माम्पा॒हीत्या॑ह य॒ज्ञाय॒ यज॑मानाया॒त्मने॒ तेभ्य॑ ए॒वाशिष॒मा शा॒स्ते\-ऽना᳚र्त्या आ॒श्राव्या॑ह दे॒वान् य॒जेति॑ ब्रह्मवा॒दिनो॑ वदन्ती॒ष्टा दे॒वता॒ अथ॑ कत॒म ए॒ते दे॒वा इति॒ छन्दा॒ꣳ॒सीति॑ ब्रूयाद्गाय॒त्रीं त्रि॒ष्टुभम्᳚~(५१)

%2.6.9.4
जग॑ती॒मित्यथो॒ खल्वा॑हुर्ब्राह्म॒णा वै छन्दा॒ꣳ॒सीति॒ ताने॒व तद्य॑जति दे॒वानां॒ वा इ॒ष्टा दे॒वता॒ आस॒न्नथा॒ग्निर्नोद॑ज्वल॒त्तं दे॒वा आहु॑तीभिरनूया॒जेष्वन्व॑विन्द॒न्॒ यद॑नूया॒जान् यज॑त्य॒ग्निमे॒व तथ्समि॑न्द्ध ए॒तदु॒र्वै नामा॑सु॒र आ॑सी॒थ्स ए॒तर्\mbox{}हि॑ य॒ज्ञस्या॒शिष॑मवृङ्क्त॒ यद्ब्रू॒यादे॒तत्~(५२)

%2.6.9.5
उ॒ द्या॒वा॒पृ॒थि॒वी॒ भ॒द्रम॑भू॒दित्ये॒तदु॑मे॒वासु॒रं य॒ज्ञस्या॒शिषं॑ गमयेदि॒दं द्या॑वापृथिवी भ॒द्रम॑भू॒दित्ये॒व ब्रू॑या॒द्यज॑मानमे॒व य॒ज्ञस्या॒ऽऽशिषं॑ गमय॒त्यार्ध्म॑ सूक्तवा॒कमु॒त न॑मोवा॒कमित्या॑\-हे॒दम॑रा॒थ्स्मेति॒ वावैतदा॒होप॑श्रितो दि॒वः पृ॑थि॒व्योरित्या॑ह॒ द्यावा॑पृथि॒व्योर्\mbox{}हि य॒ज्ञ उप॑श्रित॒ ओम॑न्वती ते॒\-ऽस्मिन् य॒ज्ञे य॑जमान॒ द्यावा॑पृथि॒वी~(५३)

%2.6.9.6
स्ता॒मित्या॑हा॒ऽ॒ऽ॒शिष॑मे॒वैतामा शा᳚स्ते॒ यद्ब्रू॒याथ्सू॑पावसा॒ना च॑ स्वध्यवसा॒ना चेति॑ प्र॒मायु॑को॒ यज॑मानः स्याद्य॒दा हि प्र॒मीय॒ते\-ऽथे॒मामु॑पाव॒स्यति॑ सूपचर॒णा च॑ स्वधिचर॒णा चेत्ये॒व ब्रू॑या॒द्वरी॑यसीमे॒वास्मै॒ गव्यू॑ति॒मा शा᳚स्ते॒ न प्र॒मायु॑को भवति॒ तयो॑रा॒विद्य॒ग्निरि॒दꣳ ह॒विर॑जुष॒तेत्या॑ह॒ या अया᳚क्ष्म~(५४)

%2.6.9.7
दे॒वता॒स्ता अ॑रीरधा॒मेति॒ वावैतदा॑ह॒ यन्न नि॑र्दि॒शेत्प्रति॑वेशं य॒ज्ञस्या॒ऽऽशीर्ग॑च्छे॒दा शा᳚स्ते॒\-ऽयं यज॑मानो॒\-ऽसावित्या॑ह नि॒र्दिश्यै॒वैनꣳ॑ सुव॒र्गं लो॒कं ग॑मय॒त्यायु॒रा शा᳚स्ते सुप्रजा॒स्त्वमा शा᳚स्त॒ इत्या॑हा॒ऽऽशिष॑मे॒वैतामा शा᳚स्ते सजातवन॒स्यामा शा᳚स्त॒ इत्या॑ह प्रा॒णा वै स॑जा॒ताः प्रा॒णाने॒व~(५५)

%2.6.9.8
नान्तरे॑ति॒ तद॒ग्निर्दे॒वो दे॒वेभ्यो॒ वन॑ते व॒यम॒ग्नेर्मानु॑षा॒ इत्या॑हा॒ग्निर्दे॒वेभ्यो॑ वनु॒ते व॒यं म॑नु॒ष्ये᳚भ्य॒ इति॒ वावैतदा॑हे॒ह गति॑र्वा॒मस्ये॒दं च॒ नमो॑ दे॒वेभ्य॒ इत्या॑ह॒ याश्चै॒व दे॒वता॒ यज॑ति॒ याश्च॒ न ताभ्य॑ ए॒वोभयी᳚भ्यो॒ नम॑स्करोत्या॒त्मनो\-ऽना᳚र्त्यै॥~(५६)

%2.6.10.0
{\anuvakamend[{श्रि॒तस्ते॒ प्र त्रि॒ष्टुभ॑मे॒तद्द्यावा॑पृथि॒वी या अया᳚क्ष्म प्रा॒णाने॒व षट्च॑त्वारिꣳशच्च}]}%~(९)

%2.6.10.1
दे॒वा वै य॒ज्ञस्य॑ स्वगाक॒र्तारं॒ नावि॑न्द॒न्ते शं॒ युम्बा॑र्\mbox{}हस्प॒त्य\-म॑ब्रुवन्नि॒मं नो॑ य॒ज्ञꣴ स्व॒गा कु॒र्विति॒ सो᳚\-ऽब्रवी॒द्वरं॑ वृणै॒ यदे॒वाब्रा᳚ह्मणो॒क्तो\-ऽश्र॑द्दधानो॒ यजा॑तै॒ सा मे॑ य॒ज्ञस्या॒शीर॑स॒दिति॒ तस्मा॒द्यदब्रा᳚ह्मणो॒क्तो\-ऽश्र॑द्दधानो॒ यज॑ते शं॒ युमे॒व तस्य॑ बार्\mbox{}हस्प॒त्यं य॒ज्ञस्या॒शीर्ग॑च्छ\-त्ये॒तन्ममेत्य॑ब्रवी॒त्किम्मे᳚ प्र॒जायाः᳚~(५७)

%2.6.10.2
इति॒ यो॑\-ऽपगु॒रातै॑ श॒तेन॑ यातया॒द्यो नि॒हन॑थ्स॒हस्रे॑ण यातया॒द्यो लोहि॑तं क॒रव॒द्याव॑तः प्र॒स्कद्य॑ पाꣳ॒सून्थ्सं॑गृ॒ह्णात्ताव॑तः संवथ्स॒रान्पि॑तृलो॒कं न प्र जा॑ना॒दिति॒ तस्मा᳚द्ब्राह्म॒णाय॒ नाप॑ गुरेत॒ न नि ह॑न्या॒न्न लोहि॑तं कुर्यादे॒ताव॑ता॒ हैन॑सा भवति॒ तच्छं॒ योरा वृ॑णीमह॒ इत्या॑ह य॒ज्ञमे॒व तथ्स्व॒गा क॑रोति॒ तत्~(५८)

%2.6.10.3
शं॒ योरा वृ॑णीमह॒ इत्या॑ह शं॒ युमे॒व बा॑र्\mbox{}हस्प॒त्यम्भा॑ग॒धेये॑न॒ सम॑र्धयति गा॒तुं य॒ज्ञाय॑ गा॒तुं य॒ज्ञप॑तय॒ इत्या॑हा॒ऽ॒ऽ॒शिष॑मे॒वैतामा शा᳚स्ते॒ सोमं॑ यजति॒ रेत॑ ए॒व तद्द॑धाति॒ त्वष्टा॑रं यजति॒ रेत॑ ए॒व हि॒तं त्वष्टा॑ रू॒पाणि॒ वि क॑रोति दे॒वाना॒म्पत्नी᳚र्यजति मिथुन॒त्वाया॒ग्निं गृ॒हप॑तिं यजति॒ प्रति॑ष्ठित्यै जा॒मि वा ए॒तद्य॒ज्ञस्य॑ क्रियते~(५९)

%2.6.10.4
यदाज्ये॑न प्रया॒जा इ॒ज्यन्त॒ आज्ये॑न पत्नीसंया॒जा ऋच॑म॒नूच्य॑ पत्नीसंया॒जाना॑मृ॒चा य॑ज॒त्यजा॑मित्वा॒याथो॑ मिथुन॒त्वाय॑ प॒ङ्क्तिप्रा॑यणो॒ वै य॒ज्ञः प॒ङ्क्त्यु॑दयनः॒ पञ्च॑ प्रया॒जा इ॑ज्यन्ते च॒त्वारः॑ पत्नीसंया॒जाः स॑मिष्टय॒जुः प॑ञ्च॒मम्प॒ङ्क्तिमे॒वानु॑ प्र॒ यन्ति॑ प॒ङ्क्तिमनूद्य॑न्ति॥~(६०)

%2.6.11.0
{\anuvakamend[{प्र॒जायाः᳚ करोति॒ तत्क्रि॑यते॒ त्रय॑स्त्रिꣳशच्च}]}%॥10॥

%2.6.11.1
यु॒क्ष्वा हि दे॑व॒हूत॑मा॒ꣳ॒ अश्वाꣳ॑ अग्ने र॒थीरि॑व। नि होता॑ पू॒र्व्यः स॑दः। उ॒त नो॑ देव दे॒वाꣳ अच्छा॑ वोचो वि॒दुष्ट॑रः। श्रद्विश्वा॒ वार्या॑ कृधि। त्वꣳ ह॒ यद्य॑विष्ठ्य॒ सह॑सः सूनवाहुत। ऋ॒तावा॑ य॒ज्ञियो॒ भुवः॑। अ॒यम॒ग्निः स॑ह॒स्रिणो॒ वाज॑स्य श॒तिन॒स्पतिः॑। मू॒र्धा क॒वी र॑यी॒णाम्। तं ने॒मिमृ॒भवो॑ य॒था न॑मस्व॒ सहू॑तिभिः। नेदी॑यो य॒ज्ञम्~(६१)

%2.6.11.2
अ॒ङ्गि॒रः॒। तस्मै॑ नू॒नम॒भिद्य॑वे वा॒चा वि॑रूप॒ नित्य॑या। वृष्णे॑ चोदस्व सुष्टु॒तिम्। कमु॑ ष्विदस्य॒ सेन॑या॒ग्नेरपा॑कचक्षसः। प॒णिं गोषु॑ स्तरामहे। मा नो॑ दे॒वानां॒ विशः॑ प्रस्ना॒तीरि॑वो॒स्राः। कृ॒शं न हा॑सु॒रघ्नि॑याः। मा नः॑ समस्य दू॒ढ्यः॑ परि॑द्वेषसो अꣳह॒तिः। ऊ॒र्मिर्न नाव॒मा व॑धीत्। नम॑स्ते अग्न॒ ओज॑से गृ॒णन्ति॑ देव कृ॒ष्टयः॑। अमैः᳚~(६२)

%2.6.11.3
अ॒मित्र॑मर्दय। कु॒विथ्सु नो॒ गवि॑ष्ट॒ये\-ऽग्ने॑ सं॒वेषि॑षो र॒यिम्। उरु॑कृदु॒रु ण॑स्कृधि। मा नो॑ अ॒स्मिन्म॑हाध॒ने परा॑ वर्ग्भार॒भृद्य॑था। सं॒वर्ग॒ꣳ॒ सꣳ र॒यिञ्ज॑य। अ॒न्यम॒स्मद्भि॒या इ॒यमग्ने॒ सिष॑क्तु दु॒च्छुना᳚। वर्धा॑ नो॒ अम॑व॒च्छवः॑। यस्याजु॑षन्नम॒स्विनः॒ शमी॒मदु॑र्मखस्य वा। तं घेद॒ग्निर्वृ॒धाव॑ति। पर॑स्या॒ अधि॑~(६३)

%2.6.11.4
सं॒वतो\-ऽव॑राꣳ अ॒भ्या त॑र। यत्रा॒हमस्मि॒ ताꣳ अ॑व। वि॒द्मा हि ते॑ पु॒रा व॒यमग्ने॑ पि॒तुर्यथाव॑सः। अधा॑ ते सु॒म्नमी॑महे। य उ॒ग्र इ॑व शर्य॒हा ति॒ग्मशृ॑ङ्गो॒ न वꣳस॑गः। अग्ने॒ पुरो॑ रु॒रोजि॑थ। सखा॑यः॒ सं वः॑ स॒म्यञ्च॒मिष॒ꣴ॒ स्तोमं॑ चा॒ग्नये᳚। वर्\mbox{}षि॑ष्ठाय क्षिती॒नामू॒र्जो नप्त्रे॒ सह॑स्वते। सꣳस॒मिद्यु॑वसे वृष॒न्नग्ने॒ विश्वा᳚न्य॒र्य आ। इ॒डस्प॒दे समि॑ध्यसे॒ स नो॒ वसू॒न्या भ॑र। प्रजा॑पते॒ स वे॑द॒ सोमा॑पूषणे॒मौ दे॒वौ॥~(६४)

%2.6.12.0
{\anuvakamend[{य॒ज्ञममै॒रधि॑ वृष॒न्नेका॒न्नविꣳ॑श॒तिश्च॑}]}%॥11॥

%2.6.12.1
उ॒शन्त॑स्त्वा हवामह उ॒शन्तः॒ समि॑धीमहि। उ॒शन्नु॑श॒त आ व॑ह॒ पि॒तॄन् ह॒विषे॒ अत्त॑वे। त्वꣳ सो॑म॒ प्रचि॑कितो मनी॒षा त्वꣳ रजि॑ष्ठ॒मनु॑ नेषि॒ पन्था᳚म्। तव॒ प्रणी॑ती पि॒तरो॑ न इन्दो दे॒वेषु॒ रत्न॑मभजन्त॒ धीराः᳚। त्वया॒ हि नः॑ पि॒तरः॑ सोम॒ पूर्वे॒ कर्मा॑णि च॒क्रुः प॑वमान॒ धीराः᳚। व॒न्वन्नवा॑तः परि॒धीꣳ रपो᳚र्णु वी॒रेभि॒रश्वै᳚र्म॒घवा॑ भव~(६५)

%2.6.12.2
नः॒। त्वꣳ सो॑म पि॒तृभिः॑ संविदा॒नो\-ऽनु॒ द्यावा॑पृथि॒वी आ त॑तन्थ। तस्मै॑ त इन्दो ह॒विषा॑ विधेम व॒यꣴ स्या॑म॒ पत॑यो रयी॒णाम्। अग्नि॑ष्वात्ताः पितर॒ एह ग॑च्छत॒ सदः॑सदः सदत सुप्रणीतयः। अ॒त्ता ह॒वीꣳषि॒ प्रय॑तानि ब॒र्॒\mbox{}हिष्यथा॑ र॒यिꣳ सर्व॑वीरं दधातन। बर्\mbox{}हि॑षदः पितर ऊ॒त्य॑र्वागि॒मा वो॑ ह॒व्या च॑कृमा जु॒षध्वम्᳚। त आ ग॒ताव॑सा॒ शन्त॑मे॒नाथा॒स्मभ्यम्᳚~(६६)

%2.6.12.3
शं योर॑र॒पो द॑धात। आहं पि॒त़ॄन्थ्सु॑वि॒दत्राꣳ॑ अविथ्सि॒ नपा॑तञ्च वि॒क्रम॑णं च॒ विष्णोः᳚। ब॒र्॒\mbox{}हि॒षदो॒ ये स्व॒धया॑ सु॒तस्य॒ भज॑न्त पि॒त्वस्त इ॒हाग॑मिष्ठाः। उप॑हूताः पि॒तरो॑ बर्\mbox{}हि॒ष्ये॑षु नि॒धिषु॑ प्रि॒येषु॑। त आग॑मन्तु॒ त इ॒ह श्रु॑व॒न्त्वधि॑ ब्रुवन्तु॒ ते अ॑वन्त्व॒स्मान्। उदी॑रता॒मव॑र॒ उत्परा॑स॒ उन्म॑ध्य॒माः पि॒तरः॑ सो॒म्यासः॑। असुम्᳚~(६७)

%2.6.12.4
य ई॒युर॑वृ॒का ऋ॑त॒ज्ञास्ते नो॑\-ऽवन्तु पि॒तरो॒ हवे॑षु। इ॒दं पि॒तृभ्यो॒ नमो॑ अस्त्व॒द्य ये पूर्वा॑सो॒ य उप॑रास ई॒युः। ये पार्थि॑वे॒ रज॒स्या निष॑त्ता॒ ये वा॑ नू॒नꣳ सु॑वृ॒जना॑सु वि॒क्षु। अधा॒ यथा॑ नः पि॒तरः॒ परा॑सः प्र॒त्नासो॑ अग्न ऋ॒तमा॑शुषा॒णाः। शुचीद॑य॒न्दीधि॑तिमुक्थ॒शासः॒ क्षामा॑ भि॒न्दन्तो॑ अरु॒णीरप॑ व्रन्न्। यद॑ग्ने~(६८)

%2.6.12.5
क॒व्य॒वा॒ह॒न॒ पि॒तॄन् यक्ष्यृ॑ता॒वृधः॑। प्र च॑ ह॒व्यानि॑ वक्ष्यसि दे॒वेभ्य॑श्च पि॒तृभ्य॒ आ। त्वम॑ग्न ईडि॒तो जा॑तवे॒दो\-ऽवा᳚ड्ढ॒व्यानि॑ सुर॒भीणि॑ कृ॒त्वा। प्रादाः᳚ पि॒तृभ्यः॑ स्व॒धया॒ ते अ॑क्षन्न॒द्धि त्वं दे॑व॒ प्रय॑ता ह॒वीꣳषि॑। मात॑ली क॒व्यैर्य॒मो अङ्गि॑रोभि॒र्बृह॒स्पति॒र्॒\mbox{}ऋक्व॑भिर्वावृधा॒नः। याꣴश्च॑ दे॒वा वा॑वृ॒धुर्ये च॑ दे॒वान्थ्\-स्वाहा॒न्ये स्व॒धया॒न्ये म॑दन्ति।~(६९)

%2.6.12.6
इ॒मं य॑म प्रस्त॒रमा हि सीदाङ्गि॑रोभिः पि॒तृभिः॑ संविदा॒नः। आ त्वा॒ मन्त्राः᳚ कविश॒स्ता व॑हन्त्वे॒ना रा॑जन् ह॒विषा॑ मादयस्व। अङ्गि॑रोभि॒रा ग॑हि य॒ज्ञिये॑भि॒र्यम॑ वैरू॒पैरि॒ह मा॑दयस्व। विव॑स्वन्तꣳ हुवे॒ यः पि॒ता ते॒\-ऽस्मिन् य॒ज्ञे ब॒र्॒\mbox{}हिष्या नि॒षद्य॑। अङ्गि॑रसो नः पि॒तरो॒ नव॑ग्वा॒ अथ॑र्वाणो॒ भृग॑वः सो॒म्यासः॑। तेषां᳚ व॒यꣳ सु॑म॒तौ य॒ज्ञिया॑ना॒मपि॑ भ॒द्रे सौ॑मन॒से स्या॑म॥~(७०)

{\prashnaend[{स॒मिधो॑ या॒ज्या॑ तस्मा॒न्नाभा॒गꣳ हि तमन्वित्या॑ह प्र॒जा वा आ॒हेत्या॑ह यु॒क्ष्वा हि स॑प्त॒तिः॥७०॥ स॒मिधः॑ सौमन॒से स्या॑म॥}]}
%%% END PRASHNA

%2.6.0.0
{\prashnaend[{स॒मिध॒श्चक्षु॑षी प्र॒जा\-प॑ति॒राज्यं॑ दे॒वस्य॒ स्फ्यम्ब्र॑ह्मवा॒दिनो॒\-ऽद्भिर॒ग्नेस्त्रयो॒ मनुः॑ पृथि॒व्याः प॒शवो॒\-ऽग्नीधे॑ दे॒वा वै य॒ज्ञस्य॑ यु॒क्ष्वोशन्त॑स्त्वा॒ द्वाद॑श}]%॥12॥
}
%%% END PRASHNA


%3.1.0.0
{\anuvakamend[{भ॒वा॒स्मभ्य॒मसुं॒ यद॑ग्ने मदन्ति सौमन॒स एक॑ञ्च}]}%॥12॥

%3.1.0.0
% {\anuvakamend[{प्र॒जा\-प॑तिरकामयतै॒ष ते॑ य॒ज्ञं वै प्र॒जाप॑ते॒र्जाय॑मानाः प्राजाप॒त्या यो वा अय॑थादेवतमि॒ष्टर्गो॑ निग्रा॒भ्याः᳚ स्थ॒ यो वै दे॒वां जुष्टो॒\-ऽग्निना॑ र॒यिमेका॑\-दश}]}%॥11॥ 
%%% END KANDAM
