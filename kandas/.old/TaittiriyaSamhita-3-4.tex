\sect{चतुर्थः प्रश्नः}
\dnsub{तैत्तिरीयसंहितायां तृतीयकाण्डे चतुर्थः प्रश्नः}
%3.4.1.0
{\scriptsize {व॒र्त॒येत्या॑ह न॒ इति॒ वै नाभ्या॒ उल्ब॑मि॒वैक॑विशतिश्च॥1॥}}

%3.4.1.1
वि वा ए॒तस्य॑ य॒ज्ञ ऋ॑ध्यते॒ यस्य॑ ह॒विर॑ति॒रिच्य॑ते॒ सूर्यो॑ दे॒वो दि॑वि॒षद्भ्य॒ इत्या॑ह॒ बृह॒स्पति॑ना चै॒वास्य॑ प्र॒जाप॑तिना च य॒ज्ञस्य॒ व्यृ॑द्ध॒मपि॑ वपति॒ रख्षाꣳ॑सि॒ वा ए॒तत्प॒शु स॑चन्ते॒ यदे॑कदेव॒त्य॑ आल॑ब्धो॒ भूया॒न्भव॑ति॒ यस्यास्ते॒ हरि॑तो॒ गर्भ॒ इत्या॑ह देव॒त्रैवैनां गमयति॒ रख्ष॑सा॒मप॑हत्या॒ आ व॑र्तन वर्त॒येत्या॑ह [1]

%3.4.1.2
ब्रह्म॑णै॒वैन॒मा व॑र्तयति॒ वि ते॑ भिनद्मि तक॒रीमित्या॑ह यथाय॒जुरे॒वैतदु॑रुद्र॒प्सो वि॒श्वरू॑प॒ इन्दु॒रित्या॑ह प्र॒जा वै प॒शव॒ इन्दुः॑ प्र॒जयै॒वैन॑म्प॒शुभिः॒ सम॑र्धयति॒ दिवं॒ वै य॒ज्ञस्य॒ व्यृ॑द्धं गच्छति पृथि॒वीमति॑रिक्त॒न्तद्यन्न श॒मये॒दार्ति॒मार्च्छे॒द्यज॑मानो म॒ही द्यौः पृ॑थि॒वी च॑ न॒ इति॑ [2]

%3.4.1.3
आ॒ह॒ द्यावा॑पृथि॒वीभ्या॑मे॒व य॒ज्ञस्य॒ व्यृ॑द्धं॒ चाति॑रिक्तं च शमयति॒ नार्ति॒मार्च्छ॑ति॒ यज॑मानो॒ भस्म॑ना॒भि समू॑हति स्व॒गाकृ॑त्या॒ अथो॑ अ॒नयो॒र्वा ए॒ष गर्भो॒ऽनयो॑रे॒वैनं॑ दधाति॒ यद॑व॒द्येदति॒ तद्रे॑चये॒द्यन्नाव॒द्येत्प॒शोराल॑ब्धस्य॒ नाव॑ द्येत् पु॒रस्ता॒न्नाभ्या॑ अ॒न्यद॑व॒द्येदु॒परि॑ष्टाद॒न्यत्पु॒रस्ता॒द्वै नाभ्यै [3]

%3.4.1.4
प्रा॒ण उ॒परि॑ष्टादपा॒नो यावा॑ने॒व प॒शुस्तस्याव॑ द्यति॒ विष्ण॑वे शिपिवि॒ष्टाय॑ जुहोति॒ यद्वै य॒ज्ञस्या॑ति॒रिच्य॑ते॒ यः प॒शोर्भू॒मा या पुष्टि॒स्तद्विष्णुः॑ शिपिवि॒ष्टोऽति॑रिक्त ए॒वाति॑रिक्तं दधा॒त्यति॑रिक्तस्य॒ शान्त्या॑ अ॒ष्टाप्रू॒ड्ढिर॑ण्यं॒ दख्षि॑णा॒ऽष्टाप॑दी॒ ह्ये॑षात्मा न॑व॒मः प॒शोराप्त्या॑ अन्तरको॒श उ॒ष्णीषे॒णावि॑ष्टितम्भवत्ये॒वमि॑व॒ हि प॒शुरुल्ब॑मिव॒ चर्मे॑व मा॒ꣳ॒समि॒वास्थी॑व॒ यावा॑ने॒व प॒शुस्तमा॒प्त्वाव॑ रुन्द्धे॒ यस्यै॒षा य॒ज्ञे प्राय॑श्चित्तिः क्रि॒यत॑ इ॒ष्ट्वा वसी॑यान्भवति॥ [4]

%3.4.2.0
{\scriptsize {सर॑स्वत्यै॒ स्वाहा॒ मनु॒स्त्रयो॑दश च॥2॥}}

%3.4.2.1
आ वा॑यो भूष शुचिपा॒ उप॑ नः स॒हस्रं॑ ते नि॒युतो॑ विश्ववार। उपो॑ ते॒ अन्धो॒ मद्य॑मयामि॒ यस्य॑ देव दधि॒षे पूर्व॒पेयम्। आकूत्यै त्वा॒ कामा॑य त्वा स॒मृधे त्वा किक्कि॒टा ते॒ मनः॑ प्र॒जाप॑तये॒ स्वाहा॑ किक्कि॒टा ते प्रा॒णं वा॒यवे॒ स्वाहा॑ किक्कि॒टा ते॒ चख्षुः॒ सूर्या॑य॒ स्वाहा॑ किक्कि॒टा ते॒ श्रोत्रं॒ द्यावा॑पृथि॒वीभ्या॒ꣳ॒ स्वाहा॑ किक्कि॒टा ते॒ वाच॒ꣳ॒ सर॑स्वत्यै॒ स्वाहा [5]

%3.4.2.2
त्वं तु॒रीया॑ व॒शिनी॑ व॒शासि॑ स॒कृद्यत्त्वा॒ मन॑सा॒ गर्भ॒ आश॑यत्। व॒शा त्वं व॒शिनी॑ गच्छ दे॒वान्थ्स॒त्याः स॑न्तु॒ यज॑मानस्य॒ कामाः। अ॒जासि॑ रयि॒ष्ठा पृ॑थि॒व्या सी॑दो॒र्ध्वान्तरि॑ख्ष॒मुप॑ तिष्ठस्व दि॒वि ते॑ बृ॒हद्भाः। तन्तुं॑ त॒न्वन्रज॑सो भा॒नुमन्वि॑हि॒ ज्योति॑ष्मतः प॒थो र॑ख्ष धि॒या कृ॒तान्। अ॒नु॒ल्ब॒णं व॑यत॒ जोगु॑वा॒मपो॒ मनु॑र्भव ज॒नया॒ दैव्यं॒ जनम्। मन॑सो ह॒विर॑सि प्र॒जाप॑ते॒र्वर्णो॒ गात्रा॑णां ते गात्र॒भाजो॑ भूयास्म॥ [6]

%3.4.3.0
{\scriptsize {यथ्स्वेन॑ सारस्व॒तीमा ल॑भेत॒ यः कामा॑य त्वा॒ कामोऽप॒ इत्य॒भ्रो द्विच॑त्वारिशच्च॥3॥}}

%3.4.3.1
इ॒मे वै स॒हास्ता॒न्ते वा॒युर्व्य॑वा॒त्ते गर्भ॑मदधाता॒न्त सोमः॒ प्राज॑नयद॒ग्निर॑ग्रसत॒ स ए॒तम्प्र॒जाप॑तिराग्ने॒यम॒ष्टाक॑पाल- मपश्य॒त्तं निर॑वप॒त्तेनै॒वैना॑म॒ग्नेरधि॒ निर॑क्रीणा॒त्तस्मा॒दप्य॑न्यदेव॒त्या॑मा॒लभ॑मान आग्ने॒यम॒ष्टाक॑पालम्पु॒रस्ता॒न्निर्व॑पेद॒ग्ने- रे॒वैना॒मधि॑ नि॒ष्क्रीया ल॑भते॒ यत् [7]

%3.4.3.2
वा॒युर्व्यवा॒त्तस्माद्वाय॒व्या॑ यदि॒मे गर्भ॒मद॑धातां॒ तस्माद्द्यावापृथि॒व्या॑ यथ्सोमः॒ प्राज॑नयद॒ग्निरग्र॑सत॒ तस्मा॑दग्नीषो॒मीया॒ यद॒नयोर्विय॒त्योर्वागव॑द॒त्तस्माथ्सारस्व॒ती यत्प्र॒जाप॑तिर॒ग्नेरधि॑ नि॒रक्री॑णा॒त्तस्मात्प्राजाप॒त्या सा वा ए॒षा स॑र्वदेव॒त्या॑ यद॒जा व॒शा वा॑य॒व्या॑मा ल॑भेत॒ भूति॑कामो वा॒युर्वै ख्षेपि॑ष्ठा दे॒वता॑ वा॒युमे॒व स्वेन॑ [8]

%3.4.3.3
भा॒ग॒धेये॒नोप॑ धावति॒ स ए॒वैन॒म्भूतिं॑ गमयति द्यावापृथि॒व्या॑मा ल॑भेत कृ॒षमा॑णः प्रति॒ष्ठाका॑मो दि॒व ए॒वास्मै॑ प॒र्जन्यो॑ वर्\mbox{}षति॒ व्य॑स्यामोष॑धयो रोहन्ति स॒मर्धु॑कमस्य स॒स्यम्भ॑वत्यग्नीषो॒मीया॒मा ल॑भेत॒ यः का॒मये॒तान्न॑वानन्ना॒दः स्या॒मित्य॒ग्निनै॒वान्न॒मव॑ रुन्द्धे॒ सोमे॑ना॒न्नाद्य॒मन्न॑वाने॒वान्ना॒दो भ॑वति सारस्व॒तीमा ल॑भेत॒ यः [9]

%3.4.3.4
ई॒श्व॒रो वा॒चो वदि॑तोः॒ सन्वाचं॒ न वदे॒द्वाग्वै सर॑स्वती॒ सर॑स्वतीमे॒व स्वेन॑ भाग॒धेये॒नोप॑ धावति॒ सैवास्मि॒न्वाचं॑ दधाति प्राजाप॒त्यामा ल॑भेत॒ यः का॒मये॒तान॑भिजितम॒भि ज॑येय॒मिति॑ प्र॒जाप॑तिः॒ सर्वा॑ दे॒वता॑ दे॒वता॑भिरे॒वान॑भि- जितम॒भि ज॑यति वाय॒व्य॑यो॒पाक॑रोति वा॒योरे॒वैना॑मव॒रुध्या ल॑भत॒ आकूत्यै त्वा॒ कामा॑य त्वा [10]

%3.4.3.5
इत्या॑ह यथाय॒जुरे॒वैतत्कि॑क्किटा॒कारं॑ जुहोति किक्किटाका॒रेण॒ वै ग्रा॒म्याः प॒शवो॑ रमन्ते॒ प्रार॒ण्याः प॑तन्ति॒ यत्कि॑क्किटा॒कारं॑ जु॒होति॑ ग्रा॒म्याणाम्पशू॒नां धृत्यै॒ पर्य॑ग्नौ क्रि॒यमा॑णे जुहोति॒ जीव॑न्तीमे॒वैनाꣳ॑ सुव॒र्गं लो॒कङ्ग॑मयति॒ त्वं तु॒रीया॑ व॒शिनी॑ व॒शासीत्या॑ह देव॒त्रैवैनां गमयति स॒त्याः स॑न्तु॒ यज॑मानस्य॒ कामा॒ इत्या॑है॒ष वै कामः॑ [11]

%3.4.3.6
यज॑मानस्य॒ यदनार्त उ॒दृचं॒ गच्छ॑ति॒ तस्मा॑दे॒वमा॑हा॒जासि॑ रयि॒ष्ठेत्या॑है॒ष्वे॑वैनां लो॒केषु॒ प्रति॑ ष्ठापयति दि॒वि ते॑ बृ॒हद्भा इत्या॑ह सुव॒र्ग ए॒वास्मै॑ लो॒के ज्योति॑र्दधाति॒ तन्तुं॑ त॒न्वन्रज॑सो भा॒नुमन्वि॒हीत्या॑हे॒माने॒वास्मै॑ लो॒काञ्ज्योति॑ष्मतः करोत्यनुल्ब॒णं व॑यत॒ जोगु॑वा॒मप॒ इति॑ [12]

%3.4.3.7
आ॒ह॒ यदे॒व य॒ज्ञ उ॒ल्बणं॑ क्रि॒यते॒ तस्यै॒वैषा शान्ति॒र्मनु॑र्भव ज॒नया॒ दैव्यं॒ जन॒मित्या॑ह मान॒व्यो॑ वै प्र॒जास्ता ए॒वाद्याः कुरुते॒ मन॑सो ह॒विर॒सीत्या॑ह स्व॒गाकृ॑त्यै॒ गात्रा॑णां ते गात्र॒भाजो॑ भूया॒स्मेत्या॑हा॒शिष॑मे॒वैतामा शास्ते॒ तस्यै॒ वा ए॒तस्या॒ एक॑मे॒वादे॑वयजनं॒ यदाल॑ब्धायाम॒भ्रः [13]

%3.4.3.8
भव॑ति॒ यदाल॑ब्धायाम॒भ्रः स्याद॒प्सु वा प्रवे॒शये॒थ्सर्वां वा॒ प्राश्नी॑या॒द्यद॒प्सु प्र॑वे॒शयेद्यज्ञवेश॒सं कु॑र्या॒थ्सर्वा॑मे॒व प्राश्नी॑यादिन्द्रि॒यमे॒वात्मन्ध॑त्ते॒ सा वा ए॒षा त्र॑या॒णामे॒वाव॑रुद्धा संवथ्सर॒सदः॑ सहस्रया॒जिनो॑ गृहमे॒धिन॒स्त ए॒वैतया॑ यजेर॒न्तेषा॑मे॒वैषाप्ता॥14॥

%3.4.4.0
{\scriptsize {उप॒ पञ्च॑विशतिश्च॥4॥}}

%3.4.4.1
चि॒त्तं च॒ चित्ति॒श्चाकू॑तं॒ चाकू॑तिश्च॒ विज्ञा॑तं च वि॒ज्ञानं॑ च॒ मन॑श्च॒ शक्व॑रीश्च॒ दर्\mbox{}श॑श्च पू॒र्णमा॑सश्च बृ॒हच्च॑ रथंत॒रं च॑ प्र॒जाप॑ति॒र्जया॒निन्द्रा॑य॒ वृष्णे॒ प्राय॑च्छदु॒ग्रः पृ॑त॒नाज्ये॑षु॒ तस्मै॒ विशः॒ सम॑नमन्त॒ सर्वाः॒ स उ॒ग्रः स हि हव्यो॑ ब॒भूव॑ देवासु॒राः संय॑त्ता आस॒न्थ्स इन्द्रः॑ प्र॒जाप॑ति॒मुपा॑धाव॒त्तस्मा॑ ए॒ताञ्जया॒न्प्राय॑च्छ॒त्तान॑जुहो॒त्ततो॒ वै दे॒वा असु॑रानजय॒न् य- दज॑य॒न्तज्जया॑नां जय॒त्व स्पर्ध॑मानेनै॒ते हो॑त॒व्या॑ जय॑त्ये॒व ताम्पृत॑नाम्॥[15]

%3.4.5.0
{\scriptsize {अ॒व॒रे॒ स॒प्तद॑श च॥5॥}}

%3.4.5.1
अ॒ग्निर्भू॒ताना॒मधि॑पतिः॒ स मा॑व॒त्विन्द्रो ज्ये॒ष्ठानां य॒मः पृ॑थि॒व्या वा॒युर॒न्तरि॑ख्षस्य॒ सूर्यो॑ दि॒वश्च॒न्द्रमा॒ नख्ष॑त्राणा॒म्बृह॒स्पति॒र्ब्रह्म॑णो मि॒त्रः स॒त्यानां॒ वरु॑णो॒ऽपा स॑मु॒द्रः स्रो॒त्याना॒मन्न॒ꣳ॒ साम्राज्याना॒मधि॑पति॒ तन्मा॑वतु॒ सोम॒ ओष॑धीना सवि॒ता प्र॑स॒वानाꣳ॑ रु॒द्रः प॑शू॒नां त्वष्टा॑ रू॒पाणां॒ विष्णुः॒ पर्व॑तानाम्म॒रुतो॑ ग॒णाना॒मधि॑पतय॒स्ते मा॑वन्तु॒ पित॑रः पितामहाः परेऽवरे॒ ततास्ततामहा इ॒ह मा॑वत। अ॒स्मिन्ब्रह्म॑न्न॒स्मिन्ख्ष॒त्रेऽस्यामा॒शिष्य॒स्याम्पु॑रो॒धाया॑- म॒स्मिन्कर्म॑न्न॒स्यां दे॒वहूत्याम्॥ [16]

%3.4.6.0
{\scriptsize {प्रा॒जा॒प॒त्याः सोऽष्टाद॑श च॥6॥}}

%3.4.6.1
दे॒वा वै यद्य॒ज्ञेऽकु॑र्वत॒ तदसु॑रा अकुर्वत॒ ते दे॒वा ए॒तान॑भ्याता॒नान॑पश्य॒न्तान॒भ्यात॑न्वत॒ यद्दे॒वानां॒ कर्मासी॒दार्ध्य॑त॒ तद्यदसु॑राणां॒ न तदार्ध्यत॒ येन॒ कर्म॒णेर्थ्से॒त्तत्र॑ होत॒व्या॑ ऋ॒ध्नोत्ये॒व तेन॒ कर्म॑णा॒ यद्विश्वे॑ दे॒वाः स॒मभ॑र॒न्तस्मा॑दभ्याता॒ना वैश्वदे॒वा यत्प्र॒जाप॑ति॒र्जया॒न्प्राय॑च्छ॒त्तस्मा॒ज्जयाः प्राजाप॒त्याः [17]

%3.4.6.2
यद्राष्ट्र॒भृद्भी॑ रा॒ष्ट्रमाद॑दत॒ तद्राष्ट्र॒भृताꣳ॑ राष्ट्रभृ॒त्त्वन्ते दे॒वा अ॑भ्याता॒नैरसु॑रान॒भ्यात॑न्वत॒ जयै॑रजयन्राष्ट्र॒भृद्भी॑ रा॒ष्ट्रमाद॑दत॒ यद्दे॒वा अ॑भ्याता॒नैरसु॑रान॒भ्यात॑न्वत॒ तद॑भ्याता॒नाना॑मभ्यातान॒त्वय्यँज्जयै॒रज॑य॒न्तज्जया॑नां जय॒त्वय्यँद्राष्ट्र॒भृद्भी॑ रा॒ष्ट्रमाद॑दत॒ तद्राष्ट्र॒भृताꣳ॑ राष्ट्रभृ॒त्त्वन्ततो॑ दे॒वा अभ॑व॒न्परासु॑रा॒ यो भ्रातृ॑व्यवा॒न्थ्स्याथ्स ए॒ताञ्जु॑हुयादभ्याता॒नैरे॒व भ्रातृ॑व्यान॒भ्यात॑नुते॒ जयैर्जयति राष्ट्र॒भृद्भी॑ रा॒ष्ट्रमा द॑त्ते॒ भव॑त्या॒त्मना॒ परास्य॒ भ्रातृ॑व्यो भवति॥ [18]

%3.4.7.0
{\scriptsize {मनो॑ऽमृड॒यष्षट्च॑त्वारिशच्च॥7॥}}

%3.4.7.1
ऋ॒ता॒षाडृ॒तधा॑मा॒ऽग्निर्ग॑न्ध॒र्वस्तस्यौष॑धयोऽप्स॒रस॒ ऊर्जो॒ नाम॒ स इ॒दम्ब्रह्म॑ ख्ष॒त्रम्पा॑तु॒ ता इ॒दम्ब्रह्म॑ ख्ष॒त्रम्पान्तु॒ तस्मै॒ स्वाहा॒ ताभ्यः॒ स्वाहा॑ सहि॒तो वि॒श्वसा॑मा॒ सूर्यो॑ गन्ध॒र्वस्तस्य॒ मरी॑चयोऽप्स॒रस॑ आ॒युवः॑ सुषु॒म्नः सूर्य॑रश्मिश्च॒न्द्रमा॑ गन्ध॒र्वस्तस्य॒ नख्ष॑त्राण्यप्स॒रसो॑ बे॒कुर॑यो भु॒ज्युः सु॑प॒र्णो य॒ज्ञो ग॑न्ध॒र्वस्तस्य॒ दख्षि॑णा अप्स॒रसः॑ स्त॒वाः प्र॒जाप॑तिर्वि॒श्वक॑र्मा॒ मनः॑ [19]

%3.4.7.2
ग॒न्ध॒र्वस्तस्य॑र्क्सा॒मान्य॑प्स॒रसो॒ वह्न॑य इषि॒रो वि॒श्वव्य॑चा॒ वातो॑ गन्ध॒र्वस्तस्यापोऽप्स॒रसो॑ मु॒दा भुव॑नस्य पते॒ यस्य॑ त उ॒परि॑ गृ॒हा इ॒ह च॑। स नो॑ रा॒स्वाज्या॑नि रा॒यस्पोषꣳ॑ सु॒वीर्यꣳ॑ संवथ्स॒रीणाꣳ॑ स्व॒स्तिम्। प॒र॒मे॒ष्ठ्यधि॑पति- र्मृ॒त्युर्ग॑न्ध॒र्वस्तस्य॒ विश्व॑मप्स॒रसो॒ भुवः॑ सुख्षि॒तिः सुभू॑तिर्भद्र॒कृथ्सुव॑र्वान्प॒र्जन्यो॑ गन्ध॒र्वस्तस्य॑ वि॒द्युतोऽप्स॒रसो॒ रुचो॑ दू॒रेहे॑तिरमृड॒यः [20]

%3.4.7.3
मृ॒त्युर्ग॑न्ध॒र्वस्तस्य॑ प्र॒जा अ॑प्स॒रसो॑ भी॒रुव॒श्चारुः॑ कृपणका॒शी कामो॑ गन्ध॒र्वस्तस्या॒धयोऽप्स॒रसः॑ शो॒चय॑न्ती॒र्नाम॒ स इ॒दम्ब्रह्म॑ ख्ष॒त्रम्पा॑तु॒ ता इ॒दम्ब्रह्म॑ ख्ष॒त्रम्पान्तु॒ तस्मै॒ स्वाहा॒ ताभ्यः॒ स्वाहा॒ स नो॑ भुवनस्य पते॒ यस्य॑ त उ॒परि॑ गृ॒हा इ॒ह च॑। उ॒रु ब्रह्म॑णे॒ऽस्मै ख्ष॒त्राय॒ महि॒ शर्म॑ यच्छ॥ [21]

%3.4.8.0
{\scriptsize {ग्रा॒मी यु॑नक्ती॒ध्मः स्व ए॒वैना॑न॒न्नाद्यं॑ यच्छ॒न्त्येका॒न्नप॑ञ्चा॒शच्च॑॥8॥}}

%3.4.8.1
रा॒ष्ट्रका॑माय होत॒व्या॑ रा॒ष्ट्रं वै राष्ट्र॒भृतो॑ रा॒ष्ट्रेणै॒वास्मै॑ रा॒ष्ट्रमव॑ रुन्द्धे रा॒ष्ट्रमे॒व भ॑वत्या॒त्मने॑ होत॒व्या॑ रा॒ष्ट्रं वै राष्ट्र॒भृतो॑ रा॒ष्ट्रम्प्र॒जा रा॒ष्ट्रम्प॒शवो॑ रा॒ष्ट्रं यच्छ्रेष्ठो॒ भव॑ति रा॒ष्ट्रेणै॒व रा॒ष्ट्रमव॑ रुन्द्धे॒ वसि॑ष्ठः समा॒नानाम्भवति॒ ग्राम॑कामाय होत॒व्या॑ रा॒ष्ट्रं वै राष्ट्र॒भृतो॑ रा॒ष्ट्र स॑जा॒ता रा॒ष्ट्रेणै॒वास्मै॑ रा॒ष्ट्र स॑जा॒तानव॑ रुन्द्धे ग्रा॒मी [22]

%3.4.8.2
ए॒व भ॑वत्यधि॒देव॑ने जुहोत्यधि॒देव॑न ए॒वास्मै॑ सजा॒तानव॑ रुन्द्धे॒ त ए॑न॒मव॑रुद्धा॒ उप॑ तिष्ठन्ते रथमु॒ख ओज॑स्कामस्य होत॒व्या॑ ओजो॒ वै राष्ट्र॒भृत॒ ओजो॒ रथ॒ ओज॑सै॒वास्मा॒ ओजोऽव॑ रुन्द्ध ओज॒स्व्ये॑व भ॑वति॒ यो रा॒ष्ट्रादप॑भूतः॒ स्यात्तस्मै॑ होत॒व्या॑ याव॑न्तोऽस्य॒ रथाः॒ स्युस्तान्ब्रू॑याद्यु॒ङ्ग्ध्वमिति॑ रा॒ष्ट्रमे॒वास्मै॑ युनक्ति [23]

%3.4.8.3
आहु॑तयो॒ वा ए॒तस्याकॢ॑प्ता॒ यस्य॑ रा॒ष्ट्रं न कल्प॑ते स्वर॒थस्य॒ दख्षि॑णं च॒क्रम्प्र॒वृह्य॑ ना॒डीम॒भि जु॑हुया॒दाहु॑तीरे॒वास्य॑ कल्पयति॒ ता अ॑स्य॒ कल्प॑माना रा॒ष्ट्रमनु॑ कल्पते संग्रा॒मे संय॑त्ते होत॒व्या॑ रा॒ष्ट्रं वै राष्ट्र॒भृतो॑ रा॒ष्ट्रे खलु॒ वा ए॒ते व्याय॑च्छन्ते॒ ये सं॑ग्रा॒म सं॒यन्ति॒ यस्य॒ पूर्व॑स्य॒ जुह्व॑ति॒ स ए॒व भ॑वति॒ जय॑ति॒ तं सं॑ग्रा॒मं मान्धु॒क इ॒ध्मः [24]

%3.4.8.4
भ॒व॒त्यङ्गा॑रा ए॒व प्र॑ति॒वेष्ट॑माना अ॒मित्रा॑णामस्य॒ सेना॒म्प्रति॑ वेष्टयन्ति॒ य उ॒न्माद्ये॒त्तस्मै॑ होत॒व्या॑ गन्धर्वाप्स॒रसो॒ वा ए॒तमुन्मा॑दयन्ति॒ य उ॒न्माद्य॑त्ये॒ते खलु॒ वै ग॑न्धर्वाप्स॒रसो॒ यद्राष्ट्र॒भृत॒स्तस्मै॒ स्वाहा॒ ताभ्यः॒ स्वाहेति॑ जुहोति॒ तेनै॒वैनाञ्छमयति॒ नैय॑ग्रोध॒ औदु॑म्बर॒ आश्व॑त्थः॒ प्लाख्ष॒ इती॒ध्मो भ॑वत्ये॒ते वै ग॑न्धर्वाप्स॒रसां गृ॒हाः स्व ए॒वैनान्॑ [25]

%3.4.8.5
आ॒यत॑ने शमयत्यभि॒चर॑ता प्रतिलो॒म हो॑त॒व्याः प्रा॒णाने॒वास्य॑ प्र॒तीचः॒ प्रति॑ यौति॒ तं ततो॒ येन॒ केन॑ च स्तृणुते॒ स्वकृ॑त॒ इरि॑णे जुहोति प्रद॒रे वै॒तद्वा अ॒स्यै निर्\mbox{}ऋ॑तिगृहीतं॒ निर्\mbox{}ऋ॑तिगृहीत ए॒वैनं॒ निर्\mbox{}ऋ॑त्या ग्राहयति॒ यद्वा॒चः क्रू॒रन्तेन॒ वष॑ट्करोति वा॒च ए॒वैनं॑ क्रू॒रेण॒ प्र वृ॑श्चति ता॒जगार्ति॒मार्च्छ॑ति॒ यस्य॑ का॒मये॑ता॒न्नाद्यम् [26]

%3.4.8.6
आ द॑दी॒येति॒ तस्य॑ स॒भाया॑मुत्ता॒नो नि॒पद्य॒ भुव॑नस्य पत॒ इति॒ तृणा॑नि॒ सं गृ॑ह्णीयात्प्र॒जाप॑तिर्वै भुव॑नस्य॒ पतिः॑ प्र॒जाप॑तिनै॒वास्या॒न्नाद्य॒मा द॑त्त इ॒दम॒हम॒मुष्या॑मुष्याय॒णस्या॒न्नाद्यꣳ॑ हरा॒मीत्या॑हा॒न्नाद्य॑मे॒वास्य॑ हरति ष॒ड्भिर्\mbox{}ह॑रति॒ षड्वा ऋ॒तवः॑ प्र॒जाप॑तिनै॒वास्या॒न्नाद्य॑मा॒दाय॒र्तवोऽस्मा॒ अनु॒ प्र य॑च्छन्ति [27]

%3.4.8.7
यो ज्ये॒ष्ठब॑न्धु॒रप॑भूतः॒ स्यात्त स्थले॑ऽव॒साय्य॑ ब्रह्मौद॒नं चतुः॑शरावम्प॒क्त्वा तस्मै॑ होत॒व्या॑ वर्ष्म॒ वै राष्ट्र॒भृतो॒ वर्ष्म॒ स्थलं॒ वर्ष्म॑णै॒वैनं॒ वर्ष्म॑ समा॒नानां गमयति॒ चतुः॑शरावो भवति दि॒ख्ष्वे॑व प्रति॑ तिष्ठति ख्षी॒रे भ॑वति॒ रुच॑मे॒वास्मि॑- न्दधा॒त्युद्ध॑रति शृत॒त्वाय॑ स॒र्पिष्वान्भवति मेध्य॒त्वाय॑ च॒त्वार॑ आर्\mbox{}षे॒याः प्राश्न॑न्ति दि॒शामे॒व ज्योति॑षि जुहोति॥ [28]

%3.4.9.0
{\scriptsize {प॒शुका॑म॒श्छन्दाꣳ॑सि॒ वै देवि॑का॒श्छन्दाꣳ॑सि॒ ग्राम॑ङ्कल्पयत्ये॒ता ए॒व निरु॑त्त॒मन्धा॒तारं॑ करोति मे॒धा न॑मत्ये॒ता ए॒व निर्व॑पेद॒ष्टौ द॑हन्ति॒ नव॑ च॥9॥ देविकाः प्रजाकामो मिथुनी पशुकाम}}

%3.4.9.1
देवि॑का॒ निर्व॑पेत्प्र॒जाका॑म॒श्छन्दाꣳ॑सि॒ वै देवि॑का॒श्छन्दाꣳ॑सीव॒ खलु॒ वै प्र॒जाश्छन्दो॑भिरे॒वास्मै प्र॒जाः प्र ज॑नयति प्रथ॒मं धा॒तारं॑ करोति मिथु॒नी ए॒व तेन॑ करो॒त्यन्वे॒वास्मा॒ अनु॑मतिर्मन्यते रा॒ते रा॒का प्र सि॑नीवा॒ली ज॑नयति प्र॒जास्वे॒व प्रजा॑तासु कु॒ह्वा॑ वाचं॑ दधात्ये॒ता ए॒व निर्व॑पेत्प॒शुका॑म॒श्छन्दाꣳ॑सि॒ वै देवि॑का॒श्छन्दाꣳ॑सि [29]

%3.4.9.2
इ॒व॒ खलु॒ वै प॒शव॒श्छन्दो॑भिरे॒वास्मै॑ प॒शून्प्र ज॑नयति प्रथ॒मं धा॒तारं॑ करोति॒ प्रैव तेन॑ वापय॒त्यन्वे॒वास्मा॒ अनु॑मतिर्मन्यते रा॒ते रा॒का प्र सि॑नीवा॒ली ज॑नयति प॒शूने॒व प्रजा॑तान्कु॒ह्वा प्रति॑ ष्ठापयत्ये॒ता ए॒व निर्व॑पे॒द्ग्राम॑काम॒श्छन्दाꣳ॑सि॒ वै देवि॑का॒श्छन्दाꣳ॑सीव॒ खलु॒ वै ग्राम॒श्छन्दो॑भिरे॒वास्मै॒ ग्रामम् [30]

%3.4.9.3
अव॑ रुन्द्धे मध्य॒तो धा॒तारं॑ करोति मध्य॒त ए॒वैनं॒ ग्राम॑स्य दधात्ये॒ता ए॒व निर्व॑पे॒ज्ज्योगा॑मयावी॒ छन्दाꣳ॑सि॒ वै देवि॑का॒श्छन्दाꣳ॑सि॒ खलु॒ वा ए॒तम॒भि म॑न्यन्ते॒ यस्य॒ ज्योगा॒मय॑ति॒ छन्दो॑भिरे॒वैन॑मग॒दं क॑रोति मध्य॒तो धा॒तारं॑ करोति मध्य॒तो वा ए॒तस्याकॢ॑प्तं॒ यस्य॒ ज्योगा॒मय॑ति मध्य॒त ए॒वास्य॒ तेन॑ कल्पयत्ये॒ता ए॒व निः [31]

%3.4.9.4
व॒पे॒द्यं य॒ज्ञो नोप॒नमे॒च्छन्दाꣳ॑सि॒ वै देवि॑का॒श्छन्दाꣳ॑सि॒ खलु॒ वा ए॒तं नोप॑ नमन्ति॒ यं य॒ज्ञो नोप॒नम॑ति प्रथ॒मं धा॒तारं॑ करोति मुख॒त ए॒वास्मै॒ छन्दाꣳ॑सि दधा॒त्युपै॑नं य॒ज्ञो न॑मत्ये॒ता ए॒व निर्व॑पेदीजा॒नश्छन्दाꣳ॑सि॒ वै देवि॑का या॒तया॑मानीव॒ खलु॒ वा ए॒तस्य॒ छन्दाꣳ॑सि॒ य ई॑जा॒न उ॑त्त॒मं धा॒तारं॑ करोति [32]

%3.4.9.5
उ॒परि॑ष्टादे॒वास्मै॒ छन्दा॒ꣳ॒स्यया॑तयामा॒न्यव॑ रुन्द्ध॒ उपै॑न॒मुत्त॑रो य॒ज्ञो न॑मत्ये॒ता ए॒व निर्व॑पे॒द्यम्मे॒धा नोप॒नमे॒च्छन्दाꣳ॑सि॒ वै देवि॑का॒श्छन्दाꣳ॑सि॒ खलु॒ वा ए॒तं नोप॑ नमन्ति॒ यम्मे॒धा नोप॒नम॑ति प्रथ॒मं धा॒तारं॑ करोति मुख॒त ए॒वास्मै॒ छन्दाꣳ॑सि दधा॒त्युपै॑नम्मे॒धा न॑मत्ये॒ता ए॒व निर्व॑पेत् [33]

%3.4.9.6
रुक्का॑म॒श्छन्दाꣳ॑सि॒ वै देवि॑का॒श्छन्दाꣳ॑सीव॒ खलु॒ वै रुक्छन्दो॑भिरे॒वास्मि॒न्रुचं॑ दधाति ख्षी॒रे भ॑वन्ति॒ रुच॑मे॒वास्मि॑न्दधति मध्य॒तो धा॒तारं॑ करोति मध्य॒त ए॒वैनꣳ॑ रु॒चो द॑धाति गाय॒त्री वा अनु॑मतिस्त्रि॒ष्टुग्रा॒का जग॑ती सिनीवाल्य॑नु॒ष्टुप्कु॒हूर्धा॒ता व॑षट्का॒रः पूर्वप॒ख्षो रा॒काप॑रप॒ख्षः कु॒हूर॑मावा॒स्या॑ सिनीवा॒ली पौर्णमा॒स्यनु॑मतिश्च॒न्द्रमा॑ धा॒ताऽष्टौ [34]

%3.4.9.7
वस॑वो॒ऽष्टाख्ष॑रा गाय॒त्र्येका॑दश रु॒द्रा एका॑दशाख्षरा त्रि॒ष्टुब्द्वाद॑शादि॒त्या द्वाद॑शाख्षरा॒ जग॑ती प्र॒जाप॑तिरनु॒ष्टुब्धा॒ता व॑षट्का॒र ए॒तद्वै देवि॑काः॒ सर्वा॑णि च॒ छन्दाꣳ॑सि॒ सर्वाश्च दे॒वता॑ वषट्का॒रस्ता यथ्स॒ह सर्वा॑ नि॒र्वपे॑दीश्व॒रा ए॑नम्प्र॒दहो॒ द्वे प्र॑थ॒मे नि॒रुप्य॑ धा॒तुस्तृ॒तीयं॒ निर्व॑पे॒त्तथो॑ ए॒वोत्त॑रे॒ निर्व॑पे॒त्तथै॑नं॒ न प्र द॑ह॒न्त्यथो॒ यस्मै॒ कामा॑य निरु॒प्यन्ते॒ तमे॒वाभि॒रुपाप्नोति॥ [35]

%3.4.10.0
{\scriptsize {ध॒त्ते॒ऽर्वा॒चीनꣵ॑ स्याथ्स॒मारो॑हयति॒ पञ्च॑चत्वारिशच्च॥10॥}}

%3.4.10.1
वास्तोष्पते॒ प्रति॑ जानीह्य॒स्मान्थ्स्वा॑वे॒शो अ॑नमी॒वो भ॑वा नः। यत्त्वेम॑हे॒ प्रति॒ तन्नो॑ जुषस्व॒ शं न॑ एधि द्वि॒पदे॒ शं चतु॑ष्पदे। वास्तोष्पते श॒ग्मया॑ स॒ꣳ॒सदा॑ ते सख्षी॒महि॑ र॒ण्वया॑ गातु॒मत्या। आवः॒ ख्षेम॑ उ॒त योगे॒ वरं॑ नो यू॒यम्पा॑त स्व॒स्तिभिः॒ सदा॑ नः। यथ्सा॒यम्प्रा॑तरग्निहो॒त्रं जु॒होत्या॑हुतीष्ट॒का ए॒व ता उप॑ धत्ते [36]

%3.4.10.2
यज॑मानोऽहोरा॒त्राणि॒ वा ए॒तस्येष्ट॑का॒ य आहि॑ताग्नि॒र्यथ्सा॒यम्प्रा॑तर्जु॒होत्य॑होरा॒त्राण्ये॒वाप्त्वेष्ट॑काः कृ॒त्वोप॑ धत्ते॒ दश॑ समा॒नत्र॑ जुहोति॒ दशाख्षरा वि॒राड्वि॒राज॑मे॒वाप्त्वेष्ट॑कां कृ॒त्वोप॑ ध॒त्तेऽथो॑ वि॒राज्ये॒व य॒ज्ञमाप्नोति॒ चित्य॑श्चित्योऽस्य भवति॒ तस्मा॒द्यत्र॒ दशो॑षि॒त्वा प्र॒याति॒ तद्य॑ज्ञवा॒स्त्ववास्त्वे॒व तद्यत्ततोऽर्वा॒चीनम् [37]

%3.4.10.3
रु॒द्रः खलु॒ वै वास्तोष्प॒तिर्यदहु॑त्वा वास्तोष्प॒तीय॑म्प्रया॒याद्रु॒द्र ए॑नम्भू॒त्वाग्निर॑नू॒त्थाय॑ हन्याद्वास्तोष्प॒तीयं॑ जुहोति भाग॒धेये॑नै॒वैनꣳ॑ शमयति॒ नार्ति॒मार्च्छ॑ति॒ यज॑मानो॒ यद्यु॒क्ते जु॑हु॒याद्यथा॒ प्रया॑ते॒ वास्ता॒वाहु॑तिं जु॒होति॑ ता॒दृगे॒व तद्यदयु॑क्ते जुहु॒याद्यथा॒ ख्षेम॒ आहु॑तिं जु॒होति॑ ता॒दृगे॒व तदहु॑तमस्य वास्तोष्प॒तीयꣵ॑ स्यात् [38]

%3.4.10.4
दख्षि॑णो यु॒क्तो भव॑ति स॒व्योऽयु॒क्तोऽथ॑ वास्तोष्प॒तीयं॑ जुहोत्यु॒भय॑मे॒वाक॒रप॑रिवर्गमे॒वैनꣳ॑ शमयति॒ यदेक॑या जुहु॒याद्द॑र्विहो॒मं कु॑र्यात्पुरोनुवा॒क्या॑म॒नूच्य॑ या॒ज्य॑या जुहोति सदेव॒त्वाय॒ यद्धु॒त आ॑द॒ध्याद्रु॒द्रं गृ॒हान॒न्वारो॑हये॒द्यद॑व॒- ख्षाणा॒न्यस॑म्प्रख्षाप्य प्रया॒याद्यथा॑ यज्ञवेश॒सं वा॒दह॑नं वा ता॒दृगे॒व तद॒यं ते॒ योनि॑र्\mbox{}ऋ॒त्विय॒ इत्य॒रण्योः स॒मारो॑हयति [39]

%3.4.10.5
ए॒ष वा अ॒ग्नेर्योनिः॒ स्व ए॒वैनं॒ योनौ॑ स॒मारो॑हय॒त्यथो॒ खल्वा॑हु॒र्यद॒रण्योः स॒मारू॑ढो॒ नश्ये॒दुद॑स्या॒ग्निः सी॑देत्पुनरा॒धेयः॑ स्या॒दिति॒ या ते॑ अग्ने य॒ज्ञिया॑ त॒नूस्तयेह्या रो॒हेत्या॒त्मन्थ्स॒मारो॑हयते॒ यज॑मानो॒ वा अ॒ग्नेर्योनिः॒ स्वाया॑मे॒वैनं॒ योन्याꣳ॑ स॒मारो॑हयते॥ [40]

%3.4.11.0
{\scriptsize {सोमो॒ गोषु॒ मा र॒यिं मन्त्रो॒ यच्छि॑थि॒रा स॒प्त च॑॥11॥}}

%3.4.11.1
त्वम॑ग्ने बृ॒हद्वयो॒ दधा॑सि देव दा॒शुषे। क॒विर्गृ॒हप॑ति॒र्युवा॥ ह॒व्य॒वाड॒ग्निर॒जरः॑ पि॒ता नो॑ वि॒भुर्वि॒भावा॑ सु॒दृशी॑को अ॒स्मे। सु॒गा॒र्\mbox{}ह॒प॒त्याः समिषो॑ दिदीह्यस्म॒द्रिय॒क्सम्मि॑मीहि॒ श्रवाꣳ॑सि। त्वं च॑ सोम नो॒ वशो॑ जी॒वातुं॒ न म॑रामहे। प्रि॒यस्तोत्रो॒ वन॒स्पतिः॑। ब्र॒ह्मा दे॒वानाम्पद॒वीः क॑वी॒नामृषि॒र्विप्रा॑णाम्महि॒षो मृ॒गाणाम्। श्ये॒नो गृ॑ध्राणा॒ꣳ॒ स्वधि॑ति॒र्वना॑ना॒ꣳ॒ सोमः॑ [41]

%3.4.11.2
प॒वित्र॒मत्ये॑ति॒ रेभन्न्॑। आ वि॒श्वदे॑व॒ꣳ॒ सत्प॑ति सू॒क्तैर॒द्या वृ॑णीमहे। स॒त्यस॑व सवि॒तारम्॥ आ स॒त्येन॒ रज॑सा॒ वर्त॑मानो निवे॒शय॑न्न॒मृत॒म्मर्त्यं॑ च। हि॒र॒ण्यये॑न सवि॒ता रथे॒ना दे॒वो या॑ति॒ भुव॑ना वि॒पश्यन्न्॑। यथा॑ नो॒ अदि॑तिः॒ कर॒त्पश्वे॒ नृभ्यो॒ यथा॒ गवे। यथा॑ तो॒काय॑ रु॒द्रियम्। मा न॑स्तो॒के तन॑ये॒ मा न॒ आयु॑षि॒ मा नो॒ गोषु॒ मा [42]

%3.4.11.3
नो॒ अश्वे॑षु रीरिषः। वी॒रान्मा नो॑ रुद्र भामि॒तो व॑धीर्\mbox{}ह॒विष्म॑न्तो॒ नम॑सा विधेम ते। उ॒द॒प्रुतो॒ न वयो॒ रख्ष॑माणा॒ वाव॑दतो अ॒भ्रिय॑स्येव॒ घोषाः। गि॒रि॒भ्रजो॒ नोर्मयो॒ मद॑न्तो॒ बृह॒स्पति॑म॒भ्य॑र्का अ॑नावन्न्। ह॒ꣳ॒सैरि॑व॒ सखि॑भि॒र्वाव॑दद्भिरश्म॒न्मया॑नि॒ नह॑ना॒ व्यस्यन्न्॑। बृह॒स्पति॑रभि॒ कनि॑क्रद॒द्गा उ॒त प्रास्तौ॒दुच्च॑ वि॒द्वा अ॑गायत्। एन्द्र॑ सान॒सि र॒यिम् [43]

%3.4.11.4
स॒जित्वा॑न सदा॒सहम्। वर्\mbox{}षि॑ष्ठमू॒तये॑ भर। प्र स॑साहिषे पुरुहूत॒ शत्रू॒ञ्ज्येष्ठ॑स्ते॒ शुष्म॑ इ॒ह रा॒तिर॑स्तु। इन्द्रा भ॑र॒ दख्षि॑णेना॒ वसू॑नि॒ पतिः॒ सिन्धू॑नामसि रे॒वती॑नाम्। त्व सु॒तस्य॑ पी॒तये॑ स॒द्यो वृ॒द्धो अ॑जायथाः। इन्द्र॒ ज्यैष्ठ्या॑य सुक्रतो। भु॒वस्त्वमि॑न्द्र॒ ब्रह्म॑णा म॒हान्भुवो॒ विश्वे॑षु॒ सव॑नेषु य॒ज्ञियः॑। भुवो॒ नॄश्च्यौ॒त्नो विश्व॑स्मि॒न्भरे॒ ज्येष्ठ॑श्च॒ मन्त्रः॑ [44]

%3.4.11.5
वि॒श्व॒च॒र्\mbox{}ष॒णे॒। मि॒त्रस्य॑ चर्\mbox{}षणी॒धृतः॒ श्रवो॑ दे॒वस्य॑ सान॒सिम्। स॒त्यं चि॒त्रश्र॑वस्तमम्। मि॒त्रो जना॑न् यातयति प्रजा॒नन्मि॒त्रो दा॑धार पृथि॒वीमु॒त द्याम्। मि॒त्रः कृ॒ष्टीरनि॑मिषा॒भि च॑ष्टे स॒त्याय॑ ह॒व्यं घृ॒तव॑द्विधेम। प्र स मि॑त्र॒ मर्तो॑ अस्तु॒ प्रय॑स्वा॒न् यस्त॑ आदित्य॒ शिख्ष॑ति व्र॒तेन॑। न ह॑न्यते॒ न जी॑यते॒ त्वोतो॒ नैन॒महो॑ अश्नो॒त्यन्ति॑तो॒ न दू॒रात्। यत् [45]

%3.4.11.6
चि॒द्धि ते॒ विशो॑ यथा॒ प्र दे॑व वरुण व्र॒तम्। मि॒नी॒मसि॒ द्यवि॑द्यवि। यत्किं चे॒दं व॑रुण॒ दैव्ये॒ जने॑ऽभिद्रो॒ह- म्म॑नु॒ष्याश्चरा॑मसि। अचि॑त्ती॒ यत्तव॒ धर्मा॑ युयोपि॒म मा न॒स्तस्मा॒देन॑सो देव रीरिषः। कि॒त॒वासो॒ यद्रि॑रि॒पुर्न दी॒वि यद्वा॑ घा स॒त्यमु॒त यन्न वि॒द्म। सर्वा॒ ता वि ष्य॑ शिथि॒रेव॑ दे॒वाथा॑ ते स्याम वरुण प्रि॒यासः॑॥ [46]

%3.5.0.0
{\scriptsize {पू॒र्णर्\mbox{}ष॑यो॒ऽग्निना॒ ये दे॒वास्सूर्यो॑ मा॒ सन्त्वा॑ नह्यामि वषट्का॒रस्स ख॑दि॒र उ॑पया॒मगृ॑हीतोऽसि॒ याव्वैँ त्वे क्रतु॒म्प्र दे॒वमेका॑दश॥11॥ पू॒र्णा स॑ह॒जान्तवाग्ने प्रा॒णैरे॒व षट्त्रिꣳ॑शत्॥36॥ पू॒र्णा सन्ति॑ दे॒वाः॥ हरिः॑ ओम्॥ श्रीकृष्णार्पणमस्तु॥}}

%3.5.0.0
{\scriptsize {॥ हरिः ओम्॥॥ तैत्तिरीयसंहिता तृतीयकाण्डे पञ्चमः प्रश्नः॥}}


%%% END PRASHNA
