\sect{प्रथमः प्रश्नः}\setcounter{anuvakam}{0}
\dnsub{तैत्तिरीयसंहितायां तृतीयकाण्डे प्रथमः प्रश्नः}
%3.1.1.0
%3.1.1.1
प्र॒जा\-प॑तिरकामयत प्र॒जाः सृ॑जे॒येति॒ स तपो॑\-ऽतप्यत॒ स स॒र्पान॑सृजत॒ सो॑\-ऽकामयत प्र॒जाः सृ॑जे॒येति॒ स द्वि॒तीय॑म\-तप्यत॒ स वयाꣴ॑स्यसृजत॒ सो॑\-ऽकामयत प्र॒जाः सृ॑जे॒येति॒ स तृ॒तीय॑मतप्यत॒ स ए॒तं दी᳚क्षितवा॒दम॑पश्य॒त्तम॑वद॒त्ततो॒ वै स प्र॒जा अ॑सृजत॒ यत्तप॑स्त॒प्त्वा दी᳚क्षितवा॒दं वद॑ति प्र॒जा ए॒व तद्यज॑मानः~(१)

%3.1.1.2
सृ॒ज॒ते॒ यद्वै दी᳚क्षि॒तो॑\-ऽमे॒ध्यम्पश्य॒त्यपा᳚स्माद्दी॒क्षा क्रा॑मति॒ नील॑मस्य॒ हरो॒ व्ये᳚त्यब॑द्ध॒म्मनो॑ द॒रिद्रं॒ चक्षुः॒ सूर्यो॒ ज्योति॑षा॒ꣴ॒ श्रेष्ठो॒ दीक्षे॒ मा मा॑ हासी॒रित्या॑ह॒ नास्मा᳚द्दी॒क्षाप॑ क्रामति॒ नास्य॒ नीलं॒ न हरो॒ व्ये॑ति॒ यद्वै दी᳚क्षि॒तम॑भि॒वर्\mbox{}ष॑ति दि॒व्या आपो\-ऽशा᳚न्ता॒ ओजो॒ बलं॑ दी॒क्षाम्~(२)

%3.1.1.3
तपो᳚\-ऽस्य॒ निर्घ्न॑न्त्युन्द॒तीर्बलं॑ ध॒त्तौजो॑ धत्त॒ बलं॑ धत्त॒ मा मे॑ दी॒क्षां मा तपो॒ निर्व॑धि॒ष्टेत्या॑है॒तदे॒व सर्व॑मा॒त्मन्ध॑त्ते॒ नास्यौजो॒ बलं॒ न दी॒क्षां न तपो॒ निर्घ्न॑न्त्य॒ग्निर्वै दी᳚क्षि॒तस्य॑ दे॒वता॒ सो᳚\-ऽस्मादे॒तर्\mbox{}हि॑ ति॒र इ॑व॒ यर्\mbox{}हि॒ याति॒ तमी᳚श्व॒रꣳ रक्षाꣳ॑सि॒ हन्तोः᳚~(३)

%3.1.1.4
भ॒द्राद॒भि श्रेयः॒ प्रेहि॒ बृह॒स्पतिः॑ पुरए॒ता ते॑ अ॒स्त्वित्या॑ह॒ ब्रह्म॒ वै दे॒वानां॒ बृह॒स्पति॒स्तमे॒वान्वार॑भते॒ स ए॑न॒ꣳ॒ सम्पा॑रय॒त्येदम॑गन्म देव॒यज॑नं पृथि॒व्या इत्या॑ह देव॒यज॑न॒ꣴ॒ ह्ये॑ष पृ॑थि॒व्या आ॒गच्छ॑ति॒ यो यज॑ते॒ विश्वे॑ दे॒वा यदजु॑षन्त॒ पूर्व॒ इत्या॑ह॒ विश्वे॒ ह्ये॑तद्दे॒वा जो॒षय॑न्ते॒ यद्ब्रा᳚ह्म॒णा ऋ॑ख्सा॒माभ्यां॒ यजु॑षा स॒न्तर॑न्त॒ इत्या॑हर्ख्सा॒माभ्या॒ꣳ॒ ह्ये॑ष यजु॑षा स॒न्तर॑ति॒ यो यज॑ते रा॒यस्पोषे॑ण॒ समि॒षा म॑दे॒मेत्या॑हा॒ऽ॒ऽ॒शिष॑मे॒वैतामा शा᳚स्ते॥~(४)

%3.1.2.0
{\anuvakamend[{यज॑मानो दी॒क्षाꣳ हन्तो᳚र्ब्राह्म॒णाश्चतु॑र्विꣳशतिश्च}]}%~(१)

%3.1.2.1
ए॒ष ते॑ गाय॒त्रो भा॒ग इति॑ मे॒ सोमा॑य ब्रूतादे॒ष ते॒ँतँरैष्टु॑भो॒ जाग॑तो भा॒ग इति॑ मे॒ सोमा॑य ब्रूताच्छन्दो॒माना॒ꣳ॒ साम्रा᳚ज्यं ग॒च्छेति॑ मे॒ सोमा॑य ब्रूता॒द्यो वै सोम॒ꣳ॒ राजा॑न॒ꣳ॒ साम्रा᳚ज्यं लो॒कं ग॑मयि॒त्वा क्री॒णाति॒ गच्छ॑ति॒ स्वाना॒ꣳ॒ साम्रा᳚ज्यं॒ छन्दाꣳ॑सि॒ खलु॒ वै सोम॑स्य॒ राज्ञः॒ साम्रा᳚ज्यो लो॒कः पु॒रस्ता॒थ्सोम॑स्य क्र॒यादे॒वम॒भि म॑न्त्रयेत॒ साम्रा᳚ज्यमे॒व~(५)

%3.1.2.2
ए॒नं॒ लो॒कं ग॑मयि॒त्वा क्री॑णाति॒ गच्छ॑ति॒ स्वाना॒ꣳ॒ साम्रा᳚ज्यं॒ यो वै ता॑नून॒प्त्रस्य॑ प्रति॒ष्ठां वेद॒ प्रत्ये॒व ति॑ष्ठति ब्रह्मवा॒दिनो॑ वदन्ति॒ न प्रा॒श्नन्ति॒ न जु॑ह्व॒त्यथ॒ क्व॑ तानून॒प्त्रं प्रति॑ तिष्ठ॒तीति॑ प्र॒जाप॑तौ॒ मन॒सीति॑ ब्रूया॒त्त्रिरव॑ जिघ्रेत्प्र॒जाप॑तौ त्वा॒ मन॑सि जुहो॒मीत्ये॒षा वै ता॑नून॒प्त्रस्य॑ प्रति॒ष्ठा य ए॒वं वेद॒ प्रत्ये॒व ति॑ष्ठति॒ यः~(६)

%3.1.2.3
वा अ॑ध्व॒र्योः प्र॑ति॒ष्ठां वेद॒ प्रत्ये॒व ति॑ष्ठति॒ यतो॒ मन्ये॒तान॑भिक्रम्य होष्या॒मीति॒ तत्तिष्ठ॒न्ना श्रा॑वयेदे॒षा वा अ॑ध्व॒र्योः प्र॑ति॒ष्ठा य ए॒वं वेद॒ प्रत्ये॒व ति॑ष्ठति॒ यद॑भि॒क्रम्य॑ जुहु॒यात्प्र॑ति॒ष्ठाया॑ इया॒त्तस्मा᳚थ्समा॒नत्र॒ तिष्ठ॑ता होत॒व्यं॑ प्रति॑ष्ठित्यै॒ यो वा अ॑ध्व॒र्योः स्वं वेद॒ स्ववा॑ने॒व भ॑वति॒ स्रुग्वा अ॑स्य॒ स्वं वा॑य॒व्य॑मस्य~(७)

%3.1.2.4
स्वं च॑म॒सो᳚\-ऽस्य॒ स्वं यद्वा॑य॒व्यं॑ वा चम॒सं वा\-ऽन॑न्वारभ्याश्रा॒वये॒थ्स्वादि॑या॒त्तस्मा॑दन्वा॒रभ्या॒श्राव्य॒ꣴ॒ स्वादे॒व नैति॒ यो वै सोम॒मप्र॑तिष्ठाप्य स्तो॒त्रमु॑पाक॒रोत्यप्र॑तिष्ठितः॒ सोमो॒ भव॒त्यप्र॑तिष्ठितः॒ स्तोमो\-ऽप्र॑तिष्ठितान्यु॒क्थान्यप्र॑तिष्ठितो॒ यज॑मा॒नो\-ऽप्र॑तिष्ठितो\-ऽध्व॒र्युर्वा॑य॒व्यं॑ वै सोम॑स्य प्रति॒ष्ठा च॑म॒सो᳚\-ऽस्य प्रति॒ष्ठा सोमः॒ स्तोम॑स्य॒ स्तोम॑ उ॒क्थानां॒ ग्रहं॑ वा गृही॒त्वा च॑म॒सं वो॒न्नीय॑ स्तो॒त्रमु॒पाकु॑र्या॒त्प्रत्ये॒व सोमꣴ॑ स्था॒पय॑ति॒ प्रति॒ स्तोमं॒ प्रत्यु॒क्थानि॒ प्रति॒ यज॑मान॒स्तिष्ठ॑ति॒ प्रत्य॑ध्व॒र्युः॥~(८)

%3.1.3.0
{\anuvakamend[{ए॒व ति॑ष्ठति॒ यो वा॑य॒व्य॑मस्य॒ ग्रहं॒ वैका॒न्नविꣳ॑श॒तिश्च॑}]}%~(२)

%3.1.3.1
य॒ज्ञं वा ए॒तथ्सम्भ॑रन्ति॒ यथ्सो॑म॒क्रय॑ण्यै प॒दं य॑ज्ञमु॒खꣳ ह॑वि॒र्धाने॒ यर्\mbox{}हि॑ हवि॒र्धाने॒ प्राची᳚ प्रव॒र्तये॑यु॒स्तर्\mbox{}हि॒ तेनाक्ष॒मुपा᳚ञ्ज्याद्यज्ञमु॒ख ए॒व य॒ज्ञमनु॒ सं त॑नोति॒ प्राञ्च॑म॒ग्निम्प्र ह॑र॒न्त्युत्पत्नी॒मा न॑य॒न्त्यन्वनाꣳ॑सि॒ प्र व॑र्तय॒न्त्यथ॒ वा अ॑स्यै॒ष धिष्णि॑यो हीयते॒ सो\-ऽनु॑ ध्यायति॒ स ई᳚श्व॒रो रु॒द्रो भू॒त्वा~(९)

%3.1.3.2
प्र॒जां प॒शून् यज॑मानस्य॒ शम॑यितो॒र्यर्\mbox{}हि॑ प॒शुमाप्री॑त॒मुद॑ञ्चं॒ नय॑न्ति॒ तर्\mbox{}हि॒ तस्य॑ पशु॒श्रप॑णꣳ हरे॒त्तेनै॒वैन॑म्भा॒गिनं॑ करोति॒ यज॑मानो॒ वा आ॑हव॒नीयो॒ यज॑मानं॒ वा ए॒तद्वि क॑र्\mbox{}षन्ते॒ यदा॑हव॒नीया᳚त्पशु॒श्रप॑ण॒ꣳ॒ हर॑न्ति॒ स वै॒व स्यान्नि॑र्म॒न्थ्यं॑ वा कुर्या॒द्यज॑मानस्य सात्म॒त्वाय॒ यदि॑ प॒शोर॑व॒दानं॒ नश्ये॒दाज्य॑स्य प्रत्या॒ख्याय॒मव॑ द्ये॒थ्सैव ततः॒ प्राय॑श्चित्ति॒र्ये प॒शुं वि॑मथ्नी॒रन् यस्तान्का॒मये॒तार्ति॒मार्च्छे॑यु॒रिति॑ कु॒विद॒ङ्गेति॒ नमो॑वृक्तिवत्य॒र्चाग्नी᳚ध्रे जुहुया॒न्नमो॑वृक्तिमे॒वैषां᳚ वृङ्क्ते ता॒जगार्ति॒मार्च्छ॑न्ति॥~(१०)

%3.1.4.0
{\anuvakamend[{भू॒त्वा ततः॒ षड्विꣳ॑शतिश्च}]}%~(३)

%3.1.4.1
प्र॒जाप॑ते॒र्जाय॑मानाः प्र॒जा जा॒ताश्च॒ या इ॒माः। तस्मै॒ प्रति॒ प्र वे॑दय चिकि॒त्वाꣳ अनु॑ मन्यताम्। इ॒मम्प॒शुम्प॑शुपते ते अ॒द्य ब॒ध्नाम्य॑ग्ने सुकृ॒तस्य॒ मध्ये᳚। अनु॑ मन्यस्व सु॒यजा॑ यजाम॒ जुष्टं॑ दे॒वाना॑मि॒दम॑स्तु ह॒व्यम्। प्र॒जा॒नन्तः॒ प्रति॑ गृह्णन्ति॒ पूर्वे᳚ प्रा॒णमङ्गे᳚भ्यः॒ पर्या॒चर॑न्तम्। सु॒व॒र्गं या॑हि प॒थिभि॑र्देव॒यानै॒रोष॑धीषु॒ प्रति॑ तिष्ठा॒ शरी॑रैः। येषा॒मीशे᳚~(११)

%3.1.4.2
प॒शु॒पतिः॑ पशू॒नां चतु॑ष्पदामु॒त च॑ द्वि॒पदा᳚म्। निष्क्री॑तो॒\-ऽयं य॒ज्ञिय॑म्भा॒गमे॑तु रा॒यस्पोषा॒ यज॑मानस्य सन्तु। ये ब॒ध्यमा॑न॒मनु॑ ब॒ध्यमा॑ना अ॒भ्यैक्ष॑न्त॒ मन॑सा॒ चक्षु॑षा च। अ॒ग्निस्ताꣳ अग्रे॒ प्र मु॑मोक्तु दे॒वः प्र॒जा\-प॑तिः प्र॒जया॑ संविदा॒नः। य आ॑र॒ण्याः प॒शवो॑ वि॒श्वरू॑पा॒ विरू॑पाः॒ सन्तो॑ बहु॒धैक॑रूपाः। वा॒युस्ताꣳ अग्रे॒ प्र मु॑मोक्तु दे॒वः प्र॒जा\-प॑तिः प्र॒जया॑ संविदा॒नः। प्र॒मु॒ञ्चमा॑नाः~(१२)

%3.1.4.3
भुव॑नस्य॒ रेतो॑ गा॒तुं ध॑त्त॒ यज॑मानाय देवाः। उ॒पाकृ॑तꣳ शशमा॒नं यदस्था᳚ज्जी॒वं दे॒वाना॒मप्ये॑तु॒ पाथः॑। नाना᳚ प्रा॒णो यज॑मानस्य प॒शुना॑ य॒ज्ञो दे॒वेभिः॑ स॒ह दे॑व॒यानः॑। जी॒वं दे॒वाना॒मप्ये॑तु॒ पाथः॑ स॒त्याः स॑न्तु॒ यज॑मानस्य॒ कामाः᳚। यत्प॒शुर्मा॒युमकृ॒तोरो॑ वा प॒द्भिरा॑ह॒ते। अ॒ग्निर्मा॒ तस्मा॒देन॑सो॒ विश्वा᳚न्मुञ्च॒त्वꣳह॑सः। शमि॑तार उ॒पेत॑न य॒ज्ञम्~(१३)

%3.1.4.4
दे॒वेभि॑रिन्वि॒तम्। पाशा᳚त्प॒शुं प्र मु॑ञ्चत ब॒न्धाद्य॒ज्ञप॑तिं॒ परि॑। अदि॑तिः॒ पाशं॒ प्र मु॑मोक्त्वे॒तं नमः॑ प॒शुभ्यः॑ पशु॒पत॑ये करोमि। अ॒रा॒ती॒यन्त॒मध॑रं कृणोमि॒ यं द्वि॒ष्मस्तस्मि॒न्प्रति॑ मुञ्चामि॒ पाशम्᳚। त्वामु॒ ते द॑धिरे हव्य॒वाहꣳ॑ शृतङ्क॒र्तार॑मु॒त य॒ज्ञियं॑ च। अग्ने॒ सद॑क्षः॒ सत॑नु॒र्॒\mbox{}हि भू॒त्वा\-ऽथ॑ ह॒व्या जा॑तवेदो जुषस्व। जात॑वेदो व॒पया॑ गच्छ दे॒वान्त्वꣳ हि होता᳚ प्रथ॒मो ब॒भूथ॑। घृ॒तेन॒ त्वं त॒नुवो॑ वर्धयस्व॒ स्वाहा॑कृतꣳ ह॒विर॑दन्तु दे॒वाः। स्वाहा॑ दे॒वेभ्यो॑ दे॒वेभ्यः॒ स्वाहा᳚॥~(१४)

%3.1.5.0
{\anuvakamend[{ईशे᳚ प्रमु॒ञ्चमा॑ना य॒ज्ञन्त्वꣳ षोड॑श च}]}%~(४)

%3.1.5.1
प्रा॒जा॒प॒त्या वै प॒शव॒स्तेषाꣳ॑ रु॒द्रो\-ऽधि॑पति॒र्यदे॒ताभ्या॑मुपाक॒रोति॒ ताभ्या॑मे॒वैनं॑ प्रति॒प्रोच्या ल॑भत आ॒त्मनो\-ऽना᳚व्रस्काय॒ द्वाभ्या॑मु॒पाक॑रोति द्वि॒पाद्यज॑मानः॒ प्रति॑ष्ठित्या उपा॒कृत्य॒ पञ्च॑ जुहोति॒ पाङ्क्ताः᳚ प॒शवः॑ प॒शूने॒वाव॑ रुन्धे मृ॒त्यवे॒ वा ए॒ष नी॑यते॒ यत्प॒शुस्तं यद॑न्वा॒रभे॑त प्र॒मायु॑को॒ यज॑मानः स्या॒न्नाना᳚ प्रा॒णो यज॑मानस्य प॒शुनेत्या॑ह॒ व्यावृ॑त्त्यै~(१५)

%3.1.5.2
यत्प॒शुर्मा॒युमकृ॒तेति॑ जुहोति॒ शान्त्यै॒ शमि॑तार उ॒पेत॒नेत्या॑ह यथाय॒जुरे॒वैतद्व॒पायां॒ वा आ᳚ह्रि॒यमा॑णायाम॒ग्नेर्मेधो\-ऽप॑ क्रामति॒ त्वामु॒ ते द॑धिरे हव्य॒वाह॒मिति॑ व॒पाम॒भि जु॑होत्य॒ग्नेरे॒व मेध॒मव॑ रु॒न्धे\-ऽथो॑ शृत॒त्वाय॑ पु॒रस्ता᳚थ्स्वाहाकृतयो॒ वा अ॒न्ये दे॒वा उ॒परि॑ष्टाथ्स्वाहाकृतयो॒\-ऽन्ये स्वाहा॑ दे॒वेभ्यो॑ दे॒वेभ्यः॒ स्वाहेत्य॒भितो॑ व॒पां जु॑होति॒ ताने॒वोभया᳚न्प्रीणाति॥~(१६)

%3.1.6.0
{\anuvakamend[{व्यावृ॑त्त्या अ॒भितो॑ व॒पां पञ्च॑ च}]}%~(५)

%3.1.6.1
यो वा अय॑थादेवतं य॒ज्ञमु॑प॒चर॒त्या दे॒वता᳚भ्यो वृश्च्यते॒ पापी॑यान्भवति॒ यो य॑थादेव॒तं न दे॒वता᳚भ्य॒ आ वृ॑श्च्यते॒ वसी॑यान्भवत्याग्ने॒य्यर्चाग्नी᳚ध्रम॒भि मृ॑शेद्वैष्ण॒व्या ह॑वि॒र्धान॑माग्ने॒य्या स्रुचो॑ वाय॒व्य॑या वाय॒व्या᳚न्यैन्द्रि॒या सदो॑ यथादेव॒तमे॒व य॒ज्ञमुप॑ चरति॒ न दे॒वता᳚भ्य॒ आ वृ॑श्च्यते॒ वसी॑यान्भवति यु॒नज्मि॑ ते पृथि॒वीं ज्योति॑षा स॒ह यु॒नज्मि॑ वा॒युम॒न्तरि॑क्षेण~(१७)

%3.1.6.2
ते॒ स॒ह यु॒नज्मि॒ वाचꣳ॑ स॒ह सूर्ये॑ण ते यु॒नज्मि॑ ति॒स्रो वि॒पृचः॒ सूर्य॑स्य ते। अ॒ग्निर्दे॒वता॑ गाय॒त्री छन्द॑ उपा॒ꣳ॒शोः पात्र॑मसि॒ सोमो॑ दे॒वता᳚ त्रि॒ष्टुप्छन्दो᳚\-ऽन्तर्या॒मस्य॒ पात्र॑म॒सीन्द्रो॑ दे॒वता॒ जग॑ती॒ छन्द॑ इन्द्रवायु॒वोः पात्र॑मसि॒ बृह॒स्पति॑र्दे॒वता॑\-ऽनु॒ष्टुप्छन्दो॑ मि॒त्रावरु॑णयोः॒ पात्र॑मस्य॒श्विनौ॑ दे॒वता॑ प॒ङ्क्तिश्छन्दो॒\-ऽश्विनोः॒ पात्र॑मसि॒ सूर्यो॑ दे॒वता॑ बृह॒ती~(१८)

%3.1.6.3
छन्दः॑ शु॒क्रस्य॒ पात्र॑मसि च॒न्द्रमा॑ दे॒वता॑ स॒तोबृ॑हती॒ छन्दो॑ म॒न्थिनः॒ पात्र॑मसि॒ विश्वे॑ दे॒वा दे॒वतो॒ष्णिहा॒ छन्द॑ आग्रय॒णस्य॒ पात्र॑म॒सीन्द्रो॑ दे॒वता॑ क॒कुच्छन्द॑ उ॒क्थाना॒म्पात्र॑मसि पृथि॒वी दे॒वता॑ वि॒राट्छन्दो᳚ ध्रु॒वस्य॒ पात्र॑मसि॥~(१९)

%3.1.7.0
{\anuvakamend[{अ॒न्तरि॑क्षेण बृह॒ती त्रय॑स्त्रिꣳशच्च}]}%~(६)

%3.1.7.1
इ॒ष्टर्गो॒ वा अ॑ध्व॒र्युर्यज॑मानस्ये॒ष्टर्गः॒ खलु॒ वै पूर्वो॒\-ऽर्ष्टुः क्षी॑यत आस॒न्या᳚न्मा॒ मन्त्रा᳚त्पाहि॒ कस्या᳚श्चिद॒भिश॑स्त्या॒ इति॑ पु॒रा प्रा॑तरनुवा॒काज्जु॑हुयादा॒त्मन॑ एव॒ तद॑ध्व॒र्युः पु॒रस्ता॒च्छर्म॑ नह्य॒ते\-ऽना᳚र्त्यै संवे॒शाय॑ त्वोपवे॒शाय॑ त्वा गायत्रि॒यास्त्रि॒ष्टुभो॒ जग॑त्या अ॒भिभू᳚त्यै॒ स्वाहा॒ प्राणा॑पानौ मृ॒त्योर्मा॑ पातं॒ प्राणा॑पानौ॒ मा मा॑ हासिष्टं दे॒वता॑सु॒ वा ए॒ते प्रा॑णापा॒नयोः᳚~(२०)

%3.1.7.2
व्याय॑च्छन्ते॒ येषा॒ꣳ॒ सोमः॑ समृ॒च्छते॑ संवे॒शाय॑ त्वोपवे॒शाय॒ त्वेत्या॑ह॒ छन्दाꣳ॑सि॒ वै सं॑वे॒श उ॑पवे॒शश्छन्दो॑भिरे॒वास्य॒ छन्दाꣳ॑सि वृङ्क्ते॒ प्रेति॑व॒न्त्याज्या॑नि भवन्त्य॒भिजि॑त्यै म॒रुत्व॑तीः प्रति॒पदो॒ विजि॑त्या उ॒भे बृ॑हद्रथन्त॒रे भ॑वत इ॒यं वाव र॑थन्त॒रम॒सौ बृ॒हदा॒भ्यामे॒वैन॑म॒न्तरे᳚त्य॒द्य वाव र॑थन्त॒रꣴ श्वो बृ॒हद॑द्या॒श्वादे॒वैन॑म॒न्तरे॑ति भू॒तम्~(२१)

%3.1.7.3
वाव र॑थन्त॒रम्भ॑वि॒ष्यद्बृ॒हद्भू॒ताच्चै॒वैनं॑ भविष्य॒तश्चा॒न्तरे॑ति॒ परि॑मितं॒ वाव र॑थन्त॒रमप॑रिमितम्बृ॒हत्परि॑मिताच्चै॒वैन॒मप॑रि\-मिताच्चा॒न्तरे॑ति विश्वामित्रजमद॒ग्नी वसि॑ष्ठेनास्पर्धेता॒ꣳ॒ स ए॒तज्ज॒मद॑ग्निर्विह॒व्य॑मपश्य॒त्तेन॒ वै स वसि॑ष्ठस्येन्द्रि॒यं वी॒र्य॑मवृङ्क्त॒ यद्वि॑ह॒व्यꣳ॑ श॒स्यत॑ इन्द्रि॒यमे॒व तद्वी॒र्यं॑ यज॑मानो॒ भ्रातृ॑व्यस्य वृङ्क्ते॒ यस्य॒ भूयाꣳ॑सो यज्ञक्र॒तव॒ इत्या॑हुः॒ स दे॒वता॑ वृङ्क्त॒ इति॒ यद्य॑ग्निष्टो॒मः सोमः॑ प॒रस्ता॒थ्स्यादु॒क्थ्यं॑ कुर्वीत॒ यद्यु॒क्थ्यः॑ स्याद॑तिरा॒त्रं कु॑र्वीत यज्ञक्र॒तुभि॑रे॒वास्य॑ दे॒वता॑ वृङ्क्ते॒ वसी॑यान्भवति॥~(२२)

%3.1.8.0
{\anuvakamend[{प्रा॒णा॒पा॒नयो᳚र्भू॒तं वृ॑ङ्क्ते॒\-ऽष्टाविꣳ॑शतिश्च}]}%~(७)

%3.1.8.1
नि॒ग्रा॒भ्याः᳚ स्थ देव॒श्रुत॒ आयु॑र्मे तर्पयत प्रा॒णं मे॑ तर्पयतापा॒नं मे॑ तर्पयत व्या॒नं मे॑ तर्पयत॒ चक्षु॑र्मे तर्पयत॒ श्रोत्रं॑ मे तर्पयत॒ मनो॑ मे तर्पयत॒ वाचं॑ मे तर्पयता॒त्मानं॑ मे तर्पय॒ताङ्गा॑नि मे तर्पयत प्र॒जां मे॑ तर्पयत प॒शून्मे॑ तर्पयत गृ॒हान्मे॑ तर्पयत ग॒णान्मे॑ तर्पयत स॒र्वग॑णं मा तर्पयत त॒र्पय॑त मा~(२३)

%3.1.8.2
ग॒णा मे॒ मा वि तृ॑ष॒न्नोष॑धयो॒ वै सोम॑स्य॒ विशो॒ विशः॒ खलु॒ वै राज्ञः॒ प्रदा॑तोरीश्व॒रा ऐ॒न्द्रः सोमो\-ऽवी॑वृधं वो॒ मन॑सा सुजाता॒ ऋत॑प्रजाता॒ भग॒ इद्वः॑ स्याम। इन्द्रे॑ण दे॒वीर्वी॒रुधः॑ संविदा॒ना अनु॑ मन्यन्ता॒ꣳ॒ सव॑नाय॒ सोम॒मित्या॒हौष॑धीभ्य ए॒वैन॒ꣴ॒ स्वायै॑ वि॒शः स्वायै॑ दे॒वता॑यै नि॒र्याच्या॒भि षु॑णोति॒ यो वै सोम॑स्याभिषू॒यमा॑णस्य~(२४)

%3.1.8.3
प्र॒थ॒मो\-ऽꣳ॑शुः स्कन्द॑ति॒ स ई᳚श्व॒र इ॑न्द्रि॒यं वी॒र्यं॑ प्र॒जां प॒शून् यज॑मानस्य॒ निर्\mbox{}ह॑न्तो॒स्तम॒भि म॑न्त्रये॒ता मा᳚स्कान्थ्स॒ह प्र॒जया॑ स॒ह रा॒यस्पोषे॑णेन्द्रि॒यं मे॑ वी॒र्यं॑ मा निर्व॑धी॒रित्या॒शिष॑मे॒वैतामा शा᳚स्त इन्द्रि॒यस्य॑ वी॒र्य॑स्य प्र॒जायै॑ पशू॒नामनि॑र्घाताय द्र॒फ्सश्च॑स्कन्द पृथि॒वीमनु॒ द्यामि॒मं च॒ योनि॒मनु॒ यश्च॒ पूर्वः॑। तृ॒तीयं॒ योनि॒मनु॑ स॒ञ्चर॑न्तं द्र॒फ्सं जु॑हो॒म्यनु॑ स॒प्त होत्राः᳚॥~(२५)

%3.1.9.0
{\anuvakamend[{त॒र्पय॑त मा\-ऽभिषू॒यमा॑णस्य॒ यश्च॒ दश॑ च}]}%~(८)

%3.1.9.1
यो वै दे॒वान्दे॑वयश॒सेना॒र्पय॑ति मनु॒ष्या᳚न्मनुष्ययश॒सेन॑ देवयश॒स्ये॑व दे॒वेषु॒ भव॑ति मनुष्ययश॒सी म॑नु॒ष्ये॑षु॒ यान्प्रा॒चीन॑माग्रय॒णाद्ग्रहा᳚न्गृह्णी॒यात्तानु॑पा॒ꣳ॒शु गृ॑ह्णीया॒द्यानू॒र्ध्वाꣴस्तानु॑पब्दि॒मतो॑ दे॒वाने॒व तद्दे॑वयश॒सेना᳚र्पयति मनु॒ष्या᳚न्मनुष्ययश॒सेन॑ देवयश॒स्ये॑व दे॒वेषु॑ भवति मनुष्ययश॒सी म॑नु॒ष्ये᳚ष्व॒ग्निः प्रा॑तःसव॒ने पा᳚त्व॒स्मान् वै᳚श्वान॒रो म॑हि॒ना वि॒श्वश॑म्भूः। स नः॑ पाव॒को द्रवि॑णं दधातु~(२६)

%3.1.9.2
आयु॑ष्मन्तः स॒हभ॑क्षाः स्याम। विश्वे॑ दे॒वा म॒रुत॒ इन्द्रो॑ अ॒स्मान॒स्मिन्द्वि॒तीये॒ सव॑ने॒ न ज॑ह्युः। आयु॑ष्मन्तः प्रि॒यमे॑षां॒ वद॑न्तो व॒यं दे॒वानाꣳ॑ सुम॒तौ स्या॑म। इ॒दं तृ॒तीय॒ꣳ॒ सव॑नं कवी॒नामृ॒तेन॒ ये च॑म॒समैर॑यन्त। ते सौ॑धन्व॒नाः सुव॑रानशा॒नाः स्वि॑ष्टिं नो अ॒भि वसी॑यो नयन्तु। आ॒यत॑नवती॒र्वा अ॒न्या आहु॑तयो हू॒यन्ते॑\-ऽनायत॒ना अ॒न्या या आ॑घा॒रव॑ती॒स्ता आ॒यत॑नवती॒र्याः~(२७)

%3.1.9.3
सौ॒म्यास्ता अ॑नायत॒ना ऐ᳚न्द्रवाय॒वमा॒दाया॑घा॒रमा घा॑रयेदध्व॒रो य॒ज्ञो॑\-ऽयम॑स्तु देवा॒ ओष॑धीभ्यः प॒शवे॑ नो॒ जना॑य॒ विश्व॑स्मै भू॒ताया᳚ध्व॒रो॑\-ऽसि॒ स पि॑न्वस्व घृत॒व॑द्देव सो॒मेति॑ सौ॒म्या ए॒व तदाहु॑तीरा॒यत॑नवतीः करोत्या॒यत॑नवान्भवति॒ य ए॒वं वेदाथो॒ द्यावा॑पृथि॒वी ए॒व घृ॒तेन॒ व्यु॑नत्ति॒ ते व्यु॑त्ते उपजीव॒नीये॑ भवत उपजीव॒नीयो॑ भवति~(२८)

%3.1.9.4
य ए॒वं वेदै॒ष ते॑ रुद्र भा॒गो यं नि॒रया॑चथा॒स्तं जु॑षस्व वि॒देर्गौ॑प॒त्यꣳ रा॒यस्पोषꣳ॑ सु॒वीर्यꣳ॑ संवथ्स॒रीणाꣴ॑ स्व॒स्तिम्। मनुः॑ पु॒त्रेभ्यो॑ दा॒यं व्य॑भज॒थ्स नाभा॒नेदि॑ष्ठं ब्रह्म॒चर्यं॒ वस॑न्तं॒ निर॑भज॒थ्स आग॑च्छ॒थ्सो᳚\-ऽब्रवीत्क॒था मा॒ निर॑भा॒गिति॒ न त्वा॒ निर॑भाक्ष॒मित्य॑ब्रवी॒दङ्गि॑रस इ॒मे स॒त्तमा॑सते॒ ते~(२९)

%3.1.9.5
सु॒व॒र्गं लो॒कं न प्र जा॑नन्ति॒ तेभ्य॑ इ॒दम्ब्राह्म॑णम्ब्रूहि॒ ते सु॑व॒र्गं लो॒कं यन्तो॒ य ए॑षाम्प॒शव॒स्ताꣴस्ते॑ दास्य॒न्तीति॒ तदे᳚भ्यो\-ऽब्रवी॒त्ते सु॑व॒र्गं लो॒कं यन्तो॒ य ए॑षाम्प॒शव॒ आस॒न्तान॑स्मा अददु॒स्तम्प॒शुभि॒श्चर॑न्तं यज्ञवा॒स्तौ रु॒द्र आग॑च्छ॒थ्सो᳚\-ऽब्रवी॒न्मम॒ वा इ॒मे प॒शव॒ इत्यदु॒र्वै~(३०)

%3.1.9.6
मह्य॑मि॒मानित्य॑ब्रवी॒न्न वै तस्य॒ त ई॑शत॒ इत्य॑ब्रवी॒द्यद्य॑ज्ञवा॒स्तौ हीय॑ते॒ मम॒ वै तदिति॒ तस्मा᳚द्यज्ञवा॒स्तु नाभ्य॒वेत्य॒ꣳ॒ सो᳚\-ऽब्रवीद्य॒ज्ञे मा भ॒जाथ॑ ते प॒शून्नाभि मꣴ॑स्य॒ इति॒ तस्मा॑ ए॒तम्म॒न्थिनः॑ सꣴस्रा॒वम॑जुहो॒त्ततो॒ वै तस्य॑ रु॒द्रः प॒शून्नाभ्य॑मन्यत॒ यत्रै॒तमे॒वं वि॒द्वान्म॒न्थिनः॑ सꣴस्रा॒वं जु॒होति॒ न तत्र॑ रु॒द्रः प॒शून॒भि म॑न्यते॥~(३१)

%3.1.10.0
{\anuvakamend[{द॒धा॒त्वा॒यत॑नवती॒र्या उ॑पजीव॒नीयो॑ भवति॒ ते\-ऽदु॒र्वै यत्रै॒तमेका॑\-दश च}]}%~(९)

%3.1.10.1
जुष्टो॑ वा॒चो भू॑यासं॒ जुष्टो॑ वा॒चस्पत॑ये॒ देवि॑ वाक्। यद्वा॒चो मधु॑म॒त्तस्मि॑न्मा धाः॒ स्वाहा॒ सर॑स्वत्यै। ऋ॒चा स्तोम॒ꣳ॒ सम॑र्धय गाय॒त्रेण॑ रथन्त॒रम्। बृ॒हद्गा॑य॒त्रव॑र्तनि। यस्ते᳚ द्र॒फ्सः स्कन्द॑ति॒ यस्ते॑ अ॒ꣳ॒शुर्बा॒हुच्यु॑तो धि॒षण॑योरु॒पस्था᳚त्। अ॒ध्व॒र्योर्वा॒ परि॒ यस्ते॑ प॒वित्रा॒थ्स्वाहा॑कृत॒मिन्द्रा॑य॒ तं जु॑होमि। यो द्र॒फ्सो अ॒ꣳ॒शुः प॑ति॒तः पृ॑थि॒व्यां प॑रिवा॒पात्~(३२)

%3.1.10.2
पु॒रो॒डाशा᳚त्कर॒म्भात्। धा॒ना॒सो॒मान्म॒न्थिन॑ इन्द्र शु॒क्राथ्स्वाहा॑कृत॒मिन्द्रा॑य॒ तं जु॑होमि। यस्ते᳚ द्र॒फ्सो मधु॑माꣳ इन्द्रि॒यावा॒न्थ्स्वाहा॑कृतः॒ पुन॑र॒प्येति॑ दे॒वान्। दि॒वः पृ॑थि॒व्याः पर्य॒न्तरि॑क्षा॒थ्स्वाहा॑कृत॒मिन्द्रा॑य॒ तं जु॑होमि। अ॒ध्व॒र्युर्वा ऋ॒त्विजां᳚ प्रथ॒मो यु॑ज्यते॒ तेन॒ स्तोमो॑ योक्त॒व्य॑ इत्या॑हु॒र्वाग॑ग्रे॒गा अग्र॑ एत्वृजु॒गा दे॒वेभ्यो॒ यशो॒ मयि॒ दध॑ती प्रा॒णान्प॒शुषु॑ प्र॒जाम्मयि॑~(३३)

%3.1.10.3
च॒ यज॑माने॒ चेत्या॑ह॒ वाच॑मे॒व तद्य॑ज्ञमु॒खे यु॑नक्ति॒ वास्तु॒ वा ए॒तद्य॒ज्ञस्य॑ क्रियते॒ यद्ग्र॒हा᳚न्गृही॒त्वा ब॑हिष्पवमा॒नꣳ सर्प॑न्ति॒ परा᳚ञ्चो॒ हि यन्ति॒ परा॑चीभिः स्तु॒वते॑ वैष्ण॒व्यर्चा पुन॒रेत्योप॑ तिष्ठते य॒ज्ञो वै विष्णु॑र्य॒ज्ञमे॒वाक॒र्विष्णो॒ त्वं नो॒ अन्त॑मः॒ शर्म॑ यच्छ सहन्त्य। प्र ते॒ धारा॑ मधु॒श्चुत॒ उथ्सं॑ दुह्रते॒ अक्षि॑त॒मित्या॑ह॒ यदे॒वास्य॒ शया॑नस्योप॒शुष्य॑ति॒ तदे॒वास्यै॒तेना प्या॑ययति॥~(३४)

%3.1.11.0
{\anuvakamend[{प॒रि॒वा॒पात्प्र॒जां मयि॑ दुह्रते॒ चतु॑र्दश च}]}%॥10॥

%3.1.11.1
अ॒ग्निना॑ र॒यिम॑श्नव॒त्पोष॑मे॒व दि॒वेदि॑वे। य॒शसं॑ वी॒रव॑त्तमम्॥ गोमाꣳ॑ अ॒ग्ने\-ऽवि॑माꣳ अ॒श्वी य॒ज्ञो नृ॒वथ्स॑खा॒ सद॒मिद॑प्रमृ॒ष्यः। इडा॑वाꣳ ए॒षो अ॑सुर प्र॒जावा᳚न्दी॒र्घो र॒यिः पृ॑थुबु॒ध्नः स॒भावान्॑॥ आ प्या॑यस्व॒ सं ते᳚॥ इ॒ह त्वष्टा॑रमग्रि॒यं वि॒श्वरू॑प॒मुप॑ ह्वये। अ॒स्माक॑मस्तु॒ केव॑लः॥ तन्न॑स्तु॒रीप॒मध॑ पोषयि॒त्नु देव॑ त्वष्ट॒र्वि र॑रा॒णः स्य॑स्व। यतो॑ वी॒रः~(३५)

%3.1.11.2
क॒र्म॒ण्यः॑ सु॒दक्षो॑ यु॒क्तग्रा॑वा॒ जाय॑ते दे॒वका॑मः। शि॒वस्त्व॑ष्टरि॒हा ग॑हि वि॒भुः पोष॑ उ॒त त्मना᳚। य॒ज्ञेय॑ज्ञे न॒ उद॑व। पि॒शङ्ग॑रूपः सु॒भरो॑ वयो॒धाः श्रु॒ष्टी वी॒रो जा॑यते दे॒वका॑मः। प्र॒जां त्वष्टा॒ वि ष्य॑तु॒ नाभि॑म॒स्मे अथा॑ दे॒वाना॒मप्ये॑तु॒ पाथः॑। प्र णो॑ दे॒व्या नो॑ दि॒वः। पी॒पि॒वाꣳस॒ꣳ॒ सर॑स्वतः॒ स्तनं॒ यो वि॒श्वद॑र्शतः। धु॒क्षी॒महि॑ प्र॒जामिषम्᳚~(३६)

%3.1.11.3
ये ते॑ सरस्व ऊ॒र्मयो॒ मधु॑मन्तो घृत॒श्चुतः॑। तेषां᳚ ते सु॒म्नमी॑महे। यस्य॑ व्र॒तम्प॒शवो॒ यन्ति॒ सर्वे॒ यस्य॑ व्र॒तमु॑प॒तिष्ठ॑न्त॒ आपः॒। यस्य॑ व्र॒ते पु॑ष्टि॒पति॒र्निवि॑ष्ट॒स्तꣳ सर॑स्वन्त॒मव॑से हुवेम। दि॒व्यꣳ सु॑प॒र्णं व॑य॒सम्बृ॒हन्त॑म॒पां गर्भं॑ वृष॒भमोष॑धीनाम्। अ॒भी॒प॒तो वृ॒ट्या त॒र्पय॑न्तं॒ तꣳ सर॑स्वन्त॒मव॑से हुवेम। सिनी॑वालि॒ पृथु॑ष्टुके॒ या दे॒वाना॒मसि॒ स्वसा᳚। जु॒षस्व॑ ह॒व्यम्~(३७)

%3.1.11.4
आहु॑तं प्र॒जां दे॑वि दिदिड्ढि नः। या सु॑पा॒णिः स्व॑ङ्गु॒रिः सु॒षूमा॑ बहु॒सूव॑री। तस्यै॑ वि॒श्पत्नि॑यै ह॒विः सि॑नीवा॒ल्यै जु॑होतन। इन्द्रं॑ वो वि॒श्वत॒स्परीन्द्रं॒ नरः॑। असि॑तवर्णा॒ हर॑यः सुप॒र्णा मिहो॒ वसा॑ना॒ दिव॒मुत्प॑तन्ति। त आ\-ऽव॑वृत्र॒न्थ्सद॑नानि कृ॒त्वादित्पृ॑थि॒वी घृ॒तैर्व्यु॑द्यते। हिर॑ण्यकेशो॒ रज॑सो विसा॒रे\-ऽहि॒र्धुनि॒र्वात॑ इव॒ ध्रजी॑मान्। शुचि॑भ्राजा उ॒षसः॑~(३८)

%3.1.11.5
नवे॑दा॒ यश॑स्वतीरप॒स्युवो॒ न स॒त्याः। आ ते॑ सुप॒र्णा अ॑मिनन्त॒ एवैः᳚ कृ॒ष्णो नो॑नाव वृष॒भो यदी॒दम्। शि॒वाभि॒र्न स्मय॑मानाभि॒रागा॒त्पत॑न्ति॒ मिहः॑ स्त॒नय॑न्त्य॒भ्रा। वा॒श्रेव॑ वि॒द्युन्मि॑माति व॒थ्सं न मा॒ता सि॑षक्ति। यदे॑षां वृ॒ष्टिरस॑र्जि। पर्व॑तश्चि॒न्महि॑ वृ॒द्धो बि॑भाय दि॒वश्चि॒थ्सानु॑ रेजत स्व॒ने वः॑। यत्क्रीड॑थ मरुतः~(३९)

%3.1.11.6
ऋ॒ष्टि॒मन्त॒ आप॑ इव स॒ध्रिय॑ञ्चो धवध्वे। अ॒भि क्र॑न्द स्त॒नय॒ गर्भ॒मा धा॑ उद॒न्वता॒ परि॑ दीया॒ रथे॑न। दृति॒ꣳ॒ सु क॑र्\mbox{}ष॒ विषि॑तं॒ न्य॑ञ्चꣳ स॒मा भ॑वन्तू॒द्वता॑ निपा॒दाः। त्वं त्या चि॒दच्यु॒ताग्ने॑ प॒शुर्न यव॑से। धामा॑ ह॒ यत्ते॑ अजर॒ वना॑ वृ॒श्चन्ति॒ शिक्व॑सः। अग्ने॒ भूरी॑णि॒ तव॑ जातवेदो॒ देव॑ स्वधावो॒\-ऽमृत॑स्य॒ धाम॑। याश्च॑~(४०)

%3.1.11.7
मा॒या मा॒यिनां᳚ विश्वमिन्व॒ त्वे पू॒र्वीः सं॑द॒धुः पृ॑ष्टबन्धो। दि॒वो नो॑ वृ॒ष्टिम्म॑रुतो ररीध्वं॒ प्र पि॑न्वत॒ वृष्णो॒ अश्व॑स्य॒ धाराः᳚। अ॒र्वाङे॒तेन॑ स्तनयि॒त्नुतेह्य॒पो नि॑षि॒ञ्चन्नसु॑रः पि॒ता नः॑। पिन्व॑न्त्य॒पो म॒रुतः॑ सु॒दान॑वः॒ पयो॑ घृ॒तव॑द्वि॒दथे᳚ष्वा॒भुवः॑। अत्यं॒ न मि॒हे वि न॑यन्ति वा॒जिन॒मुथ्सं॑ दुहन्ति स्त॒नय॑न्त॒मक्षि॑तम्। उ॒द॒प्रुतो॑ मरुत॒स्ताꣳ इ॑यर्त॒ वृष्टिम्᳚~(४१)

%3.1.11.8
ये विश्वे॑ म॒रुतो॑ जु॒नन्ति॑। क्रोशा॑ति॒ गर्दा॑ क॒न्ये॑व तु॒न्ना पेरुं॑ तुञ्जा॒ना पत्ये॑व जा॒या। घृ॒तेन॒ द्यावा॑पृथि॒वी मधु॑ना॒ समु॑क्षत॒ पय॑स्वतीः कृणु॒ताप॒ ओष॑धीः। ऊर्जं॑ च॒ तत्र॑ सुम॒तिं च॑ पिन्वथ॒ यत्रा॑ नरो मरुतः सि॒ञ्चथा॒ मधु॑। उदु॒ त्यञ्चि॒त्रम्। औ॒र्व॒भृ॒गु॒वच्छुचि॑मप्नवान॒वदा हु॑वे। अ॒ग्निꣳ स॑मु॒द्रवा॑ससम्। आ स॒वꣳ स॑वि॒तुर्य॑था॒ भग॑स्येव भु॒जिꣳ हु॑वे। अ॒ग्निꣳ स॑मु॒द्रवा॑ससम्। हु॒वे वात॑स्वनं क॒विम्प॒र्जन्य॑क्रन्द्य॒ꣳ॒ सहः॑। अ॒ग्निꣳ स॑मु॒द्रवा॑ससम्॥~(४२)

%3.2.0.0
{\anuvakamend[{वी॒र इषꣳ॑ ह॒व्यमु॒षसो॑ मरुतश्च॒ वृष्टिं॒ भग॑स्य॒ द्वाद॑श च}]}%॥11॥

%3.2.0.0

{\anuvakamend[{यो वै पव॑मानाना॒न्त्रीणि॑ परि॒भूः स्फ्यः स्व॒स्तिर्भक्षेहि॑ मही॒नां पयो॑\-ऽसि॒ देव॑ सवितरे॒तत्ते᳚ श्ये॒नाय॒ यद्वै होतो॑पयाम॒गृ॑हीतो\-ऽसि वाक्ष॒सत्प्र सो अ॑ग्न॒ एका॑\-दश}]}%॥11॥ 

{\prashnaend[{प्र॒जा\-प॑तिरकामयत प्र॒जाप॑ते॒र्जाय॑माना॒ व्याय॑च्छन्ते॒ मह्य॑मि॒मान्मा॒या मा॒यिनां॒ द्विच॑त्वारिꣳशत्॥42॥ प्र॒जा\-प॑तिरकामयता॒ग्निꣳ स॑मु॒द्रवा॑ससम्॥}]}
%%% END PRASHNA
