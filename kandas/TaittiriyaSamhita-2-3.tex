\sect{तृतीयः प्रश्नः}\setcounter{anuvakam}{0}
\dnsub{तैत्तिरीयसंहितायां द्वितीयकाण्डे तृतीयः प्रश्नः}
%2.3.1.0
%2.3.1.1
आ॒दि॒त्येभ्यो॒ भुव॑द्वद्भ्यश्च॒रुं निर्व॑पे॒द्भूति॑काम आदि॒त्या वा ए॒तम्भूत्यै॒ प्रति॑ नुदन्ते॒ यो\-ऽल॒म्भूत्यै॒ सन्भूतिं॒ न प्रा॒प्नोत्या॑दि॒त्याने॒व भुव॑द्वतः॒ स्वेन॑ भाग॒धेये॒नोप॑ धावति॒ त ए॒वैन॒म्भूतिं॑ गमयन्ति॒ भव॑त्ये॒वादि॒त्येभ्यो॑ धा॒रय॑द्वद्भ्यश्च॒रुं निर्व॑पे॒दप॑रुद्धो वा\-ऽपरु॒ध्यमा॑नो वा\-ऽ\-ऽदि॒त्या वा अ॑परो॒द्धार॑ आदि॒त्या अ॑वगमयि॒तार॑ आदि॒त्याने॒व धा॒रय॑द्वतः॥१॥

%2.3.1.2
स्वेन॑ भाग॒धेये॒नोप॑ धावति॒ त ए॒वैनं॑ वि॒शि दा᳚ध्रत्यनपरु॒ध्यो भ॑व॒त्यदि॒ते\-ऽनु॑ मन्य॒स्वेत्य॑परु॒ध्यमा॑नो\-ऽस्य प॒दमा द॑दीते॒यं वा अदि॑तिरि॒यमे॒वास्मै॑ रा॒ज्यमनु॑ मन्यते स॒त्याशीरित्या॑ह स॒त्यामे॒वाशिषं॑ कुरुत इ॒ह मन॒ इत्या॑ह प्र॒जा ए॒वास्मै॒ सम॑नसः करो॒त्युप॒ प्रेत॑ मरुतः॥२॥

%2.3.1.3
सु॒दा॒न॒व॒ ए॒ना वि॒श्पति॑ना॒भ्य॑मुꣳ राजा॑न॒मित्या॑ह मारु॒ती वै विड्ज्ये॒ष्ठो वि॒श्पति॑र्वि॒शैवनꣳ॑ रा॒ष्ट्रेण॒ सम॑र्धयति॒ यः प॒रस्ता᳚द्ग्राम्यवा॒दी स्यात्तस्य॑ गृ॒हाद्व्री॒हीना ह॑रेच्छु॒क्लाꣳश्च॑ कृ॒ष्णाꣳश्च॒ वि चि॑नुया॒द्ये शु॒क्लाः स्युस्तमा॑दि॒त्यं च॒रुं निर्व॑पेदादि॒त्या वै दे॒वत॑या॒ विड्विश॑मे॒वाव॑ गच्छति॥३॥

%2.3.1.4
अव॑गतास्य॒ विडन॑वगतꣳ रा॒ष्ट्रमित्या॑हु॒र्ये कृ॒ष्णाः स्युस्तं वा॑रु॒णं च॒रुं निर्व॑पेद्वारु॒णं वै रा॒ष्ट्रमु॒भे ए॒व विशं॑ च रा॒ष्ट्रं चाव॑ गच्छति॒ यदि॒ नाव॒गच्छे॑दि॒मम॒हमा॑दि॒त्येभ्यो॑ भा॒गं निर्व॑पा॒म्यामुष्मा॑द॒मुष्यै॑ वि॒शो\-ऽव॑गन्तो॒रिति॒ निर्व॑पेदादि॒त्या ए॒वैन॑म्भाग॒धेय॑म्प्रे॒प्सन्तो॒ विश॒मव॑॥३॥

%2.3.1.5
ग॒म॒य॒न्ति॒ यदि॒ नाव॒गच्छे॒दाश्व॑त्थान्म॒यूखा᳚न्थ्स॒प्त म॑ध्यमे॒षाया॒मुप॑ हन्यादि॒दम॒हमा॑दि॒त्यान्ब॑ध्ना॒म्यामुष्मा॑द॒मुष्यै॑ वि॒शो\-ऽव॑गन्तो॒रित्या॑दि॒त्या ए॒वैन॑म्ब॒द्धवी॑रा॒ विश॒मव॑ गमयन्ति॒ यदि॒ नाव॒गच्छे॑दे॒तमे॒वादि॒त्यं च॒रुं निर्व॑पेदि॒ध्मे\-ऽपि॑ म॒यूखा॒न्थ्सं न॑ह्येदनपरु॒ध्यमे॒वाव॑ गच्छ॒त्याश्व॑त्था भवन्ति म॒रुतां॒ वा ए॒तदोजो॒ यद॑श्व॒त्थ ओज॑सै॒व विश॒मव॑ गच्छति स॒प्त भ॑वन्ति स॒प्तग॑णा॒ वै म॒रुतो॑ गण॒श ए॒व विश॒मव॑ गच्छति॥५॥

%2.3.2.0
{\anuvakamend[{धा॒रय॑द्वतो मरुतो गच्छति॒ विश॒मवै॒तद॒ष्टाद॑श च॥१॥}]}

%2.3.2.1
दे॒वा वै मृ॒त्योर॑बिभयु॒स्ते प्र॒जाप॑ति॒मुपा॑धाव॒न्तेभ्य॑ ए॒ताम्प्रा॑जाप॒त्याꣳ श॒तकृ॑ष्णलां॒ निर॑वप॒त्तयै॒वैष्व॒मृत॑मदधा॒द्यो मृ॒त्योर्बि॑भी॒यात्तस्मा॑ ए॒ताम्प्रा॑जाप॒त्याꣳ श॒तकृ॑ष्णलां॒ निर्व॑पेत्प्र॒जाप॑तिमे॒व स्वेन॑ भाग॒धेये॒नोप॑ धावति॒ स ए॒वास्मि॒न्नायु॑र्दधाति॒ सर्व॒मायु॑रेति श॒तकृ॑ष्णला भवति श॒तायुः॒ पुरु॑षः श॒तेन्द्रि॑य॒ आयु॑ष्ये॒वेन्द्रि॒ये॥६॥

%2.3.2.2
प्रति॑ तिष्ठति घृ॒ते भ॑व॒त्यायु॒र्वै घृ॒तम॒मृत॒ꣳ॒ हिर॑ण्य॒मायु॑श्चै॒वास्मा॑ अ॒मृतं॑ च स॒मीची॑ दधाति च॒त्वारि॑चत्वारि कृ॒ष्णला॒न्यव॑ द्यति चतुरव॒त्तस्याप्त्या॑ एक॒धा ब्र॒ह्मण॒ उप॑ हरत्येक॒धैव यज॑मान॒ आयु॑र्दधात्य॒सावा॑दि॒त्यो न व्य॑रोचत॒ तस्मै॑ दे॒वाः प्राय॑श्चित्तिमैच्छ॒न्तस्मा॑ ए॒तꣳ सौ॒र्यं च॒रुं निर॑वप॒न्तेनै॒वास्मिन्न्॑॥७॥

%2.3.2.3
रुच॑मदधु॒र्यो ब्र॑ह्मवर्च॒सका॑मः॒ स्यात्तस्मा॑ ए॒तꣳ सौ॒र्यं च॒रुं निर्व॑पेद॒मुमे॒वादि॒त्यꣴ स्वेन॑ भाग॒धेये॒नोप॑ धावति॒ स ए॒वास्मि॑न्ब्रह्मवर्च॒सं द॑धाति ब्रह्मवर्च॒स्ये॑व भ॑वत्युभ॒यतो॑ रु॒क्मौ भ॑वत उभ॒यत॑ ए॒वास्मि॒न्रुचं॑ दधाति प्रया॒जेप्र॑याजे कृ॒ष्णलं॑ जुहोति दि॒ग्भ्य ए॒वास्मै᳚ ब्रह्मवर्च॒समव॑ रुन्द्ध आग्ने॒यम॒ष्टाक॑पालं॒ निर्व॑पेथ्सावि॒त्रं द्वाद॑शकपाल॒म्भूम्यै᳚॥८॥

%2.3.2.4
च॒रुं यः का॒मये॑त॒ हिर॑ण्यं विन्देय॒ हिर॑ण्य॒म्मोप॑ नमे॒दिति॒ यदा᳚ग्ने॒यो भव॑त्याग्ने॒यं वै हिर॑ण्यं॒ यस्यै॒व हिर॑ण्यं॒ तेनै॒वैन॑द्विन्दते सावि॒त्रो भ॑वति सवि॒तृप्र॑सूत ए॒वैन॑द्विन्दते॒ भूम्यै॑ च॒रुर्भ॑वत्य॒स्यामे॒वैन॑द्विन्दत॒ उपै॑न॒ꣳ॒ हिर॑ण्यं नमति॒ वि वा ए॒ष इ॑न्द्रि॒येण॑ वी॒र्ये॑णर्ध्यते॒ यो हिर॑ण्यं वि॒न्दत॑ ए॒ताम्॥९॥

%2.3.2.5
ए॒व निर्व॑पे॒द्धिर॑ण्यं वि॒त्त्वा नेन्द्रि॒येण॑ वी॒र्ये॑ण॒ व्यृ॑ध्यत ए॒तामे॒व निर्व॑पे॒द्यस्य॒ हिर॑ण्यं॒ नश्ये॒द्यदा᳚ग्ने॒यो भव॑त्याग्ने॒यं वै हिर॑ण्यं॒ यस्यै॒व हिर॑ण्यं॒ तेनै॒वैन॑द्विन्दति सावि॒त्रो भ॑वति सवि॒तृप्र॑सूत ए॒वैन॑द्विन्दति॒ भूम्यै॑ च॒रुर्भ॑वत्य॒स्यां वा ए॒तन्न॑श्यति॒ यन्नश्य॑त्य॒स्यामे॒वैन॑द्विन्द॒तीन्द्रः॑॥१०॥

%2.3.2.6
त्वष्टुः॒ सोम॑मभी॒षहा॑पिब॒थ्स विष्व॒ङ्व्या᳚र्च्छ॒थ्स इ॑न्द्रि॒येण॑ सोमपी॒थेन॒ व्या᳚र्ध्यत॒ स यदू॒र्ध्वमु॒दव॑मी॒त्ते श्या॒माका॑ अभव॒न्थ्स प्र॒जाप॑ति॒मुपा॑धाव॒त्तस्मा॑ ए॒तꣳ सो॑मे॒न्द्रꣴ श्या॑मा॒कं च॒रुं निर॑वप॒त्तेनै॒वास्मि॑न्निन्द्रि॒यꣳ सो॑मपी॒थम॑दधा॒द्वि वा ए॒ष इ॑न्द्रि॒येण॑ सोमपी॒थेन॑र्ध्यते॒ यः सोमं॒ वमि॑ति॒ यः सो॑मवा॒मी स्यात्तस्मै᳚॥११॥

%2.3.2.7
ए॒तꣳ सो॑मे॒न्द्रꣴ श्या॑मा॒कं च॒रुं निर्व॑पे॒थ्सोमं॑ चै॒वेन्द्रं॑ च॒ स्वेन॑ भाग॒धेये॒नोप॑ धावति॒ तावे॒वास्मि॑न्निन्द्रि॒यꣳ सो॑मपी॒थं ध॑त्तो॒ नेन्द्रि॒येण॑ सोमपी॒थेन॒ व्यृ॑ध्यते॒ यथ्सौ॒म्यो भव॑ति सोमपी॒थमे॒वाव॑ रुन्द्धे॒ यदै॒न्द्रो भव॑तीन्द्रि॒यं वै सो॑मपी॒थ इ॑न्द्रि॒यमे॒व सो॑मपी॒थमव॑ रुन्द्धे श्यामा॒को भ॑वत्ये॒ष वाव स सोमः॑॥१२॥

%2.3.2.8
सा॒क्षादे॒व सो॑मपी॒थमव॑ रुन्द्धे॒\-ऽग्नये॑ दा॒त्रे पु॑रो॒डाश॑म॒ष्टाक॑पालं॒ निर्व॑पे॒दिन्द्रा॑य प्रदा॒त्रे पु॑रो॒डाश॒मेका॑दशकपालम् प॒शुका॑मो॒\-ऽग्निरे॒वास्मै॑ प॒शून्प्र॑ज॒नय॑ति वृ॒द्धानिन्द्रः॒ प्र य॑च्छति॒ दधि॒ मधु॑ घृ॒तमापो॑ धा॒ना भ॑वन्त्ये॒तद्वै प॑शू॒नाꣳ रू॒पꣳ रू॒पेणै॒व प॒शूनव॑ रुन्द्धे पञ्चगृही॒तम्भ॑वति॒ पाङ्क्ता॒ हि प॒शवो॑ बहुरू॒पम्भ॑वति बहुरू॒पा हि प॒शवः॑॥१३॥

%2.3.2.9
समृ॑द्ध्यै प्राजाप॒त्यम्भ॑वति प्राजाप॒त्या वै प॒शवः॑ प्र॒जाप॑तिरे॒वास्मै॑ प॒शून्प्र ज॑नयत्या॒त्मा वै पुरु॑षस्य॒ मधु॒ यन्मध्व॒ग्नौ जु॒होत्या॒त्मान॑मे॒व तद्यज॑मानो॒\-ऽग्नौ प्र द॑धाति प॒ङ्क्त्यौ॑ याज्यानुवा॒क्ये॑ भवतः॒ पाङ्क्तः॒ पुरु॑षः॒ पाङ्क्ताः᳚ प॒शव॑ आ॒त्मान॑मे॒व मृ॒त्योर्नि॒ष्क्रीय॑ प॒शूनव॑ रुन्द्धे॥१४॥

%2.3.3.0
{\anuvakamend[{इ॒न्द्रि॒ये᳚\-ऽस्मि॒न्भूम्या॑ ए॒तामिन्द्रः॒ स्यात्तस्मै॒ सोमो॑ बहुरू॒पा हि प॒शव॒ एक॑चत्वारिꣳशच्च॥२॥}]}

%2.3.3.1
दे॒वा वै स॒त्त्रमा॑स॒तर्द्धि॑परिमितं॒ यश॑स्कामा॒स्तेषा॒ꣳ॒ सोम॒ꣳ॒ राजा॑नं॒ यश॑ आर्च्छ॒त्स गि॒रिमुदै॒त्तम॒ग्निरनूदै॒त्ताव॒ग्नीषोमौ॒ सम॑भवता॒न्ताविन्द्रो॑ य॒ज्ञवि॑भ्र॒ष्टो\-ऽनु॒ परै॒त्ताव॑ब्रवीद्या॒जय॑त॒म्मेति॒ तस्मा॑ ए॒तामिष्टिं॒ निर॑वपतामाग्ने॒यम॒ष्टाक॑पालमै॒न्द्रमेका॑दशकपालꣳ सौ॒म्यं च॒रुन्तयै॒वास्मि॒न्तेजः॑॥१५॥

%2.3.3.2
इ॒न्द्रि॒यम्ब्र॑ह्मवर्च॒सम॑धत्तां॒ यो य॒ज्ञवि॑भ्रष्टः॒ स्यात्तस्मा॑ ए॒तामिष्टिं॒ निर्व॑पेदाग्ने॒यम॒ष्टाक॑पालमै॒न्द्रमेका॑दशकपालꣳ सौ॒म्यं च॒रुं यदा᳚ग्ने॒यो भव॑ति॒ तेज॑ ए॒वास्मि॒न्तेन॑ दधाति॒ यदै॒न्द्रो भव॑तीन्द्रि॒यमे॒वास्मि॒न्तेन॑ दधाति॒ यथ्सौ॒म्यो ब्र॑ह्मवर्च॒सं तेना᳚ग्ने॒यस्य॑ च सौ॒म्यस्य॑ चै॒न्द्रे स॒माश्ले॑षये॒त्तेज॑श्चै॒वास्मि॑न्ब्रह्मवर्च॒सं च॑ स॒मीची᳚॥१६॥

%2.3.3.3
द॒धा॒त्य॒ग्नी॒षो॒मीय॒मेका॑दशकपालं॒ निर्व॑पे॒द्यं कामो॒ नोप॒नमे॑दाग्ने॒यो वै ब्रा᳚ह्म॒णः स सोम॑म्पिबति॒ स्वामे॒व दे॒वता॒ꣴ॒ स्वेन॑ भाग॒धेये॒नोप॑ धावति॒ सैवैनं॒ कामे॑न॒ सम॑र्धय॒त्युपै॑नं॒ कामो॑ नमत्यग्नीषो॒मीय॑म॒ष्टाक॑पालं॒ निर्व॑पेद्ब्रह्मवर्च॒सका॑मो॒ \-ऽग्नीषोमा॑वे॒व स्वेन॑ भाग॒धेये॒नोप॑ धावति॒ तावे॒वास्मि॑न्ब्रह्मवर्च॒सं ध॑त्तो ब्रह्मवर्च॒स्ये॑व॥१७॥

%2.3.3.4
भ॒व॒ति॒ यद॒ष्टाक॑पाल॒स्तेना᳚ग्ने॒यो यच्छ्या॑मा॒कस्तेन॑ सौ॒म्यः समृ॑द्ध्यै॒ सोमा॑य वा॒जिने᳚ श्यामा॒कं च॒रुं निर्व॑पे॒द्यः क्लैव्या᳚द्बिभी॒याद्रेतो॒ हि वा ए॒तस्मा॒द्वाजि॑नमप॒क्राम॒त्यथै॒ष क्लैब्या᳚द्बिभाय॒ सोम॑मे॒व वा॒जिन॒ꣴ॒ स्वेन॑ भाग॒धेये॒नोप॑ धावति॒ स ए॒वास्मि॒न्रेतो॒ वाजि॑नं दधाति॒ न क्ली॒बो भ॑वति ब्राह्मणस्प॒त्यमेका॑दशकपालं॒ निर्व॑पे॒द्ग्राम॑कामः॥१८॥

%2.3.3.5
ब्रह्म॑ण॒स्पति॑मे॒व स्वेन॑ भाग॒धेये॒नोप॑ धावति॒ स ए॒वास्मै॑ सजा॒तान्प्र य॑च्छति ग्रा॒म्ये॑व भ॑वति ग॒णव॑ती याज्यानुवा॒क्ये॑ भवतः सजा॒तैरे॒वैनं॑ ग॒णव॑न्तं करोत्ये॒तामे॒व निर्व॑पे॒द्यः का॒मये॑त॒ ब्रह्म॒न्विशं॒ वि ना॑शयेय॒मिति॑ मारु॒ती या᳚ज्यानुवा॒क्ये॑ कुर्या॒द्ब्रह्म॑न्ने॒व विशं॒ वि ना॑शयति॥१९॥

%2.3.4.0
{\anuvakamend[{तेजः॑ स॒मीची᳚ ब्रह्मवर्च॒स्ये॑व ग्राम॑काम॒स्त्रिच॑त्वारिꣳशच्च॥३॥}]}

%2.3.4.1
अ॒र्य॒म्णे च॒रुं निर्व॑पेथ्सुव॒र्गका॑मो॒\-ऽसौ वा आ॑दि॒त्यो᳚\-ऽर्य॒मा\-ऽर्य॒मण॑मे॒व स्वेन॑ भाग॒धेये॒नोप॑ धावति॒ स ए॒वैनꣳ॑ सुव॒र्गं लो॒कं ग॑मयत्यर्य॒म्णे च॒रुं निर्व॑पे॒द्यः का॒मये॑त॒ दान॑कामा मे प्र॒जाः स्यु॒रित्य॒सौ वा आ॑दि॒त्यो᳚\-ऽर्य॒मा यः खलु॒ वै ददा॑ति॒ सो᳚\-ऽर्य॒मा\-ऽर्य॒मण॑मे॒व स्वेन॑ भाग॒धेये॒नोप॑ धावति॒ स ए॒व॥२०॥

%2.3.4.2
अ॒स्मै॒ दान॑कामाः प्र॒जाः क॑रोति॒ दान॑कामा अस्मै प्र॒जा भ॑वन्त्यर्य॒म्णे च॒रुं निर्व॑पे॒द्यः का॒मये॑त स्व॒स्ति ज॒नता॑मिया॒मित्य॒सौ वा आ॑दि॒त्यो᳚\-ऽर्य॒मा\-ऽर्य॒मण॑मे॒व स्वेन॑ भाग॒धेये॒नोप॑ धावति॒ स ए॒वैनं॒ तद्ग॑मयति॒ यत्र॒ जिग॑मिष॒तीन्द्रो॒ वै दे॒वाना॑मानुजाव॒र आ॑सी॒थ्स प्र॒जाप॑ति॒मुपा॑धाव॒त्तस्मा॑ ए॒तमै॒न्द्रमा॑नुषू॒कमेका॑दशकपालं॒ निः॥२१॥

%2.3.4.3
अ॒व॒प॒त्तेनै॒वैन॒मग्रं॑ दे॒वता॑ना॒म्पर्य॑णयद्बु॒ध्नव॑ती॒ अग्र॑वती याज्यानुवा॒क्ये॑ अकरोद्बु॒ध्नादे॒वैन॒मग्र॒म्पर्य॑णय॒द्यो रा॑ज॒न्य॑ आनुजाव॒रः स्यात्तस्मा॑ ए॒तमै॒न्द्रमा॑नुषू॒कमेका॑दशकपालं॒ निर्व॑पे॒दिन्द्र॑मे॒व स्वेन॑ भाग॒धेये॒नोप॑ धावति॒ स ए॒वैन॒मग्रꣳ॑ समा॒नाना॒म्परि॑ णयति बु॒ध्नव॑ती॒ अग्र॑वती याज्यानुवा॒क्ये॑ भवतो बु॒ध्नादे॒वैन॒मग्रम्᳚॥२२॥

%2.3.4.4
परि॑ णयत्यानुषू॒को भ॑वत्ये॒षा ह्ये॑तस्य॑ दे॒वता॒ य आ॑नुजाव॒रः समृ॑द्ध्यै॒ यो ब्रा᳚ह्म॒ण आ॑नुजाव॒रः स्यात्तस्मा॑ ए॒तम्बा॑र्\mbox{}हस्प॒त्यमा॑नुषू॒कं च॒रुं निर्व॑पे॒द्बृह॒स्पति॑मे॒व स्वेन॑ भाग॒धेये॒नोप॑ धावति॒ स ए॒वैन॒मग्रꣳ॑ समा॒नाना॒म्परि॑ णयति बु॒ध्नव॑ती॒ अग्र॑वती याज्यानुवा॒क्ये॑ भवतो बु॒ध्नादे॒वैन॒मग्र॒म्परि॑ णयत्यानुषू॒को भ॑वत्ये॒षा ह्ये॑तस्य॑ दे॒वता॒ य आ॑नुजाव॒रः समृ॑द्ध्यै॥२३॥

%2.3.5.0
{\anuvakamend[{ए॒व निरग्र॑मे॒तस्य॑ च॒त्वारि॑ च॥४॥}]}

%2.3.5.1
प्र॒जाप॑ते॒स्त्रय॑स्त्रिꣳशद्दुहि॒तर॑ आस॒न्ताः सोमा॑य॒ राज्ञे॑\-ऽददा॒त्तासाꣳ॑ रोहि॒णीमुपै॒त्ता ईर्ष्य॑न्तीः॒ पुन॑रगच्छ॒न्ता अन्वै॒त्ताः पुन॑रयाचत॒ ता अ॑स्मै॒ न पुन॑रददा॒त्सो᳚\-ऽब्रवीदृ॒तम॑मीष्व॒ यथा॑ समाव॒च्छ उ॑पै॒ष्याम्यथ॑ ते॒ पुन॑र्दास्या॒मीति॒ स ऋ॒तमा॑मी॒त्ता अ॑स्मै॒ पुन॑रददा॒त्तासाꣳ॑ रोहि॒णीमे॒वोप॑॥२४॥

%2.3.5.2
ऐ॒त्तं यक्ष्म॑ आर्च्छ॒द्राजा॑नं॒ यक्ष्म॑ आर॒दिति॒ तद्रा॑जय॒क्ष्मस्य॒ जन्म॒ यत्पापी॑या॒नभ॑व॒त्तत्पा॑पय॒क्ष्मस्य॒ यज्जा॒याभ्यो\-ऽवि॑न्द॒त्तज्जा॒येन्य॑स्य॒ य ए॒वमे॒तेषां॒ यक्ष्मा॑णां॒ जन्म॒ वेद॒ नैन॑मे॒ते यक्ष्मा॑ विन्दन्ति॒ स ए॒ता ए॒व न॑म॒स्यन्नुपा॑धाव॒त्ता अ॑ब्रुव॒न्वरं॑ वृणामहै समाव॒च्छ ए॒व न॒ उपा॑य॒ इति॒ तस्मा॑ ए॒तम्॥२५॥

%2.3.5.3
आ॒दि॒त्यं च॒रुं निर॑वप॒न्तेनै॒वैन॑म्पा॒पाथ्स्रामा॑दमुञ्च॒न् यः पा॑पय॒क्ष्मगृ॑हीतः॒ स्यात्तस्मा॑ ए॒तमा॑दि॒त्यं च॒रुं निर्व॑पेदादि॒त्याने॒व स्वेन॑ भाग॒धेये॒नोप॑ धावति॒ त ए॒वैनं॑ पा॒पाथ्स्रामा᳚न्मुञ्चन्त्यमावा॒स्या॑यां॒ निर्व॑पेद॒मुमे॒वैन॑मा॒प्याय॑मान॒मन्वा प्या॑ययति॒ नवो॑नवो भवति॒ जाय॑मान॒ इति॑ पुरोनुवा॒क्या॑ भव॒त्यायु॑रे॒वास्मि॒न्तया॑ दधाति॒ यमा॑दि॒त्या अ॒ꣳ॒शुमा᳚प्या॒यय॒न्तीति॑ या॒ज्यैवैन॑मे॒तया᳚ प्याययति॥२६॥

%2.3.6.0
{\anuvakamend[{ए॒वोपै॒तम॑स्मि॒न्त्रयो॑दश च॥५॥}]}

%2.3.6.1
प्र॒जाप॑तिर्दे॒वेभ्यो॒\-ऽन्नाद्यं॒ व्यादि॑श॒त्सो᳚\-ऽब्रवी॒द्यदि॒माल्लोँ॒कान॒भ्य॑ति॒रिच्या॑तै॒ तन्ममा॑स॒दिति॒ तदि॒माल्लोँ॒कान॒भ्यत्य॑रिच्य॒तेन्द्र॒ꣳ॒ राजा॑न॒मिन्द्र॑मधिरा॒जमिन्द्रꣴ॑ स्व॒राजा॑न॒न्ततो॒ वै स इ॒माल्लोँ॒काꣳस्त्रे॒धादु॑ह॒त्तत्त्रि॒धातो᳚स्त्रिधातु॒त्वय्यं का॒मये॑तान्ना॒दः स्या॒दिति॒ तस्मा॑ ए॒तं त्रि॒धातुं॒ निर्व॑पे॒दिन्द्रा॑य॒ राज्ञे॑ पुरो॒डाशम्᳚॥२७॥

%2.3.6.2
एका॑दशकपाल॒मिन्द्रा॑याधिरा॒जायेन्द्रा॑य स्व॒राज्ञे॒\-ऽयं वा इन्द्रो॒ राजा॒यमिन्द्रो॑\-ऽधिरा॒जो॑\-ऽसाविन्द्रः॑ स्व॒राडि॒माने॒व लो॒कान्थ्स्वेन॑ भाग॒धेये॒नोप॑ धावति॒ त ए॒वास्मा॒ अन्न॒म्प्र य॑च्छन्त्यन्ना॒द ए॒व भ॑वति॒ यथा॑ व॒थ्सेन॒ प्रत्तां॒ गां दु॒ह ए॒वमे॒वेमाल्लोँ॒कान्प्रत्ता॒न्काम॑म॒न्नाद्यं॑ दुह उत्ता॒नेषु॑ क॒पाले॒ष्वधि॑ श्रय॒त्यया॑तयामत्वाय॒ त्रयः॑ पुरो॒डाशा॑ भवन्ति॒ त्रय॑ इ॒मे लो॒का ए॒षाल्लोँ॒काना॒माप्त्या॒ उत्त॑रउत्तरो॒ ज्याया᳚न्भवत्ये॒वमि॑व॒ हीमे लो॒काः समृ॑द्ध्यै॒ सर्वे॑षामभिग॒मय॒न्नव॑ द्य॒त्यछ॑म्बट्कारव्व्यँ॒त्यास॒मन्वा॒हानि॑र्दाहाय॥२८॥

%2.3.7.0
{\anuvakamend[{पु॒रो॒डाश॒न्त्रय॒ष्षड्विꣳ॑शतिश्च॥६॥}]}

%2.3.7.1
दे॒वा॒सु॒राः संय॑त्ता आस॒न्तां दे॒वानसु॑रा अजय॒न्ते दे॒वाः प॑राजिग्या॒ना असु॑राणां॒ वैश्य॒मुपा॑य॒न्तेभ्य॑ इन्द्रि॒यं वी॒र्य॑मपा᳚क्राम॒त्तदिन्द्रो॑\-ऽचाय॒त्तदन्वपा᳚क्राम॒त्तद॑व॒रुधं॒ नाश॑क्नो॒त्तद॑स्मादभ्य॒र्धो॑\-ऽचर॒त्स प्र॒जाप॑ति॒मुपा॑धाव॒त्तमे॒तया॒ सर्व॑पृष्ठया\-ऽयाजय॒त्तयै॒वास्मि॑न्निन्द्रि॒यं वी॒र्य॑मदधा॒द्य इ॑न्द्रि॒यका॑मः॥२९॥

%2.3.7.2
वी॒र्य॑कामः॒ स्यात्तमे॒तया॒ सर्व॑पृष्ठया याजयेदे॒ता ए॒व दे॒वताः॒ स्वेन॑ भाग॒धेये॒नोप॑ धावति॒ ता ए॒वास्मि॑न्निन्द्रि॒यं वी॒र्यं॑ दधति॒ यदिन्द्रा॑य॒ राथं॑तराय नि॒र्वप॑ति॒ यदे॒वाग्नेस्तेज॒स्तदे॒वाव॑ रुन्द्धे॒ यदिन्द्रा॑य॒ बार्\mbox{}ह॑ताय॒ यदे॒वेन्द्र॑स्य॒ तेज॒स्तदे॒वाव॑ रुन्द्धे॒ यदिन्द्रा॑य वैरू॒पाय॒ यदे॒व स॑वि॒तुस्तेज॒स्तत्॥३०॥

%2.3.7.3
ए॒वाव॑ रुन्द्धे॒ यदिन्द्रा॑य वैरा॒जाय॒ यदे॒व धा॒तुस्तेज॒स्तदे॒वाव॑ रुन्द्धे॒ यदिन्द्रा॑य शाक्व॒राय॒ यदे॒व म॒रुतां॒ तेज॒स्तदे॒वाव॑ रुन्द्धे॒ यदिन्द्रा॑य रैव॒ताय॒ यदे॒व बृह॒स्पते॒स्तेज॒स्तदे॒वाव॑ रुन्द्ध ए॒ताव॑न्ति॒ वै तेजाꣳ॑सि॒ तान्ये॒वाव॑ रुन्द्ध उत्ता॒नेषु॑ क॒पाले॒ष्वधि॑ श्रय॒त्यया॑तयामत्वाय॒ द्वाद॑शकपालः पुरो॒डाशः॑॥३१॥

%2.3.7.4
भ॒व॒ति॒ वै॒श्व॒दे॒व॒त्वाय॑ सम॒न्तम्प॒र्यव॑द्यति सम॒न्तमे॒वेन्द्रि॒यं वी॒र्यं॑ यज॑माने दधाति व्य॒त्यास॒मन्वा॒हानि॑र्दाहा॒याश्व॑ ऋष॒भो वृ॒ष्णिर्ब॒स्तः सा दक्षि॑णा वृष॒त्वायै॒तयै॒व य॑जेताभिश॒स्यमा॑न ए॒ताश्चेद्वा अ॑स्य दे॒वता॒ अन्न॑म॒दन्त्य॒दन्त्यु॑वे॒वास्य॑ मनु॒ष्याः᳚॥३२॥

%2.3.8.0
{\anuvakamend[{इ॒न्द्रि॒यका॑मः सवि॒तुस्तेज॒स्तत्पु॑रो॒डाशो॒\-ऽष्टात्रिꣳ॑शच्च॥७॥}]}

%2.3.8.1
रज॑नो॒ वै कौ॑णे॒यः क्र॑तु॒जितं॒ जान॑किं चक्षु॒र्वन्य॑मया॒त्तस्मा॑ ए॒तामिष्टिं॒ निर॑वपद॒ग्नये॒ भ्राज॑स्वते पुरो॒डाश॑म॒ष्टाक॑पालꣳ सौ॒र्यं च॒रुम॒ग्नये॒ भ्राज॑स्वते पुरो॒डाश॑म॒ष्टाक॑पाल॒न्तयै॒वास्मि॒ञ्चक्षु॑रदधा॒द्यश्चक्षु॑ष्कामः॒ स्यात्तस्मा॑ ए॒तामिष्टिं॒ निर्व॑पेद॒ग्नये॒ भ्राज॑स्वते पुरो॒डाश॑म॒ष्टाक॑पालꣳ सौ॒र्यं च॒रुम॒ग्नये॒ भ्राज॑स्वते पुरो॒डाश॑म॒ष्टाक॑पालम॒ग्नेर्वै चक्षु॑षा मनु॒ष्या॑ वि॥३३॥

%2.3.8.2
प॒श्य॒न्ति॒ सूर्य॑स्य दे॒वा अ॒ग्निं चै॒व सूर्यं॑ च॒ स्वेन॑ भाग॒धेये॒नोप॑ धावति॒ तावे॒वास्मि॒ञ्चक्षु॑र्धत्त॒श्चक्षु॑ष्माने॒व भ॑वति॒ यदा᳚ग्ने॒यौ भव॑त॒श्चक्षु॑षी ए॒वास्मि॒न्तत्प्रति॑ दधाति॒ यथ्सौ॒र्यो नासि॑कां॒ तेना॒भितः॑ सौ॒र्यमा᳚ग्ने॒यौ भ॑वत॒स्तस्मा॑द॒भितो॒ नासि॑कां॒ चक्षु॑षी॒ तस्मा॒न्नासि॑कया॒ चक्षु॑षी॒ विधृ॑ते समा॒नी या᳚ज्यानुवा॒क्ये॑ भवतः समा॒नꣳ हि चक्षुः॒ समृ॑द्ध्या॒ उदु॒ त्यं जा॒तवे॑दसꣳ स॒प्त त्वा॑ ह॒रितो॒ रथे॑ चि॒त्रं दे॒वाना॒मुद॑गा॒दनी॑क॒मिति॒ पिण्डा॒न्प्र य॑च्छति॒ चक्षु॑रे॒वास्मै॒ प्र य॑च्छति॒ यदे॒व तस्य॒ तत्॥३४॥

%2.3.9.0
{\anuvakamend[{वि ह्य॑ष्टाविꣳ॑शतिश्च॥८॥}]}

%2.3.9.1
ध्रु॒वो॑\-ऽसि ध्रु॒वो॑\-ऽहꣳ स॑जा॒तेषु॑ भूयासं॒ धीर॒श्चेत्ता॑ वसु॒विद्ध्रु॒वो॑\-ऽसि ध्रु॒वो॑\-ऽहꣳ स॑जा॒तेषु॑ भूयासमु॒ग्रश्चेत्ता॑ वसु॒विद्ध्रु॒वो॑\-ऽसि ध्रु॒वो॑\-ऽहꣳ स॑जा॒तेषु॑ भूयासमभि॒भूश्चेत्ता॑ वसु॒विदाम॑नम॒स्याम॑नस्य देवा॒ ये स॑जा॒ताः कु॑मा॒राः सम॑नस॒स्तान॒हं का॑मये हृ॒दा ते मां का॑मयन्ताꣳ हृ॒दा तान्म॒ आम॑नसः कृधि॒ स्वाहाम॑नमसि॥३५॥

%2.3.9.2
आम॑नस्य देवा॒ याः स्त्रियः॒ सम॑नस॒स्ता अ॒हं का॑मये हृ॒दा ता मां का॑मयन्ताꣳ हृ॒दा ता म॒ आम॑नसः कृधि॒ स्वाहा॑ वैश्वदे॒वीꣳ सा᳚ङ्ग्रह॒णीं निर्व॑पे॒द्ग्राम॑कामो वैश्वदे॒वा वै स॑जा॒ता विश्वा॑ने॒व दे॒वान्थ्स्वेन॑ भाग॒धेये॒नोप॑ धावति॒ त ए॒वास्मै॑ सजा॒तान्प्र य॑च्छन्ति ग्रा॒म्ये॑व भ॑वति साङ्ग्रह॒णी भ॑वति मनो॒ग्रह॑णं॒ वै सं॒ग्रह॑ण॒म्मन॑ ए॒व स॑जा॒ताना᳚म्॥३६॥

%2.3.9.3
गृ॒ह्णा॒ति॒ ध्रु॒वो॑\-ऽसि ध्रु॒वो॑\-ऽहꣳ स॑जा॒तेषु॑ भूयास॒मिति॑ परि॒धीन्परि॑ दधात्या॒शिष॑मे॒वैतामा शा॒स्ते\-ऽथो॑ ए॒तदे॒व सर्वꣳ॑ सजा॒तेष्वधि॑ भवति॒ यस्यै॒वं वि॒दुष॑ ए॒ते प॑रि॒धयः॑ परिधी॒यन्त॒ आम॑नम॒स्याम॑नस्य देवा॒ इति॑ ति॒स्र आहु॑तीर्जुहोत्ये॒ताव॑न्तो॒ वै स॑जा॒ता ये म॒हान्तो॒ ये क्षु॑ल्ल॒का याः स्त्रिय॒स्ताने॒वाव॑ रुन्द्धे॒ त ए॑न॒मव॑रुद्धा॒ उप॑ तिष्ठन्ते॥३७॥

%2.3.10.0
{\anuvakamend[{स्वाहाम॑नमसि सजा॒तानाꣳ॑ रुन्द्धे॒ पञ्च॑ च॥९॥}]}

%2.3.10.1
यन्नव॒मैत्तन्नव॑नीतमभव॒द्यदस॑र्प॒त्तथ्स॒र्पिर॑भव॒द्यदध्रि॑यत॒ तद्घृ॒तम॑भवद॒श्विनोः᳚ प्रा॒णो॑\-ऽसि॒ तस्य॑ ते दत्तां॒ ययोः᳚ प्रा॒णो\-ऽसि॒ स्वाहेन्द्र॑स्य प्रा॒णो॑\-ऽसि॒ तस्य॑ ते ददातु॒ यस्य॑ प्रा॒णो\-ऽसि॒ स्वाहा॑ मि॒त्रावरु॑णयोः प्रा॒णो॑\-ऽसि॒ तस्य॑ ते दत्तां॒ ययोः᳚ प्रा॒णो\-ऽसि॒ स्वाहा॒ विश्वे॑षां दे॒वानां᳚ प्रा॒णो॑\-ऽसि॥३८॥

%2.3.10.2
तस्य॑ ते ददतु॒ येषां᳚ प्रा॒णो\-ऽसि॒ स्वाहा॑ घृ॒तस्य॒ धारा॑म॒मृत॑स्य॒ पन्था॒मिन्द्रे॑ण द॒त्ताम्प्रय॑ताम्म॒रुद्भिः॑। तत्त्वा॒ विष्णुः॒ पर्य॑पश्य॒त्तत्त्वेडा॒ गव्यैर॑यत्। पा॒व॒मा॒नेन॑ त्वा॒ स्तोमे॑न गाय॒त्रस्य॑ वर्त॒न्योपा॒ꣳ॒शोर्वी॒र्ये॑ण दे॒वस्त्वा॑ सवि॒तोथ्सृ॑जतु जी॒वात॑वे जीवन॒स्यायै॑ बृहद्रथन्त॒रयो᳚स्त्वा॒ स्तोमे॑न त्रि॒ष्टुभो॑ वर्त॒न्या शु॒क्रस्य॑ वी॒र्ये॑ण दे॒वस्त्वा॑ सवि॒तोत्॥३९॥

%2.3.10.3
सृ॒ज॒तु॒ जी॒वात॑वे जीवन॒स्याया॑ अ॒ग्नेस्त्वा॒ मात्र॑या॒ जग॑त्यै वर्त॒न्याग्र॑य॒णस्य॑ वी॒र्ये॑ण दे॒वस्त्वा॑ सवि॒तोथ्सृ॑जतु जी॒वात॑वे जीवन॒स्याया॑ इ॒मम॑ग्न॒ आयु॑षे॒ वर्च॑से कृधि प्रि॒यꣳ रेतो॑ वरुण सोम राजन्न्। मा॒तेवा᳚स्मा अदिते॒ शर्म॑ यच्छ॒ विश्वे॑ देवा॒ जर॑दष्टि॒र्यथास॑त्। अ॒ग्निरायु॑ष्मा॒न्थ्स वन॒स्पति॑भि॒रायु॑ष्मा॒न्तेन॒ त्वायु॒षायु॑ष्मन्तं करोमि॒ सोम॒ आयु॑ष्मा॒न्थ्स ओष॑धीभिर्य॒ज्ञ आयु॑ष्मा॒न्थ्स दक्षि॑णाभि॒र्ब्रह्मायु॑ष्म॒त्तद्ब्रा᳚ह्म॒णैरायु॑ष्मद्दे॒वा आयु॑ष्मन्त॒स्ते॑\-ऽमृते॑न पि॒तर॒ आयु॑ष्मन्त॒स्ते स्व॒धयायु॑ष्मन्त॒स्तेन॒ त्वायु॒षायु॑ष्मन्तं करोमि॥४०॥

%2.3.11.0
{\anuvakamend[{विश्वे॑षां दे॒वानां᳚ प्रा॒णो॑\-ऽसि त्रि॒ष्टुभो॑ वर्त॒न्या शु॒क्रस्य॑ वी॒र्ये॑ण दे॒वस्त्वा॑ सवि॒तोत्सोम॒ आयु॑ष्मा॒न्पञ्च॑विꣳशतिश्च॥10॥}]}

%2.3.11.1
अ॒ग्निं वा ए॒तस्य॒ शरी॑रं गच्छति॒ सोम॒ꣳ॒ रसो॒ वरु॑ण एनं वरुणपा॒शेन॑ गृह्णाति॒ सर॑स्वतीं॒ वाग॒ग्नाविष्णू॑ आ॒त्मा यस्य॒ ज्योगा॒मय॑ति॒ यो ज्योगा॑मयावी॒ स्याद्यो वा॑ का॒मये॑त॒ सर्व॒मायु॑रिया॒मिति॒ तस्मा॑ ए॒तामिष्टिं॒ निर्व॑पेदाग्ने॒यम॒ष्टाक॑पालꣳ सौ॒म्यं च॒रुं वा॑रु॒णं दश॑कपालꣳ सारस्व॒तं च॒रुमा᳚ग्नावैष्ण॒वमेका॑दशकपालम॒ग्नेरे॒वास्य॒ शरी॑रं निष्क्री॒णाति॒ सोमा॒द्रसम्᳚॥४१॥

%2.3.11.2
वा॒रु॒णेनै॒वैनं॑ वरुणपा॒शान्मु॑ञ्चति सारस्व॒तेन॒ वाचं॑ दधात्य॒ग्निः सर्वा॑ दे॒वता॒ विष्णु॑र्य॒ज्ञो दे॒वता॑भिश्चै॒वैनं॑ य॒ज्ञेन॑ च भिषज्यत्यु॒त यदी॒तासु॒र्भव॑ति॒ जीव॑त्ये॒व यन्नव॒मैत्तन्नव॑नीतमभव॒दित्याज्य॒मवे᳚क्षते रू॒पमे॒वास्यै॒तन्म॑हि॒मानं॒ व्याच॑ष्टे॒\-ऽश्विनोः᳚ प्रा॒णो॑\-ऽसीत्या॑हा॒श्विनौ॒ वै दे॒वाना᳚म्॥४२॥

%2.3.11.3
भि॒षजौ॒ ताभ्या॑मे॒वास्मै॑ भेष॒जं क॑रो॒तीन्द्र॑स्य प्रा॒णो॑\-ऽसीत्या॑हेन्द्रि॒यमे॒वास्मि॑न्ने॒तेन॑ दधाति मि॒त्रावरु॑णयोः प्रा॒णो॑\-ऽसीत्या॑ह प्राणापा॒नावे॒वास्मि॑न्ने॒तेन॑ दधाति॒ विश्वे॑षां दे॒वानां᳚ प्रा॒णो॑\-ऽसीत्या॑ह वी॒र्य॑मे॒वास्मि॑न्ने॒तेन॑ दधाति घृ॒तस्य॒ धारा॑म॒मृत॑स्य॒ पन्था॒मित्या॑ह यथाय॒जुरे॒वैतत्पा॑वमा॒नेन॑ त्वा॒ स्तोमे॒नेति॑॥४३॥

%2.3.11.4
आ॒ह॒ प्रा॒णमे॒वास्मि॑न्ने॒तेन॑ दधाति बृहद्रथन्त॒रयो᳚स्त्वा॒ स्तोमे॒नेत्या॒हौज॑ ए॒वास्मि॑न्ने॒तेन॑ दधात्य॒ग्नेस्त्वा॒ मात्र॒येत्या॑हा॒त्मान॑मे॒वास्मि॑न्ने॒तेन॑ दधात्यृ॒त्विजः॒ पर्या॑हु॒र्याव॑न्त ए॒वर्त्विज॒स्त ए॑नम्भिषज्यन्ति ब्र॒ह्मणो॒ हस्त॑मन्वा॒रभ्य॒ पर्या॑हुरेक॒धैव यज॑मान॒ आयु॑र्दधति॒ यदे॒व तस्य॒ तद्धिर॑ण्यात्॥४४॥

%2.3.11.5
घृ॒तं निष्पि॑ब॒त्यायु॒र्वै घृ॒तम॒मृत॒ꣳ॒ हिर॑ण्यम॒मृता॑दे॒वायु॒र्निष्पि॑बति श॒तमा॑नम्भवति श॒तायुः॒ पुरु॑षः श॒तेन्द्रि॑य॒ आयु॑ष्ये॒वेन्द्रि॒ये प्रति॑ तिष्ठ॒त्यथो॒ खलु॒ याव॑तीः॒ समा॑ ए॒ष्यन्मन्ये॑त॒ ताव॑न्मानꣴ स्या॒थ्समृ॑द्ध्या इ॒मम॑ग्न॒ आयु॑षे॒ वर्च॑से कृ॒धीत्या॒हायु॑रे॒वास्मि॒न्वर्चो॑ दधाति॒ विश्वे॑ देवा॒ जर॑दष्टि॒र्यथास॒दित्या॑ह॒ जर॑दष्टिमे॒वैनं॑ करोत्य॒ग्निरायु॑ष्मा॒निति॒ हस्तं॑ गृह्णात्ये॒ते वै दे॒वा आयु॑ष्मन्त॒स्त ए॒वास्मि॒न्नायु॑र्दधति॒ सर्व॒मायु॑रेति॥४५॥

%2.3.12.0
{\anuvakamend[{रसं॑ दे॒वाना॒ꣴ॒ स्तोमे॒नेति॒ हिर॑ण्या॒दस॒दिति॒ द्वाविꣳ॑शतिश्च॥11॥}]}

%2.3.12.1
प्र॒जाप॑ति॒र्वरु॑णा॒याश्व॑मनय॒त्स स्वां दे॒वता॑मार्च्छ॒त्स पर्य॑दीर्यत॒ स ए॒तं वा॑रु॒णं चतु॑ष्कपालमपश्य॒त्तं निर॑वप॒त्ततो॒ वै स व॑रुणपा॒शाद॑मुच्यत॒ वरु॑णो॒ वा ए॒तं गृ॑ह्णाति॒ यो\-ऽश्वं॑ प्रतिगृ॒ह्णाति॒ याव॒तो\-ऽश्वा᳚न्प्रतिगृह्णी॒यात्ताव॑तो वारु॒णाञ्चतु॑ष्कपाला॒न्निर्व॑पे॒द्वरु॑णमे॒व स्वेन॑ भाग॒धेये॒नोप॑ धावति॒ स ए॒वैनं॑ वरुणपा॒शान्मु॑ञ्चति॥४६॥

%2.3.12.2
चतु॑ष्कपाला भवन्ति॒ चतु॑ष्पा॒द्ध्यश्वः॒ समृ॑द्ध्या॒ एक॒मति॑रिक्तं॒ निर्व॑पे॒द्यमे॒व प्र॑तिग्रा॒ही भव॑ति॒ यं वा॒ नाध्येति॒ तस्मा॑दे॒व व॑रुणपा॒शान्मु॑च्यते॒ यद्यप॑रं प्रतिग्रा॒ही स्याथ्सौ॒र्यमेक॑कपाल॒मनु॒ निर्व॑पेद॒मुमे॒वादि॒त्यमु॑च्चा॒रं कु॑रुते॒\-ऽपो॑\-ऽवभृ॒थमवै᳚त्य॒प्सु वै वरु॑णः सा॒क्षादे॒व वरु॑ण॒मव॑ यजते\-ऽपोन॒प्त्रीयं॑ च॒रुम्पुन॒रेत्य॒ निर्व॑पेद॒प्सुयो॑नि॒र्वा अश्वः॒ स्वामे॒वैनं॒ योनिं॑ गमयति॒ स ए॑नꣳ शा॒न्त उप॑ तिष्ठते॥४७॥

%2.3.13.0
{\anuvakamend[{मु॒ञ्च॒ति॒ च॒रुꣳ स॒प्तद॑श च॥12॥}]}

%2.3.13.1
या वा॑मिन्द्रावरुणा यत॒व्या॑ त॒नूस्तये॒ममꣳह॑सो मुञ्चतं॒ या वा॑मिन्द्रावरुणा सह॒स्या॑ रक्ष॒स्या॑ तेज॒स्या॑ त॒नूस्तये॒ममꣳह॑सो मुञ्चतं॒ यो वा॑मिन्द्रावरुणाव॒ग्नौ स्राम॒स्तं वा॑मे॒तेनाव॑ यजे॒ यो वा॑मिन्द्रावरुणा द्वि॒पाथ्सु॑ प॒शुषु॒ चतु॑ष्पाथ्सु गो॒ष्ठे गृ॒हेष्व॒प्स्वोष॑धीषु॒ वन॒स्पति॑षु॒ स्राम॒स्तं वा॑मे॒तेनाव॑ यज॒ इन्द्रो॒ वा ए॒तस्य॑॥४८॥

%2.3.13.2
इ॒न्द्रि॒येणाप॑ क्रामति॒ वरु॑ण एनं वरुणपा॒शेन॑ गृह्णाति॒ यः पा॒प्मना॑ गृही॒तो भव॑ति॒ यः पा॒प्मना॑ गृही॒तः स्यात्तस्मा॑ ए॒तामै᳚न्द्रावरु॒णीम्प॑य॒स्यां᳚ निर्व॑पे॒दिन्द्र॑ ए॒वास्मि॑न्निन्द्रि॒यं द॑धाति॒ वरु॑ण एनं वरुणपा॒शान्मु॑ञ्चति पय॒स्या॑ भवति॒ पयो॒ हि वा ए॒तस्मा॑दप॒क्राम॒त्यथै॒ष पा॒प्मना॑ गृही॒तो यत्प॑य॒स्या॑ भव॑ति॒ पय॑ ए॒वास्मि॒न्तया॑ दधाति पय॒स्या॑याम्॥४९॥

%2.3.13.3
पु॒रो॒डाश॒मव॑ दधात्यात्म॒न्वन्त॑मे॒वैनं॑ करो॒त्यथो॑ आ॒यत॑नवन्तमे॒व च॑तु॒र्धा व्यू॑हति दि॒क्ष्वे॑व प्रति॑ तिष्ठति॒ पुनः॒ समू॑हति दि॒ग्भ्य ए॒वास्मै॑ भेष॒जं क॑रोति स॒मूह्याव॑ द्यति॒ यथावि॑द्धं निष्कृ॒न्तति॑ ता॒दृगे॒व तद्यो वा॑मिन्द्रावरुणाव॒ग्नौ स्राम॒स्तं वा॑मे॒तेनाव॑ यज॒ इत्या॑ह॒ दुरि॑ष्ट्या ए॒वैन॑म्पाति॒ यो वा॑मिन्द्रावरुणा द्वि॒पाथ्सु॑ प॒शुषु॒ स्राम॒स्तं वा॑मे॒तेनाव॑ यज॒ इत्या॑है॒ताव॑ती॒र्वा आप॒ ओष॑धयो॒ वन॒स्पत॑यः प्र॒जाः प॒शव॑ उपजीव॒नीया॒स्ता ए॒वास्मै॑ वरुणपा॒शान्मु॑ञ्चति॥५०॥

%2.3.14.0
{\anuvakamend[{ए॒तस्य॑ पय॒स्या॑याम्पाति॒ षड्विꣳ॑शतिश्च॥13॥}]}

%2.3.14.1
स प्र॑त्न॒वन्नि काव्येन्द्रं॑ वो वि॒श्वत॒स्परीन्द्रं॒ नरः॑। त्वं नः॑ सोम वि॒श्वतो॒ रक्षा॑ राजन्नघाय॒तः। न रि॑ष्ये॒त्त्वाव॑तः॒ सखा᳚। या ते॒ धामा॑नि दि॒वि या पृ॑थि॒व्यां या पर्व॑ते॒ष्वोष॑धीष्व॒प्सु। तेभि॑र्नो॒ विश्वैः᳚ सु॒मना॒ अहे॑ड॒न्राज᳚न्थ्सोम॒ प्रति॑ ह॒व्या गृ॑भाय। अग्नी॑षोमा॒ सवे॑दसा॒ सहू॑ती वनतं॒ गिरः॑। सं दे॑व॒त्रा ब॑भूवथुः। यु॒वम्॥५१॥

%2.3.14.2
ए॒तानि॑ दि॒वि रो॑च॒नान्य॒ग्निश्च॑ सोम॒ सक्र॑तू अधत्तम्। यु॒वꣳ सिन्धूꣳ॑ र॒भिश॑स्तेरव॒द्यादग्नी॑षोमा॒वमु॑ञ्चतं गृभी॒तान्। अग्नी॑षोमावि॒मꣳ सु मे॑ शृणु॒तं वृ॑षणा॒ हवम्᳚। प्रति॑ सू॒क्तानि॑ हर्यत॒म्भव॑तं दा॒शुषे॒ मयः॑। आन्यं दि॒वो मा॑त॒रिश्वा॑ जभा॒राम॑थ्नाद॒न्यम्परि॑ श्ये॒नो अद्रेः᳚। अग्नी॑षोमा॒ ब्रह्म॑णा वावृधा॒नोरुं य॒ज्ञाय॑ चक्रथुरु लो॒कम्। अग्नी॑षोमा ह॒विषः॒ प्रस्थि॑तस्य वी॒तम्॥५२॥

%2.3.14.3
हर्य॑तं वृषणा जु॒षेथा᳚म्। सु॒शर्मा॑णा॒ स्वव॑सा॒ हि भू॒तमथा॑ धत्तं॒ यज॑मानाय॒ शं योः। आ प्या॑यस्व॒ सं ते᳚। ग॒णानां᳚ त्वा ग॒णप॑तिꣳ हवामहे क॒विं क॑वी॒नामु॑प॒मश्र॑वस्तमम्। ज्ये॒ष्ठ॒राजं॒ ब्रह्म॑णां ब्रह्मणस्पत॒ आ नः॑ शृ॒ण्वन्नू॒तिभिः॑ सीद॒ साद॑नम्। स इज्जने॑न॒ स वि॒शा स जन्म॑ना॒ स पु॒त्रैर्वाज॑म्भरते॒ धना॒ नृभिः॑। दे॒वानां॒ यः पि॒तर॑मा॒विवा॑सति॥५३॥

%2.3.14.4
श्र॒द्धाम॑ना ह॒विषा॒ ब्रह्म॑ण॒स्पतिम्᳚। स सु॒ष्टुभा॒ स ऋक्व॑ता ग॒णेन॑ व॒लꣳ रु॑रोज फलि॒गꣳ रवे॑ण। बृह॒स्पति॑रु॒स्रिया॑ हव्य॒सूदः॒ कनि॑क्रद॒द्वाव॑शती॒रुदा॑जत्। मरु॑तो॒ यद्ध॑ वो दि॒वो या वः॒ शर्म॑। अ॒र्य॒मा या॑ति वृष॒भस्तुवि॑ष्मान्दा॒ता वसू॑नाम्पुरुहू॒तो अर्\mbox{}हन्न्॑। स॒ह॒स्रा॒क्षो गो᳚त्र॒भिद्वज्र॑बाहुर॒स्मासु॑ दे॒वो द्रवि॑णं दधातु। ये ते᳚\-ऽर्यमन्ब॒हवो॑ देव॒यानाः॒ पन्था॑नः॥५४॥

%2.3.14.5
रा॒ज॒न्दि॒व आ॒चर॑न्ति। तेभि॑र्नो देव॒ महि॒ शर्म॑ यच्छ॒ शं न॑ एधि द्वि॒पदे॒ शं चतु॑ष्पदे। बु॒ध्नादग्र॒मङ्गि॑रोभिर्गृणा॒नो वि पर्व॑तस्य दृꣳहि॒तान्यै॑रत्। रु॒जद्रोधाꣳ॑सि कृ॒त्रिमा᳚ण्येषा॒ꣳ॒ सोम॑स्य॒ ता मद॒ इन्द्र॑श्चकार। बु॒ध्नादग्रे॑ण॒ वि मि॑माय॒ मानै॒र्वज्रे॑ण॒ खान्य॑तृणन्न॒दीना᳚म्। वृथा॑सृजत्प॒थिभि॑र्दीर्घया॒थैः सोम॑स्य॒ ता मद॒ इन्द्र॑श्चकार॥५॥

%2.3.14.6
प्र यो ज॒ज्ञे वि॒द्वाꣳ अ॒स्य बन्धुं॒ विश्वा॑नि दे॒वो जनि॑मा विवक्ति। ब्रह्म॒ ब्रह्म॑ण॒ उज्ज॑भार॒ मध्या᳚न्नी॒चादु॒च्चा स्व॒धया॒भि प्र त॑स्थौ। म॒हान्म॒ही अ॑स्तभाय॒द्वि जा॒तो द्याꣳ सद्म॒ पार्थि॑वं च॒ रजः॑। स बु॒ध्नादा᳚ष्ट ज॒नुषा॒भ्यग्र॒म्बृह॒स्पति॑र्दे॒वता॒ यस्य॑ स॒म्राट्। बु॒ध्नाद्यो अग्र॑म॒भ्यर्त्योज॑सा॒ बृह॒स्पति॒मा वि॑वासन्ति दे॒वाः। भि॒नद्व॒लं वि पुरो॑ दर्दरीति॒ कनि॑क्रद॒थ्सुव॑र॒पो जि॑गाय॥५६॥

%2.4.0.0
{\anuvakamend[{यु॒वं वी॒तमा॒ विवा॑सति॒ पन्था॑नो दीर्घया॒थैः सोम॑स्य॒ ता मद॒ इन्द्र॑श्चकार दे॒वा नव॑ च॥14॥}]}

%2.4.0.0

{\anuvakamend[{दे॒वा म॑नु॒ष्या॑ देवासु॒रा अ॑ब्रुवन्देवासु॒रास्तेषा᳚ङ्गाय॒त्री प्र॒जाप॑ति॒स्ता यत्राग्ने॒ गोभि॑श्चि॒त्रया॑ मारु॒तन्देवा॑ वसव्या॒ अग्ने॑ मारु॒तमिति॒ देवा॑ वसव्या॒ देवाः᳚ शर्मण्या॒स्त्वष्टा॑ ह॒तपु॑त्रो दे॒वा वै रा॑ज॒न्या᳚न्नवो॑नव॒श्चतु॑र्दश॥14॥ दे॒वा म॑नु॒ष्याः᳚ प्र॒जां प॒शून्देवा॑ वसव्याः परिद॒ध्यादि॒दमस्म्य॒ष्टाच॑त्वारिꣳशत्॥48॥ दे॒वा म॑नु॒ष्या॑ मादयध्वम्॥}]}
%%% END PRASHNA
