\sect{षष्ठमः प्रश्नः}\setcounter{anuvakam}{0}
\dnsub{तैत्तिरीयसंहितायां चतुर्थकाण्डे षष्ठमः प्रश्नः}
%4.6.1.0
%4.6.1.1
अश्म॒न्नूर्ज॒म्पर्व॑ते शिश्रिया॒णां वाते॑ प॒र्जन्ये॒ वरु॑णस्य॒ शुष्मे᳚। अ॒द्भ्य ओष॑धीभ्यो॒ वन॒स्पति॒भ्यो\-ऽधि॒ सम्भृ॑तां॒ तां न॒ इष॒मूर्जं॑ धत्त मरुतः सꣳररा॒णाः। अश्मꣴ॑स्ते॒ क्षुद॒मुं ते॒ शुगृ॑च्छतु॒ यं द्वि॒ष्मः। स॒मु॒द्रस्य॑ त्वा॒\-ऽवाक॒याग्ने॒ परि॑ व्ययामसि। पाव॒को अ॒स्मभ्यꣳ॑ शि॒वो भ॑व। हि॒मस्य॑ त्वा ज॒रायु॒णाग्ने॒ परि॑ व्ययामसि। पा॒व॒को अ॒स्मभ्यꣳ॑ शि॒वो भ॑व। उप॑॥१॥

%4.6.1.2
ज्मन्नुप॑ वेत॒से\-ऽव॑त्तरं न॒दीष्वा। अग्ने॑ पि॒त्तम॒पाम॑सि। मण्डू॑कि॒ ताभि॒रा ग॑हि॒ सेमं नो॑ य॒ज्ञम्। पा॒व॒कव॑र्णꣳ शि॒वं कृ॑धि। पा॒व॒क आ चि॒तय॑न्त्या कृ॒पा। क्षाम॑न्रुरु॒च उ॒षसो॒ न भा॒नुना᳚। तूर्व॒न्न याम॒न्नेत॑शस्य॒ नू रण॒ आ यो घृ॒णे। न त॑तृषा॒णो अ॒जरः॑। अग्ने॑ पावक रो॒चिषा॑ म॒न्द्रया॑ देव जि॒ह्वया᳚। आ दे॒वान्॥२॥

%4.6.1.3
व॒क्षि॒ यक्षि॑ च। स नः॑ पावक दीदि॒वो\-ऽग्ने॑ दे॒वाꣳ इ॒हा व॑ह। उप॑ य॒ज्ञꣳ ह॒विश्च॑ नः। अ॒पामि॒दं न्यय॑नꣳ समु॒द्रस्य॑ नि॒वेश॑नम्। अ॒न्यं ते॑ अ॒स्मत्त॑पन्तु हे॒तयः॑ पाव॒को अ॒स्मभ्यꣳ॑ शि॒वो भ॑व। नम॑स्ते॒ हर॑से शो॒चिषे॒ नम॑स्ते अस्त्व॒र्चिषे᳚। अ॒न्यं ते॑ अ॒स्मत्त॑पन्तु हे॒तयः॑ पाव॒को अ॒स्मभ्यꣳ॑ शि॒वो भ॑व। नृ॒षदे॒ वट्॥३॥

%4.6.1.4
अ॒प्सु॒षदे॒ वड्व॑न॒सदे॒ वड्ब॑र्\mbox{}हि॒षदे॒ वट्थ्सु॑व॒र्विदे॒ वट्। ये दे॒वा दे॒वानां᳚ य॒ज्ञिया॑ य॒ज्ञिया॑नाꣳ संवथ्स॒रीण॒मुप॑ भा॒गमास॑ते। अ॒हु॒तादो॑ ह॒विषो॑ य॒ज्ञे अ॒स्मिन्थ्स्व॒यं जु॑हुध्व॒म्मधु॑नो घृ॒तस्य॑। ये दे॒वा दे॒वेष्वधि॑ देव॒त्वमाय॒न् ये ब्रह्म॑णः पुरए॒तारो॑ अ॒स्य। येभ्यो॒ नर्ते पव॑ते॒ धाम॒ किं च॒न न ते दि॒वो न पृ॑थि॒व्या अधि॒ स्नुषु॑। प्रा॒ण॒दाः॥४॥

%4.6.1.5
अ॒पा॒न॒दा व्या॑न॒दाश्च॑क्षु॒र्दा व॑र्चो॒दा व॑रिवो॒दाः। अ॒न्यं ते॑ अ॒स्मत्त॑पन्तु हे॒तयः॑ पाव॒को अ॒स्मभ्यꣳ॑ शि॒वो भ॑व। अ॒ग्निस्ति॒ग्मेन॑ शो॒चिषा॒ यꣳस॒द्विश्वं॒ न्य॑त्रिणम्᳚। अ॒ग्निर्नो॑ वꣳसते र॒यिम्। सैनानी॑केन सुवि॒दत्रो॑ अ॒स्मे यष्टा॑ दे॒वाꣳ आय॑जिष्ठः स्व॒स्ति। अद॑ब्धो गो॒पा उ॒त नः॑ पर॒स्पा अग्ने᳚ द्यु॒मदु॒त रे॒वद्दि॑दीहि॥५॥

%4.6.2.0
{\anuvakamend[{उप॑ दे॒वान् वट्प्रा॑ण॒दाश्चतु॑श्चत्वारिꣳशच्च॥१॥}]}

%4.6.2.1
य इ॒मा विश्वा॒ भुव॑नानि॒ जुह्व॒दृषि॒र्\mbox{}होता॑ निष॒सादा॑ पि॒ता नः॑। स आ॒शिषा॒ द्रवि॑णमि॒च्छमा॑नः परम॒च्छदो॒ वर॒ आ वि॑वेश। वि॒श्वक॑र्मा॒ मन॑सा॒ यद्विहा॑या धा॒ता वि॑धा॒ता प॑र॒मोत सं॒दृक्। तेषा॑मि॒ष्टानि॒ समि॒षा म॑दन्ति॒ यत्र॑ सप्त॒र्\mbox{}षीन्प॒र एक॑मा॒हुः। यो नः॑ पि॒ता ज॑नि॒ता यो वि॑धा॒ता यो नः॑ स॒तो अ॒भ्या सज्ज॒जान॑।॥६॥

%4.6.2.2
यो दे॒वानां᳚ नाम॒धा एक॑ ए॒व तꣳ स॑म्प्र॒श्ञम्भुव॑ना यन्त्य॒न्या। त आय॑जन्त॒ द्रवि॑ण॒ꣳ॒ सम॑स्मा॒ ऋष॑यः॒ पूर्वे॑ जरि॒तारो॒ न भू॒ना। अ॒सूर्ता॒ सूर्ता॒ रज॑सो वि॒माने॒ ये भू॒तानि॑ स॒मकृ॑ण्वन्नि॒मानि॑। न तं वि॑दाथ॒ य इ॒दं ज॒जाना॒न्यद्यु॒ष्माक॒मन्त॑रम्भवाति। नी॒हा॒रेण॒ प्रावृ॑ता॒ जल्प्या॑ चासु॒तृप॑ उक्थ॒शास॑श्चरन्ति। प॒रो दि॒वा प॒र ए॒ना॥७॥

%4.6.2.3
पृ॒थि॒व्या प॒रो दे॒वेभि॒रसु॑रै॒र्गुहा॒ यत्। कꣴ स्वि॒द्गर्भं॑ प्रथ॒मं द॑ध्र॒ आपो॒ यत्र॑ दे॒वाः स॒मग॑च्छन्त॒ विश्वे᳚। तमिद्गर्भ॑म्प्रथ॒मं द॑ध्र॒ आपो॒ यत्र॑ दे॒वाः स॒मग॑च्छन्त॒ विश्वे᳚। अ॒जस्य॒ नाभा॒वध्येक॒मर्पि॑तं॒ यस्मि॑न्नि॒दं विश्व॒म्भुव॑न॒\-मधि॑ श्रि॒तम्। वि॒श्वक॑र्मा॒ ह्यज॑निष्ट दे॒व आदिद्ग॑न्ध॒र्वो अ॑भवद्द्वि॒तीयः॑। तृ॒तीयः॑ पि॒ता ज॑नि॒तौष॑धीनाम्॥८॥

%4.6.2.4
अ॒पां गर्भं॒ व्य॑दधात्पुरु॒त्रा। चक्षु॑षः पि॒ता मन॑सा॒ हि धीरो॑ घृ॒तमे॑ने अजन॒न्नन्न॑माने। य॒देदन्ता॒ अद॑दृꣳहन्त॒ पूर्व॒ आदिद्द्यावा॑पृथि॒वी अ॑प्रथेताम्। वि॒श्वत॑श्चक्षुरु॒त वि॒श्वतो॑मुखो वि॒श्वतो॑हस्त उ॒त वि॒श्वत॑स्पात्। सं बा॒हुभ्यां॒ नम॑ति॒ सम्पत॑त्रै॒र्द्यावा॑पृथि॒वी ज॒नयं॑ दे॒व एकः॑। किꣴ स्वि॑दासीदधि॒ष्ठान॑मा॒रम्भ॑णं कत॒मथ्स्वि॒त्किमा॑सीत्। यदी॒ भूमिं॑ ज॒नयन्न्॑॥९॥

%4.6.2.5
वि॒श्वक॑र्मा॒ वि द्यामौर्णो᳚न्महि॒ना वि॒श्वच॑क्षाः। किꣴ स्वि॒द्वनं॒ क उ॒ स वृ॒क्ष आ॑सी॒द्यतो॒ द्यावा॑पृथि॒वी नि॑ष्टत॒क्षुः। मनी॑षिणो॒ मन॑सा पृ॒च्छतेदु॒ तद्यद॒ध्यति॑ष्ठ॒द्भुव॑नानि धा॒रयन्न्॑। या ते॒ धामा॑नि पर॒माणि॒ याव॒मा या म॑ध्य॒मा वि॑श्वकर्मन्नु॒तेमा। शिक्षा॒ सखि॑भ्यो ह॒विषि॑ स्वधावः स्व॒यं य॑जस्व त॒नुवं॑ जुषा॒णः। वा॒चस्पतिं॑ वि॒श्वक॑र्माणमू॒तये᳚॥१०॥

%4.6.2.6
म॒नो॒युजं॒ वाजे॑ अ॒द्या हु॑वेम। स नो॒ नेदि॑ष्ठा॒ हव॑नानि जोषते वि॒श्वश॑म्भू॒रव॑से सा॒धुक॑र्मा। विश्व॑कर्मन् ह॒विषा॑ वावृधा॒नः स्व॒यं य॑जस्व त॒नुवं॑ जुषा॒णः। मुह्य॑न्त्व॒न्ये अ॒भितः॑ स॒पत्ना॑ इ॒हास्माक॑म्म॒घवा॑ सू॒रिर॑स्तु। विश्व॑कर्मन् ह॒विषा॒ वर्ध॑नेन त्रा॒तार॒मिन्द्र॑मकृणोरव॒ध्यम्। तस्मै॒ विशः॒ सम॑नमन्त पू॒र्वीर॒यमु॒ग्रो वि॑ह॒व्यो॑ यथास॑त्। स॒मु॒द्राय॑ व॒युना॑य॒ सिन्धू॑ना॒म्पत॑ये॒ नमः॑। न॒दीना॒ꣳ॒ सर्वा॑साम्पि॒त्रे जु॑हु॒ता वि॒श्वक॑र्मणे॒ विश्वाहाम॑र्त्यꣳ ह॒विः॥११॥

%4.6.3.0
{\anuvakamend[{ज॒जानै॒नौष॑धीनां॒ भूमिं॑ ज॒नय॑न्नू॒तये॒ नमो॒ नव॑ च॥२॥}]}

%4.6.3.1
उदे॑नमुत्त॒रां न॒याग्ने॑ घृतेनाहुत। रा॒यस्पोषे॑ण॒ सꣳ सृ॑ज प्र॒जया॑ च॒ धने॑न च। इन्द्रे॒मम्प्र॑त॒रां कृ॑धि सजा॒ताना॑मसद्व॒शी। समे॑नं॒ वर्च॑सा सृज दे॒वेभ्यो॑ भाग॒धा अ॑सत्। यस्य॑ कु॒र्मो ह॒विर्गृ॒हे तम॑ग्ने वर्धया॒ त्वम्। तस्मै॑ दे॒वा अधि॑ ब्रवन्न॒यं च॒ ब्रह्म॑ण॒स्पतिः॑। उदु॑ त्वा॒ विश्वे॑ दे॒वाः॥१२॥

%4.6.3.2
अग्ने॒ भर॑न्तु॒ चित्ति॑भिः। स नो॑ भव शि॒वत॑मः सु॒प्रती॑को वि॒भाव॑सुः। पञ्च॒ दिशो॒ दैवी᳚र्य॒ज्ञम॑वन्तु दे॒वीरपाम॑तिं दुर्म॒तिम्बाध॑मानाः। रा॒यस्पोषे॑ य॒ज्ञप॑तिमा॒भज॑न्तीः। रा॒यस्पोषे॒ अधि॑ य॒ज्ञो अ॑स्था॒थ्समि॑द्धे अ॒ग्नावधि॑ मामहा॒नः। उ॒क्थप॑त्त्र॒ ईड्यो॑ गृभी॒तस्त॒प्तं घ॒र्मम्प॑रि॒गृह्या॑यजन्त। ऊ॒र्जा यद्य॒ज्ञमश॑मन्त दे॒वा दैव्या॑य ध॒र्त्रे जोष्ट्रे᳚। दे॒व॒श्रीः श्रीम॑णाः श॒तप॑याः॥१३॥

%4.6.3.3
प॒रि॒गृह्य॑ दे॒वा य॒ज्ञमा॑यन्न्। सूर्य॑रश्मि॒र्\mbox{}हरि॑केशः पु॒रस्ता᳚थ्सवि॒ता ज्योति॒रुद॑या॒ꣳ॒ अज॑स्रम्। तस्य॑ पू॒षा प्र॑स॒वं या॑ति दे॒वः स॒म्पश्य॒न्विश्वा॒ भुव॑नानि गो॒पाः। दे॒वा दे॒वेभ्यो॑ अध्व॒र्यन्तो॑ अस्थुर्वी॒तꣳ श॑मि॒त्रे श॑मि॒ता य॒जध्यै᳚। तु॒रीयो॑ य॒ज्ञो यत्र॑ ह॒व्यमेति॒ ततः॑ पाव॒का आ॒शिषो॑ नो जुषन्ताम्। वि॒मान॑ ए॒ष दि॒वो मध्य॑ आस्त आपप्रि॒वान्रोद॑सी अ॒न्तरि॑क्षम्। स वि॒श्वाची॑र॒भि॥१४॥

%4.6.3.4
च॒ष्टे॒ घृ॒ताची॑रन्त॒रा पूर्व॒मप॑रं च के॒तुम्। उ॒क्षा स॑मु॒द्रो अ॑रु॒णः सु॑प॒र्णः पूर्व॑स्य॒ योनि॑म्पि॒तुरा वि॑वेश। मध्ये॑ दि॒वो निहि॑तः॒ पृश्ञि॒रश्मा॒ वि च॑क्रमे॒ रज॑सः पा॒त्यन्तौ᳚। इन्द्रं॒ विश्वा॑ अवीवृधन्थ्समु॒द्रव्य॑चसं॒ गिरः॑। र॒थीत॑मꣳ रथी॒नां वाजा॑ना॒ꣳ॒ सत्प॑ति॒म्पतिम्᳚। सु॒म्न॒हूर्य॒ज्ञो दे॒वाꣳ आ च॑ वक्ष॒द्यक्ष॑द॒ग्निर्दे॒वो दे॒वाꣳ आ च॑ वक्षत्। वाज॑स्य मा प्रस॒वेनो᳚द्ग्रा॒भेणोद॑ग्रभीत्। अथा॑ स॒पत्ना॒ꣳ॒ इन्द्रो॑ मे निग्रा॒भेणाध॑राꣳ अकः। उ॒द्ग्रा॒भं च॑ निग्रा॒भं च॒ ब्रह्म॑ दे॒वा अ॑वीवृधन्न्। अथा॑ स॒पत्ना॑निन्द्रा॒ग्नी मे॑ विषू॒चीना॒न्व्य॑स्यताम्॥१५॥

%4.6.4.0
{\anuvakamend[{दे॒वाः श॒तप॑या अ॒भि वाज॑स्य॒ षड्विꣳ॑शतिश्च॥३॥}]}

%4.6.4.1
आ॒शुः शिशा॑नो वृष॒भो न यु॒ध्मो घ॑नाघ॒नः क्षोभ॑णश्चर्\mbox{}षणी॒नाम्। स॒ङ्क्रन्द॑नो\-ऽनिमि॒ष ए॑कवी॒रः श॒तꣳ सेना॑ अजयथ्सा॒कमिन्द्रः॑। सं॒क्रन्द॑नेनानिमि॒षेण॑ जि॒ष्णुना॑ युत्का॒रेण॑ दुश्च्यव॒नेन॑ धृ॒ष्णुना᳚। तदिन्द्रे॑ण जयत॒ तथ्स॑हध्वं॒ युधो॑ नर॒ इषु॑हस्तेन॒ वृष्णा᳚। स इषु॑हस्तैः॒ स नि॑ष॒ङ्गिभि॑र्व॒शी सꣴस्र॑ष्टा॒ स युध॒ इन्द्रो॑ ग॒णेन। स॒ꣳ॒सृ॒ष्ट॒जिथ्सो॑म॒पा बा॑हुश॒र्ध्यू᳚र्ध्वध॑न्वा॒ प्रति॑हिताभि॒रस्ता᳚। बृह॑स्पते॒ परि॑ दीय॥१६॥

%4.6.4.2
रथे॑न रक्षो॒हामित्राꣳ॑ अप॒बाध॑मानः। प्र॒भ॒ञ्जन्थ्सेनाः᳚ प्रमृ॒णो यु॒धा जय॑न्न॒स्माक॑मेध्यवि॒ता रथा॑नाम्। गो॒त्र॒भिदं॑ गो॒विदं॒ वज्र॑बाहुं॒ जय॑न्त॒मज्म॑ प्रमृ॒णन्त॒मोज॑सा। इ॒मꣳ स॑जाता॒ अनु॑ वीरयध्व॒मिन्द्रꣳ॑ सखा॒यो\-ऽनु॒ सꣳ र॑भध्वम्। ब॒ल॒वि॒ज्ञा॒यः स्थवि॑रः॒ प्रवी॑रः॒ सह॑स्वान् वा॒जी सह॑मान उ॒ग्रः। अ॒भिवी॑रो अ॒भिस॑त्वा सहो॒जा जैत्र॑मिन्द्र॒ रथ॒मा ति॑ष्ठ गो॒वित्। अ॒भि गो॒त्राणि॒ सह॑सा॒ गाह॑मानो\-ऽदा॒यः॥१७॥

%4.6.4.3
वी॒रः श॒तम॑न्यु॒रिन्द्रः॑। दु॒श्च्य॒व॒नः पृ॑तना॒षाड॑यु॒ध्यो᳚\-ऽस्माक॒ꣳ॒ सेना॑ अवतु॒ प्र यु॒थ्सु। इन्द्र॑ आसां ने॒ता बृह॒स्पति॒र्दक्षि॑णा य॒ज्ञः पु॒र ए॑तु॒ सोमः॑। दे॒व॒से॒नाना॑मभिभञ्जती॒नां जय॑न्तीनाम्म॒रुतो॑ य॒न्त्वग्रे᳚। इन्द्र॑स्य॒ वृष्णो॒ वरु॑णस्य॒ राज्ञ॑ आदि॒त्याना᳚म्म॒रुता॒ꣳ॒ शर्ध॑ उ॒ग्रम्। म॒हाम॑नसाम्भुवनच्य॒वानां॒ घोषो॑ दे॒वानां॒ जय॑ता॒मुद॑स्थात्। अ॒स्माक॒मिन्द्रः॒ समृ॑तेषु ध्व॒जेष्व॒स्माकं॒ या इष॑व॒स्ता ज॑यन्तु।॥१८॥

%4.6.4.4
अ॒स्माकं॑ वी॒रा उत्त॑रे भवन्त्व॒स्मानु॑ देवा अवता॒ हवे॑षु। उद्ध॑र्\mbox{}षय मघव॒न्नायु॑धा॒न्युथ्सत्व॑नाम्माम॒काना॒म्महाꣳ॑सि। उद्वृ॑त्रहन्वा॒जिनां॒ वाजि॑ना॒न्युद्रथा॑नां॒ जय॑तामेतु॒ घोषः॑। उप॒ प्रेत॒ जय॑ता नरः स्थि॒रा वः॑ सन्तु बा॒हवः॑। इन्द्रो॑ वः॒ शर्म॑ यच्छत्वनाधृ॒ष्या यथास॑थ। अव॑सृष्टा॒ परा॑ पत॒ शर॑व्ये॒ ब्रह्म॑सꣳशिता। गच्छा॒मित्रा॒न्प्र॥१९॥

%4.6.4.5
वि॒श॒ मैषां॒ कं च॒नोच्छि॑षः। मर्मा॑णि ते॒ वर्म॑भिश्छादयामि॒ सोम॑स्त्वा॒ राजा॒मृते॑ना॒भि व॑स्ताम्। उ॒रोर्वरी॑यो॒ वरि॑वस्ते अस्तु॒ जय॑न्तं॒ त्वामनु॑ मदन्तु दे॒वाः। यत्र॑ बा॒णाः स॒म्पत॑न्ति कुमा॒रा वि॑शि॒खा इ॑व। इन्द्रो॑ न॒स्तत्र॑ वृत्र॒हा वि॑श्वा॒हा शर्म॑ यच्छतु॥२०॥

%4.6.5.0
{\anuvakamend[{दी॒या॒ दा॒यो ज॑यन्त्व॒मित्रा॒न्प्र च॑त्वारि॒ꣳ॒शच्च॑॥४॥}]}

%4.6.5.1
प्राची॒मनु॑ प्र॒दिश॒म्प्रेहि॑ वि॒द्वान॒ग्नेर॑ग्ने पु॒रो अ॑ग्निर्भवे॒ह। विश्वा॒ आशा॒ दीद्या॑नो॒ वि भा॒ह्यूर्जं॑ नो धेहि द्वि॒पदे॒ चतु॑ष्पदे। क्रम॑ध्वम॒ग्निना॒ नाक॒मुख्य॒ꣳ॒ हस्ते॑षु॒ बिभ्र॑तः। दि॒वः पृ॒ष्ठꣳ सुव॑र्ग॒त्वा मि॒श्रा दे॒वेभि॑राद्ध्वम्। पृ॒थि॒व्या अ॒हमुद॒न्तरि॑क्ष॒मारु॑हम॒न्तरि॑क्षा॒द्दिव॒मारु॑हम्। दि॒वो नाक॑स्य पृ॒ष्ठाथ्सुव॒र्ज्योति॑रगाम्॥२१॥

%4.6.5.2
अ॒हम्। सुव॒र्यन्तो॒ नापे᳚क्षन्त॒ आ द्याꣳ रो॑हन्ति॒ रोद॑सी। य॒ज्ञं ये वि॒श्वतो॑धार॒ꣳ॒ सुवि॑द्वाꣳसो वितेनि॒रे। अग्ने॒ प्रेहि॑ प्रथ॒मो दे॑वय॒तां चक्षु॑र्दे॒वाना॑मु॒त मर्त्या॑नाम्। इय॑क्षमाणा॒ भृगु॑भिः स॒जोषाः॒ सुव॑र्यन्तु॒ यज॑मानाः स्व॒स्ति। नक्तो॒षासा॒ सम॑नसा॒ विरू॑पे धा॒पये॑ते॒ शिशु॒मेकꣳ॑ समी॒ची। द्यावा॒ क्षामा॑ रु॒क्मो अ॒न्तर्विभा॑ति दे॒वा अ॒ग्निं धा॑रयन्द्रविणो॒दाः। अग्ने॑ सहस्राक्ष॥२२॥

%4.6.5.3
श॒त॒मू॒र्ध॒ञ्छ॒तं ते᳚ प्रा॒णाः स॒हस्र॑मपा॒नाः। त्वꣳ सा॑ह॒स्रस्य॑ रा॒य ई॑शिषे॒ तस्मै॑ ते विधेम॒ वाजा॑य॒ स्वाहा᳚। सु॒प॒र्णो॑\-ऽसि ग॒रुत्मा᳚न्पृथि॒व्याꣳ सी॑द पृ॒ष्ठे पृ॑थि॒व्याः सी॑द भा॒सान्तरि॑क्ष॒मा पृ॑ण॒ ज्योति॑षा॒ दिव॒मुत्त॑भान॒ तेज॑सा॒ दिश॒ उद्दृꣳ॑ह। आ॒जुह्वा॑नः सु॒प्रती॑कः पु॒रस्ता॒दग्ने॒ स्वां योनि॒मा सी॑द सा॒ध्या। अ॒स्मिन्थ्स॒धस्थे॒ अध्युत्त॑रस्मि॒न्विश्वे॑ देवाः॥२३॥

%4.6.5.4
यज॑मानश्च सीदत। प्रेद्धो॑ अग्ने दीदिहि पु॒रो नो\-ऽज॑स्रया सू॒र्म्या॑ यविष्ठ। त्वाꣳ शश्व॑न्त॒ उप॑ यन्ति॒ वाजाः᳚। वि॒धेम॑ ते पर॒मे जन्म॑न्नग्ने वि॒धेम॒ स्तोमै॒रव॑रे स॒धस्थे᳚। यस्मा॒द्योने॑रु॒दारि॑था॒ यजे॒ तम्प्र त्वे ह॒वीꣳषि॑ जुहुरे॒ समि॑द्धे। ताꣳ स॑वि॒तुर्वरे᳚ण्यस्य चि॒त्रामाहं वृ॑णे सुम॒तिं वि॒श्वज॑न्याम्। याम॑स्य॒ कण्वो॒ अदु॑ह॒त्प्रपी॑नाꣳ स॒हस्र॑धाराम्॥२४॥

%4.6.5.5
पय॑सा म॒हीं गाम्। स॒प्त ते॑ अग्ने स॒मिधः॑ स॒प्त जि॒ह्वाः स॒प्तर्\mbox{}ष॑यः स॒प्त धाम॑ प्रि॒याणि॑। स॒प्त होत्राः᳚ सप्त॒धा त्वा॑ यजन्ति स॒प्त योनी॒रा पृ॑णस्वा घृ॒तेन॑। ई॒दृङ्चा᳚न्या॒दृङ्चै॑ता॒दृङ्च॑ प्रति॒दृङ्च॑ मि॒तश्च॒ सम्मि॑तश्च॒ सभ॑राः। शु॒क्रज्यो॑तिश्च चि॒त्रज्यो॑तिश्च स॒त्यज्यो॑तिश्च॒ ज्योति॑ष्माꣳश्च स॒त्यश्च॑र्त॒पाश्चात्यꣳ॑हाः।॥२५॥

%4.6.5.6
ऋ॒त॒जिच्च॑ सत्य॒जिच्च॑ सेन॒जिच्च॑ सु॒षेण॒श्चान्त्य॑मित्रश्च दू॒रेअ॑मित्रश्च ग॒णः। ऋ॒तश्च॑ स॒त्यश्च॑ ध्रु॒वश्च॑ ध॒रुण॑श्च ध॒र्ता च॑ विध॒र्ता च॑ विधार॒यः। ई॒दृक्षा॑स एता॒दृक्षा॑स ऊ॒ षु णः॑ स॒दृक्षा॑सः॒ प्रति॑सदृक्षास॒ एत॑न। मि॒तास॑श्च॒ सम्मि॑तासश्च न ऊ॒तये॒ सभ॑रसो मरुतो य॒ज्ञे अ॒स्मिन्निन्द्रं॒ दैवी॒र्विशो॑ म॒रुतो\-ऽनु॑वर्त्मानो॒ यथेन्द्रं॒ दैवी॒र्विशो॑ म॒रुतो\-ऽनु॑वर्त्मान ए॒वमि॒मं यज॑मानं॒ दैवी᳚श्च॒ विशो॒ मानु॑षी॒श्चानु॑वर्त्मानो भवन्तु॥२६॥

%4.6.6.0
{\anuvakamend[{अ॒गा॒ꣳ स॒ह॒स्रा॒क्ष॒ दे॒वाः॒ स॒हस्र॑धारा॒मत्यꣳ॑हा॒ अनु॑वर्त्मानः॒ षोड॑श च॥५॥}]}

%4.6.6.1
जी॒मूत॑स्येव भवति॒ प्रती॑कं॒ यद्व॒र्मी याति॑ स॒मदा॑मु॒पस्थे᳚। अना॑विद्धया त॒नुवा॑ जय॒ त्वꣳ स त्वा॒ वर्म॑णो महि॒मा पि॑पर्तु। धन्व॑ना॒ गा धन्व॑ना॒जिं ज॑येम॒ धन्व॑ना ती॒व्राः स॒मदो॑ जयेम। धनुः॒ शत्रो॑रपका॒मं कृ॑णोति॒ धन्व॑ना॒ सर्वाः᳚ प्र॒दिशो॑ जयेम। व॒क्ष्यन्ती॒वेदा ग॑नीगन्ति॒ कर्ण॑म्प्रि॒यꣳ सखा॑यम्परिषस्वजा॒ना। योषे॑व शिङ्क्ते॒ वित॒ताधि॒ धन्वन्न्॑॥२७॥

%4.6.6.2
ज्या इ॒यꣳ सम॑ने पा॒रय॑न्ती। ते आ॒चर॑न्ती॒ सम॑नेव॒ योषा॑ मा॒तेव॑ पु॒त्रम्बि॑भृतामु॒पस्थे᳚। अप॒ शत्र न््॑विध्यताꣳ संविदा॒ने आर्त्नी॑ इ॒मे वि॑ष्फु॒रन्ती॑ अ॒मित्रान्॑। ब॒ह्वी॒नाम्पि॒ता ब॒हुर॑स्य पु॒त्रश्चि॒श्चा कृ॑णोति॒ सम॑नाव॒गत्य॑। इ॒षु॒धिः सङ्काः॒ पृत॑नाश्च॒ सर्वाः᳚ पृ॒ष्ठे निन॑द्धो जयति॒ प्रसू॑तः। रथे॒ तिष्ठ॑न्नयति वा॒जिनः॑ पु॒रो यत्र॑यत्र का॒मय॑ते सुषार॒थिः। अ॒भीशू॑नाम्महि॒मानम्᳚॥२८॥

%4.6.6.3
प॒ना॒य॒त॒ मनः॑ प॒श्चादनु॑ यच्छन्ति र॒श्मयः॑। ती॒व्रान्घोषा᳚न्कृण्वते॒ वृष॑पाण॒यो\-ऽश्वा॒ रथे॑भिः स॒ह वा॒जय॑न्तः। अ॒व॒क्राम॑न्तः॒ प्रप॑दैर॒मित्रा᳚न्क्षि॒णन्ति॒ शत्रू॒ꣳ॒रन॑पव्ययन्तः। र॒थ॒वाह॑नꣳ ह॒विर॑स्य॒ नाम॒ यत्रायु॑धं॒ निहि॑तमस्य॒ वर्म॑। तत्रा॒ रथ॒मुप॑ श॒ग्मꣳ स॑देम वि॒श्वाहा॑ व॒यꣳ सु॑मन॒स्यमा॑नाः। स्वा॒दु॒ष॒ꣳ॒सदः॑ पि॒तरो॑ वयो॒धाः कृ॑च्छ्रे॒श्रितः॒ शक्ती॑वन्तो गभी॒राः। चि॒त्रसे॑ना॒ इषु॑बला॒ अमृ॑ध्राः स॒तोवी॑रा उ॒रवो᳚ व्रातसा॒हाः। ब्राह्म॑णासः॥२९॥

%4.6.6.4
पित॑रः॒ सोम्या॑सः शि॒वे नो॒ द्यावा॑पृथि॒वी अ॑ने॒हसा᳚। पू॒षा नः॑ पातु दुरि॒तादृ॑तावृधो॒ रक्षा॒ माकि॑र्नो अ॒घशꣳ॑स ईशत। सु॒प॒र्णं व॑स्ते मृ॒गो अ॑स्या॒ दन्तो॒ गोभिः॒ संन॑द्धा पतति॒ प्रसू॑ता। यत्रा॒ नरः॒ सं च॒ वि च॒ द्रव॑न्ति॒ तत्रा॒स्मभ्य॒मिष॑वः॒ शर्म॑ यꣳसन्न्। ऋजी॑ते॒ परि॑ वृङ्ग्धि॒ नो\-ऽश्मा॑ भवतु नस्त॒नूः। सोमो॒ अधि॑ ब्रवीतु॒ नो\-ऽदि॑तिः॥३०॥

%4.6.6.5
शर्म॑ यच्छतु। आ ज॑ङ्घन्ति॒ सान्वे॑षां ज॒घना॒ꣳ॒ उप॑ जिघ्नते। अश्वा॑जनि॒ प्रचे॑त॒सो\-ऽश्वा᳚न्थ्स॒मथ्सु॑ चोदय। अहि॑रिव भो॒गैः पर्ये॑ति बा॒हुं ज्याया॑ हे॒तिम्प॑रि॒बाध॑मानः। ह॒स्त॒घ्नो विश्वा॑ व॒युना॑नि वि॒द्वान्पुमा॒न्पुमाꣳ॑स॒म्परि॑ पातु वि॒श्वतः॑। वन॑स्पते वी॒ड्व॑ङ्गो॒ हि भू॒या अ॒स्मथ्स॑खा प्र॒तर॑णः सु॒वीरः॑। गोभिः॒ संन॑द्धो असि वी॒डय॑स्वास्था॒ता ते॑ जयतु॒ जेत्वा॑नि। दि॒वः पृ॑थि॒व्याः परि॑॥३१॥

%4.6.6.6
ओज॒ उद्भृ॑तं॒ वन॒स्पति॑भ्यः॒ पर्याभृ॑त॒ꣳ॒ सहः॑। अ॒पामो॒ज्मान॒म्परि॒ गोभि॒रावृ॑त॒मिन्द्र॑स्य॒ वज्रꣳ॑ ह॒विषा॒ रथं॑ यज। इन्द्र॑स्य॒ वज्रो॑ म॒रुता॒मनी॑कम्मि॒त्रस्य॒ गर्भो॒ वरु॑णस्य॒ नाभिः॑। सेमां नो॑ ह॒व्यदा॑तिं जुषा॒णो देव॑ रथ॒ प्रति॑ ह॒व्या गृ॑भाय। उप॑ श्वासय पृथि॒वीमु॒त द्याम्पु॑रु॒त्रा ते॑ मनुतां॒ विष्ठि॑तं॒ जग॑त्। स दु॑न्दुभे स॒जूरिन्द्रे॑ण दे॒वैर्दू॒रात्॥३२॥

%4.6.6.7
दवी॑यो॒ अप॑ सेध॒ शत्रून्॑। आ क्र॑न्दय॒ बल॒मोजो॑ न॒ आ धा॒ नि ष्ट॑निहि दुरि॒ता बाध॑मानः। अप॑ प्रोथ दुन्दुभे दु॒च्छुनाꣳ॑ इ॒त इन्द्र॑स्य मु॒ष्टिर॑सि वी॒डय॑स्व। आमूर॑ज प्र॒त्याव॑र्तये॒माः के॑तु॒मद्दु॑न्दु॒भिर्वा॑वदीति। समश्व॑पर्णा॒श्चर॑न्ति नो॒ नरो॒\-ऽस्माक॑मिन्द्र र॒थिनो॑ जयन्तु॥३३॥

%4.6.7.0
{\anuvakamend[{धन्व॑न्महि॒मानं॒ ब्राह्म॑णा॒सो\-ऽदि॑तिः पृथि॒व्याः परि॑ दू॒रादेक॑चत्वारिꣳशच्च॥६॥}]}

%4.6.7.1
यदक्र॑न्दः प्रथ॒मं जाय॑मान उ॒द्यन्थ्स॑मु॒द्रादु॒त वा॒ पुरी॑षात्। श्ये॒नस्य॑ प॒क्षा ह॑रि॒णस्य॑ बा॒हू उ॑प॒स्तुत्य॒म्महि॑ जा॒तं ते॑ अर्वन्न्। य॒मेन॑ द॒त्तं त्रि॒त ए॑नमायुन॒गिन्द्र॑ एणम्प्रथ॒मो अध्य॑तिष्ठत्। ग॒न्ध॒र्वो अ॑स्य रश॒नाम॑गृभ्णा॒थ्सूरा॒दश्वं॑ वसवो॒ निर॑तष्ट। असि॑ य॒मो अस्या॑दि॒त्यो अ॑र्व॒न्नसि॑ त्रि॒तो गुह्ये॑न व्र॒तेन॑। असि॒ सोमे॑न स॒मया॒ विपृ॑क्तः॥३४॥

%4.6.7.2
आ॒हुस्ते॒ त्रीणि॑ दि॒वि बन्ध॑नानि। त्रीणि॑ त आहुर्दि॒वि बन्ध॑नानि॒ त्रीण्य॒प्सु त्रीण्य॒न्तः स॑मु॒द्रे। उ॒तेव॑ मे॒ वरु॑णश्छन्थ्स्यर्व॒न् यत्रा॑ त आ॒हुः प॑र॒मं ज॒नित्रम्᳚। इ॒मा ते॑ वाजिन्नव॒मार्ज॑नानी॒मा श॒फानाꣳ॑ सनि॒तुर्नि॒धाना᳚। अत्रा॑ ते भ॒द्रा र॑श॒ना अ॑पश्यमृ॒तस्य॒ या अ॑भि॒रक्ष॑न्ति गो॒पाः। आ॒त्मानं॑ ते॒ मन॑सा॒राद॑जानाम॒वो दि॒वा॥३५॥

%4.6.7.3
प॒तय॑न्तम्पतं॒गम्। शिरो॑ अपश्यम्प॒थिभिः॑ सु॒गेभि॑ररे॒णुभि॒र्जेह॑मानम्पत॒त्रि। अत्रा॑ ते रू॒पमु॑त्त॒मम॑पश्यं॒ जिगी॑षमाणमि॒ष आ प॒दे गोः। य॒दा ते॒ मर्तो॒ अनु॒ भोग॒मान॒डादिद्ग्रसि॑ष्ठ॒ ओष॑धीरजीगः। अनु॑ त्वा॒ रथो॒ अनु॒ मर्यो॑ अर्व॒न्ननु॒ गावो\-ऽनु॒ भगः॑ क॒नीनाम्᳚। अनु॒ व्राता॑स॒स्तव॑ स॒ख्यमी॑यु॒रनु॑ दे॒वा म॑मिरे वी॒र्यम्᳚॥३६॥

%4.6.7.4
ते॒। हिर॑ण्यशृ॒ङ्गो\-ऽयो॑ अस्य॒ पादा॒ मनो॑जवा॒ अव॑र॒ इन्द्र॑ आसीत्। दे॒वा इद॑स्य हवि॒रद्य॑माय॒न् यो अर्व॑न्तम्प्रथ॒मो अ॒ध्यति॑ष्ठत्। ई॒र्मान्ता॑सः॒ सिलि॑कमध्यमासः॒ सꣳ शूर॑णासो दि॒व्यासो॒ अत्याः᳚। ह॒ꣳ॒सा इ॑व श्रेणि॒शो य॑तन्ते॒ यदाक्षि॑षुर्दि॒व्यमज्म॒मश्वाः᳚। तव॒ शरी॑रम्पतयि॒ष्ण्व॑र्व॒न्तव॑ चि॒त्तं वात॑ इव॒ ध्रजी॑मान्। तव॒ शृङ्गा॑णि॒ विष्ठि॑ता पुरु॒त्रार॑ण्येषु॒ जर्भु॑राणा चरन्ति। उप॑॥३७॥

%4.6.7.5
प्रागा॒च्छस॑नं वा॒ज्यर्वा॑ देव॒द्रीचा॒ मन॑सा॒ दीध्या॑नः। अ॒जः पु॒रो नी॑यते॒ नाभि॑र॒स्यानु॑ प॒श्चात्क॒वयो॑ यन्ति रे॒भाः। उप॒ प्रागा᳚त्पर॒मं यथ्स॒धस्थ॒मर्वा॒ꣳ॒ अच्छा॑ पि॒तर॑म्मा॒तरं॑ च। अ॒द्या दे॒वां जुष्ट॑तमो॒ हि ग॒म्या अथा शा᳚स्ते दा॒शुषे॒ वार्या॑णि॥३८॥

%4.6.8.0
{\anuvakamend[{विपृ॑क्तो दि॒वा वी॒र्य॑मुपैका॒न्नच॑त्वारि॒ꣳ॒शच्च॑॥७॥}]}

%4.6.8.1
मा नो॑ मि॒त्रो वरु॑णो अर्य॒मायुरिन्द्र॑ ऋभु॒क्षा म॒रुतः॒ परि॑ ख्यन्न्। यद्वा॒जिनो॑ दे॒वजा॑तस्य॒ सप्तेः᳚ प्रव॒क्ष्यामो॑ वि॒दथे॑ वी॒र्या॑णि। यन्नि॒र्णिजा॒ रेक्ण॑सा॒ प्रावृ॑तस्य रा॒तिं गृ॑भी॒ताम्मु॑ख॒तो नय॑न्ति। सुप्रा॑ङ॒जो मेम्य॑द्वि॒श्वरू॑प इन्द्रापू॒ष्णोः प्रि॒यमप्ये॑ति॒ पाथः॑। ए॒ष च्छागः॑ पु॒रो अश्वे॑न वा॒जिना॑ पू॒ष्णो भा॒गो नी॑यते वि॒श्वदे᳚व्यः। अ॒भि॒प्रियं॒ यत्पु॑रो॒डाश॒मर्व॑ता॒ त्वष्टेत्॥३९॥

%4.6.8.2
ए॒न॒ꣳ॒ सौ॒श्र॒व॒साय॑ जिन्वति। यद्ध॒विष्य॑मृतु॒शो दे॑व॒यानं॒ त्रिर्मानु॑षाः॒ पर्यश्वं॒ नय॑न्ति। अत्रा॑ पू॒ष्णः प्र॑थ॒मो भा॒ग ए॑ति य॒ज्ञं दे॒वेभ्यः॑ प्रतिवे॒दय॑न्न॒जः। होता᳚ध्व॒र्युराव॑या अग्निमि॒न्धो ग्रा॑वग्रा॒भ उ॒त शꣴस्ता॒ सुवि॑प्रः। तेन॑ य॒ज्ञेन॑ स्व॑रंकृतेन॒ स्वि॑ष्टेन व॒क्षणा॒ आ पृ॑णध्वम्। यू॒प॒व्र॒स्का उ॒त ये यू॑पवा॒हाश्च॒षालं॒ ये अ॑श्वयू॒पाय॒ तक्ष॑ति। ये चार्व॑ते॒ पच॑नꣳ स॒म्भर॑न्त्यु॒तो॥४०॥

%4.6.8.3
तेषा॑म॒भिगू᳚र्तिर्न इन्वतु। उप॒ प्रागा᳚थ्सु॒मन्मे॑\-ऽधायि॒ मन्म॑ दे॒वाना॒माशा॒ उप॑ वी॒तपृ॑ष्ठः। अन्वे॑नं॒ विप्रा॒ ऋष॑यो मदन्ति दे॒वानां᳚ पु॒ष्टे च॑कृमा सु॒बन्धुम्᳚। यद्वा॒जिनो॒ दाम॑ सं॒दान॒मर्व॑तो॒ या शी॑र्\mbox{}ष॒ण्या॑ रश॒ना रज्जु॑रस्य। यद्वा॑ घास्य॒ प्रभृ॑तमा॒स्ये॑ तृण॒ꣳ॒ सर्वा॒ ता ते॒ अपि॑ दे॒वेष्व॑स्तु। यदश्व॑स्य क्र॒विषः॑॥४१॥

%4.6.8.4
मक्षि॒काश॒ यद्वा॒ स्वरौ॒ स्वधि॑तौ रि॒प्तमस्ति॑। यद्धस्त॑योः शमि॒तुर्यन्न॒खेषु॒ सर्वा॒ ता ते॒ अपि॑ दे॒वेष्व॑स्तु। यदूव॑ध्यमु॒दर॑स्याप॒वाति॒ य आ॒मस्य॑ क्र॒विषो॑ ग॒न्धो अस्ति॑। सु॒कृ॒ता तच्छ॑मि॒तारः॑ कृण्वन्तू॒त मेधꣳ॑ शृत॒पाकं॑ पचन्तु। यत्ते॒ गात्रा॑द॒ग्निना॑ प॒च्यमा॑नाद॒भि शूलं॒ निह॑तस्याव॒धाव॑ति। मा तद्भूम्या॒मा श्रि॑ष॒न्मा तृणे॑षु दे॒वेभ्य॒स्तदु॒शद्भ्यो॑ रा॒तम॑स्तु॥४२॥

%4.6.9.0
{\anuvakamend[{इदु॒तो क्र॒विषः॑ श्रिषथ्स॒प्त च॑॥८॥}]}

%4.6.9.1
ये वा॒जिन॑म्परि॒पश्य॑न्ति प॒क्वं य ई॑मा॒हुः सु॑र॒भिर्निर्\mbox{}ह॒रेति॑। ये चार्व॑तो माꣳसभि॒क्षामु॒पास॑त उ॒तो तेषा॑म॒भिगू᳚र्तिर्न इन्वतु। यन्नीक्ष॑णम्मा॒ꣳ॒स्पच॑न्या उ॒खाया॒ या पात्रा॑णि यू॒ष्ण आ॒सेच॑नानि। ऊ॒ष्म॒ण्या॑पि॒धाना॑ चरू॒णाम॒ङ्काः सू॒नाः परि॑ भूष॒न्त्यश्वम्᳚। नि॒क्रम॑णं नि॒षद॑नं वि॒वर्त॑नं॒ यच्च॒ पड्बी॑श॒मर्व॑तः। यच्च॑ प॒पौ यच्च॑ घा॒सिम्॥४३॥

%4.6.9.2
ज॒घास॒ सर्वा॒ ता ते॒ अपि॑ दे॒वेष्व॑स्तु। मा त्वा॒ग्निर्ध्व॑नयिद्धू॒मग॑न्धि॒र्मोखा भ्राज॑न्त्य॒भि वि॑क्त॒ जघ्रिः॑। इ॒ष्टं वी॒तम॒भिगू᳚र्तं॒ वष॑ट्कृतं॒ तं दे॒वासः॒ प्रति॑ गृभ्ण॒न्त्यश्वम्᳚। यदश्वा॑य॒ वास॑ उपस्तृ॒णन्त्य॑धीवा॒सं या हिर॑ण्यान्यस्मै। सं॒दान॒मर्व॑न्त॒म्पड्बी॑शम्प्रि॒या दे॒वेष्वा या॑मयन्ति। यत्ते॑ सा॒दे मह॑सा॒ शूकृ॑तस्य॒ पार्ष्णि॑या वा॒ कश॑या॥४४॥

%4.6.9.3
वा॒ तु॒तोद॑। स्रु॒चेव॒ ता ह॒विषो॑ अध्व॒रेषु॒ सर्वा॒ ता ते॒ ब्रह्म॑णा सूदयामि। चतु॑स्त्रिꣳशद्वा॒जिनो॑ दे॒वब॑न्धो॒र्वङ्क्री॒रश्व॑स्य॒ स्वधि॑तिः॒ समे॑ति। अच्छि॑द्रा॒ गात्रा॑ व॒युना॑ कृणोत॒ परु॑ष्परुरनु॒घुष्या॒ वि श॑स्त। एक॒स्त्वष्टु॒रश्व॑स्या विश॒स्ता द्वा य॒न्तारा॑ भवत॒स्तथ॒र्तुः। या ते॒ गात्रा॑णामृतु॒था कृ॒णोमि॒ ताता॒ पिण्डा॑ना॒म्प्र जु॑होम्य॒ग्नौ। मा त्वा॑ तपत्॥४५॥

%4.6.9.4
प्रि॒य आ॒त्मापि॒यन्तं॒ मा स्वधि॑तिस्त॒नुव॒ आ ति॑ष्ठिपत्ते। मा ते॑ गृ॒ध्नुर॑विश॒स्ताति॒हाय॑ छि॒द्रा गात्रा॑ण्य॒सिना॒ मिथू॑ कः। न वा उ॑ वे॒तन्म्रि॑यसे॒ न रि॑ष्यसि दे॒वाꣳ इदे॑षि प॒थिभिः॑ सु॒गेभिः॑। हरी॑ ते॒ युञ्जा॒ पृष॑ती अभूता॒मुपा᳚स्थाद्वा॒जी धु॒रि रास॑भस्य। सु॒गव्यं॑ नो वा॒जी स्वश्वि॑यम्पु॒ꣳ॒सः पु॒त्राꣳ उ॒त वि॑श्वा॒पुषꣳ॑ र॒यिम्। अ॒ना॒गा॒स्त्वं नो॒ अदि॑तिः कृणोतु क्ष॒त्रं नो॒ अश्वो॑ वनताꣳ ह॒विष्मान्॑॥४६॥

%4.7.0.0

{\anuvakamend[{घा॒सिं कश॑या तपद्र॒यिं नव॑ च॥९॥}]}
%%% END PRASHNA
