\sect{द्वितीयः प्रश्नः}\setcounter{anuvakam}{0}
\dnsub{तैत्तिरीयसंहितायां सप्तमकाण्डे द्वितीयः प्रश्नः}
%7.2.1.0
%7.2.1.1
सा॒ध्या वै दे॒वाः सु॑व॒र्गका॑मा ए॒तꣳ ष॑ड्रा॒त्रम॑पश्य॒न्तमाह॑र॒न्तेना॑यजन्त॒ ततो॒ वै ते सु॑व॒र्गं लो॒कमा॑य॒न् य ए॒वं वि॒द्वाꣳसः॑ षड्रा॒त्रमास॑ते सुव॒र्गमे॒व लो॒कं य॑न्ति देवस॒त्रं वै ष॑ड्रा॒त्रः प्र॒त्यक्ष॒ꣴ॒ ह्ये॑तानि॑ पृ॒ष्ठानि॒ य ए॒वं वि॒द्वाꣳसः॑ षड्रा॒त्रमास॑ते सा॒क्षादे॒व दे॒वता॑ अ॒भ्यारो॑हन्ति षड्रा॒त्रो भ॑वति॒ षड्वा ऋ॒तवः॒ षट्पृ॒ष्ठानि॑~(१)

%7.2.1.2
पृ॒ष्ठैरे॒वर्तून॒न्वारो॑हन्त्यृ॒तुभिः॑ संवथ्स॒रन्ते सं॑वथ्स॒र ए॒व प्रति॑ तिष्ठन्ति बृहद्रथन्त॒रा\-भ्यां᳚ यन्ती॒यं वाव र॑थन्त॒रम॒सौ बृ॒हदा॒भ्यामे॒व य॒न्त्यथो॑ अ॒नयो॑रे॒व प्रति॑ तिष्ठन्त्ये॒ते वै य॒ज्ञस्या᳚ञ्ज॒साय॑नी स्रु॒ती ताभ्या॑मे॒व सु॑व॒र्गं लो॒कं य॑न्ति त्रि॒वृद॑ग्निष्टो॒मो भ॑वति॒ तेज॑ ए॒वाव॑ रुन्धते पञ्चद॒शो भ॑वतीन्द्रि॒यमे॒वाव॑ रुन्धते सप्तद॒शः~(२)

%7.2.1.3
भ॒व॒त्य॒न्नाद्य॒स्याव॑रुद्ध्या॒ अथो॒ प्रैव तेन॑ जायन्त एकवि॒ꣳ॒शो भ॑वति॒ प्रति॑ष्ठित्या॒ अथो॒ रुच॑मे॒वात्मन्द॑धते त्रिण॒वो भ॑वति॒ विजि॑त्यै त्रयस्त्रि॒ꣳ॒शो भ॑वति॒ प्रति॑ष्ठित्यै सदोहविर्धा॒निन॑ ए॒तेन॑ षड्रा॒त्रेण॑ यजेर॒न्नाश्व॑त्थी हवि॒र्धानं॒ चाग्नी᳚ध्रं च भवत॒स्तद्धि सु॑व॒र्ग्यं॑ च॒क्रीव॑ती भवतः सुव॒र्गस्य॑ लो॒कस्य॒ सम॑ट्या उ॒लूख॑लबुध्नो॒ यूपो॑ भवति॒ प्रति॑ष्ठित्यै॒ प्राञ्चो॑ यान्ति॒ प्राङि॑व॒ हि सु॑व॒र्गः~(३)

%7.2.1.4
लो॒कः सर॑स्वत्या यान्त्ये॒ष वै दे॑व॒यानः॒ पन्था॒स्तमे॒वान्वारो॑हन्त्या॒क्रोश॑न्तो या॒न्त्यव॑र्तिमे॒वान्यस्मि॑न्प्रति॒षज्य॑ प्रति॒ष्ठां ग॑च्छन्ति य॒दा दश॑ श॒तं कु॒र्वन्त्यथैक॑मु॒त्थानꣳ॑ श॒तायुः॒ पुरु॑षः श॒तेन्द्रि॑य॒ आयु॑ष्ये॒वेन्द्रि॒ये प्रति॑ तिष्ठन्ति य॒दा श॒तꣳ स॒हस्रं॑ कु॒र्वन्त्यथैक॑मु॒त्थानꣳ॑ स॒हस्र॑सम्मितो॒ वा अ॒सौ लो॒को॑\-ऽमुमे॒व लो॒कम॒भि ज॑यन्ति य॒दैषां᳚ प्र॒मीये॑त य॒दा वा॒ जीये॑र॒न्नथैक॑मु॒त्थान॒न्तद्धि ती॒र्थम्॥~(४)

%7.2.2.0
{\anuvakamend[{पृ॒ष्ठानि॑ सप्तद॒शः सु॑व॒र्गो ज॑यन्ति य॒दैका॑\-दश च}]}%~(१)

%7.2.2.1
कु॒सु॒रु॒बिन्द॒ औद्दा॑लकिरकामयत पशु॒मान्थ्स्या॒मिति॒ स ए॒तꣳ स॑प्तरा॒त्रमाह॑र॒त्तेना॑यजत॒ तेन॒ वै स याव॑न्तो ग्रा॒म्याः प॒शव॒स्तानवा॑रुन्ध॒ य ए॒वं वि॒द्वान्थ्स॑प्तरा॒त्रेण॒ यज॑ते॒ याव॑न्त ए॒व ग्रा॒म्याः प॒शव॒स्ताने॒वाव॑ रुन्धे सप्तरा॒त्रो भ॑वति स॒प्त ग्रा॒म्याः प॒शवः॑ स॒प्तार॒ण्याः स॒प्त छन्दाꣴ॑स्यु॒भय॒स्याव॑रुद्ध्यै त्रि॒वृद॑ग्निष्टो॒मो भ॑वति॒ तेजः॑~(५)

%7.2.2.2
ए॒वाव॑ रुन्धे पञ्चद॒शो भ॑वतीन्द्रि॒यमे॒वाव॑ रुन्धे सप्तद॒शो भ॑वत्य॒न्नाद्य॒स्याव॑रुद्ध्या॒ अथो॒ प्रैव तेन॑ जायत एकवि॒ꣳ॒शो भ॑वति॒ प्रति॑ष्ठित्या॒ अथो॒ रुच॑मे॒वात्मन्ध॑त्ते त्रिण॒वो भ॑वति॒ विजि॑त्यै पञ्चवि॒ꣳ॒शो᳚\-ऽग्निष्टो॒मो भ॑वति प्र॒जाप॑ते॒राप्त्यै॑ महाव्र॒तवा॑न॒न्नाद्य॒स्याव॑रुद्ध्यै विश्व॒जिथ्सर्व॑पृष्ठो\-ऽतिरा॒त्रो भ॑वति॒ सर्व॑स्या॒भिजि॑त्यै॒ यत्प्र॒त्यक्ष॒म्पूर्वे॒ष्वहः॑सु पृ॒ष्ठान्यु॑पे॒युः प्र॒त्यक्षम्᳚~(६)

%7.2.2.3
वि॒श्व॒जिति॒ यथा॑ दु॒ग्धामु॑प॒सीद॑त्ये॒वमु॑त्त॒ममहः॑ स्या॒न्नैक॑रा॒त्रश्च॒न स्या᳚द्बृहद्रथन्त॒रे पूर्वे॒ष्वहः॒सूप॑ यन्ती॒यं वाव र॑थन्त॒रम॒सौ बृ॒हदा॒भ्यामे॒व न य॒न्त्यथो॑ अ॒नयो॑रे॒व प्रति॑ तिष्ठन्ति॒ यत्प्र॒त्यक्षं॑ विश्व॒जिति॑ पृ॒ष्ठान्यु॑प॒यन्ति॒ यथा॒ प्रत्तां᳚ दु॒हे ता॒दृगे॒व तत्॥~(७)

%7.2.3.0
{\anuvakamend[{तेज॑ उपे॒युः प्र॒त्यक्षं॒ द्विच॑त्वारिꣳशच्च}]}%~(२)

%7.2.3.1
बृह॒स्पति॑रकामयत ब्रह्मवर्च॒सी स्या॒मिति॒ स ए॒तम॑ष्टरा॒त्रम॑पश्य॒त्तमाह॑र॒त्तेना॑यजत॒ ततो॒ वै स ब्र॑ह्मवर्च॒स्य॑भव॒द्य ए॒वं वि॒द्वान॑ष्टरा॒त्रेण॒ यज॑ते ब्रह्मवर्च॒स्ये॑व भ॑वत्यष्टरा॒त्रो भ॑वत्य॒ष्टाक्ष॑रा गाय॒त्री गा॑य॒त्री ब्र॑ह्मवर्च॒सम्गा॑यत्रि॒यैव ब्र॑ह्मवर्च॒समव॑ रुन्धे\-ऽष्टरा॒त्रो भ॑वति॒ चत॑स्रो॒ वै दिश॒श्चत॑स्रो\-ऽवान्तरदि॒शा दि॒ग्भ्य ए॒व ब्र॑ह्मवर्च॒समव॑ रुन्धे~(८)

%7.2.3.2
त्रि॒वृद॑ग्निष्टो॒मो भ॑वति॒ तेज॑ ए॒वाव॑ रुन्धे पञ्चद॒शो भ॑वतीन्द्रि॒यमे॒वाव॑ रुन्धे सप्तद॒शो भ॑वत्य॒न्नाद्य॒स्याव॑रुद्ध्या॒ अथो॒ प्रैव तेन॑ जायत एकवि॒ꣳ॒शो भ॑वति॒ प्रति॑ष्ठित्या॒ अथो॒ रुच॑मे॒वात्मन्ध॑त्ते त्रिण॒वो भ॑वति॒ विजि॑त्यै त्रयस्त्रि॒ꣳ॒शो भ॑वति॒ प्रति॑ष्ठित्यै पञ्चवि॒ꣳ॒शो᳚\-ऽग्निष्टो॒मो भ॑वति प्र॒जाप॑ते॒राप्त्यै॑ महाव्र॒तवा॑न॒न्नाद्य॒स्याव॑रुद्ध्यै विश्व॒जिथ्सर्व॑पृष्ठो\-ऽतिरा॒त्रो भ॑वति॒ सर्व॑स्या॒भिजि॑त्यै॥~(९)

%7.2.4.0
{\anuvakamend[{दि॒ग्भ्य ए॒व ब्र॑ह्मवर्च॒समव॑\-रुन्धे॒\-ऽभिजि॑त्यै}]}%~(३)

%7.2.4.1
प्र॒जा\-प॑तिः प्र॒जा अ॑सृजत॒ ताः सृ॒ष्टाः क्षुधं॒ न्या॑य॒न्थ्स ए॒तं न॑वरा॒त्रम॑पश्य॒त्तमाह॑र॒त्तेना॑यजत॒ ततो॒ वै प्र॒जाभ्यो॑\-ऽ कल्पत॒ यर्\mbox{}हि॑ प्र॒जाः क्षुधं॑ नि॒गच्छे॑यु॒स्तर्\mbox{}हि॑ नवरा॒त्रेण॑ यजेते॒मे हि वा ए॒तासां᳚ लो॒का अकॢ॑प्ता॒ अथै॒ताः क्षुधं॒ नि ग॑च्छन्ती॒माने॒वाभ्यो॑ लो॒कान्क॑ल्पयति॒ तान्कल्प॑मानान्प्र॒जाभ्यो\-ऽनु॑ कल्पते॒ कल्प॑न्ते~(१०)

%7.2.4.2
अ॒स्मा॒ इ॒मे लो॒का ऊर्जं॑ प्र॒जासु॑ दधाति त्रिरा॒त्रेणै॒वेमं लो॒कं क॑ल्पयति त्रिरा॒त्रेणा॒न्तरि॑क्षं त्रिरा॒त्रेणा॒मुं लो॒कं यथा॑ गु॒णे गु॒णम॒न्वस्य॑त्ये॒वमे॒व तल्लो॒के लो॒कमन्व॑स्यति॒ धृत्या॒ अशि॑थिलम्भावाय॒ ज्योति॒र्गौरायु॒रिति॑ ज्ञा॒ताः स्तोमा॑ भवन्ती॒यं वाव ज्योति॑र॒न्तरि॑क्षं॒ गौर॒सावायु॑रे॒ष्वे॑व लो॒केषु॒ प्रति॑ तिष्ठन्ति॒ ज्ञात्रं॑ प्र॒जाना᳚म्~(११)

%7.2.4.3
ग॒च्छ॒ति॒ न॒व॒रा॒त्रो भ॑वत्यभिपू॒र्वमे॒वास्मि॒न्तेजो॑ दधाति॒ यो ज्योगा॑मयावी॒ स्याथ्स न॑वरा॒त्रेण॑ यजेत प्रा॒णा हि वा ए॒तस्याधृ॑ता॒ अथै॒तस्य॒ ज्योगा॑मयति प्रा॒णाने॒वास्मि॑न्दाधारो॒त यदी॒तासु॒र्भव॑ति॒ जीव॑त्ये॒व॥~(१२)

%7.2.5.0
{\anuvakamend[{कल्प॑न्ते प्र॒जाना॒न्त्रय॑स्त्रिꣳशच्च}]}%~(४)

%7.2.5.1
प्र॒जा\-प॑तिरकामयत॒ प्र जा॑ये॒येति॒ स ए॒तं दश॑होतारमपश्य॒त्तम॑जुहो॒त्तेन॑ दशरा॒त्रम॑सृजत॒ तेन॑ दशरा॒त्रेण॒ प्राजा॑यत दशरा॒त्राय॑ दीक्षि॒ष्यमा॑णो॒ दश॑होतारं जुहुया॒द्दश॑होत्रै॒व द॑शरा॒त्रꣳ सृ॑जते॒ तेन॑ दशरा॒त्रेण॒ प्र जा॑यते वैरा॒जो वा ए॒ष य॒ज्ञो यद्द॑शरा॒त्रो य ए॒वं वि॒द्वान्द॑शरा॒त्रेण॒ यज॑ते वि॒राज॑मे॒व ग॑च्छति प्राजाप॒त्यो वा ए॒ष य॒ज्ञो यद्द॑शरा॒त्रः~(१३)

%7.2.5.2
य ए॒वं वि॒द्वान्द॑शरा॒त्रेण॒ यज॑ते॒ प्रैव जा॑यत॒ इन्द्रो॒ वै स॒दृङ्दे॒वता॑भिरासी॒थ्स न व्या॒वृत॑मगच्छ॒थ्स प्र॒जा\-प॑ति॒मुपा॑धाव॒त् तस्मा॑ ए॒तं द॑शरा॒त्रम्प्राय॑च्छ॒त्तमाह॑र॒त्तेना॑यजत॒ ततो॒ वै सो᳚\-ऽन्याभि॑र्दे॒वता॑भिर्व्या॒वृत॑मगच्छ॒द्य ए॒वं वि॒द्वान्द॑शरा॒त्रेण॒ यज॑ते व्या॒वृत॑मे॒व पा॒प्मना॒ भ्रातृ॑व्येण गच्छति त्रिक॒कुद्वै~(१४)

%7.2.5.3
ए॒ष य॒ज्ञो यद्द॑शरा॒त्रः क॒कुत्प॑ञ्चद॒शः क॒कुदे॑कवि॒ꣳ॒शः क॒कुत्त्र॑यस्त्रि॒ꣳ॒शो य ए॒वं वि॒द्वान्द॑शरा॒त्रेण॒ यज॑ते त्रिक॒कुदे॒व स॑मा॒नानां᳚ भवति॒ यज॑मानः पञ्चद॒शो यज॑मान एकवि॒ꣳ॒शो यज॑मानस्त्रयस्त्रि॒ꣳ॒शः पुर॒ इत॑रा अभिच॒र्यमा॑णो दशरा॒त्रेण॑ यजेत देवपु॒रा ए॒व पर्यू॑हते॒ तस्य॒ न कुत॑श्च॒नोपा᳚व्या॒धो भ॑वति॒ नैन॑मभि॒चर᳚न्थ्स्तृणुते देवासु॒राः संय॑त्ता आस॒न्ते दे॒वा ए॒ताः~(१५)

%7.2.5.4
दे॒व॒पु॒रा अ॑पश्य॒न् यद्द॑शरा॒त्रस्ताः पर्यौ॑हन्त॒ तेषां॒ न कुत॑श्च॒नोपा᳚व्या॒धो॑\-ऽभव॒त्ततो॑ दे॒वा अभ॑व॒न्परासु॑रा॒ यो भ्रातृ॑व्यवा॒न्थ्स्याथ्स द॑शरा॒त्रेण॑ यजेत देवपु॒रा ए॒व पर्यू॑हते॒ तस्य॒ न कुत॑श्च॒नोपा᳚व्या॒धो भ॑वति॒ भव॑त्या॒त्मना॒ परा᳚स्य॒ भ्रातृ॑व्यो भवति॒ स्तोमः॒ स्तोम॒स्योप॑स्तिर्भवति॒ भ्रातृ॑व्यमे॒वोप॑स्तिं कुरुते जा॒मि वै~(१६)

%7.2.5.5
ए॒तत्कु॑र्वन्ति॒ यज्ज्यायाꣳ॑स॒ꣴ॒ स्तोम॑मु॒पेत्य॒ कनी॑याꣳसमुप॒यन्ति॒ यद॑ग्निष्टोमसा॒मान्य॒वस्ता᳚च्च प॒रस्ता᳚च्च॒ भव॒न्त्यजा॑मित्वाय त्रि॒वृद॑ग्निष्टो॒मो᳚\-ऽग्नि॒ष्टुदा᳚ग्ने॒यीषु॑ भवति॒ तेज॑ ए॒वाव॑ रुन्धे पञ्चद॒श उ॒क्थ्य॑ ऐ॒न्द्रीष्वि॑न्द्रि॒यमे॒वाव॑ रुन्धे त्रि॒वृद॑ग्निष्टो॒मो वै᳚श्वदे॒वीषु॒ पुष्टि॑मे॒वाव॑ रुन्धे सप्तद॒शो᳚\-ऽग्निष्टो॒मः प्रा॑जाप॒त्यासु॑ तीव्रसो॒मो᳚\-ऽन्नाद्य॒स्याव॑रुद्ध्या॒ अथो॒ प्रैव तेन॑ जायते~(१७)

%7.2.5.6
ए॒क॒वि॒ꣳ॒श उ॒क्थ्यः॑ सौ॒रीषु॒ प्रति॑ष्ठित्या॒ अथो॒ रुच॑मे॒वात्मन्ध॑त्ते सप्तद॒शो᳚\-ऽग्निष्टो॒मः प्रा॑जाप॒त्यासू॑पह॒व्य॑ उपह॒वमे॒व ग॑च्छति त्रिण॒वाव॑ग्निष्टो॒माव॒भित॑ ऐ॒न्द्रीषु॒ विजि॑त्यै त्रयस्त्रि॒ꣳ॒श उ॒क्थ्यो॑ वैश्वदे॒वीषु॒ प्रति॑ष्ठित्यै विश्व॒जिथ्सर्व॑पृष्ठो\-ऽ तिरा॒त्रो भ॑वति॒ सर्व॑स्या॒भिजि॑त्यै॥~(१८)

%7.2.6.0
{\anuvakamend[{प्रा॒जा॒प॒त्यो वा ए॒ष य॒ज्ञो यद्द॑शरा॒त्रस्त्रि॑क॒कुद्वा ए॒ता वै जा॑यत॒ एक॑त्रिꣳशच्च}]}%~(५)

%7.2.6.1
ऋ॒तवो॒ वै प्र॒जाका॑माः प्र॒जां नावि॑न्दन्त॒ ते॑\-ऽकामयन्त प्र॒जाꣳ सृ॑जेमहि प्र॒जामव॑ रुन्धीमहि प्र॒जां वि॑न्देमहि प्र॒जाव॑न्तः स्या॒मेति॒ त ए॒तमे॑कादशरा॒त्रम॑पश्य॒न्तमाह॑र॒न्तेना॑यजन्त॒ ततो॒ वै ते प्र॒जाम॑सृजन्त प्र॒जामवा॑रुन्धत प्र॒जाम॑विन्दन्त प्र॒जाव॑न्तो\-ऽभव॒न्त ऋ॒तवो॑\-ऽभव॒न्तदा᳚र्त॒वाना॑मार्तव॒त्वमृ॑तू॒नां वा ए॒ते पु॒त्रास्तस्मा᳚त्~(१९)

%7.2.6.2
आ॒र्त॒वा उ॑च्यन्ते॒ य ए॒वं वि॒द्वाꣳस॑ एकादशरा॒त्रमास॑ते प्र॒जामे॒व सृ॑जन्ते प्र॒जामव॑ रुन्धते प्र॒जां वि॑न्दन्ते प्र॒जाव॑न्तो भवन्ति॒ ज्योति॑रतिरा॒त्रो भ॑वति॒ ज्योति॑रे॒व पु॒रस्ता᳚द्दधते सुव॒र्गस्य॑ लो॒कस्यानु॑ख्यात्यै॒ पृठ्यः॑ षड॒हो भ॑वति॒ षड्वा ऋ॒तवः॒ षट्पृ॒ष्ठानि॑ पृ॒ष्ठैरे॒वर्तून॒न्वारो॑हन्त्यृ॒तुभिः॑ संवथ्स॒रन्ते सं॑वथ्स॒र ए॒व प्रति॑ तिष्ठन्ति चतुर्वि॒ꣳ॒शो भ॑वति॒ चतु॑र्विꣳशत्यक्षरा गाय॒त्री~(२०)

%7.2.6.3
गा॒य॒त्रम्ब्र॑ह्मवर्च॒सङ्गा॑यत्रि॒यामे॒व ब्र॑ह्मवर्च॒से प्रति॑ तिष्ठन्ति चतुश्चत्वारि॒ꣳ॒शो भ॑वति॒ चतु॑श्चत्वारिꣳशदक्षरा त्रि॒ष्टुगि॑न्द्रि॒यं त्रि॒ष्टुप्त्रि॒ष्टुभ्ये॒वेन्द्रि॒ये प्रति॑ तिष्ठन्त्यष्टाचत्वारि॒ꣳ॒शो भ॑वत्य॒ष्टाच॑त्वारिꣳशदक्षरा॒ जग॑ती॒ जाग॑ताः प॒शवो॒ जग॑त्यामे॒व प॒शुषु॒ प्रति॑ तिष्ठन्त्येकादशरा॒त्रो भ॑वति॒ पञ्च॒ वा ऋ॒तव॑ आर्त॒वाः पञ्च॒र्तुष्वे॒वार्त॒वेषु॑ संवथ्स॒रे प्र॑ति॒ष्ठाय॑ प्र॒जामव॑ रुन्धते\-ऽतिरा॒त्राव॒भितो॑ भवतः प्र॒जायै॒ परि॑गृहीत्यै॥~(२१)

%7.2.7.0
{\anuvakamend[{तस्मा᳚द्गाय॒त्र्येका॒न्नप॑ञ्चा॒शच्च॑}]}%~(६)

%7.2.7.1
ऐ॒न्द्र॒वा॒य॒वाग्रा᳚न्गृह्णीया॒द्यः का॒मये॑त यथापू॒र्वं प्र॒जाः क॑ल्पेर॒न्निति॑ य॒ज्ञस्य॒ वै कॢप्ति॒मनु॑ प्र॒जाः क॑ल्पन्ते य॒ज्ञस्याकॢ॑प्ति॒मनु॒ न क॑ल्पन्ते यथापू॒र्वमे॒व प्र॒जाः क॑ल्पयति॒ न ज्यायाꣳ॑सं॒ कनी॑या॒नति॑ क्रामत्यैन्द्रवाय॒वाग्रा᳚न्गृह्णीयादामया॒विनः॑ प्रा॒णेन॒ वा ए॒ष व्यृ॑ध्यते॒ यस्या॒मय॑ति प्रा॒ण ऐ᳚न्द्रवाय॒वः प्रा॒णेनै॒वैन॒ꣳ॒ सम॑र्धयति मैत्रावरु॒णाग्रा᳚न्गृह्णीर॒न् येषां᳚ दीक्षि॒तानां᳚ प्र॒मीये॑त~(२२)

%7.2.7.2
प्रा॒णा॒पा॒नाभ्यां॒ वा ए॒ते व्यृ॑ध्यन्ते॒ येषां᳚ दीक्षि॒तानां᳚ प्र॒मीय॑ते प्राणापा॒नौ मि॒त्रावरु॑णौ प्राणापा॒नावे॒व मु॑ख॒तः परि॑ हरन्त आश्वि॒नाग्रा᳚न्गृह्णीतानुजाव॒रो᳚\-ऽश्विनौ॒ वै दे॒वाना॑मानुजाव॒रौ प॒श्चेवाग्रं॒ पर्यैताम॒श्विना॑वे॒तस्य॑ दे॒वता॒ य आ॑नुजाव॒रस्तावे॒वैन॒मग्रं॒ परि॑ णयतः शु॒क्राग्रा᳚न्गृह्णीत ग॒तश्रीः᳚ प्रति॒ष्ठाका॑मो॒\-ऽसौ वा आ॑दि॒त्यः शु॒क्र ए॒षो\-ऽन्तो\-ऽन्त॑म्मनु॒ष्यः॑~(२३)

%7.2.7.3
श्रि॒यै ग॒त्वा नि व॑र्त॒ते\-ऽन्ता॑दे॒वान्त॒मा र॑भते॒ न ततः॒ पापी॑यान्भवति म॒न्थ्य॑ग्रान्गृह्णीताभि॒चर॑न्नार्तपा॒त्रं वा ए॒तद्यन्म॑न्थिपा॒त्रम्मृ॒त्युनै॒वैनं॑ ग्राहयति ता॒जगार्ति॒मार्च्छ॑त्याग्रय॒णाग्रा᳚न्गृह्णीत॒ यस्य॑ पि॒ता पि॑ताम॒हः पुण्यः॒ स्यादथ॒ तन्न प्रा᳚प्नु॒याद्वा॒चा वा ए॒ष इ॑न्द्रि॒येण॒ व्यृ॑ध्यते॒ यस्य॑ पि॒ता पि॑ताम॒हः पुण्यः॑~(२४)

%7.2.7.4
भव॒त्यथ॒ तन्न प्रा॒प्नोत्युर॑ इवै॒तद्य॒ज्ञस्य॒ वागि॑व॒ यदा᳚ग्रय॒णो वा॒चैवैन॑मिन्द्रि॒येण॒ सम॑र्धयति॒ न ततः॒ पापी॑यान्भव\-त्यु॒क्थ्या᳚ग्रान्गृह्णीताभिच॒र्यमा॑णः॒ सर्वे॑षां॒ वा ए॒तत्पात्रा॑णामिन्द्रि॒यं यदु॑क्थ्यपा॒त्रꣳ सर्वे॑णै॒वैन॑मिन्द्रि॒येणाति॒ प्र यु॑ङ्क्ते॒ सर॑स्वत्य॒भि नो॑ नेषि॒ वस्य॒ इति॑ पुरो॒रुचं॑ कुर्या॒द्वाग्वै~(२५)

%7.2.7.5
सर॑स्वती वा॒चैवैन॒मति॒ प्र यु॑ङ्क्ते॒ मा त्वत्क्षेत्रा॒ण्यर॑णानि ग॒न्मेत्या॑ह मृ॒त्योर्वै क्षेत्रा॒ण्यर॑णानि॒ तेनै॒व मृ॒त्योः क्षेत्रा॑णि॒ न ग॑च्छति पू॒र्णान्ग्रहा᳚न्गृह्णीयादामया॒विनः॑ प्रा॒णान् वा ए॒तस्य॒ शुगृ॑च्छति॒ यस्या॒मय॑ति प्रा॒णा ग्रहाः᳚ प्रा॒णाने॒वास्य॑ शु॒चो मु॑ञ्चत्यु॒त यदी॒तासु॒र्भव॑ति॒ जीव॑त्ये॒व पू॒र्णान्ग्रहा᳚न्गृह्णीया॒द्यर्\mbox{}हि॑ प॒र्जन्यो॒ न वर्\mbox{}षे᳚त्प्रा॒णान् वा ए॒तर्\mbox{}हि॑ प्र॒जाना॒ꣳ॒ शुगृ॑च्छति॒ यर्\mbox{}हि॑ प॒र्जन्यो॒ न॒ वर्\mbox{}ष॑ति प्रा॒णा ग्रहाः᳚ प्रा॒णाने॒व प्र॒जानाꣳ॑ शु॒चो मु॑ञ्चति ता॒जक्प्र व॑र्\mbox{}षति॥~(२६)

%7.2.8.0
{\anuvakamend[{प्र॒मीये॑त मनु॒ष्य॑ ऋध्यते॒ यस्य॑ पि॒ता पि॑ताम॒हः पुण्यो॒ वाग्वा ए॒व पू॒र्णान्ग्रहा॒न्पञ्च॑विꣳशतिश्च}]}%~(७)

%7.2.8.1
गा॒य॒त्रो वा ऐ᳚न्द्रवाय॒वो गा॑य॒त्रम्प्रा॑य॒णीय॒मह॒स्तस्मा᳚त्प्राय॒णीये\-ऽह॑न्नैन्द्रवाय॒वो गृ॑ह्यते॒ स्व ए॒वैन॑मा॒यत॑ने गृह्णाति॒ त्रैष्टु॑भो॒ वै शु॒क्रस्त्रैष्टु॑भं द्वि॒तीय॒मह॒स्तस्मा᳚द्द्वि॒तीये\-ऽह॑ञ्छु॒क्रो गृ॑ह्यते॒ स्व ए॒वैन॑मा॒यत॑ने गृह्णाति॒ जाग॑तो॒ वा आ᳚ग्रय॒णो जाग॑तं तृ॒तीय॒मह॒स्तस्मा᳚त्तृ॒तीये\-ऽह॑न्नाग्रय॒णो गृ॑ह्यते॒ स्व ए॒वैन॑मा॒यत॑ने गृह्णात्ये॒तद्वै~(२७)

%7.2.8.2
य॒ज्ञमा॑प॒द्यच्छन्दाꣴ॑स्या॒प्नोति॒ यदा᳚ग्रय॒णः श्वो गृ॒ह्यते॒ यत्रै॒व य॒ज्ञमदृ॑श॒न्तत॑ ए॒वैन॒म्पुनः॒ प्र यु॑ङ्क्ते॒ जग॑न्मुखो॒ वै द्वि॒तीय॑स्त्रिरा॒त्रो जाग॑त आग्रय॒णो यच्च॑तु॒र्थे\-ऽह॑न्नाग्रय॒णो गृ॒ह्यते॒ स्व ए॒वैन॑मा॒यत॑ने गृह्णा॒त्यथो॒ स्वमे॒व छन्दो\-ऽनु॑ प॒र्याव॑र्तन्ते॒ राथं॑तरो॒ वा ऐ᳚न्द्रवाय॒वो राथं॑तरं पञ्च॒ममह॒स्तस्मा᳚त्पञ्च॒मे\-ऽहन्न्॑~(२८)

%7.2.8.3
ऐ॒न्द्र॒वा॒य॒वो गृ॑ह्यते॒ स्व ए॒वैन॑मा॒यत॑ने गृह्णाति॒ बार्\mbox{}ह॑तो॒ वै शु॒क्रो बार्\mbox{}ह॑तꣳ ष॒ष्ठमह॒स्तस्मा᳚त्ष॒ष्ठे\-ऽह॑ञ्छु॒क्रो गृ॑ह्यते॒ स्व ए॒वैन॑मा॒यत॑ने गृह्णात्ये॒तद्वै द्वि॒तीयं॑ य॒ज्ञमा॑प॒द्यच्छन्दाꣴ॑स्या॒प्नोति॒ यच्छु॒क्रः श्वो गृ॒ह्यते॒ यत्रै॒व य॒ज्ञमदृ॑श॒न्तत॑ ए॒वैन॒म्पुनः॒ प्र यु॑ङ्क्ते त्रि॒ष्टुङ्मु॑खो॒ वै तृ॒तीय॑स्त्रिरा॒त्रस्त्रैष्टु॑भः~(२९)

%7.2.8.4
शु॒क्रो यथ्स॑प्त॒मे\-ऽह॑ञ्छु॒क्रो गृ॒ह्यते॒ स्व ए॒वैन॑मा॒यत॑ने गृह्णा॒त्यथो॒ स्वमे॒व छन्दो\-ऽनु॑ प॒र्याव॑र्तन्ते॒ वाग्वा आ᳚ग्रय॒णो वाग॑ष्ट॒ममह॒स्तस्मा॑दष्ट॒मे\-ऽह॑न्नाग्रय॒णो गृ॑ह्यते॒ स्व ए॒वैन॑मा॒यत॑ने गृह्णाति प्रा॒णो वा ऐ᳚न्द्रवाय॒वः प्रा॒णो न॑व॒ममह॒स्तस्मा᳚न्नव॒मे\-ऽह॑न्नैन्द्रवाय॒वो गृ॑ह्यते॒ स्व ए॒वैन॑मा॒यत॑ने गृह्णात्ये॒तत्~(३०)

%7.2.8.5
वै तृ॒तीयं॑ य॒ज्ञमा॑प॒द्यच्छन्दाꣴ॑स्या॒प्नोति॒ यदै᳚न्द्रवाय॒वः श्वो गृ॒ह्यते॒ यत्रै॒व य॒ज्ञमदृ॑श॒न्तत॑ ए॒वैन॒म्पुनः॒ प्र यु॒ङ्क्ते\-ऽथो॒ स्वमे॒व छन्दो\-ऽनु॑ प॒र्याव॑र्तन्ते प॒थो वा ए॒ते\-ऽध्यप॑थेन यन्ति॒ ये᳚\-ऽन्येनै᳚न्द्रवाय॒वात्प्र॑ति॒पद्य॒न्ते\-ऽन्तः॒ खलु॒ वा ए॒ष य॒ज्ञस्य॒ यद्द॑श॒ममह॑र्दश॒मे\-ऽह॑न्नैन्द्रवाय॒वो गृ॑ह्यते य॒ज्ञस्य॑~(३१)

%7.2.8.6
ए॒वान्तं॑ ग॒त्वाप॑था॒त्पन्था॒मपि॑ य॒न्त्यथो॒ यथा॒ वही॑यसा प्रति॒सारं॒ वह॑न्ति ता॒दृगे॒व तच्छन्दाꣴ॑स्य॒न्यो᳚न्यस्य॑ लो॒कम॒भ्य॑ध्याय॒न्तान्ये॒तेनै॒व दे॒वा व्य॑वाहयन्नैन्द्रवाय॒वस्य॒ वा ए॒तदा॒यत॑नं॒ यच्च॑तु॒र्थमह॒स्तस्मि॑न्नाग्रय॒णो गृ॑ह्यते॒ तस्मा॑दाग्रय॒णस्या॒यत॑ने नव॒मे\-ऽह॑न्नैन्द्रवाय॒वो गृ॑ह्यते शु॒क्रस्य॒ वा ए॒तदा॒यत॑नं॒ यत्प॑ञ्च॒मम्~(३२)

%7.2.8.7
अह॒स्तस्मि॑न्नैन्द्रवाय॒वो गृ॑ह्यते॒ तस्मा॑दैन्द्रवाय॒वस्या॒यत॑ने सप्त॒मे\-ऽह॑ञ्छु॒क्रो गृ॑ह्यत आग्रय॒णस्य॒ वा ए॒तदा॒यत॑नं॒ यत्ष॒ष्ठमह॒स्तस्मि॑ञ्छु॒क्रो गृ॑ह्यते॒ तस्मा᳚च्छु॒क्रस्या॒यत॑ने\-ऽष्ट॒मे\-ऽह॑न्नाग्रय॒णो गृ॑ह्यते॒ छन्दाꣴ॑स्ये॒व तद्वि वा॑हयति॒ प्र वस्य॑सो विवा॒हमा᳚प्नोति॒ य ए॒वं वेदाथो॑ दे॒वता᳚भ्य ए॒व य॒ज्ञे सं॒विदं॑ दधाति॒ तस्मा॑दि॒दमन्यो᳚न्यस्मै॑ ददाति॥~(३३)

%7.2.9.0
{\anuvakamend[{ए॒तद्वै प॑ञ्च॒मे\-ऽह॒न्त्रैष्टु॑भ ए॒तद्गृ॑ह्यते य॒ज्ञस्य॑ प़ञ्च॒मम॒न्यस्मा॒ एक॑ञ्च}]}%~(८)

%7.2.9.1
प्र॒जा\-प॑तिरकामयत॒ प्र जा॑ये॒येति॒ स ए॒तं द्वा॑दशरा॒त्रम॑पश्य॒त्तमाह॑र॒त्तेना॑यजत॒ ततो॒ वै स प्राजा॑यत॒ यः का॒मये॑त॒ प्र जा॑ये॒येति॒ स द्वा॑दशरा॒त्रेण॑ यजेत॒ प्रैव जा॑यते ब्रह्मवा॒दिनो॑ वदन्त्यग्निष्टो॒मप्रा॑यणा य॒ज्ञा अथ॒ कस्मा॑दतिरा॒त्रः पूर्वः॒ प्र यु॑ज्यत॒ इति॒ चक्षु॑षी॒ वा ए॒ते य॒ज्ञस्य॒ यद॑तिरा॒त्रौ क॒नीनि॑के अग्निष्टो॒मौ यत्~(३४)

%7.2.9.2
अ॒ग्नि॒ष्टो॒मं पूर्वं॑ प्रयुञ्जी॒रन्ब॑हि॒र्धा क॒नीनि॑के दध्यु॒स्तस्मा॑दतिरा॒त्रः पूर्वः॒ प्र यु॑ज्यते॒ चक्षु॑षी ए॒व य॒ज्ञे धि॒त्वा म॑ध्य॒तः क॒नीनि॑के॒ प्रति॑ दधति॒ यो वै गा॑य॒त्रीं ज्योतिः॑पक्षां॒ वेद॒ ज्योति॑षा भा॒सा सु॑व॒र्गं लो॒कमे॑ति॒ याव॑ग्निष्टो॒मौ तौ प॒क्षौ ये\-ऽन्त॑रे॒\-ऽष्टावु॒क्थ्याः᳚ स आ॒त्मैषा वै गा॑य॒त्री ज्योतिः॑पक्षा॒ य ए॒वं वेद॒ ज्योति॑षा भा॒सा सु॑व॒र्गं लो॒कम्~(३५)

%7.2.9.3
ए॒ति॒ प्र॒जा\-प॑तिर्वा ए॒ष द्वा॑दश॒धा विहि॑तो॒ यद्द्वा॑दशरा॒त्रो याव॑तिरा॒त्रौ तौ प॒क्षौ ये\-ऽन्त॑रे॒\-ऽष्टावु॒क्थ्याः᳚ स आ॒त्मा प्र॒जा\-प॑तिर्वावैष सन्थ्सद्ध॒ वै स॒त्रेण॑ स्पृणोति प्रा॒णा वै सत्प्रा॒णाने॒व स्पृ॑णोति॒ सर्वा॑सां॒ वा ए॒ते प्र॒जानां᳚ प्रा॒णैरा॑सते॒ ये स॒त्रमास॑ते॒ तस्मा᳚त्पृच्छन्ति॒ किमे॒ते स॒त्रिण॒ इति॑ प्रि॒यः प्र॒जाना॒मुत्थि॑तो भवति॒ य ए॒वं वेद॑॥~(३६)

%7.2.10.0
{\anuvakamend[{अ॒ग्नि॒ष्टो॒मौ यथ्सु॑व॒र्गल्लों॒कं प्रि॒यः प्र॒जानां॒ पञ्च॑ च}]}%~(९)

%7.2.10.1
न वा ए॒षो᳚\-ऽन्यतो॑वैश्वानरः सुव॒र्गाय॑ लो॒काय॒ प्राभ॑वदू॒र्ध्वो ह॒ वा ए॒ष आत॑त आसी॒त्ते दे॒वा ए॒तं वै᳚श्वान॒रं पर्यौ॑हन्थ्सुव॒र्गस्य॑ लो॒कस्य॒ प्रभू᳚त्या ऋ॒तवो॒ वा ए॒तेन॑ प्र॒जा\-प॑तिमयाजय॒न्तेष्वा᳚र्ध्नो॒दधि॒ तदृ॒ध्नोति॑ ह॒ वा ऋ॒त्विक्षु॒ य ए॒वं वि॒द्वान्द्वा॑दशा॒हेन॒ यज॑ते॒ ते᳚\-ऽस्मिन्नैच्छन्त॒ स रस॒मह॑ वस॒न्ताय॒ प्राय॑च्छत्~(३७)

%7.2.10.2
यवं॑ ग्री॒ष्मायौष॑धीर्व॒र्\mbox{}षाभ्यो᳚ व्री॒हीञ्छ॒रदे॑ माषति॒लौ हे॑मन्तशिशि॒राभ्या॒न्तेनेन्द्रं॑ प्र॒जा\-प॑तिरयाजय॒त्ततो॒ वा इन्द्र॒ इन्द्रो॑\-ऽभव॒त्तस्मा॑दाहुरानुजाव॒रस्य॑ य॒ज्ञ इति॒ स ह्ये॑तेनाग्रे\-ऽय॑जतै॒ष ह॒ वै कु॒णप॑मत्ति॒ यः स॒त्रे प्र॑तिगृ॒ह्णाति॑ पुरुषकुण॒पम॑श्वकुण॒पङ्गौर्वा अन्नं॒ येन॒ पात्रे॒णान्न॒म्बिभ्र॑ति॒ यत्तन्न नि॒र्णेनि॑जति॒ ततो\-ऽधि॑~(३८)

%7.2.10.3
मलं॑ जायत॒ एक॑ ए॒व य॑जे॒तैको॒ हि प्र॒जा\-प॑ति॒रार्ध्नो॒द्द्वाद॑श॒ रात्री᳚र्दीक्षि॒तः स्या॒द्द्वाद॑श॒ मासाः᳚ संवथ्स॒रः सं॑वथ्स॒रः प्र॒जा\-प॑तिः प्र॒जा\-प॑ति॒र्वावैष ए॒ष ह॒ त्वै जा॑यते॒ यस्तप॒सो\-ऽधि॒ जाय॑ते चतु॒र्धा वा ए॒तास्ति॒स्रस्ति॑स्रो॒ रात्र॑यो॒ यद्द्वाद॑शोप॒सदो॒ याः प्र॑थ॒मा य॒ज्ञं ताभिः॒ सम्भ॑रति॒ या द्वि॒तीया॑ य॒ज्ञं ताभि॒रा र॑भते~(३९)

%7.2.10.4
यास्तृ॒तीयाः॒ पात्रा॑णि॒ ताभि॒र्निर्णे॑निक्ते॒ याश्च॑तु॒र्थीरपि॒ ताभि॑रा॒त्मान॑मन्तर॒तः शु॑न्धते॒ यो वा अ॑स्य प॒शुमत्ति॑ मा॒ꣳ॒सꣳ सो᳚\-ऽत्ति॒ यः पु॑रो॒डाश॑म्म॒स्तिष्क॒ꣳ॒ स यः प॑रिवा॒पं पुरी॑ष॒ꣳ॒ स य आज्य॑म्म॒ज्जान॒ꣳ॒ स यः सोमꣴ॒ स्वेद॒ꣳ॒ सो\-ऽपि॑ ह॒ वा अ॑स्य शीर्\mbox{}ष॒ण्या॑ नि॒ष्पदः॒ प्रति॑ गृह्णाति॒ यो द्वा॑दशा॒हे प्र॑तिगृ॒ह्णाति॒ तस्मा᳚द्द्वादशा॒हेन॒ न याज्य॑म्पा॒प्मनो॒ व्यावृ॑त्त्यै॥~(४०)

%7.2.11.0
{\anuvakamend[{अय॑च्छ॒दधि॑ रभते द्वादशा॒हेन॑ च॒त्वारि॑ च}]}%॥10॥

%7.2.11.1
एक॑स्मै॒ स्वाहा॒ द्वाभ्या॒ꣴ॒ स्वाहा᳚ त्रि॒भ्यः स्वाहा॑ च॒तुर्भ्यः॒ स्वाहा॑ प॒ञ्चभ्यः॒ स्वाहा॑ ष॒ड्भ्यः स्वाहा॑ स॒प्तभ्यः॒ स्वाहा᳚\-ऽष्टा॒भ्यः स्वाहा॑ न॒वभ्यः॒ स्वाहा॑ द॒शभ्यः॒ स्वाहै॑काद॒शभ्यः॒ स्वाहा᳚ द्वाद॒शभ्यः॒ स्वाहा᳚ त्रयोद॒शभ्यः॒ स्वाहा॑ चतुर्द॒शभ्यः॒ स्वाहा॑ पञ्चद॒शभ्यः॒ स्वाहा॑ षोड॒शभ्यः॒ स्वाहा॑ सप्तद॒शभ्यः॒ स्वाहा᳚\-ऽष्टाद॒शभ्यः॒ स्वाहैका॒न्न विꣳ॑श॒त्यै स्वाहा॒ नव॑विꣳशत्यै॒ स्वाहैका॒न्न च॑त्वारि॒ꣳ॒शते॒ स्वाहा॒ नव॑चत्वारिꣳशते॒ स्वाहैका॒न्न ष॒ट्यै स्वाहा॒ नव॑षट्यै॒ स्वाहैका॒न्नाशी॒त्यै स्वाहा॒ नवा॑शीत्यै॒ स्वाहैका॒न्न श॒ताय॒ स्वाहा॑ श॒ताय॒ स्वाहा॒ द्वाभ्याꣳ॑ श॒ताभ्या॒ꣴ॒ स्वाहा॒ सर्व॑स्मै॒ स्वाहा᳚॥~(४१)

%7.2.12.0
{\anuvakamend[{नव॑चत्वारिꣳशते॒ स्वाहैका॒न्नैक॑विꣳशतिश्च}]}%॥11॥

%7.2.12.1
एक॑स्मै॒ स्वाहा᳚ त्रि॒भ्यः स्वाहा॑ प॒ञ्चभ्यः॒ स्वाहा॑ स॒प्तभ्यः॒ स्वाहा॑ न॒वभ्यः॒ स्वाहै॑काद॒शभ्यः॒ स्वाहा᳚ त्रयोद॒शभ्यः॒ स्वाहा॑ पञ्चद॒शभ्यः॒ स्वाहा॑ सप्तद॒शभ्यः॒ स्वाहैका॒न्न विꣳ॑श॒त्यै स्वाहा॒ नव॑विꣳशत्यै॒ स्वाहैका॒न्न च॑त्वारि॒ꣳ॒शते॒ स्वाहा॒ नव॑चत्वारिꣳशते॒ स्वाहैका॒न्न ष॒ट्यै स्वाहा॒ नव॑षट्यै॒ स्वाहैका॒न्नाशी॒त्यै स्वाहा॒ नवा॑शीत्यै॒ स्वाहैका॒न्न श॒ताय॒ स्वाहा॑ श॒ताय॒ स्वाहा॒ सर्व॑स्मै॒ स्वाहा᳚॥~(४२)

%7.2.13.0
{\anuvakamend[{एक॑स्मै त्रि॒भ्यः प॑ञ्चा॒शत्}]}%॥12॥

%7.2.13.1
द्वाभ्या॒ꣴ॒ स्वाहा॑ च॒तुर्भ्यः॒ स्वाहा॑ ष॒ड्भ्यः स्वाहा᳚\-ऽष्टा॒भ्यः स्वाहा॑ द॒शभ्यः॒ स्वाहा᳚ द्वाद॒शभ्यः॒ स्वाहा॑ चतुर्द॒शभ्यः॒ स्वाहा॑ षोड॒शभ्यः॒ स्वाहा᳚\-ऽष्टाद॒शभ्यः॒ स्वाहा॑ विꣳश॒त्यै स्वाहा॒\-ऽष्टान॑वत्यै॒ स्वाहा॑ श॒ताय॒ स्वाहा॒ सर्व॑स्मै॒ स्वाहा᳚॥~(४३)

%7.2.14.0
{\anuvakamend[{द्वाभ्या॑म॒ष्टान॑वत्यै॒ षड्विꣳ॑शतिः}]}%॥13॥

%7.2.14.1
त्रि॒भ्यः स्वाहा॑ प॒ञ्चभ्यः॒ स्वाहा॑ स॒प्तभ्यः॒ स्वाहा॑ न॒वभ्यः॒ स्वाहै॑काद॒शभ्यः॒ स्वाहा᳚ त्रयोद॒शभ्यः॒ स्वाहा॑ पञ्चद॒शभ्यः॒ स्वाहा॑ सप्तद॒शभ्यः॒ स्वाहैका॒न्न विꣳ॑श॒त्यै स्वाहा॒ नव॑विꣳशत्यै॒ स्वाहैका॒न्न च॑त्वारि॒ꣳ॒शते॒ स्वाहा॒ नव॑चत्वारिꣳशते॒ स्वाहैका॒न्न ष॒ट्यै स्वाहा॒ नव॑षट्यै॒ स्वाहैका॒न्नाशी॒त्यै स्वाहा॒ नवा॑शीत्यै॒ स्वाहैका॒न्न श॒ताय॒ स्वाहा॑ श॒ताय॒ स्वाहा॒ सर्व॑स्मै॒ स्वाहा᳚॥~(४४)

%7.2.15.0
{\anuvakamend[{त्रि॒भ्यो᳚\-ऽष्टाचत्वारि॒ꣳ॒शत्}]}%॥14॥

%7.2.15.1
च॒तुर्भ्यः॒ स्वाहा᳚\-ऽष्टा॒भ्यः स्वाहा᳚ द्वाद॒शभ्यः॒ स्वाहा॑ षोड॒शभ्यः॒ स्वाहा॑ विꣳश॒त्यै स्वाहा॒ षण्ण॑वत्यै॒ स्वाहा॑ श॒ताय॒ स्वाहा॒ सर्व॑स्मै॒ स्वाहा᳚॥~(४५)

%7.2.16.0
{\anuvakamend[{च॒तुर्भ्यः॒ षण्ण॑वत्यै॒ षोड॑श}]}%॥15॥

%7.2.16.1
प॒ञ्चभ्यः॒ स्वाहा॑ द॒शभ्यः॒ स्वाहा॑ पञ्चद॒शभ्यः॒ स्वाहा॑ विꣳश॒त्यै स्वाहा॒ पञ्च॑नवत्यै॒ स्वाहा॑ श॒ताय॒ स्वाहा॒ सर्व॑स्मै॒ स्वाहा᳚॥~(४६)

%7.2.17.0
{\anuvakamend[{प॒ञ्चभ्यः॒ प़ञ्च॑नवत्यै॒ चतु॑र्दश}]}%॥16॥

%7.2.17.1
द॒शभ्यः॒ स्वाहा॑ विꣳश॒त्यै स्वाहा᳚ त्रि॒ꣳ॒शते॒ स्वाहा॑ चत्वारि॒ꣳ॒शते॒ स्वाहा॑ पञ्चा॒शते॒ स्वाहा॑ ष॒ट्यै स्वाहा॑ सप्त॒त्यै स्वाहा॑\-ऽशी॒त्यै स्वाहा॑ नव॒त्यै स्वाहा॑ श॒ताय॒ स्वाहा॒ सर्व॑स्मै॒ स्वाहा᳚॥~(४७)

%7.2.18.0
{\anuvakamend[{द॒शभ्यो॒ द्वाविꣳ॑शतिः}]}%॥17॥

%7.2.18.1
वि॒ꣳ॒श॒त्यै स्वाहा॑ चत्वारि॒ꣳ॒शते॒ स्वाहा॑ ष॒ट्यै स्वाहा॑\-ऽशी॒त्यै स्वाहा॑ श॒ताय॒ स्वाहा॒ सर्व॑स्मै॒ स्वाहा᳚॥~(४८)

%7.2.19.0
{\anuvakamend[{वि॒ꣳ॒श॒त्यै द्वाद॑श}]}%॥18॥

%7.2.19.1
प॒ञ्चा॒शते॒ स्वाहा॑ श॒ताय॒ स्वाहा॒ द्वाभ्याꣳ॑ श॒ताभ्या॒ꣴ॒ स्वाहा᳚ त्रि॒भ्यः श॒तेभ्यः॒ स्वाहा॑ च॒तुर्भ्यः॑ श॒तेभ्यः॒ स्वाहा॑ प॒ञ्चभ्यः॑ श॒तेभ्यः॒ स्वाहा॑ ष॒ड्भ्यः श॒तेभ्यः॒ स्वाहा॑ स॒प्तभ्यः॑ श॒तेभ्यः॒ स्वाहा᳚\-ऽष्टा॒भ्यः श॒तेभ्यः॒ स्वाहा॑ न॒वभ्यः॑ श॒तेभ्यः॒ स्वाहा॑ स॒हस्रा॑य॒ स्वाहा॒ सर्व॑स्मै॒ स्वाहा᳚॥~(४९)

%7.2.20.0
{\anuvakamend[{प॒ञ्चा॒शते॒ द्वात्रिꣳ॑शत्}]}%॥19॥

%7.2.20.1
श॒ताय॒ स्वाहा॑ स॒हस्रा॑य॒ स्वाहा॒\-ऽयुता॑य॒ स्वाहा॑ नि॒युता॑य॒ स्वाहा᳚ प्र॒युता॑य॒ स्वाहा\-ऽर्बु॑दाय॒ स्वाहा॒ न्य॑र्बुदाय॒ स्वाहा॑ समु॒द्राय॒ स्वाहा॒ मध्या॑य॒ स्वाहा\-ऽन्ता॑य॒ स्वाहा॑ परा॒र्धाय॒ स्वाहो॒षसे॒ स्वाहा॒ व्यु॑ट्यै॒ स्वाहो॑देष्य॒ते स्वाहो᳚द्य॒ते स्वाहोदि॑ताय॒ स्वाहा॑ सुव॒र्गाय॒ स्वाहा॑ लो॒काय॒ स्वाहा॒ सर्व॑स्मै॒ स्वाहा᳚॥~(५०)

%7.3.0.0
{\anuvakamend[{श॒ताया॒ष्टात्रिꣳ॑शत्}]}%॥20॥

%7.3.0.0

{\anuvakamend[{प्र॒जवं॑ ब्रह्मवा॒दिनः॒ किमे॒ष वा आ॒प्त आ॑दि॒त्या उ॒भयोः᳚ प्र॒जा\-प॑ति॒रन्वा॑य॒न्निन्द्रो॒ वै स॒दृङ्ङिन्द्रो॒ वै शि॑थि॒लः प्र॒जा\-प॑तिरकामयतान्ना॒दः सा वि॒राड॒सावा॑दि॒त्यो᳚\-ऽर्वाङ्भू॒तमा मे॒\-ऽग्निना॒ स्वाहा॒धिन्द॒द्भ्यो᳚\-ऽञ्ज्ये॒ताय॑ कृ॒ष्णायौष॑धीभ्यो॒ वन॒स्पति॑भ्यो विꣳश॒तिः}]%॥20॥
}
%%% END PRASHNA
