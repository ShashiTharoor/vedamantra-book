\sect{पञ्चमः प्रश्नः}\setcounter{anuvakam}{0}
\dnsub{तैत्तिरीयसंहितायां पञ्चमकाण्डे पञ्चमः प्रश्नः}
%5.5.1.0
%5.5.1.1
यदेके॑न सꣴस्था॒पय॑ति य॒ज्ञस्य॒ सन्त॑त्या॒ अवि॑च्छेदायै॒न्द्राः प॒शवो॒ ये मु॑ष्क॒रा यदै॒न्द्राः सन्तो॒\-ऽग्निभ्य॑ आल॒भ्यन्ते॑ दे॒वता᳚भ्यः स॒मदं॑ दधात्याग्ने॒यीस्त्रि॒ष्टुभो॑ याज्यानुवा॒क्याः᳚ कुर्या॒द्यदा᳚ग्ने॒यीस्तेना᳚ग्ने॒या यत्त्रि॒ष्टुभ॒स्तेनै॒न्द्राः समृ॑द्ध्यै॒ न दे॒वता᳚भ्यः स॒मदं॑ दधाति वा॒यवे॑ नि॒युत्व॑ते तूप॒रमा ल॑भते॒ तेजो॒\-ऽग्नेर्वा॒युस्तेज॑स ए॒ष आ ल॑भ्यते॒ तस्मा᳚द्य॒द्रिय॑ङ्वा॒युः~(१)

%5.5.1.2
वाति॑ त॒द्रिय॑ङ्ङ॒ग्निर्द॑हति॒ स्वमे॒व तत्तेजो\-ऽन्वे॑ति॒ यन्न नि॒युत्व॑ते॒ स्यादुन्मा᳚द्ये॒द्यज॑मानो नि॒युत्व॑ते भवति॒ यज॑मान॒स्यानु॑न्मादाय वायु॒मती᳚ श्वे॒तव॑ती याज्यानुवा॒क्ये॑ भवतः सतेज॒स्त्वाय॑ हिरण्यग॒र्भः सम॑वर्त॒ताग्र॒ इत्या॑घा॒रमा घा॑रयति प्र॒जा\-प॑ति॒र्वै हि॑रण्यग॒र्भः प्र॒जाप॑तेरनुरूप॒त्वाय॒ सर्वा॑णि॒ वा ए॒ष रू॒पाणि॑ पशू॒नाम्प्रत्या ल॑भ्यते॒ यच्छ्म॑श्रु॒णस्तत्~(२)

%5.5.1.3
पुरु॑षाणाꣳ रू॒पं यत्तू॑प॒रस्तदश्वा॑नां॒ यद॒न्यतो॑द॒न्तद्गवां॒ यदव्या॑ इव श॒फास्तदवी॑नां॒ यद॒जस्तद॒जानां᳚ वा॒युर्वै प॑शू॒नाम्प्रि॒यं धाम॒ यद्वा॑य॒व्यो॑ भव॑त्ये॒तमे॒वैन॑म॒भि सं॑जाना॒नाः प॒शव॒ उप॑ तिष्ठन्ते वाय॒व्यः॑ का॒र्या(३)ः प्रा॑जाप॒त्या(३) इत्या॑हु॒र्यद्वा॑य॒व्यं॑ कु॒र्यात्प्र॒जाप॑तेरिया॒द्यत्प्रा॑जाप॒त्यं कु॒र्याद्वा॒योः~(३)

%5.5.1.4
इ॒या॒द्यद्वा॑य॒व्यः॑ प॒शुर्भव॑ति॒ तेन॑ वा॒योर्नैति॒ यत्प्रा॑जाप॒त्यः पु॑रो॒डाशो॒ भव॑ति॒ तेन॑ प्रा॒जाप॑ते॒र्नैति॒ यद्द्वाद॑श\-कपाल॒स्तेन॑ वैश्वान॒रान्नैत्या᳚ग्नावैष्ण॒वमेका॑\-दश\-कपालं॒ निर्व॑पति दीक्षि॒ष्यमा॑णो॒\-ऽग्निः सर्वा॑ दे॒वता॒ विष्णु॑र्य॒ज्ञो दे॒वता᳚श्चै॒व य॒ज्ञं चा र॑भते॒\-ऽग्निर॑व॒मो दे॒वता॑नां॒ विष्णुः॑ पर॒मो यदा᳚ग्नावैष्ण॒वमेका॑\-दश\-कपालं नि॒र्वप॑ति दे॒वताः᳚~(४)

%5.5.1.5
ए॒वोभ॒यतः॑ परि॒गृह्य॒ यज॑मा॒नो\-ऽव॑ रुन्धे पुरो॒डाशे॑न॒ वै दे॒वा अ॒मुष्मिँ॑ल्लो॒क आ᳚र्धुवं च॒रुणा॒स्मिन् यः का॒मये॑ता॒मुष्मिँ॑ल्लो॒क ऋ॑ध्नुया॒मिति॒ स पु॑रो॒डाशं॑ कुर्वीता॒मुष्मि॑न्ने॒व लो॒क ऋ॑ध्नोति॒ यद॒ष्टाक॑पाल॒स्तेना᳚ग्ने॒यो यत्त्रि॑कपा॒लस्तेन॑ वैष्ण॒वः समृ॑द्ध्यै॒ यः का॒मये॑ता॒स्मिँल्लो॒क ऋ॑ध्नुया॒मिति॒ स च॒रुं कु॑र्वीता॒ग्नेर्घृ॒तं विष्णो᳚स्तण्डु॒लास्तस्मा᳚त्~(५)

%5.5.1.6
च॒रुः का॒र्यो᳚\-ऽस्मिन्ने॒व लो॒क ऋ॑ध्नोत्यादि॒त्यो भ॑वती॒यं वा अदि॑तिर॒स्यामे॒व प्रति॑ तिष्ठ॒त्यथो॑ अ॒स्यामे॒वाधि॑ य॒ज्ञं त॑नुते॒ यो वै सं॑वथ्स॒रमुख्य॒मभृ॑त्वा॒ग्निं चि॑नु॒ते यथा॑ सा॒मि गर्भो॑\-ऽव॒पद्य॑ते ता॒दृगे॒व तदार्ति॒मार्च्छे᳚द्वैश्वान॒रं द्वाद॑श\-कपालम् पु॒रस्ता॒न्निर्व॑पेथ्संवथ्स॒रो वा अ॒ग्निर्वै᳚श्वान॒रो यथा॑ संवथ्स॒रमा॒प्त्वा~(६)

%5.5.1.7
का॒ल आग॑ते वि॒जाय॑त ए॒वमे॒व सं॑वथ्स॒रमा॒प्त्वा का॒ल आग॑ते॒\-ऽग्निं चि॑नुते॒ नार्ति॒मार्च्छ॑त्ये॒षा वा अ॒ग्नेः प्रि॒या त॒नूर्यद्वै᳚श्वान॒रः प्रि॒यामे॒वास्य॑ त॒नुव॒मव॑ रुन्धे॒ त्रीण्ये॒तानि॑ ह॒वीꣳषि॑ भवन्ति॒ त्रय॑ इ॒मे लो॒का ए॒षां लो॒काना॒ꣳ॒ रोहा॑य॥~(७)

%5.5.2.0
{\anuvakamend[{य॒द्रिय॑ङ्वा॒युर्यच्छ्म॑श्रु॒णस्तद्वा॒योर्नि॒र्वप॑ति दे॒वता॒स्तस्मा॑दा॒प्त्वाष्टात्रिꣳ॑शच्च}]}%~(१)

%5.5.2.1
प्र॒जा\-प॑तिः प्र॒जाः सृ॒ष्ट्वा प्रे॒णानु॒ प्रावि॑श॒त्ताभ्यः॒ पुनः॒ सम्भ॑वितुं॒ नाश॑क्नो॒थ्सो᳚\-ऽब्रवीदृ॒ध्नव॒दिथ्स यो मे॒तः पुनः॑ सञ्चि॒नव॒दिति॒ तं दे॒वाः सम॑चिन्व॒न्ततो॒ वै त आ᳚र्ध्नुव॒न् यथ्स॒मचि॑न्व॒न्तच्चित्य॑स्य चित्य॒त्वम् य ए॒वं वि॒द्वान॒ग्निं चि॑नु॒त ऋ॒ध्नोत्ये॒व कस्मै॒ कम॒ग्निश्ची॑यत॒ इत्या॑हुरग्नि॒वान्~(८)

%5.5.2.2
अ॒सा॒नीति॒ वा अ॒ग्निश्ची॑यते\-ऽग्नि॒वाने॒व भ॑वति॒ कस्मै॒ कम॒ग्निश्ची॑यत॒ इत्या॑हुर्दे॒वा मा॑ वेद॒न्निति॒ वा अ॒ग्निश्ची॑यते वि॒दुरे॑नं दे॒वाः कस्मै॒ कम॒ग्निश्ची॑यत॒ इत्या॑हुर्गृ॒ह्य॑सा॒नीति॒ वा अ॒ग्निश्ची॑यते गृ॒ह्ये॑व भ॑वति॒ कस्मै॒ कम॒ग्निश्ची॑यत॒ इत्या॑हुः पशु॒मान॑सा॒नीति॒ वा अ॒ग्निः~(९)

%5.5.2.3
ची॒य॒ते॒ प॒शु॒माने॒व भ॑वति॒ कस्मै॒ कम॒ग्निश्ची॑यत॒ इत्या॑हुः स॒प्त मा॒ पुरु॑षा॒ उप॑ जीवा॒निति॒ वा अ॒ग्निश्ची॑यते॒ त्रयः॒ प्राञ्च॒स्त्रयः॑ प्र॒त्यं च॑ आ॒त्मा स॑प्त॒म ए॒ताव॑न्त ए॒वैन॑म॒मुष्मिँ॑ल्लो॒क उप॑ जीवन्ति प्र॒जा\-प॑तिर॒ग्निम॑चिकीषत॒ तं पृ॑थि॒व्य॑ब्रवी॒न्न मय्य॒ग्निं चे᳚ष्य॒सेति॑ मा धक्ष्यति॒ सा त्वा॑तिद॒ह्यमा॑ना॒ वि ध॑विष्ये~(१०)

%5.5.2.4
स पापी॑यान्भविष्य॒सीति॒ सो᳚\-ऽब्रवी॒त्तथा॒ वा अ॒हं क॑रिष्यामि॒ यथा᳚ त्वा॒ नाति॑ध॒क्ष्यतीति॒ स इ॒माम॒भ्य॑मृशत् प्र॒जा\-प॑तिस्त्वा सादयतु॒ तया॑ दे॒वत॑याङ्गिर॒स्वद्ध्रु॒वा सी॒देती॒मामे॒वेष्ट॑कां कृ॒त्वोपा॑ध॒त्तान॑तिदाहाय॒ यत्प्रत्य॒ग्निं चि॑न्वी॒त तद॒भि मृ॑शेत्प्र॒जा\-प॑तिस्त्वा सादयतु॒ तया॑ दे॒वत॑याङ्गिर॒स्वद्ध्रु॒वा सी॑द~(११)

%5.5.2.5
इती॒मामे॒वेष्ट॑कां कृ॒त्वोप॑ ध॒त्ते\-ऽन॑तिदाहाय प्र॒जा\-प॑तिरकामयत॒ प्र जा॑ये॒येति॒ स ए॒तमुख्य॑मपश्य॒त्तꣳ सं॑वथ्स॒रम॑बिभ॒स्ततो॒ वै स प्राजा॑यत॒ तस्मा᳚थ्संवथ्स॒रम्भा॒र्यः॑ प्रैव जा॑यते॒ तं वस॑वो\-ऽब्रुव॒न्प्र त्वम॑जनिष्ठा व॒यं प्र जा॑यामहा॒ इति॒ तं वसु॑भ्यः॒ प्राय॑च्छ॒त्तं त्रीण्यहा᳚न्यबिभरु॒स्तेन॑~(१२)

%5.5.2.6
त्रीणि॑ च श॒तान्यसृ॑जन्त॒ त्रय॑स्त्रिꣳशतं च॒ तस्मा᳚त्त्र्य॒हम्भा॒र्यः॑ प्रैव जा॑यते॒ तान्रु॒द्रा अ॑ब्रुव॒न्प्र यू॒यम॑जनिढ्वं व॒यं प्र जा॑यामहा॒ इति॒ तꣳ रु॒द्रेभ्यः॒ प्राय॑च्छ॒न्तꣳ षडहा᳚न्यबिभरु॒स्तेन॒ त्रीणि॑ च श॒तान्यसृ॑जन्त॒ त्रय॑स्त्रिꣳशतं च॒ तस्मा᳚त्षड॒हम्भा॒र्यः॑ प्रैव जा॑यते॒ ताना॑दि॒त्या अ॑ब्रुव॒न्प्र यू॒यम॑जनिढ्वं व॒यं ~(१३)

%5.5.2.7
प्र जा॑यामहा॒ इति॒ तमा॑दि॒त्येभ्यः॒ प्राय॑च्छ॒न्तं द्वाद॒शाहा᳚न्यबिभरु॒स्तेन॒ त्रीणि॑ च श॒तान्यसृ॑जन्त॒ त्रय॑स्त्रिꣳशतं च॒ तस्मा᳚द्द्वादशा॒हम्भा॒र्यः॑ प्रैव जा॑यते॒ तेनै॒व ते स॒हस्र॑मसृजन्तो॒खाꣳ स॑हस्रत॒मीं य ए॒वमुख्यꣳ॑ साह॒स्रं वेद॒ प्र स॒हस्रं॑ प॒शूना᳚प्नोति॥~(१४)

%5.5.3.0
{\anuvakamend[{अ॒ग्नि॒वान्प॑शु॒मान॑सा॒नीति॒ वा अ॒ग्निर्ध॑विष्ये मृशेत्प्र॒जा\-प॑तिस्त्वा सादयतु॒ तया॑ दे॒वत॑याङ्गिर॒स्वद्ध्रु॒वा सी॑द॒ तेन॒ ताना॑दि॒त्या अ॑ब्रुव॒न्प्र यू॒यम॑जनिढ्वं व॒यञ्च॑त्वारि॒ꣳ॒शच्च॑}]}%~(२)

%5.5.3.1
यजु॑षा॒ वा ए॒षा क्रि॑यते॒ यजु॑षा पच्यते॒ यजु॑षा॒ वि मु॑च्यते॒ यदु॒खा सा वा ए॒षैतर्\mbox{}हि॑ या॒तया᳚म्नी॒ सा न पुनः॑ प्र॒युज्येत्या॑हु॒रग्ने॑ यु॒क्ष्वा हि ये तव॑ यु॒क्ष्वा हि दे॑व॒हूत॑मा॒ꣳ॒ इत्यु॒खायां᳚ जुहोति॒ तेनै॒वैना॒म्पुनः॒ प्र यु॑ङ्क्ते॒ तेनाया॑तयाम्नी॒ यो वा अ॒ग्निं योग॒ आग॑ते यु॒नक्ति॑ यु॒ङ्क्ते यु॑ञ्जा॒नेष्वग्ने᳚~(१५)

%5.5.3.2
यु॒क्ष्वा हि ये तव॑ यु॒क्ष्वा हि दे॑व॒हूत॑मा॒ꣳ॒ इत्या॑है॒ष वा अ॒ग्नेर्योग॒स्तेनै॒वैनं॑ युनक्ति यु॒ङ्क्ते यु॑ञ्जा॒नेषु॑ ब्रह्मवा॒दिनो॑ वदन्ति न्य॑ङ्ङ॒ग्निश्चे॑त॒व्या(३) उ॑त्ता॒ना(३) इति॒ वय॑सां॒ वा ए॒ष प्र॑ति॒मया॑ चीयते॒ यद॒ग्निर्यन्न्य॑ञ्चं चिनु॒यात्पृ॑ष्टि॒त ए॑न॒माहु॑तय ऋच्छेयु॒र्यदु॑त्ता॒नं न पति॑तुꣳ शक्नुया॒दसु॑वर्ग्यो\-ऽस्य स्यात्प्रा॒चीन॑मुत्ता॒नम्~(१६)

%5.5.3.3
पु॒रु॒ष॒शी॒र्\mbox{}षमुप॑ दधाति मुख॒त ए॒वैन॒माहु॑तय ऋच्छन्ति॒ नोत्ता॒नं चि॑नुते सुव॒र्ग्यो᳚\-ऽस्य भवति सौ॒र्या जु॑होति॒ चक्षु॑रे॒वास्मि॒न्प्रति॑ दधाति॒ द्विर्जु॑होति॒ द्वे हि चक्षु॑षी समा॒न्या जु॑होति समा॒नꣳ हि चक्षुः॒ समृ॑द्ध्यै देवासु॒राः संय॑त्ता आस॒न्ते वा॒मं वसु॒ सं न्य॑दधत॒ तद्दे॒वा वा॑म॒भृता॑वृञ्जत॒ तद्वा॑म॒भृतो॑ वामभृ॒त्त्वं यद्वा॑म॒भृत॑मुप॒दधा॑ति वा॒ममे॒व तया॒ वसु॒ यज॑मानो॒ भ्रातृ॑व्यस्य वृङ्क्ते॒ हिर॑ण्यमूर्ध्नी भवति॒ ज्योति॒र्वै हिर॑ण्यं॒ ज्योति॑र्वा॒मं ज्योति॑षै॒वास्य॒ ज्योति॑र्वा॒मं वृ॑ङ्क्ते द्विय॒जुर्भ॑वति॒ प्रति॑ष्ठित्यै॥~(१७)

%5.5.4.0
{\anuvakamend[{यु॒ञ्जा॒नेष्वग्ने᳚ प्रा॒चीन॑मुत्ता॒नं वा॑म॒भृत॒ञ्चतु॑र्विꣳशतिश्च}]}%~(३)

%5.5.4.1
आपो॒ वरु॑णस्य॒ पत्न॑य आस॒न्ता अ॒ग्निर॒भ्य॑ध्याय॒त्ताः सम॑भव॒त्तस्य॒ रेतः॒ परा॑पत॒त्तदि॒यम॑भव॒द्यद्द्वि॒तीय॑म्प॒राप॑त॒त्तद॒सा\-व॑भवदि॒यं वै वि॒राड॒सौ स्व॒राड्यद्वि॒राजा॑वुप॒दधा॑ती॒मे ए॒वोप॑ धत्ते॒ यद्वा अ॒सौ रेतः॑ सि॒ञ्चति॒ तद॒स्यां प्रति॑ तिष्ठति॒ तत्प्र जा॑यते॒ ता ओष॑धयः~(१८)

%5.5.4.2
वी॒रुधो॑ भवन्ति॒ ता अ॒ग्निर॑त्ति॒ य ए॒वं वेद॒ प्रैव जा॑यते\-ऽन्ना॒दो भ॑वति॒ यो रे॑त॒स्वी स्यात्प्र॑थ॒मायां॒ तस्य॒ चित्या॑मु॒भे उप॑ दध्यादि॒मे ए॒वास्मै॑ स॒मीची॒ रेतः॑ सिञ्चतो॒ यः सि॒क्तरे॑ताः॒ स्यात्प्र॑थ॒मायां॒ तस्य॒ चित्या॑म॒न्यामुप॑ दध्यादुत्त॒माया॑\-म॒न्याꣳ रेत॑ ए॒वास्य॑ सि॒क्तमा॒भ्यामु॑भ॒यतः॒ परि॑ गृह्णाति संवथ्स॒रं न कम्~(१९)

%5.5.4.3
च॒न प्र॒त्यव॑रोहे॒न्न हीमे कं च॒न प्र॑त्यव॒रोह॑त॒स्तदे॑नयोर्व्र॒तं यो वा अप॑शीर्\mbox{}षाणम॒ग्निं चि॑नु॒ते\-ऽप॑शीर्\mbox{}षा॒मुष्मिँ॑ल्लो॒के भ॑वति॒ यः सशी॑र्\mbox{}षाणं चिनु॒ते सशी॑र्\mbox{}षा॒मुष्मिँ॑ल्लो॒के भ॑वति॒ चित्तिं॑ जुहोमि॒ मन॑सा घृ॒तेन॒ यथा॑ दे॒वा इ॒हागम॑न्वी॒तिहो᳚त्रा ऋता॒वृधः॑ समु॒द्रस्य॑ व॒युन॑स्य॒ पत्म॑ञ्जु॒होमि॑ वि॒श्वक॑र्मणे॒ विश्वाहाम॑र्त्यꣳ ह॒विरिति॑ स्वयमातृ॒ण्णामु॑प॒धाय॑ जुहोति~(२०)

%5.5.4.4
ए॒तद्वा अ॒ग्नेः शिरः॒ सशी॑र्\mbox{}षाणमे॒वाग्निं चि॑नुते॒ सशी॑र्\mbox{}षा॒मुष्मिँ॑ल्लो॒के भ॑वति॒ य ए॒वं वेद॑ सुव॒र्गाय॒ वा ए॒ष लो॒काय॑ चीयते॒ यद॒ग्निस्तस्य॒ यदय॑थापूर्वं क्रि॒यते\-ऽसु॑वर्ग्यमस्य॒ तथ्सु॑व॒र्ग्यो᳚\-ऽग्निश्चिति॑मुप॒धाया॒भि मृ॑शे॒च्चित्ति॒मचि॑त्तिं चिनव॒द्वि वि॒द्वान्पृ॒ष्ठेव॑ वी॒ता वृ॑जि॒ना च॒ मर्ता᳚न्रा॒ये च॑ नः स्वप॒त्याय॑ देव॒ दितिं॑ च॒ रास्वादि॑तिमुरु॒ष्येति॑ यथापू॒र्वमे॒वैना॒मुप॑ धत्ते॒ प्राञ्च॑मेनं चिनुते सुव॒र्ग्यो᳚\-ऽस्य भवति॥~(२१)

%5.5.5.0
{\anuvakamend[{ओष॑धयः॒ कञ्जु॑होति स्वप॒त्याया॒ष्टाद॑श च}]}%~(४)

%5.5.5.1
वि॒श्वक॑र्मा दि॒शाम्पतिः॒ स नः॑ प॒शून्पा॑तु॒ सो᳚\-ऽस्मान्पा॑तु॒ तस्मै॒ नमः॑ प्र॒जाप॑ती रु॒द्रो वरु॑णो॒\-ऽग्निर्दि॒शाम्पतिः॒ स नः॑ प॒शून्पा॑तु॒ सो᳚\-ऽस्मान्पा॑तु॒ तस्मै॒ नम॑ ए॒ता वै दे॒वता॑ ए॒तेषां᳚ पशू॒नामधि॑पतय॒स्ताभ्यो॒ वा ए॒ष आ वृ॑श्च्यते॒ यः प॑शुशी॒र्\mbox{}षाण्यु॑प॒दधा॑ति हिरण्येष्ट॒का उप॑ दधात्ये॒ताभ्य॑ ए॒व दे॒वता᳚भ्यो॒ नम॑स्करोति ब्रह्मवा॒दिनः॑~(२२)

%5.5.5.2
व॒द॒न्त्य॒ग्नौ ग्रा॒म्यान्प॒शून्प्र द॑धाति शु॒चार॒ण्यान॑र्पयति॒ किं तत॒ उच्छिꣳ॑ष॒तीति॒ यद्धि॑रण्येष्ट॒का उ॑प॒दधा᳚त्य॒मृतं॒ वै हिर॑ण्यम॒मृते॑नै॒व ग्रा॒म्येभ्यः॑ प॒शुभ्यो॑ भेष॒जं क॑रोति॒ नैनान्॑ हिनस्ति प्रा॒णो वै प्र॑थ॒मा स्व॑यमातृ॒ण्णा व्या॒नो द्वि॒तीया॑पा॒नस्तृ॒तीयानु॒ प्राण्या᳚त्प्रथ॒माꣴ स्व॑यमातृ॒ण्णामु॑प॒धाय॑ प्रा॒णेनै॒व प्रा॒णꣳ सम॑र्धयति॒ व्य॑न्यात्~(२३)

%5.5.5.3
द्वि॒तीया॑मुप॒धाय॑ व्या॒नेनै॒व व्या॒नꣳ सम॑र्धय॒त्यपा᳚न्यात्तृ॒तीया॑मुप॒धाया॑पा॒नेनै॒वापा॒नꣳ सम॑र्धय॒त्यथो᳚ प्रा॒णैरे॒वैन॒ꣳ॒ समि॑न्द्धे॒ भूर्भुवः॒ सुव॒रिति॑ स्वयमातृ॒ण्णा उप॑ दधाती॒मे वै लो॒काः स्व॑यमातृ॒ण्णा ए॒ताभिः॒ खलु॒ वै व्याहृ॑तीभिः प्र॒जा\-प॑तिः॒ प्राजा॑यत॒ यदे॒ताभि॒र्व्याहृ॑तीभिः स्वयमातृ॒ण्णा उ॑प॒दधा॑ती॒माने॒व लो॒कानु॑प॒धायै॒षु~(२४)

%5.5.5.4
लो॒केष्वधि॒ प्र जा॑यते प्रा॒णाय॑ व्या॒नाया॑पा॒नाय॑ वा॒चे त्वा॒ चक्षु॑षे त्वा॒ तया॑ दे॒वत॑याङ्गिर॒स्वद्ध्रु॒वा सी॑दा॒ग्निना॒ वै दे॒वाः सु॑व॒र्गं लो॒कम॑जिगाꣳस॒न्तेन॒ पति॑तुं॒ नाश॑क्नुव॒न्त ए॒ताश्चत॑स्रः स्वयमातृ॒ण्णा अ॑पश्य॒न्ता दि॒क्षूपा॑दधत॒ तेन॑ स॒र्वत॑श्चक्षुषा सुव॒र्गं लो॒कमा॑य॒न्॒यच्चत॑स्रः स्वयमातृ॒ण्णा दि॒क्षू॑प॒दधा॑ति स॒र्वत॑श्चक्षुषै॒व तद॒ग्निना॒ यज॑मानः सुव॒र्गं लो॒कमे॑ति॥~(२५)

%5.5.6.0
{\anuvakamend[{ब्र॒ह्म॒वा॒दिनो॒ व्य॑न्यादे॒षु यज॑मान॒स्त्रीणि॑ च}]}%~(५)

%5.5.6.1
अग्न॒ आ या॑हि वी॒तय॒ इत्या॒हाह्व॑तै॒वैन॑म॒ग्निं दू॒तं वृ॑णीमह॒ इत्या॑ह हू॒त्वैवैनं॑ वृणीते॒\-ऽग्निना॒ग्निः समि॑ध्यत॒ इत्या॑ह॒ समि॑न्द्ध ए॒वैन॑म॒ग्निर्वृ॒त्राणि॑ जङ्घन॒दित्या॑ह॒ समि॑द्ध ए॒वास्मि॑न्निन्द्रि॒यं द॑धात्य॒ग्नेः स्तोम॑म्मनामह॒ इत्या॑ह मनु॒त ए॒वैन॑मे॒तानि॒ वा अह्नाꣳ॑ रू॒पाणि॑~(२६)

%5.5.6.2
अ॒न्व॒हमे॒वैनं॑ चिनु॒ते\-ऽवाह्नाꣳ॑ रू॒पाणि॑ रुन्धे ब्रह्मवा॒दिनो॑ वदन्ति॒ कस्मा᳚थ्स॒त्याद्या॒तया᳚म्नीर॒न्या इष्ट॑का॒ अया॑तयाम्नी लोकं पृ॒णेत्यै᳚न्द्रा॒ग्नी हि बा॑र्\mbox{}हस्प॒त्येति॑ ब्रूयादिन्द्रा॒ग्नी च॒ हि दे॒वानां॒ बृह॒स्पति॒श्चाया॑तयामानो\-ऽनुच॒रव॑ती भव॒त्यजा॑मित्वायानु॒ष्टुभानु॑ चरत्या॒त्मा वै लो॑कं पृ॒णा प्रा॒णो॑\-ऽनु॒ष्टुप्तस्मा᳚त्प्रा॒णः सर्वा॒ण्यङ्गा॒न्यनु॑ चरति॒ ता अ॑स्य॒ सूद॑दोहसः~(२७)

%5.5.6.3
इत्या॑ह॒ तस्मा॒त्परु॑षिपरुषि॒ रसः॒ सोमꣴ॑ श्रीणन्ति॒ पृश्ञ॑य॒ इत्या॒हान्नं॒ वै पृश्न्यन्न॑मे॒वाव॑ रुन्धे॒\-ऽर्को वा अ॒ग्निर॒र्को\-ऽन्न॒मन्न॑मे॒वाव॑ रुन्धे॒ जन्मं॑ दे॒वानां॒ विश॑स्त्रि॒ष्वा रो॑च॒ने दि॒व इत्या॑हे॒माने॒वास्मै॑ लो॒कां ज्योति॑ष्मतः करोति॒ यो वा इष्ट॑कानां प्रति॒ष्ठां वेद॒ प्रत्ये॒व ति॑ष्ठति॒ तया॑ दे॒वत॑याङ्गिर॒स्वद्ध्रु॒वा सी॒देत्या॑है॒षा वा इष्ट॑कानां प्रति॒ष्ठा य ए॒वं वेद॒ प्रत्ये॒व ति॑ष्ठति॥~(२८)

%5.5.7.0
{\anuvakamend[{रू॒पाणि॒ सूद॑दोहस॒स्तया॒ षोड॑श च}]}%~(६)

%5.5.7.1
सु॒व॒र्गाय॒ वा ए॒ष लो॒काय॑ चीयते॒ यद॒ग्निर्वज्र॑ एकाद॒शिनी॒ यद॒ग्नावे॑काद॒शिनी᳚म्मिनु॒याद्वज्रे॑णैनꣳ सुव॒र्गाल्लो॒का\-द॒न्तर्द॑ध्या॒द्यन्न मि॑नु॒याथ्स्वरु॑भिः प॒शून्व्य॑र्धयेदेकयू॒पम्मि॑नोति॒ नैनं॒ वज्रे॑ण सुव॒र्गाल्लो॒काद॑न्त॒र्दधा॑ति॒ न स्वरु॑भिः प॒शून्व्य॑र्धयति॒ वि वा ए॒ष इ॑न्द्रि॒येण॑ वी॒र्ये॑णर्ध्यते॒ यो᳚\-ऽग्निं चि॒न्वन्न॑धि॒क्राम॑त्यैन्द्रि॒या~(२९)

%5.5.7.2
ऋ॒चाक्रम॑णं॒ प्रतीष्ट॑का॒मुप॑ दध्या॒न्नेन्द्रि॒येण॑ वी॒र्ये॑ण॒ व्यृ॑ध्यते रु॒द्रो वा ए॒ष यद॒ग्निस्तस्य॑ ति॒स्रः श॑र॒व्याः᳚ प्र॒तीची॑ ति॒रश्च्य॒नूची॒ ताभ्यो॒ वा ए॒ष आ वृ॑श्च्यते॒ यो᳚\-ऽग्निं चि॑नु॒ते᳚\-ऽग्निं चि॒त्वा ति॑सृध॒न्वमया॑चितम्ब्राह्म॒णाय॑ दद्या॒त्ताभ्य॑ ए॒व नम॑स्करो॒त्यथो॒ ताभ्य॑ ए॒वात्मानं॒ निष्क्री॑णीते॒ यत्ते॑ रुद्र पु॒रः~(३०)

%5.5.7.3
धनु॒स्तद्वातो॒ अनु॑ वातु ते॒ तस्मै॑ ते रुद्र संवथ्स॒रेण॒ नम॑स्करोमि॒ यत्ते॑ रुद्र दक्षि॒णा धनु॒स्तद्वातो॒ अनु॑ वातु ते॒ तस्मै॑ ते रुद्र परिवथ्स॒रेण॒ नम॑स्करोमि॒ यत्ते॑ रुद्र प॒श्चाद्धनु॒स्तद्वातो॒ अनु॑ वातु ते॒ तस्मै॑ ते रुद्रेदावथ्स॒रेण॒ नम॑स्करोमि॒ यत्ते॑ रुद्रोत्त॒राद्धनु॒स्तत्~(३१)

%5.5.7.4
वातो॒ अनु॑ वातु ते॒ तस्मै॑ ते रुद्रेदुवथ्स॒रेण॒ नम॑स्करोमि॒ यत्ते॑ रुद्रो॒परि॒ धनु॒स्तद्वातो॒ अनु॑ वातु ते॒ तस्मै॑ ते रुद्र वथ्स॒रेण॒ नम॑स्करोमि रु॒द्रो वा ए॒ष यद॒ग्निः स यथा᳚ व्या॒घ्रः क्रु॒द्धस्तिष्ठ॑त्ये॒वं वा ए॒ष ए॒तर्\mbox{}हि॒ सञ्चि॑तमे॒तैरुप॑ तिष्ठते नमस्का॒रैरे॒वैनꣳ॑ शमयति॒ ये᳚\-ऽग्नयः॑~(३२)

%5.5.7.5
पु॒री॒ष्याः᳚ प्रवि॑ष्टाः पृथि॒वीमनु॑। तेषां॒ त्वम॑स्युत्त॒मः प्र णो॑ जी॒वात॑वे सुव। आपं॑ त्वाऽग्ने॒ मन॒सापं॑ त्वाऽग्ने॒ तप॒सापं॑ त्वाग्ने दी॒क्षयापं॑ त्वाग्न उप॒सद्भि॒रापं॑ त्वाग्ने सु॒त्ययापं॑ त्वाऽग्ने॒ दक्षि॑णाभि॒रापं॑ त्वाग्ने\-ऽवभृ॒थेनापं॑ त्वाग्ने व॒शयापं॑ त्वाग्ने स्वगाका॒रेणेत्या॑है॒षा वा अ॒ग्नेराप्ति॒स्तयै॒वैन॑माप्नोति॥~(३३)

%5.5.8.0
{\anuvakamend[{ऐ॒न्द्रि॒या पु॒र उ॑त्त॒राद्धनु॒स्तद॒ग्नय॑ आहा॒ष्टौ च॑}]}%~(७)

%5.5.8.1
गा॒य॒त्रेण॑ पु॒रस्ता॒दुप॑ तिष्ठते प्रा॒णमे॒वास्मि॑न्दधाति बृहद्रथन्त॒रा\-भ्यां᳚ प॒क्षावोज॑ ए॒वास्मि॑न्दधात्यृतु॒स्थाय॑ज्ञाय॒ज्ञिये॑न॒ पुच्छ॑मृ॒तुष्वे॒व प्रति॑ तिष्ठति पृ॒ष्ठैरुप॑ तिष्ठते॒ तेजो॒ वै पृ॒ष्ठानि॒ तेज॑ ए॒वास्मि॑न्दधाति प्र॒जा\-प॑तिर॒ग्निम॑सृजत॒ सो᳚\-ऽस्माथ्सृ॒ष्टः परा॑ङै॒त्तं वा॑रव॒न्तीये॑नावारयत॒ तद्वा॑रव॒न्तीय॑स्य वारवन्तीय॒त्वꣴ श्यै॒तेन॑ श्ये॒ती अ॑कुरुत॒ तच्छ्यै॒तस्य॑ श्यैत॒त्वम्~(३४)

%5.5.8.2
यद्वा॑रव॒न्तीये॑नोप॒तिष्ठ॑ते वा॒रय॑त ए॒वैनꣴ॑ श्यै॒तेन॑ श्ये॒ती कु॑रुते प्र॒जाप॑ते॒र्\mbox{}हृद॑येनापिप॒क्षम्प्रत्युप॑ तिष्ठते प्रे॒माण॑मे॒वास्य॑ गच्छति॒ प्राच्या᳚ त्वा दि॒शा सा॑दयामि गाय॒त्रेण॒ छन्द॑सा॒ग्निना॑ दे॒वत॑या॒ग्नेः शी॒र्ष्णाग्नेः शिर॒ उप॑ दधामि॒ दक्षि॑णया त्वा दि॒शा सा॑दयामि॒ त्रैष्टु॑भेन॒ छन्द॒सेन्द्रे॑ण दे॒वत॑या॒ग्नेः प॒क्षेणा॒ग्नेः प॒क्षमुप॑ दधामि प्र॒तीच्या᳚ त्वा दि॒शा सा॑दयामि~(३५)

%5.5.8.3
जाग॑तेन॒ छन्द॑सा सवि॒त्रा दे॒वत॑या॒ग्नेः पुच्छे॑ना॒ग्नेः पुच्छ॒मुप॑ दधा॒म्युदी᳚च्या त्वा दि॒शा सा॑दया॒म्यानु॑ष्टुभेन॒ छन्द॑सा मि॒त्रावरु॑णाभ्यां दे॒वत॑या॒ग्नेः प॒क्षेणा॒ग्नेः प॒क्षमुप॑ दधाम्यू॒र्ध्वया᳚ त्वा दि॒शा सा॑दयामि॒ पाङ्क्ते॑न॒ छन्द॑सा॒ बृह॒स्पति॑ना दे॒वत॑या॒ग्नेः पृ॒ष्ठेना॒ग्नेः पृ॒ष्ठमुप॑ दधामि॒ यो वा अपा᳚त्मानम॒ग्निं चि॑नु॒ते\-ऽपा᳚त्मा॒मुष्मिँ॑ल्लो॒के भ॑वति॒ यः सात्मा॑नं चिनु॒ते सात्मा॒मुष्मिँ॑ल्लो॒के भ॑वत्यात्मेष्ट॒का उप॑ दधात्ये॒ष वा अ॒ग्नेरा॒त्मा सात्मा॑नमे॒वाग्निं चि॑नुते॒ सात्मा॒मुष्मिँ॑ल्लो॒के भ॑वति॒ य ए॒वं वेद॑॥~(३६)

%5.5.9.0
{\anuvakamend[{श्यै॒त॒त्वं प्र॒तीच्या᳚ त्वा दि॒शा सा॑दयामि॒ यः सात्मा॑नञ्चिनु॒ते द्वाविꣳ॑शतिश्च}]}%~(८)

%5.5.9.1
अग्न॑ उदधे॒ या त॒ इषु॑र्यु॒वा नाम॒ तया॑ नो मृड॒ तस्या᳚स्ते॒ नम॒स्तस्या᳚स्त॒ उप॒ जीव॑न्तो भूया॒स्माग्ने॑ दुध्र गह्य किꣳशिल वन्य॒ या त॒ इषु॑र्यु॒वा नाम॒ तया॑ नो मृड॒ तस्या᳚स्ते॒ नम॒स्तस्या᳚स्त॒ उप॒ जीव॑न्तो भूयास्म॒ पञ्च॒ वा ए॒ते᳚\-ऽग्नयो॒ यच्चित॑य उद॒धिरे॒व नाम॑ प्रथ॒मो दु॒ध्रः~(३७)

%5.5.9.2
द्वि॒तीयो॒ गह्य॑स्तृ॒तीयः॑ किꣳशि॒लश्च॑तु॒र्थो वन्यः॑ पञ्च॒मस्तेभ्यो॒ यदाहु॑ती॒र्न जु॑हु॒याद॑ध्व॒र्युं च॒ यज॑मानं च॒ प्र द॑हेयु॒र्यदे॒ता आहु॑तीर्जु॒होति॑ भाग॒धेये॑नै॒वैना᳚ञ्छमयति॒ नार्ति॒मार्च्छ॑त्यध्व॒र्युर्न यज॑मानो॒ वाङ्म॑ आ॒सन्न॒सोः प्रा॒णो᳚\-ऽक्ष्योश्चक्षुः॒ कर्ण॑योः॒ श्रोत्र॑म्बाहु॒वोर्बल॑मूरु॒वोरोजो\-ऽरि॑ष्टा॒ विश्वा॒न्यङ्गा॑नि त॒नूः~(३८)

%5.5.9.3
त॒नुवा॑ मे स॒ह नम॑स्ते अस्तु॒ मा मा॑ हिꣳसी॒रप॒ वा ए॒तस्मा᳚त्प्रा॒णाः क्रा॑मन्ति॒ यो᳚\-ऽग्निं चि॒न्वन्न॑धि॒क्राम॑ति॒ वाङ्म॑ आ॒सन्न॒सोः प्रा॒ण इत्या॑ह प्रा॒णाने॒वात्मन्ध॑त्ते॒ यो रु॒द्रो अ॒ग्नौ यो अ॒फ्सु य ओष॑धीषु॒ यो रु॒द्रो विश्वा॒ भुव॑नावि॒वेश॒ तस्मै॑ रु॒द्राय॒ नमो॑ अ॒स्त्वाहु॑तिभागा॒ वा अ॒न्ये रु॒द्रा ह॒विर्भा॑गाः~(३९)

%5.5.9.4
अ॒न्ये श॑तरु॒द्रीयꣳ॑ हु॒त्वा गा॑वीधु॒कं च॒रुमे॒तेन॒ यजु॑षा चर॒माया॒मिष्ट॑कायां॒ नि द॑ध्याद्भाग॒धेये॑नै॒वैनꣳ॑ शमयति॒ तस्य॒ त्वै श॑तरु॒द्रीयꣳ॑ हु॒तमित्या॑हु॒र्यस्यै॒तद॒ग्नौ क्रि॒यत॒ इति॒ वस॑वस्त्वा रु॒द्रैः पु॒रस्ता᳚त्पान्तु पि॒तर॑स्त्वा य॒मरा॑जानः पि॒तृभि॑र्दक्षिण॒तः पा᳚न्त्वादि॒त्यास्त्वा॒ विश्वै᳚र्दे॒वैः प॒श्चात्पा᳚न्तु द्युता॒नस्त्वा॑ मारु॒तो म॒रुद्भि॑रुत्तर॒तः पा॑तु~(४०)

%5.5.9.5
दे॒वास्त्वेन्द्र॑ज्येष्ठा॒ वरु॑णराजानो॒\-ऽधस्ता᳚च्चो॒परि॑ष्टाच्च पान्तु॒ न वा ए॒तेन॑ पू॒तो न मेध्यो॒ न प्रोक्षि॑तो॒ यदे॑न॒मतः॑ प्रा॒चीनं॑ प्रो॒क्षति॒ यथ्सञ्चि॑त॒माज्ये॑न प्रो॒क्षति॒ तेन॑ पू॒तस्तेन॒ मेध्य॒स्तेन॒ प्रोक्षि॑तः॥~(४१)

%5.5.10.0
{\anuvakamend[{दु॒ध्रस्त॒नूर्\mbox{}ह॒विर्भा॑गाः पातु॒ द्वात्रिꣳ॑शच्च}]}%~(९)

%5.5.10.1
स॒मीची॒ नामा॑सि॒ प्राची॒ दिक्तस्या᳚स्ते॒\-ऽग्निरधि॑पतिरसि॒तो र॑क्षि॒ता यश्चाधि॑पति॒र्यश्च॑ गो॒प्ता ताभ्यां॒ नम॒स्तौ नो॑ मृडयता॒न्ते यं द्वि॒ष्मो यश्च॑ नो॒ द्वेष्टि॒ तं वां॒ जम्भे॑ दधाम्योज॒स्विनी॒ नामा॑सि दक्षि॒णा दिक्तस्या᳚स्त॒ इन्द्रो\-ऽधि॑पतिः॒ पृदा॑कुः॒ प्राची॒ नामा॑सि प्र॒तीची॒ दिक्तस्या᳚स्ते~(४२)

%5.5.10.2
सोमो\-ऽधि॑पतिः स्व॒जो॑\-ऽव॒स्थावा॒ नामा॒स्युदी॑ची॒ दिक्तस्या᳚स्ते॒ वरु॒णो\-ऽधि॑पतिस्ति॒रश्च॑राजि॒रधि॑पत्नी॒ नामा॑सि बृह॒ती दिक्तस्या᳚स्ते॒ बृह॒स्पति॒रधि॑पतिः श्वि॒त्रो व॒शिनी॒ नामा॑सी॒यं दिक्तस्या᳚स्ते य॒मो\-ऽधि॑पतिः क॒ल्माष॑ग्रीवो रक्षि॒ता यश्चाधि॑पति॒र्यश्च॑ गो॒प्ता ताभ्यां॒ नम॒स्तौ नो॑ मृडयता॒न्ते यं द्वि॒ष्मो यश्च॑~(४३)

%5.5.10.3
नो॒ द्वेष्टि॒ तं वां॒ जम्भे॑ दधाम्ये॒ता वै दे॒वता॑ अ॒ग्निं चि॒तꣳ र॑क्षन्ति॒ ताभ्यो॒ यदाहु॑ती॒र्न जु॑हु॒याद॑ध्व॒र्युं च॒ यज॑मानं च ध्यायेयु॒र्यदे॒ता आहु॑तीर्जु॒होति॑ भाग॒धेये॑नै॒वैना᳚ञ्छमयति॒ नार्ति॒मार्च्छ॑त्यध्व॒र्युर्न यज॑मानो हे॒तयो॒ नाम॑ स्थ॒ तेषां᳚ वः पु॒रो गृ॒हा अ॒ग्निर्व॒ इष॑वः सलि॒लो निलि॒म्पा नाम॑~(४४)

%5.5.10.4
स्थ॒ तेषां᳚ वो दक्षि॒णा गृ॒हाः पि॒तरो॑ व॒ इष॑वः॒ सग॑रो व॒ज्रिणो॒ नाम॑ स्थ॒ तेषां᳚ वः प॒श्चाद्गृ॒हाः स्वप्नो॑ व॒ इष॑वो॒ गह्व॑रो\-ऽव॒स्थावा॑नो॒ नाम॑ स्थ॒ तेषां᳚ व उत्त॒राद्गृ॒हा आपो॑ व॒ इष॑वः समु॒द्रो\-ऽधि॑पतयो॒ नाम॑ स्थ॒ तेषां᳚ व उ॒परि॑ गृ॒हा व॒र्\mbox{}षं व॒ इष॒वो\-ऽव॑स्वान्क्र॒व्या नाम॑ स्थ॒ पार्थि॑वा॒स्तेषां᳚ व इ॒ह गृ॒हाः~(४५)

%5.5.10.5
अन्नं॑ व॒ इष॑वो निमि॒षो वा॑तना॒मन्तेभ्यो॑ वो॒ नम॒स्ते नो॑ मृडयत॒ ते यं द्वि॒ष्मो यश्च॑ नो॒ द्वेष्टि॒ तं वो॒ जम्भे॑ दधामि हु॒तादो॒ वा अ॒न्ये दे॒वा अ॑हु॒तादो॒\-ऽन्ये तान॑ग्नि॒चिदे॒वोभया᳚न्प्रीणाति द॒ध्ना म॑धुमि॒श्रेणै॒ता आहु॑तीर्जुहोति भाग॒धेये॑नै॒वैना᳚न्प्रीणा॒त्यथो॒ खल्वा॑हु॒रिष्ट॑का॒ वै दे॒वा अ॑हु॒ताद॒ इति॑~(४६)

%5.5.10.6
अ॒नु॒प॒रि॒क्रामं॑ जुहो॒त्यप॑रिवर्गमे॒वैना᳚न्प्रीणाती॒मꣴ स्तन॒मूर्ज॑स्वन्तं धया॒पाम्प्रप्या॑तमग्ने सरि॒रस्य॒ मध्ये᳚। उथ्सं॑ जुषस्व॒ मधु॑मन्तमूर्व समु॒द्रिय॒ꣳ॒ सद॑न॒मा वि॑शस्व। यो वा अ॒ग्निम्प्र॒युज्य॒ न वि॑मु॒ञ्चति॒ यथाश्वो॑ यु॒क्तो\-ऽवि॑मुच्यमानः॒ क्षुध्य॑न्परा॒भव॑त्ये॒वम॑स्या॒ग्निः परा॑ भवति॒ तं प॑रा॒भव॑न्तं॒ यज॑मा॒नो\-ऽनु॒ परा॑ भवति॒ सो᳚\-ऽग्निं चि॒त्वा लू॒क्षः~(४७)

%5.5.10.7
भ॒व॒ती॒मꣴ स्तन॒मूर्ज॑स्वन्तं धया॒पामित्याज्य॑स्य पू॒र्णाꣴ स्रुचं॑ जुहोत्ये॒ष वा अ॒ग्नेर्वि॑मो॒को वि॒मुच्यै॒वास्मा॒ अन्न॒मपि॑ दधाति॒ तस्मा॑दाहु॒र्यश्चै॒वं वेद॒ यश्च॒ न सु॒धायꣳ॑ ह॒ वै वा॒जी सुहि॑तो दधा॒तीत्य॒ग्निर्वाव वा॒जी तमे॒व तत्प्री॑णाति॒ स ए॑नम्प्री॒तः प्री॑णाति॒ वसी॑यान्भवति॥~(४८)

%5.5.11.0
{\anuvakamend[{प्र॒तीची॒ दिक्तस्या᳚स्ते द्वि॒ष्मो यश्च॑ निलि॒म्पा नामे॒ह गृ॒हा इति॑ लू॒क्षो वसी॑यान्भवति}]}%॥10॥

%5.5.11.1
इन्द्रा॑य॒ राज्ञे॑ सूक॒रो वरु॑णाय॒ राज्ञे॒ कृष्णो॑ य॒माय॒ राज्ञ॒ ऋश्य॑ ऋष॒भाय॒ राज्ञे॑ गव॒यः शा᳚र्दू॒लाय॒ राज्ञे॑ गौ॒रः पु॑रुषरा॒जाय॑ म॒र्कटः॑ क्षिप्रश्ये॒नस्य॒ वर्ति॑का॒ नीलं॑गोः॒ क्रिमिः॒ सोम॑स्य॒ राज्ञः॑ कुलु॒ङ्गः सिन्धोः᳚ शिꣳशु॒मारो॑ हि॒मव॑तो ह॒स्ती॥~(४९)

%5.5.12.0
{\anuvakamend[{इन्द्रा॑या॒ष्टाविꣳ॑शतिः}]}%॥11॥

%5.5.12.1
म॒युः प्रा॑जाप॒त्य ऊ॒लो हली᳚क्ष्णो वृषद॒ꣳ॒शस्ते धा॒तुः सर॑स्वत्यै॒ शारिः॑ श्ये॒ता पु॑रुष॒वाख्सर॑स्वते॒ शुकः॑ श्ये॒तः पु॑रुष॒वागा॑र॒ण्यो॑\-ऽजो न॑कु॒लः शका॒ ते पौ॒ष्णा वा॒चे क्रौ॒ञ्चः॥~(५०)

%5.5.13.0
{\anuvakamend[{म॒युस्त्रयो॑विꣳशतिः}]}%॥12॥

%5.5.13.1
अ॒पां नप्त्रे॑ ज॒षो ना॒क्रो मक॑रः कुली॒कय॒स्ते\-ऽकू॑पारस्य वा॒चे पै᳚ङ्गरा॒जो भगा॑य कु॒षीत॑क आ॒ती वा॑ह॒सो दर्वि॑दा॒ ते वा॑य॒व्या॑ दि॒ग्भ्यश्च॑क्रवा॒कः॥~(५१)

%5.5.14.0
{\anuvakamend[{अ॒पामेका॒न्नविꣳ॑शतिः}]}%॥13॥

%5.5.14.1
बला॑याजग॒र आ॒खुः सृ॑ज॒या श॒यण्ड॑क॒स्ते मै॒त्रा मृ॒त्यवे॑\-ऽसि॒तो म॒न्यवे᳚ स्व॒जः कु॑म्भी॒नसः॑ पुष्करसा॒दो लो॑हिता॒हिस्ते त्वा॒ष्ट्राः प्र॑ति॒श्रुत्का॑यै वाह॒सः॥~(५२)

%5.5.15.0
{\anuvakamend[{}]}

%5.5.15.1
पु॒रु॒ष॒मृ॒गश्च॒न्द्रम॑से गो॒धा काल॑का दार्वाघा॒टस्ते वन॒स्पती॑नामे॒ण्यह्ने॒ कृष्णो॒ रात्रि॑यै पि॒कः क्ष्विङ्का॒ नील॑शीर्ष्णी॒ ते᳚\-ऽर्य॒म्णे धा॒तुः क॑त्क॒टः॥~(५३)

%5.5.16.0
{\anuvakamend[{}]}

%5.5.16.1
सौ॒री ब॒लाकर्श्यो॑ म॒यूरः॑ श्ये॒नस्ते ग॑न्ध॒र्वाणां॒ वसू॑नां क॒पिञ्ज॑लो रु॒द्राणां᳚ तित्ति॒री रो॒हित्कु॑ण्डृ॒णाची॑ गो॒लत्ति॑का॒ ता अ॑फ्स॒रसा॒मर॑ण्याय सृम॒रः॥~(५४)

%5.5.17.0
{\anuvakamend[{}]}

%5.5.17.1
पृ॒ष॒तो वै᳚श्वदे॒वः पि॒त्वो न्यङ्कुः॒ कश॒स्ते\-ऽनु॑मत्या अन्यवा॒पो᳚\-ऽर्धमा॒साना᳚म्मा॒सां क॒श्यपः॒ क्वयिः॑ कु॒टरु॑र्दात्यौ॒हस्ते सि॑नीवा॒ल्यै बृह॒स्पत॑ये शित्पु॒टः॥~(५)

%5.5.18.0
{\anuvakamend[{}]}

%5.5.18.1
शका॑ भौ॒मी पा॒न्त्रः कशो॑ मान्थी॒लव॒स्ते पि॑तृ॒णामृ॑तू॒नां जह॑का संवथ्स॒राय॒ लोपा॑ क॒पोत॒ उलू॑कः श॒शस्ते नैर्॑\mbox{}॑ऋताः कृ॑क॒वाकुः॑ सावि॒त्रः॥~(५६)

%5.5.19.0
{\anuvakamend[{बला॑य पुरुषमृ॒गः सौ॒री पृ॑ष॒तः शका॒ष्टाद॑शा॒ष्टाद॑श}]}%॥14-18॥

%5.5.19.1
रुरू॑ रौ॒द्रः कृ॑कला॒सः श॒कुनिः॒ पिप्प॑का॒ ते श॑र॒व्या॑यै हरि॒णो मा॑रु॒तो ब्रह्म॑णे शा॒र्गस्त॒रक्षुः॑ कृ॒ष्णः श्वा च॑तुर॒क्षो ग॑र्द॒भस्त इ॑तरज॒नाना॑म॒ग्नये॒ धूङ्क्ष्णा᳚॥~(५७)

%5.5.20.0
{\anuvakamend[{रुरु॑र्विꣳश॒तिः}]}%॥19॥

%5.5.20.1
अ॒ल॒ज आ᳚न्तरि॒क्ष उ॒द्रो म॒द्गुः प्ल॒वस्ते॑\-ऽपामदि॑त्यै हꣳस॒साचि॑रिन्द्रा॒ण्यै कीर्\mbox{}शा॒ गृध्रः॑ शितिक॒क्षी वा᳚र्ध्राण॒सस्ते दि॒व्या द्या॑वापृथि॒व्या᳚ श्वा॒वित्॥~(५८)

%5.5.21.0
{\anuvakamend[{}]}

%5.5.21.1
सु॒प॒र्णः पा᳚र्ज॒न्यो ह॒ꣳ॒सो वृको॑ वृषद॒ꣳ॒शस्त ऐ॒न्द्रा अ॒पामु॒द्रो᳚\-ऽर्य॒म्णे लो॑पा॒शः सि॒ꣳ॒हो न॑कु॒लो व्या॒घ्रस्ते म॑हे॒न्द्राय॒ कामा॑य॒ पर॑स्वान्॥~(५९)

%5.5.22.0
{\anuvakamend[{अ॒ल॒जः सु॑प॒र्णो᳚\-ऽष्टाद॑शाष्टा॒द॑श}]}%॥21॥

%5.5.22.1
आ॒ग्ने॒यः कृ॒ष्णग्री॑वः सारस्व॒ती मे॒षी ब॒भ्रुः सौ॒म्यः पौ॒ष्णः श्या॒मः शि॑तिपृ॒ष्ठो बा॑र्\mbox{}हस्प॒त्यः शि॒ल्पो वै᳚श्वदे॒व ऐ॒न्द्रो॑\-ऽरु॒णो मा॑रु॒तः क॒ल्माष॑ ऐन्द्रा॒ग्नः सꣳ॑हि॒तो॑\-ऽधोरा॑मः सावि॒त्रो वा॑रु॒णः पेत्वः॑॥~(६०)

%5.5.23.0
{\anuvakamend[{आ॒ग्ने॒यो द्वाविꣳ॑शतिः}]}%॥22॥

%5.5.23.1
अश्व॑स्तूप॒रो गो॑मृ॒गस्ते प्रा॑जाप॒त्या आ᳚ग्ने॒यौ कृ॒ष्णग्री॑वौ त्वा॒ष्ट्रौ लो॑मशस॒क्थौ शि॑तिपृ॒ष्ठौ बा॑र्\mbox{}हस्प॒त्यौ धा॒त्रे पृ॑षोद॒रः सौ॒र्यो ब॒लक्षः॒ पेत्वः॑॥~(६१)

%5.5.24.0
{\anuvakamend[{अश्वः॒ षोड॑श}]}%॥23॥

%5.5.24.1
अ॒ग्नये\-ऽनी॑कवते॒ रोहि॑ताञ्जिरन॒ड्वान॒धोरा॑मौ सावि॒त्रौ पौ॒ष्णौ र॑ज॒तना॑भी वैश्वदे॒वौ पि॒शङ्गौ॑ तूप॒रौ मा॑रु॒तः क॒ल्माष॑ आग्ने॒यः कृ॒ष्णो॑\-ऽजः सा॑रस्व॒ती मे॒षी वा॑रु॒णः कृ॒ष्ण एक॑शितिपा॒त्पेत्वः॑~(६२)

%5.6.0.0

%5.6.0.0
{\anuvakamend[{अ॒ग्नयो\-ऽनी॑कवते॒ द्वाविꣳ॑शतिः}]}%॥24॥

{\anuvakamend[{हिर॑ण्यवर्णा अ॒पां ग्रहा᳚न्भूतेष्ट॒काः स॒जूः सं॑वथ्स॒रं प्र॒जा\-प॑तिः॒ स क्षु॒रप॑विर॒ग्नेर्वै दी॒क्षया॑ सुव॒र्गाय॒ तं यन्न सू॒यते᳚ प्र॒जा\-प॑तिर्\mbox{}ऋ॒तुभी॒ रोहि॑तः॒ पृश्ञिः॑ शितिबा॒हुरु॑न्न॒तः क॒र्णाः शु॒ण्ठा इन्द्रा॒यादि॑त्यै सौ॒म्या वा॑रु॒णाः सोमा॒यैका॑\-दश पि॒शङ्गा॒स्त्रयो॑विꣳशतिः}]}%॥23॥
{\prashnaend{ हिर॑ण्यवर्णा भूतेष्ट॒काश्छन्दो॒ यत्कनी॑याꣳसन्त्रि॒वृद्ध्य॑ग्निर्वा॑रु॒णाश्चतुः॑पञ्चाशत्॥54॥ हिर॑ण्यवर्णा॒ निव॑क्षसः॥}}
%%% END PRASHNA
