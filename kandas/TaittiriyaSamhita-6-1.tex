\sect{प्रथमः प्रश्नः}\setcounter{anuvakam}{0}
\dnsub{तैत्तिरीयसंहितायां षष्ठमकाण्डे प्रथमः प्रश्नः}
%6.1.1.0
%6.1.1.1
प्रा॒चीन॑वꣳशं करोति देवमनु॒ष्या दिशो॒ व्य॑भजन्त॒ प्राचीं᳚ दे॒वा द॑क्षि॒णा पि॒तरः॑ प्र॒तीची᳚म्मनु॒ष्या॑ उदी॑चीꣳ रु॒द्रा यत्प्रा॒चीन॑वꣳशं क॒रोति॑ देवलो॒कमे॒व तद्यज॑मान उ॒पाव॑र्तते॒ परि॑ श्रयत्य॒न्तर्\mbox{}हि॑तो॒ हि दे॑वलो॒को म॑नुष्यलो॒का\-न्नास्माल्लो॒काथ्स्वे॑तव्यमि॒वेत्या॑हुः॒ को हि तद्वेद॒ यद्य॒मुष्मि॑ल्लोँ॒के\-ऽस्ति॑ वा॒ न वेति॑ दि॒क्ष्वती॑का॒शान्क॑रोति~(१)

%6.1.1.2
उ॒भयो᳚र्लो॒कयो॑र॒भिजि॑त्यै केशश्म॒श्रु व॑पते न॒खानि॒ नि कृ॑न्तते मृ॒ता वा ए॒षा त्वग॑मे॒ध्या यत्के॑शश्म॒श्रु मृ॒तामे॒व त्वच॑ममे॒ध्याम॑प॒हत्य॑ य॒ज्ञियो॑ भू॒त्वा मेध॒मुपै॒त्यङ्गि॑रसः सुव॒र्गं लो॒कं यन्तो॒\-ऽफ्सु दी᳚क्षात॒पसी॒ प्रावे॑शयन्न॒फ्सु स्ना॑ति सा॒क्षादे॒व दी᳚क्षात॒पसी॒ अव॑ रुन्द्धे ती॒र्थे स्ना॑ति ती॒र्थे हि ते ताम्प्रावे॑शयन्ती॒र्थे स्ना॑ति~(२)

%6.1.1.3
ती॒र्थमे॒व स॑मा॒नानां᳚ भवत्य॒पो᳚\-ऽश्ञात्यन्तर॒त ए॒व मेध्यो॑ भवति॒ वास॑सा दीक्षयति सौ॒म्यं वै क्षौमं॑ दे॒वत॑या॒ सोम॑मे॒ष दे॒वता॒मुपै॑ति॒ यो दीक्ष॑ते॒ सोम॑स्य त॒नूर॑सि त॒नुवं॑ मे पा॒हीत्या॑ह॒ स्वामे॒व दे॒वता॒मुपै॒त्यथो॑ आ॒शिष॑मे॒वैतामा शा᳚स्ते॒\-ऽग्नेस्तू॑षा॒धानं॑ वा॒योर्वा॑त॒पान॑म्पितृ॒णान्नी॒विरोष॑धीनाम्प्रघा॒तः~(३)

%6.1.1.4
आ॒दि॒त्यानां᳚ प्राचीनता॒नो विश्वे॑षां दे॒वाना॒मोतु॒र्नक्ष॑त्राणामतीका॒शास्तद्वा ए॒तथ्स॑र्वदेव॒त्यं॑ यद्वासो॒ यद्वास॑सा दी॒क्षय॑ति॒ सर्वा॑भिरे॒वैनं॑ दे॒वता॑भिर्दीक्षयति ब॒हिःप्रा॑णो॒ वै म॑नु॒ष्य॑स्तस्याश॑नं प्रा॒णो᳚\-ऽश्ञाति॒ सप्रा॑ण ए॒व दी᳚क्षत॒ आशि॑तो भवति॒ यावा॑ने॒वास्य॑ प्रा॒णस्तेन॑ स॒ह मेध॒मुपै॑ति घृ॒तं दे॒वाना॒म्मस्तु॑ पितृ॒णान्निष्प॑क्वम्मनु॒ष्या॑णा॒न्तद्वै~(४)

%6.1.1.5
ए॒तथ्स॑र्वदेव॒त्यं॑ यन्नव॑नीतं॒ यन्नव॑नीतेनाभ्य॒ङ्क्ते सर्वा॑ ए॒व दे॒वताः᳚ प्रीणाति॒ प्रच्यु॑तो॒ वा ए॒षो᳚\-ऽस्माल्लो॒कादग॑तो देवलो॒कं यो दी᳚क्षि॒तो᳚\-ऽन्त॒रेव॒ नव॑नीत॒न्तस्मा॒न्नव॑नीतेना॒भ्य॑ङ्क्ते\-ऽनुलो॒मं यजु॑षा॒ व्यावृ॑त्त्या॒ इन्द्रो॑ वृ॒त्रम॑ह॒न्तस्य॑ क॒नीनि॑का॒ परा॑पत॒त्तदाञ्ज॑नमभव॒द्यदा॒ङ्क्ते चक्षु॑रे॒व भ्रातृ॑व्यस्य वृङ्क्ते॒ दक्षि॑ण॒म्पूर्व॒माङ्क्ते᳚~(५)

%6.1.1.6
स॒व्यꣳ हि पूर्व॑म्मनु॒ष्या॑ आ॒ञ्जते॒ न नि धा॑वते॒ नीव॒ हि म॑नु॒ष्या॑ धाव॑न्ते॒ पञ्च॒ कृत्व॒ आङ्क्ते॒ पञ्चा᳚क्षरा प॒ङ्क्तिः पाङ्क्तो॑ य॒ज्ञो य॒ज्ञमे॒वाव॑ रुन्द्धे॒ परि॑मित॒माङ्क्ते\-ऽप॑रिमित॒ꣳ॒ हि म॑नु॒ष्या॑ आ॒ञ्जते॒ सतू॑ल॒याङ्क्ते\-ऽप॑तूलया॒ हि म॑नु॒ष्या॑ आ॒ञ्जते॒ व्यावृ॑त्त्यै॒ यदप॑तूलयाञ्जी॒त वज्र॑ इव स्या॒थ्सतू॑ल॒याङ्क्ते॑ मित्र॒त्वाय॑~(६)

%6.1.1.7
इन्द्रो॑ वृ॒त्रम॑ह॒न्थ्सो\-ऽ पो\-ऽ भ्य॑म्रियत॒ तासां॒ यन्मेध्यं॑ य॒ज्ञिय॒ꣳ॒ सदे॑व॒मासी॒त्तदपोद॑क्राम॒त्ते द॒र्भा अ॑भव॒न्॒ यद्द॑र्भपुञ्जी॒लैः प॒वय॑ति॒ या ए॒व मेध्या॑ य॒ज्ञियाः॒ सदे॑वा॒ आप॒स्ताभि॑रे॒वैन॑म्पवयति॒ द्वा\-भ्यां᳚ पवयत्यहोरा॒त्राभ्या॑मे॒वैन॑म्पवयति त्रि॒भिः प॑वयति॒ त्रय॑ इ॒मे लो॒का ए॒भिरे॒वैनं॑ लो॒कैः प॑वयति प॒ञ्चभिः॑~(७)

%6.1.1.8
प॒व॒य॒ति॒ पञ्चा᳚क्षरा प॒ङ्क्तिः पाङ्क्तो॑ य॒ज्ञो य॒ज्ञायै॒वैन॑म्पवयति ष॒ड्भिः प॑वयति॒ षड्वा ऋ॒तव॑ ऋ॒तुभि॑रे॒वैन॑म्पवयति स॒प्तभिः॑ पवयति स॒प्त छन्दाꣳ॑सि॒ छन्दो॑भिरे॒वैन॑म्पवयति न॒वभिः॑ पवयति॒ नव॒ वै पुरु॑षे प्रा॒णाः सप्रा॑णमे॒वैन॑म्पव\-य॒त्येक॑विꣳशत्या पवयति॒ दश॒ हस्त्या॑ अ॒ङ्गुल॑यो॒ दश॒ पद्या॑ आ॒त्मैक॑वि॒ꣳ॒शो यावा॑ने॒व पुरु॑ष॒स्तमप॑रिवर्गम्~(८)

%6.1.1.9
प॒व॒य॒ति॒ चि॒त्पति॑स्त्वा पुना॒त्वित्या॑ह॒ मनो॒ वै चि॒त्पति॒र्मन॑सै॒वैन॑म्पवयति वा॒क्पति॑स्त्वा पुना॒त्वित्या॑ह वा॒चैवैन॑म्पवयति दे॒वस्त्वा॑ सवि॒ता पु॑ना॒त्वित्या॑ह सवि॒तृप्र॑सूत ए॒वैन॑म्पवयति॒ तस्य॑ ते पवित्रपते प॒वित्रे॑ण॒ यस्मै॒ कम्पु॒ने तच्छ॑केय॒मित्या॑हा॒शिष॑मे॒वैतामा शा᳚स्ते॥~(९)

%6.1.2.0
{\anuvakamend[{अ॒ती॒का॒शान्क॑रो॒त्यवे॑शयन्ती॒र्थे स्ना॑ति प्रघा॒तो म॑नु॒ष्या॑णा॒न्तद्वा आङ्क्ते॑ मित्र॒त्वाय॑ प॒ञ्चभि॒रप॑रिवर्गम॒ष्टाच॑त्वारिꣳशच्च}]}%~(१)

%6.1.2.1
याव॑न्तो॒ वै दे॒वा य॒ज्ञायापु॑नत॒ त ए॒वाभ॑व॒न्॒ य ए॒वं वि॒द्वान् य॒ज्ञाय॑ पुनी॒ते भव॑त्ये॒व ब॒हिः प॑वयि॒त्वान्तः प्र पा॑दयति मनुष्यलो॒क ए॒वैन॑म्पवयि॒त्वा पू॒तन्दे॑वलो॒कम्प्र ण॑य॒त्यदी᳚क्षित॒ एक॒याहु॒त्येत्या॑हुः स्रु॒वेण॒ चत॑स्रो जुहोति दीक्षित॒त्वाय॑ स्रु॒चा प॑ञ्च॒मीं पञ्चा᳚क्षरा प॒ङ्क्तिः पाङ्क्तो॑ य॒ज्ञो य॒ज्ञमे॒वाव॑ रुन्द्ध॒ आकू᳚त्यै प्र॒युजे॒\-ऽग्नये᳚~(१०)

%6.1.2.2
स्वाहेत्या॒हाकू᳚त्या॒ हि पुरु॑षो य॒ज्ञम॒भि प्र॑यु॒ङ्क्ते यजे॒येति॑ मे॒धायै॒ मन॑से॒\-ऽग्नये॒ स्वाहेत्या॑ह मे॒धया॒ हि मन॑सा॒ पुरु॑षो य॒ज्ञम॑भि॒गच्छ॑ति॒ सर॑स्वत्यै पू॒ष्णे᳚\-ऽग्नये॒ स्वाहेत्या॑ह॒ वाग्वै सर॑स्वती पृथि॒वी पू॒षा वा॒चैव पृ॑थि॒व्या य॒ज्ञम्प्र यु॑ङ्क्त॒ आपो॑ देवीर्बृहतीर्विश्वशम्भुव॒ इत्या॑ह॒ या वै वर्ष्या॒स्ताः~(११)

%6.1.2.3
आपो॑ दे॒वीर्बृ॑ह॒तीर्वि॒श्वश॑म्भुवो॒ यदे॒तद्यजु॒र्न ब्रू॒याद्दि॒व्या आपो\-ऽशा᳚न्ता इ॒मल्लोँ॒कमा ग॑च्छेयु॒रापो॑ देवीर्बृहतीर्विश्वशम्भुव॒ इत्या॑हा॒स्मा ए॒वैना॑ लो॒काय॑ शमयति॒ तस्मा᳚च्छा॒न्ता इ॒मल्लोँ॒कमा ग॑च्छन्ति॒ द्यावा॑पृथि॒वी इत्या॑ह॒ द्यावा॑पृथि॒व्योर्\mbox{}हि य॒ज्ञ उ॒र्व॑न्तरि॑क्ष॒मित्या॑हा॒न्तरि॑क्षे॒ हि य॒ज्ञो बृह॒स्पति॑र्नो ह॒विषा॑ वृधातु~(१२)

%6.1.2.4
इत्या॑ह॒ ब्रह्म॒ वै दे॒वाना॒म्बृह॒स्पति॒र्ब्रह्म॑णै॒वास्मै॑ य॒ज्ञमव॑ रुन्द्धे॒ यद्ब्रू॒याद्वि॑धे॒रिति॑ यज्ञस्था॒णुमृ॑च्छेद्वृधा॒त्वित्या॑ह यज्ञस्था॒णुमे॒व परि॑ वृणक्ति प्र॒जाप॑तिर्य॒ज्ञम॑सृजत॒ सो᳚\-ऽस्माथ्सृ॒ष्टः परा॑ङै॒थ्स प्र यजु॒रव्ली॑ना॒त्प्र साम॒ तमृगुद॑यच्छ॒द्यदृगु॒दय॑च्छ॒त्तदौ᳚द्ग्रह॒णस्यौ᳚द्ग्रहण॒त्वमृ॒चा~(१३)

%6.1.2.5
जु॒हो॒ति॒ य॒ज्ञस्योद्य॑त्या अनु॒ष्टुप्छन्द॑सा॒मुद॑यच्छ॒दित्या॑हु॒स्तस्मा॑दनु॒ष्टुभा॑ जुहोति य॒ज्ञस्योद्य॑त्यै॒ द्वाद॑श वाथ्सब॒न्धान्युद॑यच्छ॒न्नित्या॑हु॒स्तस्मा᳚द्द्वाद॒शभि॑र्वाथ्सबन्ध॒विदो॑ दीक्षयन्ति॒ सा वा ए॒षर्ग॑नु॒ष्टुग्वाग॑नु॒ष्टुग्यदे॒तय॒र्चा दी॒क्षय॑ति वा॒चैवैन॒ꣳ॒ सर्व॑या दीक्षयति॒ विश्वे॑ दे॒वस्य॑ ने॒तुरित्या॑ह सावि॒त्र्ये॑तेन॒ मर्तो॑ वृणीत स॒ख्यम्~(१४)

%6.1.2.6
इत्या॑ह पितृदेव॒त्यै॑तेन॒ विश्वे॑ रा॒य इ॑षुध्य॒सीत्या॑ह वैश्वदे॒व्ये॑तेन॑ द्यु॒म्नं वृ॑णीत पु॒ष्यस॒ इत्या॑ह पौ॒ष्ण्ये॑तेन॒ सा वा ए॒षर्ख्स॑र्वदेव॒त्या॑ यदे॒तय॒र्चा दी॒क्षय॑ति॒ सर्वा॑भिरे॒वैनं॑ दे॒वता॑भिर्दीक्षयति स॒प्ताक्ष॑रम्प्रथ॒मम्प॒दम॒ष्टाक्ष॑राणि॒ त्रीणि॒ यानि॒ त्रीणि॒ तान्य॒ष्टावुप॑ यन्ति॒ यानि॑ च॒त्वारि॒ तान्य॒ष्टौ यद॒ष्टाक्ष॑रा॒ तेन॑~(१५)

%6.1.2.7
गा॒य॒त्री यदेका॑\-दशाक्षरा॒ तेन॑ त्रि॒ष्टुग्यद्द्वाद॑शाक्षरा॒ तेन॒ जग॑ती॒ सा वा ए॒षर्ख्सर्वा॑णि॒ छन्दाꣳ॑सि॒ यदे॒तय॒र्चा दी॒क्षय॑ति॒ सर्वे॑भिरे॒वैनं॒ छन्दो॑भिर्दीक्षयति स॒प्ताक्ष॑रम्प्रथ॒मम्प॒दꣳ स॒प्तप॑दा॒ शक्व॑री प॒शवः॒ शक्व॑री प॒शूने॒वाव॑ रुन्द्ध॒ एक॑स्माद॒क्षरा॒दना᳚प्तम्प्रथ॒मम्प॒दन्तस्मा॒द्यद्वा॒चो\-ऽना᳚प्त॒न्तन्म॑नु॒ष्या॑ उप॑ जीवन्ति पू॒र्णया॑ जुहोति पू॒र्ण इ॑व॒ हि प्र॒जाप॑तिः प्र॒जाप॑ते॒राप्त्यै॒ न्यू॑नया जुहोति॒ न्यू॑ना॒द्धि प्र॒जाप॑तिः प्र॒जा असृ॑जत प्र॒जाना॒ꣳ॒ सृष्ट्यै᳚॥~(१६)

%6.1.3.0
{\anuvakamend[{अ॒ग्नये॒ ता वृ॑धात्वृ॒चा स॒ख्यन्तेन॑ जुहोति॒ पञ्च॑दश च}]}%~(२)

%6.1.3.1
ऋ॒ख्सा॒मे वै दे॒वेभ्यो॑ य॒ज्ञायाति॑ष्ठमाने॒ कृष्णो॑ रू॒पं कृ॒त्वाप॒क्रम्या॑तिष्ठता॒न्ते॑\-ऽमन्यन्त॒ यं वा इ॒मे उ॑पाव॒र्थ्स्यतः॒ स इ॒दं भ॑विष्य॒तीति॒ ते उपा॑मन्त्रयन्त॒ ते अ॑होरा॒त्रयो᳚र्महि॒मान॑मपनि॒धाय॑ दे॒वानु॒पाव॑र्तेतामे॒ष वा ऋ॒चो वर्णो॒ यच्छु॒क्लं कृ॑ष्णाजि॒नस्यै॒ष साम्नो॒ यत्कृ॒ष्णमृ॑ख्सा॒मयोः॒ शिल्पे᳚ स्थ॒ इत्या॑हर्ख्सा॒मे ए॒वाव॑ रुन्ध ए॒षः~(१७)

%6.1.3.2
वा अह्नो॒ वर्णो॒ यच्छु॒क्लं कृ॑ष्णाजि॒नस्यै॒ष रात्रि॑या॒ यत्कृ॒ष्णं यदे॒वैन॑यो॒स्तत्र॒ न्य॑क्तं॒ तदे॒वाव॑ रुन्द्धे कृष्णाजि॒नेन॑ दीक्षयति॒ ब्रह्म॑णो॒ वा ए॒तद्रू॒पं यत्कृ॑ष्णाजि॒नं ब्रह्म॑णै॒वैनं॑ दीक्षयती॒मान्धिय॒ꣳ॒ शिक्ष॑माणस्य दे॒वेत्या॑ह यथाय॒जुरे॒वैतद्गर्भो॒ वा ए॒ष यद्दी᳚क्षि॒त उल्बं॒ वासः॒ प्रोर्णु॑ते॒ तस्मा᳚त्~(१८)

%6.1.3.3
गर्भाः॒ प्रावृ॑ता जायन्ते॒ न पु॒रा सोम॑स्य क्र॒यादपो᳚र्ण्वीत॒ यत्पु॒रा सोम॑स्य क्र॒याद॑पोर्ण्वी॒त गर्भाः᳚ प्र॒जानां᳚ परा॒पातु॑काः स्युः क्री॒ते सोमे\-ऽपो᳚र्णुते॒ जाय॑त ए॒व तदथो॒ यथा॒ वसी॑याꣳसम्प्रत्यपोर्णु॒ते ता॒दृगे॒व तदङ्गि॑रसः सुव॒र्गं लो॒कं यन्त॒ ऊर्जं॒ व्य॑भजन्त॒ ततो॒ यद॒त्यशि॑ष्यत॒ ते श॒रा अ॑भव॒न्नूर्ग्वै श॒रा यच्छ॑र॒मयी᳚~(१९)

%6.1.3.4
मेख॑ला॒ भव॒त्यूर्ज॑मे॒वाव॑ रुन्द्धे मध्य॒तः सन्न॑ह्यति मध्य॒त ए॒वास्मा॒ ऊर्जं॑ दधाति॒ तस्मा᳚न्मध्य॒त ऊ॒र्जा भु॑ञ्जत ऊ॒र्ध्वं वै पुरु॑षस्य॒ नाभ्यै॒ मेध्य॑मवा॒चीन॑ममे॒ध्यं यन्म॑ध्य॒तः सं॒नह्य॑ति॒ मेध्यं॑ चै॒वास्या॑मे॒ध्यं च॒ व्याव॑र्तय॒तीन्द्रो॑ वृ॒त्राय॒ वज्र॒म्प्राह॑र॒थ्स त्रे॒धा व्य॑भव॒थ्स्फ्यस्तृती॑य॒ꣳ॒ रथ॒स्तृती॑यं॒ यूप॒स्तृती॑यम्~(२०)

%6.1.3.5
ये᳚\-ऽन्तःश॒रा अशी᳚र्यन्त॒ ते श॒रा अ॑भव॒न्तच्छ॒राणाꣳ॑ शर॒त्वं वज्रो॒ वै श॒राः क्षुत्खलु॒ वै म॑नु॒ष्य॑स्य॒ भ्रातृ॑व्यो॒ यच्छ॑र॒मयी॒ मेख॑ला॒ भव॑ति॒ वज्रे॑णै॒व सा॒क्षात्क्षुध॒म्भ्रातृ॑व्यम्मध्य॒तो\-ऽप॑ हते त्रि॒वृद्भ॑वति त्रि॒वृद्वै प्रा॒णस्त्रि॒वृत॑मे॒व प्रा॒णम्म॑ध्य॒तो यज॑माने दधाति पृ॒थ्वी भ॑वति॒ रज्जू॑ना॒व्व्याँवृ॑त्त्यै॒ मेख॑लया॒ यज॑मानन्दीक्षयति॒ योक्त्रे॑ण॒ पत्नी᳚म्मिथुन॒त्वाय॑~(२१)

%6.1.3.6
य॒ज्ञो दक्षि॑णाम॒भ्य॑ध्याय॒त्ताꣳ सम॑भव॒त्तदिन्द्रो॑\-ऽचाय॒थ्सो॑\-ऽमन्यत॒ यो वा इ॒तो ज॑नि॒ष्यते॒ स इ॒दम्भ॑विष्य॒तीति॒ ताम्प्रावि॑श॒त्तस्या॒ इन्द्र॑ ए॒वाजा॑यत॒ सो॑\-ऽमन्यत॒ यो वै मदि॒तो\-ऽप॑रो जनि॒ष्यते॒ स इ॒दम्भ॑विष्य॒तीति॒ तस्या॑ अनु॒मृश्य॒ योनि॒माच्छि॑न॒थ्सा सू॒तव॑शाभव॒त्तथ्सू॒तव॑शायै॒ जन्म॑~(२२)

%6.1.3.7
ताꣳ हस्ते॒ न्य॑वेष्टयत॒ ताम्मृ॒गेषु॒ न्य॑दधा॒थ्सा कृ॑ष्णविषा॒णाभ॑व॒दिन्द्र॑स्य॒ योनि॑रसि॒ मा मा॑ हिꣳसी॒रिति॑ कृष्णविषा॒णाम्प्र य॑च्छति॒ सयो॑निमे॒व य॒ज्ञं क॑रोति॒ सयो॑नि॒न्दक्षि॑णा॒ꣳ॒ सयो॑नि॒मिन्द्रꣳ॑ सयोनि॒त्वाय॑ कृ॒ष्यै त्वा॑ सुस॒स्याया॒ इत्या॑ह॒ तस्मा॑दकृष्टप॒च्या ओष॑धयः पच्यन्ते सुपिप्प॒लाभ्य॒स्त्वौष॑धीभ्य॒ इत्या॑ह॒ तस्मा॒दोष॑धयः॒ फलं॑ गृह्णन्ति॒ यद्धस्ते॑न~(२३)

%6.1.3.8
क॒ण्डू॒येत॑ पामन॒म्भावु॑काः प्र॒जाः स्यु॒र्यथ्स्मये॑त नग्न॒म्भावु॑काः कृष्णविषा॒णया॑ कण्डूयते\-ऽपि॒गृह्य॑ स्मयते प्र॒जानां᳚ गोपी॒थाय॒ न पु॒रा दक्षि॑णाभ्यो॒ नेतोः᳚ कृष्णविषा॒णामव॑ चृते॒द्यत्पु॒रा दक्षि॑णाभ्यो॒ नेतोः᳚ कृष्णविषा॒णामव॑चृ॒तेद्योनिः॑ प्र॒जानां᳚ परा॒पातु॑का स्यान्नी॒तासु॒ दक्षि॑णासु॒ चात्वा॑ले कृष्णविषा॒णाम्प्रास्य॑ति॒ योनि॒र्वै य॒ज्ञस्य॒ चात्वा॑लं॒ योनिः॑ कृष्णविषा॒णा योना॑वे॒व योनि॑न्दधाति य॒ज्ञस्य॑ सयोनि॒त्वाय॑॥~(२४)

%6.1.4.0
{\anuvakamend[{रु॒न्ध॒ ए॒ष तस्मा᳚च्छर॒मयी॒ यूप॒स्तृती॑यम्मिथुन॒त्वाय॒ जन्म॒ हस्ते॑ना॒ष्टाच॑त्वारिꣳशच्च}]}%~(३)

%6.1.4.1
वाग्वै दे॒वेभ्यो\-ऽपा᳚क्रामद्यज्ञा॒याति॑ष्ठमाना॒ सा वन॒स्पती॒न्प्रावि॑श॒थ्सैषा वाग्वन॒स्पति॑षु वदति॒ या दु॑न्दु॒भौ या तूण॑वे॒ या वीणा॑यां॒ यद्दी᳚क्षितद॒ण्डम्प्र॒यच्छ॑ति॒ वाच॑मे॒वाव॑ रुन्द्ध॒ औदु॑म्बरो भव॒त्यूर्ग्वा उ॑दु॒म्बर॒ ऊर्ज॑मे॒वाव॑ रुन्द्धे॒ मुखे॑न॒ सम्मि॑तो भवति मुख॒त ए॒वास्मा॒ ऊर्जं॑ दधाति॒ तस्मा᳚न्मुख॒त ऊ॒र्जा भु॑ञ्जते~(२५)

%6.1.4.2
क्री॒ते सोमे॑ मैत्रावरु॒णाय॑ द॒ण्डम्प्र य॑च्छति मैत्रावरु॒णो हि पु॒रस्ता॑दृ॒त्विग्भ्यो॒ वाचं॑ वि॒भज॑ति॒ तामृ॒त्विजो॒ यज॑माने॒ प्रति॑ ष्ठापयन्ति॒ स्वाहा॑ य॒ज्ञम्मन॒सेत्या॑ह॒ मन॑सा॒ हि पुरु॑षो य॒ज्ञम॑भि॒गच्छ॑ति॒ स्वाहा॒ द्यावा॑पृथि॒वीभ्या॒मित्या॑ह॒ द्यावा॑पृथि॒व्योर्\mbox{}हि य॒ज्ञः स्वाहो॒रोर॒न्तरि॑क्षा॒दित्या॑हा॒न्तरि॑क्षे॒ हि य॒ज्ञः स्वाहा॑ य॒ज्ञं वाता॒दार॑भ॒ इत्या॑हा॒यम्~(२६)

%6.1.4.3
वाव यः पव॑ते॒ स य॒ज्ञस्तमे॒व सा॒क्षादा र॑भते मु॒ष्टी क॑रोति॒ वाचं॑ यच्छति य॒ज्ञस्य॒ धृत्या॒ अदी᳚क्षिष्टा॒यम्ब्रा᳚ह्म॒ण इति॒ त्रिरु॑पा॒ꣳ॒श्वा॑ह दे॒वेभ्य॑ ए॒वैन॒म्प्राह॒ त्रिरु॒च्चैरु॒भये᳚भ्य ए॒वैनं॑ देवमनु॒ष्येभ्यः॒ प्राह॒ न पु॒रा नक्ष॑त्रेभ्यो॒ वाचं॒ वि सृ॑जे॒द्यत्पु॒रा नक्ष॑त्रेभ्यो॒ वाचं॑ विसृ॒जेद्य॒ज्ञं विच्छि॑न्द्यात्~(२७)

%6.1.4.4
उदि॑तेषु॒ नक्ष॑त्रेषु व्र॒तं कृ॑णु॒तेति॒ वाचं॒ वि सृ॑जति य॒ज्ञव्र॑तो॒ वै दी᳚क्षि॒तो य॒ज्ञमे॒वाभि वाचं॒ वि सृ॑जति॒ यदि॑ विसृ॒जेद्वै᳚ष्ण॒वीमृच॒मनु॑ ब्रूयाद्य॒ज्ञो वै विष्णु॑र्\mbox{}य॒ज्ञेनै॒व य॒ज्ञꣳ सं त॑नोति॒ दैवी॒न्धिय॑म्मनामह॒ इत्या॑ह य॒ज्ञमे॒व तन्म्र॑दयति सुपा॒रा नो॑ अस॒द्वश॒ इत्या॑ह॒ व्यु॑ष्टिमे॒वाव॑ रुन्द्धे~(२८)

%6.1.4.5
ब्र॒ह्म॒वा॒दिनो॑ वदन्ति होत॒व्यं॑ दीक्षि॒तस्य॑ गृ॒हा(३)इ न हो॑त॒व्या(३)मिति॑ ह॒विर्वै दी᳚क्षि॒तो यज्जु॑हु॒याद्यज॑मानस्याव॒दाय॑ जुहुया॒द्यन्न जु॑हु॒याद्य॑ज्ञप॒रुर॒न्तरि॑या॒द्ये दे॒वा मनो॑जाता मनो॒युज॒ इत्या॑ह प्रा॒णा वै दे॒वा मनो॑जाता मनो॒युज॒स्तेष्वे॒व प॒रोक्षं॑ जुहोति॒ तन्नेव॑ हु॒तं नेवाहु॑तꣴ स्व॒पन्तं॒ वै दी᳚क्षि॒तꣳ रक्षाꣳ॑सि जिघाꣳसन्त्य॒ग्निः~(२९)

%6.1.4.6
खलु॒ वै र॑क्षो॒हाग्ने॒ त्वꣳ सु जा॑गृहि व॒यꣳ सु म॑न्दिषीम॒हीत्या॑हा॒ग्निमे॒वाधि॒पां कृ॒त्वा स्व॑पिति॒ रक्ष॑सा॒मप॑हत्या अव्र॒त्यमि॑व॒ वा ए॒ष क॑रोति॒ यो दी᳚क्षि॒तः स्वपि॑ति॒ त्वम॑ग्ने व्रत॒पा अ॒सीत्या॑हा॒ग्निर्वै दे॒वानां᳚ व्र॒तप॑तिः॒ स ए॒वैनं॑ व्र॒तमाल॑म्भयति दे॒व आ मर्त्ये॒ष्वेत्या॑ह दे॒वः~(३०)

%6.1.4.7
ह्ये॑ष सन्मर्त्ये॑षु॒ त्वं य॒ज्ञेष्वीड्य॒ इत्या॑है॒तꣳ हि य॒ज्ञेष्वीड॒ते\-ऽप॒ वै दी᳚क्षि॒ताथ्सु॑षु॒पुष॑ इन्द्रि॒यं दे॒वताः᳚ क्रामन्ति॒ विश्वे॑ दे॒वा अ॒भि मामाव॑वृत्र॒न्नित्या॑हेन्द्रि॒येणै॒वैनं॑ दे॒वता॑भिः॒ सं न॑यति॒ यदे॒तद्यजु॒र्न ब्रू॒याद्याव॑त ए॒व प॒शून॒भि दीक्षे॑त॒ ताव॑न्तो\-ऽस्य प॒शवः॑ स्यू॒ रास्वेय॑त्~(३१)

%6.1.4.8
सो॒मा भूयो॑ भ॒रेत्या॒हाप॑रिमिताने॒व प॒शूनव॑ रुन्द्धे च॒न्द्रम॑सि॒ मम॒ भोगा॑य भ॒वेत्या॑ह यथादेव॒तमे॒वैनाः॒ प्रति॑ गृह्णाति वा॒यवे᳚ त्वा॒ वरु॑णाय॒ त्वेति॒ यदे॒वमे॒ता नानु॑दि॒शेदय॑थादेवतं॒ दक्षि॑णा गमये॒दा दे॒वता᳚भ्यो वृश्च्येत॒ यदे॒वमे॒ता अ॑नुदि॒शति॑ यथादेव॒तमे॒व दक्षि॑णा गमयति॒ न दे॒वता᳚भ्य॒ आ~(३२)

%6.1.4.9
वृ॒श्च्य॒ते॒ देवी॑रापो अपां नपा॒दित्या॑ह॒ यद्वो॒ मेध्यं॑ य॒ज्ञिय॒ꣳ॒ सदे॑वं॒ तद्वो॒ माव॑ क्रमिष॒मिति॒ वावैतदा॒हाच्छि॑न्नं॒ तन्तुं॑ पृथि॒व्या अनु॑ गेष॒मित्या॑ह॒ सेतु॑मे॒व कृ॒त्वात्ये॑ति॥~(३३)

%6.1.5.0
{\anuvakamend[{भु॒ञ्ज॒ते॒\-ऽयञ्छि॑न्द्याद्रुन्धे॒\-ऽग्निरा॑ह दे॒व इय॑द्दे॒वता᳚भ्य॒ आ त्रय॑स्त्रिꣳशच्च}]}%~(४)

%6.1.5.1
दे॒वा वै दे॑व॒यज॑नमध्यव॒साय॒ दिशो॒ न प्राजा॑न॒न्ते\-ऽ न्यो᳚न्यमुपा॑धाव॒न्त्वया॒ प्र जा॑नाम॒ त्वयेति॒ ते\-ऽदि॑त्या॒ꣳ॒ सम॑ध्रियन्त॒ त्वया॒ प्र जा॑ना॒मेति॒ साब्र॑वी॒द्वरं॑ वृणै॒ मत्प्रा॑यणा ए॒व वो॑ य॒ज्ञा मदु॑दयना अस॒न्निति॒ तस्मा॑दादि॒त्यः प्रा॑य॒णीयो॑ य॒ज्ञाना॑मादि॒त्य उ॑दय॒नीयः॒ पञ्च॑ दे॒वता॑ यजति॒ पञ्च॒ दिशो॑ दि॒शाम्प्रज्ञा᳚त्यै~(३४)

%6.1.5.2
अथो॒ पञ्चा᳚क्षरा प॒ङ्क्तिः पाङ्क्तो॑ य॒ज्ञो य॒ज्ञमे॒वाव॑ रुन्द्धे॒ पथ्याꣴ॑ स्व॒स्तिम॑यज॒न्प्राची॑मे॒व तया॒ दिश॒म्प्राजा॑नन्न॒ग्निना॑ दक्षि॒णा सोमे॑न प्र॒तीचीꣳ॑ सवि॒त्रोदी॑ची॒मदि॑त्यो॒र्ध्वाम्पथ्याꣴ॑ स्व॒स्तिं य॑जति॒ प्राची॑मे॒व तया॒ दिश॒म्प्र जा॑नाति॒ पथ्याꣴ॑ स्व॒स्तिमि॒ष्ट्वाग्नीषोमौ॑ यजति॒ चक्षु॑षी॒ वा ए॒ते य॒ज्ञस्य॒ यद॒ग्नीषोमौ॒ ताभ्या॑मे॒वानु॑ पश्यति~(३५)

%6.1.5.3
अ॒ग्नीषोमा॑वि॒ष्ट्वा स॑वि॒तारं॑ यजति सवि॒तृप्र॑सूत ए॒वानु॑ पश्यति सवि॒तार॑मि॒ष्ट्वादि॑तिं यजती॒यं वा अदि॑तिर॒स्यामे॒व प्र॑ति॒ष्ठायानु॑ पश्य॒त्यदि॑तिमि॒ष्ट्वा मा॑रु॒तीमृच॒मन्वा॑ह म॒रुतो॒ वै दे॒वानां॒ विशो॑ देववि॒शं खलु॒ वै कल्प॑मानम्मनुष्यवि॒श\-मनु॑ कल्पते॒ यन्मा॑रु॒तीमृच॑म॒न्वाह॑ वि॒शां कॢप्त्यै᳚ ब्रह्मवा॒दिनो॑ वदन्ति प्रया॒जव॑दननूया॒जम्प्रा॑य॒णीयं॑ का॒र्य॑मनूया॒जव॑त्~(३६)

%6.1.5.4
अ॒प्र॒या॒जमु॑दय॒नीय॒मिती॒मे वै प्र॑या॒जा अ॒मी अ॑नूया॒जाः सैव सा य॒ज्ञस्य॒ सन्त॑ति॒स्तत्तथा॒ न का॒र्य॑मा॒त्मा वै प्र॑या॒जाः प्र॒जानू॑या॒जा यत्प्र॑या॒जान॑न्तरि॒यादा॒त्मान॑म॒न्तरि॑या॒द्यद॑नूया॒जान॑न्तरि॒यात्प्र॒जाम॒न्तरि॑या॒द्यतः॒ खलु॒ वै य॒ज्ञस्य॒ वित॑तस्य॒ न क्रि॒यते॒ तदनु॑ य॒ज्ञः परा॑ भवति य॒ज्ञं प॑रा॒भव॑न्तं॒ यज॑मा॒नो\-ऽनु॑~(३७)

%6.1.5.5
परा॑ भवति प्रया॒जव॑दे॒वानू॑या॒जव॑त्प्राय॒णीयं॑ का॒र्य॑म्प्रया॒जव॑दनूया॒जव॑दुदय॒नीयं॒ नात्मान॑मन्त॒रेति॒ न प्र॒जां न य॒ज्ञः प॑रा॒भव॑ति॒ न यज॑मानः प्राय॒णीय॑स्य निष्का॒स उ॑दय॒नीय॑म॒भि निर्व॑पति॒ सैव सा य॒ज्ञस्य॒ सन्त॑ति॒र्याः प्रा॑य॒णीय॑स्य या॒ज्या॑ यत्ता उ॑दय॒नीय॑स्य या॒ज्याः᳚ कु॒र्यात्परा॑ङ॒मुं लो॒कमा रो॑हेत्प्र॒मायु॑कः स्या॒द्याः प्रा॑य॒णीय॑स्य पुरोनुवा॒क्या᳚स्ता उ॑दय॒नीय॑स्य या॒ज्याः᳚ करोत्य॒स्मिन्ने॒व लो॒के प्रति॑ तिष्ठति॥~(३८)

%6.1.6.0
{\anuvakamend[{प्रज्ञा᳚त्यै पश्यत्यनूया॒जव॒द्यज॑मा॒नो\-ऽनु॑ पुरोनुवा॒क्या᳚स्ता अ॒ष्टौ च॑}]}%~(५)

%6.1.6.1
क॒द्रूश्च॒ वै सु॑प॒र्णी चा᳚त्मरू॒पयो॑रस्पर्धेता॒ꣳ॒ सा क॒द्रूः सु॑प॒र्णीम॑जय॒थ्साब्र॑वीत्तृ॒तीय॑स्यामि॒तो दि॒वि सोम॒स्तमा ह॑र॒ तेना॒त्मानं॒ निष्क्री॑णी॒ष्वेती॒यं वै क॒द्रूर॒सौ सु॑प॒र्णी छन्दाꣳ॑सि सौपर्णे॒याः साब्र॑वीद॒स्मै वै पि॒तरौ॑ पु॒त्रान्बि॑भृतस्तृ॒तीय॑स्यामि॒तो दि॒वि सोम॒स्तमा ह॑र॒ तेना॒त्मानं॒ निष्क्री॑णीष्व~(३९)

%6.1.6.2
इति॑ मा क॒द्रूर॑वोच॒दिति॒ जग॒त्युद॑पत॒च्चतु॑र्दशाक्षरा स॒ती साप्रा᳚प्य॒ न्य॑वर्तत॒ तस्यै॒ द्वे अ॒क्षरे॑ अमीयेता॒ꣳ॒ सा प॒शुभि॑श्च दी॒क्षया॒ चाग॑च्छ॒त्तस्मा॒ज्जग॑ती॒ छन्द॑साम्पश॒व्य॑तमा॒ तस्मा᳚त्पशु॒मन्तं॑ दी॒क्षोप॑ नमति त्रि॒ष्टुगुद॑पत॒त्त्रयो॑दशाक्षरा स॒ती साप्रा᳚प्य॒ न्य॑वर्तत॒ तस्यै॒ द्वे अ॒क्षरे॑ अमीयेता॒ꣳ॒ सा दक्षि॑णाभिश्च~(४०)

%6.1.6.3
तप॑सा॒ चाग॑च्छ॒त्तस्मा᳚त्त्रि॒ष्टुभो॑ लो॒के माध्यं॑दिने॒ सव॑ने॒ दक्षि॑णा नीयन्त ए॒तत्खलु॒ वाव तप॒ इत्या॑हु॒र्यः स्वं ददा॒तीति॑ गाय॒त्र्युद॑पत॒च्चतु॑रक्षरा स॒त्य॑जया॒ ज्योति॑षा॒ तम॑स्या अ॒जाभ्य॑रुन्द्ध॒ तद॒जाया॑ अज॒त्वꣳ सा सोमं॒ चाह॑रच्च॒त्वारि॑ चा॒क्षरा॑णि साष्टाक्ष॑रा॒ सम॑पद्यत ब्रह्मवा॒दिनो॑ वदन्ति~(४१)

%6.1.6.4
कस्मा᳚थ्स॒त्याद्गा॑य॒त्री कनि॑ष्ठा॒ छन्द॑साꣳ स॒ती य॑ज्ञमु॒खं परी॑या॒येति॒ यदे॒वादः सोम॒माह॑र॒त्तस्मा᳚द्यज्ञमु॒खं पर्यै॒त् तस्मा᳚त्तेज॒स्विनी॑तमा प॒द्भ्यां द्वे सव॑ने स॒मगृ॑ह्णा॒न्मुखे॒नैकं॒ यन्मुखे॑न स॒मगृ॑ह्णा॒त्तद॑धय॒त्तस्मा॒द्द्वे सव॑ने शु॒क्रव॑ती प्रातःसव॒नं च॒ माध्यं॑दिनं च॒ तस्मा᳚त्तृतीयसव॒न ऋ॑जी॒षम॒भि षु॑ण्वन्ति धी॒तमि॑व॒ हि मन्य॑न्ते~(४२)

%6.1.6.5
आ॒शिर॒मव॑ नयति सशुक्र॒त्वायाथो॒ सम्भ॑रत्ये॒वैन॒त्तꣳ सोम॑माह्रि॒यमा॑णं गन्ध॒र्वो वि॒श्वाव॑सुः॒ पर्य॑मुष्णा॒थ्स ति॒स्रो रात्रीः॒ परि॑मुषितो\-ऽवस॒त्तस्मा᳚त्ति॒स्रो रात्रीः᳚ क्री॒तः सोमो॑ वसति॒ ते दे॒वा अ॑ब्रुव॒न्थ्स्त्रीका॑मा॒ वै ग॑न्ध॒र्वाः स्त्रि॒या निष्क्री॑णा॒मेति॒ ते वाच॒ꣴ॒ स्त्रिय॒मेक॑हायनीं कृ॒त्वा तया॒ निर॑क्रीण॒न्थ्सा रो॒हिद्रू॒पं कृ॒त्वा ग॑न्ध॒र्वेभ्यः॑~(४३)

%6.1.6.6
अ॒प॒क्रम्या॑तिष्ठ॒त्तद्रो॒हितो॒ जन्म॒ ते दे॒वा अ॑ब्रुव॒न्नप॑ यु॒ष्मदक्र॑मी॒न्नास्मानु॒पाव॑र्तते॒ वि ह्व॑यामहा॒ इति॒ ब्रह्म॑ गन्ध॒र्वा अव॑द॒न्नगा॑यं दे॒वाः सा दे॒वान्गाय॑त उ॒पाव॑र्तत॒ तस्मा॒द्गाय॑न्त॒ꣴ॒ स्त्रियः॑ कामयन्ते॒ कामु॑का एन॒ꣴ॒ स्त्रियो॑ भवन्ति॒ य ए॒वं वेदाथो॒ य ए॒वं वि॒द्वानपि॒ जन्ये॑षु॒ भव॑ति॒ तेभ्य॑ ए॒व द॑दत्यु॒त यद्ब॒हुत॑याः~(४४)

%6.1.6.7
भव॒न्त्येक॑हायन्या क्रीणाति वा॒चैवैन॒ꣳ॒ सर्व॑या क्रीणाति॒ तस्मा॒देक॑हायना मनु॒ष्या॑ वाचं॑ वद॒न्त्यकू॑ट॒या\-ऽक॑र्ण॒या\-ऽ का॑ण॒या\-ऽश्लो॑ण॒या\-ऽस॑प्तशफया क्रीणाति॒ सर्व॑यै॒वैनं॑ क्रीणाति॒ यच्छ्वे॒तया᳚ क्रीणी॒याद्दु॒श्चर्मा॒ यज॑मानः स्या॒द्यत्कृ॒ष्णया॑नु॒स्तर॑णी स्यात्प्र॒मायु॑को॒ यज॑मानः स्या॒द्यद्द्वि॑रू॒पया॒ वात्र॑घ्नी स्या॒थ्स वा॒न्यं जि॑नी॒यात्तं वा॒न्यो जि॑नीयादरु॒णया॑ पिङ्गा॒क्ष्या क्री॑णात्ये॒तद्वै सोम॑स्य रू॒पꣴ स्वयै॒वैनं॑ दे॒वत॑या क्रीणाति॥~(४५)

%6.1.7.0
{\anuvakamend[{निष्क्री॑णीष्व॒ दक्षि॑णाभिश्च वदन्ति॒ मन्य॑न्ते गन्ध॒र्वेभ्यो॑ ब॒हुत॑याः पिङ्गा॒क्ष्या दश॑ च}]}%~(६)

%6.1.7.1
तद्धिर॑ण्यमभव॒त्तस्मा॑द॒द्भ्यो हिर॑ण्यम्पुनन्ति ब्रह्मवा॒दिनो॑ वदन्ति॒ कस्मा᳚थ्स॒त्याद॑न॒स्थिके॑न प्र॒जाः प्र॒वीय॑न्ते\-ऽ\-स्थ॒न्वती᳚र्जायन्त॒ इति॒ यद्धिर॑ण्यं घृ॒ते॑\-ऽव॒धाय॑ जु॒होति॒ तस्मा॑दन॒स्थिके॑न प्र॒जाः प्र वी॑यन्ते\-ऽस्थ॒न्वती᳚र्जायन्त ए॒तद्वा अ॒ग्नेः प्रि॒यं धाम॒ यद्घृ॒तं तेजो॒ हिर॑ण्यमि॒यं ते॑ शुक्र त॒नूरि॒दं वर्च॒ इत्या॑ह॒ सते॑जसमे॒वैन॒ꣳ॒ सत॑नुम्~(४६)

%6.1.7.2
क॒रो॒त्यथो॒ सम्भ॑रत्ये॒वैनं॒ यदब॑द्धमवद॒ध्याद्गर्भाः᳚ प्र॒जानां᳚ परा॒पातु॑काः स्युर्ब॒द्धमव॑ दधाति॒ गर्भा॑णां॒ धृत्यै॑ निष्ट॒र्क्य॑म्बध्नाति प्र॒जानां᳚ प्र॒जन॑नाय॒ वाग्वा ए॒षा यथ्सो॑म॒क्रय॑णी॒ जूर॒सीत्या॑ह॒ यद्धि मन॑सा॒ जव॑ते॒ तद्वा॒चा वद॑ति धृ॒ता मन॒सेत्या॑ह॒ मन॑सा॒ हि वाग्धृ॒ता जुष्टा॒ विष्ण॑व॒ इत्या॑ह~(४७)

%6.1.7.3
य॒ज्ञो वै विष्णु॑र्\mbox{}य॒ज्ञायै॒वैनां॒ जुष्टां᳚ करोति॒ तस्या᳚स्ते स॒त्यस॑वसः प्रस॒व इत्या॑ह सवि॒तृप्र॑सूतामे॒व वाच॒मव॑ रुन्द्धे॒ काण्डे॑काण्डे॒ वै क्रि॒यमा॑णे य॒ज्ञꣳ रक्षाꣳ॑सि जिघाꣳसन्त्ये॒ष खलु॒ वा अर॑क्षोहतः॒ पन्था॒ यो᳚\-ऽग्नेश्च॒ सूर्य॑स्य च॒ सूर्य॑स्य॒ चक्षु॒रारु॑हम॒ग्नेर॒क्ष्णः क॒नीनि॑का॒मित्या॑ह॒ य ए॒वार॑क्षोहतः॒ पन्था॒स्तꣳ स॒मारो॑हति~(४८)

%6.1.7.4
वाग्वा ए॒षा यथ्सो॑म॒क्रय॑णी॒ चिद॑सि म॒नासीत्या॑ह॒ शास्त्ये॒वैना॑मे॒तत्तस्मा᳚च्छि॒ष्टाः प्र॒जा जा॑यन्ते॒ चिद॒सीत्या॑ह॒ यद्धि मन॑सा चे॒तय॑ते॒ तद्वा॒चा वद॑ति म॒नासीत्या॑ह॒ यद्धि मन॑साभि॒गच्छ॑ति॒ तत्क॒रोति॒ धीर॒सीत्या॑ह॒ यद्धि मन॑सा॒ ध्याय॑ति॒ तद्वा॒चा~(४९)

%6.1.7.5
वद॑ति॒ दक्षि॑णा॒सीत्या॑ह॒ दक्षि॑णा ह्ये॑षा य॒ज्ञिया॒सीत्या॑ह य॒ज्ञिया॑मे॒वैनां᳚ करोति क्ष॒त्रिया॒सीत्या॑ह क्ष॒त्रिया॒ ह्ये॑षादि॑तिरस्युभ॒यतः॑शी॒र्\mbox{}ष्णीत्या॑ह॒ यदे॒वाऽऽदि॒त्यः प्रा॑य॒णीयो॑ य॒ज्ञाना॑मादि॒त्य उ॑दय॒नीय॒स्तस्मा॑दे॒वमा॑ह॒ यदब॑द्धा॒ स्यादय॑ता स्या॒द्यत्प॑दिब॒द्धानु॒स्तर॑णी स्यात्प्र॒मायु॑को॒ यज॑मानः स्यात्~(५०)

%6.1.7.6
यत्क॑र्णगृही॒ता वार्त्र॑घ्नी स्या॒थ्स वा॒न्यं जि॑नी॒यात्तं वा॒न्यो जि॑नीयान्मि॒त्रस्त्वा॑ प॒दि ब॑ध्ना॒त्वित्या॑ह मि॒त्रो वै शि॒वो दे॒वाना॒न्तेनै॒वैनां᳚ प॒दि ब॑ध्नाति पू॒षाध्व॑नः पा॒त्वित्या॑हे॒यं वै पू॒षेमामे॒वास्या॑ अधि॒पाम॑कः॒ सम॑ष्ट्या॒ इन्द्रा॒याध्य॑क्षा॒येत्या॒हेन्द्र॑मे॒वास्या॒ अध्य॑क्षं करोति~(५१)

%6.1.7.7
अनु॑ त्वा मा॒ता म॑न्यता॒मनु॑ पि॒तेत्या॒हानु॑मतयै॒वैन॑या क्रीणाति॒ सा दे॑वि दे॒वमच्छे॒हीत्या॑ह दे॒वी ह्ये॑षा दे॒वः सोम॒ इन्द्रा॑य॒ सोम॒मित्या॒हेन्द्रा॑य॒ हि सोम॑ आह्रि॒यते॒ यदे॒तद्यजु॒र्न ब्रू॒यात्परा᳚च्ये॒व सो॑म॒क्रय॑णीयाद्रु॒द्रस्त्वा व॑र्तय॒त्वित्या॑ह रु॒द्रो वै क्रू॒रः~(५२)

%6.1.7.8
दे॒वाना॒न्तमे॒वास्यै॑ प॒रस्ता᳚द्दधा॒त्यावृ॑त्त्यै क्रू॒रमि॑व॒ वा ए॒तत्क॑रोति॒ यद्रु॒द्रस्य॑ की॒र्तय॑ति मि॒त्रस्य॑ प॒थेत्या॑ह॒ शान्त्यै॑ वा॒चा वा ए॒ष वि क्री॑णीते॒ यः सो॑म॒क्रय॑ण्या स्व॒स्ति सोम॑सखा॒ पुन॒रेहि॑ स॒ह र॒य्येत्या॑ह वा॒चैव वि॒क्रीय॒ पुन॑रा॒त्मन्वाचं॑ ध॒त्ते\-ऽनु॑पदासुकास्य॒ वाग्भ॑वति॒ य ए॒वं वेद॑॥~(५३)

%6.1.8.0
{\anuvakamend[{सत॑नुं॒ विष्ण॑व॒ इत्या॑ह स॒मारो॑हति॒ ध्याय॑ति॒ तद्वा॒चा यज॑मानः स्यात्करोति क्रू॒रो वेद॑}]}%~(७)

%6.1.8.1
षट्प॒दान्यनु॒ नि क्रा॑मति षड॒हं वाङ्नाति॑ वदत्यु॒त सं॑वथ्स॒रस्याय॑ने॒ याव॑त्ये॒व वाक्तामव॑ रुन्द्धे सप्त॒मे प॒दे जु॑होति स॒प्तप॑दा॒ शक्व॑री प॒शवः॒ शक्व॑री प॒शूने॒वाव॑ रुन्द्धे स॒प्त ग्रा॒म्याः प॒शवः॑ स॒प्तार॒ण्याः स॒प्त छन्दाꣴ॑स्यु॒भय॒स्याव॑रुद्ध्यै॒ वस्व्य॑सि रु॒द्रासीत्या॑ह रू॒पमे॒वास्या॑ ए॒तन्म॑हि॒मानम्᳚~(५४)

%6.1.8.2
व्याच॑ष्टे॒ बृह॒स्पति॑स्त्वा सु॒म्ने र॑ण्व॒त्वित्या॑ह॒ ब्रह्म॒ वै दे॒वाना॒म्बृह॒स्पति॒र्ब्रह्म॑णै॒वास्मै॑ प॒शूनव॑ रुन्द्धे रु॒द्रो वसु॑भि॒रा चि॑के॒त्वित्या॒हावृ॑त्त्यै पृथि॒व्यास्त्वा॑ मू॒र्धन्ना जि॑घर्मि देव॒यज॑न॒ इत्या॑ह पृथि॒व्या ह्ये॑ष मू॒र्धा यद्दे॑व॒यज॑न॒मिडा॑याः प॒द इत्या॒हेडा॑यै॒ ह्ये॑तत्प॒दं यथ्सो॑म॒क्रय॑ण्यै घृ॒तव॑ति॒ स्वाहा᳚~(५)

%6.1.8.3
इत्या॑ह॒ यदे॒वास्यै॑ प॒दाद्घृ॒तमपी᳚ड्यत॒ तस्मा॑दे॒वमा॑ह॒ यद॑ध्व॒र्युर॑न॒ग्नावाहु॑तिं जुहु॒याद॒न्धो᳚\-ऽध्व॒र्युः स्या॒द्रक्षाꣳ॑सि य॒ज्ञꣳ ह॑न्यु॒र्\mbox{}हिर॑ण्यमु॒पास्य॑ जुहोत्यग्नि॒वत्ये॒व जु॑होति॒ नान्धो᳚\-ऽध्व॒र्युर्भव॑ति॒ न य॒ज्ञꣳ रक्षाꣳ॑सि घ्नन्ति॒ काण्डे॑काण्डे॒ वै क्रि॒यमा॑णे य॒ज्ञꣳ रक्षाꣳ॑सि जिघाꣳसन्ति॒ परि॑लिखित॒ꣳ॒ रक्षः॒ परि॑लिखिता॒ अरा॑तय॒ इत्या॑ह॒ रक्ष॑सा॒मप॑हत्यै~(५६)

%6.1.8.4
इ॒दम॒हꣳ रक्ष॑सो ग्री॒वा अपि॑ कृन्तामि॒ यो᳚\-ऽस्मान्द्वेष्टि॒ यं च॑ व॒यं द्वि॒ष्म इत्या॑ह॒ द्वौ वाव पुरु॑षौ॒ यं चै॒व द्वेष्टि॒ यश्चै॑नं॒ द्वेष्टि॒ तयो॑रे॒वान॑न्तरायं ग्री॒वाः कृ॑न्तति प॒शवो॒ वै सो॑म॒क्रय॑ण्यै प॒दं या॑वत्त्मू॒तꣳ सं व॑पति प॒शूने॒वाव॑ रुन्द्धे॒\-ऽस्मे राय॒ इति॒ सं व॑पत्या॒त्मान॑मे॒वाध्व॒र्युः~(५७)

%6.1.8.5
प॒शुभ्यो॒ नान्तरे॑ति॒ त्वे राय॒ इति॒ यज॑मानाय॒ प्र य॑च्छति॒ यज॑मान ए॒व र॒यिं द॑धाति॒ तोते॒ राय॒ इति॒ पत्नि॑या अ॒र्धो वा ए॒ष आ॒त्मनो॒ यत्पत्नी॒ यथा॑ गृ॒हेषु॑ निध॒त्ते ता॒दृगे॒व तत्त्वष्टी॑मती ते सपे॒येत्या॑ह॒ त्वष्टा॒ वै प॑शू॒नाम्मि॑थु॒नानाꣳ॑ रूप॒कृद्रू॒पमे॒व प॒शुषु॑ दधात्य॒स्मै वै लो॒काय॒ गार्\mbox{}ह॑पत्य॒ आ धी॑यते॒\-ऽमुष्मा॑ आहव॒नीयो॒ यद्गार्\mbox{}ह॑पत्य उप॒वपे॑द॒स्मिल्लोँ॒के प॑शु॒मान्थ्स्या॒द्यदा॑हव॒नीये॒\-ऽमुष्मि॑ल्लोँ॒के प॑शु॒मान्थ्स्या॑दु॒भयो॒रुप॑ वपत्यु॒भयो॑रे॒वैन॑ल्लोँ॒कयोः᳚ पशु॒मन्तं॑ करोति॥~(५८)

%6.1.9.0
{\anuvakamend[{म॒हि॒मान॒ꣴ॒ स्वाहाप॑हत्या अध्व॒र्युर्धी॑यते॒ चतु॑र्विHꣳशतिश्च}]}%~(८)

%6.1.9.1
ब्र॒ह्म॒वा॒दिनो॑ वदन्ति वि॒चित्यः॒ सोमा~(३) न वि॒चित्या~(३) इति॒ सोमो॒ वा ओष॑धीना॒ꣳ॒ राजा॒ तस्मि॒न् यदाप॑न्नं ग्रसि॒तमे॒वास्य॒ तद्यद्वि॑चिनु॒याद्यथा॒स्या᳚द्ग्रसि॒तं नि॑ष्खि॒दति॑ ता॒दृगे॒व तद्यन्न वि॑चिनु॒याद्यथा॒क्षन्नाप॑न्नं वि॒धाव॑ति ता॒दृगे॒व तत्क्षोधु॑को\-ऽध्व॒र्युः स्यात्क्षोधु॑को॒ यज॑मानः॒ सोम॑विक्रयि॒न्थ्सोमꣳ॑ शोध॒येत्ये॒व ब्रू॑या॒द्यदीत॑रम्~(५९)

%6.1.9.2
यदीत॑रमु॒भये॑नै॒व सो॑मविक्र॒यिण॑मर्पयति॒ तस्मा᳚थ्सोमविक्र॒यी क्षोधु॑को\-ऽरु॒णो ह॑ स्मा॒हौप॑वेशिः सोम॒क्रय॑ण ए॒वाहं तृ॑तीयसव॒नमव॑ रुन्ध॒ इति॑ पशू॒नां चर्म॑न्मिमीते प॒शूने॒वाव॑ रुन्द्धे प॒शवो॒ हि तृ॒तीय॒ꣳ॒ सव॑नं॒ यङ्का॒मये॑ताप॒शुः स्या॒दित्यृ॑क्ष॒तस्तस्य॑ मिमीत॒र्क्षं वा अ॑पश॒व्यम॑प॒शुरे॒व भ॑वति॒ यं का॒मये॑त पशु॒मान्थ्स्या᳚त्~(६०)

%6.1.9.3
इति॑ लोम॒तस्तस्य॑ मिमीतै॒तद्वै प॑शू॒नाꣳ रू॒पꣳ रू॒पेणै॒वास्मै॑ प॒शूनव॑ रुन्द्धे पशु॒माने॒व भ॑वत्य॒पामन्ते᳚ क्रीणाति॒ सर॑समे॒वैनं॑ क्रीणात्य॒मात्यो॒\-ऽसीत्या॑हा॒मैवैनं॑ कुरुते शु॒क्रस्ते॒ ग्रह॒ इत्या॑ह शु॒क्रो ह्य॑स्य॒ ग्रहो\-ऽन॒साच्छ॑ याति महि॒मान॑मे॒वास्याच्छ॑ या॒त्यन॑सा~(६१)

%6.1.9.4
अच्छ॑ याति॒ तस्मा॑दनोवा॒ह्यꣳ॑ स॒मे जीव॑नं॒ यत्र॒ खलु॒ वा ए॒तꣳ शी॒र्\mbox{}ष्णा हर॑न्ति॒ तस्मा᳚च्छीर्\mbox{}षहा॒र्यं॑ गि॒रौ जीव॑नम॒भि त्यं दे॒वꣳ स॑वि॒तार॒मित्यति॑छन्दस॒र्चा मि॑मी॒ते\-ऽति॑च्छन्दा॒ वै सर्वा॑णि॒ छन्दाꣳ॑सि॒ सर्वे॑भिरे॒वैनं॒ छन्दो॑भिर्मिमीते॒ वर्\mbox{}ष्म॒ वा ए॒षा छन्द॑सां॒ यदति॑च्छन्दा॒ यदति॑च्छन्दस॒र्चा मिमी॑ते॒ वर्\mbox{}ष्मै॒वैनꣳ॑ समा॒नानां᳚ करो॒त्येक॑यैकयो॒थ्सर्गम्᳚~(६२)

%6.1.9.5
मि॒मी॒ते\-ऽया॑तयाम्नियायातयाम्नियै॒वैन॑म्मिमीते॒ तस्मा॒न्नाना॑वीर्या अ॒ङ्गुल॑यः॒ सर्वा᳚स्वङ्गु॒ष्ठमुप॒ नि गृ॑ह्णाति॒ तस्मा᳚थ्स॒माव॑द्वीर्यो॒\-ऽन्याभि॑र॒ङ्गुलि॑भि॒स्तस्मा॒थ्सर्वा॒ अनु॒ सं च॑रति॒ यथ्स॒ह सर्वा॑भि॒र्मिमी॑त॒ सꣴश्लि॑ष्टा अ॒ङ्गुल॑यो जायेर॒न्नेक॑यैकयो॒थ्सर्ग॑म्मिमीते॒ तस्मा॒द्विभ॑क्ता जायन्ते॒ पञ्च॒ कृत्वो॒ यजु॑षा मिमीते॒ पञ्चा᳚क्षरा प॒ङ्क्तिः पाङ्क्तो॑ य॒ज्ञो य॒ज्ञमे॒वाव॑ रुन्द्धे॒ पञ्च॒ कृत्व॑स्तू॒ष्णीम्~(६३)

%6.1.9.6
दश॒ सम्प॑द्यन्ते॒ दशा᳚क्षरा वि॒राडन्नं॑ वि॒राड्वि॒राजै॒वान्नाद्य॒मव॑ रुन्द्धे॒ यद्यजु॑षा॒ मिमी॑ते भू॒तमे॒वाव॑ रुन्द्धे॒ यत्तू॒ष्णीम्भ॑वि॒ष्यद्यद्वै तावा॑ने॒व सोमः॒ स्याद्याव॑न्त॒म्मिमी॑ते॒ यज॑मानस्यै॒व स्या॒न्नापि॑ सद॒स्या॑नां प्र॒जाभ्य॒स्त्वेत्युप॒ समू॑हति सद॒स्या॑ने॒वान्वाभ॑जति॒ वास॒सोप॑ नह्यति सर्वदेव॒त्यं॑ वै~(६४)

%6.1.9.7
वासः॒ सर्वा॑भिरे॒वैनं॑ दे॒वता॑भिः॒ सम॑र्धयति प॒शवो॒ वै सोमः॑ प्रा॒णाय॒ त्वेत्युप॑ नह्यति प्रा॒णमे॒व प॒शुषु॑ दधाति व्या॒नाय॒ त्वेत्यनु॑ शृन्थति व्या॒नमे॒व प॒शुषु॑ दधाति॒ तस्मा᳚थ्स्व॒पन्तं॑ प्रा॒णा न ज॑हति॥~(६५)

%6.1.10.0
{\anuvakamend[{इत॑रम्पशु॒मान्थ्स्या᳚द्या॒त्यन॑सो॒थ्सर्ग॑न्तू॒ष्णीꣳ स॑र्वदेव॒त्यं॑ वै त्रय॑स्त्रिꣳशच्च}]}%~(९)

%6.1.10.1
यत्क॒लया॑ ते श॒फेन॑ ते क्रीणा॒नीति॒ पणे॒तागो॑अर्घ॒ꣳ॒ सोमं॑ कु॒र्यादगो॑अर्घं॒ यज॑मान॒मगो॑अर्घमध्व॒र्युङ्गोस्तु म॑हि॒मानं॒ नाव॑ तिरे॒द्गवा॑ ते क्रीणा॒नीत्ये॒व ब्रू॑याद्गोअ॒र्घमे॒व सोमं॑ क॒रोति॑ गोअ॒र्घं यज॑मानं गोअ॒र्घम॑ध्व॒र्युन्न गोर्म॑हि॒मान॒मव॑ तिरत्य॒जया᳚ क्रीणाति॒ सत॑पसमे॒वैनं॑ क्रीणाति॒ हिर॑ण्येन क्रीणाति॒ सशु॑क्रमे॒व~(६६)

%6.1.10.2
ए॒नं॒ क्री॒णा॒ति॒ धे॒न्वा क्री॑णाति॒ साशि॑रमे॒वैनं॑ क्रीणात्यृष॒भेण॑ क्रीणाति॒ सेन्द्र॑मे॒वैनं॑ क्रीणात्यन॒डुहा᳚ क्रीणाति॒ वह्नि॒र्वा अ॑न॒ड्वान् वह्नि॑नै॒व वह्नि॑ य॒ज्ञस्य॑ क्रीणाति मिथु॒ना\-भ्यां᳚ क्रीणाति मिथु॒नस्याव॑रुद्ध्यै॒ वास॑सा क्रीणाति सर्वदेव॒त्यं॑ वै वासः॒ सर्वा᳚भ्य ए॒वैनं॑ दे॒वता᳚भ्यः क्रीणाति॒ दश॒ सम्प॑द्यन्ते॒ दशा᳚क्षरा वि॒राडन्नं॑ वि॒राड्वि॒राजै॒वान्नाद्य॒मव॑ रुन्द्धे~(६७)

%6.1.10.3
तप॑सस्त॒नूर॑सि प्र॒जाप॑ते॒र्वर्ण॒ इत्या॑ह प॒शुभ्य॑ ए॒व तद॑ध्व॒र्युर्नि ह्नु॑त आ॒त्मनो\-ऽना᳚व्रस्काय॒ गच्छ॑ति॒ श्रियं प्र प॒शूना᳚प्नोति॒ य ए॒वं वेद॑ शु॒क्रं ते॑ शु॒क्रेण॑ क्रीणा॒मीत्या॑ह यथाय॒जुरे॒वैतद्दे॒वा वै येन॒ हिर॑ण्येन॒ सोम॒मक्री॑ण॒न्तद॑भी॒षहा॒ पुन॒राद॑दत॒ को हि तेज॑सा विक्रे॒ष्यत॒ इति॒ येन॒ हिर॑ण्येन~(६८)

%6.1.10.4
सोमं॑ क्रीणी॒यात्तद॑भी॒षहा॒ पुन॒रा द॑दीत॒ तेज॑ ए॒वात्मन्ध॑त्ते॒\-ऽस्मे ज्योतिः॑ सोमविक्र॒यिणि॒ तम॒ इत्या॑ह॒ ज्योति॑रे॒व यज॑माने दधाति॒ तम॑सा॒ सोमविक्र॒यिण॑मर्पयति॒ यदनु॑पग्रथ्य ह॒न्याद्द॑न्द॒शूका॒स्ताꣳ समाꣳ॑ स॒र्पाः स्यु॑रि॒दम॒हꣳ स॒र्पाणां᳚ दन्द॒शूका॑नां ग्री॒वा उप॑ ग्रथ्ना॒मीत्या॒हाद॑न्दशूका॒स्ताꣳ समाꣳ॑ स॒र्पा भ॑वन्ति॒ तम॑सा सोमविक्र॒यिणं॑ विध्यति॒ स्वान॑~(६९)

%6.1.10.5
भ्राजेत्या॑है॒ते वा अ॒मुष्मि॑ल्लोँ॒के सोम॑मरक्ष॒न्तेभ्यो\-ऽधि॒ सोम॒माह॑र॒न् यदे॒तेभ्यः॑ सोम॒क्रय॑णा॒न्नानु॑दि॒शेदक्री॑तो\-ऽस्य॒ सोमः॑ स्या॒न्नास्यै॒ते॑\-ऽमुष्मि॑ल्लोँ॒के सोमꣳ॑ रक्षेयु॒र्यदे॒तेभ्यः॑ सोम॒क्रय॑णाननुदि॒शति॑ क्री॒तो᳚\-ऽस्य॒ सोमो॑ भवत्ये॒ते᳚\-ऽस्या॒मुष्मि॑ल्लोँ॒के सोमꣳ॑ रक्षन्ति॥~(७०)

%6.1.11.0
{\anuvakamend[{सशु॑क्रमे॒व रु॑न्ध॒ इति॒ येन॒ हिर॑ण्येन॒ स्वान॒ चतु॑श्चत्वारिꣳशच्च}]}%॥10॥

%6.1.11.1
वा॒रु॒णो वै क्री॒तः सोम॒ उप॑नद्धो मि॒त्रो न॒ एहि॒ सुमि॑त्रधा॒ इत्या॑ह॒ शान्त्या॒ इन्द्र॑स्यो॒रुमा वि॑श॒ दक्षि॑ण॒मित्या॑ह दे॒वा वै यꣳ सोम॒मक्री॑ण॒न्तमिन्द्र॑स्यो॒रौ दक्षि॑ण॒ आसा॑दयन्ने॒ष खलु॒ वा ए॒तर्\mbox{}हीन्द्रो॒ यो यज॑ते॒ तस्मा॑दे॒वमा॒होदायु॑षा स्वा॒युषेत्या॑ह दे॒वता॑ ए॒वान्वा॒रभ्योत्~(७१)

%6.1.11.2
ति॒ष्ठ॒त्यु॒र्व॑न्तरि॑क्ष॒मन्वि॒हीत्या॑हान्तरिक्षदेव॒त्यो  ह्ये॑तर्\mbox{}हि॒ सोमो\-ऽदि॑त्याः॒ सदो॒\-ऽस्यदि॑त्याः॒ सद॒ आ सी॒देत्या॑ह यथाय॒जुरे॒वैतद्वि वा ए॑नमे॒तद॑र्धयति॒ यद्वा॑रु॒णꣳ सन्त॑म्मै॒त्रं क॒रोति॑ वारु॒ण्यर्चा सा॑दयति॒ स्वयै॒वैनं॑ दे॒वत॑या॒ सम॑र्धयति॒ वास॑सा प॒र्यान॑ह्यति सर्वदेव॒त्यं॑ वै वासः॒ सर्वा॑भिरे॒व~(७२)

%6.1.11.3
ए॒नं॒ दे॒वता॑भिः॒ सम॑र्धय॒त्यथो॒ रक्ष॑सा॒मप॑हत्यै॒ वने॑षु॒ व्य॑न्तरि॑क्षं तता॒नेत्या॑ह॒ वने॑षु॒ हि व्य॑न्तरि॑क्षं त॒तान॒ वाज॒मर्व॒थ्स्वित्या॑ह॒ वाज॒ꣴ॒ ह्यर्व॑थ्सु॒ पयो॑ अघ्नि॒यास्वित्या॑ह॒ पयो॒ ह्य॑घ्नि॒यासु॑ हृ॒थ्सु क्रतु॒मित्या॑ह हृ॒थ्सु हि क्रतुं॒ वरु॑णो वि॒क्ष्व॑ग्निमित्या॑ह॒ वरु॑णो॒ हि वि॒क्ष्व॑ग्निन्दि॒वि सूर्यम्᳚~(७३)

%6.1.11.4
इत्या॑ह दि॒वि हि सूर्य॒ꣳ॒ सोम॒मद्रा॒वित्या॑ह॒ ग्रावा॑णो॒ वा अद्र॑य॒स्तेषु॒ वा ए॒ष सोमं॑ दधाति॒ यो यज॑ते॒ तस्मा॑दे॒वमा॒होदु॒ त्यं जा॒तवे॑दस॒मिति॑ सौ॒र्यर्चा कृ॑ष्णाजि॒नम्प्र॒त्यान॑ह्यति॒ रक्ष॑सा॒मप॑हत्या॒ उस्रा॒वेतं॑ धूर्\mbox{}षाहा॒वित्या॑ह यथाय॒जुरे॒वैतत्प्र च्य॑वस्व भुवस्पत॒ इत्या॑ह भू॒ताना॒ꣳ॒ हि~(७४)

%6.1.11.5
ए॒ष पति॒र्विश्वा᳚न्य॒भि धामा॒नीत्या॑ह॒ विश्वा॑नि॒ ह्ये  षो॑\-ऽभि धामा॑नि प्र॒च्यव॑ते॒ मा त्वा॑ परिप॒री वि॑द॒दित्या॑ह॒ यदे॒वादः सोम॑माह्रि॒यमा॑णं गन्ध॒र्वो वि॒श्वाव॑सुः प॒र्यमु॑ष्णा॒त्तस्मा॑दे॒वमा॒हाप॑रिमोषाय॒ यज॑मानस्य स्व॒स्त्यय॑न्य॒सीत्या॑ह॒ यज॑मानस्यै॒वैष य॒ज्ञस्या᳚न्वार॒म्भो\-ऽन॑वछित्त्यै॒ वरु॑णो॒ वा ए॒ष यज॑मानम॒भ्यैति॒ यत्~(७५)

%6.1.11.6
क्री॒तः सोम॒ उप॑नद्धो॒ नमो॑ मि॒त्रस्य॒ वरु॑णस्य॒ चक्ष॑स॒ इत्या॑ह॒ शान्त्या॒ आ सोमं॒ वह॑न्त्य॒ग्निना॒ प्रति॑ तिष्ठते॒ तौ स॒म्भव॑न्तौ॒ यज॑मानम॒भि सम्भ॑वतः पु॒रा खलु॒ वावैष मेधा॑या॒त्मान॑मा॒रभ्य॑ चरति॒ यो दी᳚क्षि॒तो यद॑ग्नीषो॒मीय॑म्प॒शुमा॒लभ॑त आत्मनि॒ष्क्रय॑ण ए॒वास्य॒ स तस्मा॒त्तस्य॒ नाश्यं॑ पुरुषनि॒ष्क्रय॑ण इव॒ ह्यथो॒ खल्वा॑हुर॒ग्नीषोमा᳚भ्यां॒ वा इन्द्रो॑ वृ॒त्रम॑ह॒न्निति॒ यद॑ग्नीषो॒मीय॑म्प॒शुमा॒लभ॑ते॒ वार्त्र॑घ्न ए॒वास्य॒ स तस्मा᳚द्वा॒श्यं॑ वारु॒ण्यर्चा परि॑ चरति॒ स्वयै॒वैनं॑ दे॒वत॑या॒ परि॑ चरति॥~(७६)

%6.2.0.0
{\anuvakamend[{अ॒न्वा॒रभ्योथ्सर्वा॑भिरे॒व सूर्यं॑ भू॒ताना॒ꣳ॒ ह्ये॑ति॒ यदा॑हुः स॒प्तविꣳ॑शतिश्च}]}%॥11॥

%6.2.0.0

{\anuvakamend[{यदु॒भौ दे॑वासु॒रा मि॒थस्तेषाꣳ॑ सुव॒र्गं यद्वा अनी॑शानः पु॒रोह॑विषि॒ तेभ्यः॒ सोत्त॑रवे॒दिर्ब॒द्धं दे॒वस्याभ्रि॒ꣳ॒ शिरो॒ वा एका॑\-दश}]}%॥11॥
\prashnaend{यदु॒भावित्या॑ह दे॒वानां᳚ य॒ज्ञो दे॒वेभ्यो॒ न रथा॑य॒ यज॑मानाय प॒रस्ता॑द॒र्वाची॒न्नव॑पञ्चा॒शत्॥59॥ यदु॒भौ दु॒ह ए॒वैना᳚म्॥}
%%% END PRASHNA
