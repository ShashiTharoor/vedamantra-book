\sect{चतुर्थः प्रश्नः}\setcounter{anuvakam}{0}
\dnsub{तैत्तिरीयसंहितायां षष्ठमकाण्डे चतुर्थः प्रश्नः}
%6.4.1.0
%6.4.1.1
य॒ज्ञेन॒ वै प्र॒जा\-प॑तिः प्र॒जा अ॑सृजत॒ ता उ॑प॒यड्भि॑रे॒वासृ॑जत॒ यदु॑प॒यज॑ उप॒यज॑ति प्र॒जा ए॒व तद्यज॑मानः सृजते जघना॒र्धादव॑ द्यति जघना॒र्धाद्धि प्र॒जाः प्र॒जाय॑न्ते स्थविम॒तो\-ऽव॑ द्यति स्थविम॒तो हि प्र॒जाः प्र॒जाय॒न्ते\-ऽस॑म्भिन्द॒न्नव॑ द्यति प्रा॒णाना॒मस॑म्भेदाय॒ न प॒र्याव॑र्तयति॒ यत्प॑र्याव॒र्तये॑दुदाव॒र्तः प्र॒जा ग्राहु॑कः स्याथ्समु॒द्रं ग॑च्छ॒ स्वाहेत्या॑ह रेतः॑~(१)

%6.4.1.2
ए॒व तद्द॑धात्य॒न्तरि॑क्षं गच्छ॒ स्वाहेत्या॑हा॒न्तरि॑क्षेणै॒वास्मै᳚ प्र॒जाः प्र ज॑नयत्य॒न्तरि॑क्ष॒ꣴ॒ ह्यनु॑ प्र॒जाः प्र॒जाय॑न्ते दे॒वꣳ स॑वि॒तारं॑ गच्छ॒ स्वाहेत्या॑ह सवि॒तृप्र॑सूत ए॒वास्मै᳚ प्र॒जाः प्र ज॑नयत्यहोरा॒त्रे ग॑च्छ॒ स्वाहेत्या॑हाहोरा॒त्राभ्या॑मे॒वास्मै᳚ प्र॒जाः प्र ज॑नयत्यहोरा॒त्रे ह्यनु॑ प्र॒जाः प्र॒जाय॑न्ते मि॒त्रावरु॑णौ गच्छ॒ स्वाहा᳚~(२)

%6.4.1.3
इत्या॑ह प्र॒जास्वे॒व प्रजा॑तासु प्राणापा॒नौ द॑धाति॒ सोमं॑ गच्छ॒ स्वाहेत्या॑ह सौ॒म्या हि दे॒वत॑या प्र॒जा य॒ज्ञं ग॑च्छ॒ स्वाहेत्या॑ह प्र॒जा ए॒व य॒ज्ञियाः᳚ करोति॒ छन्दाꣳ॑सि गच्छ॒ स्वाहेत्या॑ह प॒शवो॒ वै छन्दाꣳ॑सि प॒शूने॒वाव॑ रुन्धे॒ द्यावा॑पृथि॒वी ग॑च्छ॒ स्वाहेत्या॑ह प्र॒जा ए॒व प्रजा॑ता॒ द्यावा॑पृथि॒वीभ्या॑मुभ॒यतः॒ परि॑ गृह्णाति नभः॑~(३)

%6.4.1.4
दि॒व्यं ग॑च्छ॒ स्वाहेत्या॑ह प्र॒जाभ्य॑ ए॒व प्रजा॑ताभ्यो॒ वृष्टिं॒ नि य॑च्छत्य॒ग्निं वै᳚श्वान॒रं ग॑च्छ॒ स्वाहेत्या॑ह प्र॒जा ए॒व प्रजा॑ता अ॒स्यां प्रति॑\-ष्ठापयति प्रा॒णानां॒ वा ए॒षो\-ऽव॑ द्यति॒ यो॑\-ऽव॒द्यति॑ गु॒दस्य॒ मनो॑ मे॒ हार्दि॑ य॒च्छेत्या॑ह प्रा॒णाने॒व य॑थास्था॒नमुप॑ ह्वयते प॒शोर्वा आल॑ब्धस्य॒ हृद॑य॒ꣳ॒ शुगृ॑च्छति॒ सा हृ॑दयशू॒लम्~(४)

%6.4.1.5
अ॒भि समे॑ति॒ यत्पृ॑थि॒व्याꣳ हृ॑दयशू॒लमु॑द्वा॒सये᳚त्पृथि॒वीꣳ शु॒चार्प॑ये॒द्यद॒फ्स्व॑पः शु॒चार्प॑ये॒च्छुष्क॑स्य चा॒र्द्रस्य॑ च स॒न्धावुद्वा॑सयत्यु॒भय॑स्य॒ शान्त्यै॒ यं द्वि॒ष्यात्तं ध्या॑येच्छु॒चैवैन॑मर्पयति॥~(५)

%6.4.2.0
{\anuvakamend[{रेतो॑ मि॒त्रावरु॑णौ गच्छ॒ स्वाहा॒ नभो॑ हृदयशू॒लं द्वात्रिꣳ॑शच्च}]}%~(१)

%6.4.2.1
दे॒वा वै य॒ज्ञमाग्नी᳚ध्रे॒ व्य॑भजन्त॒ ततो॒ यद॒त्यशि॑ष्यत॒ तद॑ब्रुव॒न्वस॑तु॒ नु न॑ इ॒दमिति॒ तद्व॑सती॒वरी॑णां वसतीवरि॒त्वम् तस्मि॑न्प्रा॒तर्न सम॑शक्नुव॒न्तद॒फ्सु प्रावे॑शय॒न्ता व॑सती॒वरी॑रभवन्वसती॒वरी᳚र्गृह्णाति य॒ज्ञो वै व॑सती॒वरी᳚र्य॒ज्ञमे॒वारभ्य॑ गृही॒त्वोप॑ वसति॒ यस्यागृ॑हीता अ॒भि नि॒म्रोचे॒दना॑रब्धो\-ऽस्य य॒ज्ञः स्या᳚त्~(६)

%6.4.2.2
य॒ज्ञं वि च्छि॑न्द्याज्ज्योति॒ष्या॑ वा गृह्णी॒याद्धिर॑ण्यं वाव॒धाय॒ सशु॑क्राणामे॒व गृ॑ह्णाति॒ यो वा᳚ ब्राह्म॒णो ब॑हुया॒जी तस्य॒ कुम्भ्या॑नां गृह्णीया॒थ्स हि गृ॑ही॒तव॑सतीवरीको वसती॒वरी᳚र्गृह्णाति प॒शवो॒ वै व॑सती॒वरीः᳚ प॒शूने॒वारभ्य॑ गृही॒त्वोप॑ वसति॒ यद॑न्वी॒पं तिष्ठ॑न्गृह्णी॒यान्नि॒र्मार्गु॑का अस्मात्प॒शवः॑ स्युः प्रती॒पं तिष्ठ॑न्गृह्णाति प्रति॒रुध्यै॒वास्मै॑ प॒शून्गृ॑ह्णा॒तीन्द्रः॑~(७)

%6.4.2.3
वृ॒त्रम॑ह॒न्थ्सो \-ऽपो \-ऽभ्य॑म्रियत॒ तासां॒ यन्मेध्यं॑ य॒ज्ञिय॒ꣳ॒ सदे॑व॒मासी॒त्तदत्य॑मुच्यत॒ ता वह॑न्तीरभव॒न्वह॑न्तीनां गृह्णाति॒ या ए॒व मेध्या॑ य॒ज्ञियाः॒ सदे॑वा॒ आप॒स्तासा॑मे॒व गृ॑ह्णाति॒ नान्त॒मा वह॑न्ती॒रती॑या॒द्यद॑न्त॒मा वह॑न्तीरती॒याद्य॒ज्ञमति॑ मन्येत॒ न स्था॑व॒राणां᳚ गृह्णीया॒द्वरु॑णगृहीता॒ वै स्था॑व॒रा यथ्स्था॑व॒राणां᳚ गृह्णी॒यात्~(८)

%6.4.2.4
वरु॑णेनास्य य॒ज्ञं ग्रा॑हये॒द्यद्वै दिवा॒ भव॑त्य॒पो रात्रिः॒ प्र वि॑शति॒ तस्मा᳚त्ता॒म्रा आपो॒ दिवा॑ ददृश्रे॒ यन्नक्त॒म्भव॑त्य॒पो\-ऽहः॒ प्र वि॑शति॒ तस्मा᳚च्च॒न्द्रा आपो॒ नक्तं॑ ददृश्रे छा॒यायै॑ चा॒तप॑तश्च सं॒धौ गृ॑ह्णात्यहोरा॒त्रयो॑रे॒वास्मै॒ वर्णं॑ गृह्णाति ह॒विष्म॑तीरि॒मा आप॒ इत्या॑ह ह॒विष्कृ॑तानामे॒व गृ॑ह्णाति ह॒विष्माꣳ॑ अस्तु~(९)

%6.4.2.5
सूर्य॒ इत्या॑ह॒ सशु॑क्राणामे॒व गृ॑ह्णात्यनु॒ष्टुभा॑ गृह्णाति॒ वाग्वा अ॑नु॒ष्टुग्वा॒चैवैनाः॒ सर्व॑या गृह्णाति॒ चतु॑ष्पदय॒र्चा गृ॑ह्णाति॒ त्रिः सा॑दयति स॒प्त सम्प॑द्यन्ते स॒प्तप॑दा॒ शक्व॑री प॒शवः॒ शक्व॑री प॒शूने॒वाव॑ रुन्धे॒\-ऽस्मै वै लो॒काय॒ गार्\mbox{}ह॑पत्य॒ आ धी॑यते॒\-ऽमुष्मा॑ आहव॒नीयो॒ यद्गार्\mbox{}ह॑पत्य उपसा॒दये॑द॒स्मिँल्लो॒के प॑शु॒मान्थ्स्या॒द्ययदा॑हव॒नीये॒\-ऽमुष्मिन्न्॑~(१०)

%6.4.2.6
लो॒के प॑शु॒मान्थ्स्या॑दु॒भयो॒रुप॑ सादयत्यु॒भयो॑रे॒वैनं॑ लो॒कयोः᳚ पशु॒मन्तं॑ करोति स॒र्वतः॒ परि॑ हरति॒ रक्ष॑सा॒मप॑हत्या इन्द्राग्नि॒योर्भा॑ग॒धेयीः॒ स्थेत्या॑ह यथाय॒जुरे॒वैतदाग्नी᳚ध्र॒ उप॑ वासयत्ये॒तद्वै य॒ज्ञस्याप॑राजितं॒ यदाग्नी᳚ध्रं॒ यदे॒व य॒ज्ञस्याप॑राजितं॒ तदे॒वैना॒ उप॑ वासयति॒ यतः॒ खलु॒ वै य॒ज्ञस्य॒ वित॑तस्य॒ न क्रि॒यते॒ तदनु॑ य॒ज्ञꣳ रक्षा॒ꣴ॒स्यव॑ चरन्ति॒ यद्वह॑न्तीनां गृ॒ह्णाति॑ क्रि॒यमा॑णमे॒व तद्य॒ज्ञस्य॑ शये॒ रक्ष॑सा॒मन॑न्ववचाराय॒ न ह्ये॑ता ई॒लय॒न्त्या तृ॑तीयसव॒नात्परि॑ शेरे य॒ज्ञस्य॒ सन्त॑त्यै॥~(११)

%6.4.3.0
{\anuvakamend[{स्या॒दिन्द्रो॑ गृह्णी॒याद॑स्त्व॒मुष्मि॑न्क्रि॒यते॒ षड्विꣳ॑शतिश्च}]}%~(२)

%6.4.3.1
ब्र॒ह्म॒वा॒दिनो॑ वदन्ति॒ स त्वा अ॑ध्व॒र्युः स्या॒द्यः सोम॑मुपाव॒हर॒न्थ्सर्वा᳚भ्यो दे॒वता᳚भ्य उपाव॒हरे॒दिति॑ हृ॒दे त्वेत्या॑ह मनु॒ष्ये᳚भ्य ए॒वैतेन॑ करोति॒ मन॑से॒ त्वेत्या॑ह पि॒तृभ्य॑ ए॒वैतेन॑ करोति दि॒वे त्वा॒ सूर्या॑य॒ त्वेत्या॑ह दे॒वेभ्य॑ ए॒वैतेन॑ करोत्ये॒ताव॑ती॒र्वै दे॒वता॒स्ताभ्य॑ ए॒वैन॒ꣳ॒ सर्वा᳚भ्य उ॒पाव॑हरति पु॒रा वा॒चः~(१२)

%6.4.3.2
प्रव॑दितोः प्रातरनुवा॒कमु॒पाक॑रोति॒ याव॑त्ये॒व वाक्तामव॑ रुन्धे॒\-ऽपो\-ऽग्रे॑\-ऽभि॒व्याह॑रति य॒ज्ञो वा आपो॑ य॒ज्ञमे॒वाभि वाचं॒ वि सृ॑जति॒ सर्वा॑णि॒ छन्दा॒ꣴ॒स्यन्वा॑ह प॒शवो॒ वै छन्दाꣳ॑सि प॒शूने॒वाव॑ रुन्धे गायत्रि॒या तेज॑स्कामस्य॒ परि॑ दध्यात्त्रि॒ष्टुभे᳚न्द्रि॒यका॑मस्य॒ जग॑त्या प॒शुका॑मस्यानु॒ष्टुभा᳚ प्रति॒ष्ठाका॑मस्य प॒ङ्क्त्या य॒ज्ञका॑मस्य वि॒राजान्न॑कामस्य शृ॒णोत्व॒ग्निः स॒मिधा॒ हवम्᳚~(१३)

%6.4.3.3
म॒ इत्या॑ह सवि॒तृप्र॑सूत ए॒व दे॒वता᳚भ्यो नि॒वेद्या॒पो\-ऽच्छै᳚त्य॒प इ॑ष्य होत॒रित्या॑हेषि॒तꣳ हि कर्म॑ क्रि॒यते॒ मैत्रा॑वरुणस्य चमसाध्वर्य॒वा द्र॒वेत्या॑ह मि॒त्रावरु॑णौ॒ वा अ॒पां ने॒तारौ॒ ताभ्या॑मे॒वैना॒ अच्छै॑ति॒ देवी॑रापो अपां नपा॒दित्या॒हाहु॑त्यै॒वैना॑ नि॒ष्क्रीय॑ गृह्णा॒त्यथो॑ ह॒विष्कृ॑तानामे॒वाभिघृ॑तानां गृह्णाति~(१४)

%6.4.3.4
कार्\mbox{}षि॑र॒सीत्या॑ह॒ शम॑लमे॒वासा॒मप॑ प्लावयति समु॒द्रस्य॒ वोक्षि॑त्या॒ उन्न॑य॒ इत्या॑ह॒ तस्मा॑द॒द्यमा॑नाः पी॒यमा॑ना॒ आपो॒ न क्षी॑यन्ते॒ योनि॒र्वै य॒ज्ञस्य॒ चात्वा॑लं य॒ज्ञो व॑सती॒वरीर्॑\mbox{}होतृचम॒सं च॑ मैत्रावरुणचम॒सं च॑ स॒ꣴ॒स्पर्श्य॑ वसती॒वरी॒र्व्यान॑यति य॒ज्ञस्य॑ सयोनि॒त्वायाथो॒ स्वादे॒वैना॒ योनेः॒ प्र ज॑नय॒त्यध्व॒र्यो\-ऽवे॑र॒पा~(३) इत्या॑हो॒तेम॑नन्नमुरु॒तेमाः प॒श्येति॒ वावैतदा॑ह॒ यद्य॑ग्निष्टो॒मो जु॒होति॒ यद्यु॒क्थ्यः॑ परि॒धौ नि मा᳚र्ष्टि॒ यद्य॑तिरा॒त्रो यजु॒र्वद॒न्प्र प॑द्यते यज्ञक्रतू॒नां व्यावृ॑त्त्यै॥~(१५)

%6.4.4.0
{\anuvakamend[{वा॒चो हव॑म॒भिघृ॑तानां गृह्णात्यु॒त पञ्च॑विꣳशतिश्च}]}%~(३)

%6.4.4.1
दे॒वस्य॑ त्वा सवि॒तुः प्र॑स॒व इति॒ ग्रावा॑ण॒मा द॑त्ते॒ प्रसू᳚त्या अ॒श्विनो᳚र्बा॒हुभ्या॒मित्या॑हा॒श्विनौ॒ हि दे॒वाना॑मध्व॒र्यू आस्तां᳚ पू॒ष्णो हस्ता᳚भ्या॒मित्या॑ह॒ यत्यै॑ प॒शवो॒ वै सोमो᳚ व्या॒न उ॑पाꣳशु॒सव॑नो॒ यदु॑पाꣳशु॒सव॑नम॒भि मिमी॑ते व्या॒नमे॒व प॒शुषु॑ दधा॒तीन्द्रा॑य॒ त्वेन्द्रा॑य॒ त्वेति॑ मिमीत॒ इन्द्रा॑य॒ हि सोम॑ आह्रि॒यते॒ पञ्च॒ कृत्वो॒ यजु॑षा मिमीते~(१६)

%6.4.4.2
पञ्चा᳚क्षरा प॒ङ्क्तिः पाङ्क्तो॑ य॒ज्ञो य॒ज्ञमे॒वाव॑ रुन्धे॒ पञ्च॒ कृत्व॑स्तू॒ष्णीन्दश॒ सम्प॑द्यन्ते॒ दशा᳚क्षरा वि॒राडन्नं॑ वि॒राड्वि॒राजै॒वान्नाद्य॒मव॑ रुन्धे श्वा॒त्राः स्थ॑ वृत्र॒तुर॒ इत्या॑है॒ष वा अ॒पाꣳ सो॑मपी॒थो य ए॒वं वेद॒ नाफ्स्वार्ति॒मार्च्छ॑ति॒ यत्ते॑ सोम दि॒वि ज्योति॒रित्या॑है॒भ्य ए॒वैनम्᳚~(१७)

%6.4.4.3
लो॒केभ्यः॒ सम्भ॑रति॒ सोमो॒ वै राजा॒ दिशो॒\-ऽभ्य॑ध्याय॒थ्स दिशो\-ऽनु॒ प्रावि॑श॒त्प्रागपा॒गुद॑गध॒रागित्या॑ह दि॒ग्भ्य ए॒वैन॒ꣳ॒ सम्भ॑र॒त्यथो॒ दिश॑ ए॒वास्मा॒ अव॑ रु॒न्धे\-ऽम्ब॒ नि ष्व॒रेत्या॑ह॒ कामु॑का एन॒ꣴ॒ स्त्रियो॑ भवन्ति॒ य ए॒वं वेद॒ यत्ते॑ सो॒मादा᳚भ्यं॒ नाम॒ जागृ॒वीति॑~(१८)

%6.4.4.4
आ॒है॒ष वै सोम॑स्य सोमपी॒थो य ए॒वं वेद॒ न सौ॒म्यामार्ति॒मार्च्छ॑ति॒ घ्नन्ति॒ वा ए॒तथ्सोमं॒ यद॑भिषु॒ण्वन्त्य॒ꣳ॒शूनप॑ गृह्णाति॒ त्राय॑त ए॒वैनं॑ प्रा॒णा वा अ॒ꣳ॒शवः॑ प॒शवः॒ सोमो॒\-ऽꣳ॒शून्पुन॒रपि॑ सृजति प्रा॒णाने॒व प॒शुषु॑ दधाति॒ द्वौद्वा॒वपि॑ सृजति॒ तस्मा॒द्द्वौद्वौ᳚ प्रा॒णाः॥~(१९)

%6.4.5.0
{\anuvakamend[{यजु॑षा मिमीत एनं॒ जागृ॒वीति॒ चतु॑श्चत्वारिꣳशच्च}]}%~(४)

%6.4.5.1
प्रा॒णो वा ए॒ष यदु॑पा॒ꣳ॒शुर्यदु॑पा॒ꣳ॒श्व॑ग्रा॒ ग्रहा॑ गृ॒ह्यन्ते᳚ प्रा॒णमे॒वानु॒ प्र य॑न्त्यरु॒णो ह॑ स्मा॒हौप॑वेशिः प्रातःसव॒न ए॒वाहं य॒ज्ञꣳ सꣴस्था॑पयामि॒ तेन॒ ततः॒ सꣴस्थि॑तेन चरा॒मीत्य॒ष्टौ कृत्वो\-ऽग्रे॒\-ऽभि षु॑णोत्य॒ष्टाक्ष॑रा गाय॒त्री गा॑य॒त्रम्प्रा॑तःसव॒नं प्रा॑तःसव॒नमे॒व तेना᳚ऽऽप्नो॒त्येका॑\-दश॒ कृत्वो᳚ द्वि॒तीय॒मेका॑\-दशाक्षरा त्रि॒ष्टुप्त्रैष्टु॑भ॒म्माध्यं॑दिनम्~(२०)

%6.4.5.2
सव॑न॒म्माध्यं॑दिनमे॒व सव॑नं॒ तेना᳚ऽऽप्नोति॒ द्वाद॑श॒ कृत्व॑स्तृ॒तीयं॒ द्वाद॑शाक्षरा॒ जग॑ती॒ जाग॑तं तृतीयसव॒नन्तृ॑तीयसव॒नमे॒व तेना᳚ऽऽप्नोत्ये॒ताꣳ ह॒ वाव स य॒ज्ञस्य॒ सꣴस्थि॑तिमुवा॒चास्क॑न्दा॒यास्क॑न्न॒ꣳ॒ हि तद्यद्य॒ज्ञस्य॒ सꣴस्थि॑तस्य॒ स्कन्द॒त्यथो॒ खल्वा॑हुर्गाय॒त्री वाव प्रा॑तःसव॒ने नाति॒वाद॒ इत्यन॑तिवादुक एन॒म्भ्रातृ॑व्यो भवति॒ य ए॒वं वेद॒ तस्मा॑द॒ष्टाव॑ष्टौ~(२१)

%6.4.5.3
कृत्वो॑\-ऽभि॒षुत्यं॑ ब्रह्मवा॒दिनो॑ वदन्ति प॒वित्र॑वन्तो॒\-ऽन्ये ग्रहा॑ गृ॒ह्यन्ते॒ किम्प॑वित्र उपा॒ꣳ॒शुरिति॒ वाक्प॑वित्र॒ इति॑ ब्रूयात् वा॒चस्पत॑ये पवस्व वाजि॒न्नित्या॑ह वा॒चैवैन॑म्पवयति॒ वृष्णो॑ अ॒ꣳ॒शुभ्या॒मित्या॑ह॒ वृष्णो॒ ह्ये॑ताव॒ꣳ॒शू यौ सोम॑स्य॒ गभ॑स्तिपूत॒ इत्या॑ह॒ गभ॑स्तिना॒ ह्ये॑नम्प॒वय॑ति दे॒वो दे॒वानां᳚ प॒वित्र॑म॒सीत्या॑ह दे॒वो ह्ये॑षः~(२२)

%6.4.5.4
सं दे॒वानां᳚ प॒वित्रं॒ येषां᳚ भा॒गो\-ऽसि॒ तेभ्य॒स्त्वेत्या॑ह॒ येषा॒ꣳ॒ ह्ये॑ष भा॒गस्तेभ्य॑ एनं गृ॒ह्णाति॒ स्वां कृ॑तो॒\-ऽसीत्या॑ह प्रा॒णमे॒व स्वम॑कृत॒ मधु॑मतीर्न॒ इष॑स्कृ॒धीत्या॑ह॒ सर्व॑मे॒वास्मा॑ इ॒दꣴ स्व॑दयति॒ विश्वे᳚भ्यस्त्वेन्द्रि॒येभ्यो॑ दि॒व्येभ्यः॒ पार्थि॑वेभ्य॒ इत्या॑हो॒भये᳚ष्वे॒व दे॑वमनु॒ष्येषु॑ प्रा॒णान्द॑धाति॒ मन॑स्त्वा~(२३)

%6.4.5.5
अ॒ष्ट्वित्या॑ह॒ मन॑ ए॒वाश्ञु॑त उ॒र्व॑न्तरि॑क्ष॒मन्वि॒हीत्या॑हान्तरिक्षदेव॒त्यो॑ हि प्रा॒णः स्वाहा᳚ त्वा सुभवः॒ सूर्या॒येत्या॑ह प्रा॒णा वै स्वभ॑वसो दे॒वास्तेष्वे॒व प॒रोक्षं॑ जुहोति दे॒वेभ्य॑स्त्वा मरीचि॒पेभ्य॒ इत्या॑हादि॒त्यस्य॒ वै र॒श्मयो॑ दे॒वा म॑रीचि॒पास्तेषां॒ तद्भा॑ग॒धेय॒न्ताने॒व तेन॑ प्रीणाति॒ यदि॑ का॒मये॑त॒ वर्\mbox{}षु॑कः प॒र्जन्यः॑~(२४)

%6.4.5.6
स्या॒दिति॒ नीचा॒ हस्ते॑न॒ नि मृ॑ज्या॒द्वृष्टि॑मे॒व नि य॑च्छति॒ यदि॑ का॒मये॒ताव॑र्\mbox{}षुकः स्या॒दित्यु॑त्ता॒नेन॒ नि मृ॑ज्या॒द्वृष्टि॑मे॒वोद्य॑च्छति॒ यद्य॑भि॒चरे॑द॒मुं ज॒ह्यथ॑ त्वा होष्या॒मीति॑ ब्रूया॒दाहु॑तिमे॒वैनं॑ प्रे॒फ्सन् ह॑न्ति॒ यदि॑ दू॒रे स्यादा तमि॑तोस्तिष्ठेत्प्रा॒णमे॒वास्या॑नु॒गत्य॑ हन्ति॒ यद्य॑भि॒चरे॑द॒मुष्य॑~(२५)

%6.4.5.7
त्वा॒ प्रा॒णे सा॑दया॒मीति॑ सादये॒दस॑न्नो॒ वै प्रा॒णः प्रा॒णमे॒वास्य॑ सादयति ष॒ड्भिर॒ꣳ॒शुभिः॑ पवयति॒ षड्वा ऋ॒तव॑ ऋ॒तुभि॑रे॒वैन॑म्पवयति॒ त्रिः प॑वयति॒ त्रय॑ इ॒मे लो॒का ए॒भिरे॒वैनं॑ लो॒कैः प॑वयति ब्रह्मवा॒दिनो॑ वदन्ति॒ कस्मा᳚थ्स॒त्यात्त्रयः॑ पशू॒नाꣳ हस्ता॑दाना॒ इति॒ यत्त्रिरु॑पा॒ꣳ॒शुꣳ हस्ते॑न विगृ॒ह्णाति॒ तस्मा॒त्त्रयः॑ पशू॒नाꣳ हस्ता॑दानाः॒ पुरु॑षो ह॒स्ती म॒र्कटः॑॥~(२६)

%6.4.6.0
{\anuvakamend[{माध्य॑न्दिनम॒ष्टाव॑ष्टावे॒ष मन॑स्त्वा प॒र्जन्यो॒\-ऽमुष्य॒ पुरु॑षो॒ द्वे च॑}]}%~(५)

%6.4.6.1
दे॒वा वै यद्य॒ज्ञे\-ऽकु॑र्वत॒ तदसु॑रा अकुर्वत॒ ते दे॒वा उ॑पा॒ꣳ॒शौ य॒ज्ञꣳ स॒ꣴ॒स्थाप्य॑मपश्य॒न्तमु॑पा॒ꣳ॒शौ सम॑स्थापय॒न्ते\-ऽसु॑रा॒ वज्र॑मु॒द्यत्य॑ दे॒वान॒भ्या॑यन्त॒ ते दे॒वा बिभ्य॑त॒ इन्द्र॒मुपा॑धाव॒न्तानिन्द्रो᳚\-ऽन्तर्या॒मेणा॒न्तर॑धत्त॒ तद॑न्तर्या॒मस्या᳚न्तर्याम॒त्वम् यद॑न्तर्या॒मो गृ॒ह्यते॒ भ्रातृ॑व्याने॒व तद्यज॑मानो॒\-ऽन्तर्ध॑त्ते॒\-ऽन्तस्ते᳚~(२७)

%6.4.6.2
द॒धा॒मि॒ द्यावा॑पृथि॒वी अ॒न्तरु॒र्व॑न्तरि॑क्ष॒मित्या॑है॒भिरे॒व लो॒कैर्यज॑मानो॒ भ्रातृ॑व्यान॒न्तर्ध॑त्ते॒ ते दे॒वा अ॑मन्य॒न्तेन्द्रो॒ वा इ॒दम॑भू॒द्यद्व॒यꣴ स्म इति॒ ते᳚\-ऽब्रुव॒न्मघ॑व॒न्ननु॑ न॒ आ भ॒जेति॑ स॒जोषा॑ दे॒वैरव॑रैः॒ परै॒श्चेत्य॑ब्रवी॒द्ये चै॒व दे॒वाः परे॒ ये चाव॑रे॒ तानु॒भयान्॑~(२८)

%6.4.6.3
अ॒न्वाभ॑जथ्स॒जोषा॑ दे॒वैरव॑रैः॒ परै॒श्चेत्या॑ह॒ ये चै॒व दे॒वाः परे॒ य चाव॑रे॒ तानु॒भया॑न॒न्वाभ॑जत्यन्तर्या॒मे म॑घवन्मादय॒स्वेत्या॑ह य॒ज्ञादे॒व यज॑मानं॒ नान्तरे᳚त्युपया॒मगृ॑हीतो॒\-ऽसीत्या॑हापा॒नस्य॒ धृत्यै॒ यदु॒भाव॑पवि॒त्रौ गृ॒ह्येया॑तां प्रा॒णम॑पा॒नो\-ऽनु॒ न्यृ॑च्छेत्प्र॒मायु॑कः स्यात्प॒वित्र॑वानन्तर्या॒मो गृ॑ह्यते~(२९)

%6.4.6.4
प्रा॒णा॒पा॒नयो॒र्विधृ॑त्यै प्राणापा॒नौ वा ए॒तौ यदु॑पाꣳश्वन्तर्या॒मौ व्या॒न उ॑पाꣳशु॒सव॑नो॒ यं का॒मये॑त प्र॒मायु॑कः स्या॒दित्यसꣴ॑स्पृष्टौ॒ तस्य॑ सादयेद्व्या॒नेनै॒वास्य॑ प्राणापा॒नौ वि च्छि॑नत्ति ता॒जक्प्रमी॑यते॒ यं का॒मये॑त॒ सर्व॒मायु॑रिया॒दिति॒ सꣴस्पृ॑ष्टौ॒ तस्य॑ सादयेद्व्या॒नेनै॒वास्य॑ प्राणापा॒नौ सं त॑नोति॒ सर्व॒मायु॑रेति॥~(३०)

%6.4.7.0
{\anuvakamend[{त॒ उ॒भया᳚न्गृह्यते॒ चतु॑श्चत्वारिꣳशच्च}]}%~(६)

%6.4.7.1
वाग्वा ए॒षा यदै᳚न्द्रवाय॒वो यदै᳚न्द्रवाय॒वाग्रा॒ ग्रहा॑ गृ॒ह्यन्ते॒ वाच॑मे॒वानु॒ प्र य॑न्ति वा॒युं दे॒वा अ॑ब्रुव॒न्थ्सोम॒ꣳ॒ राजा॑नꣳ हना॒मेति॒ सो᳚\-ऽब्रवी॒द्वरं॑ वृणै॒ मद॑ग्रा ए॒व वो॒ ग्रहा॑ गृह्यान्ता॒ इति॒ तस्मा॑दैन्द्रवाय॒वाग्रा॒ ग्रहा॑ गृह्यन्ते॒ तम॑घ्न॒न्थ्सो॑\-ऽपूय॒त् तं दे॒वा नोपा॑धृष्णुव॒न्ते वा॒युम॑ब्रुवन्नि॒मं नः॑ स्वदय~(३१)

%6.4.7.2
इति॒ सो᳚\-ऽब्रवी॒द्वरं॑ वृणै मद्देव॒त्या᳚न्ये॒व वः॒ पात्रा᳚ण्युच्यान्ता॒ इति॒ तस्मा᳚न्नानादेव॒त्या॑नि॒ सन्ति॑ वाय॒व्या᳚न्युच्यन्ते॒ तमे᳚भ्यो वा॒युरे॒वास्व॑दय॒त्तस्मा॒द्यत्पूय॑ति॒ तत्प्र॑वा॒ते वि ष॑जन्ति वा॒युर्\mbox{}हि तस्य॑ पवयि॒ता स्व॑दयि॒ता तस्य॑ वि॒ग्रह॑णं॒ नावि॑न्द॒न्थ्सा\-ऽदि॑तिरब्रवी॒द्वरं॑ वृणा॒ अथ॒ मया॒ वि गृ॑ह्णीध्वम्मद्देव॒त्या॑ ए॒व वः॒ सोमाः᳚~(३२)

%6.4.7.3
स॒न्ना अ॑स॒न्नित्यु॑पया॒मगृ॑हीतो॒\-ऽसीत्या॑हादितिदेव॒त्या᳚स्तेन॒ यानि॒ हि दा॑रु॒मया॑णि॒ पात्रा᳚ण्य॒स्यै तानि॒ योनेः॒ सम्भू॑तानि॒ यानि॑ मृ॒न्मया॑नि सा॒क्षात्तान्य॒स्यै तस्मा॑दे॒वमा॑ह॒ वाग्वै परा॒च्यव्या॑कृतावद॒त्ते दे॒वा इन्द्र॑मब्रुवन्नि॒मां नो॒ वाचं॒ व्याकु॒र्विति॒ सो᳚\-ऽब्रवी॒द्वरं॑ वृणै॒ मह्यं॑ चै॒वैष वा॒यवे॑ च स॒ह गृ॑ह्याता॒ इति॒ तस्मा॑दैन्द्रवाय॒वः स॒ह गृ॑ह्यते॒ तामिन्द्रो॑ मध्य॒तो॑\-ऽव॒क्रम्य॒ व्याक॑रो॒त्तस्मा॑दि॒यं व्याकृ॑ता॒ वागु॑द्यते॒ तस्मा᳚थ्स॒कृदिन्द्रा॑य मध्य॒तो गृ॑ह्यते॒ द्विर्वा॒यवे॒ द्वौ हि स वरा॒ववृ॑णीत॥~(३३)

%6.4.8.0
{\anuvakamend[{स्व॒द॒य॒ सोमाः᳚ स॒हाष्टाविꣳ॑शतिश्च}]}%~(७)

%6.4.8.1
मि॒त्रं दे॒वा अ॑ब्रुव॒न्थ्सोम॒ꣳ॒ राजा॑नꣳ हना॒मेति॒ सो᳚\-ऽब्रवी॒न्नाहꣳ सर्व॑स्य॒ वा अ॒हम्मि॒त्रम॒स्मीति॒ तम॑ब्रुव॒न्॒ हना॑मै॒वेति॒ सो᳚\-ऽब्रवी॒द्वरं॑ वृणै॒ पय॑सै॒व मे॒ सोमꣴ॑ श्रीण॒न्निति॒ तस्मा᳚न्मैत्रावरु॒णम्पय॑सा श्रीणन्ति॒ तस्मा᳚त्प॒शवो\-ऽपा᳚क्रामन् मि॒त्रः सन्क्रू॒रम॑क॒रिति॑ क्रू॒रमि॑व॒ खलु॒ वा ए॒षः~(३४)

%6.4.8.2
क॒रो॒ति॒ यः सोमे॑न॒ यज॑ते॒ तस्मा᳚त्प॒शवो\-ऽप॑ क्रामन्ति॒ यन्मै᳚त्रावरु॒णम्पय॑सा श्री॒णाति॑ प॒शुभि॑रे॒व तन्मि॒त्रꣳ स॑म॒र्धय॑ति प॒शुभि॒र्यज॑मानं पु॒रा खलु॒ वावैवम्मि॒त्रो॑\-ऽवे॒दप॒ मत्क्रू॒रं च॒क्रुषः॑ प॒शवः॑ क्रमिष्य॒न्तीति॒ तस्मा॑दे॒वम॑वृणीत॒ वरु॑णं दे॒वा अ॑ब्रुव॒न्त्वयाꣳ॑श॒भुवा॒ सोम॒ꣳ॒ राजा॑नꣳ हना॒मेति॒ सो᳚\-ऽब्रवी॒द्वरं॑ वृणै॒ मह्यं॑ च~(३५)

%6.4.8.3
ए॒वैष मि॒त्राय॑ च स॒ह गृ॑ह्याता॒ इति॒ तस्मा᳚न्मैत्रावरु॒णः स॒ह गृ॑ह्यते॒ तस्मा॒द्राज्ञा॒ राजा॑नमꣳश॒भुवा᳚ घ्नन्ति॒ वैश्ये॑न॒ वैश्यꣳ॑ शू॒द्रेण॑ शू॒द्रन्न वा इ॒दं दिवा॒ न नक्त॑मासी॒दव्या॑वृत्त॒न्ते दे॒वा मि॒त्रावरु॑णावब्रुवन्नि॒दं नो॒ वि वा॑सयत॒मिति॒ ताव॑ब्रूतां॒ वरं॑ वृणावहा॒ एक॑ ए॒वावत्पूर्वो॒ ग्रहो॑ ग्रहो गृह्याता॒ इति॒ तस्मा॑दैन्द्रवाय॒वः पूर्वो॑ मैत्रावरु॒णाद्गृ॑ह्यते प्राणापा॒नौ ह्ये॑तौ यदु॑पाꣳश्वन्तर्या॒मौ मि॒त्रो\-ऽह॒रज॑नय॒द्वरु॑णो॒ रात्रिं॒ ततो॒ वा इ॒दं व्यौ᳚च्छ॒द्यन्मै᳚त्रावरु॒णो गृ॒ह्यते॒ व्यु॑ष्ट्यै॥~(३६)

%6.4.9.0
{\anuvakamend[{ए॒ष चै᳚न्द्रवाय॒वो द्वाविꣳ॑शतिश्च}]}%~(८)

%6.4.9.1
य॒ज्ञस्य॒ शिरो᳚\-ऽच्छिद्यत॒ ते दे॒वा अ॒श्विना॑वब्रुवन्भि॒षजौ॒ वै स्थ॑ इ॒दं य॒ज्ञस्य॒ शिरः॒ प्रति॑ धत्त॒मिति॒ ताव॑ब्रूतां॒ वरं॑ वृणावहै॒ ग्रह॑ ए॒व ना॒वत्रापि॑ गृह्यता॒मिति॒ ताभ्या॑मे॒तमा᳚श्वि॒नम॑गृह्ण॒न्ततो॒ वै तौ य॒ज्ञस्य॒ शिरः॒ प्रत्य॑धत्ता॒म् यदा᳚श्वि॒नो गृ॒ह्यते॑ य॒ज्ञस्य॒ निष्कृ॑त्यै॒ तौ दे॒वा अ॑ब्रुव॒न्नपू॑तौ॒ वा इ॒मौ म॑नुष्यच॒रौ~(३७)

%6.4.9.2
भि॒षजा॒विति॒ तस्मा᳚द्ब्राह्म॒णेन॑ भेष॒जं न का॒र्य॑मपू॑तो॒ ह्ये  षो॑\-ऽमे॒ध्यो यो भि॒षक्तौ ब॑हिष्पवमा॒नेन॑ पवयि॒त्वा ताभ्या॑मे॒तमा᳚श्वि॒नम॑गृह्ण॒न्तस्मा᳚द्बहिष्पवमा॒ने स्तु॒त आ᳚श्वि॒नो गृ॑ह्यते॒ तस्मा॑दे॒वं वि॒दुषा॑ बहिष्पवमा॒न उ॑प॒सद्यः॑ प॒वित्रं॒ वै ब॑हिष्पवमा॒न आ॒त्मान॑मे॒व प॑वयते॒ तयो᳚स्त्रे॒धा भैष॑ज्यं॒ वि न्य॑दधुर॒ग्नौ तृ॑तीयम॒फ्सु तृती॑यम्ब्राह्म॒णे तृती॑य॒न्तस्मा॑दुदपा॒त्रम्~(३८)

%6.4.9.3
उ॒प॒नि॒धाय॑ ब्राह्म॒णं द॑क्षिण॒तो नि॒षाद्य॑ भेष॒जं कु॑र्या॒द्याव॑दे॒व भे॑ष॒जं तेन॑ करोति स॒मर्धु॑कमस्य कृ॒तं भ॑वति ब्रह्मवा॒दिनो॑ वदन्ति॒ कस्मा᳚थ्स॒त्यादेक॑पात्रा द्विदेव॒त्या॑ गृ॒ह्यन्ते᳚ द्वि॒पात्रा॑ हूयन्त॒ इति॒ यदेक॑पात्रा गृ॒ह्यन्ते॒ तस्मा॒देको᳚\-ऽन्तर॒तः प्रा॒णो द्वि॒पात्रा॑ हूयन्ते॒ तस्मा॒द्द्वौद्वौ॑ ब॒हिष्टा᳚त्प्रा॒णाः प्रा॒णा वा ए॒ते यद्द्वि॑देव॒त्याः᳚ प॒शव॒ इडा॒ यदिडा॒म्पूर्वां᳚ द्विदेव॒त्ये᳚भ्य उप॒ह्वये॑त~(३९)

%6.4.9.4
प॒शुभिः॑ प्रा॒णान॒न्तर्द॑धीत प्र॒मायु॑कः स्याद्द्विदेव॒त्या᳚न्भक्षयि॒त्वेडा॒मुप॑ ह्वयते प्रा॒णाने॒वात्मन्धि॒त्वा प॒शूनुप॑ ह्वयते॒ वाग्वा ऐ᳚न्द्रवाय॒वश्चक्षु॑र्मैत्रावरु॒णः श्रोत्र॑माश्वि॒नः पु॒रस्ता॑दैन्द्रवाय॒वम्भ॑क्षयति॒ तस्मा᳚त्पु॒रस्ता᳚द्वा॒चा व॑दति पु॒रस्ता᳚न्मैत्रावरु॒णं तस्मा᳚त्पु॒रस्ता॒च्चक्षु॑षा पश्यति स॒र्वतः॑ परि॒हार॑माश्वि॒नं तस्मा᳚थ्स॒र्वतः॒ श्रोत्रे॑ण शृणोति प्रा॒णा वा ए॒ते यद्द्वि॑देव॒त्याः᳚~(४०)

%6.4.9.5
अरि॑क्तानि॒ पात्रा॑णि सादयति॒ तस्मा॒दरि॑क्ता अन्तर॒तः प्रा॒णा यतः॒ खलु॒ वै य॒ज्ञस्य॒ वित॑तस्य॒ न क्रि॒यते॒ तदनु॑ य॒ज्ञꣳ रक्षा॒ꣴ॒स्यव॑ चरन्ति॒ यदरि॑क्तानि॒ पात्रा॑णि सा॒दय॑ति क्रि॒यमा॑णमे॒व तद्य॒ज्ञस्य॑ शये॒ रक्ष॑सा॒मन॑न्ववचाराय॒ दक्षि॑णस्य हवि॒र्धान॒स्योत्त॑रस्यां वर्त॒न्याꣳ सा॑दयति वा॒च्ये॑व वाचं॑ दधा॒त्या तृ॑तीयसव॒नात्परि॑ शेरे य॒ज्ञस्य॒ सन्त॑त्यै॥~(४१)

%6.4.10.0
{\anuvakamend[{म॒नु॒ष्य॒च॒रावु॑दपा॒त्रमु॑प॒ह्वये॑त द्विदेव॒त्याः᳚ षट्च॑त्वारिꣳशच्च}]}%~(९)

%6.4.10.1
बृह॒स्पति॑र्दे॒वानां᳚ पु॒रोहि॑त॒ आसी॒च्छण्डा॒मर्का॒वसु॑राणां॒ ब्रह्म॑ण्वन्तो दे॒वा आस॒न्ब्रह्म॑ण्व॒न्तो\-ऽसु॑रा॒स्ते \-ऽन्यो᳚न्यं नाश॑क्नुवन्न॒भिभ॑वितु॒न्ते दे॒वाः शण्डा॒मर्का॒वुपा॑मन्त्रयन्त॒ ताव॑ब्रूतां॒ वरं॑ वृणावहै॒ ग्रहा॑वे॒व ना॒वत्रापि॑ गृह्येता॒मिति॒ ताभ्या॑मे॒तौ शु॒क्राम॒न्थिना॑वगृह्ण॒न्ततो॑ दे॒वा अभ॑व॒न्परासु॑रा॒ यस्यै॒वं वि॒दुषः॑ शु॒क्राम॒न्थिनौ॑ गृ॒ह्येते॒ भव॑त्या॒त्मना॒ परा᳚~(४२)

%6.4.10.2
अ॒स्य॒ भ्रातृ॑व्यो भवति॒ तौ दे॒वा अ॑प॒नुद्या॒त्मन॒ इन्द्रा॑याजुहवु॒रप॑नुत्तौ॒ शण्डा॒मर्कौ॑ स॒हामुनेति॑ ब्रूया॒द्यं द्वि॒ष्याद्यमे॒व द्वेष्टि॒ तेनै॑नौ स॒हाप॑ नुदते॒ स प्र॑थ॒मः सङ्कृ॑तिर्वि॒श्वक॒र्मेत्ये॒वैना॑वा॒त्मन॒ इन्द्रा॑याजुहवु॒रिन्द्रो॒ ह्ये॑तानि॑ रू॒पाणि॒ करि॑क्र॒दच॑रद॒सौ वा आ॑दि॒त्यः शु॒क्रश्च॒न्द्रमा॑ म॒न्थ्य॑पि॒गृह्य॒ प्राञ्चौ॒ निः~(४३)

%6.4.10.3
क्रा॒म॒त॒स्तस्मा॒त्प्राञ्चौ॒ यन्तौ॒ न प॑श्यन्ति प्र॒त्यञ्चा॑वा॒वृत्य॑ जुहुत॒स्तस्मा᳚त्प्र॒त्यञ्चौ॒ यन्तौ॑ पश्यन्ति॒ चक्षु॑षी॒ वा ए॒ते य॒ज्ञस्य॒ यच्छु॒क्राम॒न्थिनौ॒ नासि॑कोत्तरवे॒दिर॒भितः॑ परि॒क्रम्य॑ जुहुत॒स्तस्मा॑द॒भितो॒ नासि॑कां॒ चक्षु॑षी॒ तस्मा॒न्नासि॑कया॒ चक्षु॑षी॒ विधृ॑ते स॒र्वतः॒ परि॑ क्रामतो॒ रक्ष॑सा॒मप॑हत्यै दे॒वा वै याः प्राची॒राहु॑ती॒रजु॑हवु॒र्ये पु॒रस्ता॒दसु॑रा॒ आस॒न्ताꣴ स्ताभिः॒ प्र~(४४)

%6.4.10.4
अ॒नु॒द॒न्त॒ याः प्र॒तीची॒र्ये प॒श्चादसु॑रा॒ आस॒न्ताꣴस्ताभि॒रपा॑नुदन्त॒ प्राची॑र॒न्या आहु॑तयो हू॒यन्ते᳚ प्र॒त्यञ्चौ॑ शु॒क्राम॒न्थिनौ॑ प॒श्चाच्चै॒व पु॒रस्ता᳚च्च॒ यज॑मानो॒ भ्रातृ॑व्या॒न्प्र णु॑दते॒ तस्मा॒त्परा॑चीः प्र॒जाः प्र वी॑यन्ते प्र॒तीची᳚र्जायन्ते शु॒क्राम॒न्थिनौ॒ वा अनु॑ प्र॒जाः प्र जा॑यन्ते॒\-ऽत्त्रीश्चा॒द्या᳚श्च सु॒वीराः᳚ प्र॒जाः प्र॑ज॒नय॒न्परी॑हि शु॒क्रः शु॒क्रशो॑चिषा~(४५)

%6.4.10.5
सु॒प्र॒जाः प्र॒जाः प्र॑ज॒नय॒न्परी॑हि म॒न्थी म॒न्थिशो॑चि॒षेत्या॑है॒ता वै सु॒वीरा॒ या अ॒त्त्रीरे॒ताः सु॑प्र॒जा या आ॒द्या॑ य ए॒वं वेदा॒त्त्र्य॑स्य प्र॒जा जा॑यते॒ नाद्या᳚ प्र॒जाप॑ते॒रक्ष्य॑श्वय॒त्तत्परा॑पत॒त्तद्विक॑ङ्कतं॒ प्रावि॑श॒त्तद्विक॑ङ्कते॒ नार॑मत॒ तद्यवं॒ प्रावि॑श॒त् तद्यवे॑\-ऽरमत॒ तद्यव॑स्य~(४६)

%6.4.10.6
य॒व॒त्वं यद्वैक॑ङ्कतम्मन्थिपा॒त्रम्भव॑ति॒ सक्तु॑भिः श्री॒णाति॑ प्र॒जाप॑तेरे॒व तच्चक्षुः॒ सम्भ॑रति ब्रह्मवा॒दिनो॑ वदन्ति॒ कस्मा᳚थ्स॒त्यान्म॑न्थिपा॒त्रꣳ सदो॒ नाश्ञु॑त॒ इत्या᳚र्तपा॒त्रꣳ हीति॑ ब्रूया॒द्यद॑श्ञुवी॒तान्धो᳚\-ऽध्व॒र्युः स्या॒दार्ति॒मार्च्छे॒त्तस्मा॒न्नाश्ञु॑ते॥~(४७)

%6.4.11.0
{\anuvakamend[{आ॒त्मना॒ परा॒ निष्प्र शु॒क्रशो॑चिषा॒ यव॑स्य स॒प्तत्रिꣳ॑शच्च}]}%॥10॥

%6.4.11.1
दे॒वा वै यद्य॒ज्ञे\-ऽकु॑र्वत॒ तदसु॑रा अकुर्वत॒ ते दे॒वा आ᳚ग्रय॒णाग्रा॒न्ग्रहा॑नपश्य॒न्तान॑गृह्णत॒ ततो॒ वै ते\-ऽग्रं॒ पर्या॑य॒न्॒ यस्यै॒वं वि॒दुष॑ आग्रय॒णाग्रा॒ ग्रहा॑ गृ॒ह्यन्ते\-ऽग्र॑मे॒व स॑मा॒नानां॒ पर्ये॑ति रु॒ग्णव॑त्य॒र्चा भ्रातृ॑व्यवतो गृह्णीया॒द्भ्रातृ॑व्यस्यै॒व रु॒क्त्वाग्रꣳ॑ समा॒नानां॒ पर्ये॑ति॒ ये दे॑वा दि॒व्येका॑\-दश॒ स्थेत्या॑ह~(४८)

%6.4.11.2
ए॒ताव॑ती॒र्वै दे॒वता॒स्ताभ्य॑ ए॒वैन॒ꣳ॒ सर्वा᳚भ्यो गृह्णात्ये॒ष ते॒ योनि॒र्विश्वे᳚भ्यस्त्वा दे॒वेभ्य॒ इत्या॑ह वैश्वदे॒वो ह्ये॑ष दे॒वत॑या॒ वाग्वै दे॒वेभ्यो\-ऽपा᳚क्रामद्य॒ज्ञायाति॑ष्ठमाना॒ ते दे॒वा वा॒च्यप॑क्रान्तायां तू॒ष्णीं ग्रहा॑नगृह्णत॒ सा\-ऽम॑न्यत॒ वाग॒न्तर्य॑न्ति॒ वै मेति॒ साग्र॑य॒णम्प्रत्याग॑च्छ॒त्तदा᳚ग्रय॒णस्या᳚ग्रयण॒त्वम्~(४९)

%6.4.11.3
तस्मा॑दाग्रय॒णे वाग्वि सृ॑ज्यते॒ यत्तू॒ष्णीम्पूर्वे॒ ग्रहा॑ गृ॒ह्यन्ते॒ यथा᳚ थ्सा॒रीय॑ति म॒ आख॒ इय॑ति॒ नाप॑ राथ्स्या॒मीत्यु॑पावसृ॒जत्ये॒वमे॒व तद॑ध्व॒र्युरा᳚ग्रय॒णं गृ॑ही॒त्वा य॒ज्ञमा॒रभ्य॒ वाचं॒ वि सृ॑जते॒ त्रिर्\mbox{}हिं क॑रोत्युद्गा॒तॄने॒व तद्वृ॑णीते प्र॒जा\-प॑ति॒र्वा ए॒ष यदा᳚ग्रय॒णो यदा᳚ग्रय॒णं गृ॑ही॒त्वा हिं॑क॒रोति॑ प्र॒जा\-प॑तिरे॒व~(५०)

%6.4.11.4
तत्प्र॒जा अ॒भि जि॑घ्रति॒ तस्मा᳚द्व॒थ्सं जा॒तं गौर॒भि जि॑घ्रत्या॒त्मा वा ए॒ष य॒ज्ञस्य॒ यदा᳚ग्रय॒णः सव॑नेसवने॒\-ऽभि गृ॑ह्णात्या॒त्मन्ने॒व य॒ज्ञꣳ सं त॑नोत्यु॒परि॑ष्टा॒दा न॑यति॒ रेत॑ ए॒व तद्द॑धात्य॒धस्ता॒दुप॑ गृह्णाति॒ प्र ज॑नयत्ये॒व तद्ब्र॑ह्मवा॒दिनो॑ वदन्ति॒ कस्मा᳚थ्स॒त्याद्गा॑य॒त्री कनि॑ष्ठा॒ छन्द॑साꣳ स॒ती सर्वा॑णि॒ सव॑नानि वह॒तीत्ये॒ष वै गा॑यत्रि॒यै व॒थ्सो यदा᳚ग्रय॒णस्तमे॒व तद॑भिनि॒वर्त॒ꣳ॒ सर्वा॑णि॒ सव॑नानि वहति॒ तस्मा᳚द्व॒थ्सम॒पाकृ॑तं॒ गौर॒भि नि व॑र्तते॥~(५१)

%6.5.0.0
{\anuvakamend[{आ॒हा॒ग्र॒य॒ण॒त्वं प्र॒जा\-प॑तिरे॒वेति॑ विꣳश॒तिश्च॑}]}%॥11॥

%6.5.0.0

{\anuvakamend[{इन्द्रो॑ वृ॒त्रायायु॒र्वै य॒ज्ञेन॑ सुव॒र्गायेन्द्रो॑ म॒रुद्भि॒रदि॑तिरन्तर्यामपा॒त्रेण॑ प्रा॒ण उ॑पाꣳशुपा॒त्रेणेन्द्रो॑ वृ॒त्रम॑ह॒न्तस्य॒ ग्रहा॒न्॒ वै प्रान्यान्येका॑\-दश}]}%॥11॥ 
{\prashnaend{इन्द्रो॑ वृ॒त्राय॒ पुन॑र्\mbox{}ऋ॒तुना॑ह मिथु॒नम्प॒शवो॒ नेष्टः॒ पत्नी॑मुपाꣳश्वन्तर्या॒मयो॒र्द्विच॑त्वारिꣳशत्॥42॥ इन्द्रो॑ वृ॒त्राय॑ पाङ्क्त॒त्वम्॥}}
%%% END PRASHNA
