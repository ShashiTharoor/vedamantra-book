\sect{चतुर्थः प्रश्नः}\setcounter{anuvakam}{0}
\dnsub{तैत्तिरीयसंहितायां द्वितीयकाण्डे चतुर्थः प्रश्नः}
%2.4.1.0
%2.4.1.1
दे॒वा म॑नु॒ष्याः᳚ पि॒तर॒स्ते᳚\-ऽन्यत॑ आस॒न्नसु॑रा॒ रक्षाꣳ॑सि पिशा॒चास्ते᳚\-ऽन्यत॒स्तेषां᳚ दे॒वाना॑मु॒त यदल्पं॒ लोहि॑त॒मकु॑र्व॒न्तद्रक्षाꣳ॑सि॒ रात्री॑भिरसुभ्न॒न्तान्थ्सु॒ब्धान्मृ॒तान॒भि व्यौ᳚च्छ॒त्ते दे॒वा अ॑विदु॒र्यो वै नो॒\-ऽयम्म्रि॒यते॒ रक्षाꣳ॑सि॒ वा इ॒मं घ्न॒न्तीति॒ ते रक्षा॒ꣴ॒स्युपा॑मन्त्रयन्त॒ तान्य॑ब्रुव॒न्वरं॑ वृणामहै॒ यत्~(१)

%2.4.1.2
असु॑रा॒ञ्जया॑म॒ तन्नः॑ स॒हास॒दिति॒ ततो॒ वै दे॒वा असु॑रानजय॒न्ते\-ऽसु॑राञ्जि॒त्वा रक्षा॒ꣴ॒स्यपा॑नुदन्त॒ तानि॒ रक्षा॒ꣴ॒स्यनृ॑तमक॒र्तेति॑ सम॒न्तं दे॒वान्पर्य॑विश॒न्ते दे॒वा अ॒ग्नाव॑नाथन्त॒ ते᳚\-ऽग्नये॒ प्रव॑ते पुरो॒डाश॑म॒ष्टाक॑पालं॒ निर॑वपन्न॒ग्नये॑ विबा॒धव॑ते॒\-ऽग्नये॒ प्रती॑कवते॒ यद॒ग्नये॒ प्रव॑ते नि॒रव॑प॒न् यान्ये॒व पु॒रस्ता॒द्रक्षाꣳ॑सि~(२)

%2.4.1.3
आस॒न्तानि॒ तेन॒ प्राणु॑दन्त॒ यद॒ग्नये॑ विबा॒धव॑ते॒ यान्ये॒वाभितो॒ रक्षा॒ꣴ॒स्यास॒न्तानि॒ तेन॒ व्य॑बाधन्त॒ यद॒ग्नये॒ प्रती॑कवते॒ यान्ये॒व प॒श्चाद्रक्षा॒ꣴ॒स्यास॒न्तानि॒ तेनापा॑नुदन्त॒ ततो॑ दे॒वा अभ॑व॒न्परासु॑रा॒ यो भ्रातृ॑व्यवा॒न्थ्स्याथ्स स्पर्ध॑मान ए॒तयेष्ट्या॑ यजेता॒ग्नये॒ प्रव॑ते पुरो॒डाश॑म॒ष्टाक॑पालं॒ निर्व॑पेद॒ग्नये॑ विबा॒धव॑ते~(३)

%2.4.1.4
अ॒ग्नये॒ प्रती॑कवते॒ यद॒ग्नये॒ प्रव॑ते नि॒र्वप॑ति॒ य ए॒वास्मा॒च्छ्रेया॒न्भ्रातृ॑व्य॒स्तं तेन॒ प्र णु॑दते॒ यद॒ग्नये॑ विबा॒धव॑ते॒ य ए॒वैने॑न स॒दृङ्तं तेन॒ वि बा॑धते॒ यद॒ग्नये॒ प्रती॑कवते॒ य ए॒वास्मा॒त्पापी॑या॒न्तं तेनाप॑ नुदते॒ प्र श्रेयाꣳ॑स॒म्भ्रातृ॑व्यं नुद॒ते\-ऽति॑ स॒दृशं॑ क्रामति॒ नैन॒म्पापी॑यानाप्नोति॒ य ए॒वं वि॒द्वाने॒तयेष्ट्या॒ यज॑ते~(४)

%2.4.2.0
{\anuvakamend[{वृ॒णा॒म॒है॒ यत्पु॒रस्ता॒द्रक्षाꣳ॑सि वपेद॒ग्नये॑ विबा॒धव॑त ए॒वं च॒त्वारि॑ च}]}%~(१)

%2.4.2.1
दे॒वा॒सु॒राः संय॑त्ता आस॒न्ते दे॒वा अ॑ब्रुव॒न् यो नो॑ वी॒र्या॑वत्तम॒स्तमनु॑ स॒मार॑भामहा॒ इति॒ त इन्द्र॑मब्रुव॒न्त्वं वै नो॑ वी॒र्या॑वत्तमो\-ऽसि॒ त्वामनु॑ स॒मार॑भामहा॒ इति॒ सो᳚\-ऽब्रवीत्ति॒स्रो म॑ इ॒मास्त॒नुवो॑ वी॒र्या॑वती॒स्ताः प्री॑णी॒ताथासु॑रान॒भि भ॑विष्य॒थेति॒ ता वै ब्रू॒हीत्य॑ब्रुवन्नि॒यमꣳ॑हो॒मुगि॒यं वि॑मृ॒धेयमि॑न्द्रि॒याव॑ती~(५)

%2.4.2.2
इत्य॑ब्रवी॒त्त इन्द्रा॑याꣳहो॒मुचे॑ पुरो॒डाश॒मेका॑दशकपालं॒ निर॑वप॒न्निन्द्रा॑य वैमृ॒धायेन्द्रा॑येन्द्रि॒याव॑ते॒ यदिन्द्रा॑याꣳहो॒मुचे॑ नि॒रव॑प॒न्नꣳह॑स ए॒व तेना॑मुच्यन्त॒ यदिन्द्रा॑य वैमृ॒धाय॒ मृध॑ ए॒व तेनापा᳚घ्नत॒ यदिन्द्रा॑येन्द्रि॒याव॑त इन्द्रि॒यमे॒व तेना॒त्मन्न॑दधत॒ त्रय॑स्त्रिꣳशत्कपालं पुरो॒डाशं॒ निर॑वप॒न्त्रय॑स्त्रिꣳश॒द्वै दे॒वता॒स्ता इन्द्र॑ आ॒त्मन्ननु॑ स॒मार॑म्भयत॒ भूत्यै᳚~(६)

%2.4.2.3
तां वाव दे॒वा विजि॑तिमुत्त॒मामसु॑रै॒र्व्य॑जयन्त॒ यो भ्रातृ॑व्यवा॒न्थ्स्याथ्स स्पर्ध॑मान ए॒तयेष्ट्या॑ यजे॒तेन्द्रा॑याꣳहो॒मुचे॑ पुरो॒डाश॒मेका॑दशकपालं॒ निर्व॑पे॒दिन्द्रा॑य वैमृ॒धायेन्द्रा॑येन्द्रि॒याव॒ते\-ऽꣳह॑सा॒ वा ए॒ष गृ॑ही॒तो यस्मा॒च्छ्रेया॒न्भ्रातृ॑व्यो॒ यदिन्द्रा॑याꣳहो॒मुचे॑ नि॒र्वप॒त्यꣳह॑स ए॒व तेन॑ मुच्यते मृ॒धा वा ए॒षो॑\-ऽभिष॑ण्णो॒ यस्मा᳚थ्समा॒नेष्व॒न्यः श्रेया॑नु॒त~(७)

%2.4.2.4
अभ्रा॑तृव्यो॒ यदिन्द्रा॑य वैमृ॒धाय॒ मृध॑ ए॒व तेनाप॑ हते॒ यदिन्द्रा॑येन्द्रि॒याव॑त इन्द्रि॒यमे॒व तेना॒त्मन्ध॑त्ते॒ त्रय॑स्त्रिꣳशत्कपालं पुरो॒डाशं॒ निर्व॑पति॒ त्रय॑स्त्रिꣳश॒द्वै दे॒वता॒स्ता ए॒व यज॑मान आ॒त्मन्ननु॑ स॒मार॑म्भयते॒ भूत्यै॒ सा वा ए॒षा विजि॑ति॒र्नामेष्टि॒र्य ए॒वं वि॒द्वाने॒तयेष्ट्या॒ यज॑त उत्त॒मामे॒व विजि॑ति॒म्भ्रातृ॑व्येण॒ वि ज॑यते~(८)

%2.4.3.0
{\anuvakamend[{इ॒न्द्रि॒याव॑ती॒ भूत्या॑ उ॒तैका॒न्नप॑ञ्चा॒शच्च॑}]}%~(२)

%2.4.3.1
दे॒वा॒सु॒राः संय॑त्ता आस॒न्तेषां᳚ गाय॒त्र्योजो॒ बल॑मिन्द्रि॒यं वी॒र्यं॑ प्र॒जां प॒शून्थ्सं॒गृह्या॒दाया॑प॒क्रम्या॑तिष्ठ॒त्ते॑\-ऽमन्यन्त यत॒रान् वा इ॒यमु॑पाव॒र्थ्स्यति॒ त इ॒दम्भ॑विष्य॒न्तीति॒ तां व्य॑ह्वयन्त॒ विश्व॑कर्म॒न्निति॑ दे॒वा दाभीत्यसु॑राः॒ सा नान्य॑त॒राꣴश्च॒ नोपाव॑र्तत॒ ते दे॒वा ए॒तद्यजु॑रपश्य॒न्नोजो॑\-ऽसि॒ सहो॑\-ऽसि॒ बल॑मसि~(९)

%2.4.3.2
भ्राजो॑\-ऽसि दे॒वानां॒ धाम॒ नामा॑सि॒ विश्व॑मसि वि॒श्वायुः॒ सर्व॑मसि स॒र्वायु॑रभि॒भूरिति॒ वाव दे॒वा असु॑राणा॒मोजो॒ बल॑मिन्द्रि॒यं वी॒र्यं॑ प्र॒जां प॒शून॑वृञ्जत॒ यद्गा॑य॒त्र्य॑प॒क्रम्याति॑ष्ठ॒त्तस्मा॑दे॒तां गा॑य॒त्रीतीष्टि॑माहुः संवथ्स॒रो वै गा॑य॒त्री सं॑वथ्स॒रो वै तद॑प॒क्रम्या॑तिष्ठ॒द्यदे॒तया॑ दे॒वा असु॑राणा॒मोजो॒ बल॑मिन्द्रि॒यं वी॒र्यम्᳚~(१०)

%2.4.3.3
प्र॒जां प॒शूनवृ॑ञ्जत॒ तस्मा॑दे॒ताꣳ सं॑व॒र्ग इतीष्टि॑माहु॒र्यो भ्रातृ॑व्यवा॒न्थ्स्याथ्स स्पर्ध॑मान ए॒तयेष्ट्या॑ यजेता॒ग्नये॑ संव॒र्गाय॑ पुरो॒डाश॑म॒ष्टाक॑पालं॒ निर्व॑पे॒त्तꣳ शृ॒तमास॑न्नमे॒तेन॒ यजु॑षा॒भि मृ॑शे॒दोज॑ ए॒व बल॑मिन्द्रि॒यं वी॒र्यं॑ प्र॒जां प॒शून्भ्रातृ॑व्यस्य वृङ्क्ते॒ भव॑त्या॒त्मना॒ परा᳚स्य॒ भ्रातृ॑व्यो भवति~(११)

%2.4.4.0
{\anuvakamend[{बल॑मस्ये॒तया॑ दे॒वा असु॑राणा॒मोजो॒ बल॑मिन्द्रि॒यं वी॒र्यं॑ पञ्च॑चत्वारिꣳशच्च}]}%~(३)

%2.4.4.1
प्र॒जाप॑तिः प्र॒जा अ॑सृजत॒ ता अ॑स्माथ्सृ॒ष्टाः परा॑चीराय॒न्ता यत्राव॑स॒न्ततो॑ ग॒र्मुदुद॑तिष्ठ॒त्ता बृह॒स्पति॑श्चा॒न्ववै॑ता॒ꣳ॒ सो᳚\-ऽब्रवी॒द्बृह॒स्पति॑र॒नया᳚ त्वा॒ प्र ति॑ष्ठा॒न्यथ॑ त्वा प्र॒जा उ॒पाव॑र्थ्स्य॒न्तीति॒ तम्प्राति॑ष्ठ॒त्ततो॒ वै प्र॒जाप॑तिं प्र॒जा उ॒पाव॑र्तन्त॒ यः प्र॒जाका॑मः॒ स्यात्तस्मा॑ ए॒तम्प्रा॑जाप॒त्यं गा᳚र्मु॒तं च॒रुं निर्व॑पेत्प्र॒जाप॑तिम्~(१२)

%2.4.4.2
ए॒व स्वेन॑ भाग॒धेये॒नोप॑ धावति॒ स ए॒वास्मै᳚ प्र॒जाम्प्र ज॑नयति प्र॒जाप॑तिः प॒शून॑सृजत॒ ते᳚\-ऽस्माथ्सृ॒ष्टाः परा᳚ञ्च आय॒न्ते यत्राव॑स॒न्ततो॑ ग॒र्मुदुद॑तिष्ठ॒त्तान्पू॒षा चा॒न्ववै॑ता॒ꣳ॒ सो᳚\-ऽब्रवीत्पू॒षानया॑ मा॒ प्र ति॒ष्ठाथ॑ त्वा प॒शव॑ उ॒पाव॑र्थ्स्य॒न्तीति॒ माम्प्र ति॒ष्ठेति॒ सोमो᳚\-ऽब्रवी॒न्मम॒ वै~(१३)

%2.4.4.3
अ॒कृ॒ष्ट॒प॒च्यमित्यु॒भौ वा॒म्प्र ति॑ष्ठा॒नीत्य॑ब्रवी॒त्तौ प्राति॑ष्ठ॒त्ततो॒ वै प्र॒जाप॑तिम्प॒शव॑ उ॒पाव॑र्तन्त॒ यः प॒शुका॑मः॒ स्यात्तस्मा॑ ए॒तꣳ सो॑मापौ॒ष्णं गा᳚र्मु॒तं च॒रुं निर्व॑पेथ्सोमापू॒षणा॑वे॒व स्वेन॑ भाग॒धेये॒नोप॑ धावति॒ तावे॒वास्मै॑ प॒शून्प्र ज॑नयतः॒ सोमो॒ वै रे॑तो॒धाः पू॒षा प॑शू॒नाम्प्र॑जनयि॒ता सोम॑ ए॒वास्मै॒ रेतो॒ दधा॑ति पू॒षा प॒शून्प्र ज॑नयति~(१४)

%2.4.5.0
{\anuvakamend[{व॒पे॒त्प्र॒जाप॑तिं॒ वै दधा॑ति पू॒षा त्रीणि॑ च}]}%~(४)

%2.4.5.1
अग्ने॒ गोभि॑र्न॒ आ ग॒हीन्दो॑ पु॒ष्ट्या जु॑षस्व नः। इन्द्रो॑ ध॒र्ता गृ॒हेषु॑ नः॥ स॒वि॒ता यः स॑ह॒स्रियः॒ स नो॑ गृ॒हेषु॑ रारणत्। आ पू॒षा ए॒त्वा वसु॑॥ धा॒ता द॑दातु नो र॒यिमीशा॑नो॒ जग॑त॒स्पतिः॑। स नः॑ पू॒र्णेन॑ वावनत्॥ त्वष्टा॒ यो वृ॑ष॒भो वृषा॒ स नो॑ गृ॒हेषु॑ रारणत्। स॒हस्रे॑णा॒युते॑न च॥ येन॑ दे॒वा अ॒मृतम्᳚~(१५)

%2.4.5.2
दी॒र्घꣴ श्रवो॑ दि॒व्यैर॑यन्त। राय॑स्पोष॒ त्वम॒स्मभ्यं॒ गवां᳚ कु॒ल्मिं जी॒वस॒ आ यु॑वस्व। अ॒ग्निर्गृ॒हप॑तिः॒ सोमो॑ विश्व॒वनिः॑ सवि॒ता सु॑मे॒धाः स्वाहा᳚। अग्ने॑ गृहपते॒ यस्ते॒ घृत्यो॑ भा॒गस्तेन॒ सह॒ ओज॑ आ॒क्रम॑माणाय धेहि॒ श्रैष्ठ्या᳚त्प॒थो मा यो॑षं मू॒र्धा भू॑यास॒ꣴ॒ स्वाहा᳚~(१६)

%2.4.6.0
{\anuvakamend[{अ॒मृत॑म॒ष्टात्रिꣳ॑शच्च}]}%~(५)

%2.4.6.1
चि॒त्रया॑ यजेत प॒शुका॑म इ॒यं वै चि॒त्रा यद्वा अ॒स्यां विश्व॑म्भू॒तमधि॑ प्र॒जाय॑ते॒ तेने॒यं चि॒त्रा य ए॒वं वि॒द्वाꣴश्चि॒त्रया॑ प॒शुका॑मो॒ यज॑ते॒ प्र प्र॒जया॑ प॒शुभि॑र्मिथु॒नैर्जा॑यते॒ प्रैवाग्ने॒येन॑ वापयति॒ रेतः॑ सौ॒म्येन॑ दधाति॒ रेत॑ ए॒व हि॒तं त्वष्टा॑ रू॒पाणि॒ वि क॑रोति सारस्व॒तौ भ॑वत ए॒तद्वै दैव्य॑म्मिथु॒नं दैव्य॑मे॒वास्मै᳚~(१७)

%2.4.6.2
मि॒थु॒नम्म॑ध्य॒तो द॑धाति॒ पुष्ट्यै᳚ प्र॒जन॑नाय सिनीवा॒ल्यै च॒रुर्भ॑वति॒ वाग्वै सि॑नीवा॒ली पुष्टिः॒ खलु॒ वै वाक्पुष्टि॑मे॒व वाच॒मुपै᳚त्यै॒न्द्र उ॑त्त॒मो भ॑वति॒ तेनै॒व तन्मि॑थु॒नꣳ स॒प्तैतानि॑ ह॒वीꣳषि॑ भवन्ति स॒प्त ग्रा॒म्याः प॒शवः॑ स॒प्तार॒ण्याः स॒प्त छन्दाꣳ॑स्यु॒भय॒स्याव॑रुद्ध्या॒ अथै॒ता आहु॑तीर्जुहोत्ये॒ते वै दे॒वाः पुष्टि॑पतय॒स्त ए॒वास्मि॒न्पुष्टिं॑ दधति॒ पुष्य॑ति प्र॒जया॑ प॒शुभि॒रथो॒ यदे॒ता आहु॑तीर्जु॒होति॒ प्रति॑ष्ठित्यै~(१८)

%2.4.7.0
{\anuvakamend[{अ॒स्मै॒ त ए॒व द्वाद॑श च}]}%~(६)

%2.4.7.1
मा॒रु॒तम॑सि म॒रुता॒मोजो॒\-ऽपां धारां᳚ भिन्द्धि र॒मय॑त मरुतः श्ये॒नमा॒यिन॒म्मनो॑जवसं॒ वृष॑णꣳ सुवृ॒क्तिम्। येन॒ शर्ध॑ उ॒ग्रमव॑सृष्ट॒मेति॒ तद॑श्विना॒ परि॑ धत्तꣴ स्व॒स्ति। पु॒रो॒वा॒तो वर्\mbox{}ष॑ञ्जि॒न्वरा॒वृथ्स्वाहा॑ वा॒ताव॒द्वर्\mbox{}ष॑न्नु॒ग्ररा॒वृथ्स्वाहा᳚ स्त॒नय॒न्वर्\mbox{}ष॑न्भी॒मरा॒वृथ्स्वाहा॑नश॒न्य॑व॒स्फूर्ज॑न्दि॒द्युद्वर्\mbox{}ष॑न्त्वे॒षरा॒वृथ्स्वाहा॑तिरा॒त्रं वर्\mbox{}ष॑न्पू॒र्तिरा॒वृत्~(१९)

%2.4.7.2
स्वाहा॑ ब॒हु हा॒यम॑वृषा॒दिति॑ श्रु॒तरा॒वृथ्स्वाहा॒तप॑ति॒ वर्\mbox{}ष॑न्वि॒राडा॒वृथ्स्वाहा॑व॒स्फूर्ज॑न्दि॒द्युद्वर्\mbox{}ष॑न्भू॒तरा॒वृथ्स्वाहा॒ मान्दा॒ वाशाः॒ शुन्ध्यू॒रजि॑राः। ज्योति॑ष्मती॒स्तम॑स्वरी॒रुन्द॑तीः॒ सुफे॑नाः। मित्र॑भृतः॒ क्षत्र॑भृतः॒ सुरा᳚ष्ट्रा इ॒ह मा॑\-ऽवत। वृष्णो॒ अश्व॑स्य सं॒दान॑मसि॒ वृष्ट्यै॒ त्वोप॑ नह्यामि~(२०)

%2.4.8.0
{\anuvakamend[{पू॒र्तिरा॒वृद्द्विच॑त्वारिꣳशच्च}]}%~(७)

%2.4.8.1
देवा॑ वसव्या॒ अग्ने॑ सोम सूर्य। देवाः᳚ शर्मण्या॒ मित्रा॑वरुणार्यमन्न्। देवाः᳚ सपीत॒यो\-ऽपां᳚ नपादाशुहेमन्न्। उ॒द्नो द॑त्तो\-ऽद॒धिम्भि॑न्त दि॒वः प॒र्जन्या॑द॒न्तरि॑क्षात्पृथि॒व्यास्ततो॑ नो॒ वृष्ट्या॑\-ऽवत। दिवा॑ चि॒त्तमः॑ कृण्वन्ति प॒र्जन्ये॑नोदवा॒हेन॑। पृ॒थि॒वीं यद्व्यु॒न्दन्ति॑। आ यं नरः॑ सु॒दान॑वो ददा॒शुषे॑ दि॒वः कोश॒मचु॑च्यवुः। वि प॒र्जन्याः᳚ सृजन्ति॒ रोद॑सी॒ अनु॒ धन्व॑ना यन्ति~(२१)

%2.4.8.2
वृ॒ष्टयः॑। उदी॑रयथा मरुतः समुद्र॒तो यू॒यं वृ॒ष्टिं व॑र्\mbox{}षयथा पुरीषिणः। न वो॑ दस्रा॒ उप॑ दस्यन्ति धे॒नवः॒ शुभं॑ या॒तामनु॒ रथा॑ अवृथ्सत। सृ॒जा वृ॒ष्टिं दि॒व आद्भिः स॑मु॒द्रं पृ॑ण। अ॒ब्जा अ॑सि प्रथम॒जा बल॑मसि समु॒द्रियम्᳚। उन्न॑म्भय पृथि॒वीम्भि॒न्द्धीदं दि॒व्यं नभः॑। उ॒द्नो दि॒व्यस्य॑ नो दे॒हीशा॑नो॒ वि सृ॑जा॒ दृतिम्᳚। ये दे॒वा दि॒विभा॑गा॒ ये᳚\-ऽन्तरि॑क्षभागा॒ ये पृ॑थि॒विभा॑गाः। त इ॒मं य॒ज्ञम॑वन्तु॒ त इ॒दं क्षेत्र॒मा वि॑शन्तु॒ त इ॒दं क्षेत्र॒मनु॒ वि वि॑शन्तु~(२२)

%2.4.9.0
{\anuvakamend[{य॒न्ति॒ दे॒वा विꣳ॑शति॒श्च॑}]}%~(८)

%2.4.9.1
मा॒रु॒तम॑सि म॒रुता॒मोज॒ इति॑ कृ॒ष्णं वासः॑ कृ॒ष्णतू॑षं॒ परि॑ धत्त ए॒तद्वै वृष्ट्यै॑ रू॒पꣳ सरू॑प ए॒व भू॒त्वा प॒र्जन्यं॑ वर्\mbox{}षयति र॒मय॑त मरुतः श्ये॒नमा॒यिन॒मिति॑ पश्चाद्वा॒तं प्रति॑ मीवति पुरोवा॒तमे॒व ज॑नयति व॒र्\mbox{}षस्याव॑रुद्ध्यै वातना॒मानि॑ जुहोति वा॒युर्वै वृष्ट्या॑ ईशे वा॒युमे॒व स्वेन॑ भाग॒धेये॒नोप॑ धावति॒ स ए॒वास्मै॑ प॒र्जन्यं॑ वर्\mbox{}षयत्य॒ष्टौ~(२३)

%2.4.9.2
जु॒हो॒ति॒ चत॑स्रो॒ वै दिश॒श्चत॑स्रो\-ऽवान्तरदि॒शा दि॒ग्भ्य ए॒व वृष्टि॒ꣳ॒ सम्प्र च्या॑वयति कृष्णाजि॒ने सं यौ॑ति ह॒विरे॒वाक॑रन्तर्वे॒दि सं यौ॒त्यव॑रुद्ध्यै॒ यती॑नाम॒द्यमा॑नानाꣳ शी॒र्\mbox{}षाणि॒ परा॑पत॒न्ते ख॒र्जूरा॑ अभव॒न्तेषा॒ꣳ॒ रस॑ ऊ॒र्ध्वो॑\-ऽपत॒त्तानि॑ क॒रीरा᳚ण्यभवन्थ्सौ॒म्यानि॒ वै क॒रीरा॑णि सौ॒म्या खलु॒ वा आहु॑तिर्दि॒वो वृष्टिं॑ च्यावयति॒ यत्क॒रीरा॑णि॒ भव॑न्ति~(२४)

%2.4.9.3
सौ॒म्ययै॒वाहु॑त्या दि॒वो वृष्टि॒मव॑ रुन्द्धे॒ मधु॑षा॒ सं यौ᳚त्य॒पां वा ए॒ष ओष॑धीना॒ꣳ॒ रसो॒ यन्मध्व॒द्भ्य ए॒वौष॑धीभ्यो वर्\mbox{}ष॒त्यथो॑ अ॒द्भ्य ए॒वौष॑धीभ्यो॒ वृष्टिं॒ नि न॑यति॒ मान्दा॒ वाशा॒ इति॒ सं यौ॑ति नाम॒धेयै॑रे॒वैना॒ अच्छै॒त्यथो॒ यथा᳚ ब्रू॒यादसा॒वेहीत्ये॒वमे॒वैना॑ नाम॒धेयै॒रा~(२५)

%2.4.9.4
च्या॒व॒य॒ति॒ वृष्णो॒ अश्व॑स्य सं॒दान॑मसि॒ वृष्ट्यै॒ त्वोप॑ नह्या॒मीत्या॑ह॒ वृषा॒ वा अश्वो॒ वृषा॑ प॒र्जन्यः॑ कृ॒ष्ण इ॑व॒ खलु॒ वै भू॒त्वा व॑र्\mbox{}षति रू॒पेणै॒वैन॒ꣳ॒ सम॑र्धयति व॒र्\mbox{}षस्याव॑रुद्ध्यै~(२६)

%2.4.10.0
{\anuvakamend[{अ॒ष्टौ भव॑न्ति नाम॒धेयै॒रैका॒न्नत्रि॒ꣳ॒शच्च॑}]}%~(९)

%2.4.10.1
देवा॑ वसव्या॒ देवाः᳚ शर्मण्या॒ देवाः᳚ सपीतय॒ इत्या ब॑ध्नाति दे॒वता॑भिरे॒वान्व॒हं वृष्टि॑मिच्छति॒ यदि॒ वर्\mbox{}षे॒त्ताव॑त्ये॒व हो॑त॒व्यं॑ यदि॒ न वर्\mbox{}षे॒च्छ्वो भू॒ते ह॒विर्निर्व॑पेदहोरा॒त्रे वै मि॒त्रावरु॑णावहोरा॒त्राभ्यां॒ खलु॒ वै प॒र्जन्यो॑ वर्\mbox{}षति॒ नक्तं॑ वा॒ हि दिवा॑ वा॒ वर्\mbox{}ष॑ति मि॒त्रावरु॑णावे॒व स्वेन॑ भाग॒धेये॒नोप॑ धावति॒ तावे॒वास्मै᳚~(२७)

%2.4.10.2
अ॒हो॒रा॒त्रा\-भ्यां᳚ प॒र्जन्यं॑ वर्\mbox{}षयतो॒\-ऽग्नये॑ धाम॒च्छदे॑ पुरो॒डाश॑म॒ष्टाक॑पालं॒ निर्व॑पेन्मारु॒तꣳ स॒प्तक॑पालꣳ सौ॒र्यमेक॑कपालम॒ग्निर्वा इ॒तो वृष्टि॒मुदी॑रयति म॒रुतः॑ सृ॒ष्टां न॑यन्ति य॒दा खलु॒ वा अ॒सावा॑दि॒त्यो न्य॑ङ्र॒श्मिभिः॑ पर्या॒वर्त॒ते\-ऽथ॑ वर्\mbox{}षति धाम॒च्छदि॑व॒ खलु॒ वै भू॒त्वा व॑र्\mbox{}षत्ये॒ता वै दे॒वता॒ वृष्ट्या॑ ईशते॒ ता ए॒व स्वेन॑ भाग॒धेये॒नोप॑ धावति॒ ताः~(२८)

%2.4.10.3
ए॒वास्मै॑ प॒र्जन्यं॑ वर्\mbox{}षयन्त्यु॒ताव॑र्\mbox{}षिष्य॒न्वर्\mbox{}ष॑त्ये॒व सृ॒जा वृ॒ष्टिं दि॒व आद्भिः स॑मु॒द्रं पृ॒णेत्या॑हे॒माश्चै॒वामूश्चा॒पः सम॑र्धय॒त्यथो॑ आ॒भिरे॒वामूरच्छै᳚त्य॒ब्जा अ॑सि प्रथम॒जा बल॑मसि समु॒द्रिय॒मित्या॑ह यथाय॒जुरे॒वैतदुन्न॑म्भय पृथि॒वीमिति॑ वर्\mbox{}षा॒ह्वां जु॑होत्ये॒षा वा ओष॑धीनां वृष्टि॒वनि॒स्तयै॒व वृष्टि॒मा च्या॑वयति॒ ये दे॒वा दि॒विभा॑गा॒ इति॑ कृष्णाजि॒नमव॑ धूनोती॒म ए॒वास्मै॑ लो॒काः प्री॒ता अ॒भीष्टा॑ भवन्ति~(२९)

%2.4.11.0
{\anuvakamend[{अ॒स्मै॒ धा॒व॒ति॒ ता वा एक॑विꣳशतिश्च}]}%॥10॥

%2.4.11.1
सर्वा॑णि॒ छन्दाꣳ॑स्ये॒तस्या॒मिष्ट्या॑म॒नूच्या॒नीत्या॑हुस्त्रि॒ष्टुभो॒ वा ए॒तद्वी॒र्यं॑ यत्क॒कुदु॒ष्णिहा॒ जग॑त्यै॒ यदु॑ष्णिहक॒कुभा॑व॒न्वाह॒ तेनै॒व सर्वा॑णि॒ छन्दा॒ꣴ॒स्यव॑ रुन्द्धे गाय॒त्री वा ए॒षा यदु॒ष्णिहा॒ यानि॑ च॒त्वार्यध्य॒क्षरा॑णि॒ चतु॑ष्पाद ए॒व ते प॒शवो॒ यथा॑ पुरो॒डाशे॑ पुरो॒डाशो\-ऽध्ये॒वमे॒व तद्यदृ॒च्यध्य॒क्षरा॑णि॒ यज्जग॑त्या~(३०)

%2.4.11.2
प॒रि॒द॒ध्यादन्तं॑ य॒ज्ञं ग॑मयेत्त्रि॒ष्टुभा॒ परि॑ दधातीन्द्रि॒यं वै वी॒र्यं॑ त्रि॒ष्टुगि॑न्द्रि॒य ए॒व वी॒र्ये॑ य॒ज्ञं प्रति॑ ष्ठापयति॒ नान्तं॑ गमय॒त्यग्ने॒ त्री ते॒ वाजि॑ना॒ त्री ष॒धस्थेति॒ त्रिव॑त्या॒ परि॑ दधाति सरूप॒त्वाय॒ सर्वो॒ वा ए॒ष य॒ज्ञो यत्त्रै॑धात॒वीय॒ङ्कामा॑यकामाय॒ प्र यु॑ज्यते॒ सर्वे᳚भ्यो॒ हि कामे᳚भ्यो य॒ज्ञः प्र॑यु॒ज्यते᳚ त्रैधात॒वीये॑न यजेताभि॒चर॒न्थ्सर्वो॒ वै~(३१)

%2.4.11.3
ए॒ष य॒ज्ञो यत्त्रै॑धात॒वीय॒ꣳ॒ सर्वे॑णै॒वैनं॑ य॒ज्ञेना॒भि च॑रति स्तृणु॒त ए॒वैन॑मे॒तयै॒व य॑जेताभिच॒र्यमा॑णः॒ सर्वो॒ वा ए॒ष य॒ज्ञो यत्त्रै॑धात॒वीय॒ꣳ॒ सर्वे॑णै॒व य॒ज्ञेन॑ यजते॒ नैन॑मभि॒चर᳚न्थ्स्तृणुत ए॒तयै॒व य॑जेत स॒हस्रे॑ण य॒क्ष्यमा॑णः॒ प्रजा॑तमे॒वैन॑द्ददात्ये॒तयै॒व य॑जेत स॒हस्रे॑णेजा॒नो\-ऽन्तं॒ वा ए॒ष प॑शू॒नां ग॑च्छति~(३२)

%2.4.11.4
यः स॒हस्रे॑ण॒ यज॑ते प्र॒जाप॑तिः॒ खलु॒ वै प॒शून॑सृजत॒ ताꣴ स्त्रै॑धात॒वीये॑नै॒वासृ॑जत॒ य ए॒वं वि॒द्वाꣴस्त्रै॑धात॒वीये॑न प॒शुका॑मो॒ यज॑ते॒ यस्मा॑दे॒व योनेः᳚ प्र॒जाप॑तिः प॒शूनसृ॑जत॒ तस्मा॑दे॒वैना᳚न्थ्सृजत॒ उपै॑न॒मुत्त॑रꣳ स॒हस्रं॑ नमति दे॒वता᳚भ्यो॒ वा ए॒ष आ वृ॑श्च्यते॒ यो य॒क्ष्य इत्यु॒क्त्वा न यज॑ते त्रैधात॒वीये॑न यजेत॒ सर्वो॒ वा ए॒ष य॒ज्ञः~(३३)

%2.4.11.5
यत्त्रै॑धात॒वीय॒ꣳ॒ सर्वे॑णै॒व य॒ज्ञेन॑ यजते॒ न दे॒वता᳚भ्य॒ आ वृ॑श्च्यते॒ द्वाद॑शकपालः पुरो॒डाशो॑ भवति॒ ते त्रय॒श्चतु॑ष्कपालास्त्रिः षमृद्ध॒त्वाय॒ त्रयः॑ पुरो॒डाशा॑ भवन्ति॒ त्रय॑ इ॒मे लो॒का ए॒षां लो॒काना॒माप्त्या॒ उत्त॑रउत्तरो॒ ज्याया᳚न्भवत्ये॒वमि॑व॒ हीमे लो॒का य॑व॒मयो॒ मध्य॑ ए॒तद्वा अ॒न्तरि॑क्षस्य रू॒पꣳ समृ॑द्ध्यै॒ सर्वे॑षामभिग॒मय॒न्नव॑ द्य॒त्यछ॑म्बट्कार॒ꣳ॒ हिर॑ण्यं ददाति॒ तेज॑ ए॒व~(३४)

%2.4.11.6
अव॑ रुन्द्धे ता॒र्प्यं द॑दाति प॒शूने॒वाव॑ रुन्द्धे धे॒नुं द॑दात्या॒शिष॑ ए॒वाव॑ रुन्द्धे॒ साम्नो॒ वा ए॒ष वर्णो॒ यद्धिर॑ण्यं॒ यजु॑षां ता॒र्प्यमु॑क्थाम॒दानां᳚ धे॒नुरे॒ताने॒व सर्वा॒न् वर्णा॒नव॑ रुन्द्धे~(३५)

%2.4.12.0
{\anuvakamend[{जग॑त्या\-ऽभि॒चर॒न्थ्सर्वो॒ वै ग॑च्छति य॒ज्ञस्तेज॑ ए॒व त्रि॒ꣳ॒शच्च॑}]}%॥11॥

%2.4.12.1
त्वष्टा॑ ह॒तपु॑त्रो॒ वीन्द्र॒ꣳ॒ सोम॒माह॑र॒त्तस्मि॒न्निन्द्र॑ उपह॒वमै᳚च्छत॒ तं नोपा᳚ह्वयत पु॒त्रम्मे॑\-ऽवधी॒रिति॒ स य॑ज्ञवेश॒सं कृ॒त्वा प्रा॒सहा॒ सोम॑मपिब॒त्तस्य॒ यद॒त्यशि॑ष्यत॒ तत्त्वष्टा॑हव॒नीय॒मुप॒ प्राव॑र्तय॒थ्स्वाहेन्द्र॑शत्रुर्वर्ध॒स्वेति॒ स याव॑दू॒र्ध्वः प॑रा॒विध्य॑ति॒ ताव॑ति स्व॒यमे॒व व्य॑रमत॒ यदि॑ वा॒ ताव॑त्प्रव॒णम्~(३६)

%2.4.12.2
आसी॒द्यदि॑ वा॒ ताव॒दध्य॒ग्नेरासी॒थ्स स॒म्भव॑न्न॒ग्नीषोमा॑व॒भि सम॑भव॒थ्स इ॑षुमा॒त्रमि॑षुमात्रं॒ विष्व॑ङ्ङवर्धत॒ स इ॒माल्लोँ॒कान॑वृणो॒द्यदि॒माल्लोँ॒कानवृ॑णो॒त्तद्वृ॒त्रस्य॑ वृत्र॒त्वन्तस्मा॒दिन्द्रो॑\-ऽबिभे॒दपि॒ त्वष्टा॒ तस्मै॒ त्वष्टा॒ वज्र॑मसिञ्च॒त्तपो॒ वै स वज्र॑ आसी॒त्तमुद्य॑न्तुं॒ नाश॑क्नो॒दथ॒ वै तर्\mbox{}हि॒ विष्णुः॑~(३७)

%2.4.12.3
अ॒न्या दे॒वता॑सी॒थ्सो᳚\-ऽब्रवी॒द्विष्ण॒वेही॒दमा ह॑रिष्यावो॒ येना॒यमि॒दमिति॒ स विष्णु॑स्त्रे॒धात्मानं॒ वि न्य॑धत्त पृथि॒व्यां तृती॑यम॒न्तरि॑क्षे॒ तृती॑यं दि॒वि तृती॑यमभिपर्याव॒र्ताद्ध्यबि॑भे॒द्यत्पृ॑थि॒व्यां तृती॑य॒मासी॒त्तेनेन्द्रो॒ वज्र॒मुद॑यच्छ॒द्विष्ण्व॑नुस्थितः॒ सो᳚\-ऽब्रवी॒न्मा मे॒ प्र हा॒रस्ति॒ वा इ॒दम्~(३८)

%2.4.12.4
मयि॑ वी॒र्यं॑ तत्ते॒ प्र दा᳚स्या॒मीति॒ तद॑स्मै॒ प्राय॑च्छ॒त्तत्प्रत्य॑गृह्णा॒दधा॒ मेति॒ तद्विष्ण॒वेति॒ प्राय॑च्छ॒त्तद्विष्णुः॒ प्रत्य॑गृह्णाद॒स्मास्विन्द्र॑ इन्द्रि॒यं द॑धा॒त्विति॒ यद॒न्तरि॑क्षे॒ तृती॑य॒मासी॒त्तेनेन्द्रो॒ वज्र॒मुद॑यच्छ॒द्विष्ण्व॑नुस्थितः॒ सो᳚\-ऽब्रवी॒न्मा मे॒ प्र हा॒रस्ति॒ वा इ॒दम्~(३९)

%2.4.12.5
मयि॑ वी॒र्यं॑ तत्ते॒ प्र दा᳚स्या॒मीति॒ तद॑स्मै॒ प्राय॑च्छ॒त्तत्प्रत्य॑गृह्णा॒द्द्विर्मा॑धा॒ इति॒ तद्विष्ण॒वेति॒ प्राय॑च्छ॒त्तद्विष्णुः॒ प्रत्य॑गृह्णाद॒स्मास्विन्द्र॑ इन्द्रि॒यं द॑धा॒त्विति॒ यद्दि॒वि तृती॑य॒मासी॒त्तेनेन्द्रो॒ वज्र॒मुद॑यच्छ॒द्विष्ण्व॑नुस्थितः॒ सो᳚\-ऽब्रवी॒न्मा मे॒ प्र हा॒र्येना॒हम्~(४०)

%2.4.12.6
इ॒दमस्मि॒ तत्ते॒ प्र दा᳚स्या॒मीति॒ त्वी (३) इत्य॑ब्रवीथ्स॒न्धान्तु सं द॑धावहै॒ त्वामे॒व प्र वि॑शा॒नीति॒ यन्माम्प्र॑वि॒शेः किम्मा॑ भुञ्ज्या॒ इत्य॑ब्रवी॒त्त्वामे॒वेन्धी॑य॒ तव॒ भोगा॑य॒ त्वाम्प्र वि॑शेय॒मित्य॑ब्रवी॒त्तं वृ॒त्रः प्रावि॑शदु॒दरं॒ वै वृ॒त्रः क्षुत्खलु॒ वै म॑नु॒ष्य॑स्य॒ भ्रातृ॑व्यो॒ यः~(४१)

%2.4.12.7
ए॒वं वेद॒ हन्ति॒ क्षुध॒म्भ्रातृ॑व्य॒न्तद॑स्मै॒ प्राय॑च्छ॒त्तत्प्रत्य॑गृह्णा॒त्त्रिर्मा॑धा॒ इति॒ तद्विष्ण॒वेति॒ प्राय॑च्छ॒त्तद्विष्णुः॒ प्रत्य॑गृह्णाद॒स्मास्विन्द्र॑ इन्द्रि॒यं द॑धा॒त्विति॒ यत्त्रिः प्राय॑च्छ॒त्त्रिः प्र॒त्यगृ॑ह्णा॒त्तत्त्रि॒धातो᳚स्त्रिधातु॒त्वं यद्विष्णु॑र॒न्वति॑ष्ठत॒ विष्ण॒वेति॒ प्राय॑च्छ॒त्तस्मा॑दैन्द्रावैष्ण॒वꣳ ह॒विर्भ॑वति॒ यद्वा इ॒दं किं च॒ तद॑स्मै॒ तत्प्राय॑च्छ॒दृचः॒ सामा॑नि॒ यजूꣳ॑षि स॒हस्रं॒ वा अ॑स्मै॒ तत्प्राय॑च्छ॒त्तस्मा᳚थ्स॒हस्र॑दक्षिणम्~(४२)

%2.4.13.0
{\anuvakamend[{प्र॒व॒णं विष्णु॒र्वा इ॒दमि॒दम॒हं यो भ॑व॒त्येक॑विꣳशतिश्च}]}%॥12॥

%2.4.13.1
दे॒वा वै रा॑ज॒न्या᳚ज्जाय॑मानादबिभयु॒स्तम॒न्तरे॒व सन्तं॒ दाम्नापौ᳚म्भ॒न्थ्स वा ए॒षो\-ऽपो᳚ब्धो जायते॒ यद्रा॑ज॒न्यो॑ यद्वा ए॒षो\-ऽन॑पोब्धो॒ जाये॑त वृ॒त्रान्घ्नꣴश्च॑रे॒द्यं का॒मये॑त राज॒न्य॑मन॑पोब्धो जायेत वृ॒त्रान्घ्नꣴश्च॑रे॒दिति॒ तस्मा॑ ए॒तमै᳚न्द्राबार्\mbox{}हस्प॒त्यं च॒रुं निर्व॑पेदै॒न्द्रो वै रा॑ज॒न्यो᳚ ब्रह्म॒ बृह॒स्पति॒र्ब्रह्म॑णै॒वैनं॒ दाम्नो॒\-ऽपोम्भ॑नान्मुञ्चति हिर॒ण्मयं॒ दाम॒ दक्षि॑णा सा॒क्षादे॒वैनं॒ दाम्नो॒\-ऽपोम्भ॑नान्मुञ्चति~(४३)

%2.4.14.0
{\anuvakamend[{ए॒न॒न्द्वाद॑श च}]}%॥13॥

%2.4.14.1
नवो॑नवो भवति॒ जाय॑मा॒नो\-ऽह्नां᳚ के॒तुरु॒षसा॑मे॒त्यग्रे᳚। भा॒गं दे॒वेभ्यो॒ वि द॑धात्या॒यन्प्र च॒न्द्रमा᳚स्तिरति दी॒र्घमायुः॑। यमा॑दि॒त्या अ॒ꣳ॒शुमा᳚प्या॒यय॑न्ति॒ यमक्षि॑त॒मक्षि॑तयः॒ पिब॑न्ति। तेन॑ नो॒ राजा॒ वरु॑णो॒ बृह॒स्पति॒रा प्या॑ययन्तु॒ भुव॑नस्य गो॒पाः। प्राच्यां᳚ दि॒शि त्वमि॑न्द्रासि॒ राजो॒तोदी᳚च्यां वृत्रहन्वृत्र॒हासि॑। यत्र॒ यन्ति॑ स्रो॒त्यास्तत्~(४४)

%2.4.14.2
जि॒तं ते॑ दक्षिण॒तो वृ॑ष॒भ ए॑धि॒ हव्यः॑। इन्द्रो॑ जयाति॒ न परा॑ जयाता अधिरा॒जो राज॑सु राजयाति। विश्वा॒ हि भू॒याः पृत॑ना अभि॒ष्टीरु॑प॒सद्यो॑ नम॒स्यो॑ यथास॑त्। अ॒स्येदे॒व प्र रि॑रिचे महि॒त्वं दि॒वः पृ॑थि॒व्याः पर्य॒न्तरि॑क्षात्। स्व॒राडिन्द्रो॒ दम॒ आ वि॒श्वगू᳚र्तः स्व॒रिरम॑त्रो ववक्षे॒ रणा॑य। अ॒भि त्वा॑ शूर नोनु॒मो\-ऽदु॑ग्धा इव धे॒नवः॑। ईशा॑नम्~(४५)

%2.4.14.3
अ॒स्य जग॑तः सुव॒र्दृश॒मीशा॑नमिन्द्र त॒स्थुषः॑। त्वामिद्धि हवा॑महे सा॒ता वाज॑स्य का॒रवः॑। त्वां वृ॒त्रेष्वि॑न्द्र॒ सत्प॑तिं॒ नर॒स्त्वां काष्ठा॒स्वर्व॑तः। यद्द्याव॑ इन्द्र ते श॒तꣳ श॒तम्भूमी॑रु॒त स्युः। न त्वा॑ वज्रिन्थ्स॒हस्र॒ꣳ॒ सूर्या॒ अनु॒ न जा॒तम॑ष्ट॒ रोद॑सी। पिबा॒ सोम॑मिन्द्र॒ मन्द॑तु त्वा॒ यं ते॑ सु॒षाव॑ हर्य॒श्वाद्रिः॑।~(४६)

%2.4.14.4
सो॒तुर्बा॒हुभ्या॒ꣳ॒ सुय॑तो॒ नार्वा᳚। रे॒वती᳚र्नः सध॒माद॒ इन्द्रे॑ सन्तु तु॒विवा॑जाः। क्षु॒मन्तो॒ याभि॒र्मदे॑म। उद॑ग्ने॒ शुच॑य॒स्तव॒ वि ज्योति॒षोदु॒ त्यं जा॒तवे॑दसꣳ स॒प्त त्वा॑ ह॒रितो॒ रथे॒ वह॑न्ति देव सूर्य। शो॒चिष्के॑शं विचक्षण। चि॒त्रं दे॒वाना॒मुद॑गा॒दनी॑कं॒ चक्षु॑र्मि॒त्रस्य॒ वरु॑णस्या॒ग्नेः। आ\-ऽप्रा॒ द्यावा॑पृथि॒वी अ॒न्तरि॑क्ष॒ꣳ॒ सूर्य॑ आ॒त्मा जग॑तस्त॒स्थुषः॑~(४७)

%2.4.14.5
च॒। विश्वे॑ दे॒वा ऋ॑ता॒वृध॑ ऋ॒तुभि॑र्\mbox{}हवन॒श्रुतः॑। जु॒षन्तां॒ युज्य॒म्पयः॑। विश्वे॑ देवाः शृणु॒तेमꣳ हव॑म्मे॒ ये अ॒न्तरि॑क्षे॒ य उप॒ द्यवि॒ ष्ठ। ये अ॑ग्निजि॒ह्वा उ॒त वा॒ यज॑त्रा आ॒सद्या॒स्मिन्ब॒र्\mbox{}हिषि॑ मादयध्वम्~(४८)

%2.5.0.0

%2.5.0.0
{\anuvakamend[{तदीशा॑न॒मद्रि॑स्त॒स्थुष॑स्त्रि॒ꣳ॒शच्च॑}]}%॥14॥

{\anuvakamend[{वि॒श्वरू॑प॒स्त्वष्टेन्द्रं॑ वृ॒त्रम्ब्र॑ह्मवा॒दिनः॒ स त्वै नासो॑मयाज्ये॒ष वै दे॑वर॒थो दे॒वा वै नर्चि नाय॒ज्ञो\-ऽग्ने॑ म॒हान्त्रीन्निवी॑त॒मायु॑ष्टे॒ द्वाद॑श}]}%॥12॥ 
\prashnaend[{वि॒श्वरू॑पो॒ नैनꣳ॑ शीतरू॒राव॒द्य वसु॑ पूर्वे॒द्युर्वाजा॒ इत्यग्ने॑ म॒हान्निवी॑तम॒न्या यन्ति॒ चतुः॑सप्ततिः॥74॥ वि॒श्वरू॒पो\-ऽनु॑ ते दायि॥}]
%%% END PRASHNA
