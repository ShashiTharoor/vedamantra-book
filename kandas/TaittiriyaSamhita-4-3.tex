\sect{तृतीयः प्रश्नः}\setcounter{anuvakam}{0}
\dnsub{तैत्तिरीयसंहितायां चतुर्थकाण्डे तृतीयः प्रश्नः}
%4.3.1.0
%4.3.1.1
अ॒पां त्वेम᳚न्थ्सादयाम्य॒पां त्वोद्म᳚न्थ्सादयाम्य॒पां त्वा॒ भस्म᳚न्थ्सादयाम्य॒पां त्वा॒ ज्योति॑षि सादयाम्य॒पां त्वाय॑ने सादयाम्यर्ण॒वे सद॑ने सीद समु॒द्रे सद॑ने सीद सलि॒ले सद॑ने सीदा॒पां ख्षये॑ सीदा॒पा सधि॑षि सीदा॒पां त्वा॒ सद॑ने सादयाम्य॒पां त्वा॑ स॒धस्थे॑ सादयाम्य॒पां त्वा॒ पुरी॑षे सादयाम्य॒पां त्वा॒ योनौ॑ सादयाम्य॒पां त्वा॒ पाथ॑सि सादयामि गाय॒त्री छन्द॑स्त्रि॒ष्टुप्छन्दो॒ जग॑ती॒ छन्दो॑\-ऽनु॒ष्टुप्छन्दः॑ प॒ङ्क्तिश्छन्दः॑॥१॥

%4.3.2.0
{\anuvakamend[{योनौ॒ पञ्च॑दश च॥१॥}]}

%4.3.2.1
अ॒यम्पु॒रो भुव॒स्तस्य॑ प्रा॒णो भौ॑वाय॒नो व॑स॒न्तः प्रा॑णाय॒नो गा॑य॒त्री वा॑स॒न्ती गा॑यत्रि॒यै गा॑य॒त्रं गा॑य॒त्रादु॑पा॒ꣳ॒शु- रु॑पा॒ꣳ॒शोस्त्रि॒वृत्त्रि॒वृतो॑ रथंत॒र र॑थंत॒राद्वसि॑ष्ठ॒ ऋषिः॑ प्र॒जाप॑तिगृहीतया॒ त्वया᳚ प्रा॒णं गृ॑ह्णामि प्र॒जाभ्यो॒\-ऽयं द॑ख्षि॒णा वि॒श्वक॑र्मा॒ तस्य॒ मनो॑ वैश्वकर्म॒णं ग्री॒ष्मो मा॑न॒सस्त्रि॒ष्टुग्ग्रै॒ष्मी त्रि॒ष्टुभ॑ ऐ॒डमै॒डाद॑न्तर्या॒मो᳚\-ऽन्तर्या॒मात् प॑ञ्चद॒शः प॑ञ्चद॒शाद्बृ॒हद्बृ॑ह॒तो भ॒रद्वा॑ज॒ ऋषिः॑ प्र॒जाप॑तिगृहीतया॒ त्वया॒ मनः॑॥२॥

%4.3.2.2
गृ॒ह्णा॒मि॒ प्र॒जाभ्यो॒\-ऽयम्प॒श्चाद्वि॒श्वव्य॑चा॒स्तस्य॒ चख्षु॑र्वैश्वव्यच॒सं व॒र्\mbox{}षाणि॑ चाख्षु॒षाणि॒ जग॑ती वा॒र्\mbox{}षी जग॑त्या॒ ऋख्ष॑म॒मृख्ष॑माच्छु॒क्रः शु॒क्राथ्स॑प्तद॒शः स॑प्तद॒शाद्वै॑रू॒पं वै॑रू॒पाद्वि॒श्वामि॑त्र॒ ऋषिः॑ प्र॒जाप॑तिगृहीतया॒ त्वया॒ चख्षु॑र्गृह्णामि प्र॒जाभ्य॑ इ॒दमु॑त्त॒राथ्सुव॒स्तस्य॒ श्रोत्रꣳ॑ सौ॒व श॒रच्छ्रौ॒त्र्य॑नु॒ष्टुप्छा॑र॒द्य॑नु॒ष्टुभः॑ स्वा॒र स्वा॒रान्म॒न्थी म॒न्थिन॑ एकवि॒ꣳ॒श ए॑कवि॒ꣳ॒शाद्वै॑रा॒जं वै॑रा॒जाज्ज॒मद॑ग्नि॒र्\mbox{}ऋषिः॑ प्र॒जाप॑तिगृहीतया॥३॥

%4.3.2.3
त्वया॒ श्रोत्रं॑ गृह्णामि प्र॒जाभ्य॑ इ॒यमु॒परि॑ म॒तिस्तस्यै॒ वाङ्मा॒ती हे॑म॒न्तो वा᳚च्याय॒नः प॒ङ्क्तिर्\mbox{}है॑म॒न्ती प॒ङ्क्त्यै नि॒धन॑वन्नि॒धन॑वत आग्रय॒ण आ᳚ग्रय॒णात्त्रि॑णवत्रयस्त्रिꣳ॒शौ त्रि॑णवत्रयस्त्रिꣳ॒शाभ्याꣳ॑ शाक्वररैव॒ते शा᳚क्वररैव॒ता\-भ्यां᳚ वि॒श्वक॒र्मर्\mbox{}षिः॑ प्र॒जाप॑तिगृहीतया॒ त्वया॒ वाचं॑ गृह्णामि प्रजाभ्यः॥४॥

%4.3.3.0
{\anuvakamend[{त्वया॒ मनो॑ ज॒मद॑ग्नि॒र्\mbox{}ऋषिः॑ प्र॒जाप॑तिगृहीतया त्रि॒ꣳ॒शच्च॑॥२॥}]}

%4.3.3.1
प्राची॑ दि॒शां व॑स॒न्त ऋ॑तू॒नाम॒ग्निर्दे॒वता॒ ब्रह्म॒ द्रवि॑णं त्रि॒वृथ्स्तोमः॒ स उ॑ पञ्चद॒शव॑र्तनि॒स्त्र्यवि॒र्वयः॑ कृ॒तमया॑नां पुरोवा॒तो वातः॒ सान॑ग॒ ऋषि॑र्दख्षि॒णा दि॒शां ग्री॒ष्म ऋ॑तू॒नामिन्द्रो॑ दे॒वता᳚ ख्ष॒त्रं द्रवि॑णम्पञ्चद॒शः स्तोमः॒ स उ॑ सप्तद॒शव॑र्तनिर्दित्य॒वाड्वय॒स्त्रेताया॑नां दख्षिणाद्वा॒तो वातः॑ सना॒तन॒ ऋषिः॑ प्र॒तीची॑ दि॒शां व॒र्\mbox{}षा ऋ॑तू॒नां विश्वे॑ दे॒वा दे॒वता॒ विट्॥५॥

%4.3.3.2
द्रवि॑ण सप्तद॒शः स्तोमः॒ स उ॑वेकवि॒ꣳ॒शव॑र्तनिस्त्रिव॒थ्सो वयो᳚ द्वाप॒रो\-ऽया॑नाम्पश्चाद्वा॒तो वातो॑\-ऽह॒भून॒ ऋषि॒रुदी॑ची दि॒शा श॒रदृ॑तू॒नाम्मि॒त्रावरु॑णौ दे॒वता॑ पु॒ष्टं द्रवि॑णमेकवि॒ꣳ॒शः स्तोमः॒ स उ॑ त्रिण॒वव॑र्तनिस्तुर्य॒वाड्वय॑ आस्क॒न्दो-\-ऽ या॑नामुत्तराद्वा॒तो वातः॑ प्र॒त्न ऋषि॑रू॒र्ध्वा दि॒शा हे॑मन्तशिशि॒रावृ॑तू॒नाम्बृह॒स्पति॑र्दे॒वता॒ वर्चो॒ द्रवि॑णं त्रिण॒वः स्तोमः॒ स उ॑ त्रयस्त्रि॒ꣳ॒शव॑र्तनिः पष्ठ॒वाद्वयो॑\-ऽभि॒भूरया॑नां विष्वग्वा॒तो वातः॑ सुप॒र्ण ऋषिः॑ पि॒तरः॑ पिताम॒हाः परे\-ऽव॑रे॒ ते नः॑ पान्तु॒ ते नो॑\-ऽवन्त्व॒स्मिन्ब्रह्म॑न्न॒स्मिन्ख्ष॒त्रे᳚\-ऽस्यामा॒शिष्य॒स्याम्पु॑रो॒धाया॑म॒स्मिन्कर्म॑न्न॒स्यां दे॒वहू᳚त्याम्॥६॥

%4.3.4.0
{\anuvakamend[{विट्प॑ष्ठ॒वाड्वयो॒\-ऽष्टाविꣳ॑शतिश्च॥३॥}]}

%4.3.4.1
ध्रु॒वख्षि॑तिर्ध्रु॒वयो॑निर्ध्रु॒वासि॑ ध्रु॒वं योनि॒मा सी॑द सा॒ध्या। उख्य॑स्य के॒तुम्प्र॑थ॒मम्पु॒रस्ता॑द॒श्विना᳚ध्व॒र्यू सा॑दयतामि॒ह त्वा᳚। स्वे दख्षे॒ दख्ष॑पिते॒ह सी॑द देव॒त्रा पृ॑थि॒वी बृ॑ह॒ती ररा॑णा। स्वा॒स॒स्था त॒नुवा॒ सं वि॑शस्व पि॒तेवै॑धि सू॒नव॒ आ सु॒शेवा॒श्विना᳚ध्व॒र्यू सा॑दयतामि॒ह त्वा᳚। कु॒ला॒यिनी॒ वसु॑मती वयो॒धा र॒यिं नो॑ वर्ध बहु॒ल सु॒वीरम्᳚।॥७॥

%4.3.4.2
अपा॑मतिं दुर्म॒तिम्बाध॑माना रा॒यस्पोषे॑ य॒ज्ञप॑तिमा॒भज॑न्ती॒ सुव॑र्धेहि॒ यज॑मानाय॒ पोष॑म॒श्विना᳚ध्व॒र्यू सा॑दयतामि॒ह त्वा᳚। अ॒ग्नेः पुरी॑षमसि देव॒यानी॒ तां त्वा॒ विश्वे॑ अ॒भि गृ॑णन्तु दे॒वाः। स्तोम॑पृष्ठा घृ॒तव॑ती॒ह सी॑द प्र॒जाव॑द॒स्मे द्रवि॒णा य॑जस्वा॒श्विना᳚ध्व॒र्यू सा॑दयतामि॒ह त्वा᳚। दि॒वो मू॒र्धासि॑ पृथि॒व्या नाभि॑र्वि॒ष्टम्भ॑नी दि॒शामधि॑पत्नी॒ भुव॑नानाम्।॥८॥

%4.3.4.3
ऊ॒र्मिर्द्र॒प्सो अ॒पाम॑सि वि॒श्वक॑र्मा त॒ ऋषि॑र॒श्विना᳚ध्व॒र्यू सा॑दयतामि॒ह त्वा᳚। स॒जूर्\mbox{}ऋ॒तुभिः॑ स॒जूर्वि॒धाभिः॑ स॒जूर्वसु॑भिः स॒जू रु॒द्रैः स॒जूरा॑दि॒त्यैः स॒जूर्विश्वै᳚र्दे॒वैः स॒जूर्दे॒वैः स॒जूर्दे॒वैर्व॑योना॒धैर॒ग्नये᳚ त्वा वैश्वान॒राया॒श्विना᳚ध्व॒र्यू सा॑दयतामि॒ह त्वा᳚। प्रा॒णम्मे॑ पाह्यपा॒नम्मे॑ पाहि व्या॒नम्मे॑ पाहि॒ चख्षु॑र्म उ॒र्व्या वि भा॑हि॒ श्रोत्र॑म्मे श्लोकया॒पस्पि॒न्वौष॑धीर्जिन्व द्वि॒पात्पा॑हि॒ चतु॑ष्पादव दि॒वो वृष्टि॒मेर॑य॥९॥

%4.3.5.0
{\anuvakamend[{सु॒वीरं॒ भुव॑नानामु॒र्व्या स॒प्तद॑श च॥४॥}]}

%4.3.5.1
त्र्यवि॒र्वय॑स्त्रि॒ष्टुप्छन्दो॑ दित्य॒वाड्वयो॑ वि॒राट्छन्दः॒ पञ्चा॑वि॒र्वयो॑ गाय॒त्री छन्द॑स्त्रिव॒थ्सो वय॑ उ॒ष्णिहा॒ छन्द॑स्तुर्य॒वाड्वयो॑\-ऽ- नु॒ष्टुप्छन्दः॑ पष्ठ॒वाद्वयो॑ बृह॒ती छन्द॑ उ॒ख्षा वयः॑ स॒तोबृ॑हती॒ छन्द॑ ऋष॒भो वयः॑ क॒कुच्छन्दो॑ धे॒नुर्वयो॒ जग॑ती॒ छन्दो॑\-ऽ- न॒ड्वान् वयः॑ प॒ङ्क्तिश्छन्दो॑ ब॒स्तो वयो॑ विव॒लं छन्दो॑ वृ॒ष्णिर्वयो॑ विशा॒लं छन्दः॒ पुरु॑षो॒ वय॑स्त॒न्द्रं छन्दो᳚ व्या॒घ्रो वयो\-ऽ- ना॑धृष्टं॒ छन्दः॑ सि॒ꣳ॒हो वय॑श्छ॒दिश्छन्दो॑ विष्ट॒म्भो वयो\-ऽधि॑पति॒श्छन्दः॑ ख्ष॒त्रं वयो॒ मयं॑दं॒ छन्दो॑ वि॒श्वक॑र्मा॒ वयः॑ परमे॒ष्ठी छन्दो मू॒र्धा वयः॑ प्र॒जाप॑ति॒श्छन्दः॑॥१०॥

%4.3.6.0
{\anuvakamend[{पुरु॑षो॒ वय॒ष्षड्विꣳ॑शतिश्च॥५॥}]}

%4.3.6.1
इन्द्रा᳚ग्नी॒ अव्य॑थमाना॒मिष्ट॑कां दृहतं यु॒वम्। पृ॒ष्ठेन॒ द्यावा॑पृथि॒वी अ॒न्तरि॑ख्षं च॒ वि बा॑धताम्॥ वि॒श्वक॑र्मा त्वा सादयत्व॒न्तरि॑ख्षस्य पृ॒ष्ठे व्यच॑स्वती॒म्प्रथ॑स्वती॒म्भास्व॑ती सूरि॒मती॒मा या द्याम्भास्या पृ॑थि॒वीमोर्व॑न्तरि॑ख्षम॒न्तरि॑ख्षं यच्छा॒न्तरि॑ख्षं दृहा॒न्तरि॑ख्ष॒म्मा हिꣳ॑सी॒र्विश्व॑स्मै प्रा॒णाया॑पा॒नाय॑ व्या॒नायो॑दा॒नाय॑ प्रति॒ष्ठायै॑ च॒रित्रा॑य वा॒युस्त्वा॒भि पा॑तु म॒ह्या स्व॒स्त्या छ॒र्दिषा᳚॥११॥

%4.3.6.2
शंत॑मेन॒ तया॑ दे॒वत॑याङ्गिर॒स्वद्ध्रु॒वा सी॑द। राज्ञ्य॑सि॒ प्राची॒ दिग्वि॒राड॑सि दख्षि॒णा दिक्स॒म्राड॑सि प्र॒तीची॒ दिक्स्व॒राड॒स्युदी॑ची॒ दिगधि॑पत्न्यसि बृह॒ती दिगायु॑र्मे पाहि प्रा॒णम्मे॑ पाह्यपा॒नम्मे॑ पाहि व्या॒नम्मे॑ पाहि॒ चख्षु॑र्मे पाहि॒ श्रोत्र॑म्मे पाहि॒ मनो॑ मे जिन्व॒ वाच॑म्मे पिन्वा॒त्मान॑म्मे पाहि॒ ज्योति॑र्मे यच्छ॥१२॥

%4.3.7.0
{\anuvakamend[{छ॒र्दिषा॑ पिन्व॒ षट्च॑॥६॥}]}

%4.3.7.1
मा छन्दः॑ प्र॒मा छन्दः॑ प्रति॒मा छन्दो᳚\-ऽस्री॒विश्छन्दः॑ प॒ङ्क्तिश्छन्द॑ उ॒ष्णिहा॒ छन्दो॑ बृह॒ती छन्दो॑\-ऽनु॒ष्टुप्छन्दो॑ वि॒राट्छन्दो॑ गाय॒त्री छन्द॑स्त्रि॒ष्टुप्छन्दो॒ जग॑ती॒ छन्दः॑ पृथि॒वी छन्दो॒\-ऽन्तरि॑ख्षं॒ छन्दो॒ द्यौश्छन्दः॒ समा॒श्छन्दो॒ नख्ष॑त्राणि॒ छन्दो॒ मन॒श्छन्दो॒ वाक्छन्दः॑ कृ॒षिश्छन्दो॒ हिर॑ण्यं॒ छन्दो॒ गौश्छन्दो॒\-ऽजा छन्दो\-ऽश्व॒श्छन्दः॑। अ॒ग्निर्दे॒वता᳚॥१३॥

%4.3.7.2
वातो॑ दे॒वता॒ सूर्यो॑ दे॒वता॑ च॒न्द्रमा॑ दे॒वता॒ वस॑वो दे॒वता॑ रु॒द्रा दे॒वता॑दि॒त्या दे॒वता॒ विश्वे॑ दे॒वा दे॒वता॑ म॒रुतो॑ दे॒वता॒ बृह॒स्पति॑र्दे॒वतेन्द्रो॑ दे॒वता॒ वरु॑णो दे॒वता॑ मू॒र्धासि॒ राड्ध्रु॒वासि॑ ध॒रुणा॑ य॒न्त्र्य॑सि॒ यमि॑त्री॒षे त्वो॒र्जे त्वा॑ कृ॒ष्यै त्वा॒ ख्षेमा॑य त्वा॒ यन्त्री॒ राड्ध्रु॒वासि॒ धर॑णी ध॒र्त्र्य॑सि॒ धरि॒त्र्यायु॑षे त्वा॒ वर्च॑से॒ त्वौज॑से त्वा॒ बला॑य त्वा॥१४॥

%4.3.8.0
{\anuvakamend[{दे॒वता\-ऽ\-ऽयु॑षे त्वा॒ षट्च॑॥७॥}]}

%4.3.8.1
आ॒शुस्त्रि॒वृद्भा॒न्तः प॑ञ्चद॒शो व्यो॑म सप्तद॒शः प्रतू᳚र्तिरष्टाद॒शस्तपो॑ नवद॒शो॑\-ऽभिव॒र्तः स॑वि॒ꣳ॒शो ध॒रुण॑ एकवि॒ꣳ॒शो वर्चो᳚ द्वावि॒ꣳ॒शः स॒म्भर॑णस्त्रयोवि॒ꣳ॒शो योनि॑श्चतुर्वि॒ꣳ॒शो गर्भाः᳚ पञ्चवि॒ꣳ॒श ओज॑स्त्रिण॒वः क्रतु॑रेकत्रि॒ꣳ॒शः प्र॑ति॒ष्ठा त्र॑यस्त्रि॒ꣳ॒शो ब्र॒ध्नस्य॑ वि॒ष्टपं॑ चतुस्त्रि॒ꣳ॒शो नाकः॑ षट्त्रि॒ꣳ॒शो वि॑व॒र्तो᳚\-ऽष्टाचत्वारि॒ꣳ॒शो ध॒र्त्रश्च॑तुष्टो॒मः॥१५॥

%4.3.9.0
{\anuvakamend[{आ॒शुः स॒प्तत्रिꣳ॑शत्॥८॥}]}

%4.3.9.1
अ॒ग्नेर्भा॒गो॑\-ऽसि दी॒ख्षाया॒ आधि॑पत्य॒म्ब्रह्म॑ स्पृ॒तं त्रि॒वृथ्स्तोम॒ इन्द्र॑स्य भा॒गो॑\-ऽसि॒ विष्णो॒राधि॑पत्यं ख्ष॒त्र स्पृ॒तम्प॑ञ्चद॒शः स्तोमो॑ नृ॒चख्ष॑साम्भा॒गो॑\-ऽसि धा॒तुराधि॑पत्यं ज॒नित्रꣵ॑ स्पृ॒त स॑प्तद॒शः स्तोमो॑ मि॒त्रस्य॑ भा॒गो॑\-ऽसि॒ वरु॑ण॒स्याधि॑पत्यं दि॒वो वृ॒ष्टिर्वाताः᳚ स्पृ॒ता ए॑कवि॒ꣳ॒शः स्तोमो\-ऽदि॑त्यै भा॒गो॑\-ऽसि पू॒ष्ण आधि॑पत्य॒मोजः॑ स्पृ॒तं त्रि॑ण॒वः स्तोमो॒ वसू॑नाम्भा॒गो॑\-ऽसि॥१६॥

%4.3.9.2
रु॒द्राणा॒माधि॑पत्यं॒ चतु॑ष्पाथ्स्पृ॒तं च॑तुर्वि॒ꣳ॒शः स्तोम॑ आदि॒त्यानां᳚ भा॒गो॑\-ऽसि म॒रुता॒माधि॑पत्यं॒ गर्भाः᳚ स्पृ॒ताः प॑ञ्चवि॒ꣳ॒शः स्तोमो॑ दे॒वस्य॑ सवि॒तुर्भा॒गो॑\-ऽसि॒ बृह॒स्पते॒राधि॑पत्य स॒मीची॒र्दिशः॑ स्पृ॒ताश्च॑तुष्टो॒मः स्तोमो॒ यावा॑नाम्भा॒गो᳚\-ऽस्यया॑वाना॒माधि॑पत्यम्प्र॒जाः स्पृ॒ताश्च॑तुश्चत्वारि॒ꣳ॒शः स्तोम॑ ऋभू॒णाम्भा॒गो॑\-ऽसि॒ विश्वे॑षां दे॒वाना॒माधि॑पत्यम्भू॒तं निशा᳚न्त स्पृ॒तं त्र॑यस्त्रि॒ꣳ॒शः स्तोमः॑॥१७॥

%4.3.10.0
{\anuvakamend[{वसू॑नां भा॒गो॑\-ऽसि॒ षट्च॑त्वारिशच्च॥९॥}]}

%4.3.10.1
एक॑यास्तुवत प्र॒जा अ॑धीयन्त प्र॒जाप॑ति॒रधि॑पतिरासीत्ति॒सृभि॑रस्तुवत॒ ब्रह्मा॑सृज्यत॒ ब्रह्म॑ण॒स्पति॒रधि॑पतिरासीत् प॒ञ्चभि॑रस्तुवत भू॒तान्य॑सृज्यन्त भू॒ताना॒म्पति॒रधि॑पतिरासीत्स॒प्तभि॑रस्तुवत सप्त॒र्\mbox{}षयो॑\-ऽसृज्यन्त धा॒ताधि॑पतिरा- सीन्न॒वभि॑रस्तुवत पि॒तरो॑\-ऽसृज्य॒न्तादि॑ति॒रधि॑पत्न्यासीदेकाद॒शभि॑रस्तुवत॒र्तवो॑\-ऽसृज्यन्तार्त॒वो\-ऽधि॑पतिरासीत् त्रयोद॒शभि॑रस्तुवत॒ मासा॑ असृज्यन्त संवथ्स॒रो\-ऽधि॑पतिः॥१८॥

%4.3.10.2
आ॒सी॒त्प॒ञ्च॒द॒शभि॑रस्तुवत ख्ष॒त्रम॑सृज्य॒तेन्द्रो\-ऽधि॑पतिरासीत्सप्तद॒शभि॑रस्तुवत प॒शवो॑\-ऽसृज्यन्त॒ बृह॒स्पति॒रधि॑पतिरासी- न्नवद॒शभि॑रस्तुवत शूद्रा॒र्याव॑सृज्येतामहोरा॒त्रे अधि॑पत्नी आस्ता॒मेक॑विशत्यास्तुव॒तैक॑शफाः प॒शवो॑\-ऽसृज्यन्त॒ वरु॒णो\-ऽधि॑पतिरासी॒त्त्रयो॑विशत्यास्तुवत ख्षु॒द्राः प॒शवो॑\-ऽसृज्यन्त पू॒षाधि॑पतिरासी॒त्पञ्च॑विशत्यास्तुवतार॒ण्याः प॒शवो॑\-ऽसृज्यन्त वा॒युरधि॑पतिरासीत्स॒प्तविꣳ॑शत्यास्तुवत॒ द्यावा॑पृथि॒वी वि॥१९॥

%4.3.10.3
ऐ॒तां॒ वस॑वो रु॒द्रा आ॑दि॒त्या अनु॒ व्या॑य॒न्तेषा॒माधि॑पत्यमासी॒न्नव॑विशत्यास्तुवत॒ वन॒स्पत॑यो\-ऽसृज्यन्त॒ सोमो\-ऽ- धि॑पतिरासी॒देक॑त्रिशतास्तुवत प्र॒जा अ॑सृज्यन्त॒ यावा॑नां॒ चाया॑वानां॒ चाधि॑पत्यमासी॒त्त्रय॑स्त्रिशतास्तुवत भू॒तान्य॑शाम्यन्प्र॒जाप॑तिः परमे॒ष्ठ्यधि॑पतिरासीत्॥२०॥

%4.3.11.0
{\anuvakamend[{सं॒ व॒थ्स॒रो\-ऽधि॑पति॒र्वि पञ्च॑त्रिशच्च॥10॥}]}

%4.3.11.1
इ॒यमे॒व सा या प्र॑थ॒मा व्यौच्छ॑द॒न्तर॒स्यां च॑रति॒ प्रवि॑ष्टा। व॒धूर्ज॑जान नव॒गज्जनि॑त्री॒ त्रय॑ एनाम्महि॒मानः॑ सचन्ते॥ छन्द॑स्वती उ॒षसा॒ पेपि॑शाने समा॒नं योनि॒मनु॑ सं॒चर॑न्ती। सूर्य॑पत्नी॒ वि च॑रतः प्रजान॒ती के॒तुं कृ॑ण्वा॒ने अ॒जरे॒ भूरि॑रेतसा॥ ऋ॒तस्य॒ पन्था॒मनु॑ ति॒स्र आगु॒स्त्रयो॑ घ॒र्मासो॒ अनु॒ ज्योति॒षागुः॑। प्र॒जामेका॒ रख्ष॒त्यूर्ज॒मेका᳚॥२१॥

%4.3.11.2
व्र॒तमेका॑ रख्षति देवयू॒नाम्॥ च॒तु॒ष्टो॒मो अ॑भव॒द्या तु॒रीया॑ य॒ज्ञस्य॑ प॒ख्षावृ॑षयो॒ भव॑न्ती। गा॒य॒त्रीं त्रि॒ष्टुभं॒ जग॑तीमनु॒ष्टुभ॑म्बृ॒हद॒र्कं यु॑ञ्जा॒नाः सुव॒राभ॑रन्नि॒दम्॥ प॒ञ्चभि॑र्धा॒ता वि द॑धावि॒दं यत्तासा॒ꣳ॒ स्वसॄ॑रजनय॒त्पञ्च॑पञ्च। तासा॑मु यन्ति प्रय॒वेण॒ पञ्च॒ नाना॑ रू॒पाणि॒ क्रत॑वो॒ वसा॑नाः॥ त्रि॒ꣳ॒शथ्स्वसा॑र॒ उप॑ यन्ति निष्कृ॒त स॑मा॒नं के॒तुम्प्र॑तिमु॒ञ्चमा॑नाः।॥२२॥

%4.3.11.3
ऋ॒तूस्त॑न्वते क॒वयः॑ प्रजान॒तीर्मध्ये॑छन्दसः॒ परि॑ यन्ति॒ भास्व॑तीः। ज्योति॑ष्मती॒ प्रति॑ मुञ्चते॒ नभो॒ रात्री॑ दे॒वी सूर्य॑स्य व्र॒तानि॑। वि प॑श्यन्ति प॒शवो॒ जाय॑माना॒ नाना॑रूपा मा॒तुर॒स्या उ॒पस्थे᳚। ए॒का॒ष्ट॒का तप॑सा॒ तप्य॑माना ज॒जान॒ गर्भ॑म्महि॒मान॒मिन्द्रम्᳚। तेन॒ दस्यू॒न्व्य॑सहन्त दे॒वा ह॒न्तासु॑राणामभव॒च्छची॑भिः। अना॑नुजामनु॒जाम्माम॑कर्त स॒त्यं वद॒न्त्यन्वि॑च्छ ए॒तत्। भू॒यासम्᳚॥२३॥

%4.3.11.4
अ॒स्य॒ सु॒म॒तौ यथा॑ यू॒यम॒न्या वो॑ अ॒न्यामति॒ मा प्र यु॑क्त। अभू॒न्मम॑ सुम॒तौ वि॒श्ववे॑दा॒ आष्ट॑ प्रति॒ष्ठामवि॑द॒द्धि गा॒धम्। भू॒यास॑मस्य सुम॒तौ यथा॑ यू॒यम॒न्या वो॑ अ॒न्यामति॒ मा प्र यु॑क्त। पञ्च॒ व्यु॑ष्टी॒रनु॒ पञ्च॒ दोहा॒ गाम्पञ्च॑नाम्नीमृ॒तवो\-ऽनु॒ पञ्च॑। पञ्च॒ दिशः॑ पञ्चद॒शेन॒ कॢ॒प्ताः स॑मा॒नमू᳚र्ध्नीर॒भि लो॒कमेकम्᳚॥२४॥

%4.3.11.5
ऋ॒तस्य॒ गर्भः॑ प्रथ॒मा व्यू॒षुष्य॒पामेका॑ महि॒मान॑म्बिभर्ति। सूर्य॒स्यैका॒ चर॑ति निष्कृ॒तेषु॑ घ॒र्मस्यैका॑ सवि॒तैकां॒ नि य॑च्छति। या प्र॑थ॒मा व्यौच्छ॒थ्सा धे॒नुर॑भवद्य॒मे। सा नः॒ पय॑स्वती धु॒क्ष्वोत्त॑रामुत्तरा॒ꣳ॒ समाम्᳚। शु॒क्रर्\mbox{}ष॑भा॒ नभ॑सा॒ ज्योति॒षागा᳚द्वि॒श्वरू॑पा शब॒लीर॒ग्निके॑तुः। स॒मा॒नमर्थꣵ॑ स्वप॒स्यमा॑ना॒ बिभ्र॑ती ज॒राम॑जर उष॒ आगाः᳚। ऋ॒तू॒नाम्पत्नी᳚ प्रथ॒मेयमागा॒दह्नां᳚ ने॒त्री ज॑नि॒त्री प्र॒जानाम्᳚। एका॑ स॒ती ब॑हु॒धोषो॒ व्यु॑च्छ॒स्यजी᳚र्णा॒ त्वं ज॑रयसि॒ सर्व॑म॒न्यत्॥२५॥

%4.3.12.0
{\anuvakamend[{ऊर्ज॒मेका᳚ प्रतिमु॒ञ्चमा॑ना भू॒यास॒मेकं॒ पत्न्येका॒न्नविꣳ॑श॒तिश्च॑॥11॥}]}

%4.3.12.1
अग्ने॑ जा॒तान्प्र णु॑दा नः स॒पत्ना॒न्प्रत्यजा॑ताञ्जातवेदो नुदस्व। अ॒स्मे दी॑दिहि सु॒मना॒ अहे॑ड॒न्तव॑ स्या॒ꣳ॒ शर्म॑न्त्रि॒वरू॑थ उ॒द्भित्। सह॑सा जा॒तान्प्र णु॑दा नः स॒पत्ना॒न्प्रत्यजा॑ताञ्जातवेदो नुदस्व। अधि॑ नो ब्रूहि सुमन॒स्यमा॑नो व॒य स्या॑म॒ प्र णु॑दा नः स॒पत्नान्॑। च॒तु॒श्च॒त्वा॒रि॒ꣳ॒शः स्तोमो॒ वर्चो॒ द्रवि॑ण षोड॒शः स्तोम॒ ओजो॒ द्रवि॑णम्पृथि॒व्याः पुरी॑षमसि॥२६॥

%4.3.12.2
अप्सो॒ नाम॑। एव॒श्छन्दो॒ वरि॑व॒श्छन्दः॑ श॒म्भूश्छन्दः॑ परि॒भूश्छन्द॑ आ॒च्छच्छन्दो॒ मन॒श्छन्दो॒ व्यच॒श्छन्दः॒ सिन्धु॒श्छन्दः॑ समु॒द्रं छन्दः॑ सलि॒लं छन्दः॑ सं॒यच्छन्दो॑ वि॒यच्छन्दो॑ बृ॒हच्छन्दो॑ रथंत॒रं छन्दो॑ निका॒यश्छन्दो॑ विव॒धश्छन्दो॒ गिर॒श्छन्दो॒ भ्रज॒श्छन्दः॑ स॒ष्टुप्छन्दो॑\-ऽनु॒ष्टुप्छन्दः॑ क॒कुच्छन्द॑स्त्रिक॒कुच्छन्दः॑ का॒व्यं छन्दो᳚\-ऽङ्कु॒पं छन्दः॑॥२७॥

%4.3.12.3
प॒दप॑ङ्क्ति॒श्छन्दो॒\-ऽख्षर॑पङ्क्ति॒श्छन्दो॑ विष्टा॒रप॑ङ्क्ति॒श्छन्दः॑ ख्षु॒रो भृज्वा॒ञ्छन्दः॑ प्र॒च्छच्छन्दः॑ प॒ख्षश्छन्द॒ एव॒श्छन्दो॒ वरि॑व॒श्छन्दो॒ वय॒श्छन्दो॑ वय॒स्कृच्छन्दो॑ विशा॒लं छन्दो॒ विष्प॑र्धा॒श्छन्द॑श्छ॒दिश्छन्दो॑ दूरोह॒णं छन्द॑स्त॒न्द्रं छन्दो᳚\-ऽङ्का॒ङ्कं छन्दः॑॥२८॥

%4.3.13.0
{\anuvakamend[{अ॒स्य॒ङ्कु॒पञ्छन्द॒स्त्रय॑स्त्रिशच्च॥12॥}]}

%4.3.13.1
अ॒ग्निर्वृ॒त्राणि॑ जङ्घनद्द्रविण॒स्युर्वि॑प॒न्यया᳚। समि॑द्धः शु॒क्र आहु॑तः॥ त्व सो॑मासि॒ सत्प॑ति॒स्त्व राजो॒त वृ॑त्र॒हा। त्वम्भ॒द्रो अ॑सि॒ क्रतुः॑॥ भ॒द्रा ते॑ अग्ने स्वनीक सं॒दृग्घो॒रस्य॑ स॒तो विषु॑णस्य॒ चारुः॑। न यत्ते॑ शो॒चिस्तम॑सा॒ वर॑न्त॒ न ध्व॒स्मान॑स्त॒नुवि॒ रेप॒ आ धुः॑॥ भ॒द्रं ते॑ अग्ने सहसि॒न्ननी॑कमुपा॒क आ रो॑चते॒ सूर्य॑स्य।॥२९॥

%4.3.13.2
रुश॑द्दृ॒शे द॑दृशे नक्त॒या चि॒दरू᳚ख्षितं दृ॒श आ रू॒पे अन्नम्᳚। सैनानी॑केन सुवि॒दत्रो॑ अ॒स्मे यष्टा॑ दे॒वा आय॑जिष्ठः स्व॒स्ति। अद॑ब्धो गो॒पा उ॒त नः॑ पर॒स्पा अग्ने᳚ द्यु॒मदु॒त रे॒वद्दि॑दीहि। स्व॒स्ति नो॑ दि॒वो अ॑ग्ने पृथि॒व्या वि॒श्वायु॑र्धेहि य॒जथा॑य देव। यथ्सी॒महि॑ दिविजात॒ प्रश॑स्तं॒ तद॒स्मासु॒ द्रवि॑णं धेहि चि॒त्रम्। यथा॑ होत॒र्मनु॑षः॥३०॥

%4.3.13.3
दे॒वता॑ता य॒ज्ञेभिः॑ सूनो सहसो॒ यजा॑सि। ए॒वानो॑ अ॒द्य स॑म॒ना स॑मा॒नानु॒शन्न॑ग्न उश॒तो य॑ख्षि दे॒वान्॥ अ॒ग्निमी॑डे पु॒रोहि॑तं य॒ज्ञस्य॑ दे॒वमृ॒त्विजम्᳚। होता॑र रत्न॒धात॑मम्॥ वृषा॑ सोम द्यु॒मा अ॑सि॒ वृषा॑ देव॒ वृष॑व्रतः। वृषा॒ धर्मा॑णि दधिषे॥ सान्त॑पना इ॒द ह॒विर्मरु॑त॒स्तज्जु॑जुष्टन। यु॒ष्माको॒ती रि॑शादसः॥ यो नो॒ मर्तो॑ वसवो दुर्\mbox{}हृणा॒युस्ति॒रः स॒त्यानि॑ मरुतः॥३१॥

%4.3.13.4
जिघाꣳ॑सात्। द्रु॒हः पाश॒म्प्रति॒ स मु॑चीष्ट॒ तपि॑ष्ठेन॒ तप॑सा हन्तना॒ तम्। सं॒व॒थ्स॒रीणा॑ म॒रुतः॑ स्व॒र्का उ॑रु॒ख्षयाः॒ सग॑णा॒ मानु॑षेषु। ते᳚\-ऽस्मत्पाशा॒न्प्र मु॑ञ्च॒न्त्वह॑सः सांतप॒ना म॑दि॒रा मा॑दयि॒ष्णवः॑। पि॒प्री॒हि दे॒वा उ॑श॒तो य॑विष्ठ वि॒द्वा ऋ॒तूर्\mbox{}ऋ॑तुपते यजे॒ह। ये दैव्या॑ ऋ॒त्विज॒स्तेभि॑रग्ने॒ त्व होतॄ॑णाम॒स्याय॑जिष्ठः। अग्ने॒ यद॒द्य वि॒शो अ॑ध्वरस्य होतः॒ पाव॑क॥३२॥

%4.3.13.5
शो॒चे॒ वेष्ट्व हि यज्वा᳚। ऋ॒ता य॑जासि महि॒ना वि यद्भूर्\mbox{}ह॒व्या व॑ह यविष्ठ॒ या ते॑ अ॒द्य। अ॒ग्निना॑ र॒यिम॑श्नव॒त्पोष॑मे॒व दि॒वेदि॑वे। य॒शसं॑ वी॒रव॑त्तमम्॥ ग॒य॒स्फानो॑ अमीव॒हा व॑सु॒वित्पु॑ष्टि॒वर्ध॑नः। सु॒मि॒त्रः सो॑म नो भव। गृह॑मेधास॒ आ ग॑त॒ मरु॑तो॒ माप॑ भूतन। प्र॒मु॒ञ्चन्तो॑ नो॒ अह॑सः। पू॒र्वीभि॒र्\mbox{}हि द॑दाशि॒म श॒रद्भि॑र्मरुतो व॒यम्। महो॑भिः॥३३॥

%4.3.13.6
च॒र्\mbox{}ष॒णी॒नाम्। प्र बु॒ध्निया॑ ईरते वो॒ महाꣳ॑सि॒ प्र णामा॑नि प्रयज्यवस्तिरध्वम्। स॒ह॒स्रियं॒ दम्य॑म्भा॒गमे॒तं गृ॑हमे॒धीय॑म्मरुतो जुषध्वम्। उप॒ यमेति॑ युव॒तिः सु॒दख्षं॑ दो॒षा वस्तोर्\mbox{}॑ह॒विष्म॑ती घृ॒ताची᳚। उप॒ स्वैन॑म॒रम॑तिर्वसू॒युः। इ॒मो अ॑ग्ने वी॒तत॑मानि ह॒व्याज॑स्रो वख्षि दे॒वता॑ति॒मच्छ॑। प्रति॑ न ई सुर॒भीणि॑ वियन्तु। क्री॒डं वः॒ शर्धो॒ मारु॑तमन॒र्वाणꣳ॑ रथे॒शुभम्᳚।॥३४॥

%4.3.13.7
कण्वा॑ अ॒भि प्र गा॑यत। अत्या॑सो॒ न ये म॒रुतः॒ स्वञ्चो॑ यख्ष॒दृशो॒ न शु॒भय॑न्त॒ मर्याः᳚। ते ह॑र्म्ये॒ष्ठाः शिश॑वो॒ न शु॒भ्रा व॒थ्सासो॒ न प्र॑क्री॒डिनः॑ पयो॒धाः। प्रैषा॒मज्मे॑षु विथु॒रेव॑ रेजते॒ भूमि॒र्यामे॑षु॒ यद्ध॑ यु॒ञ्जते॑ शु॒भे। ते क्री॒डयो॒ धुन॑यो॒ भ्राज॑दृष्टयः स्व॒यम्म॑हि॒त्वम्प॑नयन्त॒ धूत॑यः। उ॒प॒ह्व॒रेषु॒ यदचि॑ध्वं य॒यिं वय॑ इव मरुतः॒ केन॑॥३५॥

%4.3.13.8
चि॒त्प॒था। श्चोत॑न्ति॒ कोशा॒ उप॑ वो॒ रथे॒ष्वा घृ॒तमु॑ख्षता॒ मधु॑वर्ण॒मर्च॑ते। अ॒ग्निम॑ग्नि॒ꣳ॒ हवी॑मभिः॒ सदा॑ हवन्त वि॒श्पतिम्᳚। ह॒व्य॒वाह॑म्पुरुप्रि॒यम्। त हि शश्व॑न्त॒ ईड॑ते स्रु॒चा दे॒वं घृ॑त॒श्चुता᳚। अ॒ग्नि ह॒व्याय॒ वोढ॑वे। इन्द्रा᳚ग्नी रोच॒ना दि॒वः श्नथ॑द्वृ॒त्रमिन्द्रं॑ वो वि॒श्वत॒स्परीन्द्रं॒ नरो॒ विश्व॑कर्मन् ह॒विषा॑ वावृधा॒नो विश्व॑कर्मन् ह॒विषा॒ वर्ध॑नेन॥३६॥

%4.4.0.0
{\anuvakamend[{सूर्य॑स्य॒ मनु॑षो मरुतः॒ पाव॑क॒ महो॑भी रथे॒शुभं॒ केन॒ षट्च॑त्वारिशच्च॥13॥}]}

%4.4.0.0

{\anuvakamend[{र॒श्मिर॑सि॒ राज्ञ्य॑स्य॒यं पु॒रो हरि॑केशो॒\-ऽग्निर्मू॒र्धेन्द्रा॒ग्निभ्यां॒ बृह॒स्पति॑र्भूय॒स्कृद॑स्य॒ग्निना॑ विश्वा॒षाट्प्र॒जाप॑ति॒र्मन॑सा॒ कृत्ति॑का॒ मधु॑श्च स॒मिद्दि॒शान्द्वाद॑श॥12॥ र॒श्मिर॑सि॒ प्रति॑ धे॒नुम॑सि स्तनयित्नु॒सनि॑रस्यादि॒त्यानाꣳ॑ स॒प्तत्रिꣳ॑शत्॥37॥ र॒श्मिर॑सि॒ को अ॒द्य यु॑ङ्क्ते॥}]}
%%% END PRASHNA
