\sect{प्रथमः प्रश्नः}\setcounter{anuvakam}{0}
\dnsub{तैत्तिरीयसंहितायां चतुर्थकाण्डे प्रथमः प्रश्नः}
%4.1.1.0
%4.1.1.1
यु॒ञ्जा॒नः प्र॑थ॒मम्मन॑स्त॒त्वाय॑ सवि॒ता धियः॑। अ॒ग्निं ज्योति॑र्नि॒चाय्य॑ पृथि॒व्या अध्याभ॑रत्। यु॒क्त्वाय॒ मन॑सा दे॒वान्त्सुव॑र्य॒तो धि॒या दिवम्᳚। बृ॒हज्ज्योतिः॑ करिष्य॒तः स॑वि॒ता प्र सु॑वति॒ तान्। यु॒क्तेन॒ मन॑सा व॒यं दे॒वस्य॑ सवि॒तुः स॒वे। सु॒व॒र्गेया॑य॒ शक्त्यै᳚। यु॒ञ्जते॒ मन॑ उ॒त यु॑ञ्जते॒ धियो॒ विप्रा॒ विप्र॑स्य बृह॒तो वि॑प॒श्चितः॑। वि होत्रा॑ दधे वयुना॒विदेक॒ इत्॥१॥

%4.1.1.2
म॒ही दे॒वस्य॑ सवि॒तुः परि॑ष्टुतिः। यु॒जे वां॒ ब्रह्म॑ पू॒र्व्यं नमो॑भि॒र्वि श्लोका॑ यन्ति प॒थ्ये॑व॒ सूराः᳚। शृ॒ण्वन्ति॒ विश्वे॑ अ॒मृत॑स्य पु॒त्रा आ ये धामा॑नि दि॒व्यानि॑ त॒स्थुः। यस्य॑ प्र॒याण॒मन्व॒न्य इद्य॒युर्दे॒वा दे॒वस्य॑ महि॒मान॒मर्च॑तः। यः पार्थि॑वानि विम॒मे स एत॑शो॒ रजाꣳ॑सि दे॒वः स॑वि॒ता म॑हित्व॒ना। देव॑ सवितः॒ प्र सु॑व य॒ज्ञम्प्र सु॑व॥२॥

%4.1.1.3
य॒ज्ञप॑ति॒म्भगा॑य दि॒व्यो ग॑न्ध॒र्वः। के॒त॒पूः केतं॑ नः पुनातु वा॒चस्पति॒र्वाच॑म॒द्य स्व॑दाति नः। इ॒मं नो॑ देव सवितर्य॒ज्ञं प्र सु॑व देवा॒युवꣳ॑ सखि॒विदꣳ॑ सत्रा॒जितं॑ धन॒जितꣳ॑ सुव॒र्जितम्᳚। ऋ॒चा स्तोम॒ꣳ॒ सम॑र्धय गाय॒त्रेण॑ रथंत॒रम्। बृ॒हद्गा॑य॒त्रव॑र्तनि। दे॒वस्य॑ त्वा सवि॒तुः प्र॑स॒वे᳚\-ऽश्विनो᳚र्बा॒हु\-भ्यां᳚ पू॒ष्णो हस्ता᳚भ्याम्गाय॒त्रेण॒ छन्द॒सा\-ऽ\-ऽद॑दे \-ऽङ्गिर॒स्वदभ्रि॑रसि॒ नारिः॑॥३॥

%4.1.1.4
अ॒सि॒ पृ॒थि॒व्याः स॒धस्था॑द॒ग्निम्पु॑री॒ष्य॑मङ्गिर॒स्वदा भ॑र॒ त्रैष्टु॑भेन त्वा॒ छन्द॒सा\-ऽ\-ऽद॑दे\-ऽङ्गिर॒स्वद्बभ्रि॑रसि॒ नारि॑रसि॒ त्वया॑ व॒यꣳ स॒धस्थ॒ आग्निꣳ श॑केम॒ खनि॑तुम्पुरी॒ष्यं॑ जाग॑तेन त्वा॒ छन्द॒सा\-ऽ\-ऽद॑दे\-ऽङ्गिर॒स्वद्धस्त॑ आ॒धाय॑ सवि॒ता बिभ्र॒दभ्रिꣳ॑ हिर॒ण्ययीम्᳚। तया॒ ज्योति॒रज॑स्र॒मिद॒ग्निं खा॒त्वी न॒ आ भ॒रानु॑ष्टुभेन त्वा॒ छन्द॒सा\-ऽ\-ऽद॑दे\-ऽङ्गिर॒स्वत्॥

%4.1.2.0
{\anuvakamend[{इद्य॒ज्ञं प्र सु॑व॒ नारि॒रानु॑ष्टुभेन त्वा॒ छन्द॑सा॒ त्रीणि॑ च॥१॥}]}

%4.1.2.1
इ॒माम॑गृभ्णन्रश॒नामृ॒तस्य॒ पूर्व॒ आयु॑षि वि॒दथे॑षु क॒व्या। तया॑ दे॒वाः सु॒तमा ब॑भूवुर्\mbox{}ऋ॒तस्य॒ साम᳚न्त्स॒रमा॒रप॑न्ती। प्रतू᳚र्तं वाजि॒न्ना द्र॑व॒ वरि॑ष्ठा॒मनु॑ सं॒वतम्᳚। दि॒वि ते॒ जन्म॑ पर॒मम॒न्तरि॑क्षे॒ नाभिः॑ पृथि॒व्यामधि॒ योनिः॑। यु॒ञ्जाथा॒ꣳ॒ रास॑भं यु॒वम॒स्मिन् यामे॑ वृषण्वसू। अ॒ग्निम्भर॑न्तमस्म॒युम्। योगे॑योगे त॒वस्त॑रं॒ वाजे॑वाजे हवामहे। सखा॑य॒ इन्द्र॑मू॒तये᳚। प्र॒तूर्वन्न्॑॥५॥

%4.1.2.2
एह्य॑व॒क्राम॒न्नश॑स्ती रु॒द्रस्य॒ गाण॑पत्यान्मयो॒भूरेहि॑। उ॒र्व॑न्तरि॑क्ष॒मन्वि॑हि स्व॒स्तिग॑व्यूति॒रभ॑यानि कृ॒ण्वन्न्। पू॒ष्णा स॒युजा॑ स॒ह। पृ॒थि॒व्याः स॒धस्था॑द॒ग्निम्पु॑री॒ष्य॑मङ्गिर॒स्वदच्छे᳚ह्य॒ग्निम्पु॑री॒ष्य॑मङ्गिर॒स्वदच्छे॑मो॒\-ऽग्निम्पु॑री॒ष्य॑मङ्गिर॒- स्वद्भ॑रिष्यामो॒\-ऽग्निम्पु॑री॒ष्य॑मङ्गिर॒स्वद्भ॑रामः। अन्व॒ग्निरु॒षसा॒मग्र॑मख्य॒दन्वहा॑नि प्रथ॒मो जा॒तवे॑दाः। अनु॒ सूर्य॑स्य॥६॥

%4.1.2.3
पु॒रु॒त्रा च॑ र॒श्मीननु॒ द्यावा॑पृथि॒वी आ त॑तान। आ॒गत्य॑ वा॒ज्यध्व॑नः॒ सर्वा॒ मृधो॒ वि धू॑नुते। अ॒ग्निꣳ स॒धस्थे॑ मह॒ति चक्षु॑षा॒ नि चि॑कीषते। आ॒क्रम्य॑ वाजिन्पृथि॒वीम॒ग्निमि॑च्छ रु॒चा त्वम्। भूम्या॑ वृ॒त्वाय॑ नो ब्रूहि॒ यतः॒ खना॑म॒ तं व॒यम्। द्यौस्ते॑ पृ॒ष्ठम्पृ॑थि॒वी स॒धस्थ॑मा॒त्मान्तरि॑क्षꣳ समु॒द्रस्ते॒ योनिः॑। वि॒ख्याय॒ चक्षु॑षा॒ त्वम॒भि ति॑ष्ठ॥७॥

%4.1.2.4
पृ॒त॒न्य॒तः। उत्क्रा॑म मह॒ते सौभ॑गाया॒स्मादा॒स्थाना᳚द्द्रविणो॒दा वा॑जिन्न्। व॒यꣴ स्या॑म सुम॒तौ पृ॑थि॒व्या अ॒ग्निं ख॑नि॒ष्यन्त॑ उ॒पस्थे॑ अस्याः। उद॑क्रमीद्द्रविणो॒दा वा॒ज्यर्वाकः॒ स लो॒कꣳ सुकृ॑तम्पृथि॒व्याः। ततः॑ खनेम सु॒प्रती॑कम॒ग्निꣳ सुवो॒ रुहा॑णा॒ अधि॒ नाक॑ उत्त॒मे। अ॒पो दे॒वीरुप॑ सृज॒ मधु॑मतीरय॒क्ष्माय॑ प्र॒जाभ्यः॑। तासा॒ꣳ॒ स्थाना॒दुज्जि॑हता॒मोष॑धयः सुपिप्प॒लाः। जिघ॑र्मि॥८॥

%4.1.2.5
अ॒ग्निम्मन॑सा घृ॒तेन॑ प्रति॒क्ष्यन्त॒म्भुव॑नानि॒ विश्वा᳚। पृ॒थुं ति॑र॒श्चा वय॑सा बृ॒हन्तं॒ व्यचि॑ष्ठ॒मन्नꣳ॑ रभ॒सं विदा॑नम्। आ त्वा॑ जिघर्मि॒ वच॑सा घृ॒तेना॑र॒क्षसा॒ मन॑सा॒ तज्जु॑षस्व। मर्य॑श्रीः स्पृह॒यद्व॑र्णो अ॒ग्निर्नाभि॒मृशे॑ त॒नुवा॒ जर्\mbox{}हृ॑षाणः। परि॒ वाज॑पतिः क॒विर॒ग्निर्\mbox{}ह॒व्या न्य॑क्रमीत्। दध॒द्रत्ना॑नि दा॒शुषे᳚। परि॑ त्वाग्ने॒ पुरं॑ व॒यं विप्रꣳ॑ सहस्य धीमहि। धृ॒षद्व॑र्णं दि॒वेदि॑वे भे॒त्तार॑म्भङ्गु॒॒राव॑तः। त्वम॑ग्ने॒ द्युभि॒स्त्वमा॑शुशु॒क्षणि॒स्त्वम॒द्भ्यस्त्वमश्म॑न॒स्परि॑। त्वं वने᳚भ्य॒स्त्वमोष॑धीभ्य॒स्त्वं नृ॒णां नृ॑पते जायसे॒ शुचिः॑॥९॥

%4.1.3.0
{\anuvakamend[{प्र॒तूर्व॒न्थ्सूर्य॑स्य तिष्ठ॒ जिघ॑र्मि भे॒त्तारं॑ विꣳश॒तिश्च॑॥२॥}]}

%4.1.3.1
दे॒वस्य॑ त्वा सवि॒तुः प्र॑स॒वे᳚\-ऽश्विनो᳚र्बा॒हु\-भ्यां᳚ पू॒ष्णो हस्ता᳚भ्याम्पृथि॒व्याः स॒धस्थे॒\-ऽग्निम्पु॑री॒ष्य॑मङ्गिर॒स्व- त्ख॑नामि। ज्योति॑ष्मन्तं त्वाग्ने सु॒प्रती॑क॒मज॑स्रेण भा॒नुना॒ दीद्या॑नम्। शि॒वं प्र॒जाभ्यो\-ऽहिꣳ॑सन्त- म्पृथि॒व्याः स॒धस्थे॒\-ऽग्निं पु॑री॒ष्य॑मङ्गिर॒स्वत्ख॑नामि। अ॒पाम्पृ॒ष्ठम॑सि स॒प्रथा॑ उ॒र्व॑ग्निम्भ॑रि॒ष्यदप॑रावपिष्ठम्। वर्ध॑मानम्म॒ह आ च॒ पुष्क॑रं दि॒वो मात्र॑या वरि॒णा प्र॑थस्व। शर्म॑ च स्थः॥१०॥

%4.1.3.2
वर्म॑ च स्थो॒ अच्छि॑द्रे बहु॒ले उ॒भे। व्यच॑स्वती॒ सं व॑साथाम्भ॒र्तम॒ग्निम्पु॑री॒ष्यम्᳚। सं व॑साथाꣳ सुव॒र्विदा॑ स॒मीची॒ उर॑सा॒ त्मना᳚। अ॒ग्निम॒न्तर्भ॑रि॒ष्यन्ती॒ ज्योति॑ष्मन्त॒मज॑स्र॒मित्। पु॒री॒ष्यो॑\-ऽसि वि॒श्वभ॑राः। अथ॑र्वा त्वा प्रथ॒मो निर॑मन्थदग्ने। त्वाम॑ग्ने॒ पुष्क॑रा॒दध्यथ॑र्वा॒ निर॑मन्थत। मू॒र्ध्नो विश्व॑स्य वा॒घतः॑। तमु॑ त्वा द॒ध्यङ्ङृषिः॑ पु॒त्र ई॑धे॥११॥

%4.1.3.3
अथ॑र्वणः। वृ॒त्र॒हण॑म्पुरन्द॒रम्। तमु॑ त्वा पा॒थ्यो वृषा॒ समी॑धे दस्यु॒हन्त॑मम्। ध॒नं॒ज॒यꣳ रणे॑रणे। सीद॑ होतः॒ स्व उ॑ लो॒के चि॑कि॒त्वान्त्सा॒दया॑ य॒ज्ञꣳ सु॑कृ॒तस्य॒ योनौ᳚। दे॒वा॒वीर्दे॒वान् ह॒विषा॑ यजा॒स्यग्ने॑ बृ॒हद्यज॑माने॒ वयो॑ धाः। नि होता॑ होतृ॒षद॑ने॒ विदा॑नस्त्वे॒षो दी॑दि॒वाꣳ अ॑सदत्सु॒दक्षः॑। अद॑ब्धव्रतप्रमति॒र्वसि॑ष्ठः सहस्रम्भ॒रः शुचि॑जिह्वो अ॒ग्निः। सꣳ सी॑दस्व म॒हाꣳ अ॑सि॒ शोच॑स्व॥१२॥

%4.1.3.4
दे॒व॒वीत॑मः। वि धू॒मम॑ग्ने अरु॒षम्मि॑येध्य सृ॒ज प्र॑शस्त दर्\mbox{}श॒तम्। जनि॑ष्वा॒ हि जेन्यो॒ अग्रे॒ अह्नाꣳ॑ हि॒तो हि॒तेष्व॑रु॒षो वने॑षु। दमे॑दमे स॒प्त रत्ना॒ दधा॑नो॒\-ऽग्निर्\mbox{}होता॒ नि ष॑सादा॒ यजी॑यान्॥१३॥

%4.1.4.0
{\anuvakamend[{स्थ॒ ई॒धे॒ शोच॑स्व स॒प्तविꣳ॑शतिश्च॥३॥}]}

%4.1.4.1
सं ते॑ वा॒युर्मा॑त॒रिश्वा॑ दधातूत्ता॒नायै॒ हृद॑यं॒ यद्विलि॑ष्टम्। दे॒वानां॒ यश्चर॑ति प्रा॒णथे॑न॒ तस्मै॑ च देवि॒ वष॑डस्तु॒ तुभ्यम्᳚। सुजा॑तो॒ ज्योति॑षा स॒ह शर्म॒ वरू॑थ॒मास॑दः॒ सुवः॑। वासो॑ अग्ने वि॒श्वरू॑प॒ꣳ॒ सं व्य॑यस्व विभावसो। उदु॑ तिष्ठ स्वध्व॒रावा॑ नो दे॒व्या कृ॒पा। दृ॒शे च॑ भा॒सा बृ॑ह॒ता सु॑शु॒क्वनि॒राग्ने॑ याहि सुश॒स्तिभिः॑।॥१४॥

%4.1.4.2
ऊ॒र्ध्व ऊ॒ षु ण॑ ऊ॒तये॒ तिष्ठा॑ दे॒वो न स॑वि॒ता। ऊ॒र्ध्वो वाज॑स्य॒ सनि॑ता॒ यद॒ञ्जिभि॑र्वा॒घद्भि॑र्वि॒ह्वया॑महे। स जा॒तो गर्भो॑ असि॒ रोद॑स्यो॒रग्ने॒ चारु॒र्विभृ॑त॒ ओष॑धीषु। चि॒त्रः शिशुः॒ परि॒ तमाꣳ॑स्य॒क्तः प्र मा॒तृभ्यो॒ अधि॒ कनि॑क्रदद्गाः। स्थि॒रो भ॑व वी॒ड्व॑ङ्ग आ॒शुर्भ॑व वा॒ज्य॑र्वन्न्। पृ॒थुर्भ॑व सु॒षद॒स्त्वम॒ग्नेः पु॑रीष॒वाह॑नः। शि॒वो भ॑व॥१५॥

%4.1.4.3
प्र॒जाभ्यो॒ मानु॑षीभ्य॒स्त्वम॑ङ्गिरः। मा द्यावा॑पृथि॒वी अ॒भि शू॑शुचो॒ मान्तरि॑क्षं॒ मा वन॒स्पतीन्॑। प्रैतु॑ वा॒जी कनि॑क्रद॒न्नान॑द॒द्रास॑भः॒ पत्वा᳚। भर॑न्न॒ग्निम्पु॑री॒ष्यं॑ मा पा॒द्यायु॑षः पु॒रा। रास॑भो वां॒ कनि॑क्रद॒त्सुयु॑क्तो वृषणा॒ रथे᳚। स वा॑म॒ग्निम्पु॑री॒ष्य॑मा॒शुर्दू॒तो व॑हादि॒तः। वृषा॒ग्निं वृ॑षण॒म्भर॑न्न॒पां गर्भꣳ॑ समु॒द्रियम्᳚। अग्न॒ आ या॑हि॥१६॥

%4.1.4.4
वी॒तय॑ ऋ॒तꣳ स॒त्यम्। ओष॑धयः॒ प्रति॑ गृह्णीता॒ग्निमे॒तꣳ शि॒वमा॒यन्त॑म॒भ्यत्र॑ यु॒ष्मान्। व्यस्य॒न्विश्वा॒ अम॑ती॒ररा॑तीर्नि॒षीद॑न्नो॒ अप॑ दुर्म॒तिꣳ ह॑नत्। ओष॑धयः॒ प्रति॑ मोदध्वमेन॒म्पुष्पा॑वतीः सुपिप्प॒लाः। अ॒यं वो॒ गर्भ॑ ऋ॒त्वियः॑ प्र॒त्नꣳ स॒धस्थ॒मास॑दत्॥१७॥

%4.1.5.0
{\anuvakamend[{सु॒श॒स्तिभिः॑ शि॒वो भ॑व याहि॒ षट्त्रिꣳ॑शच्च॥४॥}]}

%4.1.5.1
वि पाज॑सा पृ॒थुना॒ शोशु॑चानो॒ बाध॑स्व द्वि॒षो र॒क्षसो॒ अमी॑वाः। सु॒शर्म॑णो बृह॒तः शर्म॑णि स्याम॒ग्नेर॒हꣳ सु॒हव॑स्य॒ प्रणी॑तौ। आपो॒ हि ष्ठा म॑यो॒भुव॒स्ता न॑ ऊ॒र्जे द॑धातन। म॒हे रणा॑य॒ चक्ष॑से। यो वः॑ शि॒वत॑मो॒ रस॒स्तस्य॑ भाजयते॒ह नः॑। उ॒श॒तीरि॑व मा॒तरः॑। तस्मा॒ अरं॑ गमाम वो॒ यस्य॒ क्षया॑य॒ जिन्व॑थ। आपो॑ ज॒नय॑था च नः। मि॒त्रः॥१८॥

%4.1.5.2
स॒ꣳ॒सृज्य॑ पृथि॒वीम्भूमिं॑ च॒ ज्योति॑षा स॒ह। सुजा॑तं जा॒तवे॑दसम॒ग्निं वै᳚श्वान॒रं वि॒भुम्। अ॒य॒क्ष्माय॑ त्वा॒ सꣳ सृ॑जामि प्र॒जाभ्यः॑। विश्वे᳚ त्वा दे॒वा वै᳚श्वान॒राः सꣳ सृ॑ज॒न्त्वानु॑ष्टुभेन॒ छन्द॑साङ्गिर॒स्वत्। रु॒द्राः स॒म्भृत्य॑ पृथि॒वीम्बृ॒हज्ज्योतिः॒ समी॑धिरे। तेषां᳚ भा॒नुरज॑स्र॒ इच्छु॒क्रो दे॒वेषु॑ रोचते। सꣳसृ॑ष्टां॒ वसु॑भी रु॒द्रैर्धीरैः᳚ कर्म॒ण्या᳚म्मृदम्᳚। हस्ता᳚भ्याम्मृ॒द्वीं कृ॒त्वा सि॑नीवा॒ली क॑रोतु॥१९॥

%4.1.5.3
ताम्। सि॒नी॒वा॒ली सु॑कप॒र्दा सु॑कुरी॒रा स्वौ॑प॒शा। सा तुभ्य॑मदिते मह॒ ओखां द॑धातु॒ हस्त॑योः। उ॒खां क॑रोतु॒ शक्त्या॑ बा॒हुभ्या॒मदि॑तिर्धि॒या। मा॒ता पु॒त्रं यथो॒पस्थे॒ साग्निम्बि॑भर्तु॒ गर्भ॒ आ। म॒खस्य॒ शिरो॑\-ऽसि य॒ज्ञस्य॑ प॒दे स्थः॑। वस॑वस्त्वा कृण्वन्तु गाय॒त्रेण॒ छन्द॑साङ्गिर॒स्वत्पृ॑थि॒व्य॑सि रु॒द्रास्त्वा॑ कृण्वन्तु॒ त्रैष्टु॑भेन॒ छन्द॑साङ्गिर॒स्वद॒न्तरि॑क्षमसि॥२०॥

%4.1.5.4
आ॒दि॒त्यास्त्वा॑ कृण्वन्तु॒ जाग॑तेन॒ छन्द॑साङ्गिर॒स्वद्द्यौर॑सि॒ विश्वे᳚ त्वा दे॒वा वै᳚श्वान॒राः कृ॑ण्व॒न्त्वानु॑ष्टुभेन॒ छन्द॑साङ्गिर॒स्वद्दिशो॑\-ऽसि ध्रु॒वासि॑ धा॒रया॒ मयि॑ प्र॒जाꣳ रा॒यस्पोषं॑ गौप॒त्यꣳ सु॒वीर्यꣳ॑ सजा॒तान् यज॑माना॒यादि॑त्यै॒ रास्ना॒स्यदि॑तिस्ते॒ बिलं॑ गृह्णातु॒ पाङ्क्ते॑न॒ छन्द॑साङ्गिर॒स्वत्। कृ॒त्वाय॒ सा म॒हीमु॒खाम्मृ॒न्मयीं॒ योनि॑म॒ग्नये᳚। ताम्पु॒त्रेभ्यः॒ सम्प्राय॑च्छ॒ददि॑तिः श्र॒पया॒निति॑॥२१॥

%4.1.6.0
{\anuvakamend[{मि॒त्रः क॑रोत्व॒न्तरि॑क्षमसि॒ प्र च॒त्वारि॑ च॥५॥}]}

%4.1.6.1
वस॑वस्त्वा धूपयन्तु गाय॒त्रेण॒ छन्द॑साङ्गिर॒स्वद्रु॒द्रास्त्वा॑ धूपयन्तु॒ त्रैष्टु॑भेन॒ छन्द॑साङ्गिर॒स्वदा॑दि॒त्यास्त्वा॑ धूपयन्तु॒ जाग॑तेन॒ छन्द॑साङ्गिर॒स्वद्विश्वे᳚ त्वा दे॒वा वै᳚श्वान॒रा धू॑पय॒न्त्वानु॑ष्टुभेन॒ छन्द॑साङ्गिर॒स्वदिन्द्र॑स्त्वा धूपयत्वङ्गिर॒स्वद्विष्णु॑स्त्वा धूपयत्वङ्गिर॒स्वद्वरु॑णस्त्वा धूपयत्वङ्गिर॒स्वददि॑तिस्त्वा दे॒वी वि॒श्वदे᳚व्यावती पृथि॒व्याः स॒धस्थे᳚\-ऽङ्गिर॒स्वत्ख॑नत्ववट दे॒वानां᳚ त्वा॒ पत्नीः᳚॥२२॥

%4.1.6.2
दे॒वीर्वि॒श्वदे᳚व्यावतीः पृथि॒व्याः स॒धस्थे᳚\-ऽङ्गिर॒स्वद्द॑धतूखे धि॒षणा᳚स्त्वा दे॒वीर्वि॒श्वदे᳚व्यावतीः पृथि॒व्याः स॒धस्थे᳚ \-ऽङ्गिर॒स्वद॒भीन्ध॑तामुखे॒ ग्नास्त्वा॑ दे॒वीर्वि॒श्वदे᳚व्यावतीः पृथि॒व्याः स॒धस्थे᳚\-ऽङ्गिर॒स्वच्छ्र॑पयन्तूखे॒ वरू᳚त्रयो॒ जन॑यस्त्वा दे॒वीर्वि॒श्वदे᳚व्यावतीः पृथि॒व्याः स॒धस्थे᳚\-ऽङ्गिर॒स्वत्प॑चन्तूखे। मित्रै॒तामु॒खाम्प॑चै॒षा मा भे॑दि। एातां ते॒ परि॑ ददा॒म्यभि॑त्त्यै। अ॒भीमाम्॥२३॥

%4.1.6.3
म॒हि॒ना दिव॑म्मि॒त्रो ब॑भूव स॒प्रथाः᳚। उ॒त श्रव॑सा पृथि॒वीम्। मि॒त्रस्य॑ चर्\mbox{}षणी॒धृतः॒ श्रवो॑ दे॒वस्य॑ सान॒सिम्। द्यु॒म्नं चि॒त्रश्र॑वस्तमम्। दे॒वस्त्वा॑ सवि॒तोद्व॑पतु सुपा॒णिः स्व॑ङ्गु॒॒रिः। सु॒बा॒हुरु॒त शक्त्या᳚। अप॑द्यमाना पृथि॒व्याशा॒ दिश॒ आ पृ॑ण। उत्ति॑ष्ठ बृह॒ती भ॑वो॒र्ध्वा ति॑ष्ठ ध्रु॒वा त्वम्। वस॑व॒स्त्वाच्छृ॑न्दन्तु गाय॒त्रेण॒ छन्द॑साङ्गिर॒स्वद्रु॒द्रास्त्वा च्छृ॑न्दन्तु॒ त्रैष्टु॑भेन॒ छन्द॑साङ्गिर॒स्वदा॑दि॒त्यास्त्वाच्छृ॑न्दन्तु॒ जाग॑तेन॒ छन्द॑साङ्गिर॒स्वद्विश्वे᳚ त्वा दे॒वा वै᳚श्वान॒रा आच्छृ॑न्द॒न्त्वानु॑ष्टुभेन॒ छन्द॑साङ्गिर॒स्वत्॥२४॥

%4.1.7.0
{\anuvakamend[{पत्नी॑रि॒माꣳ रु॒द्रास्त्वाच्छृ॑न्द॒न्त्वेका॒न्नविꣳ॑श॒तिश्च॑॥६॥}]}

%4.1.7.1
समा᳚स्त्वाग्न ऋ॒तवो॑ वर्धयन्तु संवत्स॒रा ऋष॑यो॒ यानि॑ स॒त्या। सं दि॒व्येन॑ दीदिहि रोच॒नेन॒ विश्वा॒ आ भा॑हि प्र॒दिशः॑ पृथि॒व्याः। सं चे॒ध्यस्वा᳚ग्ने॒ प्र च॑ बोधयैन॒मुच्च॑ तिष्ठ मह॒ते सौभ॑गाय। मा च॑ रिषदुपस॒त्ता ते॑ अग्ने ब्र॒ह्माण॑स्ते य॒शसः॑ सन्तु॒ मान्ये। त्वाम॑ग्ने वृणते ब्राह्म॒णा इ॒मे शि॒वो अ॑ग्ने॥२५॥

%4.1.7.2
सं॒वर॑णे भवा नः। स॒प॒त्न॒हा नो॑ अभिमाति॒जिच्च॒ स्वे गये॑ जागृ॒ह्यप्र॑युच्छन्न्। इ॒हैवाग्ने॒ अधि॑ धारया र॒यिं मा त्वा॒ नि क्र॑न्पूर्व॒चितो॑ निका॒रिणः॑। क्ष॒त्रम॑ग्ने सु॒यम॑मस्तु॒ तुभ्य॑मुपस॒त्ता व॑र्धतां ते॒ अनि॑ष्टृतः। क्ष॒त्रेणा᳚ग्ने॒ स्वायुः॒ सꣳ र॑भस्व मि॒त्रेणा᳚ग्ने मित्र॒धेये॑ यतस्व। स॒जा॒ताना᳚म्मध्यम॒स्था ए॑धि॒ राज्ञा॑मग्ने विह॒व्यो॑ दीदिही॒ह। अति॑॥२६॥

%4.1.7.3
निहो॒ अति॒ स्रिधो\-ऽत्यचि॑त्ति॒मत्यरा॑तिमग्ने। विश्वा॒ ह्य॑ग्ने दुरि॒ता सह॒स्वाथा॒स्मभ्यꣳ॑ स॒हवी॑राꣳ र॒यिं दाः᳚। अ॒ना॒धृ॒ष्यो जा॒तवे॑दा॒ अनि॑ष्टृतो वि॒राड॑ग्ने क्षत्र॒भृद्दी॑दिही॒ह। विश्वा॒ आशाः᳚ प्रमु॒ञ्चन्मानु॑षीर्भि॒यः शि॒वाभि॑र॒द्य परि॑ पाहि नो वृ॒धे। बृह॑स्पते सवितर्बो॒धयै॑न॒ꣳ॒ सꣳशि॑तं चित्संत॒राꣳ सꣳ शि॑शाधि। व॒र्धयै॑नम्मह॒ते सौभ॑गाय॥२७॥

%4.1.7.4
विश्व॑ एन॒मनु॑ मदन्तु दे॒वाः। अ॒मु॒त्र॒भूया॒दध॒ यद्य॒मस्य॒ बृह॑स्पते अ॒भिश॑स्ते॒रमु॑ञ्चः। प्रत्यौ॑हताम॒श्विना॑ मृ॒त्युम॑स्माद्दे॒वाना॑मग्ने भि॒षजा॒ शची॑भिः। उद्व॒यं तम॑स॒स्परि॒ पश्य॑न्तो॒ ज्योति॒रुत्त॑रम्। दे॒वं दे॑व॒त्रा सूर्य॒मग॑न्म॒ ज्योति॑रुत्त॒मम्॥२८॥

%4.1.8.0
{\anuvakamend[{इ॒मे शि॒वो अ॒ग्ने\-ऽति॒ सौभ॑गाय॒ चतु॑स्त्रिꣳशच्च॥७॥}]}

%4.1.8.1
ऊ॒र्ध्वा अ॑स्य स॒मिधो॑ भवन्त्यू॒र्ध्वा शु॒क्रा शो॒चीꣳष्य॒ग्नेः। द्यु॒मत्त॑मा सु॒प्रती॑कस्य सू॒नोः। तनू॒नपा॒दसु॑रो वि॒श्ववे॑दा दे॒वो दे॒वेषु॑ दे॒वः। प॒थ आन॑क्ति॒ मध्वा॑ घृ॒तेन॑। मध्वा॑ य॒ज्ञं न॑क्षसे प्रीणा॒नो नरा॒शꣳसो॑ अग्ने। सु॒कृद्दे॒वः स॑वि॒ता वि॒श्ववा॑रः। अच्छा॒यमे॑ति॒ शव॑सा घृ॒तेने॑डा॒नो वह्नि॒र्नम॑सा। अ॒ग्निꣴ स्रुचो॑ अध्व॒रेषु॑ प्र॒यत्सु॑। स य॑क्षदस्य महि॒मान॑म॒ग्नेः सः॥२९॥

%4.1.8.2
ई॒ म॒न्द्रासु॑ प्र॒यसः॑। वसु॒श्चेति॑ष्ठो वसु॒धात॑मश्च। द्वारो॑ दे॒वीरन्व॑स्य॒ विश्वे᳚ व्र॒ता द॑दन्ते अ॒ग्नेः। उ॒रु॒व्यच॑सो॒ धाम्ना॒ पत्य॑मानाः। ते अ॑स्य॒ योष॑णे दि॒व्ये न योना॑वु॒षासा॒नक्ता᳚। इ॒मं य॒ज्ञम॑वतामध्व॒रं नः॑। दैव्या॑ होतारावू॒र्ध्वम॑ध्व॒रं नो॒\-ऽग्नेर्जि॒ह्वाम॒भि गृ॑णीतम्। कृ॒णु॒तं नः॒ स्वि॑ष्टिम्। ति॒स्रो दे॒वीर्ब॒र्हिरेदꣳ स॑द॒न्त्विडा॒ सर॑स्वती॥३०॥

%4.1.8.3
भार॑ती। म॒ही गृ॑णा॒ना। तन्न॑स्तु॒रीप॒मद्भु॑तम्पुरु॒क्षु त्वष्टा॑ सु॒वीरम्᳚। रा॒यस्पोषं॒ वि ष्य॑तु॒ नाभि॑म॒स्मे। वन॑स्प॒ते\-ऽव॑ सृजा॒ ररा॑ण॒स्त्मना॑ दे॒वेषु॑। अ॒ग्निर्\mbox{}ह॒व्यꣳ श॑मि॒ता सू॑दयाति। अग्ने॒ स्वाहा॑ कृणुहि जातवेद॒ इन्द्रा॑य ह॒व्यम्। विश्वे॑ दे॒वा ह॒विरि॒दं जु॑षन्ताम्। हि॒र॒ण्य॒ग॒र्भः सम॑वर्त॒ताग्रे॑ भू॒तस्य॑ जा॒तः पति॒रेक॑ आसीत्। स दा॑धार पृथि॒वीं द्याम्॥३१॥

%4.1.8.4
उ॒तेमां कस्मै॑ दे॒वाय॑ ह॒विषा॑ विधेम। यः प्रा॑ण॒तो नि॑मिष॒तो म॑हि॒त्वैक॒ इद्राजा॒ जग॑तो ब॒भूव॑। य ईशे॑ अ॒स्य द्वि॒पद॒श्चतु॑ष्पदः॒ कस्मै॑ दे॒वाय॑ ह॒विषा॑ विधेम। य आ᳚त्म॒दा ब॑ल॒दा यस्य॒ विश्व॑ उ॒पास॑ते प्र॒शिषं॒ यस्य॑ दे॒वाः। यस्य॑ छा॒यामृतं॒ यस्य॑ मृ॒त्युः कस्मै॑ दे॒वाय॑ ह॒विषा॑ विधेम। यस्ये॒मे हि॒मव॑न्तो महि॒त्वा यस्य॑ समु॒द्रꣳ र॒सया॑ स॒ह॥३२॥

%4.1.8.5
आ॒हुः। यस्ये॒माः प्र॒दिशो॒ यस्य॑ बा॒हू कस्मै॑ दे॒वाय॑ ह॒विषा॑ विधेम। यं क्रन्द॑सी॒ अव॑सा तस्तभा॒ने अ॒भ्यैक्षे॑ता॒म्मन॑सा॒ रेज॑माने। यत्राधि॒ सूर॒ उदि॑तौ॒ व्येति॒ कस्मै॑ दे॒वाय॑ ह॒विषा॑ विधेम। येन॒ द्यौरु॒ग्रा पृ॑थि॒वी च॑ दृ॒ढे येन॒ सुवः॑ स्तभि॒तं येन॒ नाकः॑। यो अ॒न्तरि॑क्षे॒ रज॑सो वि॒मानः॒ कस्मै॑ दे॒वाय॑ ह॒विषा॑ विधेम। आपो॑ ह॒ यन्म॑ह॒तीर्विश्वम्᳚॥३३॥

%4.1.8.6
आय॒न्दक्षं॒ दधा॑ना ज॒नय॑न्तीर॒ग्निम्। ततो॑ दे॒वानां॒ निर॑वर्त॒तासु॒रेकः॒ कस्मै॑ दे॒वाय॑ ह॒विषा॑ विधेम। यश्चि॒दापो॑ महि॒ना प॒र्यप॑श्य॒द्दक्षं॒ दधा॑ना ज॒नय॑न्तीर॒ग्निम्। यो दे॒वेष्वधि॑ दे॒व एक॒ आसी॒त्कस्मै॑ दे॒वाय॑ ह॒विषा॑ विधेम॥३४॥

%4.1.9.0
{\anuvakamend[{अ॒ग्नेः स सर॑स्वती॒ द्याꣳ स॒ह विश्व॒ञ्चतु॑स्त्रिHꣳशश्च॥८॥}]}

%4.1.9.1
आकू॑तिम॒ग्निम्प्र॒युज॒ꣴ॒ स्वाहा॒ मनो॑ मे॒धाम॒ग्निम्प्र॒युज॒ꣴ॒ स्वाहा॑ चि॒त्तं विज्ञा॑तम॒ग्निम्प्र॒युज॒ꣴ॒ स्वाहा॑ वा॒चो विधृ॑तिम॒ग्निम्प्र॒युज॒ꣴ॒ स्वाहा᳚ प्र॒जाप॑तये॒ मन॑वे॒ स्वाहा॒ग्नये॑ वैश्वान॒राय॒ स्वाहा॒ विश्वे॑ दे॒वस्य॑ ने॒तुर्मर्तो॑ वृणीत स॒ख्यं विश्वे॑ रा॒य इ॑षुध्यसि द्यु॒म्नं वृ॑णीत पु॒ष्यसे॒ स्वाहा॒ मा सु भि॑त्था॒ मा सु रि॑षो॒ दृꣳह॑स्व वी॒डय॑स्व॒ सु। अम्ब॑ धृष्णु वी॒रय॑स्व॥३५॥

%4.1.9.2
अ॒ग्निश्चे॒दं क॑रिष्यथः। दृꣳह॑स्व देवि पृथिवि स्व॒स्तय॑ आसु॒री मा॒या स्व॒धया॑ कृ॒तासि॑। जुष्टं॑ दे॒वाना॑मि॒दम॑स्तु ह॒व्यमरि॑ष्टा॒ त्वमुदि॑हि य॒ज्ञे अ॒स्मिन्न्। मित्रै॒तामु॒खां त॑पै॒षा मा भे॑दि। ए॒तान्ते॒ परि॑ ददा॒म्यभि॑त्त्यै। द्र्व॑न्नः स॒र्पिरा॑सुतिः प्र॒त्नो होता॒ वरे᳚ण्यः। सह॑सस्पु॒त्रो अद्भु॑तः। पर॑स्या॒ अधि॑ सं॒वतो\-ऽव॑राꣳ अभ्या॥३६॥

%4.1.9.3
त॒र॒। यत्रा॒हमस्मि॒ ताꣳ अ॑व। प॒र॒मस्याः᳚ परा॒वतो॑ रो॒हिद॑श्व इ॒हा ग॑हि। पु॒री॒ष्यः॑ पुरुप्रि॒यो\-ऽग्ने॒ त्वं त॑रा॒ मृधः॑। सीद॒ त्वम्मा॒तुर॒स्या उ॒पस्थे॒ विश्वा᳚न्यग्ने व॒युना॑नि वि॒द्वान्। मैना॑म॒र्चिषा॒ मा तप॑सा॒भि शू॑शुचो॒\-ऽन्तर॑स्याꣳ शु॒क्रज्यो॑ति॒र्वि भा॑हि। अ॒न्तर॑ग्ने रु॒चा त्वमु॒खायै॒ सद॑ने॒ स्वे। तस्या॒स्त्वꣳ हर॑सा॒ तप॒ञ्जात॑वेदः शि॒वो भ॑व। शि॒वो भू॒त्वा मह्य॑म॒ग्ने\-ऽथो॑ सीद शि॒वस्त्वम्। शि॒वाः कृ॒त्वा दिशः॒ सर्वाः॒ स्वां योनि॑मि॒हास॑दः॥३७॥

%4.1.10.0
{\anuvakamend[{वी॒रय॒स्वा तप॑न्विꣳश॒तिश्च॑॥९॥}]}

%4.1.10.1
यद॑ग्ने॒ यानि॒ कानि॒ चा ते॒ दारू॑णि द॒ध्मसि॑। तद॑स्तु॒ तुभ्य॒मिद्घृ॒तं तज्जु॑षस्व यविष्ठ्य। यदत्त्यु॑प॒जिह्वि॑का॒ यद्व॒म्रो अ॑ति॒सर्प॑ति। सर्वं॒ तद॑स्तु ते घृ॒तं तज्जु॑षस्व यविष्ठ्य। रात्रि॑ꣳरात्रि॒मप्र॑याव॒म्भर॒न्तो\-ऽश्वा॑येव॒ तिष्ठ॑ते घा॒समस्मै। रा॒यस्पोषे॑ण॒ समि॒षा मद॒न्तो\-ऽग्ने॒ मा ते॒ प्रति॑वेशा रिषाम। नाभा᳚॥३८॥

%4.1.10.2
पृ॒थि॒व्याः स॑मिधा॒नम॒ग्निꣳ रा॒यस्पोषा॑य बृह॒ते ह॑वामहे। इ॒र॒म्म॒दम्बृ॒हदु॑क्थं॒ यज॑त्रं॒ जेता॑रम॒ग्निम्पृ॑तनासु सास॒हिम्। याः सेना॑ अ॒भीत्व॑रीराव्या॒धिनी॒रुग॑णा उ॒त। ये स्ते॒ना ये च॒ तस्क॑रा॒स्ताꣳस्ते॑ अ॒ग्ने\-ऽपि॑ दधाम्या॒स्ये᳚। दꣴष्ट्रा᳚भ्याम्म॒लिम्लू॒ञ्जम्भ्यै॒स्तस्क॑राꣳ उ॒त। हनू᳚भ्याꣳस्ते॒नान्भ॑गव॒स्ताꣳस्त्वं खा॑द॒ सुखा॑दितान्। ये जने॑षु म॒लिम्ल॑वः स्ते॒नास॒स्तस्क॑रा॒ वने᳚। ये॥३९॥

%4.1.10.3
कक्षे᳚ष्वघा॒यव॒स्ताꣳस्ते॑ दधामि॒ जम्भ॑योः। यो अ॒स्मभ्य॑मराती॒याद्यश्च॑ नो॒ द्वेष॑ते॒ जनः॑। निन्दा॒द्यो अ॒स्मान् दिप्सा᳚च्च॒ सर्वं॒ तम्म॑स्म॒सा कु॑रु। सꣳशि॑तं मे॒ ब्रह्म॒ सꣳशि॑तं वीर्यं॑ बलम्᳚। सꣳशि॑तं क्ष॒त्रं जि॒ष्णु यस्या॒हमस्मि॑ पु॒रोहि॑तः। उदे॑षाम्बा॒हू अ॑तिर॒मुद्वर्च॒ उदू॒ बलम्᳚। क्षि॒णोमि॒ ब्रह्म॑णा॒मित्रा॒नुन्न॑यामि॥४०॥

%4.1.10.4
स्वाꣳ अ॒हम्। दृ॒शा॒नो रु॒क्म उ॒र्व्या व्य॑द्यौद्दु॒र्मर्\mbox{}ष॒मायुः॑ श्रि॒ये रु॑चा॒नः। अ॒ग्निर॒मृतो॑ अभव॒द्वयो॑भि॒र्यदे॑नं॒ द्यौरज॑नयत्सु॒रेताः᳚। विश्वा॑ रू॒पाणि॒ प्रति॑ मुञ्चते क॒विः प्रासा॑वीद्भ॒द्रं द्वि॒पदे॒ चतु॑ष्पदे। वि नाक॑मख्यत्सवि॒ता वरे॒ण्यो\-ऽनु॑ प्र॒याण॑मु॒षसो॒ वि रा॑जति। नक्तो॒षासा॒ सम॑नसा॒ विरू॑पे धा॒पये॑ते॒ शिशु॒मेकꣳ॑ समी॒ची। द्यावा॒ क्षामा॑ रु॒क्मः॥४१॥

%4.1.10.5
अ॒न्तर्वि भा॑ति दे॒वा अ॒ग्निं धा॑रयन्द्रविणो॒दाः। सु॒प॒र्णो\-ऽसि ग॒रुत्मा᳚न्त्रि॒वृत्ते॒ शिरो॑ गाय॒त्रं चक्षुः॒ स्तोम॑ आ॒त्मा साम॑ ते त॒नूर्वा॑मदे॒व्यम्बृ॑हद्रथन्त॒रे प॒क्षौ य॑ज्ञाय॒ज्ञिय॒म्पुच्छं॒ छन्दा॒ꣳ॒स्यङ्गा॑नि॒ धिष्णि॑याः श॒फा॒ यजूꣳ॑षि॒ नाम॑। सु॒प॒र्णो॑\-ऽसि ग॒रुत्मा॒न्दिवं॑ गच्छ॒ सुवः॑ पत॥४२॥

%4.1.11.0
{\anuvakamend[{नाभा॒ वने॒ येन॑ यामि॒ क्षामा॑ रु॒क्मो᳚\-ऽष्टात्रिꣳ॑शच्च॥10॥}]}

%4.1.11.1
अग्ने॒ यं य॒ज्ञम॑ध्व॒रं वि॒श्वतः॑ परि॒भूरसि॑। स इद्दे॒वेषु॑ गच्छति। सोम॒ यास्ते॑ मयो॒भुव॑ ऊ॒तयः॒ सन्ति॑ दा॒शुषे᳚। ताभि॑र्नो\-ऽवि॒ता भ॑व। अ॒ग्निर्मू॒र्धा भुवः॑। त्वं नः॑ सोम॒ या ते॒ धामा॑नि। तत्स॑वि॒तुर्वरे᳚ण्य॒म्भर्गो॑ दे॒वस्य॑ धीमहि। धियो॒ यो नः॑ प्रचो॒दया᳚त्। अचि॑त्ती॒ यच्च॑कृ॒मा दैव्ये॒ जने॑ दी॒नैर्दक्षैः॒ प्रभू॑ती पूरुष॒त्वता᳚।॥४३॥

%4.1.11.2
दे॒वेषु॑ च सवित॒र्मानु॑षेषु च॒ त्वं नो॒ अत्र॑ सुवता॒दना॑गसः। चो॒द॒यि॒त्री सू॒नृता॑नां॒ चेत॑न्ती सुमती॒नाम्। य॒ज्ञं द॑धे॒ सर॑स्वती। पावी॑रवी क॒न्या॑ चि॒त्रायुः॒ सर॑स्वती वी॒रप॑त्नी॒ धियं॑ धात्। ग्नाभि॒रच्छि॑द्रꣳ शर॒णꣳ स॒जोषा॑ दुरा॒धर्\mbox{}षं॑ गृण॒ते शर्म॑ यꣳसत्। पू॒षा गा अन्वे॑तु नः पू॒षा र॑क्ष॒त्वर्व॑तः। पू॒षा वाजꣳ॑ सनोतु नः। शु॒क्रं ते॑ अ॒न्यद्य॑ज॒तं ते॑ अ॒न्यत्॥४४॥

%4.1.11.3
विषु॑रूपे॒ अह॑नी॒ द्यौरि॑वासि। विश्वा॒ हि मा॒या अव॑सि स्वधावो भ॒द्रा ते॑ पूषन्नि॒ह रा॒तिर॑स्तु। ते॑\-ऽवर्धन्त॒ स्वत॑वसो महित्व॒ना नाकं॑ त॒स्थुरु॒रु च॑क्रिरे॒ सदः॑। विष्णु॒र्यद्धाव॒द्वृष॑णम्मद॒च्युतं॒ वयो॒ न सी॑द॒न्नधि॑ ब॒र्\mbox{}हिषि॑ प्रि॒ये। प्र चि॒त्रम॒र्कं गृ॑ण॒ते तु॒राय॒ मारु॑ताय॒ स्वत॑वसे भरध्वम्। ये सहाꣳ॑सि॒ सह॑सा॒ सह॑न्ते॥४५॥

%4.1.11.4
रेज॑ते अग्ने पृथि॒वी म॒खेभ्यः॑। विश्वे॑ दे॒वा विश्वे॑ देवाः। द्यावा॑ नः पृथि॒वी इ॒मꣳ सि॒ध्रम॒द्य दि॑वि॒स्पृशम्᳚। य॒ज्ञं दे॒वेषु॑ यच्छताम्। प्र पू᳚र्व॒जे पि॒तरा॒ नव्य॑सीभिर्गी॒र्भिः कृ॑णुध्व॒ꣳ॒ सद॑ने ऋ॒तस्य॑। आ नो᳚ द्यावापृथिवी॒ दैव्ये॑न॒ जने॑न यात॒म्महि॑ वां॒ वरू॑थम्। अ॒ग्निꣴ स्तोमे॑न बोधय समिधा॒नो अम॑र्त्यम्। ह॒व्या दे॒वेषु॑ नो दधत्। स ह॑व्य॒वाडम॑र्त्य उ॒शिग्दू॒तश्चनो॑हितः। अ॒ग्निर्धि॒या समृ॑ण्वति। शं नो॑ भवन्तु॒ वाजे॑वाजे॥४६॥

%4.2.0.0
{\anuvakamend[{पू॒रु॒ष॒त्वता॑ यज॒तन्ते॑ अ॒न्यथ्सह॑न्ते॒ चनो॑हितो॒\-ऽष्टौ च॑॥11॥}]}

%4.2.0.0

{\anuvakamend[{विष्णोः॒ क्रमो॑\-ऽसि दि॒वस्पर्यन्न॑प॒ते\-ऽपे॑त॒ समि॑तं॒ या जा॒ता मा नो॑ हिꣳसीद्ध्रु॒वा\-ऽस्या॑दि॒त्यङ्गर्भ॒मिन्द्रा᳚ग्नी रोच॒नैका॑दश॥11॥ विष्णो॑रस्मिन् ह॒व्येति॑ त्वा॒\-ऽहं धी॒तिभि॒र्\mbox{}होत्रा॑ अ॒ष्टाच॑त्वारिꣳशत्॥48॥ विष्णोः॒ क्रमो॑\-ऽसि॒ स त्वन्नो॑ अग्ने॥}]}
%%% END PRASHNA
