\sect{पञ्चमः प्रश्नः}\setcounter{anuvakam}{0}
\dnsub{तैत्तिरीयसंहितायां तृतीयकाण्डे पञ्चमः प्रश्नः}
%3.5.1.0
%3.5.1.1
पू॒र्णा प॒श्चादु॒त पू॒र्णा पु॒रस्ता॒दुन्म॑ध्य॒तः पौ᳚र्णमा॒सी जि॑गाय। तस्यां᳚ दे॒वा अधि॑ सं॒वस॑न्त उत्त॒मे नाक॑ इ॒ह मा॑दयन्ताम्। यत्ते॑ दे॒वा अद॑धुर्भाग॒धेय॒ममा॑वास्ये सं॒वस॑न्तो महि॒त्वा। सा नो॑ य॒ज्ञम्पि॑पृहि विश्ववारे र॒यिं नो॑ धेहि सुभगे सु॒वीरम्᳚। नि॒वेश॑नी सं॒गम॑नी॒ वसू॑नां॒ विश्वा॑ रू॒पाणि॒ वसू᳚न्यावे॒शय॑न्ती। स॒ह॒स्र॒पो॒षꣳ सु॒भगा॒ ररा॑णा॒ सा न॒ आ ग॒न्वर्च॑सा॥१॥

%3.5.1.2
सं॒वि॒दा॒ना। अग्नी॑षोमौ प्रथ॒मौ वी॒र्ये॑ण॒ वसू᳚न्रु॒द्राना॑दि॒त्यानि॒ह जि॑न्वतम्। मा॒ध्यꣳ हि पौ᳚र्णमा॒सं जु॒षेथां॒ ब्रह्म॑णा वृ॒द्धौ सु॑कृ॒तेन॑ सा॒तावथा॒स्मभ्यꣳ॑ स॒हवी॑राꣳ र॒यिं नि य॑च्छतम्। आ॒दि॒त्याश्चाङ्गि॑रसश्चा॒ग्नीनाद॑धत॒ ते द॑र्\mbox{}शपूर्णमा॒सौ प्रैफ्स॒न्तेषा॒मङ्गि॑रसां॒ निरु॑प्तꣳ ह॒विरासी॒दथा॑दि॒त्या ए॒तौ होमा॑वपश्य॒न्ताव॑जुहवु॒स्ततो॒ वै ते द॑र्\mbox{}शपूर्णमा॒सौ॥२॥

%3.5.1.3
पूर्व॒ आल॑भन्त दर्\mbox{}शपूर्णमा॒सावा॒लभ॑मान ए॒तौ होमौ॑ पु॒रस्ता᳚ज्जुहुयाथ्सा॒क्षादे॒व द॑र्\mbox{}शपूर्णमा॒सावा ल॑भते ब्रह्मवा॒दिनो॑ वदन्ति॒ स त्वै द॑र्\mbox{}शपूर्णमा॒सावाल॑भेत॒ य ए॑नयोरनुलो॒मं च॑ प्रतिलो॒मं च॑ वि॒द्यादित्य॑मावा॒स्या॑या ऊ॒र्ध्वं तद॑नुलो॒म\-म्पौ᳚र्णमा॒स्यै प्र॑ती॒चीनं॒ तत्प्र॑तिलो॒मं यत्पौ᳚र्णमा॒सीम्पूर्वा॑मा॒लभे॑त प्रतिलो॒ममे॑ना॒वा ल॑भेता॒मुम॑प॒क्षीय॑माण॒मन्वप॑॥३॥

%3.5.1.4
क्षी॒ये॒त॒ सा॒र॒स्व॒तौ होमौ॑ पु॒रस्ता᳚ज्जुहुयादमावा॒स्या॑ वै सर॑स्वत्यनुलो॒ममे॒वैना॒वा ल॑भते॒\-ऽमुमा॒प्याय॑मान॒मन्वा प्या॑यत आग्नावैष्ण॒वमेका॑दशकपालम्पु॒रस्ता॒न्निर्व॑पे॒थ्सर॑स्वत्यै च॒रुꣳ सर॑स्वते॒ द्वाद॑शकपालं॒ यदा᳚ग्ने॒यो भव॑त्य॒ग्निर्वै य॑ज्ञमु॒खं य॑ज्ञमु॒खमे॒वर्द्धि॑म्पु॒रस्ता᳚द्धत्ते॒ यद्वै᳚ष्ण॒वो भव॑ति य॒ज्ञो वै विष्णु॑र्य॒ज्ञमे॒वारभ्य॒ प्र त॑नुते॒ सर॑स्वत्यै च॒रुर्भ॑वति॒ सर॑स्वते॒ द्वाद॑शकपालो\-ऽमावा॒स्या॑ वै सर॑स्वती पू॒र्णमा॑सः॒ सर॑स्वा॒न्तावे॒व सा॒क्षादा र॑भत ऋ॒ध्नोत्या᳚भ्या॒न्द्वाद॑शकपालः॒ सर॑स्वते भवति मिथुन॒त्वाय॒ प्रजा᳚त्यै मिथु॒नौ गावौ॒ दक्षि॑णा॒ समृ॑द्ध्यै॥४॥

%3.5.2.0
{\anuvakamend[{वर्च॑सा॒ वै ते द॑र्\mbox{}शपूर्णमा॒सावप॑ तनुते॒ सर॑स्वत्यै॒ पञ्च॑विꣳशतिश्च}]}%॥१॥

%3.5.2.1
ऋष॑यो॒ वा इन्द्र॑म्प्र॒त्यक्षं॒ नाप॑श्य॒न्तं वसि॑ष्ठः प्र॒त्यक्ष॑म्पश्य॒थ्सो᳚\-ऽब्रवी॒द्ब्राह्म॑णं ते वक्ष्यामि॒ यथा॒ त्वत्पु॑रोहिताः प्र॒जाः प्र॑जनि॒ष्यन्ते\-ऽथ॒ मेत॑रेभ्य॒ ऋषि॑भ्यो॒ मा प्र वो॑च॒ इति॒ तस्मा॑ ए॒तान्थ्स्तोम॑भागानब्रवी॒त्ततो॒ वसि॑ष्ठपुरोहिताः प्र॒जाः प्राजा॑यन्त॒ तस्मा᳚द्वासि॒ष्ठो ब्र॒ह्मा का॒र्यः॑ प्रैव जा॑यते र॒श्मिर॑सि॒ क्षया॑य त्वा॒ क्षयं॑ जि॒न्वेति॑॥५॥

%3.5.2.2
आ॒ह॒ दे॒वा वै क्षयो॑ दे॒वेभ्य॑ ए॒व य॒ज्ञम्प्राह॒ प्रेति॑रसि॒ धर्मा॑य त्वा॒ धर्मं॑ जि॒न्वेत्या॑ह मनु॒ष्या॑ वै धर्मो॑ मनु॒ष्ये᳚भ्य ए॒व य॒ज्ञम्प्राहान्वि॑तिरसि दि॒वे त्वा॒ दिवं॑ जि॒न्वेत्या॑है॒भ्य ए॒व लो॒केभ्यो॑ य॒ज्ञम्प्राह॑ विष्ट॒म्भो॑\-ऽसि॒ वृष्ट्यै᳚ त्वा॒ वृष्टिं॑ जि॒न्वेत्या॑ह॒ वृष्टि॑मे॒वाव॑॥६॥

%3.5.2.3
रु॒न्द्धे॒ प्र॒वास्य॑नु॒वासीत्या॑ह मिथुन॒त्वायो॒शिग॑सि॒ वसु॑भ्यस्त्वा॒ वसू᳚ञ्जि॒न्वेत्या॑हा॒ष्टौ वस॑व॒ एका॑दश रु॒द्रा द्वाद॑शादि॒त्या ए॒ताव॑न्तो॒ वै दे॒वास्तेभ्य॑ ए॒व य॒ज्ञम्प्राहौजो॑\-ऽसि पि॒तृभ्य॑स्त्वा पि॒तॄञ्जि॒न्वेत्या॑ह दे॒वाने॒व पि॒तॄननु॒ सं त॑नोति॒ तन्तु॑रसि प्र॒जाभ्य॑स्त्वा प्र॒जा जि॑न्व॥७॥

%3.5.2.4
इत्या॑ह पि॒तॄने॒व प्र॒जा अनु॒ सं त॑नोति पृतना॒षाड॑सि प॒शुभ्य॑स्त्वा प॒शूञ्जि॒न्वेत्या॑ह प्र॒जा ए॒व प॒शूननु॒ सं त॑नोति रे॒वद॒स्योष॑धीभ्य॒स्त्वौष॑धीर्जि॒न्वेत्या॒हौष॑धीष्वे॒व प॒शून्प्रति॑ ष्ठापयत्यभि॒जिद॑सि यु॒क्तग्रा॒वेन्द्रा॑य॒ त्वेन्द्रं॑ जि॒न्वेत्या॑हा॒भिजि॑त्या॒ अधि॑पतिरसि प्रा॒णाय॑ त्वा प्रा॒णम्॥८॥

%3.5.2.5
जि॒न्वेत्या॑ह प्र॒जास्वे॒व प्रा॒णान्द॑धाति त्रि॒वृद॑सि प्र॒वृद॒सीत्या॑ह मिथुन॒त्वाय॑ सꣳरो॒हो॑\-ऽसि नीरो॒हो॑\-ऽसीत्या॑ह॒ प्रजा᳚त्यै वसु॒को॑\-ऽसि॒ वेष॑श्रिरसि॒ वस्य॑ष्टिर॒सीत्या॑ह॒ प्रति॑ष्ठित्यै॥९॥

%3.5.3.0
{\anuvakamend[{जि॒न्वेत्यव॑ प्र॒जा जि॑न्व प्रा॒णन्त्रि॒ꣳ॒शच्च॑}]}%॥२॥

%3.5.3.1
अ॒ग्निना॑ दे॒वेन॒ पृत॑ना जयामि गाय॒त्रेण॒ छन्द॑सा त्रि॒वृता॒ स्तोमे॑न रथन्त॒रेण॒ साम्ना॑ वषट्का॒रेण॒ वज्रे॑ण पूर्व॒जान्भ्रातृ॑व्या॒नध॑रान्पादया॒म्यवै॑नान्बाधे॒ प्रत्ये॑नान्नुदे॒\-ऽस्मिन्क्षये॒\-ऽस्मिन्भू॑मिलो॒के यो᳚\-ऽस्मान्द्वेष्टि॒ यं च॑ व॒यं द्वि॒ष्मो विष्णोः॒ क्रमे॒णात्ये॑नान्क्रामा॒मीन्द्रे॑ण दे॒वेन॒ पृत॑ना जयामि॒ त्रैष्टु॑भेन॒ छन्द॑सा पञ्चद॒शेन॒ स्तोमे॑न बृह॒ता साम्ना॑ वषट्का॒रेण॒ वज्रे॑ण॥१०॥

%3.5.3.2
स॒ह॒जान् विश्वे॑भिर्दे॒वेभिः॒ पृत॑ना जयामि॒ जाग॑तेन॒ छन्द॑सा सप्तद॒शेन॒ स्तोमे॑न वामदे॒व्येन॒ साम्ना॑ वषट्का॒रेण॒ वज्रे॑णापर॒जानिन्द्रे॑ण स॒युजो॑ व॒यꣳ सा॑स॒ह्याम॑ पृतन्य॒तः। घ्नन्तो॑ वृ॒त्राण्य॑प्र॒ति। यत्ते॑ अग्ने॒ तेज॒स्तेना॒हं ते॑ज॒स्वी भू॑यासं॒ यत्ते॑ अग्ने॒ वर्च॒स्तेना॒हं व॑च॒स्वी भू॑यासं॒ यत्ते॑ अग्ने॒ हर॒स्तेना॒हꣳ ह॑र॒स्वी भू॑यासम्॥११॥

%3.5.4.0
{\anuvakamend[{बृ॒ह॒ता साम्ना॑ वषट्का॒रेण॒ वज्रे॑ण॒ षट्च॑त्वारिꣳशच्च}]}%॥३॥

%3.5.4.1
ये दे॒वा य॑ज्ञ॒हनो॑ यज्ञ॒मुषः॑ पृथि॒व्यामध्यास॑ते। अ॒ग्निर्मा॒ तेभ्यो॑ रक्षतु॒ गच्छे॑म सु॒कृतो॑ व॒यम्। आग॑न्म मित्रावरुणा वरेण्या॒ रात्री॑णाम्भा॒गो यु॒वयो॒र्यो अस्ति॑। नाकं॑ गृह्णा॒नाः सु॑कृ॒तस्य॑ लो॒के तृ॒तीये॑ पृ॒ष्ठे अधि॑ रोच॒ने दि॒वः। ये दे॒वा य॑ज्ञ॒हनो॑ यज्ञ॒मुषो॒\-ऽन्तरि॒क्षे\-ऽध्यास॑ते। वा॒युर्मा॒ तेभ्यो॑ रक्षतु॒ गच्छे॑म सु॒कृतो॑ व॒यम्। यास्ते॒ रात्रीः᳚ सवितः॥१२॥

%3.5.4.2
दे॒व॒यानी॑रन्त॒रा द्यावा॑पृथि॒वी वि॒यन्ति॑। गृ॒हैश्च॒ सर्वैः᳚ प्र॒जया॒ न्वग्रे॒ सुवो॒ रुहा॑णास्तरता॒ रजाꣳ॑सि। ये दे॒वा य॑ज्ञ॒हनो॑ यज्ञ॒मुषो॑ दि॒व्यध्यास॑ते। सूर्यो॑ मा॒ तेभ्यो॑ रक्षतु॒ गच्छे॑म सु॒कृतो॑ व॒यम्। येनेन्द्रा॑य स॒मभ॑रः॒ पयाꣳ॑स्युत्त॒\-मेन॑ ह॒विषा॑ जातवेदः। तेना᳚ग्ने॒ त्वमु॒त व॑र्धये॒मꣳ स॑जा॒ताना॒ꣴ॒ श्रैष्ठ्य॒ आ धे᳚ह्येनम्। य॒ज्ञ॒हनो॒ वै दे॒वा य॑ज्ञ॒मुषः॑॥१३॥

%3.5.4.3
स॒न्ति॒ त ए॒षु लो॒केष्वा॑सत आ॒ददा॑ना विमथ्ना॒ना यो ददा॑ति॒ यो यज॑ते॒ तस्य॑। ये दे॒वा य॑ज्ञ॒हनः॑ पृथि॒व्यामध्यास॑ते॒ ये अ॒न्तरि॑क्षे॒ ये दि॒वीत्या॑हे॒माने॒व लो॒काꣴस्ती॒र्त्वा सगृ॑हः॒ सप॑शुः सुव॒र्गं लो॒कमे॒त्यप॒ वै सोमे॑नेजा॒नाद्दे॒वता᳚श्च य॒ज्ञश्च॑ क्रामन्त्याग्ने॒यं पञ्च॑कपालमुदवसा॒नीयं॒ निर्व॑पेद॒ग्निः सर्वा॑ दे॒वताः᳚॥१४॥

%3.5.4.4
पाङ्क्तो॑ य॒ज्ञो दे॒वता᳚श्चै॒व य॒ज्ञं चाव॑ रुन्द्धे गाय॒त्रो वा अ॒ग्निर्गा॑य॒त्रछ॑न्दा॒स्तं छन्द॑सा॒ व्य॑र्धयति॒ यत्पञ्च॑कपालं क॒रोत्य॒ष्टाक॑पालः का॒र्यो᳚\-ऽष्टाक्ष॑रा गाय॒त्री गा॑य॒त्रो᳚\-ऽग्निर्गा॑य॒त्रछ॑न्दाः॒ स्वेनै॒वैनं॒ छन्द॑सा॒ सम॑र्धयति प॒ङ्क्त्यौ॑ याज्यानुवा॒क्ये॑ भवतः॒ पाङ्क्तो॑ य॒ज्ञस्तेनै॒व य॒ज्ञान्नैति॑॥१५॥

%3.5.5.0
{\anuvakamend[{स॒वि॒त॒र्दे॒वा य॑ज्ञ॒मुषः॒ सर्वा॑ दे॒वता॒स्त्रिच॑त्वारिꣳशच्च}]}%॥४॥

%3.5.5.1
सूर्यो॑ मा दे॒वो दे॒वेभ्यः॑ पातु वा॒युर॒न्तरि॑क्षा॒द्यज॑मानो॒\-ऽग्निर्मा॑ पातु॒ चक्षु॑षः। सक्ष॒ शूष॒ सवि॑त॒र्विश्व॑चर्\mbox{}षण ए॒तेभिः॑ सोम॒ नाम॑भिर्विधेम ते॒ तेभिः॑ सोम॒ नाम॑भिर्विधेम ते। अ॒हम्प॒रस्ता॑द॒हम॒वस्ता॑द॒हं ज्योति॑षा॒ वि तमो॑ ववार। यद॒न्तरि॑क्षं॒ तदु॑ मे पि॒ताभू॑द॒हꣳ सूर्य॑मुभ॒यतो॑ ददर्\mbox{}शा॒हम्भू॑यासमुत्त॒मः स॑मा॒नाना᳚म्॥१६॥

%3.5.5.2
आ स॑मु॒द्रादा\-ऽन्तरि॑क्षात्प्र॒जाप॑तिरुद॒धिं च्या॑वया॒तीन्द्रः॒ प्र स्नौ॑तु म॒रुतो॑ वर्\mbox{}षय॒न्तून्न॑म्भय पृथि॒वीम्भि॒न्द्धीदं दि॒व्यं नभः॑। उ॒द्नो दि॒व्यस्य॑ नो दे॒हीशा॑नो॒ वि सृ॑जा॒ दृतिम्᳚। प॒शवो॒ वा ए॒ते यदा॑दि॒त्य ए॒ष रु॒द्रो यद॒ग्निरोष॑धीः॒ प्रास्या॒ग्नावा॑दि॒त्यं जु॑होति रु॒द्रादे॒व प॒शून॒न्तर्द॑धा॒त्यथो॒ ओष॑धीष्वे॒व प॒शून्॥१७॥

%3.5.5.3
प्रति॑ ष्ठापयति क॒विर्य॒ज्ञस्य॒ वि त॑नोति॒ पन्थां॒ नाक॑स्य पृ॒ष्ठे अधि॑ रोच॒ने दि॒वः। येन॑ ह॒व्यं वह॑सि॒ यासि॑ दू॒त इ॒तः प्रचे॑ता अ॒मुतः॒ सनी॑यान्। यास्ते॒ विश्वाः᳚ स॒मिधः॒ सन्त्य॑ग्ने॒ याः पृ॑थि॒व्याम्ब॒र्\mbox{}हिषि॒ सूर्ये॒ याः। तास्ते॑ गच्छ॒न्त्वाहु॑तिं घृ॒तस्य॑ देवाय॒ते यज॑मानाय॒ शर्म॑। आ॒शासा॑नः सु॒वीर्यꣳ॑ रा॒यस्पोष॒ꣴ॒ स्वश्वि॑यम्। बृह॒स्पति॑ना रा॒या स्व॒गाकृ॑तो॒ मह्यं॒ यज॑मानाय तिष्ठ॥१८॥

%3.5.6.0
{\anuvakamend[{स॒मा॒नाना॒मोष॑धीष्वे॒व प॒शून्मह्यं॒ यज॑माना॒यैक॑ञ्च}]}%॥५॥

%3.5.6.1
सं त्वा॑ नह्यामि॒ पय॑सा घृ॒तेन॒ सं त्वा॑ नह्याम्य॒प ओष॑धीभिः। सं त्वा॑ नह्यामि प्र॒जया॒हम॒द्य सा दी᳚क्षि॒ता स॑नवो॒ वाज॑म॒स्मे। प्रैतु॒ ब्रह्म॑ण॒स्पत्नी॒ वेदिं॒ वर्णे॑न सीदतु। अथा॒हम॑नुका॒मिनी॒ स्वे लो॒के वि॒शा इ॒ह। सु॒प्र॒जस॑स्त्वा व॒यꣳ सु॒पत्नी॒रुप॑ सेदिम। अग्ने॑ सपत्न॒दम्भ॑न॒मद॑ब्धासो॒ अदा᳚भ्यम्। इ॒मं वि ष्या॑मि॒ वरु॑णस्य॒ पाशम्᳚॥१९॥

%3.5.6.2
यमब॑ध्नीत सवि॒ता सु॒केतः॑। धा॒तुश्च॒ योनौ॑ सुकृ॒तस्य॑ लो॒के स्यो॒नं मे॑ स॒ह पत्या॑ करोमि। प्रेह्यु॒देह्यृ॒तस्य॑ वा॒मीरन्व॒ग्निस्ते\-ऽग्रं॑ नय॒त्वदि॑ति॒र्मध्यं॑ ददताꣳ रु॒द्राव॑सृष्टासि यु॒वा नाम॒ मा मा॑ हिꣳसी॒र्वसु॑भ्यो रु॒द्रेभ्य॑ आदि॒त्येभ्यो॒ विश्वे᳚भ्यो वो दे॒वेभ्यः॑ प॒न्नेज॑नीर्गृह्णामि य॒ज्ञाय॑ वः प॒न्नेज॑नीः सादयामि॒ विश्व॑स्य ते॒ विश्वा॑वतो॒ वृष्णि॑यावतः॥२०॥

%3.5.6.3
तवा᳚ग्ने वा॒मीरनु॑ सं॒दृशि॒ विश्वा॒ रेताꣳ॑सि धिषी॒यागं॑ दे॒वान् य॒ज्ञो नि दे॒वीर्दे॒वेभ्यो॑ य॒ज्ञम॑शिषन्न॒स्मिन्थ्सु॑न्व॒ति यज॑मान आ॒शिषः॒ स्वाहा॑कृताः समुद्रे॒ष्ठा ग॑न्ध॒र्वमा ति॑ष्ठ॒ता\-ऽनु॑। वात॑स्य॒ पत्म॑न्नि॒ड ई॑डि॒ताः॥२१॥

%3.5.7.0
{\anuvakamend[{पाशं॒ वृष्णि॑यावतस्त्रि॒ꣳ॒शच्च॑}]}%॥६॥

%3.5.7.1
व॒ष॒ट्का॒रो वै गा॑यत्रि॒यै शिरो᳚\-ऽच्छिन॒त्तस्यै॒ रसः॒ परा॑पत॒थ्स पृ॑थि॒वीम्प्रावि॑श॒थ्स ख॑दि॒रो॑\-ऽभव॒द्यस्य॑ खादि॒रः स्रु॒वो भव॑ति॒ छन्द॑सामे॒व रसे॒नाव॑ द्यति॒ सर॑सा अ॒स्याहु॑तयो भवन्ति तृ॒तीय॑स्यामि॒तो दि॒वि सोम॑ आसी॒त्तं गा॑य॒त्र्याह॑र॒त्तस्य॑ प॒र्णम॑च्छिद्यत॒ तत्प॒र्णो॑\-ऽभव॒त्तत्प॒र्णस्य॑ पर्ण॒त्वं यस्य॑ पर्ण॒मयी॑ जु॒हूः॥२२॥

%3.5.7.2
भव॑ति सौ॒म्या अ॒स्याहु॑तयो भवन्ति जु॒षन्ते᳚\-ऽस्य दे॒वा आहु॑तीर्दे॒वा वै ब्रह्म॑न्नवदन्त॒ तत्प॒र्ण उपा॑शृणोथ्सु॒श्रवा॒ वै नाम॒ यस्य॑ पर्ण॒मयी॑ जु॒हूर्भव॑ति॒ न पा॒पꣴ श्लोकꣳ॑ शृणोति॒ ब्रह्म॒ वै प॒र्णो विण्म॒रुतो\-ऽन्नं॒ विण्मा॑रु॒तो᳚\-ऽश्व॒त्थो यस्य॑ पर्ण॒मयी॑ जु॒हूर्भव॒त्याश्व॑त्थ्युप॒भृद्ब्रह्म॑णै॒वान्न॒मव॑ रु॒न्द्धे\-ऽथो॒ ब्रह्म॑॥२३॥

%3.5.7.3
ए॒व वि॒श्यध्यू॑हति रा॒ष्ट्रं वै प॒र्णो विड॑श्व॒त्थो यत्प॑र्ण॒मयी॑ जु॒हूर्भव॒त्याश्व॑त्थ्युप॒भृद्रा॒ष्ट्रमे॒व वि॒श्यध्यू॑हति प्र॒जाप॑ति॒र्वा अ॑जुहो॒थ्सा यत्राहु॑तिः प्र॒त्यति॑ष्ठ॒त्ततो॒ विक॑ङ्कत॒ उद॑तिष्ठ॒त्ततः॑ प्र॒जा अ॑सृजत॒ यस्य॒ वैक॑ङ्कती ध्रु॒वा भव॑ति॒ प्रत्ये॒वास्याहु॑तयस्तिष्ठ॒न्त्यथो॒ प्रैव जा॑यत ए॒तद्वै स्रु॒चाꣳ रू॒पं यस्यै॒वꣳरू॑पाः॒ स्रुचो॒ भव॑न्ति॒ सर्वा᳚ण्ये॒वैनꣳ॑ रू॒पाणि॑ पशू॒नामुप॑ तिष्ठन्ते॒ नास्याप॑रूपमा॒त्मञ्जा॑यते॥२४॥

%3.5.8.0
{\anuvakamend[{जु॒हूरथो॒ ब्रह्म॑ स्रु॒चाꣳ स॒प्तद॑श च}]}%॥७॥

%3.5.8.1
उ॒प॒या॒मगृ॑हीतो\-ऽसि प्र॒जाप॑तये त्वा॒ ज्योति॑ष्मते॒ ज्योति॑ष्मन्तं गृह्णामि॒ दक्षा॑य दक्ष॒वृधे॑ रा॒तं दे॒वेभ्यो᳚\-ऽग्निजि॒ह्वेभ्य॑\-स्त्वर्ता॒युभ्य॒ इन्द्र॑ज्येष्ठेभ्यो॒ वरु॑णराजभ्यो॒ वाता॑पिभ्यः प॒र्जन्या᳚त्मभ्यो दि॒वे त्वा॒न्तरि॑क्षाय त्वा पृथि॒व्यै त्वापे᳚न्द्र द्विष॒तो मनो\-ऽप॒ जिज्या॑सतो ज॒ह्यप॒ यो नो॑\-ऽराती॒यति॒ तं ज॑हि प्रा॒णाय॑ त्वापा॒नाय॑ त्वा व्या॒नाय॑ त्वा स॒ते त्वास॑ते त्वा॒द्भ्यस्त्वौष॑धीभ्यो॒ विश्वे᳚भ्यस्त्वा भू॒तेभ्यो॒ यतः॑ प्र॒जा अक्खि॑द्रा॒ अजा॑यन्त॒ तस्मै᳚ त्वा प्र॒जाप॑तये विभू॒दाव्ने॒ ज्योति॑ष्मते॒ ज्योति॑ष्मन्तं जुहोमि॥२५॥

%3.5.9.0
{\anuvakamend[{ओष॑धीभ्य॒श्चतु॑र्दश च}]}%॥८॥

%3.5.9.1
यां वा अ॑ध्व॒र्युश्च॒ यज॑मानश्च दे॒वता॑मन्तरि॒तस्तस्या॒ आ वृ॑श्च्येते प्राजाप॒त्यं द॑धिग्र॒हं गृ॑ह्णीयात्प्र॒जाप॑तिः॒ सर्वा॑ दे॒वता॑ दे॒वता᳚भ्य ए॒व नि ह्नु॑वाते ज्ये॒ष्ठो वा ए॒ष ग्रहा॑णां॒ यस्यै॒ष गृ॒ह्यते॒ ज्यैष्ठ्य॑मे॒व ग॑च्छति॒ सर्वा॑सां॒ वा ए॒तद्दे॒वता॑नाꣳ रू॒पं यदे॒ष ग्रहो॒ यस्यै॒ष गृ॒ह्यते॒ सर्वा᳚ण्ये॒वैनꣳ॑ रू॒पाणि॑ पशू॒नामुप॑ तिष्ठन्त उपया॒मगृ॑हीतः॥२६॥

%3.5.9.2
अ॒सि॒ प्र॒जाप॑तये त्वा॒ ज्योति॑ष्मते॒ ज्योति॑ष्मन्तं गृह्णा॒मीत्या॑ह॒ ज्योति॑रे॒वैनꣳ॑ समा॒नानां᳚ करोत्यग्निजि॒ह्वेभ्य॑स्त्वर्ता॒युभ्य॒ इत्या॑है॒ताव॑ती॒र्वै दे॒वता॒स्ताभ्य॑ ए॒वैन॒ꣳ॒ सर्वा᳚भ्यो गृह्णा॒त्यपे᳚न्द्र द्विष॒तो मन॒ इत्या॑ह॒ भ्रातृ॑व्यापनुत्त्यै प्रा॒णाय॑ त्वापा॒नाय॒ त्वेत्या॑ह प्रा॒णाने॒व यज॑माने दधाति॒ तस्मै᳚ त्वा प्र॒जाप॑तये विभू॒दाव्ने॒ ज्योति॑ष्मते॒ ज्योति॑ष्मन्तं जुहोमि॥२७॥

%3.5.9.3
इत्या॑ह प्र॒जाप॑तिः॒ सर्वा॑ दे॒वताः॒ सर्वा᳚भ्य ए॒वैनं॑ दे॒वता᳚भ्यो जुहोत्याज्यग्र॒हं गृ॑ह्णीया॒त्तेज॑स्कामस्य॒ तेजो॒ वा आज्य॑न्तेज॒स्व्ये॑व भ॑वति सोमग्र॒हं गृ॑ह्णीयाद्ब्रह्मवर्च॒सका॑मस्य ब्रह्मवर्च॒सं वै सोमो᳚ ब्रह्मवर्च॒स्ये॑व भ॑वति दधिग्र॒हं गृ॑ह्णीयात्प॒शुका॑म॒स्योर्ग्वै दध्यूर्क्प॒शव॑ ऊ॒र्जैवास्मा॒ ऊर्जं॑ प॒शूनव॑ रुन्द्धे॥२८॥

%3.5.10.0
{\anuvakamend[{उ॒प॒या॒मगृ॑हीतो जुहोमि॒ त्रिच॑त्वारिꣳशच्च}]}%॥९॥

%3.5.10.1
त्वे क्रतु॒मपि॑ वृञ्जन्ति॒ विश्वे॒ द्विर्यदे॒ते त्रिर्भव॒न्त्यूमाः᳚। स्वा॒दोः स्वादी॑यः स्वा॒दुना॑ सृजा॒ समत॑ ऊ॒ षु मधु॒ मधु॑ना॒भि यो॑धि। उ॒प॒या॒मगृ॑हीतो\-ऽसि प्र॒जाप॑तये त्वा॒ जुष्टं॑ गृह्णाम्ये॒ष ते॒ योनिः॑ प्र॒जाप॑तये त्वा। प्रा॒ण॒ग्र॒हान्गृ॑ह्णात्ये॒ताव॒द्वा अ॑स्ति॒ याव॑दे॒ते ग्रहाः॒ स्तोमा॒श्छन्दाꣳ॑सि पृ॒ष्ठानि॒ दिशो॒ याव॑दे॒वास्ति॒ तत्॥२९॥

%3.5.10.2
अव॑ रुन्द्धे ज्ये॒ष्ठा वा ए॒तान्ब्रा᳚ह्म॒णाः पु॒रा वि॒द्वाम॑क्र॒न्तस्मा॒त्तेषा॒ꣳ॒ सर्वा॒ दिशो॒\-ऽभिजि॑ता अभूव॒न् यस्यै॒ते गृ॒ह्यन्ते॒ ज्यैष्ठ्य॑मे॒व ग॑च्छत्य॒भि दिशो॑ जयति॒ पञ्च॑ गृह्यन्ते॒ पञ्च॒ दिशः॒ सर्वा᳚स्वे॒व दि॒क्ष्वृ॑ध्नुवन्ति॒ नव॑नव गृह्यन्ते॒ नव॒ वै पुरु॑षे प्रा॒णाः प्रा॒णाने॒व यज॑मानेषु दधति प्राय॒णीये॑ चोदय॒नीये॑ च गृह्यन्ते प्रा॒णा वै प्रा॑णग्र॒हाः॥३०॥

%3.5.10.3
प्रा॒णैरे॒व प्र॒यन्ति॑ प्रा॒णैरुद्य॑न्ति दश॒मे\-ऽह॑न्गृह्यन्ते प्रा॒णा वै प्रा॑णग्र॒हाः प्रा॒णेभ्यः॒ खलु॒ वा ए॒तत्प्र॒जा य॑न्ति॒ यद्वा॑मदे॒व्यं योने॒श्च्यव॑ते दश॒मे\-ऽह॑न्वामदे॒व्यं योने᳚श्च्यवते॒ यद्द॑श॒मे\-ऽह॑न्गृ॒ह्यन्ते᳚ प्रा॒णेभ्य॑ ए॒व तत्प्र॒जा न य॑न्ति॥३१॥

%3.5.11.0
{\anuvakamend[{तत्प्रा॑णग्र॒हाः स॒प्तत्रिꣳ॑शच्च}]}%॥10॥

%3.5.11.1
प्र दे॒वं दे॒व्या धि॒या भर॑ता जा॒तवे॑दसम्। ह॒व्या नो॑ वक्षदानु॒षक्। अ॒यमु॒ ष्य प्र दे॑व॒युर्\mbox{}होता॑ य॒ज्ञाय॑ नीयते। रथो॒ न योर॒भीवृ॑तो॒ घृणी॑वाञ्चेतति॒ त्मना᳚। अ॒यम॒ग्निरु॑रुष्यत्य॒मृता॑दिव॒ जन्म॑नः। सह॑सश्चि॒थ्सही॑यां दे॒वो जी॒वात॑वे कृ॒तः। इडा॑यास्त्वा प॒दे व॒यं नाभा॑ पृथि॒व्या अधि॑। जात॑वेदो॒ नि धी॑म॒ह्यग्ने॑ ह॒व्याय॒ वोढ॑वे।॥३२॥

%3.5.11.2
अग्ने॒ विश्वे॑भिः स्वनीक दे॒वैरूर्णा॑वन्तम्प्रथ॒मः सी॑द॒ योनिम्᳚। कु॒ला॒यिनं॑ घृ॒तव॑न्तꣳ सवि॒त्रे य॒ज्ञं न॑य॒ यज॑मानाय सा॒धु। सीद॑ होतः॒ स्व उ॑ लो॒के चि॑कि॒त्वान्थ्सा॒दया॑ य॒ज्ञꣳ सु॑कृ॒तस्य॒ योनौ᳚। दे॒वा॒वीर्दे॒वान् ह॒विषा॑ यजा॒स्यग्ने॑ बृ॒हद्यज॑माने॒ वयो॑ धाः। नि होता॑ होतृ॒षद॑ने॒ विदा॑नस्त्वे॒षो दी॑दि॒वाꣳ अ॑सदथ्सु॒दक्षः॑। अद॑ब्धव्रतप्रमति॒र्वसि॑ष्ठः सहस्रम्भ॒रः शुचि॑जिह्वो अ॒ग्निः। त्वं दू॒तस्त्वम्॥३३॥

%3.5.11.3
उ॒ नः॒ प॒र॒स्पास्त्वं वस्य॒ आ वृ॑षभ प्रणे॒ता। अग्ने॑ तो॒कस्य॑ न॒स्तने॑ त॒नूना॒मप्र॑युच्छ॒न्दीद्य॑द्बोधि गो॒पाः। अ॒भि त्वा॑ देव सवित॒रीशा॑नं॒ वार्या॑णाम्। सदा॑वन्भा॒गमी॑महे। म॒ही द्यौः पृ॑थि॒वी च॑ न इ॒मं य॒ज्ञम्मि॑मिक्षताम्। पि॒पृ॒तां नो॒ भरी॑मभिः। त्वाम॑ग्ने॒ पुष्क॑रा॒दध्यथ॑र्वा॒ निर॑मन्थत। मू॒र्ध्नो विश्व॑स्य वा॒घतः॑। तमु॑॥३४॥

%3.5.11.4
त्वा॒ द॒ध्यङ्ङृषिः॑ पु॒त्र ई॑धे॒ अथ॑र्वणः। वृ॒त्र॒हणं॑ पुरन्द॒रम्। तमु॑ त्वा पा॒थ्यो वृषा॒ समी॑धे दस्यु॒हन्त॑मम्। ध॒नं॒ज॒यꣳ रणे॑रणे। उ॒त ब्रु॑वन्तु ज॒न्तव॒ उद॒ग्निर्वृ॑त्र॒हाज॑नि। ध॒नं॒ज॒यो रणे॑रणे। आ यꣳ हस्ते॒ न खा॒दिन॒ꣳ॒ शिशुं॑ जा॒तं न बिभ्र॑ति। वि॒शाम॒ग्निꣴ स्व॑ध्व॒रम्। प्र दे॒वं दे॒ववी॑तये॒ भर॑ता वसु॒वित्त॑मम्। आ स्वे योनौ॒ नि षी॑दतु। आ॥३५॥


%3.5.11.5
जा॒तं जा॒तवे॑दसि प्रि॒यꣳ शि॑शी॒ताति॑थिम्। स्यो॒न आ गृ॒हप॑तिम्। अ॒ग्निना॒ऽग्निः समि॑ध्यते क॒विर्गृ॒हप॑ति॒र्युवा᳚। ह॒व्य॒वाड्जु॒ह्वा᳚स्यः। त्वꣴ ह्य॑ग्ने अ॒ग्निना॒ विप्रो॒ विप्रे॑ण॒ सन्थ्स॒ता। सखा॒ सख्या॑ समि॒ध्यसे᳚। तम्म॑र्जयन्त सु॒क्रतुं॑ पुरो॒यावा॑नमा॒जिषु॑। स्वेषु॒ क्षये॑षु वा॒जिनम्᳚। य॒ज्ञेन॑ य॒ज्ञम॑यजन्त दे॒वास्तानि॒ धर्मा॑णि प्रथ॒मान्या॑सन्न्। ते ह॒ नाक॑म्महि॒मानः॑ सचन्ते॒ यत्र॒ पूर्वे॑ सा॒ध्याः सन्ति॑ दे॒वाः॥३६॥

%4.1.0.0

%4.1.0.0
{\anuvakamend[{वोढ॑वे दू॒तस्त्वन्तमु॑ सीद॒त्वा यत्र॑ च॒त्वारि॑ च}]}%॥11॥

{\anuvakamend[यु॒ञ्जा॒न इ॒माम॑गृभ्णं दे॒वस्य॒ सन्ते॒ वि पाज॑सा॒ वस॑वस्त्वा॒ समा᳚स्त्वो॒र्ध्वा अ॒स्याकू॑तिं॒ यद॑ग्ने॒ यान्यग्ने॒ यं य॒ज्ञमेका॑दश॥11॥ यु॒ञ्जा॒नो वर्म॑ च स्थ आदि॒त्यास्त्वा॒ भार॑ती॒ स्वाꣳ अहꣳ षट्च॑त्वारिꣳशत्॥46॥ यु॒ञ्जा॒नो वाजे॑वाजे॥]}
%%% END KANDAM

