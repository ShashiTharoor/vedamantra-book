\sect{प्रथमः प्रश्नः}\setcounter{anuvakam}{0}
\dnsub{तैत्तिरीयसंहितायां प्रथमकाण्डे प्रथमः प्रश्नः}
%1.1.1.1
इ॒षे त्वो॒र्जे त्वा॑ वा॒यवः॑ स्थोपा॒यवः॑ स्थ दे॒वो वः॑ सवि॒ता प्रार्प॑यतु॒ श्रेष्ठ॑तमाय॒ कर्म॑ण॒ आ प्या॑यध्वमघ्निया देवभा॒गमूर्ज॑स्वती॒ पय॑स्वतीः प्र॒जाव॑तीरनमी॒वा अ॑य॒क्ष्मा मा वः॑ स्ते॒न ई॑शत॒ मा\-ऽघशꣳ॑सो रु॒द्रस्य॑ हे॒तिः परि॑ वो वृणक्तु ध्रु॒वा अ॒स्मिन्गोप॑तौ स्यात ब॒ह्वीर्यज॑मानस्य प॒शून्पा॑हि॥~(१)

%1.1.2.0
{\anuvakamend[{इ॒षे त्रिच॑त्वारिꣳशत्}]}

%1.1.2.1
य॒ज्ञस्य॑ घो॒षद॑सि॒ प्रत्यु॑ष्ट॒ꣳ॒ रक्ष॒ प्रत्यु॑ष्टा॒ अरा॑तय॒ प्रेयम॑गाद्धि॒षणा॑ ब॒र्॒हिरच्छ॒ मनु॑ना कृ॒ता स्व॒धया॒ वित॑ष्टा॒ त आव॑हन्ति क॒वय॑ पु॒रस्ता᳚द्दे॒वेभ्यो॒ जुष्ट॑मि॒ह ब॒र्॒हिरा॒सदे॑ दे॒वानां᳚ परिषू॒तम॑सि व॒र्॒षवृ॑द्धमसि॒ देव॑बर्\mbox{}हि॒र्मा त्वा॒\-ऽन्वङ्मा ति॒र्यक्पर्व॑ ते राध्यासमाच्छे॒त्ता ते॒ मा रि॑षं॒ देव॑बर्\mbox{}हिः श॒तव॑ल्\mbox{}शं॒ वि रो॑ह स॒हस्र॑वल्\mbox{}शा॒~(२)

%1.1.2.2
वि व॒यꣳ रु॑हेम पृथि॒व्याः स॒म्पृच॑ पाहि सुस॒म्भृता᳚ त्वा॒ सम्भ॑रा॒म्यदि॑त्यै॒ रास्ना॑\-ऽसीन्द्रा॒ण्यै स॒न्नह॑नं पू॒षा ते᳚ ग्र॒न्थिं ग्र॑थ्नातु॒ स ते॒ मा\-ऽ\-ऽस्था॒दिन्द्र॑स्य त्वा बा॒हुभ्या॒मुद्य॑च्छे॒ बृह॒स्पते᳚र्मू॒र्ध्ना ह॑राम्यु॒र्व॑न्तरि॑क्ष॒मन्वि॑हि देवङ्ग॒मम॑सि॥~(३)

%1.1.3.0
{\anuvakamend[{स॒हस्र॑वल्\mbox{}शा अ॒ष्टात्रिꣳ॑शच्च}]}

%1.1.3.1
शुन्ध॑ध्वं॒ दैव्या॑य॒ कर्म॑णे देवय॒ज्यायै॑ मात॒रिश्व॑नो घ॒र्मो॑\-ऽसि॒ द्यौर॑सि पृथि॒व्य॑सि वि॒श्वधा॑या असि पर॒मेण॒ धाम्ना॒ दृꣳह॑स्व॒ मा ह्वा॒र्वसू॑नां प॒वित्र॑मसि श॒तधा॑रं॒ वसू॑नां प॒वित्र॑मसि स॒हस्र॑धारꣳ हु॒तः स्तो॒को हु॒तो द्र॒फ्सो᳚\-ऽग्नये॑ बृह॒ते नाका॑य॒ स्वाहा॒ द्यावा॑पृथि॒वीभ्या॒ꣳ॒ सा वि॒श्वायुः॒ सा वि॒श्वव्य॑चाः॒ सा वि॒श्वक॑र्मा॒ सम्पृ॑च्यध्वमृतावरीरू॒र्मिणी॒र्मधु॑मत्तमा म॒न्द्रा धन॑स्य सा॒तये॒ सोमे॑न॒ त्वा\-ऽ\-ऽत॑न॒च्मीन्द्रा॑य॒ दधि॒ विष्णो॑ ह॒व्यꣳ र॑क्षस्व॥~(४)

%1.1.4.0
{\anuvakamend[{सोमे॑ना॒ष्टौ च॑}]}

%1.1.4.1
कर्म॑णे वां दे॒वेभ्यः॑ शकेयं॒ वेषा॑य त्वा॒ प्रत्यु॑ष्ट॒ꣳ॒ रक्ष॒ प्रत्यु॑ष्टा॒ अरा॑तयो॒ धूर॑सि॒ धूर्व॒ धूर्व॑न्तं॒ धूर्व॒ तं यो᳚\-ऽस्मान्धूर्व॑ति॒ तं धू᳚र्व॒ यं व॒यं धूर्वा॑म॒स्त्वं दे॒वाना॑मसि॒ सस्नि॑तमं॒ पप्रि॑तमं॒ जुष्ट॑तमं॒ वह्नि॑तमं देव॒हूत॑म॒मह्रु॑तमसि हवि॒र्धानं॒ दृꣳह॑स्व॒ मा ह्वा᳚र्मि॒त्रस्य॑ त्वा॒ चक्षु॑षा॒ प्रेक्षे॒ मा भेर्मा सं वि॑क्था॒ मा त्वा॑~(५)

%1.1.4.2
हिꣳसिषमु॒रु वाता॑य दे॒वस्य॑ त्वा सवि॒तुः प्र॑स॒वे᳚\-ऽश्विनो᳚र्बा॒हु\-भ्यां᳚ पू॒ष्णो हस्ता᳚भ्याम॒ग्नये॒ जुष्टं॒ निर्व॑पाम्य॒ग्नी\-षोमा᳚भ्यामि॒दं दे॒वाना॑मि॒दमु॑ नः स॒ह स्फा॒त्यै त्वा॒ नारा᳚त्यै॒ सुव॑र॒भि वि ख्ये॑षं वैश्वान॒रं ज्योति॒र्दृꣳह॑न्ता॒न्दुर्या॒ द्यावा॑पृथि॒व्योरु॒र्व॑न्तरि॑क्ष॒मन्वि॒ह्यदि॑त्यास्त्वो॒पस्थे॑ सादया॒म्यग्ने॑ ह॒व्यꣳ र॑क्षस्व॥~(६)

%1.1.5.0
{\anuvakamend[{मा त्वा॒ षट्च॑त्वारिꣳशच्च}]}

%1.1.5.1
दे॒वो वः॑ सवि॒तोत्पु॑ना॒त्वच्छि॑द्रेण प॒वित्रे॑ण॒ वसोः॒ सूर्य॑स्य र॒श्मिभि॒रापो॑ देवीरग्रेपुवो अग्रेगु॒वो\-ऽग्र॑ इ॒मं य॒ज्ञं न॑य॒ताग्रे॑ य॒ज्ञप॑तिं धत्त यु॒ष्मानिन्द्रो॑\-ऽवृणीत वृत्र॒तूर्ये॑ यू॒यमिन्द्र॑मवृणीध्वं वृत्र॒तूर्ये॒ प्रोक्षि॑ताः स्था॒ग्नये॑ वो॒ जुष्टं॒ प्रोक्षा᳚म्य॒ग्नीषोमा᳚भ्या॒ꣳ॒ शुन्ध॑ध्वं॒ दैव्या॑य॒ कर्म॑णे देवय॒ज्याया॒ अव॑धूत॒ꣳ॒ रक्षो\-ऽव॑धूता॒ अरा॑त॒यो\-ऽदि॑त्या॒स्त्वग॑सि॒ प्रति॑ त्वा~(७)

%1.1.5.2
पृथि॒वी वे᳚त्त्वधि॒षव॑णमसि वानस्प॒त्यं प्रति॒ त्वा\-ऽदि॑त्या॒स्त्वग्वे᳚त्त्व॒ग्नेस्त॒नूर॑सि वा॒चो वि॒सर्ज॑नं दे॒ववी॑तये त्वा गृह्णा॒म्यद्रि॑रसि वानस्प॒त्यः स इ॒दं दे॒वेभ्यो॑ ह॒व्यꣳ सु॒शमि॑ शमि॒ष्वेष॒मा व॒दोर्ज॒मा व॑द द्यु॒मद्व॑दत व॒यꣳ स॑ङ्घा॒तं जे᳚ष्म व॒र्॒षवृ॑द्धमसि॒ प्रति॑ त्वा व॒र्॒षवृ॑द्धं वेत्तु॒ परा॑पूत॒ꣳ॒ रक्ष॒ परा॑पूता॒ अरा॑तयो॒ रक्ष॑सां भा॒गो॑\-ऽसि वा॒युर्वो॒ विवि॑नक्तु दे॒वो वः॑ सवि॒ता हिर॑ण्यपाणि॒ प्रति॑ गृह्णातु॥~(८)

%1.1.6.0
{\anuvakamend[{त्वा॒ भा॒ग एका॑दश च}]}

%1.1.6.1
अव॑धूत॒ꣳ॒ रक्षो\-ऽव॑धूता॒ अरा॑त॒यो\-ऽदि॑त्या॒स्त्वग॑सि॒ प्रति॑ त्वा पृथि॒वी वे᳚त्तु दि॒वः स्क॑म्भ॒निर॑सि॒ प्रति॒ त्वा\-ऽदि॑त्या॒स्त्वग्वे᳚त्तु धि॒षणा॑\-ऽसि पर्व॒त्या प्रति॑ त्वा दि॒वः स्क॑म्भ॒निर्वे᳚त्तु धि॒षणा॑\-ऽसि पार्वते॒यी प्रति॑ त्वा पर्व॒तिर्वे᳚त्तु दे॒वस्य॑ त्वा सवि॒तुः प्र॑स॒वे᳚\-ऽश्विनो᳚र्बा॒हु\-भ्यां᳚ पू॒ष्णो हस्ता᳚भ्या॒मधि॑वपामि धा॒न्य॑मसि धिनु॒हि दे॒वान्प्रा॒णाय॑ त्वा\-ऽपा॒नाय॑ त्वा व्या॒नाय॑ त्वा दी॒र्घामनु॒ प्रसि॑ति॒मायु॑षे धां दे॒वो वः॑ सवि॒ता हिर॑ण्यपाणि॒ प्रति॑ गृह्णातु॥~(९)

%1.1.7.0
{\anuvakamend[{प्रा॒णाय॑ त्वा॒ पञ्च॑दश च}]}

%1.1.7.1
धृष्टि॑रसि॒ ब्रह्म॑ य॒च्छापा᳚\-ऽग्ने॒\-ऽग्निमा॒मादं॑ जहि॒ निष्क्र॒व्यादꣳ॑ से॒धा दे॑व॒यजं॑ वह॒ निर्द॑ग्ध॒ꣳ॒ रक्षो॒ निर्द॑ग्धा॒ अरा॑तयो ध्रु॒वम॑सि पृथि॒वीं दृ॒ꣳ॒हाऽऽयु॑र्दृꣳह प्र॒जां दृꣳ॑ह सजा॒तान॒स्मै यज॑मानाय॒ पर्यू॑ह ध॒र्त्रम॑स्य॒न्तरि॑क्षं दृꣳह प्रा॒णं दृꣳ॑हापा॒नं दृꣳ॑ह सजा॒ता\-न॒स्मै यज॑मानाय॒ पर्यू॑ह ध॒रुण॑मसि॒ दिवं॑ दृꣳह॒ चक्षु॑र्~(१०)

%1.1.7.2
दृꣳह॒ श्रोत्रं॑ दृꣳह सजा॒तान॒स्मै यज॑मानाय॒ पर्यू॑ह॒ धर्मा॑\-ऽसि॒ दिशो॑ दृꣳह॒ योनिं॑ दृꣳह प्र॒जां दृꣳ॑ह सजा॒तान॒स्मै यज॑मानाय॒ पर्यू॑ह॒ चितः॑ स्थ प्र॒जाम॒स्मै र॒यिम॒स्मै स॑जा॒तान॒स्मै यज॑मानाय॒ पर्यू॑ह॒ भृगू॑णा॒मङ्गि॑रसां॒ तप॑सा तप्यध्वं॒ यानि॑ घ॒र्मे क॒पाला᳚न्युपचि॒न्वन्ति॑ वे॒धसः॑। पू॒ष्णस्तान्यपि॑ व्र॒त इ॑न्द्रवा॒यू वि मु॑ञ्चताम्॥~(११)

%1.1.8.0
{\anuvakamend[{चक्षु॑र॒ष्टाच॑त्वारिꣳशच्च}]}

%1.1.8.1
सं व॑पामि॒ समापो॑ अ॒द्भिर॑ग्मत॒ समोष॑धयो॒ रसे॑न॒ सꣳ रे॒वती॒र्जग॑तीभि॒र्मधु॑मती॒र्मधु॑मतीभिः सृज्यध्वम॒द्भ्यः परि॒ प्रजा॑ताः स्थ॒ सम॒द्भिः पृ॑च्यध्वं॒ जन॑यत्यै त्वा॒ सं यौ᳚म्य॒ग्नये᳚ त्वा॒\-ऽग्नीषोमा᳚भ्यां म॒खस्य॒ शिरो॑\-ऽसि घ॒र्मो॑\-ऽसि वि॒श्वायु॑रु॒रु प्र॑थस्वो॒रु ते॑ य॒ज्ञप॑तिः प्रथतां॒ त्वचं॑ गृह्णीष्वा॒\-ऽन्तरि॑त॒ꣳ॒ रक्षो॒\-ऽन्तरि॑ता॒ अरा॑तयो दे॒वस्त्वा॑ सवि॒ता श्र॑पयतु॒ वर्\mbox{}षि॑ष्ठे॒ अधि॒ नाके॒\-ऽग्निस्ते॑ त॒नुवं॒ मा\-ऽति॑ धा॒गग्ने॑ ह॒व्यꣳ र॑क्षस्व॒ सं ब्रह्म॑णा पृच्यस्वैक॒ताय॒ स्वाहा᳚ द्वि॒ताय॒ स्वाहा᳚ त्रि॒ताय॒ स्वाहा᳚॥~(१२)

%1.1.9.0
{\anuvakamend[{स॒वि॒ता द्वाविꣳ॑शतिश्च}]}

%1.1.9.1
आद॑द॒ इन्द्र॑स्य बा॒हुर॑सि॒ दक्षि॑णः स॒हस्र॑भृष्टिः श॒तते॑जा वा॒युर॑सि ति॒ग्मते॑जा॒ पृथि॑वि देवयज॒न्योष॑ध्यास्ते॒ मूलं॒ मा हिꣳ॑सिष॒मप॑हतो॒\-ऽररु॑ पृथि॒व्यै व्र॒जं ग॑च्छ गो॒स्थानं॒ वर्\mbox{}ष॑तु ते॒ द्यौर्ब॑धा॒न दे॑व सवितः पर॒मस्यां᳚ परा॒वति॑ श॒तेन॒ पाशै॒र्यो᳚\-ऽस्मान्द्वेष्टि॒ यं च॑ व॒यं द्वि॒ष्मस्तमतो॒ मा मौ॒गप॑हतो॒\-ऽररु॑ पृथि॒व्यै दे॑व॒यज॑न्यै व्र॒जं~(१३)

%1.1.9.2
ग॑च्छ गो॒स्थानं॒ वर्\mbox{}ष॑तु ते॒ द्यौर्ब॑धा॒न दे॑व सवितः पर॒मस्यां᳚ परा॒वति॑ श॒तेन॒ पाशै॒र्यो᳚\-ऽस्मान्द्वेष्टि॒ यं च॑ व॒यं द्वि॒ष्मस्तमतो॒ मा मौ॒गप॑हतो॒\-ऽररु॑ पृथि॒व्या अदे॑वयजनो व्र॒जं ग॑च्छ गो॒स्थानं॒ वर्\mbox{}ष॑तु ते॒ द्यौर्ब॑धा॒न दे॑व सवितः पर॒मस्यां᳚ परा॒वति॑ श॒तेन॒ पाशै॒र्यो᳚\-ऽस्मान्द्वेष्टि॒ यं च॑ व॒यं द्वि॒ष्मस्तमतो॒ मा~(१४)

%1.1.9.3
मौ॑ग॒ररु॑स्ते॒ दिवं॒ मा स्का॒न्॒ वस॑वस्त्वा॒ परि॑गृह्णन्तु गाय॒त्रेण॒ छन्द॑सा रु॒द्रास्त्वा॒ परि॑गृह्णन्तु॒ त्रैष्टु॑भेन॒ छन्द॑सा\-ऽ\-ऽदि॒त्यास्त्वा॒ परि॑गृह्णन्तु॒ जाग॑तेन॒ छन्द॑सा दे॒वस्य॑ सवि॒तुः स॒वे कर्म॑ कृण्वन्ति वे॒धस॑ ऋ॒तम॑स्यृत॒सद॑नमस्यृत॒श्रीर॑सि॒ धा अ॑सि स्व॒धा अ॑स्यु॒र्वी चासि॒ वस्वी॑ चासि पु॒रा क्रू॒रस्य॑ वि॒सृपो॑ विरफ्शिन्नुदा॒दाय॑ पृथि॒वीं जी॒रदा॑नु॒र्यामैर॑यं च॒न्द्रम॑सि स्व॒धाभि॒स्तान्धीरा॑सो अनु॒दृश्य॑ यजन्ते॥~(१५)

%1.1.10.0
{\anuvakamend[{दे॒व॒यज॑न्यै व्र॒जन्तमतो॒ मा वि॑रफ्शि॒न्नेका॑दश च}]}

%1.1.10.1
प्रत्यु॑ष्ट॒ꣳ॒ रक्ष॒ प्रत्यु॑ष्टा॒ अरा॑तयो॒\-ऽग्नेर्व॒स्तेजि॑ष्ठेन॒ तेज॑सा॒ निष्ट॑पामि गो॒ष्ठं मा निर्मृ॑क्षं वा॒जिनं॑ त्वा सपत्नसा॒हꣳ सम्मा᳚र्ज्मि॒ वाचं॑ प्रा॒णं चक्षुः॒ श्रोत्रं॑ प्र॒जां योनिं॒ मा निर्मृ॑क्षं वा॒जिनीं᳚ त्वा सपत्नसा॒हीꣳ सम्मा᳚र्ज्म्या॒शासा॑ना सौमन॒सं प्र॒जाꣳ सौभा᳚ग्यं त॒नूम्। अ॒ग्नेरनु॑व्रता भू॒त्वा सन्न॑ह्ये सुकृ॒ताय॒ कम्। सु॒प्र॒जस॑स्त्वा व॒यꣳ सु॒पत्नी॒रुप॑~(१६)

%1.1.10.2
सेदिम। अग्ने॑ सपत्न॒दम्भ॑न॒मद॑ब्धासो॒ अदा᳚भ्यम्। इ॒मं विष्या॑मि॒ वरु॑णस्य॒ पाशं॒ यमब॑ध्नीत सवि॒ता सु॒शेवः॑। धा॒तुश्च॒ योनौ॑ सुकृ॒तस्य॑ लो॒के स्यो॒नं मे॑ स॒ह पत्या॑ करोमि। समायु॑षा॒ सम्प्र॒जया॒ सम॑ग्ने॒ वर्च॑सा॒ पुनः॑। सम्पत्नी॒ पत्या॒\-ऽहं ग॑च्छे॒ समा॒त्मा त॒नुवा॒ मम॑। म॒ही॒नां पयो॒\-ऽस्योष॑धीना॒ꣳ॒ रस॒स्तस्य॒ ते\-ऽक्षी॑यमाणस्य॒ निर्~(१७)

%1.1.10.3
व॑पामि मही॒नां पयो॒\-ऽस्योष॑धीना॒ꣳ॒ रसो\-ऽद॑ब्धेन त्वा॒ चक्षु॒षा\-ऽवे᳚क्षे सुप्रजा॒स्त्वाय॒ तेजो॑\-ऽसि॒ तेजो\-ऽनु॒ प्रेह्य॒ग्निस्ते॒ तेजो॒ मा वि नै॑द॒ग्नेर्जि॒ह्वा\-ऽसि॑ सु॒भूर्दे॒वानां॒ धाम्ने॑धाम्ने दे॒वेभ्यो॒ यजु॑षेयजुषे भव शु॒क्रम॑सि॒ ज्योति॑रसि॒ तेजो॑\-ऽसि दे॒वो वः॑ सवि॒तोत्पु॑ना॒त्वच्छि॑द्रेण प॒वित्रे॑ण॒ वसोः॒ सूर्य॑स्य र॒श्मिभिः॑ शु॒क्रं त्वा॑ शु॒क्रायां॒ धाम्ने॑धाम्ने दे॒वेभ्यो॒ यजु॑षेयजुषे गृह्णामि॒ ज्योति॑स्त्वा॒ ज्योति॑ष्य॒र्चिस्त्वा॒\-ऽर्चिषि॒ धाम्ने॑धाम्ने दे॒वेभ्यो॒ यजु॑षेयजुषे गृह्णामि॥~(१८)

%1.1.11.0
{\anuvakamend[{उप॒ नी र॒श्मिभिः॑ शु॒क्रꣳ षोड॑श च}]}

%1.1.11.1
कृष्णो᳚\-ऽस्याखरे॒ष्ठो᳚\-ऽग्नये᳚ त्वा॒ स्वाहा॒ वेदि॑रसि ब॒र्॒हिषे᳚ त्वा॒ स्वाहा॑ ब॒र्॒हिर॑सि स्रु॒ग्भ्यस्त्वा॒ स्वाहा॑ दि॒वे त्वा॒\-ऽन्तरि॑क्षाय त्वा पृथि॒व्यै त्वा᳚ स्व॒धा पि॒तृभ्य॒ ऊर्ग्भ॑व बर्\mbox{}हि॒षद्भ्य॑ ऊ॒र्जा पृ॑थि॒वीं ग॑च्छत॒ विष्णोः॒ स्तूपो॒\-ऽस्यूर्णा᳚म्रदसं त्वा स्तृणामि स्वास॒स्थं दे॒वेभ्यो॑ गन्ध॒र्वो॑\-ऽसि वि॒श्वाव॑सु॒र्विश्व॑स्मा॒दीष॑तो॒ यज॑मानस्य परि॒धिरि॒ड ई॑डि॒त इन्द्र॑स्य बा॒हुर॑सि॒~(१९)

%1.1.11.2
दक्षि॑णो॒ यज॑मानस्य परि॒धिरि॒ड ई॑डि॒तो मि॒त्रावरु॑णौ त्वोत्तर॒तः परि॑धत्तां ध्रु॒वेण॒ धर्म॑णा॒ यज॑मानस्य परि॒धिरि॒ड ई॑डि॒तः सूर्य॑स्त्वा पु॒रस्ता᳚त्पातु॒ कस्या᳚श्चिद॒भिश॑स्त्या वी॒तिहो᳚त्रं त्वा कवे द्यु॒मन्त॒ꣳ॒ समि॑धीम॒ह्यग्ने॑ बृ॒हन्त॑मध्व॒रे वि॒शो य॒न्त्रे स्थो॒ वसू॑नाꣳ रु॒द्राणा॑मादि॒त्याना॒ꣳ॒ सद॑सि सीद जु॒हूरु॑प॒भृद्ध्रु॒वा\-ऽसि॑ घृ॒ताची॒ नाम्ना᳚ प्रि॒येण॒ नाम्ना᳚ प्रि॒ये सद॑सि सीदै॒ता अ॑सदन्थ्सुकृ॒तस्य॑ लो॒के ता वि॑ष्णो पाहि पा॒हि य॒ज्ञं पा॒हि य॒ज्ञप॑तिं पा॒हि मां य॑ज्ञ॒नियम्᳚॥~(२०)

%1.1.12.0
{\anuvakamend[{बा॒हुर॑सि प्रि॒ये सद॑सि पञ्च॑दश च}]}

%1.1.12.1
भुव॑नमसि॒ वि प्र॑थ॒स्वाग्ने॒ यष्ट॑रि॒दं नमः॑। जुह्वेह्य॒ग्निस्त्वा᳚ ह्वयति देवय॒ज्याया॒ उप॑भृ॒देहि॑ दे॒वस्त्वा॑ सवि॒ता ह्व॑यति देवय॒ज्याया॒ अग्ना॑विष्णू॒ मा वा॒मव॑ क्रमिषं॒ वि जि॑हाथां॒ मा मा॒ सन्ता᳚प्तं लो॒कं मे॑ लोककृतौ कृणुतं॒ विष्णोः॒ स्थान॑मसी॒त इन्द्रो॑ अकृणोद्वी॒र्या॑णि समा॒रभ्यो॒र्ध्वो अ॑ध्व॒रो दि॑वि॒स्पृश॒मह्रु॑तो य॒ज्ञो य॒ज्ञप॑ते॒रिन्द्रा॑वा॒न्थ्स्वाहा॑ बृ॒हद्भाः पा॒हि मा᳚ऽग्ने॒ दुश्च॑रिता॒दा मा॒ सुच॑रिते भज म॒खस्य॒ शिरो॑\-ऽसि॒ सं ज्योति॑षा॒ ज्योति॑रङ्क्ताम्॥~(२१)

%1.1.13.0
{\anuvakamend[{अह्रु॑त॒ एक॑विꣳशतिश्च}]}

%1.1.13.1
वाज॑स्य मा प्रस॒वेनो᳚द्ग्रा॒भेणोद॑ग्रभीत्। अथा॑ स॒पत्ना॒ꣳ॒ इन्द्रो॑ मे निग्रा॒भेणाध॑राꣳ अकः। उ॒द्ग्रा॒भं च॑ निग्रा॒भं च॒ ब्रह्म॑ दे॒वा अ॑वीवृधन्न्। अथा॑ स॒पत्ना॑निन्द्रा॒ग्नी मे॑ विषू॒चीना॒न्व्य॑स्यताम्। वसु॑भ्यस्त्वा रु॒द्रेभ्य॑स्त्वा\-ऽ\-ऽदि॒त्येभ्य॑स्त्वा॒ऽक्तꣳ रिहा॑णा वि॒यन्तु॒ वयः॑। प्र॒जां योनिं॒ मा निर्मृ॑क्ष॒मा प्या॑यन्ता॒माप॒ ओष॑धयो म॒रुतां॒ पृष॑तयः स्थ॒ दिवं॑~(२२)

%1.1.13.2
गच्छ॒ ततो॑ नो॒ वृष्टि॒मेर॑य। आ॒यु॒ष्पा अ॑ग्ने॒\-ऽस्यायु॑र्मे पाहि चक्षु॒ष्पा अ॑ग्ने\-ऽसि॒ चक्षु॑र्मे पाहि ध्रु॒वा\-ऽसि॒ यं प॑रि॒धिं प॒र्यध॑त्था॒ अग्ने॑ देव प॒णिभि॑र्वी॒यमा॑णः। तन्त॑ ए॒तमनु॒ जोषं॑ भरामि॒ नेदे॒ष त्वद॑पचे॒तया॑तै य॒ज्ञस्य॒ पाथ॒ उप॒ समि॑तꣳ सꣴस्रा॒वभा॑गाः स्थे॒षा बृ॒हन्त॑ प्रस्तरे॒ष्ठा ब॑र्\mbox{}हि॒षद॑श्च~(२३)

%1.1.13.3
दे॒वा इ॒मां वाच॑म॒भि विश्वे॑ गृ॒णन्त॑ आ॒सद्या॒स्मिन्ब॒र्॒हिषि॑ मादयध्वम॒ग्नेर्वा॒मप॑न्नगृहस्य॒ सद॑सि सादयामि सु॒म्नाय॑ सुम्निनी सु॒म्ने मा॑ धत्तं धु॒रि धु॒र्यौ॑ पात॒मग्ने॑\-ऽदब्धायो\-ऽशीततनो पा॒हि मा॒\-ऽद्य दि॒वः पा॒हि प्रसि॑त्यै पा॒हि दुरि॑ष्ट्यै पा॒हि दु॑रद्म॒न्यै पा॒हि दुश्च॑रिता॒दवि॑षन्नः पि॒तुं कृ॑णु सु॒षदा॒ योनि॒ꣴ॒ स्वाहा॒ देवा॑ गातुविदो गा॒तुं वि॒त्वा गा॒तुमि॑त॒ मन॑सस्पत इ॒मं नो॑ देव दे॒वेषु॑ य॒ज्ञꣴ स्वाहा॑ वा॒चि स्वाहा॒ वाते॑ धाः॥~(२४)

%1.1.14.0
{\anuvakamend[{दिव॑ञ्च वि॒त्वा गा॒तुन्त्रयो॑दश च}]}

%1.1.14.1
उ॒भा वा॑मिन्द्राग्नी आहु॒वध्या॑ उ॒भा राध॑सः स॒ह मा॑द॒यध्यै᳚। उ॒भा दा॒तारा॑वि॒षाꣳ र॑यी॒णामु॒भा वाज॑स्य सा॒तये॑ हुवे वाम्। अश्र॑व॒ꣳ॒ हि भू॑रि॒दाव॑त्तरा वां॒ वि जा॑मातुरु॒त वा॑ घा स्या॒लात्। अथा॒ सोम॑स्य॒ प्रय॑ती यु॒वभ्या॒मिन्द्रा᳚ग्नी॒ स्तोमं॑ जनयामि॒ नव्यम्᳚। इन्द्रा᳚ग्नी नव॒तिं पुरो॑ दा॒सप॑त्नीरधूनुतम्। सा॒कमेके॑न॒ कर्म॑णा। शुचिं॒ नु स्तोमं॒ नव॑जातम॒द्येन्द्रा᳚ग्नी वृत्रहणा जु॒षेथा᳚म्॥~(२५)

%1.1.14.2
उ॒भा हि वाꣳ॑ सु॒हवा॒ जोह॑वीमि॒ ता वाजꣳ॑ स॒द्य उ॑श॒ते धेष्ठा᳚। व॒यमु॑ त्वा पथस्पते॒ रथं॒ न वाज॑सातये। धि॒ये पू॑षन्नयुज्महि। प॒थस्प॑थः॒ परि॑पतिं वच॒स्या कामे॑न कृ॒तो अ॒भ्या॑नड॒र्कम्। स नो॑ रासच्छु॒रुध॑श्च॒न्द्राग्रा॒ धियं॑ धियꣳ सीषधाति॒ प्र पू॒षा। क्षेत्र॑स्य॒ पति॑ना व॒यꣳ हि॒तेने॑व जयामसि। गामश्वं॑ पोषयि॒त्न्वा स नो॑~(२६)

%1.1.14.3
मृडाती॒दृशे᳚। क्षेत्र॑स्य पते॒ मधु॑मन्तमू॒र्मिं धे॒नुरि॑व॒ पयो॑ अ॒स्मासु॑ धुक्ष्व। म॒धु॒श्चुतं॑ घृ॒तमि॑व॒ सुपू॑तमृ॒तस्य॑ न॒ पत॑यो मृडयन्तु। अग्ने॒ नय॑ सु॒पथा॑ रा॒ये अ॒स्मान् विश्वा॑नि देव व॒युना॑नि वि॒द्वान्। यु॒यो॒ध्य॑स्मज्जु॑हुरा॒णमेनो॒ भूयि॑ष्ठान्ते॒ नम॑ उक्तिं विधेम। आ दे॒वाना॒मपि॒ पन्था॑मगन्म॒ यच्छ॒क्नवा॑म॒ तदनु॒ प्रवो॑ढुम्। अ॒ग्निर्वि॒द्वान्थ्स य॑जा॒थ्~(२७)

%1.1.14.4
सेदु॒ होता॒ सो अ॑ध्व॒रान्थ्स ऋ॒तून्क॑ल्पयाति। यद्वा\-हि॑ष्ठं॒ तद॒ग्नये॑ बृ॒हद॑र्च विभावसो। महि॑षीव॒ त्वद्र॒यिस्त्वद्वाजा॒ उदी॑रते। अग्ने॒ त्वं पा॑रया॒ नव्यो॑ अ॒स्मान्थ्स्व॒स्तिभि॒रति॑ दु॒र्गाणि॒ विश्वा᳚। पूश्च॑ पृ॒थ्वी ब॑हु॒ला न॑ उ॒र्वी भवा॑ तो॒काय॒ तन॑याय॒ शं योः। त्वम॑ग्ने व्रत॒पा अ॑सि दे॒व आ मर्त्ये॒ष्वा। त्वं य॒ज्ञेष्वीड्यः॑। यद्वो॑ व॒यं प्र॑मि॒नाम॑ व्र॒तानि॑ वि॒दुषां᳚ देवा॒ अवि॑दुष्टरासः। अ॒ग्निष्टद्विश्व॒मा पृ॑णाति वि॒द्वान् येभि॑र्दे॒वाꣳ ऋ॒तुभिः॑ क॒ल्पया॑ति~(२८)

{\anuvakamend[{जु॒षेथा॒मा स नो॑ यजा॒दा त्रयो॑विꣳशतिश्च}]}
%1.1.1.0

{\prashnaend[{इ॒षे त्वा॑ य॒ज्ञस्य॒ शुन्ध॑ध्वं॒ कर्म॑णे दे॒वो\-ऽव॑धूत॒न्धृष्टिः॒ सं व॑पा॒म्या द॑दे॒
प्रत्यु॑ष्टं॒ कृष्णो॑\-ऽसि॒ भुव॑नमसि॒ वाज॑स्यो॒भा वां॒ चतु॑र्दश॥14॥ इ॒षे दृꣳ॑ह॒ भुव॑नम॒ष्टाविꣳ॑शतिः॥28॥ इ॒षे त्वा॑ क॒ल्पया॑ति॥}]}

%%% END PRASHNA
