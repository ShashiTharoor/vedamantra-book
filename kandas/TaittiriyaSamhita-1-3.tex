\sect{तृतीयः प्रश्नः}\setcounter{anuvakam}{0}
\dnsub{तैत्तिरीयसंहितायां प्रथमकाण्डे तृतीयः प्रश्नः}
%1.3.1.0
%1.3.1.1
दे॒वस्य॑ त्वा सवि॒तुः प्र॑स॒वे᳚\-ऽश्विनो᳚र्बा॒हु\-भ्यां᳚ पू॒ष्णो हस्ता᳚भ्या॒माद॒दे\-ऽभ्रि॑रसि॒ नारि॑रसि॒ परि॑लिखित॒ꣳ॒ रक्षः॒ परि॑लिखिता॒ अरा॑तय इ॒दम॒हꣳ रक्ष॑सो ग्री॒वा अपि॑ कृन्तामि॒ यो᳚\-ऽस्मान् द्वेष्टि॒ यं च॑ व॒यं द्वि॒ष्म इ॒दम॑स्य ग्री॒वा अपि॑ कृन्तामि दि॒वे त्वा॒\-ऽन्तरि॑क्षाय त्वा पृथि॒व्यै त्वा॒ शुन्ध॑तां लो॒कः पि॑तृ॒षद॑नो॒ यवो॑\-ऽसि य॒वया॒स्मद्द्वेषः॑॥१॥

%1.3.1.2
य॒वयारा॑तीः पितृ॒णाꣳ सद॑नम॒स्युद्दिवꣴ॑ स्तभा॒ना\-ऽन्तरि॑क्षं पृण पृथि॒वीं दृꣳ॑ह द्युता॒नस्त्वा॑ मारु॒तो मि॑नोतु मि॒त्रावरु॑णयोर्ध्रु॒वेण॒ धर्म॑णा ब्रह्म॒वनिं॑ त्वा क्षत्र॒वनिꣳ॑ सुप्रजा॒वनिꣳ॑ रायस्पोष॒वनिं॒ पर्यू॑हामि॒ ब्रह्म॑ दृꣳह क्ष॒त्रं दृꣳ॑ह प्र॒जां दृꣳ॑ह रा॒यस्पोषं॑ दृꣳह घृ॒तेन॑ द्यावापृथिवी॒ आ पृ॑णेथा॒मिन्द्र॑स्य॒ सदो॑\-ऽसि विश्वज॒नस्य॑ छा॒या परि॑ त्वा गिर्वणो॒ गिर॑ इ॒मा भ॑वन्तु वि॒श्वतो॑ वृ॒द्धायु॒मनु॒ वृद्ध॑यो॒ जुष्टा॑ भवन्तु॒ जुष्ट॑य॒ इन्द्र॑स्य॒ स्यूर॒सीन्द्र॑स्य ध्रु॒वम॑स्यै॒न्द्रम॒सीन्द्रा॑य त्वा॥२॥

%1.3.2.0
{\anuvakamend[{द्वेष॑ इ॒मा अ॒ष्टाद॑श च॥१॥}]}

%1.3.2.1
र॒क्षो॒हणो॑ वलग॒हनो॑ वैष्ण॒वान्ख॑नामी॒दम॒हं तं व॑ल॒गमुद्व॑पामि॒ यं नः॑ समा॒नो यमस॑मानो निच॒खाने॒दमे॑न॒मध॑रं करोमि॒ यो नः॑ समा॒नो यो\-ऽस॑मानो\-ऽराती॒यति॑ गाय॒त्रेण॒ छन्द॒सा\-ऽव॑बाढो वल॒गः किमत्र॑ भ॒द्रं तन्नौ॑ स॒ह वि॒राड॑सि सपत्न॒हा स॒म्राड॑सि भ्रातृव्य॒हा स्व॒राड॑स्यभिमाति॒हा वि॑श्वा॒राड॑सि॒ विश्वा॑सां ना॒ष्ट्राणाꣳ॑ ह॒न्ता॥३॥

%1.3.2.2
र॒क्षो॒हणो॑ वलग॒हनः॒ प्रोक्षा॑मि वैष्ण॒वान् र॑क्षो॒हणो॑ वलग॒हनो\-ऽव॑ नयामि वैष्ण॒वान् यवो॑\-ऽसि य॒वया॒स्मद्द्वेषो॑ य॒वयारा॑ती रक्षो॒हणो॑ वलग॒हनो\-ऽव॑ स्तृणामि वैष्ण॒वान् र॑क्षो॒हणो॑ वलग॒हनो॒\-ऽभि जु॑होमि वैष्ण॒वान् र॑क्षो॒हणौ॑ वलग॒हना॒वुप॑ दधामि वैष्ण॒वी र॑क्षो॒हणौ॑ वलग॒हनौ॒ पर्यू॑हामि वैष्ण॒वी र॑क्षो॒हणौ॑ वलग॒हनौ॒ परि॑ स्तृणामि वैष्ण॒वी र॑क्षो॒हणौ॑ वलग॒हनौ॑ वैष्ण॒वी बृ॒हन्न॑सि बृ॒हद्ग्रा॑वा बृह॒तीमिन्द्रा॑य॒ वाचं॑ वद॥४॥

%1.3.3.0
{\anuvakamend[{ह॒न्तेन्द्रा॑य॒ द्वे च॑॥२॥}]}

%1.3.3.1
वि॒भूर॑सि प्र॒वाह॑णो॒ वह्नि॑रसि हव्य॒वाह॑नः श्वा॒त्रो॑\-ऽसि॒ प्रचे॑तास्तु॒थो॑\-ऽसि वि॒श्ववे॑दा उ॒शिग॑सि क॒विरङ्घा॑रिरसि॒ बम्भा॑रिरव॒स्युर॑सि॒ दुव॑स्वाञ्छु॒न्ध्यूर॑सि मार्जा॒लीयः॑ स॒म्राड॑सि कृ॒शानुः॑ परि॒षद्यो॑\-ऽसि॒ पव॑मानः प्र॒तक्वा॑\-ऽसि॒ नभ॑स्वा॒नसं॑मृष्टो\-ऽसि हव्य॒सूद॑ ऋ॒तधा॑मा\-ऽसि॒ सुव॑र्ज्योति॒र्ब्रह्म॑ज्योतिरसि॒ सुव॑र्धामा॒\-ऽजो᳚\-ऽस्येक॑पा॒दहि॑रसि बु॒ध्नियो॒ रौद्रे॒णानी॑केन पा॒हि मा᳚\-ऽग्ने पिपृ॒हि मा॒ मा मा॑ हिꣳसीः॥५॥

%1.3.4.0
{\anuvakamend[{अनी॑केना॒ष्टौ च॑॥३॥}]}

%1.3.4.1
त्वꣳ सो॑म तनू॒कृद्भ्यो॒ द्वेषो᳚भ्यो॒\-ऽन्यकृ॑तेभ्य उ॒रु य॒न्तासि॒ वरू॑थ॒ꣴ॒ स्वाहा॑ जुषा॒णो अ॒प्तुराज्य॑स्य वेतु॒ स्वाहा॒\-ऽयं नो॑ अ॒ग्निर्वरि॑वः कृणोत्व॒यं मृधः॑ पु॒र ए॑तु प्रभि॒न्दन्न्। अ॒यꣳ शत्रू᳚ञ्जयतु॒ जर्हृ॑षाणो॒ \-ऽयं वाजं॑ जयतु॒ वाज॑सातौ॥ उ॒रु वि॑ष्णो॒ वि क्र॑मस्वो॒रु क्षया॑य नः कृधि। घृ॒तं घृ॑तयोने पिब॒ प्रप्र॑ य॒ज्ञप॑तिं तिर॥ सोमो॑ जिगाति गातु॒विद्॥६॥

%1.3.4.2
दे॒वाना॑मेति निष्कृ॒तमृ॒तस्य॒ योनि॑मा॒सद॒मदि॑त्याः॒ सदो॒\-ऽस्यदि॑त्याः॒ सद॒ आ सी॑दै॒ष वो॑ देव सवितः॒ सोम॒स्तꣳ र॑क्षध्वं॒ मा वो॑ दभदे॒तत् त्वꣳ सो॑म दे॒वो दे॒वानुपा॑गा इ॒दम॒हं म॑नु॒ष्यो॑ मनु॒ष्या᳚न्थ्स॒ह प्र॒जया॑ स॒ह रा॒यस्पोषे॑ण॒ नमो॑ दे॒वेभ्यः॑ स्व॒धा पि॒तृभ्य॑ इ॒दम॒हं निर्वरु॑णस्य॒ पाशा॒त् सुव॑र॒भि॥७॥

%1.3.4.3
वि ख्ये॑षं वैश्वान॒रं ज्योति॒रग्ने᳚ व्रतपते॒ त्वं व्र॒तानां᳚ व्र॒तप॑तिरसि॒ या मम॑ त॒नूस्त्वय्यभू॑दि॒यꣳ सा मयि॒ या तव॑ त॒नूर्मय्यभू॑दे॒षा सा त्वयि॑ यथाय॒थं नौ᳚ व्रतपते व्र॒तिनो᳚र्व्र॒तानि॑॥८॥

%1.3.5.0
{\anuvakamend[{गा॒तु॒विद॒भ्येक॑त्रिꣳशच्च॥४॥}]}

%1.3.5.1
अत्य॒न्यानगां॒ नान्यानुपा॑गाम॒र्वाक्त्वा॒ परै॑रविदं प॒रो\-ऽव॑रै॒स्तं त्वा॑ जुषे वैष्ण॒वं दे॑वय॒ज्यायै॑ दे॒वस्त्वा॑ सवि॒ता मध्वा॑\-ऽन॒क्त्वोष॑धे॒ त्राय॑स्वैन॒ꣴ॒ स्वधि॑ते॒ मैनꣳ॑ हिꣳसी॒र्दिव॒मग्रे॑ण॒ मा ले॑खीर॒न्तरि॑क्षं॒ मध्ये॑न॒ मा हिꣳ॑सीः पृथि॒व्या सं भ॑व॒ वन॑स्पते श॒तव॑ल्\mbox{}शो॒ वि रो॑ह स॒हस्र॑वल्\mbox{}शा॒ वि व॒यꣳ रु॑हेम॒ यं त्वा॒\-ऽयꣴ स्वधि॑ति॒स्तेति॑जानः प्रणि॒नाय॑ मह॒ते सौभ॑गा॒या\-ऽच्छि॑न्नो॒ रायः॑ सु॒वीरः॑॥९॥

%1.3.6.0
{\anuvakamend[{यं दश॑ च॥५॥}]}

%1.3.6.1
पृ॒थि॒व्यै त्वा॒\-ऽन्तरि॑क्षाय त्वा दि॒वे त्वा॒ शुन्ध॑तां लो॒कः पि॑तृ॒षद॑नो॒ यवो॑\-ऽसि य॒वया॒स्मद् द्वेषो॑ य॒वयारा॑तीः पितृ॒णाꣳ सद॑नमसि स्वावे॒शो᳚\-ऽस्यग्रे॒गा ने॑तृ॒णां वन॒स्पति॒रधि॑ त्वा स्थास्यति॒ तस्य॑ वित्ताद्दे॒वस्त्वा॑ सवि॒ता मध्वा॑\-ऽनक्तु सुपिप्प॒लाभ्य॒स्त्वौष॑धीभ्य॒ उद्दिवꣴ॑ स्तभा॒नान्तरि॑क्षं पृण पृथि॒वीमुप॑रेण दृꣳह॒ ते ते॒ धामा᳚न्युश्मसि॥१०॥

%1.3.6.2
ग॒मध्ये॒ गावो॒ यत्र॒ भूरि॑शृङ्गा अ॒यासः॑। अत्राह॒ तदु॑रुगा॒यस्य॒ विष्णोः᳚ पर॒मं प॒दमव॑ भाति॒ भूरेः᳚॥ विष्णोः॒ कर्मा॑णि पश्यत॒ यतो᳚ व्र॒तानि॑ पस्प॒शे। इन्द्र॑स्य॒ युज्यः॒ सखा᳚॥ तद्विष्णोः᳚ पर॒मं प॒दꣳ सदा॑ पश्यन्ति सू॒रयः॑। दि॒वीव॒ चक्षु॒रात॑तम्॥ ब्र॒ह्म॒वनिं॑ त्वा क्षत्त्र॒वनिꣳ॑ सुप्रजा॒वनिꣳ॑ रायस्पोष॒वनिं॒ पर्यू॑हामि॒ ब्रह्म॑ दृꣳह क्ष॒त्त्रं दृꣳ॑ह प्र॒जां दृꣳ॑ह रा॒यस्पोषं॑ दृꣳह परि॒वीर॑सि॒ परि॑ त्वा॒ दैवी॒र्विशो᳚ व्ययन्तां॒ परी॒मꣳ रा॒यस्पोषो॒ यज॑मानं मनु॒ष्या॑ अ॒न्तरि॑क्षस्य त्वा॒ साना॒वव॑ गूहामि॥११॥

%1.3.7.0
{\anuvakamend[{उ॒श्म॒सी॒ पोष॒मेका॒न्नविꣳ॑श॒तिश्च॑॥६॥}]}

%1.3.7.1
इ॒षे त्वो॑प॒वीर॒स्युपो॑ दे॒वान्दैवी॒र्विशः॒ प्रागु॒र्वह्नी॑रु॒शिजो॒ बृह॑स्पते धा॒रया॒ वसू॑नि ह॒व्या ते᳚ स्वदन्तां॒ देव॑ त्वष्ट॒र्वसु॑ रण्व॒ रेव॑ती॒ रम॑ध्वम॒ग्नेर्ज॒नित्र॑मसि॒ वृष॑णौ स्थ उ॒र्वश्य॑स्या॒युर॑सि पुरू॒रवा॑ घृ॒तेना॒क्ते वृष॑णं दधाथां गाय॒त्रं छन्दो\-ऽनु॒ प्र जा॑यस्व॒ त्रैष्टु॑भं॒ जाग॑तं॒ छन्दो\-ऽनु॒ प्रजा॑यस्व॒ भव॑तं॥१२॥

%1.3.7.2
नः॒ सम॑नसौ॒ समो॑कसावरे॒पसौ᳚। मा य॒ज्ञꣳ हिꣳ॑सिष्टं॒ मा य॒ज्ञप॑तिं जातवेदसौ शि॒वौ भ॑वतम॒द्य नः॑॥ अ॒ग्नाव॒ग्निश्च॑रति॒ प्रवि॑ष्ट॒ ऋषी॑णां पु॒त्त्रो अ॑धिरा॒ज ए॒षः। स्वा॒हा॒कृत्य॒ ब्रह्म॑णा ते जुहोमि॒ मा दे॒वानां᳚ मिथु॒याक॑र्भाग॒धेयम्᳚॥१३॥

%1.3.8.0
{\anuvakamend[{भव॑त॒मेक॑त्रिꣳशच्च॥७॥}]}

%1.3.8.1
आ द॑द ऋ॒तस्य॑ त्वा देवहविः॒ पाशे॒ना\-ऽ\-ऽर॑भे॒ धर्\mbox{}षा॒ मानु॑षान॒द्भ्यस्त्वौष॑धीभ्यः॒ प्रोक्षा᳚म्य॒पां पे॒रुर॑सि स्वा॒त्तं चि॒त् सदे॑वꣳ ह॒व्यमापो॑ देवीः॒ स्वद॑तैन॒ꣳ॒ सं ते᳚ प्रा॒णो वा॒युना॑ गच्छता॒ꣳ॒ सं यज॑त्रै॒रङ्गा॑नि॒ सं य॒ज्ञप॑तिरा॒शिषा॑ घृ॒तेना॒क्तौ प॒शुं त्रा॑येथा॒ꣳ॒ रेव॑तीर्य॒ज्ञप॑तिं प्रिय॒धा\-ऽ\-ऽवि॑श॒तोरो॑ अन्तरिक्ष स॒जूर्दे॒वेन॑॥१४॥

%1.3.8.2
वाते॑ना॒\-ऽस्य ह॒विष॒स्त्मना॑ यज॒ सम॑स्य त॒नुवा॑ भव॒ वर्\mbox{}षी॑यो॒ वर्\mbox{}षी॑यसि य॒ज्ञे य॒ज्ञप॑तिं धाः पृथि॒व्याः स॒म्पृचः॑ पाहि॒ नम॑स्त आताना\-ऽन॒र्वा प्रेहि॑ घृ॒तस्य॑ कु॒ल्यामनु॑ स॒ह प्र॒जया॑ स॒ह रा॒यस्पोषे॒णा\-ऽ\-ऽपो॑ देवीः शुद्धायुवः शु॒द्धा यू॒यं दे॒वाꣳ ऊ᳚ड्ढ्वꣳ शु॒द्धा व॒यं परि॑विष्टाः परिवे॒ष्टारो॑ वो भूयास्म॥१५॥

%1.3.9.0
{\anuvakamend[{दे॒वेन॒ चतु॑श्चत्वारिꣳशच्च॥८॥}]}

%1.3.9.1
वाक्त॒ आ प्या॑यतां प्रा॒णस्त॒ आ प्या॑यतां॒ चक्षु॑स्त॒ आ प्या॑यता॒ꣳ॒ श्रोत्रं॑ त॒ आ प्या॑यतां॒ या ते᳚ प्रा॒णाञ्छुग्ज॒गाम॒ या चक्षु॒र्या श्रोत्रं॒ यत् ते᳚ क्रू॒रं यदास्थि॑तं॒ तत् त॒ आ प्या॑यतां॒ तत् त॑ ए॒तेन॑ शुन्धतां॒ नाभि॑स्त॒ आ प्या॑यतां पा॒युस्त॒ आ प्या॑यताꣳ शु॒द्धाश्च॒रित्राः॒ शम॒द्भ्यः॥१६॥

%1.3.9.2
शमोष॑धीभ्यः॒ शं पृ॑थि॒व्यै शमहो᳚भ्या॒मोष॑धे॒ त्राय॑स्वैन॒ꣴ॒ स्वधि॑ते॒ मैनꣳ॑ हिꣳसी॒ रक्ष॑सां भा॒गो॑\-ऽसी॒दम॒हꣳ रक्षो॑\-ऽध॒मं तमो॑ नयामि॒ यो᳚\-ऽस्मान् द्वेष्टि॒ यं च॑ व॒यं द्वि॒ष्म इ॒दमे॑नमध॒मं तमो॑ नयामी॒षे त्वा॑ घृ॒तेन॑ द्यावापृथिवी॒ प्रोर्ण्वा॑था॒मच्छि॑न्नो॒ रायः॑ सु॒वीर॑ उ॒र्व॑न्तरि॑क्ष॒मन्वि॑हि॒ वायो॒ वीहि॑ स्तो॒काना॒ꣳ॒ स्वाहो॒र्ध्वन॑भसं मारु॒तं ग॑च्छतम्॥१७॥

%1.3.10.0
{\anuvakamend[{अ॒द्भ्यो वीहि॒ पञ्च॑ च॥९॥}]}

%1.3.10.1
सं ते॒ मन॑सा॒ मनः॒ सं प्रा॒णेन॑ प्रा॒णो जुष्टं॑ दे॒वेभ्यो॑ ह॒व्यं घृ॒तव॒त् स्वाहै॒न्द्रः प्रा॒णो अङ्गे॑अङ्गे॒ नि दे᳚ध्यदै॒न्द्रो॑\-ऽपा॒नो अङ्गे॑अङ्गे॒ वि बो॑भुव॒द्देव॑ त्वष्ट॒र्भूरि॑ ते॒ सꣳस॑मेतु॒ विषु॑रूपा॒ यत् सल॑क्ष्माणो॒ भव॑थ देव॒त्रा यन्त॒मव॑से॒ सखा॒यो\-ऽनु॑ त्वा मा॒ता पि॒तरो॑ मदन्तु॒ श्रीर॑स्य॒ग्निस्त्वा᳚ श्रीणा॒त्वापः॒ सम॑रिण॒न्वात॑स्य॥१८॥

%1.3.10.2
त्वा॒ ध्रज्यै॑ पू॒ष्णो रꣴह्या॑ अ॒पामोष॑धीना॒ꣳ॒ रोहि॑ष्यै घृ॒तं घृ॑तपावानः पिबत॒ वसां᳚ वसापावानः पिबता॒न्तरि॑क्षस्य ह॒विर॑सि॒ स्वाहा᳚ त्वा॒\-ऽन्तरि॑क्षाय॒ दिशः॑ प्र॒दिश॑ आ॒दिशो॑ वि॒दिश॑ उ॒द्दिशः॒ स्वाहा॑ दि॒ग्भ्यो नमो॑ दि॒ग्भ्यः॥१९॥

%1.3.11.0
{\anuvakamend[{वात॑स्या॒ष्टाविꣳ॑शतिश्च॥10॥}]}

%1.3.11.1
स॒मु॒द्रं ग॑च्छ॒ स्वाहा॒\-ऽन्तरि॑क्षं गच्छ॒ स्वाहा॑ दे॒वꣳ स॑वि॒तारं॑ गच्छ॒ स्वाहा॑\-ऽहोरा॒त्रे ग॑च्छ॒ स्वाहा॑ मि॒त्रावरु॑णौ गच्छ॒ स्वाहा॒ सोमं॑ गच्छ॒ स्वाहा॑ य॒ज्ञं ग॑च्छ॒ स्वाहा॒ छन्दाꣳ॑सि गच्छ॒ स्वाहा॒ द्यावा॑पृथि॒वी ग॑च्छ॒ स्वाहा॒ नभो॑ दि॒व्यं ग॑च्छ॒ स्वाहा॒\-ऽग्निं वै᳚श्वान॒रं ग॑च्छ॒ स्वाहा॒\-ऽद्भ्यस्त्वौष॑धीभ्यो॒ मनो॑ मे॒ हार्दि॑ यच्छ त॒नूं त्वचं॑ पु॒त्त्रं नप्ता॑रमशीय॒ शुग॑सि॒ तम॒भि शो॑च॒ यो᳚\-ऽस्मान् द्वेष्टि॒ यं च॑ व॒यं द्वि॒ष्मो धाम्नो॑धाम्नो राजन्नि॒तो व॑रुण नो मुञ्च॒ यदापो॒ अघ्नि॑या॒ वरु॒णेति॒ शपा॑महे॒ ततो॑ वरुण नो मुञ्च॥२०॥

%1.3.12.0
{\anuvakamend[{अ॒सि॒ षड्विꣳ॑शतिश्च॥11॥}]}

%1.3.12.1
ह॒विष्म॑तीरि॒मा आपो॑ ह॒विष्मा᳚न् दे॒वो अ॑ध्व॒रो ह॒विष्मा॒ꣳ॒ आ वि॑वासति ह॒विष्माꣳ॑ अस्तु॒ सूर्यः॑॥ अ॒ग्नेर्वो\-ऽप॑न्नगृहस्य॒ सद॑सि सादयामि सु॒म्नाय॑ सुम्निनीः सु॒म्ने मा॑ धत्तेन्द्राग्नि॒योर्भा॑ग॒धेयीः᳚ स्थ मि॒त्रावरु॑णयोर्भाग॒धेयीः᳚ स्थ॒ विश्वे॑षां दे॒वानां᳚ भाग॒धेयीः᳚ स्थ य॒ज्ञे जा॑गृत॥२१॥

%1.3.13.0
{\anuvakamend[{ह॒विष्म॑ती॒श्चतु॑स्रिꣳशत्॥12॥}]}

%1.3.13.1
हृ॒दे त्वा॒ मन॑से त्वा दि॒वे त्वा॒ सूर्या॑य त्वो॒र्ध्वमि॒मम॑ध्व॒रं कृ॑धि दि॒वि दे॒वेषु॒ होत्रा॑ यच्छ॒ सोम॑ राज॒न्नेह्यव॑ रोह॒ मा भेर्मा सं वि॑क्था॒ मा त्वा॑ हिꣳसिषं प्र॒जास्त्वमु॒पाव॑रोह प्र॒जास्त्वामु॒पाव॑ रोहन्तु शृ॒णोत्व॒ग्निः स॒मिधा॒ हवं॑ मे शृ॒ण्वन्त्वापो॑ धि॒षणा᳚श्च दे॒वीः। शृ॒णोत॑ ग्रावाणो वि॒दुषो॒ नु॥२२॥

%1.3.13.2
य॒ज्ञꣳ शृ॒णोतु॑ दे॒वः स॑वि॒ता हवं॑ मे। देवी॑रापो अपां नपा॒द्य ऊ॒र्मिर्\mbox{}ह॑वि॒ष्य॑ इन्द्रि॒यावा᳚न्म॒दिन्त॑म॒स्तं दे॒वेभ्यो॑ देव॒त्रा ध॑त्त शु॒क्रꣳ शु॑क्र॒पेभ्यो॒ येषां᳚ भा॒गः स्थ स्वाहा॒ कार्\mbox{}षि॑र॒स्यपा॒पां मृ॒ध्रꣳ स॑मु॒द्रस्य॒ वोक्षि॑त्या॒ उन्न॑ये। यम॑ग्ने पृ॒त्सु मर्त्य॒मावो॒ वाजे॑षु॒ यं जु॒नाः। स यन्ता॒ शश्व॑ती॒रिषः॑॥ (1)॥२३॥

%1.3.14.0
{\anuvakamend[{नु स॒प्तच॑त्वारिꣳशच्च॥13॥}]}

%1.3.14.1
त्वम॑ग्ने रु॒द्रो असु॑रो म॒हो दि॒वस्त्वꣳ शर्धो॒ मारु॑तं पृ॒क्ष ई॑शिषे। त्वं वातै॑ररु॒णैर्या॑सि शङ्ग॒यस्त्वं पू॒षा वि॑ध॒तः पा॑सि॒ नु त्मना᳚॥ आ वो॒ राजा॑नमध्व॒रस्य॑ रु॒द्रꣳ होता॑रꣳ सत्य॒यज॒ꣳ॒ रोद॑स्योः। अ॒ग्निं पु॒रा त॑नयि॒त्नोर॒चित्ता॒द्धिर॑ण्यरूप॒मव॑से कृणुध्वम्॥ अ॒ग्निर्होता॒ निष॑सादा॒ यजी॑यानु॒पस्थे॑ मा॒तुः सु॑र॒भावु॑ लो॒के। युवा॑ क॒विः पु॑रुनि॒ष्ठः॥२४॥

%1.3.14.2
ऋ॒तावा॑ ध॒र्ता कृ॑ष्टी॒नामु॒त मध्य॑ इ॒द्धः॥ सा॒ध्वीम॑कर्दे॒ववी॑तिं नो अ॒द्य य॒ज्ञस्य॑ जि॒ह्वाम॑विदाम॒ गुह्याम्᳚। स आयु॒रागा᳚थ्सुर॒भिर्वसा॑नो भ॒द्राम॑कर्दे॒वहू॑तिं नो अ॒द्य॥ अक्र॑न्दद॒ग्निः स्त॒नय॑न्निव॒ द्यौः क्षामा॒ रेरि॑हद्वी॒रुधः॑ सम॒ञ्जन्न्। स॒द्यो ज॑ज्ञा॒नो विहीमि॒द्धो अख्य॒दा रोद॑सी भा॒नुना॑ भात्य॒न्तः॥ त्वे वसू॑नि पुर्वणीक॥२५॥

%1.3.14.3
हो॒त॒र्दो॒षा वस्तो॒रेरि॑रे य॒ज्ञिया॑सः। क्षामे॑व॒ विश्वा॒ भुव॑नानि॒ यस्मि॒न्थ्सꣳ सौभ॑गानि दधि॒रे पा॑व॒के॥ तुभ्यं॒ ता अ॑ङ्गिरस्तम॒ विश्वाः᳚ सुक्षि॒तयः॒ पृथ॑क्। अग्ने॒ कामा॑य येमिरे॥ अ॒श्याम॒ तं काम॑मग्ने॒ तवो॒त्य॑श्याम॑ र॒यिꣳ र॑यिवः सु॒वीरम्᳚। अ॒श्याम॒ वाज॑म॒भि वा॒जय॑न्तो॒\-ऽश्याम॑ द्यु॒म्नम॑जरा॒जरं॑ ते॥ श्रेष्ठं॑ यविष्ठ भार॒ताग्ने᳚ द्यु॒मन्त॒माभ॑र।॥२६॥

%1.3.14.4
वसो॑ पुरु॒स्पृहꣳ॑ र॒यिम्॥ स श्वि॑ता॒नस्त॑न्य॒तू रो॑चन॒स्था अ॒जरे॑भि॒र्नान॑दद्भि॒र्यवि॑ष्ठः। यः पा॑व॒कः पु॑रु॒तमः॑ पु॒रूणि॑ पृ॒थून्य॒ग्निर॑नु॒याति॒ भर्वन्न्॑॥ आयु॑ष्टे वि॒श्वतो॑ दधद॒यम॒ग्निर्वरे᳚ण्यः। पुन॑स्ते प्रा॒ण आय॑ति॒ परा॒ यक्ष्मꣳ॑ सुवामि ते॥ आ॒यु॒र्दा अ॑ग्ने ह॒विषो॑ जुषा॒णो घृ॒तप्र॑तीको घृ॒तयो॑निरेधि। घृ॒तं पी॒त्वा मधु॒ चारु॒ गव्यं॑ पि॒तेव॑ पु॒त्रम॒भि॥२७॥

%1.3.14.5
र॒क्ष॒ता॒दि॒मम्॥ तस्मै॑ ते प्रति॒हर्य॑ते॒ जात॑वेदो॒ विच॑र्\mbox{}षणे। अग्ने॒ जना॑मि सुष्टु॒तिम्॥ दि॒वस्परि॑ प्रथ॒मं ज॑ज्ञे अ॒ग्निर॒स्मद् द्वि॒तीयं॒ परि॑ जा॒तवे॑दाः। तृ॒तीय॑म॒फ्सु नृ॒मणा॒ अज॑स्र॒मिन्धा॑न एनं जरते स्वा॒धीः॥ शुचिः॑ पावक॒ वन्द्यो\-ऽग्ने॑ बृ॒हद्वि रो॑चसे। त्वं घृ॒तेभि॒राहु॑तः॥ दृ॒शा॒नो रु॒क्म उ॒र्व्या व्य॑द्यौद् दु॒र्मर्\mbox{}ष॒मायुः॑ श्रि॒ये रु॑चा॒नः। अ॒ग्निर॒मृतो॑ अभव॒द्वयो॑भिः॥२८॥

%1.3.14.6
यदे॑नं॒ द्यौरज॑नयत्सु॒रेताः᳚॥ आ यदि॒षे नृ॒पतिं॒ तेज॒ आन॒ट्छुचि॒ रेतो॒ निषि॑क्तं॒ द्यौर॒भीके᳚। अ॒ग्निः शर्ध॑मनव॒द्यं युवा॑नꣴ स्वा॒धियं॑ जनयत्सू॒दय॑च्च॥ स तेजी॑यसा॒ मन॑सा॒ त्वोत॑ उ॒त शि॑क्ष स्वप॒त्यस्य॑ शि॒क्षोः। अग्ने॑ रा॒यो नृत॑मस्य॒ प्रभू॑तौ भू॒याम॑ ते सुष्टु॒तय॑श्च॒ वस्वः॑॥ अग्ने॒ सह॑न्त॒मा भ॑र द्यु॒म्नस्य॑ प्रा॒सहा॑ र॒यिम्। विश्वा॒ यः॥२९॥

%1.3.14.7
च॒र्\mbox{}ष॒णीर॒भ्या॑सा वाजे॑षु सा॒सह॑त्॥ तम॑ग्ने पृतना॒सहꣳ॑ र॒यिꣳ स॑हस्व॒ आ भ॑र। त्वꣳ हि स॒त्यो अद्भु॑तो दा॒ता वाज॑स्य॒ गोम॑तः॥ उ॒क्षान्ना॑य व॒शान्ना॑य॒ सोम॑पृष्ठाय वे॒धसे᳚। स्तोमै᳚र्विधेमा॒ग्नये᳚॥ व॒द्मा हि सू॑नो॒ अस्य॑द्म॒सद्वा॑ च॒क्रे अ॒ग्निर्ज॒नुषाज्मान्नम्᳚। स त्वं न॑ ऊर्जसन॒ ऊर्जं॑ धा॒ राजे॑व जेरवृ॒के क्षे᳚ष्य॒न्तः॥ अग्न॒ आयूꣳ॑षि॥३०॥

%1.3.14.8
प॒व॒स॒ आ सु॒वोर्ज॒मिषं॑ च नः। आ॒रे बा॑धस्व दु॒च्छुनाम्᳚॥ अग्ने॒ पव॑स्व॒ स्वपा॑ अ॒स्मे वर्चः॑ सु॒वीर्यम्᳚। दध॒त्पोषꣳ॑ र॒यिं मयि॑॥ अग्ने॑ पावक रो॒चिषा॑ म॒न्द्रया॑ देव जि॒ह्वया᳚। आ दे॒वान् व॑क्षि॒ यक्षि॑ च॥ स नः॑ पावक दीदि॒वो\-ऽग्ने॑ दे॒वाꣳ इ॒हा व॑ह। उप॑ य॒ज्ञꣳ ह॒विश्च॑ नः॥ अ॒ग्निः शुचि॑व्रततमः॒ शुचि॒र्विप्रः॒ शुचिः॑ क॒विः। शुची॑ रोचत॒ आहु॑तः॥ उद॑ग्ने॒ शुच॑य॒स्तव॑ शु॒क्रा भ्राज॑न्त ईरते। तव॒ ज्योतीꣳ॑ष्य॒र्चयः॑॥३१॥

{\anuvakamend[{पु॒रु॒नि॒ष्ठः पु॑र्वणीक भरा॒\-ऽभि वयो॑भि॒र्य आयूꣳ॑षि॒ विप्रः॒ शुचि॒श्चतु॑र्दश च॥14॥}]}
%%% END PRASHNA
