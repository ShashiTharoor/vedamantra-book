\sect{अष्टमः प्रश्नः}\setcounter{anuvakam}{0}
\dnsub{तैत्तिरीयसंहितायां प्रथमकाण्डे अष्टमः प्रश्नः}
%1.8.1.0
%1.8.1.1
अनु॑मत्यै पुरो॒डाश॑म॒ष्टा\-क॑पालं॒ निर्व॑पति धे॒नुर्दक्षि॑णा॒ ये प्र॒त्यञ्चः॒ शम्या॑या अव॒शीय॑न्ते॒ तन्नैर्॑ऋ॒तमेक॑कपालं कृ॒ष्णं वासः॑ कृ॒ष्णतू॑षं॒ दक्षि॑णा॒ वीहि॒ स्वाहा\-ऽ\-ऽहु॑तिं जुषा॒ण ए॒ष ते॑ निर्\mbox{}ऋते भा॒गो भूते॑ ह॒विष्म॑त्यसि मु॒ञ्चेममꣳह॑सः॒ स्वाहा॒ नमो॒ य इ॒दं च॒कारा॑\-ऽ\-ऽदि॒त्यं च॒रुं निर्व॑पति॒ वरो॒ दक्षि॑णा\-ऽ\-ऽग्नावैष्ण॒वमेका॑\-दश\-कपालं वाम॒नो व॒ही दक्षि॑णा\-ऽग्नीषो॒मीय॒-~(१)

%1.8.1.2
मेका॑\-दश\-कपाल॒ꣳ॒ हिर॑ण्यं॒ दक्षि॑णै॒न्द्रमेका॑\-दश\-कपालमृष॒भो व॒ही दक्षि॑णा\-ऽ\-ऽग्ने॒यम॒ष्टाक॑पालमै॒न्द्रं दध्यृ॑ष॒भो व॒ही दक्षि॑णैन्द्रा॒ग्नं द्वाद॑श\-कपालं वैश्वदे॒वं च॒रुं प्र॑थम॒जो व॒थ्सो दक्षि॑णा सौ॒म्यꣴ श्या॑मा॒कं च॒रुं वासो॒ दक्षि॑णा॒ सर॑स्वत्यै च॒रुꣳ सर॑स्वते च॒रुं मि॑थु॒नौ गावौ॒ दक्षि॑णा॥~(२)

%1.8.2.0
{\anuvakamend[{अ॒ग्नी॒षो॒मीयं॒ चतु॑स्त्रिꣳशच्च}]}%~(१)

%1.8.2.1
आ॒ग्ने॒यम॒ष्टा\-क॑पालं॒ निर्व॑पति सौ॒म्यं च॒रुꣳ सा॑वि॒त्रं द्वाद॑श\-कपालꣳ सारस्व॒तं च॒रुं पौ॒ष्णं च॒रुं मा॑रु॒तꣳ स॒प्तक॑पालं वैश्वदे॒वीमा॒मिक्षां᳚ द्यावापृथि॒व्य॑मेक॑कपालम्॥~(३)

%1.8.3.0
{\anuvakamend[{आ॒ग्ने॒यम॒ष्टाद॑श}]}%~(२)

%1.8.3.1
ऐ॒न्द्रा॒ग्नमेका॑\-दश\-कपालं मारु॒तीमा॒मिक्षां᳚ वारु॒णीमा॒मिक्षां᳚ का॒यमेक॑कपालं प्रघा॒स्यान्॑ हवामहे म॒रुतो॑ य॒ज्ञवा॑हसः कर॒म्भेण॑ स॒जोष॑सः॥ मो षू ण॑ इन्द्र पृ॒थ्सु दे॒वास्तु॑ स्म ते शुष्मिन्नव॒या। म॒ही ह्य॑स्य मी॒ढुषो॑ य॒व्या। ह॒विष्म॑तो म॒रुतो॒ वन्द॑ते॒ गीः॥ यद् ग्रामे॒ यदर॑ण्ये॒ यथ्स॒भायां॒ यदि॑न्द्रि॒ये। यच्छू॒द्रे यद॒र्य॑ एन॑श्चकृ॒मा व॒यम्। यदेक॒स्याधि॒ धर्म॑णि॒ तस्या॑व॒यज॑नमसि॒ स्वाहा᳚॥ अक्र॒न् कर्म॑ कर्म॒कृतः॑ स॒ह वा॒चा म॑योभु॒वा॥ दे॒वेभ्यः॒ कर्म॑ कृ॒त्वा\-ऽस्तं॒ प्रेत॑ सुदानवः॥~(४)

%1.8.4.0
{\anuvakamend[{व॒यं यद् विꣳ॑श॒तिश्च॑}]}%~(३)

%1.8.4.1
अ॒ग्नये\-ऽनी॑कवते पुरो॒डाश॑म॒ष्टा\-क॑पालं॒ निर्व॑पति सा॒कꣳ सूर्ये॑णोद्य॒ता म॒रुद्भ्यः॑ सान्तप॒नेभ्यो॑ म॒ध्यन्दि॑ने च॒रुं म॒रुद्भ्यो॑ गृहमे॒धिभ्यः॒ सर्वा॑सां दु॒ग्धे सा॒यं च॒रुं पू॒र्णा द॑र्वि॒ परा॑ पत॒ सुपू᳚र्णा॒ पुन॒राप॑त। व॒स्नेव॒ वि क्री॑णावहा॒ इष॒मूर्जꣳ॑ शतक्रतो॥ दे॒हि मे॒ ददा॑मि ते॒ नि मे॑ धेहि॒ नि ते॑ दधे। नि॒हार॒मिन्नि मे॑ हरा नि॒हारं॒~(५)


%1.8.4.2
नि ह॑रामि ते॥ म॒रुद्भ्यः॑ क्री॒डिभ्यः॑ पुरो॒डाशꣳ॑ स॒प्त\-क॑पालं॒ निर्व॑पति सा॒कꣳ सूर्ये॑णोद्य॒ताग्ने॒यम॒ष्टा\-क॑पालं॒ निर्व॑पति सौ॒म्यं च॒रुꣳ सा॑वि॒त्रं द्वाद॑श\-कपालꣳ सारस्व॒तं च॒रुं पौ॒ष्णं च॒रुमै᳚न्द्रा॒ग्नमेका॑\-दश\-कपालमै॒न्द्रं च॒रुं वै᳚श्वकर्म॒णमेक॑कपालम्॥~(६)

%1.8.5.0
{\anuvakamend[{ह॒रा॒ नि॒हारं॑ त्रि॒ꣳ॒शच्च॑}]}%~(४)

%1.8.5.1
सोमा॑य पितृ॒मते॑ पुरो॒डाश॒ꣳ॒ षट्\-क॑पालं॒ निर्व॑पति पि॒तृभ्यो॑ बर्\mbox{}हि॒षद्भ्यो॑ धा॒नाः पि॒तृभ्यो᳚\-ऽग्निष्वा॒त्तेभ्यो॑\-ऽभिवा॒न्या॑यै दु॒ग्धे म॒न्थमे॒तत् ते॑ तत॒ ये च॒ त्वामन्वे॒तत् ते॑ पितामह प्रपितामह॒ ये च॒ त्वामन्वत्र॑ पितरो यथाभा॒गं म॑न्दध्वꣳ सुस॒न्दृशं॑ त्वा व॒यं मघ॑वन् मन्दिषी॒महि॑॥ प्र नू॒नं पू॒र्णव॑न्धुरः स्तु॒तो या॑सि॒ वशा॒ꣳ॒ अनु॑॥ योजा॒ न्वि॑न्द्र ते॒ हरी᳚॥~(७)

%1.8.5.2
अक्ष॒न्नमी॑मदन्त॒ ह्यव॑ प्रि॒या अ॑धूषत॥ अस्तो॑षत॒ स्वभा॑नवो॒ विप्रा॒ नवि॑ष्ठया म॒ती॥ योजा॒ न्वि॑न्द्र ते॒ हरी᳚॥ अक्ष॑न् पि॒तरो\-ऽमी॑मदन्त पि॒तरो\-ऽती॑तृपन्त पि॒तरो\-ऽमी॑मृजन्त पि॒तरः॑॥ परे॑त पितरः सोम्या गम्भी॒रैः प॒थिभिः॑ पू॒र्व्यैः॥ अथा॑ पि॒तॄन्थ्सु॑वि॒दत्रा॒ꣳ॒ अपी॑त य॒मेन॒ ये स॑ध॒मादं॒ मद॑न्ति॥ मनो॒ न्वा हु॑वामहे नाराश॒ꣳ॒सेन॒ स्तोमे॑न पितृ॒णां च॒ मन्म॑भिः॥ आ~(८)

%1.8.5.3
न॑ एतु॒ मनः॒ पुनः॒ क्रत्वे॒ दक्षा॑य जी॒वसे᳚॥ ज्योक् च॒ सूर्यं॑ दृ॒शे॥ पुन॑र्नः पि॒तरो॒ मनो॒ ददा॑तु॒ दैव्यो॒ जनः॑॥ जी॒वं व्रातꣳ॑ सचेमहि॥ यद॒न्तरि॑क्षं पृथि॒वीमु॒त द्यां यन्मा॒तरं॑ पि॒तरं॑ वा जिहिꣳसि॒म॥ अ॒ग्निर्मा॒ तस्मा॒देन॑सो॒ गार्\mbox{}ह॑पत्यः॒ प्र मु॑ञ्चतु दुरि॒ता यानि॑ चकृ॒म क॒रोतु॒ माम॑ने॒नसम्᳚॥~(९)

%1.8.6.0
{\anuvakamend[{हरी॒ मन्म॑भि॒रा चतु॑श्चत्वारिꣳशच्च}]}%~(५)

%1.8.6.1
प्र॒ति॒पू॒रु॒षमेक॑कपाला॒न्निर्व॑प॒त्येक॒मति॑रिक्तं॒ याव॑न्तो गृ॒ह्याः᳚ स्मस्तेभ्यः॒ कम॑करं पशू॒नाꣳ शर्मा॑सि॒ शर्म॒ यज॑मानस्य॒ शर्म॑ मे य॒च्छैक॑ ए॒व रु॒द्रो न द्वि॒तीया॑य तस्थ आ॒खुस्ते॑ रुद्र प॒शुस्तं जु॑षस्वै॒ष ते॑ रुद्र भा॒गः स॒ह स्वस्रा\-ऽम्बि॑कया॒ तं जु॑षस्व भेष॒जं गवे\-ऽश्वा॑य॒ पुरु॑षाय भेष॒जमथो॑ अ॒स्मभ्यं॑ भेष॒जꣳ सुभे॑षजं॒~(१०)

%1.8.6.2
यथा\-ऽस॑ति॥ सु॒गं मे॒षाय॑ मे॒ष्या॑ अवा᳚म्ब रु॒द्रम॑दिम॒ह्यव॑ दे॒वं त्र्य॑म्बकम्॥ यथा॑ नः॒ श्रेय॑सः॒ कर॒द्यथा॑ नो॒ वस्य॑सः॒ कर॒द्यथा॑ नः पशु॒मतः॒ कर॒द्यथा॑ नो व्यवसा॒यया᳚त्॥ त्र्य॑म्बकं यजामहे सुग॒न्धिं पु॑ष्टि॒वर्ध॑नम्॥ उ॒र्वा॒रु॒कमि॑व॒ बन्ध॑नान्मृ॒त्योर्मु॑क्षीय॒ मा\-ऽमृता᳚त्॥ ए॒ष ते॑ रुद्र भा॒गस्तं जु॑षस्व॒ तेना॑व॒सेन॑ प॒रो मूज॑व॒तो\-ऽती॒ह्यव॑ततधन्वा॒ पिना॑कहस्तः॒ कृत्ति॑वासाः॥~(११)

%1.8.7.0
{\anuvakamend[{सुभे॑षजमिहि॒ त्रीणि॑ च}]}%~(६)

%1.8.7.1
ऐ॒न्द्रा॒ग्नं द्वाद॑श\-कपालं वैश्वदे॒वं च॒रुमिन्द्रा॑य॒ शुना॒सीरा॑य पुरो॒डाशं॒ द्वाद॑श\-कपालं वाय॒व्यं॑ पयः॑ सौ॒र्यमेक॑कपालं द्वादशग॒वꣳ सीरं॒ दक्षि॑णा\-ऽ\-ऽग्ने॒यम॒ष्टा\-क॑पालं॒ निर्व॑पति रौ॒द्रं गा॑वीधु॒कं च॒रुमै॒न्द्रं दधि॑ वारु॒णं य॑व॒मयं॑ च॒रुं व॒हिनी॑ धे॒नुर्दक्षि॑णा॒ ये दे॒वाः पु॑रः॒सदो॒\-ऽग्निने᳚त्रा दक्षिण॒सदो॑ य॒मने᳚त्राः पश्चा॒थ्सदः॑ सवि॒तृने᳚त्रा उत्तर॒सदो॒ वरु॑णनेत्रा उपरि॒षदो॒ बृह॒स्पति॑नेत्रा रक्षो॒हण॒स्ते नः॑ पान्तु॒ ते नो॑\-ऽवन्तु॒ तेभ्यो॒~(१२)

%1.8.7.2
नम॒स्तेभ्यः॒ स्वाहा॒ समू॑ढ॒ꣳ॒ रक्षः॒ सन्द॑ग्ध॒ꣳ॒ रक्ष॑ इ॒दम॒हꣳ रक्षो॒\-ऽभि सं द॑हाम्य॒ग्नये॑ रक्षो॒घ्ने स्वाहा॑ य॒माय॑ सवि॒त्रे वरु॑णाय॒ बृह॒स्पत॑ये॒ दुव॑स्वते रक्षो॒घ्ने स्वाहा᳚ प्रष्टिवा॒ही रथो॒ दक्षि॑णा दे॒वस्य॑ त्वा सवि॒तुः प्र॑स॒वे᳚\-ऽश्विनो᳚र्बा॒हु\-भ्यां᳚ पू॒ष्णो हस्ता᳚भ्या॒ꣳ॒ रक्ष॑सो व॒धं जु॑होमि ह॒तꣳ रक्षो\-ऽव॑धिष्म॒ रक्षो॒ यद्वस्ते॒ तद्दक्षि॑णा~(१३)


%1.8.8.0
{\anuvakamend[{तेभ्यः॒ पञ्च॑चत्वारिꣳशच्च}]}%~(७)

%1.8.8.1
धा॒त्रे पु॑रो॒डाशं॒ द्वाद॑श\-कपालं॒ निर्व॑प॒त्यनु॑मत्यै च॒रुꣳ रा॒कायै॑ च॒रुꣳ सि॑नीवा॒ल्यै च॒रुं कु॒ह्वै॑ च॒रुं मि॑थु॒नौ गावौ॒ दक्षि॑णा\-ऽ\-ऽग्नावैष्णव॒मेका॑\-दश\-कपालं॒ निर्व॑पत्यैन्द्रावैष्ण॒वमेका॑\-दश\-कपालं वैष्ण॒वं त्रि॑कपा॒लं वा॑म॒नो व॒ही दक्षि॑णा\-ऽग्नीषो॒मीय॒मेका॑\-दश\-कपालं॒ निर्व॑पतीन्द्रा\-सो॒मीय॒\-मेका॑\-दश\-कपालꣳ सौ॒म्यं च॒रुं ब॒भ्रुर्दक्षि॑णा सोमापौ॒ष्णं च॒रुं निर्व॑पत्यैन्द्रापौ॒ष्णं च॒रुं पौ॒ष्णं च॒रुꣴ श्या॒मो दक्षि॑णा वैश्वान॒रं द्वाद॑श\-कपालं॒ निर्व॑पति॒ हिर॑ण्यं॒ दक्षि॑णा वारु॒णं य॑व॒मयं॑ च॒रुमश्वो॒ दक्षि॑णा॥~(१४)

%1.8.9.0
{\anuvakamend[{निर॒ष्टौ च}]}%~(८)

%1.8.9.1
बा॒र्॒\mbox{}ह॒स्प॒त्यं च॒रुं निर्व॑पति ब्र॒ह्मणो॑ गृ॒हे शि॑तिपृ॒ष्ठो दक्षि॑णै॒न्द्रमेका॑\-दश\-कपालꣳ राज॒न्य॑स्य गृ॒ह ऋ॑ष॒भो दक्षि॑णा\-ऽ\-ऽदि॒त्यं च॒रुं महि॑ष्यै गृ॒हे धे॒नुर्दक्षि॑णा नैर्\mbox{}ऋ॒तं च॒रुं प॑रिवृ॒क्त्यै॑ गृ॒हे कृ॒ष्णानां᳚ व्रीही॒णां न॒खनि॑र्भिन्नं कृ॒ष्णा कू॒टा दक्षि॑णा\-ऽ\-ऽग्ने॒यम॒ष्टाक॑पालꣳ सेना॒न्यो॑ गृ॒हे हिर॑ण्यं॒ दक्षि॑णा वारु॒णं दश॑\-कपालꣳ सू॒तस्य॑ गृ॒हे म॒हानि॑रष्टो॒ दक्षि॑णा मारु॒तꣳ स॒प्तक॑पालं ग्राम॒ण्यो॑ गृ॒हे पृश्ञि॒र्दक्षि॑णा सावि॒त्रं द्वाद॑श\-कपालं~(१५)

%1.8.9.2
क्ष॒त्तुर्गृ॒ह उ॑पध्व॒स्तो दक्षि॑णा\-ऽ\-ऽश्वि॒नं द्वि॑कपा॒लꣳ स॑ङ्ग्रही॒तुर्गृ॒हे स॑वा॒त्यौ॑ दक्षि॑णा पौ॒ष्णं च॒रुं भा॑गदु॒घस्य॑ गृ॒हे श्या॒मो दक्षि॑णा रौ॒द्रं गा॑वीधु॒कं च॒रुम॑क्षावा॒पस्य॑ गृ॒हे श॒बल॒ उद्वा॑रो॒ दक्षि॒णेन्द्रा॑य सु॒त्राम्णे॑ पुरो॒डाश॒मेका॑\-दश\-कपालं॒ प्रति॒ निर्व॑प॒तीन्द्रा॑याꣳहो॒मुचे॒\-ऽयं नो॒ राजा॑ वृत्र॒हा राजा॑ भू॒त्वा वृ॒त्रं व॑ध्यान्मैत्राबार्\mbox{}हस्प॒त्यं भ॑वति श्वे॒तायै᳚ श्वे॒तव॑थ्सायै दु॒ग्धे स्व॑यम्मू॒र्ते स्व॑यम्मथि॒त आज्य॒ आश्व॑त्थे॒~(१६)

%1.8.9.3
पात्रे॒ चतुः॑स्रक्तौ स्वयमवप॒न्नायै॒ शाखा॑यै क॒र्णाꣴश्चा\-क॑र्णाꣴश्च तण्डु॒लान् वि चि॑नुया॒द्ये क॒र्णाः स पय॑सि बार्\mbox{}हस्प॒त्यो ये\-ऽक॑र्णाः॒ स आज्ये॑ मै॒त्रः स्व॑यं कृ॒ता वेदि॑र्भवति स्वयन्दि॒नं ब॒र्\mbox{}हिः स्व॑यं कृ॒त इ॒ध्मः सैव श्वे॒ता श्वे॒तव॑थ्सा॒ दक्षि॑णा~(१७)


%1.8.10.0
{\anuvakamend[{सा॒वि॒त्रं द्वाद॑श\-कपाल॒माश्व॑त्थे॒ त्रय॑स्त्रिꣳशच्च}]}%~(९)

%1.8.10.1
अ॒ग्नये॑ गृ॒हप॑तये पुरो॒डाश॑म॒ष्टा\-क॑पालं॒ निर्व॑पति कृ॒ष्णानां᳚ व्रीही॒णाꣳ सोमा॑य॒ वन॒स्पत॑ये श्यामा॒कं च॒रुꣳ स॑वि॒त्रे स॒त्यप्र॑सवाय पुरो॒डाशं॒ द्वाद॑श\-कपालमाशू॒नां व्री॑ही॒णाꣳ रु॒द्राय॑ पशु॒पत॑ये गावीधु॒कं च॒रुं बृह॒स्पत॑ये वा॒चस्पत॑ये नैवा॒रं च॒रुमिन्द्रा॑य ज्ये॒ष्ठाय॑ पुरो॒डाश॒मेका॑\-दश\-कपालं म॒हाव्री॑हीणां मि॒त्राय॑ स॒त्याया॒\-ऽ\-ऽम्बानां᳚ च॒रुं वरु॑णाय॒ धर्म॑पतये यव॒मयं॑ च॒रुꣳ स॑वि॒ता त्वा᳚ प्रस॒वानाꣳ॑ सुवताम॒ग्निर्गृ॒हप॑तीना॒ꣳ॒ सोमो॒ वन॒स्पती॑नाꣳ रु॒द्रः प॑शू॒नां~(१८)

%1.8.10.2
बृह॒स्पति॑र्वा॒चामिन्द्रो᳚ ज्ये॒ष्ठानां᳚ मि॒त्रः स॒त्यानां॒ वरु॑णो॒ धर्म॑पतीनां॒ ये दे॑वा देव॒सुवः॒ स्थ त इ॒ममा॑मुष्याय॒णम॑\-नमि॒त्राय॑ सुवध्वं मह॒ते क्ष॒त्राय॑ मह॒त आधि॑पत्याय मह॒ते जान॑राज्यायै॒ष वो॑ भरता॒ राजा॒ सोमो॒\-ऽस्माकं॑ ब्राह्म॒णाना॒ꣳ॒ राजा॒ प्रति॒ त्यन्नाम॑ रा॒ज्यम॑धायि॒ स्वां त॒नुवं॒ वरु॑णो अशिश्रे॒च्छुचे᳚र्मि॒त्रस्य॒ व्रत्या॑ अभू॒माम॑न्महि मह॒त ऋ॒तस्य॒ नाम॒ सर्वे॒ व्राता॒ वरु॑णस्याभूव॒न्वि मि॒त्र एवै॒ररा॑तिमतारी॒दसू॑षुदन्त य॒ज्ञिया॑ ऋ॒तेन॒ व्यु॑ त्रि॒तो ज॑रि॒माणं॑ न आन॒ड् विष्णोः॒ क्रमो॑\-ऽसि॒ विष्णोः᳚ क्रा॒न्तम॑सि॒ विष्णो॒र्विक्रा᳚न्तमसि॥~(१९)

%1.8.11.0
{\anuvakamend[{प॒शू॒नां व्राताः॒ पञ्च॑विꣳशतिश्च}]}%॥10॥

%1.8.11.1
अ॒र्थेतः॑ स्था॒\-ऽपां पति॑रसि॒ वृषा᳚स्यू॒र्मिर्वृ॑षसे॒नो॑\-ऽसि व्रज॒क्षितः॑ स्थ म॒रुता॒मोजः॑ स्थ॒ सूर्य॑वर्चसः स्थ॒ सूर्य॑त्वचसः स्थ॒ मान्दाः᳚ स्थ॒ वाशाः᳚ स्थ॒ शक्व॑रीः स्थ विश्व॒भृतः॑ स्थ जन॒भृतः॑ स्था॒\-ऽग्नेस्ते॑ज॒स्याः᳚ स्था॒\-ऽपामोष॑धीना॒ꣳ॒ रसः॑ स्था॒\-ऽपो दे॒वीर्मधु॑मतीरगृह्ण॒न्नूर्ज॑स्वती राज॒सूया॑य॒ चिता॑नाः। याभि॑र्मि॒त्रावरु॑णाव॒भ्यषि॑ञ्च॒न्॒ याभि॒रिन्द्र॒मन॑य॒न्नत्यरा॑तीः॥ रा॒ष्ट्र॒दाः स्थ॑ रा॒ष्ट्रं द॑त्त॒ स्वाहा॑ राष्ट्र॒दाः स्थ॑ रा॒ष्ट्रम॒मुष्मै॑ दत्त॥~(२०)

%1.8.12.0
{\anuvakamend[{अत्येका॑\-दश च}]}%॥11॥

%1.8.12.1
देवी॑रापः॒ सं मधु॑मती॒र्मधु॑मतीभिः सृज्यध्वं॒ महि॒ वर्चः॑ क्ष॒त्रिया॑य वन्वा॒ना अना॑धृष्टाः सीद॒तोर्ज॑स्वती॒र्महि॒ वर्चः॑ क्ष॒त्रिया॑य॒ दध॑ती॒रनि॑भृष्टमसि वा॒चो बन्धु॑स्तपो॒जाः सोम॑स्य दा॒त्रम॑सि शु॒क्रा वः॑ शु॒क्रेणोत्पु॑नामि च॒न्द्राश्च॒न्द्रेणा॒मृता॑ अ॒मृते॑न॒ स्वाहा॑ राज॒सूया॑य॒ चिता॑नाः॥ स॒ध॒मादो᳚ द्यु॒म्निनी॒रूर्ज॑ ए॒ता अनि॑भृष्टा अप॒स्युवो॒ वसा॑नः। प॒स्त्या॑सु चक्रे॒ वरु॑णः स॒धस्थ॑म॒पाꣳ शिशु॑र्-~(२१)

%1.8.12.2
मा॒तृत॑मास्व॒न्तः॥ क्ष॒त्रस्योल्ब॑मसि क्ष॒त्रस्य॒ योनि॑र॒स्यावि॑न्नो अ॒ग्निर्गृ॒हप॑ति॒रावि॑न्न॒ इन्द्रो॑ वृ॒द्धश्र॑वा॒ आवि॑न्नः पू॒षा वि॒श्ववे॑दा॒ आवि॑न्नौ मि॒त्रावरु॑णावृता॒वृधा॒वावि॑न्ने॒ द्यावा॑पृथि॒वी धृ॒तव्र॑ते॒ आवि॑न्ना दे॒व्यदि॑तिर्विश्वरू॒प्यावि॑न्नो॒\-ऽयम॒सावा॑मुष्याय॒णो᳚\-ऽस्यां वि॒श्य॑स्मिन् रा॒ष्ट्रे म॑ह॒ते क्ष॒त्राय॑ मह॒त आधि॑पत्याय मह॒ते जान॑राज्यायै॒ष वो॑ भरता॒ राजा॒ सोमो॒\-ऽस्माकं॑ ब्राह्म॒णाना॒ꣳ॒ राजेन्द्र॑स्य॒~(२२)

%1.8.12.3
वज्रो॑\-ऽसि॒ वार्त्र॑घ्न॒स्त्वया॒ऽयं वृ॒त्रं व॑ध्याच्छत्रु॒बाध॑नाः स्थ पा॒त मा᳚ प्र॒त्यञ्चं॑ पा॒त मा॑ ति॒र्यञ्च॑म॒न्वञ्चं॑ मा पात दि॒ग्भ्यो मा॑ पात॒ विश्वा᳚भ्यो मा ना॒ष्ट्राभ्यः॑ पात॒ हिर॑ण्यवर्णावु॒षसां᳚ विरो॒के\-ऽयः॑ स्थूणा॒वुदि॑तौ॒ सूर्य॒स्या\-ऽ\-ऽरो॑हतं वरुण मित्र॒ गर्तं॒ तत॑श्चक्षाथा॒मदि॑तिं॒ दितिं॑ च॥~(२३)

%1.8.13.0
{\anuvakamend[{शिशु॒रिन्द्र॒स्यैक॑चत्वारिꣳशच्च}]}%॥12॥

%1.8.13.1
स॒मिध॒मा ति॑ष्ठ गाय॒त्री त्वा॒ छन्द॑सामवतु त्रि॒वृथ्स्तोमो॑ रथन्त॒रꣳ सामा॒ग्निर्दे॒वता॒ ब्रह्म॒ द्रवि॑णमु॒ग्रामा ति॑ष्ठ त्रि॒ष्टुप् त्वा॒ छन्द॑सामवतु पञ्चद॒शः स्तोमो॑ बृ॒हथ्सामेन्द्रो॑ दे॒वता᳚ क्ष॒त्रं द्रवि॑णं वि॒राज॒मा ति॑ष्ठ॒ जग॑ती त्वा॒ छन्द॑सामवतु सप्तद॒शः स्तोमो॑ वैरू॒पꣳ साम॑ म॒रुतो॑ दे॒वता॒ विड्द्रवि॑ण॒मुदी॑ची॒मा ति॑ष्ठानु॒ष्टुप् त्वा॒~(२४)

%1.8.13.2
छन्द॑सामवत्वेकवि॒ꣳ॒शः स्तोमो॑ वैरा॒जꣳ साम॑ मि॒त्रावरु॑णौ दे॒वता॒ बलं॒ द्रवि॑णमू॒र्ध्वामा ति॑ष्ठ प॒ङ्क्तिस्त्वा॒ छन्द॑सामवतु त्रिणवत्रयस्त्रि॒ꣳ॒शौ स्तोमौ॑ शाक्वररैव॒ते साम॑नी॒ बृह॒स्पति॑र्दे॒वता॒ वर्चो॒ द्रवि॑णमी॒दृङ् चा᳚न्या॒दृङ् चै॑ता॒दृङ् च॑ प्रति॒दृङ् च॑ मि॒तश्च॒ सम्मि॑तश्च॒ सभ॑राः। शु॒क्रज्यो॑तिश्च चि॒त्रज्यो॑तिश्च स॒त्यज्यो॑तिश्च॒ ज्योति॑ष्माꣴश्च स॒त्यश्च॑र्त॒पाश्चा-~(२५)

%1.8.13.3
त्यꣳ॑हाः। अ॒ग्नये॒ स्वाहा॒ सोमा॑य॒ स्वाहा॑ सवि॒त्रे स्वाहा॒ सर॑स्वत्यै॒ स्वाहा॑ पू॒ष्णे स्वाहा॒ बृह॒स्पत॑ये॒ स्वाहेन्द्रा॑य॒ स्वाहा॒ घोषा॑य॒ स्वाहा॒ श्लोका॑य॒ स्वाहा\-ऽꣳशा॑य॒ स्वाहा॒ भगा॑य॒ स्वाहा॒ क्षेत्र॑स्य॒ पत॑ये॒ स्वाहा॑ पृथि॒व्यै स्वाहा॒\-ऽन्तरि॑क्षाय॒ स्वाहा॑ दि॒वे स्वाहा॒ सूर्या॑य॒ स्वाहा॑ च॒न्द्रम॑से॒ स्वाहा॒ नक्ष॑त्रेभ्यः॒ स्वाहा॒\-ऽद्भ्यः स्वाहौष॑धीभ्यः॒ स्वाहा॒ वन॒स्पति॑भ्यः॒ स्वाहा॑ चराच॒रेभ्यः॒ स्वाहा॑ परिप्ल॒वेभ्यः॒ स्वाहा॑ सरीसृ॒पेभ्यः॒ स्वाहा᳚॥~(२६)

%1.8.14.0
{\anuvakamend[{अ॒नु॒ष्टुप्त्व॑र्त॒पाश्च॑ सरीसृ॒पेभ्यः॒ स्वाहा᳚}]}%॥13॥

%1.8.14.1
सोम॑स्य॒ त्विषि॑रसि॒ तवे॑व मे॒ त्विषि॑र्भूयाद॒मृत॑मसि मृ॒त्योर्मा॑ पाहि दि॒द्योन्मा॑ पा॒ह्यवे᳚ष्टा दन्द॒शूका॒ निर॑स्तं॒ नमु॑चेः॒ शिरः॑॥ सोमो॒ राजा॒ वरु॑णो दे॒वा ध॑र्म॒सुव॑श्च॒ ये। ते ते॒ वाचꣳ॑ सुवन्तां॒ ते ते᳚ प्रा॒णꣳ सु॑वन्तां॒ ते ते॒ चक्षुः॑ सुवन्तां॒ ते ते॒ श्रोत्रꣳ॑ सुवन्ता॒ꣳ॒ सोम॑स्य त्वा द्यु॒म्नेना॒भिषि॑ञ्चाम्य॒ग्नेस्-~(२७)

%1.8.14.2
तेज॑सा॒ सूर्य॑स्य॒ वर्च॒सेन्द्र॑स्येन्द्रि॒येण॑ मि॒त्रावरु॑णयोर्वी॒र्ये॑ण म॒रुता॒मोज॑सा क्ष॒त्राणां᳚ क्ष॒त्रप॑तिर॒स्यति॑ दि॒वस्पा॑हि स॒माव॑वृत्रन्नध॒रागुदी॑ची॒रहिं॑ बु॒ध्निय॒मनु॑ स॒ञ्चर॑न्ती॒स्ताः पर्व॑तस्य वृष॒भस्य॑ पृ॒ष्ठे नाव॑श्चरन्ति स्व॒सिच॑ इया॒नाः॥ रुद्र॒ यत्ते॒ क्रयी॒ परं॒ नाम॒ तस्मै॑ हु॒तम॑सि य॒मेष्ट॑मसि। प्रजा॑पते॒ न त्वदे॒तान्य॒न्यो विश्वा॑ जा॒तानि॒ परि॒ ता ब॑भूव। यत्का॑मास्ते जुहु॒मस्तन्नो॑ अस्तु व॒यꣴ स्या॑म॒ पत॑यो रयी॒णाम्॥~(२८)

%1.8.15.0
{\anuvakamend[{अ॒ग्नेस्तैका॑\-दश च}]}%॥14॥

%1.8.15.1
इन्द्र॑स्य॒ वज्रो॑\-ऽसि॒ वार्त्र॑घ्न॒स्त्वया॒\-ऽयं वृ॒त्रं व॑ध्यान्मि॒त्रावरु॑ण\-योस्त्वा प्रशा॒स्त्रोः प्र॒शिषा॑ युनज्मि य॒ज्ञस्य॒ योगे॑न॒ विष्णोः॒ क्रमो॑\-ऽसि॒ विष्णोः᳚ क्रा॒न्तम॑सि॒ विष्णो॒र्विक्रा᳚न्तमसि म॒रुतां᳚ प्रस॒वे जे॑षमा॒प्तं मनः॒ सम॒हमि॑न्द्रि॒येण॑ वी॒र्ये॑ण पशू॒नां म॒न्युर॑सि॒ तवे॑व मे म॒न्युर्भू॑या॒न्नमो॑ मा॒त्रे पृ॑थि॒व्यै मा\-ऽहं मा॒तरं॑ पृथि॒वीꣳ हिꣳ॑सिषं॒ मा~(२९)

%1.8.15.2
मां मा॒ता पृ॑थि॒वी हिꣳ॑सी॒दिय॑द॒स्यायु॑र॒स्यायु॑र्मे धे॒ह्यूर्ग॒स्यूर्जं॑ मे धेहि॒ युङ्ङ॑सि॒ वर्चो॑\-ऽसि॒ वर्चो॒ मयि॑ धेह्य॒ग्नये॑ गृ॒हप॑तये॒ स्वाहा॒ सोमा॑य॒ वन॒स्पत॑ये॒ स्वाहेन्द्र॑स्य॒ बला॑य॒ स्वाहा॑ म॒रुता॒मोज॑से॒ स्वाहा॑ ह॒ꣳ॒सः शु॑चि॒षद्वसु॑रन्तरिक्ष॒\-सद्धोता॑ वेदि॒षदति॑थिर्दुरोण॒सत्। नृ॒षद्व॑र॒सदृ॑त॒सद्व्यो॑म॒सद॒ब्जा गो॒जा ऋ॑त॒जा अ॑द्रि॒जा ऋ॒तं बृ॒हत्॥~(३०)

%1.8.16.0
{\anuvakamend[{हि॒ꣳ॒सि॒षं॒ मर्त॒जास्त्रीणि॑ च}]}%॥15॥

%1.8.16.1
मि॒त्रो॑\-ऽसि॒ वरु॑णो\-ऽसि॒ सम॒हं विश्वै᳚र्दे॒वैः क्ष॒त्रस्य॒ नाभि॑रसि क्ष॒त्रस्य॒ योनि॑रसि स्यो॒नामा सी॑द सु॒षदा॒मा सी॑द॒ मा त्वा॑ हिꣳसी॒न्मा मा॑ हिꣳसी॒न्निष॑साद धृ॒तव्र॑तो॒ वरु॑णः प॒स्त्या᳚स्वा साम्रा᳚ज्याय सु॒क्रतु॒र्ब्रह्मा(३)न् त्वꣳ रा॑जन् ब्र॒ह्मा\-ऽसि॑ सवि॒ता\-ऽसि॑ स॒त्यस॑वो॒ ब्रह्मा(३)न् त्वꣳ रा॑जन् ब्र॒ह्मा\-ऽसीन्द्रो॑\-ऽसि स॒त्यौजा॒~(३१)

%1.8.16.2
ब्रह्मा(३)न् त्वꣳ रा॑जन् ब्र॒ह्मा\-ऽसि॑ मि॒त्रो॑\-ऽसि सु॒शेवो॒ ब्रह्मा(३)न् त्वꣳ रा॑जन् ब्र॒ह्मा\-ऽसि॒ वरु॑णो\-ऽसि स॒त्यध॒र्मेन्द्र॑स्य॒ वज्रो॑\-ऽसि॒ वार्त्र॑घ्न॒स्तेन॑ मे रध्य॒ दिशो॒\-ऽभ्य॑यꣳ राजा॑\-ऽभू॒थ्सुश्लो॒काँ(४) सुम॑ङ्ग॒लाँ(४) सत्य॑रा॒जा(३)न्। अ॒पां नप्त्रे॒ स्वाहो॒र्जो नप्त्रे॒ स्वाहा॒\-ऽग्नये॑ गृ॒हप॑तये॒ स्वाहा᳚॥~(३२)

%1.8.17.0
{\anuvakamend[{स॒त्यौजा᳚श्चत्वारि॒ꣳ॒शच्च॑}]}%॥16॥

%1.8.17.1
आ॒ग्ने॒यम॒ष्टा\-क॑पालं॒ निर्व॑पति॒ हिर॑ण्यं॒ दक्षि॑णा सारस्व॒तं च॒रुं व॑थ्सत॒री दक्षि॑णा सावि॒त्रं द्वाद॑श\-कपालमुपध्व॒स्तो दक्षि॑णा पौ॒ष्णं च॒रुꣴ श्या॒मो दक्षि॑णा बार्\mbox{}हस्प॒त्यं च॒रुꣳ शि॑तिपृ॒ष्ठो दक्षि॑णै॒न्द्रमेका॑\-दश\-कपालमृष॒भो दक्षि॑णा वारु॒णं दश॑\-कपालं म॒हानि॑रष्टो॒ दक्षि॑णा सौ॒म्यं च॒रुं ब॒भ्रुर्दक्षि॑णा त्वा॒ष्ट्रम॒ष्टाक॑पालꣳ शु॒ण्ठो दक्षि॑णा वैष्ण॒वं त्रि॑कपा॒लं वा॑म॒नो दक्षि॑णा॥~(३३)

%1.8.18.0
{\anuvakamend[{आ॒ग्ने॒यं द्विच॑त्वारिꣳशत्}]}%॥17॥

%1.8.18.1
स॒द्यो दी᳚क्षयन्ति स॒द्यः सोमं॑ क्रीणन्ति पुण्डरिस्र॒जां प्र य॑च्छति द॒शभि॑र्वथ्सत॒रैः सोमं॑ क्रीणाति दश॒पेयो॑ भवति श॒तं ब्रा᳚ह्म॒णाः पि॑बन्ति सप्तद॒शꣴ स्तो॒त्रं भ॑वति प्राका॒शाव॑ध्व॒र्यवे॑ ददाति॒ स्रज॑मुद्गा॒त्रे रु॒क्मꣳ होत्रे\-ऽश्वं॑ प्रस्तोतृप्रतिह॒र्तृभ्यां॒ द्वाद॑श पष्ठौ॒हीर्ब्र॒ह्मणे॑ व॒शां मै᳚त्रावरु॒णाय॑र्\mbox{}ष॒भं ब्रा᳚ह्मणाच्छ॒ꣳ॒सिने॒ वास॑सी नेष्टापो॒तृभ्या॒ꣴ॒ स्थूरि॑ यवाचि॒तम॑च्छावा॒काया॑न॒ड्वाह॑म॒ग्नीधे॑ भार्ग॒वो होता॑ भवति श्राय॒न्तीयं॑ ब्रह्मसा॒मं भ॑वति वारव॒न्तीय॑मग्निष्टोमसा॒मꣳ सा॑रस्व॒तीर॒पो गृ॑ह्णाति॥~(३४)

%1.8.19.0
{\anuvakamend[{वा॒र॒व॒न्तीयं॑ च॒त्वारि॑ च}]}%॥18॥

%1.8.19.1
आ॒ग्ने॒यम॒ष्टा\-क॑पालं॒ निर्व॑पति॒ हिर॑ण्यं॒ दक्षि॑णै॒न्द्रमेका॑\-दश\-कपालमृष॒भो दक्षि॑णा वैश्वदे॒वं च॒रुं पि॒शङ्गी॑ पष्ठौ॒ही दक्षि॑णा मैत्रावरु॒णीमा॒मिक्षां᳚ व॒शा दक्षि॑णा बार्\mbox{}हस्प॒त्यं च॒रुꣳ शि॑तिपृ॒ष्ठो दक्षि॑णा\-ऽ\-ऽदि॒त्यां म॒ल्॒\mbox{}हां ग॒र्भिणी॒मा ल॑भते मारु॒तीं पृश्निं॑ पष्ठौ॒हीम॒श्वि\-भ्यां᳚ पू॒ष्णे पु॑रो॒डाशं॒ द्वाद॑श\-कपालं॒ निर्व॑पति॒ सर॑स्वते सत्य॒वाचे॑ च॒रुꣳ स॑वि॒त्रे स॒त्यप्र॑सवाय पुरो॒डाशं॒ द्वाद॑श\-कपालं तिसृध॒न्वꣳ शु॑ष्कदृ॒तिर्दक्षि॑णा॥~(३५)

%1.8.20.0
{\anuvakamend[{आ॒ग्ने॒यꣳ स॒प्तच॑त्वारिꣳशत्}]}%॥19॥

%1.8.20.1
आ॒ग्ने॒यम॒ष्टा\-क॑पालं॒ निर्व॑पति सौ॒म्यं च॒रुꣳ सा॑वि॒त्रं द्वाद॑श\-कपालं बार्\mbox{}हस्प॒त्यं च॒रुं त्वा॒ष्ट्रम॒ष्टाक॑पालं वैश्वान॒रं द्वाद॑श\-कपालं॒ दक्षि॑णो रथवाहनवा॒हो दक्षि॑णा सारस्व॒तं च॒रुं निर्व॑पति पौ॒ष्णं च॒रुं मै॒त्रं च॒रुं वा॑रु॒णं च॒रुं क्षै᳚त्रप॒त्यं च॒रुमा॑दि॒त्यं च॒रुमुत्त॑रो रथवाहनवा॒हो दक्षि॑णा॥~(३६)

%1.8.21.0
{\anuvakamend[{आ॒ग्ने॒यं चतु॑स्त्रिꣳशत्}]}%॥20॥

%1.8.21.1
स्वा॒द्वीं त्वा᳚ स्वा॒दुना॑ ती॒व्रां ती॒व्रेणा॒मृता॑म॒मृते॑न सृ॒जामि॒ सꣳसोमे॑न॒ सोमो᳚\-ऽस्य॒श्वि\-भ्यां᳚ पच्यस्व॒ सर॑स्वत्यै पच्य॒स्वेन्द्रा॑य सु॒त्राम्णे॑ पच्यस्व पु॒नातु॑ ते परि॒स्रुत॒ꣳ॒ सोम॒ꣳ॒ सूर्य॑स्य दुहि॒ता। वारे॑ण॒ शश्व॑ता॒ तना᳚॥ वा॒युः पू॒तः प॒वित्रे॑ण प्र॒त्यङ्ख्सोमो॒ अति॑द्रुतः। इन्द्र॑स्य॒ युज्यः॒ सखा᳚॥ कु॒विद॒ङ्ग यव॑मन्तो॒ यवं॑ चि॒द्यथा॒ दान्त्य॑नुपू॒र्वं वि॒यूय॑। इ॒हेहै॑षां कृणुत॒ भोज॑नानि॒ ये ब॒र्॒\mbox{}हिषो॒ नमो॑वृक्तिं॒ न ज॒ग्मुः॥ आ॒श्वि॒नं धू॒म्रमा ल॑भते सारस्व॒तं मे॒षमै॒न्द्रमृ॑ष॒भमै॒न्द्रमेका॑\-दश\-कपालं॒ निर्व॑पति सावि॒त्रं द्वाद॑श\-कपालं वारु॒णं दश॑\-कपाल॒ꣳ॒ सोम॑प्रतीकाः पितरस्तृप्णुत॒ वड॑बा॒ दक्षि॑णा॥~(३७)

%1.8.22.0
{\anuvakamend[{भोज॑नानि॒ षड्विꣳ॑शतिश्च}]}%॥21॥

%1.8.22.1
अग्ना॑विष्णू॒ महि॒ तद्वां᳚ महि॒त्वं वी॒तं घृ॒तस्य॒ गुह्या॑नि॒ नाम॑। दमे॑दमे स॒प्त रत्ना॒ दधा॑ना॒ प्रति॑ वां जि॒ह्वा घृ॒तमा च॑रण्येत्॥ अग्ना॑विष्णू॒ महि॒ धाम॑ प्रि॒यं वां᳚ वी॒थो घृ॒तस्य॒ गुह्या॑ जुषा॒णा। दमे॑दमे सुष्टु॒तीर्वा॑वृधा॒ना प्रति॑ वां जि॒ह्वा घृ॒तमुच्च॑रण्येत्॥ प्र णो॑ दे॒वी सर॑स्वती॒ वाजे॑भिर्वा॒जिनी॑वती। धी॒नाम॑वि॒त्र्य॑वतु। आ नो॑ दि॒वो बृ॑ह॒तः~(३८)

%1.8.22.2
पर्व॑ता॒दा सर॑स्वती यज॒ता ग॑न्तु य॒ज्ञम्। हवं॑ दे॒वी जु॑जुषा॒णा घृ॒ताची॑ श॒ग्मां नो॒ वाच॑मुश॒ती शृ॑णोतु॥ बृह॑स्पते जु॒षस्व॑ नो ह॒व्यानि॑ विश्वदेव्य। रास्व॒ रत्ना॑नि दा॒शुषे᳚॥ ए॒वा पि॒त्रे वि॒श्वदे॑वाय॒ वृष्णे॑ य॒ज्ञैर्वि॑धेम॒ नम॑सा ह॒विर्भिः॑। बृह॑स्पते सुप्र॒जा वी॒रव॑न्तो व॒यꣴ स्या॑म॒ पत॑यो रयी॒णाम्॥ बृह॑स्पते॒ अति॒ यद॒र्यो अर्\mbox{}हा᳚द्द्यु॒मद्वि॒भाति॒ क्रतु॑म॒ज्जने॑षु। यद्दी॒दय॒च्छव॑स-~(३९)

%1.8.22.3
-र्तप्रजात॒ तद॒स्मासु॒ द्रवि॑णं धेहि चि॒त्रम्॥ आ नो॑ मित्रावरुणा घृ॒तैर्गव्यू॑तिमुक्षतम्। मध्वा॒ रजाꣳ॑सि सुक्रतू॥ प्र बा॒हवा॑ सिसृतं जी॒वसे॑ न॒ आ नो॒ गव्यू॑तिमुक्षतं घृ॒तेन॑। आ नो॒ जने᳚ श्रवयतं युवाना श्रु॒तं मे॑ मित्रावरुणा॒ हवे॒मा॥ अ॒ग्निं वः॑ पू॒र्व्यं गि॒रा दे॒वमी॑डे॒ वसू॑नाम्। स॒प॒र्यन्तः॑ पुरुप्रि॒यं मि॒त्रं न क्षे᳚त्र॒साध॑सम्॥ म॒क्षू दे॒वव॑तो॒ रथः॒~(४०)

%1.8.22.4
शूरो॑ वा पृ॒थ्सु कासु॑ चित्। दे॒वानां॒ य इन्मनो॒ यज॑मान॒ इय॑क्षत्य॒भीदय॑ज्वनो भुवत्॥ न य॑जमान रिष्यसि॒ न सु॑न्वान॒ न दे॑वयो॥ अस॒दत्र॑ सु॒वीर्य॑मु॒त त्यदा॒श्वश्वि॑यम्॥ नकि॒ष्टं कर्म॑णा नश॒न्न प्र यो॑ष॒न्न यो॑षति॥ उप॑ क्षरन्ति॒ सिन्ध॑वो मयो॒भुव॑ ईजा॒नं च॑ य॒क्ष्यमा॑णं च धे॒नवः॑। पृ॒णन्तं॑ च॒ पपु॑रिं च~(४१)

%1.8.22.5
श्रव॒स्यवो॑ घृ॒तस्य॒ धारा॒ उप॑ यन्ति वि॒श्वतः॑॥ सोमा॑रुद्रा॒ वि वृ॑हतं॒ विषू॑ची॒ममी॑वा॒ या नो॒ गय॑मावि॒वेश॑। आ॒रे बा॑धेथां॒ निर्\mbox{}ऋ॑तिं परा॒चैः कृ॒तं चि॒देनः॒ प्रमु॑मुक्तम॒स्मत्॥ सोमा॑रुद्रा यु॒वमे॒तान्य॒स्मे विश्वा॑ त॒नूषु॑ भेष॒जानि॑ धत्तम्। अव॑ स्यतं मु॒ञ्चतं॒ यन्नो॒ अस्ति॑ त॒नूषु॑ ब॒द्धं कृ॒तमेनो॑ अ॒स्मत्॥ सोमा॑पूषणा॒ जन॑ना रयी॒णां जन॑ना दि॒वो जन॑ना पृथि॒व्याः। जा॒तौ विश्व॑स्य॒ भुव॑नस्य गो॒पौ दे॒वा अ॑कृण्वन्न॒मृत॑स्य॒ नाभिम्᳚॥ इ॒मौ दे॒वौ जाय॑मानौ जुषन्ते॒मौ तमाꣳ॑सि गूहता॒मजु॑ष्टा। आ॒भ्यामिन्द्रः॑ प॒क्वमा॒मास्व॒न्तः सो॑मापू॒ष\-भ्यां᳚ जनदु॒स्रिया॑सु~(४२)

{\anuvakamend[{बृ॒ह॒तः शव॑सा॒ रथः॒ पपु॑रिं च दि॒वो जन॑ना॒ पञ्च॑विꣳशतिश्च}]}%॥22॥
%%% END KANDAM

