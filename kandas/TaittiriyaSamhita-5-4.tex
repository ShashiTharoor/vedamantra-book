\sect{चतुर्थः प्रश्नः}\setcounter{anuvakam}{0}
\dnsub{तैत्तिरीयसंहितायां पञ्चमकाण्डे चतुर्थः प्रश्नः}
%5.4.1.0
%5.4.1.1
दे॒वा॒सु॒राः संय॑त्ता आस॒न्ते न व्य॑जयन्त॒ स ए॒ता इन्द्र॑स्त॒नूर॑पश्य॒त्ता उपा॑धत्त॒ ताभि॒र्वै स त॒नुव॑मिन्द्रि॒यं वी॑र्यमा॒त्मन्न॑धत्त॒ ततो॑ दे॒वा अभ॑व॒न्परासु॑रा॒ यदि॑न्द्रत॒नूरु॑प॒दधा॑ति त॒नुव॑मे॒व ताभि॑रिन्द्रि॒यं वी॒र्यं॑ यज॑मान आ॒त्मन्ध॒त्ते\-ऽथो॒ सेन्द्र॑मे॒वाग्निꣳ सत॑नुं चिनुते॒ भव॑त्या॒त्मना॒ परा᳚स्य॒ भ्रातृ॑व्यः~(१)

%5.4.1.2
भ॒व॒ति॒ य॒ज्ञो दे॒वेभ्यो\-ऽपा᳚क्राम॒त्तम॑व॒रुधं॒ नाश॑क्नुव॒न्त ए॒ता य॑ज्ञत॒नूर॑पश्य॒न्ता उपा॑दधत॒ ताभि॒र्वै ते य॒ज्ञमवा॑रुन्धत॒ यद्य॑ज्ञत॒नूरु॑प॒दधा॑ति य॒ज्ञमे॒व ताभि॒र्यज॑मा॒नो\-ऽव॑ रुन्द्धे॒ त्रय॑स्त्रिꣳशत॒मुप॑ दधाति॒ त्रय॑स्त्रिꣳश॒द्वै दे॒वता॑ दे॒वता॑ ए॒वाव॑ रु॒न्द्धे\-ऽथो॒ सात्मा॑नमे॒वाग्निꣳ सत॑नुं चिनुते॒ सात्मा॒मुष्मि॑ल्लोँ॒के~(२)

%5.4.1.3
भ॒व॒ति॒ य ए॒वं वेद॒ ज्योति॑ष्मती॒रुप॑ दधाति॒ ज्योति॑रे॒वास्मि॑न्दधात्ये॒ताभि॒र्वा अ॒ग्निश्चि॒तो ज्व॑लति॒ ताभि॑रे॒वैन॒ꣳ॒ समि॑न्द्ध उ॒भयो॑रस्मै लो॒कयो॒र्ज्योति॑र्भवति नक्षत्रेष्ट॒का उप॑ दधात्ये॒तानि॒ वै दि॒वो ज्योतीꣳ॑षि॒ तान्ये॒वाव॑ रुन्द्धे सु॒कृतां॒ वा ए॒तानि॒ ज्योतीꣳ॑षि॒ यन्नक्ष॑त्राणि॒ तान्ये॒वाप्नो॒त्यथो॑ अनूका॒शमे॒वैतानि॑~(३)

%5.4.1.4
ज्योतीꣳ॑षि कुरुते सुव॒र्गस्य॑ लो॒कस्यानु॑ख्यात्यै॒ यथ्सꣴस्पृ॑ष्टा उपद॒ध्याद्वृष्ट्यै॑ लो॒कमपि॑ दध्या॒दव॑र्\mbox{}षुकः प॒र्जन्यः॑ स्या॒दसꣴ॑स्पृष्टा॒ उप॑ दधाति॒ वृष्ट्या॑ ए॒व लो॒कं क॑रोति॒ वर्\mbox{}षु॑कः प॒र्जन्यो॑ भवति पु॒रस्ता॑द॒न्याः प्र॒तीची॒रुप॑ दधाति प॒श्चाद॒न्याः प्राची॒स्तस्मा᳚त्प्रा॒चीना॑नि च प्रती॒चीना॑नि च॒ नक्ष॑त्रा॒ण्या व॑र्तन्ते॥~(४)

%5.4.2.0
{\anuvakamend[{भ्रातृ॑व्यो लो॒क ए॒वैतान्येक॑चत्वारिꣳशच्च}]}%~(१)

%5.4.2.1
ऋ॒त॒व्या॑ उप॑ दधात्यृतू॒नां कॢप्त्यै᳚ द्वं॒द्वमुप॑ दधाति॒ तस्मा᳚द्द्वं॒द्वमृ॒तवो\-ऽधृ॑तेव॒ वा ए॒षा यन्म॑ध्य॒मा चिति॑र॒न्तरि॑क्षमिव॒ वा ए॒षा द्वं॒द्वम॒न्यासु॒ चिती॒षूप॑ दधाति॒ चत॑स्रो॒ मध्ये॒ धृत्या॑ अन्तः॒श्लेष॑णं॒ वा ए॒ताश्चिती॑नां॒ यदृ॑त॒व्या॑ यदृ॑त॒व्या॑ उप॒दधा॑ति॒ चितीनां॒ विधृ॑त्या॒ अव॑का॒मनूप॑ दधात्ये॒षा वा अ॒ग्नेर्योनिः॒ सयो॑निम्~(५)

%5.4.2.2
ए॒वाग्निं चि॑नुत उ॒वाच॑ ह वि॒श्वामि॒त्रो\-ऽद॒दिथ्स ब्रह्म॒णान्नं॒ यस्यै॒ता उ॑पधी॒यान्तै॒ य उ॑ चैना ए॒वं वेद॒दिति॑ संवथ्स॒रो वा ए॒तम्प्र॑ति॒ष्ठायै॑ नुदते॒ यो᳚\-ऽग्निं चि॒त्वा न प्र॑ति॒तिष्ठ॑ति॒ पञ्च॒ पूर्वा॒श्चित॑यो भव॒न्त्यथ॑ ष॒ष्ठीं चितिं॑ चिनुते॒ षड्वा ऋ॒तवः॑ संवथ्स॒र ऋ॒तुष्वे॒व सं॑वथ्स॒रे प्रति॑ तिष्ठत्ये॒ता वै~(६)

%5.4.2.3
अधि॑पत्नी॒र्नामेष्ट॑का॒ यस्यै॒ता उ॑पधी॒यन्ते\-ऽधि॑पतिरे॒व स॑मा॒नानां᳚ भवति॒ यं द्वि॒ष्यात्तमु॑प॒दध॑द्ध्यायेदे॒ताभ्य॑ ए॒वैनं॑ दे॒वता᳚भ्य॒ आ वृ॑श्चति ता॒जगार्ति॒मार्च्छ॒त्यङ्गि॑रसः सुव॒र्गं लो॒कं यन्तो॒ या य॒ज्ञस्य॒ निष्कृ॑ति॒रासी॒त्तामृषि॑भ्यः॒ प्रत्यौ॑ह॒न् तद्धिर॑ण्यमभव॒द्यद्धि॑रण्यश॒ल्कैः प्रो॒क्षति॑ य॒ज्ञस्य॒ निष्कृ॑त्या॒ अथो॑ भेष॒जमे॒वास्मै॑ करोति~(७)

%5.4.2.4
अथो॑ रू॒पेणै॒वैन॒ꣳ॒ सम॑र्धय॒त्यथो॒ हिर॑ण्यज्योतिषै॒व सु॑व॒र्गं लो॒कमे॑ति साह॒स्रव॑ता॒ प्रोक्ष॑ति साह॒स्रः प्र॒जाप॑तिः प्र॒जाप॑ते॒राप्त्या॑ इ॒मा मे॑ अग्न॒ इष्ट॑का धे॒नवः॑ स॒न्त्वित्या॑ह धे॒नूरे॒वैनाः᳚ कुरुते॒ ता ए॑नं काम॒दुघा॑ अ॒मुत्रा॒मुष्मि॑ल्लोँ॒क उप॑ तिष्ठन्ते॥~(८)

%5.4.3.0
{\anuvakamend[{सयो॑निमे॒ता वै क॑रो॒त्येका॒न्नच॑त्वारि॒ꣳ॒शच्च॑}]}%~(२)

%5.4.3.1
रु॒द्रो वा ए॒ष यद॒ग्निः स ए॒तर्\mbox{}हि॑ जा॒तो यर्\mbox{}हि॒ सर्व॑श्चि॒तः स यथा॑ व॒थ्सो जा॒तः स्तन॑म्प्रे॒फ्सत्ये॒वं वा ए॒ष ए॒तर्\mbox{}हि॑ भाग॒धेय॒म्प्रेफ्स॑ति॒ तस्मै॒ यदाहु॑तिं॒ न जु॑हु॒याद॑ध्व॒र्युं च॒ यज॑मानं च ध्यायेच्छतरु॒द्रीयं॑ जुहोति भाग॒धेये॑नै॒वैनꣳ॑ शमयति॒ नार्ति॒मार्च्छ॑त्यध्व॒र्युर्न यज॑मानो॒ यद्ग्रा॒म्याणां᳚ पशू॒नाम्~(९)

%5.4.3.2
पय॑सा जुहु॒याद्ग्रा॒म्यान्प॒शूञ्छु॒चार्प॑ये॒द्यदा॑र॒ण्याना॑मार॒ण्याञ्ज॑र्तिलयवा॒ग्वा॑ वा जुहु॒याद्ग॑वीधुकयवा॒ग्वा॑ वा॒ न ग्रा॒म्यान्प॒शून् हि॒नस्ति॒ नार॒ण्यानथो॒ खल्वा॑हु॒रना॑हुति॒र्वै ज॒र्तिला᳚श्च ग॒वीधु॑का॒श्चेत्य॑जक्षी॒रेण॑ जुहोत्याग्ने॒यी वा ए॒षा यद॒जाहु॑त्यै॒व जु॑होति॒ न ग्रा॒म्यान्प॒शून् हि॒नस्ति॒ नार॒ण्यानङ्गि॑रसः सुव॒र्गं लो॒कं यन्तः॑~(१०)

%5.4.3.3
अ॒जायां᳚ घ॒र्मम्प्रासि॑ञ्च॒न्थ्सा शोच॑न्ती प॒र्णं परा॑जिहीत॒ सो \-ऽर्को॑\-ऽभव॒त्तद॒र्कस्या᳚र्क॒त्वम॑र्कप॒र्णेन॑ जुहोति सयोनि॒त्वायोद॒ङ्तिष्ठ॑ञ्जुहोत्ये॒षा वै रु॒द्रस्य॒ दिख्स्वाया॑मे॒व दि॒शि रु॒द्रं नि॒रव॑दयते चर॒माया॒मिष्ट॑कायां जुहोत्यन्त॒त ए॒व रु॒द्रं नि॒रव॑दयते त्रेधाविभ॒क्तं जु॑होति॒ त्रय॑ इ॒मे लो॒का इ॒माने॒व लो॒कान्थ्स॒माव॑द्वीर्यान्करो॒तीय॒त्यग्रे॑ जुहोति~(११)

%5.4.3.4
अथेय॒त्यथेय॑ति॒ त्रय॑ इ॒मे लो॒का ए॒भ्य ए॒वैनं॑ लो॒केभ्यः॑ शमयति ति॒स्र उत्त॑रा॒ आहु॑तीर्जुहोति॒ षट्थ्सम्प॑द्य॒न्ते षड्वा ऋ॒तव॑ ऋ॒तुभि॑रे॒वैनꣳ॑ शमयति॒ यद॑नुपरि॒क्रामं॑ जुहु॒याद॑न्तरवचा॒रिणꣳ॑ रु॒द्रं कु॑र्या॒दथो॒ खल्वा॑हुः॒ कस्यां॒ वाह॑ दि॒शि रु॒द्रः कस्यां॒ वेत्य॑नुपरि॒क्राम॑मे॒व हो॑त॒व्य॑मप॑रिवर्गमे॒वैनꣳ॑ शमयति~(१२)

%5.4.3.5
ए॒ता वै दे॒वताः᳚ सुव॒र्ग्या॑ या उ॑त्त॒मास्ता यज॑मानं वाचयति॒ ताभि॑रे॒वैनꣳ॑ सुव॒र्गं लो॒कं ग॑मयति॒ यं द्वि॒ष्यात्तस्य॑ सञ्च॒रे प॑शू॒नां न्य॑स्ये॒द्यः प्र॑थ॒मः प॒शुर॑भि॒तिष्ठ॑ति॒ स आर्ति॒मार्च्छ॑ति॥~(१३)

%5.4.4.0
{\anuvakamend[{प॒शू॒नां यन्तो\-ऽग्रे॑ जुहो॒त्यप॑रिवर्गमे॒वैनꣳ॑ शमयति त्रि॒ꣳ॒शच्च॑}]}%~(३)

%5.4.4.1
अश्म॒न्नूर्ज॒मिति॒ परि॑ षिञ्चति मा॒र्जय॑त्ये॒वैन॒मथो॑ त॒र्पय॑त्ये॒व स ए॑नं तृ॒प्तो\-ऽक्षु॑ध्य॒न्नशो॑चन्न॒मुष्मि॑ल्लोँ॒क उप॑ तिष्ठते॒ तृप्य॑ति प्र॒जया॑ प॒शुभि॒र्य ए॒वं वेद॒ तां न॒ इष॒मूर्जं॑ धत्त मरुतः सꣳररा॒णा इत्या॒हान्नं॒ वा ऊर्गन्न॑म्म॒रुतो\-ऽन्न॑मे॒वाव॑ रु॒न्द्धे\-ऽश्मꣴ॑स्ते॒ क्षुद॒मुं ते॒ शुक्~(१४)

%5.4.4.2
ऋ॒च्छ॒तु॒ यं द्वि॒ष्म इत्या॑ह॒ यमे॒व द्वेष्टि॒ तम॑स्य क्षु॒धा च॑ शु॒चा चा᳚र्पयति॒ त्रिः प॑रिषि॒ञ्चन्पर्ये॑ति त्रि॒वृद्वा अ॒ग्निर्यावा॑ने॒वाग्निस्तस्य॒ शुचꣳ॑ शमयति॒ त्रिः पुनः॒ पर्ये॑ति॒ षट्थ्सम्प॑द्यन्ते॒ षड्वा ऋ॒तव॑ ऋ॒तुभि॑रे॒वास्य॒ शुचꣳ॑ शमयत्य॒पां वा ए॒तत्पुष्पं॒ यद्वे॑त॒सो॑\-ऽपाम्~(१५)

%5.4.4.3
शरो\-ऽव॑का वेतसशा॒खया॒ चाव॑काभिश्च॒ वि क॑र्\mbox{}ष॒त्यापो॒ वै शा॒न्ताः शा॒न्ताभि॑रे॒वास्य॒ शुचꣳ॑ शमयति॒ यो वा अ॒ग्निं चि॒तम्प्र॑थ॒मः प॒शुर॑धि॒क्राम॑तीश्व॒रो वै तꣳ शु॒चा प्र॒दहो म॒ण्डूके॑न॒ वि क॑र्\mbox{}षत्ये॒ष वै प॑शू॒नाम॑नुपजीवनी॒यो न वा ए॒ष ग्रा॒म्येषु॑ प॒शुषु॑ हि॒तो नार॒ण्येषु॒ तमे॒व शु॒चार्प॑यत्यष्टा॒भिर्वि क॑र्\mbox{}षति~(१६)

%5.4.4.4
अ॒ष्टाक्ष॑रा गाय॒त्री गा॑य॒त्रो᳚\-ऽग्निर्यावा॑ने॒वाग्निस्तस्य॒ शुचꣳ॑ शमयति पाव॒कव॑तीभि॒रन्नं॒ वै पा॑व॒को\-ऽन्ने॑नै॒वास्य॒ शुचꣳ॑ शमयति मृ॒त्युर्वा ए॒ष यद॒ग्निर्ब्रह्म॑ण ए॒तद्रू॒पं यत्कृ॑ष्णाजि॒नम् कार्\mbox{}ष्णी॑ उपा॒नहा॒वुप॑ मुञ्चते॒ ब्रह्म॑णै॒व मृ॒त्योर॒न्तर्ध॑त्ते॒\-ऽन्तर्मृ॒त्योर्ध॑त्ते॒\-ऽन्तर॒न्नाद्या॒दित्या॑हुर॒न्यामु॑पमु॒ञ्चते॒\-ऽन्यां नान्तः~(१७)

%5.4.4.5
ए॒व मृ॒त्योर्ध॒त्ते\-ऽवा॒न्नाद्यꣳ॑ रुन्द्धे॒ नम॑स्ते॒ हर॑से शो॒चिष॒ इत्या॑ह नम॒स्कृत्य॒ हि वसी॑याꣳसमुप॒चर॑न्त्य॒न्यं ते॑ अ॒स्मत्त॑पन्तु हे॒तय॒ इत्या॑ह॒ यमे॒व द्वेष्टि॒ तम॑स्य शु॒चार्प॑यति पाव॒को अ॒स्मभ्यꣳ॑ शि॒वो भ॒वेत्या॒हान्नं॒ वै पा॑व॒को\-ऽन्न॑मे॒वाव॑ रुन्द्धे॒ द्वाभ्या॒मधि॑ क्रामति॒ प्रति॑ष्ठित्या अप॒स्य॑वतीभ्या॒ꣳ॒ शान्त्यै᳚॥~(१८)

%5.4.5.0
{\anuvakamend[{शुग्वे॑त॒सो॑\-ऽपाम॑ष्टा॒भिर्विक॑र्\mbox{}षति॒ नान्तरेका॒न्नप॑ञ्चा॒शच्च॑}]}%~(४)

%5.4.5.1
नृ॒षदे॒ वडिति॒ व्याघा॑रयति प॒ङ्क्त्याहु॑त्या यज्ञमु॒खमा र॑भते\-ऽक्ष्ण॒या व्याघा॑रयति॒ तस्मा॑दक्ष्ण॒या प॒शवो\-ऽङ्गा॑नि॒ प्र ह॑रन्ति॒ प्रति॑ष्ठित्यै॒ यद्व॑षट्कु॒र्याद्या॒तया॑मास्य वषट्का॒रः स्या॒द्यन्न व॑षट्कु॒॒र्याद्रक्षाꣳ॑सि य॒ज्ञꣳ ह॑न्यु॒र्वडित्या॑ह प॒रोक्ष॑मे॒व वष॑ट्करोति॒ नास्य॑ या॒तया॑मा वषट्का॒रो भव॑ति॒ न य॒ज्ञꣳ रक्षाꣳ॑सि घ्नन्ति हु॒तादो॒ वा अ॒न्ये दे॒वाः~(१९)

%5.4.5.2
अ॒हु॒तादो॒\-ऽन्ये तान॑ग्नि॒चिदे॒वोभया᳚न्प्रीणाति॒ ये दे॒वा दे॒वाना॒मिति॑ द॒ध्ना म॑धुमि॒श्रेणावो᳚क्षति हु॒ताद॑श्चै॒व दे॒वान॑हु॒ताद॑श्च॒ यज॑मानः प्रीणाति॒ ते यज॑मानम्प्रीणन्ति द॒ध्नैव हु॒तादः॑ प्री॒णाति॒ मधु॑षाहु॒तादो᳚ ग्रा॒म्यं वा ए॒तदन्नं॒ यद्दध्या॑र॒ण्यम्मधु॒ यद्द॒ध्ना म॑धुमि॒श्रेणा॒वोक्ष॑त्यु॒भय॒स्याव॑रुद्ध्यै ग्रुमु॒ष्टिनावो᳚क्षति प्राजाप॒त्यः~(२०)

%5.4.5.3
वै ग्रु॑मु॒ष्टिः स॑योनि॒त्वाय॒ द्वाभ्यां॒ प्रति॑ष्ठित्या अनुपरि॒चार॒मवो᳚क्ष॒त्यप॑रिवर्गमे॒वैना᳚न्प्रीणाति॒ वि वा ए॒ष प्रा॒णैः प्र॒जया॑ प॒शुभि॑र्\mbox{}ऋध्यते॒ यो᳚\-ऽग्निं चि॒न्वन्न॑धि॒क्राम॑ति प्राण॒दा अ॑पान॒दा इत्या॑ह प्रा॒णाने॒वात्मन्ध॑त्ते वर्चो॒दा व॑रिवो॒दा इत्या॑ह प्र॒जा वै वर्चः॑ प॒शवो॒ वरि॑वः प्र॒जामे॒व प॒शूना॒त्मन्ध॑त्त॒ इन्द्रो॑ वृ॒त्रम॑ह॒न्तं वृ॒त्रः~(२१)

%5.4.5.4
ह॒तः षो॑ड॒शभि॑र्भो॒गैर॑सिना॒थ्स ए॒ताम॒ग्नये\-ऽनी॑कवत॒ आहु॑तिमपश्य॒त्ताम॑जुहो॒त्तस्या॒ग्निरनी॑कवा॒न्थ्स्वेन॑ भाग॒धेये॑न प्री॒तः षो॑डश॒धा वृ॒त्रस्य॑ भो॒गानप्य॑दहद्वैश्वकर्म॒णेन॑ पा॒प्मनो॒ निर॑मुच्यत॒ यद॒ग्नये\-ऽनी॑कवत॒ आहु॑तिं जु॒होत्य॒ग्निरे॒वास्यानी॑कवा॒न्थ्स्वेन॑ भाग॒धेये॑न प्री॒तः पा॒प्मान॒मपि॑ दहति वैश्वकर्म॒णेन॑ पा॒प्मनो॒ निर्मु॑च्यते॒ यं का॒मये॑त चि॒रम्पा॒प्मनः॑~(२२)

%5.4.5.5
निर्मु॑च्ये॒तेत्येकै॑कं॒ तस्य॑ जुहुयाच्चि॒रमे॒व पा॒प्मनो॒ निर्मु॑च्यते॒ यं का॒मये॑त ता॒जक्पा॒प्मनो॒ निर्मु॑च्ये॒तेति॒ सर्वा॑णि॒ तस्या॑नु॒द्रुत्य॑ जुहुयात्ता॒जगे॒व पा॒प्मनो॒ निर्मु॑च्य॒ते\-ऽथो॒ खलु॒ नानै॒व सू॒क्ता\-भ्यां᳚ जुहोति॒ नानै॒व सू॒क्तयो᳚र्वी॒र्यं॑ दधा॒त्यथो॒ प्रति॑ष्ठित्यै॥~(२३)

%5.4.6.0
{\anuvakamend[{दे॒वाः प्रा॑जाप॒त्यो वृ॒त्रश्चि॒रं पा॒प्मन॑श्चत्वारि॒ꣳ॒शच्च॑}]}%~(५)

%5.4.6.1
उदे॑नमुत्त॒रां न॒येति॑ स॒मिध॒ आ द॑धाति॒ यथा॒ जनं॑ य॒ते॑\-ऽव॒सं क॒रोति॑ ता॒दृगे॒व तत्ति॒स्र आ द॑धाति त्रि॒वृद्वा अ॒ग्निर्यावा॑ने॒वाग्निस्तस्मै॑ भाग॒धेयं॑ करो॒त्यौदु॑म्बरीर्भव॒न्त्यूर्ग्वा उ॑दु॒म्बर॒ ऊर्ज॑मे॒वास्मा॒ अपि॑ दधा॒त्युदु॑ त्वा॒ विश्वे॑ दे॒वा इत्या॑ह प्रा॒णा वै विश्वे॑ दे॒वाः प्रा॒णैः~(२४)

%5.4.6.2
ए॒वैन॒मुद्य॑च्छ॒ते\-ऽग्ने॒ भर॑न्तु॒ चित्ति॑भि॒रित्या॑ह॒ यस्मा॑ ए॒वैनं॑ चि॒त्तायो॒द्यच्छ॑ते॒ तेनै॒वैन॒ꣳ॒ सम॑र्धयति॒ पञ्च॒ दिशो॒ दैवी᳚र्य॒ज्ञम॑वन्तु दे॒वीरित्या॑ह॒ दिशो॒ ह्ये॑षो\-ऽनु॑ प्र॒च्यव॒ते\-ऽपाम॑तिं दुर्म॒तिम्बाध॑माना॒ इत्या॑ह॒ रक्ष॑सा॒मप॑हत्यै रा॒यस्पोषे॑ य॒ज्ञप॑तिमा॒भज॑न्ती॒रित्या॑ह प॒शवो॒ वै रा॒यस्पोषः॑~(२५)

%5.4.6.3
प॒शूने॒वाव॑ रुन्द्धे ष॒ड्भिर्\mbox{}ह॑रति॒ षड्वा ऋ॒तव॑ ऋ॒तुभि॑रे॒वैनꣳ॑ हरति॒ द्वे प॑रि॒गृह्य॑वती भवतो॒ रक्ष॑सा॒मप॑हत्यै॒ सूर्य॑रश्मि॒र्\mbox{}हरि॑केशः पु॒रस्ता॒दित्या॑ह॒ प्रसू᳚त्यै॒ ततः॑ पाव॒का आ॒शिषो॑ नो जुषन्ता॒मित्या॒हान्नं॒ वै पा॑व॒को\-ऽन्न॑मे॒वाव॑ रुन्द्धे देवासु॒राः संय॑त्ता आस॒न्ते दे॒वा ए॒तदप्र॑तिरथमपश्य॒न्तेन॒ वै ते᳚\-ऽप्र॒ति~(२६)

%5.4.6.4
असु॑रानजय॒न्तदप्र॑तिरथस्याप्रतिरथ॒त्वं यदप्र॑तिरथं द्वि॒तीयो॒ होता॒न्वाहा᳚प्र॒त्ये॑व तेन॒ यज॑मानो॒ भ्रातृ॑व्याञ्जय॒त्यथो॒ अन॑भिजितमे॒वाभि ज॑यति दश॒र्चम्भ॑वति॒ दशा᳚क्षरा वि॒राड्वि॒राजे॒मौ लो॒कौ विधृ॑ताव॒नयो᳚र्लो॒कयो॒र्विधृ॑त्या॒ अथो॒ दशा᳚क्षरा वि॒राडन्नं॑ वि॒राड्वि॒राज्ये॒वान्नाद्ये॒ प्रति॑ तिष्ठ॒त्यस॑दिव॒ वा अ॒न्तरि॑क्षम॒न्तरि॑क्षमि॒वाग्नी᳚ध्र॒माग्नी᳚ध्रे~(२७)

%5.4.6.5
अश्मा॑नं॒ नि द॑धाति स॒त्त्वाय॒ द्वाभ्यां॒ प्रति॑ष्ठित्यै वि॒मान॑ ए॒ष दि॒वो मध्य॑ आस्त॒ इत्या॑ह॒ व्ये॑वैतया॑ मिमीते॒ मध्ये॑ दि॒वो निहि॑तः पृश्नि॒रश्मेत्या॒हान्नं॒ वै पृश्न्यन्न॑मे॒वाव॑ रुन्द्धे चत॒सृभि॒रा पुच्छा॑देति च॒त्वारि॒ छन्दाꣳ॑सि॒ छन्दो॑भिरे॒वेन्द्रं॒ विश्वा॑ अवीवृध॒न्नित्या॑ह॒ वृद्धि॑मे॒वोपाव॑र्तते॒ वाजा॑ना॒ꣳ॒ सत्प॑ति॒म्पतिम्᳚~(२८)

%5.4.6.6
इत्या॒हान्नं॒ वै वाजो\-ऽन्न॑मे॒वाव॑ रुन्द्धे सुम्न॒हूर्य॒ज्ञो दे॒वाꣳ आ च॑ वक्ष॒दित्या॑ह प्र॒जा वै प॒शवः॑ सु॒म्नं प्र॒जामे॒व प॒शूना॒त्मन्ध॑त्ते॒ यक्ष॑द॒ग्निर्दे॒वो दे॒वाꣳ आ च॑ वक्ष॒दित्या॑ह स्व॒गाकृ॑त्यै॒ वाज॑स्य मा प्रस॒वेनो᳚द्ग्रा॒भेणोद॑ग्रभी॒दित्या॑हा॒सौ वा आ॑दि॒त्य उ॒द्यन्नु॑द्ग्रा॒भ ए॒ष नि॒म्रोच॑न्निग्रा॒भो ब्रह्म॑णै॒वात्मान॑मुद्गृ॒ह्णाति॒ ब्रह्म॑णा॒ भ्रातृ॑व्यं॒ नि गृ॑ह्णाति॥~(२९)

%5.4.7.0
{\anuvakamend[{प्रा॒णैः पोषो᳚\-ऽप्र॒त्याग्नी᳚ध्रे॒ पति॑मे॒ष दश॑ च}]}%~(६)

%5.4.7.1
प्राची॒मनु॑ प्र॒दिश॒म्प्रेहि॑ वि॒द्वानित्या॑ह देवलो॒कमे॒वैतयो॒पाव॑र्तते॒ क्रम॑ध्वम॒ग्निना॒ नाक॒मित्या॑हे॒माने॒वैतया॑ लो॒कान्क्र॑मते पृथि॒व्या अ॒हमुद॒न्तरि॑क्ष॒मारु॑ह॒मित्या॑हे॒माने॒वैतया॑ लो॒कान्थ्स॒मारो॑हति॒ सुव॒र्यन्तो॒ नापे᳚क्षन्त॒ इत्या॑ह सुव॒र्गमे॒वैतया॑ लो॒कमे॒त्यग्ने॒ प्रेहि॑~(३०)

%5.4.7.2
प्र॒थ॒मो दे॑वय॒तामित्या॑हो॒भये᳚ष्वे॒वैतया॑ देवमनु॒ष्येषु॒ चक्षु॑र्दधाति प॒ञ्चभि॒रधि॑ क्रामति॒ पाङ्क्तो॑ य॒ज्ञो यावा॑ने॒व य॒ज्ञस्तेन॑ स॒ह सु॑व॒र्गं लो॒कमे॑ति॒ नक्तो॒षासेति॑ पुरोनुवा॒क्या॑मन्वा॑ह॒ प्रत्त्या॒ अग्ने॑ सहस्रा॒क्षेत्या॑ह साह॒स्रः प्र॒जाप॑तिः प्र॒जाप॑ते॒राप्त्यै॒ तस्मै॑ ते विधेम॒ वाजा॑य॒ स्वाहेत्या॒हान्नं॒ वै वाजो\-ऽन्न॑मे॒वाव॑~(३१)

%5.4.7.3
रु॒न्द्धे॒ द॒ध्नः पू॒र्णामौदु॑म्बरीꣴ स्वयमातृ॒ण्णायां᳚ जुहो॒त्यूर्ग्वै दध्यूर्गु॑दु॒म्बरो॒\-ऽसौ स्व॑यमातृ॒ण्णामुष्या॑मे॒वोर्जं॑ दधाति॒ तस्मा॑द॒मुतो॒\-ऽर्वाची॒मूर्ज॒मुप॑ जीवामस्ति॒सृभिः॑ सादयति त्रि॒वृद्वा अ॒ग्निर्यावा॑ने॒वाग्निस्तम्प्र॑ति॒ष्ठां ग॑मयति॒ प्रेद्धो॑ अग्ने दीदिहि पु॒रो न॒ इत्यौदुम्ब॑री॒मा द॑धात्ये॒षा वै सू॒र्मी कर्ण॑कावत्ये॒तया॑ ह स्म~(३२)

%5.4.7.4
वै दे॒वा असु॑राणाꣳ शतत॒र्\mbox{}हाꣴस्तृꣳ॑हन्ति॒ यदे॒तया॑ स॒मिध॑मा॒दधा॑ति॒ वज्र॑मे॒वैतच्छ॑त॒घ्नीं यज॑मानो॒ भ्रातृ॑व्याय॒ प्र ह॑रति॒ स्तृत्या॒ अछ॑म्बट्कारं वि॒धेम॑ ते पर॒मे जन्म॑न्नग्न॒ इति॒ वैक॑ङ्कती॒मा द॑धाति॒ भा ए॒वाव॑ रुन्द्धे॒ ताꣳ स॑वि॒तुर्वरे᳚ण्यस्य चि॒त्रामिति॑ शमी॒मयी॒ꣳ॒ शान्त्या॑ अ॒ग्निर्वा॑ ह॒ वा अ॑ग्नि॒चितं॑ दु॒हे᳚\-ऽग्नि॒चिद्वा॒ग्निं दु॑हे॒ ताम्~(३३)

%5.4.7.5
स॒वि॒तुर्वरे᳚ण्यस्य चि॒त्रामित्या॑है॒ष वा अ॒ग्नेर्दोह॒स्तम॑स्य॒ कण्व॑ ए॒व श्रा॑य॒सो॑\-ऽवे॒त्तेन॑ ह स्मैन॒ꣳ॒ स दु॑हे॒ यदे॒तया॑ स॒मिध॑मा॒दधा᳚त्यग्नि॒चिदे॒व तद॒ग्निं दु॑हे स॒प्त ते॑ अग्ने स॒मिधः॑ स॒प्त जि॒ह्वा इत्या॑ह स॒प्तैवास्य॒ साप्ता॑नि प्रीणाति पू॒र्णया॑ जुहोति पू॒र्ण इ॑व॒ हि प्र॒जाप॑तिः प्र॒जाप॑तेः~(३४)

%5.4.7.6
आप्त्यै॒ न्यू॑नया जुहोति॒ न्यू॑ना॒द्धि प्र॒जाप॑तिः प्र॒जा असृ॑जत प्र॒जाना॒ꣳ॒ सृष्ट्या॑ अ॒ग्निर्दे॒वेभ्यो॒ निला॑यत॒ स दिशो\-ऽनु॒ प्रावि॑श॒ज्जुह्व॒न्मन॑सा॒ दिशो᳚ ध्यायेद्दि॒ग्भ्य ए॒वैन॒मव॑ रुन्द्धे द॒ध्ना पु॒रस्ता᳚ज्जुहो॒त्याज्ये॑नो॒परि॑ष्टा॒त्तेज॑श्चै॒वास्मा॑ इन्द्रि॒यं च॑ स॒मीची॑ दधाति॒ द्वाद॑श\-कपालो वैश्वान॒रो भ॑वति॒ द्वाद॑श॒ मासाः᳚ संवथ्स॒रः सं॑वथ्स॒रो᳚\-ऽग्निर्वै᳚श्वान॒रः सा॒क्षात्~(३५)

%5.4.7.7
ए॒व वै᳚श्वान॒रमव॑ रुन्द्धे॒ यत्प्र॑याजानूया॒जान्कु॒र्याद्विक॑स्तिः॒ सा य॒ज्ञस्य॑ दर्विहो॒मं क॑रोति य॒ज्ञस्य॒ प्रति॑ष्ठित्यै रा॒ष्ट्रं वै वै᳚श्वान॒रो विण्म॒रुतो वैश्वान॒रꣳ हु॒त्वा मा॑रु॒ताञ्जु॑होति रा॒ष्ट्र ए॒व विश॒मनु॑ बध्नात्यु॒च्चैर्वै᳚श्वान॒रस्या श्रा॑वयत्युपा॒ꣳ॒शु मा॑रु॒ताञ्जु॑होति॒ तस्मा᳚द्रा॒ष्ट्रं विश॒मति॑ वदति मारु॒ता भ॑वन्ति म॒रुतो॒ वै दे॒वानां॒ विशो॑ देववि॒शेनै॒वास्मै॑ मनुष्यवि॒शमव॑ रुन्द्धे स॒प्त भ॑वन्ति स॒प्तग॑णा॒ वै म॒रुतो॑ गण॒श ए॒व विश॒मव॑ रुन्द्धे ग॒णेन॑ ग॒णम॑नु॒द्रुत्य॑ जुहोति॒ विश॑मे॒वास्मा॒ अनु॑वर्त्मानं करोति॥~(३६)

%5.4.8.0
{\anuvakamend[{अग्ने॒ प्रेह्यव॑ स्म दुहे॒ तां प्र॒जाप॑तेः सा॒क्षान्म॑नुष्यवि॒शमेक॑विꣳशतिश्च}]}%~(७)

%5.4.8.1
वसो॒र्धारां᳚ जुहोति॒ वसो᳚र्मे॒ धारा॑स॒दिति॒ वा ए॒षा हू॑यते घृ॒तस्य॒ वा ए॑नमे॒षा धारा॒मुष्मि॑ल्लोँ॒के पिन्व॑मा॒नोप॑ तिष्ठत॒ आज्ये॑न जुहोति॒ तेजो॒ वा आज्यं॒ तेजो॒ वसो॒र्धारा॒ तेज॑सै॒वास्मै॒ तेजो\-ऽव॑ रु॒न्द्धे\-ऽथो॒ कामा॒ वै वसो॒र्धारा॒ कामा॑ने॒वाव॑ रुन्द्धे॒ यं का॒मये॑त प्रा॒णान॑स्या॒न्नाद्यं॒ वि~(३७)

%5.4.8.2
छि॒न्द्या॒मिति॑ वि॒ग्राहं॒ तस्य॑ जुहुयात्प्रा॒णाने॒वास्या॒न्नाद्यं॒ विच्छि॑नत्ति॒ यं का॒मये॑त प्रा॒णान॑स्या॒न्नाद्य॒ꣳ॒ सं त॑नुया॒मिति॒ सन्त॑तां॒ तस्य॑ जुहुयात्प्रा॒णाने॒वास्या॒न्नाद्य॒ꣳ॒ सं त॑नोति॒ द्वाद॑श द्वाद॒शानि॑ जुहोति॒ द्वाद॑श॒ मासाः᳚ संवथ्स॒रः संवथ्स॒रेणै॒वास्मा॒ अन्न॒मव॑ रु॒न्द्धे\-ऽन्नं॑ च॒ मे\-ऽक्षु॑च्च म॒ इत्या॑है॒तद्वै~(३८)

%5.4.8.3
अन्न॑स्य रू॒पꣳ रू॒पेणै॒वान्न॒मव॑ रुन्द्धे॒\-ऽग्निश्च॑ म॒ आप॑श्च म॒ इत्या॑है॒षा वा अन्न॑स्य॒ योनिः॒ सयो᳚न्ये॒वान्न॒मव॑ रुन्द्धे\-ऽर्धे॒न्द्राणि॑ जुहोति दे॒वता॑ ए॒वाव॑ रुन्द्धे॒ यथ्सर्वे॑षाम॒र्धमिन्द्रः॒ प्रति॒ तस्मा॒दिन्द्रो॑ दे॒वता॑नाम्भूयिष्ठ॒भाक्त॑म॒ इन्द्र॒मुत्त॑रमाहेन्द्रि॒यमे॒वास्मि॑न्नु॒परि॑ष्टाद्दधाति यज्ञायु॒धानि॑ जुहोति य॒ज्ञः~(३९)

%5.4.8.4
वै य॑ज्ञायु॒धानि॑ य॒ज्ञमे॒वाव॑ रु॒न्द्धे\-ऽथो॑ ए॒तद्वै य॒ज्ञस्य॑ रू॒पꣳ रू॒पेणै॒व य॒ज्ञमव॑ रुन्द्धे\-ऽवभृ॒थश्च॑ मे स्वगाका॒रश्च॑ म॒ इत्या॑ह स्व॒गाकृ॑त्या अ॒ग्निश्च॑ मे घ॒र्मश्च॑ म॒ इत्या॑है॒तद्वै ब्र॑ह्मवर्च॒सस्य॑ रू॒पꣳ रू॒पेणै॒व ब्र॑ह्मवर्च॒समव॑ रुन्द्ध॒ ऋक्च॑ मे॒ साम॑ च म॒ इत्या॑ह~(४०)

%5.4.8.5
ए॒तद्वै छन्द॑साꣳ रू॒पꣳ रू॒पेणै॒व छन्दा॒ꣴ॒स्यव॑ रुन्द्धे॒ गर्भा᳚श्च मे व॒थ्साश्च॑ म॒ इत्या॑है॒तद्वै प॑शू॒नाꣳ रू॒पꣳ रू॒पेणै॒व प॒शूनव॑ रुन्द्धे॒ कल्पा᳚ञ्जुहो॒त्यकॢ॑प्तस्य॒ कॢप्त्यै॑ युग्मदयु॒जे जु॑होति मिथुन॒त्वायो᳚त्त॒राव॑ती भवतो॒\-ऽभिक्रा᳚न्त्या॒ एका॑ च मे ति॒स्रश्च॑ म॒ इत्या॑ह देवछन्द॒सं वा एका॑ च ति॒स्रश्च॑~(४१)

%5.4.8.6
म॒नु॒ष्य॒छ॒न्द॒सं चत॑स्रश्चा॒ष्टौ च॑ देवछन्द॒सं चै॒व म॑नुष्यछन्द॒सं चाव॑ रुन्द्ध॒ आ त्रय॑स्त्रिꣳशतो जुहोति॒ त्रय॑स्त्रिꣳश॒द्वै दे॒वता॑ दे॒वता॑ ए॒वाव॑ रुन्द्ध॒ आष्टाच॑त्वारिꣳशतो जुहोत्य॒ष्टाच॑त्वारिꣳशदक्षरा॒ जग॑ती॒ जाग॑ताः प॒शवो॒ जग॑त्यै॒वास्मै॑ प॒शूनव॑ रुन्द्धे॒ वाज॑श्च प्रस॒वश्चेति॑ द्वाद॒शं जु॑होति॒ द्वाद॑श॒ मासाः᳚ संवथ्स॒रः सं॑वथ्स॒र ए॒व प्रति॑ तिष्ठति॥~(४२)

%5.4.9.0
{\anuvakamend[{वि वै य॒ज्ञः साम॑ च म॒ इत्या॑ह च ति॒स्रश्चैका॒न्नप॑ञ्चा॒शच्च॑}]}%~(८)

%5.4.9.1
अ॒ग्निर्दे॒वेभ्यो\-ऽपा᳚क्रामद्भाग॒धेय॑मि॒च्छमा॑न॒स्तं दे॒वा अ॑ब्रुव॒न्नुप॑ न॒ आ व॑र्तस्व ह॒व्यं नो॑ व॒हेति॒ सो᳚\-ऽब्रवी॒द्वरं॑ वृणै॒ मह्य॑मे॒व वा॑जप्रस॒वीयं॑ जुहव॒न्निति॒ तस्मा॑द॒ग्नये॑ वाजप्रस॒वीयं॑ जुह्वति॒ यद्वा॑जप्रस॒वीयं॑ जु॒होत्य॒ग्निमे॒व तद्भा॑ग॒धेये॑न॒ सम॑र्धय॒त्यथो॑ अभिषे॒क ए॒वास्य॒ स च॑तुर्द॒शभि॑र्जुहोति स॒प्त ग्रा॒म्या ओष॑धयः स॒प्त~(४३)

%5.4.9.2
आ॒र॒ण्या उ॒भयी॑षा॒मव॑रुद्ध्या॒ अन्न॑स्यान्नस्य जुहो॒त्यन्न॑स्यान्न॒स्याव॑रुद्ध्या॒ औदु॑म्बरेण स्रु॒वेण॑ जुहो॒त्यूर्ग्वा उ॑दु॒म्बर॒ ऊर्गन्न॑मू॒र्जैवास्मा॒ ऊर्ज॒मन्न॒मव॑ रुन्द्धे॒\-ऽग्निर्वै दे॒वाना॑म॒भिषि॑क्तो\-ऽग्नि॒चिन्म॑नु॒ष्या॑णा॒न्तस्मा॑दग्नि॒चिद्वर्\mbox{}ष॑ति॒ न धा॑वे॒दव॑रुद्ध॒ꣴ॒ ह्य॑स्यान्न॒मन्न॑मिव॒ खलु॒ वै व॒र्\mbox{}षं यद्धावे॑द॒न्नाद्या᳚द्धावेदु॒पाव॑र्तेता॒न्नाद्य॑मे॒वाभि~(४४)

%5.4.9.3
उ॒पाव॑र्तते॒ नक्तो॒षासेति॑ कृ॒ष्णायै᳚ श्वे॒तव॑थ्सायै॒ पय॑सा जुहो॒त्यह्नै॒वास्मै॒ रात्रि॒म्प्र दा॑पयति॒ रात्रि॒याह॑रहोरा॒त्रे ए॒वास्मै॒ प्रत्ते॒ काम॑म॒न्नाद्यं॑ दुहाते राष्ट्र॒भृतो॑ जुहोति रा॒ष्ट्रमे॒वाव॑ रुन्द्धे ष॒ड्भिर्जु॑होति॒ षड्वा ऋ॒तव॑ ऋ॒तुष्वे॒व प्रति॑ तिष्ठति॒ भुव॑नस्य पत॒ इति॑ रथमु॒खे पञ्चाहु॑तीर्जुहोति॒ वज्रो॒ वै रथो॒ वज्रे॑णै॒व दिशः॑~(४५)

%5.4.9.4
अ॒भि ज॑यत्यग्नि॒चितꣳ॑ ह॒ वा अ॒मुष्मि॑ल्लोँ॒के वातो॒\-ऽभि प॑वते वातना॒मानि॑ जुहोत्य॒भ्ये॑वैन॑म॒मुष्मि॑ल्लोँ॒के वातः॑ पवते॒ त्रीणि॑ जुहोति॒ त्रय॑ इ॒मे लो॒का ए॒भ्य ए॒व लो॒केभ्यो॒ वात॒मव॑ रुन्द्धे समु॒द्रो॑\-ऽसि॒ नभ॑स्वा॒नित्या॑है॒तद्वै वात॑स्य रू॒पꣳ रू॒पेणै॒व वात॒मव॑ रुन्द्धे\-ऽञ्ज॒लिना॑ जुहोति॒ न ह्ये॑तेषा॑म॒न्यथाहु॑तिरव॒कल्प॑ते॥~(४६)

%5.4.10.0
{\anuvakamend[{ओष॑धयः स॒प्ताभि दिशो॒\-ऽन्यथा॒ द्वे च॑}]}%~(९)

%5.4.10.1
सुव॒र्गाय॒ वै लो॒काय॑ देवर॒थो यु॑ज्यते यत्राकू॒ताय॑ मनुष्यर॒थ ए॒ष खलु॒ वै दे॑वर॒थो यद॒ग्निर॒ग्निं यु॑नज्मि॒ शव॑सा घृ॒तेनेत्या॑ह यु॒नक्त्ये॒वैन॒ꣳ॒ स ए॑नं यु॒क्तः सु॑व॒र्गं लो॒कम॒भि व॑हति॒ यथ्सर्वा॑भिः प॒ञ्चभि॑र्यु॒ञ्ज्याद्यु॒क्तो᳚\-ऽस्या॒ग्निः प्रच्यु॑तः स्या॒दप्र॑तिष्ठिता॒ आहु॑तयः॒ स्युरप्र॑तिष्ठिताः॒ स्तोमा॒ अप्र॑तिष्ठितान्यु॒क्थानि॑ ति॒सृभिः॑ प्रातःसव॒ने॑\-ऽभि मृ॑शति त्रि॒वृत्~(४७)

%5.4.10.2
वा अ॒ग्निर्यावा॑ने॒वाग्निस्तं यु॑नक्ति॒ यथान॑सि यु॒क्त आ॑धी॒यत॑ ए॒वमे॒व तत्प्रत्याहु॑तय॒स्तिष्ठ॑न्ति॒ प्रति॒ स्तोमाः॒ प्रत्यु॒क्थानि॑ यज्ञाय॒ज्ञिय॑स्य स्तो॒त्रे द्वाभ्या॑म॒भि मृ॑शत्ये॒तावा॒न् वै य॒ज्ञो यावा॑नग्निष्टो॒मो भू॒मा त्वा अ॒स्यात॑ ऊ॒र्ध्वः क्रि॑यते॒ यावा॑ने॒व य॒ज्ञस्तम॑न्त॒तो᳚\-ऽन्वारो॑हति॒ द्वाभ्यां॒ प्रति॑ष्ठित्या॒ एक॒याप्र॑स्तुत॒म्भव॒त्यथ॑~(४८)

%5.4.10.3
अ॒भि मृ॑श॒त्युपै॑न॒मुत्त॑रो य॒ज्ञो न॑म॒त्यथो॒ सन्त॑त्यै॒ प्र वा ए॒षो᳚\-ऽस्माल्लो॒काच्च्य॑वते॒ यो᳚\-ऽग्निं चि॑नु॒ते न वा ए॒तस्या॑निष्ट॒क आहु॑ति॒रव॑ कल्पते॒ यां वा ए॒षो॑\-ऽनिष्ट॒क आहु॑तिं जु॒होति॒ स्रव॑ति॒ वै सा ताꣴ स्रव॑न्तीं य॒ज्ञो\-ऽनु॒ परा॑ भवति य॒ज्ञं यज॑मानो॒ यत्पु॑नश्चि॒तिं चि॑नु॒त आहु॑तीनां॒ प्रति॑ष्ठित्यै॒ प्रत्याहु॑तय॒स्तिष्ठ॑न्ति~(४९)

%5.4.10.4
न य॒ज्ञः प॑रा॒भव॑ति॒ न यज॑मानो॒\-ऽष्टावुप॑ दधात्य॒ष्टाक्ष॑रा गाय॒त्री गा॑य॒त्रेणै॒वैनं॒ छन्द॑सा चिनुते॒ यदेका॑\-दश॒ त्रैष्टु॑भेन॒ यद्द्वाद॑श॒ जाग॑तेन॒ छन्दो॑भिरे॒वैनं॑ चिनुते नपा॒त्को वै नामै॒षो᳚\-ऽग्निर्यत्पु॑नश्चि॒तिर्य ए॒वं वि॒द्वान्पु॑नश्चि॒तिं चि॑नु॒त आ तृ॒तीया॒त्पुरु॑षा॒दन्न॑मत्ति॒ यथा॒ वै पु॑नरा॒धेय॑ ए॒वम्पु॑नश्चि॒तिर्यो᳚\-ऽग्न्या॒धेये॑न॒ न~(५०)

%5.4.10.5
ऋ॒ध्नोति॒ स पु॑नरा॒धेय॒मा ध॑त्ते॒ यो᳚\-ऽग्निं चि॒त्वा नर्ध्नोति॒ स पु॑नश्चि॒तिं चि॑नुते॒ यत्पु॑नश्चि॒तिं चि॑नु॒त ऋद्ध्या॒ अथो॒ खल्वा॑हु॒र्न चे॑त॒व्येति॑ रु॒द्रो वा ए॒ष यद॒ग्निर्यथा᳚ व्या॒घ्रꣳ सु॒प्तम्बो॒धय॑ति ता॒दृगे॒व तदथो॒ खल्वा॑हुश्चेत॒व्येति॒ यथा॒ वसी॑याꣳसम्भाग॒धेये॑न बो॒धय॑ति ता॒दृगे॒व तन्मनु॑र॒ग्निम॑चिनुत॒ तेन॒ नार्ध्नो॒थ्स ए॒ताम्पु॑नश्चि॒तिम॑पश्य॒त्ताम॑चिनुत॒ तया॒ वै स आ᳚र्ध्नो॒द्यत्पु॑नश्चि॒तिं चि॑नु॒त ऋद्ध्यै᳚॥~(५१)

%5.4.11.0
{\anuvakamend[{त्रि॒वृदथ॒ तिष्ठ॑न्त्यग्न्या॒धेये॑न॒ नाचि॑नुत स॒प्तद॑श च}]}%॥10॥

%5.4.11.1
छ॒न्द॒श्चितं॑ चिन्वीत प॒शुका॑मः प॒शवो॒ वै छन्दाꣳ॑सि पशु॒माने॒व भ॑वति श्येन॒चितं॑ चिन्वीत सुव॒र्गका॑मः श्ये॒नो वै वय॑सा॒म्पति॑ष्ठः श्ये॒न ए॒व भू॒त्वा सु॑व॒र्गं लो॒कम्प॑तति कङ्क॒चितं॑ चिन्वीत॒ यः का॒मये॑त शीर्\mbox{}ष॒ण्वान॒मुष्मि॑ल्लोँ॒के स्या॒मिति॑ शीर्\mbox{}ष॒ण्वाने॒वामुष्मि॑ल्लोँ॒के भ॑वत्यलज॒चितं॑ चिन्वीत॒ चतुः॑सीतं प्रति॒ष्ठाका॑म॒श्चत॑स्रो॒ दिशो॑ दि॒क्ष्वे॑व प्रति॑ तिष्ठति प्रउग॒चितं॑ चिन्वीत॒ भ्रातृ॑व्यवा॒न्प्र~(५२)

%5.4.11.2
ए॒व भ्रातृ॑व्यान्नुदत उभ॒यतः॑प्रउगं चिन्वीत॒ यः का॒मये॑त॒ प्र जा॒तान्भ्रातृ॑व्यान्नु॒देय॒ प्रति॑जनि॒ष्यमा॑णा॒निति॒ प्रैव जा॒तान्भ्रातृ॑व्यान्नु॒दते॒ प्रति॑ जनि॒ष्यमा॑णान्रथचक्र॒चितं॑ चिन्वीत॒ भ्रातृ॑व्य॒वान् वज्रो॒ वै रथो॒ वज्र॑मे॒व भ्रातृ॑व्येभ्यः॒ प्र ह॑रति द्रोण॒चितं॑ चिन्वी॒तान्न॑कामो॒ द्रोणे॒ वा अन्न॑म्भ्रियते॒ सयो᳚न्ये॒वान्न॒मव॑ रुन्द्धे समू॒ह्यं॑ चिन्वीत प॒शुका॑मः पशु॒माने॒व भ॑वति~(५३)

%5.4.11.3
प॒रि॒चा॒य्यं॑ चिन्वीत॒ ग्राम॑कामो ग्रा॒म्ये॑व भ॑वति श्मशान॒चितं॑ चिन्वीत॒ यः का॒मये॑त पितृलो॒क ऋ॑ध्नुया॒मिति॑ पितृलो॒क ए॒वर्ध्नो॑ति विश्वामित्रजमद॒ग्नी वसि॑ष्ठेनास्पर्धेता॒ꣳ॒ स ए॒ता ज॒मद॑ग्निर्विह॒व्या॑ अपश्य॒त्ता उपा॑धत्त॒ ताभि॒र्वै स वसि॑ष्ठस्येन्द्रि॒यं वी॒र्य॑मवृङ्क्त॒ यद्वि॑ह॒व्या॑ उप॒दधा॑तीन्द्रि॒यमे॒व ताभि॑र्वी॒र्यं॑ यज॑मानो॒ भ्रातृ॑व्यस्य वृङ्क्ते॒ होतु॒र्धिष्णि॑य॒ उप॑ दधाति यजमानायत॒नं वै~(५४)

%5.4.11.4
होता॒ स्व ए॒वास्मा॑ आ॒यत॑न इन्द्रि॒यं वी॒र्य॑मव॑ रुन्द्धे॒ द्वाद॒शोप॑ दधाति॒ द्वाद॑शाक्षरा॒ जग॑ती॒ जाग॑ताः प॒शवो॒ जग॑त्यै॒वास्मै॑ प॒शूनव॑ रुन्द्धे॒\-ऽष्टाव॑ष्टाव॒न्येषु॒ धिष्णि॑ये॒षूप॑ दधात्य॒ष्टाश॑फाः प॒शवः॑ प॒शूने॒वाव॑ रुन्द्धे॒ षण्मा᳚र्जा॒लीये॒ षड्वा ऋ॒तव॑ ऋ॒तवः॒ खलु॒ वै दे॒वाः पि॒तर॑ ऋ॒तूने॒व दे॒वान्पि॒तॄन्प्री॑णाति॥~(५)

%5.4.12.0
{\anuvakamend[{प्र भ॑वति यजमानायत॒नं वा अ॒ष्टाच॑त्वारिꣳशच्च}]}%॥11॥

%5.4.12.1
पव॑स्व॒ वाज॑सातय॒ इत्य॑नु॒ष्टुक्प्र॑ति॒पद्भ॑वति ति॒स्रो॑\-ऽनु॒ष्टुभ॒श्चत॑स्रो गाय॒त्रियो॒ यत्ति॒स्रो॑\-ऽनु॒ष्टुभ॒स्तस्मा॒दः श्व॑स्त्रि॒भिस्तिष्ठꣴ॑ स्तिष्ठति॒ यच्चत॑स्रो गाय॒त्रिय॒स्तस्मा॒थ्सर्वाꣳ॑श्च॒तुरः॑ प॒दः प्र॑ति॒दध॒त्पला॑यते पर॒मा वा ए॒षा छन्द॑सां॒ यद॑नु॒ष्टुक्प॑र॒मश्च॑तुष्टो॒मः स्तोमा॑नां पर॒मस्त्रि॑रा॒त्रो य॒ज्ञानां᳚ पर॒मो\-ऽश्वः॑ पशू॒नां पर॑मेणै॒वैनं॑ पर॒मतां᳚ गमयत्येकवि॒ꣳ॒शमह॑र्भवति~(५६)

%5.4.12.2
यस्मि॒न्नश्व॑ आल॒भ्यते॒ द्वाद॑श॒ मासाः॒ पञ्च॒र्तव॒स्त्रय॑ इ॒मे लो॒का अ॒सावा॑दि॒त्य ए॑कवि॒ꣳ॒श ए॒ष प्र॒जाप॑तिः प्राजाप॒त्यो\-ऽश्व॒स्तमे॒व सा॒क्षादृ॑ध्नोति॒ शक्व॑रयः पृ॒ष्ठम्भ॑वन्त्य॒न्यद॑न्य॒च्छन्दो॒\-ऽन्ये᳚न्ये॒ वा ए॒ते प॒शव॒ आ ल॑भ्यन्त उ॒तेव॑ ग्रा॒म्या उ॒तेवा॑र॒ण्या यच्छक्व॑रयः पृ॒ष्ठम्भव॒न्त्यश्व॑स्य सर्व॒त्वाय॑ पार्थुर॒श्मम्ब्र॑ह्मसा॒मम्भ॑वति र॒श्मिना॒ वा अश्वः॑~(५७)

%5.4.12.3
य॒त ई᳚श्व॒रो वा अश्वो\-ऽय॒तो\-ऽप्र॑तिष्ठितः॒ परां᳚ परा॒वतं॒ गन्तो॒र्त्पा᳚र्थुर॒श्मम्ब्र॑ह्मसा॒मम्भव॒त्यश्व॑स्य॒ यत्यै॒ धृत्यै॒ सङ्कृ॑त्यच्छावाकसा॒मम्भ॑वत्युथ्सन्नय॒ज्ञो वा ए॒ष यद॑श्वमे॒धः कस्तद्वे॒देत्या॑हु॒र्यदि॒ सर्वो॑ वा क्रि॒यते॒ न वा॒ सर्व॒ इति॒ यथ्सङ्कृ॑त्यच्छावाकसा॒मम्भव॒त्यश्व॑स्य सर्व॒त्वाय॒ पर्या᳚प्त्या॒ अन॑न्तरायाय॒ सर्व॑स्तोमो\-ऽतिरा॒त्र उ॑त्त॒ममह॑र्भवति॒ सर्व॒स्याप्त्यै॒ सर्व॑स्य॒ जित्यै॒ सर्व॑मे॒व तेना᳚ऽऽप्नोति॒ सर्वं॑ जयति॥~(५८)

%5.5.0.0
{\anuvakamend[{अह॑र्भवति॒ वा अश्वो\-ऽह॑र्भवति॒ दश॑ च}]}%॥12॥

%5.5.0.0

{\anuvakamend[{यदेके॑न प्र॒जाप॑तिः प्रे॒णानु॒ यजु॒षापो॑ वि॒श्वक॒र्माग्न॒ आ या॑हि सुव॒र्गाय॒ वज्रो॑ गाय॒त्रेणाग्न॑ उदधे स॒मीचीन्द्रा॑य म॒युर॒पां बला॑य पुरुषमृ॒गः सौ॒री पृ॑ष॒तः शका॒ रुरु॑रल॒जः सु॑प॒र्ण आ᳚ग्ने॒यो\-ऽश्वो॒\-ऽग्नये\-ऽनी॑कवते॒ चतु॑र्विꣳशतिः}]}%॥24॥ 
\prashnaend{यदेके॑न॒ स पापी॑याने॒तद्वा अ॒ग्नेर्धनु॒स्तद्दे॒वास्त्वेन्द्र॑ज्येष्ठा अ॒पां नप्रे\-ऽश्व॑स्तूप॒रो द्विष॑ष्टिः॥62॥ यदेके॒नैक॑शितिपा॒त्पेत्वः॑॥}
%%% END PRASHNA
