\chapt{काण्डम् ७}
\sect{प्रथमः प्रश्नः}\setcounter{anuvakam}{0}
\dnsub{तैत्तिरीयसंहितायां सप्तमकाण्डे प्रथमः प्रश्नः}
%7.1.1.0
%7.1.1.1
प्र॒जन॑नं॒ ज्योति॑र॒ग्निर्दे॒वता॑नां॒ ज्योति॑र्वि॒राट्छन्द॑सां॒ ज्योति॑र्वि॒राड्वा॒चो᳚\-ऽग्नौ सं ति॑ष्ठते वि॒राज॑म॒भि सम्प॑द्यते॒ तस्मा॒\-त्तज्ज्योति॑रुच्यते॒ द्वौ स्तोमौ᳚ प्रातःसव॒नं व॑हतो॒ यथा᳚ प्रा॒णश्चा॑पा॒नश्च॒ द्वौ माध्यं॑दिन॒ꣳ॒ सव॑नं॒ यथा॒ चक्षु॑श्च॒ श्रोत्रं॑ च॒ द्वौ तृ॑तीयसव॒नं यथा॒ वाक्च॑ प्रति॒ष्ठा च॒ पुरु॑षसम्मितो॒ वा ए॒ष य॒ज्ञो\-ऽस्थू॑रिः~(१)

%7.1.1.2
यं कामं॑ का॒मय॑ते॒ तमे॒तेना॒भ्य॑श्ञुते॒ सर्व॒ꣴ॒ ह्यस्थू॑रिणाभ्यश्ञु॒ते᳚\-ऽग्निष्टो॒मेन॒ वै प्र॒जा\-प॑तिः प्र॒जा अ॑सृजत॒ ता अ॑ग्निष्टो॒मेनै॒व पर्य॑गृह्णा॒त्तासां॒ परि॑गृहीतानामश्वत॒रो\-ऽत्य॑प्रवत॒ तस्या॑नु॒हाय॒ रेत॒ आद॑त्त॒ तद्ग॑र्द॒भे न्य॑मा॒र्ट्तस्मा᳚द्गर्द॒भो द्वि॒रेता॒ अथो॑ आहु॒र्वड॑बायां॒ न्य॑मा॒र्डिति॒ तस्मा॒द्वड॑बा द्वि॒रेता॒ अथो॑ आहु॒रोष॑धीषु~(२)

%7.1.1.3
न्य॑मा॒र्डिति॒ तस्मा॒दोष॑ध॒यो\-ऽन॑भ्यक्ता रेभ॒न्त्यथो॑ आहुः प्र॒जासु॒ न्य॑मा॒र्डिति॒ तस्मा᳚द्य॒मौ जा॑येते॒ तस्मा॑दश्वत॒रो न प्र जा॑यत॒ आत्त॑रेता॒ हि तस्मा᳚द्ब॒र्॒\mbox{}हिष्यन॑वकॢप्तः सर्ववेद॒से वा॑ स॒हस्रे॒ वाव॑कॢ॒प्तो\-ऽति॒ ह्यप्र॑वत॒ य ए॒वं वि॒द्वान॑ग्निष्टो॒मेन॒ यज॑ते॒ प्राजा॑ताः प्र॒जा ज॒नय॑ति॒ परि॒ प्रजा॑ता गृह्णाति॒ तस्मा॑दाहुर्ज्येष्ठय॒ज्ञ इति॑~(३)

%7.1.1.4
प्र॒जा\-प॑ति॒र्वाव ज्येष्ठः॒ स ह्ये॑तेनाग्रे\-ऽय॑जत प्र॒जा\-प॑तिरकामयत॒ प्र जा॑ये॒येति॒ स मु॑ख॒तस्त्रि॒वृतं॒ निर॑मिमीत॒ तम॒ग्नि\-र्दे॒वतान्व॑सृज्यत गाय॒त्री छन्दो॑ रथन्त॒रꣳ साम॑ ब्राह्म॒णो म॑नु॒ष्या॑णाम॒जः प॑शू॒नान्तस्मा॒त्ते मुख्या॑ मुख॒तो ह्यसृ॑ज्य॒न्तोर॑सो बा॒हु\-भ्यां᳚ पञ्चद॒शं निर॑मिमीत॒ तमिन्द्रो॑ दे॒वतान्व॑सृज्यत त्रि॒ष्टुप्छन्दो॑ बृ॒हत्~(४)

%7.1.1.5
साम॑ राज॒न्यो॑ मनु॒ष्या॑णा॒मविः॑ पशू॒नान्तस्मा॒त्ते वी॒र्या॑वन्तो वी॒र्या᳚द्ध्यसृ॑ज्यन्त मध्य॒तः स॑प्तद॒शं निर॑मिमीत॒ तं विश्वे॑ दे॒वा दे॒वता॒ अन्व॑सृज्यन्त॒ जग॑ती॒ छन्दो॑ वैरू॒पꣳ साम॒ वैश्यो॑ मनु॒ष्या॑णां॒ गावः॑ पशू॒नान्तस्मा॒त्त आ॒द्या॑ अन्न॒धाना॒\-द्ध्यसृ॑ज्यन्त॒ तस्मा॒द्भूयाꣳ॑सो॒\-ऽन्येभ्यो॒ भूयि॑ष्ठा॒ हि दे॒वता॒ अन्वसृ॑ज्यन्त प॒त्त ए॑कवि॒ꣳ॒शं निर॑मिमीत॒ तम॑नु॒ष्टुप्छन्दः॑ [5[]

%7.1.1.6
अन्व॑सृज्यत वैरा॒जꣳ साम॑ शू॒द्रो म॑नु॒ष्या॑णा॒मश्वः॑ पशू॒नान्तस्मा॒त्तौ भू॑तसङ्क्रा॒मिणा॒वश्व॑श्च शू॒द्रश्च॒ तस्मा᳚च्छू॒द्रो य॒ज्ञे\-ऽन॑वकॢप्तो॒ न हि दे॒वता॒ अन्वसृ॑ज्यत॒ तस्मा॒त्पादा॒वुप॑ जीवतः प॒त्तो ह्यसृ॑ज्येतां प्रा॒णा वै त्रि॒वृद॑र्धमा॒साः प॑ञ्चद॒शः प्र॒जा\-प॑तिः सप्तद॒शस्त्रय॑ इ॒मे लो॒का अ॒सावा॑दि॒त्य ए॑कवि॒ꣳ॒श ए॒तस्मि॒न्वा ए॒ते श्रि॒ता ए॒तस्मि॒न्प्रति॑ष्ठिता॒ य ए॒वं वेदै॒तस्मि॑न्ने॒व श्र॑यत ए॒तस्मि॒न्प्रति॑ तिष्ठति॥~(६)

%7.1.2.0
{\anuvakamend[{अस्थू॑रि॒रोष॑धीषु ज्येष्ठय॒ज्ञ इति॑ बृ॒हद॑नु॒ष्टुप्छन्दः॒ प्रति॑ष्ठिता॒ नव॑ च}]}%~(१)

%7.1.2.1
प्रा॒तः॒स॒व॒ने वै गा॑य॒त्रेण॒ छन्द॑सा त्रि॒वृते॒ स्तोमा॑य॒ ज्योति॒र्दध॑देति त्रि॒वृता᳚ ब्रह्मवर्च॒सेन॑ पञ्चद॒शाय॒ ज्योति॒र्दध॑देति पञ्चद॒शेनौज॑सा वी॒र्ये॑ण सप्तद॒शाय॒ ज्योति॒र्दध॑देति सप्तद॒शेन॑ प्राजाप॒त्येन॑ प्र॒जन॑नेनैकवि॒ꣳ॒शाय॒ ज्योति॒र्दध॑देति॒ स्तोम॑ ए॒व तथ्स्तोमा॑य॒ ज्योति॒र्दध॑दे॒त्यथो॒ स्तोम॑ ए॒व स्तोम॑म॒भि प्र ण॑यति॒ याव॑न्तो॒ वै स्तोमा॒स्ताव॑न्तः॒ कामा॒स्ताव॑न्तो लो॒कास्ताव॑न्ति॒ ज्योतीꣴ॑ष्ये॒ताव॑त ए॒व स्तोमा॑ने॒ताव॑तः॒ कामा॑ने॒ताव॑तो लो॒काने॒ताव॑न्ति॒ ज्योती॒ꣳ॒ष्यव॑ रुन्धे॥~(७)

%7.1.3.0
{\anuvakamend[{ताव॑न्तो लो॒कास्त्रयो॑दश च}]}%~(२)

%7.1.3.1
ब्र॒ह्म॒वा॒दिनो॑ वदन्ति॒ स त्वै य॑जेत॒ यो᳚\-ऽग्निष्टो॒मेन॒ यज॑मा॒नो\-ऽथ॒ सर्व॑स्तोमेन॒ यजे॒तेति॒ यस्य॑ त्रि॒वृत॑मन्त॒र्यन्ति॑ प्रा॒णाꣴस्तस्या॒न्तर्य॑न्ति प्रा॒णेषु॒ मे\-ऽप्य॑स॒दिति॒ खलु॒ वै य॒ज्ञेन॒ यज॑मानो यजते॒ यस्य॑ पञ्चद॒शम॑न्त॒र्यन्ति॑ वी॒र्यं॑ तस्या॒न्तर्य॑न्ति वी॒र्ये॑ मे\-ऽप्य॑स॒दिति॒ खलु॒ वै य॒ज्ञेन॒ यज॑मानो यजते॒ यस्य॑ सप्तद॒शम॑न्त॒र्यन्ति॑~(८)

%7.1.3.2
प्र॒जां तस्या॒न्तर्य॑न्ति प्र॒जाया॒म्मे\-ऽप्य॑स॒दिति॒ खलु॒ वै य॒ज्ञेन॒ यज॑मानो यजते॒ यस्यै॑कवि॒ꣳ॒शम॑न्त॒र्यन्ति॑ प्रति॒ष्ठां तस्या॒न्तर्य॑न्ति प्रति॒ष्ठाया॒म्मे\-ऽप्य॑स॒दिति॒ खलु॒ वै य॒ज्ञेन॒ यज॑मानो यजते॒ यस्य॑ त्रिण॒वम॑न्त॒र्यन्त्यृ॒तूꣴश्च॒ तस्य॑ नक्ष॒त्रियां᳚ च वि॒राज॑म॒न्तर्य॑न्त्यृ॒तुषु॒ मे\-ऽप्य॑सन्नक्ष॒त्रिया॑यां च वि॒राजीति॑~(९)

%7.1.3.3
खलु॒ वै य॒ज्ञेन॒ यज॑मानो यजते॒ यस्य॑ त्रयस्त्रि॒ꣳ॒शम॑न्त॒र्यन्ति॑ दे॒वता॒स्तस्या॒न्तर्य॑न्ति दे॒वता॑सु॒ मे\-ऽप्य॑स॒दिति॒ खलु॒ वै य॒ज्ञेन॒ यज॑मानो यजते॒ यो वै स्तोमा॑नामव॒मं प॑र॒मतां॒ गच्छ॑न्तं॒ वेद॑ पर॒मता॑मे॒व ग॑च्छति त्रि॒वृद्वै स्तोमा॑नामव॒म\-स्त्रि॒वृत्प॑र॒मो य ए॒वं वेद॑ पर॒मता॑मे॒व ग॑च्छति॥~(१०)

%7.1.4.0
{\anuvakamend[{स॒प्त॒द॒शम॑न्त॒र्यन्ति॑ वि॒राजीति॒ चतु॑श्चत्वारिꣳशच्च}]}%~(३)

%7.1.4.1
अङ्गि॑रसो॒ वै स॒त्रमा॑सत॒ ते सु॑व॒र्गं लो॒कमा॑य॒न्तेषाꣳ॑ ह॒विष्माꣴ॑श्च हवि॒ष्कृच्चा॑हीयेता॒न्ताव॑कामयेताꣳ सुव॒र्गं लो॒कमि॑या॒वेति॒ तावे॒तं द्वि॑रा॒त्रम॑पश्यतां॒ तमाह॑रतां॒ तेना॑यजेतां॒ ततो॒ वै तौ सु॑व॒र्गं लो॒कमै॑तां॒ य ए॒वं वि॒द्वान्द्वि॑रा॒त्रेण॒ यज॑ते सुव॒र्गमे॒व लो॒कमे॑ति॒ तावैतां॒ पूर्वे॒णा\-ऽह्ना\-ऽग॑च्छता॒मुत्त॑रेण~(११)

%7.1.4.2
अ॒भि॒प्ल॒वः पूर्व॒मह॑र्भवति॒ गति॒रुत्त॑रं॒ ज्योति॑ष्टोमो\-ऽग्निष्टो॒मः पूर्व॒मह॑र्भवति॒ तेज॒स्तेनाव॑ रुन्धे॒ सर्व॑स्तोमो\-ऽतिरा॒त्र उत्त॑र॒ꣳ॒ सर्व॒स्याप्त्यै॒ सर्व॒स्याव॑रुद्ध्यै गाय॒त्रम्पूर्वेह॒न्थ्साम॑ भवति॒ तेजो॒ वै गा॑य॒त्री गा॑य॒त्री ब्र॑ह्मवर्च॒सं तेज॑ ए॒व ब्र॑ह्मवर्च॒स\-मा॒त्मन्ध॑त्ते॒ त्रैष्टु॑भ॒मुत्त॑र॒ ओजो॒ वै वी॒र्यं॑ त्रि॒ष्टुगोज॑ ए॒व वी॒र्य॑मा॒त्मन्ध॑त्ते रथन्त॒रम्पूर्वे᳚~(१२)

%7.1.4.3
अह॒न्थ्साम॑ भवती॒यं वै र॑थन्त॒रम॒स्यामे॒व प्रति॑ तिष्ठति बृ॒हदुत्त॑रे॒\-ऽसौ वै बृ॒हद॒मुष्या॑मे॒व प्रति॑ तिष्ठति॒ तदा॑हुः॒ क्व॑ जग॑ती चानु॒ष्टुप्चेति॑ वैखान॒सम्पूर्वे\-ऽह॒न्थ्साम॑ भवति॒ तेन॒ जग॑त्यै॒ नैति॑ षोड॒श्युत्त॑रे॒ तेना॑नु॒ष्टुभो\-ऽथा॑हु॒र्यथ्स॑मा॒ने᳚\-ऽ र्धमा॒से स्याता॑मन्यत॒रस्याह्नो॑ वी॒र्य॑मनु॑ पद्ये॒तेत्य॑मावा॒स्या॑या॒म्पूर्व॒मह॑र्भव॒त्युत्त॑रस्मि॒न्नुत्त॑र॒न्नानै॒वार्ध॑मा॒सयो᳚र्भवतो॒ नाना॑वीर्ये भवतो ह॒विष्म॑न्निधन॒म्पूर्व॒मह॑र्भवति हवि॒ष्कृन्नि॑धन॒मुत्त॑रं॒ प्रति॑ष्ठित्यै॥~(१३)

%7.1.5.0
{\anuvakamend[{उत्त॑रेण रथन्त॒रम्पूर्वे\-ऽन्वेक॑विꣳशतिश्च}]}%~(४)

%7.1.5.1
आपो॒ वा इ॒दमग्रे॑ सलि॒लमा॑सी॒त्तस्मि॑न्प्र॒जा\-प॑तिर्वा॒युर्भू॒त्वाच॑र॒थ्स इ॒माम॑पश्य॒त्तां व॑रा॒हो भू॒त्वाह॑र॒त्तां वि॒श्वक॑र्मा भू॒त्वा व्य॑मा॒र्ट्थ्साप्र॑थत॒ सा पृ॑थि॒व्य॑भव॒त्तत्पृ॑थि॒व्यै पृ॑थिवि॒त्वन्तस्या॑मश्राम्यत्प्र॒जा\-प॑तिः॒ स दे॒वान॑सृजत॒ वसू᳚न्रु॒द्राना॑दि॒त्यान्ते दे॒वाः प्र॒जा\-प॑तिमब्रुव॒न्प्र जा॑यामहा॒ इति॒ सो᳚\-ऽब्रवीत्~(१४)

%7.1.5.2
यथा॒हं यु॒ष्माꣴस्तप॒सासृ॑क्ष्ये॒वं तप॑सि प्र॒जन॑नमिच्छध्व॒मिति॒ तेभ्यो॒\-ऽग्निमा॒यत॑नं॒ प्राय॑च्छदे॒तेना॒यत॑नेन श्राम्य॒तेति॒ ते᳚\-ऽग्निना॒यत॑नेनाश्राम्य॒न्ते सं॑वथ्स॒र एकां॒ गाम॑सृजन्त॒ तां वसु॑भ्यो रु॒द्रेभ्य॑ आदि॒त्येभ्यः॒ प्राय॑च्छन्ने॒ताꣳ र॑क्षध्व॒मिति॒ तां वस॑वो रु॒द्रा आ॑दि॒त्या अ॑रक्षन्त॒ सा वसु॑भ्यो रु॒द्रेभ्य॑ आदि॒त्येभ्यः॒ प्राजा॑यत॒ त्रीणि॑ च~(१५)

%7.1.5.3
श॒तानि॒ त्रय॑स्त्रिꣳशतं॒ चाथ॒ सैव स॑हस्रत॒म्य॑भव॒त्ते दे॒वाः प्र॒जा\-प॑तिमब्रुवन्थ्स॒हस्रे॑ण नो याज॒येति॒ सो᳚\-ऽग्निष्टो॒मेन॒ वसू॑नयाजय॒त्त इ॒मं लो॒कम॑जय॒न्तच्चा॑ददुः॒ स उ॒क्थ्ये॑न रु॒द्रान॑याजय॒त्ते᳚\-ऽन्तरि॑क्षमजय॒न्तच्चा॑ददुः॒ सो॑\-ऽतिरा॒त्रेणा॑\-दि॒त्यान॑याजय॒त्ते॑\-ऽमुं लो॒कम॑जय॒न्तच्चा॑ददु॒स्तद॒न्तरि॑क्षम्~(१६)

%7.1.5.4
व्यवै᳚र्यत॒ तस्मा᳚द्रु॒द्रा घातु॑का अनायत॒ना हि तस्मा॑दाहुः शिथि॒लं वै म॑ध्य॒ममह॑स्त्रिरा॒त्रस्य॒ वि हि तद॒वैर्य॒तेति॒ त्रैष्टु॑भम्मध्य॒मस्याह्न॒ आज्य॑म्भवति सं॒याना॑नि सू॒क्तानि॑ शꣳसति षोड॒शिनꣳ॑ शꣳस॒त्यह्नो॒ धृत्या॒ अशि॑थिलम्भावाय॒ तस्मा᳚त्त्रिरा॒त्रस्या᳚ग्निष्टो॒म ए॒व प्र॑थ॒ममहः॑ स्या॒दथो॒क्थ्यो\-ऽथा॑तिरा॒त्र ए॒षां लो॒कानां॒ विधृ॑त्यै॒ त्रीणि॑त्रीणि श॒तान्य॑नूचीना॒हमव्य॑वच्छिन्नानि ददाति~(१७)

%7.1.5.5
ए॒षां लो॒काना॒मनु॒ सन्त॑त्यै द॒शतं॒ न विच्छि॑न्द्याद्वि॒राजं॒ नेद्वि॑च्छि॒नदा॒नीत्यथ॒ या स॑हस्रत॒म्यासी॒त्तस्या॒मिन्द्र॑श्च॒ विष्णु॑श्च॒ व्याय॑च्छेता॒ꣳ॒ स इन्द्रो॑\-ऽमन्यता॒नया॒ वा इ॒दं विष्णुः॑ स॒हस्रं॑ वर्क्ष्यत॒ इति॒ तस्या॑मकल्पेतां॒ द्विभा॑ग॒ इन्द्र॒स्तृती॑ये॒ विष्णु॒स्तद्वा ए॒षाभ्यनू᳚च्यत उ॒भा जि॑ग्यथु॒रिति॒ तां वा ए॒ताम॑च्छावा॒कः~(१८)

%7.1.5.6
ए॒व शꣳ॑स॒त्यथ॒ या स॑हस्रत॒मी सा होत्रे॒ देयेति॒ होता॑रं॒ वा अ॒भ्यति॑रिच्यते॒ यद॑ति॒रिच्य॑ते॒ होताना᳚प्तस्यापयि॒ता\-था॑हुरुन्ने॒त्रे देयेत्यति॑रिक्ता॒ वा ए॒षा स॒हस्र॒स्याति॑रिक्त उन्ने॒तर्त्विजा॒मथा॑हुः॒ सर्वे᳚भ्यः सद॒स्ये᳚भ्यो॒ देयेत्यथा॑हुरुदा॒कृत्या॒ सा वशं॑ चरे॒दित्यथा॑हुर्ब्र॒ह्मणे॑ चा॒ग्नीधे॑ च॒ देयेति॑~(१९)

%7.1.5.7
द्विभा॑गम्ब्र॒ह्मणे॒ तृती॑यम॒ग्नीध॑ ऐ॒न्द्रो वै ब्र॒ह्मा वै᳚ष्ण॒वो᳚\-ऽग्नीद्यथै॒व तावक॑ल्पेता॒मित्यथा॑हु॒र्या क॑ल्या॒णी ब॑हुरू॒पा सा देयेत्यथा॑हु॒र्या द्वि॑रू॒पोभ॒यत॑एनी॒ सा देयेति॑ स॒हस्र॑स्य॒ परि॑गृहीत्यै॒ तद्वा ए॒तथ्स॒हस्र॒स्याय॑नꣳ स॒हस्रꣴ॑ स्तो॒त्रीयाः᳚ स॒हस्रं॒ दक्षि॑णाः स॒हस्र॑सम्मितः सुव॒र्गो लो॒कः सु॑व॒र्गस्य॑ लो॒कस्या॒भिजि॑त्यै॥~(२०)

%7.1.6.0
{\anuvakamend[{अ॒ब्र॒वी॒च्च॒ तद॒न्तरि॑क्षन्ददात्यच्छावा॒कश्च॒ देयेति॑ स॒प्तच॑त्वारिꣳशच्च}]}%~(५)

%7.1.6.1
सोमो॒ वै स॒हस्र॑मविन्द॒त्तमिन्द्रो\-ऽन्व॑विन्द॒त्तौ य॒मो न्याग॑च्छ॒त्ताव॑ब्रवी॒दस्तु॒ मे\-ऽत्रापीत्यस्तु॒ ही(३) इत्य॑ब्रूता॒ꣳ॒ स य॒म एक॑स्यां वी॒र्यं॑ पर्य॑पश्यदि॒यं वा अ॒स्य स॒हस्र॑स्य वी॒र्य॑म्बिभ॒र्तीति॒ ताव॑ब्रवीदि॒यम्ममास्त्वे॒तद्यु॒वयो॒रिति॒ ताव॑ब्रूता॒ꣳ॒ सर्वे॒ वा ए॒तदे॒तस्यां᳚ वी॒र्यम्᳚~(२१)

%7.1.6.2
परि॑ पश्या॒मो\-ऽꣳश॒मा ह॑रामहा॒ इति॒ तस्या॒मꣳश॒माह॑रन्त॒ ताम॒फ्सु प्रावे॑शय॒न्थ्सोमा॑यो॒देहीति॒ सा रोहि॑णी पिङ्ग॒लैक॑हायनी रू॒पं कृ॒त्वा त्रय॑स्त्रिꣳशता च त्रि॒भिश्च॑ श॒तैः स॒होदैत्तस्मा॒द्रोहि॑ण्या पिङ्ग॒लयैक॑हायन्या॒ सोमं॑ क्रीणीया॒द्य ए॒वं वि॒द्वान्रोहि॑ण्या पिङ्ग॒लयैक॑हायन्या॒ सोमं॑ क्री॒णाति॒ त्रय॑स्त्रिꣳशता चै॒वास्य॑ त्रि॒भिश्च॑~(२२)

%7.1.6.3
श॒तैः सोमः॑ क्री॒तो भ॑वति॒ सुक्री॑तेन यजते॒ ताम॒फ्सु प्रावे॑शय॒न्निन्द्रा॑यो॒देहीति॒ सा रोहि॑णी लक्ष्म॒णा प॑ष्ठौ॒ही वार्त्र॑घ्नी रू॒पं कृ॒त्वा त्रय॑स्त्रिꣳशता च त्रि॒भिश्च॑ श॒तैः स॒होदैत्तस्मा॒द्रोहि॑णीं लक्ष्म॒णाम्प॑ष्ठौ॒हीं वार्त्र॑घ्नीं दद्या॒द्य ए॒वं वि॒द्वान्रोहि॑णीं लक्ष्म॒णाम्प॑ष्ठौ॒हीं वार्त्र॑घ्नीं॒ ददा॑ति॒ त्रय॑स्त्रिꣳशच्चै॒वास्य॒ त्रीणि॑ च श॒तानि॒ सा द॒त्ता~(२३)

%7.1.6.4
भ॒व॒ति॒ ताम॒फ्सु प्रावे॑शयन् य॒मायो॒देहीति॒ सा जर॑ती मू॒र्खा त॑ज्जघ॒न्या रू॒पं कृ॒त्वा त्रय॑स्त्रिꣳशता च त्रि॒भिश्च॑ श॒तैः स॒होदैत्तस्मा॒ज्जर॑तीम्मू॒र्खां त॑ज्जघ॒न्याम॑नु॒स्तर॑णीं कुर्वीत॒ य ए॒वं वि॒द्वाञ्जर॑तीम्मू॒र्खां त॑ज्जघ॒न्याम॑नु॒स्तर॑णीं कुरु॒ते त्रय॑स्त्रिꣳशच्चै॒वास्य॒ त्रीणि॑ च श॒तानि॒ सामुष्मिँ॑ल्लो॒के भ॑वति॒ वागे॒व स॑हस्रत॒मी तस्मा᳚त्~(२४)

%7.1.6.5
वरो॒ देयः॒ सा हि वरः॑ स॒हस्र॑मस्य॒ सा द॒त्ता भ॑वति॒ तस्मा॒द्वरो॒ न प्र॑ति॒गृह्यः॒ सा हि वरः॑ स॒हस्र॑मस्य॒ प्रति॑गृहीत\-म्भवती॒यं वर॒ इति॑ ब्रूया॒दथा॒न्याम्ब्रू॑यादि॒यम्ममेति॒ तथा᳚स्य॒ तथ्स॒हस्र॒मप्र॑तिगृहीतम्भवत्युभयतए॒नी स्या॒त्तदा॑हुरन्यत\-ए॒नी स्या᳚थ्स॒हस्र॑म्प॒रस्ता॒देत॒मिति॒ यैव वरः॑~(२५)

%7.1.6.6
क॒ल्या॒णी रू॒पस॑मृद्धा॒ सा स्या॒थ्सा हि वरः॒ समृ॑द्ध्यै॒ तामुत्त॑रे॒णाग्नी᳚ध्रं पर्या॒णीया॑हव॒नीय॒स्यान्ते᳚ द्रोणकल॒शमव॑ घ्रापये॒दा जि॑घ्र क॒लश॑म्मह्यु॒रुधा॑रा॒ पय॑स्व॒त्या त्वा॑ विश॒न्त्विन्द॑वः समु॒द्रमि॑व॒ सिन्ध॑वः॒ सा मा॑ स॒हस्र॒ आ भ॑ज प्र॒जया॑ प॒शुभिः॑ स॒ह पुन॒र्मा वि॑शताद्र॒यिरिति॑ प्र॒जयै॒वैनं॑ प॒शुभी॑ र॒य्या सम्~(२६)

%7.1.6.7
अ॒र्ध॒य॒ति॒ प्र॒जावा᳚न्पशु॒मान्र॑यि॒मान्भ॑वति॒ य ए॒वं वेद॒ तया॑ स॒हाग्नी᳚ध्रम्प॒रेत्य॑ पु॒रस्ता᳚त्प्र॒तीच्यां॒ तिष्ठ॑न्त्यां जुहुयादु॒भा जि॑ग्यथु॒र्न परा॑ जयेथे॒ न परा॑ जिग्ये कत॒रश्च॒नैनोः᳚। इन्द्र॑श्च विष्णो॒ यदप॑स्पृधेथां त्रे॒धा स॒हस्रं॒ वि तदै॑रयेथा॒मिति॑ त्रेधाविभ॒क्तं वै त्रि॑रा॒त्रे स॒हस्रꣳ॑ साह॒स्रीमे॒वैनां᳚ करोति स॒हस्र॑स्यै॒वैना॒म्मात्रा᳚म्~(२७)

%7.1.6.8
क॒रो॒ति॒ रू॒पाणि॑ जुहोति रू॒पैरे॒वैना॒ꣳ॒ सम॑र्धयति॒ तस्या॑ उपो॒त्थाय॒ कर्ण॒मा ज॑पे॒दिडे॒ रन्ते\-ऽदि॑ते॒ सर॑स्वति॒ प्रिये॒ प्रेय॑सि॒ महि॒ विश्रु॑त्ये॒तानि॑ ते अघ्निये॒ नामा॑नि सु॒कृतं॑ मा दे॒वेषु॑ ब्रूता॒दिति॑ दे॒वेभ्य॑ ए॒वैन॒मा वे॑दय॒त्यन्वे॑नं दे॒वा बु॑ध्यन्ते॥~(२८)

%7.1.7.0
{\anuvakamend[{ए॒तदे॒तस्यां᳚ वी॒र्य॑मस्य त्रि॒भिश्च॑ द॒त्ता स॑हस्रत॒मी तस्मा॑दे॒व वरः॒ सम्मात्रा॒मेका॒न्नच॑त्वारि॒ꣳ॒शच्च॑}]}%~(६)

%7.1.7.1
स॒ह॒स्र॒त॒म्या॑ वै यज॑मानः सुव॒र्गं लो॒कमे॑ति॒ सैनꣳ॑ सुव॒र्गं लो॒कं ग॑मयति॒ सा मा॑ सुव॒र्गं लो॒कं ग॑म॒येत्या॑ह सुव॒र्गमे॒वैनं॑ लो॒कं ग॑मयति॒ सा मा॒ ज्योति॑ष्मन्तं लो॒कं ग॑म॒येत्या॑ह॒ ज्योति॑ष्मन्तमे॒वैनं॑ लो॒कं ग॑मय॒ति सा मा॒ सर्वा॒न्पुण्याँ᳚ल्लो॒कान्ग॑म॒येत्या॑ह॒ सर्वा॑ने॒वैनं॒ पुण्याँ᳚ल्लो॒कान्ग॑मयति॒ सा~(२९)

%7.1.7.2
मा॒ प्र॒ति॒ष्ठां ग॑मय प्र॒जया॑ प॒शुभिः॑ स॒ह पुन॒र्मा वि॑शताद्र॒यिरिति॑ प्र॒जयै॒वैनं॑ प॒शुभी॑ र॒य्यां प्रति॑\-ष्ठापयति प्र॒जावा᳚न्पशु॒मान्र॑यि॒मान्भ॑वति॒ य ए॒वं वेद॒ ताम॒ग्नीधे॑ वा ब्र॒ह्मणे॑ वा॒ होत्रे॑ वोद्गा॒त्रे वा᳚ध्व॒र्यवे॑ वा दद्याथ्स॒हस्र॑मस्य॒ सा द॒त्ता भ॑वति स॒हस्र॑मस्य॒ प्रति॑गृहीतम्भवति॒ यस्तामवि॑द्वान्~(३०)

%7.1.7.3
प्र॒ति॒गृ॒ह्णाति॒ तां प्रति॑ गृह्णीया॒देका॑सि॒ न स॒हस्र॒मेकां᳚ त्वा भू॒तां प्रति॑ गृह्णामि॒ न स॒हस्र॒मेका॑ मा भू॒ता वि॑श॒ मा स॒हस्र॒मित्येका॑मे॒वैनां᳚ भू॒तां प्रति॑ गृह्णाति॒ न स॒हस्रं॒ य ए॒वं वेद॑ स्यो॒नासि॑ सु॒षदा॑ सु॒शेवा᳚ स्यो॒ना मा वि॑श सु॒षदा॒ मा वि॑श सु॒शेवा॒ मा वि॑श~(३१)

%7.1.7.4
इत्या॑ह स्यो॒नैवैनꣳ॑ सु॒षदा॑ सु॒शेवा॑ भू॒ता वि॑शति॒ नैनꣳ॑ हिनस्ति ब्रह्मवा॒दिनो॑ वदन्ति स॒हस्र॑ꣳ सहस्रत॒म्यन्वे॒ती(३) स॑हस्रत॒मीꣳ स॒हस्रा(३)मिति॒ यत्प्राची॑मुथ्सृ॒जेथ्स॒हस्रꣳ॑ सहस्रत॒म्यन्वि॑या॒त्तथ्स॒हस्र॑मप्रज्ञा॒त्रꣳ सु॑व॒र्गं लो॒कं न प्र जा॑नीयात्प्र॒तीची॒मुथ्सृ॑जति॒ ताꣳ स॒हस्र॒मनु॑ प॒र्याव॑र्तते॒ सा प्र॑जान॒ती सु॑व॒र्गं लो॒कमे॑ति॒ यज॑मानम॒भ्युथ्सृ॑जति क्षि॒प्रे स॒हस्रं॒ प्र जा॑यत उत्त॒मा नी॒यते᳚ प्रथ॒मा दे॒वान्ग॑च्छति॥~(३२)

%7.1.8.0
{\anuvakamend[{लो॒कान्ग॑मयति॒ सावि॑द्वान्थ्सु॒शेवा॒ मावि॑श॒ यज॑मानं॒ द्वाद॑श च}]}%~(७)

%7.1.8.1
अत्रि॑रददा॒दौर्वा॑य प्र॒जां पु॒त्रका॑माय॒ स रि॑रिचा॒नो॑\-ऽमन्यत॒ निर्वी᳚र्यः शिथि॒लो या॒तया॑मा॒ स ए॒तं च॑तूरा॒त्रम॑पश्य॒त् तमाह॑र॒त्तेना॑यजत॒ ततो॒ वै तस्य॑ च॒त्वारो॑ वी॒रा आजा॑यन्त॒ सुहो॑ता॒ सू᳚द्गाता॒ स्व॑ध्वर्युः॒ सुस॑भेयो॒ य ए॒वं वि॒द्वाꣴश्च॑तूरा॒त्रेण॒ यज॑त॒ आस्य॑ च॒त्वारो॑ वी॒रा जा॑यन्ते॒ सुहो॑ता॒ सू᳚द्गाता॒ स्व॑ध्वर्युः॒ सुस॑भेयो॒ ये च॑तुर्वि॒ꣳ॒शाः पव॑माना ब्रह्मवर्च॒सं तत्~(३३)

%7.1.8.2
य उ॒द्यन्तः॒ स्तोमाः॒ श्रीः सात्रिꣴ॑ श्र॒द्धादे॑वं॒ यज॑मानं च॒त्वारि॑ वीर्याणि॒ नोपा॑नम॒न्तेज॑ इन्द्रि॒यम्ब्र॑ह्मवर्च॒सम॒न्नाद्य॒ꣳ॒ स ए॒ताꣴश्च॒तुर॒श्चतु॑ष्टोमा॒न्थ्सोमा॑नपश्य॒त्तानाह॑र॒त्तैर॑यजत॒ तेज॑ ए॒व प्र॑थ॒मेनावा॑रुन्धेन्द्रि॒यं द्वि॒तीये॑न ब्रह्मवर्च॒सं तृ॒तीये॑ना॒न्नाद्यं॑ चतु॒र्थेन॒ य ए॒वं वि॒द्वाꣴश्च॒तुर॒श्चतु॑ष्टोमा॒न्थ्सोमा॑ना॒हर॑ति॒ तैर्यज॑ते॒ तेज॑ ए॒व प्र॑थ॒मेनाव॑ रुन्ध इन्द्रि॒यं द्वि॒तीये॑न ब्रह्मवर्च॒सं तृ॒तीये॑ना॒न्नाद्यं॑ चतु॒र्थेन॒ यामे॒वात्रि॒र्॒\mbox{}ऋद्धि॒मार्ध्नो॒त्तामे॒व यज॑मान ऋध्नोति॥~(३४)

%7.1.9.0
{\anuvakamend[{तत्तेज॑ ए॒वाष्टाद॑श च}]}%~(८)

%7.1.9.1
ज॒मद॑ग्निः॒ पुष्टि॑कामश्चतूरा॒त्रेणा॑यजत॒ स ए॒तान्पोषाꣳ॑ अपुष्य॒त्तस्मा᳚त्पलि॒तौ जाम॑दग्नियौ॒ न सं जा॑नाते ए॒ताने॒व पोषा᳚न्पुष्यति॒ य ए॒वं वि॒द्वाꣴश्च॑तूरा॒त्रेण॒ यज॑ते पुरोडा॒शिन्य॑ उप॒सदो॑ भवन्ति प॒शवो॒ वै पु॑रो॒डाशः॑ प॒शूने॒वाव॑ रु॒न्धे\-ऽन्नं॒ वै पु॑रो॒डाशो\-ऽन्न॑मे॒वाव॑ रुन्धे\-ऽन्ना॒दः प॑शु॒मान्भ॑वति॒ य ए॒वं वि॒द्वाꣴश्च॑तूरा॒त्रेण॒ यज॑ते॥~(३५)

%7.1.10.0
{\anuvakamend[{ज॒मद॑ग्निर॒ष्टाच॑त्वारिꣳशत्}]}%~(९)

%7.1.10.1
सं॒व॒थ्स॒रो वा इ॒दमेक॑ आसी॒थ्सो॑\-ऽकामयत॒र्तून्थ्सृ॑जे॒येति॒ स ए॒तम्प॑ञ्चरा॒त्रम॑पश्य॒त्तमाह॑र॒त्तेना॑यजत॒ ततो॒ वै स ऋ॒तून॑सृजत॒ य ए॒वं वि॒द्वान्प॑ञ्चरा॒त्रेण॒ यज॑ते॒ प्रैव जा॑यते॒ त ऋ॒तवः॑ सृ॒ष्टा न व्याव॑र्तन्त॒ त ए॒तम्प॑ञ्चरा॒त्रम॑पश्य॒न् तमाह॑र॒न्तेना॑यजन्त॒ ततो॒ वै ते व्याव॑र्तन्त~(३६)

%7.1.10.2
य ए॒वं वि॒द्वान्प॑ञ्चरा॒त्रेण॒ यज॑ते॒ वि पा॒प्मना॒ भ्रातृ॑व्ये॒णा व॑र्तते॒ सार्व॑सेनिः शौचे॒यो॑\-ऽकामयत पशु॒मान्थ्स्या॒मिति॒ स ए॒तम्प॑ञ्चरा॒त्रमाह॑र॒त्तेना॑यजत॒ ततो॒ वै स स॒हस्रं॑ प॒शून्प्राप्नो॒द्य ए॒वं वि॒द्वान्प॑ञ्चरा॒त्रेण॒ यज॑ते॒ प्र स॒हस्रं॑ प॒शूना᳚प्नोति बब॒रः प्रावा॑हणिरकामयत वा॒चः प्र॑वदि॒ता स्या॒मिति॒ स ए॒तम्प॑ञ्चरा॒त्रमा~(३७)

%7.1.10.3
अ॒ह॒र॒त्तेना॑यजत॒ ततो॒ वै स वा॒चः प्र॑वदि॒ताभ॑व॒द्य ए॒वं वि॒द्वान्प॑ञ्चरा॒त्रेण॒ यज॑ते प्रवदि॒तैव वा॒चो भ॑व॒त्यथो॑ एनं वा॒चस्पति॒रित्या॑हु॒रना᳚प्तश्चतूरा॒त्रो\-ऽति॑रिक्तः षड्रा॒त्रो\-ऽथ॒ वा ए॒ष सं॑ प्र॒ति य॒ज्ञो यत्प॑ञ्चरा॒त्रो य ए॒वं वि॒द्वान्प॑ञ्चरा॒त्रेण॒ यज॑ते सम्प्र॒त्ये॑व य॒ज्ञेन॑ यजते पञ्चरा॒त्रो भ॑वति॒ पञ्च॒ वा ऋ॒तवः॑ संवथ्स॒रः~(३८)

%7.1.10.4
ऋ॒तुष्वे॒व सं॑वथ्स॒रे प्रति॑ तिष्ठ॒त्यथो॒ पञ्चा᳚क्षरा प॒ङ्क्तिः पाङ्क्तो॑ य॒ज्ञो य॒ज्ञमे॒वाव॑ रुन्धे त्रि॒वृद॑ग्निष्टो॒मो भ॑वति॒ तेज॑ ए॒वाव॑ रुन्धे पञ्चद॒शो भ॑वतीन्द्रि॒यमे॒वाव॑ रुन्धे सप्तद॒शो भ॑वत्य॒न्नाद्य॒स्याव॑रुद्ध्या॒ अथो॒ प्रैव तेन॑ जायते पञ्चवि॒ꣳ॒शो᳚\-ऽग्निष्टो॒मो भ॑वति प्र॒जाप॑ते॒राप्त्यै॑ महाव्र॒तवा॑न॒न्नाद्य॒स्याव॑रुद्ध्यै विश्व॒जिथ्सर्व॑पृष्ठो\-ऽतिरा॒त्रो भ॑वति॒ सर्व॑स्या॒भिजि॑त्यै॥~(३९)

%7.1.11.0
{\anuvakamend[{ते व्याव॑र्तन्त प्रवदि॒ता स्या॒मिति॒ स ए॒तम्प़॑ञ्चरा॒त्रमा सं॑वथ्स॒रो॑\-ऽभिजि॑त्यै}]}%॥10॥

%7.1.11.1
दे॒वस्य॑ त्वा सवि॒तुः प्र॑स॒वे᳚\-ऽश्विनो᳚र्बा॒हु\-भ्यां᳚ पू॒ष्णो हस्ता᳚भ्या॒मा द॑द इ॒माम॑गृभ्णन्रश॒नामृ॒तस्य॒ पूर्व॒ आयु॑षि वि॒दथे॑षु क॒व्या। तया॑ दे॒वाः सु॒तमा ब॑भूवुर्\mbox{}ऋ॒तस्य॒ साम᳚न्थ्स॒रमा॒रप॑न्ती। अ॒भि॒धा अ॑सि॒ भुव॑नमसि य॒न्तासि॑ ध॒र्तासि॒ सो᳚\-ऽग्निं वै᳚श्वान॒रꣳ सप्र॑थसं गच्छ॒ स्वाहा॑कृतः पृथि॒व्यां य॒न्ता राड्य॒न्तासि॒ यम॑नो ध॒र्तासि॑ ध॒रुणः॑ कृ॒ष्यै त्वा॒ क्षेमा॑य त्वा र॒य्यै त्वा॒ पोषा॑य त्वा पृथि॒व्यै त्वा॒न्तरि॑क्षाय त्वा दि॒वे त्वा॑ स॒ते त्वास॑ते त्वा॒द्भ्यस्त्वौष॑धीभ्यस्त्वा॒ विश्वे᳚भ्यस्त्वा भू॒तेभ्यः॑॥~(४०)

%7.1.12.0
{\anuvakamend[{ध॒रुणः॒ प़ञ्च॑विꣳशतिश्च}]}%॥11॥

%7.1.12.1
वि॒भूर्मा॒त्रा प्र॒भूः पि॒त्राश्वो॑\-ऽसि॒ हयो॒\-ऽस्यत्यो॑\-ऽसि॒ नरो॒\-ऽस्यर्वा॑सि॒ सप्ति॑रसि वा॒ज्य॑सि॒ वृषा॑सि नृ॒मणा॑ असि॒ ययु॒र्नामा᳚स्यादि॒त्याना॒म्पत्वान्वि॑ह्य॒ग्नये॒ स्वाहा॒ स्वाहे᳚न्द्रा॒ग्निभ्या॒ꣴ॒ स्वाहा᳚ प्र॒जाप॑तये॒ स्वाहा॒ विश्वे᳚भ्यो दे॒वेभ्यः॒ स्वाहा॒ सर्वा᳚भ्यो दे॒वेता᳚भ्य इ॒ह धृतिः॒ स्वाहे॒ह विधृ॑तिः॒ स्वाहे॒ह रन्तिः॒ स्वाहे॒ह रम॑तिः॒ स्वाहा॒ भूर॑सि भु॒वे त्वा॒ भव्या॑य त्वा भविष्य॒ते त्वा॒ विश्वे᳚भ्यस्त्वा भू॒तेभ्यो॒ देवा॑ आशापाला ए॒तं दे॒वेभ्यो\-ऽश्व॒म्मेधा॑य॒ प्रोक्षि॑तं गोपायत॥~(४१)

%7.1.13.0
{\anuvakamend[{रन्तिः॒ स्वाहा॒ द्वाविꣳ॑शतिश्च}]}%॥12॥

%7.1.13.1
आय॑नाय॒ स्वाहा॒ प्राय॑णाय॒ स्वाहो᳚द्द्रा॒वाय॒ स्वाहोद्द्रु॑ताय॒ स्वाहा॑ शूका॒राय॒ स्वाहा॒ शूकृ॑ताय॒ स्वाहा॒ पला॑यिताय॒ स्वाहा॒\-ऽ\-ऽपला॑यिताय॒ स्वाहा॒\-ऽ\-ऽवल्ग॑ते॒ स्वाहा॑ परा॒वल्ग॑ते॒ स्वाहा॑\-ऽ\-ऽय॒ते स्वाहा᳚ प्रय॒ते स्वाहा॒ सर्व॑स्मै॒ स्वाहा᳚॥~(४२)

%7.1.14.0
{\anuvakamend[{आय॑ना॒योत्त॑रमा॒पला॑यिताय॒ षड्विꣳ॑शतिः}]}%॥13॥

%7.1.14.1
अ॒ग्नये॒ स्वाहा॒ सोमा॑य॒ स्वाहा॑ वा॒यवे॒ स्वाहा॒पाम्मोदा॑य॒ स्वाहा॑ सवि॒त्रे स्वाहा॒ सर॑स्वत्यै॒ स्वाहेन्द्रा॑य॒ स्वाहा॒ बृह॒स्पत॑ये॒ स्वाहा॑ मि॒त्राय॒ स्वाहा॒ वरु॑णाय॒ स्वाहा॒ सर्व॑स्मै॒ स्वाहा᳚॥~(४३)

%7.1.15.0
{\anuvakamend[{}]}

%7.1.15.1
पृ॒थि॒व्यै स्वाहा॒\-ऽन्तरि॑क्षाय॒ स्वाहा॑ दि॒वे स्वाहा॒ सूर्या॑य॒ स्वाहा॑ च॒न्द्रम॑से॒ स्वाहा॒ नक्ष॑त्रेभ्यः॒ स्वाहा॒ प्राच्यै॑ दि॒शे स्वाहा॒ दक्षि॑णायै दि॒शे स्वाहा᳚ प्र॒तीच्यै॑ दि॒शे स्वाहोदी᳚च्यै दि॒शे स्वाहो॒र्ध्वायै॑ दि॒शे स्वाहा॑ दि॒ग्भ्यः स्वाहा॑\-ऽ\-वान्तरदि॒शाभ्यः॒ स्वाहा॒ समा᳚भ्यः॒ स्वाहा॑ श॒रद्भ्यः॒ स्वाहा॑\-ऽहोरा॒त्रेभ्यः॒ स्वाहा᳚\-ऽर्धमा॒सेभ्यः॒ स्वाहा॒ मासे᳚भ्यः॒ स्वाह॒र्तुभ्यः॒ स्वाहा॑ संवथ्स॒राय॒ स्वाहा॒ सर्व॑स्मै॒ स्वाहा᳚॥~(४४)

%7.1.16.0
{\anuvakamend[{}]}

%7.1.16.1
अ॒ग्नये॒ स्वाहा॒ सोमा॑य॒ स्वाहा॑ सवि॒त्रे स्वाहा॒ सर॑स्वत्यै॒ स्वाहा॑ पू॒ष्णे स्वाहा॒ बृह॒स्पत॑ये॒ स्वाहा॒\-ऽपाम्मोदा॑य॒ स्वाहा॑ वा॒यवे॒ स्वाहा॑ मि॒त्राय॒ स्वाहा॒ वरु॑णाय॒ स्वाहा॒ सर्व॑स्मै॒ स्वाहा᳚॥~(४५)

%7.1.17.0
{\anuvakamend[{}]}

%7.1.17.1
पृ॒थि॒व्यै स्वाहा॒\-ऽन्तरि॑क्षाय॒ स्वाहा॑ दि॒वे स्वाहा॒\-ऽग्नये॒ स्वाहा॒ सोमा॑य॒ स्वाहा॒ सूर्या॑य॒ स्वाहा॑ च॒न्द्रम॑से॒ स्वाहा\-ऽह्ने॒ स्वाहा॒ रात्रि॑यै॒ स्वाह॒र्जवे॒ स्वाहा॑ सा॒धवे॒ स्वाहा॑ सुक्षि॒त्यै स्वाहा᳚ क्षु॒धे स्वाहा॑\-ऽ\-ऽशिति॒म्ने स्वाहा॒ रोगा॑य॒ स्वाहा॑ हि॒माय॒ स्वाहा॑ शी॒ताय॒ स्वाहा॑\-ऽ\-ऽत॒पाय॒ स्वाहा\-ऽर॑ण्याय॒ स्वाहा॑ सुव॒र्गाय॒ स्वाहा॑ लो॒काय॒ स्वाहा॒ सर्व॑स्मै॒ स्वाहा᳚॥~(४६)

%7.1.18.0
{\anuvakamend[{}]}

%7.1.18.1
भुवो॑ दे॒वानां॒ कर्म॑णा॒पस॒र्तस्य॑ प॒थ्या॑सि॒ वसु॑भिर्दे॒वेभि॑र्दे॒वत॑या गाय॒त्रेण॑ त्वा॒ छन्द॑सा युनज्मि वस॒न्तेन॑ त्व॒र्तुना॑ ह॒विषा॑ दीक्षयामि रु॒द्रेभि॑र्दे॒वेभि॑र्दे॒वत॑या॒ त्रैष्टु॑भेन त्वा॒ छन्द॑सा युनज्मि ग्री॒ष्मेण॑ त्व॒र्तुना॑ ह॒विषा॑ दीक्षयाम्यादि॒त्येभि॑\-र्दे॒वेभि॑र्दे॒वत॑या॒ जाग॑तेन त्वा॒ छन्द॑सा युनज्मि व॒र्॒\mbox{}षाभि॑स्त्व॒र्तुना॑ ह॒विषा॑ दीक्षयामि॒ विश्वे॑भिर्दे॒वेभि॑र्दे॒वत॒यानु॑ष्टुभेन त्वा॒ छन्द॑सा युनज्मि~(४७)

%7.1.18.2
श॒रदा᳚ त्व॒र्तुना॑ ह॒विषा॑ दीक्षया॒म्यङ्गि॑रोभिर्दे॒वेभि॑र्दे॒वत॑या॒ पाङ्क्ते॑न त्वा॒ छन्द॑सा युनज्मि हेमन्तशिशि॒रा\-भ्यां᳚ त्व॒र्तुना॑ ह॒विषा॑ दीक्षया॒म्याहं दी॒क्षाम॑रुहमृ॒तस्य॒ पत्नीं᳚ गाय॒त्रेण॒ छन्द॑सा॒ ब्रह्म॑णा च॒र्तꣳ स॒त्ये॑\-ऽधाꣳ स॒त्यमृ॒ते॑\-ऽधाम्। म॒हीमू॒ षु सु॒त्रामा॑णमि॒ह धृतिः॒ स्वाहे॒ह विधृ॑तिः॒ स्वाहे॒ह रन्तिः॒ स्वाहे॒ह रम॑तिः॒ स्वाहा᳚॥~(४८)

%7.1.19.0
{\anuvakamend[{}]}

%7.1.19.1
ई॒ङ्का॒राय॒ स्वाहें कृ॑ताय॒ स्वाहा॒ क्रन्द॑ते॒ स्वाहा॑\-ऽव॒क्रन्द॑ते॒ स्वाहा॒ प्रोथ॑ते॒ स्वाहा᳚ प्र॒प्रोथ॑ते॒ स्वाहा॑ ग॒न्धाय॒ स्वाहा᳚ घ्रा॒ताय॒ स्वाहा᳚ प्रा॒णाय॒ स्वाहा᳚ व्या॒नाय॒ स्वाहा॑\-ऽपा॒नाय॒ स्वाहा॑ सन्दी॒यमा॑नाय॒ स्वाहा॒ सन्दि॑ताय॒ स्वाहा॑ विचृ॒त्यमा॑नाय॒ स्वाहा॒ विचृ॑त्ताय॒ स्वाहा॑ पलायि॒ष्यमा॑णाय॒ स्वाहा॒ पला॑यिताय॒ स्वाहो॑परꣴस्य॒ते स्वाहोप॑रताय॒ स्वाहा॑ निवेक्ष्य॒ते स्वाहा॑ निवि॒शमा॑नाय॒ स्वाहा॒ निवि॑ष्टाय॒ स्वाहा॑ निषथ्स्य॒ते स्वाहा॑ नि॒षीद॑ते॒ स्वाहा॒ निष॑ण्णाय॒ स्वाहा॑~(४९)



%7.1.19.2
आ॒सि॒ष्य॒ते स्वाहा\-ऽ\-ऽसी॑नाय॒ स्वाहा॑\-ऽ\-ऽसि॒ताय॒ स्वाहा॑ निपथ्स्य॒ते स्वाहा॑ नि॒पद्य॑मानाय॒ स्वाहा॒ निप॑न्नाय॒ स्वाहा॑ शयिष्य॒ते स्वाहा॒ शया॑नाय॒ स्वाहा॑ शयि॒ताय॒ स्वाहा॑ सम्मीलिष्य॒ते स्वाहा॑ स॒म्मील॑ते॒ स्वाहा॒ सम्मी॑लिताय॒ स्वाहा᳚ स्वफ्स्य॒ते स्वाहा᳚ स्वप॒ते स्वाहा॑ सु॒प्ताय॒ स्वाहा᳚ प्रभोथ्स्य॒ते स्वाहा᳚ प्र॒बुध्य॑मानाय॒ स्वाहा॒ प्रबु॑द्धाय॒ स्वाहा॑ जागरिष्य॒ते स्वाहा॒ जाग्र॑ते॒ स्वाहा॑ जागरि॒ताय॒ स्वाहा॒ शुश्रू॑षमाणाय॒ स्वाहा॑ शृण्व॒ते स्वाहा᳚ श्रु॒ताय॒ स्वाहा॑ वीक्षिष्य॒ते स्वाहा᳚~(५०)

%7.1.19.3
वीक्ष॑माणाय॒ स्वाहा॒ वीक्षि॑ताय॒ स्वाहा॑ सꣳहास्य॒ते स्वाहा॑ स॒ञ्जिहा॑नाय॒ स्वाहो॒ज्जिहा॑नाय॒ स्वाहा॑ विवर्थ्स्य॒ते स्वाहा॑ वि॒वर्त॑मानाय॒ स्वाहा॒ विवृ॑त्ताय॒ स्वाहो᳚त्थास्य॒ते स्वाहो॒त्तिष्ठ॑ते॒ स्वाहोत्थि॑ताय॒ स्वाहा॑ विधविष्य॒ते स्वाहा॑ विधून्वा॒नाय॒ स्वाहा॒ विधू॑ताय॒ स्वाहो᳚त्क्रꣴस्य॒ते स्वाहो॒त्क्राम॑ते॒ स्वाहोत्क्रा᳚न्ताय॒ स्वाहा॑ चङ्क्रमिष्य॒ते स्वाहा॑ चङ्क्र॒म्यमा॑णाय॒ स्वाहा॑ चङ्क्रमि॒ताय॒ स्वाहा॑ कण्डूयिष्य॒ते स्वाहा॑ कण्डू॒यमा॑नाय॒ स्वाहा॑ कण्डूयि॒ताय॒ स्वाहा॑ निकषिष्य॒ते स्वाहा॑ नि॒कष॑माणाय॒ स्वाहा॒ निक॑षिताय॒ स्वाहा॒ यदत्ति॒ तस्मै॒ स्वाहा॒ यत्पिब॑ति॒ तस्मै॒ स्वाहा॒ यन्मेह॑ति॒ तस्मै॒ स्वाहा॒ यच्छकृ॑त्क॒रोति॒ तस्मै॒ स्वाहा॒ रेत॑से॒ स्वाहा᳚ प्र॒जाभ्यः॒ स्वाहा᳚ प्रजन॑नाय॒ स्वाहा॒ सर्व॑स्मै॒ स्वाहा᳚॥~(५१)

%7.1.20.0
{\anuvakamend[{}]}

%7.1.20.1
अ॒ग्नये॒ स्वाहा॑ वा॒यवे॒ स्वाहा॒ सूर्या॑य॒ स्वाह॒र्तम॑स्यृ॒तस्य॒र्तम॑सि स॒त्यम॑सि स॒त्यस्य॑ स॒त्यम॑स्यृ॒तस्य॒ पन्था॑ असि दे॒वानां᳚ छा॒यामृ॑तस्य॒ नाम॒ तथ्स॒त्यं यत्त्वं प्र॒जा\-प॑ति॒रस्यधि॒ यद॑स्मिन्वा॒जिनी॑व॒ शुभः॒ स्पर्ध॑न्ते॒ दिवः॒ सूर्ये॑ण॒ विशो॒\-ऽपो वृ॑णा॒नः प॑वते क॒व्यन्प॒शुं न गो॒पा इर्यः॒ परि॑ज्मा~(५२)

%7.2.0.0

%7.2.0.0
{\anuvakamend[{}]}

%%% END PRASHNA

\sect{द्वितीयः प्रश्नः}\setcounter{anuvakam}{0}
\dnsub{तैत्तिरीयसंहितायां सप्तमकाण्डे द्वितीयः प्रश्नः}
%7.2.1.0
%7.2.1.1
सा॒ध्या वै दे॒वाः सु॑व॒र्गका॑मा ए॒तꣳ ष॑ड्रा॒त्रम॑पश्य॒न्तमाह॑र॒न्तेना॑यजन्त॒ ततो॒ वै ते सु॑व॒र्गं लो॒कमा॑य॒न्॒ य ए॒वं वि॒द्वाꣳसः॑ षड्रा॒त्रमास॑ते सुव॒र्गमे॒व लो॒कं य॑न्ति देवस॒त्रं वै ष॑ड्रा॒त्रः प्र॒त्यक्ष॒ꣴ॒ ह्ये॑तानि॑ पृ॒ष्ठानि॒ य ए॒वं वि॒द्वाꣳसः॑ षड्रा॒त्रमास॑ते सा॒क्षादे॒व दे॒वता॑ अ॒भ्यारो॑हन्ति षड्रा॒त्रो भ॑वति॒ षड्वा ऋ॒तवः॒ षट्पृ॒ष्ठानि॑~(१)

%7.2.1.2
पृ॒ष्ठैरे॒वर्तून॒न्वारो॑हन्त्यृ॒तुभिः॑ संवथ्स॒रन्ते सं॑वथ्स॒र ए॒व प्रति॑ तिष्ठन्ति बृहद्रथन्त॒रा\-भ्यां᳚ यन्ती॒यं वाव र॑थन्त॒रम॒सौ बृ॒हदा॒भ्यामे॒व य॒न्त्यथो॑ अ॒नयो॑रे॒व प्रति॑ तिष्ठन्त्ये॒ते वै य॒ज्ञस्या᳚ञ्ज॒साय॑नी स्रु॒ती ताभ्या॑मे॒व सु॑व॒र्गं लो॒कं य॑न्ति त्रि॒वृद॑ग्निष्टो॒मो भ॑वति॒ तेज॑ ए॒वाव॑ रुन्धते पञ्चद॒शो भ॑वतीन्द्रि॒यमे॒वाव॑ रुन्धते सप्तद॒शः~(२)

%7.2.1.3
भ॒व॒त्य॒न्नाद्य॒स्याव॑रुद्ध्या॒ अथो॒ प्रैव तेन॑ जायन्त एकवि॒ꣳ॒शो भ॑वति॒ प्रति॑ष्ठित्या॒ अथो॒ रुच॑मे॒वात्मन्द॑धते त्रिण॒वो भ॑वति॒ विजि॑त्यै त्रयस्त्रि॒ꣳ॒शो भ॑वति॒ प्रति॑ष्ठित्यै सदोहविर्धा॒निन॑ ए॒तेन॑ षड्रा॒त्रेण॑ यजेर॒न्नाश्व॑त्थी हवि॒र्धानं॒ चाग्नी᳚ध्रं च भवत॒स्तद्धि सु॑व॒र्ग्यं॑ च॒क्रीव॑ती भवतः सुव॒र्गस्य॑ लो॒कस्य॒ सम॑ष्ट्या उ॒लूख॑लबुध्नो॒ यूपो॑ भवति॒ प्रति॑ष्ठित्यै॒ प्राञ्चो॑ यान्ति॒ प्राङि॑व॒ हि सु॑व॒र्गः~(३)

%7.2.1.4
लो॒कः सर॑स्वत्या यान्त्ये॒ष वै दे॑व॒यानः॒ पन्था॒स्तमे॒वान्वारो॑हन्त्या॒क्रोश॑न्तो या॒न्त्यव॑र्तिमे॒वान्यस्मि॑न्प्रति॒षज्य॑ प्रति॒ष्ठां ग॑च्छन्ति य॒दा दश॑ श॒तं कु॒र्वन्त्यथैक॑मु॒त्थानꣳ॑ श॒तायुः॒ पुरु॑षः श॒तेन्द्रि॑य॒ आयु॑ष्ये॒वेन्द्रि॒ये प्रति॑ तिष्ठन्ति य॒दा श॒तꣳ स॒हस्रं॑ कु॒र्वन्त्यथैक॑मु॒त्थानꣳ॑ स॒हस्र॑सम्मितो॒ वा अ॒सौ लो॒को॑\-ऽमुमे॒व लो॒कम॒भि ज॑यन्ति य॒दैषां᳚ प्र॒मीये॑त य॒दा वा॒ जीये॑र॒न्नथैक॑मु॒त्थान॒न्तद्धि ती॒र्थम्॥~(४)

%7.2.2.0
{\anuvakamend[{पृ॒ष्ठानि॑ सप्तद॒शः सु॑व॒र्गो ज॑यन्ति य॒दैका॑\-दश च}]}%~(१)

%7.2.2.1
कु॒सु॒रु॒बिन्द॒ औद्दा॑लकिरकामयत पशु॒मान्थ्स्या॒मिति॒ स ए॒तꣳ स॑प्तरा॒त्रमाह॑र॒त्तेना॑यजत॒ तेन॒ वै स याव॑न्तो ग्रा॒म्याः प॒शव॒स्तानवा॑रुन्ध॒ य ए॒वं वि॒द्वान्थ्स॑प्तरा॒त्रेण॒ यज॑ते॒ याव॑न्त ए॒व ग्रा॒म्याः प॒शव॒स्ताने॒वाव॑ रुन्धे सप्तरा॒त्रो भ॑वति स॒प्त ग्रा॒म्याः प॒शवः॑ स॒प्तार॒ण्याः स॒प्त छन्दाꣴ॑स्यु॒भय॒स्याव॑रुद्ध्यै त्रि॒वृद॑ग्निष्टो॒मो भ॑वति॒ तेजः॑~(५)

%7.2.2.2
ए॒वाव॑ रुन्धे पञ्चद॒शो भ॑वतीन्द्रि॒यमे॒वाव॑ रुन्धे सप्तद॒शो भ॑वत्य॒न्नाद्य॒स्याव॑रुद्ध्या॒ अथो॒ प्रैव तेन॑ जायत एकवि॒ꣳ॒शो भ॑वति॒ प्रति॑ष्ठित्या॒ अथो॒ रुच॑मे॒वात्मन्ध॑त्ते त्रिण॒वो भ॑वति॒ विजि॑त्यै पञ्चवि॒ꣳ॒शो᳚\-ऽग्निष्टो॒मो भ॑वति प्र॒जाप॑ते॒राप्त्यै॑ महाव्र॒तवा॑न॒न्नाद्य॒स्याव॑रुद्ध्यै विश्व॒जिथ्सर्व॑पृष्ठो\-ऽतिरा॒त्रो भ॑वति॒ सर्व॑स्या॒भिजि॑त्यै॒ यत्प्र॒त्यक्ष॒म्पूर्वे॒ष्वहः॑सु पृ॒ष्ठान्यु॑पे॒युः प्र॒त्यक्षम्᳚~(६)

%7.2.2.3
वि॒श्व॒जिति॒ यथा॑ दु॒ग्धामु॑प॒सीद॑त्ये॒वमु॑त्त॒ममहः॑ स्या॒न्नैक॑रा॒त्रश्च॒न स्या᳚द्बृहद्रथन्त॒रे पूर्वे॒ष्वहः॒सूप॑ यन्ती॒यं वाव र॑थन्त॒रम॒सौ बृ॒हदा॒भ्यामे॒व न य॒न्त्यथो॑ अ॒नयो॑रे॒व प्रति॑ तिष्ठन्ति॒ यत्प्र॒त्यक्षं॑ विश्व॒जिति॑ पृ॒ष्ठान्यु॑प॒यन्ति॒ यथा॒ प्रत्तां᳚ दु॒हे ता॒दृगे॒व तत्॥~(७)

%7.2.3.0
{\anuvakamend[{तेज॑ उपे॒युः प्र॒त्यक्षं॒ द्विच॑त्वारिꣳशच्च}]}%~(२)

%7.2.3.1
बृह॒स्पति॑रकामयत ब्रह्मवर्च॒सी स्या॒मिति॒ स ए॒तम॑ष्टरा॒त्रम॑पश्य॒त्तमाह॑र॒त्तेना॑यजत॒ ततो॒ वै स ब्र॑ह्मवर्च॒स्य॑भव॒द्य ए॒वं वि॒द्वान॑ष्टरा॒त्रेण॒ यज॑ते ब्रह्मवर्च॒स्ये॑व भ॑वत्यष्टरा॒त्रो भ॑वत्य॒ष्टाक्ष॑रा गाय॒त्री गा॑य॒त्री ब्र॑ह्मवर्च॒सम्गा॑यत्रि॒यैव ब्र॑ह्मवर्च॒समव॑ रुन्धे\-ऽष्टरा॒त्रो भ॑वति॒ चत॑स्रो॒ वै दिश॒श्चत॑स्रो\-ऽवान्तरदि॒शा दि॒ग्भ्य ए॒व ब्र॑ह्मवर्च॒समव॑ रुन्धे~(८)

%7.2.3.2
त्रि॒वृद॑ग्निष्टो॒मो भ॑वति॒ तेज॑ ए॒वाव॑ रुन्धे पञ्चद॒शो भ॑वतीन्द्रि॒यमे॒वाव॑ रुन्धे सप्तद॒शो भ॑वत्य॒न्नाद्य॒स्याव॑रुद्ध्या॒ अथो॒ प्रैव तेन॑ जायत एकवि॒ꣳ॒शो भ॑वति॒ प्रति॑ष्ठित्या॒ अथो॒ रुच॑मे॒वात्मन्ध॑त्ते त्रिण॒वो भ॑वति॒ विजि॑त्यै त्रयस्त्रि॒ꣳ॒शो भ॑वति॒ प्रति॑ष्ठित्यै पञ्चवि॒ꣳ॒शो᳚\-ऽग्निष्टो॒मो भ॑वति प्र॒जाप॑ते॒राप्त्यै॑ महाव्र॒तवा॑न॒न्नाद्य॒स्याव॑रुद्ध्यै विश्व॒जिथ्सर्व॑पृष्ठो\-ऽतिरा॒त्रो भ॑वति॒ सर्व॑स्या॒भिजि॑त्यै॥~(९)

%7.2.4.0
{\anuvakamend[{दि॒ग्भ्य ए॒व ब्र॑ह्मवर्च॒समव॑\-रुन्धे॒\-ऽभिजि॑त्यै}]}%~(३)

%7.2.4.1
प्र॒जा\-प॑तिः प्र॒जा अ॑सृजत॒ ताः सृ॒ष्टाः क्षुधं॒ न्या॑य॒न्थ्स ए॒तं न॑वरा॒त्रम॑पश्य॒त्तमाह॑र॒त्तेना॑यजत॒ ततो॒ वै प्र॒जाभ्यो॑\-ऽ कल्पत॒ यर्\mbox{}हि॑ प्र॒जाः क्षुधं॑ नि॒गच्छे॑यु॒स्तर्\mbox{}हि॑ नवरा॒त्रेण॑ यजेते॒मे हि वा ए॒तासां᳚ लो॒का अकॢ॑प्ता॒ अथै॒ताः क्षुधं॒ नि ग॑च्छन्ती॒माने॒वाभ्यो॑ लो॒कान्क॑ल्पयति॒ तान्कल्प॑मानान्प्र॒जाभ्यो\-ऽनु॑ कल्पते॒ कल्प॑न्ते~(१०)

%7.2.4.2
अ॒स्मा॒ इ॒मे लो॒का ऊर्जं॑ प्र॒जासु॑ दधाति त्रिरा॒त्रेणै॒वेमं लो॒कं क॑ल्पयति त्रिरा॒त्रेणा॒न्तरि॑क्षं त्रिरा॒त्रेणा॒मुं लो॒कं यथा॑ गु॒णे गु॒णम॒न्वस्य॑त्ये॒वमे॒व तल्लो॒के लो॒कमन्व॑स्यति॒ धृत्या॒ अशि॑थिलम्भावाय॒ ज्योति॒र्गौरायु॒रिति॑ ज्ञा॒ताः स्तोमा॑ भवन्ती॒यं वाव ज्योति॑र॒न्तरि॑क्षं॒ गौर॒सावायु॑रे॒ष्वे॑व लो॒केषु॒ प्रति॑ तिष्ठन्ति॒ ज्ञात्रं॑ प्र॒जाना᳚म्~(११)

%7.2.4.3
ग॒च्छ॒ति॒ न॒व॒रा॒त्रो भ॑वत्यभिपू॒र्वमे॒वास्मि॒न्तेजो॑ दधाति॒ यो ज्योगा॑मयावी॒ स्याथ्स न॑वरा॒त्रेण॑ यजेत प्रा॒णा हि वा ए॒तस्याधृ॑ता॒ अथै॒तस्य॒ ज्योगा॑मयति प्रा॒णाने॒वास्मि॑न्दाधारो॒त यदी॒तासु॒र्भव॑ति॒ जीव॑त्ये॒व॥~(१२)

%7.2.5.0
{\anuvakamend[{कल्प॑न्ते प्र॒जाना॒न्त्रय॑स्त्रिꣳशच्च}]}%~(४)

%7.2.5.1
प्र॒जा\-प॑तिरकामयत॒ प्र जा॑ये॒येति॒ स ए॒तं दश॑होतारमपश्य॒त्तम॑जुहो॒त्तेन॑ दशरा॒त्रम॑सृजत॒ तेन॑ दशरा॒त्रेण॒ प्राजा॑यत दशरा॒त्राय॑ दीक्षि॒ष्यमा॑णो॒ दश॑होतारं जुहुया॒द्दश॑होत्रै॒व द॑शरा॒त्रꣳ सृ॑जते॒ तेन॑ दशरा॒त्रेण॒ प्र जा॑यते वैरा॒जो वा ए॒ष य॒ज्ञो यद्द॑शरा॒त्रो य ए॒वं वि॒द्वान्द॑शरा॒त्रेण॒ यज॑ते वि॒राज॑मे॒व ग॑च्छति प्राजाप॒त्यो वा ए॒ष य॒ज्ञो यद्द॑शरा॒त्रः~(१३)

%7.2.5.2
य ए॒वं वि॒द्वान्द॑शरा॒त्रेण॒ यज॑ते॒ प्रैव जा॑यत॒ इन्द्रो॒ वै स॒दृङ्दे॒वता॑भिरासी॒थ्स न व्या॒वृत॑मगच्छ॒थ्स प्र॒जा\-प॑ति॒मुपा॑धाव॒त् तस्मा॑ ए॒तं द॑शरा॒त्रम्प्राय॑च्छ॒त्तमाह॑र॒त्तेना॑यजत॒ ततो॒ वै सो᳚\-ऽन्याभि॑र्दे॒वता॑भिर्व्या॒वृत॑मगच्छ॒द्य ए॒वं वि॒द्वान्द॑शरा॒त्रेण॒ यज॑ते व्या॒वृत॑मे॒व पा॒प्मना॒ भ्रातृ॑व्येण गच्छति त्रिक॒कुद्वै~(१४)

%7.2.5.3
ए॒ष य॒ज्ञो यद्द॑शरा॒त्रः क॒कुत्प॑ञ्चद॒शः क॒कुदे॑कवि॒ꣳ॒शः क॒कुत्त्र॑यस्त्रि॒ꣳ॒शो य ए॒वं वि॒द्वान्द॑शरा॒त्रेण॒ यज॑ते त्रिक॒कुदे॒व स॑मा॒नानां᳚ भवति॒ यज॑मानः पञ्चद॒शो यज॑मान एकवि॒ꣳ॒शो यज॑मानस्त्रयस्त्रि॒ꣳ॒शः पुर॒ इत॑रा अभिच॒र्यमा॑णो दशरा॒त्रेण॑ यजेत देवपु॒रा ए॒व पर्यू॑हते॒ तस्य॒ न कुत॑श्च॒नोपा᳚व्या॒धो भ॑वति॒ नैन॑मभि॒चर᳚न्थ्स्तृणुते देवासु॒राः संय॑त्ता आस॒न्ते दे॒वा ए॒ताः~(१५)

%7.2.5.4
दे॒व॒पु॒रा अ॑पश्य॒न्॒ यद्द॑शरा॒त्रस्ताः पर्यौ॑हन्त॒ तेषां॒ न कुत॑श्च॒नोपा᳚व्या॒धो॑\-ऽभव॒त्ततो॑ दे॒वा अभ॑व॒न्परासु॑रा॒ यो भ्रातृ॑व्यवा॒न्थ्स्याथ्स द॑शरा॒त्रेण॑ यजेत देवपु॒रा ए॒व पर्यू॑हते॒ तस्य॒ न कुत॑श्च॒नोपा᳚व्या॒धो भ॑वति॒ भव॑त्या॒त्मना॒ परा᳚स्य॒ भ्रातृ॑व्यो भवति॒ स्तोमः॒ स्तोम॒स्योप॑स्तिर्भवति॒ भ्रातृ॑व्यमे॒वोप॑स्तिं कुरुते जा॒मि वै~(१६)

%7.2.5.5
ए॒तत्कु॑र्वन्ति॒ यज्ज्यायाꣳ॑स॒ꣴ॒ स्तोम॑मु॒पेत्य॒ कनी॑याꣳसमुप॒यन्ति॒ यद॑ग्निष्टोमसा॒मान्य॒वस्ता᳚च्च प॒रस्ता᳚च्च॒ भव॒न्त्यजा॑मित्वाय त्रि॒वृद॑ग्निष्टो॒मो᳚\-ऽग्नि॒ष्टुदा᳚ग्ने॒यीषु॑ भवति॒ तेज॑ ए॒वाव॑ रुन्धे पञ्चद॒श उ॒क्थ्य॑ ऐ॒न्द्रीष्वि॑न्द्रि॒यमे॒वाव॑ रुन्धे त्रि॒वृद॑ग्निष्टो॒मो वै᳚श्वदे॒वीषु॒ पुष्टि॑मे॒वाव॑ रुन्धे सप्तद॒शो᳚\-ऽग्निष्टो॒मः प्रा॑जाप॒त्यासु॑ तीव्रसो॒मो᳚\-ऽन्नाद्य॒स्याव॑रुद्ध्या॒ अथो॒ प्रैव तेन॑ जायते~(१७)

%7.2.5.6
ए॒क॒वि॒ꣳ॒श उ॒क्थ्यः॑ सौ॒रीषु॒ प्रति॑ष्ठित्या॒ अथो॒ रुच॑मे॒वात्मन्ध॑त्ते सप्तद॒शो᳚\-ऽग्निष्टो॒मः प्रा॑जाप॒त्यासू॑पह॒व्य॑ उपह॒वमे॒व ग॑च्छति त्रिण॒वाव॑ग्निष्टो॒माव॒भित॑ ऐ॒न्द्रीषु॒ विजि॑त्यै त्रयस्त्रि॒ꣳ॒श उ॒क्थ्यो॑ वैश्वदे॒वीषु॒ प्रति॑ष्ठित्यै विश्व॒जिथ्सर्व॑पृष्ठो\-ऽ तिरा॒त्रो भ॑वति॒ सर्व॑स्या॒भिजि॑त्यै॥~(१८)

%7.2.6.0
{\anuvakamend[{प्रा॒जा॒प॒त्यो वा ए॒ष य॒ज्ञो यद्द॑शरा॒त्रस्त्रि॑क॒कुद्वा ए॒ता वै जा॑यत॒ एक॑त्रिꣳशच्च}]}%~(५)

%7.2.6.1
ऋ॒तवो॒ वै प्र॒जाका॑माः प्र॒जां नावि॑न्दन्त॒ ते॑\-ऽकामयन्त प्र॒जाꣳ सृ॑जेमहि प्र॒जामव॑ रुन्धीमहि प्र॒जां वि॑न्देमहि प्र॒जाव॑न्तः स्या॒मेति॒ त ए॒तमे॑कादशरा॒त्रम॑पश्य॒न्तमाह॑र॒न्तेना॑यजन्त॒ ततो॒ वै ते प्र॒जाम॑सृजन्त प्र॒जामवा॑रुन्धत प्र॒जाम॑विन्दन्त प्र॒जाव॑न्तो\-ऽभव॒न्त ऋ॒तवो॑\-ऽभव॒न्तदा᳚र्त॒वाना॑मार्तव॒त्वमृ॑तू॒नां वा ए॒ते पु॒त्रास्तस्मा᳚त्~(१९)

%7.2.6.2
आ॒र्त॒वा उ॑च्यन्ते॒ य ए॒वं वि॒द्वाꣳस॑ एकादशरा॒त्रमास॑ते प्र॒जामे॒व सृ॑जन्ते प्र॒जामव॑ रुन्धते प्र॒जां वि॑न्दन्ते प्र॒जाव॑न्तो भवन्ति॒ ज्योति॑रतिरा॒त्रो भ॑वति॒ ज्योति॑रे॒व पु॒रस्ता᳚द्दधते सुव॒र्गस्य॑ लो॒कस्यानु॑ख्यात्यै॒ पृष्ठ्यः॑ षड॒हो भ॑वति॒ षड्वा ऋ॒तवः॒ षट्पृ॒ष्ठानि॑ पृ॒ष्ठैरे॒वर्तून॒न्वारो॑हन्त्यृ॒तुभिः॑ संवथ्स॒रन्ते सं॑वथ्स॒र ए॒व प्रति॑ तिष्ठन्ति चतुर्वि॒ꣳ॒शो भ॑वति॒ चतु॑र्विꣳशत्यक्षरा गाय॒त्री~(२०)

%7.2.6.3
गा॒य॒त्रम्ब्र॑ह्मवर्च॒सङ्गा॑यत्रि॒यामे॒व ब्र॑ह्मवर्च॒से प्रति॑ तिष्ठन्ति चतुश्चत्वारि॒ꣳ॒शो भ॑वति॒ चतु॑श्चत्वारिꣳशदक्षरा त्रि॒ष्टुगि॑न्द्रि॒यं त्रि॒ष्टुप्त्रि॒ष्टुभ्ये॒वेन्द्रि॒ये प्रति॑ तिष्ठन्त्यष्टाचत्वारि॒ꣳ॒शो भ॑वत्य॒ष्टाच॑त्वारिꣳशदक्षरा॒ जग॑ती॒ जाग॑ताः प॒शवो॒ जग॑त्यामे॒व प॒शुषु॒ प्रति॑ तिष्ठन्त्येकादशरा॒त्रो भ॑वति॒ पञ्च॒ वा ऋ॒तव॑ आर्त॒वाः पञ्च॒र्तुष्वे॒वार्त॒वेषु॑ संवथ्स॒रे प्र॑ति॒ष्ठाय॑ प्र॒जामव॑ रुन्धते\-ऽतिरा॒त्राव॒भितो॑ भवतः प्र॒जायै॒ परि॑गृहीत्यै॥~(२१)

%7.2.7.0
{\anuvakamend[{तस्मा᳚द्गाय॒त्र्येका॒न्नप॑ञ्चा॒शच्च॑}]}%~(६)

%7.2.7.1
ऐ॒न्द्र॒वा॒य॒वाग्रा᳚न्गृह्णीया॒द्यः का॒मये॑त यथापू॒र्वं प्र॒जाः क॑ल्पेर॒न्निति॑ य॒ज्ञस्य॒ वै कॢप्ति॒मनु॑ प्र॒जाः क॑ल्पन्ते य॒ज्ञस्याकॢ॑प्ति॒मनु॒ न क॑ल्पन्ते यथापू॒र्वमे॒व प्र॒जाः क॑ल्पयति॒ न ज्यायाꣳ॑सं॒ कनी॑या॒नति॑ क्रामत्यैन्द्रवाय॒वाग्रा᳚न्गृह्णीयादामया॒विनः॑ प्रा॒णेन॒ वा ए॒ष व्यृ॑ध्यते॒ यस्या॒मय॑ति प्रा॒ण ऐ᳚न्द्रवाय॒वः प्रा॒णेनै॒वैन॒ꣳ॒ सम॑र्धयति मैत्रावरु॒णाग्रा᳚न्गृह्णीर॒न्॒ येषां᳚ दीक्षि॒तानां᳚ प्र॒मीये॑त~(२२)

%7.2.7.2
प्रा॒णा॒पा॒नाभ्यां॒ वा ए॒ते व्यृ॑ध्यन्ते॒ येषां᳚ दीक्षि॒तानां᳚ प्र॒मीय॑ते प्राणापा॒नौ मि॒त्रावरु॑णौ प्राणापा॒नावे॒व मु॑ख॒तः परि॑ हरन्त आश्वि॒नाग्रा᳚न्गृह्णीतानुजाव॒रो᳚\-ऽश्विनौ॒ वै दे॒वाना॑मानुजाव॒रौ प॒श्चेवाग्रं॒ पर्यैताम॒श्विना॑वे॒तस्य॑ दे॒वता॒ य आ॑नुजाव॒रस्तावे॒वैन॒मग्रं॒ परि॑ णयतः शु॒क्राग्रा᳚न्गृह्णीत ग॒तश्रीः᳚ प्रति॒ष्ठाका॑मो॒\-ऽसौ वा आ॑दि॒त्यः शु॒क्र ए॒षो\-ऽन्तो\-ऽन्त॑म्मनु॒ष्यः॑~(२३)

%7.2.7.3
श्रि॒यै ग॒त्वा नि व॑र्त॒ते\-ऽन्ता॑दे॒वान्त॒मा र॑भते॒ न ततः॒ पापी॑यान्भवति म॒न्थ्य॑ग्रान्गृह्णीताभि॒चर॑न्नार्तपा॒त्रं वा ए॒तद्यन्म॑न्थिपा॒त्रम्मृ॒त्युनै॒वैनं॑ ग्राहयति ता॒जगार्ति॒मार्च्छ॑त्याग्रय॒णाग्रा᳚न्गृह्णीत॒ यस्य॑ पि॒ता पि॑ताम॒हः पुण्यः॒ स्यादथ॒ तन्न प्रा᳚प्नु॒याद्वा॒चा वा ए॒ष इ॑न्द्रि॒येण॒ व्यृ॑ध्यते॒ यस्य॑ पि॒ता पि॑ताम॒हः पुण्यः॑~(२४)

%7.2.7.4
भव॒त्यथ॒ तन्न प्रा॒प्नोत्युर॑ इवै॒तद्य॒ज्ञस्य॒ वागि॑व॒ यदा᳚ग्रय॒णो वा॒चैवैन॑मिन्द्रि॒येण॒ सम॑र्धयति॒ न ततः॒ पापी॑यान्भव\-त्यु॒क्थ्या᳚ग्रान्गृह्णीताभिच॒र्यमा॑णः॒ सर्वे॑षां॒ वा ए॒तत्पात्रा॑णामिन्द्रि॒यं यदु॑क्थ्यपा॒त्रꣳ सर्वे॑णै॒वैन॑मिन्द्रि॒येणाति॒ प्र यु॑ङ्क्ते॒ सर॑स्वत्य॒भि नो॑ नेषि॒ वस्य॒ इति॑ पुरो॒रुचं॑ कुर्या॒द्वाग्वै~(२५)

%7.2.7.5
सर॑स्वती वा॒चैवैन॒मति॒ प्र यु॑ङ्क्ते॒ मा त्वत्क्षेत्रा॒ण्यर॑णानि ग॒न्मेत्या॑ह मृ॒त्योर्वै क्षेत्रा॒ण्यर॑णानि॒ तेनै॒व मृ॒त्योः क्षेत्रा॑णि॒ न ग॑च्छति पू॒र्णान्ग्रहा᳚न्गृह्णीयादामया॒विनः॑ प्रा॒णान् वा ए॒तस्य॒ शुगृ॑च्छति॒ यस्या॒मय॑ति प्रा॒णा ग्रहाः᳚ प्रा॒णाने॒वास्य॑ शु॒चो मु॑ञ्चत्यु॒त यदी॒तासु॒र्भव॑ति॒ जीव॑त्ये॒व पू॒र्णान्ग्रहा᳚न्गृह्णीया॒द्यर्\mbox{}हि॑ प॒र्जन्यो॒ न वर्\mbox{}षे᳚त्प्रा॒णान् वा ए॒तर्\mbox{}हि॑ प्र॒जाना॒ꣳ॒ शुगृ॑च्छति॒ यर्\mbox{}हि॑ प॒र्जन्यो॒ न॒ वर्\mbox{}ष॑ति प्रा॒णा ग्रहाः᳚ प्रा॒णाने॒व प्र॒जानाꣳ॑ शु॒चो मु॑ञ्चति ता॒जक्प्र व॑र्\mbox{}षति॥~(२६)

%7.2.8.0
{\anuvakamend[{प्र॒मीये॑त मनु॒ष्य॑ ऋध्यते॒ यस्य॑ पि॒ता पि॑ताम॒हः पुण्यो॒ वाग्वा ए॒व पू॒र्णान्ग्रहा॒न्पञ्च॑विꣳशतिश्च}]}%~(७)

%7.2.8.1
गा॒य॒त्रो वा ऐ᳚न्द्रवाय॒वो गा॑य॒त्रम्प्रा॑य॒णीय॒मह॒स्तस्मा᳚त्प्राय॒णीये\-ऽह॑न्नैन्द्रवाय॒वो गृ॑ह्यते॒ स्व ए॒वैन॑मा॒यत॑ने गृह्णाति॒ त्रैष्टु॑भो॒ वै शु॒क्रस्त्रैष्टु॑भं द्वि॒तीय॒मह॒स्तस्मा᳚द्द्वि॒तीये\-ऽह॑ञ्छु॒क्रो गृ॑ह्यते॒ स्व ए॒वैन॑मा॒यत॑ने गृह्णाति॒ जाग॑तो॒ वा आ᳚ग्रय॒णो जाग॑तं तृ॒तीय॒मह॒स्तस्मा᳚त्तृ॒तीये\-ऽह॑न्नाग्रय॒णो गृ॑ह्यते॒ स्व ए॒वैन॑मा॒यत॑ने गृह्णात्ये॒तद्वै~(२७)

%7.2.8.2
य॒ज्ञमा॑प॒द्यच्छन्दाꣴ॑स्या॒प्नोति॒ यदा᳚ग्रय॒णः श्वो गृ॒ह्यते॒ यत्रै॒व य॒ज्ञमदृ॑श॒न्तत॑ ए॒वैनं॒ पुनः॒ प्र यु॑ङ्क्ते॒ जग॑न्मुखो॒ वै द्वि॒तीय॑स्त्रिरा॒त्रो जाग॑त आग्रय॒णो यच्च॑तु॒र्थे\-ऽह॑न्नाग्रय॒णो गृ॒ह्यते॒ स्व ए॒वैन॑मा॒यत॑ने गृह्णा॒त्यथो॒ स्वमे॒व छन्दो\-ऽनु॑ प॒र्याव॑र्तन्ते॒ राथ॑न्तरो॒ वा ऐ᳚न्द्रवाय॒वो राथ॑न्तरं पञ्च॒ममह॒स्तस्मा᳚त्पञ्च॒मे\-ऽहन्न्॑~(२८)

%7.2.8.3
ऐ॒न्द्र॒वा॒य॒वो गृ॑ह्यते॒ स्व ए॒वैन॑मा॒यत॑ने गृह्णाति॒ बार्\mbox{}ह॑तो॒ वै शु॒क्रो बार्\mbox{}ह॑तꣳ ष॒ष्ठमह॒स्तस्मा᳚त्ष॒ष्ठे\-ऽह॑ञ्छु॒क्रो गृ॑ह्यते॒ स्व ए॒वैन॑मा॒यत॑ने गृह्णात्ये॒तद्वै द्वि॒तीयं॑ य॒ज्ञमा॑प॒द्यच्छन्दाꣴ॑स्या॒प्नोति॒ यच्छु॒क्रः श्वो गृ॒ह्यते॒ यत्रै॒व य॒ज्ञमदृ॑श॒न्तत॑ ए॒वैनं॒ पुनः॒ प्र यु॑ङ्क्ते त्रि॒ष्टुङ्मु॑खो॒ वै तृ॒तीय॑स्त्रिरा॒त्रस्त्रैष्टु॑भः~(२९)

%7.2.8.4
शु॒क्रो यथ्स॑प्त॒मे\-ऽह॑ञ्छु॒क्रो गृ॒ह्यते॒ स्व ए॒वैन॑मा॒यत॑ने गृह्णा॒त्यथो॒ स्वमे॒व छन्दो\-ऽनु॑ प॒र्याव॑र्तन्ते॒ वाग्वा आ᳚ग्रय॒णो वाग॑ष्ट॒ममह॒स्तस्मा॑दष्ट॒मे\-ऽह॑न्नाग्रय॒णो गृ॑ह्यते॒ स्व ए॒वैन॑मा॒यत॑ने गृह्णाति प्रा॒णो वा ऐ᳚न्द्रवाय॒वः प्रा॒णो न॑व॒ममह॒स्तस्मा᳚न्नव॒मे\-ऽह॑न्नैन्द्रवाय॒वो गृ॑ह्यते॒ स्व ए॒वैन॑मा॒यत॑ने गृह्णात्ये॒तत्~(३०)

%7.2.8.5
वै तृ॒तीयं॑ य॒ज्ञमा॑प॒द्यच्छन्दाꣴ॑स्या॒प्नोति॒ यदै᳚न्द्रवाय॒वः श्वो गृ॒ह्यते॒ यत्रै॒व य॒ज्ञमदृ॑श॒न्तत॑ ए॒वैनं॒ पुनः॒ प्र यु॒ङ्क्ते\-ऽथो॒ स्वमे॒व छन्दो\-ऽनु॑ प॒र्याव॑र्तन्ते प॒थो वा ए॒ते\-ऽध्यप॑थेन यन्ति॒ ये᳚\-ऽन्येनै᳚न्द्रवाय॒वात्प्र॑ति॒पद्य॒न्ते\-ऽन्तः॒ खलु॒ वा ए॒ष य॒ज्ञस्य॒ यद्द॑श॒ममह॑र्दश॒मे\-ऽह॑न्नैन्द्रवाय॒वो गृ॑ह्यते य॒ज्ञस्य॑~(३१)

%7.2.8.6
ए॒वान्तं॑ ग॒त्वाप॑था॒त्पन्था॒मपि॑ य॒न्त्यथो॒ यथा॒ वही॑यसा प्रति॒सारं॒ वह॑न्ति ता॒दृगे॒व तच्छन्दाꣴ॑स्य॒न्यो᳚न्यस्य॑ लो॒कम॒भ्य॑ध्याय॒न्तान्ये॒तेनै॒व दे॒वा व्य॑वाहयन्नैन्द्रवाय॒वस्य॒ वा ए॒तदा॒यत॑नं॒ यच्च॑तु॒र्थमह॒स्तस्मि॑न्नाग्रय॒णो गृ॑ह्यते॒ तस्मा॑दाग्रय॒णस्या॒यत॑ने नव॒मे\-ऽह॑न्नैन्द्रवाय॒वो गृ॑ह्यते शु॒क्रस्य॒ वा ए॒तदा॒यत॑नं॒ यत्प॑ञ्च॒मम्~(३२)

%7.2.8.7
अह॒स्तस्मि॑न्नैन्द्रवाय॒वो गृ॑ह्यते॒ तस्मा॑दैन्द्रवाय॒वस्या॒यत॑ने सप्त॒मे\-ऽह॑ञ्छु॒क्रो गृ॑ह्यत आग्रय॒णस्य॒ वा ए॒तदा॒यत॑नं॒ यत्ष॒ष्ठमह॒स्तस्मि॑ञ्छु॒क्रो गृ॑ह्यते॒ तस्मा᳚च्छु॒क्रस्या॒यत॑ने\-ऽष्ट॒मे\-ऽह॑न्नाग्रय॒णो गृ॑ह्यते॒ छन्दाꣴ॑स्ये॒व तद्वि वा॑हयति॒ प्र वस्य॑सो विवा॒हमा᳚प्नोति॒ य ए॒वं वेदाथो॑ दे॒वता᳚भ्य ए॒व य॒ज्ञे सं॒विदं॑ दधाति॒ तस्मा॑दि॒दमन्यो᳚न्यस्मै॑ ददाति॥~(३३)

%7.2.9.0
{\anuvakamend[{ए॒तद्वै प॑ञ्च॒मे\-ऽह॒न्त्रैष्टु॑भ ए॒तद्गृ॑ह्यते य॒ज्ञस्य॑ प़ञ्च॒मम॒न्यस्मा॒ एक॑ञ्च}]}%~(८)

%7.2.9.1
प्र॒जा\-प॑तिरकामयत॒ प्र जा॑ये॒येति॒ स ए॒तं द्वा॑दशरा॒त्रम॑पश्य॒त्तमाह॑र॒त्तेना॑यजत॒ ततो॒ वै स प्राजा॑यत॒ यः का॒मये॑त॒ प्र जा॑ये॒येति॒ स द्वा॑दशरा॒त्रेण॑ यजेत॒ प्रैव जा॑यते ब्रह्मवा॒दिनो॑ वदन्त्यग्निष्टो॒मप्रा॑यणा य॒ज्ञा अथ॒ कस्मा॑दतिरा॒त्रः पूर्वः॒ प्र यु॑ज्यत॒ इति॒ चक्षु॑षी॒ वा ए॒ते य॒ज्ञस्य॒ यद॑तिरा॒त्रौ क॒नीनि॑के अग्निष्टो॒मौ यत्~(३४)

%7.2.9.2
अ॒ग्नि॒ष्टो॒मं पूर्वं॑ प्रयुञ्जी॒रन्ब॑हि॒र्धा क॒नीनि॑के दध्यु॒स्तस्मा॑दतिरा॒त्रः पूर्वः॒ प्र यु॑ज्यते॒ चक्षु॑षी ए॒व य॒ज्ञे धि॒त्वा म॑ध्य॒तः क॒नीनि॑के॒ प्रति॑ दधति॒ यो वै गा॑य॒त्रीं ज्योतिः॑पक्षां॒ वेद॒ ज्योति॑षा भा॒सा सु॑व॒र्गं लो॒कमे॑ति॒ याव॑ग्निष्टो॒मौ तौ प॒क्षौ ये\-ऽन्त॑रे॒\-ऽष्टावु॒क्थ्याः᳚ स आ॒त्मैषा वै गा॑य॒त्री ज्योतिः॑पक्षा॒ य ए॒वं वेद॒ ज्योति॑षा भा॒सा सु॑व॒र्गं लो॒कम्~(३५)

%7.2.9.3
ए॒ति॒ प्र॒जा\-प॑तिर्वा ए॒ष द्वा॑दश॒धा विहि॑तो॒ यद्द्वा॑दशरा॒त्रो याव॑तिरा॒त्रौ तौ प॒क्षौ ये\-ऽन्त॑रे॒\-ऽष्टावु॒क्थ्याः᳚ स आ॒त्मा प्र॒जा\-प॑तिर्वावैष सन्थ्सद्ध॒ वै स॒त्रेण॑ स्पृणोति प्रा॒णा वै सत्प्रा॒णाने॒व स्पृ॑णोति॒ सर्वा॑सां॒ वा ए॒ते प्र॒जानां᳚ प्रा॒णैरा॑सते॒ ये स॒त्रमास॑ते॒ तस्मा᳚त्पृच्छन्ति॒ किमे॒ते स॒त्रिण॒ इति॑ प्रि॒यः प्र॒जाना॒मुत्थि॑तो भवति॒ य ए॒वं वेद॑॥~(३६)

%7.2.10.0
{\anuvakamend[{अ॒ग्नि॒ष्टो॒मौ यथ्सु॑व॒र्गल्लों॒कं प्रि॒यः प्र॒जानां॒ पञ्च॑ च}]}%~(९)

%7.2.10.1
न वा ए॒षो᳚\-ऽन्यतो॑वैश्वानरः सुव॒र्गाय॑ लो॒काय॒ प्राभ॑वदू॒र्ध्वो ह॒ वा ए॒ष आत॑त आसी॒त्ते दे॒वा ए॒तं वै᳚श्वान॒रं पर्यौ॑हन्थ्सुव॒र्गस्य॑ लो॒कस्य॒ प्रभू᳚त्या ऋ॒तवो॒ वा ए॒तेन॑ प्र॒जा\-प॑तिमयाजय॒न्तेष्वा᳚र्ध्नो॒दधि॒ तदृ॒ध्नोति॑ ह॒ वा ऋ॒त्विक्षु॒ य ए॒वं वि॒द्वान्द्वा॑दशा॒हेन॒ यज॑ते॒ ते᳚\-ऽस्मिन्नैच्छन्त॒ स रस॒मह॑ वस॒न्ताय॒ प्राय॑च्छत्~(३७)

%7.2.10.2
यवं॑ ग्री॒ष्मायौष॑धीर्व॒र्॒\mbox{}षाभ्यो᳚ व्री॒हीञ्छ॒रदे॑ माषति॒लौ हे॑मन्तशिशि॒राभ्या॒न्तेनेन्द्रं॑ प्र॒जा\-प॑तिरयाजय॒त्ततो॒ वा इन्द्र॒ इन्द्रो॑\-ऽभव॒त्तस्मा॑दाहुरानुजाव॒रस्य॑ य॒ज्ञ इति॒ स ह्ये॑तेनाग्रे\-ऽय॑जतै॒ष ह॒ वै कु॒णप॑मत्ति॒ यः स॒त्रे प्र॑तिगृ॒ह्णाति॑ पुरुषकुण॒पम॑श्वकुण॒पङ्गौर्वा अन्नं॒ येन॒ पात्रे॒णान्न॒म्बिभ्र॑ति॒ यत्तन्न नि॒र्णेनि॑जति॒ ततो\-ऽधि॑~(३८)

%7.2.10.3
मलं॑ जायत॒ एक॑ ए॒व य॑जे॒तैको॒ हि प्र॒जा\-प॑ति॒रार्ध्नो॒द्द्वाद॑श॒ रात्री᳚र्दीक्षि॒तः स्या॒द्द्वाद॑श॒ मासाः᳚ संवथ्स॒रः सं॑वथ्स॒रः प्र॒जा\-प॑तिः प्र॒जा\-प॑ति॒र्वावैष ए॒ष ह॒ त्वै जा॑यते॒ यस्तप॒सो\-ऽधि॒ जाय॑ते चतु॒र्धा वा ए॒तास्ति॒स्रस्ति॑स्रो॒ रात्र॑यो॒ यद्द्वाद॑शोप॒सदो॒ याः प्र॑थ॒मा य॒ज्ञं ताभिः॒ सम्भ॑रति॒ या द्वि॒तीया॑ य॒ज्ञं ताभि॒रा र॑भते~(३९)

%7.2.10.4
यास्तृ॒तीयाः॒ पात्रा॑णि॒ ताभि॒र्निर्णे॑निक्ते॒ याश्च॑तु॒र्थीरपि॒ ताभि॑रा॒त्मान॑मन्तर॒तः शु॑न्धते॒ यो वा अ॑स्य प॒शुमत्ति॑ मा॒ꣳ॒सꣳ सो᳚\-ऽत्ति॒ यः पु॑रो॒डाश॑म्म॒स्तिष्क॒ꣳ॒ स यः प॑रिवा॒पं पुरी॑ष॒ꣳ॒ स य आज्य॑म्म॒ज्जान॒ꣳ॒ स यः सोमꣴ॒ स्वेद॒ꣳ॒ सो\-ऽपि॑ ह॒ वा अ॑स्य शीर्\mbox{}ष॒ण्या॑ नि॒ष्पदः॒ प्रति॑ गृह्णाति॒ यो द्वा॑दशा॒हे प्र॑तिगृ॒ह्णाति॒ तस्मा᳚द्द्वादशा॒हेन॒ न याज्य॑म्पा॒प्मनो॒ व्यावृ॑त्त्यै॥~(४०)

%7.2.11.0
{\anuvakamend[{अय॑च्छ॒दधि॑ रभते द्वादशा॒हेन॑ च॒त्वारि॑ च}]}%॥10॥

%7.2.11.1
एक॑स्मै॒ स्वाहा॒ द्वाभ्या॒ꣴ॒ स्वाहा᳚ त्रि॒भ्यः स्वाहा॑ च॒तुर्भ्यः॒ स्वाहा॑ प॒ञ्चभ्यः॒ स्वाहा॑ ष॒ड्भ्यः स्वाहा॑ स॒प्तभ्यः॒ स्वाहा᳚\-ऽष्टा॒भ्यः स्वाहा॑ न॒वभ्यः॒ स्वाहा॑ द॒शभ्यः॒ स्वाहै॑काद॒शभ्यः॒ स्वाहा᳚ द्वाद॒शभ्यः॒ स्वाहा᳚ त्रयोद॒शभ्यः॒ स्वाहा॑ चतुर्द॒शभ्यः॒ स्वाहा॑ पञ्चद॒शभ्यः॒ स्वाहा॑ षोड॒शभ्यः॒ स्वाहा॑ सप्तद॒शभ्यः॒ स्वाहा᳚\-ऽष्टाद॒शभ्यः॒ स्वाहैका॒न्न विꣳ॑श॒त्यै स्वाहा॒ नव॑विꣳशत्यै॒ स्वाहैका॒न्न च॑त्वारि॒ꣳ॒शते॒ स्वाहा॒ नव॑चत्वारिꣳशते॒ स्वाहैका॒न्न ष॒ष्ट्यै स्वाहा॒ नव॑षष्ट्यै॒ स्वाहैका॒न्नाशी॒त्यै स्वाहा॒ नवा॑शीत्यै॒ स्वाहैका॒न्न श॒ताय॒ स्वाहा॑ श॒ताय॒ स्वाहा॒ द्वाभ्याꣳ॑ श॒ताभ्या॒ꣴ॒ स्वाहा॒ सर्व॑स्मै॒ स्वाहा᳚॥~(४१)

%7.2.12.0
{\anuvakamend[{नव॑चत्वारिꣳशते॒ स्वाहैका॒न्नैक॑विꣳशतिश्च}]}%॥11॥

%7.2.12.1
एक॑स्मै॒ स्वाहा᳚ त्रि॒भ्यः स्वाहा॑ प॒ञ्चभ्यः॒ स्वाहा॑ स॒प्तभ्यः॒ स्वाहा॑ न॒वभ्यः॒ स्वाहै॑काद॒शभ्यः॒ स्वाहा᳚ त्रयोद॒शभ्यः॒ स्वाहा॑ पञ्चद॒शभ्यः॒ स्वाहा॑ सप्तद॒शभ्यः॒ स्वाहैका॒न्न विꣳ॑श॒त्यै स्वाहा॒ नव॑विꣳशत्यै॒ स्वाहैका॒न्न च॑त्वारि॒ꣳ॒शते॒ स्वाहा॒ नव॑चत्वारिꣳशते॒ स्वाहैका॒न्न ष॒ष्ट्यै स्वाहा॒ नव॑षष्ट्यै॒ स्वाहैका॒न्नाशी॒त्यै स्वाहा॒ नवा॑शीत्यै॒ स्वाहैका॒न्न श॒ताय॒ स्वाहा॑ श॒ताय॒ स्वाहा॒ सर्व॑स्मै॒ स्वाहा᳚॥~(४२)

%7.2.13.0
{\anuvakamend[{एक॑स्मै त्रि॒भ्यः प॑ञ्चा॒शत्}]}%॥12॥

%7.2.13.1
द्वाभ्या॒ꣴ॒ स्वाहा॑ च॒तुर्भ्यः॒ स्वाहा॑ ष॒ड्भ्यः स्वाहा᳚\-ऽष्टा॒भ्यः स्वाहा॑ द॒शभ्यः॒ स्वाहा᳚ द्वाद॒शभ्यः॒ स्वाहा॑ चतुर्द॒शभ्यः॒ स्वाहा॑ षोड॒शभ्यः॒ स्वाहा᳚\-ऽष्टाद॒शभ्यः॒ स्वाहा॑ विꣳश॒त्यै स्वाहा॒\-ऽष्टान॑वत्यै॒ स्वाहा॑ श॒ताय॒ स्वाहा॒ सर्व॑स्मै॒ स्वाहा᳚॥~(४३)

%7.2.14.0
{\anuvakamend[{द्वाभ्या॑म॒ष्टान॑वत्यै॒ षड्विꣳ॑शतिः}]}%॥13॥

%7.2.14.1
त्रि॒भ्यः स्वाहा॑ प॒ञ्चभ्यः॒ स्वाहा॑ स॒प्तभ्यः॒ स्वाहा॑ न॒वभ्यः॒ स्वाहै॑काद॒शभ्यः॒ स्वाहा᳚ त्रयोद॒शभ्यः॒ स्वाहा॑ पञ्चद॒शभ्यः॒ स्वाहा॑ सप्तद॒शभ्यः॒ स्वाहैका॒न्न विꣳ॑श॒त्यै स्वाहा॒ नव॑विꣳशत्यै॒ स्वाहैका॒न्न च॑त्वारि॒ꣳ॒शते॒ स्वाहा॒ नव॑चत्वारिꣳशते॒ स्वाहैका॒न्न ष॒ष्ट्यै स्वाहा॒ नव॑षष्ट्यै॒ स्वाहैका॒न्नाशी॒त्यै स्वाहा॒ नवा॑शीत्यै॒ स्वाहैका॒न्न श॒ताय॒ स्वाहा॑ श॒ताय॒ स्वाहा॒ सर्व॑स्मै॒ स्वाहा᳚॥~(४४)

%7.2.15.0
{\anuvakamend[{त्रि॒भ्यो᳚\-ऽष्टाचत्वारि॒ꣳ॒शत्}]}%॥14॥

%7.2.15.1
च॒तुर्भ्यः॒ स्वाहा᳚\-ऽष्टा॒भ्यः स्वाहा᳚ द्वाद॒शभ्यः॒ स्वाहा॑ षोड॒शभ्यः॒ स्वाहा॑ विꣳश॒त्यै स्वाहा॒ षण्ण॑वत्यै॒ स्वाहा॑ श॒ताय॒ स्वाहा॒ सर्व॑स्मै॒ स्वाहा᳚॥~(४५)

%7.2.16.0
{\anuvakamend[{च॒तुर्भ्यः॒ षण्ण॑वत्यै॒ षोड॑श}]}%॥15॥

%7.2.16.1
प॒ञ्चभ्यः॒ स्वाहा॑ द॒शभ्यः॒ स्वाहा॑ पञ्चद॒शभ्यः॒ स्वाहा॑ विꣳश॒त्यै स्वाहा॒ पञ्च॑नवत्यै॒ स्वाहा॑ श॒ताय॒ स्वाहा॒ सर्व॑स्मै॒ स्वाहा᳚॥~(४६)

%7.2.17.0
{\anuvakamend[{प॒ञ्चभ्यः॒ प़ञ्च॑नवत्यै॒ चतु॑र्दश}]}%॥16॥

%7.2.17.1
द॒शभ्यः॒ स्वाहा॑ विꣳश॒त्यै स्वाहा᳚ त्रि॒ꣳ॒शते॒ स्वाहा॑ चत्वारि॒ꣳ॒शते॒ स्वाहा॑ पञ्चा॒शते॒ स्वाहा॑ ष॒ष्ट्यै स्वाहा॑ सप्त॒त्यै स्वाहा॑\-ऽशी॒त्यै स्वाहा॑ नव॒त्यै स्वाहा॑ श॒ताय॒ स्वाहा॒ सर्व॑स्मै॒ स्वाहा᳚॥~(४७)

%7.2.18.0
{\anuvakamend[{द॒शभ्यो॒ द्वाविꣳ॑शतिः}]}%॥17॥

%7.2.18.1
वि॒ꣳ॒श॒त्यै स्वाहा॑ चत्वारि॒ꣳ॒शते॒ स्वाहा॑ ष॒ष्ट्यै स्वाहा॑\-ऽशी॒त्यै स्वाहा॑ श॒ताय॒ स्वाहा॒ सर्व॑स्मै॒ स्वाहा᳚॥~(४८)

%7.2.19.0
{\anuvakamend[{वि॒ꣳ॒श॒त्यै द्वाद॑श}]}%॥18॥

%7.2.19.1
प॒ञ्चा॒शते॒ स्वाहा॑ श॒ताय॒ स्वाहा॒ द्वाभ्याꣳ॑ श॒ताभ्या॒ꣴ॒ स्वाहा᳚ त्रि॒भ्यः श॒तेभ्यः॒ स्वाहा॑ च॒तुर्भ्यः॑ श॒तेभ्यः॒ स्वाहा॑ प॒ञ्चभ्यः॑ श॒तेभ्यः॒ स्वाहा॑ ष॒ड्भ्यः श॒तेभ्यः॒ स्वाहा॑ स॒प्तभ्यः॑ श॒तेभ्यः॒ स्वाहा᳚\-ऽष्टा॒भ्यः श॒तेभ्यः॒ स्वाहा॑ न॒वभ्यः॑ श॒तेभ्यः॒ स्वाहा॑ स॒हस्रा॑य॒ स्वाहा॒ सर्व॑स्मै॒ स्वाहा᳚॥~(४९)

%7.2.20.0
{\anuvakamend[{प॒ञ्चा॒शते॒ द्वात्रिꣳ॑शत्}]}%॥19॥

%7.2.20.1
श॒ताय॒ स्वाहा॑ स॒हस्रा॑य॒ स्वाहा॒\-ऽयुता॑य॒ स्वाहा॑ नि॒युता॑य॒ स्वाहा᳚ प्र॒युता॑य॒ स्वाहा\-ऽर्बु॑दाय॒ स्वाहा॒ न्य॑र्बुदाय॒ स्वाहा॑ समु॒द्राय॒ स्वाहा॒ मध्या॑य॒ स्वाहा\-ऽन्ता॑य॒ स्वाहा॑ परा॒र्धाय॒ स्वाहो॒षसे॒ स्वाहा॒ व्यु॑ष्ट्यै॒ स्वाहो॑देष्य॒ते स्वाहो᳚द्य॒ते स्वाहोदि॑ताय॒ स्वाहा॑ सुव॒र्गाय॒ स्वाहा॑ लो॒काय॒ स्वाहा॒ सर्व॑स्मै॒ स्वाहा᳚॥~(५०)

%7.3.0.0
{\anuvakamend[{श॒ताया॒ष्टात्रिꣳ॑शत्}]}%॥20॥

%7.3.0.0

{\anuvakamend[{प्र॒जवं॑ ब्रह्मवा॒दिनः॒ किमे॒ष वा आ॒प्त आ॑दि॒त्या उ॒भयोः᳚ प्र॒जा\-प॑ति॒रन्वा॑य॒न्निन्द्रो॒ वै स॒दृङ्ङिन्द्रो॒ वै शि॑थि॒लः प्र॒जा\-प॑तिरकामयतान्ना॒दः सा वि॒राड॒सावा॑दि॒त्यो᳚\-ऽर्वाङ्भू॒तमा मे॒\-ऽग्निना॒ स्वाहा॒धिन्द॒द्भ्यो᳚\-ऽञ्ज्ये॒ताय॑ कृ॒ष्णायौष॑धीभ्यो॒ वन॒स्पति॑भ्यो विꣳश॒तिः}]%॥20॥
}
%%% END PRASHNA

\sect{तृतीयः प्रश्नः}\setcounter{anuvakam}{0}
\dnsub{तैत्तिरीयसंहितायां सप्तमकाण्डे तृतीयः प्रश्नः}
%7.3.1.0
%7.3.1.1
प्र॒जवं॒ वा ए॒तेन॑ यन्ति॒ यद्द॑श॒ममहः॑ पापाव॒हीयं॒ वा ए॒तेन॑ भवन्ति॒ यद्द॑श॒ममह॒र्यो वै प्र॒जवं॑ य॒तामप॑थेन प्रति॒पद्य॑ते॒ यः स्था॒णुꣳ हन्ति॒ यो भ्रेषं॒ न्येति॒ स ही॑यते॒ स यो वै द॑श॒मे\-ऽह॑न्नविवा॒क्य उ॑पह॒न्यते॒ स ही॑यते॒ तस्मै॒ य उप॑हताय॒ व्याह॒ तमे॒वान्वा॒रभ्य॒ सम॑श्ञु॒ते\-ऽथ॒ यो व्याह॒ सः~(१)

%7.3.1.2
ही॒य॒ते॒ तस्मा᳚द्दश॒मे\-ऽह॑न्नविवा॒क्य उप॑हताय॒ न व्युच्य॒मथो॒ खल्वा॑हुर्य॒ज्ञस्य॒ वै समृ॑द्धेन दे॒वाः सु॑व॒र्गं लो॒कमा॑यन् य॒ज्ञस्य॒ व्यृ॑द्धे॒नासु॑रा॒न्परा॑भावय॒न्निति॒ यत्खलु॒ वै य॒ज्ञस्य॒ समृ॑द्धं॒ तद्यज॑मानस्य॒ यद्व्यृ॑द्धं॒ तद्भ्रातृ॑व्यस्य॒ स यो वै द॑श॒मे\-ऽह॑न्नविवा॒क्य उ॑पह॒न्यते॒ स ए॒वाति॑ रेचयति॒ ते ये बाह्या॑ दृशी॒कवः॑~(२)

%7.3.1.3
स्युस्ते वि ब्रू॑यु॒र्यदि॒ तत्र॒ न वि॒न्देयु॑रन्तःसद॒साद्व्युच्यं॒ यदि॒ तत्र॒ न वि॒न्देयु॑र्गृ॒हप॑तिना॒ व्युच्य॒न्तद्व्युच्य॑मे॒वाथ॒ वा ए॒तथ्स॑र्परा॒ज्ञिया॑ ऋ॒ग्भिः स्तु॑वन्ती॒यं वै सर्प॑तो॒ राज्ञी॒ यद्वा अ॒स्यां किं चार्च॑न्ति॒ यदा॑नृ॒चुस्तेने॒यꣳ स॑र्परा॒ज्ञी ते यदे॒व किं च॑ वा॒चानृ॒चुर्यद॒तो\-ऽध्य॑र्चि॒तारः॑~(३)

%7.3.1.4
तदु॒भय॑मा॒प्त्वाव॒रुध्योत्ति॑ष्ठा॒मेति॒ ताभि॒र्मन॑सा स्तुवते॒ न वा इ॒माम॑श्वर॒थो नाश्व॑तरीर॒थः स॒द्यः पर्या᳚प्तुमर्\mbox{}हति॒ मनो॒ वा इ॒माꣳ स॒द्यः पर्या᳚प्तुमर्\mbox{}हति॒ मनः॒ परि॑भवितु॒मथ॒ ब्रह्म॑ वदन्ति॒ परि॑मिता॒ वा ऋचः॒ परि॑मितानि॒ सामा॑नि॒ परि॑मितानि॒ यजू॒ꣳ॒ष्यथै॒तस्यै॒वान्तो॒ नास्ति॒ यद्ब्रह्म॒ तत्प्र॑तिगृण॒त आ च॑क्षीत॒ स प्र॑तिग॒रः॥~(४)

%7.3.2.0
{\anuvakamend[{व्याह॒ स दृ॑शी॒कवो᳚\-ऽर्चि॒तारः॒ स एक॑ञ्च}]}%~(१)

%7.3.2.1
ब्र॒ह्म॒वा॒दिनो॑ वदन्ति॒ किं द्वा॑दशा॒हस्य॑ प्रथ॒मेनाह्न॒र्त्विजां॒ यज॑मानो वृङ्क्त॒ इति॒ तेज॑ इन्द्रि॒यमिति॒ किं द्वि॒तीये॒नेति॑ प्रा॒णान॒न्नाद्य॒मिति॒ किं तृ॒तीये॒नेति॒ त्रीनि॒माल्लोण॒कानिति॒ किं च॑तु॒र्थेनेति॒ चतु॑ष्पदः प॒शूनिति॒ किम्प॑ञ्च॒मेनेति॒ पञ्चा᳚क्षराम्प॒ङ्क्तिमिति॒ किꣳ ष॒ष्ठेनेति॒ षडृ॒तूनिति॒ किꣳ स॑प्त॒मेनेति॑ स॒प्तप॑दा॒ꣳ॒ शक्व॑री॒मिति॑~(५)

%7.3.2.2
किम॑ष्ट॒मेनेत्य॒ष्टाक्ष॑रां गाय॒त्रीमिति॒ किं न॑व॒मेनेति॑ त्रि॒वृत॒ꣴ॒ स्तोम॒मिति॒ किं द॑श॒मेनेति॒ दशा᳚क्षरां वि॒राज॒मिति॒ किमे॑काद॒शेनेत्येका॑\-दशाक्षरां त्रि॒ष्टुभ॒मिति॒ किं द्वा॑द॒शेनेति॒ द्वाद॑शाक्षरां॒ जग॑ती॒मित्ये॒ताव॒द्वा अ॑स्ति॒ याव॑दे॒तद्याव॑दे॒वास्ति॒ तदे॑षां वृङ्क्ते॥~(६)

%7.3.3.0
{\anuvakamend[{शक्व॑री॒मित्येक॑चत्वारिꣳशच्च}]}%~(२)

%7.3.3.1
ए॒ष वा आ॒प्तो द्वा॑दशा॒हो यत्त्र॑योदशरा॒त्रः स॑मा॒नꣴ ह्ये॑तदह॒र्यत्प्रा॑य॒णीय॑श्चोदय॒नीय॑श्च॒ त्र्य॑तिरात्रो भवति॒ त्रय॑ इ॒मे लो॒का ए॒षां लो॒काना॒माप्त्यै᳚ प्रा॒णो वै प्र॑थ॒मो॑\-ऽतिरा॒त्रो व्या॒नो द्वि॒तीयो॑\-ऽपा॒नस्तृ॒तीयः॑ प्राणापानोदा॒नेष्वे॒वान्नाद्ये॒ प्रति॑ तिष्ठन्ति॒ सर्व॒मायु॑र्यन्ति॒ य ए॒वं वि॒द्वाꣳ॑सस्त्रयोदशरा॒त्रमास॑ते॒ तदा॑हु॒र्वाग्वा ए॒षा वित॑ता~(७)

%7.3.3.2
यद्द्वा॑दशा॒हस्तां विच्छि॑न्द्यु॒र्यन्मध्ये॑\-ऽतिरा॒त्रं कु॒र्युरु॑प॒दासु॑का गृ॒हप॑ते॒र्वाख्स्या॑दु॒परि॑ष्टाच्छन्दो॒माना᳚म्महाव्र॒तं कु॑र्वन्ति॒ सन्त॑तामे॒व वाच॒मव॑ रुन्ध॒ते\-ऽनु॑पदासुका गृ॒हप॑ते॒र्वाग्भ॑वति प॒शवो॒ वै छ॑न्दो॒मा अन्न॑म्महाव्र॒तं यदु॒परि॑ष्टाच्छन्दो॒माना᳚\-म्महाव्र॒तं कु॒र्वन्ति॑ प॒शुषु॑ चै॒वान्नाद्ये॑ च॒ प्रति॑ तिष्ठन्ति॥~(८)

%7.3.4.0
{\anuvakamend[{वित॑ता॒ त्रिच॑त्वारिꣳशच्च}]}%~(३)

%7.3.4.1
आ॒दि॒त्या अ॑कामयन्तो॒भयो᳚र्लो॒कयोर्॑ऋध्नुया॒मेति॒ त ए॒तं च॑तुर्दशरा॒त्रम॑पश्य॒न्तमाह॑र॒न्तेना॑यजन्त॒ ततो॒ वै त उ॒भयो᳚र्लो॒कयो॑रार्ध्नुवन्न॒स्मिꣴश्चा॒मुष्मिꣴ॑श्च॒ य ए॒वं वि॒द्वाꣳस॑श्चतुर्दशरा॒त्रमास॑त उ॒भयो॑रे॒व लो॒कयोर्॑\mbox{}॑ध्नुवन्त्य॒\-स्मिꣴश्चा॒मुष्मिꣴ॑श्च चतुर्दशरा॒त्रो भ॑वति स॒प्त ग्रा॒म्या ओष॑धयः स॒प्तार॒ण्या उ॒भयी॑षा॒मव॑रुद्ध्यै॒ यत्प॑रा॒चीना॑नि पृ॒ष्ठानि॑~(९)

%7.3.4.2
भव॑न्त्य॒मुमे॒व तैर्लो॒कम॒भि ज॑यन्ति॒ यत्प्र॑ती॒चीना॑नि पृ॒ष्ठानि॒ भव॑न्ती॒ममे॒व तैर्लो॒कम॒भि ज॑यन्ति त्रयस्त्रि॒ꣳ॒शौ म॑ध्य॒तः स्तोमौ॑ भवतः॒ साम्रा᳚ज्यमे॒व ग॑च्छन्त्यधिरा॒जौ भ॑वतो\-ऽधिरा॒जा ए॒व स॑मा॒नानां᳚ भवन्त्यतिरा॒त्राव॒भितो॑ भवतः॒ परि॑गृहीत्यै॥~(१०)

%7.3.5.0
{\anuvakamend[{पृ॒ष्ठानि॒ चतु॑स्त्रिꣳशच्च}]}%~(४)

%7.3.5.1
प्र॒जा\-प॑तिः सुव॒र्गं लो॒कमै॒त्तं दे॒वा अन्वा॑य॒न्ताना॑दि॒त्याश्च॑ प॒शव॒श्चान्वा॑य॒न्ते दे॒वा अ॑ब्रुव॒न्॒ यान्प॒शूनु॒पाजी॑विष्म॒ त इ॒मे᳚\-ऽन्वाग्म॒न्निति॒ तेभ्य॑ ए॒तं च॑तुर्दशरा॒त्रम्प्रत्यौ॑ह॒न्त आ॑दि॒त्याः पृ॒ष्ठैः सु॑व॒र्गं लो॒कमारो॑हन्त्र्य॒हाभ्या॑म॒स्मिँल्लो॒के प॒शून्प्रत्यौ॑हन्पृ॒ष्ठैरा॑दि॒त्या अ॒मुष्मिँ॑ल्लो॒क आर्ध्नु॑वन्त्र्य॒हाभ्या॑म॒स्मिन्~(११)

%7.3.5.2
लो॒के प॒शवो॒ य ए॒वं वि॒द्वाꣳस॑श्चतुर्दशरा॒त्रमास॑त उ॒भयो॑रे॒व लो॒कयोर्॑ऋध्नुवन्त्य॒स्मिꣴश्चा॒मुष्मिꣴ॑श्च पृ॒ष्ठैरे॒वामुष्मिँ॑ल्लो॒क ऋ॑ध्नु॒वन्ति॑ त्र्य॒हाभ्या॑म॒स्मिँल्लो॒के ज्योति॒र्गौरायु॒रिति॑ त्र्य॒हो भ॑वती॒यं वाव ज्योति॑र॒न्तरि॑क्षं॒ गौर॒सावायु॑रि॒माने॒व लो॒कान॒भ्यारो॑हन्ति॒ यद॒न्यतः॑ पृ॒ष्ठानि॒ स्युर्विवि॑वधꣴ स्या॒न्मध्ये॑ पृ॒ष्ठानि॑ भवन्ति सविवध॒त्वाय॑~(१२)

%7.3.5.3
ओजो॒ वै वी॒र्यं॑ पृ॒ष्ठान्योज॑ ए॒व वी॒र्य॑म्मध्य॒तो द॑धते बृहद्रथन्त॒रा\-भ्यां᳚ यन्ती॒यं वाव र॑थन्त॒रम॒सौ बृ॒हदा॒भ्यामे॒व य॒न्त्यथो॑ अ॒नयो॑रे॒व प्रति॑ तिष्ठन्त्ये॒ते वै य॒ज्ञस्या᳚ञ्ज॒साय॑नी स्रु॒ती ताभ्या॑मे॒व सु॑व॒र्गं लो॒कं य॑न्ति॒ परा᳚ञ्चो॒ वा ए॒ते सु॑व॒र्गं लो॒कम॒भ्यारो॑हन्ति॒ ये प॑रा॒चीना॑नि पृ॒ष्ठान्यु॑प॒यन्ति॑ प्र॒त्यङ्त्र्य॒हो भ॑वति प्र॒त्यव॑रूढ्या॒ अथो॒ प्रति॑ष्ठित्या उ॒भयो᳚र्लो॒कयोर्॑\mbox{}॑ऋद्ध्वोत्ति॑ष्ठन्ति॒ चतु॑र्दशै॒तास्तासां॒ या दश॒ दशा᳚क्षरा वि॒राडन्नं॑ वि॒राड्वि॒राजै॒वान्नाद्य॒मव॑ रुन्धते॒ याश्चत॑स्र॒श्चत॑स्रो॒ दिशो॑ दि॒क्ष्वे॑व प्रति॑ तिष्ठन्त्यतिरा॒त्राव॒भितो॑ भवतः॒ परि॑गृहीत्यै॥~(१३)

%7.3.6.0
{\anuvakamend[{आर्ध्नु॑वन्त्र्य॒हाभ्या॑म॒स्मिन्थ्स॑विवध॒त्वाय॒ प्रति॑ष्ठित्या॒ एक॑त्रिꣳशच्च}]}%~(५)

%7.3.6.1
इन्द्रो॒ वै स॒दृङ्दे॒वता॑भिरासी॒थ्स न व्या॒वृत॑मगच्छ॒थ्स प्र॒जा\-प॑ति॒मुपा॑धाव॒त्तस्मा॑ ए॒तम्प॑ञ्चदशरा॒त्रम्प्राय॑च्छ॒त्तमाह॑र॒त् तेना॑यजत॒ ततो॒ वै सो᳚\-ऽन्याभि॑र्दे॒वता॑भिर्व्या॒वृत॑मगच्छ॒द्य ए॒वं वि॒द्वाꣳसः॑ पञ्चदशरा॒त्रमास॑ते व्या॒वृत॑मे॒व पा॒प्मना॒ भ्रातृ॑व्येण गच्छन्ति॒ ज्योति॒र्गौरायु॒रिति॑ त्र्य॒हो भ॑वती॒यं वाव ज्योति॑र॒न्तरि॑क्षम्~(१४)

%7.3.6.2
गौर॒सावायु॑रे॒ष्वे॑व लो॒केषु॒ प्रति॑ तिष्ठ॒न्त्यस॑त्त्रं॒ वा ए॒तद्यद॑छन्दो॒मं यच्छ॑न्दो॒मा भव॑न्ति॒ तेन॑ स॒त्रं दे॒वता॑ ए॒व पृ॒ष्ठैरव॑ रुन्धते प॒शूञ्छ॑न्दो॒मैरोजो॒ वा वी॒र्यं॑ पृ॒ष्ठानि॑ प॒शव॑श्छन्दो॒मा ओज॑स्ये॒व वी॒र्ये॑ प॒शुषु॒ प्रति॑ तिष्ठन्ति पञ्चदशरा॒त्रो भ॑वति पञ्चद॒शो वज्रो॒ वज्र॑मे॒व भ्रातृ॑व्येभ्यः॒ प्र ह॑रन्त्यतिरा॒त्राव॒भितो॑ भवत इन्द्रि॒यस्य॒ परि॑गृहीत्यै॥~(१५)

%7.3.7.0
{\anuvakamend[{अ॒न्तरि॑क्षमिन्द्रि॒यस्यैक॑ञ्च}]}%~(६)

%7.3.7.1
इन्द्रो॒ वै शि॑थि॒ल इ॒वाप्र॑तिष्ठित आसी॒थ्सो\-ऽसु॑रेभ्यो\-ऽबिभे॒थ्स प्र॒जा\-प॑ति॒मुपा॑धाव॒त्तस्मा॑ ए॒तम्प॑ञ्चदशरा॒त्रं वज्रं॒ प्राय॑च्छ॒त् तेनासु॑रान्परा॒भाव्य॑ वि॒जित्य॒ श्रिय॑मगच्छदग्नि॒ष्टुता॑ पा॒प्मानं॒ निर॑दहत पञ्चदशरा॒त्रेणौजो॒ बल॑मिन्द्रि॒यं वी॒र्य॑मा॒त्मन्न॑धत्त॒ य ए॒वं वि॒द्वाꣳसः॑ पञ्चदशरा॒त्रमास॑ते॒ भ्रातृ॑व्याने॒व प॑रा॒भाव्य॑ वि॒जित्य॒ श्रियं॑ गच्छन्त्यग्नि॒ष्टुता॑ पा॒प्मानं॒ निः~(१६)

%7.3.7.2
द॒ह॒न्ते॒ प॒ञ्च॒द॒श॒रा॒त्रेणौजो॒ बल॑मिन्द्रि॒यं वी॒र्य॑मा॒त्मन्द॑धत ए॒ता ए॒व प॑श॒व्याः᳚ पञ्च॑दश॒ वा अ॑र्धमा॒सस्य॒ रात्र॑यो\-ऽ\-र्धमास॒शः सं॑वथ्स॒र आ᳚प्यते संवथ्स॒रम्प॒शवो\-ऽनु॒ प्र जा॑यन्ते॒ तस्मा᳚त्पश॒व्या॑ ए॒ता ए॒व सु॑व॒र्ग्याः᳚ पञ्च॑दश॒ वा अ॑र्धमा॒सस्य॒ रात्र॑यो\-ऽर्धमास॒शः सं॑वथ्स॒र आ᳚प्यते संवथ्स॒रः सु॑व॒र्गो लो॒कस्तस्मा᳚थ्सुव॒र्ग्या᳚ ज्योति॒र्गौरायु॒रिति॑ त्र्य॒हो भ॑वती॒यं वाव ज्योति॑र॒न्तरि॑क्षम्~(१७)

%7.3.7.3
गौर॒सावायु॑रि॒माने॒व लो॒कान॒भ्यारो॑हन्ति॒ यद॒न्यतः॑ पृ॒ष्ठानि॒ स्युर्विवि॑वधꣴ स्या॒न्मध्ये॑ पृ॒ष्ठानि॑ भवन्ति सविवध॒त्वायौजो॒ वै वी॒र्यं॑ पृ॒ष्ठान्योज॑ ए॒व वी॒र्य॑म्मध्य॒तो द॑धते बृहद्रथन्त॒रा\-भ्यां᳚ यन्ती॒यं वाव र॑थन्त॒रम॒सौ बृ॒हदा॒भ्यामे॒व य॒न्त्यथो॑ अ॒नयो॑रे॒व प्रति॑ तिष्ठन्त्ये॒ते वै य॒ज्ञस्या᳚ञ्ज॒साय॑नी स्रु॒ती ताभ्या॑मे॒व सु॑व॒र्गं लो॒कम्~(१८)

%7.3.7.4
य॒न्ति॒ परा᳚ञ्चो॒ वा ए॒ते सु॑व॒र्गं लो॒कम॒भ्यारो॑हन्ति॒ ये प॑रा॒चीना॑नि पृ॒ष्ठान्यु॑प॒यन्ति॑ प्र॒त्यङ्त्र्य॒हो भ॑वति प्र॒त्यव॑रूढ्या॒ अथो॒ प्रति॑ष्ठित्या उ॒भयो᳚र्लो॒कयोर्॑ऋ॒द्ध्वोत्ति॑ष्ठन्ति॒ पञ्च॑दशै॒तास्तासां॒ या दश॒ दशा᳚क्षरा वि॒राडन्नं॑ वि॒राड्वि॒राजै॒वान्नाद्य॒\-मव॑ रुन्धते॒ याः पञ्च॒ पञ्च॒ दिशो॑ दि॒क्ष्वे॑व प्रति॑ तिष्ठन्त्यतिरा॒त्राव॒भितो॑ भवत इन्द्रि॒यस्य॑ वी॒र्य॑स्य प्र॒जायै॑ पशू॒नां परि॑गृहीत्यै॥~(१९)

%7.3.8.0
{\anuvakamend[{ग॒च्छ॒न्त्य॒ग्नि॒ष्टुता॑ पा॒प्मान॒न्निर॒न्तरि॑क्षं लो॒कं प्र॒जायै॒ द्वे च॑}]}%~(७)

%7.3.8.1
प्र॒जा\-प॑तिरकामयतान्ना॒दः स्या॒मिति॒ स ए॒तꣳ स॑प्तदशरा॒त्रम॑पश्य॒त्तमाह॑र॒त्तेना॑यजत॒ ततो॒ वै सो᳚\-ऽन्ना॒दो॑\-ऽभव॒द्य ए॒वं वि॒द्वाꣳसः॑ सप्तदशरा॒त्रमास॑ते\-ऽन्ना॒दा ए॒व भ॑वन्ति पञ्चा॒हो भ॑वति॒ पञ्च॒ वा ऋ॒तवः॑ संवथ्स॒र ऋ॒तुष्वे॒व सं॑वथ्स॒रे प्रति॑ तिष्ठ॒न्त्यथो॒ पञ्चा᳚क्षरा प॒ङ्क्तिः पाङ्क्तो॑ य॒ज्ञो य॒ज्ञमे॒वाव॑ रुन्ध॒ते\-ऽस॑त्त्रं॒ वा ए॒तत्~(२०)

%7.3.8.2
यद॑छन्दो॒मं यच्छ॑न्दो॒मा भव॑न्ति॒ तेन॑ स॒त्रं दे॒वता॑ ए॒व पृ॒ष्ठैरव॑ रुन्धते प॒शूञ्छ॑न्दो॒मैरोजो॒ वै वी॒र्यं॑ पृ॒ष्ठानि॑ प॒शव॑श्छन्दो॒मा ओज॑स्ये॒व वी॒र्ये॑ प॒शुषु॒ प्रति॑ तिष्ठन्ति सप्तदशरा॒त्रो भ॑वति सप्तद॒शः प्र॒जा\-प॑तिः प्र॒जाप॑ते॒राप्त्या॑ अतिरा॒त्राव॒भितो॑ भवतो॒\-ऽन्नाद्य॑स्य॒ परि॑गृहीत्यै॥~(२१)

%7.3.9.0
{\anuvakamend[{ए॒तथ्स॒प्तत्रिꣴ॑श्चच्च}]}%~(८)

%7.3.9.1
सा वि॒राड्वि॒क्रम्या॑तिष्ठ॒द्ब्रह्म॑णा दे॒वेष्वन्ने॒नासु॑रेषु॒ ते दे॒वा अ॑कामयन्तो॒भय॒ꣳ॒ सं वृ॑ञ्जीमहि॒ ब्रह्म॒ चान्नं॒ चेति॒ त ए॒ता विꣳ॑श॒तिꣳ रात्री॑रपश्य॒न्ततो॒ वै त उ॒भय॒ꣳ॒ सम॑वृञ्जत॒ ब्रह्म॒ चान्नं॑ च ब्रह्मवर्च॒सिनो᳚\-ऽन्ना॒दा अ॑भव॒न्॒ य ए॒वं वि॒द्वाꣳस॑ ए॒ता आस॑त उ॒भय॑मे॒व सं वृ॑ञ्जते॒ ब्रह्म॒ चान्नं॑ च~(२२)

%7.3.9.2
ब्र॒ह्म॒व॒र्च॒सिनो᳚\-ऽन्ना॒दा भ॑वन्ति॒ द्वे वा ए॒ते वि॒राजौ॒ तयो॑रे॒व नाना॒ प्रति॑ तिष्ठन्ति वि॒ꣳ॒शो वै पुरु॑षो॒ दश॒ हस्त्या॑ अ॒ङ्गुल॑यो॒ दश॒ पद्या॒ यावा॑ने॒व पुरु॑ष॒स्तमा॒प्त्वोत्ति॑ष्ठन्ति॒ ज्योति॒र्गौरायु॒रिति॑ त्र्य॒हा भ॑वन्ती॒यं वाव ज्योति॑र॒न्तरि॑क्षं॒ गौर॒सावायु॑रि॒माने॒व लो॒कान॒भ्यारो॑हन्त्यभिपू॒र्वं त्र्य॒हा भ॑वन्त्यभिपू॒र्वमे॒व सु॑व॒र्गम्~(२३)

%7.3.9.3
लो॒कम॒भ्यारो॑हन्ति॒ यद॒न्यतः॑ पृ॒ष्ठानि॒ स्युर्विवि॑वधꣴ स्या॒न्मध्ये॑ पृ॒ष्ठानि॑ भवन्ति सविवध॒त्वायौजो॒ वै वी॒र्यं॑ पृ॒ष्ठान्योज॑ ए॒व वी॒र्य॑म्मध्य॒तो द॑धते बृहद्रथन्त॒रा\-भ्यां᳚ यन्ती॒यं वाव र॑थन्त॒रम॒सौ बृ॒हदाभ्यामे॒व य॒न्त्यथो॑ अ॒नयो॑रे॒व प्रति॑ तिष्ठन्त्ये॒ते वै य॒ज्ञस्या᳚ञ्ज॒साय॑नी स्रु॒ती ताभ्या॑मे॒व सु॑व॒र्गं लो॒कं य॑न्ति॒ परा᳚ञ्चो॒ वा ए॒ते सु॑व॒र्गं लो॒कम॒भ्यारो॑हन्ति॒ ये प॑रा॒चीना॑नि पृ॒ष्ठान्यु॑प॒यन्ति॑ प्र॒त्यङ्त्र्य॒हो भ॑वति प्र॒त्यव॑रूढ्या॒ अथो॒ प्रति॑ष्ठित्या उ॒भयो᳚र्लो॒कयोर्॑ ऋ॒द्ध्वोत्ति॑ष्ठन्त्यतिरा॒त्राव॒भितो॑ भवतो ब्रह्मवर्च॒सस्या॒न्नाद्य॑स्य॒ परि॑गृहीत्यै॥~(२४)

%7.3.10.0
{\anuvakamend[{वृ॒ञ्ज॒ते॒ ब्रह्म॒ चान्न॑ञ्च सुव॒र्गमे॒ते सु॑व॒र्गन्त्रयो॑विꣳशतिश्च}]}%~(९)

%7.3.10.1
अ॒सावा॑दि॒त्यो᳚\-ऽस्मिँल्लो॒क आ॑सी॒त्तं दे॒वाः पृ॒ष्ठैः प॑रि॒गृह्य॑ सुव॒र्गं लो॒कम॑गमय॒न्परै॑र॒वस्ता॒त्पर्य॑गृह्णन्दिवाकी॒र्त्ये॑न सुव॒र्गे लो॒के प्रत्य॑स्थापय॒न्परैः᳚ प॒रस्ता॒त्पर्य॑गृह्णन्पृ॒ष्ठैरु॒पावा॑रोह॒न्थ्स वा अ॒सावा॑दि॒त्यो॑\-ऽमुष्मिँ॑ल्लो॒के परै॑रुभ॒यतः॒ परि॑गृहीतो॒ यत्पृ॒ष्ठानि॒ भव॑न्ति सुव॒र्गमे॒व तैर्लो॒कं यज॑माना यन्ति॒ परै॑र॒वस्ता॒त्परि॑ गृह्णन्ति दिवाकी॒र्त्ये॑न~(२५)

%7.3.10.2
सु॒व॒र्गे लो॒के प्रति॑ तिष्ठन्ति॒ परैः᳚ प॒रस्ता॒त्परि॑ गृह्णन्ति पृ॒ष्ठैरु॒पाव॑रोहन्ति॒ यत्परे॑ प॒रस्ता॒न्न स्युः परा᳚ञ्चः सुव॒र्गाल्लो॒कान्निष्प॑द्येर॒न्॒ यद॒वस्ता॒न्न स्युः प्र॒जा निर्द॑हेयुर॒भितो॑ दिवाकी॒र्त्यं॑ परः॑सामानो भवन्ति सुव॒र्ग ए॒वैनां᳚ लो॒क उ॑भ॒यतः॒ परि॑ गृह्णन्ति॒ यज॑माना॒ वै दि॑वाकी॒र्त्यꣳ॑ संवथ्स॒रः परः॑सामानो॒\-ऽभितो॑ दिवाकी॒र्त्यं॑ परः॑ सामानो भवन्ति संवथ्स॒र ए॒वोभ॒यतः॑~(२६)

%7.3.10.3
प्रति॑ तिष्ठन्ति पृ॒ष्ठं वै दि॑वाकी॒र्त्य॑म्पा॒र्श्वे परः॑सामानो॒\-ऽभितो॑ दिवाकी॒र्त्यं॑ परः॑सामानो भवन्ति॒ तस्मा॑द॒भितः॑ पृ॒ष्ठम्पा॒र्श्वे भूयि॑ष्ठा॒ ग्रहा॑ गृह्यन्ते॒ भूयि॑ष्ठꣳ शस्यते य॒ज्ञस्यै॒व तन्म॑ध्य॒तो ग्र॒न्थं ग्र॑थ्न॒न्त्यवि॑स्रꣳसाय स॒प्त गृ॑ह्यन्ते स॒प्त वै शी॑र्\mbox{}ष॒ण्याः᳚ प्रा॒णाः प्रा॒णाने॒व यज॑मानेषु दधति॒ यत्प॑रा॒चीना॑नि पृ॒ष्ठानि॒ भव॑न्त्य॒मुमे॒व तैर्लो॒कम॒भ्यारो॑हन्ति॒ यदि॒मं लो॒कं न~(२७)

%7.3.10.4
प्र॒त्य॒व॒रोहे॑यु॒रुद्वा॒ माद्ये॑यु॒र्यज॑मानाः॒ प्र वा॑ मीयेर॒न्॒ यत्प्र॑ती॒चीना॑नि पृ॒ष्ठानि॒ भव॑न्ती॒ममे॒व तैर्लो॒कम्प्र॒त्यव॑रोह॒न्त्यथो॑ अ॒स्मिन्ने॒व लो॒के प्रति॑ तिष्ठ॒न्त्यनु॑न्मादा॒येन्द्रो॒ वा अप्र॑तिष्ठित आसी॒थ्स प्र॒जा\-प॑ति॒मुपा॑धाव॒त्तस्मा॑ ए॒तमे॑कविꣳशतिरा॒त्रम्प्राय॑च्छ॒त्तमाह॑र॒त्तेना॑यजत॒ ततो॒ वै स प्रत्य॑तिष्ठ॒द्ये ब॑हुया॒जिनो\-ऽप्र॑तिष्ठिताः~(२८)

%7.3.10.5
स्युस्त ए॑कविꣳशतिरा॒त्रमा॑सीर॒न्द्वाद॑श॒ मासाः॒ पञ्च॒र्तव॒स्त्रय॑ इ॒मे लो॒का अ॒सावा॑दि॒त्य ए॑कवि॒ꣳ॒श ए॒ताव॑न्तो॒ वै दे॑वलो॒कास्तेष्वे॒व य॑थापू॒र्वं प्रति॑ तिष्ठन्त्य॒सावा॑दि॒त्यो न व्य॑रोचत॒ स प्र॒जा\-प॑ति॒मुपा॑धाव॒त्तस्मा॑ ए॒तमे॑कविꣳशति\-रा॒त्रम्प्राय॑च्छ॒त्तमाह॑र॒त्तेना॑यजत॒ ततो॒ वै सो॑\-ऽरोचत॒ य ए॒वं वि॒द्वाꣳ॑स एकविꣳशतिरा॒त्रमास॑ते॒ रोच॑न्त ए॒वैक॑विꣳशतिरा॒त्रो भ॑वति॒ रुग्वा ए॑कवि॒ꣳ॒शो रुच॑मे॒व ग॑च्छ॒न्त्यथो᳚ प्रति॒ष्ठामे॒व प्र॑ति॒ष्ठा ह्ये॑कवि॒ꣳ॒शो॑-\-ऽ तिरा॒त्राव॒भितो॑ भवतो ब्रह्मवर्च॒सस्य॒ परि॑गृहीत्यै॥~(२९)

%7.3.11.0
{\anuvakamend[{गृ॒ह्ण॒न्ति॒ दि॒वा॒की॒र्त्ये॑नै॒वोभ॒यतो॒ नाप्र॑तिष्ठिता॒ आस॑त॒ एक॑विꣳशतिश्च}]}%॥10॥

%7.3.11.1
अ॒र्वाङ्य॒ज्ञः सं क्रा॑मत्व॒मुष्मा॒दधि॒ माम॒भि। ऋषी॑णां॒ यः पु॒रोहि॑तः। निर्दे॑वं॒ निर्वी॑रं कृ॒त्वा विष्क॑न्धं॒ तस्मि॑न् हीयतां॒ यो᳚\-ऽस्मान्द्वेष्टि॑। शरी॑रं यज्ञशम॒लं कुसी॑दं॒ तस्मि᳚न्थ्सीदतु॒ यो᳚\-ऽस्मान्द्वेष्टि॑। यज्ञ॑ य॒ज्ञस्य॒ यत्तेज॒स्तेन॒ सं क्रा॑म॒ माम॒भि। ब्रा॒ह्म॒णानृ॒त्विजो॑ दे॒वान् य॒ज्ञस्य॒ तप॑सा॒ ते सवा॒हमा हु॑वे। इ॒ष्टेन॑ प॒क्वमुप॑~(३०)

%7.3.11.2
ते॒ हु॒वे॒ स॒वा॒हम्। सन्ते॑ वृञ्जे सुकृ॒तꣳ सं प्र॒जां प॒शून्। प्रै॒षान्थ्सा॑मिधे॒नीरा॑घा॒रावाज्य॑भागा॒वाश्रु॑तम्प्र॒त्याश्रु॑त॒मा शृ॑णामि ते। प्र॒या॒जा॒नू॒या॒जान्थ्स्वि॑ष्ट॒कृत॒मिडा॑मा॒शिष॒ आ वृ॑ञ्जे॒ सुवः॑। अ॒ग्निनेन्द्रे॑ण॒ सोमे॑न॒ सर॑स्वत्या॒ विष्णु॑ना दे॒वता॑भिः। या॒ज्या॒नु॒वा॒क्या᳚भ्या॒मुप॑ ते हुवे स्वा॒हं य॒ज्ञमा द॑दे ते॒ वष॑ट्कृतम्। स्तु॒तꣳ श॒स्त्रम्प्र॑तिग॒रं ग्रह॒मिडा॑मा॒शिषः॑~(३१)

%7.3.11.3
आ वृ॑ञ्जे॒ सुवः॑। प॒त्नी॒सं॒या॒जानुप॑ ते हुवे सवा॒हꣳ स॑मिष्टय॒जुरा द॑दे॒ तव॑। प॒शून्थ्सु॒तं पु॑रो॒डाशा॒न्थ्सव॑ना॒न्योत य॒ज्ञम्। दे॒वान्थ्सेन्द्रा॒नुप॑ ते हुवे सवा॒हम॒ग्निमु॑खा॒न्थ्सोम॑वतो॒ ये च॒ विश्वे᳚॥~(३२)

%7.3.12.0
{\anuvakamend[{उप॒ ग्रह॒मिडा॑मा॒शिषो॒ द्वात्रिꣳ॑शच्च}]}%॥11॥

%7.3.12.1
भू॒तम्भव्य॑म्भवि॒ष्यद्वष॒ट्थ्\-स्वाहा॒ नम॒ ऋख्साम॒ यजु॒र्वष॒ट्थ्\-स्वाहा॒ नमो॑ गाय॒त्री त्रि॒ष्टुब्जग॑ती॒ वष॒ट्थ्\-स्वाहा॒ नमः॑ पृथि॒व्य॑न्तरि॑क्षं॒ द्यौर्वष॒ट्थ्\-स्वाहा॒ नमो॒\-ऽग्निर्वा॒युः सूर्यो॒ वष॒ट्थ्\-स्वाहा॒ नमः॑ प्रा॒णो व्या॒नो॑\-ऽपा॒नो वष॒ट्थ्\-स्वाहा॒ नमो\-ऽन्नं॑ कृ॒षिर्वृष्टि॒र्वष॒ट्थ्\-स्वाहा॒ नमः॑ पि॒ता पु॒त्रः पौत्रो॒ वष॒ट्थ्\-स्वाहा॒ नमो॒ भूर्भुवः॒सुव॒र्वष॒ट्थ्\-स्वाहा॒ नमः॑॥~(३३)

%7.3.13.0
{\anuvakamend[{भुव॑श्च॒त्वारि॑ च}]}%॥12॥

%7.3.13.1
आ मे॑ गृ॒हा भ॑वं᳚ त्वा प्र॒जा म॒ आ मा॑ य॒ज्ञो वि॑शतु वी॒र्या॑वान्। आपो॑ दे॒वीर्य॒ज्ञिया॒ मा वि॑शन्तु स॒हस्र॑स्य मा भू॒मा मा प्र हा॑सीत्। आ मे॒ ग्रहो॑ भव॒त्वा पु॑रो॒रुख्स्तु॑तश॒स्त्रे मा वि॑शताꣳ स॒मीची᳚। आ॒दि॒त्या रु॒द्रा वस॑वो मे सद॒स्याः᳚ स॒हस्र॑स्य मा भू॒मा मा प्र हा॑सीत्। आ मा᳚ग्निष्टो॒मो वि॑शतू॒क्थ्य॑श्चातिरा॒त्रो मा वि॑शत्वापिशर्व॒रः। ति॒रोअ॑ह्निया मा॒ सुहु॑ता॒ आ वि॑शन्तु स॒हस्र॑स्य मा भू॒मा मा॒ प्र हा॑सीत्॥~(३४)

%7.3.14.0
{\anuvakamend[{अ॒ग्नि॒ष्टो॒मो वि॑शत्व॒ष्टाद॑श च}]}%॥13॥

%7.3.14.1
अ॒ग्निना॒ तपो\-ऽन्व॑भवद्वा॒चा ब्रह्म॑ म॒णिना॑ रू॒पाणीन्द्रे॑ण दे॒वान् वाते॑न प्रा॒णान्थ्सूर्ये॑ण॒ द्याञ्च॒न्द्रम॑सा॒ नक्ष॑त्राणि य॒मेन॑ पितॄन्राज्ञा॑ मनु॒ष्या᳚न्फ॒लेन॑ नादे॒यान॑जग॒रेण॑ स॒र्पान्व्या॒घ्रेणा॑र॒ण्यान्प॒शूञ्छ्ये॒नेन॑ पत॒त्रिणो॒ वृष्णाश्वा॑नृष॒भेण॒ गा ब॒स्तेना॒जा वृ॒ष्णिनावी᳚र्व्री॒हिणान्ना॑नि॒ यवे॒नौष॑धीर्न्य॒ग्रोधे॑न॒ वन॒स्पती॑नुदु॒म्बरे॒णोर्जं॑ गायत्रि॒या छन्दाꣳ॑सि त्रि॒वृता॒ स्तोमा᳚न्ब्राह्म॒णेन॒ वाचम्᳚॥~(३५)

%7.3.15.0
{\anuvakamend[{ब्रा॒ह्म॒णेनैक॑ञ्च}]}%॥14॥

%7.3.15.1
स्वाहा॒धिमाधी॑ताय॒ स्वाहा॒ स्वाहाधी॑त॒म्मन॑से॒ स्वाहा॒ स्वाहा॒ मनः॑ प्र॒जाप॑तये॒ स्वाहा॒ काय॒ स्वाहा॒ कस्मै॒ स्वाहा॑ कत॒मस्मै॒ स्वाहादि॑त्यै॒ स्वाहादि॑त्यै म॒ह्यै᳚ स्वाहादि॑त्यै सुमृडी॒कायै॒ स्वाहा॒ सर॑स्वत्यै॒ स्वाहा॒ सर॑स्वत्यै बृह॒त्यै᳚ स्वाहा॒ सर॑स्वत्यै पाव॒कायै॒ स्वाहा॑ पू॒ष्णे स्वाहा॑ पू॒ष्णे प्र॑प॒थ्या॑य॒ स्वाहा॑ पू॒ष्णे न॒रन्धि॑षाय॒ स्वाहा॒ त्वष्ट्रे॒ स्वाहा॒ त्वष्ट्रे॑ तु॒रीपा॑य॒ स्वाहा॒ त्वष्ट्रे॑ पुरु॒रूपा॑य॒ स्वाहा॒ विष्ण॑वे॒ स्वाहा॒ विष्ण॑वे निखुर्य॒पाय॒ स्वाहा॒ विष्ण॑वे निभूय॒पाय॒ स्वाहा॒ सर्वस्मै॒ स्वाहा᳚॥~(३६)

%7.3.16.0
{\anuvakamend[{पु॒रु॒रूपा॑य॒ स्वाहा॒ दश॑ च}]}%॥15॥

%7.3.16.1
द॒द्भ्यः स्वाहा॒ हनू᳚भ्या॒ꣴ॒ स्वाहोष्ठा᳚भ्या॒ꣴ॒ स्वाहा॒ मुखा॑य॒ स्वाहा॒ नासि॑काभ्या॒ꣴ॒ स्वाहा॒क्षीभ्या॒ꣴ॒ स्वाहा॒ कर्णा᳚भ्या॒ꣴ॒ स्वाहा॑ पा॒र इ॒क्षवो॑\-ऽवा॒र्ये᳚भ्यः॒ पक्ष्म॑भ्यः॒ स्वाहा॑वा॒र इ॒क्षवः॑ पा॒र्ये᳚भ्यः॒ पक्ष्म॑भ्यः॒ स्वाहा॑ शी॒र्॒\mbox{}ष्णे स्वाहा᳚ भ्रू॒भ्याꣴ स्वाहा॑ ल॒लाटा॑य॒ स्वाहा॑ मू॒र्ध्ने स्वाहा॑ म॒स्तिष्का॑य॒ स्वाहा॒ केशे᳚भ्यः॒ स्वाहा॒ वहा॑य॒ स्वाहा᳚ ग्री॒वाभ्यः॒ स्वाहा᳚ स्क॒न्धेभ्यः॒ स्वाहा॒ कीक॑साभ्यः॒ स्वाहा॑ पृ॒ष्टीभ्यः॒ स्वाहा॑ पाज॒स्या॑य॒ स्वाहा॑ पा॒र्श्वाभ्या॒ꣴ॒ स्वाहा᳚~(३७)

%7.3.16.2
अꣳसा᳚भ्या॒ꣴ॒ स्वाहा॑ दो॒षभ्या॒ꣴ॒ स्वाहा॑ बा॒हुभ्या॒ꣴ॒ स्वाहा॒ जङ्घा᳚भ्या॒ꣴ॒ स्वाहा॒ श्रोणी᳚भ्या॒ꣴ॒ स्वाहो॒रुभ्या॒ꣴ॒ स्वाहा᳚ष्ठी॒वद्भ्या॒ꣴ॒ स्वाहा॒ जङ्घा᳚भ्या॒ꣴ॒ स्वाहा॑ भ॒सदे॒ स्वाहा॑ शिख॒ण्डेभ्यः॒ स्वाहा॑ वाल॒धाना॑य॒ स्वाहा॒ण्डाभ्या॒ꣴ॒ स्वाहा॒ शेपा॑य॒ स्वाहा॒ रेत॑से॒ स्वाहा᳚ प्र॒जाभ्यः॒ स्वाहा᳚ प्र॒जन॑नाय॒ स्वाहा॑ प॒द्भ्यः स्वाहा॑ श॒फेभ्यः॒ स्वाहा॒ लोम॑भ्यः॒ स्वाहा᳚ त्व॒चे स्वाहा॒ लोहि॑ताय॒ स्वाहा॑ मा॒ꣳ॒साय॒ स्वाहा॒ स्नाव॑भ्यः॒ स्वाहा॒स्थभ्यः॒ स्वाहा॑ म॒ज्जभ्यः॒ स्वाहाङ्गे᳚भ्यः॒ स्वाहा॒त्मने॒ स्वाहा॒ सर्व॑स्मै॒ स्वाहा᳚॥~(३८)

%7.3.17.0
{\anuvakamend[{पा॒र्श्वाभ्या॒ꣴ॒ स्वाहा॑ म॒ज्जभ्यः॒ स्वाहा॒ षट्च॑}]}%॥16॥

%7.3.17.1
अ॒ञ्ज्ये॒ताय॒ स्वाहा᳚ञ्जिस॒क्थाय॒ स्वाहा॑ शिति॒पदे॒ स्वाहा॒ शिति॑ककुदे॒ स्वाहा॑ शिति॒रन्ध्रा॑य॒ स्वाहा॑ शितिपृ॒ष्ठाय॒ स्वाहा॑ शि॒त्यꣳसा॑य॒ स्वाहा॑ पुष्प॒कर्णा॑य॒ स्वाहा॑ शि॒त्योष्ठा॑य॒ स्वाहा॑ शिति॒भ्रवे॒ स्वाहा॒ शिति॑भसदे॒ स्वाहा᳚ श्वे॒तानू॑काशाय॒ स्वाहा॒ञ्जये॒ स्वाहा॑ ल॒लामा॑य॒ स्वाहासि॑तज्ञवे॒ स्वाहा॑ कृष्णै॒ताय॒ स्वाहा॑ रोहितै॒ताय॒ स्वाहा॑रुणै॒ताय॒ स्वाहे॒दृशा॑य॒ स्वाहा॑ की॒दृशा॑य॒ स्वाहा॑ ता॒दृशा॑य॒ स्वाहा॑ स॒दृशा॑य॒ स्वाहा॒ विस॑दृशाय॒ स्वाहा॒ सुस॑दृ॒शाय॒ स्वाहा॑ रू॒पाय॒ स्वाहा॒ सर्व॑स्मै॒ स्वाहा᳚॥~(३९)

%7.3.18.0
{\anuvakamend[{रू॒पाय॒ स्वाहा॒ द्वे च॑}]}%॥17॥

%7.3.18.1
कृ॒ष्णाय॒ स्वाहा᳚ श्वे॒ताय॒ स्वाहा॑ पि॒शङ्गा॑य॒ स्वाहा॑ सा॒रङ्गा॑य॒ स्वाहा॑रु॒णाय॒ स्वाहा॑ गौ॒राय॒ स्वाहा॑ ब॒भ्रवे॒ स्वाहा॑ नकु॒लाय॒ स्वाहा॒ रोहि॑ताय॒ स्वाहा॒ शोणा॑य॒ स्वाहा᳚ श्या॒वाय॒ स्वाहा᳚ श्या॒माय॒ स्वाहा॑ पाक॒लाय॒ स्वाहा॑ सुरू॒पाय॒ स्वाहानु॑रूपाय॒ स्वाहा॒ विरू॑पाय॒ स्वाहा॒ सरू॑पाय॒ स्वाहा॒ प्रति॑रूपाय॒ स्वाहा॑ श॒बला॑य॒ स्वाहा॑ कम॒लाय॒ स्वाहा॒ पृश्ञ॑ये॒ स्वाहा॑ पृश्ञिस॒क्थाय॒ स्वाहा॒ सर्व॑स्मै॒ स्वाहा᳚॥~(४०)

%7.3.19.0
{\anuvakamend[{कृ॒ष्णाय॒ षट्च॑त्वारिꣳशत्}]}%॥18॥

%7.3.19.1
ओष॑धीभ्यः॒ स्वाहा॒ मूले᳚भ्यः॒ स्वाहा॒ तूले᳚भ्यः॒ स्वाहा॒ काण्डे᳚भ्यः॒ स्वाहा॒ वल्\mbox{}शे᳚भ्यः॒ स्वाहा॒ पुष्पे᳚भ्यः॒ स्वाहा॒ फले᳚भ्यः॒ स्वाहा॑ गृही॒तेभ्यः॒ स्वाहागृ॑हीतेभ्यः॒ स्वाहाव॑पन्नेभ्यः॒ स्वाहा॒ शया॑नेभ्यः॒ स्वाहा॒ सर्व॑स्मै॒ स्वाहा᳚॥~(४१)

%7.3.20.0
{\anuvakamend[{ओष॑धीभ्य॒श्चतु॑र्विꣳशतिः}]}%॥19॥

%7.3.20.1
वन॒स्पति॑भ्यः॒ स्वाहा॒ मूले᳚भ्यः॒ स्वाहा॒ तूले᳚भ्यः॒ स्वाहा॒ स्कन्धो᳚भ्यः॒ स्वाहा॒ शाखा᳚भ्यः॒ स्वाहा॑ प॒र्णेभ्यः॒ स्वाहा॒ पुष्पे᳚भ्यः॒ स्वाहा॒ फले᳚भ्यः॒ स्वाहा॑ गृही॒तेभ्यः॒ स्वाहागृ॑हीतेभ्यः॒ स्वाहाव॑पन्नेभ्यः॒ स्वाहा॒ शया॑नेभ्यः॒ स्वाहा॑ शि॒ष्टाय॒ स्वाहाति॑शिष्टाय॒ स्वाहा॒ परि॑शिष्टाय॒ स्वाहा॒ सꣳशि॑ष्टाय॒ स्वाहोच्छि॑ष्टाय॒ स्वाहा॑ रि॒क्ताय॒ स्वाहारि॑क्ताय॒ स्वाहा॒ प्ररि॑क्ताय॒ स्वाहा॒ सꣳरि॑क्ताय॒ स्वाहोद्रि॑क्ताय॒ स्वाहा॒ सर्व॑स्मै॒ स्वाहा᳚॥~(४२)

%7.4.0.0
{\anuvakamend[{वन॒स्पति॑भ्यः॒ स्कन्धो᳚भ्यः शि॒ष्टाय॑ रि॒क्ताय॒ षट्च॑त्वारिꣳशत्}]}%॥20॥

{\prashnaend[{प्र॒जवं॑ प्र॒जा\-प॑ति॒र्यद॑छन्दो॒मन्ते॑ हुवे सवा॒हमोष॑धीभ्यो॒ द्विच॑त्वारिꣳशत्॥42॥ प्र॒जव॒ꣳ॒ सर्व॑स्मै॒ स्वाहा᳚॥}]}
%%% END PRASHNA

\sect{चतुर्थः प्रश्नः}\setcounter{anuvakam}{0}
\dnsub{तैत्तिरीयसंहितायां सप्तमकाण्डे चतुर्थः प्रश्नः}
%7.4.1.0
%7.4.1.1
बृह॒स्पति॑रकामयत॒ श्रन्मे॑ दे॒वा दधी॑र॒न्गच्छे॑यं पुरो॒धामिति॒ स ए॒तं च॑तुर्विꣳशतिरा॒त्रम॑पश्य॒त्तमाह॑र॒त्तेना॑यजत॒ ततो॒ वै तस्मै॒ श्रद्दे॒वा अद॑ध॒ताग॑च्छत्पुरो॒धां य ए॒वं वि॒द्वाꣳस॑श्चतुर्विꣳशतिरा॒त्रमास॑ते॒ श्रदे᳚भ्यो मनु॒ष्या॑ दधते॒ गच्छ॑न्ति पुरो॒धां ज्योति॒र्गौरायु॒रिति॑ त्र्य॒हा भ॑वन्ती॒यं वाव ज्योति॑र॒न्तरि॑क्षं॒ गौर॒सावायुः॑~(१)

%7.4.1.2
इ॒माने॒व लो॒कान॒भ्यारो॑हन्त्यभिपू॒र्वं त्र्य॒हा भ॑वन्त्यभिपू॒र्वमे॒व सु॑व॒र्गं लो॒कम॒भ्यारो॑ह॒न्त्यस॑त्त्रं॒ वा ए॒तद्यद॑छन्दो॒मं यच्छ॑न्दो॒मा भव॑न्ति॒ तेन॑ स॒त्रं दे॒वता॑ ए॒व पृ॒ष्ठैरव॑ रुन्धते प॒शूञ्छ॑न्दो॒मैरोजो॒ वै वी॒र्यं॑ पृ॒ष्ठानि॑ प॒शव॑श्छन्दो॒मा ओज॑स्ये॒व वी॒र्ये॑ प॒शुषु॒ प्रति॑ तिष्ठन्ति बृहद्रथन्त॒रा\-भ्यां᳚ यन्ती॒यं वाव र॑थन्त॒रम॒सौ बृ॒हदा॒भ्यामे॒व~(२)

%7.4.1.3
य॒न्त्यथो॑ अ॒नयो॑रे॒व प्रति॑ तिष्ठन्त्ये॒ते वै य॒ज्ञस्या᳚ञ्ज॒साय॑नी स्रु॒ती ताभ्या॑मे॒व सु॑व॒र्गं लो॒कं य॑न्ति चतुर्विꣳशतिरा॒त्रो भ॑वति॒ चतु॑र्विꣳशतिरर्धमा॒साः सं॑वथ्स॒रः सं॑वथ्स॒रः सु॑व॒र्गो लो॒कः सं॑वथ्स॒र ए॒व सु॑व॒र्गे लो॒के प्रति॑ तिष्ठ॒न्त्यथो॒ चतु॑र्विꣳशत्यक्षरा गाय॒त्री गा॑य॒त्री ब्र॑ह्मवर्च॒सङ्गा॑यत्रि॒यैव ब्र॑ह्मवर्च॒समव॑ रुन्धते\-ऽतिरा॒त्राव॒भितो॑ भवतो ब्रह्मवर्च॒सस्य॒ परि॑गृहीत्यै॥~(३)

%7.4.2.0
{\anuvakamend[{अ॒सावायु॑रा॒भ्यामे॒व पञ्च॑चत्वारिꣳशच्च}]}%~(१)

%7.4.2.1
यथा॒ वै म॑नु॒ष्या॑ ए॒वं दे॒वा अग्र॑ आस॒न्ते॑\-ऽकामय॒न्ताव॑र्तिम्पा॒प्मान॑म्मृ॒त्युम॑प॒हत्य॒ दैवीꣳ॑ स॒ꣳ॒सदं॑ गच्छे॒मेति॒ त ए॒तं च॑तुर्विꣳशतिरा॒त्रम॑पश्य॒न्तमाह॑र॒न्तेना॑यजन्त॒ ततो॒ वै ते\-ऽव॑र्तिम्पा॒प्मान॑म्मृ॒त्युम॑प॒हत्य॒ दैवीꣳ॑ स॒ꣳ॒सद॑मगच्छ॒न्॒ य ए॒वं वि॒द्वाꣳ॑सश्चतुर्विꣳशतिरा॒त्रमास॒ते\-ऽव॑र्तिमे॒व पा॒प्मान॑मप॒हत्य॒ श्रियं॑ गच्छन्ति॒ श्रीर्\mbox{}हि म॑नु॒ष्य॑स्य~(४)

%7.4.2.2
दैवी॑ स॒ꣳ॒सज्ज्योति॑रतिरा॒त्रो भ॑वति सुव॒र्गस्य॑ लो॒कस्यानु॑ख्यात्यै॒ पृष्ठ्यः॑ षड॒हो भ॑वति॒ षड्वा ऋ॒तवः॑ संवथ्स॒रस्तम्मासा॑ अर्धमा॒सा ऋ॒तवः॑ प्र॒विश्य॒ दैवीꣳ॑ स॒ꣳ॒सद॑मगच्छ॒न्॒ य ए॒वं वि॒द्वाꣳ॑सश्चतुर्विꣳशतिरा॒त्रमास॑ते संवथ्स॒रमे॒व प्र॒विश्य॒ वस्य॑सीꣳ स॒ꣳ॒सदं॑ गच्छन्ति॒ त्रय॑स्त्रयस्त्रि॒ꣳ॒शा अ॒वस्ता᳚द्भवन्ति॒ त्रय॑स्त्रयस्त्रि॒ꣳ॒शाः प॒रस्ता᳚त्त्रयस्त्रि॒ꣳ॒शैरे॒वोभ॒यतो\-ऽव॑र्तिम्पा॒प्मान॑मप॒हत्य॒ दैवीꣳ॑ स॒ꣳ॒सद॑म्मध्य॒तः~(५)

%7.4.2.3
ग॒च्छ॒न्ति॒ पृ॒ष्ठानि॒ हि दैवी॑ स॒ꣳ॒सज्जा॒मि वा ए॒तत्कु॑र्वन्ति॒ यत्त्रय॑स्त्रयस्त्रि॒ꣳ॒शा अ॒न्वञ्चो॒ मध्ये\-ऽनि॑रुक्तो भवति॒ तेनाजा᳚म्यू॒र्ध्वानि॑ पृ॒ष्ठानि॑ भवन्त्यू॒र्ध्वाश्छ॑न्दो॒मा उ॒भाभ्याꣳ॑ रू॒पाभ्याꣳ॑ सुव॒र्गं लो॒कं य॒न्त्यस॑त्त्रं॒ वा ए॒तद्यद॑छन्दो॒मं यच्छ॑न्दो॒मा भव॑न्ति॒ तेन॑ स॒त्रं दे॒वता॑ ए॒व पृ॒त्ष्ठैरव॑ रुन्धते प॒शूञ्छ॑न्दो॒मैरोजो॒ वै वी॒र्यं॑ पृ॒ष्ठानि॑ प॒शवः॑~(६)

%7.4.2.4
छ॒न्दो॒मा ओज॑स्ये॒व वी॒र्ये॑ प॒शुषु॒ प्रति॑ तिष्ठन्ति॒ त्रय॑स्त्रयस्त्रि॒ꣳ॒शा अ॒वस्ता᳚द्भवन्ति॒ त्रय॑स्त्रयस्त्रि॒ꣳ॒शाः प॒रस्ता॒न्मध्ये॑ पृ॒ष्ठान्युरो॒ वै त्र॑यस्त्रि॒ꣳ॒शा आ॒त्मा पृ॒ष्ठान्या॒त्मन॑ ए॒व तद्यज॑मानाः॒ शर्म॑ नह्य॒न्ते\-ऽना᳚र्त्यै बृहद्रथन्त॒रा\-भ्यां᳚ यन्ती॒यं वाव र॑थन्त॒रम॒सौ बृ॒हदा॒भ्यामे॒व य॒न्त्यथो॑ अ॒नयो॑रे॒व प्रति॑ तिष्ठन्त्ये॒ते वै य॒ज्ञस्या᳚ञ्ज॒साय॑नी स्रु॒ती ताभ्या॑मे॒व~(७)

%7.4.2.5
सु॒व॒र्गं लो॒कं य॑न्ति॒ परा᳚ञ्चो॒ वा ए॒ते सु॑व॒र्गं लो॒कम॒भ्यारो॑हन्ति॒ ये प॑रा॒चीना॑नि पृ॒ष्ठान्यु॑प॒यन्ति॑ प्र॒त्यङ्क्ष॑ड॒हो भ॑वति प्र॒त्यव॑रूढ्या॒ अथो॒ प्रति॑ष्ठित्या उ॒भयो᳚र्लो॒कयोर्॑\mbox{}ऋ॒द्ध्वोत्ति॑ष्ठन्ति त्रि॒वृतो\-ऽधि॑ त्रि॒वृत॒मुप॑ यन्ति॒ स्तोमा॑ना॒ꣳ॒ सम्प॑त्त्यै प्रभ॒वाय॒ ज्योति॑रग्निष्टो॒मो भ॑वत्य॒यं वाव स क्षयो॒\-ऽस्मादे॒व तेन॒ क्षया॒न्न य॑न्ति चतुर्विꣳशतिरा॒त्रो भ॑वति॒ चतु॑र्विꣳशतिरर्धमा॒साः सं॑वथ्स॒रः सं॑वथ्स॒रः सु॑व॒र्गो लो॒कः सं॑वथ्स॒र ए॒व सु॑व॒र्गे लो॒के प्रति॑ तिष्ठ॒न्त्यथो॒ चतु॑र्विꣳशत्यक्षरा गाय॒त्री गा॑य॒त्री ब्र॑ह्मवर्च॒सङ्गा॑यत्रि॒यैव ब्र॑ह्मवर्च॒समव॑ रुन्धते\-ऽतिरा॒त्राव॒भितो॑ भवतो ब्रह्मवर्च॒सस्य॒ परि॑गृहीत्यै॥~(८)

%7.4.3.0
{\anuvakamend[{म॒नु॒ष्य॑स्य मध्य॒तः प॒शव॒स्ताभ्या॑मे॒व सं॑वथ्स॒रश्चतु॑र्विꣳशतिश्च}]}%~(२)

%7.4.3.1
ऋ॒क्षा वा इ॒यम॑लो॒मका॑सी॒थ्साका॑मय॒तौष॑धीभि॒र्वन॒स्पति॑भिः॒ प्र जा॑ये॒येति॒ सैतास्त्रि॒ꣳ॒शत॒ꣳ॒ रात्री॑रपश्य॒त्ततो॒ वा इ॒यमोष॑धीभि॒र्वन॒स्पति॑भिः॒ प्राजा॑यत॒ ये प्र॒जाका॑माः प॒शुका॑माः॒ स्युस्त ए॒ता आ॑सीर॒न्प्रैव जा॑यन्ते प्र॒जया॑ प॒शुभि॑रि॒यं वा अ॑क्षुध्य॒थ्सैतां वि॒राज॑मपश्य॒त्तामा॒त्मन्धि॒त्वान्नाद्य॒मवा॑रु॒न्धौष॑धीः~(९)

%7.4.3.2
वन॒स्पती᳚न्प्र॒जां प॒शून्तेना॑वर्धत॒ सा जे॒मान॑म्महि॒मान॑मगच्छ॒द्य ए॒वं वि॒द्वाꣳस॑ ए॒ता आस॑ते वि॒राज॑मे॒वात्मन्धि॒त्वा\-ऽन्नाद्य॒मव॑ रुन्धते॒ वर्ध॑न्ते प्र॒जया॑ प॒शुभि॑र्जे॒मान॑म्महि॒मानं॑ गच्छन्ति॒ ज्योति॑रतिरा॒त्रो भ॑वति सुव॒र्गस्य॑ लो॒कस्यानु॑\-ख्यात्यै॒ पृष्ठ्यः॑ षड॒हो भ॑वति॒ षड्वा ऋ॒तवः॒ षट्पृ॒ष्ठानि॑ पृ॒ष्ठैरे॒वर्तून॒न्वारो॑हन्त्यृ॒तुभिः॑ संवथ्स॒रन्ते सं॑वथ्स॒र ए॒व~(१०)

%7.4.3.3
प्रति॑ तिष्ठन्ति त्रयस्त्रि॒ꣳ॒शात्त्र॑यस्त्रि॒ꣳ॒शमुप॑ यन्ति य॒ज्ञस्य॒ सन्त॑त्या॒ अथो᳚ प्र॒जा\-प॑ति॒र्वै त्र॑यस्त्रि॒ꣳ॒शः प्र॒जा\-प॑तिमे॒वा र॑भन्ते॒ प्रति॑ष्ठित्यै त्रिण॒वो भ॑वति॒ विजि॑त्या एकवि॒ꣳ॒शो भ॑वति॒ प्रति॑ष्ठित्या॒ अथो॒ रुच॑मे॒वात्मन्द॑धते त्रि॒वृद॑ग्नि॒ष्टुद्भ॑वति पा॒प्मान॑मे॒व तेन॒ निर्द॑ह॒न्ते\-ऽथो॒ तेजो॒ वै त्रि॒वृत्तेज॑ ए॒वात्मन्द॑धते पञ्चद॒श इ॑न्द्रस्तो॒मो भ॑वतीन्द्रि॒यमे॒वाव॑~(११)

%7.4.3.4
रु॒न्ध॒ते॒ स॒प्त॒द॒शो भ॑वत्य॒न्नाद्य॒स्याव॑रुद्ध्या॒ अथो॒ प्रैव तेन॑ जायन्त एकवि॒ꣳ॒शो भ॑वति॒ प्रति॑ष्ठित्या॒ अथो॒ रुच॑मे॒वात्मन्द॑धते चतुर्वि॒ꣳ॒शो भ॑वति॒ चतु॑र्विꣳशतिरर्धमा॒साः सं॑वथ्स॒रः सं॑वथ्स॒रः सु॑व॒र्गो लो॒कः सं॑वथ्स॒र ए॒व सु॑व॒र्गे लो॒के प्रति॑ तिष्ठ॒न्त्यथो॑ ए॒ष वै वि॑षू॒वान् वि॑षू॒वन्तो॑ भवन्ति॒ य ए॒वं वि॒द्वाꣳस॑ ए॒ता आस॑ते चतुर्वि॒ꣳ॒शात्पृ॒ष्ठान्युप॑ यन्ति संवथ्स॒र ए॒व प्र॑ति॒ष्ठाय॑~(१२)

%7.4.3.5
दे॒वता॑ अ॒भ्यारो॑हन्ति त्रयस्त्रि॒ꣳ॒शात्त्र॑यस्त्रि॒ꣳ॒शमुप॑ यन्ति॒ त्रय॑स्त्रिꣳश॒द्वै दे॒वता॑ दे॒वता᳚स्वे॒व प्रति॑ तिष्ठन्ति त्रिण॒वो भ॑वती॒मे वै लो॒कास्त्रि॑ण॒व ए॒ष्वे॑व लो॒केषु॒ प्रति॑ तिष्ठन्ति॒ द्वावे॑कवि॒ꣳ॒शौ भ॑वतः॒ प्रति॑ष्ठित्या॒ अथो॒ रुच॑मे॒वात्मन्द॑धते ब॒हवः॑ षोड॒शिनो॑ भवन्ति॒ तस्मा᳚द्ब॒हवः॑ प्र॒जासु॒ वृषा॑णो॒ यदे॒ते स्तोमा॒ व्यति॑षक्ता॒ भव॑न्ति॒ तस्मा॑दि॒यमोष॑धीभि॒र्वन॒स्पति॑भि॒र्व्यति॑षक्ता~(१३)

%7.4.3.6
व्यति॑षज्यन्ते प्र॒जया॑ प॒शुभि॒र्य ए॒वं वि॒द्वाꣳस॑ ए॒ता आस॒ते\-ऽकॢ॑प्ता॒ वा ए॒ते सु॑व॒र्गं लो॒कं य॑न्त्युच्चाव॒चान् हि स्तोमा॑नुप॒यन्ति॒ यदे॒त ऊ॒र्ध्वाः कॢ॒प्ताः स्तोमा॒ भव॑न्ति कॢ॒प्ता ए॒व सु॑व॒र्गं लो॒कं य॑न्त्यु॒भयो॑रेभ्यो लो॒कयोः᳚ कल्पते त्रि॒ꣳ॒शदे॒तास्त्रि॒ꣳ॒शद॑क्षरा वि॒राडन्नं॑ वि॒राड्वि॒राजै॒वान्नाद्य॒मव॑ रुन्धते\-ऽतिरा॒त्राव॒भितो॑ भवतो॒\-ऽन्नाद्य॑स्य॒ परि॑गृहीत्यै॥~(१४)

%7.4.4.0
{\anuvakamend[{ओष॑धीः संवथ्स॒र ए॒वाव॑ प्रति॒ष्ठाय॒ व्यति॑ष॒क्तैका॒न्नप॑ञ्चा॒शच्च॑}]}%~(३)

%7.4.4.1
प्र॒जा\-प॑तिः सुव॒र्गं लो॒कमै॒त्तं दे॒वा येन॑येन॒ छन्द॒सानु॒ प्रायु॑ञ्जत॒ तेन॒ नाप्नु॑व॒न्त ए॒ता द्वात्रिꣳ॑शत॒ꣳ॒ रात्री॑रपश्य॒न् द्वात्रिꣳ॑शदक्षरानु॒ष्टुगानु॑ष्टुभः प्र॒जा\-प॑तिः॒ स्वेनै॒व छन्द॑सा प्र॒जा\-प॑तिमा॒प्त्वाभ्या॒रुह्य॑ सुव॒र्गं लो॒कमा॑य॒न्॒ य ए॒वं वि॒द्वाꣳस॑ ए॒ता आस॑ते॒ द्वात्रिꣳ॑शदे॒ता द्वात्रिꣳ॑शदक्षरानु॒ष्टुगानु॑ष्टुभः प्र॒जा\-प॑तिः॒ स्वेनै॒व छन्द॑सा प्र॒जा\-प॑तिमा॒प्त्वा श्रियं॑ गच्छन्ति~(१५)

%7.4.4.2
श्रीर्\mbox{}हि म॑नु॒ष्य॑स्य सुव॒र्गो लो॒को द्वात्रिꣳ॑शदे॒ता द्वात्रिꣳ॑शदक्षरानु॒ष्टुग्वाग॑नु॒ष्टुफ्सर्वा॑मे॒व वाच॑माप्नुवन्ति॒ सर्वे॑ वा॒चो व॑दि॒तारो॑ भवन्ति॒ सर्वे॒ हि श्रियं॒ गच्छ॑न्ति॒ ज्योति॒र्गौरायु॒रिति॑ त्र्य॒हा भ॑वन्ती॒यं वाव ज्योति॑र॒न्तरि॑क्षं॒ गौर॒सावायु॑\-रि॒माने॒व लो॒कान॒भ्यारो॑हन्त्यभिपू॒र्वं त्र्य॒हा भ॑वन्त्यभिपू॒र्वमे॒व सु॑व॒र्गं लो॒कम॒भ्यारो॑हन्ति बृहद्रथन्त॒रा\-भ्यां᳚ यन्ति~(१६)

%7.4.4.3
इ॒यं वाव र॑थन्त॒रम॒सौ बृ॒हदा॒भ्यामे॒व य॒न्त्यथो॑ अ॒नयो॑रे॒व प्रति॑ तिष्ठन्त्ये॒ते वै य॒ज्ञस्या᳚ञ्ज॒साय॑नी स्रु॒ती ताभ्या॑मे॒व सु॑व॒र्गं लो॒कं य॑न्ति॒ परा᳚ञ्चो॒ वा ए॒ते सु॑व॒र्गं लो॒कम॒भ्यारो॑हन्ति॒ ये परा॑चस्त्र्य॒हानु॑प॒यन्ति॑ प्र॒त्यङ्त्र्य॒हो भ॑वति॒ प्र॒त्यव॑रूढ्या॒ अथो॒ प्रति॑ष्ठित्या उ॒भयो᳚र्लो॒कयोर्॑\mbox{}॑ऋद्ध्वोत्ति॑ष्ठन्ति॒ द्वात्रिꣳ॑शदे॒तास्तासां॒ यास्त्रि॒ꣳ॒शत्त्रि॒ꣳ॒शद॑क्षरा वि॒राडन्नं॑ वि॒राड्वि॒राजै॒वान्नाद्य॒मव॑ रुन्धते॒ ये द्वे अ॑होरा॒त्रे ए॒व ते उ॒भाभ्याꣳ॑ रू॒पाभ्याꣳ॑ सुव॒र्गं लो॒कं य॑न्त्यतिरा॒त्राव॒भितो॑ भवतः॒ परि॑गृहीत्यै॥~(१७)

%7.4.5.0
{\anuvakamend[{ग॒च्छ॒न्ति॒ य॒न्ति॒ त्रि॒ꣳ॒शद॑क्षरा॒ द्वाविꣳ॑शतिश्च}]}%~(४)

%7.4.5.1
द्वे वाव दे॑वस॒त्रे द्वा॑दशा॒हश्चै॒व त्र॑यस्त्रिꣳशद॒हश्च॒ य ए॒वं वि॒द्वाꣳस॑स्त्रयस्त्रिꣳशद॒हमास॑ते सा॒क्षादे॒व दे॒वता॑ अ॒भ्यारो॑हन्ति॒ यथा॒ खलु॒ वै श्रेया॑न॒भ्यारू॑ढः का॒मय॑ते॒ तथा॑ करोति॒ यद्य॑व॒विध्य॑ति॒ पापी॑यान्भवति॒ यदि॒ नाव॒विध्य॑ति स॒दृङ्य ए॒वं वि॒द्वाꣳस॑स्त्रयस्त्रिꣳशद॒हमास॑ते॒ वि पा॒प्मना॒ भ्रातृ॑व्ये॒णा व॑र्तन्ते\-ऽह॒र्भाजो॒ वा ए॒ता दे॒वा अग्र॒ आह॑रन्न्~(१८)

%7.4.5.2
अह॒रेको\-ऽभ॑ज॒ताह॒रेक॒स्ताभि॒र्वै ते प्र॒बाहु॑गार्ध्नुव॒न्॒ य ए॒वं वि॒द्वाꣳस॑स्त्रयस्त्रिꣳशद॒हमास॑ते॒ सर्व॑ ए॒व प्र॒बाहु॑गृध्नुवन्ति॒ सर्वे॒ ग्राम॑णीयं॒ प्राप्नु॑वन्ति पञ्चा॒हा भ॑वन्ति॒ पञ्च॒ वा ऋ॒तवः॑ संवथ्स॒र ऋ॒तुष्वे॒व सं॑वथ्स॒रे प्रति॑ तिष्ठ॒न्त्यथो॒ पञ्चा᳚क्षरा प॒ङ्क्तिः पाङ्क्तो॑ य॒ज्ञ य॒ज्ञमे॒वाव॑ रुन्धते॒ त्रीण्या᳚श्वि॒नानि॑ भवन्ति॒ त्रय॑ इ॒मे लो॒का ए॒षु~(१९)

%7.4.5.3
ए॒व लो॒केषु॒ प्रति॑ तिष्ठ॒न्त्यथो॒ त्रीणि॒ वै य॒ज्ञस्ये᳚न्द्रि॒याणि॒ तान्ये॒वाव॑ रुन्धते विश्व॒जिद्भ॑वत्य॒न्नाद्य॒स्याव॑रुद्ध्यै॒ सर्व॑पृष्ठो भवति॒ सर्व॑स्या॒भिजि॑त्यै॒ वाग्वै द्वा॑दशा॒हो यत्पु॒रस्ता᳚द्द्वादशा॒हमु॑पे॒युरना᳚प्तां॒ वाच॒मुपे॑युरुप॒दासु॑कैषां॒ वाख्स्या॑दु॒परि॑ष्टाद्द्वादशा॒हमुप॑ यन्त्या॒प्तामे॒व वाच॒मुप॑ यन्ति॒ तस्मा॑दु॒परि॑ष्टाद्वा॒चा व॑दामो\-ऽवान्त॒रम्~(२०)

%7.4.5.4
वै द॑शरा॒त्रेण॑ प्र॒जा\-प॑तिः प्र॒जा अ॑सृजत॒ यद्द॑शरा॒त्रो भव॑ति प्र॒जा ए॒व तद्यज॑मानाः सृजन्त ए॒ताꣳ ह॒ वा उ॑द॒ङ्कः शौ᳚ल्बाय॒नः स॒त्रस्यर्द्धि॑मुवाच॒ यद्द॑शरा॒त्रो यद्द॑शरा॒त्रो भव॑ति स॒त्रस्यर्द्ध्या॒ अथो॒ यदे॒व पूर्वे॒ष्वहः॑सु॒ विलो॑म क्रि॒यते॒ तस्यै॒वैषा शान्ति॑र्द्व्यनी॒का वा ए॒ता रात्र॑यो॒ यज॑माना विश्व॒जिथ्स॒हाति॑रा॒त्रेण॒ पूर्वाः॒ षोड॑श स॒हाति॑रा॒त्रेणोत्त॑राः॒ षोड॑श॒ य ए॒वं वि॒द्वाꣳस॑स्त्रयस्त्रिꣳशद॒हमास॑त॒ ऐ॑षां᳚ द्व्यनी॒का प्र॒जा जा॑यते\-ऽतिरा॒त्राव॒भितो॑ भवतः॒ परि॑गृहीत्यै॥~(२१)

%7.4.6.0
{\anuvakamend[{अ॒ह॒र॒न्ने॒ष्व॑वान्त॒रꣳ षोड॑श स॒ह स॒प्तद॑श च}]}%~(५)

%7.4.6.1
आ॒दि॒त्या अ॑कामयन्त सुव॒र्गं लो॒कमि॑या॒मेति॒ ते सु॑व॒र्गं लो॒कं न प्राजा॑न॒न्न सु॑व॒र्गं लो॒कमा॑य॒न्त ए॒तꣳ ष॑ट्त्रिꣳशद्रा॒त्रम॑पश्य॒न्तमाह॑र॒न्तेना॑यजन्त॒ ततो॒ वै ते सु॑व॒र्गं लो॒कम्प्राजा॑नन्थ्सुव॒र्गं लो॒कमा॑य॒न्॒ य ए॒वं वि॒द्वाꣳसः॑ षट्त्रिꣳशद्रा॒त्रमास॑ते सुव॒र्गमे॒व लो॒कम्प्र जा॑नन्ति सुव॒र्गं लो॒कं य॑न्ति॒ ज्योति॑रतिरा॒त्रः~(२२)

%7.4.6.2
भ॒व॒ति॒ ज्योति॑रे॒व पु॒रस्ता᳚द्दधते सुव॒र्गस्य॑ लो॒कस्यानु॑ख्यात्यै षड॒हा भ॑वन्ति॒ षड्वा ऋ॒तव॑ ऋ॒तुष्वे॒व प्रति॑ तिष्ठन्ति च॒त्वारो॑ भवन्ति॒ चत॑स्रो॒ दिशो॑ दि॒क्ष्वे॑व प्रति॑ तिष्ठ॒न्त्यस॑त्त्रं॒ वा ए॒तद्यद॑छन्दो॒मं यच्छ॑न्दो॒मा भव॑न्ति॒ तेन॑ स॒त्रं दे॒वता॑ ए॒व पृ॒ष्ठैरव॑ रुन्धते प॒शूञ्छ॑न्दो॒मैरोजो॒ वै वी॒र्यं॑ पृ॒ष्ठानि॑ प॒शव॑श्छन्दो॒मा ओज॑स्ये॒व~(२३)

%7.4.6.3
वी॒र्ये॑ प॒शुषु॒ प्रति॑ तिष्ठन्ति षट्त्रिꣳशद्रा॒त्रो भ॑वति॒ षट्त्रिꣳ॑शदक्षरा बृह॒ती बार्\mbox{}ह॑ताः प॒शवो॑ बृह॒त्यैव प॒शूनव॑ रुन्धते बृह॒ती छन्द॑सा॒ꣴ॒ स्वारा᳚ज्यमाश्ञुताश्ञु॒वते॒ स्वारा᳚ज्यं॒ य ए॒वं वि॒द्वाꣳसः॑ षट्त्रिꣳशद्रा॒त्रमास॑ते सुव॒र्गमे॒व लो॒कं य॑न्त्यतिरा॒त्राव॒भितो॑ भवतः सुव॒र्गस्य॑ लो॒कस्य॒ परि॑गृहीत्यै॥~(२४)

%7.4.7.0
{\anuvakamend[{अ॒ति॒रा॒त्र ओज॑स्ये॒व षट्त्रिꣳ॑शच्च}]}%~(६)

%7.4.7.1
वसि॑ष्ठो ह॒तपु॑त्रो\-ऽकामयत वि॒न्देय॑ प्र॒जाम॒भि सौ॑दा॒सान्भ॑वेय॒मिति॒ स ए॒तमे॑कस्मान्नपञ्चा॒शम॑पश्य॒त्तमाह॑र॒त्तेना॑यजत॒ ततो॒ वै सो\-ऽवि॑न्दत प्र॒जाम॒भि सौ॑दा॒सान॑भव॒द्य ए॒वं वि॒द्वाꣳस॑ एकस्मान्नपञ्चा॒शमास॑ते वि॒न्दन्ते᳚ प्र॒जाम॒भि भ्रातृ॑व्यान्भवन्ति॒ त्रय॑स्त्रि॒वृतो᳚\-ऽग्निष्टो॒मा भ॑वन्ति॒ वज्र॑स्यै॒व मुख॒ꣳ॒ सꣴ श्य॑न्ति॒ दश॑ पञ्चद॒शा भ॑वन्ति पञ्चद॒शो वज्रः॑~(२५)

%7.4.7.2
वज्र॑मे॒व भ्रातृ॑व्येभ्यः॒ प्र ह॑रन्ति षोडशि॒मद्द॑श॒ममह॑र्भवति॒ वज्र॑ ए॒व वी॒र्यं॑ दधति॒ द्वाद॑श सप्तद॒शा भ॑वन्त्य॒न्नाद्य॒स्याव॑रुद्ध्या॒ अथो॒ प्रैव तैर्जा॑यन्ते॒ पृष्ठ्यः॑ षड॒हो भ॑वति॒ षड्वा ऋ॒तवः॒ षट्पृ॒ष्ठानि॑ पृ॒ष्ठैरे॒वर्तून॒न्वारो॑हन्त्यृ॒तुभिः॑ संवथ्स॒रन्ते सं॑वथ्स॒र ए॒व प्रति॑ तिष्ठन्ति॒ द्वाद॑शैकवि॒ꣳ॒शा भ॑वन्ति॒ प्रति॑ष्ठित्या॒ अथो॒ रुच॑मे॒वात्मन्न्~(२६)

%7.4.7.3
द॒ध॒ते॒ ब॒हवः॑ षोड॒शिनो॑ भवन्ति॒ विजि॑त्यै॒ षडा᳚श्वि॒नानि॑ भवन्ति॒ षड्वा ऋ॒तव॑ ऋ॒तुष्वे॒व प्रति॑ तिष्ठन्त्यूनातिरि॒क्ता वा ए॒ता रात्र॑य ऊ॒नास्तद्यदेक॑स्यै॒ न प॑ञ्चा॒शदति॑रिक्ता॒स्तद्यद्भूय॑सीर॒ष्टाच॑त्वारिꣳशत ऊ॒नाच्च॒ खलु॒ वा अति॑रिक्ताच्च प्र॒जा\-प॑तिः॒ प्राजा॑यत॒ ये प्र॒जाका॑माः प॒शुका॑माः॒ स्युस्त ए॒ता आ॑सीर॒न्प्रैव जा॑यन्ते प्र॒जया॑ प॒शुभि॑र्वैरा॒जो वा ए॒ष य॒ज्ञो यदे॑कस्मान्नपञ्चा॒शो य ए॒वं वि॒द्वाꣳस॑ एकस्मान्नपञ्चा॒शमास॑ते वि॒राज॑मे॒व ग॑च्छन्त्यन्ना॒दा भ॑वन्त्यतिरा॒त्राव॒भितो॑ भवतो॒\-ऽन्नाद्य॑स्य॒ परि॑गृहीत्यै॥~(२७)

%7.4.8.0
{\anuvakamend[{वज्र॑ आ॒त्मन्प्र॒जया॒ द्वाविꣳ॑शतिश्च}]}%~(७)

%7.4.8.1
सं॒व॒थ्स॒राय॑ दीक्षि॒ष्यमा॑णा एकाष्ट॒कायां᳚ दीक्षेरन्ने॒षा वै सं॑वथ्स॒रस्य॒ पत्नी॒ यदे॑काष्ट॒कैतस्यां॒ वा ए॒ष ए॒ताꣳ रात्रिं॑ वसति सा॒क्षादे॒व सं॑वथ्स॒रमा॒रभ्य॑ दीक्षन्त॒ आर्तं॒ वा ए॒ते सं॑वथ्स॒रस्या॒भि दी᳚क्षन्ते॒ य ए॑काष्ट॒कायां॒ दीक्ष॒न्ते\-ऽन्त॑नामानावृ॒तू भ॑वतो॒ व्य॑स्तं॒ वा ए॒ते सं॑वथ्स॒रस्या॒भि दी᳚क्षन्ते॒ य ए॑काष्ट॒कायां॒ दीक्ष॒न्ते\-ऽन्त॑नामानावृ॒तू भ॑वतः फल्गुनीपूर्णमा॒से दी᳚क्षेर॒न्मुखं॒ वा ए॒तत्~(२८)

%7.4.8.2
सं॒व॒थ्स॒रस्य॒ यत्फ॑ल्गुनीपूर्णमा॒सो मु॑ख॒त ए॒व सं॑वथ्स॒रमा॒रभ्य॑ दीक्षन्ते॒ तस्यैकै॒व नि॒र्या यथ्साम्मे᳚घ्ये विषू॒वान्थ्स॒म्पद्य॑ते चित्रापूर्णमा॒से दी᳚क्षेर॒न्मुखं॒ वा ए॒तथ्सं॑वथ्स॒रस्य॒ यच्चि॑त्रापूर्णमा॒सो मु॑ख॒त ए॒व सं॑वथ्स॒रमा॒रभ्य॑ दीक्षन्ते॒ तस्य॒ न का च॒न नि॒र्या भ॑वति चतुर॒हे पु॒रस्ता᳚त्पौर्णमा॒स्यै दी᳚क्षेर॒न्तेषा॑मेकाष्ट॒कायां᳚ क्र॒यः सम्प॑द्यते॒ तेनै॑काष्ट॒कां न छ॒म्बट्कु॑र्वन्ति॒ तेषा᳚म्~(२९)

%7.4.8.3
पू॒र्व॒प॒क्षे सु॒त्या सम्प॑द्यते पूर्वप॒क्षम्मासा॑ अ॒भि सम्प॑द्यन्ते॒ ते पू᳚र्वप॒क्ष उत्ति॑ष्ठन्ति॒ तानु॒त्तिष्ठ॑त॒ ओष॑धयो॒ वन॒स्पत॒यो\-ऽनूत्ति॑ष्ठन्ति॒ तान्क॑ल्या॒णी की॒र्तिरनूत्ति॑ष्ठ॒त्यरा᳚थ्सुरि॒मे यज॑माना॒ इति॒ तदनु॒ सर्वे॑ राध्नुवन्ति॥~(३०)

%7.4.9.0
{\anuvakamend[{ए॒तच्छ॒म्बट्कु॑र्वन्ति॒ तेषा॒ञ्चतु॑स्त्रिꣳशच्च}]}%~(८)

%7.4.9.1
सु॒व॒र्गं वा ए॒ते लो॒कं य॑न्ति॒ ये स॒त्रमु॑प॒यन्त्य॒भीन्ध॑त ए॒व दी॒क्षाभि॑रा॒त्मानꣴ॑ श्रपयन्त उप॒सद्भि॒र्द्वाभ्यां॒ लोमाव॑ द्यन्ति॒ द्वाभ्या॒न्त्वचं॒ द्वाभ्या॒मसृ॒द्द्वाभ्या᳚म्मा॒ꣳ॒सं द्वाभ्या॒मस्थि॒ द्वाभ्या᳚म्म॒ज्जान॑मा॒त्मद॑क्षिणं॒ वै स॒त्रमा॒त्मान॑मे॒व दक्षि॑णां नी॒त्वा सु॑व॒र्गं लो॒कं य॑न्ति॒ शिखा॒मनु॒ प्र व॑पन्त॒ ऋद्ध्या॒ अथो॒ रघी॑याꣳसः सुव॒र्गं लो॒कम॑या॒मेति॑॥~(३१)

%7.4.10.0
{\anuvakamend[{सु॒व॒र्गम्प॑ञ्चा॒शत्}]}%~(९)

%7.4.10.1
ब्र॒ह्म॒वा॒दिनो॑ वदन्त्यतिरा॒त्रः प॑र॒मो य॑ज्ञक्रतू॒नां कस्मा॒त्तम्प्र॑थ॒ममुप॑ य॒न्तीत्ये॒तद्वा अ॑ग्निष्टो॒मम्प्र॑थ॒ममुप॑ य॒न्त्यथो॒क्थ्य॑मथ॑ षोड॒शिन॒मथा॑तिरा॒त्रम॑नुपू॒र्वमे॒वैतद्य॑ज्ञक्र॒तूनु॒पेत्य॒ ताना॒लभ्य॑ परि॒गृह्य॒ सोम॑मे॒वैतत्पिब॑न्त आसते॒ ज्योति॑ष्टोमम्प्रथ॒ममुप॑ यन्ति॒ ज्योति॑ष्टोमो॒ वै स्तोमा॑ना॒म्मुख॑म्मुख॒त ए॒व स्तोमा॒न्प्र यु॑ञ्जते॒ ते~(३२)

%7.4.10.2
सꣴस्तु॑ता वि॒राज॑म॒भि सम्प॑द्यन्ते॒ द्वे चर्चा॒वति॑ रिच्येते॒ एक॑या॒ गौरति॑रिक्त॒ एक॒यायु॑रू॒नः सु॑व॒र्गो वै लो॒को ज्योति॒रूर्ग्वि॒राट्सु॑व॒र्गमे॒व तेन॑ लो॒कं य॑न्ति रथन्त॒रं दिवा॒ भव॑ति रथन्त॒रं नक्त॒मित्या॑हुर्ब्रह्मवा॒दिनः॒ केन॒ तदजा॒मीति॑ सौभ॒रं तृ॑तीयसव॒ने ब्र॑ह्मसा॒मम्बृ॒हत्तन्म॑ध्य॒तो द॑धति॒ विधृ॑त्यै॒ तेनाजा॑मि॥~(३३)

%7.4.11.0
{\anuvakamend[{त एका॒न्नप॑ञ्चा॒शच्च॑}]}%॥10॥

%7.4.11.1
ज्योति॑ष्टोमम्प्रथ॒ममुप॑ यन्त्य॒स्मिन्ने॒व तेन॑ लो॒के प्रति॑ तिष्ठन्ति॒ गोष्टो॑मं द्वि॒तीय॒मुप॑ यन्त्य॒न्तरि॑क्ष ए॒व तेन॒ प्रति॑ तिष्ठ॒न्त्यायु॑ष्टोमं तृ॒तीय॒मुप॑ यन्त्य॒मुष्मि॑न्ने॒व तेन॑ लो॒के प्रति॑ तिष्ठन्ती॒यं वाव ज्योति॑र॒न्तरि॑क्षं॒ गौर॒सावायु॒र्यदे॒तान्थ्स्तोमा॑नुप॒यन्त्ये॒ष्वे॑व तल्लो॒केषु॑ स॒त्रिणः॑ प्रति॒तिष्ठ॑न्तो यन्ति॒ ते सꣴस्तु॑ता वि॒राजम्᳚~(३४)

%7.4.11.2
अ॒भि सम्प॑द्यन्ते॒ द्वे चर्चा॒वति॑ रिच्येते॒ एक॑या॒ गौरति॑रिक्त॒ एक॒यायु॑रू॒नः सु॑व॒र्गो वै लो॒को ज्योति॒रूर्ग्वि॒राडूर्ज॑मे॒वाव॑ रुन्धते॒ ते न क्षु॒धार्ति॒मार्च्छ॒न्त्यक्षो॑धुका भवन्ति॒ क्षुथ्स॑म्बाधा इव॒ हि स॒त्रिणो᳚\-ऽग्निष्टो॒माव॒भितः॑ प्र॒धी तावु॒क्थ्या॑ मध्ये॒ नभ्यं॒ तत्तदे॒तत्प॑रि॒यद्दे॑वच॒क्रं यदे॒तेन॑~(३५)

%7.4.11.3
ष॒ड॒हेन॒ यन्ति॑ देवच॒क्रमे॒व स॒मारो॑ह॒न्त्यरि॑ष्ट्यै॒ ते स्व॒स्ति सम॑श्ञुवते षड॒हेन॑ यन्ति॒ षड्वा ऋ॒तव॑ ऋ॒तुष्वे॒व प्रति॑ तिष्ठन्त्युभ॒यतो᳚ज्योतिषा यन्त्युभ॒यत॑ ए॒व सु॑व॒र्गे लो॒के प्र॑ति॒तिष्ठ॑न्तो यन्ति॒ द्वौ ष॑ड॒हौ भ॑वत॒स्तानि॒ द्वाद॒शाहा॑नि॒ सम्प॑द्यन्ते द्वाद॒शो वै पुरु॑षो॒ द्वे स॒क्थ्यौ᳚ द्वौ बा॒हू आ॒त्मा च॒ शिर॑श्च च॒त्वार्यङ्गा॑नि॒ स्तनौ᳚ द्वाद॒शौ~(३६)

%7.4.11.4
तत्पुरु॑ष॒मनु॑ प॒र्याव॑र्तन्ते॒ त्रयः॑ षड॒हा भ॑वन्ति॒ तान्य॒ष्टाद॒शाहा॑नि॒ सम्प॑द्यन्ते॒ नवा॒न्यानि॒ नवा॒न्यानि॒ नव॒ वै पुरु॑षे प्रा॒णास्तत्प्रा॒णाननु॑ प॒र्याव॑र्तन्ते च॒त्वारः॑ षड॒हा भ॑वन्ति॒ तानि॒ चतु॑र्विꣳशति॒रहा॑नि॒ सम्प॑द्यन्ते॒ चतु॑र्विꣳशतिरर्धमा॒साः सं॑वथ्स॒रस्तथ्सं॑वथ्स॒रमनु॑ प॒र्याव॑र्त॒न्ते\-ऽप्र॑तिष्ठितः संवथ्स॒र इति॒ खलु॒ वा आ॑हु॒र्वर्\mbox{}षी॑यान्प्रति॒ष्ठाया॒ इत्ये॒ताव॒द्वै सं॑वथ्स॒रस्य॒ ब्राह्म॑णं॒ याव॑न्मा॒सो मा॒सिमा᳚स्ये॒व प्र॑ति॒तिष्ठ॑न्तो यन्ति॥~(३७)

%7.4.12.0
{\anuvakamend[{वि॒राज॑मे॒तेन॑ द्वाद॒शावे॒ताव॒द्वा अ॒ष्टौ च॑}]}%॥11॥

%7.4.12.1
मे॒षस्त्वा॑ पच॒तैर॑वतु॒ लोहि॑तग्रीव॒श्छागैः᳚ शल्म॒लिर्वृद्ध्या॑ प॒र्णो ब्रह्म॑णा प्ल॒क्षो मेधे॑न न्य॒ग्रोध॑श्चम॒सैरु॑दु॒म्बर॑ ऊ॒र्जा गा॑य॒त्री छन्दो॑भिस्त्रि॒वृथ्स्तोमै॒रव॑न्तीः॒ स्थाव॑न्तीस्त्वावन्तु प्रि॒यं त्वा᳚ प्रि॒याणां॒ वर्\mbox{}षि॑ष्ठ॒माप्या॑नां निधी॒नां त्वा॑ निधि॒पतिꣳ॑ हवामहे वसो मम॥~(३८)

%7.4.13.0
{\anuvakamend[{मे॒षः षट्त्रिꣳ॑शत्}]}%॥12॥

%7.4.13.1
कूप्या᳚भ्यः॒ स्वाहा॒ कूल्या᳚भ्यः॒ स्वाहा॑ विक॒र्या᳚भ्यः॒ स्वाहा॑\-ऽव॒ट्या᳚भ्यः॒ स्वाहा॒ खन्या᳚भ्यः॒ स्वाहा॒ ह्रद्या᳚भ्यः॒ स्वाहा॒ सूद्या᳚भ्यः॒ स्वाहा॑ सर॒स्या᳚भ्यः॒ स्वाहा॑ वैश॒न्तीभ्यः॒ स्वाहा॑ पल्व॒ल्या᳚भ्यः॒ स्वाहा॒ वर्ष्या᳚भ्यः॒ स्वाहा॑\-ऽव॒र्ष्याभ्यः॒ स्वाहा᳚ ह्रा॒दुनी᳚भ्यः॒ स्वाहा॒ पृष्वा᳚भ्यः॒ स्वाहा॒ स्यन्द॑मानाभ्यः॒ स्वाहा᳚ स्थाव॒राभ्यः॒ स्वाहा॑ नादे॒यीभ्यः॒ स्वाहा॑ सैन्ध॒वीभ्यः॒ स्वाहा॑ समु॒द्रिया᳚भ्यः॒ स्वाहा॒ सर्वा᳚भ्यः॒ स्वाहा᳚॥~(३९)

%7.4.14.0
{\anuvakamend[{कूप्या᳚भ्यश्चत्वारि॒ꣳ॒शत्}]}%॥13॥

%7.4.14.1
अ॒द्भ्यः स्वाहा॒ वह॑न्तीभ्यः॒ स्वाहा॑ परि॒वह॑न्तीभ्यः॒ स्वाहा॑ सम॒न्तं वह॑न्तीभ्यः॒ स्वाहा॒ शीघ्रं॒ वह॑न्तीभ्यः॒ स्वाहा॒ शीभं॒ वह॑न्तीभ्यः॒ स्वाहो॒ग्रं वह॑न्तीभ्यः॒ स्वाहा॑ भी॒मं वह॑न्तीभ्यः॒ स्वाहा\-ऽम्भो᳚भ्यः॒ स्वाहा॒ नभो᳚भ्यः॒ स्वाहा॒ महो᳚भ्यः॒ स्वाहा॒ सर्व॑स्मै॒ स्वाहा᳚॥~(४०)

%7.4.15.0
{\anuvakamend[{अ॒द्भ्य एका॒न्नत्रि॒ꣳ॒शत्}]}%॥14॥

%7.4.15.1
यो अर्व॑न्तं॒ जिघाꣳ॑सति॒ तम॒भ्य॑मीति॒ वरु॑णः। प॒रो मर्तः॑ प॒रः श्वा। अ॒हं च॒ त्वं च॑ वृत्रह॒न्थ्सम्ब॑भूव स॒निभ्य॒ आ। अ॒रा॒ती॒वा चि॑दद्रि॒वो\-ऽनु॑ नौ शूर मꣳसतै भ॒द्रा इन्द्र॑स्य रा॒तयः॑। अ॒भि क्रत्वे᳚न्द्र भू॒रध॒ ज्मन्न ते॑ विव्यङ्महि॒मान॒ꣳ॒ रजाꣳ॑सि। स्वेना॒ हि वृ॒त्रꣳ शव॑सा ज॒घन्थ॒ न शत्रु॒रन्तं॑ विविदद्यु॒धा ते᳚॥~(४१)

%7.4.16.0
{\anuvakamend[{वि॒वि॒दद्द्वे च॑}]}%॥15॥

%7.4.16.1
नमो॒ राज्ञे॒ नमो॒ वरु॑णाय॒ नमो\-ऽश्वा॑य॒ नमः॑ प्र॒जाप॑तये॒ नमो\-ऽधि॑पत॒ये\-ऽधि॑पतिर॒स्यधि॑पतिं मा कु॒र्वधि॑पतिर॒हं प्र॒जानां᳚ भूयास॒म्मां धे॑हि॒ मयि॑ धेह्यु॒पाकृ॑ताय॒ स्वाहा\-ऽ\-ऽल॑ब्धाय॒ स्वाहा॑ हु॒ताय॒ स्वाहा᳚॥~(४२)

%7.4.17.0
{\anuvakamend[{नम॒ एका॒न्नत्रि॒ꣳ॒शत्}]}%॥16॥

%7.4.17.1
म॒यो॒भूर्वातो॑ अ॒भि वा॑तू॒स्रा ऊर्ज॑स्वती॒रोष॑धी॒रा रि॑शन्ताम्। पीव॑स्वतीर्जी॒वध॑न्याः पिबन्त्वव॒साय॑ प॒द्वते॑ रुद्र मृड। याः सरू॑पा॒ विरू॑पा॒ एक॑रूपा॒ यासा॑म॒ग्निरिष्ट्या॒ नामा॑नि॒ वेद॑। या अङ्गि॑रस॒स्तप॑से॒ह च॒क्रुस्ताभ्यः॑ पर्जन्य॒ महि॒ शर्म॑ यच्छ। या दे॒वेषु॑ त॒नुव॒मैर॑यन्त॒ यासा॒ꣳ॒ सोमो॒ विश्वा॑ रू॒पाणि॒ वेद॑। ता अ॒स्मभ्य॒म्पय॑सा॒ पिन्व॑मानाः प्र॒जाव॑तीरिन्द्र~(४३)

%7.4.17.2
गो॒ष्ठे रि॑रीहि। प्र॒जा\-प॑ति॒र्मह्य॑मे॒ता ररा॑णो॒ विश्वै᳚र्दे॒वैः पि॒तृभिः॑ संविदा॒नः। शि॒वाः स॒तीरुप॑ नो गो॒ष्ठमाक॒स्तासां᳚ व॒यं प्र॒जया॒ सꣳ स॑देम। इ॒ह धृतिः॒ स्वाहे॒ह विधृ॑तिः॒ स्वाहे॒ह रन्तिः॒ स्वाहे॒ह रम॑तिः॒ स्वाहा॑ म॒हीमू॒ षु सु॒त्रामा॑णम्॥~(४४)

%7.4.18.0
{\anuvakamend[{इ॒न्द्रा॒ष्टात्रिꣳ॑शच्च}]}%॥17॥

%7.4.18.1
किꣴ स्वि॑दासीत्पू॒र्वचि॑त्तिः॒ किꣴ स्वि॑दासीद्बृ॒हद्वयः॑। किꣴ स्वि॑दासीत्पिशङ्गि॒ला किꣴ स्वि॑दासीत्पिलिप्पि॒ला। द्यौरा॑सीत्पू॒र्वचि॑त्ति॒रश्व॑ आसीद्बृ॒हद्वयः॑। रात्रि॑रासीत्पिशङ्गि॒लावि॑रासीत्पिलिप्पि॒ला। कः स्वि॑देका॒की च॑रति॒ क उ॑ स्विज्जायते॒ पुनः॑। किꣴ स्वि॑द्धि॒मस्य॑ भेष॒जं किꣴ स्वि॑दा॒वप॑नम्म॒हत्। सूर्य॑ एका॒की च॑रति~(४५)

%7.4.18.2
च॒न्द्रमा॑ जायते॒ पुनः॑। अ॒ग्निर्\mbox{}हि॒मस्य॑ भेष॒जं भूमि॑रा॒वप॑नम्म॒हत्। पृ॒च्छामि॑ त्वा॒ पर॒मन्तं॑ पृथि॒व्याः पृ॒च्छामि॑ त्वा॒ भुव॑नस्य॒ नाभिम्᳚। पृ॒च्छामि॑ त्वा॒ वृष्णो॒ अश्व॑स्य॒ रेतः॑ पृ॒च्छामि॑ वा॒चः प॑र॒मं व्यो॑म। वेदि॑माहुः॒ पर॒मन्तं॑ पृथि॒व्या य॒ज्ञमा॑हु॒र्भुव॑नस्य॒ नाभिम्᳚। सोम॑माहु॒र्वृष्णो॒ अश्व॑स्य॒ रेतो॒ ब्रह्मै॒व वा॒चः प॑र॒मं व्यो॑म॥~(४६)

%7.4.19.0
{\anuvakamend[{सूर्य॑ एका॒की च॑रति॒ षट्च॑त्वारिꣳशच्च}]}%॥18॥

%7.4.19.1
अम्बे॒ अम्बा॒ल्यम्बि॑के॒ न मा॑ नयति॒ कश्च॒न। स॒सस्त्य॑श्व॒कः। सुभ॑गे॒ काम्पी॑लवासिनि सुव॒र्गे लो॒के सं प्रोर्ण्वा॑थाम्। आहम॑जानि गर्भ॒धमा त्वम॑जासि गर्भ॒धम्। तौ स॒ह च॒तुरः॑ प॒दः सम्प्र सा॑रयावहै। वृषा॑ वाꣳ रेतो॒धा रेतो॑ दधा॒तूथ्स॒क्थ्यो᳚र्गृ॒दं धे᳚ह्य॒ञ्जिमुद॑ञ्जि॒मन्व॑ज। यः स्त्री॒णां जी॑व॒भोज॑नो॒ य आ॑साम्~(४७)

%7.4.19.2
बि॒ल॒धाव॑नः। प्रि॒यः स्त्री॒णाम॑पी॒च्यः॑। य आ॑सां कृ॒ष्णे लक्ष्म॑णि॒ सर्दि॑गृदिम्प॒राव॑धीत्। अम्बे॒ अम्बा॒ल्यम्बि॑के॒ न मा॑ यभति॒ कश्च॒न। स॒सस्त्य॑श्व॒कः। ऊ॒र्ध्वामे॑ना॒मुच्छ्र॑यताद्वेणुभा॒रं गि॒रावि॑व। अथा᳚स्या॒ मध्य॑मेधताꣳ शी॒ते वाते॑ पु॒नन्नि॑व। अम्बे॒ अम्बा॒ल्यम्बि॑के॒ न मा॑ यभति॒ कश्च॒न। स॒सस्त्य॑श्व॒कः। यद्ध॑रि॒णी यव॒मत्ति॒ न~(४८)

%7.4.19.3
पु॒ष्टं प॒शु म॑न्यते। शू॒द्रा यदर्य॑जारा॒ न पोषा॑य धनायति। अम्बे॒ अम्बा॒ल्यम्बि॑के॒ न मा॑ यभति॒ कश्च॒न। स॒सस्त्य॑श्व॒कः। इ॒यं य॒का श॑कुन्ति॒काहल॒मिति॒ सर्प॑ति। आह॑तं ग॒भे पसो॒ नि ज॑ल्गुलीति॒ धाणि॑का। अम्बे॒ अम्बा॒ल्यम्बि॑के॒ न मा॑ यभति॒ कश्च॒न। स॒सस्त्य॑श्व॒कः। मा॒ता च॑ ते पि॒ता च॒ ते\-ऽग्रं॑ वृ॒क्षस्य॑ रोहतः।~(४९)

%7.4.19.4
प्र सु॑ला॒मीति॑ ते पि॒ता ग॒भे मु॒ष्टिम॑तꣳसयत्। द॒धि॒क्राव्ण्णो॑ अकारिषं जि॒ष्णोरश्व॑स्य वा॒जिनः॑। सु॒र॒भि नो॒ मुखा॑ कर॒त्प्र ण॒ आयूꣳ॑षि तारिषत्। आपो॒ हि\-ष्ठा म॑यो॒भुव॒स्ता न॑ ऊ॒र्जे द॑धातन। म॒हे रणा॑य॒ चक्ष॑से। यो वः॑ शि॒वत॑मो॒ रस॒स्तस्य॑ भाजयते॒ह नः॑। उ॒श॒तीरि॑व मा॒तरः॑। तस्मा॒ अरं॑ गमाम वो॒ यस्य॒ क्षया॑य॒ जिन्व॑थ। आपो॑ ज॒नय॑था च नः॥~(५०)

%7.4.20.0
{\anuvakamend[{आ॒सा॒मत्ति॒ न रो॑हतो॒ जिन्व॑थ च॒त्वारि॑ च}]}%॥19॥

%7.4.20.1
भूर्भुवः॒ सुव॒र्वस॑वस्त्वाञ्जन्तु गाय॒त्रेण॒ छन्द॑सा रु॒द्रास्त्वा᳚ञ्जन्तु॒ त्रैष्टु॑भेन॒ छन्द॑सादि॒त्यास्त्वा᳚ञ्जन्तु॒ जाग॑तेन॒ छन्द॑सा॒ यद्वातो॑ अ॒पो अग॑म॒दिन्द्र॑स्य त॒नुवं॑ प्रि॒याम्। ए॒तꣴ स्तो॑तरे॒तेन॑ प॒था पुन॒रश्व॒मा व॑र्तयासि नः। लाजी~(३) ञ्छाची~(३) न् यशो॑ म॒मा~(४)म्। य॒व्यायै॑ ग॒व्याया॑ ए॒तद्दे॑वा॒ अन्न॑मत्तै॒तदन्न॑मद्धि प्रजापते। यु॒ञ्जन्ति॑ ब्र॒ध्नम॑रु॒षं चर॑न्तं॒ परि॑ त॒स्थुषः॑। रोच॑न्ते रोच॒ना दि॒वि। यु॒ञ्जन्त्य॑स्य॒ काम्या॒ हरी॒ विप॑क्षसा॒ रथे᳚। शोणा॑ धृ॒ष्णू नृ॒वाह॑सा। के॒तुं कृ॒ण्वन्न॑के॒तवे॒ पेशो॑ मर्या अपे॒शसे᳚। समु॒षद्भि॑रजायथाः॥~(५१)

%7.4.21.0
{\anuvakamend[{ब्र॒ध्नं पञ्च॑विꣳशतिश्च}]}%॥20॥

%7.4.21.1
प्रा॒णाय॒ स्वाहा᳚ व्या॒नाय॒ स्वाहा॑\-ऽपा॒नाय॒ स्वाहा॒ स्नाव॑भ्यः॒ स्वाहा॑ सन्ता॒नेभ्यः॒ स्वाहा॒ परि॑सन्तानेभ्यः॒ स्वाहा॒ पर्व॑भ्यः॒ स्वाहा॑ स॒न्धाने᳚भ्यः॒ स्वाहा॒ शरी॑रेभ्यः॒ स्वाहा॑ य॒ज्ञाय॒ स्वाहा॒ दक्षि॑णाभ्यः॒ स्वाहा॑ सुव॒र्गाय॒ स्वाहा॑ लो॒काय॒ स्वाहा॒ सर्व॑स्मै॒ स्वाहा᳚॥~(५२)

%7.4.22.0
{\anuvakamend[{प्रा॒णाया॒ष्टाविꣳ॑शतिः}]}%॥21॥

%7.4.22.1
सि॒ताय॒ स्वाहा\-ऽसि॑ताय॒ स्वाहा॒\-ऽभिहि॑ताय॒ स्वाहा\-ऽन॑भिहिताय॒ स्वाहा॑ यु॒क्ताय॒ स्वाहाH\-ऽयु॑क्ताय॒ स्वाहा॒ सुयु॑क्ताय॒ स्वाहोद्यु॑क्ताय॒ स्वाहा॒ विमु॑क्ताय॒ स्वाहा॒ प्रमु॑क्ताय॒ स्वाहा॒ वञ्च॑ते॒ स्वाहा॑ परि॒वञ्च॑ते॒ स्वाहा॑ सं॒वञ्च॑ते॒ स्वाहा॑\-ऽनु॒वञ्च॑ते॒ स्वाहो॒द्वञ्च॑ते॒ स्वाहा॑ य॒ते स्वाहा॒ धाव॑ते॒ स्वाहा॒ तिष्ठ॑ते॒ स्वाहा॒ सर्व॑स्मै॒ स्वाहा᳚॥~(५३)

%7.5.0.0
{\anuvakamend[{सि॒ताया॒ष्टात्रिꣳ॑शत्}]}%॥22॥

{\anuvakamend[{बृह॒स्पतिः॒ श्रद्यथा॒ वा ऋ॒क्षा वै प्र॒जा\-प॑ति॒र्येन॑येन॒ द्वे वाव दे॑वस॒त्रे आ॑दि॒त्या अ॑कामयन्त सुव॒र्गं वसि॑ष्ठः संवथ्स॒राय॑ सुव॒र्गं ये स॒त्रम्ब्र॑ह्मवा॒दिनो॑\-ऽतिरा॒त्रो ज्योति॑ष्टोमं मे॒षः कूप्या᳚भ्यो॒\-ऽद्भ्यो यो नमो॑ मयो॒भूः किꣴ स्वि॒दम्बे॒ भूः प्रा॒णाय॑ सि॒ताय॒ द्वाविꣳ॑शतिः}]%॥22॥

{\prashnaend[{बृह॒स्पतिः॒ प्रति॑तिष्ठन्ति॒ वै द॑शरा॒त्रेण॑ सुव॒र्गं यो अर्व॑न्तं॒ भूस्त्रिप़॑ञ्चा॒शत्॥53॥ बृह॒स्पतिः॒ सर्व॑स्मै॒ स्वाहा᳚॥}]}

%7.5.0.0

%%% END PRASHNA

\sect{पञ्चमः प्रश्नः}\setcounter{anuvakam}{0}
\dnsub{तैत्तिरीयसंहितायां सप्तमकाण्डे पञ्चमः प्रश्नः}
%7.5.1.0
%7.5.1.1
गावो॒ वा ए॒तथ्स॒त्रमा॑सताशृ॒ङ्गाः स॒तीः शृङ्गा॑णि नो जायन्ता॒ इति॒ कामे॑न॑ तासां॒ दश॒ मासा॒ निष॑ण्णा॒ आस॒न्नथ॒ शृङ्गा᳚ण्यजायन्त॒ ता उद॑तिष्ठ॒न्नरा॒थ्स्मेत्यथ॒ यासां॒ नाजा॑यन्त॒ ताः सं॑वथ्स॒रमा॒प्त्वोद॑तिष्ठ॒न्नरा॒थ्स्मेति॒ यासां॒ चाजा॑यन्त॒ यासां᳚ च॒ न ता उ॒भयी॒रुद॑तिष्ठ॒न्नरा॒थ्स्मेति॑ गोस॒त्रं वै~(१)

%7.5.1.2
सं॒व॒थ्स॒रो य ए॒वं वि॒द्वाꣳसः॑ संवथ्स॒रमु॑प॒यन्त्यृ॑ध्नु॒वन्त्ये॒व तस्मा᳚त्तूप॒रा वार्\mbox{}षि॑कौ॒ मासौ॒ पर्त्वा॑ चरति स॒त्राभि॑जित॒ꣴ॒ ह्य॑स्यै॒ तस्मा᳚थ्संवथ्सर॒सदो॒ यत्किं च॑ गृ॒हे क्रि॒यते॒ तदा॒प्तमव॑रुद्धम॒भिजि॑तं क्रियते समु॒द्रं वा ए॒ते प्र प्ल॑वन्ते॒ ये सं॑वथ्स॒रमु॑प॒यन्ति॒ यो वै स॑मु॒द्रस्य॒ पारं॒ न पश्य॑ति॒ न वै स तत॒ उदे॑ति संवथ्स॒रः~(२)

%7.5.1.3
वै स॑मु॒द्रस्तस्यै॒तत्पा॒रं यद॑तिरा॒त्रौ य ए॒वं वि॒द्वाꣳसः॑ संवथ्स॒रमु॑प॒यन्त्यना᳚र्ता ए॒वोदृचं॑ गच्छन्ती॒यं वै पूर्वो॑\-ऽतिरा॒त्रो॑\-ऽ\-सावुत्त॑रो॒ मनः॒ पूर्वो॒ वागुत्त॑रः प्रा॒णः पूर्वो॑\-ऽपा॒न उत्त॑रः प्र॒रोध॑न॒म्पूर्व॑ उ॒दय॑न॒मुत्त॑रो॒ ज्योति॑ष्टोमो वैश्वान॒रो॑\-ऽतिरा॒त्रो भ॑वति॒ ज्योति॑रे॒व पु॒रस्ता᳚द्दधते सुव॒र्गस्य॑ लो॒कस्यानु॑ख्यात्यै चतुर्वि॒ꣳ॒शः प्रा॑य॒णीयो॑ भवति चतु॑र्विꣳशतिरर्धमा॒साः~(३)

%7.5.1.4
सं॒व॒थ्स॒रः प्र॒यन्त॑ ए॒व सं॑वथ्स॒रे प्रति॑ तिष्ठन्ति॒ तस्य॒ त्रीणि॑ च श॒तानि॑ ष॒ष्टिश्च॑ स्तो॒त्रीया॒स्ताव॑तीः संवथ्स॒रस्य॒ रात्र॑य उ॒भे ए॒व सं॑वथ्स॒रस्य॑ रू॒पे आ᳚प्नुवन्ति॒ ते सꣴस्थि॑त्या॒ अरि॑ष्ट्या॒ उत्त॑रै॒रहो॑भिश्चरन्ति षड॒हा भ॑वन्ति॒ षड्वा ऋ॒तवः॑ संवथ्स॒र ऋ॒तुष्वे॒व सं॑वथ्स॒रे प्रति॑ तिष्ठन्ति॒ गौश्चायु॑श्च मध्य॒तः स्तोमौ॑ भवतः संवथ्स॒रस्यै॒व तन्मि॑थु॒नम्म॑ध्य॒तः~(४)

%7.5.1.5
द॒ध॒ति॒ प्र॒जन॑नाय॒ ज्योति॑र॒भितो॑ भवति वि॒मोच॑नमे॒व तच्छन्दाꣴ॑स्ये॒व तद्वि॒मोकं॑ य॒न्त्यथो॑ उभ॒यतो᳚ज्योतिषै॒व ष॑ड॒हेन॑ सुव॒र्गं लो॒कं य॑न्ति ब्रह्मवा॒दिनो॑ वद॒न्त्यास॑ते॒ केन॑ य॒न्तीति॑ देव॒याने॑न प॒थेति॑ ब्रूया॒च्छन्दाꣳ॑सि॒ वै दे॑व॒यानः॒ पन्था॑ गाय॒त्री त्रि॒ष्टुब्जग॑ती॒ ज्योति॒र्वै गा॑य॒त्री गौस्त्रि॒ष्टुगायु॒र्जग॑ती यदे॒ते स्तोमा॒ भव॑न्ति देव॒याने॑नै॒व~(५)

%7.5.1.6
तत्प॒था य॑न्ति समा॒नꣳ साम॑ भवति देवलो॒को वै साम॑ देवलो॒कादे॒व न य॑न्त्य॒न्याअ॑न्या॒ ऋचो॑ भवन्ति मनुष्यलो॒को वा ऋचो॑ मनुष्यलो॒कादे॒वान्यम॑न्यं देवलो॒कम॑भ्या॒रोह॑न्तो यन्त्यभिव॒र्तो ब्र॑ह्मसा॒मं भ॑वति सुव॒र्गस्य॑ लो॒कस्या॒भिवृ॑त्त्या अभि॒जिद्भ॑वति सुव॒र्गस्य॑ लो॒कस्या॒भिजि॑त्यै विश्व॒जिद्भ॑वति॒ विश्व॑स्य॒ जित्यै॑ मा॒सिमा॑सि पृ॒ष्ठान्युप॑ यन्ति मा॒सिमा᳚स्यतिग्रा॒ह्या॑ गृह्यन्ते मा॒सिमा᳚स्ये॒व वी॒र्यं॑ दधति मा॒सां प्रति॑ष्ठित्या उ॒परि॑ष्टान्मा॒सां पृ॒ष्ठान्युप॑ यन्ति॒ तस्मा॑दु॒परि॑ष्टा॒दोष॑धयः॒ फलं॑ गृह्णन्ति॥~(६)

%7.5.2.0
{\anuvakamend[{गो॒स॒त्रं वा ए॑ति संवथ्स॒रो᳚\-ऽर्धमा॒सा मि॑थु॒नम्म॑ध्य॒तो दे॑व॒याने॑नै॒व वी॒र्य॑न्त्रयो॑दश च}]}%~(१)

%7.5.2.1
गावो॒ वा ए॒तथ्स॒त्रमा॑सताशृ॒ङ्गाः स॒तीः शृ॑ङ्गाणि॒ सिषा॑सन्ती॒स्तासां॒ दश॒ मासा॒ निष॑ण्णा॒ आस॒न्नथ॒ शृङ्गा᳚ण्यजायन्त॒ ता अ॑ब्रुव॒न्नरा॒थ्स्मोत्ति॑ष्ठा॒माव॒ तं काम॑मरुथ्स्महि॒ येन॒ कामे॑न॒ न्यष॑दा॒मेति॒ तासा॑मु॒ त्वा अ॑ब्रुवन्न॒र्धा वा॒ याव॑ती॒र्वासा॑महा ए॒वेमौ द्वा॑द॒शौ मासौ॑ संवथ्स॒रꣳ स॒म्पाद्योत्ति॑ष्ठा॒मेति॒ तासा᳚म्~(७)

%7.5.2.2
द्वा॒द॒शे मा॒सि शृङ्गा॑णि॒ प्राव॑र्तन्त श्र॒द्धया॒ वाश्र॑द्धया वा॒ ता इ॒मा यास्तू॑प॒रा उ॒भय्यो॒ वाव ता आ᳚र्ध्नुव॒न्॒ याश्च॒ शृङ्गा॒ण्यस॑न्व॒न्॒ याश्चोर्ज॑म॒वारु॑न्धत॒र्ध्नोति॑ द॒शसु॑ मा॒सू᳚त्तिष्ठ॑न्नृ॒ध्नोति॑ द्वाद॒शसु॒ य ए॒वं वेद॑ प॒देन॒ खलु॒ वा ए॒ते य॑न्ति वि॒न्दति॒ खलु॒ वै प॒देन॒ यन्तद्वा ए॒तदृ॒द्धमय॑न॒न्तस्मा॑दे॒तद्गो॒सनि॑॥~(८)

%7.5.3.0
{\anuvakamend[{ति॒ष्ठा॒मेति॒ तासा॒न्तस्मा॒द्द्वे च॑}]}%~(२)

%7.5.3.1
प्र॒थ॒मे मा॒सि पृ॒ष्ठान्युप॑ यन्ति मध्य॒म उप॑ यन्त्युत्त॒म उप॑ यन्ति॒ तदा॑हु॒र्यां वै त्रिरेक॒स्याह्न॑ उप॒सीद॑न्ति द॒ह्रं वै साप॑राभ्यां॒ दोहा᳚भ्यां दु॒हे\-ऽथ॒ कुतः॒ सा धो᳚क्ष्यते॒ यां द्वाद॑श॒ कृत्व॑ उप॒सीद॒न्तीति॑ संवथ्स॒रꣳ स॒म्पाद्यो᳚त्त॒मे मा॒सि स॒कृत्पृ॒ष्ठान्युपे॑यु॒स्तद्यज॑माना य॒ज्ञं प॒शूनव॑ रुन्धते समु॒द्रं वै~(९)

%7.5.3.2
ए॒ते॑\-ऽनवा॒रम॑पा॒रम्प्र प्ल॑वन्ते॒ ये सं॑वथ्स॒रमु॑प॒यन्ति॒ यद्बृ॑हद्रथन्त॒रे अ॒न्वर्जे॑यु॒र्यथा॒ मध्ये॑ समु॒द्रस्य॑ प्ल॒वम॒न्वर्जे॑युस्ता॒दृक्त\-दनु॑थ्सर्गम्बृहद्रथन्त॒राभ्या॑मि॒त्वा प्र॑ति॒ष्ठां ग॑च्छन्ति॒ सर्वे᳚भ्यो॒ वै कामे᳚भ्यः स॒न्धिर्दु॑हे॒ तद्यज॑मानाः॒ सर्वा॒न्कामा॒नव॑ रुन्धते॥~(१०)

%7.5.4.0
{\anuvakamend[{स॒मु॒द्रं वै चतु॑स्त्रिꣳशच्च}]}%~(३)

%7.5.4.1
स॒मा॒न्य॑ ऋचो॑ भवन्ति मनुष्यलो॒को वा ऋचो॑ मनुष्यलो॒कादे॒व न य॑न्त्य॒न्यद॑न्य॒थ्साम॑ भवति देवलो॒को वै साम॑ देवलो॒कादे॒वान्यम॑न्यम्मनुष्यलो॒कम्प्र॑त्यव॒रोह॑न्तो यन्ति॒ जग॑ती॒मग्र॒ उप॑ यन्ति॒ जग॑तीं॒ वै छन्दाꣳ॑सि प्र॒त्यव॑रोहन्त्याग्रय॒णं ग्रहा॑ बृ॒हत्पृ॒ष्ठानि॑ त्रयस्त्रि॒ꣳ॒शꣴ स्तोमा॒स्तस्मा॒ज्ज्यायाꣳ॑सं॒ कनी॑यान्प्र॒त्यव॑रोहति वैश्वकर्म॒णो गृ॑ह्यते॒ विश्वा᳚न्ये॒व तेन॒ कर्मा॑णि॒ यज॑माना॒ अव॑ रुन्धत आदि॒त्यः~(११)

%7.5.4.2
गृ॒ह्य॒त॒ इ॒यं वा अदि॑तिर॒स्यामे॒व प्रति॑ तिष्ठन्त्य॒न्यो᳚न्यो गृह्येते मिथुन॒त्वाय॒ प्रजा᳚त्या अवान्त॒रं वै द॑शरा॒त्रेण॑ प्र॒जा\-प॑तिः प्र॒जा अ॑सृजत॒ यद्द॑शरा॒त्रो भव॑ति प्र॒जा ए॒व तद्यज॑मानाः सृजन्त ए॒ताꣳ ह॒ वा उ॑द॒ङ्कः शौ᳚ल्बाय॒नः स॒त्रस्यर्द्धि॑मुवाच॒ यद्द॑शरा॒त्रो यद्द॑शरा॒त्रो भव॑ति स॒त्रस्यर्द्ध्या॒ अथो॒ यदे॒व पूर्वे॒ष्वहः॑सु॒ विलो॑म क्रि॒यते॒ तस्यै॒वैषा शान्तिः॑॥~(१२)

%7.5.5.0
{\anuvakamend[{आ॒दि॒त्यस्तस्यै॒व द्वे च॑}]}%~(४)

%7.5.5.1
यदि॒ सोमौ॒ सꣳसु॑तौ॒ स्याता᳚म्मह॒ति रात्रि॑यै प्रातरनुवा॒कमु॒पाकु॑र्या॒त्पूर्वो॒ वाच॒म्पूर्वो॑ दे॒वताः॒ पूर्व॒श्छन्दाꣳ॑सि वृङ्क्ते॒ वृष॑ण्वतीं प्रति॒पदं॑ कुर्यात्प्रातःसव॒नादे॒वैषा॒मिन्द्रं॑ वृ॒ङ्क्ते\-ऽथो॒ खल्वा॑हुः सवनमु॒खेस॑वनमुखे का॒र्येति॑ सवनमु॒खाथ्स॑वनमुखादे॒वैषा॒मिन्द्रं॑ वृङ्क्ते संवे॒शायो॑पवे॒शाय॑ गायत्रि॒यास्त्रि॒ष्टुभो॒ जग॑त्या अनु॒ष्टुभः॑ प॒ङ्क्त्या अ॒भिभू᳚त्यै॒ स्वाहा॒ छन्दाꣳ॑सि॒ वै सं॑वे॒श उ॑पवे॒शश्छन्दो॑भिरे॒वैषा᳚म्~(१३)

%7.5.5.2
छन्दाꣳ॑सि वृङ्क्ते सज॒नीय॒ꣳ॒ शस्यं॑ विह॒व्यꣳ॑ शस्य॑म॒गस्त्य॑स्य कयाशु॒भीय॒ꣳ॒ शस्य॑मे॒ताव॒द्वा अ॑स्ति॒ याव॑दे॒तद्याव॑दे॒वास्ति॒ तदे॑षां वृङ्क्ते॒ यदि॑ प्रातःसव॒ने क॒लशो॒ दीर्ये॑त वैष्ण॒वीषु॑ शिपिवि॒ष्टव॑तीषु स्तुवीर॒न्॒ यद्वै य॒ज्ञस्या॑ति॒रिच्य॑ते॒ विष्णुं॒ तच्छि॑पिवि॒ष्टम॒भ्यति॑ रिच्यते॒ तद्विष्णुः॑ शिविपि॒ष्टो\-ऽति॑रिक्त ए॒वाति॑रिक्तं दधा॒त्यथो॒ अति॑रिक्तेनै॒वाति॑रिक्तमा॒प्त्वाव॑ रुन्धते॒ यदि॑ म॒ध्यन्दि॑ने॒ दीर्ये॑त वषट्का॒रनि॑धन॒ꣳ॒ साम॑ कुर्युर्वषट्का॒रो वै य॒ज्ञस्य॑ प्रति॒ष्ठा प्र॑ति॒ष्ठामे॒वैन॑द्गमयन्ति॒ यदि॑ तृतीयसव॒न ए॒तदे॒व॥~(१४)


%7.5.6.0
{\anuvakamend[{छन्दो॑भिरे॒वैषा॒मवैका॒न्नविꣳ॑श॒तिश्च॑}]}%~(५)

%7.5.6.1
ष॒ड॒हैर्मासा᳚न्थ्स॒म्पाद्याह॒रुथ्सृ॑जन्ति षड॒हैर्\mbox{}हि मासा᳚न्थ्स॒म्पश्य॑न्त्यर्धमा॒सैर्मासा᳚न्थ्स॒म्पाद्याह॒रुथ्सृ॑जन्त्यर्धमा॒सैर्\mbox{}हि मासा᳚न्थ्स॒म्पश्य॑न्त्यमावा॒स्य॑या॒ मासा᳚न्थ्स॒म्पाद्याह॒रुथ्सृ॑जन्त्यमावा॒स्य॑या॒ हि मासा᳚न्थ्स॒म्पश्य॑न्ति पौर्णमा॒स्या मासा᳚न्थ्स॒म्पाद्याह॒रुथ्सृ॑जन्ति पौर्णमा॒स्या हि मासा᳚न्थ्स॒म्पश्य॑न्ति॒ यो वै पू॒र्ण आ॑सि॒ञ्चति॒ परा॒ स सि॑ञ्चति॒ यः पू॒र्णादु॒दच॑ति~(१५)

%7.5.6.2
प्रा॒णम॑स्मि॒न्थ्स द॑धाति॒ यत्पौ᳚र्णमा॒स्या मासा᳚न्थ्स॒म्पाद्याह॑रुथ्सृ॒जन्ति॑ संवथ्स॒रायै॒व तत्प्रा॒णं द॑धति॒ तदनु॑ स॒त्रिणः॒ प्राण॑न्ति॒ यदह॒र्नोथ्सृ॒जेयु॒र्यथा॒ दृति॒रुप॑नद्धो वि॒पत॑त्ये॒वꣳ सं॑वथ्स॒रो वि प॑ते॒दार्ति॒मार्च्छे॑यु॒र्यत्पौ᳚र्णमा॒स्या मासा᳚न्थ्स॒म्पाद्याह॑रुथ्सृ॒जन्ति॑ संवथ्स॒रायै॒व तदु॑दा॒नं द॑धति॒ तदनु॑ स॒त्रिण॒ उत्~(१६)

%7.5.6.3
अ॒न॒न्ति॒ नार्ति॒मार्च्छ॑न्ति पू॒र्णमा॑से॒ वै दे॒वानाꣳ॑ सु॒तो यत्पौ᳚र्णमा॒स्या मासा᳚न्थ्स॒म्पाद्याह॑रुथ्सृ॒जन्ति॑ दे॒वाना॑मे॒व तद्य॒ज्ञेन॑ य॒ज्ञम्प्र॒त्यव॑रोहन्ति॒ वि वा ए॒तद्य॒ज्ञं छि॑न्दन्ति॒ यत्ष॑ड॒हस॑न्तत॒ꣳ॒ सन्त॒मथाह॑रुथ्सृ॒जन्ति॑ प्राजाप॒त्यं प॒शुमाल॑भन्ते प्र॒जा\-प॑तिः॒ सर्वा॑ दे॒वता॑ दे॒वता॑भिरे॒व य॒ज्ञꣳ सं त॑न्वन्ति॒ यन्ति॒ वा ए॒ते सव॑ना॒द्ये\-ऽहः॑~(१७)

%7.5.6.4
उ॒थ्सृ॒जन्ति॑ तु॒रीयं॒ खलु॒ वा ए॒तथ्सव॑नं॒ यथ्सा᳚न्ना॒य्यं यथ्सा᳚न्ना॒य्यम्भव॑ति॒ तेनै॒व सव॑ना॒न्न य॑न्ति समुप॒हूय॑ भक्षयन्त्ये॒तथ्सो॑मपीथा॒ ह्ये॑तर्\mbox{}हि॑ यथायत॒नं वा ए॒तेषाꣳ॑ सवन॒भाजो॑ दे॒वता॑ गच्छन्ति॒ ये\-ऽह॑रुथ्सृ॒जन्त्य॑नुसव॒नं पु॑रो॒डाशा॒न्निर्व॑पन्ति यथायत॒नादे॒व स॑वन॒भाजो॑ दे॒वता॒ अव॑ रुन्धते॒\-ऽष्टाक॑पालान्प्रातःसव॒न एका॑\-दश\-कपाला॒\-न्माध्य॑न्दिने॒ सव॑ने॒ द्वाद॑श\-कपालाꣴस्तृतीयसव॒ने छन्दाꣴ॑स्ये॒वाप्त्वाव॑ रुन्धते वैश्वदे॒वं च॒रुं तृ॑तीयसव॒ने निर्व॑पन्ति वैश्वदे॒वं वै तृ॑तीयसव॒नन्तेनै॒व तृ॑तीयसव॒नान्न य॑न्ति॥~(१८)

%7.5.7.0
{\anuvakamend[{उ॒दच॒त्युद्ये\-ऽह॑रा॒प्त्वा पञ्च॑दश च}]}%~(६)

%7.5.7.1
उ॒थ्सृज्या~(३) न्नोथ्सृज्या~(३) मिति॑ मीमाꣳसन्ते ब्रह्मवा॒दिन॒स्तद्वा॑हुरु॒थ्सृज्य॑मे॒वेत्य॑मावा॒स्या॑यां च पौर्णमा॒स्यां चो॒थ्सृज्य॒मित्या॑हुरे॒ते हि य॒ज्ञं वह॑त॒ इति॒ ते त्वाव नोथ्सृज्ये॒ इत्या॑हु॒र्ये अ॑वान्त॒रं य॒ज्ञम्भे॒जाते॒ इति॒ या प्र॑थ॒मा व्य॑ष्टका॒ तस्या॑मु॒थ्सृज्य॒मित्या॑हुरे॒ष वै मा॒सो वि॑श॒र इति॒ नादि॑ष्टम्~(१९)

%7.5.7.2
उथ्सृ॑जेयु॒र्यदादि॑ष्टमुथ्सृ॒जेयु॑र्या॒दृशे॒ पुनः॑ पर्याप्ला॒वे मध्ये॑ षड॒हस्य॑ स॒म्पद्ये॑त षड॒हैर्मासा᳚न्थ्स॒म्पाद्य॒ यथ्स॑प्त॒ममह॒\-स्तस्मि॒न्नुथ्सृ॑ज्येयु॒स्तद॒ग्नये॒ वसु॑मते पुरो॒डाश॑\-म॒ष्टा\-क॑पालं॒ निर्व॑पेयुरै॒न्द्रं दधीन्द्रा॑य म॒रुत्व॑ते पुरो॒डाश॒मेका॑\-दश\-कपालं वैश्वदे॒वं द्वाद॑श\-कपालम॒ग्नेर्वै वसु॑मतः प्रातःसव॒नं यद॒ग्नये॒ वसु॑मते पुरो॒डाश॑\-म॒ष्टाक॑पालं नि॒र्वप॑न्ति दे॒वता॑मे॒व तद्भा॒गिनीं᳚ कु॒र्वन्ति॑~(२०)

%7.5.7.3
सव॑नमष्टा॒भिरुप॑ यन्ति॒ यदैन्द्रं दधि॒ भव॒तीन्द्र॑मे॒व तद्भा॑ग॒धेया॒न्न च्या॑वय॒न्तीन्द्र॑स्य॒ वै म॒रुत्व॑तो॒ माध्यं॑दिन॒ꣳ॒ सव॑नं॒ यदिन्द्रा॑य म॒रुत्व॑ते पुरो॒डाश॒मेका॑\-दश\-कपालं नि॒र्वप॑न्ति दे॒वता॑मे॒व तद्भा॒गिनीं᳚ कु॒र्वन्ति॒ सव॑नमेकाद॒शभि॒रुप॑ यन्ति॒ विश्वे॑षां॒ वै दे॒वाना॑मृभु॒मतां᳚ तृतीयसव॒नं यद्वै᳚श्वदे॒वं द्वाद॑श\-कपालं नि॒र्वप॑न्ति दे॒वता॑ ए॒व तद्भा॒गिनीः᳚ कु॒र्वन्ति॒ सव॑नं द्वाद॒शभिः॑~(२१)

%7.5.7.4
उप॑ यन्ति प्राजाप॒त्यं प॒शुमा ल॑भन्ते य॒ज्ञो वै प्र॒जा\-प॑तिर्य॒ज्ञस्यान॑नुसर्गायाभिव॒र्त इ॒तः षण्मा॒सो ब्र॑ह्मसा॒मं भ॑वति॒ ब्रह्म॒ वा अ॑भिव॒र्तो ब्रह्म॑णै॒व तथ्सु॑व॒र्गं लो॒कम॑भिव॒र्तय॑न्तो यन्ति प्रतिकू॒लमि॑व॒ हीतः सु॑व॒र्गो लो॒क इन्द्र॒ क्रतुं॑ न॒ आ भ॑र पि॒ता पु॒त्रेभ्यो॒ यथा᳚। शिक्षा॑ नो अ॒स्मिन्पु॑रुहूत॒ याम॑नि जी॒वा ज्योति॑रशीम॒हीत्य॒मुत॑ आय॒ताꣳ षण्मा॒सो ब्र॑ह्मसा॒मम्भ॑वत्य॒यं वै लो॒को ज्योतिः॑ प्र॒जा ज्योति॑रि॒ममे॒व तल्लो॒कम्पश्य॑न्तो\-ऽभि॒वद॑न्त॒ आ य॑न्ति॥~(२२)

%7.5.8.0
{\anuvakamend[{नादि॑ष्टङ्कु॒र्वन्ति॑ द्वाद॒शभि॒रिति॑ विꣳश॒तिश्च॑}]}%~(७)

%7.5.8.1
दे॒वानां॒ वा अन्तं॑ ज॒ग्मुषा॑मिन्द्रि॒यं वी॒र्य॑मपा᳚क्राम॒त्तत्क्रो॒शेनावा॑ रुन्धत॒ तत्क्रो॒शस्य॑ क्रोश॒त्वं यत्क्रो॒शेन॒ चात्वा॑ल॒स्यान्ते᳚ स्तु॒वन्ति॑ य॒ज्ञस्यै॒वान्तं॑ ग॒त्वेन्द्रि॒यं वी॒र्य॑मव॑ रुन्धते स॒त्रस्यर्द्ध्या॑हव॒नीय॒स्यान्ते᳚ स्तुवन्त्य॒ग्निमे॒वोप॑द्र॒ष्टारं॑ कृ॒त्वर्द्धि॒मुप॑ यन्ति प्र॒जाप॑ते॒र्॒\mbox{}हृद॑येन हवि॒र्धाने॒\-ऽन्तः स्तु॑वन्ति प्रे॒माण॑मे॒वास्य॑ गच्छन्ति श्लो॒केन॑ पु॒रस्ता॒थ्सद॑सः~(२३)

%7.5.8.2
स्तु॒व॒न्त्यनु॑श्लोकेन प॒श्चाद्य॒ज्ञस्यै॒वान्तं॑ ग॒त्वा श्लो॑क॒भाजो॑ भवन्ति न॒वभि॑रध्व॒र्युरुद्गा॑यति॒ नव॒ वै पुरु॑षे प्रा॒णाः प्रा॒णाने॒व यज॑मानेषु दधाति॒ सर्वा॑ ऐ॒न्द्रियो॑ भवन्ति प्रा॒णेष्वे॒वेन्द्रि॒यं द॑ध॒त्यप्र॑तिहृताभि॒रुद्गा॑यति॒ तस्मा॒त्पुरु॑षः॒ सर्वा᳚ण्य॒न्यानि॑ शी॒र्ष्णो\-ऽङ्गा॑नि॒ प्रत्य॑चति॒ शिर॑ ए॒व न पञ्च॑द॒शꣳ र॑थन्त॒रम्भ॑वतीन्द्रि॒यमे॒वाव॑ रुन्धते सप्तद॒शम्~(२४)

%7.5.8.3
बृ॒हद॒न्नाद्य॒स्याव॑रुद्ध्या॒ अथो॒ प्रैव तेन॑ जायन्त एकवि॒ꣳ॒शम्भ॒द्रं द्वि॒पदा॑सु॒ प्रति॑ष्ठित्यै॒ पत्न॑य॒ उप॑ गायन्ति मिथुन॒त्वाय॒ प्रजा᳚त्यै प्र॒जा\-प॑तिः प्र॒जा अ॑सृजत॒ सो॑\-ऽकामयता॒साम॒हꣳ रा॒ज्यं परी॑या॒मिति॒ तासाꣳ॑ राज॒नेनै॒व रा॒ज्यं पर्यै॒त्तद्रा॑ज॒नस्य॑ राजन॒त्वं यद्रा॑ज॒नम्भव॑ति प्र॒जाना॑मे॒व तद्यज॑माना रा॒ज्यं परि॑ यन्ति पञ्चवि॒ꣳ॒शं भ॑वति प्र॒जाप॑तेः~(२५)

%7.5.8.4
आप्त्यै॑ प॒ञ्चभि॒स्तिष्ठ॑न्तः स्तुवन्ति देवलो॒कमे॒वाभि ज॑यन्ति प॒ञ्चभि॒रासी॑ना मनुष्यलो॒कमे॒वाभि ज॑यन्ति॒ दश॒ सम्प॑द्यन्ते॒ दशा᳚क्षरा वि॒राडन्नं॑ वि॒राजै॒वान्नाद्य॒मव॑ रुन्धते पञ्च॒धा वि॑नि॒षद्य॑ स्तुवन्ति॒ पञ्च॒ दिशो॑ दि॒क्ष्वे॑व प्रति॑ तिष्ठ॒न्त्येकै॑क॒यास्तु॑तया स॒माय॑न्ति दि॒ग्भ्य ए॒वान्नाद्य॒ꣳ॒ सम्भ॑रन्ति॒ ताभि॑रुद्गा॒तोद्गा॑यति दि॒ग्भ्य ए॒वान्नाद्यम्᳚~(२६)

%7.5.8.5
स॒म्भृत्य॒ तेज॑ आ॒त्मन्द॑धते॒ तस्मा॒देकः॑ प्रा॒णः सर्वा॒ण्यङ्गा᳚न्यव॒त्यथो॒ यथा॑ सुप॒र्ण उ॑त्पति॒ष्यञ्छिर॑ उत्त॒मं कु॑रु॒त ए॒वमे॒व तद्यज॑मानाः प्र॒जाना॑मुत्त॒मा भ॑वन्त्यास॒न्दीमु॑द्गा॒ता रो॑हति॒ साम्रा᳚ज्यमे॒व ग॑च्छन्ति प्ले॒ङ्खꣳ होता॒ नाक॑स्यै॒व पृ॒ष्ठꣳ रो॑हन्ति कू॒र्चाव॑ध्व॒र्युर्ब्र॒ध्नस्यै॒व वि॒ष्टपं॑ गच्छन्त्ये॒ताव॑न्तो॒ वै दे॑वलो॒कास्तेष्वे॒व य॑थापू॒र्वं प्रति॑ तिष्ठ॒न्त्यथो॑ आ॒क्रम॑णमे॒व तथ्सेतुं॒ यज॑मानाः कुर्वते सुव॒र्गस्य॑ लो॒कस्य॒ सम॑ष्ट्यै॥~(२७)

%7.5.9.0
{\anuvakamend[{सद॑सः सप्तद॒शं प्र॒जाप॑तेर्गायति दि॒ग्भ्य ए॒वान्नाद्यं॒ प्रत्येका॑\-दश च}]}%~(८)

%7.5.9.1
अ॒र्क्ये॑ण॒ वै स॑हस्र॒शः प्र॒जा\-प॑तिः प्र॒जा अ॑सृजत॒ ताभ्य॒ इला᳚न्दे॒नेरां॒ लूता॒मवा॑रुन्ध॒ यद॒र्क्य॑म्भव॑ति प्र॒जा ए॒व तद्यज॑मानाः सृजन्त॒ इला᳚न्दम्भवति प्र॒जाभ्य॑ ए॒व सृ॒ष्टाभ्य॒ इरां॒ लूता॒मव॑ रुन्धते॒ तस्मा॒द्याꣳ समाꣳ॑ स॒त्रꣳ समृ॑द्धं॒ क्षोधु॑का॒स्ताꣳ समां᳚ प्र॒जा इष॒ꣴ॒ ह्या॑सा॒मूर्ज॑मा॒दद॑ते॒ याꣳ समां॒ व्यृ॑द्ध॒मक्षो॑धुका॒स्ताꣳ समां᳚ प्र॒जाः~(२८)

%7.5.9.2
न ह्या॑सा॒मिष॒मूर्ज॑मा॒दद॑त उत्क्रो॒दं कु॑र्वते॒ यथा॑ ब॒न्धान्मु॑मुचा॒ना उ॑त्क्रो॒दं कु॒र्वत॑ ए॒वमे॒व तद्यज॑माना देवब॒न्धान्मु॑मुचा॒ना उ॑त्क्रो॒दं कु॑र्वत॒ इष॒मूर्ज॑मा॒त्मन्दधा॑ना वा॒णः श॒तत॑न्तुर्भवति श॒तायुः॒ पुरु॑षः श॒तेन्द्रि॑य॒ आयु॑ष्ये॒वेन्द्रि॒ये प्रति॑ तिष्ठन्त्या॒जिं धा॑व॒न्त्यन॑भिजितस्या॒भिजि॑त्यै दुन्दु॒भीन्थ्स॒माघ्न॑न्ति पर॒मा वा ए॒षा वाग्या दु॑न्दु॒भौ प॑र॒मामे॒व~(२९)

%7.5.9.3
वाच॒मव॑ रुन्धते भूमिदुन्दु॒भिमा घ्न॑न्ति॒ यैवेमां वाक्प्रवि॑ष्टा॒ तामे॒वाव॑ रुन्ध॒ते\-ऽथो॑ इ॒मामे॒व ज॑यन्ति॒ सर्वा॒ वाचो॑ वदन्ति॒ सर्वा॑सां वा॒चामव॑रुद्ध्या आ॒र्द्रे चर्म॒न्व्याय॑च्छेते इन्द्रि॒यस्याव॑रुद्ध्या॒ आन्यः क्रोश॑ति॒ प्रान्यः शꣳ॑सति॒ य आ॒क्रोश॑ति पु॒नात्ये॒वैना॒न्थ्स स यः प्र॒शꣳस॑ति पू॒तेष्वे॒वान्नाद्यं॑ दधा॒त्यृषि॑कृतं च~(३०)

%7.5.9.4
वा ए॒ते दे॒वकृ॑तं च॒ पूर्वै॒र्मासै॒रव॑ रुन्धते॒ यद्भू॑ते॒च्छदा॒ꣳ॒ सामा॑नि॒ भव॑न्त्यु॒भय॒स्याव॑रुद्ध्यै॒ यन्ति॒ वा ए॒ते मि॑थु॒नाद्ये सं॑वथ्स॒रमु॑प॒यन्त्य॑न्तर्वे॒दि मि॑थु॒नौ सम्भ॑वत॒स्तेनै॒व मि॑थु॒नान्न य॑न्ति॥~(३१)

%7.5.10.0
{\anuvakamend[{व्यृ॑द्ध॒मक्षो॑धुका॒स्ताꣳ समां᳚ प्र॒जाः प॑र॒मामे॒व च॑ त्रि॒ꣳ॒शच्च॑}]}%~(९)

%7.5.10.1
चर्माव॑ भिन्दन्ति पा॒प्मान॑मे॒वैषा॒मव॑ भिन्दन्ति॒ माप॑ राथ्सी॒र्माति॑ व्याथ्सी॒रित्या॑ह सम्प्र॒त्ये॑वैषां᳚ पा॒प्मान॒मव॑ भिन्दन्त्युदकु॒म्भान॑धिनि॒धाय॑ दा॒स्यो॑ मार्जा॒लीयं॒ परि॑ नृत्यन्ति प॒दो नि॑घ्न॒तीरि॒दम्म॑धुं॒ गाय॑न्त्यो॒ मधु॒ वै दे॒वानां᳚ पर॒मम॒न्नाद्यं॑ पर॒ममे॒वान्नाद्य॒मव॑ रुन्धते प॒दो नि घ्न॑न्ति मही॒यामे॒वैषु॑ दधति॥~(३२)

%7.5.11.0
{\anuvakamend[{चर्मैका॒न्नप॑ञ्चा॒शत्}]}%॥10॥

%7.5.11.1
पृ॒थि॒व्यै स्वाहा॒न्तरि॑क्षाय॒ स्वाहा॑ दि॒वे स्वाहा॑ सम्प्लोष्य॒ते स्वाहा॑ स॒म्प्लव॑मानाय॒ स्वाहा॒ सम्प्लु॑ताय॒ स्वाहा॑ मेघायिष्य॒ते स्वाहा॑ मेघाय॒ते स्वाहा॑ मेघि॒ताय॒ स्वाहा॑ मे॒घाय॒ स्वाहा॑ नीहा॒राय॒ स्वाहा॑ नि॒हाका॑यै॒ स्वाहा᳚ प्रास॒चाय॒ स्वाहा᳚ प्रच॒लाका॑यै॒ स्वाहा॑ विद्योतिष्य॒ते स्वाहा॑ वि॒द्योत॑मानाय॒ स्वाहा॑ संवि॒द्योत॑मानाय॒ स्वाहा᳚ स्तनयिष्य॒ते स्वाहा᳚ स्त॒नय॑ते॒ स्वाहो॒ग्रꣴ स्त॒नय॑ते॒ स्वाहा॑ वर्\mbox{}षिष्य॒ते स्वाहा॒ वर्\mbox{}ष॑ते॒ स्वाहा॑भि॒वर्\mbox{}ष॑ते॒ स्वाहा॑ परि॒वर्\mbox{}ष॑ते॒ स्वाहा॑ सं॒वर्\mbox{}ष॑ते~(३३)

%7.5.11.2
स्वाहा॑नु॒वर्\mbox{}ष॑ते॒ स्वाहा॑ शीकायिष्य॒ते स्वाहा॑ शीकाय॒ते स्वाहा॑ शीकि॒ताय॒ स्वाहा᳚ प्रोषिष्य॒ते स्वाहा᳚ प्रुष्ण॒ते स्वाहा॑ परिप्रुष्ण॒ते स्वाहो᳚द्ग्रहीष्य॒ते स्वाहो᳚द्गृह्ण॒ते स्वाहोद्गृ॑हीताय॒ स्वाहा॑ विप्लोष्य॒ते स्वाहा॑ वि॒प्लव॑मानाय॒ स्वाहा॒ विप्लु॑ताय॒ स्वाहा॑तफ्स्य॒ते स्वाहा॒तप॑ते॒ स्वाहो॒ग्रमा॒तप॑ते॒ स्वाह॒र्ग्भ्यः स्वाहा॒ यजु॑र्भ्यः॒ स्वाहा॒ साम॑भ्यः॒ स्वाहाङ्गि॑रोभ्यः॒ स्वाहा॒ वेदे᳚भ्यः॒ स्वाहा॒ गाथा᳚भ्यः॒ स्वाहा॑ नाराश॒ꣳ॒सीभ्यः॒ स्वाहा॒ रैभी᳚भ्यः॒ स्वाहा॒ सर्व॑स्मै॒ स्वाहा᳚॥~(३४)

%7.5.12.0
{\anuvakamend[{सं॒ वर्\mbox{}ष॑ते॒ रैभी᳚भ्यः॒ स्वाहा॒ द्वे च॑}]}%॥11॥

%7.5.12.1
द॒त्वते॒ स्वाहा॑\-ऽद॒न्तका॑य॒ स्वाहा᳚ प्रा॒णिने॒ स्वाहा᳚\-ऽप्रा॒णाय॒ स्वाहा॒ मुख॑वते॒ स्वाहा॑\-ऽमु॒खाय॒ स्वाहा॒ नासि॑कवते॒ स्वाहा॑\-ऽनासि॒काय॒ स्वाहा᳚\-ऽक्ष॒ण्वते॒ स्वाहा॑\-ऽन॒क्षिका॑य॒ स्वाहा॑ क॒र्णिने॒ स्वाहा॑\-ऽक॒र्णका॑य॒ स्वाहा॑ शीर्\mbox{}ष॒ण्वते॒ स्वाहा॑\-ऽ\-शी॒र्॒\mbox{}षका॑य॒ स्वाहा॑ प॒द्वते॒ स्वाहा॑\-ऽपा॒दका॑य॒ स्वाहा᳚ प्राण॒ते स्वाहा\-ऽप्रा॑णते॒ स्वाहा॒ वद॑ते॒ स्वाहा\-ऽव॑दते॒ स्वाहा॒ पश्य॑ते॒ स्वाहा\-ऽप॑श्यते॒ स्वाहा॑ शृण्व॒ते स्वाहा\-ऽशृ॑ण्वते॒ स्वाहा॑ मन॒स्विने॒ स्वाहा᳚~(३५)

%7.5.12.2
अ॒म॒नसे॒ स्वाहा॑ रेत॒स्विने॒ स्वाहा॑\-ऽरे॒तस्का॑य॒ स्वाहा᳚ प्र॒जाभ्यः॒ स्वाहा᳚ प्र॒जन॑नाय॒ स्वाहा॒ लोम॑वते॒ स्वाहा॑\-ऽलो॒मका॑य॒ स्वाहा᳚ त्व॒चे स्वाहा॒\-ऽत्वक्का॑य॒ स्वाहा॒ चर्म॑ण्वते॒ स्वाहा॑\-ऽच॒र्मका॑य॒ स्वाहा॒ लोहि॑तवते॒ स्वाहा॑\-ऽलोहि॒ताय॒ स्वाहा॑ माꣳस॒न्वते॒ स्वाहा॑\-ऽमा॒ꣳ॒सका॑य॒ स्वाहा॒ स्नाव॑भ्यः॒ स्वाहा᳚\-ऽस्ना॒वका॑य॒ स्वाहा᳚\-ऽस्थ॒न्वते॒ स्वाहा॑\-ऽन॒स्थिका॑य॒ स्वाहा॑ मज्ज॒न्वते॒ स्वाहा॑\-ऽम॒ज्जका॑य॒ स्वाहा॒\-ऽङ्गिने॒ स्वाहा॑\-ऽन॒ङ्गाय॒ स्वाहा॒\-ऽ\-ऽत्मने॒ स्वाहा\-ऽना᳚त्मने॒ स्वाहा॒ सर्व॑स्मै॒ स्वाहा᳚॥~(३६)

%7.5.13.0
{\anuvakamend[{म॒न॒स्विने॒ स्वाहा\-ऽना᳚त्मने॒ स्वाहा॒ द्वे च॑}]}%॥12॥

%7.5.13.1
कस्त्वा॑ युनक्ति॒ स त्वा॑ युनक्तु॒ विष्णु॑स्त्वा युनक्त्व॒स्य य॒ज्ञस्यर्द्ध्यै॒ मह्य॒ꣳ॒ सन्न॑त्या अ॒मुष्मै॒ कामा॒यायु॑षे त्वा प्रा॒णाय॑ त्वा\-ऽपा॒नाय॑ त्वा व्या॒नाय॑ त्वा॒ व्यु॑ष्ट्यै त्वा र॒य्यै त्वा॒ राध॑से त्वा॒ घोषा॑य त्वा॒ पोषा॑य त्वाराद् घो॒षाय॑ त्वा॒ प्रच्यु॑त्यै त्वा॥~(३७)

%7.5.14.0
{\anuvakamend[{कस्त्वा॒\-ऽष्टात्रिꣳ॑शत्}]}%॥13॥

%7.5.14.1
अ॒ग्नये॑ गाय॒त्राय॑ त्रि॒वृते॒ राथ॑न्तराय वास॒न्ताया॒ष्टाक॑पाल॒ इन्द्रा॑य॒ त्रैष्टु॑भाय पञ्चद॒शाय॒ बार्\mbox{}ह॑ताय॒ ग्रैष्मा॒यैका॑\-दश\-कपालो॒ विश्वे᳚भ्यो दे॒वेभ्यो॒ जाग॑तेभ्यः सप्तद॒शेभ्यो॑ वैरू॒पेभ्यो॒ वार्\mbox{}षि॑केभ्यो॒ द्वाद॑श\-कपालो मि॒त्रावरु॑णाभ्या॒मानु॑ष्टुभाभ्यामेक\-वि॒ꣳ॒शा\-भ्यां᳚ वैरा॒जाभ्याꣳ॑ शार॒दा\-भ्यां᳚ पय॒स्या॑ बृह॒स्पत॑ये॒ पाङ्क्ता॑य त्रिण॒वाय॑ शाक्व॒राय॒ हैम॑न्तिकाय च॒रुः स॑वि॒त्र आ॑तिच्छन्द॒साय॑ त्रयस्त्रि॒ꣳ॒शाय॑ रैव॒ताय॑ शैशि॒राय॒ द्वाद॑श\-कपा॒लो\-ऽदि॑त्यै॒ विष्णु॑पत्न्यै च॒रुर॒ग्नये॑ वैश्वान॒राय॒ द्वाद॑श\-कपा॒लो\-ऽनु॑मत्यै च॒रुः का॒य एक॑कपालः॥~(३८)

%7.5.15.0
{\anuvakamend[{अ॒ग्नये\-ऽदि॑त्या॒ अनु॑मत्यै स॒प्तच॑त्वारिꣳशत्}]}%॥14॥

%7.5.15.1
यो वा अ॒ग्नाव॒ग्निः प्र॑ह्रि॒यते॒ यश्च॒ सोमो॒ राजा॒ तयो॑रे॒ष आ॑ति॒थ्यं यद॑ग्नीषो॒मीयो\-ऽथै॒ष रु॒द्रो यश्ची॒यते॒ यथ्सञ्चि॑ते॒\-ऽग्नावे॒तानि॑ ह॒वीꣳषि॒ न नि॒र्वपे॑दे॒ष ए॒व रु॒द्रो\-ऽशा᳚न्त उपो॒त्थाय॑ प्र॒जां प॒शून् यज॑मानस्या॒भि म॑न्येत॒ यथ्सञ्चि॑ते॒\-ऽग्नावे॒तानि॑ ह॒वीꣳषि॑ नि॒र्वप॑ति भाग॒धेये॑नै॒वैनꣳ॑ शमयति॒ नास्य॑ रु॒द्रो\-ऽशा᳚न्तः~(३९)

%7.5.15.2
उ॒पो॒त्थाय॑ प्र॒जां प॒शून॒भि म॑न्यते॒ दश॑ ह॒वीꣳषि॑ भवन्ति॒ नव॒ वै पुरु॑षे प्रा॒णा नाभि॑र्दश॒मी प्रा॒णाने॒व यज॑माने दधा॒त्यथो॒ दशा᳚क्षरा वि॒राडन्नं॑ वि॒राज्ये॒वान्नाद्ये॒ प्रति॑ तिष्ठत्यृ॒तुभि॒र्वा ए॒ष छन्दो॑भिः॒ स्तोमैः᳚ पृ॒ष्ठैश्चे॑त॒व्य॑ इत्या॑हु॒र्यदे॒तानि॑ ह॒वीꣳषि॑ नि॒र्वप॑त्यृ॒तुभि॑रे॒वैनं॒ छन्दो॑भिः॒ स्तोमैः᳚ पृ॒ष्ठैश्चि॑नुते॒ दिशः॑ सुषुवा॒णेन॑~(४०)

%7.5.15.3
अ॒भि॒जित्या॒ इत्या॑हु॒र्यदे॒तानि॑ ह॒वीꣳषि॑ नि॒र्वप॑ति दि॒शाम॒भिजि॑त्या ए॒तया॒ वा इन्द्रं॑ दे॒वा अ॑याजय॒न्तस्मा॑दिन्द्रस॒व ए॒तया॒ मनु॑म्मनु॒ष्या᳚स्तस्मा᳚न्मनुस॒वो यथेन्द्रो॑ दे॒वानां॒ यथा॒ मनु॑र्मनु॒ष्या॑णामे॒वं भ॑वति॒ य ए॒वं वि॒द्वाने॒तयेष्ट्या॒ यज॑ते॒ दिग्व॑तीः पुरोनुवा॒क्या॑ भवन्ति॒ सर्वा॑सां दि॒शाम॒भिजि॑त्यै॥~(४१)

%7.5.16.0
{\anuvakamend[{अशा᳚न्तः सुषुवा॒णेनैक॑चत्वारिꣳशच्च}]}%॥15॥

%7.5.16.1
यः प्रा॑ण॒तो नि॑मिष॒तो म॑हि॒त्वैक॒ इद्राजा॒ जग॑तो ब॒भूव॑। य ईशे॑ अ॒स्य द्वि॒पद॒श्चतु॑ष्पदः॒ कस्मै॑ दे॒वाय॑ ह॒विषा॑ विधेम। उ॒प॒या॒मगृ॑हीतो\-ऽसि प्र॒जाप॑तये त्वा॒ जुष्टं॑ गृह्णामि॒ तस्य॑ ते॒ द्यौर्म॑हि॒मा नक्ष॑त्राणि रू॒पमा॑दि॒त्यस्ते॒ तेज॒स्तस्मै᳚ त्वा महि॒म्ने प्र॒जाप॑तये॒ स्वाहा᳚॥~(४२)

%7.5.17.0
{\anuvakamend[{यः प्रा॑ण॒तो द्यौरा॑दि॒त्यो᳚\-ऽष्टात्रिꣳ॑शत्}]}%॥16॥

%7.5.17.1
य आ᳚त्म॒दा ब॑ल॒दा यस्य॒ विश्व॑ उ॒पास॑ते प्र॒शिषं॒ यस्य॑ दे॒वाः। यस्य॑ छा॒यामृतं॒ यस्य॑ मृ॒त्युः कस्मै॑ दे॒वाय॑ ह॒विषा॑ विधेम। उ॒प॒या॒मगृ॑हीतो\-ऽसि प्र॒जाप॑तये त्वा॒ जुष्टं॑ गृह्णामि॒ तस्य॑ ते पृथि॒वी म॑हि॒मौष॑धयो॒ वन॒स्पत॑यो रू॒पम॒ग्निस्ते॒ तेज॒स्तस्मै᳚ त्वा महि॒म्ने प्र॒जाप॑तये॒ स्वाहा᳚॥~(४३)

%7.5.18.0
{\anuvakamend[{य आ᳚त्म॒दाः पृ॑थि॒व्य॑ग्निरेका॒न्नच॑त्वारि॒ꣳ॒शत्}]}%॥17॥

%7.5.18.1
आ ब्रह्म॑न्ब्राह्म॒णो ब्र॑ह्मवर्च॒सी जा॑यता॒मा\-ऽस्मिन्रा॒ष्ट्रे रा॑ज॒न्य॑ इष॒व्यः॑ शूरो॑ महार॒थो जा॑यता॒न्दोग्ध्री॑ धे॒नुर्वोढा॑\-ऽ\-न॒ड्वाना॒शुः सप्तिः॒ पुरं॑धि॒र्योषा॑ जि॒ष्णू र॑थे॒ष्ठाः स॒भेयो॒ युवा\-ऽस्य यज॑मानस्य वी॒रो जा॑यतान्निका॒मेनि॑कामे नः प॒र्जन्यो॑ वर्\mbox{}षतु फ॒लिन्यो॑ न॒ ओष॑धयः पच्यन्तां योगक्षे॒मो नः॑ कल्पताम्॥~(४४)

%7.5.19.0
{\anuvakamend[{आ ब्रह्म॒न्नेक॑चत्वारिꣳशत्}]}%॥18॥

%7.5.19.1
आक्रान्॑ वा॒जी पृ॑थि॒वीम॒ग्निं युज॑मकृत वा॒ज्यर्वाक्रान्॑ वा॒ज्य॑न्तरि॑क्षं वा॒युं युज॑मकृत वा॒ज्यर्वा॒ द्यां वा॒ज्या\-ऽक्रꣴ॑स्त॒ सूर्यं॒ युज॑मकृत वा॒ज्यर्वा॒ग्निस्ते॑ वाजि॒न् युङ्ङनु॒ त्वा र॑भे स्व॒स्ति मा॒ सम्पा॑रय वा॒युस्ते॑ वाजि॒न् युङ्ङनु॒ त्वा र॑भे स्व॒स्ति मा॒ सम्~(४५)

%7.5.19.2
पा॒र॒यादि॒त्यस्ते॑ वाजि॒न् युङ्ङनु॒ त्वा र॑भे स्व॒स्ति मा॒ सम्पा॑रय प्राण॒धृग॑सि प्रा॒णं मे॑ दृꣳह व्यान॒धृग॑सि व्या॒नं मे॑ दृꣳहापान॒धृग॑स्यपा॒नं मे॑ दृꣳह॒ चक्षु॑रसि॒ चक्षु॒र्मयि॑ धेहि॒ श्रोत्र॑मसि॒ श्रोत्र॒म्मयि॑ धे॒ह्यायु॑र॒स्यायु॒र्मयि॑ धेहि॥~(४६)

%7.5.20.0
{\anuvakamend[{वा॒युस्ते॑ वाजि॒न् युङ्ङनु॒ त्वा र॑भे स्व॒स्ति मा॒ सन्त्रिच॑त्वारिꣳशच्च}]}%॥19॥

%7.5.20.1
जज्ञि॒ बीजं॒ वर्\mbox{}ष्टा॑ प॒र्जन्यः॒ पक्ता॑ स॒स्यꣳ सु॑पिप्प॒ला ओष॑धयः स्वधिचर॒णेयꣳ सू॑पसद॒नो᳚\-ऽग्निः स्व॑ध्य॒क्षम॒न्तरि॑क्षꣳ सुपा॒वः पव॑मानः सूपस्था॒ना द्यौः शि॒वम॒सौ तप॑न् यथापू॒र्वम॑होरा॒त्रे प॑ञ्चद॒शिनो᳚\-ऽर्धमा॒सास्त्रि॒ꣳ॒शिनो॒ मासाः᳚ कॢ॒प्ता ऋ॒तवः॑ शा॒न्तः सं॑वथ्स॒रः॥~(४७)

%7.5.21.0
{\anuvakamend[{जज्ञि॒ बीज॒मेक॑त्रिꣳशत्}]}%॥20॥

%7.5.21.1
आ॒ग्ने॒यो᳚\-ऽष्टाक॑पालः सौ॒म्यश्च॒रुः सा॑वि॒त्रो᳚\-ऽष्टाक॑पालः पौ॒ष्णश्च॒रू रौ॒द्रश्च॒रुर॒ग्नये॑ वैश्वान॒राय॒ द्वाद॑श\-कपालो मृगाख॒रे यदि॒ नागच्छे॑द॒ग्नये\-ऽꣳ॑हो॒मुचे॒\-ऽष्टाक॑पालः सौ॒र्यम्पयो॑ वाय॒व्य॑ आज्य॑भागः॥~(४८)

%7.5.22.0
{\anuvakamend[{आ॒ग्ने॒यश्चतु॑र्विꣳशतिः}]}%॥21॥

%7.5.22.1
अ॒ग्नये\-ऽꣳ॑हो॒मुचे॒\-ऽष्टाक॑पाल॒ इन्द्रा॑याꣳहो॒मुच॒ एका॑\-दश\-कपालो मि॒त्रावरु॑णाभ्यामागो॒मुग्\-भ्यां᳚ पय॒स्या॑ वायोसावि॒त्र आ॑गो॒मुग्\-भ्यां᳚ च॒रुर॒श्विभ्या॑मागो॒मुग्\-भ्यां᳚ धा॒ना म॒रुद्भ्य॑ एनो॒मुग्भ्यः॑ स॒प्तक॑पालो॒ विश्वे᳚भ्यो दे॒वेभ्य॑ एनो॒मुग्भ्यो॒ द्वाद॑श\-कपा॒लो\-ऽनु॑मत्यै च॒रुर॒ग्नये॑ वैश्वान॒राय॒ द्वाद॑श\-कपालो॒ द्यावा॑पृथि॒वीभ्या॑मꣳहो॒मुग्\-भ्यां᳚ द्विकपा॒लः॥~(४९)

%7.5.23.0
{\anuvakamend[{अ॒ग्नये\-ऽꣳ॑हो॒मुचे᳚ त्रि॒ꣳ॒शत्}]}%॥22॥

%7.5.23.1
अ॒ग्नये॒ सम॑नमत्पृथि॒व्यै सम॑नम॒द्यथा॒ग्निः पृ॑थि॒व्या स॒मन॑मदे॒वम्मह्य॑म्भ॒द्राः सन्न॑तयः॒ सं न॑मन्तु वा॒यवे॒ सम॑नमद॒न्तरि॑क्षाय॒ सम॑नम॒द्यथा॑ वा॒युर॒न्तरि॑क्षेण॒ सूर्या॑य॒ सम॑नमद्दि॒वे सम॑नम॒द्यथा॒ सूर्यो॑ दि॒वा च॒न्द्रम॑से॒ सम॑नम॒न्नक्ष॑त्रेभ्यः॒ सम॑नम॒द्यथा॑ च॒न्द्रमा॒ नक्ष॑त्रै॒र्वरु॑णाय॒ सम॑नमद॒द्भ्यः सम॑नम॒द्यथा᳚~(५०)

%7.5.23.2
वरु॑णो॒\-ऽद्भिः साम्ने॒ सम॑नमदृ॒चे सम॑नम॒द्यथा॒ साम॒र्चा ब्रह्म॑णे॒ सम॑नमत्क्ष॒त्राय॒ सम॑नम॒द्यथा॒ ब्रह्म॑ क्ष॒त्रेण॒ राज्ञे॒ सम॑नमद्वि॒शे सम॑नम॒द्यथा॒ राजा॑ वि॒शा रथा॑य॒ सम॑नम॒दश्वे᳚भ्यः॒ सम॑नम॒द्यथा॒ रथो\-ऽश्वैः᳚ प्र॒जाप॑तये॒ सम॑नमद्भू॒तेभ्यः॒ सम॑नम॒द्यथा᳚ प्र॒जा\-प॑तिर्भू॒तैः स॒मन॑मदे॒वम्मह्य॑म्भ॒द्राः सन्न॑तयः॒ सं न॑मन्तु॥~(५१)

%7.5.24.0
{\anuvakamend[{अ॒द्भ्यः सम॑नम॒द्यथा॒ मह्यं॑ च॒त्वारि॑ च}]}%॥23॥

%7.5.24.1
ये ते॒ पन्था॑नः सवितः पू॒र्व्यासो॑\-ऽरे॒णवो॒ वित॑ता अ॒न्तरि॑क्षे। तेभि॑र्नो अ॒द्य प॒थिभिः॑ सु॒गेभी॒ रक्षा॑ च नो॒ अधि॑ च देव ब्रूहि। नमो॒\-ऽग्नये॑ पृथिवि॒क्षिते॑ लोक॒स्पृते॑ लो॒कम॒स्मै यज॑मानाय देहि॒ नमो॑ वा॒यवे᳚\-ऽन्तरिक्ष॒क्षिते॑ लोक॒स्पृते॑ लो॒कम॒स्मै यज॑मानाय देहि॒ नमः॒ सूर्या॑य दिवि॒क्षिते॑ लोक॒स्पृते॑ लो॒कम॒स्मै यज॑मानाय देहि॥~(५२)

%7.5.25.0
{\anuvakamend[{ये ते॒ चतु॑श्चत्वारिꣳशत्}]}%॥24॥

%7.5.25.1
यो वा अश्व॑स्य॒ मेध्य॑स्य॒ शिरो॒ वेद॑ शीर्\mbox{}ष॒ण्वान्मेध्यो॑ भवत्यु॒षा वा अश्व॑स्य॒ मेध्य॑स्य॒ शिरः॒ सूर्य॒श्चक्षु॒र्वातः॑ प्रा॒णश्च॒न्द्रमाः॒ श्रोत्र॒न्दिशः॒ पादा॑ अवान्तरदि॒शाः पर्\mbox{}श॑वो\-ऽहोरा॒त्रे नि॑मे॒षो᳚\-ऽर्धमा॒साः पर्वा॑णि॒ मासाः᳚ स॒न्धाना᳚न्यृ॒तवो\-ऽङ्गा॑नि संवथ्स॒र आ॒त्मा र॒श्मयः॒ केशा॒ नक्ष॑त्राणि रू॒पन्तार॑का अ॒स्थानि॒ नभो॑ मा॒ꣳ॒सान्योष॑धयो॒ लोमा॑नि॒ वन॒स्पत॑यो॒ वाला॑ अ॒ग्निर्मुखं॑ वैश्वान॒रो व्यात्तम्᳚~(५३)

%7.5.25.2
स॒मु॒द्र उ॒दर॑म॒न्तरि॑क्षम्पा॒युर्द्यावा॑पृथि॒वी आ॒ण्डौ ग्रावा॒ शेपः॒ सोमो॒ रेतो॒ यज्ज॑ञ्ज॒भ्यते॒ तद्वि द्यो॑तते॒ यद्वि॑धूनु॒ते तथ्स्त॑नयति॒ यन्मेह॑ति॒ तद्व॑र्\mbox{}षति॒ वागे॒वास्य॒ वागह॒र्वा अश्व॑स्य॒ जाय॑मानस्य महि॒मा पु॒रस्ता᳚ज्जायते॒ रात्रि॑रेनम्महि॒मा प॒श्चादनु॑ जायत ए॒तौ वै म॑हि॒माना॒वश्व॑म॒भितः॒ सम्ब॑भूवतु॒र्॒\mbox{}हयो॑ दे॒वान॑वह॒दर्वासु॑रान् वा॒जी ग॑न्ध॒र्वानश्वो॑ मनु॒ष्या᳚न्थ्समु॒द्रो वा अश्व॑स्य॒ योनिः॑ समु॒द्रो बन्धुः॑॥~(५४)

{\anuvakamend[{व्यात्त॑मवह॒द्द्वाद॑श च}]}%॥25॥

{\anuvakamend[{गावो॒ गावः॒ सिषा॑सन्तीः प्रथ॒मे मा॒सि स॑मा॒न्यो॑ यदि॒ सोमौ॑ षड॒हैरु॒थ्सृज्या(३)ं दे॒वाना॑म॒र्क्ये॑ण॒ चर्माव॑ पृथि॒व्यै द॒त्वते॒ कस्त्वा॒ग्नये॒ यो वै यः प्रा॑ण॒तो य आ᳚त्म॒दा आ ब्रह्म॒न्नाक्रा॒ञ्जज्ञि॒ बीज॑माग्ने॒यो᳚\-ऽष्टाक॑पालो॒\-ऽग्नये\-ऽꣳ॑हो॒मुचे॒\-ऽ\-ष्टाक॑पालो॒\-ऽग्नये॒ सम॑नम॒द्ये ते॒ पन्था॑नो॒ यो वा अश्व॑स्य॒ मेध्य॑स्य॒ शिरः॒ प़ञ्च॑विꣳशतिः}]%॥25॥
}

{\prashnaend[{गावः॑ समा॒न्यः॑ सव॑नमष्टा॒भिर्वा ए॒ते दे॒वकृ॑तञ्चाभि॒जित्या॒ इत्या॑हु॒र्वरु॑णो॒\-ऽद्भिः साम्ने॒ चतुः॑पञ्चा॒शत्॥54॥ गावो॒ योनिः॑ समु॒द्रो बन्धुः॑॥}]}
%%% END PRASHNA
%%% END KANDAM
