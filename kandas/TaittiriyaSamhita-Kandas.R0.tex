% !TeX program = XeLaTeX
% !TeX root = ../SamhitaBook-kindle.tex

%1.1.0.0

%1.1.0.0
\chapt{काण्डम् १}
\sect{प्रथमः प्रश्नः}\setcounter{anuvakam}{0}
\dnsub{तैत्तिरीयसंहितायां प्रथमकाण्डे प्रथमः प्रश्नः}
%1.1.1.1
इ॒षे त्वो॒र्जे त्वा॑ वा॒यवः॑ स्थोपा॒यवः॑ स्थ दे॒वो वः॑ सवि॒ता प्रार्प॑यतु॒ श्रेष्ठ॑तमाय॒ कर्म॑ण॒ आ प्या॑यध्वमघ्निया देवभा॒गमूर्ज॑स्वती॒ पय॑स्वतीः प्र॒जाव॑तीरनमी॒वा अ॑य॒क्ष्मा मा वः॑ स्ते॒न ई॑शत॒ मा\-ऽघशꣳ॑सो रु॒द्रस्य॑ हे॒तिः परि॑ वो वृणक्तु ध्रु॒वा अ॒स्मिन्गोप॑तौ स्यात ब॒ह्वीर्यज॑मानस्य प॒शून्पा॑हि॥१॥

%1.1.2.0
{\anuvakamend[{इ॒षे त्रिच॑त्वारिꣳशत्}]}

%1.1.2.1
य॒ज्ञस्य॑ घो॒षद॑सि॒ प्रत्यु॑ष्ट॒ꣳ॒ रक्ष॒ प्रत्यु॑ष्टा॒ अरा॑तय॒ प्रेयम॑गाद्धि॒षणा॑ ब॒र्॒हिरच्छ॒ मनु॑ना कृ॒ता स्व॒धया॒ वित॑ष्टा॒ त आव॑हन्ति क॒वय॑ पु॒रस्ता᳚द्दे॒वेभ्यो॒ जुष्ट॑मि॒ह ब॒र्॒हिरा॒सदे॑ दे॒वानां᳚ परिषू॒तम॑सि व॒र्॒षवृ॑द्धमसि॒ देव॑बर्\mbox{}हि॒र्मा त्वा॒\-ऽन्वङ्मा ति॒र्यक्पर्व॑ ते राध्यासमाच्छे॒त्ता ते॒ मा रि॑षं॒ देव॑बर्\mbox{}हिः श॒तव॑ल्\mbox{}शं॒ वि रो॑ह स॒हस्र॑वल्\mbox{}शा॒॥२॥

%1.1.2.2
वि व॒यꣳ रु॑हेम पृथि॒व्याः स॒म्पृच॑ पाहि सुस॒म्भृता᳚ त्वा॒ सम्भ॑रा॒म्यदि॑त्यै॒ रास्ना॑\-ऽसीन्द्रा॒ण्यै स॒न्नह॑नं पू॒षा ते᳚ ग्र॒न्थिं ग्र॑थ्नातु॒ स ते॒ मा\-ऽ\-ऽस्था॒दिन्द्र॑स्य त्वा बा॒हुभ्या॒मुद्य॑च्छे॒ बृह॒स्पते᳚र्मू॒र्ध्ना ह॑राम्यु॒र्व॑न्तरि॑क्ष॒मन्वि॑हि देवङ्ग॒मम॑सि॥३॥

%1.1.3.0
{\anuvakamend[{स॒हस्र॑वल्\mbox{}शा अ॒ष्टात्रिꣳ॑शच्च}]}

%1.1.3.1
शुन्ध॑ध्वं॒ दैव्या॑य॒ कर्म॑णे देवय॒ज्यायै॑ मात॒रिश्व॑नो घ॒र्मो॑\-ऽसि॒ द्यौर॑सि पृथि॒व्य॑सि वि॒श्वधा॑या असि पर॒मेण॒ धाम्ना॒ दृꣳह॑स्व॒ मा ह्वा॒र्वसू॑नां प॒वित्र॑मसि श॒तधा॑रं॒ वसू॑नां प॒वित्र॑मसि स॒हस्र॑धारꣳ हु॒तः स्तो॒को हु॒तो द्र॒फ्सो᳚\-ऽग्नये॑ बृह॒ते नाका॑य॒ स्वाहा॒ द्यावा॑पृथि॒वीभ्या॒ꣳ॒ सा वि॒श्वायुः॒ सा वि॒श्वव्य॑चाः॒ सा वि॒श्वक॑र्मा॒ सम्पृ॑च्यध्वमृतावरीरू॒र्मिणी॒र्मधु॑मत्तमा म॒न्द्रा धन॑स्य सा॒तये॒ सोमे॑न॒ त्वा\-ऽ\-ऽत॑न॒च्मीन्द्रा॑य॒ दधि॒ विष्णो॑ ह॒व्यꣳ र॑क्षस्व॥४॥

%1.1.4.0
{\anuvakamend[{सोमे॑ना॒ष्टौ च॑}]}

%1.1.4.1
कर्म॑णे वां दे॒वेभ्यः॑ शकेयं॒ वेषा॑य त्वा॒ प्रत्यु॑ष्ट॒ꣳ॒ रक्ष॒ प्रत्यु॑ष्टा॒ अरा॑तयो॒ धूर॑सि॒ धूर्व॒ धूर्व॑न्तं॒ धूर्व॒ तं यो᳚\-ऽस्मान्धूर्व॑ति॒ तं धू᳚र्व॒ यं व॒यं धूर्वा॑म॒स्त्वं दे॒वाना॑मसि॒ सस्नि॑तमं॒ पप्रि॑तमं॒ जुष्ट॑तमं॒ वह्नि॑तमं देव॒हूत॑म॒मह्रु॑तमसि हवि॒र्धानं॒ दृꣳह॑स्व॒ मा ह्वा᳚र्मि॒त्रस्य॑ त्वा॒ चक्षु॑षा॒ प्रेक्षे॒ मा भेर्मा सं वि॑क्था॒ मा त्वा॑॥५॥

%1.1.4.2
हिꣳसिषमु॒रु वाता॑य दे॒वस्य॑ त्वा सवि॒तुः प्र॑स॒वे᳚\-ऽश्विनो᳚र्बा॒हु\-भ्यां᳚ पू॒ष्णो हस्ता᳚भ्याम॒ग्नये॒ जुष्टं॒ निर्व॑पाम्य॒ग्नी\-षोमा᳚भ्यामि॒दं दे॒वाना॑मि॒दमु॑ नः स॒ह स्फा॒त्यै त्वा॒ नारा᳚त्यै॒ सुव॑र॒भि वि ख्ये॑षं वैश्वान॒रं ज्योति॒र्दृꣳह॑न्ता॒न्दुर्या॒ द्यावा॑पृथि॒व्योरु॒र्व॑न्तरि॑क्ष॒मन्वि॒ह्यदि॑त्यास्त्वो॒पस्थे॑ सादया॒म्यग्ने॑ ह॒व्यꣳ र॑क्षस्व॥६॥

%1.1.5.0
{\anuvakamend[{मा त्वा॒ षट्च॑त्वारिꣳशच्च}]}

%1.1.5.1
दे॒वो वः॑ सवि॒तोत्पु॑ना॒त्वच्छि॑द्रेण प॒वित्रे॑ण॒ वसोः॒ सूर्य॑स्य र॒श्मिभि॒रापो॑ देवीरग्रेपुवो अग्रेगु॒वो\-ऽग्र॑ इ॒मं य॒ज्ञं न॑य॒ताग्रे॑ य॒ज्ञप॑तिं धत्त यु॒ष्मानिन्द्रो॑\-ऽवृणीत वृत्र॒तूर्ये॑ यू॒यमिन्द्र॑मवृणीध्वं वृत्र॒तूर्ये॒ प्रोक्षि॑ताः स्था॒ग्नये॑ वो॒ जुष्टं॒ प्रोक्षा᳚म्य॒ग्नीषोमा᳚भ्या॒ꣳ॒ शुन्ध॑ध्वं॒ दैव्या॑य॒ कर्म॑णे देवय॒ज्याया॒ अव॑धूत॒ꣳ॒ रक्षो\-ऽव॑धूता॒ अरा॑त॒यो\-ऽदि॑त्या॒स्त्वग॑सि॒ प्रति॑ त्वा॥७॥

%1.1.5.2
पृथि॒वी वे᳚त्त्वधि॒षव॑णमसि वानस्प॒त्यं प्रति॒ त्वा\-ऽदि॑त्या॒स्त्वग्वे᳚त्त्व॒ग्नेस्त॒नूर॑सि वा॒चो वि॒सर्ज॑नं दे॒ववी॑तये त्वा गृह्णा॒म्यद्रि॑रसि वानस्प॒त्यः स इ॒दं दे॒वेभ्यो॑ ह॒व्यꣳ सु॒शमि॑ शमि॒ष्वेष॒मा व॒दोर्ज॒मा व॑द द्यु॒मद्व॑दत व॒यꣳ स॑ङ्घा॒तं जे᳚ष्म व॒र्॒षवृ॑द्धमसि॒ प्रति॑ त्वा व॒र्॒षवृ॑द्धं वेत्तु॒ परा॑पूत॒ꣳ॒ रक्ष॒ परा॑पूता॒ अरा॑तयो॒ रक्ष॑सां भा॒गो॑\-ऽसि वा॒युर्वो॒ विवि॑नक्तु दे॒वो वः॑ सवि॒ता हिर॑ण्यपाणि॒ प्रति॑ गृह्णातु॥८॥

%1.1.6.0
{\anuvakamend[{त्वा॒ भा॒ग एका॑दश च}]}

%1.1.6.1
अव॑धूत॒ꣳ॒ रक्षो\-ऽव॑धूता॒ अरा॑त॒यो\-ऽदि॑त्या॒स्त्वग॑सि॒ प्रति॑ त्वा पृथि॒वी वे᳚त्तु दि॒वः स्क॑म्भ॒निर॑सि॒ प्रति॒ त्वा\-ऽदि॑त्या॒स्त्वग्वे᳚त्तु धि॒षणा॑\-ऽसि पर्व॒त्या प्रति॑ त्वा दि॒वः स्क॑म्भ॒निर्वे᳚त्तु धि॒षणा॑\-ऽसि पार्वते॒यी प्रति॑ त्वा पर्व॒तिर्वे᳚त्तु दे॒वस्य॑ त्वा सवि॒तुः प्र॑स॒वे᳚\-ऽश्विनो᳚र्बा॒हु\-भ्यां᳚ पू॒ष्णो हस्ता᳚भ्या॒मधि॑वपामि धा॒न्य॑मसि धिनु॒हि दे॒वान्प्रा॒णाय॑ त्वा\-ऽपा॒नाय॑ त्वा व्या॒नाय॑ त्वा दी॒र्घामनु॒ प्रसि॑ति॒मायु॑षे धां दे॒वो वः॑ सवि॒ता हिर॑ण्यपाणि॒ प्रति॑ गृह्णातु॥९॥

%1.1.7.0
{\anuvakamend[{प्रा॒णाय॑ त्वा॒ पञ्च॑दश च}]}

%1.1.7.1
धृष्टि॑रसि॒ ब्रह्म॑ य॒च्छापा᳚\-ऽग्ने॒\-ऽग्निमा॒मादं॑ जहि॒ निष्क्र॒व्यादꣳ॑ से॒धा दे॑व॒यजं॑ वह॒ निर्द॑ग्ध॒ꣳ॒ रक्षो॒ निर्द॑ग्धा॒ अरा॑तयो ध्रु॒वम॑सि पृथि॒वीं दृ॒ꣳ॒हाऽऽयु॑र्दृꣳह प्र॒जां दृꣳ॑ह सजा॒तान॒स्मै यज॑मानाय॒ पर्यू॑ह ध॒र्त्रम॑स्य॒न्तरि॑क्षं दृꣳह प्रा॒णं दृꣳ॑हापा॒नं दृꣳ॑ह सजा॒ता\-न॒स्मै यज॑मानाय॒ पर्यू॑ह ध॒रुण॑मसि॒ दिवं॑ दृꣳह॒ चक्षु॑र्॥१०॥

%1.1.7.2
दृꣳह॒ श्रोत्रं॑ दृꣳह सजा॒तान॒स्मै यज॑मानाय॒ पर्यू॑ह॒ धर्मा॑\-ऽसि॒ दिशो॑ दृꣳह॒ योनिं॑ दृꣳह प्र॒जां दृꣳ॑ह सजा॒तान॒स्मै यज॑मानाय॒ पर्यू॑ह॒ चितः॑ स्थ प्र॒जाम॒स्मै र॒यिम॒स्मै स॑जा॒तान॒स्मै यज॑मानाय॒ पर्यू॑ह॒ भृगू॑णा॒मङ्गि॑रसां॒ तप॑सा तप्यध्वं॒ यानि॑ घ॒र्मे क॒पाला᳚न्युपचि॒न्वन्ति॑ वे॒धसः॑। पू॒ष्णस्तान्यपि॑ व्र॒त इ॑न्द्रवा॒यू वि मु॑ञ्चताम्॥११॥

%1.1.8.0
{\anuvakamend[{चक्षु॑र॒ष्टाच॑त्वारिꣳशच्च}]}

%1.1.8.1
सं व॑पामि॒ समापो॑ अ॒द्भिर॑ग्मत॒ समोष॑धयो॒ रसे॑न॒ सꣳ रे॒वती॒र्जग॑तीभि॒र्मधु॑मती॒र्मधु॑मतीभिः सृज्यध्वम॒द्भ्यः परि॒ प्रजा॑ताः स्थ॒ सम॒द्भिः पृ॑च्यध्वं॒ जन॑यत्यै त्वा॒ सं यौ᳚म्य॒ग्नये᳚ त्वा॒\-ऽग्नीषोमा᳚भ्यां म॒खस्य॒ शिरो॑\-ऽसि घ॒र्मो॑\-ऽसि वि॒श्वायु॑रु॒रु प्र॑थस्वो॒रु ते॑ य॒ज्ञप॑तिः प्रथतां॒ त्वचं॑ गृह्णीष्वा॒\-ऽन्तरि॑त॒ꣳ॒ रक्षो॒\-ऽन्तरि॑ता॒ अरा॑तयो दे॒वस्त्वा॑ सवि॒ता श्र॑पयतु॒ वर्\mbox{}षि॑ष्ठे॒ अधि॒ नाके॒\-ऽग्निस्ते॑ त॒नुवं॒ मा\-ऽति॑ धा॒गग्ने॑ ह॒व्यꣳ र॑क्षस्व॒ सं ब्रह्म॑णा पृच्यस्वैक॒ताय॒ स्वाहा᳚ द्वि॒ताय॒ स्वाहा᳚ त्रि॒ताय॒ स्वाहा᳚॥१२॥

%1.1.9.0
{\anuvakamend[{स॒वि॒ता द्वाविꣳ॑शतिश्च}]}

%1.1.9.1
आद॑द॒ इन्द्र॑स्य बा॒हुर॑सि॒ दक्षि॑णः स॒हस्र॑भृष्टिः श॒तते॑जा वा॒युर॑सि ति॒ग्मते॑जा॒ पृथि॑वि देवयज॒न्योष॑ध्यास्ते॒ मूलं॒ मा हिꣳ॑सिष॒मप॑हतो॒\-ऽररु॑ पृथि॒व्यै व्र॒जं ग॑च्छ गो॒स्थानं॒ वर्\mbox{}ष॑तु ते॒ द्यौर्ब॑धा॒न दे॑व सवितः पर॒मस्यां᳚ परा॒वति॑ श॒तेन॒ पाशै॒र्यो᳚\-ऽस्मान्द्वेष्टि॒ यं च॑ व॒यं द्वि॒ष्मस्तमतो॒ मा मौ॒गप॑हतो॒\-ऽररु॑ पृथि॒व्यै दे॑व॒यज॑न्यै व्र॒जं॥१३॥

%1.1.9.2
ग॑च्छ गो॒स्थानं॒ वर्\mbox{}ष॑तु ते॒ द्यौर्ब॑धा॒न दे॑व सवितः पर॒मस्यां᳚ परा॒वति॑ श॒तेन॒ पाशै॒र्यो᳚\-ऽस्मान्द्वेष्टि॒ यं च॑ व॒यं द्वि॒ष्मस्तमतो॒ मा मौ॒गप॑हतो॒\-ऽररु॑ पृथि॒व्या अदे॑वयजनो व्र॒जं ग॑च्छ गो॒स्थानं॒ वर्\mbox{}ष॑तु ते॒ द्यौर्ब॑धा॒न दे॑व सवितः पर॒मस्यां᳚ परा॒वति॑ श॒तेन॒ पाशै॒र्यो᳚\-ऽस्मान्द्वेष्टि॒ यं च॑ व॒यं द्वि॒ष्मस्तमतो॒ मा॥१४॥

%1.1.9.3
मौ॑ग॒ररु॑स्ते॒ दिवं॒ मा स्का॒न्॒ वस॑वस्त्वा॒ परि॑गृह्णन्तु गाय॒त्रेण॒ छन्द॑सा रु॒द्रास्त्वा॒ परि॑गृह्णन्तु॒ त्रैष्टु॑भेन॒ छन्द॑सा\-ऽ\-ऽदि॒त्यास्त्वा॒ परि॑गृह्णन्तु॒ जाग॑तेन॒ छन्द॑सा दे॒वस्य॑ सवि॒तुः स॒वे कर्म॑ कृण्वन्ति वे॒धस॑ ऋ॒तम॑स्यृत॒सद॑नमस्यृत॒श्रीर॑सि॒ धा अ॑सि स्व॒धा अ॑स्यु॒र्वी चासि॒ वस्वी॑ चासि पु॒रा क्रू॒रस्य॑ वि॒सृपो॑ विरफ्शिन्नुदा॒दाय॑ पृथि॒वीं जी॒रदा॑नु॒र्यामैर॑यं च॒न्द्रम॑सि स्व॒धाभि॒स्तान्धीरा॑सो अनु॒दृश्य॑ यजन्ते॥१५॥

%1.1.10.0
{\anuvakamend[{दे॒व॒यज॑न्यै व्र॒जन्तमतो॒ मा वि॑रफ्शि॒न्नेका॑दश च}]}

%1.1.10.1
प्रत्यु॑ष्ट॒ꣳ॒ रक्ष॒ प्रत्यु॑ष्टा॒ अरा॑तयो॒\-ऽग्नेर्व॒स्तेजि॑ष्ठेन॒ तेज॑सा॒ निष्ट॑पामि गो॒ष्ठं मा निर्मृ॑क्षं वा॒जिनं॑ त्वा सपत्नसा॒हꣳ सम्मा᳚र्ज्मि॒ वाचं॑ प्रा॒णं चक्षुः॒ श्रोत्रं॑ प्र॒जां योनिं॒ मा निर्मृ॑क्षं वा॒जिनीं᳚ त्वा सपत्नसा॒हीꣳ सम्मा᳚र्ज्म्या॒शासा॑ना सौमन॒सं प्र॒जाꣳ सौभा᳚ग्यं त॒नूम्। अ॒ग्नेरनु॑व्रता भू॒त्वा सन्न॑ह्ये सुकृ॒ताय॒ कम्। सु॒प्र॒जस॑स्त्वा व॒यꣳ सु॒पत्नी॒रुप॑॥१६॥

%1.1.10.2
सेदिम। अग्ने॑ सपत्न॒दम्भ॑न॒मद॑ब्धासो॒ अदा᳚भ्यम्। इ॒मं विष्या॑मि॒ वरु॑णस्य॒ पाशं॒ यमब॑ध्नीत सवि॒ता सु॒शेवः॑। धा॒तुश्च॒ योनौ॑ सुकृ॒तस्य॑ लो॒के स्यो॒नं मे॑ स॒ह पत्या॑ करोमि। समायु॑षा॒ सम्प्र॒जया॒ सम॑ग्ने॒ वर्च॑सा॒ पुनः॑। सम्पत्नी॒ पत्या॒\-ऽहं ग॑च्छे॒ समा॒त्मा त॒नुवा॒ मम॑। म॒ही॒नां पयो॒\-ऽस्योष॑धीना॒ꣳ॒ रस॒स्तस्य॒ ते\-ऽक्षी॑यमाणस्य॒ निर्॥१७॥

%1.1.10.3
व॑पामि मही॒नां पयो॒\-ऽस्योष॑धीना॒ꣳ॒ रसो\-ऽद॑ब्धेन त्वा॒ चक्षु॒षा\-ऽवे᳚क्षे सुप्रजा॒स्त्वाय॒ तेजो॑\-ऽसि॒ तेजो\-ऽनु॒ प्रेह्य॒ग्निस्ते॒ तेजो॒ मा वि नै॑द॒ग्नेर्जि॒ह्वा\-ऽसि॑ सु॒भूर्दे॒वानां॒ धाम्ने॑धाम्ने दे॒वेभ्यो॒ यजु॑षेयजुषे भव शु॒क्रम॑सि॒ ज्योति॑रसि॒ तेजो॑\-ऽसि दे॒वो वः॑ सवि॒तोत्पु॑ना॒त्वच्छि॑द्रेण प॒वित्रे॑ण॒ वसोः॒ सूर्य॑स्य र॒श्मिभिः॑ शु॒क्रं त्वा॑ शु॒क्रायां॒ धाम्ने॑धाम्ने दे॒वेभ्यो॒ यजु॑षेयजुषे गृह्णामि॒ ज्योति॑स्त्वा॒ ज्योति॑ष्य॒र्चिस्त्वा॒\-ऽर्चिषि॒ धाम्ने॑धाम्ने दे॒वेभ्यो॒ यजु॑षेयजुषे गृह्णामि॥१८॥

%1.1.11.0
{\anuvakamend[{उप॒ नी र॒श्मिभिः॑ शु॒क्रꣳ षोड॑श च}]}

%1.1.11.1
कृष्णो᳚\-ऽस्याखरे॒ष्ठो᳚\-ऽग्नये᳚ त्वा॒ स्वाहा॒ वेदि॑रसि ब॒र्॒हिषे᳚ त्वा॒ स्वाहा॑ ब॒र्॒हिर॑सि स्रु॒ग्भ्यस्त्वा॒ स्वाहा॑ दि॒वे त्वा॒\-ऽन्तरि॑क्षाय त्वा पृथि॒व्यै त्वा᳚ स्व॒धा पि॒तृभ्य॒ ऊर्ग्भ॑व बर्\mbox{}हि॒षद्भ्य॑ ऊ॒र्जा पृ॑थि॒वीं ग॑च्छत॒ विष्णोः॒ स्तूपो॒\-ऽस्यूर्णा᳚म्रदसं त्वा स्तृणामि स्वास॒स्थं दे॒वेभ्यो॑ गन्ध॒र्वो॑\-ऽसि वि॒श्वाव॑सु॒र्विश्व॑स्मा॒दीष॑तो॒ यज॑मानस्य परि॒धिरि॒ड ई॑डि॒त इन्द्र॑स्य बा॒हुर॑सि॒॥१९॥

%1.1.11.2
दक्षि॑णो॒ यज॑मानस्य परि॒धिरि॒ड ई॑डि॒तो मि॒त्रावरु॑णौ त्वोत्तर॒तः परि॑धत्तां ध्रु॒वेण॒ धर्म॑णा॒ यज॑मानस्य परि॒धिरि॒ड ई॑डि॒तः सूर्य॑स्त्वा पु॒रस्ता᳚त्पातु॒ कस्या᳚श्चिद॒भिश॑स्त्या वी॒तिहो᳚त्रं त्वा कवे द्यु॒मन्त॒ꣳ॒ समि॑धीम॒ह्यग्ने॑ बृ॒हन्त॑मध्व॒रे वि॒शो य॒न्त्रे स्थो॒ वसू॑नाꣳ रु॒द्राणा॑मादि॒त्याना॒ꣳ॒ सद॑सि सीद जु॒हूरु॑प॒भृद्ध्रु॒वा\-ऽसि॑ घृ॒ताची॒ नाम्ना᳚ प्रि॒येण॒ नाम्ना᳚ प्रि॒ये सद॑सि सीदै॒ता अ॑सदन्थ्सुकृ॒तस्य॑ लो॒के ता वि॑ष्णो पाहि पा॒हि य॒ज्ञं पा॒हि य॒ज्ञप॑तिं पा॒हि मां य॑ज्ञ॒नियम्᳚॥२०॥

%1.1.12.0
{\anuvakamend[{बा॒हुर॑सि प्रि॒ये सद॑सि पञ्च॑दश च}]}

%1.1.12.1
भुव॑नमसि॒ वि प्र॑थ॒स्वाग्ने॒ यष्ट॑रि॒दं नमः॑। जुह्वेह्य॒ग्निस्त्वा᳚ ह्वयति देवय॒ज्याया॒ उप॑भृ॒देहि॑ दे॒वस्त्वा॑ सवि॒ता ह्व॑यति देवय॒ज्याया॒ अग्ना॑विष्णू॒ मा वा॒मव॑ क्रमिषं॒ वि जि॑हाथां॒ मा मा॒ सन्ता᳚प्तं लो॒कं मे॑ लोककृतौ कृणुतं॒ विष्णोः॒ स्थान॑मसी॒त इन्द्रो॑ अकृणोद्वी॒र्या॑णि समा॒रभ्यो॒र्ध्वो अ॑ध्व॒रो दि॑वि॒स्पृश॒मह्रु॑तो य॒ज्ञो य॒ज्ञप॑ते॒रिन्द्रा॑वा॒न्थ्स्वाहा॑ बृ॒हद्भाः पा॒हि मा᳚ऽग्ने॒ दुश्च॑रिता॒दा मा॒ सुच॑रिते भज म॒खस्य॒ शिरो॑\-ऽसि॒ सं ज्योति॑षा॒ ज्योति॑रङ्क्ताम्॥२१॥

%1.1.13.0
{\anuvakamend[{अह्रु॑त॒ एक॑विꣳशतिश्च}]}

%1.1.13.1
वाज॑स्य मा प्रस॒वेनो᳚द्ग्रा॒भेणोद॑ग्रभीत्। अथा॑ स॒पत्ना॒ꣳ॒ इन्द्रो॑ मे निग्रा॒भेणाध॑राꣳ अकः। उ॒द्ग्रा॒भं च॑ निग्रा॒भं च॒ ब्रह्म॑ दे॒वा अ॑वीवृधन्न्। अथा॑ स॒पत्ना॑निन्द्रा॒ग्नी मे॑ विषू॒चीना॒न्व्य॑स्यताम्। वसु॑भ्यस्त्वा रु॒द्रेभ्य॑स्त्वा\-ऽ\-ऽदि॒त्येभ्य॑स्त्वा॒ऽक्तꣳ रिहा॑णा वि॒यन्तु॒ वयः॑। प्र॒जां योनिं॒ मा निर्मृ॑क्ष॒मा प्या॑यन्ता॒माप॒ ओष॑धयो म॒रुतां॒ पृष॑तयः स्थ॒ दिवं॑॥२२॥

%1.1.13.2
गच्छ॒ ततो॑ नो॒ वृष्टि॒मेर॑य। आ॒यु॒ष्पा अ॑ग्ने॒\-ऽस्यायु॑र्मे पाहि चक्षु॒ष्पा अ॑ग्ने\-ऽसि॒ चक्षु॑र्मे पाहि ध्रु॒वा\-ऽसि॒ यं प॑रि॒धिं प॒र्यध॑त्था॒ अग्ने॑ देव प॒णिभि॑र्वी॒यमा॑णः। तन्त॑ ए॒तमनु॒ जोषं॑ भरामि॒ नेदे॒ष त्वद॑पचे॒तया॑तै य॒ज्ञस्य॒ पाथ॒ उप॒ समि॑तꣳ सꣴस्रा॒वभा॑गाः स्थे॒षा बृ॒हन्त॑ प्रस्तरे॒ष्ठा ब॑र्\mbox{}हि॒षद॑श्च॥२३॥

%1.1.13.3
दे॒वा इ॒मां वाच॑म॒भि विश्वे॑ गृ॒णन्त॑ आ॒सद्या॒स्मिन्ब॒र्॒हिषि॑ मादयध्वम॒ग्नेर्वा॒मप॑न्नगृहस्य॒ सद॑सि सादयामि सु॒म्नाय॑ सुम्निनी सु॒म्ने मा॑ धत्तं धु॒रि धु॒र्यौ॑ पात॒मग्ने॑\-ऽदब्धायो\-ऽशीततनो पा॒हि मा॒\-ऽद्य दि॒वः पा॒हि प्रसि॑त्यै पा॒हि दुरि॑ष्ट्यै पा॒हि दु॑रद्म॒न्यै पा॒हि दुश्च॑रिता॒दवि॑षन्नः पि॒तुं कृ॑णु सु॒षदा॒ योनि॒ꣴ॒ स्वाहा॒ देवा॑ गातुविदो गा॒तुं वि॒त्वा गा॒तुमि॑त॒ मन॑सस्पत इ॒मं नो॑ देव दे॒वेषु॑ य॒ज्ञꣴ स्वाहा॑ वा॒चि स्वाहा॒ वाते॑ धाः॥२४॥

%1.1.14.0
{\anuvakamend[{दिव॑ञ्च वि॒त्वा गा॒तुन्त्रयो॑दश च}]}

%1.1.14.1
उ॒भा वा॑मिन्द्राग्नी आहु॒वध्या॑ उ॒भा राध॑सः स॒ह मा॑द॒यध्यै᳚। उ॒भा दा॒तारा॑वि॒षाꣳ र॑यी॒णामु॒भा वाज॑स्य सा॒तये॑ हुवे वाम्। अश्र॑व॒ꣳ॒ हि भू॑रि॒दाव॑त्तरा वां॒ वि जा॑मातुरु॒त वा॑ घा स्या॒लात्। अथा॒ सोम॑स्य॒ प्रय॑ती यु॒वभ्या॒मिन्द्रा᳚ग्नी॒ स्तोमं॑ जनयामि॒ नव्यम्᳚। इन्द्रा᳚ग्नी नव॒तिं पुरो॑ दा॒सप॑त्नीरधूनुतम्। सा॒कमेके॑न॒ कर्म॑णा। शुचिं॒ नु स्तोमं॒ नव॑जातम॒द्येन्द्रा᳚ग्नी वृत्रहणा जु॒षेथा᳚म्॥॥२५॥

%1.1.14.2
उ॒भा हि वाꣳ॑ सु॒हवा॒ जोह॑वीमि॒ ता वाजꣳ॑ स॒द्य उ॑श॒ते धेष्ठा᳚। व॒यमु॑ त्वा पथस्पते॒ रथं॒ न वाज॑सातये। धि॒ये पू॑षन्नयुज्महि। प॒थस्प॑थः॒ परि॑पतिं वच॒स्या कामे॑न कृ॒तो अ॒भ्या॑नड॒र्कम्। स नो॑ रासच्छु॒रुध॑श्च॒न्द्राग्रा॒ धियं॑ धियꣳ सीषधाति॒ प्र पू॒षा। क्षेत्र॑स्य॒ पति॑ना व॒यꣳ हि॒तेने॑व जयामसि। गामश्वं॑ पोषयि॒त्न्वा स नो॑॥२६॥

%1.1.14.3
मृडाती॒दृशे᳚। क्षेत्र॑स्य पते॒ मधु॑मन्तमू॒र्मिं धे॒नुरि॑व॒ पयो॑ अ॒स्मासु॑ धुक्ष्व। म॒धु॒श्चुतं॑ घृ॒तमि॑व॒ सुपू॑तमृ॒तस्य॑ न॒ पत॑यो मृडयन्तु। अग्ने॒ नय॑ सु॒पथा॑ रा॒ये अ॒स्मान् विश्वा॑नि देव व॒युना॑नि वि॒द्वान्। यु॒यो॒ध्य॑स्मज्जु॑हुरा॒णमेनो॒ भूयि॑ष्ठान्ते॒ नम॑ उक्तिं विधेम। आ दे॒वाना॒मपि॒ पन्था॑मगन्म॒ यच्छ॒क्नवा॑म॒ तदनु॒ प्रवो॑ढुम्। अ॒ग्निर्वि॒द्वान्थ्स य॑जा॒थ्॥२७॥

%1.1.14.4
सेदु॒ होता॒ सो अ॑ध्व॒रान्थ्स ऋ॒तून्क॑ल्पयाति। यद्वा\-हि॑ष्ठं॒ तद॒ग्नये॑ बृ॒हद॑र्च विभावसो। महि॑षीव॒ त्वद्र॒यिस्त्वद्वाजा॒ उदी॑रते। अग्ने॒ त्वं पा॑रया॒ नव्यो॑ अ॒स्मान्थ्स्व॒स्तिभि॒रति॑ दु॒र्गाणि॒ विश्वा᳚। पूश्च॑ पृ॒थ्वी ब॑हु॒ला न॑ उ॒र्वी भवा॑ तो॒काय॒ तन॑याय॒ शं योः। त्वम॑ग्ने व्रत॒पा अ॑सि दे॒व आ मर्त्ये॒ष्वा। त्वं य॒ज्ञेष्वीड्यः॑। यद्वो॑ व॒यं प्र॑मि॒नाम॑ व्र॒तानि॑ वि॒दुषां᳚ देवा॒ अवि॑दुष्टरासः। अ॒ग्निष्टद्विश्व॒मा पृ॑णाति वि॒द्वान् येभि॑र्दे॒वाꣳ ऋ॒तुभिः॑ क॒ल्पया॑ति॥२८॥

{\anuvakamend[{जु॒षेथा॒मा स नो॑ यजा॒दा त्रयो॑विꣳशतिश्च}]}
%1.1.1.0

{\prashnaend[{इ॒षे त्वा॑ य॒ज्ञस्य॒ शुन्ध॑ध्वं॒ कर्म॑णे दे॒वो\-ऽव॑धूत॒न्धृष्टिः॒ सं व॑पा॒म्या द॑दे॒
प्रत्यु॑ष्टं॒ कृष्णो॑\-ऽसि॒ भुव॑नमसि॒ वाज॑स्यो॒भा वां॒ चतु॑र्दश॥14॥ इ॒षे दृꣳ॑ह॒ भुव॑नम॒ष्टाविꣳ॑शतिः॥28॥ इ॒षे त्वा॑ क॒ल्पया॑ति॥}]}

%%% END PRASHNA

\sect{द्वितीयः प्रश्नः}\setcounter{anuvakam}{0}
\dnsub{तैत्तिरीयसंहितायां प्रथमकाण्डे द्वितीयः प्रश्नः}
%1.2.1.0
%1.2.1.1
आप॑ उन्दन्तु जी॒वसे॑ दीर्घायु॒त्वाय॒ वर्च॑स॒ ओष॑धे॒ त्राय॑स्वैन॒ꣴ॒ स्वधि॑ते॒ मैनꣳ॑ हिꣳसीर्देव॒श्रूरे॒तानि॒ प्र व॑पे स्व॒स्त्युत्त॑राण्यशी॒या\-ऽ\-ऽपो॑ अ॒स्मान्मा॒तरः॑ शुन्धन्तु घृ॒तेन॑ नो घृत॒पुवः॑ पुनन्तु॒ विश्व॑म॒स्मत्प्र व॑हन्तु रि॒प्रमुदा᳚भ्यः॒ शुचि॒रा पू॒त ए॑मि॒ सोम॑स्य त॒नूर॑सि त॒नुवं॑ मे पाहि मही॒नां पयो॑\-ऽसि वर्चो॒धा अ॑सि॒ वर्चो॒॥१॥

%1.2.1.2
मयि॑ धेहि वृ॒त्रस्य॑ क॒नीनि॑का\-ऽसि चक्षु॒ष्पा अ॑सि॒ चक्षु॑र्मे पाहि चि॒त्पति॑स्त्वा पुनातु वा॒क्पति॑स्त्वा पुनातु दे॒वस्त्वा॑ सवि॒ता पु॑ना॒त्वच्छि॑द्रेण प॒वित्रे॑ण॒ वसोः॒ सूर्य॑स्य र॒श्मिभि॒स्तस्य॑ ते पवित्रपते प॒वित्रे॑ण॒ यस्मै॒ कं पु॒ने तच्छ॑केय॒मा वो॑ देवास ईमहे॒ सत्य॑धर्माणो अध्व॒रे यद्वो॑ देवास आगु॒रे यज्ञि॑यासो॒ हवा॑मह॒ इन्द्रा᳚ग्नी॒ द्यावा॑पृथिवी॒ आप॑ ओषधी॒स्त्वं दी॒क्षाणा॒मधि॑पतिरसी॒ह मा॒ सन्तं॑ पाहि॥२॥

%1.2.2.0
{\anuvakamend[{वर्च॑ ओषधीर॒ष्टौ च॑}]}%॥१॥

%1.2.2.1
आकू᳚त्यै प्र॒युजे॒\-ऽग्नये॒ स्वाहा॑ मे॒धायै॒ मन॑से॒\-ऽग्नये॒ स्वाहा॑ दी॒क्षायै॒ तप॑से॒\-ऽग्नये॒ स्वाहा॒ सर॑स्वत्यै पू॒ष्णे᳚\-ऽग्नये॒ स्वाहा\-ऽ\-ऽपो॑ देवीर्बृहतीर्विश्वशम्भुवो॒ द्यावा॑पृथि॒वी उ॒र्व॑न्तरि॑क्षं॒ बृह॒स्पति॑र्नो ह॒विषा॑ वृधातु॒ स्वाहा॒ विश्वे॑ दे॒वस्य॑ ने॒तुर्मर्तो॑\-ऽवृणीत स॒ख्यं विश्वे॑ रा॒य इ॑षुध्यसि द्यु॒म्नं वृ॑णीत पु॒ष्यसे॒ स्वाह॑र्ख्सा॒मयोः॒ शिल्पे᳚ स्थ॒स्ते वा॒मा र॑भे॒ ते मा॑॥३॥

%1.2.2.2
पात॒मा\-ऽस्य य॒ज्ञस्यो॒दृच॑ इ॒मां धिय॒ꣳ॒ शिक्ष॑माणस्य देव॒ क्रतुं॒ दक्षं॑ वरुण॒ सꣳशि॑शाधि॒ यया\-ऽति॒ विश्वा॑ दुरि॒ता तरे॑म सु॒तर्मा॑ण॒मधि॒ नावꣳ॑ रुहे॒मोर्ग॑स्याङ्गिर॒स्यूर्ण॑म्रदा॒ ऊर्जं॑ मे यच्छ पा॒हि मा॒ मा मा॑ हिꣳसी॒र्विष्णोः॒ शर्मा॑सि॒ शर्म॒ यज॑मानस्य॒ शर्म॑ मे यच्छ॒ नक्ष॑त्राणां मा\-ऽतीका॒शात् पा॒हीन्द्र॑स्य॒ योनि॑रसि॒॥४॥

%1.2.2.3
मा मा॑ हिꣳसीः कृ॒ष्यै त्वा॑ सुस॒स्यायै॑ सुपिप्प॒लाभ्य॒स्त्वौष॑\-धीभ्यः सूप॒स्था दे॒वो वन॒स्पति॑रू॒र्ध्वो मा॑ पा॒ह्योदृचः॒ स्वाहा॑ य॒ज्ञं मन॑सा॒ स्वाहा॒ द्यावा॑पृथि॒वीभ्या॒ꣴ॒ स्वाहो॒रोर॒न्तरि॑क्षा॒थ्स्वाहा॑ य॒ज्ञं वाता॒दा र॑भे॥५॥

%1.2.3.0
{\anuvakamend[{मा॒ योनि॑रसि त्रि॒ꣳ॒शच्च॑}]}%॥२॥

%1.2.3.1
दैवीं॒ धियं॑ मनामहे सुमृडी॒काम॒भिष्ट॑ये वर्चो॒धां य॒ज्ञवा॑हसꣳ सुपा॒रा नो॑ अस॒द्वशे᳚। ये दे॒वा मनो॑जाता मनो॒युजः॑ सु॒दक्षा॒ दक्ष॑पितार॒स्ते नः॑ पान्तु॒ ते नो॑\-ऽवन्तु॒ तेभ्यो॒ नम॒स्तेभ्यः॒ स्वाहा\-ऽग्ने॒ त्वꣳ सु जा॑गृहि व॒यꣳ सु म॑न्दिषीमहि गोपा॒य नः॑ स्व॒स्तये᳚ प्र॒बुधे॑ नः॒ पुन॑र्ददः। त्वम॑ग्ने व्रत॒पा अ॑सि दे॒व आ मर्त्ये॒ष्वा। त्वं॥६॥

%1.2.3.2
य॒ज्ञेष्वीड्यः॑॥ विश्वे॑ दे॒वा अ॒भि मा मा\-ऽव॑वृत्रन् पू॒षा स॒न्या सोमो॒ राध॑सा दे॒वः स॑वि॒ता वसो᳚र्वसु॒दावा॒ रास्वेय॑थ्सो॒मा\-ऽ\-ऽभूयो॑ भर॒ मा पृ॒णन्पू॒र्त्या वि रा॑धि॒ मा\-ऽहमायु॑षा च॒न्द्रम॑सि॒ मम॒ भोगा॑य भव॒ वस्त्र॑मसि॒ मम॒ भोगा॑य भवो॒स्रा\-ऽसि॒ मम॒ भोगा॑य भव॒ हयो॑\-ऽसि॒ मम॒ भोगा॑य भव॒॥७॥

%1.2.3.3
छागो॑\-ऽसि॒ मम॒ भोगा॑य भव मे॒षो॑\-ऽसि॒ मम॒ भोगा॑य भव वा॒यवे᳚ त्वा॒ वरु॑णाय त्वा॒ निर्\mbox{}ऋ॑त्यै त्वा रु॒द्राय॑ त्वा॒ देवी॑रापो अपां नपा॒द्य ऊ॒र्मिर्\mbox{}ह॑वि॒ष्य॑ इन्द्रि॒यावा᳚न्म॒दिन्त॑म॒स्तं वो॒ मा\-ऽव॑क्रमिष॒मच्छि॑न्नं॒ तन्तुं॑ पृथि॒व्या अनु॑ गेषं भ॒द्राद॒भि श्रेयः॒ प्रेहि॒ बृह॒स्पतिः॑ पुरए॒ता ते॑ अ॒स्त्वथे॒मव॑ स्य॒ वर॒ आ पृ॑थि॒व्या आ॒रे शत्रू᳚न् कृणुहि॒ सर्व॑वीर॒ एदम॑गन्म देव॒यज॑नं पृथि॒व्या विश्वे॑ दे॒वा यदजु॑षन्त॒ पूर्व॑ ऋख्सा॒माभ्यां॒ यजु॑षा स॒न्तर॑न्तो रा॒यस्पोषे॑ण॒ समि॒षा म॑देम॥८॥

%1.2.4.0
{\anuvakamend[{आ त्वꣳ हयो॑\-ऽसि॒ मम॒ भोगा॑य भव स्य॒ पञ्च॑विꣳशतिश्च}]}%॥३॥

%1.2.4.1
इ॒यं ते॑ शुक्र त॒नूरि॒दं वर्च॒स्तया॒ सं भ॑व॒ भ्राजं॑ गच्छ॒ जूर॑सि धृ॒ता मन॑सा॒ जुष्टा॒ विष्ण॑वे॒ तस्या᳚स्ते स॒त्यस॑वसः प्रस॒वे वा॒चो य॒न्त्रम॑शीय॒ स्वाहा॑ शु॒क्रम॑स्य॒मृत॑मसि वैश्वदे॒वꣳ ह॒विः सूर्य॑स्य॒ चक्षु॒रा\-ऽरु॑हम॒ग्नेर॒क्ष्णः क॒नीनि॑कां॒ यदेत॑शेभि॒रीय॑से॒ भ्राज॑मानो विप॒श्चिता॒ चिद॑सि म॒ना\-ऽसि॒ धीर॑सि॒ दक्षि॑णा-॥९॥

%1.2.4.2
ऽसि य॒ज्ञिया॑\-ऽसि क्ष॒त्रिया॒\-ऽस्यदि॑तिरस्युभ॒यतः॑शीर्ष्णी॒ सा नः॒ सुप्रा॑ची॒ सुप्र॑तीची॒ सं भ॑व मि॒त्रस्त्वा॑ प॒दि ब॑ध्नातु पू॒षा\-ऽध्व॑नः पा॒त्विन्द्रा॒याध्य॑क्षा॒यानु॑ त्वा मा॒ता म॑न्यता॒मनु॑ पि॒ता\-ऽनु॒ भ्राता॒ सग॒र्भ्यो\-ऽनु॒ सखा॒ सयू᳚थ्यः॒ सा दे॑वि दे॒वमच्छे॒हीन्द्रा॑य॒ सोमꣳ॑ रु॒द्रस्त्वा\-ऽ\-ऽव॑र्तयतु मि॒त्रस्य॑ प॒था स्व॒स्ति सोम॑सखा॒ पुन॒रेहि॑ स॒ह र॒य्या॥१०॥

%1.2.5.0
{\anuvakamend[{दक्षि॑णा॒ सोम॑सखा॒ पञ्च॑ च}]}%॥४॥

%1.2.5.1
वस्व्य॑सि रु॒द्रा\-ऽस्यदि॑तिरस्यादि॒त्या\-ऽसि॑ शु॒क्रा\-ऽसि॑ च॒न्द्रा\-ऽसि॒ बृह॒स्पति॑स्त्वा सु॒म्ने र॑ण्वतु रु॒द्रो वसु॑भि॒रा चि॑केतु पृथि॒व्यास्त्वा॑ मू॒र्धन्ना जि॑घर्मि देव॒यज॑न॒ इडा॑याः प॒दे घृ॒तव॑ति॒ स्वाहा॒ परि॑लिखित॒ꣳ॒ रक्षः॒ परि॑लिखिता॒ अरा॑तय इ॒दम॒हꣳ रक्ष॑सो ग्री॒वा अपि॑ कृन्तामि॒ यो᳚\-ऽस्मान् द्वेष्टि॒ यं च॑ व॒यं द्वि॒ष्म इ॒दम॑स्य ग्री॒वा॥११॥

%1.2.5.2
अपि॑ कृन्ताम्य॒स्मे राय॒स्त्वे राय॒स्तोते॒ रायः॒ सं दे॑वि दे॒व्योर्वश्या॑ पश्यस्व॒ त्वष्टी॑मती ते सपेय सु॒रेता॒ रेतो॒ दधा॑ना वी॒रं वि॑देय॒ तव॑ स॒न्दृशि॒ मा\-ऽहꣳ रा॒यस्पोषे॑ण॒ वि यो॑षम्॥१२॥

%1.2.6.0
{\anuvakamend[{अ॒स्य॒ ग्री॒वा एका॒न्नत्रि॒ꣳ॒शच्च॑}]}%॥५॥

%1.2.6.1
अ॒ꣳ॒शुना॑ ते अ॒ꣳ॒शुः पृ॑च्यतां॒ परु॑षा॒ परु॑र्ग॒न्धस्ते॒ काम॑मवतु॒ मदा॑य॒ रसो॒ अच्यु॑तो॒\-ऽमात्यो॑\-ऽसि शु॒क्रस्ते॒ ग्रहो॒\-ऽभि त्यं दे॒वꣳ स॑वि॒तार॑मू॒ण्योः᳚ क॒विक्र॑तु॒मर्चा॑मि स॒त्यस॑वसꣳ रत्न॒धाम॒भि प्रि॒यं म॒तिमू॒र्ध्वा यस्या॒मति॒र्भा अदि॑द्युत॒थ्सवी॑मनि॒ हिर॑ण्यपाणिरमिमीत सु॒क्रतुः॑ कृ॒पा सुवः॑। प्र॒जाभ्य॑स्त्वा प्रा॒णाय॑ त्वा व्या॒नाय॑ त्वा प्र॒जास्त्वमनु॒ प्राणि॑हि प्र॒जास्त्वामनु॒ प्राण॑न्तु॥१३॥

%1.2.7.0
{\anuvakamend[{अनु॑ स॒प्त च॑}]}%॥६॥

%1.2.7.1
सोमं॑ ते क्रीणा॒म्यूर्ज॑स्वन्तं॒ पय॑स्वन्तं वी॒र्या॑वन्तमभिमाति॒\-षाहꣳ॑ शु॒क्रं ते॑ शु॒क्रेण॑ क्रीणामि च॒न्द्रं च॒न्द्रेणा॒मृत॑म॒मृते॑न स॒म्यत्ते॒ गोर॒स्मे च॒न्द्राणि॒ तप॑सस्त॒नूर॑सि प्र॒जाप॑ते॒र्वर्ण॒स्तस्या᳚स्ते सहस्रपो॒षं पुष्य॑न्त्याश्चर॒मेण॑ प॒शुना᳚ क्रीणाम्य॒स्मे ते॒ बन्धु॒र्मयि॑ ते॒ रायः॑ श्रयन्ताम॒स्मे ज्योतिः॑ सोमविक्र॒यिणि॒ तमो॑ मि॒त्रो न॒ एहि॒ सुमि॑त्रधा॒ इन्द्र॑स्यो॒रु मा वि॑श॒ दक्षि॑णमु॒शन्नु॒शन्तꣴ॑ स्यो॒नः स्यो॒नꣴ स्वान॒ भ्राजाङ्घा॑रे॒ बम्भा॑रे॒ हस्त॒ सुह॑स्त॒ कृशा॑नवे॒ते वः॑ सोम॒क्रय॑णा॒स्तान्र॑क्षध्वं॒ मा वो॑ दभन्न्॥१४॥

%1.2.8.0
{\anuvakamend[{ऊ॒रुं द्वाविꣳ॑शतिश्व}]}%॥७॥

%1.2.8.1
उदायु॑षा स्वा॒युषोदोष॑धीना॒ꣳ॒ रसे॒नोत्प॒र्जन्य॑स्य॒ शुष्मे॒णोद॑स्थाम॒मृता॒ꣳ॒ अनु॑। उ॒र्व॑न्तरि॑क्ष॒मन्वि॒ह्यदि॑त्याः॒ सदो॒\-ऽस्यदि॑त्याः॒ सद॒ आसी॒दास्त॑भ्ना॒द्द्यामृ॑ष॒भो अ॒न्तरि॑क्ष॒ममि॑मीत वरि॒माणं॑ पृथि॒व्या आसी॑द॒द्विश्वा॒ भुव॑नानि स॒म्राड्विश्वेत्तानि॒ वरु॑णस्य व्र॒तानि॒ वने॑षु॒ व्य॑न्तरि॑क्षं ततान॒ वाज॒मर्व॑थ्सु॒ पयो॑ अघ्नि॒यासु॑ हृ॒थ्सु॥१५॥

%1.2.8.2
क्रतुं॒ वरु॑णो वि॒क्ष्व॑ग्निं दि॒वि सूर्य॑मदधा॒थ्सोम॒मद्रा॒वुदु॒त्यं जा॒तवे॑दसं दे॒वं व॑हन्ति के॒तवः॑। दृ॒शे विश्वा॑य॒ सूर्यम्᳚॥ उस्रा॒वेतं॑ धूर्\mbox{}षाहावन॒श्रू अवी॑रहणौ ब्रह्म॒चोद॑नौ॒ वरु॑णस्य॒ स्कम्भ॑नमसि॒ वरु॑णस्य स्कम्भ॒सर्ज॑नमसि॒ प्रत्य॑स्तो॒ वरु॑णस्य॒ पाशः॑॥१६॥

%1.2.9.0
{\anuvakamend[{हृ॒थ्सु पञ्च॑त्रिꣳशच्च}]}%॥८॥

%1.2.9.1
प्रच्य॑वस्व भुवस्पते॒ विश्वा᳚न्य॒भि धामा॑नि॒ मा त्वा॑ परिप॒री वि॑द॒न्मा त्वा॑ परिप॒न्थिनो॑ विद॒न्मा त्वा॒ वृका॑ अघा॒यवो॒ मा ग॑न्ध॒र्वो वि॒श्वाव॑सु॒रा द॑घच्छ्ये॒नो भू॒त्वा परा॑ पत॒ यज॑मानस्य नो गृ॒हे दे॒वैः सꣴ॑स्कृ॒तं यज॑मानस्य स्व॒स्त्यय॑न्य॒स्यपि॒ पन्था॑मगस्महि स्वस्ति॒गाम॑ने॒हसं॒ येन॒ विश्वाः॒ परि॒ द्विषो॑ वृ॒णक्ति॑ वि॒न्दते॒ वसु॒ नमो॑ मि॒त्रस्य॒ वरु॑णस्य॒ चक्ष॑से म॒हो दे॒वाय॒ तदृ॒तꣳ स॑पर्यत दूरे॒दृशे॑ दे॒वजा॑ताय के॒तवे॑ दि॒वस्पु॒त्राय॒ सूर्या॑य शꣳसत॒ वरु॑णस्य॒ स्कम्भ॑नमसि॒ वरु॑णस्य स्कम्भ॒सर्ज॑नम॒स्युन्मु॑क्तो॒ वरु॑णस्य॒ पाशः॑॥१७॥

%1.2.10.0
{\anuvakamend[{मि॒त्रस्य॒ त्रयो॑विꣳशतिश्च}]}%॥९॥

%1.2.10.1
अ॒ग्नेरा॑ति॒थ्यम॑सि॒ विष्ण॑वे त्वा॒ सोम॑स्या\-ऽ\-ऽति॒थ्यम॑सि॒ विष्ण॑वे॒ त्वा\-ऽति॑थेराति॒थ्यम॑सि॒ विष्ण॑वे त्वा॒\-ऽग्नये᳚ त्वा रायस्पोष॒दाव्न्ने॒ विष्ण॑वे त्वा श्ये॒नाय॑ त्वा सोम॒भृते॒ विष्ण॑वे त्वा॒ या ते॒ धामा॑नि ह॒विषा॒ यज॑न्ति॒ ता ते॒ विश्वा॑ परि॒भूर॑स्तु य॒ज्ञं ग॑य॒स्फानः॑ प्र॒तर॑णः सु॒वीरो\-ऽवी॑रहा॒ प्र च॑रा सोम॒ दुर्या॒नदि॑त्याः॒ सदो॒\-ऽस्यदि॑त्याः॒ सद॒ आ॥१८॥

%1.2.10.2
सी॑द॒ वरु॑णो\-ऽसि धृ॒तव्र॑तो वारु॒णम॑सि शं॒योर्दे॒वानाꣳ॑ स॒ख्यान्मा दे॒वाना॑म॒पस॑श्छिथ्स्म॒ह्याप॑तये त्वा गृह्णामि॒ परि॑पतये त्वा गृह्णामि॒ तनू॒नप्त्रे᳚ त्वा गृह्णामि शाक्व॒राय॑ त्वा गृह्णामि॒ शक्म॒न्नोजि॑ष्ठाय त्वा गृह्णा॒म्यना॑धृष्टम\-स्यनाधृ॒ष्यं दे॒वाना॒मोजो॑\-ऽभिशस्ति॒पा अ॑नभिशस्ते॒\-ऽन्यमनु॑ मे दी॒क्षां दी॒क्षाप॑तिर्मन्यता॒मनु॒ तप॒स्तप॑स्पति॒रञ्ज॑सा स॒त्यमुप॑ गेषꣳ सुवि॒ते मा॑ धाः॥१९॥

%1.2.11.0
{\anuvakamend[{आ मैकं॑ च}]}%॥10॥

%1.2.11.1
अ॒ꣳ॒शुरꣳ॑शुस्ते देव सो॒मा\-ऽ\-ऽप्या॑यता॒मिन्द्रा॑यैकधन॒विद॒ आ तुभ्य॒मिन्द्रः॑ प्यायता॒मा त्वमिन्द्रा॑य प्याय॒स्वा\-ऽ\-ऽ प्या॑यय॒ सखी᳚न्थ्स॒न्या मे॒धया᳚ स्व॒स्ति ते॑ देव सोम सु॒त्याम॑शी॒येष्टा॒ रायः॒ प्रेषे भगा॑य॒र्तमृ॑तवा॒दिभ्यो॒ नमो॑ दि॒वे नमः॑ पृथि॒व्या अग्ने᳚ व्रतपते॒ त्वं व्र॒तानां᳚ व्र॒तप॑तिरसि॒ या मम॑ त॒नूरे॒षा सा त्वयि॒॥२०॥

%1.2.11.2
या तव॑ त॒नूरि॒यꣳ सा मयि॑ स॒ह नौ᳚ व्रतपते व्र॒तिनो᳚र्व्र॒तानि॒ या ते॑ अग्ने॒ रुद्रि॑या त॒नूस्तया॑ नः पाहि॒ तस्या᳚स्ते॒ स्वाहा॒ या ते॑ अग्ने\-ऽयाश॒या र॑जाश॒या ह॑राश॒या त॒नूर्वर्\mbox{}षि॑ष्ठा गह्वरे॒ष्ठोग्रं वचो॒ अपा॑वधीं त्वे॒षं वचो॒ अपा॑वधी॒ꣴ॒ स्वाहा᳚॥२१॥

%1.2.12.0
{\anuvakamend[{त्वयि॑ चत्वारि॒ꣳ॒शच्च॑}]}%॥11॥

%1.2.12.1
वि॒त्ताय॑नी मे\-ऽसि ति॒क्ताय॑नी मे॒\-ऽस्यव॑तान्मा नाथि॒तमव॑तान्मा व्यथि॒तं वि॒देर॒ग्निर्नभो॒ नामाग्ने॑ अङ्गिरो॒ यो᳚\-ऽस्यां पृ॑थि॒व्यामस्यायु॑षा॒ नाम्नेहि॒ यत्ते\-ऽना॑धृष्टं॒ नाम॑ य॒ज्ञियं॒ तेन॒ त्वा\-ऽ\-ऽद॒धे\-ऽग्ने॑ अङ्गिरो॒ यो द्वि॒तीय॑स्यां तृ॒तीय॑स्यां पृथि॒व्यामस्यायु॑षा॒ नाम्नेहि॒ यत्ते\-ऽना॑धृष्टं॒ नाम॑॥२२॥

%1.2.12.2
य॒ज्ञियं॒ तेन॒ त्वा\-ऽ\-ऽद॑धे सि॒ꣳ॒हीर॑सि महि॒षीर॑स्यु॒रु प्र॑थस्वो॒रु ते॑ य॒ज्ञप॑तिः प्रथतां ध्रु॒वा\-ऽसि॑ दे॒वेभ्यः॑ शुन्धस्व दे॒वेभ्यः॑ शुम्भस्वेन्द्रघो॒षस्त्वा॒ वसु॑भिः पु॒रस्ता᳚त्पातु॒ मनो॑जवास्त्वा पि॒तृभि॑र्दक्षिण॒तः पा॑तु॒ प्रचे॑तास्त्वा रु॒द्रैः प॒श्चात्पा॑तु वि॒श्वक॑र्मा त्वा\-ऽ\-ऽदि॒त्यैरु॑त्तर॒तः पा॑तु सि॒ꣳ॒हीर॑सि सपत्नसा॒ही स्वाहा॑ सि॒ꣳ॒हीर॑सि सुप्रजा॒वनिः॒ स्वाहा॑ सि॒ꣳ॒ही-॥२३॥

%1.2.12.3
-र॑सि रायस्पोष॒वनिः॒ स्वाहा॑ सि॒ꣳ॒हीर॑स्यादित्य॒वनिः॒ स्वाहा॑ सि॒ꣳ॒हीर॒स्या व॑ह दे॒वान्दे॑वय॒ते यज॑मानाय॒ स्वाहा॑ भू॒तेभ्य॑स्त्वा वि॒श्वायु॑रसि पृथि॒वीं दृꣳ॑ह ध्रुव॒क्षिद॑स्य॒न्तरि॑क्षं दृꣳहाच्युत॒क्षिद॑सि॒ दिवं॑ दृꣳहा॒ग्नेर्भस्मा᳚स्य॒ग्नेः पुरी॑षमसि॥२४॥

%1.2.13.0
{\anuvakamend[{नाम॑ सुप्रजा॒वनिः॒ स्वाहा॑ सि॒ꣳ॒हीः पञ्च॑त्रिꣳशच्च}]}%॥12॥

%1.2.13.1
यु॒ञ्जते॒ मन॑ उ॒त यु॑ञ्जते॒ धियो॒ विप्रा॒ विप्र॑स्य बृह॒तो वि॑प॒श्चितः॑। वि होत्रा॑ दधे वयुना॒विदेक॒ इन्म॒ही दे॒वस्य॑ सवि॒तुः परि॑ष्टुतिः॥ सु॒वाग्दे॑व॒ दुर्या॒ꣳ॒ आ व॑द देव॒श्रुतौ॑ दे॒वेष्वा घो॑षेथा॒मा नो॑ वी॒रो जा॑यतां कर्म॒ण्यो॑ यꣳ सर्वे॑\-ऽनु॒जीवा॑म॒ यो ब॑हू॒नामस॑द्व॒शी। इ॒दं विष्णु॒र्वि च॑क्रमे त्रे॒धा नि द॑धे प॒दम्। समू॑ढमस्य॥२५॥

%1.2.13.2
पाꣳसु॒र इरा॑वती धेनु॒मती॒ हि भू॒तꣳ सू॑यव॒सिनी॒ मन॑वे यश॒स्ये᳚। व्य॑स्कभ्ना॒द्रोद॑सी॒ विष्णु॑रे॒ते दा॒धार॑ पृथि॒वीम॒भितो॑ म॒यूखैः᳚॥ प्राची॒ प्रेत॑मध्व॒रं क॒ल्पय॑न्ती ऊ॒र्ध्वं य॒ज्ञं न॑यतं॒ मा जी᳚ह्वरत॒मत्र॑ रमेथां॒ वर्ष्म॑न्पृथि॒व्या दि॒वो वा॑ विष्णवु॒त वा॑ पृथि॒व्या म॒हो वा॑ विष्णवु॒त वा॒\-ऽन्तरि॑क्षा॒द्धस्तौ॑ पृणस्व ब॒हुभि॑र्वस॒व्यै॑रा प्र य॑च्छ॒॥२६॥

%1.2.13.3
दक्षि॑णा॒दोत स॒व्यात्। विष्णो॒र्नुकं॑ वी॒र्या॑णि॒ प्र वो॑चं॒ यः पार्थि॑वानि विम॒मे रजाꣳ॑सि॒ यो अस्क॑भाय॒दुत्त॑रꣳ स॒धस्थं॑ विचक्रमा॒णस्त्रे॒धोरु॑गा॒यो विष्णो॑ र॒राट॑मसि॒ विष्णोः᳚ पृ॒ष्ठम॑सि॒ विष्णोः॒ श्ञप्त्रे᳚ स्थो॒ विष्णोः॒ स्यूर॑सि॒ विष्णो᳚र्ध्रु॒वम॑सि वैष्ण॒वम॑सि॒ विष्ण॑वे त्वा॥२७॥

%1.2.14.0
{\anuvakamend[{अ॒स्य॒ य॒च्छैका॒न्नच॑त्वारि॒ꣳ॒शच्च॑}]}%॥13॥

%1.2.14.1
कृ॒णु॒ष्व पाजः॒ प्रसि॑तिं॒ न पृ॒थ्वीं या॒हि राजे॒वाम॑वा॒ꣳ॒ इभे॑न। तृ॒ष्वीमनु॒ प्रसि॑तिं द्रूणा॒नो\-ऽस्ता॑सि॒ विध्य॑ र॒क्षस॒स्तपि॑ष्ठैः॥ तव॑ भ्र॒मास॑ आशु॒या प॑त॒न्त्यनु॑ स्पृश धृष॒ता शोशु॑चानः। तपूꣴ॑ष्यग्ने जु॒ह्वा॑ पत॒ङ्गानस॑न्दितो॒ वि सृ॑ज॒ विष्व॑गु॒ल्काः॥ प्रति॒ स्पशो॒ वि सृ॑ज॒ तूर्णि॑तमो॒ भवा॑ पा॒युर्वि॒शो अ॒स्या अद॑ब्धः। यो नो॑ दू॒रे अ॒घशꣳ॑सो॒॥२८॥

%1.2.14.2
यो अन्त्यग्ने॒ माकि॑ष्टे॒ व्यथि॒रा द॑धर्षीत्। उद॑ग्ने तिष्ठ॒ प्रत्या\-ऽ\-ऽत॑नुष्व॒ न्य॑मित्राꣳ॑ ओषतात्तिग्महेते। यो नो॒ अरा॑तिꣳ समिधान च॒क्रे नी॒चा तं ध॑क्ष्यत॒सं न शुष्कम्᳚॥ ऊ॒र्ध्वो भ॑व॒ प्रति॑ वि॒ध्याध्य॒स्मदा॒विष्कृ॑णुष्व॒ दैव्या᳚न्यग्ने। अव॑ स्थि॒रा त॑नुहि यातु॒जूनां᳚ जा॒मिमजा॑मिं॒ प्र मृ॑णीहि॒ शत्रून्॑॥ स ते॑॥२९॥

%1.2.14.3
जानाति सुम॒तिं य॑विष्ठ॒ य ईव॑ते॒ ब्रह्म॑णे गा॒तुमैर॑त्। विश्वा᳚न्यस्मै सु॒दिना॑नि रा॒यो द्यु॒म्नान्य॒र्यो वि दुरो॑ अ॒भि द्यौ᳚त्॥ सेद॑ग्ने अस्तु सु॒भगः॑ सु॒दानु॒र्यस्त्वा॒ नित्ये॑न ह॒विषा॒ य उ॒क्थैः। पिप्री॑षति॒ स्व आयु॑षि दुरो॒णे विश्वेद॑स्मै सु॒दिना॒ सा\-ऽस॑दि॒ष्टिः॥ अर्चा॑मि ते सुम॒तिं घोष्य॒र्वाख्सं ते॑ वा॒वाता॑ जरता-॥३०॥

%1.2.14.4
मि॒यङ्गीः। स्वश्वा᳚स्त्वा सु॒रथा॑ मर्जयेमा॒स्मे क्ष॒त्राणि॑ धारये॒रनु॒ द्यून्॥ इ॒ह त्वा॒ भूर्या च॑रे॒दुप॒ त्मन्दोषा॑वस्तर्दीदि॒वाꣳ\-स॒मनु॒ द्यून्। कीड॑न्तस्त्वा सु॒मन॑सः सपेमा॒भि द्यु॒म्ना त॑स्थि॒वाꣳसो॒ जना॑नाम्॥ यस्त्वा॒ स्वश्वः॑ सुहिर॒ण्यो अ॑ग्न उप॒याति॒ वसु॑मता॒ रथे॑न। तस्य॑ त्रा॒ता भ॑वसि॒ तस्य॒ सखा॒ यस्त॑ आति॒थ्यमा॑नु॒षग्जुजो॑षत्॥ म॒हो रु॑जामि॥३१॥

%1.2.14.5
ब॒न्धुता॒ वचो॑भि॒स्तन्मा॑ पि॒तुर्गोत॑मा॒दन्वि॑याय॥ त्वं नो॑ अ॒स्य वच॑सश्चिकिद्धि॒ होत॑र्यविष्ठ सुक्रतो॒ दमू॑नाः॥ अस्व॑प्नजस्त॒रण॑यः सु॒शेवा॒ अत॑न्द्रासो\-ऽवृ॒का अश्र॑मिष्ठाः। ते पा॒यवः॑ स॒ध्रिय॑ञ्चो नि॒षद्या\-ऽग्ने॒ तव॑ नः पान्त्वमूर॥ ये पा॒यवो॑ मामते॒यं ते॑ अग्ने॒ पश्य॑न्तो अ॒न्धं दु॑रि॒तादर॑क्षन्। र॒रक्ष॒ तान्थ्सु॒कृतो॑ वि॒श्ववे॑दा॒ दिफ्स॑न्त॒ इद्रि॒पवो॒ ना ह॑॥३२॥

%1.2.14.6
देभुः॥ त्वया॑ व॒यꣳ स॑ध॒न्य॑स्त्वोता॒स्तव॒ प्रणी᳚त्यश्याम॒ वाजान्॑। उ॒भा शꣳसा॑ सूदय सत्यताते\-ऽनुष्ठु॒या कृ॑णुह्यह्रयाण॥ अ॒या ते॑ अग्ने स॒मिधा॑ विधेम॒ प्रति॒ स्तोमꣳ॑ श॒स्यमा॑नं गृभाय। दहा॒शसो॑ र॒क्षसः॑ पा॒ह्य॑स्मान्द्रु॒हो नि॒दोऽमि॑त्रमहो अव॒द्यात्॥ र॒क्षो॒हणं॑ वा॒जिन॒मा\-ऽ\-ऽजि॑घर्मि मि॒त्रं प्रथि॑ष्ठ॒मुप॑ यामि॒ शर्म॑। शिशा॑नो अ॒ग्निः क्रतु॑भिः॒ समि॑द्धः॒ स नो॒ दिवा॒॥३३॥

%1.2.14.7
स रि॒षः पा॑तु॒ नक्तम्᳚॥ वि ज्योति॑षा बृह॒ता भा᳚त्य॒ग्निरा॒विर्विश्वा॑नि कृणुते महि॒त्वा। प्रादे॑वीर्मा॒याः स॑हते दु॒रेवाः॒ शिशी॑ते॒ शृङ्गे॒ रक्ष॑से वि॒निक्षे᳚॥ उ॒त स्वा॒नासो॑ दि॒विष॑न्त्व॒ग्नेस्ति॒ग्मायु॑धा॒ रक्ष॑से॒ हन्त॒वा उ॑। मदे॑ चिदस्य॒ प्ररु॑जन्ति॒ भामा॒ न व॑रन्ते परि॒बाधो॒ अदे॑वीः॥३४॥

%1.3.0.0
{\anuvakamend[{अ॒घशꣳ॑सः॒ स ते॑ जरताꣳ रुजामि ह॒ दिवैक॑चत्वारिꣳशच्च}]}

{\prashnaend[{दे॒वस्य॑ रक्षो॒हणो॑ वि॒भूस्त्वꣳ सो॒मात्य॒न्यानगां᳚ पृथि॒व्या इ॒षे त्वा\-ऽ\-ऽद॑दे॒ वाक्ते॒ सं ते॑ समु॒द्रꣳ ह॒विष्म॑तीर्\mbox{}हृ॒दे त्वम॑ग्ने रु॒द्रश्चतु॑र्दश॥ दे॒वस्य॑ ग॒मध्ये॑ ह॒विष्म॑तीः पवस॒ एक॑त्रिꣳशत्॥ दे॒वस्या॒र्चयः॑॥}]}
%%% END PRASHNA

\sect{तृतीयः प्रश्नः}\setcounter{anuvakam}{0}
\dnsub{तैत्तिरीयसंहितायां प्रथमकाण्डे तृतीयः प्रश्नः}
%1.3.1.0
%1.3.1.1
दे॒वस्य॑ त्वा सवि॒तुः प्र॑स॒वे᳚\-ऽश्विनो᳚र्बा॒हु\-भ्यां᳚ पू॒ष्णो हस्ता᳚भ्या॒माद॒दे\-ऽभ्रि॑रसि॒ नारि॑रसि॒ परि॑लिखित॒ꣳ॒ रक्षः॒ परि॑लिखिता॒ अरा॑तय इ॒दम॒हꣳ रक्ष॑सो ग्री॒वा अपि॑ कृन्तामि॒ यो᳚\-ऽस्मान् द्वेष्टि॒ यं च॑ व॒यं द्वि॒ष्म इ॒दम॑स्य ग्री॒वा अपि॑ कृन्तामि दि॒वे त्वा॒\-ऽन्तरि॑क्षाय त्वा पृथि॒व्यै त्वा॒ शुन्ध॑तां लो॒कः पि॑तृ॒षद॑नो॒ यवो॑\-ऽसि य॒वया॒स्मद्द्वेषो॑॥१॥

%1.3.1.2
य॒वयारा॑तीः पितृ॒णाꣳ सद॑नम॒स्युद्दिवꣴ॑ स्तभा॒ना\-ऽन्तरि॑क्षं पृण पृथि॒वीं दृꣳ॑ह द्युता॒नस्त्वा॑ मारु॒तो मि॑नोतु मि॒त्रावरु॑णयोर्ध्रु॒वेण॒ धर्म॑णा ब्रह्म॒वनिं॑ त्वा क्षत्र॒वनिꣳ॑ सुप्रजा॒वनिꣳ॑ रायस्पोष॒वनिं॒ पर्यू॑हामि॒ ब्रह्म॑ दृꣳह क्ष॒त्रं दृꣳ॑ह प्र॒जां दृꣳ॑ह रा॒यस्पोषं॑ दृꣳह घृ॒तेन॑ द्यावापृथिवी॒ आ पृ॑णेथा॒मिन्द्र॑स्य॒ सदो॑\-ऽसि विश्वज॒नस्य॑ छा॒या परि॑ त्वा गिर्वणो॒ गिर॑ इ॒मा भ॑वन्तु वि॒श्वतो॑ वृ॒द्धायु॒मनु॒ वृद्ध॑यो॒ जुष्टा॑ भवन्तु॒ जुष्ट॑य॒ इन्द्र॑स्य॒ स्यूर॒सीन्द्र॑स्य ध्रु॒वम॑स्यै॒न्द्रम॒सीन्द्रा॑य त्वा॥२॥

%1.3.2.0
{\anuvakamend[{द्वेष॑ इ॒मा अ॒ष्टाद॑श च}]}%॥१॥

%1.3.2.1
र॒क्षो॒हणो॑ वलग॒हनो॑ वैष्ण॒वान्ख॑नामी॒दम॒हं तं व॑ल॒गमुद्व॑पामि॒ यं नः॑ समा॒नो यमस॑मानो निच॒खाने॒दमे॑न॒मध॑रं करोमि॒ यो नः॑ समा॒नो यो\-ऽस॑मानो\-ऽराती॒यति॑ गाय॒त्रेण॒ छन्द॒सा\-ऽव॑बाढो वल॒गः किमत्र॑ भ॒द्रं तन्नौ॑ स॒ह वि॒राड॑सि सपत्न॒हा स॒म्राड॑सि भ्रातृव्य॒हा स्व॒राड॑स्यभिमाति॒हा वि॑श्वा॒राड॑सि॒ विश्वा॑सां ना॒ष्ट्राणाꣳ॑ ह॒न्ता॥३॥

%1.3.2.2
र॑क्षो॒हणो॑ वलग॒हनः॒ प्रोक्षा॑मि वैष्ण॒वान् र॑क्षो॒हणो॑ वलग॒हनो\-ऽव॑ नयामि वैष्ण॒वान् यवो॑\-ऽसि य॒वया॒स्मद्द्वेषो॑ य॒वयारा॑ती रक्षो॒हणो॑ वलग॒हनो\-ऽव॑ स्तृणामि वैष्ण॒वान् र॑क्षो॒हणो॑ वलग॒हनो॒\-ऽभि जु॑होमि वैष्ण॒वान् र॑क्षो॒हणौ॑ वलग॒हना॒वुप॑ दधामि वैष्ण॒वी र॑क्षो॒हणौ॑ वलग॒हनौ॒ पर्यू॑हामि वैष्ण॒वी र॑क्षो॒हणौ॑ वलग॒हनौ॒ परि॑ स्तृणामि वैष्ण॒वी र॑क्षो॒हणौ॑ वलग॒हनौ॑ वैष्ण॒वी बृ॒हन्न॑सि बृ॒हद्ग्रा॑वा बृह॒तीमिन्द्रा॑य॒ वाचं॑ वद॥४॥

%1.3.3.0
{\anuvakamend[{ह॒न्तेन्द्रा॑य॒ द्वे च॑}]}%॥२॥

%1.3.3.1
वि॒भूर॑सि प्र॒वाह॑णो॒ वह्नि॑रसि हव्य॒वाह॑नः श्वा॒त्रो॑\-ऽसि॒ प्रचे॑तास्तु॒थो॑\-ऽसि वि॒श्ववे॑दा उ॒शिग॑सि क॒विरङ्घा॑रिरसि॒ बम्भा॑रिरव॒स्युर॑सि॒ दुव॑स्वाञ्छु॒न्ध्यूर॑सि मार्जा॒लीयः॑ स॒म्राड॑सि कृ॒शानुः॑ परि॒षद्यो॑\-ऽसि॒ पव॑मानः प्र॒तक्वा॑\-ऽसि॒ नभ॑स्वा॒नस॑म्मृष्टो\-ऽसि हव्य॒सूद॑ ऋ॒तधा॑मा\-ऽसि॒ सुव॑र्ज्योति॒र्ब्रह्म॑ज्योतिरसि॒ सुव॑र्धामा॒\-ऽजो᳚\-ऽस्येक॑पा॒दहि॑रसि बु॒ध्नियो॒ रौद्रे॒णानी॑केन पा॒हि मा᳚\-ऽग्ने पिपृ॒हि मा॒ मा मा॑ हिꣳसीः॥५॥

%1.3.4.0
{\anuvakamend[{अनी॑केना॒ष्टौ च॑}]}%॥३॥

%1.3.4.1
त्वꣳ सो॑म तनू॒कृद्भ्यो॒ द्वेषो᳚भ्यो॒\-ऽन्यकृ॑तेभ्य उ॒रु य॒न्तासि॒ वरू॑थ॒ꣴ॒ स्वाहा॑ जुषा॒णो अ॒प्तुराज्य॑स्य वेतु॒ स्वाहा॒\-ऽयं नो॑ अ॒ग्निर्वरि॑वः कृणोत्व॒यं मृधः॑ पु॒र ए॑तु प्रभि॒न्दन्न्। अ॒यꣳ शत्रू᳚ञ्जयतु॒ जर्\mbox{}हृ॑षाणो॒\-ऽयं वाजं॑ जयतु॒ वाज॑सातौ॥ उ॒रु वि॑ष्णो॒ वि क्र॑मस्वो॒रु क्षया॑य नः कृधि। घृ॒तं घृ॑तयोने पिब॒ प्रप्र॑ य॒ज्ञप॑तिं तिर॥ सोमो॑ जिगाति गातु॒विद्॥६॥

%1.3.4.2
दे॒वाना॑मेति निष्कृ॒तमृ॒तस्य॒ योनि॑मा॒सद॒मदि॑त्याः॒ सदो॒\-ऽस्यदि॑त्याः॒ सद॒ आ सी॑दै॒ष वो॑ देव सवितः॒ सोम॒स्तꣳ र॑क्षध्वं॒ मा वो॑ दभदे॒तत् त्वꣳ सो॑म दे॒वो दे॒वानुपा॑गा इ॒दम॒हं म॑नु॒ष्यो॑ मनु॒ष्या᳚न्थ्स॒ह प्र॒जया॑ स॒ह रा॒यस्पोषे॑ण॒ नमो॑ दे॒वेभ्यः॑ स्व॒धा पि॒तृभ्य॑ इ॒दम॒हं निर्वरु॑णस्य॒ पाशा॒थ्सुव॑र॒भि॥७॥

%1.3.4.3
वि ख्ये॑षं वैश्वान॒रं ज्योति॒रग्ने᳚ व्रतपते॒ त्वं व्र॒तानां᳚ व्र॒तप॑तिरसि॒ या मम॑ त॒नूस्त्वय्यभू॑दि॒यꣳ सा मयि॒ या तव॑ त॒नूर्मय्यभू॑दे॒षा सा त्वयि॑ यथाय॒थं नौ᳚ व्रतपते व्र॒तिनो᳚र्व्र॒तानि॑॥८॥

%1.3.5.0
{\anuvakamend[{गा॒तु॒विद॒भ्येक॑त्रिꣳशच्च}]}%॥४॥

%1.3.5.1
अत्य॒न्यानगां॒ नान्यानुपा॑गाम॒र्वाक्त्वा॒ परै॑रविदं प॒रो\-ऽव॑रै॒स्तं त्वा॑ जुषे वैष्ण॒वं दे॑वय॒ज्यायै॑ दे॒वस्त्वा॑ सवि॒ता मध्वा॑\-ऽन॒क्त्वोष॑धे॒ त्राय॑स्वैन॒ꣴ॒ स्वधि॑ते॒ मैनꣳ॑ हिꣳसी॒र्दिव॒मग्रे॑ण॒ मा ले॑खीर॒न्तरि॑क्षं॒ मध्ये॑न॒ मा हिꣳ॑सीः पृथि॒व्या सं भ॑व॒ वन॑स्पते श॒तव॑ल्\mbox{}शो॒ वि रो॑ह स॒हस्र॑वल्\mbox{}शा॒ वि व॒यꣳ रु॑हेम॒ यं त्वा॒\-ऽयꣴ स्वधि॑ति॒स्तेति॑जानः प्रणि॒नाय॑ मह॒ते सौभ॑गा॒या\-ऽच्छि॑न्नो॒ रायः॑ सु॒वीरः॑॥९॥

%1.3.6.0
{\anuvakamend[{यं दश॑ च}]}%॥५॥

%1.3.6.1
पृ॒थि॒व्यै त्वा॒\-ऽन्तरि॑क्षाय त्वा दि॒वे त्वा॒ शुन्ध॑तां लो॒कः पि॑तृ॒षद॑नो॒ यवो॑\-ऽसि य॒वया॒स्मद् द्वेषो॑ य॒वयारा॑तीः पितृ॒णाꣳ सद॑नमसि स्वावे॒शो᳚\-ऽस्यग्रे॒गा ने॑तृ॒णां वन॒स्पति॒रधि॑ त्वा स्थास्यति॒ तस्य॑ वित्ताद्दे॒वस्त्वा॑ सवि॒ता मध्वा॑\-ऽनक्तु सुपिप्प॒लाभ्य॒स्त्वौष॑धीभ्य॒ उद्दिवꣴ॑ स्तभा॒नान्तरि॑क्षं पृण पृथि॒वीमुप॑रेण दृꣳह॒ ते ते॒ धामा᳚न्युश्मसी॥१०॥

%1.3.6.2
ग॒मध्ये॒ गावो॒ यत्र॒ भूरि॑शृङ्गा अ॒यासः॑। अत्राह॒ तदु॑रुगा॒यस्य॒ विष्णोः᳚ पर॒मं प॒दमव॑ भाति॒ भूरेः᳚॥ विष्णोः॒ कर्मा॑णि पश्यत॒ यतो᳚ व्र॒तानि॑ पस्प॒शे। इन्द्र॑स्य॒ युज्यः॒ सखा᳚॥ तद्विष्णोः᳚ पर॒मं प॒दꣳ सदा॑ पश्यन्ति सू॒रयः॑। दि॒वीव॒ चक्षु॒रात॑तम्॥ ब्र॒ह्म॒वनिं॑ त्वा क्षत्र॒वनिꣳ॑ सुप्रजा॒वनिꣳ॑ रायस्पोष॒वनिं॒ पर्यू॑हामि॒ ब्रह्म॑ दृꣳह क्ष॒त्रं दृꣳ॑ह प्र॒जां दृꣳ॑ह रा॒यस्पोषं॑ दृꣳह परि॒वीर॑सि॒ परि॑ त्वा॒ दैवी॒र्विशो᳚ व्ययन्तां॒ परी॒मꣳ रा॒यस्पोषो॒ यज॑मानं मनु॒ष्या॑ अ॒न्तरि॑क्षस्य त्वा॒ साना॒वव॑ गूहामि॥११॥

%1.3.7.0
{\anuvakamend[{उ॒श्म॒सी॒ पोष॒मेका॒न्नविꣳ॑श॒तिश्च॑}]}%॥६॥

%1.3.7.1
इ॒षे त्वो॑प॒वीर॒स्युपो॑ दे॒वान्दैवी॒र्विशः॒ प्रागु॒र्वह्नी॑रु॒शिजो॒ बृह॑स्पते धा॒रया॒ वसू॑नि ह॒व्या ते᳚ स्वदन्तां॒ देव॑ त्वष्ट॒र्वसु॑ रण्व॒ रेव॑ती॒ रम॑ध्वम॒ग्नेर्ज॒नित्र॑मसि॒ वृष॑णौ स्थ उ॒र्वश्य॑स्या॒युर॑सि पुरू॒रवा॑ घृ॒तेना॒क्ते वृष॑णं दधाथां गाय॒त्रं छन्दो\-ऽनु॒ प्र जा॑यस्व॒ त्रैष्टु॑भं॒ जाग॑तं॒ छन्दो\-ऽनु॒ प्रजा॑यस्व॒ भव॑तं॥१२॥

%1.3.7.2
नः॒ सम॑नसौ॒ समो॑कसावरे॒पसौ᳚। मा य॒ज्ञꣳ हिꣳ॑सिष्टं॒ मा य॒ज्ञप॑तिं जातवेदसौ शि॒वौ भ॑वतम॒द्य नः॑॥ अ॒ग्नाव॒ग्निश्च॑रति॒ प्रवि॑ष्ट॒ ऋषी॑णां पु॒त्रो अ॑धिरा॒ज ए॒षः। स्वा॒हा॒कृत्य॒ ब्रह्म॑णा ते जुहोमि॒ मा दे॒वानां᳚ मिथु॒याक॑र्भाग॒धेयम्᳚॥१३॥

%1.3.8.0
{\anuvakamend[{भव॑त॒मेक॑त्रिꣳशच्च}]}%॥७॥

%1.3.8.1
आ द॑द ऋ॒तस्य॑ त्वा देवहविः॒ पाशे॒ना\-ऽ\-ऽर॑भे॒ धर्\mbox{}षा॒ मानु॑षान॒द्भ्यस्त्वौष॑धीभ्यः॒ प्रोक्षा᳚म्य॒पां पे॒रुर॑सि स्वा॒त्तं चि॒थ्सदे॑वꣳ ह॒व्यमापो॑ देवीः॒ स्वद॑तैन॒ꣳ॒ सं ते᳚ प्रा॒णो वा॒युना॑ गच्छता॒ꣳ॒ सं यज॑त्रै॒रङ्गा॑नि॒ सं य॒ज्ञप॑तिरा॒शिषा॑ घृ॒तेना॒क्तौ प॒शुं त्रा॑येथा॒ꣳ॒ रेव॑तीर्य॒ज्ञप॑तिं प्रिय॒धा\-ऽ\-ऽवि॑श॒तोरो॑ अन्तरिक्ष स॒जूर्दे॒वेन॒॥१४॥

%1.3.8.2
वाते॑ना॒\-ऽस्य ह॒विष॒स्त्मना॑ यज॒ सम॑स्य त॒नुवा॑ भव॒ वर्\mbox{}षी॑यो॒ वर्\mbox{}षी॑यसि य॒ज्ञे य॒ज्ञप॑तिं धाः पृथि॒व्याः स॒म्पृचः॑ पाहि॒ नम॑स्त आताना\-ऽन॒र्वा प्रेहि॑ घृ॒तस्य॑ कु॒ल्यामनु॑ स॒ह प्र॒जया॑ स॒ह रा॒यस्पोषे॒णा\-ऽ\-ऽपो॑ देवीः शुद्धायुवः शु॒द्धा यू॒यं दे॒वाꣳ ऊ᳚ड्ढ्वꣳ शु॒द्धा व॒यं परि॑विष्टाः परिवे॒ष्टारो॑ वो भूयास्म॥१५॥

%1.3.9.0
{\anuvakamend[{दे॒वेन॒ चतु॑श्चत्वारिꣳशच्च}]}%॥८॥

%1.3.9.1
वाक्त॒ आ प्या॑यतां प्रा॒णस्त॒ आ प्या॑यतां॒ चक्षु॑स्त॒ आ प्या॑यता॒ꣴ॒ श्रोत्रं॑ त॒ आ प्या॑यतां॒ या ते᳚ प्रा॒णाञ्छुग्ज॒गाम॒ या चक्षु॒र्या श्रोत्रं॒ यत् ते᳚ क्रू॒रं यदास्थि॑तं॒ तत् त॒ आ प्या॑यतां॒ तत् त॑ ए॒तेन॑ शुन्धतां॒ नाभि॑स्त॒ आ प्या॑यतां पा॒युस्त॒ आ प्या॑यताꣳ शु॒द्धाश्च॒रित्राः॒ शम॒द्भ्यः॥१६॥

%1.3.9.2
शमोष॑धीभ्यः॒ शं पृ॑थि॒व्यै शमहो᳚भ्या॒मोष॑धे॒ त्राय॑स्वैन॒ꣴ॒ स्वधि॑ते॒ मैनꣳ॑ हिꣳसी॒ रक्ष॑सां भा॒गो॑\-ऽसी॒दम॒हꣳ रक्षो॑\-ऽध॒मं तमो॑ नयामि॒ यो᳚\-ऽस्मान् द्वेष्टि॒ यं च॑ व॒यं द्वि॒ष्म इ॒दमे॑नमध॒मं तमो॑ नयामी॒षे त्वा॑ घृ॒तेन॑ द्यावापृथिवी॒ प्रोर्ण्वा॑था॒मच्छि॑न्नो॒ रायः॑ सु॒वीर॑ उ॒र्व॑न्तरि॑क्ष॒मन्वि॑हि॒ वायो॒ वीहि॑ स्तो॒काना॒ꣴ॒ स्वाहो॒र्ध्वन॑भसं मारु॒तं ग॑च्छतम्॥१७॥

%1.3.10.0
{\anuvakamend[{अ॒द्भ्यो वीहि॒ पञ्च॑ च}]}%॥९॥

%1.3.10.1
सं ते॒ मन॑सा॒ मनः॒ सं प्रा॒णेन॑ प्रा॒णो जुष्टं॑ दे॒वेभ्यो॑ ह॒व्यं घृ॒तव॒थ्स्वाहै॒न्द्रः प्रा॒णो अङ्गे॑अङ्गे॒ नि दे᳚ध्यदै॒न्द्रो॑\-ऽपा॒नो अङ्गे॑अङ्गे॒ वि बो॑भुव॒द्देव॑ त्वष्ट॒र्भूरि॑ ते॒ सꣳस॑मेतु॒ विषु॑रूपा॒ यथ्सल॑क्ष्माणो॒ भव॑थ देव॒त्रा यन्त॒मव॑से॒ सखा॒यो\-ऽनु॑ त्वा मा॒ता पि॒तरो॑ मदन्तु॒ श्रीर॑स्य॒ग्निस्त्वा᳚ श्रीणा॒त्वापः॒ सम॑रिण॒न्वात॑स्य॥१८॥

%1.3.10.2
त्वा॒ ध्रज्यै॑ पू॒ष्णो रꣴह्या॑ अ॒पामोष॑धीना॒ꣳ॒ रोहि॑ष्यै घृ॒तं घृ॑तपावानः पिबत॒ वसां᳚ वसापावानः पिबता॒न्तरि॑क्षस्य ह॒विर॑सि॒ स्वाहा᳚ त्वा॒\-ऽन्तरि॑क्षाय॒ दिशः॑ प्र॒दिश॑ आ॒दिशो॑ वि॒दिश॑ उ॒द्दिशः॒ स्वाहा॑ दि॒ग्भ्यो नमो॑ दि॒ग्भ्यः॥१९॥

%1.3.11.0
{\anuvakamend[{वात॑स्या॒ष्टाविꣳ॑शतिश्च}]}%॥10॥

%1.3.11.1
स॒मु॒द्रं ग॑च्छ॒ स्वाहा॒\-ऽन्तरि॑क्षं गच्छ॒ स्वाहा॑ दे॒वꣳ स॑वि॒तारं॑ गच्छ॒ स्वाहा॑\-ऽहोरा॒त्रे ग॑च्छ॒ स्वाहा॑ मि॒त्रावरु॑णौ गच्छ॒ स्वाहा॒ सोमं॑ गच्छ॒ स्वाहा॑ य॒ज्ञं ग॑च्छ॒ स्वाहा॒ छन्दाꣳ॑सि गच्छ॒ स्वाहा॒ द्यावा॑पृथि॒वी ग॑च्छ॒ स्वाहा॒ नभो॑ दि॒व्यं ग॑च्छ॒ स्वाहा॒\-ऽग्निं वै᳚श्वान॒रं ग॑च्छ॒ स्वाहा॒\-ऽद्भ्यस्त्वौष॑धीभ्यो॒ मनो॑ मे॒ हार्दि॑ यच्छ त॒नूं त्वचं॑ पु॒त्रं नप्ता॑रमशीय॒ शुग॑सि॒ तम॒भि शो॑च॒ यो᳚\-ऽस्मान् द्वेष्टि॒ यं च॑ व॒यं द्वि॒ष्मो धाम्नो॑धाम्नो राजन्नि॒तो व॑रुण नो मुञ्च॒ यदापो॒ अघ्नि॑या॒ वरु॒णेति॒ शपा॑महे॒ ततो॑ वरुण नो मुञ्च॥२०॥

%1.3.12.0
{\anuvakamend[{अ॒सि॒ षड्विꣳ॑शतिश्च}]}%॥11॥

%1.3.12.1
ह॒विष्म॑तीरि॒मा आपो॑ ह॒विष्मा᳚न् दे॒वो अ॑ध्व॒रो ह॒विष्मा॒ꣳ॒ आ वि॑वासति ह॒विष्माꣳ॑ अस्तु॒ सूर्यः॑॥ अ॒ग्नेर्वो\-ऽप॑न्नगृहस्य॒ सद॑सि सादयामि सु॒म्नाय॑ सुम्निनीः सु॒म्ने मा॑ धत्तेन्द्राग्नि॒योर्भा॑ग॒धेयीः᳚ स्थ मि॒त्रावरु॑णयोर्भाग॒धेयीः᳚ स्थ॒ विश्वे॑षां दे॒वानां᳚ भाग॒धेयीः᳚ स्थ य॒ज्ञे जा॑गृत॥२१॥

%1.3.13.0
{\anuvakamend[{ह॒विष्म॑ती॒श्चतु॑स्रिꣳशत्}]}%॥12॥

%1.3.13.1
हृ॒दे त्वा॒ मन॑से त्वा दि॒वे त्वा॒ सूर्या॑य त्वो॒र्ध्वमि॒मम॑ध्व॒रं कृ॑धि दि॒वि दे॒वेषु॒ होत्रा॑ यच्छ॒ सोम॑ राज॒न्नेह्यव॑ रोह॒ मा भेर्मा सं वि॑क्था॒ मा त्वा॑ हिꣳसिषं प्र॒जास्त्वमु॒पाव॑रोह प्र॒जास्त्वामु॒पाव॑ रोहन्तु शृ॒णोत्व॒ग्निः स॒मिधा॒ हवं॑ मे शृ॒ण्वन्त्वापो॑ धि॒षणा᳚श्च दे॒वीः। शृ॒णोत॑ ग्रावाणो वि॒दुषो॒ नु॥२२॥

%1.3.13.2
य॒ज्ञꣳ शृ॒णोतु॑ दे॒वः स॑वि॒ता हवं॑ मे। देवी॑रापो अपां नपा॒द्य ऊ॒र्मिर्\mbox{}ह॑वि॒ष्य॑ इन्द्रि॒यावा᳚न्म॒दिन्त॑म॒स्तं दे॒वेभ्यो॑ देव॒त्रा ध॑त्त शु॒क्रꣳ शु॑क्र॒पेभ्यो॒ येषां᳚ भा॒गः स्थ स्वाहा॒ कार्\mbox{}षि॑र॒स्यपा॒पां मृ॒ध्रꣳ स॑मु॒द्रस्य॒ वोक्षि॑त्या॒ उन्न॑ये। यम॑ग्ने पृ॒थ्सु मर्त्य॒मावो॒ वाजे॑षु॒ यं जु॒नाः। स यन्ता॒ शश्व॑ती॒रिषः॑॥२३॥

%1.3.14.0
{\anuvakamend[{नु स॒प्तच॑त्वारिꣳशच्च}]}%॥13॥

%1.3.14.1
त्वम॑ग्ने रु॒द्रो असु॑रो म॒हो दि॒वस्त्वꣳ शर्धो॒ मारु॑तं पृ॒क्ष ई॑शिषे। त्वं वातै॑ररु॒णैर्या॑सि शङ्ग॒यस्त्वं पू॒षा वि॑ध॒तः पा॑सि॒ नु त्मना᳚॥ आ वो॒ राजा॑नमध्व॒रस्य॑ रु॒द्रꣳ होता॑रꣳ सत्य॒यज॒ꣳ॒ रोद॑स्योः। अ॒ग्निं पु॒रा त॑नयि॒त्नोर॒चित्ता॒द्धिर॑ण्यरूप॒मव॑से कृणुध्वम्॥ अ॒ग्निर्\mbox{}होता॒ निष॑सादा॒ यजी॑यानु॒पस्थे॑ मा॒तुः सु॑र॒भावु॑ लो॒के। युवा॑ क॒विः पु॑रुनि॒ष्ठ -॥२४॥

%1.3.14.2
ऋ॒तावा॑ ध॒र्ता कृ॑ष्टी॒नामु॒त मध्य॑ इ॒द्धः॥ सा॒ध्वीम॑कर्दे॒ववी॑तिं नो अ॒द्य य॒ज्ञस्य॑ जि॒ह्वाम॑विदाम॒ गुह्या᳚म्। स आयु॒रागा᳚थ्सुर॒\-भिर्वसा॑नो भ॒द्राम॑कर्दे॒वहू॑तिं नो अ॒द्य॥ अक्र॑न्दद॒ग्निः स्त॒नय॑न्निव॒ द्यौः क्षामा॒ रेरि॑हद्वी॒रुधः॑ सम॒ञ्जन्न्। स॒द्यो ज॑ज्ञा॒नो विहीमि॒द्धो अख्य॒दा रोद॑सी भा॒नुना॑ भात्य॒न्तः॥ त्वे वसू॑नि पुर्वणीक॥२५॥

%1.3.14.3
होतर्दो॒षा वस्तो॒रेरि॑रे य॒ज्ञिया॑सः। क्षामे॑व॒ विश्वा॒ भुव॑नानि॒ यस्मि॒न्थ्सꣳ सौभ॑गानि दधि॒रे पा॑व॒के॥ तुभ्यं॒ ता अ॑ङ्गिरस्तम॒ विश्वाः᳚ सुक्षि॒तयः॒ पृथ॑क्। अग्ने॒ कामा॑य येमिरे॥ अ॒श्याम॒ तं काम॑मग्ने॒ तवो॒त्य॑श्याम॑ र॒यिꣳ र॑यिवः सु॒वीरम्᳚। अ॒श्याम॒ वाज॑म॒भि वा॒जय॑न्तो॒\-ऽश्याम॑ द्यु॒म्नम॑जरा॒जरं॑ ते॥ श्रेष्ठं॑ यविष्ठ भार॒ताग्ने᳚ द्यु॒मन्त॒माभ॑र।॥२६॥

%1.3.14.4
वसो॑ पुरु॒स्पृहꣳ॑ र॒यिम्॥ स श्वि॑ता॒नस्त॑न्य॒तू रो॑चन॒स्था अ॒जरे॑भि॒र्नान॑दद्भि॒र्यवि॑ष्ठः। यः पा॑व॒कः पु॑रु॒तमः॑ पु॒रूणि॑ पृ॒थून्य॒ग्निर॑नु॒याति॒ भर्वन्न्॑॥ आयु॑ष्टे वि॒श्वतो॑ दधद॒यम॒ग्निर्वरे᳚ण्यः। पुन॑स्ते प्रा॒ण आय॑ति॒ परा॒ यक्ष्मꣳ॑ सुवामि ते॥ आ॒यु॒र्दा अ॑ग्ने ह॒विषो॑ जुषा॒णो घृ॒तप्र॑तीको घृ॒तयो॑निरेधि। घृ॒तं पी॒त्वा मधु॒ चारु॒ गव्यं॑ पि॒तेव॑ पु॒त्रम॒भि॥२७॥

%1.3.14.5
र॑क्षतादि॒मम्॥ तस्मै॑ ते प्रति॒हर्य॑ते॒ जात॑वेदो॒ विच॑र्\mbox{}षणे। अग्ने॒ जना॑मि सुष्टु॒तिम्॥ दि॒वस्परि॑ प्रथ॒मं ज॑ज्ञे अ॒ग्निर॒स्मद् द्वि॒तीयं॒ परि॑ जा॒तवे॑दाः। तृ॒तीय॑म॒फ्सु नृ॒मणा॒ अज॑स्र॒मिन्धा॑न एनं जरते स्वा॒धीः॥ शुचिः॑ पावक॒ वन्द्यो\-ऽग्ने॑ बृ॒हद्वि रो॑चसे। त्वं घृ॒तेभि॒राहु॑तः॥ दृ॒शा॒नो रु॒क्म उ॒र्व्या व्य॑द्यौद् दु॒र्मर्\mbox{}ष॒मायुः॑ श्रि॒ये रु॑चा॒नः। अ॒ग्निर॒मृतो॑ अभव॒द्वयो॑भि॒र्-॥२८॥

%1.3.14.6
यदे॑नं॒ द्यौरज॑नयथ्सु॒रेताः᳚॥ आ यदि॒षे नृ॒पतिं॒ तेज॒ आन॒ट्छुचि॒ रेतो॒ निषि॑क्तं॒ द्यौर॒भीके᳚। अ॒ग्निः शर्ध॑मनव॒द्यं युवा॑नꣴ स्वा॒धियं॑ जनयथ्सू॒दय॑च्च॥ स तेजी॑यसा॒ मन॑सा॒ त्वोत॑ उ॒त शि॑क्ष स्वप॒त्यस्य॑ शि॒क्षोः। अग्ने॑ रा॒यो नृत॑मस्य॒ प्रभू॑तौ भू॒याम॑ ते सुष्टु॒तय॑श्च॒ वस्वः॑॥ अग्ने॒ सह॑न्त॒मा भ॑र द्यु॒म्नस्य॑ प्रा॒सहा॑ र॒यिम्। विश्वा॒ यश्-॥२९॥

%1.3.14.7
च॑र्\mbox{}ष॒णीर॒भ्या॑सा वाजे॑षु सा॒सह॑त्॥ तम॑ग्ने पृतना॒सहꣳ॑ र॒यिꣳ स॑हस्व॒ आ भ॑र। त्वꣳ हि स॒त्यो अद्भु॑तो दा॒ता वाज॑स्य॒ गोम॑तः॥ उ॒क्षान्ना॑य व॒शान्ना॑य॒ सोम॑पृष्ठाय वे॒धसे᳚। स्तोमै᳚र्विधेमा॒ग्नये᳚॥ व॒द्मा हि सू॑नो॒ अस्य॑द्म॒सद्वा॑ च॒क्रे अ॒ग्निर्ज॒नुषाज्मान्नम्᳚। स त्वं न॑ ऊर्जसन॒ ऊर्जं॑ धा॒ राजे॑व जेरवृ॒के क्षे᳚ष्य॒न्तः॥ अग्न॒ आयूꣳ॑षि॥३०॥

%1.3.14.8
पवस॒ आ सु॒वोर्ज॒मिषं॑ च नः। आ॒रे बा॑धस्व दु॒च्छुना᳚म्॥ अग्ने॒ पव॑स्व॒ स्वपा॑ अ॒स्मे वर्चः॑ सु॒वीर्यम्᳚। दध॒त्पोषꣳ॑ र॒यिं मयि॑॥ अग्ने॑ पावक रो॒चिषा॑ म॒न्द्रया॑ देव जि॒ह्वया᳚। आ दे॒वान् व॑क्षि॒ यक्षि॑ च॥ स नः॑ पावक दीदि॒वो\-ऽग्ने॑ दे॒वाꣳ इ॒हा व॑ह। उप॑ य॒ज्ञꣳ ह॒विश्च॑ नः॥ अ॒ग्निः शुचि॑व्रततमः॒ शुचि॒र्विप्रः॒ शुचिः॑ क॒विः। शुची॑ रोचत॒ आहु॑तः॥ उद॑ग्ने॒ शुच॑य॒स्तव॑ शु॒क्रा भ्राज॑न्त ईरते। तव॒ ज्योतीꣴ॑ष्य॒र्चयः॑॥३१॥

{\anuvakamend[{पु॒रु॒नि॒ष्ठः पु॑र्वणीक भरा॒\-ऽभि वयो॑भि॒र्य आयूꣳ॑षि॒ विप्रः॒ शुचि॒श्चतु॑र्दश च}]}%॥14॥
%%% END PRASHNA

\sect{चतुर्थः प्रश्नः}\setcounter{anuvakam}{0}
\dnsub{तैत्तिरीयसंहितायां प्रथमकाण्डे चतुर्थः प्रश्नः}
%1.4.1.0
%1.4.1.1
आ द॑दे॒ ग्रावा᳚स्यध्वर॒कृद् दे॒वेभ्यो॑ गम्भी॒रमि॒मम॑ध्व॒रं कृ॑ध्युत्त॒मेन॑ प॒विनेन्द्रा॑य॒ सोम॒ꣳ॒ सुषु॑तं॒ मधु॑मन्तं॒ पय॑स्वन्तं वृष्टि॒वनि॒मिन्द्रा॑य त्वा वृत्र॒घ्न इन्द्रा॑य त्वा वृत्र॒तुर॒ इन्द्रा॑य त्वा\-ऽभिमाति॒घ्न इन्द्रा॑य त्वा\-ऽ\-ऽदि॒त्यव॑त॒ इन्द्रा॑य त्वा वि॒श्वदे᳚व्यावते श्वा॒त्राः स्थ॑ वृत्र॒तुरो॒ राधो॑गूर्ता अ॒मृत॑स्य॒ पत्नी॒स्ता दे॑वीर्देव॒त्रेमं य॒ज्ञं ध॒त्तोप॑हूताः॒ सोम॑स्य पिब॒तोप॑हूतो यु॒ष्माक॒ꣳ॒॥१॥

%1.4.1.2
सोमः॑ पिबतु॒ यत्ते॑ सोम दि॒वि ज्योति॒र्यत् पृ॑थि॒व्यां यदु॒राव॒न्तरि॑क्षे॒ तेना॒स्मै यज॑मानायो॒रु रा॒या कृ॒ध्यधि॑ दा॒त्रे वो॑चो॒ धिष॑णे वी॒डू स॒ती वी॑डयेथा॒मूर्जं॑ दधाथा॒मूर्जं॑ मे धत्तं॒ मा वाꣳ॑ हिꣳसिषं॒ मा मा॑ हिꣳसिष्टं॒ प्रागपा॒गुद॑गध॒राक्तास्त्वा॒ दिश॒ आ धा॑व॒न्त्वम्ब॒ नि ष्व॑र। यत्ते॑ सो॒मादा᳚भ्यं॒ नाम॒ जागृ॑वि॒ तस्मै॑ ते सोम॒ सोमा॑य॒ स्वाहा᳚॥२॥

%1.4.2.0
{\anuvakamend[{यु॒ष्माकꣴ॑ स्वर॒ यत्ते॒ नव॑ च}]}%॥१॥

%1.4.2.1
वा॒चस्पत॑ये पवस्व वाजि॒न् वृषा॒ वृष्णो॑ अ॒ꣳ॒शुभ्यां॒ गभ॑स्तिपूतो दे॒वो दे॒वानां᳚ प॒वित्र॑मसि॒ येषां᳚ भा॒गो\-ऽसि॒ तेभ्य॑स्त्वा॒ स्वां कृ॑तो\-ऽसि॒ मधु॑मतीर्न॒ इष॑स्कृधि॒ विश्वे᳚भ्यस्त्वेन्द्रि॒येभ्यो॑ दि॒व्येभ्यः॒ पार्थि॑वेभ्यो॒ मन॑स्त्वाष्टू॒र्व॑न्त\-रि॑क्ष॒मन्वि॑हि॒ स्वाहा᳚ त्वा सुभवः॒ सूर्या॑य दे॒वेभ्य॑स्त्वा मरीचि॒पेभ्य॑ ए॒ष ते॒ योनिः॑ प्रा॒णाय॑ त्वा॥३॥

%1.4.3.0
{\anuvakamend[{वा॒चः स॒प्तच॑त्वारिꣳशत्}]}%॥२॥

%1.4.3.1
उ॒प॒या॒मगृ॑हीतो\-ऽस्य॒न्तर्य॑च्छ मघवन् पा॒हि सोम॑मुरु॒ष्य रायः॒ समिषो॑ यजस्वा॒न्तस्ते॑ दधामि॒ द्यावा॑पृथि॒वी अ॒न्तरु॒र्व॑न्तरि॑क्षꣳ स॒जोषा॑ दे॒वैरव॑रैः॒ परै᳚श्चान्तर्या॒मे म॑घवन् मादयस्व॒ स्वां कृ॑तो\-ऽसि॒ मधु॑मतीर्न॒ इष॑स्कृधि॒ विश्वे᳚भ्यस्त्वेन्द्रि॒येभ्यो॑ दि॒व्येभ्यः॒ पार्थि॑वेभ्यो॒ मन॑स्त्वाष्टू॒र्व॑न्त\-रि॑क्ष॒मन्वि॑हि॒ स्वाहा᳚ त्वा सुभवः॒ सूर्या॑य दे॒वेभ्य॑स्त्वा मरीचि॒पेभ्य॑ ए॒ष ते॒ योनि॑रपा॒नाय॑ त्वा॥४॥

%1.4.4.0
{\anuvakamend[{दे॒वेभ्यः॑ स॒प्त च॑}]}%॥३॥

%1.4.4.1
आ वा॑यो भूष शुचिपा॒ उप॑ नः स॒हस्रं॑ ते नि॒युतो॑ विश्ववार। उपो॑ ते॒ अन्धो॒ मद्य॑मयामि॒ यस्य॑ देव दधि॒षे पू᳚र्व॒पेयम्᳚॥ उ॒प॒या॒मगृ॑हीतो\-ऽसि वा॒यवे॒ त्वेन्द्र॑वायू इ॒मे सु॒ताः। उप॒ प्रयो॑भि॒रा ग॑त॒मिन्द॑वो वामु॒शन्ति॒ हि॥ उ॒प॒या॒मगृ॑हीतो\-ऽसीन्द्रवा॒यु\-भ्यां᳚ त्वै॒ष ते॒ योनिः॑ स॒जोषा᳚भ्यां त्वा॥५॥

%1.4.5.0
{\anuvakamend[{आ वा॑यो॒ त्रिच॑त्वारिꣳशत्}]}%॥४॥

%1.4.5.1
अ॒यं वां᳚ मित्रावरुणा सु॒तः सोम॑ ऋतावृधा। ममेदि॒ह श्रु॑त॒ꣳ॒ हवम्᳚। उ॒प॒या॒मगृ॑हीतो\-ऽसि मि॒त्रावरु॑णाभ्यां त्वै॒ष ते॒ योनि॑र् ऋता॒यु\-भ्यां᳚ त्वा॥६॥

%1.4.6.0
{\anuvakamend[{अ॒यं वां᳚ विꣳश॒तिः}]}%॥५॥

%1.4.6.1
या वां॒ कशा॒ मधु॑म॒त्यश्वि॑ना सू॒नृता॑वती। तया॑ य॒ज्ञं मि॑मिक्षतम्। उ॒प॒या॒मगृ॑हीतो\-ऽस्य॒श्वि\-भ्यां᳚ त्वै॒ष ते॒ योनि॒र्माध्वी᳚भ्यां त्वा॥७॥

%1.4.7.0
{\anuvakamend[{या वा॑म॒ष्टाद॑श}]}%॥६॥

%1.4.7.1
प्रा॒त॒र्युजौ॒ वि मु॑च्येथा॒मश्वि॑ना॒वेह ग॑च्छतम्। अ॒स्य सोम॑स्य पी॒तये᳚॥ उ॒प॒या॒मगृ॑हीतो\-ऽस्य॒श्वि\-भ्यां᳚ त्वै॒ष ते॒ योनि॑र॒श्वि\-भ्यां᳚ त्वा॥८॥

%1.4.8.0
{\anuvakamend[{प्रा॒त॒र्युजा॒वेका॒न्नविꣳ॑शतिः}]}%॥७॥

%1.4.8.1
अ॒यं वे॒नश्चो॑दय॒त् पृश्नि॑गर्भा॒ ज्योति॑र्जरायू॒ रज॑सो वि॒माने᳚। इ॒मम॒पाꣳ स॑ङ्ग॒मे सूर्य॑स्य॒ शिशुं॒ न विप्रा॑ म॒तिभी॑ रिहन्ति॥ उ॒प॒या॒मगृ॑हीतो\-ऽसि॒ शण्डा॑य त्वै॒ष ते॒ योनि॑र्वी॒रतां᳚ पाहि॥९॥

%1.4.9.0
{\anuvakamend[{अ॒यं वे॒नः पञ्च॑विꣳशतिः}]}%॥८॥

%1.4.9.1
तं प्र॒त्नथा॑ पू॒र्वथा॑ वि॒श्वथे॒मथा᳚ ज्ये॒ष्ठता॑तिं बर्\mbox{}हि॒षदꣳ॑ सुव॒र्विदं॑ प्रतीची॒नं वृ॒जनं॑ दोहसे गि॒रा\-ऽ\-ऽशुं जय॑न्त॒मनु॒ यासु॒ वर्ध॑से। उ॒प॒या॒मगृ॑हीतो\-ऽसि॒ मर्का॑य त्वै॒ष ते॒ योनिः॑ प्र॒जाः पा॑हि॥१०॥

%1.4.10.0
{\anuvakamend[{तꣳ षड्विꣳ॑शतिः}]}%॥९॥

%1.4.10.1
ये दे॑वा दि॒व्येका॑दश॒ स्थ पृ॑थि॒व्यामध्येका॑दश॒ स्था\-ऽफ्सु॒षदो॑ महि॒नैका॑दश॒ स्थ ते दे॑वा य॒ज्ञमि॒मं जु॑षध्वमुपया॒मगृ॑हीतो\-ऽस्याग्रय॒णो॑\-ऽसि॒ स्वा᳚ग्रयणो॒ जिन्व॑ य॒ज्ञं जिन्व॑ य॒ज्ञप॑तिम॒भि सव॑ना पाहि॒ विष्णु॒स्त्वां पा॑तु॒ विशं॒ त्वं पा॑हीन्द्रि॒येणै॒ष ते॒ योनि॒र्विश्वे᳚भ्यस्त्वा दे॒वेभ्यः॑॥११॥

%1.4.12.0
{\anuvakamend[{ये दे॑वा॒स्त्रिच॑त्वारिꣳशत्}]}%॥10॥

%1.4.11.1
त्रि॒ꣳ॒शत्त्रय॑श्च ग॒णिनो॑ रु॒जन्तो॒ दिवꣳ॑ रु॒द्राः पृ॑थि॒वीं च॑ सचन्ते। ए॒का॒द॒शासो॑ अफ्सु॒षदः॑ सु॒तꣳ सोमं॑ जुषन्ता॒ꣳ॒ सव॑नाय॒ विश्वे᳚॥ उ॒प॒या॒मगृ॑हीतो\-ऽस्याग्रय॒णो॑\-ऽसि॒ स्वा᳚ग्रयणो॒ जिन्व॑ य॒ज्ञं जिन्व॑ य॒ज्ञप॑तिम॒भि सव॑ना पाहि॒ विष्णु॒स्त्वां पा॑तु॒ विशं॒ त्वं पा॑हीन्द्रि॒येणै॒ष ते॒ योनि॒र्विश्वे᳚भ्यस्त्वा दे॒वेभ्यः॑॥१२॥

%1.4.10.0
{\anuvakamend[{त्रि॒ꣳ॒शद् द्विच॑त्वारिꣳशत्}]}%॥11॥

%1.4.12.1
उ॒प॒या॒मगृ॑हीतो॒\-ऽसीन्द्रा॑य त्वा बृ॒हद्व॑ते॒ वय॑स्वत उक्था॒युवे॒ यत् त॑ इन्द्र बृ॒हद्वय॒स्तस्मै᳚ त्वा॒ विष्ण॑वे त्वै॒ष ते॒ योनि॒रिन्द्रा॑य त्वोक्था॒युवे᳚॥१३॥

%1.4.13.0
{\anuvakamend[{उ॒प॒या॒मगृ॑हीतो॒ द्वाविꣳ॑शतिः}]}%॥12॥

%1.4.13.1
मू॒र्धानं॑ दि॒वो अ॑र॒तिं पृ॑थि॒व्या वै᳚श्वान॒रमृ॒ताय॑ जा॒तम॒ग्निम्। क॒विꣳ स॒म्राज॒मति॑थिं॒ जना॑नामा॒सन्ना पात्रं॑ जनयन्त दे॒वाः॥ उ॒प॒या॒मगृ॑हीतो\-ऽस्य॒ग्नये᳚ त्वा वैश्वान॒राय॑ ध्रु॒वो॑\-ऽसि ध्रु॒वक्षि॑तिर्ध्रु॒वाणां᳚ ध्रु॒वत॒मो\-ऽच्यु॑तानामच्युत॒क्षित्त॑म ए॒ष ते॒ योनि॑र॒ग्नये᳚ त्वा वैश्वान॒राय॑॥१४॥

%1.4.14.0
{\anuvakamend[{मू॒र्धानं॒ पञ्च॑त्रिꣳशत्}]}%॥13॥

%1.4.14.1
मधु॑श्च॒ माध॑वश्च शु॒क्रश्च॒ शुचि॑श्च॒ नभ॑श्च नभ॒स्य॑श्चे॒षश्चो॒र्जश्च॒ सह॑श्च सह॒स्य॑श्च॒ तप॑श्च तप॒स्य॑श्चोपया॒मगृ॑हीतो\-ऽसि स॒ꣳ॒सर्पो᳚\-ऽस्यꣳहस्प॒त्याय॑ त्वा॥१५॥

%1.4.15.0
{\anuvakamend[{मधु॑स्त्रि॒ꣳ॒शत्}]}%॥14॥

%1.4.15.1
इन्द्रा᳚ग्नी॒ आ ग॑तꣳ सु॒तं गी॒र्भिर्नभो॒ वरे᳚ण्यम्। अ॒स्य पा॑तं धि॒येषि॒ता॥ उ॒प॒या॒मगृ॑हीतो\-ऽसीन्द्रा॒ग्नि\-भ्यां᳚ त्वै॒ष ते॒ योनि॑रिन्द्रा॒ग्नि\-भ्यां᳚ त्वा॥१६॥

%1.4.16.0
{\anuvakamend[{इन्द्रा᳚ग्नी विꣳश॒तिः}]}%॥15॥

%1.4.16.1
ओमा॑सश्चर्\mbox{}षणीधृतो॒ विश्वे॑ देवास॒ आ ग॑त। दा॒श्वाꣳसो॑ दा॒शुषः॑ सु॒तम्॥ उ॒प॒या॒मगृ॑हीतो\-ऽसि॒ विश्वे᳚भ्यस्त्वा दे॒वेभ्य॑ ए॒ष ते॒ योनि॒र्विश्वे᳚भ्यस्त्वा दे॒वेभ्यः॑॥१७॥

%1.4.17.0
{\anuvakamend[{इन्द्रा᳚ग्नी॒ ओमा॑सो विꣳश॒तिर्विꣳ॑शतिः}]}%॥16॥

%1.4.17.1
म॒रुत्व॑न्तं वृष॒भं वा॑वृधा॒नमक॑वारिं दि॒व्यꣳ शा॒समिन्द्रम्᳚। वि॒श्वा॒साह॒मव॑से॒ नूत॑नायो॒ग्रꣳ स॑हो॒दामि॒ह तꣳ हु॑वेम॥ उ॒प॒या॒मगृ॑हीतो॒\-ऽसीन्द्रा॑य त्वा म॒रुत्व॑त ए॒ष ते॒ योनि॒रिन्द्रा॑य त्वा म॒रुत्व॑ते॥१८॥

%1.4.18.0
{\anuvakamend[{म॒रुत्व॑न्त॒ꣳ॒ षड्विꣳ॑शतिः}]}%॥17॥

%1.4.18.1
इन्द्र॑ मरुत्व इ॒ह पा॑हि॒ सोमं॒ यथा॑ शार्या॒ते अपि॑बः सु॒तस्य॑। तव॒ प्रणी॑ती॒ तव॑ शूर॒ शर्म॒न्ना वि॑वासन्ति क॒वयः॑ सुय॒ज्ञाः॥ उ॒प॒या॒मगृ॑हीतो॒\-ऽसीन्द्रा॑य त्वा म॒रुत्व॑त ए॒ष ते॒ योनि॒रिन्द्रा॑य त्वा म॒रुत्व॑ते॥१९॥

%1.4.19.0
{\anuvakamend[{इन्द्रैका॒न्नत्रि॒ꣳ॒शत्}]}%॥18॥

%1.4.19.1
म॒रुत्वाꣳ॑ इन्द्र वृष॒भो रणा॑य॒ पिबा॒ सोम॑मनुष्व॒धं मदा॑य। आ सि॑ञ्चस्व ज॒ठरे॒ मध्व॑ ऊ॒र्मिं त्वꣳ राजा॑सि प्र॒दिवः॑ सु॒ताना᳚म्॥ उ॒प॒या॒मगृ॑हीतो॒\-ऽसीन्द्रा॑य त्वा म॒रुत्व॑त ए॒ष ते॒ योनि॒रिन्द्रा॑य त्वा म॒रुत्व॑ते॥२०॥

%1.4.20.0
{\anuvakamend[{इन्द्र॑ मरुत्वो म॒रुत्वा॒नेका॒न्न त्रि॒ꣳ॒शदेका॒न्न त्रि॒ꣳ॒शत्}]}%॥19॥

%1.4.20.1
म॒हाꣳ इन्द्रो॒ य ओज॑सा प॒र्जन्यो॑ वृष्टि॒माꣳ इ॑व। स्तोमै᳚र्व॒थ्सस्य॑ वावृधे॥ उ॒प॒या॒मगृ॑हीतो\-ऽसि महे॒न्द्राय॑ त्वै॒ष ते॒ योनि॑र्महे॒न्द्राय॑ त्वा॥२१॥

%1.4.21.0
{\anuvakamend[{म॒हानेका॒न्नविꣳ॑शतिः}]}%॥20॥

%1.4.21.1
म॒हाꣳ इन्द्रो॑ नृ॒वदा च॑र्\mbox{}षणि॒प्रा उ॒त द्वि॒बर्\mbox{}हा॑ अमि॒नः सहो॑भिः। अ॒स्म॒द्रिय॑ग्वावृधे वी॒र्या॑यो॒रुः पृ॒थुः सुकृ॑तः क॒र्तृभि॑र्भूत्॥ उ॒प॒या॒मगृ॑हीतो\-ऽसि महे॒न्द्राय॑ त्वै॒ष ते॒ योनि॑र्महे॒न्द्राय॑ त्वा॥२२॥

%1.4.22.0
{\anuvakamend[{म॒हान्नृ॒वत्षड्विꣳ॑शतिः}]}%॥21॥

%1.4.22.1
क॒दा च॒न स्त॒रीर॑सि॒ नेन्द्र॑ सश्चसि दा॒शुषे᳚। उपो॒पेन्नु म॑घव॒न् भूय॒ इन्नु ते॒ दानं॑ दे॒वस्य॑ पृच्यते॥ उ॒प॒या॒मगृ॑हीतो\-ऽस्यादि॒त्येभ्य॑स्त्वा॥ क॒दा च॒न प्र यु॑च्छस्यु॒भे नि पा॑सि॒ जन्म॑नी। तुरी॑यादित्य॒ सव॑नं त इन्द्रि॒यमा त॑स्थाव॒मृतं॑ दि॒वि॥ य॒ज्ञो दे॒वानां॒ प्रत्ये॑ति सु॒म्नमादि॑त्यासो॒ भव॑ता मृड॒यन्तः॑। आ वो॒ऽर्वाची॑ सुम॒तिर्व॑वृत्याद॒ꣳ॒होश्चि॒द्या व॑रिवो॒वित्त॒रास॑त्॥ विव॑स्व आदित्यै॒ष ते॑ सोमपी॒थस्तेन॑ मन्दस्व॒ तेन॑ तृप्य तृ॒प्यास्म॑ ते व॒यं त॑र्पयि॒तारो॒ या दि॒व्या वृष्टि॒स्तया᳚ त्वा श्रीणामि॥२३॥

%1.4.23.0
{\anuvakamend[{वः॒ स॒प्तविꣳ॑शतिश्च}]}%॥2॥

%1.4.23.1
वा॒मम॒द्य स॑वितर्वा॒ममु॒ श्वो दि॒वेदि॑वे वा॒मम॒स्मभ्यꣳ॑ सावीः। वा॒मस्य॒ हि क्षय॑स्य देव॒ भूरे॑र॒या धि॒या वा॑म॒भाजः॑ स्याम॥ उ॒प॒या॒मगृ॑हीतो\-ऽसि दे॒वाय॑ त्वा सवि॒त्रे॥२४॥

%1.4.24.0
{\anuvakamend[{वा॒मं चतु॑र्विꣳशतिः}]}%॥23॥

%1.4.24.1
अद॑ब्धेभिः सवितः पा॒युभि॒ष्ट्वꣳ शि॒वेभि॑र॒द्य परि॑ पाहि नो॒ गयम्᳚। हिर॑ण्यजिह्वः सुवि॒ताय॒ नव्य॑से॒ रक्षा॒ माकि॑र्नो अ॒घशꣳ॑स ईशत॥ उ॒प॒या॒मगृ॑हीतो\-ऽसि दे॒वाय॑ त्वा सवि॒त्रे॥२५॥

%1.4.25.0
{\anuvakamend[{अद॑ब्धेभि॒स्त्रयो॑विꣳशतिः}]}%॥24॥

%1.4.25.1
हिर॑ण्यपाणिमू॒तये॑ सवि॒तार॒मुप॑ ह्वये। स चेत्ता॑ दे॒वता॑ प॒दम्॥ उ॒प॒या॒मगृ॑हीतो\-ऽसि दे॒वाय॑ त्वा सवि॒त्रे॥२६॥

%1.4.26.0
{\anuvakamend[{हिर॑ण्यपाणिं॒ चतु॑र्दश}]}%॥25॥

%1.4.26.1
सु॒शर्मा॑\-ऽसि सुप्रतिष्ठा॒नो बृ॒हदु॒क्षे नम॑ ए॒ष ते॒ योनि॒र्विश्वे᳚भ्यस्त्वा दे॒वेभ्यः॑॥२७॥

%1.4.27.0
{\anuvakamend[{सु॒शर्मा॒ द्वाद॑श}]}%॥26॥

%1.4.27.1
बृह॒स्पति॑सुतस्य त इन्दो इन्द्रि॒याव॑तः॒ पत्नी॑वन्तं॒ ग्रहं॑ गृह्णा॒म्यग्ना(३)इ पत्नी॒वा(३)स्स॒जूर्दे॒वेन॒ त्वष्ट्रा॒ सोमं॑ पिब॒ स्वाहा᳚॥२८॥

%1.4.28.0
{\anuvakamend[{बृह॒स्पति॑सुतस्य॒ पञ्च॑दश}]}%॥27॥

%1.4.28.1
हरि॑रसि हारियोज॒नो हर्योः᳚ स्था॒ता वज्र॑स्य भ॒र्ता पृश्ञेः᳚ प्रे॒ता तस्य॑ ते देव सोमे॒ष्टय॑जुषः स्तु॒तस्तो॑मस्य श॒स्तोक्थ॑स्य॒ हरि॑वन्तं॒ ग्रहं॑ गृह्णामि ह॒रीः स्थ॒ हर्यो᳚र्धा॒नाः स॒हसो॑मा॒ इन्द्रा॑य॒ स्वाहा᳚॥२९॥

%1.4.29.0
{\anuvakamend[{हरिः॒ षड्विꣳ॑शतिः}]}%॥28॥

%1.4.29.1
अग्न॒ आयूꣳ॑षि पवस॒ आ सु॒वोर्ज॒मिषं॑ च नः। आ॒रे बा॑धस्व दु॒च्छुना᳚म्॥ उ॒प॒या॒मगृ॑हीतो\-ऽस्य॒ग्नये᳚ त्वा॒ तेज॑स्वत ए॒ष ते॒ योनि॑र॒ग्नये᳚ त्वा॒ तेज॑स्वते॥३०॥

%1.4.30.0
{\anuvakamend[{अग्न॒ आयूꣳ॑षि॒ त्रयो॑वि ꣳशतिः}]}%॥29॥

%1.4.30.1
उ॒त्तिष्ठ॒न्नोज॑सा स॒ह पी॒त्वा शिप्रे॑ अवेपयः। सोम॑मिन्द्र च॒मू सु॒तम्॥ उ॒प॒या॒मगृ॑हीतो॒\-ऽसीन्द्रा॑य॒ त्वौज॑स्वत ए॒ष ते॒ योनि॒रिन्द्रा॑य॒ त्वौज॑स्वते॥३१॥

%1.4.31.0
{\anuvakamend[{उ॒त्तिष्ठ॒न्नेक॑विꣳशतिः}]}%॥30॥

%1.4.31.1
त॒रणि॑र्वि॒श्वद॑र्\mbox{}शतो ज्योति॒ष्कृद॑सि सूर्य। विश्व॒मा भा॑सि रोच॒नम्॥ उ॒प॒या॒मगृ॑हीतो\-ऽसि॒ सूर्या॑य त्वा॒ भ्राज॑स्वत ए॒ष ते॒ योनिः॒ सूर्या॑य त्वा॒ भ्राज॑स्वते॥३२॥

%1.4.32.0
{\anuvakamend[{त॒रणि॑र्विꣳश॒तिः}]}%॥31॥

%1.4.32.1
आ प्या॑यस्व मदिन्तम॒ सोम॒ विश्वा॑भिरू॒तिभिः॑। भवा॑ नः स॒प्रथ॑स्तमः॥३३॥

%1.4.33.0
{\anuvakamend[{आ प्या॑यस्व॒ नव॑}]}%॥32॥

%1.4.33.1
ई॒युष्टे ये पूर्व॑तरा॒मप॑श्यन् व्यु॒च्छन्ती॑मु॒षसं॒ मर्त्या॑सः। अ॒स्माभि॑रू॒ नु प्र॑ति॒चक्ष्या॑\-ऽभू॒दो ते य॑न्ति॒ ये अ॑प॒रीषु॒ पश्यान्॑॥३४॥

%1.4.34.0
{\anuvakamend[{ई॒युरेका॒न्नविꣳ॑शतिः}]}%॥33॥

%1.4.34.1
ज्योति॑ष्मतीं त्वा सादयामि ज्योति॒ष्कृतं॑ त्वा सादयामि ज्योति॒र्विदं॑ त्वा सादयामि॒ भास्व॑तीं त्वा सादयामि॒ ज्वल॑न्तीं त्वा सादयामि मल्मला॒भव॑न्तीं त्वा सादयामि॒ दीप्य॑मानां त्वा सादयामि॒ रोच॑मानां त्वा सादया॒म्यज॑स्रां त्वा सादयामि बृ॒हज्ज्यो॑तिषं त्वा सादयामि बो॒धय॑न्तीं त्वा सादयामि॒ जाग्र॑तीं त्वा सादयामि॥३५॥

%1.4.35.0
{\anuvakamend[{ज्योति॑ष्मती॒ꣳ॒ षट्त्रिꣳ॑शत्}]}%॥34॥

%1.4.35.1
प्र॒या॒साय॒ स्वाहा॑\-ऽ\-ऽया॒साय॒ स्वाहा॑ विया॒साय॒ स्वाहा॑ संया॒साय॒ स्वाहो᳚द्या॒साय॒ स्वाहा॑\-ऽवया॒साय॒ स्वाहा॑ शु॒चे स्वाहा॒ शोका॑य॒ स्वाहा॑ तप्य॒त्वै स्वाहा॒ तप॑ते॒ स्वाहा᳚ ब्रह्मह॒त्यायै॒ स्वाहा॒ सर्व॑स्मै॒ स्वाहा᳚॥३६॥

%1.4.36.0
{\anuvakamend[{प्र॒या॒साय॒ चतु॑र्विꣳशतिः}]}%॥35॥

%1.4.36.1
चि॒त्तꣳ स॑न्ता॒नेन॑ भ॒वं य॒क्ना रु॒द्रं तनि॑म्ना पशु॒पतिꣴ॑ स्थूलहृद॒येना॒ग्निꣳ हृद॑येन रु॒द्रं लोहि॑तेन श॒र्वं मत॑स्नाभ्यां महादे॒वम॒न्तःपा᳚र्श्वेनौषिष्ठ॒हनꣳ॑ शिङ्गीनिको॒श्या᳚भ्याम्॥३७॥

%1.4.37.0
{\anuvakamend[{चि॒त्तम॒ष्टाद॑श}]}%॥36॥

%1.4.37.1
आ ति॑ष्ठ वृत्रह॒न् रथं॑ यु॒क्ता ते॒ ब्रह्म॑णा॒ हरी᳚। अ॒र्वा॒चीन॒ꣳ॒ सु ते॒ मनो॒ ग्रावा॑ कृणोतु व॒ग्नुना᳚॥ उ॒प॒या॒मगृ॑हीतो॒\-ऽसीन्द्रा॑य त्वा षोड॒शिन॑ ए॒ष ते॒ योनि॒रिन्द्रा॑य त्वा षोड॒शिने᳚॥३८॥

%1.4.38.0
{\anuvakamend[{आ ति॑ष्ठ॒ षड्विꣳ॑शतिः}]}%॥37॥

%1.4.38.1
इन्द्र॒मिद्धरी॑ वह॒तो\-ऽप्र॑तिधृष्टशवस॒मृषी॑णां च स्तु॒तीरुप॑ य॒ज्ञं च॒ मानु॑षाणाम्॥ उ॒प॒या॒मगृ॑हीतो॒\-ऽसीन्द्रा॑य त्वा षोड॒शिन॑ ए॒ष ते॒ योनि॒रिन्द्रा॑य त्वा षोड॒शिने᳚॥३९॥

%1.4.39.0
{\anuvakamend[{इन्द्र॒मित्त्रयो॑विꣳशतिः}]}%॥38॥

%1.4.39.1
असा॑वि॒ सोम॑ इन्द्र ते॒ शवि॑ष्ठ धृष्ण॒वा ग॑हि। आ त्वा॑ पृणक्त्विन्द्रि॒यꣳ रजः॒ सूर्यं॒ न र॒श्मिभिः॑॥ उ॒प॒या॒मगृ॑हीतो॒\-ऽसीन्द्रा॑य त्वा षोड॒शिन॑ ए॒ष ते॒ योनि॒रिन्द्रा॑य त्वा षोड॒शिने᳚॥४०॥

%1.4.40.0
{\anuvakamend[{असा॑वि स॒प्तविꣳ॑शतिः}]}%॥39॥

%1.4.40.1
सर्व॑स्य प्रति॒शीव॑री॒ भूमि॑स्त्वो॒पस्थ॒ आ\-ऽधि॑त। स्यो॒नास्मै॑ सु॒षदा॑ भव॒ यच्छा᳚स्मै॒ शर्म॑ स॒प्रथाः᳚॥ उ॒प॒या॒मगृ॑हीतो॒\-ऽसीन्द्रा॑य त्वा षोड॒शिन॑ ए॒ष ते॒ योनि॒रिन्द्रा॑य त्वा षोड॒शिने᳚॥४१॥

%1.4.41.0
{\anuvakamend[{सर्व॑स्य॒ षड्विꣳ॑शतिः}]}%॥40॥

%1.4.41.1
म॒हाꣳ इन्द्रो॒ वज्र॑बाहुः षोड॒शी शर्म॑ यच्छतु। स्व॒स्ति नो॑ म॒घवा॑ करोतु॒ हन्तु॑ पा॒प्मानं॒ यो᳚\-ऽस्मान् द्वेष्टि॑॥ उ॒प॒या॒मगृ॑हीतो॒\-ऽसीन्द्रा॑य त्वा षोड॒शिन॑ ए॒ष ते॒ योनि॒रिन्द्रा॑य त्वा षोड॒शिने᳚॥४२॥

%1.4.42.0
{\anuvakamend[{सर्व॑स्य म॒हान्थ्षड्विꣳ॑शति॒ष्षड्विꣳ॑शतिः}]}%॥41॥

%1.4.42.1
स॒जोषा॑ इन्द्र॒ सग॑णो म॒रुद्भिः॒ सोमं॑ पिब वृत्रहञ्छूर वि॒द्वान्। ज॒हि शत्रू॒ꣳ॒ रप॒ मृधो॑ नुद॒स्वा\-ऽथाभ॑यं कृणुहि वि॒श्वतो॑ नः॥ उ॒प॒या॒मगृ॑हीतो॒\-ऽसीन्द्रा॑य त्वा षोड॒शिन॑ ए॒ष ते॒ योनि॒रिन्द्रा॑य त्वा षोड॒शिने᳚॥४३॥

%1.4.43.0
{\anuvakamend[{स॒जोषा᳚स्त्रि॒ꣳ॒शत्}]}%॥42॥

%1.4.43.1
उदु॒ त्यं जा॒तवे॑दसं दे॒वं व॑हन्ति के॒तवः॑। दृ॒शे विश्वा॑य॒ सूर्यम्᳚॥ चि॒त्रं दे॒वाना॒मुद॑गा॒दनी॑कं॒ चक्षु॑र्मि॒त्रस्य॒ वरु॑णस्या॒ग्नेः। आ\-ऽप्रा॒ द्यावा॑पृथि॒वी अ॒न्तरि॑क्ष॒ꣳ॒ सूर्य॑ आ॒त्मा जग॑तस्त॒स्थुष॑श्च॥ अग्ने॒ नय॑ सु॒पथा॑ रा॒ये अ॒स्मान् विश्वा॑नि देव व॒युना॑नि वि॒द्वान्। यु॒यो॒ध्य॑स्मज्जु॑हुरा॒णमेनो॒ भूयि॑ष्ठां ते॒ नम॑उक्तिं विधेम॥ दिवं॑ गच्छ॒ सुवः॑ पत रू॒पेण॑॥४४॥

%1.4.43.2
वो रू॒पम॒भ्यैमि॒ वय॑सा॒ वयः॑। तु॒थो वो॑ वि॒श्ववे॑दा॒ वि भ॑जतु॒ वर्\mbox{}षि॑ष्ठे॒ अधि॒ नाके᳚॥ ए॒तत् ते॑ अग्ने॒ राध॒ ऐति॒ सोम॑च्युतं॒ तन्मि॒त्रस्य॑ प॒था न॑य॒र्तस्य॑ प॒था प्रेत॑ च॒न्द्रद॑क्षिणा य॒ज्ञस्य॑ प॒था सु॑वि॒ता नय॑न्तीर्ब्राह्म॒णम॒द्य रा᳚ध्यास॒मृषि॑मार्\mbox{}षे॒यं पि॑तृ॒मन्तं॑ पैतृम॒त्यꣳ सु॒धातु॑दक्षिणं॒ वि सुवः॒ पश्य॒ व्य॑न्तरि॑क्षं॒ यत॑स्व सद॒स्यै॑र॒स्मद्दा᳚त्रा देव॒त्रा ग॑च्छत॒ मधु॑मतीः प्रदा॒तार॒मा वि॑श॒तान॑वहाया॒स्मान् दे॑व॒याने॑न प॒थेत॑ सु॒कृतां᳚ लो॒के सी॑दत॒ तन्नः॑ सꣴस्कृ॒तम्॥४५॥

%1.4.44.0
{\anuvakamend[{रू॒पेण॑ सद॒स्यै॑र॒ष्टाद॑श च}]}%॥43 (37)॥

%1.4.44.1
धा॒ता रा॒तिः स॑वि॒तेदं जु॑षन्तां प्र॒जाप॑तिर्निधि॒पति॑र्नो अ॒ग्निः। त्वष्टा॒ विष्णुः॑ प्र॒जया॑ सꣳररा॒णो यज॑मानाय॒ द्रवि॑णं दधातु॥ समि॑न्द्र णो॒ मन॑सा नेषि॒ गोभिः॒ सꣳ सू॒रिभि॑र्मघव॒न्थ्सꣴ स्व॒स्त्या। सं ब्रह्म॑णा दे॒वकृ॑तं॒ यदस्ति॒ सं दे॒वानाꣳ॑ सुम॒त्या य॒ज्ञिया॑नाम्॥ सं वर्च॑सा॒ पय॑सा॒ सं त॒नूभि॒रग॑न्महि॒ मन॑सा॒ सꣳ शि॒वेन॑। त्वष्टा॑ नो॒ अत्र॒ वरि॑वः कृणो॒-॥४६॥

%1.4.44.2
त्वनु॑ मार्ष्टु त॒नुवो॒ यद्विलि॑ष्टम्॥ यद॒द्य त्वा᳚ प्रय॒ति य॒ज्ञे अ॒स्मिन्नग्ने॒ होता॑र॒मवृ॑णीमही॒ह। ऋध॑गया॒डृध॑गु॒ताश॑मिष्ठाः प्रजा॒नन् य॒ज्ञमुप॑याहि वि॒द्वान्॥ स्व॒गा वो॑ देवाः॒ सद॑नमकर्म॒ य आ॑ज॒ग्म सव॑ने॒दं जु॑षा॒णाः। ज॒क्षि॒वाꣳसः॑ पपि॒वाꣳस॑श्च॒ विश्वे॒\-ऽस्मे ध॑त्त वसवो॒ वसू॑नि॥ याना\-ऽव॑ह उश॒तो दे॑व दे॒वान्तान्॥४७॥

%1.4.44.3
प्रेर॑य॒ स्वे अ॑ग्ने स॒धस्थे᳚। वह॑माना॒ भर॑माणा ह॒वीꣳषि॒ वसुं॑ घ॒र्मं दिव॒मा ति॑ष्ठ॒तानु॑॥ यज्ञ॑ य॒ज्ञं ग॑च्छ य॒ज्ञप॑तिं गच्छ॒ स्वां योनिं॑ गच्छ॒ स्वाहै॒ष ते॑ य॒ज्ञो य॑ज्ञपते स॒हसू᳚क्तवाकः सु॒वीरः॒ स्वाहा॒ देवा॑ गातुविदो गा॒तुं वि॒त्वा गा॒तुमि॑त॒ मन॑सस्पत इ॒मं नो॑ देव दे॒वेषु॑ य॒ज्ञꣴ स्वाहा॑ वा॒चि स्वाहा॒ वाते॑ धाः॥४८॥

%1.4.45.0
{\anuvakamend[{कृ॒णो॒तु॒ तान॒ष्टाच॑त्वारिꣳशच्च}]}%॥44 (38)॥

%1.4.45.1
उ॒रुꣳ हि राजा॒ वरु॑णश्च॒कार॒ सूर्या॑य॒ पन्था॒मन्वे॑त॒वा उ॑। अ॒पदे॒ पादा॒ प्रति॑धातवे\-ऽकरु॒ताप॑व॒क्ता हृ॑दया॒विध॑श्चित्॥ श॒तं ते॑ राजन् भि॒षजः॑ स॒हस्र॑मु॒र्वी ग॑म्भी॒रा सु॑म॒तिष्टे॑ अस्तु। बाध॑स्व॒ द्वेषो॒ निर्\mbox{}ऋ॑तिं परा॒चैः कृ॒तं चि॒देनः॒ प्र मु॑मुग्ध्य॒स्मत्॥ अ॒भिष्ठि॑तो॒ वरु॑णस्य॒ पाशो॒\-ऽग्नेरनी॑कम॒प आ वि॑वेश। अपां᳚ नपात् प्रति॒रक्ष॑न्नसु॒र्यं॑ दमे॑दमे॥४९॥

%1.4.45.2
स॒मिधं॑ यक्ष्यग्ने॥ प्रति॑ ते जि॒ह्वा घृ॒तमुच्च॑रण्येथ्समु॒द्रे ते॒ हृद॑यम॒फ्स्व॑न्तः। सं त्वा॑ विश॒न्त्वोष॑धीरु॒ता\-ऽ\-ऽपो॑ य॒ज्ञस्य॑ त्वा यज्ञपते ह॒विर्भिः॑॥ सू॒क्त॒वा॒के न॑मोवा॒के वि॑धे॒माव॑भृथ निचङ्कुण निचे॒रुर॑सि निचङ्कु॒णाव॑ दे॒वैर्दे॒वकृ॑त॒मेनो॑\-ऽया॒डव॒ मर्त्यै॒र्मर्त्य॑कृतमु॒रोरा नो॑ देव रि॒षस्पा॑हि सुमि॒त्रा न॒ आप॒ ओष॑धयः॥५०॥

%1.4.45.3
सन्तु दुर्मि॒त्रास्तस्मै॑ भूयासु॒र्यो᳚\-ऽस्मान् द्वेष्टि॒ यं च॑ व॒यं द्वि॒ष्मो देवी॑राप ए॒ष वो॒ गर्भ॒स्तं वः॒ सुप्री॑त॒ꣳ॒ सुभृ॑तमकर्म दे॒वेषु॑ नः सु॒कृतो᳚ ब्रूता॒त् प्रति॑युतो॒ वरु॑णस्य॒ पाशः॒ प्रत्य॑स्तो॒ वरु॑णस्य॒ पाश॒ एधो᳚\-ऽस्येधिषी॒महि॑ स॒मिद॑सि॒ तेजो॑\-ऽसि॒ तेजो॒ मयि॑ धेह्य॒पो अन्व॑चारिष॒ꣳ॒ रसे॑न॒ सम॑सृक्ष्महि। पय॑स्वाꣳ अग्न॒ आ\-ऽग॑मं॒ तं मा॒ सꣳ सृ॑ज॒ वर्च॑सा॥५१॥

%1.4.46.0
{\anuvakamend[{दमे॑दम॒ ओष॑धय॒ आ षट् च॑}]}%॥45 (39)॥

%1.4.46.1
यस्त्वा॑ हृ॒दा की॒रिणा॒ मन्य॑मा॒नो\-ऽम॑र्त्यं॒ मर्त्यो॒ जोह॑वीमि। जात॑वेदो॒ यशो॑ अ॒स्मासु॑ धेहि प्र॒जाभि॑रग्ने अमृत॒त्वम॑श्याम्॥ यस्मै॒ त्वꣳ सु॒कृते॑ जातवेद॒ उ लो॒कम॑ग्ने कृ॒णवः॑ स्यो॒नम्। अ॒श्विन॒ꣳ॒ स पु॒त्रिणं॑ वी॒रव॑न्तं॒ गोम॑न्तꣳ र॒यिं न॑शते स्व॒स्ति॥ त्वे सु पु॑त्र शव॒सो\-ऽवृ॑त्र॒न् काम॑कातयः। न त्वामि॒न्द्राति॑ रिच्यते॥ उ॒क्थउ॑क्थे॒ सोम॒ इन्द्रं॑ ममाद नी॒थेनी॑थे म॒घवा॑नꣳ॥५२॥

%1.4.46.2
सु॒तासः॑। यदीꣳ॑ स॒बाधः॑ पि॒तरं॒ न पु॒त्राः स॑मा॒नद॑क्षा॒ अव॑से॒ हव॑न्ते॥ अग्ने॒ रसे॑न॒ तेज॑सा॒ जात॑वेदो॒ वि रो॑चसे। र॒क्षो॒हा\-ऽमी॑व॒चात॑नः॥ अ॒पो अन्व॑चारिष॒ꣳ॒ रसे॑न॒ सम॑सृक्ष्महि। पय॑स्वाꣳ अग्न॒ आ\-ऽग॑मं॒ तं मा॒ सꣳ सृ॑ज॒ वर्च॑सा॥ वसु॒र्वसु॑पति॒र्\mbox{}हिक॒मस्य॑ग्ने वि॒भाव॑सुः। स्याम॑ ते सुम॒तावपि॑॥ त्वाम॑ग्ने॒ वसु॑पतिं॒ वसू॑नाम॒भि प्र म॑न्दे॥५३॥

%1.4.46.3
अध्व॒रेषु॑ राजन्न्। त्वया॒ वाजं॑ वाज॒यन्तो॑ जयेमा॒भि ष्या॑म पृथ्सु॒तीर्मर्त्या॑नाम्। त्वाम॑ग्ने वाज॒सात॑मं॒ विप्रा॑ वर्धन्ति॒ सुष्टु॑तम्। स नो॑ रास्व सु॒वीर्यम्᳚॥ अ॒यं नो॑ अ॒ग्निर्वरि॑वः कृणोत्व॒यं मृधः॑ पु॒र ए॑तु प्रभि॒न्दन्न्। अ॒यꣳ शत्रू᳚ञ्जयतु॒ जर्\mbox{}हृ॑षाणो॒\-ऽयं वाजं॑ जयतु॒ वाज॑सातौ॥ अ॒ग्निना॒ऽग्निः समि॑ध्यते क॒विर्गृ॒हप॑ति॒र्युवा᳚। ह॒व्य॒वाड् जु॒ह्वा᳚स्यः॥ त्वꣴ ह्य॑ग्ने अ॒ग्निना॒ विप्रो॒ विप्रे॑ण॒ सन्थ्स॒ता। सखा॒ सख्या॑ समि॒ध्यसे᳚॥ उद॑ग्ने॒ शुच॑य॒स्तव॒ वि ज्योति॑षा॥५४॥

{\anuvakamend[{म॒घवा॑नं मन्दे॒ ह्य॑ग्ने॒ चतु॑र्दश च}]}%॥46॥
%%% END PRASHNA

\sect{पञ्चमः प्रश्नः}\setcounter{anuvakam}{0}
\dnsub{तैत्तिरीयसंहितायां प्रथमकाण्डे पञ्चमः प्रश्नः}
%1.5.1.0
%1.5.1.1
दे॒वा॒सु॒राः संय॑त्ता आस॒न् ते दे॒वा वि॑ज॒यमु॑प॒यन्तो॒\-ऽग्नौ वा॒मं वसु॒ सं न्य॑दधते॒दमु॑ नो भविष्यति॒ यदि॑ नो जे॒ष्यन्तीति॒ तद॒ग्निर्न्य॑कामयत॒ तेनापा᳚क्राम॒त् तद्दे॒वा वि॒जित्या॑व॒रुरु॑थ्समाना॒ अन्वा॑य॒न् तद॑स्य॒ सह॒सा\-ऽदि॑थ्सन्त॒ सो॑\-ऽरोदी॒द्यदरो॑दी॒त् तद्रु॒द्रस्य॑ रुद्र॒त्वं यदश्र्वशी॑यत॒ तद्॥१॥

%1.5.1.2
र॑ज॒तꣳ हिर॑ण्यमभव॒त् तस्मा᳚द्रज॒तꣳ हिर॑ण्यमदक्षि॒ण्य\-म॑श्रु॒जꣳ हि यो ब॒र्॒\mbox{}हिषि॒ ददा॑ति पु॒रा\-ऽस्य॑ संवथ्स॒राद्गृ॒हे रु॑दन्ति॒ तस्मा᳚द्ब॒र्॒\mbox{}हिषि॒ न देय॒ꣳ॒ सो᳚\-ऽग्निर॑ब्रवीद्भा॒ग्य॑सा॒न्यथ॑ व इ॒दमिति॑ पुनरा॒धेयं॑ ते॒ केव॑ल॒मित्य॑ब्रुवन्नृ॒ध्नव॒त् खलु॒ स इत्य॑ब्रवी॒द्यो म॑द्देव॒त्य॑म॒ग्निमा॒दधा॑ता॒ इति॒ तं पू॒षा\-ऽ\-ऽध॑त्त॒ तेन॑॥२॥

%1.5.1.3
पू॒षा\-ऽ\-ऽर्ध्नो॒त् तस्मा᳚त् पौ॒ष्णाः प॒शव॑ उच्यन्ते॒ तं त्वष्टा\-ऽ\-ऽध॑त्त॒ तेन॒ त्वष्टा᳚\-ऽ\-ऽर्ध्नो॒त् तस्मा᳚त् त्वा॒ष्ट्राः प॒शव॑ उच्यन्ते॒ तं मनु॒रा\-ऽध॑त्त॒ तेन॒ मनु॑रार्ध्नो॒त् तस्मा᳚न्मान॒व्यः॑ प्र॒जा उ॑च्यन्ते॒ तं धा॒ता\-ऽ\-ऽध॑त्त॒ तेन॑ धा॒ता\-ऽ\-ऽर्ध्नो᳚थ्संवथ्स॒रो वै धा॒ता तस्मा᳚थ्संवथ्स॒रं प्र॒जाः प॒शवो\-ऽनु॒ प्र जा॑यन्ते॒ य ए॒वं पु॑नरा॒धेय॒स्यर्द्धिं॒ वे-॥३॥

%1.5.1.4
द॒र्ध्नोत्ये॒व यो᳚\-ऽस्यै॒वं ब॒न्धुतां॒ वेद॒ बन्धु॑मान् भवति भाग॒धेयं॒ वा अ॒ग्निराहि॑त इ॒च्छमा॑नः प्र॒जां प॒शून् यज॑मान॒स्योप॑ दोद्रावो॒द्वास्य॒ पुन॒रा द॑धीत भाग॒धेये॑नै॒वैन॒ꣳ॒ सम॑र्धय॒त्यथो॒ शान्ति॑रे॒वास्यै॒षा पुन॑र्वस्वो॒रा द॑धीतै॒तद्वै पु॑नरा॒धेय॑स्य॒ नक्ष॑त्रं॒ यत्पुन॑र्वसू॒ स्वाया॑मे॒वैनं॑ दे॒वता॑यामा॒धाय॑ ब्रह्मवर्च॒सी भ॑वति द॒र्भैरा द॑धा॒त्यया॑तयामत्वाय द॒र्भैरा द॑धात्य॒द्भ्य ए॒वैन॒मोष॑धीभ्यो\-ऽव॒रुध्या\-ऽ\-ऽध॑त्ते॒ पञ्च॑कपालः पुरो॒डाशो॑ भवति॒ पञ्च॒ वा ऋ॒तव॑ ऋ॒तुभ्य॑ ए॒वैन॑मव॒रुध्या\-ऽ\-ऽध॑त्ते॥४॥

%1.5.2.0
{\anuvakamend[{अशी॑यत॒ तत् तेन॒ वेद॑ द॒र्भैः पञ्च॑विꣳशतिश्च}]}%॥१॥

%1.5.2.1
परा॒ वा ए॒ष य॒ज्ञं प॒शून् व॑पति॒ यो᳚\-ऽग्निमु॑द्वा॒सय॑ते॒ पञ्च॑कपालः पुरो॒डाशो॑ भवति॒ पाङ्क्तो॑ य॒ज्ञः पाङ्क्ताः᳚ प॒शवो॑ य॒ज्ञमे॒व प॒शूनव॑ रुन्धे वीर॒हा वा ए॒ष दे॒वानां॒ यो᳚\-ऽग्निमु॑द्वा॒सय॑ते॒ न वा ए॒तस्य॑ ब्राह्म॒णा ऋ॑ता॒यवः॑ पु॒रा\-ऽन्न॑मक्षन् प॒ङ्क्त्यो॑ याज्यानुवा॒क्या॑ भवन्ति॒ पाङ्क्तो॑ य॒ज्ञः पाङ्क्तः॒ पुरु॑षो दे॒वाने॒व वी॒रं नि॑रव॒दाया॒ग्निं पुन॒रा॥५॥

%1.5.2.2
ध॑त्ते श॒ताक्ष॑रा भवन्ति श॒तायुः॒ पुरु॑षः श॒तेन्द्रि॑य॒ आयु॑ष्ये॒वेन्द्रि॒ये प्रति॑ तिष्ठति॒ यद्वा अ॒ग्निराहि॑तो॒ नर्ध्यते॒ ज्यायो॑ भाग॒धेयं॑ निका॒मय॑मानो॒ यदा᳚ग्ने॒यꣳ सर्वं॒ भव॑ति॒ सैवास्यर्धिः॒ सं वा ए॒तस्य॑ गृ॒हे वाक् सृ॑ज्यते॒ यो᳚\-ऽग्निमु॑द्वा॒सय॑ते॒ स वाच॒ꣳ॒ सꣳसृ॑ष्टां॒ यज॑मान ईश्व॒रो\-ऽनु॒ परा॑भवितो॒र्विभ॑क्तयो भवन्ति वा॒चो विधृ॑त्यै॒ यज॑मान॒स्याप॑राभावाय॒॥६॥

%1.5.2.3
विभ॑क्तिं करोति॒ ब्रह्मै॒व तद॑करुपा॒ꣳ॒शु य॑जति॒ यथा॑ वा॒मं वसु॑ विविदा॒नो गूह॑ति ता॒दृगे॒व तद॒ग्निं प्रति॑ स्विष्ट॒कृतं॒ निरा॑ह॒ यथा॑ वा॒मं वसु॑ विविदा॒नः प्र॑का॒शं जिग॑मिषति ता॒दृगे॒व तद्विभ॑क्तिमु॒क्त्वा प्र॑या॒जेन॒ वष॑ट्करोत्या॒यत॑नादे॒व नैति॒ यज॑मानो॒ वै पु॑रो॒डाशः॑ प॒शव॑ ए॒ते आहु॑ती॒ यद॒भितः॑ पुरो॒डाश॑मे॒ते आहु॑ती॥७॥

%1.5.2.4
जु॒होति॒ यज॑मानमे॒वोभ॒यतः॑ प॒शुभिः॒ परि॑ गृह्णाति कृ॒तय॑जुः॒ सम्भृ॑तसम्भार॒ इत्या॑हु॒र्न स॒म्भृत्याः᳚ सम्भा॒रा न यजुः॑ कर्त॒व्य॑मित्यथो॒ खलु॑ स॒म्भृत्या॑ ए॒व स॑म्भा॒राः क॑र्त॒व्यं॑ यजु॑र्य॒ज्ञस्य॒ समृ॑द्ध्यै पुनर्निष्कृ॒तो रथो॒ दक्षि॑णा पुनरुथ्स्यू॒तं वासः॑ पुनरुथ्सृ॒ष्टो॑\-ऽन॒ड्वान् पु॑नरा॒धेय॑स्य॒ समृ॑द्ध्यै स॒प्त ते॑ अग्ने स॒मिधः॑ स॒प्त जि॒ह्वा इत्य॑ग्निहो॒त्रं जु॑होति॒ यत्र॑यत्रै॒वास्य॒ न्य॑क्तं॒ तत॑॥८॥

%1.5.2.5
ए॒वैन॒मव॑ रुन्धे वीर॒हा वा ए॒ष दे॒वानां॒ यो᳚\-ऽग्निमु॑द्वा॒सय॑ते॒ तस्य॒ वरु॑ण ए॒वर्ण॒यादा᳚ग्निवारु॒णमेका॑दशकपाल॒मनु॒ निर्व॑पे॒द्यं चै॒व हन्ति॒ यश्चा᳚स्यर्ण॒यात्तौ भा॑ग॒धेये॑न प्रीणाति॒ ना\-ऽ\-ऽर्ति॒मार्च्छ॑ति॒ यज॑मानः॥९॥

%1.5.3.0
{\anuvakamend[{आ\-ऽप॑राभावाय पुरो॒डाश॑मे॒ते आहु॑ती॒ तत॒ष्षट्त्रिꣳ॑शच्च}]}%॥२॥

%1.5.3.1
भूमि॑र्भू॒म्ना द्यौर्व॑रि॒णा\-ऽन्तरि॑क्षं महि॒त्वा। उ॒पस्थे॑ ते देव्यदिते॒\-ऽग्निम॑न्ना॒दम॒न्नाद्या॒या\-ऽ\-ऽद॑धे॥ आ\-ऽयं गौः पृश्ञि॑रक्रमी॒दस॑नन्मा॒तरं॒ पुनः॑। पि॒तरं॑ च प्र॒यन्थ्सुवः॑॥ त्रि॒ꣳ॒शद्धाम॒ वि रा॑जति॒ वाक्प॑त॒ङ्गाय॑ शिश्रिये। प्रत्य॑स्य वह॒ द्युभिः॑॥ अ॒स्य प्रा॒णाद॑पान॒त्य॑न्तश्च॑रति रोच॒ना। व्य॑ख्यन्महि॒षः सुवः॑॥ यत् त्वा᳚॥१०॥

%1.5.3.2
क्रु॒द्धः प॑रो॒वप॑ म॒न्युना॒ यदव॑र्त्या। सु॒कल्प॑मग्ने॒ तत् तव॒ पुन॒स्त्वोद्दी॑पयामसि॥ यत् ते॑ म॒न्युप॑रोप्तस्य पृथि॒वीमनु॑ दध्व॒से। आ॒दि॒त्या विश्वे॒ तद्दे॒वा वस॑वश्च स॒माभ॑रन्न्॥ मनो॒ ज्योति॑र्जुषता॒माज्यं॒ विच्छि॑न्नं य॒ज्ञꣳ समि॒मं द॑धातु। बृह॒स्पति॑स्तनुतामि॒मं नो॒ विश्वे॑ दे॒वा इ॒ह मा॑दयन्ताम्॥ स॒प्त ते॑ अग्ने स॒मिधः॑ स॒प्त जि॒ह्वाः स॒प्त॥११॥

%1.5.3.3
ऋष॑यः स॒प्त धाम॑ प्रि॒याणि॑। स॒प्त होत्राः᳚ सप्त॒धा त्वा॑ यजन्ति स॒प्त योनी॒रा पृ॑णस्वा घृ॒तेन॑॥ पुन॑रू॒र्जा नि व॑र्तस्व॒ पुन॑रग्न इ॒षा\-ऽ\-ऽयु॑षा। पुन॑र्नः पाहि वि॒श्वतः॑॥ स॒ह र॒य्या नि व॑र्त॒स्वाग्ने॒ पिन्व॑स्व॒ धार॑या। वि॒श्वफ्स्नि॑या वि॒श्वत॒स्परि॑॥ लेकः॒ सले॑कः सु॒लेक॒स्ते न॑ आदि॒त्या आज्यं॑ जुषा॒णा वि॑यन्तु॒ केतः॒ सके॑तः सु॒केत॒स्ते न॑ आदि॒त्या आज्यं॑ जुषा॒णा वि॑यन्तु॒ विव॑स्वा॒ꣳ॒ अदि॑ति॒र्देव॑जूति॒स्ते न॑ आदि॒त्या आज्यं॑ जुषा॒णा वि॑यन्तु॥१२॥

%1.5.4.0
{\anuvakamend[{त्वा॒ जि॒ह्वाः स॒प्त सु॒केत॒स्ते न॒स्त्रयो॑दश च}]}%॥३॥

%1.5.4.1
भूमि॑र्भू॒म्ना द्यौर्व॑रि॒णेत्या॑हा॒\-ऽ\-ऽशिषै॒वैन॒मा ध॑त्ते स॒र्पा वै जीर्य॑न्तो\-ऽमन्यन्त॒ स ए॒तं क॑स॒र्णीरः॑ काद्रवे॒यो मन्त्र॑मपश्य॒त् ततो॒ वै ते जी॒र्णास्त॒नूरपा᳚घ्नत सर्परा॒ज्ञिया॑ ऋ॒ग्भिर्गार्\mbox{}ह॑पत्य॒मा द॑धाति पुनर्न॒वमे॒वैन॑म॒जरं॑ कृ॒त्वा\-ऽ\-ऽध॒त्ते\-ऽथो॑ पू॒तमे॒व पृ॑थि॒वीम॒न्नाद्यं॒ नोपा॑नम॒थ्सैतं॥१३॥

%1.5.4.2
मन्त्र॑मपश्य॒त् ततो॒ वै ताम॒न्नाद्य॒मुपा॑नम॒द्यथ्स॑र्परा॒ज्ञिया॑ ऋ॒ग्भिर्गार्\mbox{}ह॑पत्यमा॒दधा᳚त्य॒न्नाद्य॒स्याव॑रुद्ध्या॒ अथो॑ अ॒स्यामे॒वैनं॒ प्रति॑ष्ठित॒मा ध॑त्ते॒ यत्त्वा᳚ क्रु॒द्धः प॑रो॒वपेत्या॒हाप॑ह्नुत ए॒वास्मै॒ तत् पुन॒स्त्वोद्दी॑पयाम॒सीत्या॑ह॒ समि॑न्ध ए॒वैनं॒ यत्ते॑ म॒न्युप॑रोप्त॒स्येत्या॑ह दे॒वता॑भिरे॒-॥१४॥

%1.5.4.3
वैन॒ꣳ॒ सं भ॑रति॒ वि वा ए॒तस्य॑ य॒ज्ञश्छि॑द्यते॒ यो᳚\-ऽग्निमु॑द्वा॒सय॑ते॒ बृह॒स्पति॑वत्य॒र्चोप॑ तिष्ठते॒ ब्रह्म॒ वै दे॒वानां॒ बृह॒स्पति॒र्ब्रह्म॑णै॒व य॒ज्ञꣳ सं द॑धाति॒ विच्छि॑न्नं य॒ज्ञꣳ समि॒मं द॑धा॒त्वित्या॑ह॒ सन्त॑त्यै॒ विश्वे॑ दे॒वा इ॒ह मा॑दयन्ता॒मित्या॑ह स॒न्तत्यै॒व य॒ज्ञं दे॒वेभ्यो\-ऽनु॑ दिशति स॒प्त ते॑ अग्ने स॒मिधः॑ स॒प्त जि॒ह्वाः॥१५॥

%1.5.4.4
इत्या॑ह स॒प्तस॑प्त॒ वै स॑प्त॒धा\-ऽग्नेः प्रि॒यास्त॒नुव॒स्ता ए॒वाव॑ रुन्धे॒ पुन॑रू॒र्जा स॒ह र॒य्येत्य॒भितः॑ पुरो॒डाश॒माहु॑ती जुहोति॒ यज॑मानमे॒वोर्जा च॑ र॒य्या चो॑भ॒यतः॒ परि॑ गृह्णात्यादि॒त्या वा अ॒स्माल्लो॒काद॒मुं लो॒कमा॑य॒न्ते॑\-ऽमुष्मि॑ल्लोँ॒के व्य॑तृष्य॒न्त इ॒मं लो॒कं पुन॑रभ्य॒वेत्या॒ग्निमा॒धायै॒तान् होमा॑नजुहवु॒स्त आ᳚र्ध्नुव॒न् ते सु॑व॒र्गँल्लो॒कमा॑य॒न्॒ यः प॑रा॒चीनं॑ पुनरा॒धेया॑द॒ग्निमा॒दधी॑त॒ स ए॒तान् होमा᳚ञ्जुहुया॒द्यामे॒वा\-ऽ\-ऽदि॒त्या ऋद्धि॒मार्ध्नु॑व॒न् तामे॒वर्ध्नो॑ति॥१६॥

%1.5.5.0
{\anuvakamend[{ए॒तमे॒व जि॒ह्वा ए॒तान् पञ्च॑विꣳशतिश्च}]}%॥४॥

%1.5.5.1
उ॒प॒प्र॒यन्तो॑ अध्व॒रं मन्त्रं॑ वोचेमा॒ग्नये᳚। आ॒रे अ॒स्मे च॑ शृण्व॒ते॥ अ॒स्य प्र॒त्नामनु॒ द्युतꣳ॑ शु॒क्रं दु॑दुह्रे॒ अह्र॑यः। पयः॑ सहस्र॒सामृषिम्᳚॥ अ॒ग्निर्मू॒र्धा दि॒वः क॒कुत् पतिः॑ पृथि॒व्या अ॒यम्। अ॒पाꣳ रेताꣳ॑सि जिन्वति॥ अ॒यमि॒ह प्र॑थ॒मो धा॑यि धा॒तृभि॒र्\mbox{}होता॒ यजि॑ष्ठो अध्व॒रेष्वीड्यः॑। यमप्न॑वानो॒ भृग॑वो विरुरु॒चुर्वने॑षु चि॒त्रं वि॒भुवं॑ वि॒शेवि॑शे॥ उ॒भा वा॑मिन्द्राग्नी आहु॒वध्या॑॥१७॥

%1.5.5.2
उ॒भा राध॑सः स॒ह मा॑द॒यध्यै᳚। उ॒भा दा॒तारा॑वि॒षाꣳ र॑यी॒णामु॒भा वाज॑स्य सा॒तये॑ हुवे वाम्॥ अ॒यं ते॒ योनि॑र्\mbox{}ऋ॒त्वियो॒ यतो॑ जा॒तो अरो॑चथाः। तं जा॒नन्न॑ग्न॒ आ रो॒हाथा॑ नो वर्धया र॒यिम्॥ अग्न॒ आयूꣳ॑षि पवस॒ आ सु॒वोर्ज॒मिषं॑ च नः। आ॒रे बा॑धस्व दु॒च्छुना᳚म्॥ अग्ने॒ पव॑स्व॒ स्वपा॑ अ॒स्मे वर्चः॑ सु॒वीर्यम्᳚। दध॒त्पोषꣳ॑ र॒यिं॥१८॥

%1.5.5.3
मयि॑॥ अग्ने॑ पावक रो॒चिषा॑ म॒न्द्रया॑ देव जि॒ह्वया᳚। आ दे॒वान् व॑क्षि॒ यक्षि॑ च॥ स नः॑ पावक दीदि॒वो\-ऽग्ने॑ दे॒वाꣳ इ॒हा\-ऽ\-ऽव॑ह। उप॑ य॒ज्ञꣳ ह॒विश्च॑ नः॥ अ॒ग्निः शुचि॑व्रततमः॒ शुचि॒र्विप्रः॒ शुचिः॑ क॒विः। शुची॑ रोचत॒ आहु॑तः॥ उद॑ग्ने॒ शुच॑य॒स्तव॑ शु॒क्रा भ्राज॑न्त ईरते। तव॒ ज्योतीꣴ॑ष्य॒र्चयः॑॥ आ॒यु॒र्दा अ॑ग्ने॒\-ऽस्यायु॑र्मे॥१९॥

%1.5.5.4
देहि वर्चो॒दा अ॑ग्ने\-ऽसि॒ वर्चो॑ मे देहि तनू॒पा अ॑ग्ने\-ऽसि त॒नुवं॑ मे पा॒ह्यग्ने॒ यन्मे॑ त॒नुवा॑ ऊ॒नं तन्म॒ आ पृ॑ण॒ चित्रा॑वसो स्व॒स्ति ते॑ पा॒रम॑शी॒येन्धा॑नास्त्वा श॒तꣳ हिमा᳚ द्यु॒मन्तः॒ समि॑धीमहि॒ वय॑स्वन्तो वय॒स्कृतं॒ यश॑स्वन्तो यश॒स्कृतꣳ॑ सु॒वीरा॑सो॒ अदा᳚भ्यम्। अग्ने॑ सपत्न॒दम्भ॑नं॒ वर्\mbox{}षि॑ष्ठे॒ अधि॒ नाके᳚॥ सं त्वम॑ग्ने॒ सूर्य॑स्य॒ वर्च॑सा\-ऽगथाः॒ समृषी॑णाꣴ स्तु॒तेन॒ सं प्रि॒येण॒ धाम्ना᳚। त्वम॑ग्ने॒ सूर्य॑वर्चा असि॒ सं मामायु॑षा॒ वर्च॑सा प्र॒जया॑ सृज॥२०॥

%1.5.6.0
{\anuvakamend[{आ॒हु॒वध्यै॑ र॒यिं मे॒ वर्च॑सा स॒प्तद॑श च}]}%॥५॥

%1.5.6.1
सं प॑श्यामि प्र॒जा अ॒हमिड॑प्रजसो मान॒वीः। सर्वा॑ भवन्तु नो गृ॒हे॥ अम्भः॒ स्थाम्भो॑ वो भक्षीय॒ महः॑ स्थ॒ महो॑ वो भक्षीय॒ सहः॑ स्थ॒ सहो॑ वो भक्षी॒योर्जः॒ स्थोर्जं॑ वो भक्षीय॒ रेव॑ती॒ रम॑ध्वम॒स्मिल्लोँ॒के᳚\-ऽस्मिन् गो॒ष्ठे᳚\-ऽस्मिन् क्षये॒\-ऽस्मिन् योना॑वि॒हैव स्ते॒तो मा\-ऽप॑ गात ब॒ह्वीर्मे॑ भूयास्त॥२१॥

%1.5.6.2
सꣳहि॒तासि॑ विश्वरू॒पीरा मो॒र्जा वि॒शा\-ऽ\-ऽगौ॑प॒त्येना\-ऽ\-ऽरा॒यस्पोषे॑ण सहस्रपो॒षं वः॑ पुष्यासं॒ मयि॑ वो॒ रायः॑ श्रयन्ताम्॥ उप॑ त्वा\-ऽग्ने दि॒वेदि॑वे॒ दोषा॑वस्तर्धि॒या व॒यम्। नमो॒ भर॑न्त॒ एम॑सि। राज॑न्तमध्व॒राणां᳚ गो॒पामृ॒तस्य॒ दीदि॑विम्। वर्ध॑मान॒ꣴ॒ स्वे दमे᳚॥ स नः॑ पि॒तेव॑ सू॒नवे\-ऽग्ने॑ सूपाय॒नो भ॑व। सच॑स्वा नः स्व॒स्तये᳚॥ अग्ने॒॥२२॥

%1.5.6.3
त्वं नो॒ अन्त॑मः। उ॒त त्रा॒ता शि॒वो भ॑व वरू॒थ्यः॑॥ तं त्वा॑ शोचिष्ठ दीदिवः। सु॒म्नाय॑ नू॒नमी॑महे॒ सखि॑भ्यः॥ वसु॑र॒ग्निर्वसु॑श्रवाः। अच्छा॑ नक्षि द्यु॒मत्त॑मो र॒यिं दाः᳚॥ ऊ॒र्जा वः॑ पश्याम्यू॒र्जा मा॑ पश्यत रा॒यस्पोषे॑ण वः पश्यामि रा॒यस्पोषे॑ण मा पश्य॒तेडाः᳚ स्थ मधु॒कृतः॑ स्यो॒ना मा\-ऽ\-ऽवि॑श॒तेरा॒ मदः॑। स॒ह॒स्र॒पो॒षं वः॑ पुष्यासं॒॥२३॥

%1.5.6.4
मयि॑ वो॒ रायः॑ श्रयन्ताम्॥ तथ्स॑वि॒तुर्वरे᳚ण्यं॒ भर्गो॑ दे॒वस्य॑ धीमहि। धियो॒ यो नः॑ प्रचो॒दया᳚त्॥ सो॒मान॒ꣴ॒ स्वर॑णं कृणु॒हि ब्र॑ह्मणस्पते। क॒क्षीव॑न्तं॒ य औ॑शि॒जम्॥ क॒दा च॒न स्त॒रीर॑सि॒ नेन्द्र॑ सश्चसि दा॒शुषे᳚। उपो॒पेन्नु म॑घव॒न् भूय॒ इन्नु ते॒ दानं॑ दे॒वस्य॑ पृच्यते॥ परि॑ त्वाऽग्ने॒ पुरं॑ व॒यं विप्रꣳ॑ सहस्य धीमहि। धृ॒षद्व॑र्णं दि॒वेदि॑वे भे॒त्तारं॑ भङ्गु॒राव॑तः॥ अग्ने॑ गृहपते सुगृहप॒तिर॒हं त्वया॑ गृ॒हप॑तिना भूयासꣳ सुगृहप॒तिर्मया॒ त्वं गृ॒हप॑तिना भूयाः श॒तꣳ हिमा॒स्तामा॒शिष॒मा शा॑से॒ तन्त॑वे॒ ज्योति॑ष्मतीं॒ तामा॒शिष॒मा शा॑से॒\-ऽमुष्मै॒ ज्योति॑ष्मतीम्॥२४॥

%1.5.7.0
{\anuvakamend[{भू॒या॒स्त॒ स्व॒स्तये\-ऽग्ने॑ पुष्यासं धृ॒षद्व॑र्ण॒मेका॒न्नत्रि॒ꣳ॒शच्च॑}]}%॥६॥

%1.5.7.1
अय॑ज्ञो॒ वा ए॒ष यो॑\-ऽसा॒मोप॑प्र॒यन्तो॑ अध्व॒रमित्या॑ह॒ स्तोम॑मे॒वास्मै॑ युन॒क्त्युपेत्या॑ह प्र॒जा वै प॒शव॒ उपे॒मं लो॒कं प्र॒जामे॒व प॒शूनि॒मं लो॒कमुपै᳚त्य॒स्य प्र॒त्नामनु॒द्युत॒मित्या॑ह सुव॒र्गो वै लो॒कः प्र॒त्नः सु॑व॒र्गमे॒व लो॒कꣳ स॒मारो॑हत्य॒ग्निर्मू॒र्धा दि॒वः क॒कुदित्या॑ह मू॒र्धान॑-॥२५॥

%1.5.7.2
मे॒वैनꣳ॑ समा॒नानां᳚ करो॒त्यथो॑ देवलो॒कादे॒व म॑नुष्यलो॒के प्रति॑तिष्ठत्य॒यमि॒ह प्र॑थ॒मो धा॑यि धा॒तृभि॒रित्या॑ह॒ मुख्य॑मे॒वैनं॑ करोत्यु॒भा वा॑मिन्द्राग्नी आहु॒वध्या॒ इत्या॒हौजो॒ बल॑मे॒वाव॑ रुन्धे॒\-ऽयं ते॒ योनि॑र्\mbox{}ऋ॒त्विय॒ इत्या॑ह प॒शवो॒ वै र॒यिः प॒शूने॒वाव॑ रुन्धे ष॒ड्भिरुप॑ तिष्ठते॒ षड्वा-॥२६॥

%1.5.7.3
ऋ॒तव॑ ऋ॒तुष्वे॒व प्रति॑ तिष्ठति ष॒ड्भिरुत्त॑राभि॒रुप॑ तिष्ठते॒ द्वाद॑श॒ सं प॑द्यन्ते॒ द्वाद॑श॒ मासाः᳚ संवथ्स॒रः सं॑वथ्स॒र ए॒व प्रति॑ तिष्ठति॒ यथा॒ वै पुरु॒षो\-ऽश्वो॒ गौर्जीर्य॑त्ये॒वम॒ग्निराहि॑तो जीर्यति संवथ्स॒रस्य॑ प॒रस्ता॑दाग्निपावमा॒नीभि॒रुप॑ तिष्ठते पुनर्न॒वमे॒वैन॑म॒जरं॑ करो॒त्यथो॑ पु॒नात्ये॒वोप॑ तिष्ठते॒ योग॑ ए॒वास्यै॒ष उप॑ तिष्ठते॒॥२७॥

%1.5.7.4
दम॑ ए॒वास्यै॒ष उप॑ तिष्ठते या॒च्ञैवास्यै॒षोप॑ तिष्ठते॒ यथा॒ पापी॑या॒ञ्छ्रेय॑स आ॒हृत्य॑ नम॒स्यति॑ ता॒दृगे॒व तदा॑यु॒र्दा अ॑ग्ने॒\-ऽस्यायु॑र्मे दे॒हीत्या॑हा\-ऽ\-ऽयु॒र्दा ह्ये॑ष व॑र्चो॒दा अ॑ग्ने\-ऽसि॒ वर्चो॑ मे दे॒हीत्या॑ह वर्चो॒दा ह्ये॑ष त॑नू॒पा अ॑ग्ने\-ऽसि त॒नुवं॑ मे पा॒हीत्या॑ह॥२८॥

%1.5.7.5
तनू॒पा ह्ये॑षो\-ऽग्ने॒ यन्मे॑ त॒नुवा॑ ऊ॒नं तन्म॒ आ पृ॒णेत्या॑ह॒ यन्मे᳚ प्र॒जायै॑ पशू॒नामू॒नं तन्म॒ आ पू॑र॒येति॒ वावैतदा॑ह॒ चित्रा॑वसो स्व॒स्ति ते॑ पा॒रम॑शी॒येत्या॑ह॒ रात्रि॒र्वै चि॒त्राव॑सु॒रव्यु॑ष्ट्यै॒ वा ए॒तस्यै॑ पु॒रा ब्रा᳚ह्म॒णा अ॑भैषु॒र्व्यु॑ष्टिमे॒वाव॑ रुन्ध॒ इन्धा॑नास्त्वा श॒तꣳ॥२९॥

%1.5.7.6
हिमा॒ इत्या॑ह श॒तायुः॒ पुरु॑षः श॒तेन्द्रि॑य॒ आयु॑ष्ये॒वेन्द्रि॒ये प्रति॑ तिष्ठत्ये॒षा वै सू॒र्मी कर्ण॑कावत्ये॒तया॑ ह स्म॒ वै दे॒वा असु॑राणाꣳ शतत॒र्॒\mbox{}हाꣴ स्तृꣳ॑हन्ति॒ यदे॒तया॑ स॒मिध॑मा॒दधा॑ति॒ वज्र॑मे॒वैतच्छ॑त॒घ्नीं यज॑मानो॒ भ्रातृ॑व्याय॒ प्रह॑रति॒ स्तृत्या॒ अछ॑म्बट्कार॒ꣳ॒ सं त्वम॑ग्ने॒ सूर्य॑स्य॒ वर्च॑सा गथा॒ इत्या॑है॒तत्त्वमसी॒दम॒हं भू॑यास॒मिति॒ वावैतदा॑ह॒ त्वम॑ग्ने॒ सूर्य॑वर्चा अ॒सीत्या॑हा॒\-ऽ\-ऽशिष॑मे॒वैतामा शा᳚स्ते॥३०॥

%1.5.8.0
{\anuvakamend[{मू॒र्धानं॒ वै तिष्ठ॑त आह श॒तम॒हꣳ षोड॑श च}]}%॥७॥

%1.5.8.1
सं प॑श्यामि प्र॒जा अ॒हमित्या॑ह॒ याव॑न्त ए॒व ग्रा॒म्याः प॒शव॒स्ताने॒वाव॑ रु॒न्धे\-ऽम्भः॒ स्थाम्भो॑ वो भक्षी॒येत्या॒हाम्भो॒ ह्ये॑ता महः॑ स्थ॒ महो॑ वो भक्षी॒येत्या॑ह॒ महो॒ ह्ये॑ताः सहः॑ स्थ॒ सहो॑ वो भक्षी॒येत्या॑ह॒ सहो॒ ह्ये॑ता ऊर्ज॒स्थोर्जं॑ वो भक्षी॒ये-॥३१॥

%1.5.8.2
त्या॒होर्जो॒ ह्ये॑ता रेव॑ती॒ रम॑ध्व॒मित्या॑ह प॒शवो॒ वै रे॒वतीः᳚ प॒शूने॒वात्मन् र॑मयत इ॒हैव स्ते॒तो मा\-ऽप॑ गा॒तेत्या॑ह ध्रु॒वा ए॒वैना॒ अन॑पगाः कुरुत इष्टक॒चिद्वा अ॒न्यो᳚\-ऽग्निः प॑शु॒चिद॒न्यः सꣳ॑हि॒तासि॑ विश्वरू॒पीरिति॑ व॒थ्सम॒भि मृ॑श॒त्युपै॒वैनं॑ धत्ते पशु॒चित॑मेनं कुरुते॒ प्र॥३२॥

%1.5.8.3
वा ए॒षो᳚\-ऽस्माल्लो॒काच्च्य॑वते॒ य आ॑हव॒नीय॑मुप॒तिष्ठ॑ते॒ गार्\mbox{}ह॑पत्य॒मुप॑ तिष्ठते॒\-ऽस्मिन्ने॒व लो॒के प्रति॑ तिष्ठ॒त्यथो॒ गार्\mbox{}ह॑पत्यायै॒व नि ह्नु॑ते गाय॒त्रीभि॒रुप॑ तिष्ठते॒ तेजो॒ वै गा॑य॒त्री तेज॑ ए॒वात्मन् ध॒त्ते\-ऽथो॒ यदे॒तं तृ॒चम॒न्वाह॒ सन्त॑त्यै॒ गार्\mbox{}ह॑पत्यं॒ वा अनु॑ द्वि॒पादो॑ वी॒राः प्रजा॑यन्ते॒ य ए॒वं वि॒द्वान् द्वि॒पदा॑भि॒र्गार्\mbox{}ह॑पत्यमुप॒तिष्ठ॑त॒ -॥३३॥

%1.5.8.4
आ\-ऽस्य॑ वी॒रो जा॑यत ऊ॒र्जा वः॑ पश्याम्यू॒र्जा मा॑ पश्य॒तेत्या॑हा॒\-ऽ\-ऽशिष॑मे॒वैतामा शा᳚स्ते॒ तथ्स॑वि॒तुर्वरे᳚ण्य॒मित्या॑ह॒ प्रसू᳚त्यै सो॒मान॒ꣴ॒ स्वर॑ण॒मित्या॑ह सोमपी॒थमे॒वाव॑ रुन्धे कृणु॒हि ब्र॑ह्मणस्पत॒ इत्या॑ह ब्रह्मवर्च॒समे॒वाव॑ रुन्धे क॒दा च॒न स्त॒रीर॒सीत्या॑ह॒ न स्त॒रीꣳ रात्रिं॑ वसति॒॥३४॥

%1.5.8.5
य ए॒वं वि॒द्वान॒ग्निमु॑प॒तिष्ठ॑ते॒ परि॑ त्वाऽग्ने॒ पुरं॑ व॒यमित्या॑ह परि॒धिमे॒वैतं परि॑ दधा॒त्यस्क॑न्दा॒याग्ने॑ गृहपत॒ इत्या॑ह यथाय॒जुरे॒वैतच्छ॒तꣳ हिमा॒ इत्या॑ह श॒तं त्वा॑ हेम॒न्तानि॑न्धिषी॒येति॒ वावैतदा॑ह पु॒त्रस्य॒ नाम॑ गृह्णात्यन्ना॒दमे॒वैनं॑ करोति॒ तामा॒शिष॒मा शा॑से॒ तन्त॑वे॒ ज्योति॑ष्मती॒मिति॑ ब्रूया॒द्यस्य॑ पु॒त्रो\-ऽजा॑तः॒ स्यात्ते॑ज॒स्व्ये॑वास्य॑ ब्रह्मवर्च॒सी पु॒त्रो जा॑यते॒ तामा॒शिष॒मा शा॑से॒\-ऽमुष्मै॒ ज्योति॑ष्मती॒मिति॑ ब्रूया॒द्यस्य॑ पु॒त्रो जा॒तः स्यात् तेज॑ ए॒वास्मि॑न् ब्रह्मवर्च॒सं द॑धाति॥३५॥

%1.5.9.0
{\anuvakamend[{ऊर्जं॑ वो भक्षी॒येति॒ प्र गार्\mbox{}ह॑पत्यमुप॒तिष्ठ॑ते वसति॒ ज्योति॑ष्मती॒मेका॒न्नत्रि॒ꣳ॒शच्च॑}]}%॥८॥

%1.5.9.1
अ॒ग्नि॒हो॒त्रं जु॑होति॒ यदे॒व किं च॒ यज॑मानस्य॒ स्वं तस्यै॒व तद्रेतः॑ सिञ्चति प्र॒जन॑ने प्र॒जन॑न॒ꣳ॒ हि वा अ॒ग्निरथौष॑धी॒रन्त॑गता दहति॒ तास्ततो॒ भूय॑सीः॒ प्रजा॑यन्ते॒ यथ्सा॒यं जु॒होति॒ रेत॑ ए॒व तथ्सि॑ञ्चति॒ प्रैव प्रा॑त॒स्तने॑न जनयति॒ तद्रेतः॑ सि॒क्तं न त्वष्ट्रा\-ऽवि॑कृतं॒ प्रजा॑यते याव॒च्छो वै रेत॑सः सि॒क्तस्य॒॥३६॥

%1.5.9.2
त्वष्टा॑ रू॒पाणि॑ विक॒रोति॑ ताव॒च्छो वै तत्प्रजा॑यत ए॒ष वै दैव्य॒स्त्वष्टा॒ यो यज॑ते ब॒ह्वीभि॒रुप॑ तिष्ठते॒ रेत॑स ए॒व सि॒क्तस्य॑ बहु॒शो रू॒पाणि॒ वि क॑रोति॒ स प्रैव जा॑यते॒ श्वःश्वो॒ भूया᳚न् भवति॒ य ए॒वं वि॒द्वान॒ग्निमु॑प॒तिष्ठ॒ते\-ऽह॑र्दे॒वाना॒मासी॒द्रात्रि॒रसु॑राणां॒ ते\-ऽसु॑रा॒ यद्दे॒वानां᳚ वि॒त्तं वेद्य॒मासी॒त्तेन॑ स॒ह॥३७॥

%1.5.9.3
रात्रिं॒ प्रावि॑श॒न् ते दे॒वा ही॒ना अ॑मन्यन्त॒ ते॑\-ऽपश्यन्नाग्ने॒यी रात्रि॑राग्ने॒याः प॒शव॑ इ॒ममे॒वाग्निꣴ स्त॑वाम॒ स नः॑ स्तु॒तः प॒शून् पुन॑र्दास्य॒तीति॒ ते᳚\-ऽग्निम॑स्तुव॒न्थ्स ए᳚भ्यः स्तु॒तो रात्रि॑या॒ अध्यह॑र॒भि प॒शून्निरा᳚र्ज॒त् ते दे॒वाः प॒शून् वि॒त्वा कामाꣳ॑ अकुर्वत॒ य ए॒वं वि॒द्वान॒ग्निमु॑प॒तिष्ठ॑ते पशु॒मान् भ॑वत्या-॥३८॥

%1.5.9.4
दि॒त्यो वा अ॒स्माल्लो॒काद॒मुं लो॒कमै॒थ्सो॑\-ऽमुं लो॒कं ग॒त्वा पुन॑रि॒मं लो॒कम॒भ्य॑ध्याय॒थ्स इ॒मं लो॒कमा॒गत्य॑ मृ॒त्योर॑बिभेन्मृ॒त्युसं॑युत इव॒ ह्य॑यं लो॒कः सो॑\-ऽमन्यते॒ममे॒वाग्निꣴ स्त॑वानि॒ स मा᳚ स्तु॒तः सु॑व॒र्गं लो॒कं ग॑मयिष्य॒तीति॒ सो᳚\-ऽग्निम॑स्तौ॒थ्स ए॑नꣴ स्तु॒तः सु॑व॒र्गं लो॒कम॑गमय॒द्य -॥३९॥

%1.5.9.5
ए॒वं वि॒द्वान॒ग्निमु॑प॒तिष्ठ॑ते सुव॒र्गमे॒व लो॒कमे॑ति॒ सर्व॒मायु॑रेत्य॒भि वा ए॒षो᳚\-ऽग्नी आ रो॑हति॒ य ए॑नावुप॒तिष्ठ॑ते॒ यथा॒ खलु॒ वै श्रेया॑न॒भ्यारू॑ढः का॒मय॑ते॒ तथा॑ करोति॒ नक्त॒मुप॑ तिष्ठते॒ न प्रा॒तः सꣳ हि नक्तं॑ व्र॒तानि॑ सृ॒ज्यन्ते॑ स॒ह श्रेयाꣴ॑श्च॒ पापी॑याꣴश्चासाते॒ ज्योति॒र्वा अ॒ग्निस्तमो॒ रात्रि॒र्यन्-॥४०॥

%1.5.9.6
नक्त॑मुप॒तिष्ठ॑ते॒ ज्योति॑षै॒व तम॑स्तरत्युप॒स्थेयो॒\-ऽग्नी(३)र्नोप॒\-स्थेया(३) इत्या॑हुर्मनु॒ष्या॑येन्न्वै यो\-ऽह॑रहरा॒हृत्या\-थै॑नं॒ याच॑ति॒ स इन्न्वै तमुपा᳚र्च्छ॒त्यथ॒ को दे॒वानह॑रहर्याचिष्य॒तीति॒ तस्मा॒न्नोप॒स्थेयो\-ऽथो॒ खल्वा॑हुरा॒शिषे॒ वै कं यज॑मानो यजत॒ इत्ये॒षा खलु॒ वा -॥४१॥

%1.5.9.7
आहि॑ताग्नेरा॒शीर्यद॒ग्निमु॑प॒तिष्ठ॑ते॒ तस्मा॑दुप॒स्थेयः॑ प्र॒जाप॑तिः प॒शून॑सृजत॒ ते सृ॒ष्टा अ॑होरा॒त्रे प्रावि॑श॒न् ताञ्छन्दो॑\-भि॒रन्व॑\-विन्द॒द्यच्छन्दो॑भिरुप॒तिष्ठ॑ते॒ स्वमे॒व तदन्वि॑च्छति॒ न तत्र॑ जा॒म्य॑स्तीत्या॑हु॒र्यो\-ऽह॑रहरुप॒तिष्ठ॑त॒ इति॒ यो वा अ॒ग्निं प्र॒त्यङ्ङु॑प॒तिष्ठ॑ते॒ प्रत्ये॑नमोषति॒ यः परा॒ङ्॒ विष्व॑ङ् प्र॒जया॑ प॒शुभि॑रेति॒ कवा॑तिर्यङ्ङि॒वोप॑ तिष्ठेत॒ नैनं॑ प्र॒त्योष॑ति॒ न विष्व॑ङ् प्र॒जया॑ प॒शुभि॑रेति॥४२॥

%1.5.10.0
{\anuvakamend[{सि॒क्तस्य॑ स॒ह भ॑वति॒ यो यत्खलु॒ वै प॒शुभि॒स्त्रयो॑दश च}]}%॥९॥

%1.5.10.1
मम॒ नाम॑ प्रथ॒मं जा॑तवेदः पि॒ता मा॒ता च॑ दधतु॒र्यदग्रे᳚। तत्त्वं बि॑भृहि॒ पुन॒रा मदैतो॒स्तवा॒हं नाम॑ बिभराण्यग्ने॥ मम॒ नाम॒ तव॑ च जातवेदो॒ वास॑सी इव वि॒वसा॑नौ॒ ये चरा॑वः। आयु॑षे॒ त्वं जी॒वसे॑ व॒यं य॑थाय॒थं वि परि॑ दधावहै॒ पुन॒स्ते॥ नमो॒\-ऽग्नये\-ऽप्र॑तिविद्धाय॒ नमो\-ऽना॑धृष्टाय॒ नमः॑ स॒म्राजे᳚। अषा॑ढो॥४३॥

%1.5.10.2
अ॒ग्निर्बृ॒हद्व॑या विश्व॒जिथ्सह॑न्त्यः॒ श्रेष्ठो॑ गन्ध॒र्वः॥ त्वत्पि॑तारो अग्ने दे॒वास्त्वामा॑हुतय॒स्त्वद्वि॑वाचनाः। सं मामायु॑षा॒ सं गौ॑प॒त्येन॒ सुहि॑ते मा धाः॥ अ॒यम॒ग्निः श्रेष्ठ॑तमो॒\-ऽयं भग॑वत्तमो॒\-ऽयꣳ स॑हस्र॒सात॑मः। अ॒स्मा अ॑स्तु सु॒वीर्यम्᳚॥ मनो॒ ज्योति॑र्जुषता॒माज्यं॒ विच्छि॑न्नं य॒ज्ञꣳ समि॒मं द॑धातु। या इ॒ष्टा उ॒षसो॑ नि॒म्रुच॑श्च॒ ताः सं द॑धामि ह॒विषा॑ घृ॒तेन॑॥ पय॑स्वती॒रोष॑धयः॒॥४४॥

%1.5.10.3
पय॑स्वद्वी॒रुधां॒ पयः॑। अ॒पां पय॑सो॒ यत्पय॒स्तेन॒ मामि॑न्द्र॒ सꣳ सृ॑ज॥ अग्ने᳚ व्रतपते व्र॒तं च॑रिष्यामि॒ तच्छ॑केयं॒ तन्मे॑ राध्यताम्॥ अ॒ग्निꣳ होता॑रमि॒ह तꣳ हु॑वे दे॒वान् य॒ज्ञिया॑नि॒ह यान् हवा॑महे॥ आ य॑न्तु दे॒वाः सु॑मन॒स्यमा॑ना वि॒यन्तु॑ दे॒वा ह॒विषो॑ मे अ॒स्य॥ कस्त्वा॑ युनक्ति॒ स त्वा॑ युनक्तु॒ यानि॑ घ॒र्मे क॒पाला᳚न्युपचि॒न्वन्ति॑॥४५॥

%1.5.10.4
वे॒धसः॑। पू॒ष्णस्तान्यपि॑ व्र॒त इ॑न्द्रवा॒यू विमु॑ञ्चताम्॥ अभि॑न्नो घ॒र्मो जी॒रदा॑नु॒र्यत॒ आत्त॒स्तद॑ग॒न् पुनः॑। इ॒ध्मो वेदिः॑ परि॒धय॑श्च॒ सर्वे॑ य॒ज्ञस्या\-ऽ\-ऽयु॒रनु॒ सं च॑रन्ति॥ त्रय॑स्त्रिꣳश॒त्तन्त॑वो॒ ये वि॑तत्नि॒रे य इ॒मं य॒ज्ञꣴ स्व॒धया॒ दद॑न्ते॒ तेषां᳚ छि॒न्नं प्रत्ये॒तद्द॑धामि॒ स्वाहा॑ घ॒र्मो दे॒वाꣳ अप्ये॑तु॥४६॥

%1.5.11.0
{\anuvakamend[{अषा॑ढ॒ ओष॑धय उपचि॒न्वन्ति॒ पञ्च॑चत्वारिꣳशच्च}]}%॥10॥

%1.5.11.1
वै॒श्वा॒न॒रो न॑ ऊ॒त्या\-ऽ\-ऽप्र या॑तु परा॒वतः॑। अ॒ग्निरु॒क्थेन॒ वाह॑सा॥ ऋ॒तावा॑नं वैश्वान॒रमृ॒तस्य॒ ज्योति॑ष॒स्पतिम्᳚। अज॑स्रं घ॒र्ममी॑महे॥ वै॒श्वा॒न॒रस्य॑ द॒ꣳ॒सना᳚भ्यो बृ॒हदरि॑णा॒देकः॑ स्वप॒स्य॑या क॒विः। उ॒भा पि॒तरा॑ म॒हय॑न्नजायता॒ग्निर्द्यावा॑पृथि॒वी भूरि॑रेतसा॥ पृ॒ष्टो दि॒वि पृ॒ष्टो अ॒ग्निः पृ॑थि॒व्यां पृ॒ष्टो विश्वा॒ ओष॑धी॒रा वि॑वेश। वै॒श्वा॒न॒रः सह॑सा पृ॒ष्टो अ॒ग्निः स नो॒ दिवा॒ स॥४७॥

%1.5.11.2
रि॒षः पा॑तु॒ नक्तम्᳚॥ जा॒तो यद॑ग्ने॒ भुव॑ना॒ व्यख्यः॑ प॒शुं न गो॒पा इर्यः॒ परि॑ज्मा। वैश्वा॑नर॒ ब्रह्म॑णे विन्द गा॒तुं यू॒यं पा॑त स्व॒स्तिभिः॒ सदा॑ नः॥ त्वम॑ग्ने शो॒चिषा॒ शोशु॑चान॒ आ रोद॑सी अपृणा॒ जाय॑मानः। त्वं दे॒वाꣳ अ॒भिश॑स्तेरमुञ्चो॒ वैश्वा॑नर जातवेदो महि॒त्वा॥ अ॒स्माक॑मग्ने म॒घव॑थ्सु धार॒याना॑मि क्ष॒त्रम॒जरꣳ॑ सु॒वीर्यम्᳚। व॒यं ज॑येम श॒तिनꣳ॑ सह॒स्रिणं॒ वैश्वा॑नर॒॥४८॥

%1.5.11.3
वाज॑मग्ने॒ तवो॒तिभिः॑॥ वै॒श्वा॒न॒रस्य॑ सुम॒तौ स्या॑म॒ राजा॒ हिकं॒ भुव॑नानामभि॒श्रीः। इ॒तो जा॒तो विश्व॑मि॒दं वि च॑ष्टे वैश्वान॒रो य॑तते॒ सूर्ये॑ण॥ अव॑ ते॒ हेडो॑ वरुण॒ नमो॑\-भि॒रव॑ य॒ज्ञेभि॑रीमहे ह॒विर्भिः॑। क्षय॑न्न॒स्मभ्य॑मसुर प्रचेतो॒ राज॒न्नेनाꣳ॑सि शिश्रथः कृ॒तानि॑॥ उदु॑त्त॒मं व॑रुण॒ पाश॑\-म॒स्मद\-वा॑ध॒मं वि म॑ध्य॒मꣴ श्र॑थाय। अथा॑ व॒यमा॑दित्य॥४९॥ 

%1.5.11.4
व्र॒ते तवाना॑गसो॒ अदि॑तये स्याम॥ द॒धि॒क्राव्ण्णो॑ अकारिषं जि॒ष्णोरश्व॑स्य वा॒जिनः॑॥ सु॒र॒भि नो॒ मुखा॑ कर॒त् प्र ण॒ आयूꣳ॑षि तारिषत्॥ आ द॑धि॒क्राः शव॑सा॒ पञ्च॑ कृ॒ष्टीः सूर्य॑ इव॒ ज्योति॑षा॒\-ऽपस्त॑तान। स॒ह॒स्र॒साः श॑त॒सा वा॒ज्यर्वा॑ पृ॒णक्तु॒ मध्वा॒ समि॒मा वचाꣳ॑सि॥ अ॒ग्निर्मू॒र्धा भुवः॑। मरु॑तो॒ यद्ध॑ वो दि॒वः सु॑म्ना॒यन्तो॒ हवा॑महे। आ तू न॒॥५०॥

%1.5.11.5
उप॑ गन्तन॥ या वः॒ शर्म॑ शशमा॒नाय॒ सन्ति॑ त्रि॒धातू॑नि दा॒शुषे॑ यच्छ॒ताधि॑। अ॒स्मभ्यं॒ तानि॑ मरुतो॒ वि य॑न्त र॒यिं नो॑ धत्त वृषणः सु॒वीरम्᳚॥ अदि॑तिर्न उरुष्य॒त्वदि॑तिः॒ शर्म॑ यच्छतु। अदि॑तिः पा॒त्वꣳह॑सः॥ म॒हीमू॒ षु मा॒तरꣳ॑ सुव्र॒ता\-ना॑\-मृ॒तस्य॒ पत्नी॒मव॑से हुवेम। तु॒वि॒क्ष॒त्रा\-म॒जर॑न्ती\-मुरू॒चीꣳ सु॒शर्मा॑ण॒मदि॑तिꣳ सु॒प्रणी॑तिम्॥ सु॒त्रामा॑णं पृथि॒वीं द्याम॑ने॒हसꣳ॑ सु॒शर्मा॑ण॒मदि॑तिꣳ सु॒प्रणी॑तिम्। दैवीं॒ नावꣴ॑ स्वरि॒त्रा\-मना॑\-गस॒मस्र॑वन्ती॒मा रु॑हेमा स्व॒स्तये᳚॥ इ॒माꣳ सु नाव॒मा\-ऽरु॑हꣳ श॒तारि॑त्राꣳ श॒तस्फ्या᳚म्। अच्छि॑द्रां पारयि॒ष्णुम्॥५१॥

{\anuvakamend[{दिवा॒ स स॑ह॒स्रिणं॒ वैश्वा॑नरा\-ऽ\-ऽदित्य॒ तू नो॑\-ऽने॒हसꣳ॑ सु॒शर्मा॑ण॒मेका॒न्नविꣳ॑श॒तिश्च॑}]}%॥11॥
%%% END PRASHNA

\sect{षष्ठमः प्रश्नः}\setcounter{anuvakam}{0}
\dnsub{तैत्तिरीयसंहितायां प्रथमकाण्डे षष्ठमः प्रश्नः}
%1.6.1.0
%1.6.1.1
सं त्वा॑ सिञ्चामि॒ यजु॑षा प्र॒जामायु॒र्धनं॑ च। बृह॒स्पति॑प्रसूतो॒ यज॑मान इ॒ह मा रि॑षत्॥ आज्य॑मसि स॒त्यम॑सि स॒त्यस्याध्य॑क्षमसि ह॒विर॑सि वैश्वान॒रं वै᳚श्वदे॒वमुत्पू॑तशुष्मꣳ स॒त्यौजाः॒ सहो॑\-ऽसि॒ सह॑मानमसि॒ सह॒स्वारा॑तीः॒ सह॑स्वारातीय॒तः सह॑स्व॒ पृत॑नाः॒ सह॑स्व पृतन्य॒तः। स॒हस्र॑वीर्यमसि॒ तन्मा॑ जि॒न्वा\-ऽ\-ऽज्य॒स्या\-ऽ\-ऽज्य॑मसि स॒त्यस्य॑ स॒त्यम॑सि स॒त्यायु॑-॥१॥

%1.6.1.2
रसि स॒त्यशु॑ष्ममसि स॒त्येन॑ त्वा॒\-ऽभि घा॑रयामि॒ तस्य॑ ते भक्षीय\\
पञ्चा॒नां त्वा॒ वाता॑नां य॒न्त्राय॑ ध॒र्त्राय॑ गृह्णामि\\
पञ्चा॒नां त्व॑र्तू॒नां य॒न्त्राय॑ ध॒र्त्राय॑ गृह्णामि\\
पञ्चा॒नां त्वा॑ दि॒शां य॒न्त्राय॑ ध॒र्त्राय॑ गृह्णामि\\
पञ्चा॒नां त्वा॑ पञ्चज॒नानां᳚ य॒न्त्राय॑ ध॒र्त्राय॑ गृह्णामि\\
च॒रोस्त्वा॒ पञ्च॑बिलस्य य॒न्त्राय॑ ध॒र्त्राय॑ गृह्णामि॒\\
ब्रह्म॑णस्त्वा॒ तेज॑से य॒न्त्राय॑ ध॒र्त्राय॑ गृह्णामि\\
क्ष॒त्रस्य॒ त्वौज॑से य॒न्त्राय॑॥२॥

%1.6.1.3
ध॒र्त्राय॑ गृह्णामि\\
वि॒शे त्वा॑ य॒न्त्राय॑ ध॒र्त्राय॑ गृह्णामि\\
सु॒वीर्या॑य त्वा गृह्णामि सुप्रजा॒स्त्वाय॑ त्वा गृह्णामि रा॒यस्पोषा॑य त्वा गृह्णामि ब्रह्मवर्च॒साय॑ त्वा गृह्णामि॒ भूर॒स्माकꣳ॑ ह॒विर्दे॒वाना॑मा॒शिषो॒ यज॑मानस्य दे॒वानां᳚ त्वा दे॒वता᳚भ्यो गृह्णामि॒ कामा॑य त्वा गृह्णामि॥३॥

%1.6.2.0
{\anuvakamend[{स॒त्यायु॒रोज॑से य॒न्त्राय॒ त्रय॑स्त्रिꣳशच्च}]}%॥१॥

%1.6.2.1
ध्रु॒वो॑\-ऽसि ध्रु॒वो॑\-ऽहꣳ स॑जा॒तेषु॑ भूयासं॒\\
धीर॒श्चेत्ता॑ वसु॒विदु॒ग्रो᳚\-ऽस्यु॒ग्रो॑\-ऽहꣳ स॑जा॒तेषु॑ भूयास-\\
मु॒ग्रश्चेत्ता॑ वसु॒विद॑भि॒भूर॑स्यभि॒भूर॒हꣳ स॑जा॒तेषु॑ भूयास-\\
मभि॒भूश्चेत्ता॑ वसु॒विद्यु॒नज्मि॑ त्वा॒ ब्रह्म॑णा॒ दैव्ये॑न ह॒व्याया॒स्मै वोढ॒वे जा॑तवेदः। इन्धा॑नास्त्वा सुप्र॒जसः॑ सु॒वीरा॒ ज्योग्जी॑वेम बलि॒हृतो॑ व॒यं ते᳚॥ यन्मे॑ अग्ने अ॒स्य य॒ज्ञस्य॒ रिष्या॒-॥४॥

%1.6.2.2
द्यद्वा॒ स्कन्दा॒दाज्य॑स्यो॒त वि॑ष्णो। तेन॑ हन्मि स॒पत्नं॑ दुर्मरा॒युमैनं॑ दधामि॒ निर्\mbox{}ऋ॑त्या उ॒पस्थे᳚। भूर्भुवः॒ सुव॒रुच्छु॑ष्मो अग्ने॒ यज॑मानायैधि॒ निशु॑ष्मो अभि॒दास॑ते। अग्ने॒ देवे᳚द्ध॒ मन्वि॑द्ध॒ मन्द्र॑जि॒ह्वाम॑र्त्यस्य ते होतर्मू॒र्धन्ना जि॑घर्मि रा॒यस्पोषा॑य सुप्रजा॒स्त्वाय॑ सु॒वीर्या॑य॒ मनो॑\-ऽसि प्राजाप॒त्यं मन॑सा मा भू॒तेना\-ऽ\-ऽवि॑श॒ वाग॑स्यै॒न्द्री स॑पत्न॒क्षय॑णी॥५॥

%1.6.2.3
वा॒चा मे᳚न्द्रि॒येणा\-ऽ\-ऽवि॑श\\
वस॒न्तमृ॑तू॒नां प्री॑णामि॒ स मा᳚ प्री॒तः प्री॑णातु\\
ग्री॒ष्ममृ॑तू॒नां प्री॑णामि॒ स मा᳚ प्री॒तः प्री॑णातु\\
व॒र्॒\mbox{}षा ऋ॑तू॒नां प्री॑णामि॒ ता मा᳚ प्री॒ताः प्री॑णन्तु\\
श॒रद॑मृतू॒नां प्री॑णामि॒ सा मा᳚ प्री॒ता प्री॑णातु\\
हेमन्तशिशि॒रावृ॑तू॒नां प्री॑णामि॒ तौ मा᳚ प्री॒तौ प्री॑णीताम॒ग्नीषोम॑योर॒हं\\
दे॑वय॒ज्यया॒ चक्षु॑ष्मान् भूयासम॒-\\
ग्नेर॒हं दे॑वय॒ज्यया᳚न्ना॒दो भू॑यासं॒॥६॥

%1.6.2.4
दब्धि॑र॒स्यद॑ब्धो भूयास\-
म॒मुं द॑भेयम॒ग्नीषोम॑योर॒हं दे॑वय॒ज्यया॑ वृत्र॒हा भू॑यास-\\
मिन्द्राग्नि॒योर॒हं दे॑वय॒ज्यये᳚न्द्रिया॒\-व्य॑न्ना॒दो भू॑यास॒-\\
मिन्द्र॑स्या॒हं दे॑वय॒ज्यये᳚न्द्रिया॒वी भू॑यासं\\
महे॒न्द्रस्या॒हं दे॑वय॒ज्यया॑ जे॒मानं॑ महि॒मानं॑ गमेयम॒ग्नेः स्वि॑ष्ट॒कृतो॒\-ऽहं दे॑वय॒ज्यया\-ऽ\-ऽयु॑ष्मान् य॒ज्ञेन॑ प्रति॒ष्ठां ग॑मेयम्॥७॥

%1.6.3.0
{\anuvakamend[{रिष्या᳚थ्सपत्न॒क्षय॑ण्यन्ना॒दो भू॑यास॒ꣳ॒ षट्त्रिꣳ॑शच्च}]}%॥२॥

%1.6.3.1
अ॒ग्निर्मा॒ दुरि॑ष्टात् पातु सवि॒ता\-ऽघशꣳ॑सा॒द्यो मे\-ऽन्ति॑ दू॒रे॑\-ऽराती॒यति॒ तमे॒तेन॑ जेष॒ꣳ॒ सुरू॑पवर्\mbox{}षवर्ण॒ एही॒मान् भ॒द्रान् दुर्याꣳ॑ अ॒भ्येहि॒ मामनु॑व्रता॒ न्यु॑ शी॒र्\mbox{}षाणि॑ मृढ्व॒मिड॒ एह्यदि॑त॒ एहि॒ सर॑स्व॒त्येहि॒ रन्ति॑रसि॒ रम॑तिरसि सू॒नर्य॑सि॒ जुष्टे॒ जुष्टिं॑ ते\-ऽशी॒योप॑हूत उपह॒वं॥८॥

%1.6.3.2
ते॑\-ऽशीय॒ सा मे॑ स॒त्याशीर॒स्य य॒ज्ञस्य॑ भूया॒दरे॑डता॒ मन॑सा॒ तच्छ॑केयं य॒ज्ञो दिवꣳ॑ रोहतु य॒ज्ञो दिवं॑ गच्छतु॒ यो दे॑व॒यानः॒ पन्था॒स्तेन॑ य॒ज्ञो दे॒वाꣳ अप्ये᳚त्व॒स्मास्विन्द्र॑ इन्द्रि॒यं द॑धात्व॒स्मान्राय॑ उ॒त य॒ज्ञाः स॑चन्ताम॒स्मासु॑ सन्त्वा॒शिषः॒ सा नः॑ प्रि॒या सु॒प्रतू᳚र्तिर्म॒घोनी॒ जुष्टि॑रसि जु॒षस्व॑ नो॒ जुष्टा॑ नो-॥९॥

%1.6.3.3
ऽसि॒ जुष्टिं॑ ते गमेयं॒ मनो॒ ज्योति॑र्जुषता॒माज्यं॒ विच्छि॑न्नं य॒ज्ञꣳ समि॒मं द॑धातु। बृह॒स्पति॑स्तनुतामि॒मं नो॒ विश्वे॑ दे॒वा इ॒ह मा॑दयन्ताम्॥ ब्रध्न॒ पिन्व॑स्व॒ दद॑तो मे॒ मा क्षा॑यि कुर्व॒तो मे॒ मोप॑दसत् प्र॒जाप॑तेर्भा॒गो᳚\-ऽस्यूर्ज॑स्वा॒न् पय॑स्वान् प्राणापा॒नौ मे॑ पाहि समानव्या॒नौ मे॑ पाह्युदानव्या॒नौ मे॑ पा॒ह्यक्षि॑तो॒\-ऽस्यक्षि॑त्यै त्वा॒ मा मे᳚ क्षेष्ठा अ॒मुत्रा॒मुष्मि॑न् लो॒के॥१०॥

%1.6.4.0
{\anuvakamend[{उ॒प॒ह॒वं जुष्टा॑ नस्त्वा॒ षट् च॑}]}%॥३॥

%1.6.4.1
ब॒र्॒\mbox{}हिषो॒\-ऽहं दे॑वय॒ज्यया᳚ प्र॒जावा᳚न् भूयासं॒ नरा॒शꣳस॑स्या॒हं दे॑वय॒ज्यया॑ पशु॒मान् भू॑यासम॒ग्नेः स्वि॑ष्ट॒कृतो॒\-ऽहं दे॑वय॒ज्यया\-ऽ\-ऽयु॑ष्मान् य॒ज्ञेन॑ प्रति॒ष्ठां ग॑मेयम॒ग्नेर॒हमुज्जि॑ति॒मनूज्जे॑ष॒ꣳ॒ सोम॑\-स्या॒\-हमुज्जि॑ति॒\-मनूज्जे॑षम॒ग्नेर॒हमुज्जि॑ति॒\-मनूज्जे॑षम॒ग्नी\-षोम॑यो\-र॒ह\-मु\-ज्जि॑ति॒\-मनूज्जे॑ष\-मिन्द्राग्नि॒यो\-र॒हमुज्जि॑ति॒\-मनूज्जे॑ष॒\-मिन्द्र॑स्या॒हमु-॥११॥

%1.6.4.2
ज्जि॑ति॒मनूज्जे॑षं महे॒न्द्रस्या॒हमुज्जि॑ति॒मनूज्जे॑षम॒ग्नेः स्वि॑ष्ट॒कृतो॒\-ऽहमुज्जि॑ति॒मनूज्जे॑षं॒ वाज॑स्य मा प्रस॒वेनो᳚द्ग्रा॒भेणोद॑ग्रभीत्। अथा॑ स॒पत्ना॒ꣳ॒ इन्द्रो॑ मे निग्रा॒भेणाध॑राꣳ अकः॥ उ॒द्ग्रा॒भं च॑ निग्रा॒भं च॒ ब्रह्म॑ दे॒वा अ॑वीवृधन्न्। अथा॑ स॒पत्ना॑निन्द्रा॒ग्नी मे॑ विषू॒चीना॒न्व्य॑स्यताम्॥ एमा अ॑ग्मन्ना॒शिषो॒ दोह॑कामा॒ इन्द्र॑वन्तो॥१२॥

%1.6.4.3
वनामहे॒ धुक्षी॒महि॑ प्र॒जामिषम्᳚॥ रोहि॑तेन त्वा॒\-ऽग्निर्दे॒वतां᳚ गमयतु॒ हरि॑भ्यां॒ त्वेन्द्रो॑ दे॒वतां᳚ गमय॒त्वेत॑शेन त्वा॒ सूर्यो॑ दे॒वतां᳚ गमयतु॒ वि ते॑ मुञ्चामि रश॒ना वि र॒श्मीन् वि योक्त्रा॒ यानि॑ परि॒चर्त॑नानि ध॒त्ताद॒स्मासु॒ द्रवि॑णं॒ यच्च॑ भ॒द्रं प्र णो᳚ ब्रूताद्भाग॒धान् दे॒वता॑सु॥ विष्णोः᳚ शं॒योर॒हं दे॑वय॒ज्यया॑ य॒ज्ञेन॑ प्रति॒ष्ठां ग॑मेय॒ꣳ॒ सोम॑स्या॒हं दे॑वय॒ज्यया᳚॥१३॥

%1.6.4.4
सु॒रेता॒ रेतो॑ धिषीय॒ त्वष्टु॑र॒हं दे॑वय॒ज्यया॑ पशू॒नाꣳ रू॒पं पु॑षेयं दे॒वानां॒ पत्नी॑र॒ग्निर्गृ॒हप॑तिर्य॒ज्ञस्य॑ मिथु॒नं तयो॑र॒हं दे॑वय॒ज्यया॑ मिथु॒नेन॒ प्र भू॑यासं वे॒दो॑\-ऽसि॒ वित्ति॑रसि वि॒देय॒ कर्मा॑सि क॒रुण॑मसि क्रि॒यासꣳ॑ स॒निर॑सि सनि॒तासि॑ स॒नेयं॑ घृ॒तव॑न्तं कुला॒यिनꣳ॑ रा॒यस्पोषꣳ॑ सह॒स्रिणं॑ वे॒दो द॑दातु वा॒जिनम्᳚॥१४॥

%1.6.5.0
{\anuvakamend[{इन्द्र॑स्या॒हमिन्द्र॑वन्तः॒ सोम॑स्या॒हं दे॑वय॒ज्यया॒ चतु॑श्चत्वारिꣳशच्च}]}%॥४॥

%1.6.5.1
आ प्या॑यतां ध्रु॒वा घृ॒तेन॑ य॒ज्ञं य॑ज्ञं॒ प्रति॑ देव॒यद्भ्यः॑। सू॒र्याया॒ ऊधो\-ऽदि॑त्या उ॒पस्थ॑ उ॒रुधा॑रा पृथि॒वी य॒ज्ञे अ॒स्मिन्॥ प्र॒जाप॑तेर्वि॒भान्नाम॑ लो॒कस्तस्मिꣴ॑स्त्वा दधामि स॒ह यज॑मानेन॒ सद॑सि॒ सन्मे॑ भूयाः॒ सर्व॑मसि॒ सर्वं॑ मे भूयाः पू॒र्णम॑सि पू॒र्णं मे॑ भूया॒ अक्षि॑तमसि॒ मा मे᳚ क्षेष्ठाः॒ प्राच्यां᳚ दि॒शि दे॒वा ऋ॒त्विजो॑ मार्जयन्तां॒ दक्षि॑णायां॥१५॥

%1.6.5.2
दि॒शि मासाः᳚ पि॒तरो॑ मार्जयन्तां प्र॒तीच्यां᳚ दि॒शि गृ॒हाः प॒शवो॑ मार्जयन्ता॒मुदी᳚च्यां दि॒श्याप॒ ओष॑धयो॒ वन॒स्पत॑यो मार्जयन्तामू॒र्ध्वायां᳚ दि॒शि य॒ज्ञः सं॑वथ्स॒रो य॒ज्ञप॑तिर्मार्जयन्तां॒ विष्णोः॒ क्रमो᳚\-ऽस्यभिमाति॒हा गा॑य॒त्रेण॒ छन्द॑सा पृथि॒वीमनु॒ वि क्र॑मे॒ निर्भ॑क्तः॒ स यं द्वि॒ष्मो विष्णोः॒ क्रमो᳚\-ऽस्यभिशस्ति॒हा त्रैष्टु॑भेन॒ छन्द॑सा॒\-ऽन्तरि॑क्ष॒मनु॒ वि क्र॑मे॒ निर्भ॑क्तः॒ स यं द्वि॒ष्मो विष्णोः॒ क्रमो᳚\-ऽस्यरातीय॒तो ह॒न्ता जाग॑तेन॒ छन्द॑सा॒ दिव॒मनु॒ वि क्र॑मे॒ निर्भ॑क्तः॒ स यं द्वि॒ष्मो विष्णोः॒ क्रमो॑\-ऽसि शत्रूय॒तो ह॒न्ता\-ऽ\-ऽनु॑ष्टुभेन॒ छन्द॑सा॒ दिशो\-ऽनु॒ वि क्र॑मे॒ निर्भ॑क्तः॒ स यं द्वि॒ष्मः॥१६॥

%1.6.6.0
{\anuvakamend[{दक्षि॑णायां द्वि॒ष्मो विष्णो॒रेका॒न्नत्रि॒ꣳ॒शच्च॑}]}%॥५॥

%1.6.6.1
अग॑न्म॒ सुवः॒ सुव॑रगन्म सं॒दृश॑स्ते॒ मा छि॑थ्सि॒ यत्ते॒ तप॒स्तस्मै॑ ते॒ मा\-ऽ\-ऽवृ॑क्षि सु॒भूर॑सि॒ श्रेष्ठो॑ रश्मी॒नामा॑यु॒र्धा अ॒स्यायु॑र्मे धेहि वर्चो॒धा अ॑सि॒ वर्चो॒ मयि॑ धेही॒दम॒हम॒मुं भ्रातृ॑व्यमा॒भ्यो दि॒ग्भ्यो᳚\-ऽस्यै दि॒वो᳚\-ऽस्माद॒न्तरि॑क्षाद॒स्यै पृ॑थि॒व्या अ॒स्माद॒न्नाद्या॒न्निर्भ॑जामि॒ निर्भ॑क्तः॒ स यं द्वि॒ष्मः।॥१७॥

%1.6.6.2
सं ज्योति॑षा\-ऽभूवमै॒न्द्रीमा॒वृत॑म॒न्वाव॑र्ते॒ सम॒हं प्र॒जया॒ सं मया᳚ प्र॒जा सम॒हꣳ रा॒यस्पोषे॑ण॒ सं मया॑ रा॒यस्पोषः॒ समि॑द्धो अग्ने मे दीदिहि समे॒द्धा ते॑ अग्ने दीद्यासं॒ वसु॑मान् य॒ज्ञो वसी॑यान् भूयास॒मग्न॒ आयूꣳ॑षि पवस॒ आ सु॒वोर्ज॒मिषं॑ च नः। आ॒रे बा॑धस्व दु॒च्छुना᳚म्॥ अग्ने॒ पव॑स्व॒ स्वपा॑ अ॒स्मे वर्चः॑ सु॒वीर्यम्᳚।॥१८॥

%1.6.6.3
दध॒त् पोषꣳ॑ र॒यिं मयि॑। अग्ने॑ गृहपते सुगृहप॒तिर॒हं त्वया॑ गृ॒हप॑तिना भूयासꣳ सुगृहप॒तिर्मया॒ त्वं गृ॒हप॑तिना भूयाः श॒तꣳ हिमा॒स्तामा॒शिष॒मा शा॑से॒ तन्त॑वे॒ ज्योति॑ष्मतीं॒ तामा॒शिष॒माशा॑से॒\-ऽमुष्मै॒ ज्योति॑ष्मतीं॒ कस्त्वा॑ युनक्ति॒ स त्वा॒ विमु॑ञ्च॒त्वग्ने᳚ व्रतपते व्र॒तम॑चारिषं॒ तद॑शकं॒ तन्मे॑\-ऽराधि य॒ज्ञो ब॑भूव॒ स आ॥१९॥

%1.6.6.4
ब॒भू॒व॒ स प्र ज॑ज्ञे॒ स वा॑वृधे। स दे॒वाना॒मधि॑पतिर्बभूव॒ सो अ॒स्माꣳ अधि॑पतीन् करोतु व॒यꣴ स्या॑म॒ पत॑यो रयी॒णाम्॥ गोमाꣳ॑ अ॒ग्ने\-ऽवि॑माꣳ अ॒श्वी य॒ज्ञो नृ॒वथ्स॑खा॒ सद॒मिद॑प्रमृ॒ष्यः। इडा॑वाꣳ ए॒षो अ॑सुर प्र॒जावा᳚न् दी॒र्घो र॒यिः पृ॑थुबु॒ध्नः स॒भावान्॑॥२०॥

%1.6.7.0
{\anuvakamend[{द्वि॒ष्मः सु॒वीर्य॒ꣳ॒ स आ पञ्च॑त्रिꣳशच्च}]}%॥६॥

%1.6.7.1
यथा॒ वै स॑मृतसो॒मा ए॒वं वा ए॒ते स॑मृतय॒ज्ञा यद्द॑र्\mbox{}शपूर्णमा॒सौ कस्य॒ वाह॑ दे॒वा य॒ज्ञमा॒ गच्छ॑न्ति॒ कस्य॑ वा॒ न ब॑हू॒नां यज॑मानानां॒ यो वै दे॒वताः॒ पूर्वः॑ परिगृ॒ह्णाति॒ स ए॑नाः॒ श्वो भू॒ते य॑जत ए॒तद्वै दे॒वाना॑मा॒यत॑नं॒ यदा॑हव॒नीयो᳚\-ऽन्त॒राग्नी प॑शू॒नां गार्\mbox{}ह॑पत्यो मनु॒ष्या॑णामन्वाहार्य॒पच॑नः पितृ॒णाम॒ग्निं गृ॑ह्णाति॒ स्व ए॒वायत॑ने दे॒वताः॒ परि॑॥२१॥

%1.6.7.2
गृ॒ह्णा॒ति॒ ताः श्वो भू॒ते य॑जते व्र॒तेन॒ वै मेध्यो॒\-ऽग्निर्व्र॒तप॑तिर्ब्राह्म॒णो व्र॑त॒भृद् व्र॒तमु॑पै॒ष्यन् ब्रू॑या॒दग्ने᳚ व्रतपते व्र॒तं च॑रिष्या॒मीत्य॒ग्निर्वै दे॒वानां᳚ व्र॒तप॑ति॒स्तस्मा॑ ए॒व प्र॑ति॒प्रोच्य॑ व्र॒तमाल॑भते ब॒र्॒\mbox{}हिषा॑ पू॒र्णमा॑से व्र॒तमुपै॑ति व॒थ्सैर॑मावा॒स्या॑यामे॒तद्ध्ये॑तयो॑रा॒यत॑नमुप॒स्तीर्यः॒ पूर्व॑श्चा॒ग्निरप॑र॒श्चेत्या॑हुर्मनु॒ष्याः᳚॥२२॥

%1.6.7.3
इन्न्वा उप॑स्तीर्णमि॒च्छन्ति॒ किमु॑ दे॒वा येषां॒ नवा॑वसान॒मुपा᳚स्मि॒ञ्छ्वो य॒क्ष्यमा॑णे दे॒वता॑ वसन्ति॒ य ए॒वं वि॒द्वान॒ग्निमु॑पस्तृ॒णाति॒ यज॑मानेन ग्रा॒म्याश्च॑ प॒शवो॑\-ऽव॒रुध्या॑ आर॒ण्याश्चेत्या॑हु॒र्यद्ग्रा॒म्यानु॑प॒वस॑ति॒ तेन॑ ग्रा॒म्यानव॑ रुन्धे॒ यदा॑र॒ण्यस्या॒श्ञाति॒ तेना॑र॒ण्यान् यदना᳚श्वानुप॒वसे᳚त् पितृदेव॒त्यः॑ स्यादार॒ण्यस्या᳚श्ञातीन्द्रि॒यं॥२३॥

%1.6.7.4
वा आ॑र॒ण्यमि॑न्द्रि॒यमे॒वा\-ऽ\-ऽत्मन् ध॑त्ते॒ यदना᳚श्वानुप॒वसे॒त् क्षोधु॑कः स्या॒द्यद॑श्ञी॒याद्रु॒द्रो᳚\-ऽस्य प॒शून॒भिम॑न्येता॒पो᳚\-ऽश्ञाति॒ तन्नेवा॑शि॒तं नेवान॑शितं॒ न क्षोधु॑को॒ भव॑ति॒ नास्य॑ रु॒द्रः प॒शून॒भि म॑न्यते॒ वज्रो॒ वै य॒ज्ञः क्षुत्खलु॒ वै म॑नु॒ष्य॑स्य॒ भ्रातृ॑व्यो॒ यदना᳚श्वानुप॒वस॑ति॒ वज्रे॑णै॒व सा॒क्षात्क्षुधं॒ भ्रातृ॑व्यꣳ हन्ति॥२४॥

%1.6.8.0
{\anuvakamend[{परि॑ मनु॒ष्या॑ इन्द्रि॒यꣳ सा॒क्षात् त्रीणि॑ च}]}%॥७॥

%1.6.8.1
यो वै श्र॒द्धामना॑रभ्य य॒ज्ञेन॒ यज॑ते॒ नास्ये॒ष्टाय॒ श्रद्द॑धते॒\-ऽपः प्र ण॑यति श्र॒द्धा वा आपः॑ श्र॒द्धामे॒वा\-ऽ\-ऽरभ्य॑ य॒ज्ञेन॑ यजत उ॒भये᳚\-ऽस्य देवमनु॒ष्या इ॒ष्टाय॒ श्रद्द॑धते॒ तदा॑हु॒रति॒ वा ए॒ता वर्त्र॑न्नेद॒न्त्यति॒ वाचं॒ मनो॒ वावैता नाति॑ नेद॒न्तीति॒ मन॑सा॒ प्र ण॑यती॒यं वै मनः॑॥२५॥

%1.6.8.2
अ॒नयै॒वैनाः॒ प्र ण॑य॒त्यस्क॑न्नहविर्भवति॒ य ए॒वं वेद॑ यज्ञायु॒धानि॒ सम्भ॑रति य॒ज्ञो वै य॑ज्ञायु॒धानि॑ य॒ज्ञमे॒व तथ्सम्भ॑रति॒ यदेक॑मेकꣳ सं॒भरे᳚त् पितृदेव॒त्या॑नि स्यु॒र्यथ्स॒ह सर्वा॑णि मानु॒षाणि॒ द्वेद्वे॒ सम्भ॑रति याज्यानुवा॒क्य॑योरे॒व रू॒पं क॑रो॒त्यथो॑ मिथु॒नमे॒व यो वै दश॑ यज्ञायु॒धानि॒ वेद॑ मुख॒तो᳚\-ऽस्य य॒ज्ञः क॑ल्पते॒ स्फ्यः॥२६॥

%1.6.8.3
च॒ क॒पाला॑नि चाग्निहोत्र॒हव॑णी च॒ शूर्पं॑ च कृष्णाजि॒नं च॒ शम्या॑ चो॒लूख॑लं च॒ मुस॑लं च दृ॒षच्चोप॑ला चै॒तानि॒ वै दश॑ यज्ञायु॒धानि॒ य ए॒वं वेद॑ मुख॒तो᳚\-ऽस्य य॒ज्ञः क॑ल्पते॒ यो वै दे॒वेभ्यः॑ प्रति॒प्रोच्य॑ य॒ज्ञेन॒ यज॑ते जु॒षन्ते᳚\-ऽस्य दे॒वा ह॒व्यꣳ ह॒विर्नि॑रु॒प्यमा॑णम॒भि म॑न्त्रयेता॒ग्निꣳ होता॑रमि॒ह तꣳ हु॑व॒ इति॑॥२७॥

%1.6.8.4
दे॒वेभ्य॑ ए॒व प्र॑ति॒प्रोच्य॑ य॒ज्ञेन॑ यजते जु॒षन्ते᳚\-ऽस्य दे॒वा ह॒व्यमे॒ष वै य॒ज्ञस्य॒ ग्रहो॑ गृही॒त्वैव य॒ज्ञेन॑ यजते॒ तदु॑दि॒त्वा वाचं॑ यच्छति य॒ज्ञस्य॒ धृत्या॒ अथो॒ मन॑सा॒ वै प्र॒जाप॑तिर्य॒ज्ञम॑तनुत॒ मन॑सै॒व तद्य॒ज्ञं त॑नुते॒ रक्ष॑सा॒मन॑न्ववचाराय॒ यो वै य॒ज्ञं योग॒ आग॑ते यु॒नक्ति॑ यु॒ङ्क्ते यु॑ञ्जा॒नेषु॒ कस्त्वा॑ युनक्ति॒ स त्वा॑ युन॒क्त्वित्या॑ह प्र॒जाप॑ति॒र्वै कः प्र॒जाप॑तिनै॒वैनं॑ युनक्ति यु॒ङ्क्ते यु॑ञ्जा॒नेषु॑॥२८॥

%1.6.9.0
{\anuvakamend[{वै मनः॒ स्फ्य इति॑ युन॒क्त्वेका॑दश च}]}%॥८॥

%1.6.9.1
प्र॒जाप॑तिर्य॒ज्ञान॑सृजताग्निहो॒त्रं चा᳚ग्निष्टो॒मं च॑ पौर्णमा॒सीं चो॒क्थ्यं॑ चामावा॒स्यां᳚ चातिरा॒त्रं च॒ तानुद॑मिमीत॒ याव॑दग्निहो॒त्रमासी॒त् तावा॑नग्निष्टो॒मो याव॑ती पौर्णमा॒सी तावा॑नु॒क्थ्यो॑ याव॑त्यमावा॒स्या॑ तावा॑नतिरा॒त्रो य ए॒वं वि॒द्वान॑ग्निहो॒त्रं जु॒होति॒ याव॑दग्निष्टो॒मेनो॑पा॒प्नोति॒ ताव॒दुपा᳚\-ऽ\-ऽप्नोति॒ य ए॒वं वि॒द्वान् पौ᳚र्णमा॒सीं यज॑ते॒ याव॑दु॒क्थ्ये॑नोपा॒प्नोति॑॥२९॥

%1.6.9.2
ताव॒दुपा᳚\-ऽ\-ऽप्नोति॒ य ए॒वं वि॒द्वान॑मावा॒स्यां᳚ यज॑ते॒ याव॑दतिरा॒त्रेणो॑पा॒प्नोति॒ ताव॒दुपा᳚\-ऽ\-ऽप्नोति परमे॒ष्ठिनो॒ वा ए॒ष य॒ज्ञो\-ऽग्र॑ आसी॒त् तेन॒ स प॑र॒मां काष्ठा॑मगच्छ॒त् तेन॑ प्र॒जाप॑तिं नि॒रवा॑सायय॒त् तेन॑ प्र॒जाप॑तिः पर॒मां काष्ठा॑मगच्छ॒त् तेनेन्द्रं॑ नि॒रवा॑सायय॒त् तेनेन्द्रः॑ पर॒मां काष्ठा॑मगच्छ॒त् तेना॒ग्नीषोमौ॑ नि॒रवा॑सायय॒त् तेना॒ग्नीषोमौ॑ पर॒मां काष्ठा॑मगच्छतां॒ यः॥३०॥

%1.6.9.3
ए॒वं वि॒द्वान् द॑र्\mbox{}शपूर्णमा॒सौ यज॑ते पर॒मामे॒व काष्ठां᳚ गच्छति॒ यो वै प्रजा॑तेन य॒ज्ञेन॒ यज॑ते॒ प्र प्र॒जया॑ प॒शुभि॑र्मिथु॒नैर्जा॑यते॒ द्वाद॑श॒ मासाः᳚ संवथ्स॒रो द्वाद॑श द्व॒न्द्वानि॑ दर्\mbox{}शपूर्णमा॒सयो॒स्तानि॑ सं॒पाद्या॒नीत्या॑हुर्व॒थ्सं चो॑पावसृ॒जत्यु॒खां चाधि॑ श्रय॒त्यव॑ च॒ हन्ति॑ दृ॒षदौ॑ च स॒माह॒न्त्यधि॑ च॒ वप॑ते क॒पाला॑नि॒ चोप॑ दधाति पुरो॒डाशं॑ च॥३१॥

%1.6.9.4
अ॒धि॒श्रय॒त्याज्यं॑ च स्तम्बय॒जुश्च॒ हर॑त्य॒भि च॑ गृह्णाति॒ वेदिं॑ च परिगृ॒ह्णाति॒ पत्नीं᳚ च॒ सं न॑ह्यति॒ प्रोक्ष॑णीश्चा\-ऽ\-ऽसा॒दय॒त्याज्यं॑ चै॒तानि॒ वै द्वाद॑श द्व॒न्द्वानि॑ दर्\mbox{}शपूर्णमा॒सयो॒स्तानि॒ य ए॒वꣳ सं॒पाद्य॒ यज॑ते॒ प्रजा॑तेनै॒व य॒ज्ञेन॑ यजते॒ प्र प्र॒जया॑ प॒शुभि॑र्मिथु॒नैर्जा॑यते॥३२॥

%1.6.10.0
{\anuvakamend[{उ॒क्थ्ये॑नोपा॒प्नोत्य॑गच्छतां॒ यः पु॑रो॒डाशं॑ च चत्वारि॒ꣳ॒शच्च॑}]}%॥९॥

%1.6.10.1
ध्रु॒वो॑\-ऽसि ध्रु॒वो॑\-ऽहꣳ स॑जा॒तेषु॑ भूयास॒मित्या॑ह ध्रु॒वाने॒वैना᳚न् कुरुत उ॒ग्रो᳚\-ऽस्यु॒ग्रो॑\-ऽहꣳ स॑जा॒तेषु॑ भूयास॒मित्या॒हाप्र॑तिवादिन ए॒वैना᳚न्कुरुते\-ऽभि॒भूर॑स्यभि॒भूर॒हꣳ स॑जा॒तेषु॑ भूयास॒मित्या॑ह॒ य ए॒वैनं॑ प्रत्यु॒त्पिपी॑ते॒ तमुपा᳚स्यते यु॒नज्मि॑ त्वा॒ ब्रह्म॑णा॒ दैव्ये॒नेत्या॑है॒ष वा अ॒ग्नेर्योग॒स्तेन॑॥३३॥

%1.6.10.2
ए॒वैनं॑ युनक्ति य॒ज्ञस्य॒ वै समृ॑द्धेन दे॒वाः सु॑व॒र्गं लो॒कमा॑यन् य॒ज्ञस्य॒ व्यृ॑द्धे॒नासु॑रा॒न् परा॑भावय॒न्॒ यन्मे॑ अग्ने अ॒स्य य॒ज्ञस्य॒ रिष्या॒दित्या॑ह य॒ज्ञस्यै॒व तथ्समृ॑द्धेन॒ यज॑मानः सुव॒र्गं लो॒कमे॑ति य॒ज्ञस्य॒ व्यृ॑द्धेन॒ भ्रातृ॑व्या॒न् परा॑ भावयत्यग्निहो॒त्रमे॒ताभि॒र्व्याहृ॑तीभि॒रुप॑ सादयेद्यज्ञमु॒खं वा अ॑ग्निहो॒त्रं ब्रह्मै॒ता व्याहृ॑तयो यज्ञमु॒ख ए॒व ब्रह्म॑॥३४॥

%1.6.10.3
कु॒रु॒ते॒ सं॒व॒थ्स॒रे प॒र्याग॑त ए॒ताभि॑रे॒वोप॑सादये॒द् ब्रह्म॑णै॒वोभ॒यतः॑ संवथ्स॒रं परि॑गृह्णाति दर्\mbox{}शपूर्णमा॒सौ चा॑तुर्मा॒स्यान्या॒लभ॑मान ए॒ताभि॒र्व्याहृ॑तीभिर्\mbox{} ह॒वीꣳष्यासा॑दयेद्यज्ञमु॒खं वै द॑र्\mbox{}शपूर्णमा॒सौ चा॑तुर्मा॒स्यानि॒ ब्रह्मै॒ता व्याहृ॑तयो यज्ञमु॒ख ए॒व ब्रह्म॑ कुरुते संवथ्स॒रे प॒र्याग॑त ए॒ताभि॑रे॒वासा॑दये॒द् ब्रह्म॑णै॒वोभ॒यतः॑ संवथ्स॒रं परि॑गृह्णाति॒ यद्वै य॒ज्ञस्य॒ साम्ना᳚ क्रि॒यते॑ रा॒ष्ट्रं॥३५॥

%1.6.10.4
य॒ज्ञस्या॒\-ऽ\-ऽशीर्ग॑च्छति॒ यदृ॒चा विशं॑ य॒ज्ञस्या॒\-ऽ\-ऽशीर्ग॑च्छ॒त्यथ॑ ब्राह्म॒णो॑\-ऽना॒शीर्के॑ण य॒ज्ञेन॑ यजते सामिधे॒नीर॑नुव॒क्ष्यन्ने॒ता व्याहृ॑तीः पु॒रस्ता᳚द्दध्या॒द् ब्रह्मै॒व प्र॑ति॒पदं॑ कुरुते॒ तथा᳚ ब्राह्म॒णः साशी᳚र्केण य॒ज्ञेन॑ यजते॒ यं का॒मये॑त॒ यज॑मानं॒ भ्रातृ॑व्यमस्य य॒ज्ञस्या॒\-ऽ\-ऽशीर्ग॑च्छे॒दिति॒ तस्यै॒ता व्याहृ॑तीः पुरो\-ऽनुवा॒क्या॑यां दध्याद् भ्रातृव्यदेव॒त्या॑ वै पु॑रो\-ऽनुवा॒क्या᳚ भ्रातृ॑व्यमे॒वास्य॑ य॒ज्ञस्य॑॥३६॥

%1.6.10.5
आ॒शीर्ग॑च्छति॒ यान् का॒मये॑त॒ यज॑मानान्थ्स॒माव॑त्येनान् य॒ज्ञस्या॒\-ऽ\-ऽशीर्ग॑च्छे॒दिति॒ तेषा॑मे॒ता व्याहृ॑तीः पुरो\-ऽनुवा॒क्या॑या अर्ध॒र्च एकां᳚ दध्याद्या॒ज्या॑यै पु॒रस्ता॒देकां᳚ या॒ज्या॑या अर्ध॒र्च एकां॒ तथै॑नान्थ्स॒माव॑ती य॒ज्ञस्या॒\-ऽ\-ऽशीर्ग॑च्छति॒ यथा॒ वै प॒र्जन्यः॒ सुवृ॑ष्टं॒ वर्\mbox{}ष॑त्ये॒वं य॒ज्ञो यज॑मानाय वर्\mbox{}षति॒ स्थल॑योद॒कं प॑रिगृ॒ह्णन्त्या॒शिषा॑ य॒ज्ञं यज॑मानः॒ परि॑गृह्णाति॒ मनो॑\-ऽसि प्राजाप॒त्यं॥३७॥

%1.6.10.6
मन॑सा मा भू॒तेना\-ऽ\-ऽवि॒शेत्या॑ह॒ मनो॒ वै प्रा॑जाप॒त्यं प्रा॑जाप॒त्यो य॒ज्ञो मन॑ ए॒व य॒ज्ञमा॒त्मन् ध॑त्ते॒ वाग॑स्यै॒न्द्री स॑पत्न॒क्षय॑णी वा॒चा मे᳚न्द्रि॒येणा\-ऽ\-ऽवि॒शेत्या॑है॒न्द्री वै वाग्वाच॑मे॒वैन्द्रीमा॒त्मन् ध॑त्ते॥३८॥

%1.6.11.0
{\anuvakamend[{तेनै॒व ब्रह्म॑ रा॒ष्ट्रमे॒वास्य॑ य॒ज्ञस्य॑ प्राजाप॒त्यꣳ षट्त्रिꣳ॑शच्च}]}%॥10॥

%1.6.11.1
यो वै स॑प्तद॒शं प्र॒जाप॑तिं य॒ज्ञम॒न्वाय॑त्तं॒ वेद॒ प्रति॑ य॒ज्ञेन॑ तिष्ठति॒ न य॒ज्ञाद् भ्रꣳ॑शत॒ आ श्रा॑व॒येति॒ चतु॑रक्षर॒मस्तु॒ श्रौष॒डिति॒ चतु॑रक्षरं॒ यजेति॒ द्व्य॑क्षरं॒ ये यजा॑मह॒ इति॒ पञ्चा᳚क्षरं द्व्यक्ष॒रो व॑षट्का॒र ए॒ष वै स॑प्तद॒शः प्र॒जाप॑तिर्य॒ज्ञम॒न्वाय॑त्तो॒ य ए॒वं वेद॒ प्रति॑ य॒ज्ञेन॑ तिष्ठति॒ न य॒ज्ञाद् भ्रꣳ॑शते॒ यो वै य॒ज्ञस्य॒ प्राय॑णं प्रति॒ष्ठाम्॥३९॥

%1.6.11.2
उ॒दय॑नं॒ वेद॒ प्रति॑ष्ठिते॒नारि॑ष्टेन य॒ज्ञेन॑ स॒ꣴ॒स्थां ग॑च्छ॒त्या श्रा॑व॒यास्तु॒ श्रौष॒ड्यज॒ ये यजा॑महे वषट्का॒र ए॒तद्वै य॒ज्ञस्य॒ प्राय॑णमे॒षा प्र॑ति॒ष्ठैतदु॒दय॑नं॒ य ए॒वं वेद॒ प्रति॑ष्ठिते॒नारि॑ष्टेन य॒ज्ञेन॑ स॒ꣴ॒स्थां ग॑च्छति॒ यो वै सू॒नृता॑यै॒ दोहं॒ वेद॑ दु॒ह ए॒वैनां᳚ य॒ज्ञो वै सू॒नृता\-ऽ\-ऽश्रा॑व॒येत्यैवैना॑मह्व॒दस्तु॑॥४०॥

%1.6.11.3
श्रौष॒डित्यु॒पावा᳚स्रा॒ग्यजेत्युद॑नैषी॒द्ये यजा॑मह॒ इत्युपा॑सदद्वषट्का॒रेण॑ दोग्ध्ये॒ष वै सू॒नृता॑यै॒ दोहो॒ य ए॒वं वेद॑ दु॒ह ए॒वैनां᳚ दे॒वा वै स॒त्रमा॑सत॒ तेषां॒ दिशो॑\-ऽदस्य॒न्त ए॒तामा॒र्द्रां प॒ङ्क्तिम॑पश्य॒न्ना श्रा॑व॒येति॑ पुरोवा॒तम॑जनय॒न्नस्तु॒ श्रौष॒डित्य॒ब्भ्रꣳ सम॑प्लावय॒न्॒ यजेति॑ वि॒द्युतम्᳚॥४१॥

%1.6.11.4
अ॒ज॒न॒य॒न्॒ ये यजा॑मह॒ इति॒ प्राव॑र्\mbox{}षयन्न॒भ्य॑स्तनयन् वषट्का॒रेण॒ ततो॒ वै तेभ्यो॒ दिशः॒ प्राप्या॑यन्त॒ य ए॒वं वेद॒ प्रास्मै॒ दिशः॑ प्यायन्ते प्र॒जाप॑तिं त्वो॒वेद॑ प्र॒जाप॑तिस्त्वंवेद॒ यं प्र॒जाप॑ति॒र्वेद॒ स पुण्यो॑ भवत्ये॒ष वै छ॑न्द॒स्यः॑ प्र॒जाप॑ति॒रा श्रा॑व॒यास्तु॒ श्रौष॒ड्यज॒ ये यजा॑महे वषट्का॒रो य ए॒वं वेद॒ पुण्यो॑ भवति वस॒न्तम्॥४२॥

%1.6.11.5
ऋ॒तू॒नां प्री॑णा॒मीत्या॑ह॒र्तवो॒ वै प्र॑या॒जा ऋ॒तूने॒व प्री॑णाति॒ ते᳚\-ऽस्मै प्री॒ता य॑थापू॒र्वं क॑ल्पन्ते॒ कल्प॑न्ते\-ऽस्मा ऋ॒तवो॒ य ए॒वं वेदा॒ग्नीषोम॑योर॒हं दे॑वय॒ज्यया॒ चक्षु॑ष्मान् भूयास॒मित्या॑हा॒ग्नीषोमा᳚भ्यां॒ वै य॒ज्ञश्चक्षु॑ष्मा॒न् ताभ्या॑मे॒व चक्षु॑रा॒त्मन् ध॑त्ते॒\-ऽग्नेर॒हं दे॑वय॒ज्यया᳚न्ना॒दो भू॑यास॒मित्या॑हा॒ग्निर्वै दे॒वाना॑मन्ना॒दस्तेनै॒व॥४३॥

%1.6.11.6
अ॒न्नाद्य॑मा॒त्मन् ध॑त्ते॒ दब्धि॑र॒स्यद॑ब्धो भूयासम॒मुं द॑भेय॒मित्या॑है॒तया॒ वै दब्ध्या॑ दे॒वा असु॑रानदभ्नुव॒न् तयै॒व भ्रातृ॑व्यं दभ्नोत्य॒ग्नीषोम॑योर॒हं दे॑वय॒ज्यया॑ वृत्र॒हा भू॑यास॒मित्या॑हा॒ग्नीषोमा᳚भ्यां॒ वा इन्द्रो॑ वृ॒त्रम॑ह॒न् ताभ्या॑मे॒व भ्रातृ॑व्यꣴ स्तृणुत इन्द्राग्नि॒योर॒हं दे॑वय॒ज्यये᳚न्द्रिया॒व्य॑न्ना॒दो भू॑यास॒मित्या॑हेन्द्रिया॒व्ये॑वान्ना॒दो भ॑व॒तीन्द्र॑स्य॥४४॥

%1.6.11.7
अ॒हं दे॑वय॒ज्यये᳚न्द्रिया॒वी भू॑यास॒मित्या॑हेन्द्रिया॒व्ये॑व भ॑वति महे॒न्द्रस्या॒हं दे॑वय॒ज्यया॑ जे॒मानं॑ महि॒मानं॑ गमेय॒मित्या॑ह जे॒मान॑मे॒व म॑हि॒मानं॑ गच्छत्य॒ग्नेः स्वि॑ष्ट॒कृतो॒\-ऽहं दे॑वय॒ज्यया\-ऽ\-ऽयु॑ष्मान् य॒ज्ञेन॑ प्रति॒ष्ठां ग॑मेय॒मित्या॒हायु॑रे॒वात्मन् ध॑त्ते॒ प्रति॑ य॒ज्ञेन॑ तिष्ठति॥४५॥

%1.6.12.0
{\anuvakamend[{प्र॒ति॒ष्ठाम॑ह्व॒दस्तु॑ वि॒द्युतं॑ वस॒न्तमे॒वेन्द्र॑स्या॒\-ऽष्टात्रिꣳ॑शच्च}]}%॥11॥

%1.6.12.1
इन्द्रं॑ वो वि॒श्वत॒स्परि॒ हवा॑महे॒ जने᳚भ्यः। अ॒स्माक॑मस्तु॒ केव॑लः॥ इन्द्रं॒ नरो॑ ने॒मधि॑ता हवन्ते॒ यत्पार्या॑ यु॒नज॑ते॒ धिय॒स्ताः। शूरो॒ नृषा॑ता॒ शव॑सश्चका॒न आ गोम॑ति व्र॒जे भ॑जा॒ त्वं नः॑॥ इ॒न्द्रि॒याणि॑ शतक्रतो॒ या ते॒ जने॑षु प॒ञ्चसु॑। इन्द्र॒ तानि॑ त॒ आ वृ॑णे॥ अनु॑ ते दायि म॒ह इ॑न्द्रि॒याय॑ स॒त्रा ते॒ विश्व॒मनु॑ वृत्र॒हत्ये᳚। अनु॑॥४६॥

%1.6.12.2
क्ष॒त्रमनु॒ सहो॑ यज॒त्रेन्द्र॑ दे॒वेभि॒रनु॑ ते नृ॒षह्ये᳚॥ आ यस्मि᳚न्थ्स॒प्त वा॑स॒वास्तिष्ठ॑न्ति स्वा॒रुहो॑ यथा। ऋषि॑र्\mbox{}ह दीर्घ॒श्रुत्त॑म॒ इन्द्र॑स्य घ॒र्मो अति॑थिः॥ आ॒मासु॑ प॒क्वमैर॑य॒ आ सूर्यꣳ॑ रोहयो दि॒वि। घ॒र्मं न साम॑न्तपता सुवृ॒क्तिभि॒र्जुष्टं॒ गिर्व॑णसे॒ गिरः॑॥ इन्द्र॒मिद्गा॒थिनो॑ बृ॒हदिन्द्र॑म॒र्केभि॑र॒र्किणः॑। इन्द्रं॒ वाणी॑रनूषत॥ गाय॑न्ति त्वा गाय॒त्रिणः॑॥४७॥

%1.6.12.3
अर्च॑न्त्य॒र्कम॒र्किणः॑। ब्र॒ह्माण॑स्त्वा शतक्रत॒वुद्व॒ꣳ॒शमि॑व येमिरे॥ अ॒ꣳ॒हो॒मुचे॒ प्र भ॑रेमा मनी॒षामो॑षिष्ठ॒दाव्न्ने॑ सुम॒तिं गृ॑णा॒नाः। इ॒दमि॑न्द्र॒ प्रति॑ ह॒व्यं गृ॑भाय स॒त्याः स॑न्तु॒ यज॑मानस्य॒ कामाः᳚॥ वि॒वेष॒ यन्मा॑ धि॒षणा॑ ज॒जान॒ स्तवै॑ पु॒रा पार्या॒दिन्द्र॒मह्नः॑। अꣳह॑सो॒ यत्र॑ पी॒पर॒द्यथा॑ नो ना॒वेव॒ यान्त॑मु॒भये॑ हवन्ते॥ प्र स॒म्राजं॑ प्रथ॒मम॑ध्व॒राणा᳚म्॥४८॥

%1.6.12.4
अ॒ꣳ॒हो॒मुचं॑ वृष॒भं य॒ज्ञिया॑नाम्। अ॒पां नपा॑तमश्विना॒ हय॑न्तम॒स्मिन्न॑र इन्द्रि॒यं ध॑त्त॒मोजः॑॥ वि न॑ इन्द्र॒ मृधो॑ जहि नी॒चा य॑च्छ पृतन्य॒तः। अ॒ध॒स्प॒दं तमीं᳚ कृधि॒ यो अ॒स्माꣳ अ॑भि॒दास॑ति॥ इन्द्र॑ क्ष॒त्रम॒भि वा॒ममोजो\-ऽजा॑यथा वृषभ चर्\mbox{}षणी॒नाम्। अपा॑नुदो॒ जन॑ममित्र॒यन्त॑मु॒रुं दे॒वेभ्यो॑ अकृणोरु लो॒कम्॥ मृ॒गो न भी॒मः कु॑च॒रो गि॑रि॒ष्ठाः प॑रा॒वतः॑॥४९॥

%1.6.12.5
आ ज॑गामा॒ पर॑स्याः। सृ॒कꣳ स॒ꣳ॒शाय॑ प॒विमि॑न्द्र ति॒ग्मं वि शत्रू᳚न् ताढि॒ वि मृधो॑ नुदस्व॥ वि शत्रू॒न्॒ वि मृधो॑ नुद॒ वि वृ॒त्रस्य॒ हनू॑ रुज। वि म॒न्युमि॑न्द्र भामि॒तो॑\-ऽमित्र॑स्याभि॒दास॑तः॥ त्रा॒तार॒मिन्द्र॑मवि॒तार॒मिन्द्र॒ꣳ॒ हवे॑हवे सु॒हव॒ꣳ॒ शूर॒मिन्द्रम्᳚। हु॒वे नु श॒क्रं पु॑रुहू॒तमिन्द्रꣴ॑ स्व॒स्ति नो॑ म॒घवा॑ धा॒त्विन्द्रः॑॥ मा ते॑ अ॒स्याम्॥५०॥

%1.6.12.6
स॒ह॒सा॒व॒न् परि॑ष्टाव॒घाय॑ भूम हरिवः परा॒दै। त्राय॑स्व नो\-ऽवृ॒केभि॒र्वरू॑थै॒स्तव॑ प्रि॒यासः॑ सू॒रिषु॑ स्याम॥ अन॑वस्ते॒ रथ॒मश्वा॑य तक्ष॒न् त्वष्टा॒ वज्रं॑ पुरुहूत द्यु॒मन्तम्᳚। ब्र॒ह्माण॒ इन्द्रं॑ म॒हय॑न्तो अ॒र्कैरव॑र्धय॒न्नह॑ये॒ हन्त॒वा उ॑॥ वृष्णे॒ यत् ते॒ वृष॑णो अ॒र्कमर्चा॒निन्द्र॒ ग्रावा॑णो॒ अदि॑तिः स॒जोषाः᳚। अ॒न॒श्वासो॒ ये प॒वयो॑\-ऽर॒था इन्द्रे॑षिता अ॒भ्यव॑र्तन्त॒ दस्यून्॑॥५१॥

{\anuvakamend[{वृ॒त्र॒हत्ये\-ऽनु॑ गाय॒त्रिणो᳚\-ऽध्व॒राणां᳚ परा॒वतो॒\-ऽस्याम॒ष्टाच॑त्वारिꣳशच्च}]}%॥12॥
%%% END PRASHNA

\sect{सप्तमः प्रश्नः}\setcounter{anuvakam}{0}
\dnsub{तैत्तिरीयसंहितायां प्रथमकाण्डे सप्तमः प्रश्नः}
%1.7.1.0
%1.7.1.1
पा॒क॒य॒ज्ञं वा अन्वाहि॑ताग्नेः प॒शव॒ उप॑ तिष्ठन्त॒ इडा॒ खलु॒ वै पा॑कय॒ज्ञः सैषान्त॒रा प्र॑याजानूया॒जान् यज॑मानस्य लो॒के\-ऽव॑हिता॒ तामा᳚ह्रि॒यमा॑णाम॒भि म॑न्त्रयेत॒ सुरू॑पवर्\mbox{}षवर्ण॒ एहीति॑ प॒शवो॒ वा इडा॑ प॒शूने॒वोप॑ ह्वयते य॒ज्ञं वै दे॒वा अदु॑ह्रन् य॒ज्ञो\-ऽसु॑राꣳ अदुह॒त् ते\-ऽसु॑रा य॒ज्ञदु॑ग्धाः॒ परा॑\-ऽभव॒न्॒ यो वै य॒ज्ञस्य॒ दोहं॑ वि॒द्वान्॥१॥

%1.7.1.2
यज॒ते\-ऽप्य॒न्यं यज॑मानं दुहे॒ सा मे॑ स॒त्या\-ऽ\-ऽशीर॒स्य य॒ज्ञस्य॑ भूया॒दित्या॑है॒ष वै य॒ज्ञस्य॒ दोह॒स्तेनै॒वैनं॑ दुहे॒ प्रत्ता॒ वै गौर्दु॑हे॒ प्रत्तेडा॒ यज॑मानाय दुह ए॒ते वा इडा॑यै॒ स्तना॒ इडोप॑हू॒तेति॑ वा॒युर्व॒थ्सो यर्\mbox{}हि॒ होतेडा॑मुप॒ह्वये॑त॒ तर्\mbox{}हि॒ यज॑मानो॒ होता॑र॒मीक्ष॑माणो वा॒युं मन॑सा ध्यायेत्॥२॥

%1.7.1.3
मा॒त्रे व॒थ्समु॒पाव॑सृजति॒ सर्वे॑ण॒ वै य॒ज्ञेन॑ दे॒वाः सु॑व॒र्गं लो॒कमा॑यन् पाकय॒ज्ञेन॒ मनु॑रश्राम्य॒थ्सेडा॒ मनु॑मु॒पाव॑र्तत॒ तान्दे॑वासु॒रा व्य॑ह्वयन्त प्र॒तीचीं᳚ दे॒वाः परा॑ची॒मसु॑राः॒ सा दे॒वानु॒पाव॑र्तत प॒शवो॒ वै तद्दे॒वान॑वृणत प॒शवो\-ऽसु॑रानजहु॒र्यं का॒मये॑ताप॒शुः स्या॒दिति॒ परा॑चीं॒ तस्येडा॒मुप॑ह्वयेताप॒शुरे॒व भ॑वति॒ यं॥३॥

%1.7.1.4
का॒मये॑त पशु॒मान्थ्स्या॒दिति॑ प्र॒तीचीं॒ तस्येडा॒मुप॑ह्वयेत पशु॒माने॒व भ॑वति ब्रह्मवा॒दिनो॑ वदन्ति॒ स त्वा इडा॒मुप॑ह्वयेत॒ य इडा॑मुप॒हूया॒त्मान॒मिडा॑यामुप॒ह्वये॒तेति॒ सा नः॑ प्रि॒या सु॒प्रतू᳚र्तिर्म॒घोनीत्या॒हेडा॑मे॒वोप॒हूया॒\-ऽ\-ऽत्मान॒मिडा॑या॒मुप॑ ह्वयते॒ व्य॑स्तमिव॒ वा ए॒तद्य॒ज्ञस्य॒ यदिडा॑ सा॒मि प्रा॒श्ञन्ति॑॥४॥

%1.7.1.5
सा॒मि मा᳚र्जयन्त ए॒तत् प्रति॒ वा असु॑राणां य॒ज्ञो व्य॑च्छिद्यत॒ ब्रह्म॑णा दे॒वाः सम॑दधु॒र्बृह॒स्पति॑स्तनुतामि॒मं न॒ इत्या॑ह॒ ब्रह्म॒ वै दे॒वानां॒ बृह॒स्पति॒र्ब्रह्म॑णै॒व य॒ज्ञꣳ सन्द॑धाति॒ विच्छि॑न्नं य॒ज्ञꣳ समि॒मं द॑धा॒त्वित्या॑ह॒ सन्त॑त्यै॒ विश्वे॑ दे॒वा इ॒ह मा॑दयन्ता॒मित्या॑ह स॒न्तत्यै॒व य॒ज्ञं दे॒वेभ्यो\-ऽनु॑ दिशति॒ यां वै॥५॥


%1.7.1.6
य॒ज्ञे दक्षि॑णां॒ ददा॑ति॒ ताम॑स्य प॒शवो\-ऽनु॒ सङ्क्रा॑मन्ति॒ स ए॒ष ई॑जा॒नो॑\-ऽप॒शुर्भावु॑को॒ यज॑मानेन॒ खलु॒ वै तत्का॒र्य॑मित्या॑हु॒र्यथा॑ देव॒त्रा द॒त्तं कु॑रँवी॒तात्मन् प॒शून् र॒मये॒तेति॒ ब्रध्न॒ पिन्व॒स्वेत्या॑ह य॒ज्ञो वै ब्र॒ध्नो य॒ज्ञमे॒व तन्म॑हय॒त्यथो॑ देव॒त्रैव द॒त्तं कु॑रुत आ॒त्मन् प॒शून् र॑मयते॒ दद॑तो मे॒ मा क्षा॒यीत्या॒हाक्षि॑तिमे॒वोपै॑ति कुर्व॒तो मे॒ मोप॑ दस॒दित्या॑ह भू॒मान॑मे॒वोपै॑ति॥६॥

%1.7.2.0
{\anuvakamend[{वि॒द्वान्ध्या॑ये॒ द्यं प्रा॒श्ञन्ति॒ यां वै म॒ एका॒न्नविꣳ॑श॒तिश्च॑}]}%॥१॥

%1.7.2.1
सꣴश्र॑वा ह सौवर्चन॒सस्तुमि॑ञ्ज॒मौपो॑दितिमुवाच॒ यथ्स॒त्रिणा॒ꣳ॒ होता\-ऽभूः॒ कामिडा॒मुपा᳚ह्वथा॒ इति॒ तामुपा᳚ह्व॒ इति॑ होवाच॒ या प्रा॒णेन॑ दे॒वान् दा॒धार॑ व्या॒नेन॑ मनु॒ष्या॑नपा॒नेन॑ पि॒तॄनिति॑ छि॒नत्ति॒ सा न छि॑न॒त्ती (३) इति॑ छि॒नत्तीति॑ होवाच॒ शरी॑रं॒ वा अ॑स्यै॒ तदुपा᳚ह्वथा॒ इति॑ होवाच॒ गौर्वै॥७॥

%1.7.2.2
अ॒स्यै॒ शरी॑रं॒ गां वाव तौ तत्पर्य॑वदतां॒ या य॒ज्ञे दी॒यते॒ सा प्रा॒णेन॑ दे॒वान् दा॑धार॒ यया॑ मनु॒ष्या॑ जीव॑न्ति॒ सा व्या॒नेन॑ मनु॒ष्यान्॑ यां पि॒तृभ्यो॒ घ्नन्ति॒ सा\-ऽपा॒नेन॑ पि॒तॄन् य ए॒वं वेद॑ पशु॒मान् भ॑व॒त्यथ॒ वै तामुपा᳚ह्व॒ इति॑ होवाच॒ या प्र॒जाः प्र॒भव॑न्तीः॒ प्रत्या॒भव॒तीत्यन्नं॒ वा अ॑स्यै॒ तत्॥८॥

%1.7.2.3
उपा᳚ह्वथा॒ इति॑ होवा॒चौष॑धयो॒ वा अ॑स्या॒ अन्न॒मोष॑धयो॒ वै प्र॒जाः प्र॒भव॑न्तीः॒ प्रत्या भ॑वन्ति॒ य ए॒वं वेदा᳚न्ना॒दो भ॑व॒त्यथ॒ वै तामुपा᳚ह्व॒ इति॑ होवाच॒ या प्र॒जाः प॑रा॒भव॑न्तीरनुगृ॒ह्णाति॒ प्रत्या॒भव॑न्तीर्गृ॒ह्णातीति॑ प्रति॒ष्ठां वा अ॑स्यै॒ तदुपा᳚ह्वथा॒ इति॑ होवाचे॒यं वा अ॑स्यै प्रति॒ष्ठा॥९॥

%1.7.2.4
इ॒यं वै प्र॒जाः प॑रा॒भव॑न्ती॒रनु॑गृह्णाति॒ प्रत्या॒भव॑न्तीर्गृह्णाति॒ य ए॒वं वेद॒ प्रत्ये॒व ति॑ष्ठ॒त्यथ॒ वै तामुपा᳚ह्व॒ इति॑ होवाच॒ यस्यै॑ नि॒क्रम॑णे घृ॒तं प्र॒जाः सं॒जीव॑न्तीः॒ पिब॒न्तीति॑ छि॒नत्ति॒ सा न छि॑न॒त्ती (३) इति॒ न छि॑न॒त्तीति॑ होवाच॒ प्र तु ज॑नय॒तीत्ये॒ष वा इडा॒मुपा᳚ह्वथा॒ इति॑ होवाच॒ वृष्टि॒र्वा इडा॒ वृष्ट्यै॒ वै नि॒क्रम॑णे घृ॒तं प्र॒जाः सं॒जीव॑न्तीः पिबन्ति॒ य ए॒वं वेद॒ प्रैव जा॑यते\-ऽन्ना॒दो भ॑वति॥१०॥

%1.7.3.0
{\anuvakamend[{गौर्वा अ॑स्यै॒ तत् प्र॑ति॒ष्ठा\-ऽह्व॑था॒ इति॑ विꣳश॒तिश्च॑}]}%॥२॥

%1.7.3.1
प॒रोक्षं॒ वा अ॒न्ये दे॒वा इ॒ज्यन्ते᳚ प्र॒त्यक्ष॑म॒न्ये यद्यज॑ते॒ य ए॒व दे॒वाः प॒रोक्ष॑मि॒ज्यन्ते॒ ताने॒व तद्य॑जति॒ यद॑न्वाहा॒र्य॑मा॒हर॑त्ये॒ते वै दे॒वाः प्र॒त्यक्षं॒ यद् ब्रा᳚ह्म॒णास्ताने॒व तेन॑ प्रीणा॒त्यथो॒ दक्षि॑णै॒वास्यै॒षा\-ऽथो॑ य॒ज्ञस्यै॒व छि॒द्रमपि॑ दधाति॒ यद्वै य॒ज्ञस्य॑ क्रू॒रं यद्विलि॑ष्टं॒ तद॑न्वाहा॒र्ये॑ण॥११॥

%1.7.3.2
अ॒न्वाह॑रति॒ तद॑न्वाहा॒र्य॑स्यान्वाहार्य॒त्वं दे॑वदू॒ता वा ए॒ते यदृ॒त्विजो॒ यद॑न्वाहा॒र्य॑मा॒हर॑ति देवदू॒ताने॒व प्री॑णाति प्र॒जाप॑तिर्दे॒वेभ्यो॑ य॒ज्ञान् व्यादि॑श॒थ्स रि॑रिचा॒नो॑\-ऽमन्यत॒ स ए॒तम॑न्वाहा॒र्य॑मभ॑क्तमपश्य॒त् तमा॒त्मन्न॑धत्त॒ स वा ए॒ष प्रा॑जाप॒त्यो यद॑न्वाहा॒र्यो॑ यस्यै॒वं वि॒दुषो᳚\-ऽन्वाहा॒र्य॑ आह्रि॒यते॑ सा॒क्षादे॒व प्र॒जाप॑तिमृध्नो॒त्यप॑रिमितो नि॒रुप्यो\-ऽप॑रिमितः प्र॒जाप॑तिः प्र॒जाप॑तेः॥१२॥

%1.7.3.3
आप्त्यै॑ दे॒वा वै यद्य॒ज्ञे\-ऽकु॑र्वत॒ तदसु॑रा अकुर्वत॒ ते दे॒वा ए॒तं प्रा॑जाप॒त्यम॑न्वाहा॒र्य॑मपश्य॒न् तम॒न्वाह॑रन्त॒ ततो॑ दे॒वा अभ॑व॒न् परासु॑रा॒ यस्यै॒वं वि॒दुषो᳚\-ऽन्वाहा॒र्य॑ आह्रि॒यते॒ भव॑त्या॒त्मना॒ परा᳚स्य॒ भ्रातृ॑व्यो भवति य॒ज्ञेन॒ वा इ॒ष्टी प॒क्वेन॑ पू॒र्ती यस्यै॒वं वि॒दुषो᳚\-ऽन्वाहा॒र्य॑ आह्रि॒यते॒ स त्वे॑वेष्टा॑पू॒र्ती प्र॒जाप॑तेर्भा॒गो॑\-ऽसि॥१३॥

%1.7.3.4
इत्या॑ह प्र॒जाप॑तिमे॒व भा॑ग॒धेये॑न॒ सम॑र्धय॒त्यूर्ज॑स्वा॒न् पय॑स्वा॒नित्या॒होर्ज॑मे॒वास्मि॒न् पयो॑ दधाति प्राणापा॒नौ मे॑ पाहि समानव्या॒नौ मे॑ पा॒हीत्या॑हा॒\-ऽ\-ऽशिष॑मे॒वैतामा शा॒स्ते\-ऽक्षि॑तो॒\-ऽस्यक्षि॑त्यै त्वा॒ मा मे᳚ क्षेष्ठा अ॒मुत्रा॒मुष्मि॑ल्लोँ॒क इत्या॑ह॒ क्षीय॑ते॒ वा अ॒मुष्मि॑ल्लोँ॒के\-ऽन्न॑मि॒तः प्र॑दान॒ꣴ॒ ह्य॑मुष्मि॑ल्लोँ॒के प्र॒जा उ॑प॒जीव॑न्ति॒ यदे॒वम॑भिमृ॒शत्यक्षि॑तिमे॒वैन॑द्गमयति॒ नास्या॒मुष्मि॑ल्लोँ॒के\-ऽन्नं॑ क्षीयते॥१४॥

%1.7.4.0
{\anuvakamend[{अ॒न्वा॒हा॒र्ये॑ण प्र॒जाप॑तेरसि॒ ह्य॑मुष्मि॑ल्लोँ॒के पञ्च॑दश च}]}%॥३॥

%1.7.4.1
ब॒र्॒\mbox{}हिषो॒\-ऽहं दे॑वय॒ज्यया᳚ प्र॒जावा᳚न् भूयास॒मित्या॑ह ब॒र्॒\mbox{}हिषा॒ वै प्र॒जाप॑तिः प्र॒जा अ॑सृजत॒ तेनै॒व प्र॒जाः सृ॑जते॒ नरा॒शꣳस॑स्या॒हं दे॑वय॒ज्यया॑ पशु॒मान् भू॑यास॒मित्या॑ह॒ नरा॒शꣳसे॑न॒ वै प्र॒जाप॑तिः प॒शून॑सृजत॒ तेनै॒व प॒शून्थ्सृ॑जते॒\-ऽग्नेः स्वि॑ष्ट॒कृतो॒\-ऽहं दे॑वय॒ज्यया\-ऽ\-ऽयु॑ष्मान् य॒ज्ञेन॑ प्रति॒ष्ठां ग॑मेय॒मित्या॒हा\-ऽ\-ऽयु॑रे॒वात्मन् ध॑त्ते॒ प्रति॑ य॒ज्ञेन॑ तिष्ठति दर्\mbox{}शपूर्णमा॒सयोः᳚॥१५॥

%1.7.4.2
र्वै दे॒वा उज्जि॑ति॒मनूद॑जयन् दर्\mbox{}शपूर्णमा॒साभ्या॒मसु॑रा॒नपा॑नुदन्ता॒ग्नेर॒हमुज्जि॑ति॒मनूज्जे॑ष॒मित्याह दर्\mbox{}शपूर्णमा॒सयो॑रे॒व दे॒वता॑नां॒ यज॑मान॒ उज्जि॑ति॒मनूज्ज॑यति दर्\mbox{}शपूर्णमा॒साभ्यां॒ भ्रातृ॑व्या॒नप॑ नुदते॒ वाज॑वतीभ्यां॒ व्यू॑ह॒त्यन्नं॒ वै वाजो\-ऽन्न॑मे॒वाव॑रुन्धे॒ द्वाभ्यां॒ प्रति॑ष्ठित्यै॒ यो वै य॒ज्ञस्य॒ द्वौ दोहौ॑ वि॒द्वान् यज॑त उभ॒यतः॑॥१६॥

%1.7.4.3
ए॒व य॒ज्ञं दु॑हे पु॒रस्ता᳚च्चो॒परि॑ष्टाच्चै॒ष वा अ॒न्यो य॒ज्ञस्य॒ दोह॒ इडा॑याम॒न्यो यर्\mbox{}हि॒ होता॒ यज॑मानस्य॒ नाम॑ गृह्णी॒यात् तर्\mbox{}हि॑ ब्रूया॒देमा अ॑ग्मन्ना॒शिषो॒ दोह॑कामा॒ इति॒ सꣴस्तु॑ता ए॒व दे॒वता॑ दु॒हे\-ऽथो॑ उभ॒यत॑ ए॒व य॒ज्ञं दु॑हे पु॒रस्ता᳚च्चो॒परि॑ष्टाच्च॒ रोहि॑तेन त्वा॒\-ऽग्निर्दे॒वतां᳚ गमय॒त्वित्या॑है॒ते वै दे॑वा॒श्वाः॥१७॥

%1.7.4.4
यज॑मानः प्रस्त॒रो यदे॒तैः प्र॑स्त॒रं प्र॒हर॑ति देवा॒श्वैरे॒व यज॑मानꣳ सुव॒र्गं लो॒कं ग॑मयति॒ वि ते॑ मुञ्चामि रश॒ना वि र॒श्मीनित्या॑है॒ष वा अ॒ग्नेर्वि॑मो॒कस्तेनै॒वैनं॒ वि मु॑ञ्चति॒ विष्णोः᳚ शं॒योर॒हं दे॑वय॒ज्यया॑ य॒ज्ञेन॑ प्रति॒ष्ठां ग॑मेय॒मित्या॑ह य॒ज्ञो वै विष्णु॑र्य॒ज्ञ ए॒वान्त॒तः प्रति॑ तिष्ठति॒ सोम॑स्या॒हं दे॑वय॒ज्यया॑ सु॒रेताः᳚॥१८॥

%1.7.4.5
रेतो॑ धिषी॒येत्या॑ह॒ सोमो॒ वै रे॑तो॒धास्तेनै॒व रेत॑ आ॒त्मन् ध॑त्ते॒ त्वष्टु॑र॒हं दे॑वय॒ज्यया॑ पशू॒नाꣳ रू॒पं पु॑षेय॒मित्या॑ह॒ त्वष्टा॒ वै प॑शू॒नां मि॑थु॒नानाꣳ॑ रूप॒कृत्तेनै॒व प॑शू॒नाꣳ रू॒पमा॒त्मन् ध॑त्ते दे॒वानां॒ पत्नी॑र॒ग्निर्गृ॒हप॑तिर्य॒ज्ञस्य॑ मिथु॒नं तयो॑र॒हं दे॑वय॒ज्यया॑ मिथु॒नेन॒ प्र भू॑यास॒मित्या॑है॒तस्मा॒द्वै मि॑थु॒नात्प्र॒जाप॑तिर्मिथु॒नेन॑॥१९॥

%1.7.4.6
प्राजा॑यत॒ तस्मा॑दे॒व यज॑मानो मिथु॒नेन॒ प्र जा॑यते वे॒दो॑\-ऽसि॒ वित्ति॑रसि वि॒देयेत्या॑ह वे॒देन॒ वै दे॒वा असु॑राणां वि॒त्तं वेद्य॑मविन्दन्त॒ तद्वे॒दस्य॑ वेद॒त्वं यद्य॒द् भ्रातृ॑व्यस्याभि॒ध्याये॒त् तस्य॒ नाम॑ गृह्णीया॒त् तदे॒वास्य॒ सर्वं॑ वृङ्क्ते घृ॒तव॑न्तं कुला॒यिनꣳ॑ रा॒यस्पोषꣳ॑ सह॒स्रिणं॑ वे॒दो द॑दातु वा॒जिन॒मित्या॑ह॒ प्र स॒हस्रं॑ प॒शूना᳚प्नो॒त्या\-ऽस्य॑ प्र॒जायां᳚ वा॒जी जा॑यते॒ य ए॒वं वेद॑॥२०॥

%1.7.5.0
{\anuvakamend[{द॒र्॒\mbox{}श॒पू॒र्ण॒मा॒सयो॑रुभ॒यतो॑ देवा॒श्वाः सु॒रेताः᳚ प्र॒जाप॑तिर्मिथु॒नेना᳚\-ऽ\-ऽप्नोत्य॒ष्टौ च॑}]}%॥४॥

%1.7.5.1
ध्रु॒वां वै रिच्य॑मानां य॒ज्ञो\-ऽनु॑ रिच्यते य॒ज्ञं यज॑मानो॒ यज॑मानं प्र॒जा ध्रु॒वामा॒प्याय॑मानां य॒ज्ञो\-ऽन्वा प्या॑यते य॒ज्ञं यज॑मानो॒ यज॑मानं प्र॒जा आ प्या॑यतां ध्रु॒वा घृ॒तेनेत्या॑ह ध्रु॒वामे॒वा\-ऽ\-ऽप्या॑ययति॒ तामा॒प्याय॑मानां य॒ज्ञो\-ऽन्वा प्या॑यते य॒ज्ञं यज॑मानो॒ यज॑मानं प्र॒जाः प्र॒जाप॑तेर्वि॒भान्नाम॑ लो॒कस्तस्मिꣴ॑ स्त्वा दधामि स॒ह यज॑माने॒नेति॑॥२१॥

%1.7.5.2
आ॒हा॒यं वै प्र॒जाप॑तेर्वि॒भान्नाम॑ लो॒कस्तस्मि॑न्ने॒वैनं॑ दधाति स॒ह यज॑मानेन॒ रिच्य॑त इव॒ वा ए॒तद्यद्यज॑ते॒ यद्य॑जमानभा॒गं प्रा॒श्ञात्या॒त्मान॑मे॒व प्री॑णात्ये॒तावा॒न्॒ वै य॒ज्ञो यावान्॑ यजमानभा॒गो य॒ज्ञो यज॑मानो॒ यद्य॑जमानभा॒गं प्रा॒श्ञाति॑ य॒ज्ञ ए॒व य॒ज्ञं प्रति॑ष्ठापयत्ये॒तद्वै सू॒यव॑स॒ꣳ॒ सोद॑कं॒ यद्ब॒र्॒\mbox{}हिश्चा\-ऽ\-ऽप॑श्चै॒तत्॥२२॥

%1.7.5.3
यज॑मानस्या॒\-ऽ\-ऽयत॑नं॒ यद्वेदि॒र्यत् पू᳚र्णपा॒त्रम॑न्तर्वे॒दि नि॒नय॑ति॒ स्व ए॒वा\-ऽ\-ऽयत॑ने सू॒यव॑स॒ꣳ॒ सोद॑कं कुरुते॒ सद॑सि॒ सन्मे॑ भूया॒ इत्या॒हा\-ऽ\-ऽपो॒ वै य॒ज्ञ आपो॒\-ऽमृतं॑ य॒ज्ञमे॒वामृत॑मा॒त्मन्ध॑त्ते॒ सर्वा॑णि॒ वै भू॒तानि॑ व्र॒तमु॑प॒यन्त॒मनूप॑ यन्ति॒ प्राच्यां᳚ दि॒शि दे॒वा ऋ॒त्विजो॑ मार्जयन्ता॒मित्या॑है॒ष वै द॑र्\mbox{}शपूर्णमा॒सयो॑रवभृ॒थः॥२३॥

%1.7.5.4
यान्ये॒वैनं॑ भू॒तानि॑ व्र॒तमु॑प॒यन्त॒मनूप॑यन्ति॒ तैरे॒व स॒हाव॑भृ॒थमवै॑ति॒ विष्णु॑मुखा॒ वै दे॒वाश्छन्दो॑भिरि॒माल्लोँ॒कान॑नप\-ज॒य्यम॒भ्य॑जय॒न्॒ यद्वि॑ष्णुक्र॒मान् क्रम॑ते॒ विष्णु॑रे॒व भू॒त्वा यज॑मान॒श्छन्दो॑भिरि॒माल्लोँ॒कान॑नपज॒य्यम॒भि ज॑यति॒ विष्णोः॒ क्रमो᳚\-ऽस्यभिमाति॒हेत्या॑ह गाय॒त्री वै पृ॑थि॒वी त्रैष्टु॑भम॒न्तरि॑क्षं॒ जाग॑ती॒ द्यौरानु॑ष्टुभी॒र्दिश॒श्छन्दो॑भिरे॒वेमाल्लोँ॒कान् य॑थापू॒र्वम॒भि ज॑यति॥२४॥

%1.7.6.0
{\anuvakamend[{इत्ये॒तद॑वभृ॒थो दिशः॑ स॒प्त च॑}]}%॥५॥

%1.7.6.1
अग॑न्म॒ सुवः॒ सुव॑रग॒न्मेत्या॑ह सुव॒र्गमे॒व लो॒कमे॑ति सं॒दृश॑स्ते॒ मा छि॑थ्सि॒ यत्ते॒ तप॒स्तस्मै॑ ते॒ मा वृ॒क्षीत्या॑ह यथाय॒जुरे॒वैतथ्सु॒भूर॑सि॒ श्रेष्ठो॑ रश्मी॒नामा॑यु॒र्धा अ॒स्यायु॑र्मे धे॒हीत्या॑हा॒\-ऽ\-ऽशिष॑मे॒वैतामा शा᳚स्ते॒ प्र वा ए॒षो᳚\-ऽस्माल्लो॒काच्च्य॑वते॒ यः॥२५॥

%1.7.6.2
वि॒ष्णु॒क्र॒मान् क्रम॑ते सुव॒र्गाय॒ हि लो॒काय॑ विष्णुक्र॒माः क्र॒म्यन्ते᳚ ब्रह्मवा॒दिनो॑ वदन्ति॒ स त्वै वि॑ष्णुक्र॒मान् क्र॑मेत॒ य इ॒माल्लोँ॒कान् भ्रातृ॑व्यस्य सं॒विद्य॒ पुन॑रि॒मं लो॒कं प्र॑त्यव॒रोहे॒दित्ये॒ष वा अ॒स्य लो॒कस्य॑ प्रत्यवरो॒हो यदाहे॒दम॒हम॒मुं भ्रातृ॑व्यमा॒भ्यो दि॒ग्भ्यो᳚\-ऽस्यै दि॒व इती॒माने॒व लो॒कान्भ्रातृ॑व्यस्य सं॒विद्य॒ पुन॑रि॒मं लो॒कं प्र॒त्यव॑रोहति॒ सं॥२६॥

%1.7.6.3
ज्योति॑षा\-ऽभूव॒मित्या॑हा॒स्मिन्ने॒व लो॒के प्रति॑ तिष्ठत्यै॒न्द्रीमा॒वृत॑म॒न्वाव॑र्त॒ इत्या॑हा॒सौ वा आ॑दि॒त्य इन्द्र॒स्तस्यै॒वा\-ऽ\-ऽ\-वृत॒मनु॑ प॒र्याव॑र्तते दक्षि॒णा प॒र्याव॑र्तते॒ स्वमे॒व वी॒र्य॑मनु॑ प॒र्याव॑र्तते॒ तस्मा॒द्दक्षि॒णो\-ऽर्ध॑ आ॒त्मनो॑ वी॒र्या॑वत्त॒रो\-ऽथो॑ आदि॒त्यस्यै॒वा\-ऽ\-ऽवृत॒मनु॑ प॒र्याव॑र्तते॒ सम॒हं प्र॒जया॒ सं मया᳚ प्र॒जेत्या॑हा॒\-ऽ\-ऽशिषम्᳚॥२७॥

%1.7.6.4
ए॒वैतामा शा᳚स्ते॒ समि॑द्धो अग्ने मे दीदिहि समे॒द्धा ते॑ अग्ने दीद्यास॒मित्या॑ह यथाय॒जुरे॒वैतद्वसु॑मान् य॒ज्ञो वसी॑यान् भूयास॒मित्या॑हा॒\-ऽ\-ऽशिष॑मे॒वैतामा शा᳚स्ते ब॒हु वै गार्\mbox{}ह॑पत्य॒स्यान्ते॑ मि॒श्रमि॑व चर्यत आग्निपावमा॒नीभ्यां॒ गार्\mbox{}ह॑पत्य॒मुप॑ तिष्ठते पु॒नात्ये॒वाग्निं पु॑नी॒त आ॒त्मानं॒ द्वाभ्यां॒ प्रति॑ष्ठित्या॒ अग्ने॑ गृहपत॒ इत्या॑ह॥२८॥

%1.7.6.5
य॒था॒य॒जुरे॒वैतच्छ॒तꣳ हिमा॒ इत्या॑ह श॒तं त्वा॑ हेम॒न्तानि॑न्धिषी॒येति॒ वावैतदा॑ह पु॒त्रस्य॒ नाम॑ गृह्णात्यन्ना॒दमे॒वैनं॑ करोति॒ तामा॒शिष॒मा शा॑से॒ तन्त॑वे॒ ज्योति॑ष्मती॒मिति॑ ब्रूया॒द्यस्य॑ पु॒त्रो\-ऽजा॑तः॒ स्यात् ते॑ज॒स्व्ये॑वास्य॑ ब्रह्मवर्च॒सी पु॒त्रो जा॑यते॒ तामा॒शिष॒मा शा॑से॒\-ऽमुष्मै॒ ज्योति॑ष्मती॒मिति॑ ब्रूया॒द्यस्य॑ पु॒त्रः॥२९॥

%1.7.6.6
जा॒तः स्यात् तेज॑ ए॒वास्मि॑न् ब्रह्मवर्च॒सं द॑धाति॒ यो वै य॒ज्ञं प्र॒युज्य॒ न वि॑मु॒ञ्चत्य॑प्रतिष्ठा॒नो वै स भ॑वति॒ कस्त्वा॑ युनक्ति॒ स त्वा॒ वि मु॑ञ्च॒त्वित्या॑ह प्र॒जाप॑ति॒र्वै कः प्र॒जाप॑तिनै॒वैनं॑ यु॒नक्ति॑ प्र॒जाप॑तिना॒ वि मु॑ञ्चति॒ प्रति॑ष्ठित्या ईश्व॒रं वै व्र॒तमवि॑सृष्टं प्र॒दहो\-ऽग्ने᳚ व्रतपते व्र॒तम॑चारिष॒मित्या॑ह व्र॒तमे॒व॥३०॥

%1.7.6.7
वि सृ॑जते॒ शान्त्या॒ अप्र॑दाहाय॒ परा॒ङ्॒ वाव य॒ज्ञ ए॑ति॒ न नि व॑र्तते॒ पुन॒र्यो वै य॒ज्ञस्य॑ पुनराल॒म्भं वि॒द्वान् यज॑ते॒ तम॒भि नि व॑र्तते य॒ज्ञो ब॑भूव॒ स आ ब॑भू॒वेत्या॑है॒ष वै य॒ज्ञस्य॑ पुनराल॒म्भस्तेनै॒वैनं॒ पुन॒राल॑भ॒ते\-ऽन॑वरुद्धा॒ वा ए॒तस्य॑ वि॒राड्य आहि॑ताग्निः॒ सन्न॑स॒भः प॒शवः॒ खलु॒ वै ब्रा᳚ह्म॒णस्य॑ स॒भेष्ट्वा प्राङु॒त्क्रम्य॑ ब्रूया॒द्गोमाꣳ॑ अ॒ग्ने\-ऽवि॑माꣳ अ॒श्वी य॒ज्ञ इत्यव॑ स॒भाꣳ रु॒न्धे प्र स॒हस्रं॑ प॒शूना᳚प्नो॒त्यास्य॑ प्र॒जायां᳚ वा॒जी जा॑यते॥३१॥

%1.7.7.0
{\anuvakamend[{यः स मा॒शिषं॑ गृहपत॒ इत्या॑ह॒ यस्य॑ पु॒त्रो व्र॒तमे॒व खलु॒ वै चतु॑र्विꣳशतिश्च}]}%॥६॥

%1.7.7.1
देव॑ सवितः॒ प्रसु॑व य॒ज्ञं प्रसु॑व य॒ज्ञप॑तिं॒ भगा॑य दि॒व्यो ग॑न्ध॒र्वः। के॒त॒पूः केतं॑ नः पुनातु वा॒चस्पति॒र्वाच॑म॒द्य स्व॑दाति नः॥ इन्द्र॑स्य॒ वज्रो॑\-ऽसि॒ वार्त्र॑घ्न॒स्त्वया॒\-ऽयं वृ॒त्रं व॑ध्यात्॥ वाज॑स्य॒ नु प्र॑स॒वे मा॒तरं॑ म॒हीमदि॑तिं॒ नाम॒ वच॑सा करामहे। यस्या॑मि॒दं विश्वं॒ भुव॑नमावि॒वेश॒ तस्यां᳚ नो दे॒वः स॑वि॒ता धर्म॑ साविषत्॥ अ॒फ्सु॥३२॥

%1.7.7.2
अ॒न्तर॒मृत॑म॒फ्सु भे॑ष॒जम॒पामु॒त प्रश॑स्ति॒ष्वश्वा॑ भवथ वाजिनः॥ वा॒युर्वा᳚ त्वा॒ मनु॑र्वा त्वा गन्ध॒र्वाः स॒प्तविꣳ॑शतिः। ते अग्रे॒ अश्व॑मायुञ्ज॒न्ते अ॑स्मिञ्ज॒वमाद॑धुः॥ अपां᳚ नपादाशुहेम॒न्॒ य ऊ॒र्मिः क॒कुद्मा॒न् प्रतू᳚र्तिर्वाज॒सात॑म॒स्तेना॒यं वाजꣳ॑ सेत्॥ विष्णोः॒ क्रमो॑\-ऽसि॒ विष्णोः᳚ क्रा॒न्तम॑सि॒ विष्णो॒र्विक्रा᳚न्तमस्य॒ङ्कौ न्य॒ङ्काव॒भितो॒ रथं॒ यौ ध्वा॒न्तं वा॑ता॒ग्रमनु॑ सं॒चर॑न्तौ दू॒रेहे॑तिरिन्द्रि॒यावा᳚न्पत॒त्री ते नो॒\-ऽग्नयः॒ पप्र॑यः पारयन्तु॥३३॥

%1.7.8.0
{\anuvakamend[{अ॒फ्सु न्य॒ङ्कौ पञ्च॑दश च}]}%॥७॥

%1.7.8.1
दे॒वस्या॒हꣳ स॑वि॒तुः प्र॑स॒वे बृह॒स्पति॑ना वाज॒जिता॒ वाजं॑ जेषं दे॒वस्या॒हꣳ स॑वि॒तुः प्र॑स॒वे बृह॒स्पति॑ना वाज॒जिता॒ वर्\mbox{}षि॑ष्ठं॒ नाकꣳ॑ रुहेय॒मिन्द्रा॑य॒ वाचं॑ वद॒तेन्द्रं॒ वाजं॑ जापय॒तेन्द्रो॒ वाज॑मजयित्। अश्वा॑जनि वाजिनि॒ वाजे॑षु वाजिनीव॒त्यश्वा᳚न्थ्स॒मथ्सु॑ वाजय॥ अर्वा॑सि॒ सप्ति॑रसि वा॒ज्य॑सि॒ वाजि॑नो॒ वाजं॑ धावत म॒रुतां᳚ प्रस॒वे ज॑यत॒ वि योज॑ना मिमीध्व॒मध्व॑नः स्कभ्नीत॥३४॥

%1.7.8.2
काष्ठां᳚ गच्छत॒ वाजे॑वाजे\-ऽवत वाजिनो नो॒ धने॑षु विप्रा अमृता ऋतज्ञाः॥ अ॒स्य मध्वः॑ पिबत मा॒दय॑ध्वं तृ॒प्ता या॑त प॒थिभि॑र्देव॒यानैः᳚॥ ते नो॒ अर्व॑न्तो हवन॒श्रुतो॒ हवं॒ विश्वे॑ शृण्वन्तु वा॒जिनः॑॥ मि॒तद्र॑वः सहस्र॒सा मे॒धसा॑ता सनि॒ष्यवः॑। म॒हो ये रत्नꣳ॑ समि॒थेषु॑ जभ्रि॒रे शं नो॑ भवन्तु वा॒जिनो॒ हवे॑षु॥ दे॒वता॑ता मि॒तद्र॑वः स्व॒र्काः। ज॒म्भय॒न्तो\-ऽहिं॒ वृक॒ꣳ॒ रक्षाꣳ॑सि॒ सने᳚म्य॒स्मद्यु॑यवन्न्॥३५॥

%1.7.8.3
अमी॑वाः॥ ए॒ष स्य वा॒जी क्षि॑प॒णिं तु॑रण्यति ग्री॒वायां᳚ ब॒द्धो अ॑पिक॒क्ष आ॒सनि॑। क्रतुं॑ दधि॒क्रा अनु॑ स॒न्तवी᳚त्वत् प॒थामङ्का॒ꣳ॒स्यन्वा॒पनी॑फणत्॥ उ॒त स्मा᳚स्य॒ द्रव॑तस्तुरण्य॒तः प॒र्णं न वेरनु॑ वाति प्रग॒र्धिनः॑। श्ये॒नस्ये॑व॒ ध्रज॑तो अङ्क॒सं परि॑ दधि॒क्राव्ण्णः॑ स॒होर्जा तरि॑त्रतः॥ आ मा॒ वाज॑स्य प्रस॒वो ज॑गम्या॒दा द्यावा॑पृथि॒वी वि॒श्वश॑म्भू। आ मा॑ गन्तां पि॒तरा᳚॥३६॥

%1.7.8.4
मा॒तरा॒ चा\-ऽ\-ऽमा॒ सोमो॑ अमृत॒त्वाय॑ गम्यात्॥ वाजि॑नो वाजजितो॒ वाजꣳ॑ सरि॒ष्यन्तो॒ वाजं॑ जे॒ष्यन्तो॒ बृह॒स्पते᳚र्भा॒गमव॑ जिघ्रत॒ वाजि॑नो वाजजितो॒ वाजꣳ॑ ससृ॒वाꣳसो॒ वाजं॑ जिगि॒वँꣳसो॒ बृह॒स्पते᳚र्भा॒गे नि मृ॑ढ्वमि॒यं वः॒ सा स॒त्या सं॒धा\-ऽभू॒द्यामिन्द्रे॑ण स॒मध॑ध्व॒मजी॑जिपत वनस्पतय॒ इन्द्रं॒ वाजं॒ विमु॑च्यध्वम्॥३७॥

%1.7.9.0
{\anuvakamend[{स्क॒भ्नी॒त॒ यु॒य॒व॒न्पि॒तरा॒ द्विच॑त्वारिꣳशच्च}]}%॥८॥

%1.7.9.1
क्ष॒त्रस्योल्ब॑मसि क्ष॒त्रस्य॒ योनि॑रसि॒ जाय॒ एहि॒ सुवो॒ रोहा॑व॒ रोहा॑व॒ हि सुव॑र॒हं ना॑वु॒भयोः॒ सुवो॑ रोक्ष्यामि॒ वाज॑श्च प्रस॒वश्चा॑पि॒जश्च॒ क्रतु॑श्च॒ सुव॑श्च मू॒र्धा च॒ व्यश्ञि॑यश्चा\-ऽ\-ऽन्त्याय॒नश्चान्त्य॑श्च भौव॒नश्च॒ भुव॑न॒श्चाधि॑पतिश्च। आयु॑र्य॒ज्ञेन॑ कल्पतां प्रा॒णो य॒ज्ञेन॑ कल्पतामपा॒नः॥३८॥

%1.7.9.2
य॒ज्ञेन॑ कल्पतां व्या॒नो य॒ज्ञेन॑ कल्पतां॒ चक्षु॑र्य॒ज्ञेन॑ कल्पता॒ꣴ॒ श्रोत्रं॑ य॒ज्ञेन॑ कल्पतां॒ मनो॑ य॒ज्ञेन॑ कल्पतां॒ वाग्य॒ज्ञेन॑ कल्पतामा॒त्मा य॒ज्ञेन॑ कल्पतां य॒ज्ञो य॒ज्ञेन॑ कल्पता॒ꣳ॒ सुव॑र्दे॒वाꣳ अ॑गन्मा॒मृता॑ अभूम प्र॒जाप॑तेः प्र॒जा अ॑भूम॒ सम॒हं प्र॒जया॒ सं मया᳚ प्र॒जा सम॒हꣳ रा॒यस्पोषे॑ण॒ सं मया॑ रा॒यस्पोषो\-ऽन्ना॑य त्वा॒\-ऽन्नाद्या॑य त्वा॒ वाजा॑य त्वा वाजजि॒त्यायै᳚ त्वा॒\-ऽमृत॑मसि॒ पुष्टि॑रसि प्र॒जन॑नमसि॥३९॥

%1.7.10.0
{\anuvakamend[{अ॒पा॒नो वाजा॑य॒ नव॑ च}]}%॥९॥

%1.7.10.1
वाज॑स्ये॒मं प्र॑स॒वः सु॑षुवे॒ अग्रे॒ सोम॒ꣳ॒ राजा॑न॒मोष॑धीष्व॒फ्सु। ता अ॒स्मभ्यं॒ मधु॑मतीर्भवन्तु व॒यꣳ रा॒ष्ट्रे जा᳚ग्रियाम पु॒रोहि॑ताः॥ वाज॑स्ये॒दं प्र॑स॒व आ ब॑भूवे॒मा च॒ विश्वा॒ भुव॑नानि स॒र्वतः॑। स वि॒राजं॒ पर्ये॑ति प्रजा॒नन् प्र॒जां पुष्टिं॑ व॒र्धय॑मानो अ॒स्मे॥ वाज॑स्ये॒मां प्र॑स॒वः शि॑श्रिये॒ दिव॑मि॒मा च॒ विश्वा॒ भुव॑नानि स॒म्राट्। अदि॑थ्सन्तं दापयतु प्रजा॒नन् र॒यिं॥४०॥

%1.7.10.2
च॒ नः॒ सर्व॑वीरां॒ नि य॑च्छतु॥ अग्ने॒ अच्छा॑ वदे॒ह नः॒ प्रति॑ नः सु॒मना॑ भव। प्र णो॑ यच्छ भुवस्पते धन॒दा अ॑सि न॒स्त्वम्॥ प्र णो॑ यच्छत्वर्य॒मा प्र भगः॒ प्र बृह॒स्पतिः॑। प्र दे॒वाः प्रोत सू॒नृता॒ प्र वाग्दे॒वी द॑दातु नः॥ अ॒र्य॒मणं॒ बृह॒स्पति॒मिन्द्रं॒ दाना॑य चोदय। वाचं॒ विष्णु॒ꣳ॒ सर॑स्वतीꣳ सवि॒तारम्᳚॥४१॥

%1.7.10.3
च॒ वा॒जिनम्᳚॥ सोम॒ꣳ॒ राजा॑नं॒ वरु॑णम॒ग्निम॒न्वार॑भामहे। आ॒दि॒त्यान् विष्णु॒ꣳ॒ सूर्यं॑ ब्र॒ह्माणं॑ च॒ बृह॒स्पतिम्᳚॥ दे॒वस्य॑ त्वा सवि॒तुः प्र॑स॒वे᳚\-ऽश्विनो᳚र्बा॒हु\-भ्यां᳚ पू॒ष्णो हस्ता᳚भ्या॒ꣳ॒ सर॑स्वत्यै वा॒चो य॒न्तुर्य॒न्त्रेणा॒ग्नेस्त्वा॒ साम्रा᳚ज्येना॒भिषि॑ञ्चा॒मीन्द्र॑स्य॒ बृह॒स्पते᳚स्त्वा॒ साम्रा᳚ज्येना॒भिषि॑ञ्चामि॥४२॥

%1.7.11.0
{\anuvakamend[{र॒यिꣳ स॑वि॒तार॒ꣳ॒ षट्त्रिꣳ॑शच्च}]}%॥10॥

%1.7.11.1
अ॒ग्निरेका᳚क्षरेण॒ वाच॒मुद॑जयद॒श्विनौ॒ द्व्य॑क्षरेण प्राणापा॒नावुद॑जयतां॒ विष्णु॒स्त्र्य॑क्षरेण॒ त्रील्लोँ॒कानुद॑जय॒थ्सोम॒श्चतु॑रक्षरेण॒ चतु॑ष्पदः प॒शूनुद॑जयत् पू॒षा पञ्चा᳚क्षरेण प॒ङ्क्तिमुद॑जयद्धा॒ता षड॑क्षरेण॒ षडृ॒तूनुद॑जयन्म॒रुतः॑ स॒प्ताक्ष॑रेण स॒प्तप॑दा॒ꣳ॒ शक्व॑री॒मुद॑जय॒न् बृह॒स्पति॑र॒ष्टाक्ष॑रेण गाय॒त्रीमुद॑जयन्मि॒त्रो नवा᳚क्षरेण त्रि॒वृत॒ꣴ॒ स्तोम॒मुद॑जयत्॥४३॥

%1.7.11.2
वरु॑णो॒ दशा᳚क्षरेण वि॒राज॒मुद॑जय॒दिन्द्र॒ एका॑दशाक्षरेण त्रि॒ष्टुभ॒मुद॑जय॒द् विश्वे॑ दे॒वा द्वाद॑शाक्षरेण॒ जग॑ती॒मुद॑जय॒न् वस॑व॒स्त्रयो॑दशाक्षरेण त्रयोद॒शꣴस्तोम॒मुद॑जयन् रु॒द्राश्चतु॑र्दशाक्षरेण चतुर्द॒शꣴ स्तोम॒मुद॑जयन्नादि॒त्याः पञ्च॑दशाक्षरेण पञ्चद॒शꣴ स्तोम॒मुद॑जय॒न्नदि॑ति॒ष्षोड॑शाक्षरेण षोड॒शꣴ स्तोम॒मुद॑जयत् प्र॒जाप॑तिः स॒प्तद॑शाक्षरेण सप्तद॒शꣴ स्तोम॒मुद॑जयत्॥४४॥

%1.7.12.0
{\anuvakamend[{अ॒ज॒य॒त् षट्च॑त्वारिꣳशच्च}]}%॥11॥

%1.7.12.1
उ॒प॒या॒मगृ॑हीतो\-ऽसि नृ॒षदं॑ त्वा द्रु॒षदं॑ भुवन॒सद॒मिन्द्रा॑य॒ जुष्टं॑ गृह्णाम्ये॒ष ते॒ योनि॒रिन्द्रा॑य त्वोपया॒मगृ॑हीतो\-ऽस्यफ्सु॒षदं॑ त्वा घृत॒सदं॑ व्योम॒सद॒मिन्द्रा॑य॒ जुष्टं॑ गृह्णाम्ये॒ष ते॒ योनि॒रिन्द्रा॑य त्वोपया॒मगृ॑हीतो\-ऽसि पृथिवि॒षदं॑ त्वा\-ऽन्तरिक्ष॒सदं॑ नाक॒सद॒मिन्द्रा॑य॒ जुष्टं॑ गृह्णाम्ये॒ष ते॒ योनि॒रिन्द्रा॑य त्वा॥ ये ग्रहाः᳚ पञ्चज॒नीना॒ येषां᳚ ति॒स्रः प॑रम॒जाः। दैव्यः॒ कोशः॑॥४५॥

%1.7.12.2
समु॑ब्जितः। तेषां॒ विशि॑प्रियाणा॒मिष॒मूर्ज॒ꣳ॒ सम॑ग्रभीमे॒ष ते॒ योनि॒रिन्द्रा॑य त्वा॥ अ॒पाꣳ रस॒मुद्व॑यस॒ꣳ॒ सूर्य॑रश्मिꣳ स॒माभृ॑तम्। अ॒पाꣳ रस॑स्य॒ यो रस॒स्तं वो॑ गृह्णाम्युत्त॒ममे॒ष ते॒ योनि॒रिन्द्रा॑य त्वा॥ अ॒या वि॒ष्ठा ज॒नय॒न्कर्व॑राणि॒ स हि घृणि॑रु॒रुर्वरा॑य गा॒तुः। स प्रत्युदै᳚द्ध॒रुणो॒ मध्वो॒ अग्र॒ꣴ॒ स्वायां॒ यत्त॒नुवां᳚ त॒नूमैर॑यत। उ॒प॒या॒मगृ॑हीतो\-ऽसि प्र॒जाप॑तये त्वा॒ जुष्टं॑ गृह्णाम्ये॒ष ते॒ योनिः॑ प्र॒जाप॑तये त्वा॥४६॥

%1.7.13.0
{\anuvakamend[{कोश॑स्त॒नुवां॒ त्रयो॑दश च}]}%॥12॥

%1.7.13.1
अन्वह॒ मासा॒ अन्विद्वना॒न्यन्वोष॑धी॒रनु॒ पर्व॑तासः। अन्विन्द्र॒ꣳ॒ रोद॑सी वावशा॒ने अन्वापो॑ अजिहत॒ जाय॑मानम्॥ अनु॑ ते दायि म॒ह इ॑न्द्रि॒याय॑ स॒त्रा ते॒ विश्व॒मनु॑ वृत्र॒हत्ये᳚। अनु॒ क्ष॒त्रमनु॒ सहो॑ यज॒त्रेन्द्र॑ दे॒वेभि॒रनु॑ ते नृ॒षह्ये᳚॥ इ॒न्द्रा॒णीमा॒सु नारि॑षु सु॒पत्नी॑म॒हम॑श्रवम्। न ह्य॒स्या अप॒रं च॒न ज॒रसा᳚॥४७॥

%1.7.13.2
मर॑ते॒ पतिः॑॥ नाहमि॑न्द्राणि रारण॒ सख्यु॑र्वृ॒षाक॑पेर् ऋ॒ते। यस्ये॒दमप्यꣳ॑ ह॒विः प्रि॒यं दे॒वेषु॒ गच्छ॑ति॥ यो जा॒त ए॒व प्र॑थ॒मो मन॑स्वान् दे॒वो दे॒वान् क्रतु॑ना प॒र्यभू॑षत्। यस्य॒ शुष्मा॒द्रोद॑सी॒ अभ्य॑सेतां नृ॒म्णस्य॑ म॒ह्ना स ज॑नास॒ इन्द्रः॑॥ आ ते॑ म॒ह इ॑न्द्रो॒त्यु॑ग्र॒ सम॑न्यवो॒ यथ्स॒मर॑न्त॒ सेनाः᳚। पता॑ति दि॒द्युन्नर्य॑स्य बाहु॒वोर्मा ते᳚॥४८॥

%1.7.13.3
मनो॑ विष्व॒द्रिय॒ग्विचा॑रीत्॥ मा नो॑ मर्धी॒रा भ॑रा द॒द्धि तन्नः॒ प्र दा॒शुषे॒ दात॑वे॒ भूरि॒ यत् ते᳚। नव्ये॑ दे॒ष्णे श॒स्ते अ॒स्मिन् त॑ उ॒क्थे प्र ब्र॑वाम व॒यमि॑न्द्र स्तु॒वन्तः॑॥ आ तू भ॑र॒ माकि॑रे॒तत् परि॑ष्ठाद्वि॒द्मा हि त्वा॒ वसु॑पतिं॒ वसू॑नाम्। इन्द्र॒ यत् ते॒ माहि॑नं॒ दत्र॒मस्त्य॒स्मभ्यं॒ तद्ध॑र्यश्व॥४९॥

%1.7.13.4
प्र य॑न्धि॥ प्र॒दा॒तारꣳ॑ हवामह॒ इन्द्र॒मा ह॒विषा॑ व॒यम्। उ॒भा हि हस्ता॒ वसु॑ना पृ॒णस्वा\-ऽ\-ऽप्र य॑च्छ॒ दक्षि॑णा॒दोत स॒व्यात्॥ प्र॒दा॒ता व॒ज्री वृ॑ष॒भस्तु॑रा॒षाट्छु॒ष्मी राजा॑ वृत्र॒हा सो॑म॒पावा᳚। अ॒स्मिन् य॒ज्ञे ब॒र्\mbox{}हिष्या नि॒षद्याथा॑ भव॒ यज॑मानाय॒ शं योः॥ इन्द्रः॑ सु॒त्रामा॒ स्ववा॒ꣳ॒ अवो॑भिः सुमृडी॒को भ॑वतु वि॒श्ववे॑दाः। बाध॑तां॒ द्वेषो॒ अभ॑यं कृणोतु सु॒वीर्य॑स्य॥५०॥

%1.7.13.5
पत॑यः स्याम॥ तस्य॑ व॒यꣳ सु॑म॒तौ य॒ज्ञिय॒स्यापि॑ भ॒द्रे सौ॑मन॒से स्या॑म। स सु॒त्रामा॒ स्ववा॒ꣳ॒ इन्द्रो॑ अ॒स्मे आ॒राच्चि॒द्द्वेषः॑ सनु॒तर्यु॑योतु॥ रे॒वती᳚र्नः सध॒माद॒ इन्द्रे॑ सन्तु तु॒विवा॑जाः। क्षु॒मन्तो॒ याभि॒र्मदे॑म॥ प्रो ष्व॑स्मै पुरोर॒थमिन्द्रा॑य शू॒षम॑र्चत। अ॒भीके॑ चिदु लोक॒कृथ्स॒ङ्गे स॒मथ्सु॑ वृत्र॒हा। अ॒स्माकं॑ बोधि चोदि॒ता नभ॑न्तामन्य॒केषा᳚म्। ज्या॒का अधि॒ धन्व॑सु॥५१॥

{\anuvakamend[{ज॒रसा॒ मा ते॑ हर्यश्व सु॒वीर्य॒स्याध्येकं॑ च}]}%॥13॥
%%% END PRASHNA

\sect{अष्टमः प्रश्नः}\setcounter{anuvakam}{0}
\dnsub{तैत्तिरीयसंहितायां प्रथमकाण्डे अष्टमः प्रश्नः}
%1.8.1.0
%1.8.1.1
अनु॑मत्यै पुरो॒डाश॑म॒ष्टाक॑पालं॒ निर्व॑पति धे॒नुर्दक्षि॑णा॒ ये प्र॒त्यञ्चः॒ शम्या॑या अव॒शीय॑न्ते॒ तन्नैर्॑ऋ॒तमेक॑कपालं कृ॒ष्णं वासः॑ कृ॒ष्णतू॑षं॒ दक्षि॑णा॒ वीहि॒ स्वाहा\-ऽ\-ऽहु॑तिं जुषा॒ण ए॒ष ते॑ निर्\mbox{}ऋते भा॒गो भूते॑ ह॒विष्म॑त्यसि मु॒ञ्चेममꣳह॑सः॒ स्वाहा॒ नमो॒ य इ॒दं च॒कारा॑\-ऽ\-ऽदि॒त्यं च॒रुं निर्व॑पति॒ वरो॒ दक्षि॑णा\-ऽ\-ऽग्नावैष्ण॒वमेका॑दशकपालं वाम॒नो व॒ही दक्षि॑णा\-ऽग्नीषो॒मीयम्᳚॥१॥

%1.8.1.2
एका॑दशकपाल॒ꣳ॒ हिर॑ण्यं॒ दक्षि॑णै॒न्द्रमेका॑दशकपालमृष॒भो व॒ही दक्षि॑णा\-ऽ\-ऽग्ने॒यम॒ष्टाक॑पालमै॒न्द्रं दध्यृ॑ष॒भो व॒ही दक्षि॑णैन्द्रा॒ग्नं द्वाद॑शकपालं वैश्वदे॒वं च॒रुं प्र॑थम॒जो व॒थ्सो दक्षि॑णा सौ॒म्यꣴ श्या॑मा॒कं च॒रुं वासो॒ दक्षि॑णा॒ सर॑स्वत्यै च॒रुꣳ सर॑स्वते च॒रुं मि॑थु॒नौ गावौ॒ दक्षि॑णा॥२॥

%1.8.2.0
{\anuvakamend[{अ॒ग्नी॒षो॒मीयं॒ चतु॑स्त्रिꣳशच्च}]}%॥१॥

%1.8.2.1
आ॒ग्ने॒यम॒ष्टाक॑पालं॒ निर्व॑पति सौ॒म्यं च॒रुꣳ सा॑वि॒त्रं द्वाद॑शकपालꣳ सारस्व॒तं च॒रुं पौ॒ष्णं च॒रुं मा॑रु॒तꣳ स॒प्तक॑पालं वैश्वदे॒वीमा॒मिक्षां᳚ द्यावापृथि॒व्य॑मेक॑कपालम्॥३॥

%1.8.3.0
{\anuvakamend[{आ॒ग्ने॒यम॒ष्टाद॑श}]}%॥२॥

%1.8.3.1
ऐ॒न्द्रा॒ग्नमेका॑दशकपालं मारु॒तीमा॒मिक्षां᳚ वारु॒णीमा॒मिक्षां᳚ का॒यमेक॑कपालं प्रघा॒स्यान्॑ हवामहे म॒रुतो॑ य॒ज्ञवा॑हसः कर॒म्भेण॑ स॒जोष॑सः॥ मो षू ण॑ इन्द्र पृ॒थ्सु दे॒वास्तु॑ स्म ते शुष्मिन्नव॒या। म॒ही ह्य॑स्य मी॒ढुषो॑ य॒व्या। ह॒विष्म॑तो म॒रुतो॒ वन्द॑ते॒ गीः॥ यद् ग्रामे॒ यदर॑ण्ये॒ यथ्स॒भायां॒ यदि॑न्द्रि॒ये। यच्छू॒द्रे यद॒र्य॑ एन॑श्चकृ॒मा व॒यम्। यदेक॒स्याधि॒ धर्म॑णि॒ तस्या॑व॒यज॑नमसि॒ स्वाहा᳚॥ अक्र॒न् कर्म॑ कर्म॒कृतः॑ स॒ह वा॒चा म॑योभु॒वा॥ दे॒वेभ्यः॒ कर्म॑ कृ॒त्वा\-ऽस्तं॒ प्रेत॑ सुदानवः॥४॥

%1.8.4.0
{\anuvakamend[{व॒यं यद् विꣳ॑श॒तिश्च॑}]}%॥३॥

%1.8.4.1
अ॒ग्नये\-ऽनी॑कवते पुरो॒डाश॑म॒ष्टाक॑पालं॒ निर्व॑पति सा॒कꣳ सूर्ये॑णोद्य॒ता म॒रुद्भ्यः॑ सान्तप॒नेभ्यो॑ म॒ध्यन्दि॑ने च॒रुं म॒रुद्भ्यो॑ गृहमे॒धिभ्यः॒ सर्वा॑सां दु॒ग्धे सा॒यं च॒रुं पू॒र्णा द॑र्वि॒ परा॑ पत॒ सुपू᳚र्णा॒ पुन॒राप॑त। व॒स्नेव॒ वि क्री॑णावहा॒ इष॒मूर्जꣳ॑ शतक्रतो॥ दे॒हि मे॒ ददा॑मि ते॒ नि मे॑ धेहि॒ नि ते॑ दधे। नि॒हार॒मिन्नि मे॑ हरा नि॒हारम्᳚॥५॥


%1.8.4.2
नि ह॑रामि ते॥ म॒रुद्भ्यः॑ क्री॒डिभ्यः॑ पुरो॒डाशꣳ॑ स॒प्तक॑पालं॒ निर्व॑पति सा॒कꣳ सूर्ये॑णोद्य॒ताग्ने॒यम॒ष्टाक॑पालं॒ निर्व॑पति सौ॒म्यं च॒रुꣳ सा॑वि॒त्रं द्वाद॑शकपालꣳ सारस्व॒तं च॒रुं पौ॒ष्णं च॒रुमै᳚न्द्रा॒ग्नमेका॑दशकपालमै॒न्द्रं च॒रुं वै᳚श्वकर्म॒णमेक॑कपालम्॥६॥

%1.8.5.0
{\anuvakamend[{ह॒रा॒ नि॒हारं॑ त्रि॒ꣳ॒शच्च॑}]}%॥४॥

%1.8.5.1
सोमा॑य पितृ॒मते॑ पुरो॒डाश॒ꣳ॒ षट्क॑पालं॒ निर्व॑पति पि॒तृभ्यो॑ बर्\mbox{}हि॒षद्भ्यो॑ धा॒नाः पि॒तृभ्यो᳚\-ऽग्निष्वा॒त्तेभ्यो॑\-ऽभिवा॒न्या॑यै दु॒ग्धे म॒न्थमे॒तत् ते॑ तत॒ ये च॒ त्वामन्वे॒तत् ते॑ पितामह प्रपितामह॒ ये च॒ त्वामन्वत्र॑ पितरो यथाभा॒गं म॑न्दध्वꣳ सुसं॒दृशं॑ त्वा व॒यं मघ॑वन् मन्दिषी॒महि॑॥ प्र नू॒नं पू॒र्णव॑न्धुरः स्तु॒तो या॑सि॒ वशा॒ꣳ॒ अनु॑॥ योजा॒ न्वि॑न्द्र ते॒ हरी᳚॥७॥

%1.8.5.2
अक्ष॒न्नमी॑मदन्त॒ ह्यव॑ प्रि॒या अ॑धूषत॥ अस्तो॑षत॒ स्वभा॑नवो॒ विप्रा॒ नवि॑ष्ठया म॒ती॥ योजा॒ न्वि॑न्द्र ते॒ हरी᳚॥ अक्ष॑न् पि॒तरो\-ऽमी॑मदन्त पि॒तरो\-ऽती॑तृपन्त पि॒तरो\-ऽमी॑मृजन्त पि॒तरः॑॥ परे॑त पितरः सोम्या गम्भी॒रैः प॒थिभिः॑ पू॒र्व्यैः॥ अथा॑ पि॒तॄन्थ्सु॑वि॒दत्रा॒ꣳ॒ अपी॑त य॒मेन॒ ये स॑ध॒मादं॒ मद॑न्ति॥ मनो॒ न्वा हु॑वामहे नाराश॒ꣳ॒सेन॒ स्तोमे॑न पितृ॒णां च॒ मन्म॑भिः॥ आ॥८॥

%1.8.5.3
न॒ ए॒तु॒ मनः॒ पुनः॒ क्रत्वे॒ दक्षा॑य जी॒वसे᳚॥ ज्योक् च॒ सूर्यं॑ दृ॒शे॥ पुन॑र्नः पि॒तरो॒ मनो॒ ददा॑तु॒ दैव्यो॒ जनः॑॥ जी॒वं व्रातꣳ॑ सचेमहि॥ यद॒न्तरि॑क्षं पृथि॒वीमु॒त द्यां यन्मा॒तरं॑ पि॒तरं॑ वा जिहिꣳसि॒म॥ अ॒ग्निर्मा॒ तस्मा॒देन॑सो॒ गार्\mbox{}ह॑पत्यः॒ प्र मु॑ञ्चतु दुरि॒ता यानि॑ चकृ॒म क॒रोतु॒ माम॑ने॒नसम्᳚॥९॥

%1.8.6.0
{\anuvakamend[{हरी॒ मन्म॑भि॒रा चतु॑श्चत्वारिꣳशच्च}]}%॥५॥

%1.8.6.1
प्र॒ति॒पू॒रु॒षमेक॑कपाला॒न्निर्व॑प॒त्येक॒मति॑रिक्तं॒ याव॑न्तो गृ॒ह्याः᳚ स्मस्तेभ्यः॒ कम॑करं पशू॒नाꣳ शर्मा॑सि॒ शर्म॒ यज॑मानस्य॒ शर्म॑ मे य॒च्छैक॑ ए॒व रु॒द्रो न द्वि॒तीया॑य तस्थ आ॒खुस्ते॑ रुद्र प॒शुस्तं जु॑षस्वै॒ष ते॑ रुद्र भा॒गः स॒ह स्वस्रा\-ऽम्बि॑कया॒ तं जु॑षस्व भेष॒जं गवे\-ऽश्वा॑य॒ पुरु॑षाय भेष॒जमथो॑ अ॒स्मभ्यं॑ भेष॒जꣳ सुभे॑षजम्॥१०॥

%1.8.6.2
यथा\-ऽस॑ति॥ सु॒गं मे॒षाय॑ मे॒ष्या॑ अवा᳚म्ब रु॒द्रम॑दिम॒ह्यव॑ दे॒वं त्र्य॑म्बकम्॥ यथा॑ नः॒ श्रेय॑सः॒ कर॒द्यथा॑ नो॒ वस्य॑सः॒ कर॒द्यथा॑ नः पशु॒मतः॒ कर॒द्यथा॑ नो व्यवसा॒यया᳚त्॥ त्र्य॑म्बकं यजामहे सुग॒न्धिं पु॑ष्टि॒वर्ध॑नम्॥ उ॒र्वा॒रु॒कमि॑व॒ बन्ध॑नान्मृ॒त्योर्मु॑क्षीय॒ मा\-ऽमृता᳚त्॥ ए॒ष ते॑ रुद्र भा॒गस्तं जु॑षस्व॒ तेना॑व॒सेन॑ प॒रो मूज॑व॒तो\-ऽती॒ह्यव॑ततधन्वा॒ पिना॑कहस्तः॒ कृत्ति॑वासाः॥११॥

%1.8.7.0
{\anuvakamend[{सुभे॑षजमिहि॒ त्रीणि॑ च}]}%॥६॥

%1.8.7.1
ऐ॒न्द्रा॒ग्नं द्वाद॑शकपालं वैश्वदे॒वं च॒रुमिन्द्रा॑य॒ शुना॒सीरा॑य पुरो॒डाशं॒ द्वाद॑शकपालं वाय॒व्यं॑ पयः॑ सौ॒र्यमेक॑कपालं द्वादशग॒वꣳ सीरं॒ दक्षि॑णा\-ऽ\-ऽग्ने॒यम॒ष्टाक॑पालं॒ निर्व॑पति रौ॒द्रं गा॑वीधु॒कं च॒रुमै॒न्द्रं दधि॑ वारु॒णं य॑व॒मयं॑ च॒रुं व॒हिनी॑ धे॒नुर्दक्षि॑णा॒ ये दे॒वाः पु॑रः॒सदो॒\-ऽग्निने᳚त्रा दक्षिण॒सदो॑ य॒मने᳚त्राः पश्चा॒थ्सदः॑ सवि॒तृने᳚त्रा उत्तर॒सदो॒ वरु॑णनेत्रा उपरि॒षदो॒ बृह॒स्पति॑नेत्रा रक्षो॒हण॒स्ते नः॑ पान्तु॒ ते नो॑\-ऽवन्तु॒ तेभ्यः॑॥१२॥

%1.8.7.2
नम॒स्तेभ्यः॒ स्वाहा॒ समू॑ढ॒ꣳ॒ रक्षः॒ सन्द॑ग्ध॒ꣳ॒ रक्ष॑ इ॒दम॒हꣳ रक्षो॒\-ऽभि सं द॑हाम्य॒ग्नये॑ रक्षो॒घ्ने स्वाहा॑ य॒माय॑ सवि॒त्रे वरु॑णाय॒ बृह॒स्पत॑ये॒ दुव॑स्वते रक्षो॒घ्ने स्वाहा᳚ प्रष्टिवा॒ही रथो॒ दक्षि॑णा दे॒वस्य॑ त्वा सवि॒तुः प्र॑स॒वे᳚\-ऽश्विनो᳚र्बा॒हु\-भ्यां᳚ पू॒ष्णो हस्ता᳚भ्या॒ꣳ॒ रक्ष॑सो व॒धं जु॑होमि ह॒तꣳ रक्षो\-ऽव॑धिष्म॒ रक्षो॒ यद्वस्ते॒ तद्दक्षि॑णा॥१३॥


%1.8.8.0
{\anuvakamend[{तेभ्यः॒ पञ्च॑चत्वारिꣳशच्च}]}%॥७॥

%1.8.8.1
धा॒त्रे पु॑रो॒डाशं॒ द्वाद॑शकपालं॒ निर्व॑प॒त्यनु॑मत्यै च॒रुꣳ रा॒कायै॑ च॒रुꣳ सि॑नीवा॒ल्यै च॒रुं कु॒ह्वै॑ च॒रुं मि॑थु॒नौ गावौ॒ दक्षि॑णा\-ऽ\-ऽग्नावैष्णव॒मेका॑दशकपालं॒ निर्व॑पत्यैन्द्रावैष्ण॒वमेका॑दशकपालं वैष्ण॒वं त्रि॑कपा॒लं वा॑म॒नो व॒ही दक्षि॑णा\-ऽग्नीषो॒मीय॒मेका॑दशकपालं॒ निर्व॑पतीन्द्रासो॒मीय॒मेका॑दशकपालꣳ सौ॒म्यं च॒रुं ब॒भ्रुर्दक्षि॑णा सोमापौ॒ष्णं च॒रुं निर्व॑पत्यैन्द्रापौ॒ष्णं च॒रुं पौ॒ष्णं च॒रुꣴ श्या॒मो दक्षि॑णा वैश्वान॒रं द्वाद॑शकपालं॒ निर्व॑पति॒ हिर॑ण्यं॒ दक्षि॑णा वारु॒णं य॑व॒मयं॑ च॒रुमश्वो॒ दक्षि॑णा॥१४॥

%1.8.9.0
{\anuvakamend[{निर॒ष्टौ च}]}%॥८॥

%1.8.9.1
बा॒र्॒\mbox{}ह॒स्प॒त्यं च॒रुं निर्व॑पति ब्र॒ह्मणो॑ गृ॒हे शि॑तिपृ॒ष्ठो दक्षि॑णै॒न्द्रमेका॑दशकपालꣳ राज॒न्य॑स्य गृ॒ह ऋ॑ष॒भो दक्षि॑णा\-ऽ\-ऽदि॒त्यं च॒रुं महि॑ष्यै गृ॒हे धे॒नुर्दक्षि॑णा नैर्\mbox{}ऋ॒तं च॒रुं प॑रिवृ॒क्त्यै॑ गृ॒हे कृ॒ष्णानां᳚ व्रीही॒णां न॒खनि॑र्भिन्नं कृ॒ष्णा कू॒टा दक्षि॑णा\-ऽ\-ऽग्ने॒यम॒ष्टाक॑पालꣳ सेना॒न्यो॑ गृ॒हे हिर॑ण्यं॒ दक्षि॑णा वारु॒णं दश॑कपालꣳ सू॒तस्य॑ गृ॒हे म॒हानि॑रष्टो॒ दक्षि॑णा मारु॒तꣳ स॒प्तक॑पालं ग्राम॒ण्यो॑ गृ॒हे पृश्ञि॒र्दक्षि॑णा सावि॒त्रं द्वाद॑शकपालं॥१५॥

%1.8.9.2
क्ष॒त्तुर्गृ॒ह उ॑पध्व॒स्तो दक्षि॑णा\-ऽ\-ऽश्वि॒नं द्वि॑कपा॒लꣳ सं॑ग्रही॒तुर्गृ॒हे स॑वा॒त्यौ॑ दक्षि॑णा पौ॒ष्णं च॒रुं भा॑गदु॒घस्य॑ गृ॒हे श्या॒मो दक्षि॑णा रौ॒द्रं गा॑वीधु॒कं च॒रुम॑क्षावा॒पस्य॑ गृ॒हे श॒बल॒ उद्वा॑रो॒ दक्षि॒णेन्द्रा॑य सु॒त्राम्णे॑ पुरो॒डाश॒मेका॑दशकपालं॒ प्रति॒ निर्व॑प॒तीन्द्रा॑याꣳहो॒मुचे॒\-ऽयं नो॒ राजा॑ वृत्र॒हा राजा॑ भू॒त्वा वृ॒त्रं व॑ध्यान्मैत्राबार्\mbox{}हस्प॒त्यं भ॑वति श्वे॒तायै᳚ श्वे॒तव॑थ्सायै दु॒ग्धे स्व॑यम्मू॒र्ते स्व॑यम्मथि॒त आज्य॒ आश्व॑त्थे॥१६॥

%1.8.9.3
पात्रे॒ चतुः॑स्रक्तौ स्वयमवप॒न्नायै॒ शाखा॑यै क॒र्णाꣴश्चाक॑र्णाꣴश्च तण्डु॒लान् वि चि॑नुया॒द्ये क॒र्णाः स पय॑सि बार्\mbox{}हस्प॒त्यो ये\-ऽक॑र्णाः॒ स आज्ये॑ मै॒त्रः स्व॑यं कृ॒ता वेदि॑र्भवति स्वयन्दि॒नं ब॒र्\mbox{}हिः स्व॑यं कृ॒त इ॒ध्मः सैव श्वे॒ता श्वे॒तव॑थ्सा॒ दक्षि॑णा॥१७॥


%1.8.10.0
{\anuvakamend[{सा॒वि॒त्रं द्वाद॑शकपाल॒माश्व॑त्थे॒ त्रय॑स्त्रिꣳशच्च}]}%॥९॥

%1.8.10.1
अ॒ग्नये॑ गृ॒हप॑तये पुरो॒डाश॑म॒ष्टाक॑पालं॒ निर्व॑पति कृ॒ष्णानां᳚ व्रीही॒णाꣳ सोमा॑य॒ वन॒स्पत॑ये श्यामा॒कं च॒रुꣳ स॑वि॒त्रे स॒त्यप्र॑सवाय पुरो॒डाशं॒ द्वाद॑शकपालमाशू॒नां व्री॑ही॒णाꣳ रु॒द्राय॑ पशु॒पत॑ये गावीधु॒कं च॒रुं बृह॒स्पत॑ये वा॒चस्पत॑ये नैवा॒रं च॒रुमिन्द्रा॑य ज्ये॒ष्ठाय॑ पुरो॒डाश॒मेका॑दशकपालं म॒हाव्री॑हीणां मि॒त्राय॑ स॒त्याया॒\-ऽ\-ऽम्बानां᳚ च॒रुं वरु॑णाय॒ धर्म॑पतये यव॒मयं॑ च॒रुꣳ स॑वि॒ता त्वा᳚ प्रस॒वानाꣳ॑ सुवताम॒ग्निर्गृ॒हप॑तीना॒ꣳ॒ सोमो॒ वन॒स्पती॑नाꣳ रु॒द्रः प॑शू॒नां॥१८॥

%1.8.10.2
बृह॒स्पति॑र्वा॒चामिन्द्रो᳚ ज्ये॒ष्ठानां᳚ मि॒त्रः स॒त्यानां॒ वरु॑णो॒ धर्म॑पतीनां॒ ये दे॑वा देव॒सुवः॒ स्थ त इ॒ममा॑मुष्याय॒णम॑\-नमि॒त्राय॑ सुवध्वं मह॒ते क्ष॒त्राय॑ मह॒त आधि॑पत्याय मह॒ते जान॑राज्यायै॒ष वो॑ भरता॒ राजा॒ सोमो॒\-ऽस्माकं॑ ब्राह्म॒णाना॒ꣳ॒ राजा॒ प्रति॒ त्यन्नाम॑ रा॒ज्यम॑धायि॒ स्वां त॒नुवं॒ वरु॑णो अशिश्रे॒च्छुचे᳚र्मि॒त्रस्य॒ व्रत्या॑ अभू॒माम॑न्महि मह॒त ऋ॒तस्य॒ नाम॒ सर्वे॒ व्राता॒ वरु॑णस्याभूव॒न्वि मि॒त्र एवै॒ररा॑तिमतारी॒दसू॑षुदन्त य॒ज्ञिया॑ ऋ॒तेन॒ व्यु॑ त्रि॒तो ज॑रि॒माणं॑ न आन॒ड् विष्णोः॒ क्रमो॑\-ऽसि॒ विष्णोः᳚ क्रा॒न्तम॑सि॒ विष्णो॒र्विक्रा᳚न्तमसि॥१९॥

%1.8.11.0
{\anuvakamend[{प॒शू॒नां व्राताः॒ पञ्च॑विꣳशतिश्च}]}%॥10॥

%1.8.11.1
अ॒र्थेतः॑ स्था॒\-ऽपां पति॑रसि॒ वृषा᳚स्यू॒र्मिर्वृ॑षसे॒नो॑\-ऽसि व्रज॒क्षितः॑ स्थ म॒रुता॒मोजः॑ स्थ॒ सूर्य॑वर्चसः स्थ॒ सूर्य॑त्वचसः स्थ॒ मान्दाः᳚ स्थ॒ वाशाः᳚ स्थ॒ शक्व॑रीः स्थ विश्व॒भृतः॑ स्थ जन॒भृतः॑ स्था॒\-ऽग्नेस्ते॑ज॒स्याः᳚ स्था॒\-ऽपामोष॑धीना॒ꣳ॒ रसः॑ स्था॒\-ऽपो दे॒वीर्मधु॑मतीरगृह्ण॒न्नूर्ज॑स्वती राज॒सूया॑य॒ चिता॑नाः। याभि॑र्मि॒त्रावरु॑णाव॒भ्यषि॑ञ्च॒न्॒ याभि॒रिन्द्र॒मन॑य॒न्नत्यरा॑तीः॥ रा॒ष्ट्र॒दाः स्थ॑ रा॒ष्ट्रं द॑त्त॒ स्वाहा॑ राष्ट्र॒दाः स्थ॑ रा॒ष्ट्रम॒मुष्मै॑ दत्त॥२०॥

%1.8.12.0
{\anuvakamend[{अत्येका॑दश च}]}%॥11॥

%1.8.12.1
देवी॑रापः॒ सं मधु॑मती॒र्मधु॑मतीभिः सृज्यध्वं॒ महि॒ वर्चः॑ क्ष॒त्रिया॑य वन्वा॒ना अना॑धृष्टाः सीद॒तोर्ज॑स्वती॒र्महि॒ वर्चः॑ क्ष॒त्रिया॑य॒ दध॑ती॒रनि॑भृष्टमसि वा॒चो बन्धु॑स्तपो॒जाः सोम॑स्य दा॒त्रम॑सि शु॒क्रा वः॑ शु॒क्रेणोत्पु॑नामि च॒न्द्राश्च॒न्द्रेणा॒मृता॑ अ॒मृते॑न॒ स्वाहा॑ राज॒सूया॑य॒ चिता॑नाः॥ स॒ध॒मादो᳚ द्यु॒म्निनी॒रूर्ज॑ ए॒ता अनि॑भृष्टा अप॒स्युवो॒ वसा॑नः। प॒स्त्या॑सु चक्रे॒ वरु॑णः स॒धस्थ॑म॒पाꣳ शिशुः॑॥२१॥

%1.8.12.2
मा॒तृत॑मास्व॒न्तः॥ क्ष॒त्रस्योल्ब॑मसि क्ष॒त्रस्य॒ योनि॑र॒स्यावि॑न्नो अ॒ग्निर्गृ॒हप॑ति॒रावि॑न्न॒ इन्द्रो॑ वृ॒द्धश्र॑वा॒ आवि॑न्नः पू॒षा वि॒श्ववे॑दा॒ आवि॑न्नौ मि॒त्रावरु॑णावृता॒वृधा॒वावि॑न्ने॒ द्यावा॑पृथि॒वी धृ॒तव्र॑ते॒ आवि॑न्ना दे॒व्यदि॑तिर्विश्वरू॒प्यावि॑न्नो॒\-ऽयम॒सावा॑मुष्याय॒णो᳚\-ऽस्यां वि॒श्य॑स्मिन् रा॒ष्ट्रे म॑ह॒ते क्ष॒त्राय॑ मह॒त आधि॑पत्याय मह॒ते जान॑राज्यायै॒ष वो॑ भरता॒ राजा॒ सोमो॒\-ऽस्माकं॑ ब्राह्म॒णाना॒ꣳ॒ राजेन्द्र॑स्य॥२२॥

%1.8.12.3
वज्रो॑\-ऽसि॒ वार्त्र॑घ्न॒स्त्वया॒यं वृ॒त्रं व॑ध्याच्छत्रु॒बाध॑नाः स्थ पा॒त मा᳚ प्र॒त्यञ्चं॑ पा॒त मा॑ ति॒र्यञ्च॑म॒न्वञ्चं॑ मा पात दि॒ग्भ्यो मा॑ पात॒ विश्वा᳚भ्यो मा ना॒ष्ट्राभ्यः॑ पात॒ हिर॑ण्यवर्णावु॒षसां᳚ विरो॒के\-ऽयः॑ स्थूणा॒वुदि॑तौ॒ सूर्य॒स्या\-ऽ\-ऽरो॑हतं वरुण मित्र॒ गर्तं॒ तत॑श्चक्षाथा॒मदि॑तिं॒ दितिं॑ च॥२३॥

%1.8.13.0
{\anuvakamend[{शिशु॒रिन्द्र॒स्यैक॑चत्वारिꣳशच्च}]}%॥12॥

%1.8.13.1
स॒मिध॒मा ति॑ष्ठ गाय॒त्री त्वा॒ छन्द॑सामवतु त्रि॒वृथ्स्तोमो॑ रथन्त॒रꣳ सामा॒ग्निर्दे॒वता॒ ब्रह्म॒ द्रवि॑णमु॒ग्रामा ति॑ष्ठ त्रि॒ष्टुप् त्वा॒ छन्द॑सामवतु पञ्चद॒शः स्तोमो॑ बृ॒हथ्सामेन्द्रो॑ दे॒वता᳚ क्ष॒त्रं द्रवि॑णं वि॒राज॒मा ति॑ष्ठ॒ जग॑ती त्वा॒ छन्द॑सामवतु सप्तद॒शः स्तोमो॑ वैरू॒पꣳ साम॑ म॒रुतो॑ दे॒वता॒ विड्द्रवि॑ण॒मुदी॑ची॒मा ति॑ष्ठानु॒ष्टुप् त्वा᳚॥२४॥

%1.8.13.2
छन्द॑सामवत्वेकवि॒ꣳ॒शः स्तोमो॑ वैरा॒जꣳ साम॑ मि॒त्रावरु॑णौ दे॒वता॒ बलं॒ द्रवि॑णमू॒र्ध्वामा ति॑ष्ठ प॒ङ्क्तिस्त्वा॒ छन्द॑सामवतु त्रिणवत्रयस्त्रि॒ꣳ॒शौ स्तोमौ॑ शाक्वररैव॒ते साम॑नी॒ बृह॒स्पति॑र्दे॒वता॒ वर्चो॒ द्रवि॑णमी॒दृङ् चा᳚न्या॒दृङ् चै॑ता॒दृङ् च॑ प्रति॒दृङ् च॑ मि॒तश्च॒ सम्मि॑तश्च॒ सभ॑राः। शु॒क्रज्यो॑तिश्च चि॒त्रज्यो॑तिश्च स॒त्यज्यो॑तिश्च॒ ज्योति॑ष्माꣴश्च स॒त्यश्च॑र्त॒पाश्च॑॥२५॥

%1.8.13.3
अत्यꣳ॑हाः। अ॒ग्नये॒ स्वाहा॒ सोमा॑य॒ स्वाहा॑ सवि॒त्रे स्वाहा॒ सर॑स्वत्यै॒ स्वाहा॑ पू॒ष्णे स्वाहा॒ बृह॒स्पत॑ये॒ स्वाहेन्द्रा॑य॒ स्वाहा॒ घोषा॑य॒ स्वाहा॒ श्लोका॑य॒ स्वाहा\-ऽꣳशा॑य॒ स्वाहा॒ भगा॑य॒ स्वाहा॒ क्षेत्र॑स्य॒ पत॑ये॒ स्वाहा॑ पृथि॒व्यै स्वाहा॒\-ऽन्तरि॑क्षाय॒ स्वाहा॑ दि॒वे स्वाहा॒ सूर्या॑य॒ स्वाहा॑ च॒न्द्रम॑से॒ स्वाहा॒ नक्ष॑त्रेभ्यः॒ स्वाहा॒\-ऽद्भ्यः स्वाहौष॑धीभ्यः॒ स्वाहा॒ वन॒स्पति॑भ्यः॒ स्वाहा॑ चराच॒रेभ्यः॒ स्वाहा॑ परिप्ल॒वेभ्यः॒ स्वाहा॑ सरीसृ॒पेभ्यः॒ स्वाहा᳚॥२६॥

%1.8.14.0
{\anuvakamend[{अ॒नु॒ष्टुप्त्व॑र्त॒पाश्च॑ सरीसृ॒पेभ्यः॒ स्वाहा᳚}]}%॥13॥

%1.8.14.1
सोम॑स्य॒ त्विषि॑रसि॒ तवे॑व मे॒ त्विषि॑र्भूयाद॒मृत॑मसि मृ॒त्योर्मा॑ पाहि दि॒द्योन्मा॑ पा॒ह्यवे᳚ष्टा दन्द॒शूका॒ निर॑स्तं॒ नमु॑चेः॒ शिरः॑॥ सोमो॒ राजा॒ वरु॑णो दे॒वा ध॑र्म॒सुव॑श्च॒ ये। ते ते॒ वाचꣳ॑ सुवन्तां॒ ते ते᳚ प्रा॒णꣳ सु॑वन्तां॒ ते ते॒ चक्षुः॑ सुवन्तां॒ ते ते॒ श्रोत्रꣳ॑ सुवन्ता॒ꣳ॒ सोम॑स्य त्वा द्यु॒म्नेना॒भिषि॑ञ्चाम्य॒ग्नेः॥२७॥

%1.8.14.2
तेज॑सा॒ सूर्य॑स्य॒ वर्च॒सेन्द्र॑स्येन्द्रि॒येण॑ मि॒त्रावरु॑णयोर्वी॒र्ये॑ण म॒रुता॒मोज॑सा क्ष॒त्राणां᳚ क्ष॒त्रप॑तिर॒स्यति॑ दि॒वस्पा॑हि स॒माव॑वृत्रन्नध॒रागुदी॑ची॒रहिं॑ बु॒ध्निय॒मनु॑ सं॒चर॑न्ती॒स्ताः पर्व॑तस्य वृष॒भस्य॑ पृ॒ष्ठे नाव॑श्चरन्ति स्व॒सिच॑ इया॒नाः॥ रुद्र॒ यत्ते॒ क्रयी॒ परं॒ नाम॒ तस्मै॑ हु॒तम॑सि य॒मेष्ट॑मसि। प्रजा॑पते॒ न त्वदे॒तान्य॒न्यो विश्वा॑ जा॒तानि॒ परि॒ ता ब॑भूव। यत्का॑मास्ते जुहु॒मस्तन्नो॑ अस्तु व॒यꣴ स्या॑म॒ पत॑यो रयी॒णाम्॥२८॥

%1.8.15.0
{\anuvakamend[{अ॒ग्नेस्तैका॑दश च}]}%॥14॥

%1.8.15.1
इन्द्र॑स्य॒ वज्रो॑\-ऽसि॒ वार्त्र॑घ्न॒स्त्वया॒\-ऽयं वृ॒त्रं व॑ध्यान्मि॒त्रावरु॑णयोस्त्वा प्रशा॒स्त्रोः प्र॒शिषा॑ युनज्मि य॒ज्ञस्य॒ योगे॑न॒ विष्णोः॒ क्रमो॑\-ऽसि॒ विष्णोः᳚ क्रा॒न्तम॑सि॒ विष्णो॒र्विक्रा᳚न्तमसि म॒रुतां᳚ प्रस॒वे जे॑षमा॒प्तं मनः॒ सम॒हमि॑न्द्रि॒येण॑ वी॒र्ये॑ण पशू॒नां म॒न्युर॑सि॒ तवे॑व मे म॒न्युर्भू॑या॒न्नमो॑ मा॒त्रे पृ॑थि॒व्यै मा\-ऽहं मा॒तरं॑ पृथि॒वीꣳ हिꣳ॑सिषं॒ मा॥२९॥

%1.8.15.2
मां मा॒ता पृ॑थि॒वी हिꣳ॑सी॒दिय॑द॒स्यायु॑र॒स्यायु॑र्मे धे॒ह्यूर्ग॒स्यूर्जं॑ मे धेहि॒ युङ्ङ॑सि॒ वर्चो॑\-ऽसि॒ वर्चो॒ मयि॑ धेह्य॒ग्नये॑ गृ॒हप॑तये॒ स्वाहा॒ सोमा॑य॒ वन॒स्पत॑ये॒ स्वाहेन्द्र॑स्य॒ बला॑य॒ स्वाहा॑ म॒रुता॒मोज॑से॒ स्वाहा॑ ह॒ꣳ॒सः शु॑चि॒षद्वसु॑रन्तरिक्ष॒\-सद्धोता॑ वेदि॒षदति॑थिर्दुरोण॒सत्। नृ॒षद्व॑र॒सदृ॑त॒सद्व्यो॑म॒सद॒ब्जा गो॒जा ऋ॑त॒जा अ॑द्रि॒जा ऋ॒तं बृ॒हत्॥३०॥

%1.8.16.0
{\anuvakamend[{हि॒ꣳ॒सि॒षं॒ मर्त॒जास्त्रीणि॑ च}]}%॥15॥

%1.8.16.1
मि॒त्रो॑\-ऽसि॒ वरु॑णो\-ऽसि॒ सम॒हं विश्वै᳚र्दे॒वैः क्ष॒त्रस्य॒ नाभि॑रसि क्ष॒त्रस्य॒ योनि॑रसि स्यो॒नामा सी॑द सु॒षदा॒मा सी॑द॒ मा त्वा॑ हिꣳसी॒न्मा मा॑ हिꣳसी॒न्निष॑साद धृ॒तव्र॑तो॒ वरु॑णः प॒स्त्या᳚स्वा साम्रा᳚ज्याय सु॒क्रतु॒र्ब्रह्मा(३)न् त्वꣳ रा॑जन् ब्र॒ह्मा\-ऽसि॑ सवि॒ता\-ऽसि॑ स॒त्यस॑वो॒ ब्रह्मा(३)न् त्वꣳ रा॑जन् ब्र॒ह्मा\-ऽसीन्द्रो॑\-ऽसि स॒त्यौजाः᳚॥३१॥

%1.8.16.2
ब्रह्मा(३)न् त्वꣳ रा॑जन् ब्र॒ह्मा\-ऽसि॑ मि॒त्रो॑\-ऽसि सु॒शेवो॒ ब्रह्मा(३)न् त्वꣳ रा॑जन् ब्र॒ह्मा\-ऽसि॒ वरु॑णो\-ऽसि स॒त्यध॒र्मेन्द्र॑स्य॒ वज्रो॑\-ऽसि॒ वार्त्र॑घ्न॒स्तेन॑ मे रध्य॒ दिशो॒\-ऽभ्य॑यꣳ राजा॑\-ऽभू॒थ्सुश्लो॒काँ(४) सुम॑ङ्ग॒लाँ(४) सत्य॑रा॒जा(३)न्। अ॒पां नप्त्रे॒ स्वाहो॒र्जो नप्त्रे॒ स्वाहा॒\-ऽग्नये॑ गृ॒हप॑तये॒ स्वाहा᳚॥३२॥

%1.8.17.0
{\anuvakamend[{स॒त्यौजा᳚श्चत्वारि॒ꣳ॒शच्च॑}]}%॥16॥

%1.8.17.1
आ॒ग्ने॒यम॒ष्टाक॑पालं॒ निर्व॑पति॒ हिर॑ण्यं॒ दक्षि॑णा सारस्व॒तं च॒रुं व॑थ्सत॒री दक्षि॑णा सावि॒त्रं द्वाद॑शकपालमुपध्व॒स्तो दक्षि॑णा पौ॒ष्णं च॒रुꣴ श्या॒मो दक्षि॑णा बार्\mbox{}हस्प॒त्यं च॒रुꣳ शि॑तिपृ॒ष्ठो दक्षि॑णै॒न्द्रमेका॑दशकपालमृष॒भो दक्षि॑णा वारु॒णं दश॑कपालं म॒हानि॑रष्टो॒ दक्षि॑णा सौ॒म्यं च॒रुं ब॒भ्रुर्दक्षि॑णा त्वा॒ष्ट्रम॒ष्टाक॑पालꣳ शु॒ण्ठो दक्षि॑णा वैष्ण॒वं त्रि॑कपा॒लं वा॑म॒नो दक्षि॑णा॥३३॥

%1.8.18.0
{\anuvakamend[{आ॒ग्ने॒यं द्विच॑त्वारिꣳशत्}]}%॥17॥

%1.8.18.1
स॒द्यो दी᳚क्षयन्ति स॒द्यः सोमं॑ क्रीणन्ति पुण्डरिस्र॒जां प्र य॑च्छति द॒शभि॑र्वथ्सत॒रैः सोमं॑ क्रीणाति दश॒पेयो॑ भवति श॒तं ब्रा᳚ह्म॒णाः पि॑बन्ति सप्तद॒शꣴ स्तो॒त्रं भ॑वति प्राका॒शाव॑ध्व॒र्यवे॑ ददाति॒ स्रज॑मुद्गा॒त्रे रु॒क्मꣳ होत्रे\-ऽश्वं॑ प्रस्तोतृप्रतिह॒र्तृभ्यां॒ द्वाद॑श पष्ठौ॒हीर्ब्र॒ह्मणे॑ व॒शां मै᳚त्रावरु॒णाय॑र्\mbox{}ष॒भं ब्रा᳚ह्मणाच्छ॒ꣳ॒सिने॒ वास॑सी नेष्टापो॒तृभ्या॒ꣴ॒ स्थूरि॑ यवाचि॒तम॑च्छावा॒काया॑न॒ड्वाह॑म॒ग्नीधे॑ भार्ग॒वो होता॑ भवति श्राय॒न्तीयं॑ ब्रह्मसा॒मं भ॑वति वारव॒न्तीय॑मग्निष्टोमसा॒मꣳ सा॑रस्व॒तीर॒पो गृ॑ह्णाति॥३४॥

%1.8.19.0
{\anuvakamend[{वा॒र॒व॒न्तीयं॑ च॒त्वारि॑ च}]}%॥18॥

%1.8.19.1
आ॒ग्ने॒यम॒ष्टाक॑पालं॒ निर्व॑पति॒ हिर॑ण्यं॒ दक्षि॑णै॒न्द्रमेका॑दशकपालमृष॒भो दक्षि॑णा वैश्वदे॒वं च॒रुं पि॒शङ्गी॑ पष्ठौ॒ही दक्षि॑णा मैत्रावरु॒णीमा॒मिक्षां᳚ व॒शा दक्षि॑णा बार्\mbox{}हस्प॒त्यं च॒रुꣳ शि॑तिपृ॒ष्ठो दक्षि॑णा\-ऽ\-ऽदि॒त्यां म॒ल्॒\mbox{}हां ग॒र्भिणी॒मा ल॑भते मारु॒तीं पृश्निं॑ पष्ठौ॒हीम॒श्वि\-भ्यां᳚ पू॒ष्णे पु॑रो॒डाशं॒ द्वाद॑शकपालं॒ निर्व॑पति॒ सर॑स्वते सत्य॒वाचे॑ च॒रुꣳ स॑वि॒त्रे स॒त्यप्र॑सवाय पुरो॒डाशं॒ द्वाद॑शकपालं तिसृध॒न्वꣳ शु॑ष्कदृ॒तिर्दक्षि॑णा॥३५॥

%1.8.20.0
{\anuvakamend[{आ॒ग्ने॒यꣳ स॒प्तच॑त्वारिꣳशत्}]}%॥19॥

%1.8.20.1
आ॒ग्ने॒यम॒ष्टाक॑पालं॒ निर्व॑पति सौ॒म्यं च॒रुꣳ सा॑वि॒त्रं द्वाद॑शकपालं बार्\mbox{}हस्प॒त्यं च॒रुं त्वा॒ष्ट्रम॒ष्टाक॑पालं वैश्वान॒रं द्वाद॑शकपालं॒ दक्षि॑णो रथवाहनवा॒हो दक्षि॑णा सारस्व॒तं च॒रुं निर्व॑पति पौ॒ष्णं च॒रुं मै॒त्रं च॒रुं वा॑रु॒णं च॒रुं क्षै᳚त्रप॒त्यं च॒रुमा॑दि॒त्यं च॒रुमुत्त॑रो रथवाहनवा॒हो दक्षि॑णा॥३६॥

%1.8.21.0
{\anuvakamend[{आ॒ग्ने॒यं चतु॑स्त्रिꣳशत्}]}%॥20॥

%1.8.21.1
स्वा॒द्वीं त्वा᳚ स्वा॒दुना॑ ती॒व्रां ती॒व्रेणा॒मृता॑म॒मृते॑न सृ॒जामि॒ सꣳसोमे॑न॒ सोमो᳚\-ऽस्य॒श्वि\-भ्यां᳚ पच्यस्व॒ सर॑स्वत्यै पच्य॒स्वेन्द्रा॑य सु॒त्राम्णे॑ पच्यस्व पु॒नातु॑ ते परि॒स्रुत॒ꣳ॒ सोम॒ꣳ॒ सूर्य॑स्य दुहि॒ता। वारे॑ण॒ शश्व॑ता॒ तना᳚॥ वा॒युः पू॒तः प॒वित्रे॑ण प्र॒त्यङ्ख्सोमो॒ अति॑द्रुतः। इन्द्र॑स्य॒ युज्यः॒ सखा᳚॥ कु॒विद॒ङ्ग यव॑मन्तो॒ यवं॑ चि॒द्यथा॒ दान्त्य॑नुपू॒र्वं वि॒यूय॑। इ॒हेहै॑षां कृणुत॒ भोज॑नानि॒ ये ब॒र्॒\mbox{}हिषो॒ नमो॑वृक्तिं॒ न ज॒ग्मुः॥ आ॒श्वि॒नं धू॒म्रमा ल॑भते सारस्व॒तं मे॒षमै॒न्द्रमृ॑ष॒भमै॒न्द्रमेका॑दशकपालं॒ निर्व॑पति सावि॒त्रं द्वाद॑शकपालं वारु॒णं दश॑कपाल॒ꣳ॒ सोम॑प्रतीकाः पितरस्तृप्णुत॒ वड॑बा॒ दक्षि॑णा॥३७॥

%1.8.22.0
{\anuvakamend[{भोज॑नानि॒ षड्विꣳ॑शतिश्च}]}%॥21॥

%1.8.22.1
अग्ना॑विष्णू॒ महि॒ तद्वां᳚ महि॒त्वं वी॒तं घृ॒तस्य॒ गुह्या॑नि॒ नाम॑। दमे॑दमे स॒प्त रत्ना॒ दधा॑ना॒ प्रति॑ वां जि॒ह्वा घृ॒तमा च॑रण्येत्॥ अग्ना॑विष्णू॒ महि॒ धाम॑ प्रि॒यं वां᳚ वी॒थो घृ॒तस्य॒ गुह्या॑ जुषा॒णा। दमे॑दमे सुष्टु॒तीर्वा॑वृधा॒ना प्रति॑ वां जि॒ह्वा घृ॒तमुच्च॑रण्येत्॥ प्र णो॑ दे॒वी सर॑स्वती॒ वाजे॑भिर्वा॒जिनी॑वती। धी॒नाम॑वि॒त्र्य॑वतु। आ नो॑ दि॒वो बृ॑ह॒तः॥३८॥

%1.8.22.2
पर्व॑ता॒दा सर॑स्वती यज॒ता ग॑न्तु य॒ज्ञम्। हवं॑ दे॒वी जु॑जुषा॒णा घृ॒ताची॑ श॒ग्मां नो॒ वाच॑मुश॒ती शृ॑णोतु॥ बृह॑स्पते जु॒षस्व॑ नो ह॒व्यानि॑ विश्वदेव्य। रास्व॒ रत्ना॑नि दा॒शुषे᳚॥ ए॒वा पि॒त्रे वि॒श्वदे॑वाय॒ वृष्णे॑ य॒ज्ञैर्वि॑धेम॒ नम॑सा ह॒विर्भिः॑। बृह॑स्पते सुप्र॒जा वी॒रव॑न्तो व॒यꣴ स्या॑म॒ पत॑यो रयी॒णाम्॥ बृह॑स्पते॒ अति॒ यद॒र्यो अर्\mbox{}हा᳚द्द्यु॒मद्वि॒भाति॒ क्रतु॑म॒ज्जने॑षु। यद्दी॒दय॒च्छव॑सा॥३९॥

%1.8.22.3
ऋ॒त॒प्र॒जा॒त॒ तद॒स्मासु॒ द्रवि॑णं धेहि चि॒त्रम्॥ आ नो॑ मित्रावरुणा घृ॒तैर्गव्यू॑तिमुक्षतम्। मध्वा॒ रजाꣳ॑सि सुक्रतू॥ प्र बा॒हवा॑ सिसृतं जी॒वसे॑ न॒ आ नो॒ गव्यू॑तिमुक्षतं घृ॒तेन॑। आ नो॒ जने᳚ श्रवयतं युवाना श्रु॒तं मे॑ मित्रावरुणा॒ हवे॒मा॥ अ॒ग्निं वः॑ पू॒र्व्यं गि॒रा दे॒वमी॑डे॒ वसू॑नाम्। स॒प॒र्यन्तः॑ पुरुप्रि॒यं मि॒त्रं न क्षे᳚त्र॒साध॑सम्॥ म॒क्षू दे॒वव॑तो॒ रथः॑॥४०॥

%1.8.22.4
शूरो॑ वा पृ॒थ्सु कासु॑ चित्। दे॒वानां॒ य इन्मनो॒ यज॑मान॒ इय॑क्षत्य॒भीदय॑ज्वनो भुवत्॥ न य॑जमान रिष्यसि॒ न सु॑न्वान॒ न दे॑वयो॥ अस॒दत्र॑ सु॒वीर्य॑मु॒त त्यदा॒श्वश्वि॑यम्॥ नकि॒ष्टं कर्म॑णा नश॒न्न प्र यो॑ष॒न्न यो॑षति॥ उप॑ क्षरन्ति॒ सिन्ध॑वो मयो॒भुव॑ ईजा॒नं च॑ य॒क्ष्यमा॑णं च धे॒नवः॑। पृ॒णन्तं॑ च॒ पपु॑रिं च॥४१॥

%1.8.22.5
श्र॒व॒स्यवो॑ घृ॒तस्य॒ धारा॒ उप॑ यन्ति वि॒श्वतः॑॥ सोमा॑रुद्रा॒ वि वृ॑हतं॒ विषू॑ची॒ममी॑वा॒ या नो॒ गय॑मावि॒वेश॑। आ॒रे बा॑धेथां॒ निर्\mbox{}ऋ॑तिं परा॒चैः कृ॒तं चि॒देनः॒ प्रमु॑मुक्तम॒स्मत्॥ सोमा॑रुद्रा यु॒वमे॒तान्य॒स्मे विश्वा॑ त॒नूषु॑ भेष॒जानि॑ धत्तम्। अव॑ स्यतं मु॒ञ्चतं॒ यन्नो॒ अस्ति॑ त॒नूषु॑ ब॒द्धं कृ॒तमेनो॑ अ॒स्मत्॥ सोमा॑पूषणा॒ जन॑ना रयी॒णां जन॑ना दि॒वो जन॑ना पृथि॒व्याः। जा॒तौ विश्व॑स्य॒ भुव॑नस्य गो॒पौ दे॒वा अ॑कृण्वन्न॒मृत॑स्य॒ नाभिम्᳚॥ इ॒मौ दे॒वौ जाय॑मानौ जुषन्ते॒मौ तमाꣳ॑सि गूहता॒मजु॑ष्टा। आ॒भ्यामिन्द्रः॑ प॒क्वमा॒मास्व॒न्तः सो॑मापू॒ष\-भ्यां᳚ जनदु॒स्रिया॑सु॥४२॥

%2.1.0.0
{\anuvakamend[{बृ॒ह॒तः शव॑सा॒ रथः॒ पपु॑रिं च दि॒वो जन॑ना॒ पञ्च॑विꣳशतिश्च}]}%॥22॥

%2.1.0.0

{\anuvakamend[{वा॒य॒व्यं॑ प्र॒जाप॑ति॒स्ता वरु॑णं देवासु॒रा ए॒ष्व॑सावा॑दि॒त्यो दश॑र्\mbox{}षभा॒मिन्द्रो॑ व॒लस्य॑ बार्\mbox{}हस्प॒त्यं व॑षट्का॒रो॑\-ऽसौ सौ॒रीं वरु॑णमाश्वि॒नमिन्द्रं॑ वो॒ नर॒ एका॑दश॥11॥ वा॒य॒व्य॑माग्ने॒यीं कृ॑ष्णग्री॒वीम॒सावा॑दि॒त्यो वा अ॑होरा॒त्राणि॑ वषट्का॒रः प्र॑जनयि॒ता हु॑वे तु॒राणां॒ पञ्च॑षष्टिः॥65॥ वा॒य॒व्यं॑ प्रमो॑षीः॥ ---------------------॥ प्रथमः प्रश्नः समाप्तः॥ ------------------------}]}
%%% END KANDAM

\chapt{काण्डम् २}
\sect{प्रथमः प्रश्नः}\setcounter{anuvakam}{0}
\dnsub{तैत्तिरीयसंहितायां द्वितीयकाण्डे प्रथमः प्रश्नः}
%2.1.1.0
%2.1.1.1
वा॒य॒व्यꣴ॑ श्वे॒तमाल॑भेत॒ भूति॑कामो वा॒युर्वैH क्षेपि॑ष्ठा दे॒वता॑ वा॒युमे॒व स्वेन॑ भाग॒धेये॒नोप॑धावति॒ स ए॒वैनं॒ भूति॑ङ्गमयति॒ भव॑त्ये॒वाति॑क्षिप्रा दे॒वतेत्या॑हुः॒ सैन॑मीश्व॒रा प्र॒दह॒ इत्ये॒तमे॒व सन्तं॑ वा॒यवे॑ नि॒युत्व॑त॒ आल॑भेत नि॒युद्वा अ॑स्य॒ धृति॑र्द्धृ॒त ए॒व भूति॒मुपै॒त्यप्र॑दाहाय॒ भव॑त्ये॒व (1)

%2.1.1.2
वा॒यवे॑ नि॒युत्व॑त॒ आल॑भेत॒ ग्राम॑कामो वा॒युर्वा इ॒माः प्र॒जा न॑स्यो॒ता ने॑नीयते वा॒युमे॒व नि॒युत्व॑न्त॒ꣴ॒ स्वेन॑ भाग॒धेये॒नोप॑धावति॒ स ए॒वास्मै᳚ प्र॒जा न॑स्यो॒ता निय॑च्छति ग्रा॒म्ये॑व भ॑वति नि॒युत्व॑ते भवति द्ध्रु॒वा ए॒वास्मा॒ अन॑पगाः करोति वा॒यवे॑ नि॒युत्व॑त॒ आल॑भेत प्र॒जाका॑मः प्रा॒णो वै वा॒युर॑पा॒नो नि॒युत्प्रा॑णापा॒नौ खलु॒ वा ए॒तस्य॑ प्र॒जायाः᳚ (2)

%2.1.1.3
अप॑क्रामतो॒ यो\-ऽलं॑ प्र॒जायै॒ सन्प्र॒जान्न वि॒न्दते॑ वा॒युमे॒व नि॒युत्व॑न्त॒ꣴ॒ स्वेन॑ भाग॒धेये॒नोप॑धावति॒ स ए॒वास्मै᳚ प्राणापा॒ना\-भ्यां᳚ प्र॒जां प्रज॑नयति वि॒न्दते᳚ प्र॒जाँव्वा॒यवे॑ नि॒युत्व॑त॒ आल॑भेत॒ ज्योगा॑मयावी प्रा॒णो वै वा॒युर॑पा॒नो नि॒युत् प्रा॑णापा॒नौ खलु॒ वा ए॒तस्मा॒दप॑क्रामतो॒ यस्य॒ ज्योगा॒मय॑ति वा॒युमे॒व नि॒युत्व॑न्त॒ꣴ॒ स्वेन॑ भाग॒धेये॒नोप॑ (3)

%2.1.1.4
धा॒व॒ति॒ स ए॒वास्मि॑न्प्राणापा॒नौ द॑धात्यु॒त यदी॒तासु॒र्भव॑ति॒ जीव॑त्ये॒व प्र॒जाप॑ति॒र्वा इ॒दमेक॑ आसी॒थ्सो॑\-ऽकामयत प्र॒जाः प॒शून्थ्सृ॑जे॒येति॒ स आ॒त्मनो॑ व॒पामुद॑क्खिद॒त्ताम॒ग्नौ प्रागृ॑ह्णा॒त्ततो॒\-ऽजस्तू॑प॒रः सम॑भव॒त्तꣴ स्वायै॑ दे॒वता॑या॒ आ\-ऽल॑भत॒ ततो॒ वै स प्र॒जाः प॒शून॑सृजत॒ यः प्र॒जाका॑मः (4)

%2.1.1.5
प॒शुका॑मः॒ स्याथ्स ए॒तं प्रा॑जाप॒त्यम॒जन्तू॑प॒रमाल॑भेत प्र॒जाप॑तिमे॒व स्वेन॑ भाग॒धेये॒नोप॑धावति॒ स ए॒वास्मै᳚ प्र॒जां प॒शून्प्रज॑नयति॒ यच्छ्म॑श्रु॒णस्तत्पुरु॑षाणाꣳ रू॒पं यत्तू॑प॒रस्तदश्वा॑नां॒ यद॒न्यतो॑द॒न्तद्गवां॒ यदव्या॑ इव श॒फास्तदवी॑नां॒ यद॒ज\-स्तद॒जाना॑मे॒ताव॑न्तो॒ वै ग्रा॒म्याः प॒शव॒स्तान् (5)

%2.1.1.6
रू॒पेणै॒वाव॑रुन्धे सोमापौ॒ष्णन्त्रै॒तमाल॑भेत प॒शुका॑मो॒ द्वौ वा अ॒जायै॒ स्तनौ॒ नानै॒व द्वाव॒भिजाये॑ते॒ ऊर्जं॒ पुष्टि॑न्तृ॒तीयः॑ सोमापू॒षणा॑वे॒व स्वेन॑ भाग॒धेये॒नोप॑धावति॒ तावे॒वास्मै॑ प॒शून्प्रज॑नयतः॒ सोमो॒ वै रे॑तो॒धाः पू॒षा प॑शू॒नां प्र॑जनयि॒ता सोम॑ ए॒वास्मै॒ रेतो॒ दधा॑ति पू॒षा प॒शून्प्रज॑नय॒त्यौदु॑म्बरो॒ यूपो॑ भव॒त्यूर्ग्वा उ॑दुं॒बर॒ ऊर्क्प॒शव॑ ऊ॒र्जैवास्मा॒ ऊर्जं॑ प॒शूनव॑रुन्धे॥६॥

%2.1.2.0
{\anuvakamend{भव॑त्ये॒व प्र॒जाया॑ आ॒मय॑ति वा॒युमे॒व नि॒युत्व॑न्त॒ꣴ॒ स्वेन॑ भाग॒धेये॒नोप॑ प्र॒जाका॑म॒स्तान् यूप॒स्त्रयो॑दश च।]}}

%2.1.2.1
प्र॒जाप॑तिः प्र॒जा अ॑सृजत॒ ता अ॑स्माथ्सृ॒ष्टाः परा॑चीराय॒न्ता वरु॑णमगच्छ॒न्ता अन्वै॒त्ताः पुन॑रयाचत॒ ता अ॑स्मै॒ न पुन॑रददा॒थ्सो᳚\-ऽब्रवी॒द्वरं॑ वृणी॒ष्वाथ॑ मे॒ पुन॑र्दे॒हीति॒ तासां॒ वर॒मा\-ऽल॑भत॒ स कृ॒ष्ण एक॑शितिपादभव॒द्यो वरु॑णगृहीतः॒ स्याथ्स ए॒तं वा॑रु॒णं कृ॒ष्णमेक॑शितिपाद॒माल॑भेत॒ वरु॑णम् (7)

%2.1.2.2
ए॒व स्वेन॑ भाग॒धेये॒नोप॑धावति॒ स ए॒वैनं॑ वरुणपा॒शान्मु॑ञ्चति कृ॒ष्ण एक॑शितिपाद्भवति वारु॒णो ह्ये॑ष दे॒वत॑या॒ समृ॑द्ध्यै॒ सुव॑र्भानुरासु॒रः सूर्य॒न्तम॑सा\-ऽविद्ध्य॒त्तस्मै॑ दे॒वाः प्राय॑श्चित्तिमैच्छ॒न्तस्य॒ यत्प्र॑थ॒मन्तमो॒\-ऽपाघ्न॒न्थ्सा कृ॒ष्णा\-ऽवि॑रभव॒द्यद्द्वि॒तीय॒ꣳ॒ सा फल्गु॑नी॒ यत्तृ॒तीय॒ꣳ॒ सा ब॑ल॒क्षी यद॑द्ध्य॒स्थाद॒पाकृ॑न्त॒न्थ्सा\-ऽवि॑\-ऽर्व॒शा (8)

%2.1.2.3
सम॑भव॒त्ते दे॒वा अ॑ब्रुवन्देवप॒शुर्वा अ॒यꣳ सम॑भू॒त्कस्मा॑ इ॒ममाल॑फ्स्यामह॒ इत्यथ॒ वै तर्ह्यल्पा॑ पृथि॒व्यासी॒दजा॑ता॒ ओष॑धय॒स्तामविं॑ व॒शामा॑दि॒त्येभ्यः॒ कामा॒या\-ऽल॑भन्त॒ ततो॒ वा अप्र॑थत पृथि॒व्यजा॑य॒न्तौष॑धयो॒ यः का॒मये॑त॒ प्रथे॑य प॒शुभिः॒ प्र प्र॒जया॑ जाये॒येति॒ स ए॒तामविं॑ व॒शामा॑दि॒त्येभ्यः॒ कामा॑य (9)

%2.1.2.4
आ ल॑भेतादि॒त्याने॒व काम॒ꣴ॒ स्वेन॑ भाग॒धेये॒नोप॑धावति॒ त ए॒वैनं॑ प्र॒थय॑न्ति प॒शुभिः॒ प्र प्र॒जया॑ जनयन्त्य॒सावा॑दि॒त्यो न व्य॑रोचत॒ तस्मै॑ दे॒वाः प्राय॑श्चित्तिमैच्छ॒न्तस्मा॑ ए॒ता म॒ल्\mbox{}हा आल॑\-ऽभन्ताग्ने॒यीं कृ॑ष्णग्री॒वीꣳ सꣳ॑हि॒तामै॒न्द्रीꣴ श्वे॒तां बा॑र्\mbox{}हस्प॒त्यान्ताभि॑रे॒वास्मि॒न्रुच॑मदधु॒र्यो ब्र॑ह्मवर्च॒सका॑मः॒ स्यात्तस्मा॑ ए॒ता म॒ल्\mbox{}हा आल॑भेत (10)

%2.1.2.5
आ॒ग्ने॒यीं कृ॑ष्णग्री॒वीꣳ सꣳ॑हि॒तामै॒न्द्रीꣴ श्वे॒तां बा॑र्\mbox{}हस्प॒त्यामे॒ता ए॒व दे॒वताः॒ स्वेन॑ भाग॒धेये॒नोप॑धावति॒ ता ए॒वास्मि॑न्ब्रह्मवर्च॒सन्द॑धति ब्रह्मवर्च॒स्ये॑व भ॑वति व॒सन्ता᳚ प्रा॒तरा᳚ग्ने॒यीं कृ॑ष्णग्री॒वीमाल॑भेत ग्री॒ष्मे म॒द्ध्यन्दि॑ने सꣳहि॒तामै॒न्द्रीꣳ श॒रद्य॑परा॒ह्णे श्वे॒तां बा॑र्\mbox{}हस्प॒त्यान्त्रीणि॒ वा आ॑दि॒त्यस्य॒ तेजाꣳ॑सि व॒सन्ता᳚ प्रा॒तर्ग्री॒ष्मे म॒द्ध्यन्दि॑ने श॒रद्य॑परा॒ह्णे याव॑न्त्ये॒व तेजाꣳ॑सि॒ तान्ये॒व (11)

%2.1.2.6
अव॑ रुन्धे सं वथ्स॒रं प॒र्याल॑भ्यन्ते सं वथ्स॒रो वै ब्र॑ह्मवर्च॒सस्य॑ प्रदा॒ता सं॑ वथ्स॒र ए॒वास्मै᳚ ब्रह्मवर्च॒सं प्रय॑च्छति ब्रह्मवर्च॒स्ये॑व भ॑वति ग॒र्भिण॑यो भवन्तीन्द्रि॒यं वै गर्भ॑ इन्द्रि॒यमे॒वास्मि॑न्दधति सारस्व॒तीं मे॒षीमाल॑भेत॒ य ई᳚श्व॒रो वा॒चो वदि॑तोः॒ सन्वाच॒न्न वदे॒द्वाग्वै सर॑स्वती॒ सर॑स्वतीमे॒व स्वेन॑ भाग॒धेये॒नोप॑धावति॒ सैवास्मिन्न्॑ (12)

%2.1.2.7
वाच॑न्दधाति प्रवदि॒ता वा॒चो भ॑व॒त्यप॑न्नदती भवति॒ तस्मा᳚न्मनु॒ष्याः᳚ सर्वां॒ वाचं॑ वदन्त्याग्ने॒यं कृ॒ष्णग्री॑व॒मा ल॑भेत सौ॒म्यं ब॒भ्रुं ज्योगा॑मयाव्य॒ग्निं वा ए॒तस्य॒ शरी॑रं गच्छति॒ सोम॒ꣳ॒ रसो॒ यस्य॒ ज्योगा॒मय॑त्य॒ग्नेरे॒वास्य॒ शरी॑रन्निष्क्री॒णाति॒ सोमा॒द्रस॑मु॒त यदी॒तासु॒र्भव॑ति॒ जीव॑त्ये॒व सौ॒म्यं ब॒भ्रुमाल॑भेताग्ने॒यं कृ॒ष्णग्री॑वं प्र॒जाका॑मः॒ सोमः॑ (13)

%2.1.2.8
वै रे॑तो॒धा अ॒ग्निः प्र॒जानां᳚ प्रजनयि॒ता सोम॑ ए॒वास्मै॒ रेतो॒ दधा᳚त्य॒ग्निः प्र॒जां प्रज॑नयति वि॒न्दते᳚ प्र॒जामा᳚ग्ने॒यं कृ॒ष्ण\-ग्री॑व॒माल॑भेत सौ॒म्यं ब॒भ्रुं यो ब्रा᳚ह्म॒णो वि॒द्याम॒नूच्य॒ न वि॒रोचे॑त॒ यदा᳚ग्ने॒यो भव॑ति॒ तेज॑ ए॒वास्मि॒न्तेन॑ दधाति॒ यथ्सौ॒म्यो ब्र॑ह्मवर्च॒सन्तेन॑ कृ॒ष्णग्री॑व आग्ने॒यो भ॑वति॒ तम॑ ए॒वास्मा॒दप॑हन्ति श्वे॒तो भ॑वति (14)

%2.1.2.9
रुच॑मे॒वास्मि॑न्दधाति ब॒भ्रुः सौ॒म्यो भ॑वति ब्रह्मवर्च॒समे॒वास्मि॒न्त्विषि॑न्दधात्याग्ने॒यं कृ॒ष्णग्री॑व॒माल॑भेत सौ॒म्यं ब॒भ्रुमा᳚ग्ने॒यं कृ॒ष्णग्री॑वं पुरो॒धाया॒ꣴ॒ स्पर्ध॑मान आग्ने॒यो वै ब्रा᳚ह्म॒णः सौ॒म्यो रा॑ज॒न्यो॑\-ऽभितः॑ सौ॒म्यमा᳚ग्ने॒यौ भ॑वत॒स्तेज॑सै॒व ब्रह्म॑णोभ॒यतो॑ रा॒ष्ट्रं परि॑गृह्णात्येक॒धा स॒मावृ॑ङ्क्ते पु॒र ए॑नन्दधते॥ (15)

%2.1.3.0
{\anuvakamend[{ल॒भे॒त॒ वरु॑णं व॒शैतामविं॑ व॒शामा॑दि॒त्येभ्यः॒ कामा॑य म॒ल्\mbox{}हा आल॑भेत॒ तान्ये॒व सैवास्मि॒न्थ्सोमः॑ श्वे॒तो भ॑वति॒ त्रिच॑त्वारिꣳशच्च। (2)।}]}

%2.1.3.1
दे॒वा॒सु॒रा ए॒षु लो॒केष्व॑स्पर्द्धन्त॒ स ए॒तं विष्णु॑र्वाम॒नम॑पश्य॒त्तꣴ स्वायै॑ दे॒वता॑या॒ आ\-ऽल॑भत॒ ततो॒ वै स इ॒माल्लोँ॒कान॒भ्य॑जयद्वैष्ण॒वं वा॑म॒नमाल॑भेत॒ स्पर्द्ध॑मानो॒ विष्णु॑रे॒व भू॒त्वेमाल्लोँ॒कान॒भिज॑यति॒ विष॑म॒ आल॑भेत॒ विष॑मा इव॒ हीमे लो॒काः समृ॑द्ध्या॒ इन्द्रा॑य मन्यु॒मते॒ मन॑स्वते ल॒लामं॑ प्राशॄ॒ङ्गमाल॑भेत सङ्ग्रा॒मे (16)

%2.1.3.2
सं य॑त्त इन्द्रि॒येण॒ वै म॒न्युना॒ मन॑सा सङ्ग्रा॒मञ्ज॑य॒तीन्द्र॑मे॒व म॑न्यु॒मन्तं॒ मन॑स्वन्त॒ꣴ॒ स्वेन॑ भाग॒धेये॒नोप॑धावति॒ स ए॒वास्मि॑न्निन्द्रि॒यं म॒न्युं मनो॑ दधाति॒ जय॑ति॒ तꣳ स॑ङ्ग्रा॒ममिन्द्रा॑य म॒रुत्व॑ते पृश्ञिस॒क्थमाल॑भेत॒ ग्राम॑काम॒ इन्द्र॑मे॒व म॒रुत्व॑न्त॒ꣴ॒ स्वेन॑ भाग॒धेये॒नोप॑धावति॒ स ए॒वास्मै॑ सजा॒तान्प्रय॑च्छति ग्रा॒म्ये॑व भ॑वति॒ यदृ॑ष॒भस्तेन॑ (17)

%2.1.3.3
ऐ॒न्द्रो यत्पृश्ञि॒स्तेन॑ मारु॒तः समृ॑द्ध्यै प॒श्चात्पृ॑श्ञिस॒क्थो भ॑वति पश्चादन्ववसा॒यिनी॑मे॒वास्मै॒ विश॑ङ्करोति सौ॒म्यं ब॒भ्रुमाल॑भे॒तान्न॑कामः सौ॒म्यं वा अन्न॒ꣳ॒ सोम॑मे॒व स्वेन॑ भाग॒धेये॒नोप॑धावति॒ स ए॒वास्मा॒ अन्नं॒ प्रय॑च्छत्यन्ना॒द ए॒व भ॑वति ब॒भ्रुर्भ॑वत्ये॒तद्वा अन्न॑स्य रू॒पꣳ समृ॑द्ध्यै सौ॒म्यं ब॒भ्रुमाल॑भेत॒ यमलम्᳚ (18)

%2.1.3.4
रा॒ज्याय॒ सन्तꣳ॑ रा॒ज्यन्नोप॒नमे᳚थ्सौ॒म्यं वै रा॒ज्यꣳ सोम॑मे॒व स्वेन॑ भाग॒धेये॒नोप॑धावति॒ स ए॒वास्मै॑ रा॒ज्यं प्रय॑च्छ॒त्युपै॑नꣳ रा॒ज्यन्न॑मति ब॒भ्रुर्भ॑वत्ये॒तद्वै सोम॑स्य रू॒पꣳ समृ॑द्ध्या॒ इन्द्रा॑य वृत्र॒तुरे॑ ल॒लामं॑ प्राशृ॒ङ्गमाल॑भेत ग॒तश्रीः᳚ प्रति॒ष्ठाका॑मः पा॒प्मान॑मे॒व वृ॒त्रन्ती॒र्त्वा प्र॑ति॒ष्ठां ग॑च्छ॒तीन्द्रा॑याभिमाति॒घ्ने ल॒लामं॑ प्राशृ॒ङ्गमा (19)

%2.1.3.5
ल॒भे॒त॒ यः पा॒प्मना॑ गृही॒तः स्यात्पा॒प्मा वा अ॒भिमा॑ति॒रिन्द्र॑मे॒वाभि॑माति॒हन॒ꣴ॒ स्वेन॑ भाग॒धेये॒नोप॑धावति॒ स ए॒वास्मा᳚त्पा॒प्मान॑म॒भिमा॑तिं॒ प्रणु॑दत॒ इन्द्रा॑य व॒ज्रिणे॑ ल॒लामं॑ प्राशृ॒ङ्गमाल॑भेत॒ यमलꣳ॑ रा॒ज्याय॒ सन्तꣳ॑ रा॒ज्यन्नोप॒नमे॒दिन्द्र॑मे॒व व॒ज्रिण॒ꣴ॒ स्वेन॑ भाग॒धेये॒नोप॑धावति॒ स ए॒वास्मै॒ वज्रं॒ प्रय॑च्छति॒ स ए॑नं॒ वज्रो॒ भूत्या॑ इन्ध॒ उपै॑नꣳ रा॒ज्यन्न॑मति ल॒लामः॑ प्राशृ॒ङ्गो भ॑वत्ये॒तद्वै वज्र॑स्य रू॒पꣳ समृ॑द्ध्यै॥ (20)

%2.1.4.0
{\anuvakamend[{स॒ङ्ग्रा॒मे तेनाल॑मभिमाति॒घ्ने ल॒लामं॑ प्राशृ॒ङ्गमैनं॒ पञ्च॑दश च।3।}]}

%2.1.4.1
अ॒सावा॑दि॒त्यो न व्य॑रोचत॒ तस्मै॑ दे॒वाः प्राय॑श्चित्तिमैच्छ॒न्तस्मा॑ ए॒तान्दश॑र्\mbox{}षभा॒मा\-ऽल॑भन्त॒ तयै॒वास्मि॒न्रुच॑मदधु॒र्यो ब्र॑ह्मवर्च॒सका॑मः॒ स्यात्तस्मा॑ ए॒तान्दश॑र्\mbox{}षभा॒मा\-ऽल॑भेता॒मुमे॒वादि॒त्यꣴ स्वेन॑ भाग॒धेये॒नोप॑धावति॒ स ए॒वास्मि॑न्ब्रह्म\-वर्च॒सन्द॑धाति ब्रह्मवर्च॒स्ये॑व भ॑वति व॒सन्ता᳚ प्रा॒तस्त्रील्लँ॒लामा॒नाल॑भेत ग्री॒ष्मे म॒द्ध्यन्दि॑ने (21)

%2.1.4.2
त्रीञ्छि॑तिपृ॒ष्ठाञ्छ॒रद्य॑परा॒ह्णे त्रीञ्छि॑ति॒वारा॒न्त्रीणि॒ वा आ॑दि॒त्यस्य॒ तेजाꣳ॑सि व॒सन्ता᳚ प्रा॒तर्ग्री॒ष्मे म॒द्ध्यन्दि॑ने श॒रद्य॑परा॒ह्णे याव॑न्त्ये॒व तेजाꣳ॑सि॒ तान्ये॒वाव॑रुन्धे॒ त्रय॑स्त्रय॒ आल॑भ्यन्ते\-ऽभिपू॒र्वमे॒वास्मि॒न्तेजो॑ दधाति सं वथ्स॒रं प॒र्याल॑भ्यन्ते सं वथ्स॒रो वै ब्र॑ह्मवर्च॒सस्य॑ प्रदा॒ता सं॑ वथ्स॒र ए॒वास्मै᳚ ब्रह्मवर्च॒सं प्रय॑च्छति ब्रह्मवर्च॒स्ये॑व भ॑वति सं वथ्स॒रस्य॑ प॒रस्ता᳚त्प्राजाप॒त्यङ्कद्रुम्᳚ (22)

%2.1.4.3
आल॑भेत प्र॒जाप॑तिः॒ सर्वा॑ दे॒वता॑ दे॒वता᳚स्वे॒व प्रति॑तिष्ठति॒ यदि॑ बिभी॒याद्दु॒श्चर्मा॑ भविष्या॒मीति॑ सोमापौ॒ष्णꣴ श्या॒ममाल॑भेत सौ॒म्यो वै दे॒वत॑या॒ पुरु॑षः पौ॒ष्णाः प॒शवः॒ स्वयै॒वास्मै॑ दे॒वत॑या प॒शुभि॒स्त्वच॑ङ्करोति॒ न दु॒श्चर्मा॑ भवति दे॒वाश्च॒ वै य॒मश्चा॒स्मिल्लोँ॒के᳚\-ऽस्पर्द्धन्त॒ स य॒मो दे॒वाना॑मिन्द्रि॒यं वी॒र्य॑मयुवत॒ तद्य॒मस्य॑ (23)

%2.1.4.4
य॒म॒त्वन्ते दे॒वा अ॑मन्यन्त य॒मो वा इ॒दम॑भू॒द्यद्व॒यꣴ स्म इति॒ ते प्र॒जाप॑ति॒मुपा॑धाव॒न्थ्स ए॒तौ प्र॒जाप॑तिरा॒त्मन॑ उक्षव॒शौ निर॑मिमीत॒ ते दे॒वा वै᳚ष्णावरु॒णीं व॒शामा\-ऽल॑भन्तै॒न्द्रमु॒क्षाण॒न्तं वरु॑णेनै॒व ग्रा॑हयि॒त्वा विष्णु॑ना य॒ज्ञेन॒ प्राणु॑दन्तै॒न्द्रेणै॒वास्ये᳚न्द्रि॒यम॑वृञ्जत॒ यो भ्रातृ॑व्यवा॒न्थ्स्याथ्स स्पर्द्ध॑मानो वैष्णावरु॒णीम् (24)

%2.1.4.5
व॒शामाल॑भेतै॒न्द्रमु॒क्षाणं॒ वरु॑णेनै॒व भ्रातृ॑व्यङ्ग्राहयि॒त्वा विष्णु॑ना य॒ज्ञेन॒ प्रणु॑दत ऐ॒न्द्रेणै॒वास्ये᳚न्द्रि॒यं वृ॑ङ्क्ते॒ भव॑त्या॒त्मना॒ परा᳚स्य॒ भ्रातृ॑व्यो भव॒तीन्द्रो॑ वृ॒त्रम॑ह॒न्तं वृ॒त्रो ह॒तष्षो॑ड॒शभि॑र्भो॒गैर॑सिना॒त्तस्य॑ वृ॒त्रस्य॑ शीर्\mbox{}ष॒तो गाव॒ उदा॑य॒न्ता वै॑दे॒ह्यो॑\-ऽभव॒न्तासा॑मृष॒भो ज॒घने\-ऽनूदै॒त्तमिन्द्रः॑ (25)

%2.1.4.6
अ॒चा॒य॒थ्सो॑\-ऽमन्यत॒ यो वा इ॒ममा॒लभे॑त॒ मुच्ये॑ता॒स्मात्पा॒प्मन॒ इति॒ स आ᳚ग्ने॒यं कृ॒ष्णग्री॑व॒माल॑भतै॒न्द्रमृ॑ष॒भन्तस्या॒ग्निरे॒व स्वेन॑ भाग॒धेये॒नोप॑सृतष्षोडश॒धा वृ॒त्रस्य॑ भो॒गानप्य॑दहदै॒न्द्रेणे᳚न्द्रि॒यमा॒त्मन्न॑धत्त॒ यः पा॒प्मना॑ गृही॒तः स्याथ्स आ᳚ग्ने॒यं कृ॒ष्णग्री॑व॒माल॑भेतै॒न्द्रमृ॑ष॒भम॒ग्निरे॒वास्य॒ स्वेन॑ भाग॒धेये॒नोप॑सृतः (26)

%2.1.4.7
पा॒प्मान॒मपि॑ दहत्यै॒न्द्रेणे᳚न्द्रि॒यमा॒त्मन्ध॑त्ते॒ मुच्य॑ते पा॒प्मनो॒ भव॑त्ये॒व द्या॑वापृथि॒व्या᳚न्धे॒नुमाल॑भेत॒ ज्योग॑परुद्धो॒\-ऽनयो॒र्\mbox{}हि वा ए॒षो\-ऽप्र॑तिष्ठि॒तो\-ऽथै॒ष ज्योगप॑रुद्धो॒ द्यावा॑पृथि॒वी ए॒व स्वेन॑ भाग॒धेये॒नोप॑धावति॒ ते ए॒वैनं॑ प्रति॒ष्ठाङ्ग॑मयतः॒ प्रत्ये॒व ति॑ष्ठति पर्या॒रिणी॑ भवति पर्या॒रीव॒ ह्ये॑तस्य॑ रा॒ष्ट्रं यो ज्योग॑परुद्धः॒ समृ॑द्ध्यै वाय॒व्यम्᳚ (27)

%2.1.4.8
व॒थ्समा ल॑भेत वा॒युर्वा अ॒नयो᳚र्व॒थ्स इ॒मे वा ए॒तस्मै॑ लो॒का अप॑शुष्का॒ विडप॑शु॒ष्का\-ऽथै॒ष ज्योगप॑रुद्धो वा॒युमे॒व स्वेन॑ भाग॒धेये॒नोप॑धावति॒ स ए॒वास्मा॑ इ॒माल्लोँ॒काऩ् विशं॒ प्रदा॑पयति॒ प्रास्मा॑ इ॒मे लो॒काः स्नु॑वन्ति भुञ्ज॒त्ये॑नं॒ विडुप॑तिष्ठते॥ (28)

%2.1.5.0
{\anuvakamend[{म॒द्ध्यन्दि॑ने॒ कद्रुं॑ य॒मस्य॒ स्पर्द्ध॑मानो वैष्णावरु॒णीन्तमिन्द्रो᳚\-ऽस्य॒ स्वेन॑ भाग॒धेये॒नोप॑सृतो वाय॒व्यं॑ द्विच॑त्वारिꣳशच्च। (4)।}]}

%2.1.5.1
इन्द्रो॑ व॒लस्य॒ बिल॒मपौ᳚र्णो॒थ्स य उ॑त्त॒मः प॒शुरासी॒त्तं पृ॒ष्ठं प्रति॑ स॒ङ्गृह्योद॑क्खिद॒त्तꣳ स॒हस्रं॑ प॒शवो\-ऽनूदा॑य॒न्थ्स उ॑न्न॒तो॑\-ऽभव॒द्यः प॒शुका॑मः॒ स्याथ्स ए॒तमै॒न्द्रमु॑न्न॒तमाल॑भे॒तेन्द्र॑मे॒व स्वेन॑ भाग॒धेये॒नोप॑धावति॒ स ए॒वास्मै॑ प॒शून्प्रय॑च्छति पशु॒माने॒व भ॑वत्युन्न॒तः (29)

%2.1.5.2
भ॒व॒ति॒ सा॒ह॒स्री वा ए॒षा ल॒क्ष्मी यदु॑न्न॒तो ल॒क्ष्मियै॒व प॒शूनव॑रुन्धे य॒दा स॒हस्रं॑ प॒शून्प्रा᳚प्मु॒यादथ॑ वैष्ण॒वं वा॑म॒नमा ल॑भेतै॒तस्मि॒न्वै तथ्स॒हस्र॒मद्ध्य॑तिष्ठ॒त्तस्मा॑दे॒ष वा॑म॒नः समी॑षितः प॒शुभ्य॑ ए॒व प्रजा॑तेभ्यः प्रति॒ष्ठान्द॑धाति॒ को॑\-ऽर्\mbox{}हति स॒हस्रं॑ प॒शून्प्राप्तु॒मित्या॑हुरहोरा॒त्राण्ये॒व स॒हस्रꣳ॑ सं॒पाद्याल॑भेत प॒शवः॑ (30)

%2.1.5.3
वा अ॑होरा॒त्राणि॑ प॒शूने॒व प्रजा॑तान्प्रति॒ष्ठाङ्ग॑मय॒त्योष॑धीभ्यो वे॒हत॒माल॑भेत प्र॒जाका॑म॒ ओष॑धयो॒ वा ए॒तं प्र॒जायै॒ परि॑बाधन्ते॒ यो\-ऽलं॑ प्र॒जायै॒ सन्प्र॒जान्न वि॒न्दत॒ ओष॑धयः॒ खलु॒ वा ए॒तस्यै॒ सूतु॒मपि॑ घ्नन्ति॒ या वे॒हद्भव॒त्योष॑धीरे॒व स्वेन॑ भाग॒धेये॒नोप॑धावति॒ ता ए॒वास्मै॒ स्वाद्योनेः᳚ प्र॒जां प्रज॑नयन्ति वि॒न्दते᳚ (31)

%2.1.5.4
प्र॒जामापो॒ वा ओष॑ध॒यो\-ऽस॒त्पुरु॑ष॒ आप॑ ए॒वास्मा॒ अस॑तः॒ सद्द॑दति॒ तस्मा॑दाहु॒र्यश्चै॒वं वेद॒ यश्च॒ नाप॒स्त्वावास॑त॒ सद्द॑द॒तीत्यै॒न्द्रीꣳ सू॒तव॑शा॒माल॑भेत॒ भूति॑का॒मो\-ऽजा॑तो॒ वा ए॒ष यो\-ऽलं॒ भूत्यै॒ सन्भूति॒न्न प्रा॒प्नोतीन्द्रं॒ खलु॒ वा ए॒षा सू॒त्वा व॒शा\-ऽभ॑वत् (32)

%2.1.5.5
इन्द्र॑मे॒व स्वेन॑ भाग॒धेये॒नोप॑धावति॒ स ए॒वैनं॒ भूति॑ङ्गमयति॒ भव॑त्ये॒व यꣳ सू॒त्वा व॒शा स्यात्तमै॒न्द्रमे॒वाल॑भेतै॒तद्वाव तदि॑न्द्रि॒यꣳ सा॒क्षादे॒वेन्द्रि॒यमव॑रुन्ध ऐन्द्रा॒ग्नं पु॑नरुथ्सृ॒ष्टमाल॑भेत॒ य आ तृ॒तीया॒त्पुरु॑षा॒थ्सोम॒न्न पिबे॒द्विच्छि॑न्नो॒ वा ए॒तस्य॑ सोमपी॒थो यो ब्रा᳚ह्म॒णः सन्ना (33)

%2.1.5.6
तृ॒तीया॒त्पुरु॑षा॒थ्सोम॒न्न पिब॑तीन्द्रा॒ग्नी ए॒व स्वेन॑ भाग॒धेये॒नोप॑धावति॒ तावे॒वास्मै॑ सोमपी॒थं प्रय॑च्छत॒ उपै॑नꣳ सोमपी॒थो न॑मति॒ यदै॒न्द्रो भव॑तीन्द्रि॒यं वै सो॑मपी॒थ इ॑न्द्रि॒यमे॒व सो॑मपी॒थमव॑रुन्धे॒ यदा᳚ग्ने॒यो भव॑त्याग्ने॒यो वै ब्रा᳚ह्म॒णः स्वामे॒व दे॒वता॒मनु॒ सन्त॑नोति पुनरुथ्सृ॒ष्टो भ॑वति पुनरुथ्सृ॒ष्ट इ॑व॒ ह्ये॑तस्य॑ (34)

%2.1.5.7
सो॒म॒पी॒थः समृ॑द्ध्यै ब्राह्मणस्प॒त्यन्तू॑प॒रमाल॑भेताभि॒चर॒न्ब्रह्म॑ण॒स्पति॑मे॒व स्वेन॑ भाग॒धेये॒नोप॑धावति॒ तस्मा॑ ए॒वैन॒मा वृ॑श्चति ता॒जगार्ति॒मार्च्छ॑ति तूप॒रो भ॑वति क्षु॒रप॑वि॒र्वा ए॒षा ल॒क्ष्मी यत्तू॑प॒रः समृ॑द्ध्यै॒ स्फ्यो यूपो॑ भवति॒ वज्रो॒ वै स्फ्यो वज्र॑मे॒वास्मै॒ प्रह॑रति शर॒मयं॑ ब॒र्\mbox{}हिः शृ॒णात्ये॒वैनं॒ वैभी॑दक इ॒द्ध्मो भि॒नत्त्ये॒वैनम्᳚॥ (35)

%2.1.6.0
{\anuvakamend[{भ॒व॒त्यु॒न्न॒तः प॒शवो॑ जनयन्ति वि॒न्दते॑\-ऽभव॒थ्सन्नैतस्ये॒द्ध्मस्त्रीणि॑ च}]}%॥ (5)॥

%2.1.6.1
बा॒र्\mbox{}ह॒स्प॒त्यꣳ शि॑तिपृ॒ष्ठमाल॑भेत॒ ग्राम॑कामो॒ यः का॒मये॑त पृ॒ष्ठꣳ स॑मा॒नानाꣴ॑ स्या॒मिति॒ बृह॒स्पति॑मे॒व स्वेन॑ भाग॒धेये॒नोप॑धावति॒ स ए॒वैनं॑ पृ॒ष्ठꣳ स॑मा॒नाना᳚ङ्करोति ग्रा॒म्ये॑व भ॑वति शितिपृ॒ष्ठो भ॑वति बार्\mbox{}हस्प॒त्यो ह्ये॑ष दे॒वत॑या॒ समृ॑द्ध्यै पौ॒ष्णꣴ श्या॒ममाल॑भे॒तान्न॑का॒मो\-ऽन्नं॒ वै पू॒षा पू॒षण॑मे॒व स्वेन॑ भाग॒धेये॒नोप॑धावति॒ स ए॒वास्मै᳚ (36)

%2.1.6.2
अन्नं॒ प्रय॑च्छत्यन्ना॒द ए॒व भ॑वति श्या॒मो भ॑वत्ये॒तद्वा अन्न॑स्य रू॒पꣳ समृ॑द्ध्यै मारु॒तं पृश्ञि॒माल॑भे॒तान्न॑का॒मो\-ऽन्नं॒ वै म॒रुतो॑ म॒रुत॑ ए॒व स्वेन॑ भाग॒धेये॒नोप॑धावति॒ त ए॒वास्मा॒ अन्नं॒ प्रय॑च्छन्त्यन्ना॒द ए॒व भ॑वति॒ पृश्ञि॑र्भवत्ये॒तद्वा अन्न॑स्य रू॒पꣳ समृ॑द्ध्या ऐ॒न्द्रम॑रु॒णमाल॑भेतेन्द्रि॒यका॑म॒ इन्द्र॑मे॒व (37)

%2.1.6.3
स्वेन॑ भाग॒धेये॒नोप॑धावति॒ स ए॒वास्मि॑न्निन्द्रि॒यन्द॑धातीन्द्रिया॒व्ये॑व भ॑वत्यरु॒णो भ्रूमा᳚न्भवत्ये॒तद्वा इन्द्र॑स्य रू॒पꣳ समृ॑द्ध्यै सावि॒त्रमु॑पद्ध्व॒स्तमाल॑भेत स॒निका॑मः सवि॒ता वै प्र॑स॒वाना॑मीशे सवि॒तार॑मे॒व स्वेन॑ भाग॒धेये॒नोप॑धावति॒ स ए॒वास्मै॑ स॒निं प्रसु॑वति॒ दान॑कामा अस्मै प्र॒जा भ॑वन्त्युपद्ध्व॒स्तो भ॑वति सावि॒त्रो ह्ये॑षः (38)

%2.1.6.4
दे॒वत॑या॒ समृ॑द्ध्यै वैश्वदे॒वं ब॑हुरू॒पमाल॑भे॒तान्न॑कामो वैश्वदे॒वं वा अन्नं॒ विश्वा॑ने॒व दे॒वान्थ्स्वेन॑ भाग॒धेये॒नोप॑धावति॒ त ए॒वास्मा॒ अन्नं॒ प्रय॑च्छन्त्यन्ना॒द ए॒व भ॑वति बहुरू॒पो भ॑वति बहुरू॒पꣴ ह्यन्न॒ꣳ॒ समृ॑द्ध्यै वैश्वदे॒वं ब॑हुरू॒पमाल॑भेत॒ ग्राम॑कामो वैश्वदे॒वा वै स॑जा॒ता विश्वा॑ने॒व दे॒वान्थ्स्वेन॑ भाग॒धेये॒नोप॑धावति॒ त ए॒वास्मै᳚ (39)

%2.1.6.5
स॒जा॒तान्प्रय॑च्छन्ति ग्रा॒म्ये॑व भ॑वति बहुरू॒पो भ॑वति बहुदेव॒त्यो  ह्ये॑ष समृ॑द्ध्यै प्राजाप॒त्यन्तू॑प॒रमाल॑भेत॒ यस्याना᳚ज्ञातमिव॒ ज्योगा॒मये᳚त्प्राजाप॒त्यो वै पुरु॑षः प्र॒जाप॑तिः॒ खलु॒ वै तस्य॑ वेद॒ यस्याना᳚ज्ञातमिव॒ ज्योगा॒मय॑ति प्र॒जाप॑तिमे॒व स्वेन॑ भाग॒धेये॒नोप॑धावति॒ स ए॒वैन॒न्तस्मा॒थ्स्रामा᳚न्मुञ्चति तूप॒रो भ॑वति प्राजाप॒त्यो ह्ये॑ष दे॒वत॑या॒ समृ॑द्ध्यै॥ (40)

%2.1.7.0
{\anuvakamend[{अ॒स्मा॒ इन्द्र॑मे॒वैष स॑जा॒ता विश्वा॑ने॒व दे॒वान्थ्स्वेन॑ भाग॒धेये॒नोप॑धावति॒ त ए॒वास्मै᳚ प्राजाप॒त्यो हि त्रीणि॑ च}]}%॥ (6)॥

%2.1.7.1
व॒ष॒ट्का॒रो वै गा॑यत्रि॒यै शिरो᳚\-ऽच्छिन॒त्तस्यै॒ रसः॒ परा॑\-ऽपत॒त्तं बृह॒स्पति॒रुपा॑गृह्णा॒थ्सा शि॑तिपृ॒ष्ठा व॒शा\-ऽभ॑व॒द्यो द्वि॒तीयः॑ प॒राप॑त॒त्तं मि॒त्रावरु॑णा॒वुपा॑गृह्णीता॒ꣳ॒ सा द्वि॑रू॒पा व॒शा\-ऽभ॑व॒द्यस्तृ॒तीयः॑ प॒राप॑त॒त्तं विश्वे॑ दे॒वा उपा॑गृह्ण॒न्थ्सा ब॑हुरू॒पा व॒शा\-ऽभ॑व॒द्यश्च॑तु॒र्थः प॒राप॑त॒थ्स पृ॑थि॒वीं प्रावि॑श॒त्तं बृह॒स्पति॑र॒भि (41)

%2.1.7.2
अ॒गृ॒ह्णा॒दस्त्वे॒वायं भोगा॒येति॒ स उ॑क्षव॒शः सम॑भव॒द्यल्लोहि॑तं प॒राप॑त॒त्तद्रु॒द्र उपा॑गृह्णा॒थ्सा रौ॒द्री रोहि॑णी व॒शा\-ऽभ॑व\-द्बार्\mbox{}हस्प॒त्याꣳ शि॑तिपृ॒ष्ठामाल॑भेत ब्रह्मवर्च॒सका॑मो॒ बृह॒स्पति॑मे॒व स्वेन॑ भाग॒धेये॒नोप॑धावति॒ स ए॒वास्मि॑न्ब्रह्म\-वर्च॒सन्द॑धाति ब्रह्मवर्च॒स्ये॑व भ॑वति॒ छन्द॑सां॒ वा ए॒ष रसो॒ यद्व॒शा रस॑ इव॒ खलु॑ (42)

%2.1.7.3
वै ब्र॑ह्मवर्च॒सञ्छन्द॑सामे॒व रसे॑न॒ रसं॑ ब्रह्मवर्च॒समव॑रुन्धे मैत्रावरु॒णीं द्वि॑रू॒पामाल॑भेत॒ वृष्टि॑कामो मै॒त्रं वा अह॑र्वारु॒णी रात्रि॑रहोरा॒त्राभ्या॒ङ्खलु॒ वै प॒र्जन्यो॑ वर्\mbox{}षति मि॒त्रावरु॑णावे॒व स्वेन॑ भाग॒धेये॒नोप॑धावति॒ तावे॒वास्मा॑ अहोरा॒त्रा\-भ्यां᳚ प॒र्जन्यं॑ वर्\mbox{}षयत॒श्छन्द॑सां॒ वा ए॒ष रसो॒ यद्व॒शा रस॑ इव॒ खलु॒ वै वृष्टि॒श्छन्द॑सामे॒व रसे॑न (43)

%2.1.7.4
रसं॒ वृष्टि॒मव॑रुन्धे मैत्रावरु॒णीं द्वि॑रू॒पामाल॑भेत प्र॒जाका॑मो मै॒त्रं वा अह॑र्वारु॒णी रात्रि॑रहोरा॒त्राभ्या॒ङ्खलु॒ वै प्र॒जाः प्रजा॑यन्ते मि॒त्रावरु॑णावे॒व स्वेन॑ भाग॒धेये॒नोप॑धावति॒ तावे॒वास्मा॑ अहोरा॒त्रा\-भ्यां᳚ प्र॒जां प्रज॑नयत॒श्छन्द॑सां॒ वा ए॒ष रसो॒ यद्व॒शा रस॑ इव॒ खलु॒ वै प्र॒जा छन्द॑सामे॒व रसे॑न॒ रसं॑ प्र॒जामव॑ (44)

%2.1.7.5
रु॒न्धे॒ वै॒श्व॒दे॒वीं ब॑हुरू॒पामाल॑भे॒तान्न॑कामो वैश्वदे॒वं वा अन्नं॒ विश्वा॑ने॒व दे॒वान्थ्स्वेन॑ भाग॒धेये॒नोप॑धावति॒ त ए॒वास्मा॒ अन्नं॒ प्रय॑च्छन्त्यन्ना॒द ए॒व भ॑वति॒ छन्द॑सां॒ वा ए॒ष रसो॒ यद्व॒शा रस॑ इव॒ खलु॒ वा अन्न॒ञ्छन्द॑सामे॒व रसे॑न॒ रस॒मन्न॒मव॑रुन्धे वैश्वदे॒वीं ब॑हुरू॒पामाल॑भेत॒ ग्राम॑कामो वैश्वदे॒वा वै (45)

%2.1.7.6
स॒जा॒ता विश्वा॑ने॒व दे॒वान्थ्स्वेन॑ भाग॒धेये॒नोप॑धावति॒ त ए॒वास्मै॑ सजा॒तान्प्रय॑च्छन्ति ग्रा॒म्ये॑व भ॑वति॒ छन्द॑सां॒ वा ए॒ष रसो॒ यद्व॒शा रस॑ इव॒ खलु॒ वै स॑जा॒ताश्छन्द॑सामे॒व रसे॑न॒ रसꣳ॑ सजा॒तानव॑रुन्धे बार्\mbox{}हस्प॒त्यमु॑क्षव॒शमाल॑भेत ब्रह्मवर्च॒सका॑मो॒ बृह॒स्पति॑मे॒व स्वेन॑ भाग॒धेये॒नोप॑धाव॒ति स ए॒वास्मि॑न्ब्रह्मवर्च॒सम् (46)

%2.1.7.7
द॒धा॒ति॒ ब्र॒ह्म॒व॒र्च॒स्ये॑व भ॑वति॒ वशं॒ वा ए॒ष च॑रति॒ यदु॒क्षा वश॑ इव॒ खलु॒ वै ब्र॑ह्मवर्च॒सं वशे॑नै॒व वशं॑ ब्रह्मवर्च॒समव॑रुन्धे रौ॒द्रीꣳ रोहि॑णी॒माल॑भेताभि॒चर॑न्रु॒द्रमे॒व स्वेन॑ भाग॒धेये॒नोप॑धावति॒ तस्मा॑ ए॒वैन॒मावृ॑श्चति ता॒जगार्ति॒मार्च्छ॑ति॒ रोहि॑णी भवति रौ॒द्री ह्ये॑षा दे॒वत॑या॒ समृ॑द्ध्यै॒ स्फ्यो यूपो॑ भवति॒ वज्रो॒ वै स्फ्यो वज्र॑मे॒वास्मै॒ प्रह॑रति शर॒मयं॑ ब॒र्\mbox{}हिः शृ॒णात्ये॒वैनं॒ वैभी॑दक इ॒द्ध्मो भि॒नत्त्ये॒वैनम्᳚॥ (47)

%2.1.8.0
{\anuvakamend[{अ॒भि खलु॒ वृष्टि॒श्छन्द॑सामे॒व रसे॑न प्र॒जामव॑ वैश्वदे॒वा वै ब्र॑ह्मवर्च॒सं यूप॒ एका॒न्नविꣳ॑श॒तिश्च॑। (7)।}]}

%2.1.8.1
अ॒सावा॑दि॒त्यो न व्य॑रोचत॒ तस्मै॑ दे॒वाः प्राय॑श्चित्तिमैच्छ॒न्तस्मा॑ ए॒ताꣳ सौ॒रीꣴ श्वे॒तां व॒शामा\-ऽल॑भन्त॒ तयै॒वास्मि॒न्रुच॑मदधु॒र्यो ब्र॑ह्मवर्च॒सका॑मः॒ स्यात्तस्मा॑ ए॒ताꣳ सौ॒रीꣴ श्वे॒तां व॒शामाल॑भेता॒मुमे॒वादि॒त्यꣴ स्वेन॑ भाग॒धेये॒नोप॑धावति॒ स ए॒वास्मि॑न्ब्रह्मवर्च॒सन्द॑धाति ब्रह्मवर्च॒स्ये॑व भ॑वति बै॒ल्\mbox{}वो यूपो॑ भवत्य॒सौ (48)

%2.1.8.2
वा आ॑दि॒त्यो यतो\-ऽजा॑यत॒ ततो॑ बि॒ल्व॑ उद॑तिष्ठ॒थ्सयो᳚न्ये॒व ब्र॑ह्मवर्च॒समव॑रुन्धे ब्राह्मणस्प॒त्यां ब॑भ्रुक॒र्णीमा ल॑भेताभि॒चर॑न्वारु॒णन्दश॑कपालं पु॒रस्ता॒न्निर्व॑पे॒द्वरु॑णेनै॒व भ्रातृ॑व्यङ्ग्राहयि॒त्वा ब्रह्म॑णा स्तृणुते बभ्रुक॒र्णी भ॑वत्ये॒तद्वै ब्रह्म॑णो रू॒पꣳ समृ॑द्ध्यै॒ स्फ्यो यूपो॑ भवति॒ वज्रो॒ वै स्फ्यो वज्र॑मे॒वास्मै॒ प्रह॑रति शर॒मयं॑ ब॒र्\mbox{}हिः शृ॒णाति॑ (49)

%2.1.8.3
ए॒वैनं॒ वैभी॑दक इ॒द्ध्मो भि॒नत्त्ये॒वैनं॑ वैष्ण॒वं वा॑म॒नमाल॑भेत॒ यं य॒ज्ञो नोप॒नमे॒द्विष्णु॒र्वै य॒ज्ञो विष्णु॑मे॒व स्वेन॑ भाग॒धेये॒नोप॑धावति॒ स ए॒वास्मै॑ य॒ज्ञं प्रय॑च्छ॒त्युपै॑नं य॒ज्ञो न॑मति वाम॒नो भ॑वति वैष्ण॒वो ह्ये॑ष दे॒वत॑या॒ समृ॑द्ध्यै त्वा॒ष्ट्रं व॑ड॒बमाल॑भेत प॒शुका॑म॒स्त्वष्टा॒ वै प॑शू॒नां मि॑थु॒नाना᳚म् (50)

%2.1.8.4
प्र॒ज॒न॒यि॒ता त्वष्टा॑रमे॒व स्वेन॑ भाग॒धेये॒नोप॑धावति॒ स ए॒वास्मै॑ प॒शून्मि॑थु॒नान्प्रज॑नयति प्र॒जा हि वा ए॒तस्मि॑न्प॒शवः॒ प्रवि॑ष्टा॒ अथै॒ष पुमा॒न्थ्सन्व॑ड॒बः सा॒क्षादे॒व प्र॒जां प॒शूनव॑रुन्धे मै॒त्रꣴ श्वे॒तमाल॑भेत सङ्ग्रा॒मे सं य॑त्ते सम॒यका॑मो मि॒त्रमे॒व स्वेन॑ भाग॒धेये॒नोप॑धावति॒ स ए॒वैनं॑ मि॒त्रेण॒ सन्न॑यति (51)

%2.1.8.5
वि॒शा॒लो भ॑वति॒ व्यव॑साययत्ये॒वैनं॑ प्राजाप॒त्यं कृ॒ष्णमाल॑भेत॒ वृष्टि॑कामः प्र॒जाप॑ति॒र्वै वृष्ट्या॑ ईशे प्र॒जाप॑तिमे॒व स्वेन॑ भाग॒धेये॒नोप॑धावति॒ स ए॒वास्मै॑ प॒र्जन्यं॑ वर्\mbox{}षयति कृ॒ष्णो भ॑वत्ये॒तद्वै वृष्ट्यै॑ रू॒पꣳ रू॒पेणै॒व वृष्टि॒मव॑रुन्धे श॒बलो॑ भवति वि॒द्युत॑मे॒वास्मै॑ जनयि॒त्वा व॑र्\mbox{}षयत्यवाशृ॒ङ्गो भ॑वति॒ वृष्टि॑मे॒वास्मै॒ निय॑च्छति॥ (52)

%2.1.9.0
{\anuvakamend{श़ृ॒णाति॑ मिथु॒नाना᳚न्नयति यच्छति॥]}}

%2.1.9.1
वरु॑णꣳ सुषुवा॒णम॒न्नाद्य॒न्नोपा॑नम॒थ्स ए॒तां वा॑रु॒णीं कृ॒ष्णां व॒शाम॑पश्य॒त्ताꣴ स्वायै॑ दे॒वता॑या॒ आ\-ऽल॑भत॒ ततो॒ वै तम॒न्नाद्य॒मुपा॑नम॒द्यमल॑म॒न्नाद्या॑य॒ सन्त॑म॒न्नाद्य॒न्नोप॒नमे॒थ्स ए॒तां वा॑रु॒णीं कृ॒ष्णां व॒शामाल॑भेत॒ वरु॑णमे॒व स्वेन॑ भाग॒धेये॒नोप॑धावति॒ स ए॒वास्मा॒ अन्नं॒ प्रय॑च्छत्यन्ना॒दः (53)

%2.1.9.2
ए॒व भ॑वति कृ॒ष्णा भ॑वति वारु॒णी ह्ये॑षा दे॒वत॑या॒ समृ॑द्ध्यै मै॒त्रꣴ श्वे॒तमाल॑भेत वारु॒णं कृ॒ष्णम॒पा़ञ्चौष॑धीनाञ्च स॒न्धावन्न॑कामो मै॒त्रीर्वा ओष॑धयो वारु॒णीरापो॒\-ऽपां च॒ खलु॒ वा ओष॑धीनां च॒ रस॒मुप॑जीवामो मि॒त्रावरु॑णावे॒व स्वेन॑ भाग॒धेये॒नोप॑धावति॒ तावे॒वास्मा॒ अन्नं॒ प्रय॑च्छतो\-ऽन्ना॒द ए॒व भ॑वति (54)

%2.1.9.3
अ॒पाञ्चौष॑धीनाञ्च स॒न्धावाल॑भत उ॒भय॒स्याव॑रुद्ध्यै॒ विशा॑खो॒ यूपो॑ भवति॒ द्वे ह्ये॑ते दे॒वते॒ समृ॑द्ध्यै मै॒त्रꣴ श्वे॒तमा ल॑भेत वारु॒णं कृ॒ष्णं ज्योगा॑मयावी॒ यन्मै॒त्रो भव॑ति मि॒त्रेणै॒वास्मै॒ वरु॑णꣳ शमयति॒ यद्वा॑रु॒णः सा॒क्षादे॒वैनं॑ वरुण\-पा॒शान्मु॑ञ्चत्यु॒त यदी॒तासु॒र्भव॑ति॒ जीव॑त्ये॒व दे॒वा वै पुष्टि॒न्नावि॑न्दन्न् (55)

%2.1.9.4
तां मि॑थु॒ने॑\-ऽपश्य॒न्तस्या॒न्न सम॑राधय॒न्ताव॒श्विना॑वब्रूतामा॒वयो॒र्वा ए॒षा मैतस्या᳚ं वदद्ध्व॒मिति॒ सा\-ऽश्विनो॑रे॒वाभ॑व॒द्यः पुष्टि॑कामः॒ स्याथ्स ए॒तामा᳚श्वि॒नीं य॒मीं व॒शामाल॑भेता॒श्विना॑वे॒व स्वेन॑ भाग॒धेये॒नोप॑धावति॒ तावे॒वास्मि॒न्पुष्टिं॑ धत्तः॒ पुष्य॑ति प्र॒जया॑ प॒शुभिः॑॥ (56)

%2.1.10.0
{\anuvakamend[{अ॒न्ना॒दो᳚\-ऽन्ना॒द ए॒व भ॑वत्यविन्द॒न्पञ्च॑चत्वारिꣳशच्च।9।}]}

%2.1.10.1
आ॒श्वि॒नन्धू॒म्रल॑लाम॒माल॑भेत॒ यो दुर्ब्रा᳚ह्मणः॒ सोमं॒ पिपा॑सेद॒श्विनौ॒ वै दे॒वाना॒मसो॑मपावास्ता॒न्तौ प॒श्चा सो॑मपी॒थं प्राप्नु॑ताम॒श्विना॑वे॒तस्य॑ दे॒वता॒ यो दुर्ब्रा᳚ह्मणः॒ सोमं॒ पिपा॑सत्य॒श्विना॑वे॒व स्वेन॑ भाग॒धेये॒नोप॑धावति॒ तावे॒वास्मै॑ सोमपी॒थं प्रय॑च्छत॒ उपै॑नꣳ सोमपी॒थो न॑मति॒ यद्धू॒म्रो भव॑ति धूम्रि॒माण॑मे॒वास्मा॒दप॑हन्ति ल॒लामः॑ (57)

%2.1.10.2
भ॒व॒ति॒ मु॒ख॒त ए॒वास्मि॒न्तेजो॑ दधाति वाय॒व्य॑ङ्गोमृ॒गमाल॑भेत॒ यमज॑घ्निवाꣳसमभि॒शꣳसे॑यु॒रपू॑ता॒ वा ए॒तं वागृ॑च्छति॒ यमज॑घ्निवाꣳसमभि॒शꣳस॑न्ति॒ नैष ग्रा॒म्यः प॒शुर्नार॒ण्यो यद्गो॑मृ॒गो नेवै॒ष ग्रामे॒ नार॑ण्ये॒ यमज॑घ्निवाꣳसमभि॒शꣳस॑न्ति वा॒युर्वै दे॒वानां᳚ प॒वित्रं॑ वा॒युमे॒व स्वेन॑ भाग॒धेये॒नोप॑धावति॒ स ए॒व (58)

%2.1.10.3
ए॒नं॒ प॒व॒य॒ति॒ परा॑ची॒ वा ए॒तस्मै᳚ व्यु॒च्छन्ती॒ व्यु॑च्छति॒ तमः॑ पा॒प्मानं॒ प्रवि॑शति॒ यस्या᳚श्वि॒ने श॒स्यमा॑ने॒ सूर्यो॒ नाविर्भव॑ति सौ॒र्यं ब॑हुरू॒पमाल॑भेता॒मुमे॒वादि॒त्यꣴ स्वेन॑ भाग॒धेये॒नोप॑धावति॒ स ए॒वास्मा॒त्तमः॑ पा॒प्मान॒मप॑हन्ति प्र॒तीच्य॑स्मै व्यु॒च्छन्ती॒ व्यु॑च्छ॒त्यप॒ तमः॑ पा॒प्मानꣳ॑ हते॥ (59)

%2.1.11.0
{\anuvakamend[{ल॒लामः॒ स ए॒व षट्च॑त्वारिꣳशच्च}]}%॥10॥

%2.1.11.1
इन्द्रं॑ वो वि॒श्वत॒स्परीन्द्र॒न्नरो॒ मरु॑तो॒ यद्ध॑ वो दि॒वो या वः॒ शर्म॑। भरे॒ष्विन्द्रꣳ॑ सु॒हवꣳ॑ हवामहे\-ऽꣳहो॒मुचꣳ॑ सु॒कृतं॒ दैव्यं॒ जनम्᳚। अ॒ग्निं मि॒त्रं वरु॑णꣳ सा॒तये॒ भग॒न्द्यावा॑पृथि॒वी म॒रुतः॑ स्व॒स्तये᳚। म॒मत्तु॑ नः॒ परि॑ज्मा वस॒र्\mbox{}हा म॒मत्तु॒ वातो॑ अ॒पां वृष॑ण्वान्। शि॒शी॒तमि॑न्द्रापर्वता यु॒वन्न॒स्तन्नो॒ विश्वे॑ वरिवस्यन्तु दे॒वाः। प्रि॒या वो॒ नाम॑ (60)

%2.1.11.2
हु॒वे॒ तु॒राणा᳚म्। आयत्तृ॒पन्म॑रुतो वावशा॒नाः। श्रि॒यसे॒ कं भा॒नुभिः॒ सम्मि॑मिक्षिरे॒ ते र॒श्मिभि॒स्त ऋक्व॑भिः सुखा॒दयः॑। ते वाशी॑मन्त इ॒ष्मिणो॒ अभी॑रवो वि॒द्रे प्रि॒यस्य॒ मारु॑तस्य॒ धाम्नः॑। अ॒ग्निः प्र॑थ॒मो वसु॑भिर्नो अव्या॒थ्सोमो॑ रु॒द्रेभि॑र॒भिर॑क्षतु॒ त्मना᳚। इन्द्रो॑ म॒रुद्भि॑र्\mbox{}ऋतु॒धा कृ॑णोत्वादि॒त्यैर्नो॒ वरु॑णः॒ सꣳशि॑शातु। सन्नो॑ दे॒वो वसु॑भिर॒ग्निः सम् (61)

%2.1.11.3
सोम॑स्त॒नूभी॑ रु॒द्रिया॑भिः। समिन्द्रो॑ म॒रुद्भि॑र्य॒ज्ञियैः॒ समा॑दि॒त्यैर्नो॒ वरु॑णो अजिज्ञिपत्। यथा॑\-ऽ\-ऽदि॒त्या वसु॑भिः सम्बभू॒वुर्म॒रुद्भी॑ रु॒द्राः स॒मजा॑नता॒भि। ए॒वा त्रि॑णाम॒न्नहृ॑णीयमाना॒ विश्वे॑ दे॒वाः सम॑नसो भवन्तु। कुत्रा॑चि॒द्यस्य॒ समृ॑तौ र॒ण्वा नरो॑ नृ॒षद॑ने। अर्\mbox{}ह॑न्तश्चि॒द्यमि॑न्ध॒ते स॑ञ्ज॒नय॑न्ति ज॒न्तवः॑। सं यदि॒षो वना॑महे॒ सꣳ ह॒व्या मानु॑षाणाम्। उ॒त द्यु॒म्नस्य॒ शव॑सः (62)

%2.1.11.4
ऋ॒तस्य॑ र॒श्मिमाद॑दे। य॒ज्ञो दे॒वानां॒ प्रत्ये॑ति सु॒म्नमादि॑त्यासो॒ भव॑ता मृड॒यन्तः॑। आवो॒\-ऽर्वाची॑ सुम॒तिर्व॑वृत्या\-द॒ꣳ॒होश्चि॒द्या व॑रिवो॒वित्त॒रा\-ऽस॑त्। शुचि॑र॒पः सू॒यव॑सा॒ अद॑ब्ध॒ उप॑क्षेति वृ॒द्धव॑याः सु॒वीरः॑। नकि॒ष्टं घ्न॒न्त्यन्ति॑तो॒ न दू॒राद्य आ॑दि॒त्यानां॒ भव॑ति॒ प्रणी॑तौ। धा॒रय॑न्त आदि॒त्यासो॒ जग॒थ्स्था दे॒वा विश्व॑स्य॒ भुव॑नस्य गो॒पाः। दी॒र्घाधि॑यो॒ रक्ष॑माणाः (63)

%2.1.11.5
अ॒सु॒र्य॑मृ॒तावा॑न॒श्चय॑माना ऋ॒णानि॑। ति॒स्रो भूमी᳚र्धारय॒न्त्रीꣳ रु॒त द्यून्त्रीणि॑ व्र॒ता वि॒दथे॑ अ॒न्तरे॑षाम्। ऋ॒तेना॑दित्या॒ महि॑ वो महि॒त्वन्तद॑र्यमन्वरुण मित्र॒ चारु॑। त्यान्नु क्ष॒त्रिया॒ꣳ॒ अव॑ आदि॒त्यान् या॑चिषामहे। सु॒मृ॒डी॒काꣳ अ॒भिष्ट॑ये। न द॑क्षि॒णा विचि॑किते॒ न स॒व्या न प्रा॒चीन॑मादित्या॒ नोत प॒श्चा। पा॒क्या॑चिद्वसवो धी॒र्या॑चित् (64)

%2.1.11.6
यु॒ष्मानी॑तो॒ अभ॑यं॒ ज्योति॑रश्याम्। आ॒दि॒त्याना॒मव॑सा॒ नूत॑नेन सक्षी॒महि॒ शर्म॑णा॒ शन्त॑मेन। अ॒ना॒गा॒स्त्वे अ॑दिति॒त्वे तु॒रास॑ इ॒मं य॒ज्ञन्द॑धतु॒ श्रोष॑माणाः। इ॒मं मे॑ वरुण श्रुधी॒ हव॑म॒द्या च॑ मृडय। त्वाम॑व॒स्युराच॑के। तत्त्वा॑ यामि॒ ब्रह्म॑णा॒ वन्द॑मान॒स्तदाशा᳚स्ते॒ यज॑मानो ह॒विर्भिः॑। अहे॑डमानो वरुणे॒ह बो॒द्ध्युरु॑शꣳस॒ मा न॒ आयुः॒ प्रमो॑षीः॥ (65)

%2.2.0.0

{\anuvakamend[{नामा॒ग्निः सꣳ शव॑सो॒ रक्ष॑माणा धी॒र्या॑चि॒देका॒न्नप॑ञ्चा॒शच्च॑}]}%॥11॥
%%% END PRASHNA

\sect{द्वितीयः प्रश्नः}\setcounter{anuvakam}{0}
\dnsub{तैत्तिरीयसंहितायां द्वितीयकाण्डे द्वितीयः प्रश्नः}
%2.2.1.0
%2.2.1.1
प्र॒जाप॑तिः प्र॒जा अ॑सृजत॒ ताः सृ॒ष्टा इ॑न्द्रा॒ग्नी अपा॑गूहता॒ꣳ॒ सो॑\-ऽचायत्प्र॒जाप॑तिरिन्द्रा॒ग्नी वै मे᳚ प्र॒जा अपा॑घुक्षता॒मिति॒ स ए॒तमै᳚न्द्रा॒ग्नमेका॑दशकपालमपश्य॒त्तन्निर॑वप॒त्ताव॑स्मै प्र॒जाः प्रासा॑धयतामिन्द्रा॒ग्नी वा ए॒तस्य॑ प्र॒जामप॑गूहतो॒ यो\-ऽलं॑ प्र॒जायै॒ सन्प्र॒जान्न वि॒न्दत॑ ऐन्द्रा॒ग्नमेका॑दशकपाल॒न्निर्व॑पेत्प्र॒जाका॑म इन्द्रा॒ग्नी (1)

%2.2.1.2
ए॒व स्वेन॑ भाग॒धेये॒नोप॑धावति॒ तावे॒वास्मै᳚ प्र॒जां प्रसा॑धयतो वि॒न्दते᳚ प्र॒जामै᳚न्द्रा॒ग्नमेका॑दशकपाल॒न्निर्व॑पे॒थ्स्पर्द्ध॑मानः॒ क्षेत्रे॑ वा सजा॒तेषु॑ वेन्द्रा॒ग्नी ए॒व स्वेन॑ भाग॒धेये॒नोप॑धावति॒ ताभ्या॑मे॒वेन्द्रि॒यं वी॒र्यं॑ भ्रातृ॑व्यस्य वृङ्क्ते॒ वि पा॒प्मना॒ भ्रातृ॑व्येण जय॒ते\-ऽप॒ वा ए॒तस्मा॑दिन्द्रि॒यं वी॒र्यं॑ क्रामति॒ यः स॑ङ्ग्रा॒ममु॑पप्र॒यात्यै᳚न्द्रा॒ग्नमेका॑दशकपाल॒न्निः (2)

%2.2.1.3
व॒पे॒थ्स॒ङ्ग्रा॒ममु॑पप्रया॒स्यन्नि॑न्द्रा॒ग्नी ए॒व स्वेन॑ भाग॒धेये॒नोप॑धावति॒ तावे॒वास्मि॑न्निन्द्रि॒यं वी॒र्यं॑ धत्तः स॒हेन्द्रि॒येण॑ वी॒र्ये॑णोप॒प्रया॑ति॒ जय॑ति॒ तꣳ स॑ङ्ग्रा॒मं वि वा ए॒ष इ॑न्द्रि॒येण॑ वी॒र्ये॑णर्द्ध्यते॒ यः स॑ङ्ग्रा॒मञ्जय॑त्यैन्द्रा॒ग्नमेका॑दश\-कपाल॒न्निर्व॑पेथ्सङ्ग्रा॒मञ्जि॒त्वेन्द्रा॒ग्नी ए॒व स्वेन॑ भाग॒धेये॒नोप॑धावति॒ तावे॒वास्मि॑न्निन्द्रि॒यं वी॒र्यम्᳚ (3)

%2.2.1.4
ध॒त्तो॒ नेन्द्रि॒येण॑ वी॒र्ये॑ण॒ व्यृ॑द्ध्य॒ते\-ऽप॒ वा ए॒तस्मा॑दिन्द्रि॒यं वी॒र्य॑ङ्क्रामति॒ य एति॑ ज॒नता॑मैन्द्रा॒ग्नमेका॑दशकपाल॒न्निर्व॑पेज्ज॒नता॑मे॒ष्यन्नि॑न्द्रा॒ग्नी ए॒व स्वेन॑ भाग॒धेये॒नोप॑धावति॒ तावे॒वास्मि॑न्निन्द्रि॒यं वी॒र्यं॑ धत्तः स॒हेन्द्रि॒येण॑ वी॒र्ये॑ण ज॒नता॑मेति पौ॒ष्णं च॒रुमनु॒निर्व॑पेत्पू॒षा वा इ॑न्द्रि॒यस्य॑ वी॒र्य॑स्यानुप्रदा॒ता पू॒षण॑मे॒व (4)

%2.2.1.5
स्वेन॑ भाग॒धेये॒नोप॑धावति॒ स ए॒वास्मा॑ इन्द्रि॒यं वी॒र्य॑मनु॒ प्रय॑च्छति क्षैत्रप॒त्यं च॒रुं निर्व॑पेज्ज॒नता॑मा॒गत्ये॒यं वै क्षेत्र॑स्य॒ पति॑र॒स्यामे॒व प्रति॑तिष्ठत्यैन्द्रा॒ग्नमेका॑दशकपालमु॒परि॑ष्टा॒न्निर्व॑पेद॒स्यामे॒व प्र॑ति॒ष्ठाये᳚न्द्रि॒यं वी॒र्य॑मु॒परि॑ष्टादा॒त्मन्ध॑त्ते॥ (5)

%2.2.2.0
{\anuvakamend[{प्र॒जाका॑म इन्द्रा॒ग्नी उ॑पप्र॒यात्यै᳚न्द्रा॒ग्नमेका॑दशकपाल॒न्निर्वी॒र्यं॑ पू॒षण॑मे॒वैका॒न्नच॑त्वारि॒ꣳ॒शच्च॑॥1।}]}

%2.2.2.1
अ॒ग्नये॑ पथि॒कृते॑ पुरो॒डाश॑म॒ष्टाक॑पाल॒न्निर्व॑पे॒द्यो द॑र्\mbox{}शपूर्णमासया॒जी सन्न॑मावा॒स्या᳚ं वा पौर्णमा॒सीं वा॑\-ऽतिपा॒दये᳚त्प॒थो वा ए॒षो\-ऽद्ध्यप॑थेनैति॒ यो द॑र्\mbox{}शपूर्णमासया॒जी सन्न॑मावा॒स्या᳚ं वा पौर्णमा॒सीं वा॑\-ऽतिपा॒दय॑त्य॒ग्निमे॒व प॑थि॒कृत॒ꣴ॒ स्वेन॑ भाग॒धेये॒नोप॑धावति॒ स ए॒वैन॒मप॑था॒त्पन्था॒मपि॑ नयत्यन॒ड्वान्दक्षि॑णा व॒ही ह्ये॑ष समृ॑द्ध्या अ॒ग्नये᳚ व्र॒तप॑तये (6)

%2.2.2.2
पु॒रो॒डाश॑म॒ष्टाक॑पाल॒न्निर्व॑पे॒द्य आहि॑ताग्निः॒ सन्न॑व्र॒त्यमि॑व॒ चरे॑द॒ग्निमे॒व व्र॒तप॑ति॒ꣴ॒ स्वेन॑ भाग॒धेये॒नोप॑धावति॒ स ए॒वैन॑व्व्रँ॒तमालं॑भयति॒ व्रत्यो॑ भवत्य॒ग्नये॑ रक्षो॒घ्ने पु॑रो॒डाश॑म॒ष्टाक॑पाल॒न्निर्व॑पे॒द्यꣳ रक्षाꣳ॑सि॒ सचे॑रन्न॒ग्निमे॒व र॑क्षो॒हण॒ꣴ॒ स्वेन॑ भाग॒धेये॒नोप॑धावति॒ स ए॒वास्मा॒द्रक्षा॒ꣳ॒स्यप॑हन्ति॒ निशि॑ताया॒न्निर्व॑पेत् (7)

%2.2.2.3
निशि॑ताया॒ꣳ॒ हि रक्षाꣳ॑सि प्रे॒रते॑ सं॒प्रेर्णा᳚न्ये॒वैना॑नि हन्ति॒ परि॑श्रिते याजये॒द्रक्ष॑सा॒मन॑न्ववचाराय रक्षो॒घ्नी या᳚ज्यानुवा॒क्ये॑ भवतो॒ रक्ष॑सा॒ꣴ॒ स्तृत्या॑ अ॒ग्नये॑ रु॒द्रव॑ते पुरो॒डाश॑म॒ष्टाक॑पाल॒न्निर्व॑पेदभि॒चर॑न्ने॒षा वा अ॑स्य घो॒रा त॒नूर्यद्रु॒द्रस्तस्मा॑ ए॒वैन॒मावृ॑श्चति ता॒जगार्ति॒मार्च्छ॑त्य॒ग्नये॑ सुरभि॒मते॑ पुरो॒डाश॑म॒ष्टाक॑पाल॒न्निर्व॑पे॒द्यस्य॒ गावो॑ वा॒ पुरु॑षाः (8)

%2.2.2.4
वा॒ प्र॒मीये॑र॒न्॒ यो वा॑ बिभी॒यादे॒षा वा अ॑स्य भेष॒ज्या॑ त॒नूर्यथ्सु॑रभि॒मती॒ तयै॒वास्मै॑ भेष॒जङ्क॑रोति सुरभि॒मते॑ भवति पूतीग॒न्धस्याप॑हत्या अ॒ग्नये॒ क्षाम॑वते पुरो॒डाश॑म॒ष्टाक॑पाल॒न्निर्व॑पेथ्सङ्ग्रा॒मे सं य॑त्ते भाग॒धेये॑नै॒वैनꣳ॑ शमयि॒त्वा परा॑न॒भि निर्दि॑शति॒ यमव॑रेषां॒ विद्ध्य॑न्ति॒ जीव॑ति॒ स यं परे॑षां॒ प्र स मी॑यते॒ जय॑ति॒ तꣳ स॑ङ्ग्रा॒मम् (9)

%2.2.2.5
अ॒भि वा ए॒ष ए॒तानु॑च्यति॒ येषां᳚ पूर्वाप॒रा अ॒न्वञ्चः॑ प्र॒मीय॑न्ते पुरुषाहु॒तिर्ह्य॑स्य प्रि॒यत॑मा॒\-ऽग्नये॒ क्षाम॑वते पुरो॒डाश॑म॒ष्टाक॑पाल॒न्निर्व॑पेद्भाग॒धेये॑नै॒वैनꣳ॑ शमयति॒ नैषां᳚ पु॒रा\-ऽ\-ऽयु॒षो\-ऽप॑रः॒ प्रमी॑यते॒\-ऽभि वा ए॒ष ए॒तस्य॑ गृ॒हानु॑च्यति॒ यस्य॑ गृ॒हान्दह॑त्य॒ग्नये॒ क्षाम॑वते पुरो॒डाश॑म॒ष्टाक॑पाल॒न्निर्व॑पेद्भाग॒धेये॑नै॒वैनꣳ॑ शमयति॒ नास्याप॑रङ्गृ॒हान्द॑हति॥ (10)

%2.2.3.0
{\anuvakamend[{व्र॒तप॑तये॒ निशि॑ताया॒न्निर्व॑पे॒त्पुरु॑षाः सङ्ग्रा॒मन्न च॒त्वारि॑ च}]}%॥२॥

%2.2.3.1
अ॒ग्नये॒ कामा॑य पुरो॒डाश॑म॒ष्टाक॑पाल॒न्निर्व॑पे॒द्यङ्कामो॒ नोप॒नमे॑द॒ग्निमे॒व काम॒ꣴ॒ स्वेन॑ भाग॒धेये॒नोप॑ धावति॒ स ए॒वैन॒ङ्कामे॑न॒ सम॑र्द्धय॒त्युपै॑न॒ङ्कामो॑ नमत्य॒ग्नये॒ यवि॑ष्ठाय पुरो॒डाश॑म॒ष्टाक॑पाल॒न्निर्व॑पे॒थ्स्पर्द्ध॑मानः॒ क्षेत्रे॑ वा सजा॒तेषु॑ वा॒\-ऽग्निमे॒व यवि॑ष्ठ॒ꣴ॒ स्वेन॑ भाग॒धेये॒नोप॑धावति॒ तेनै॒वेन्द्रि॒यं वी॒र्यं॑ भ्रातृ॑व्यस्य (11)

%2.2.3.2
यु॒व॒ते॒ वि पा॒प्मना॒ भ्रातृ॑व्येण जयते॒\-ऽग्नये॒ यवि॑ष्ठाय पुरो॒डाश॑म॒ष्टाक॑पाल॒न्निर्व॑पेदभिच॒र्यमा॑णो॒\-ऽग्निमे॒व यवि॑ष्ठ॒ꣴ॒ स्वेन॑ भाग॒धेये॒नोप॑धावति॒ स ए॒वास्मा॒द्रक्षाꣳ॑सि यवयति॒ नैन॑मभि॒चर᳚न्थ्स्तृणुते॒\-ऽग्नय॒ आयु॑ष्मते पुरो॒डाश॑म॒ष्टाक॑पाल॒न्निर्व॑पे॒द्यः का॒मये॑त॒ सर्व॒मायु॑रिया॒मित्य॒ग्निमे॒वायु॑ष्मन्त॒ꣴ॒ स्वेन॑ भाग॒धेये॒नोप॑धावति॒ स ए॒वास्मिन्न्॑ (12)

%2.2.3.3
आयु॑र्दधाति॒ सर्व॒मायु॑रेत्य॒ग्नये॑ जा॒तवे॑दसे पुरो॒डाश॑म॒ष्टाक॑पाल॒न्निर्व॑पे॒द्भूति॑कामो॒\-ऽग्निमे॒व जा॒तवे॑दस॒ꣴ॒ स्वेन॑ भाग॒धेये॒नोप॑धावति॒ स ए॒वैनं॒ भूति॑ङ्गमयति॒ भव॑त्ये॒वाग्नये॒ रुक्म॑ते पुरो॒डाश॑म॒ष्टाक॑पाल॒न्निर्व॑पे॒द्रुक्का॑मो॒\-ऽग्निमे॒व रुक्म॑न्त॒ꣴ॒ स्वेन॑ भाग॒धेये॒नोप॑धावति॒ स ए॒वास्मि॒न्रुच॑न्दधाति॒ रोच॑त ए॒वाग्नये॒ तेज॑स्वते पुरो॒डाशम्᳚ (13)

%2.2.3.4
अ॒ष्टाक॑पाल॒न्निर्व॑पे॒त्तेज॑स्कामो॒\-ऽग्निमे॒व तेज॑स्वन्त॒ꣴ॒ स्वेन॑ भाग॒धेये॒नोप॑धावति॒ स ए॒वास्मि॒न्तेजो॑ दधाति तेज॒स्व्ये॑व भ॑वत्य॒ग्नये॑ साह॒न्त्याय॑ पुरो॒डाश॑म॒ष्टाक॑पाल॒न्निर्व॑पे॒थ्सीक्ष॑माणो॒\-ऽग्निमे॒व सा॑ह॒न्त्यꣴ स्वेन॑ भाग॒धेये॒नोप॑धावति॒ तेनै॒व स॑हते॒ यꣳ सीक्ष॑ते॥ (14)

%2.2.4.0
{\anuvakamend[{भ्रातृ॑व्यस्यास्मि॒न्तेज॑स्वते पुरो॒डाश॑म॒ष्टात्रिꣳ॑शच्च}]}%॥३॥

%2.2.4.1
अ॒ग्नये\-ऽन्न॑वते पुरो॒डाश॑म॒ष्टाक॑पाल॒न्निर्व॑पे॒द्यः का॒मये॒तान्न॑वान्थ्स्या॒मित्य॒ग्निमे॒वान्न॑वन्त॒ꣴ॒ स्वेन॑ भाग॒धेये॒नोप॑धावति॒ स ए॒वैन॒मन्न॑वन्तङ्करो॒त्यन्न॑वाने॒व भ॑वत्य॒ग्नये᳚\-ऽन्ना॒दाय॑ पुरो॒डाश॑म॒ष्टाक॑पाल॒ न्निर्व॑पे॒द्यः का॒मये॑तान्ना॒दः स्या॒मित्य॒ग्निमे॒वान्ना॒दꣴ स्वेन॑ भाग॒धेये॒नोप॑धावति॒ स ए॒वैन॑मन्ना॒दङ्क॑रोत्यन्ना॒दः (15)

%2.2.4.2
ए॒व भ॑वत्य॒ग्नये\-ऽन्न॑पतये पुरो॒डाश॑म॒ष्टाक॑पाल॒न्निर्व॑पे॒द्यः का॒मये॒तान्न॑पतिः स्या॒मित्य॒ग्निमे॒वान्न॑पति॒ꣴ॒ स्वेन॑ भाग॒धेये॒नोप॑धावति॒ स ए॒वैन॒मन्न॑पतिङ्करो॒त्यन्न॑पतिरे॒व भ॑वत्य॒ग्नये॒ पव॑मानाय पुरो॒डाश॑म॒ष्टाक॑पाल॒न्निर्व॑पेद॒ग्नये॑ पाव॒काया॒ग्नये॒ शुच॑ये॒ ज्योगा॑मयावी॒ यद॒ग्नये॒ पव॑मानाय नि॒र्वप॑ति प्रा॒णमे॒वास्मि॒न्तेन॑ दधाति॒ यद॒ग्नये᳚ (16)

%2.2.4.3
पा॒व॒काय॒ वाच॑मे॒वास्मि॒न्तेन॑ दधाति॒ यद॒ग्नये॒ शुच॑य॒ आयु॑रे॒वास्मि॒न्तेन॑ दधात्यु॒त यदी॒तासु॒र्भव॑ति॒ जीव॑त्ये॒वैतामे॒व निर्व॑पे॒च्चक्षु॑ष्कामो॒ यद॒ग्नये॒ पव॑मानाय नि॒र्वप॑ति प्रा॒णमे॒वास्मि॒न्तेन॑ दधाति॒ यद॒ग्नये॑ पाव॒काय॒ वाच॑मे॒वास्मि॒न्तेन॑ दधाति॒ यद॒ग्नये॒ शुच॑ये॒ चक्षु॑रे॒वास्मि॒न्तेन॑ दधाति (17)

%2.2.4.4
उ॒त यद्य॒न्धो भव॑ति॒ प्रैव प॑श्यत्य॒ग्नये॑ पु॒त्रव॑ते पुरो॒डाश॑म॒ष्टाक॑पाल॒न्निर्व॑पे॒दिन्द्रा॑य पु॒त्रिणे॑ पुरो॒डाश॒मेका॑दशकपालं प्र॒जाका॑मो़॒\-ऽग्निरे॒वास्मै᳚ प्र॒जां प्र॑ज॒नय॑ति वृ॒द्धामिन्द्रः॒ प्रय॑च्छत्य॒ग्नये॒ रस॑वते\-ऽजक्षी॒रे च॒रुं निर्व॑पे॒द्यः का॒मये॑त॒ रस॑वान्थ्स्या॒मित्य॒ग्निमे॒व रस॑वन्त॒ꣴ॒ स्वेन॑ भाग॒धेये॒नोप॑धावति॒ स ए॒वैन॒ꣳ॒ रस॑वन्तङ्करोति (18)

%2.2.4.5
रस॑वाने॒व भ॑वत्यजक्षी॒रे भ॑वत्याग्ने॒यी वा ए॒षा यद॒जा सा॒क्षादे॒व रस॒मव॑रुन्धे॒\-ऽग्नये॒ वसु॑मते पुरो॒डाश॑म॒ष्टाक॑पाल॒न्निर्व॑पे॒द्यः का॒मये॑त॒ वसु॑मान्थ्स्या॒मित्य॒ग्निमे॒व वसु॑मन्त॒ꣴ॒ स्वेन॑ भाग॒धेये॒नोप॑ धावति॒ स ए॒वैनं॒ वसु॑मन्तङ्करोति॒ वसु॑माने॒व भ॑वत्य॒ग्नये॑ वाज॒सृते॑ पुरो॒डाश॑म॒ष्टाक॑पाल॒न्निर्व॑पेथ्सङ्ग्रा॒मे सं य॑त्ते॒ वाजम्᳚ (19)

%2.2.4.6
वा ए॒ष सि॑सीर्\mbox{}षति॒ यः स॑ङ्ग्रा॒मञ्जिगी॑षत्य॒ग्निः खलु॒ वै दे॒वाना᳚ं वाज॒सृद॒ग्निमे॒व वा॑ज॒सृत॒ꣴ॒ स्वेन॑ भाग॒धेये॒नोप॑धावति॒ धाव॑ति॒ वाज॒ꣳ॒ हन्ति॑ वृ॒त्रञ्जय॑ति॒ तꣳ स॑ङ्ग्रा॒ममथो॑ अ॒ग्निरि॑व॒ न प्र॑ति॒धृषे॑ भवत्य॒ग्नये᳚\-ऽग्नि॒वते॑ पुरो॒डाश॑म॒ष्टाक॑पाल॒न्निर्व॑पे॒द्यस्या॒ग्नाव॒ग्निम॑भ्यु॒द्धरे॑यु॒र्निर्दि॑ष्टभागो॒ वा ए॒तयो॑र॒न्यो\-ऽनि॑र्दिष्टभागो॒\-ऽन्यस्तौ सं॒भव॑न्तौ॒ यज॑मानम् (20)

%2.2.4.7
अ॒भिसम्भ॑वतः॒ स ई᳚श्व॒र आर्ति॒मार्तो॒र्यद॒ग्नये᳚\-ऽग्नि॒वते॑ नि॒र्वप॑ति भाग॒धेये॑नै॒वैनौ॑ शमयति॒ नार्ति॒मार्छ॑ति॒ यज॑मानो॒\-ऽग्नये॒ ज्योति॑ष्मते पुरो॒डाश॑म॒ष्टाक॑पाल॒न्निर्व॑पे॒द्यस्या॒ग्निरुद्धृ॒तो\-ऽहु॑ते\-ऽग्निहो॒त्र उ॒द्वाये॒दप॑र आ॒दीप्या॑नू॒द्धृत्य॒ इत्या॑हु॒स्तत्तथा॒ न का॒र्यं॑ यद्भा॑ग॒धेय॑म॒भि पूर्व॑ उद्ध्रि॒यते॒ किमप॑रो॒\-ऽभ्युत् (21)

%2.2.4.8
ह्रि॒ये॒तेति॒ तान्ये॒वाव॒क्षाणा॑नि सन्नि॒धाय॑ मन्थेदि॒तः प्र॑थ॒मञ्ज॑ज्ञे अ॒ग्निः स्वाद्योने॒रधि॑ जा॒तवे॑दाः। स गा॑यत्रि॒या त्रि॒ष्टुभा॒ जग॑त्या दे॒वेभ्यो॑ ह॒व्यं व॑हतु प्रजा॒नन्निति॒ छन्दो॑भिरे॒वैन॒ꣴ॒ स्वाद्योनेः॒ प्रज॑नयत्ये॒ष वाव सो᳚\-ऽग्निरित्या॑हु॒र्ज्योति॒स्त्वा अ॑स्य॒ परा॑पतित॒मिति॒ यद॒ग्नये॒ ज्योति॑ष्मते नि॒र्वप॑ति॒ यदे॒वास्य॒ ज्योतिः॒ परा॑पतित॒न्तदे॒वाव॑रुन्धे॥ (22)

%2.2.5.0
{\anuvakamend[{क॒रो॒त्य॒न्ना॒दो द॑धाति॒ यद॒ग्नये॒ शुच॑ये॒ चक्षु॑रे॒वास्मि॒न्तेन॑ दधाति करोति॒ वाजं॒ यज॑मान॒मुदे॒वास्य॒ षट्च॑}]}%॥४॥

%2.2.5.1
वै॒श्वा॒न॒रन्द्वाद॑शकपाल॒न्निर्व॑पेद्वारु॒णं च॒रुन्द॑धि॒क्राव्ण्णे॑ च॒रुम॑भिश॒स्यमा॑नो॒ यद्वै᳚श्वान॒रो द्वाद॑शकपालो॒ भव॑ति सं वथ्स॒रो वा अ॒ग्निर्वै᳚श्वान॒रः सं॑ वथ्स॒रेणै॒वैनꣴ॑ स्वदय॒त्यप॑ पा॒पं वर्णꣳ॑ हते वारु॒णेनै॒वैनं॑ वरुणपा॒शान्मु॑ञ्चति दधि॒क्राव्ण्णा॑ पुनाति॒ हिर॑ण्य॒न्दक्षि॑णा प॒वित्रं॒ वै हिर॑ण्यं पु॒नात्ये॒वैन॑मा॒द्य॑म॒स्यान्नं॑ भवत्ये॒तामे॒व निर्व॑पेत्प्र॒जाका॑मः सं वथ्स॒रः (23)

%2.2.5.2
वा ए॒तस्याशा᳚न्तो॒ योनिं॑ प्र॒जायै॑ पशू॒नान्निर्द॑हति॒ यो\-ऽलं॑ प्र॒जायै॒ सन्प्र॒जान्न वि॒न्दते॒ यद्वै᳚श्वान॒रो द्वाद॑शकपालो॒ भव॑ति सं वथ्स॒रो वा अ॒ग्निर्वै᳚श्वान॒रः सं॑ वथ्स॒रमे॒व भा॑ग॒धेये॑न शमयति॒ सो᳚\-ऽस्मै शा॒न्तः स्वाद्योनेः᳚ प्र॒जां प्रज॑नयति वारु॒णेनै॒वैनं॑ वरुणपा॒शान्मु॑ञ्चति दधि॒क्राव्ण्णा॑ पुनाति॒ हिर॑ण्य॒न्दक्षि॑णा प॒वित्रं॒ वै हिर॑ण्यं पु॒नात्ये॒वैनम्᳚ (24)

%2.2.5.3
वि॒न्दते᳚ प्र॒जां वै᳚श्वान॒रन्द्वाद॑शकपालं॒ निर्व॑पेत्पु॒त्रे जा॒ते यद॒ष्टाक॑पालो॒ भव॑ति गायत्रि॒यैवैनं॑ ब्रह्मवर्च॒सेन॑ पुनाति॒ यन्नव॑कपालस्त्रि॒वृतै॒वास्मि॒न्तेजो॑ दधाति॒ यद्दश॑कपालो वि॒राजै॒वास्मि॑न्न॒न्नाद्य॑न्दधाति॒ यदेका॑दशकपालस्त्रि॒ष्टुभै॒वा\-स्मि॑न्निन्द्रि॒यं द॑धाति॒ यद्द्वाद॑शकपालो॒ जग॑त्यै॒वास्मि॑न्प॒शून्द॑धाति॒ यस्मि॑ञ्जा॒त ए॒तामिष्टि॑न्नि॒र्वप॑ति पू॒तः (25)

%2.2.5.4
ए॒व ते॑ज॒स्व्य॑न्ना॒द इ॑न्द्रिया॒वी प॑शु॒मान्भ॑व॒त्यव॒ वा ए॒ष सु॑व॒र्गाल्लो॒काच्छि॑द्यते॒ यो द॑र्\mbox{}शपूर्णमासया॒जी सन्न॑मावा॒स्या᳚ं वा पौर्णमा॒सीं वा॑\-ऽतिपा॒दय॑ति सुव॒र्गाय॒ हि लो॒काय॑ दर्\mbox{}शपूर्णमा॒सावि॒ज्येते॑ वैश्वान॒रन्द्वाद॑श\-कपाल॒न्निर्व॑पेदमावा॒स्या᳚ं वा पौर्णमा॒सीं वा॑\-ऽति॒पाद्य॑ सं वथ्स॒रो वा अ॒ग्निर्वै᳚श्वान॒रः सं॑ वथ्स॒रमे॒व प्री॑णा॒त्यथो॑ सं वथ्स॒रमे॒वास्मा॒ उप॑दधाति सुव॒र्गस्य॑ लो॒कस्य॒ सम॑ष्ट्यै (26)

%2.2.5.5
अथो॑ दे॒वता॑ ए॒वान्वा॒रभ्य॑ सुव॒र्गल्लोँ॒कमे॑ति वीर॒हा वा ए॒ष दे॒वानां॒ यो᳚\-ऽग्निमु॑द्वा॒सय॑ते॒ न वा ए॒तस्य॑ ब्राह्म॒णा ऋ॑ता॒यवः॑ पु॒रा\-ऽन्न॑मक्षन्नाग्ने॒यम॒ष्टाक॑पाल॒न्निर्व॑पेद्वैश्वान॒रन्द्वाद॑शकपालम॒ग्निमु॑द्वासयि॒ष्यन् यद॒ष्टाक॑पालो॒ भव॑त्य॒ष्टाक्ष॑रा गाय॒त्री गा॑य॒त्रो᳚\-ऽग्निर्यावा॑ने॒वाग्निस्तस्मा॑ आति॒थ्यङ्क॑रो॒त्यथो॒ यथा॒ जनं॑ य॒ते॑\-ऽव॒सङ्क॒रोति॑ ता॒दृक् (27)

%2.2.5.6
ए॒व तद्द्वाद॑शकपालो वैश्वान॒रो भ॑वति॒ द्वाद॑श॒ मासाः᳚ सं वथ्स॒रः सं॑ वथ्स॒रः खलु॒ वा अ॒ग्नेर्योनिः॒ स्वामे॒वैनं॒ योनि॑ङ्गमयत्या॒द्य॑म॒स्यान्नं॑ भवति वैश्वान॒रन्द्वाद॑शकपाल॒न्निर्व॑पेन्मारु॒तꣳ स॒प्तक॑पाल॒ङ्ग्राम॑काम आहव॒नीये॑ वैश्वान॒रमधि॑श्रयति॒ गार्\mbox{}ह॑पत्ये मारु॒तं पा॑पवस्य॒सस्य॒ विधृ॑त्यै॒ द्वाद॑शकपालो वैश्वान॒रो भ॑वति॒ द्वाद॑श॒ मासाः᳚ सं वथ्स॒रः सं॑ वथ्स॒रेणै॒वास्मै॑ सजा॒ताꣴश्च्या॑वयति मारु॒तो भ॑वति (28)

%2.2.5.7
म॒रुतो॒ वै दे॒वानां॒ विशो॑ देववि॒शेनै॒वास्मै॑ मनुष्यवि॒शमव॑रुन्धे स॒प्तक॑पालो भवति स॒प्तग॑णा॒ वै म॒रुतो॑ गण॒श ए॒वास्मै॑ सजा॒तानव॑रुन्धे\-ऽनू॒च्यमा॑न॒ आसा॑दयति॒ विश॑मे॒वास्मा॒ अनु॑वर्त्मानङ्करोति॥ (29)

%2.2.6.0
{\anuvakamend[{प्र॒जाका॑मः सं वथ्स॒रः पु॒नात्ये॒वैनं॑ पू॒तः सम॑ष्ट्यै ता॒दृङ्मा॑रु॒तो भ॑व॒त्येका॒न्नत्रि॒ꣳ॒शच्च॑}]}%॥५॥

%2.2.6.1
आ॒दि॒त्यं च॒रुं निर्व॑पेथ्सङ्ग्रा॒ममु॑पप्रया॒स्यन्नि॒यं वा अदि॑तिर॒स्यामे॒व पूर्वे॒ प्रति॑तिष्ठन्ति वैश्वान॒रन्द्वाद॑शकपाल॒न्निर्व॑पेदा॒यतन॑ङ्ग॒त्वा सं॑ वथ्स॒रो वा अ॒ग्निर्वै᳚श्वान॒रः सं॑ वथ्स॒रः खलु॒ वै दे॒वाना॑मा॒यत॑नमे॒तस्मा॒द्वा आ॒यत॑नाद्दे॒वा असु॑रानजय॒न्॒ यद्वै᳚श्वान॒रन्द्वाद॑शकपालन्नि॒र्वप॑ति दे॒वाना॑मे॒वायत॑ने यतते॒ जय॑ति॒ तꣳ स॑ङ्ग्रा॒ममे॒तस्मि॒न्वा ए॒तौ मृ॑जाते (30)

%2.2.6.2
यो वि॑द्विषा॒णयो॒रन्न॒मत्ति॑ वैश्वान॒रन्द्वाद॑शकपाल॒न्निर्व॑पेद्विद्विषा॒णयो॒रन्न॑ञ्ज॒ग्ध्वा सं॑ वथ्स॒रो वा अ॒ग्निर्वै᳚श्वान॒रः सं॑ वथ्स॒रस्व॑दितमे॒वात्ति॒ नास्मि॑न्मृजाते सं वथ्स॒राय॒ वा ए॒तौ सम॑माते॒ यौ स॑म॒माते॒ तयो॒र्यः पूर्वो॑\-ऽभि॒द्रुह्य॑ति॒ तं वरु॑णो गृह्णाति वैश्वान॒रन्द्वाद॑शकपाल॒न्निर्व॑पेथ्सममा॒नयोः॒ पूर्वो॑\-ऽभि॒द्रुह्य॑ सं वथ्स॒रो वा अ॒ग्निर्वै᳚श्वान॒रः सं॑ वथ्स॒रमे॒वाप्त्वा नि॑र्वरु॒णम् (31)

%2.2.6.3
प॒रस्ता॑द॒भिद्रु॑ह्यति॒ नैनं॒ वरु॑णो गृह्णात्या॒व्यं॑ वा ए॒ष प्रति॑गृह्णाति॒ यो\-ऽविं॑ प्रतिगृ॒ह्णाति॑ वैश्वान॒रन्द्वाद॑शकपाल॒न्निर्व॑पे॒दविं॑ प्रति॒गृह्य॑ सं वथ्स॒रो वा अ॒ग्निर्वै᳚श्वान॒रः सं॑ वथ्स॒रस्व॑दितामे॒व प्रति॑गृह्णाति॒ नाव्यं॑ प्रति॑गृह्णात्या॒त्मनो॒ वा ए॒ष मात्रा॑माप्नोति॒ य उ॑भ॒याद॑त्प्रतिगृ॒ह्णात्यश्वं॑ वा॒ पुरु॑षं वा वैश्वान॒रन्द्वाद॑शकपाल॒न्निर्व॑पेदुभ॒याद॑त् (32)

%2.2.6.4
प्र॒ति॒गृह्य॑ सं वथ्स॒रो वा अ॒ग्निर्वै᳚श्वान॒रः सं॑ वथ्स॒रस्व॑दितमे॒व प्रति॑गृह्णाति॒ नात्मनो॒ मात्रा॑माप्नोति वैश्वान॒रन्द्वाद॑शकपाल॒न्निर्व॑पेथ्स॒निमे॒ष्यन्थ्सं॑ वथ्स॒रो वा अ॒ग्निर्वै᳚श्वान॒रो य॒दा खलु॒ वै सं॑ वथ्स॒रञ्ज॒नता॑या॒ञ्चर॒त्यथ॒ स ध॑ना॒र्घो भ॑वति॒ यद्वै᳚श्वान॒रन्द्वाद॑शकपालन्नि॒र्वप॑ति सं वथ्स॒रसा॑तामे॒व स॒निम॒भि प्रच्य॑वते॒ दान॑कामा अस्मै प्र॒जा भ॑वन्ति॒ यो वै सं॑ वथ्स॒रम् (33)

%2.2.6.5
प्र॒युज्य॒ न वि॑मु॒ञ्चत्य॑प्रतिष्ठा॒नो वै स भ॑वत्ये॒तमे॒व वै᳚श्वान॒रं पुन॑रा॒गत्य॒ निर्व॑पे॒द्यमे॒व प्र॑यु॒ङ्क्ते तं भा॑ग॒धेये॑न॒ विमु॑ञ्चति॒ प्रति॑ष्ठित्यै॒ यया॒ रज्ज्वो᳚त्त॒माङ्गामा॒जेत्तां भ्रातृ॑व्याय॒ प्रहि॑णुया॒न्निर्\mbox{}ऋ॑तिमे॒वास्मै॒ प्रहि॑णोति॥ (34)

%2.2.7.0
{\anuvakamend{नि॒र्व॒रु॒णं व॑पेदुभ॒याद॒द्यो वै सं॑ वथ्स॒रꣳ षट्त्रिꣳ॑शच्च।]}}

%2.2.7.1
ऐ॒न्द्रं च॒रुं निर्व॑पेत्प॒शुका॑म ऐ॒न्द्रा वै प॒शव॒ इन्द्र॑मे॒व स्वेन॑ भाग॒धेये॒नोप॑धावति॒ स ए॒वास्मै॑ प॒शून्प्रय॑च्छति पशु॒माने॒व भ॑वति च॒रुर्भ॑वति॒ स्वादे॒वास्मै॒ योनेः᳚ प॒शून्प्रज॑नय॒तीन्द्रा॑येन्द्रि॒याव॑ते पुरो॒डाश॒मेका॑दशकपाल॒न्निर्व॑पेत्प॒शुका॑म इन्द्रि॒यं वै प॒शव॒ इन्द्र॑मे॒वेन्द्रि॒याव॑न्त॒ꣴ॒ स्वेन॑ भाग॒धेये॒नोप॑ धावति॒ सः (35)

%2.2.7.2
ए॒वास्मा॑ इन्द्रि॒यं प॒शून्प्रय॑च्छति पशु॒माने॒व भ॑व॒तीन्द्रा॑य घ॒र्मव॑ते पुरो॒डाश॒मेका॑दशकपाल॒न्निर्व॑पेद्ब्रह्मवर्च॒सका॑मो ब्रह्मवर्च॒सं वै घ॒र्म इन्द्र॑मे॒व घ॒र्मव॑न्त॒ꣴ॒ स्वेन॑ भाग॒धेये॒नोप॑धावति॒ स ए॒वास्मि॑न्ब्रह्मवर्च॒सन्द॑धाति ब्रह्मवर्च॒स्ये॑व भ॑व॒तीन्द्रा॑या॒र्कव॑ते पुरो॒डाश॒मेका॑दशकपाल॒न्निर्व॑पे॒दन्न॑कामो॒\-ऽर्को वै दे॒वाना॒मन्न॒मिन्द्र॑मे॒वार्कव॑न्त॒ꣴ॒ स्वेन॑ भाग॒धेये॑न (36)

%2.2.7.3
उप॑धावति॒ स ए॒वास्मा॒ अन्नं॒ प्रय॑च्छत्यन्ना॒द ए॒व भ॑व॒तीन्द्रा॑य घ॒र्मव॑ते पुरो॒डाश॒मेका॑दशकपाल॒न्निर्व॑पे॒दिन्द्रा॑येन्द्रि॒याव॑त॒ इन्द्रा॑या॒र्कव॑ते॒ भूति॑कामो॒ यदिन्द्रा॑य घ॒र्मव॑ते नि॒र्वप॑ति॒ शिर॑ ए॒वास्य॒ तेन॑ करोति॒ यदिन्द्रा॑येन्द्रि॒याव॑त आ॒त्मान॑मे॒वास्य॒ तेन॑ करोति॒ यदिन्द्रा॑या॒र्कव॑ते भू॒त ए॒वान्नाद्ये॒ प्रति॑तिष्ठति॒ भव॑त्ये॒वेन्द्रा॑य (37)

%2.2.7.4
अ॒ꣳ॒हो॒मुचे॑ पुरो॒डाश॒मेका॑दशकपाल॒न्निर्व॑पे॒द्यः पा॒प्मना॑ गृही॒तः स्यात्पा॒प्मा वा अꣳह॒ इन्द्र॑मे॒वाꣳहो॒मुच॒ꣴ॒ स्वेन॑ भाग॒धेये॒नोप॑धावति॒ स ए॒वैनं॑ पा॒प्मनो\-ऽꣳह॑सो मुञ्च॒तीन्द्रा॑य वैमृ॒धाय॑ पुरो॒डाश॒मेका॑दशकपाल॒न्निर्व॑पे॒द्यं मृधो॒\-ऽभि प्र॒वेपे॑रन्रा॒ष्ट्राणि॑ वा॒\-ऽभिस॑मि॒युरिन्द्र॑मे॒व वै॑मृ॒धꣴ स्वेन॑ भाग॒धेये॒नोप॑धावति॒ स ए॒वास्मा॒न्मृधः॑ (38)

%2.2.7.5
अप॑ह॒न्तीन्द्रा॑य त्रा॒त्रे पु॑रो॒डाश॒मेका॑दशकपाल॒न्निर्व॑पेद्ब॒द्धो वा॒ परि॑यत्तो॒ वेन्द्र॑मे॒व त्रा॒तार॒ꣴ॒ स्वेन॑ भाग॒धेये॒नोप॑धावति॒ स ए॒वैन॑न्त्रायत॒ इन्द्रा॑यार्कश्वमे॒धव॑ते पुरो॒डाश॒मेका॑दशकपाल॒न्निर्व॑पे॒द्यं म॑हाय॒ज्ञो नोप॒नमे॑दे॒ते वै म॑हाय॒ज्ञस्यान्त्ये॑ त॒नू यद॑र्काश्वमे॒धाविन्द्र॑मे॒वार्का᳚श्वमे॒धव॑न्त॒ꣴ॒ स्वेन॑ भाग॒धेये॒नोप॑धावति॒ स ए॒वास्मा॑ अन्त॒तो म॑हाय॒ज्ञञ्च्या॑वय॒त्युपै॑नं महाय॒ज्ञो न॑मति॥ (39)

%2.2.8.0
{\anuvakamend[{इ॒न्द्रि॒याव॑न्त॒ꣴ॒ स्वेन॑ भाग॒धेये॒नोप॑धावति॒ सो᳚\-ऽर्कव॑न्त॒ꣴ॒ स्वेन॑ भाग॒धेये॑नै॒वेन्द्रा॑यास्मा॒न्मृधो᳚\-ऽस्मै स॒प्त च॑।7।}]}

%2.2.8.1
इन्द्रा॒यान्वृ॑जवे पुरो॒डाश॒मेका॑दशकपाल॒न्निर्व॑पे॒द्ग्राम॑काम॒ इन्द्र॑मे॒वान्वृ॑जु॒ꣴ॒ स्वेन॑ भाग॒धेये॒नोप॑ धावति॒ स ए॒वास्मै॑ सजा॒ताननु॑कान्करोति ग्रा॒म्ये॑व भ॑वतीन्द्रा॒ण्यै च॒रुं निर्व॑पे॒द्यस्य॒ सेना\-ऽसꣳ॑शितेव॒ स्यादि॑न्द्रा॒णी वै सेना॑यै दे॒वते᳚न्द्रा॒णीमे॒व स्वेन॑ भाग॒धेये॒नोप॑धावति॒ सैवास्य॒ सेना॒ꣳ॒ सꣴश्य॑ति॒ बल्ब॑जा॒नपि॑ (40)

%2.2.8.2
इ॒द्ध्मे सन्न॑ह्ये॒द्गौर्यत्राधि॑ष्कन्ना॒ न्यमे॑ह॒त्ततो॒ बल्ब॑जा॒ उद॑तिष्ठ॒न्गवा॑मे॒वैनं॑ न्या॒यम॑पि॒नीय॒ गा वे॑दय॒तीन्द्रा॑य मन्यु॒मते॒ मन॑स्वते पुरो॒डाश॒मेका॑दशकपाल॒न्निर्व॑पेथ्सङ्ग्रा॒मे सं य॑त्त इन्द्रि॒येण॒ वै म॒न्युना॒ मन॑सा सङ्ग्रा॒मञ्ज॑य॒तीन्द्र॑मे॒व म॑न्यु॒मन्तं॒ मन॑स्वन्त॒ꣴ॒ स्वेन॑ भाग॒धेये॒नोप॑ धावति॒ स ए॒वास्मि॑न्निन्द्रि॒यं म॒न्युं मनो॑ दधाति॒ जय॑ति॒ तम् (41)

%2.2.8.3
स॒ङ्ग्रा॒ममे॒तामे॒व निर्व॑पे॒द्यो ह॒तम॑नाः स्व॒यं पा॑प इव॒ स्यादे॒तानि॒ हि वा ए॒तस्मा॒दप॑क्रान्ता॒न्यथै॒ष ह॒तम॑नाः स्व॒यं पा॑प॒ इन्द्र॑मे॒व म॑न्यु॒मन्तं॒ मन॑स्वन्त॒ꣴ॒ स्वेन॑ भाग॒धेये॒नोप॑धावति॒ स ए॒वास्मि॑न्निन्द्रि॒यं म॒न्युं मनो॑ दधाति॒ न ह॒तम॑नाः स्व॒यं पा॑पो भव॒तीन्द्रा॑य दा॒त्रे पु॑रो॒डाश॒मेका॑दशकपाल॒न्निर्व॑पे॒द्यः का॒मये॑त॒ दान॑कामा मे प्र॒जाः स्युः॑ (42)

%2.2.8.4
इतीन्द्र॑मे॒व दा॒तार॒ꣴ॒ स्वेन॑ भाग॒धेये॒नोप॑धावति॒ स ए॒वास्मै॒ दान॑कामाः प्र॒जाः क॑रोति॒ दान॑कामा अस्मै प्र॒जा भ॑व॒न्तीन्द्रा॑य प्रदा॒त्रे पु॑रो॒डाश॒मेका॑दशकपाल॒न्निर्व॑पे॒द्यस्मै॒ प्रत्त॑मिव॒ सन्न प्र॑दी॒येतेन्द्र॑मे॒व प्र॑दा॒तार॒ꣴ॒ स्वेन॑ भाग॒धेये॒नोप॑धावति॒ स ए॒वास्मै॒ प्रदा॑पय॒तीन्द्रा॑य सु॒त्राम्णे॑ पुरो॒डाश॒मेका॑दशकपाल॒न्निर्व॑पे॒दप॑रुद्धो वा (43)

%2.2.8.5
अ॒प॒रु॒द्ध्यमा॑नो॒ वेन्द्र॑मे॒व सु॒त्रामा॑ण॒ꣴ॒ स्वेन॑ भाग॒धेये॒नोप॑धावति॒ स ए॒वैन॑न्त्रायते\-ऽनपरु॒द्ध्यो भ॑व॒तीन्द्रो॒ वै स॒दृङ् दे॒वता॑भिरासी॒थ्स न व्या॒वृत॑मगच्छ॒थ्स प्र॒जाप॑ति॒मुपा॑धाव॒त्तस्मा॑ ए॒तमै॒न्द्रमेका॑दशकपाल॒न्निर॑वप॒त्तेनै॒वास्मि॑न्निन्द्रि॒य म॑दधा॒च्छक्व॑री याज्यानुवा॒क्ये॑ अकरो॒द्वज्रो॒ वै शक्व॑री॒ स ए॑नं॒ वज्रो॒ भूत्या॑ ऐन्ध (44)

%2.2.8.6
सो॑\-ऽभव॒थ्सो॑\-ऽबिभेद्भू॒तः प्र मा॑ धक्ष्य॒तीति॒ स प्र॒जाप॑तिं॒ पुन॒रुपा॑धाव॒थ्स प्र॒जाप॑तिः॒ शक्व॑र्या॒ अधि॑ रे॒वती॒न्निर॑मिमीत॒ शान्त्या॒ अप्र॑दाहाय॒ यो\-ऽलꣴ॑ श्रि॒यै सन्थ्स॒दृङ्ख्स॑मा॒नैः स्यात्तस्मा॑ ए॒तमै॒न्द्रमेका॑दशकपाल॒न्निर्व॑पे॒दिन्द्र॑मे॒व स्वेन॑ भाग॒धेये॒नोप॑धावति॒ स ए॒वास्मि॑न्निन्द्रि॒यन्द॑धाति रे॒वती॑ पुरोनुवा॒क्या॑ भवति॒ शान्त्या॒ अप्र॑दाहाय॒ शक्व॑री या॒ज्या॑ वज्रो॒ वै शक्व॑री॒ स ए॑नं॒ वज्रो॒ भूत्या॑ इन्धे॒ भव॑त्ये॒व॥ (45)

%2.2.9.0
{\anuvakamend[{अपि॒ तꣴ स्यु॑र्वैन्ध भवति॒ चतु॑र्दश च}]}%॥७॥

%2.2.9.1
आ॒ग्ना॒वै॒ष्ण॒वमेका॑दकपाल॒न्निर्व॑पेदभि॒चर॒न्थ्सर॑स्व॒त्याज्य॑भागा॒ स्याद्बा॑र्\mbox{}हस्प॒त्यश्च॒रुर्यदा᳚ग्नावैष्ण॒व एका॑दशकपालो॒ भव॑त्य॒ग्निः सर्वा॑ दे॒वता॒ विष्णु॑र्य॒ज्ञो दे॒वता॑भिश्चै॒वैनं॑ य॒ज्ञेन॑ चा॒भिच॑रति॒ सर॑स्व॒त्याज्य॑भागा भवति॒ वाग्वै सर॑स्वती वा॒चैवैन॑म॒भिच॑रति बार्\mbox{}हस्प॒त्यश्च॒रुर्भ॑वति॒ ब्रह्म॒ वै दे॒वानां॒ बृह॒स्पति॒र्ब्रह्म॑णै॒वैन॑म॒भिच॑रति (46)

%2.2.9.2
प्रति॒ वै प॒रस्ता॑दभि॒चर॑न्तम॒भिच॑रन्ति॒ द्वेद्वे॑ पुरोनुवा॒क्ये॑ कुर्या॒दति॒प्रयु॑क्त्या ए॒तयै॒व य॑जेताभिच॒र्यमा॑णो दे॒वता॑भिरे॒व दे॒वताः᳚ प्रति॒चर॑ति य॒ज्ञेन॑ य॒ज्ञं वा॒चा वाचं॒ ब्रह्म॑णा॒ ब्रह्म॒ स दे॒वता᳚श्चै॒व य॒ज्ञं च॑ मद्ध्य॒तो व्यव॑सर्पति॒ तस्य॒ न कुत॑श्च॒नोपा᳚व्या॒धो भ॑वति॒ नैन॑मभि॒चर᳚न्स्तृणुत आग्नावैष्ण॒वमेका॑दशकपाल॒न्निर्व॑पे॒द्यं य॒ज्ञो न (47)

%2.2.9.3
उ॒प॒नमे॑द॒ग्निः सर्वा॑ दे॒वता॒ विष्णु॑र्य॒ज्ञो᳚\-ऽग्निञ्चै॒व विष्णुं॑ च॒ स्वेन॑ भाग॒धेये॒नोप॑धावति॒ तावे॒वास्मै॑ य॒ज्ञं प्रय॑च्छत॒ उपै॑नं य॒ज्ञो न॑मत्याग्नावैष्ण॒वं घृ॒ते च॒रुं निर्व॑पे॒च्चक्षु॑ष्कामो॒\-ऽग्नेर्वै चक्षु॑षा मनु॒ष्या॑ वि प॑श्यन्ति य॒ज्ञस्य॑ दे॒वा अ॒ग्निञ्चै॒व विष्णुं॑ च॒ स्वेन॑ भाग॒धेये॒नोप॑धावति॒ तावे॒व (48)

%2.2.9.4
अ॒स्मि॒ञ्चक्षु॑र्धत्त॒श्चक्षु॑ष्माने॒व भ॑वति धे॒न्वै वा ए॒तद्रेतो॒ यदाज्य॑मन॒डुह॑स्तण्डु॒ला मि॑थु॒नादे॒वास्मै॒ चक्षुः॒ प्रज॑नयति घृ॒ते भ॑वति॒ तेजो॒ वै घृ॒तन्तेज॒श्चक्षु॒स्तेज॑सै॒वास्मै॒ तेज॒श्चक्षु॒रव॑रुन्ध इन्द्रि॒यं वै वी॒र्यं॑ वृङ्क्ते॒ भ्रातृ॑व्यो॒ यज॑मा॒नो\-ऽय॑जमानस्याद्ध्व॒रक॑ल्पां॒ प्रति॒ निर्व॑पे॒द्भ्रातृ॑व्ये॒ यज॑माने॒ नास्ये᳚न्द्रि॒यम् (49)

%2.2.9.5
वी॒र्यं॑ वृङ्क्ते पु॒रा वा॒चः प्रव॑दितो॒र्निर्व॑पे॒द्याव॑त्ये॒व वाक्तामप्रो॑दितां॒ भ्रातृ॑व्यस्य वृङ्क्ते॒ ताम॑स्य॒ वाचं॑ प्र॒वद॑न्तीम॒न्या वाचो\-ऽनु॒ प्रव॑दन्ति॒ ता इ॑न्द्रि॒यं वी॒र्यं॑ यज॑माने दधत्याग्नावैष्ण॒वम॒ष्टाक॑पाल॒न्निर्व॑पेत्प्रातः सव॒नस्या॑का॒ले सर॑स्व॒त्याज्य॑भागा॒ स्याद्बा॑र्\mbox{}हस्प॒त्यश्च॒रुर्यद॒ष्टाक॑पालो॒ भव॑त्य॒ष्टाक्ष॑रा गाय॒त्री गा॑य॒त्रं प्रा॑तः सव॒नं प्रा॑तः सव॒नमे॒व तेना᳚प्नोति (50)

%2.2.9.6
आ॒ग्ना॒वै॒ष्ण॒वमेका॑दशकपाल॒न्निर्व॑पे॒न्माद्ध्य॑न्दिनस्य॒ सव॑नस्याका॒ले सर॑स्व॒त्याज्य॑भागा॒ स्याद्बा॑र्\mbox{}हस्प॒त्यश्च॒रुर्यदेका॑दशकपालो॒ भव॒त्येका॑दशाक्षरा त्रि॒ष्टुप्त्रैष्टु॑भं॒ माद्ध्य॑न्दिन॒ꣳ॒ स॑वनं॒ माद्ध्य॑न्दिनमे॒व सव॑न॒न्तेना᳚प्नोत्याग्नावैष्ण॒वन्द्वाद॑शकपाल॒न्निर्व॑पेत्तृतीयसव॒नस्या॑का॒ले सर॑स्व॒त्याज्य॑भागा॒ स्याद्बा॑र्\mbox{}हस्प॒त्यश्च॒रुर्यद्द्वाद॑शकपालो॒ भव॑ति॒ द्वाद॑शाक्षरा॒ जग॑ती॒ जाग॑तन्त़ृतीयसव॒नन्तृ॑तीयसव॒नमे॒व तेना᳚प्नोति दे॒वता॑भिरे॒व दे॒वताः᳚ (51)

%2.2.9.7
प्र॒ति॒चर॑ति य॒ज्ञेन॑ य॒ज्ञं वा॒चा वाचं॒ ब्रह्म॑णा॒ ब्रह्म॑ क॒पालै॑रे॒व छन्दाꣳ॑स्या॒प्नोति॑ पुरो॒डाशैः॒ सव॑नानि मैत्रावरु॒णमेक॑कपाल॒न्निर्व॑पेद्व॒शायै॑ का॒ले यैवासौ भ्रातृ॑व्यस्य व॒शा\-ऽनू॑ब॒न्ध्या॑ सो ए॒वैषैतस्यैक॑कपालो भवति॒ नहि क॒पालैः᳚ प॒शुमर्\mbox{}ह॒त्याप्तुम्᳚॥ (52)

%2.2.10.0
{\anuvakamend[{ब्रह्म॑णै॒वैन॑म॒भिच॑रति य॒ज्ञो न तावे॒वास्ये᳚न्द्रि॒यमा᳚प्नोति दे॒वताः᳚ स॒प्तत्रिꣳ॑शच्च।9।}]}

%2.2.10.1
अ॒सावा॑दि॒त्यो न व्य॑रोचत॒ तस्मै॑ दे॒वाः प्राय॑श्चित्तिमैच्छ॒न्तस्मा॑ ए॒तꣳ सो॑मारौ॒द्रं च॒रुं निर॑वप॒न्तेनै॒वास्मि॒न्रुच॑मदधु॒र्यो ब्र॑ह्मवर्च॒सका॑मः॒ स्यात्तस्मा॑ ए॒तꣳ सो॑मारौ॒द्रं च॒रुं निर्व॑पे॒थ्सोम॑ञ्चै॒व रु॒द्रं च॒ स्वेन॑ भाग॒धेये॒नोप॑धावति॒ तावे॒वास्मि॑न्ब्रह्मवर्च॒सन्ध॑त्तो ब्रह्मवर्च॒स्ये॑व भ॑वति तिष्यापूर्णमा॒से निर्व॑पेद्रु॒द्रः (53)

%2.2.10.2
वै ति॒ष्यः॑ सोमः॑ पू॒र्णमा॑सः सा॒क्षादे॒व ब्र॑ह्मवर्च॒समव॑रुन्धे॒ परि॑श्रिते याजयति ब्रह्मवर्च॒सस्य॒ परि॑गृहीत्यै श्वे॒तायै᳚ श्वे॒तव॑थ्सायै दु॒ग्धं म॑थि॒तमाज्यं॑ भव॒त्याज्यं॒ प्रोक्ष॑ण॒माज्ये॑न मार्जयन्ते॒ याव॑दे॒व ब्र॑ह्मवर्च॒सन्तथ्सर्व॑ङ्करो॒त्यति॑ ब्रह्मवर्च॒सङ्क्रि॑यत॒ इत्या॑हुरीश्व॒रो दु॒श्चर्मा॒ भवि॑तो॒रिति॑ मान॒वी ऋचौ॑ धा॒य्ये॑ कुर्या॒द्यद्वै किं च॒ मनु॒रव॑द॒त्तद्भे॑ष॒जम् (54)

%2.2.10.3
भे॒ष॒जमे॒वास्मै॑ करोति॒ यदि॑ बिभी॒याद्दु॒श्चर्मा॑ भविष्या॒मीति॑ सोमापौ॒ष्णं च॒रुं निर्व॑पेथ्सौ॒म्यो वै दे॒वत॑या॒ पुरु॑षः पौ॒ष्णाः प॒शवः॒ स्वयै॒वास्मै॑ दे॒वत॑या प॒शुभि॒स्त्वच॑ङ्करोति॒ न दु॒श्चर्मा॑ भवति सोमारौ॒द्रं च॒रुं निर्व॑पेत्प्र॒जाका॑मः॒ सोमो॒ वै रे॑तो॒धा अ॒ग्निः प्र॒जानां᳚ प्रजनयि॒ता सोम॑ ए॒वास्मै॒ रेतो॒ दधा᳚त्य॒ग्निः प्र॒जां प्रज॑नयति वि॒न्दते᳚ (55)

%2.2.10.4
प्र॒जाꣳ सो॑मारौ॒द्रं च॒रुं निर्व॑पेदभि॒चर᳚न्थ्सौ॒म्यो वै दे॒वत॑या॒ पुरु॑ष ए॒ष रु॒द्रो यद॒ग्निः स्वाया॑ ए॒वैनं॑ दे॒वता॑यै नि॒ष्क्रीय॑ रु॒द्रायापि॑ दधाति ता॒जगार्ति॒मार्च्छ॑ति सोमारौ॒द्रं च॒रुं निर्व॑पे॒ज्ज्योगा॑मयावी॒ सोमं॒ वा ए॒तस्य॒ रसो॑ गच्छत्य॒ग्निꣳ शरी॑रं॒ यस्य॒ ज्योगा॒मय॑ति॒ सोमा॑दे॒वास्य॒ रस॑न्निष्क्री॒णात्य॒ग्नेः शरी॑रमु॒त यदि॑ (56)

%2.2.10.5
इ॒तासु॒र्भव॑ति॒ जीव॑त्ये॒व सो॑मारु॒द्रयो॒र्वा ए॒तङ्ग्र॑सि॒तꣳ होता॒ निष्खि॑दति॒ स ई᳚श्व॒र आर्ति॒मार्तो॑रन॒ड्वान् होत्रा॒ देयो॒ वह्नि॒र्वा अ॑न॒ड्वान् वह्नि॒र्\mbox{}होता॒ वह्नि॑नै॒व वह्नि॑मा॒त्मानꣴ॑ स्पृणोति सोमारौ॒द्रं च॒रुं निर्व॑पे॒द्यः का॒मये॑त॒ स्वे᳚\-ऽस्मा आ॒यत॑ने॒ भ्रातृ॑व्यं जनयेय॒मिति॒ वेदिं॑ परि॒गृह्या॒र्द्धमु॑द्ध॒न्याद॒र्द्धन्नार्द्धं ब॒र्\mbox{}हिषः॑ स्तृणी॒याद॒र्द्धं नार्द्धमि॒द्ध्मस्या᳚भ्याद॒द्ध्याद॒र्द्धं न स्व ए॒वास्मा॑ आ॒यत॑ने॒ भ्रातृ॑व्यं जनयति॥ (57)

%2.2.11.0
{\anuvakamend[{रु॒द्रो भे॑ष॒जं वि॒न्दते॒ यदि॑ स्तृणी॒याद॒र्द्धन्द्वाद॑श च।10।}]}

%2.2.11.1
ऐ॒न्द्रमेका॑दशकपाल॒न्निर्व॑पेन्मारु॒तꣳ स॒प्तक॑पाल॒ङ्ग्राम॑काम॒ इन्द्र॑ञ्चै॒व म॒रुत॑श्च॒ स्वेन॑ भाग॒धेये॒नोप॑ धावति॒ त ए॒वास्मै॑ सजा॒तान्प्रय॑च्छन्ति ग्रा॒म्ये॑व भ॑वत्याहव॒नीय॑ ऐ॒न्द्रमधि॑श्रयति॒ गार्\mbox{}ह॑पत्ये मारु॒तं पा॑पवस्य॒सस्य॒ विधृ॑त्यै स॒प्तक॑पालो मारु॒तो भ॑वति स॒प्तग॑णा॒ वै म॒रुतो॑ गण॒श ए॒वास्मै॑ सजा॒तानव॑रुन्धे\-ऽनू॒च्यमा॑न॒ आसा॑दयति॒ विश॑मे॒व (58)

%2.2.11.2
अ॒स्मा॒ अनु॑वर्त्मानङ्करोत्ये॒तामे॒व निर्व॑पे॒द्यः का॒मये॑त क्ष॒त्राय॑ च वि॒शे च॑ स॒मद॑न्दद्ध्या॒मित्यै॒न्द्रस्या॑व॒द्यन्ब्रू॑या॒दिन्द्रा॒यानु॑ ब्रू॒हीत्या॒श्राव्य॑ ब्रूयान्म॒रुतो॑ य॒जेति॑ मारु॒तस्या॑व॒द्यन्ब्रू॑यान्म॒रुद्भ्यो\-ऽनु॑ब्रू॒हीत्या॒श्राव्य॑ ब्रूया॒दिन्द्रं॑ य॒जेति॒ स्व ए॒वैभ्यो॑ भाग॒धेये॑ स॒मद॑न्दधाति वितृꣳहा॒णास्ति॑ष्ठन्त्ये॒तामे॒व (59)

%2.2.11.3
निर्व॑पे॒द्यः का॒मये॑त॒ कल्पे॑र॒न्निति॑ यथादेव॒तम॑व॒दाय॑ यथादेव॒तं य॑जेद्भाग॒धेये॑नै॒वैनान्॑ यथाय॒थङ्क॑ल्पयति॒ कल्प॑न्त ए॒वैन्द्रमेका॑दशकपाल॒न्निर्व॑पेद्वैश्वदे॒वन्द्वाद॑शकपाल॒ङ्ग्राम॑काम॒ इन्द्र॑ञ्चै॒व विश्वाꣳ॑श्च दे॒वान्थ्स्वेन॑ भाग॒धेये॒नोप॑ धावति॒ त ए॒वास्मै॑ सजा॒तान्प्रय॑च्छन्ति ग्रा॒म्ये॑व भ॑वत्यै॒न्द्रस्या॑व॒दाय॑ वैश्वदे॒वस्याव॑द्ये॒दथै॒न्द्रस्य॑ (60)

%2.2.11.4
उ॒परि॑ष्टादिन्द्रि॒येणै॒वास्मा॑ उभ॒यतः॑ सजा॒तान्परि॑गृह्णात्युपाधा॒य्य॑पूर्वयं॒ वासो॒ दक्षि॑णा सजा॒ताना॒मुप॑हित्यै॒ पृश्ञि॑यै दु॒ग्धे प्रैय॑ङ्गवं च॒रुं निर्व॑पेन्म॒रुद्भ्यो॒ ग्राम॑कामः॒ पृश्ञि॑यै॒ वै पय॑सो म॒रुतो॑ जा॒ताः पृश्ञि॑यै प्रि॒यङ्ग॑वो मारु॒ताः खलु॒ वै दे॒वत॑या सजा॒ता म॒रुत॑ ए॒व स्वेन॑ भाग॒धेये॒नोप॑धावति॒ त ए॒वास्मै॑ सजा॒तान्प्रय॑च्छन्ति ग्रा॒म्ये॑व भ॑वति प्रि॒यव॑ती याज्यानुवा॒क्ये᳚ (61)

%2.2.11.5
भ॒व॒तः॒ प्रि॒यमे॒वैनꣳ॑ समा॒नाना᳚ङ्करोति द्वि॒पदा॑ पुरोनुवा॒क्या॑ भवति द्वि॒पद॑ ए॒वाव॑रुन्धे॒ चतु॑ष्पदा या॒ज्या॑ चतु॑ष्पद ए॒व प॒शूनव॑रुन्धे देवासु॒राः सं य॑त्ता आस॒न्ते दे॒वा मि॒थो विप्रि॑या आस॒न्ते \-ऽन्यो᳚न्यस्मै॒ ज्यैष्ठ्या॒याति॑ष्ठमानाश्चतु॒र्धा व्य॑क्रामन्न॒ग्निर्वसु॑भिः॒ सोमो॑ रु॒द्रैरिन्द्रो॑ म॒रुद्भि॒र्वरु॑ण आदि॒त्यैः स इन्द्रः॑ प्र॒जाप॑ति॒मुपा॑धाव॒त्तम् (62)

%2.2.11.6
ए॒तया॑ सं॒ज्ञान्या॑\-ऽयाजयद॒ग्नये॒ वसु॑मते पुरो॒डाश॑म॒ष्टाक॑पाल॒न्निर॑वप॒थ्सोमा॑य रु॒द्रव॑ते च॒रुमिन्द्रा॑य म॒रुत्व॑ते पुरो॒डाश॒मेका॑दशकपालं॒ वरु॑णाया\-ऽ\-ऽदि॒त्यव॑ते च॒रुन्ततो॒ वा इन्द्रं॑ दे॒वा ज्यैष्ठ्या॑या॒भि सम॑जानत॒ यः स॑मा॒नैर्मि॒थो विप्रि॑यः॒ स्यात्तमे॒तया॑ सं॒ज्ञान्या॑ याजयेद॒ग्नये॒ वसु॑मते पुरो॒डाश॑म॒ष्टाक॑पाल॒न्निर्व॑पे॒थ्सोमा॑य रु॒द्रव॑ते च॒रुमिन्द्रा॑य म॒रुत्व॑ते पुरो॒डाश॒मेका॑दशकपालं॒ वरु॑णाया\-ऽ\-ऽदि॒त्यव॑ते च॒रुमिन्द्र॑मे॒वैनं॑ भू॒तञ्ज्यैष्ठ्या॑य समा॒ना अ॒भिसञ्जा॑नते॒ वसि॑ष्ठः समा॒नानां᳚ भवति॥ (63)

%2.2.12.0
{\anuvakamend[{विश॑मे॒व ति॑ष्ठन्त्ये॒तामे॒वाथै॒न्द्रस्य॑ याज्यानुवा॒क्ये॑ तं वरु॑णाय॒ चतु॑र्दश च।11।}]}

%2.2.12.0

%2.2.12.1
हि॒र॒ण्य॒ग॒र्भ आपो॑ ह॒ यत्प्रजा॑पते। स वे॑द पु॒त्रः पि॒तर॒ꣳ॒ स मा॒तर॒ꣳ॒ स सू॒नुर्भु॑व॒थ्स भु॑व॒त्पुन॑र्मघः। स द्यामौर्णो॑द॒न्तरि॑क्ष॒ꣳ॒ स सुवः॒ स विश्वा॒ भुवो॑ अभव॒थ्स आ\-ऽभ॑वत्। उदु॒त्यञ्चि॒त्रम्। सप्र॑त्न॒वन्नवी॑य॒सा\-ऽग्ने᳚ द्यु॒म्नेन॑ सं॒ यता᳚। बृ॒हत्त॑तन्थ भा॒नुना᳚। निकाव्या॑ वे॒धसः॒ शश्व॑तस्क॒र्\mbox{}हस्ते॒ दधा॑नः (64)

%2.2.12.2
नर्या॑ पु॒रूणि॑। अ॒ग्निर्भु॑वद्रयि॒पती॑ रयी॒णाꣳ स॒त्रा च॑क्रा॒णो अ॒मृता॑नि॒ विश्वा᳚। हिर॑ण्यपाणिमू॒तये॑ सवि॒तार॒मुप॑ ह्वये। स चेत्ता॑ दे॒वता॑ प॒दम्। वा॒मम॒द्य स॑वितर्वा॒ममु॒ श्वो दि॒वेदि॑वे वा॒मम॒स्मभ्यꣳ॑ सावीः। वा॒मस्य॒ हि क्षय॑स्य देव॒ भूरे॑र॒या धि॒या वा॑म॒भाजः॑ स्याम। बडि॒त्था पर्व॑तानाङ्खि॒द्रं बि॑भर्\mbox{}षि पृथिवि। प्र या भू॑मि प्रवत्वति म॒ह्ना जि॒नोषि॑ (65)

%2.2.12.3
म॒हि॒नि॒। स्तोमा॑सस्त्वा विचारिणि॒ प्रति॑ष्टोभन्त्य॒क्तुभिः॑। प्र या वाज॒न्न हेष॑न्तं पे॒रुमस्य॑स्यर्जुनि। ऋ॒दू॒दरे॑ण॒ सख्या॑ सचेय॒ यो मा॒ न रिष्ये᳚द्धर्यश्व पी॒तः। अ॒यं यः सोमो॒ न्यधा᳚य्य॒स्मे तस्मा॒ इन्द्रं॑ प्र॒तिर॑मे॒म्यच्छ॑। आपा᳚न्तमन्युस्तृ॒पल॑प्रभर्मा॒ धुनिः॒ शिमी॑वा॒ञ्छरु॑माꣳ ऋजी॒षी। सोमो॒ विश्वा᳚न्यत॒सा वना॑नि॒ नार्वागिन्द्रं॑ प्रति॒माना॑नि देभुः। प्र (66)

%2.2.12.4
सु॒वा॒नः सोम॑ ऋत॒युश्चि॑के॒तेन्द्रा॑य॒ ब्रह्म॑ ज॒मद॑ग्नि॒रर्चन्न्॑। वृषा॑ य॒न्तासि॒ शव॑सस्तु॒रस्या॒न्तर्य॑च्छ गृण॒ते ध॒र्त्रं दृꣳ॑ह। स॒बाध॑स्ते॒ मद॑ञ्च शुष्म॒यं च॒ ब्रह्म॒ नरो᳚ ब्रह्म॒कृतः॑ सपर्यन्न्। अ॒र्को वा॒ यत्तु॒रते॒ सोम॑चक्षा॒स्तत्रेदिन्द्रो॑ दधते पृ॒थ्सु तु॒र्याम्। वष॑ट्ते विष्णवा॒स आ कृ॑णोमि॒ तन्मे॑ जुषस्व शिपिविष्ट ह॒व्यम्। (67)

%2.2.12.5
वर्ध॑न्तु त्वा सुष्टु॒तयो॒ गिरो॑ मे यू॒यं पा॑त स्व॒स्तिभिः॒ सदा॑ नः। प्र तत्ते॑ अ॒द्य शि॑पिविष्ट॒ नामा॒र्यः शꣳ॑सामि व॒युना॑नि वि॒द्वान्। तं त्वा॑ गृणामि त॒वस॒मत॑वीया॒न्क्षय॑न्तम॒स्य रज॑सः परा॒के। किमित्ते॑ विष्णो परि॒चक्ष्यं॑ भू॒त्प्रयद्व॑व॒क्षे शि॑पिवि॒ष्टो अ॑स्मि। मा वर्पो॑ अ॒स्मदप॑गूह ए॒तद्यद॒न्यरू॑पः समि॒थे ब॒भूथ॑। (68)

%2.2.12.6
अग्ने॒ दा दा॒शुषे॑ र॒यिं वी॒रव॑न्तं॒ परी॑णसम्। शि॒शी॒हि नः॑ सूनु॒मतः॑। दा नो॑ अग्ने श॒तिनो॒ दाः स॑ह॒स्रिणो॑ दु॒रो न वाज॒ꣴ॒ श्रुत्या॒ अपा॑वृधि। प्राची॒ द्यावा॑पृथि॒वी ब्रह्म॑णा कृधि॒ सुव॒र्ण शु॒क्रमु॒षसो॒ विदि॑द्युतुः। अ॒ग्निर्दा॒ द्रवि॑णं वी॒रपे॑शा अ॒ग्निर्\mbox{}ऋषिं॒ यः स॒हस्रा॑ स॒नोति॑। अ॒ग्निर्दि॒वि ह॒व्यमात॑ताना॒ग्नेर्धामा॑नि॒ विभृ॑ता पुरु॒त्रा। मा (69)

%2.2.12.7
नो॒ म॒र्द्धी॒रा तू भ॑र। घृ॒तं न पू॒तं त॒नूर॑रे॒पाः शुचि॒ हिर॑ण्यम्। तत्ते॑ रु॒क्मो न रो॑चत स्वधावः। उ॒भे सु॑श्चन्द्र स॒र्पिषो॒ दर्वी᳚ श्रीणीष आ॒सनि॑। उ॒तो न॒ उत्पु॑पूर्या उ॒क्थेषु॑ शवसस्पत॒ इषꣴ॑ स्तो॒तृभ्य॒ आ भ॑र। वायो॑ श॒तꣳ हरी॑णां यु॒वस्व॒ पोष्या॑णाम्। उ॒त वा॑ ते सह॒स्रिणो॒ रथ॒ आ या॑तु॒ पाज॑सा। प्र याभिः॑ (70)

%2.2.12.8
यासि॑ दा॒श्वाꣳस॒मच्छा॑ नि॒युद्भि॑र्वायवि॒ष्टये॑ दुरो॒णे। नि नो॑ र॒यिꣳ सु॒भोज॑सं युवे॒ह नि वी॒रव॒द्गव्य॒मश्वि॑यं च॒ राधः॑। रे॒वती᳚र्नः सध॒माद॒ इन्द्रे॑ सन्तु तु॒विवा॑जाः। क्षु॒मन्तो॒ याभि॒र्मदे॑म। रे॒वाꣳ इद्रे॒वतः॑ स्तो॒ता स्यात्त्वाव॑तो म॒घोनः॑। प्रेदु॑ हरिवः श्रु॒तस्य॑॥ (71)

{\anuvakamend[जि॒नोषि॑ देभुः॒ प्र ह॒व्यं ब॒भूथ॒ मा याभि॑श्चत्वारि॒ꣳ॒शच्च॑॥ (12)]}
%2.3.0.0

{\scriptsize प्र॒जाप॑ति॒स्ताः सृ॒ष्टा अ॒ग्नये॑ पथि॒कृते॒\-ऽग्नये॒ कामा॑या॒ग्नये\-ऽन्न॑वते वैश्वान॒रमा॑दि॒त्यं च॒रुमै॒न्द्रं च॒रुमिन्द्रा॒यान्वृ॑जव आग्नावैष्ण॒वम॒सौ सो॑मारौ॒द्रमै॒न्द्रमेका॑दशकपालꣳ हिरण्यग॒र्भो द्वाद॑श॥ (12) प्र॒जाप॑तिर॒ग्नये॒ कामा॑या॒भि सम्भ॑वतो॒ यो वि॑द्विषा॒णयो॑रि॒द्ध्मे सन्न॑ह्येदाग्नावैष्ण॒वमु॒परि॑ष्टा॒द्यासि॑ दा॒श्वाꣳस॒मेक॑सप्ततिः॥ (71) प्र॒जाप॑तिः॒ प्रेदु॑ हरिवः श्रु॒तस्य॑॥}

%2.3.0.0

{\scriptsize {आ॒दि॒त्येभ्यो॑ दे॒वा वै मृ॒त्योर्दे॒वा वै स॒त्रम॑र्य॒म्णे प्र॒जाप॑ते॒स्त्रय॑स्त्रिꣳशत्प्र॒जाप॑तिर्दे॒वेभ्यो॒\-ऽन्नाद्य॑न्देवासु॒रास्तान्रज॑नो द्ध्रु॒वो॑\-ऽसि॒ यन्नव॑म॒ग्निं वै प्र॒जाप॑ति॒र्वरु॑णाय॒ या वा॑मिन्द्रावरुणा॒ सप्र॑त्न॒वच्चतु॑र्दश॥14॥ आ॒दि॒त्येभ्य॒स्त्वष्टु॑रस्मै॒ दान॑कामा ए॒वाव॑रुन्धे॒\-ऽग्निं वै सप्र॑त्न॒वथ्षट्प॑ञ्चा॒शत्॥56॥ आ॒दि॒त्येभ्यः॒ सुव॑र॒पो जि॑गाय॥}}
%%% END PRASHNA
\sect{तृतीयः प्रश्नः}\setcounter{anuvakam}{0}
\dnsub{तैत्तिरीयसंहितायां द्वितीयकाण्डे तृतीयः प्रश्नः}
%2.3.1.0
%2.3.1.1
आ॒दि॒त्येभ्यो॒ भुव॑द्वद्भ्यश्च॒रुं निर्व॑पे॒द्भूति॑काम आदि॒त्या वा ए॒तम्भूत्यै॒ प्रति॑ नुदन्ते॒ यो\-ऽल॒म्भूत्यै॒ सन्भूतिं॒ न प्रा॒प्नोत्या॑दि॒त्याने॒व भुव॑द्वतः॒ स्वेन॑ भाग॒धेये॒नोप॑ धावति॒ त ए॒वैन॒म्भूतिं॑ गमयन्ति॒ भव॑त्ये॒वादि॒त्येभ्यो॑ धा॒रय॑द्वद्भ्यश्च॒रुं निर्व॑पे॒दप॑रुद्धो वा\-ऽपरु॒ध्यमा॑नो वा\-ऽ\-ऽदि॒त्या वा अ॑परो॒द्धार॑ आदि॒त्या अ॑वगमयि॒तार॑ आदि॒त्याने॒व धा॒रय॑द्वतः॥१॥

%2.3.1.2
स्वेन॑ भाग॒धेये॒नोप॑ धावति॒ त ए॒वैनं॑ वि॒शि दा᳚ध्रत्यनपरु॒ध्यो भ॑व॒त्यदि॒ते\-ऽनु॑ मन्य॒स्वेत्य॑परु॒ध्यमा॑नो\-ऽस्य प॒दमा द॑दीते॒यं वा अदि॑तिरि॒यमे॒वास्मै॑ रा॒ज्यमनु॑ मन्यते स॒त्याशीरित्या॑ह स॒त्यामे॒वाशिषं॑ कुरुत इ॒ह मन॒ इत्या॑ह प्र॒जा ए॒वास्मै॒ सम॑नसः करो॒त्युप॒ प्रेत॑ मरुतः॥२॥

%2.3.1.3
सु॒दा॒न॒व॒ ए॒ना वि॒श्पति॑ना॒भ्य॑मुꣳ राजा॑न॒मित्या॑ह मारु॒ती वै विड्ज्ये॒ष्ठो वि॒श्पति॑र्वि॒शैवनꣳ॑ रा॒ष्ट्रेण॒ सम॑र्धयति॒ यः प॒रस्ता᳚द्ग्राम्यवा॒दी स्यात्तस्य॑ गृ॒हाद्व्री॒हीना ह॑रेच्छु॒क्लाꣴश्च॑ कृ॒ष्णाꣴश्च॒ वि चि॑नुया॒द्ये शु॒क्लाः स्युस्तमा॑दि॒त्यं च॒रुं निर्व॑पेदादि॒त्या वै दे॒वत॑या॒ विड्विश॑मे॒वाव॑ गच्छति॥३॥

%2.3.1.4
अव॑गतास्य॒ विडन॑वगतꣳ रा॒ष्ट्रमित्या॑हु॒र्ये कृ॒ष्णाः स्युस्तं वा॑रु॒णं च॒रुं निर्व॑पेद्वारु॒णं वै रा॒ष्ट्रमु॒भे ए॒व विशं॑ च रा॒ष्ट्रं चाव॑ गच्छति॒ यदि॒ नाव॒गच्छे॑दि॒मम॒हमा॑दि॒त्येभ्यो॑ भा॒गं निर्व॑पा॒म्यामुष्मा॑द॒मुष्यै॑ वि॒शो\-ऽव॑गन्तो॒रिति॒ निर्व॑पेदादि॒त्या ए॒वैन॑म्भाग॒धेय॑म्प्रे॒फ्सन्तो॒ विश॒मव॑॥३॥

%2.3.1.5
ग॒म॒य॒न्ति॒ यदि॒ नाव॒गच्छे॒दाश्व॑त्थान्म॒यूखा᳚न्थ्स॒प्त म॑ध्यमे॒षाया॒मुप॑ हन्यादि॒दम॒हमा॑दि॒त्यान्ब॑ध्ना॒म्यामुष्मा॑द॒मुष्यै॑ वि॒शो\-ऽव॑गन्तो॒रित्या॑दि॒त्या ए॒वैन॑म्ब॒द्धवी॑रा॒ विश॒मव॑ गमयन्ति॒ यदि॒ नाव॒गच्छे॑दे॒तमे॒वादि॒त्यं च॒रुं निर्व॑पेदि॒ध्मे\-ऽपि॑ म॒यूखा॒न्थ्सं न॑ह्येदनपरु॒ध्यमे॒वाव॑ गच्छ॒त्याश्व॑त्था भवन्ति म॒रुतां॒ वा ए॒तदोजो॒ यद॑श्व॒त्थ ओज॑सै॒व विश॒मव॑ गच्छति स॒प्त भ॑वन्ति स॒प्तग॑णा॒ वै म॒रुतो॑ गण॒श ए॒व विश॒मव॑ गच्छति॥५॥

%2.3.2.0
{\anuvakamend[{धा॒रय॑द्वतो मरुतो गच्छति॒ विश॒मवै॒तद॒ष्टाद॑श च}]}%॥१॥

%2.3.2.1
दे॒वा वै मृ॒त्योर॑बिभयु॒स्ते प्र॒जाप॑ति॒मुपा॑धाव॒न्तेभ्य॑ ए॒ताम्प्रा॑जाप॒त्याꣳ श॒तकृ॑ष्णलां॒ निर॑वप॒त्तयै॒वैष्व॒मृत॑मदधा॒द्यो मृ॒त्योर्बि॑भी॒यात्तस्मा॑ ए॒ताम्प्रा॑जाप॒त्याꣳ श॒तकृ॑ष्णलां॒ निर्व॑पेत्प्र॒जाप॑तिमे॒व स्वेन॑ भाग॒धेये॒नोप॑ धावति॒ स ए॒वास्मि॒न्नायु॑र्दधाति॒ सर्व॒मायु॑रेति श॒तकृ॑ष्णला भवति श॒तायुः॒ पुरु॑षः श॒तेन्द्रि॑य॒ आयु॑ष्ये॒वेन्द्रि॒ये॥६॥

%2.3.2.2
प्रति॑ तिष्ठति घृ॒ते भ॑व॒त्यायु॒र्वै घृ॒तम॒मृत॒ꣳ॒ हिर॑ण्य॒मायु॑श्चै॒वास्मा॑ अ॒मृतं॑ च स॒मीची॑ दधाति च॒त्वारि॑चत्वारि कृ॒ष्णला॒न्यव॑ द्यति चतुरव॒त्तस्याप्त्या॑ एक॒धा ब्र॒ह्मण॒ उप॑ हरत्येक॒धैव यज॑मान॒ आयु॑र्दधात्य॒सावा॑दि॒त्यो न व्य॑रोचत॒ तस्मै॑ दे॒वाः प्राय॑श्चित्तिमैच्छ॒न्तस्मा॑ ए॒तꣳ सौ॒र्यं च॒रुं निर॑वप॒न्तेनै॒वास्मिन्न्॑॥७॥

%2.3.2.3
रुच॑मदधु॒र्यो ब्र॑ह्मवर्च॒सका॑मः॒ स्यात्तस्मा॑ ए॒तꣳ सौ॒र्यं च॒रुं निर्व॑पेद॒मुमे॒वादि॒त्यꣴ स्वेन॑ भाग॒धेये॒नोप॑ धावति॒ स ए॒वास्मि॑न्ब्रह्मवर्च॒सं द॑धाति ब्रह्मवर्च॒स्ये॑व भ॑वत्युभ॒यतो॑ रु॒क्मौ भ॑वत उभ॒यत॑ ए॒वास्मि॒न्रुचं॑ दधाति प्रया॒जेप्र॑याजे कृ॒ष्णलं॑ जुहोति दि॒ग्भ्य ए॒वास्मै᳚ ब्रह्मवर्च॒समव॑ रुन्द्ध आग्ने॒यम॒ष्टाक॑पालं॒ निर्व॑पेथ्सावि॒त्रं द्वाद॑शकपाल॒म्भूम्यै᳚॥८॥

%2.3.2.4
च॒रुं यः का॒मये॑त॒ हिर॑ण्यं विन्देय॒ हिर॑ण्य॒म्मोप॑ नमे॒दिति॒ यदा᳚ग्ने॒यो भव॑त्याग्ने॒यं वै हिर॑ण्यं॒ यस्यै॒व हिर॑ण्यं॒ तेनै॒वैन॑द्विन्दते सावि॒त्रो भ॑वति सवि॒तृप्र॑सूत ए॒वैन॑द्विन्दते॒ भूम्यै॑ च॒रुर्भ॑वत्य॒स्यामे॒वैन॑द्विन्दत॒ उपै॑न॒ꣳ॒ हिर॑ण्यं नमति॒ वि वा ए॒ष इ॑न्द्रि॒येण॑ वी॒र्ये॑णर्ध्यते॒ यो हिर॑ण्यं वि॒न्दत॑ ए॒ताम्॥९॥

%2.3.2.5
ए॒व निर्व॑पे॒द्धिर॑ण्यं वि॒त्त्वा नेन्द्रि॒येण॑ वी॒र्ये॑ण॒ व्यृ॑ध्यत ए॒तामे॒व निर्व॑पे॒द्यस्य॒ हिर॑ण्यं॒ नश्ये॒द्यदा᳚ग्ने॒यो भव॑त्याग्ने॒यं वै हिर॑ण्यं॒ यस्यै॒व हिर॑ण्यं॒ तेनै॒वैन॑द्विन्दति सावि॒त्रो भ॑वति सवि॒तृप्र॑सूत ए॒वैन॑द्विन्दति॒ भूम्यै॑ च॒रुर्भ॑वत्य॒स्यां वा ए॒तन्न॑श्यति॒ यन्नश्य॑त्य॒स्यामे॒वैन॑द्विन्द॒तीन्द्रः॑॥१०॥

%2.3.2.6
त्वष्टुः॒ सोम॑मभी॒षहा॑पिब॒थ्स विष्व॒ङ्व्या᳚र्च्छ॒थ्स इ॑न्द्रि॒येण॑ सोमपी॒थेन॒ व्या᳚र्ध्यत॒ स यदू॒र्ध्वमु॒दव॑मी॒त्ते श्या॒माका॑ अभव॒न्थ्स प्र॒जाप॑ति॒मुपा॑धाव॒त्तस्मा॑ ए॒तꣳ सो॑मे॒न्द्रꣴ श्या॑मा॒कं च॒रुं निर॑वप॒त्तेनै॒वास्मि॑न्निन्द्रि॒यꣳ सो॑मपी॒थम॑दधा॒द्वि वा ए॒ष इ॑न्द्रि॒येण॑ सोमपी॒थेन॑र्ध्यते॒ यः सोमं॒ वमि॑ति॒ यः सो॑मवा॒मी स्यात्तस्मै᳚॥११॥

%2.3.2.7
ए॒तꣳ सो॑मे॒न्द्रꣴ श्या॑मा॒कं च॒रुं निर्व॑पे॒थ्सोमं॑ चै॒वेन्द्रं॑ च॒ स्वेन॑ भाग॒धेये॒नोप॑ धावति॒ तावे॒वास्मि॑न्निन्द्रि॒यꣳ सो॑मपी॒थं ध॑त्तो॒ नेन्द्रि॒येण॑ सोमपी॒थेन॒ व्यृ॑ध्यते॒ यथ्सौ॒म्यो भव॑ति सोमपी॒थमे॒वाव॑ रुन्द्धे॒ यदै॒न्द्रो भव॑तीन्द्रि॒यं वै सो॑मपी॒थ इ॑न्द्रि॒यमे॒व सो॑मपी॒थमव॑ रुन्द्धे श्यामा॒को भ॑वत्ये॒ष वाव स सोमः॑॥१२॥

%2.3.2.8
सा॒क्षादे॒व सो॑मपी॒थमव॑ रुन्द्धे॒\-ऽग्नये॑ दा॒त्रे पु॑रो॒डाश॑म॒ष्टाक॑पालं॒ निर्व॑पे॒दिन्द्रा॑य प्रदा॒त्रे पु॑रो॒डाश॒मेका॑दशकपालम् प॒शुका॑मो॒\-ऽग्निरे॒वास्मै॑ प॒शून्प्र॑ज॒नय॑ति वृ॒द्धानिन्द्रः॒ प्र य॑च्छति॒ दधि॒ मधु॑ घृ॒तमापो॑ धा॒ना भ॑वन्त्ये॒तद्वै प॑शू॒नाꣳ रू॒पꣳ रू॒पेणै॒व प॒शूनव॑ रुन्द्धे पञ्चगृही॒तम्भ॑वति॒ पाङ्क्ता॒ हि प॒शवो॑ बहुरू॒पम्भ॑वति बहुरू॒पा हि प॒शवः॑॥१३॥

%2.3.2.9
समृ॑द्ध्यै प्राजाप॒त्यम्भ॑वति प्राजाप॒त्या वै प॒शवः॑ प्र॒जाप॑तिरे॒वास्मै॑ प॒शून्प्र ज॑नयत्या॒त्मा वै पुरु॑षस्य॒ मधु॒ यन्मध्व॒ग्नौ जु॒होत्या॒त्मान॑मे॒व तद्यज॑मानो॒\-ऽग्नौ प्र द॑धाति प॒ङ्क्त्यौ॑ याज्यानुवा॒क्ये॑ भवतः॒ पाङ्क्तः॒ पुरु॑षः॒ पाङ्क्ताः᳚ प॒शव॑ आ॒त्मान॑मे॒व मृ॒त्योर्नि॒ष्क्रीय॑ प॒शूनव॑ रुन्द्धे॥१४॥

%2.3.3.0
{\anuvakamend[{इ॒न्द्रि॒ये᳚\-ऽस्मि॒न्भूम्या॑ ए॒तामिन्द्रः॒ स्यात्तस्मै॒ सोमो॑ बहुरू॒पा हि प॒शव॒ एक॑चत्वारिꣳशच्च}]}%॥२॥

%2.3.3.1
दे॒वा वै स॒त्त्रमा॑स॒तर्द्धि॑परिमितं॒ यश॑स्कामा॒स्तेषा॒ꣳ॒ सोम॒ꣳ॒ राजा॑नं॒ यश॑ आर्च्छ॒थ्स गि॒रिमुदै॒त्तम॒ग्निरनूदै॒त्ताव॒ग्नीषोमौ॒ सम॑भवता॒न्ताविन्द्रो॑ य॒ज्ञवि॑भ्र॒ष्टो\-ऽनु॒ परै॒त्ताव॑ब्रवीद्या॒जय॑त॒म्मेति॒ तस्मा॑ ए॒तामिष्टिं॒ निर॑वपतामाग्ने॒यम॒ष्टाक॑पालमै॒न्द्रमेका॑दशकपालꣳ सौ॒म्यं च॒रुन्तयै॒वास्मि॒न्तेजः॑॥१५॥

%2.3.3.2
इ॒न्द्रि॒यम्ब्र॑ह्मवर्च॒सम॑धत्तां॒ यो य॒ज्ञवि॑भ्रष्टः॒ स्यात्तस्मा॑ ए॒तामिष्टिं॒ निर्व॑पेदाग्ने॒यम॒ष्टाक॑पालमै॒न्द्रमेका॑दशकपालꣳ सौ॒म्यं च॒रुं यदा᳚ग्ने॒यो भव॑ति॒ तेज॑ ए॒वास्मि॒न्तेन॑ दधाति॒ यदै॒न्द्रो भव॑तीन्द्रि॒यमे॒वास्मि॒न्तेन॑ दधाति॒ यथ्सौ॒म्यो ब्र॑ह्मवर्च॒सं तेना᳚ग्ने॒यस्य॑ च सौ॒म्यस्य॑ चै॒न्द्रे स॒माश्ले॑षये॒त्तेज॑श्चै॒वास्मि॑न्ब्रह्मवर्च॒सं च॑ स॒मीची᳚॥१६॥

%2.3.3.3
द॒धा॒त्य॒ग्नी॒षो॒मीय॒मेका॑दशकपालं॒ निर्व॑पे॒द्यं कामो॒ नोप॒नमे॑दाग्ने॒यो वै ब्रा᳚ह्म॒णः स सोम॑म्पिबति॒ स्वामे॒व दे॒वता॒ꣴ॒ स्वेन॑ भाग॒धेये॒नोप॑ धावति॒ सैवैनं॒ कामे॑न॒ सम॑र्धय॒त्युपै॑नं॒ कामो॑ नमत्यग्नीषो॒मीय॑म॒ष्टाक॑पालं॒ निर्व॑पेद्ब्रह्मवर्च॒सका॑मो॒\-ऽग्नीषोमा॑वे॒व स्वेन॑ भाग॒धेये॒नोप॑ धावति॒ तावे॒वास्मि॑न्ब्रह्मवर्च॒सं ध॑त्तो ब्रह्मवर्च॒स्ये॑व॥१७॥

%2.3.3.4
भ॒व॒ति॒ यद॒ष्टाक॑पाल॒स्तेना᳚ग्ने॒यो यच्छ्या॑मा॒कस्तेन॑ सौ॒म्यः समृ॑द्ध्यै॒ सोमा॑य वा॒जिने᳚ श्यामा॒कं च॒रुं निर्व॑पे॒द्यः क्लैव्या᳚द्बिभी॒याद्रेतो॒ हि वा ए॒तस्मा॒द्वाजि॑नमप॒क्राम॒त्यथै॒ष क्लैब्या᳚द्बिभाय॒ सोम॑मे॒व वा॒जिन॒ꣴ॒ स्वेन॑ भाग॒धेये॒नोप॑ धावति॒ स ए॒वास्मि॒न्रेतो॒ वाजि॑नं दधाति॒ न क्ली॒बो भ॑वति ब्राह्मणस्प॒त्यमेका॑दशकपालं॒ निर्व॑पे॒द्ग्राम॑कामः॥१८॥

%2.3.3.5
ब्रह्म॑ण॒स्पति॑मे॒व स्वेन॑ भाग॒धेये॒नोप॑ धावति॒ स ए॒वास्मै॑ सजा॒तान्प्र य॑च्छति ग्रा॒म्ये॑व भ॑वति ग॒णव॑ती याज्यानुवा॒क्ये॑ भवतः सजा॒तैरे॒वैनं॑ ग॒णव॑न्तं करोत्ये॒तामे॒व निर्व॑पे॒द्यः का॒मये॑त॒ ब्रह्म॒न्विशं॒ वि ना॑शयेय॒मिति॑ मारु॒ती या᳚ज्यानुवा॒क्ये॑ कुर्या॒द्ब्रह्म॑न्ने॒व विशं॒ वि ना॑शयति॥१९॥

%2.3.4.0
{\anuvakamend[{तेजः॑ स॒मीची᳚ ब्रह्मवर्च॒स्ये॑व ग्राम॑काम॒स्त्रिच॑त्वारिꣳशच्च}]}%॥३॥

%2.3.4.1
अ॒र्य॒म्णे च॒रुं निर्व॑पेथ्सुव॒र्गका॑मो॒\-ऽसौ वा आ॑दि॒त्यो᳚\-ऽर्य॒मा\-ऽर्य॒मण॑मे॒व स्वेन॑ भाग॒धेये॒नोप॑ धावति॒ स ए॒वैनꣳ॑ सुव॒र्गं लो॒कं ग॑मयत्यर्य॒म्णे च॒रुं निर्व॑पे॒द्यः का॒मये॑त॒ दान॑कामा मे प्र॒जाः स्यु॒रित्य॒सौ वा आ॑दि॒त्यो᳚\-ऽर्य॒मा यः खलु॒ वै ददा॑ति॒ सो᳚\-ऽर्य॒मा\-ऽर्य॒मण॑मे॒व स्वेन॑ भाग॒धेये॒नोप॑ धावति॒ स ए॒व॥२०॥

%2.3.4.2
अ॒स्मै॒ दान॑कामाः प्र॒जाः क॑रोति॒ दान॑कामा अस्मै प्र॒जा भ॑वन्त्यर्य॒म्णे च॒रुं निर्व॑पे॒द्यः का॒मये॑त स्व॒स्ति ज॒नता॑मिया॒मित्य॒सौ वा आ॑दि॒त्यो᳚\-ऽर्य॒मा\-ऽर्य॒मण॑मे॒व स्वेन॑ भाग॒धेये॒नोप॑ धावति॒ स ए॒वैनं॒ तद्ग॑मयति॒ यत्र॒ जिग॑मिष॒तीन्द्रो॒ वै दे॒वाना॑मानुजाव॒र आ॑सी॒थ्स प्र॒जाप॑ति॒मुपा॑धाव॒त्तस्मा॑ ए॒तमै॒न्द्रमा॑नुषू॒कमेका॑दशकपालं॒ निः॥२१॥

%2.3.4.3
अ॒व॒प॒त्तेनै॒वैन॒मग्रं॑ दे॒वता॑नां॒ पर्य॑णयद्बु॒ध्नव॑ती॒ अग्र॑वती याज्यानुवा॒क्ये॑ अकरोद्बु॒ध्नादे॒वैन॒मग्रं॒ पर्य॑णय॒द्यो रा॑ज॒न्य॑ आनुजाव॒रः स्यात्तस्मा॑ ए॒तमै॒न्द्रमा॑नुषू॒कमेका॑दशकपालं॒ निर्व॑पे॒दिन्द्र॑मे॒व स्वेन॑ भाग॒धेये॒नोप॑ धावति॒ स ए॒वैन॒मग्रꣳ॑ समा॒नानां॒ परि॑ णयति बु॒ध्नव॑ती॒ अग्र॑वती याज्यानुवा॒क्ये॑ भवतो बु॒ध्नादे॒वैन॒मग्रम्᳚॥२२॥

%2.3.4.4
परि॑ णयत्यानुषू॒को भ॑वत्ये॒षा ह्ये॑तस्य॑ दे॒वता॒ य आ॑नुजाव॒रः समृ॑द्ध्यै॒ यो ब्रा᳚ह्म॒ण आ॑नुजाव॒रः स्यात्तस्मा॑ ए॒तम्बा॑र्\mbox{}हस्प॒त्यमा॑नुषू॒कं च॒रुं निर्व॑पे॒द्बृह॒स्पति॑मे॒व स्वेन॑ भाग॒धेये॒नोप॑ धावति॒ स ए॒वैन॒मग्रꣳ॑ समा॒नानां॒ परि॑ णयति बु॒ध्नव॑ती॒ अग्र॑वती याज्यानुवा॒क्ये॑ भवतो बु॒ध्नादे॒वैन॒मग्रं॒ परि॑ णयत्यानुषू॒को भ॑वत्ये॒षा ह्ये॑तस्य॑ दे॒वता॒ य आ॑नुजाव॒रः समृ॑द्ध्यै॥२३॥

%2.3.5.0
{\anuvakamend[{ए॒व निरग्र॑मे॒तस्य॑ च॒त्वारि॑ च}]}%॥४॥

%2.3.5.1
प्र॒जाप॑ते॒स्त्रय॑स्त्रिꣳशद्दुहि॒तर॑ आस॒न्ताः सोमा॑य॒ राज्ञे॑\-ऽददा॒त्तासाꣳ॑ रोहि॒णीमुपै॒त्ता ईर्ष्य॑न्तीः॒ पुन॑रगच्छ॒न्ता अन्वै॒त्ताः पुन॑रयाचत॒ ता अ॑स्मै॒ न पुन॑रददा॒थ्सो᳚\-ऽब्रवीदृ॒तम॑मीष्व॒ यथा॑ समाव॒च्छ उ॑पै॒ष्याम्यथ॑ ते॒ पुन॑र्दास्या॒मीति॒ स ऋ॒तमा॑मी॒त्ता अ॑स्मै॒ पुन॑रददा॒त्तासाꣳ॑ रोहि॒णीमे॒वोप॑॥२४॥

%2.3.5.2
ऐ॒त्तं यक्ष्म॑ आर्च्छ॒द्राजा॑नं॒ यक्ष्म॑ आर॒दिति॒ तद्रा॑जय॒क्ष्मस्य॒ जन्म॒ यत्पापी॑या॒नभ॑व॒त्तत्पा॑पय॒क्ष्मस्य॒ यज्जा॒याभ्यो\-ऽवि॑न्द॒त्तज्जा॒येन्य॑स्य॒ य ए॒वमे॒तेषां॒ यक्ष्मा॑णां॒ जन्म॒ वेद॒ नैन॑मे॒ते यक्ष्मा॑ विन्दन्ति॒ स ए॒ता ए॒व न॑म॒स्यन्नुपा॑धाव॒त्ता अ॑ब्रुव॒न्वरं॑ वृणामहै समाव॒च्छ ए॒व न॒ उपा॑य॒ इति॒ तस्मा॑ ए॒तम्॥२५॥

%2.3.5.3
आ॒दि॒त्यं च॒रुं निर॑वप॒न्तेनै॒वैन॑म्पा॒पाथ्स्रामा॑दमुञ्च॒न् यः पा॑पय॒क्ष्मगृ॑हीतः॒ स्यात्तस्मा॑ ए॒तमा॑दि॒त्यं च॒रुं निर्व॑पेदादि॒त्याने॒व स्वेन॑ भाग॒धेये॒नोप॑ धावति॒ त ए॒वैनं॑ पा॒पाथ्स्रामा᳚न्मुञ्चन्त्यमावा॒स्या॑यां॒ निर्व॑पेद॒मुमे॒वैन॑मा॒प्याय॑मान॒मन्वा प्या॑ययति॒ नवो॑नवो भवति॒ जाय॑मान॒ इति॑ पुरोनुवा॒क्या॑ भव॒त्यायु॑रे॒वास्मि॒न्तया॑ दधाति॒ यमा॑दि॒त्या अ॒ꣳ॒शुमा᳚प्या॒यय॒न्तीति॑ या॒ज्यैवैन॑मे॒तया᳚ प्याययति॥२६॥

%2.3.6.0
{\anuvakamend[{ए॒वोपै॒तम॑स्मि॒न्त्रयो॑दश च}]}%॥५॥

%2.3.6.1
प्र॒जाप॑तिर्दे॒वेभ्यो॒\-ऽन्नाद्यं॒ व्यादि॑श॒थ्सो᳚\-ऽब्रवी॒द्यदि॒माल्लोँ॒कान॒भ्य॑ति॒रिच्या॑तै॒ तन्ममा॑स॒दिति॒ तदि॒माल्लोँ॒कान॒भ्यत्य॑रिच्य॒तेन्द्र॒ꣳ॒ राजा॑न॒मिन्द्र॑मधिरा॒जमिन्द्रꣴ॑ स्व॒राजा॑न॒न्ततो॒ वै स इ॒माल्लोँ॒काꣴस्त्रे॒धादु॑ह॒त्तत्त्रि॒धातो᳚स्त्रिधातु॒त्वय्यं का॒मये॑तान्ना॒दः स्या॒दिति॒ तस्मा॑ ए॒तं त्रि॒धातुं॒ निर्व॑पे॒दिन्द्रा॑य॒ राज्ञे॑ पुरो॒डाशम्᳚॥२७॥

%2.3.6.2
एका॑दशकपाल॒मिन्द्रा॑याधिरा॒जायेन्द्रा॑य स्व॒राज्ञे॒\-ऽयं वा इन्द्रो॒ राजा॒यमिन्द्रो॑\-ऽधिरा॒जो॑\-ऽसाविन्द्रः॑ स्व॒राडि॒माने॒व लो॒कान्थ्स्वेन॑ भाग॒धेये॒नोप॑ धावति॒ त ए॒वास्मा॒ अन्न॒म्प्र य॑च्छन्त्यन्ना॒द ए॒व भ॑वति॒ यथा॑ व॒थ्सेन॒ प्रत्तां॒ गां दु॒ह ए॒वमे॒वेमाल्लोँ॒कान्प्रत्ता॒न्काम॑म॒न्नाद्यं॑ दुह उत्ता॒नेषु॑ क॒पाले॒ष्वधि॑ श्रय॒त्यया॑तयामत्वाय॒ त्रयः॑ पुरो॒डाशा॑ भवन्ति॒ त्रय॑ इ॒मे लो॒का ए॒षाल्लोँ॒काना॒माप्त्या॒ उत्त॑रउत्तरो॒ ज्याया᳚न्भवत्ये॒वमि॑व॒ हीमे लो॒काः समृ॑द्ध्यै॒ सर्वे॑षामभिग॒मय॒न्नव॑ द्य॒त्यछ॑म्बट्कारव्व्यँ॒त्यास॒मन्वा॒हानि॑र्दाहाय॥२८॥

%2.3.7.0
{\anuvakamend[{पु॒रो॒डाश॒न्त्रय॒ष्षड्विꣳ॑शतिश्च}]}%॥६॥

%2.3.7.1
दे॒वा॒सु॒राः संय॑त्ता आस॒न्तां दे॒वानसु॑रा अजय॒न्ते दे॒वाः प॑राजिग्या॒ना असु॑राणां॒ वैश्य॒मुपा॑य॒न्तेभ्य॑ इन्द्रि॒यं वी॒र्य॑मपा᳚क्राम॒त्तदिन्द्रो॑\-ऽचाय॒त्तदन्वपा᳚क्राम॒त्तद॑व॒रुधं॒ नाश॑क्नो॒त्तद॑स्मादभ्य॒र्धो॑\-ऽचर॒थ्स प्र॒जाप॑ति॒मुपा॑धाव॒त्तमे॒तया॒ सर्व॑पृष्ठया\-ऽयाजय॒त्तयै॒वास्मि॑न्निन्द्रि॒यं वी॒र्य॑मदधा॒द्य इ॑न्द्रि॒यका॑मः॥२९॥

%2.3.7.2
वी॒र्य॑कामः॒ स्यात्तमे॒तया॒ सर्व॑पृष्ठया याजयेदे॒ता ए॒व दे॒वताः॒ स्वेन॑ भाग॒धेये॒नोप॑ धावति॒ ता ए॒वास्मि॑न्निन्द्रि॒यं वी॒र्यं॑ दधति॒ यदिन्द्रा॑य॒ राथं॑तराय नि॒र्वप॑ति॒ यदे॒वाग्नेस्तेज॒स्तदे॒वाव॑ रुन्द्धे॒ यदिन्द्रा॑य॒ बार्\mbox{}ह॑ताय॒ यदे॒वेन्द्र॑स्य॒ तेज॒स्तदे॒वाव॑ रुन्द्धे॒ यदिन्द्रा॑य वैरू॒पाय॒ यदे॒व स॑वि॒तुस्तेज॒स्तत्॥३०॥

%2.3.7.3
ए॒वाव॑ रुन्द्धे॒ यदिन्द्रा॑य वैरा॒जाय॒ यदे॒व धा॒तुस्तेज॒स्तदे॒वाव॑ रुन्द्धे॒ यदिन्द्रा॑य शाक्व॒राय॒ यदे॒व म॒रुतां॒ तेज॒स्तदे॒वाव॑ रुन्द्धे॒ यदिन्द्रा॑य रैव॒ताय॒ यदे॒व बृह॒स्पते॒स्तेज॒स्तदे॒वाव॑ रुन्द्ध ए॒ताव॑न्ति॒ वै तेजाꣳ॑सि॒ तान्ये॒वाव॑ रुन्द्ध उत्ता॒नेषु॑ क॒पाले॒ष्वधि॑ श्रय॒त्यया॑तयामत्वाय॒ द्वाद॑शकपालः पुरो॒डाशः॑॥३१॥

%2.3.7.4
भ॒व॒ति॒ वै॒श्व॒दे॒व॒त्वाय॑ सम॒न्तम्प॒र्यव॑द्यति सम॒न्तमे॒वेन्द्रि॒यं वी॒र्यं॑ यज॑माने दधाति व्य॒त्यास॒मन्वा॒हानि॑र्दाहा॒याश्व॑ ऋष॒भो वृ॒ष्णिर्ब॒स्तः सा दक्षि॑णा वृष॒त्वायै॒तयै॒व य॑जेताभिश॒स्यमा॑न ए॒ताश्चेद्वा अ॑स्य दे॒वता॒ अन्न॑म॒दन्त्य॒दन्त्यु॑वे॒वास्य॑ मनु॒ष्याः᳚॥३२॥

%2.3.8.0
{\anuvakamend[{इ॒न्द्रि॒यका॑मः सवि॒तुस्तेज॒स्तत्पु॑रो॒डाशो॒\-ऽष्टात्रिꣳ॑शच्च}]}%॥७॥

%2.3.8.1
रज॑नो॒ वै कौ॑णे॒यः क्र॑तु॒जितं॒ जान॑किं चक्षु॒र्वन्य॑मया॒त्तस्मा॑ ए॒तामिष्टिं॒ निर॑वपद॒ग्नये॒ भ्राज॑स्वते पुरो॒डाश॑म॒ष्टाक॑पालꣳ सौ॒र्यं च॒रुम॒ग्नये॒ भ्राज॑स्वते पुरो॒डाश॑म॒ष्टाक॑पाल॒न्तयै॒वास्मि॒ञ्चक्षु॑रदधा॒द्यश्चक्षु॑ष्कामः॒ स्यात्तस्मा॑ ए॒तामिष्टिं॒ निर्व॑पेद॒ग्नये॒ भ्राज॑स्वते पुरो॒डाश॑म॒ष्टाक॑पालꣳ सौ॒र्यं च॒रुम॒ग्नये॒ भ्राज॑स्वते पुरो॒डाश॑म॒ष्टाक॑पालम॒ग्नेर्वै चक्षु॑षा मनु॒ष्या॑ वि॥३३॥

%2.3.8.2
प॒श्य॒न्ति॒ सूर्य॑स्य दे॒वा अ॒ग्निं चै॒व सूर्यं॑ च॒ स्वेन॑ भाग॒धेये॒नोप॑ धावति॒ तावे॒वास्मि॒ञ्चक्षु॑र्धत्त॒श्चक्षु॑ष्माने॒व भ॑वति॒ यदा᳚ग्ने॒यौ भव॑त॒श्चक्षु॑षी ए॒वास्मि॒न्तत्प्रति॑ दधाति॒ यथ्सौ॒र्यो नासि॑कां॒ तेना॒भितः॑ सौ॒र्यमा᳚ग्ने॒यौ भ॑वत॒स्तस्मा॑द॒भितो॒ नासि॑कां॒ चक्षु॑षी॒ तस्मा॒न्नासि॑कया॒ चक्षु॑षी॒ विधृ॑ते समा॒नी या᳚ज्यानुवा॒क्ये॑ भवतः समा॒नꣳ हि चक्षुः॒ समृ॑द्ध्या॒ उदु॒ त्यं जा॒तवे॑दसꣳ स॒प्त त्वा॑ ह॒रितो॒ रथे॑ चि॒त्रं दे॒वाना॒मुद॑गा॒दनी॑क॒मिति॒ पिण्डा॒न्प्र य॑च्छति॒ चक्षु॑रे॒वास्मै॒ प्र य॑च्छति॒ यदे॒व तस्य॒ तत्॥३४॥

%2.3.9.0
{\anuvakamend[{वि ह्य॑ष्टाविꣳ॑शतिश्च}]}%॥८॥

%2.3.9.1
ध्रु॒वो॑\-ऽसि ध्रु॒वो॑\-ऽहꣳ स॑जा॒तेषु॑ भूयासं॒ धीर॒श्चेत्ता॑ वसु॒विद्ध्रु॒वो॑\-ऽसि ध्रु॒वो॑\-ऽहꣳ स॑जा॒तेषु॑ भूयासमु॒ग्रश्चेत्ता॑ वसु॒विद्ध्रु॒वो॑\-ऽसि ध्रु॒वो॑\-ऽहꣳ स॑जा॒तेषु॑ भूयासमभि॒भूश्चेत्ता॑ वसु॒विदाम॑नम॒स्याम॑नस्य देवा॒ ये स॑जा॒ताः कु॑मा॒राः सम॑नस॒स्तान॒हं का॑मये हृ॒दा ते मां का॑मयन्ताꣳ हृ॒दा तान्म॒ आम॑नसः कृधि॒ स्वाहाम॑नमसि॥३५॥

%2.3.9.2
आम॑नस्य देवा॒ याः स्त्रियः॒ सम॑नस॒स्ता अ॒हं का॑मये हृ॒दा ता मां का॑मयन्ताꣳ हृ॒दा ता म॒ आम॑नसः कृधि॒ स्वाहा॑ वैश्वदे॒वीꣳ सा᳚ङ्ग्रह॒णीं निर्व॑पे॒द्ग्राम॑कामो वैश्वदे॒वा वै स॑जा॒ता विश्वा॑ने॒व दे॒वान्थ्स्वेन॑ भाग॒धेये॒नोप॑ धावति॒ त ए॒वास्मै॑ सजा॒तान्प्र य॑च्छन्ति ग्रा॒म्ये॑व भ॑वति साङ्ग्रह॒णी भ॑वति मनो॒ग्रह॑णं॒ वै सं॒ग्रह॑ण॒म्मन॑ ए॒व स॑जा॒ताना᳚म्॥३६॥

%2.3.9.3
गृ॒ह्णा॒ति॒ ध्रु॒वो॑\-ऽसि ध्रु॒वो॑\-ऽहꣳ स॑जा॒तेषु॑ भूयास॒मिति॑ परि॒धीन्परि॑ दधात्या॒शिष॑मे॒वैतामा शा॒स्ते\-ऽथो॑ ए॒तदे॒व सर्वꣳ॑ सजा॒तेष्वधि॑ भवति॒ यस्यै॒वं वि॒दुष॑ ए॒ते प॑रि॒धयः॑ परिधी॒यन्त॒ आम॑नम॒स्याम॑नस्य देवा॒ इति॑ ति॒स्र आहु॑तीर्जुहोत्ये॒ताव॑न्तो॒ वै स॑जा॒ता ये म॒हान्तो॒ ये क्षु॑ल्ल॒का याः स्त्रिय॒स्ताने॒वाव॑ रुन्द्धे॒ त ए॑न॒मव॑रुद्धा॒ उप॑ तिष्ठन्ते॥३७॥

%2.3.10.0
{\anuvakamend[{स्वाहाम॑नमसि सजा॒तानाꣳ॑ रुन्द्धे॒ पञ्च॑ च}]}%॥९॥

%2.3.10.1
यन्नव॒मैत्तन्नव॑नीतमभव॒द्यदस॑र्प॒त्तथ्स॒र्पिर॑भव॒द्यदध्रि॑यत॒ तद्घृ॒तम॑भवद॒श्विनोः᳚ प्रा॒णो॑\-ऽसि॒ तस्य॑ ते दत्तां॒ ययोः᳚ प्रा॒णो\-ऽसि॒ स्वाहेन्द्र॑स्य प्रा॒णो॑\-ऽसि॒ तस्य॑ ते ददातु॒ यस्य॑ प्रा॒णो\-ऽसि॒ स्वाहा॑ मि॒त्रावरु॑णयोः प्रा॒णो॑\-ऽसि॒ तस्य॑ ते दत्तां॒ ययोः᳚ प्रा॒णो\-ऽसि॒ स्वाहा॒ विश्वे॑षां दे॒वानां᳚ प्रा॒णो॑\-ऽसि॥३८॥

%2.3.10.2
तस्य॑ ते ददतु॒ येषां᳚ प्रा॒णो\-ऽसि॒ स्वाहा॑ घृ॒तस्य॒ धारा॑म॒मृत॑स्य॒ पन्था॒मिन्द्रे॑ण द॒त्ताम्प्रय॑ताम्म॒रुद्भिः॑। तत्त्वा॒ विष्णुः॒ पर्य॑पश्य॒त्तत्त्वेडा॒ गव्यैर॑यत्। पा॒व॒मा॒नेन॑ त्वा॒ स्तोमे॑न गाय॒त्रस्य॑ वर्त॒न्योपा॒ꣳ॒शोर्वी॒र्ये॑ण दे॒वस्त्वा॑ सवि॒तोथ्सृ॑जतु जी॒वात॑वे जीवन॒स्यायै॑ बृहद्रथन्त॒रयो᳚स्त्वा॒ स्तोमे॑न त्रि॒ष्टुभो॑ वर्त॒न्या शु॒क्रस्य॑ वी॒र्ये॑ण दे॒वस्त्वा॑ सवि॒तोत्॥३९॥

%2.3.10.3
सृ॒ज॒तु॒ जी॒वात॑वे जीवन॒स्याया॑ अ॒ग्नेस्त्वा॒ मात्र॑या॒ जग॑त्यै वर्त॒न्याग्र॑य॒णस्य॑ वी॒र्ये॑ण दे॒वस्त्वा॑ सवि॒तोथ्सृ॑जतु जी॒वात॑वे जीवन॒स्याया॑ इ॒मम॑ग्न॒ आयु॑षे॒ वर्च॑से कृधि प्रि॒यꣳ रेतो॑ वरुण सोम राजन्न्। मा॒तेवा᳚स्मा अदिते॒ शर्म॑ यच्छ॒ विश्वे॑ देवा॒ जर॑दष्टि॒र्यथास॑त्। अ॒ग्निरायु॑ष्मा॒न्थ्स वन॒स्पति॑भि॒रायु॑ष्मा॒न्तेन॒ त्वायु॒षायु॑ष्मन्तं करोमि॒ सोम॒ आयु॑ष्मा॒न्थ्स ओष॑धीभिर्य॒ज्ञ आयु॑ष्मा॒न्थ्स दक्षि॑णाभि॒र्ब्रह्मायु॑ष्म॒त्तद्ब्रा᳚ह्म॒णैरायु॑ष्मद्दे॒वा आयु॑ष्मन्त॒स्ते॑\-ऽमृते॑न पि॒तर॒ आयु॑ष्मन्त॒स्ते स्व॒धयायु॑ष्मन्त॒स्तेन॒ त्वायु॒षायु॑ष्मन्तं करोमि॥४०॥

%2.3.11.0
{\anuvakamend[{विश्वे॑षां दे॒वानां᳚ प्रा॒णो॑\-ऽसि त्रि॒ष्टुभो॑ वर्त॒न्या शु॒क्रस्य॑ वी॒र्ये॑ण दे॒वस्त्वा॑ सवि॒तोथ्सोम॒ आयु॑ष्मा॒न्पञ्च॑विꣳशतिश्च}]}%॥10॥

%2.3.11.1
अ॒ग्निं वा ए॒तस्य॒ शरी॑रं गच्छति॒ सोम॒ꣳ॒ रसो॒ वरु॑ण एनं वरुणपा॒शेन॑ गृह्णाति॒ सर॑स्वतीं॒ वाग॒ग्नाविष्णू॑ आ॒त्मा यस्य॒ ज्योगा॒मय॑ति॒ यो ज्योगा॑मयावी॒ स्याद्यो वा॑ का॒मये॑त॒ सर्व॒मायु॑रिया॒मिति॒ तस्मा॑ ए॒तामिष्टिं॒ निर्व॑पेदाग्ने॒यम॒ष्टाक॑पालꣳ सौ॒म्यं च॒रुं वा॑रु॒णं दश॑कपालꣳ सारस्व॒तं च॒रुमा᳚ग्नावैष्ण॒वमेका॑दशकपालम॒ग्नेरे॒वास्य॒ शरी॑रं निष्क्री॒णाति॒ सोमा॒द्रसम्᳚॥४१॥

%2.3.11.2
वा॒रु॒णेनै॒वैनं॑ वरुणपा॒शान्मु॑ञ्चति सारस्व॒तेन॒ वाचं॑ दधात्य॒ग्निः सर्वा॑ दे॒वता॒ विष्णु॑र्य॒ज्ञो दे॒वता॑भिश्चै॒वैनं॑ य॒ज्ञेन॑ च भिषज्यत्यु॒त यदी॒तासु॒र्भव॑ति॒ जीव॑त्ये॒व यन्नव॒मैत्तन्नव॑नीतमभव॒दित्याज्य॒मवे᳚क्षते रू॒पमे॒वास्यै॒तन्म॑हि॒मानं॒ व्याच॑ष्टे॒\-ऽश्विनोः᳚ प्रा॒णो॑\-ऽसीत्या॑हा॒श्विनौ॒ वै दे॒वाना᳚म्॥४२॥

%2.3.11.3
भि॒षजौ॒ ताभ्या॑मे॒वास्मै॑ भेष॒जं क॑रो॒तीन्द्र॑स्य प्रा॒णो॑\-ऽसीत्या॑हेन्द्रि॒यमे॒वास्मि॑न्ने॒तेन॑ दधाति मि॒त्रावरु॑णयोः प्रा॒णो॑\-ऽसीत्या॑ह प्राणापा॒नावे॒वास्मि॑न्ने॒तेन॑ दधाति॒ विश्वे॑षां दे॒वानां᳚ प्रा॒णो॑\-ऽसीत्या॑ह वी॒र्य॑मे॒वास्मि॑न्ने॒तेन॑ दधाति घृ॒तस्य॒ धारा॑म॒मृत॑स्य॒ पन्था॒मित्या॑ह यथाय॒जुरे॒वैतत्पा॑वमा॒नेन॑ त्वा॒ स्तोमे॒नेति॑॥४३॥

%2.3.11.4
आ॒ह॒ प्रा॒णमे॒वास्मि॑न्ने॒तेन॑ दधाति बृहद्रथन्त॒रयो᳚स्त्वा॒ स्तोमे॒नेत्या॒हौज॑ ए॒वास्मि॑न्ने॒तेन॑ दधात्य॒ग्नेस्त्वा॒ मात्र॒येत्या॑हा॒त्मान॑मे॒वास्मि॑न्ने॒तेन॑ दधात्यृ॒त्विजः॒ पर्या॑हु॒र्याव॑न्त ए॒वर्त्विज॒स्त ए॑नम्भिषज्यन्ति ब्र॒ह्मणो॒ हस्त॑मन्वा॒रभ्य॒ पर्या॑हुरेक॒धैव यज॑मान॒ आयु॑र्दधति॒ यदे॒व तस्य॒ तद्धिर॑ण्यात्॥४४॥

%2.3.11.5
घृ॒तं निष्पि॑ब॒त्यायु॒र्वै घृ॒तम॒मृत॒ꣳ॒ हिर॑ण्यम॒मृता॑दे॒वायु॒र्निष्पि॑बति श॒तमा॑नम्भवति श॒तायुः॒ पुरु॑षः श॒तेन्द्रि॑य॒ आयु॑ष्ये॒वेन्द्रि॒ये प्रति॑ तिष्ठ॒त्यथो॒ खलु॒ याव॑तीः॒ समा॑ ए॒ष्यन्मन्ये॑त॒ ताव॑न्मानꣴ स्या॒थ्समृ॑द्ध्या इ॒मम॑ग्न॒ आयु॑षे॒ वर्च॑से कृ॒धीत्या॒हायु॑रे॒वास्मि॒न्वर्चो॑ दधाति॒ विश्वे॑ देवा॒ जर॑दष्टि॒र्यथास॒दित्या॑ह॒ जर॑दष्टिमे॒वैनं॑ करोत्य॒ग्निरायु॑ष्मा॒निति॒ हस्तं॑ गृह्णात्ये॒ते वै दे॒वा आयु॑ष्मन्त॒स्त ए॒वास्मि॒न्नायु॑र्दधति॒ सर्व॒मायु॑रेति॥४५॥

%2.3.12.0
{\anuvakamend[{रसं॑ दे॒वाना॒ꣴ॒ स्तोमे॒नेति॒ हिर॑ण्या॒दस॒दिति॒ द्वाविꣳ॑शतिश्च}]}%॥11॥

%2.3.12.1
प्र॒जाप॑ति॒र्वरु॑णा॒याश्व॑मनय॒थ्स स्वां दे॒वता॑मार्च्छ॒थ्स पर्य॑दीर्यत॒ स ए॒तं वा॑रु॒णं चतु॑ष्कपालमपश्य॒त्तं निर॑वप॒त्ततो॒ वै स व॑रुणपा॒शाद॑मुच्यत॒ वरु॑णो॒ वा ए॒तं गृ॑ह्णाति॒ यो\-ऽश्वं॑ प्रतिगृ॒ह्णाति॒ याव॒तो\-ऽश्वा᳚न्प्रतिगृह्णी॒यात्ताव॑तो वारु॒णाञ्चतु॑ष्कपाला॒न्निर्व॑पे॒द्वरु॑णमे॒व स्वेन॑ भाग॒धेये॒नोप॑ धावति॒ स ए॒वैनं॑ वरुणपा॒शान्मु॑ञ्चति॥४६॥

%2.3.12.2
चतु॑ष्कपाला भवन्ति॒ चतु॑ष्पा॒द्ध्यश्वः॒ समृ॑द्ध्या॒ एक॒मति॑रिक्तं॒ निर्व॑पे॒द्यमे॒व प्र॑तिग्रा॒ही भव॑ति॒ यं वा॒ नाध्येति॒ तस्मा॑दे॒व व॑रुणपा॒शान्मु॑च्यते॒ यद्यप॑रं प्रतिग्रा॒ही स्याथ्सौ॒र्यमेक॑कपाल॒मनु॒ निर्व॑पेद॒मुमे॒वादि॒त्यमु॑च्चा॒रं कु॑रुते॒\-ऽपो॑\-ऽवभृ॒थमवै᳚त्य॒फ्सु वै वरु॑णः सा॒क्षादे॒व वरु॑ण॒मव॑ यजते\-ऽपोन॒प्त्रीयं॑ च॒रुम्पुन॒रेत्य॒ निर्व॑पेद॒फ्सुयो॑नि॒र्वा अश्वः॒ स्वामे॒वैनं॒ योनिं॑ गमयति॒ स ए॑नꣳ शा॒न्त उप॑ तिष्ठते॥४७॥

%2.3.13.0
{\anuvakamend[{मु॒ञ्च॒ति॒ च॒रुꣳ स॒प्तद॑श च}]}%॥12॥

%2.3.13.1
या वा॑मिन्द्रावरुणा यत॒व्या॑ त॒नूस्तये॒ममꣳह॑सो मुञ्चतं॒ या वा॑मिन्द्रावरुणा सह॒स्या॑ रक्ष॒स्या॑ तेज॒स्या॑ त॒नूस्तये॒ममꣳह॑सो मुञ्चतं॒ यो वा॑मिन्द्रावरुणाव॒ग्नौ स्राम॒स्तं वा॑मे॒तेनाव॑ यजे॒ यो वा॑मिन्द्रावरुणा द्वि॒पाथ्सु॑ प॒शुषु॒ चतु॑ष्पाथ्सु गो॒ष्ठे गृ॒हेष्व॒फ्स्वोष॑धीषु॒ वन॒स्पति॑षु॒ स्राम॒स्तं वा॑मे॒तेनाव॑ यज॒ इन्द्रो॒ वा ए॒तस्य॑॥४८॥

%2.3.13.2
इ॒न्द्रि॒येणाप॑ क्रामति॒ वरु॑ण एनं वरुणपा॒शेन॑ गृह्णाति॒ यः पा॒प्मना॑ गृही॒तो भव॑ति॒ यः पा॒प्मना॑ गृही॒तः स्यात्तस्मा॑ ए॒तामै᳚न्द्रावरु॒णीम्प॑य॒स्यां᳚ निर्व॑पे॒दिन्द्र॑ ए॒वास्मि॑न्निन्द्रि॒यं द॑धाति॒ वरु॑ण एनं वरुणपा॒शान्मु॑ञ्चति पय॒स्या॑ भवति॒ पयो॒ हि वा ए॒तस्मा॑दप॒क्राम॒त्यथै॒ष पा॒प्मना॑ गृही॒तो यत्प॑य॒स्या॑ भव॑ति॒ पय॑ ए॒वास्मि॒न्तया॑ दधाति पय॒स्या॑याम्॥४९॥

%2.3.13.3
पु॒रो॒डाश॒मव॑ दधात्यात्म॒न्वन्त॑मे॒वैनं॑ करो॒त्यथो॑ आ॒यत॑नवन्तमे॒व च॑तु॒र्धा व्यू॑हति दि॒क्ष्वे॑व प्रति॑ तिष्ठति॒ पुनः॒ समू॑हति दि॒ग्भ्य ए॒वास्मै॑ भेष॒जं क॑रोति स॒मूह्याव॑ द्यति॒ यथावि॑द्धं निष्कृ॒न्तति॑ ता॒दृगे॒व तद्यो वा॑मिन्द्रावरुणाव॒ग्नौ स्राम॒स्तं वा॑मे॒तेनाव॑ यज॒ इत्या॑ह॒ दुरि॑ष्ट्या ए॒वैन॑म्पाति॒ यो वा॑मिन्द्रावरुणा द्वि॒पाथ्सु॑ प॒शुषु॒ स्राम॒स्तं वा॑मे॒तेनाव॑ यज॒ इत्या॑है॒ताव॑ती॒र्वा आप॒ ओष॑धयो॒ वन॒स्पत॑यः प्र॒जाः प॒शव॑ उपजीव॒नीया॒स्ता ए॒वास्मै॑ वरुणपा॒शान्मु॑ञ्चति॥५०॥

%2.3.14.0
{\anuvakamend[{ए॒तस्य॑ पय॒स्या॑याम्पाति॒ षड्विꣳ॑शतिश्च}]}%॥13॥

%2.3.14.1
स प्र॑त्न॒वन्नि काव्येन्द्रं॑ वो वि॒श्वत॒स्परीन्द्रं॒ नरः॑। त्वं नः॑ सोम वि॒श्वतो॒ रक्षा॑ राजन्नघाय॒तः। न रि॑ष्ये॒त्त्वाव॑तः॒ सखा᳚। या ते॒ धामा॑नि दि॒वि या पृ॑थि॒व्यां या पर्व॑ते॒ष्वोष॑धीष्व॒फ्सु। तेभि॑र्नो॒ विश्वैः᳚ सु॒मना॒ अहे॑ड॒न्राज᳚न्थ्सोम॒ प्रति॑ ह॒व्या गृ॑भाय। अग्नी॑षोमा॒ सवे॑दसा॒ सहू॑ती वनतं॒ गिरः॑। सं दे॑व॒त्रा ब॑भूवथुः। यु॒वम्॥५१॥

%2.3.14.2
ए॒तानि॑ दि॒वि रो॑च॒नान्य॒ग्निश्च॑ सोम॒ सक्र॑तू अधत्तम्। यु॒वꣳ सिन्धूꣳ॑ र॒भिश॑स्तेरव॒द्यादग्नी॑षोमा॒वमु॑ञ्चतं गृभी॒तान्। अग्नी॑षोमावि॒मꣳ सु मे॑ शृणु॒तं वृ॑षणा॒ हवम्᳚। प्रति॑ सू॒क्तानि॑ हर्यत॒म्भव॑तं दा॒शुषे॒ मयः॑। आन्यं दि॒वो मा॑त॒रिश्वा॑ जभा॒राम॑थ्नाद॒न्यं परि॑ श्ये॒नो अद्रेः᳚। अग्नी॑षोमा॒ ब्रह्म॑णा वावृधा॒नोरुं य॒ज्ञाय॑ चक्रथुरु लो॒कम्। अग्नी॑षोमा ह॒विषः॒ प्रस्थि॑तस्य वी॒तम्॥५२॥

%2.3.14.3
हर्य॑तं वृषणा जु॒षेथा᳚म्। सु॒शर्मा॑णा॒ स्वव॑सा॒ हि भू॒तमथा॑ धत्तं॒ यज॑मानाय॒ शं योः। आ प्या॑यस्व॒ सं ते᳚। ग॒णानां᳚ त्वा ग॒णप॑तिꣳ हवामहे क॒विं क॑वी॒नामु॑प॒मश्र॑वस्तमम्। ज्ये॒ष्ठ॒राजं॒ ब्रह्म॑णां ब्रह्मणस्पत॒ आ नः॑ शृ॒ण्वन्नू॒तिभिः॑ सीद॒ साद॑नम्। स इज्जने॑न॒ स वि॒शा स जन्म॑ना॒ स पु॒त्रैर्वाज॑म्भरते॒ धना॒ नृभिः॑। दे॒वानां॒ यः पि॒तर॑मा॒विवा॑सति॥५३॥

%2.3.14.4
श्र॒द्धाम॑ना ह॒विषा॒ ब्रह्म॑ण॒स्पतिम्᳚। स सु॒ष्टुभा॒ स ऋक्व॑ता ग॒णेन॑ व॒लꣳ रु॑रोज फलि॒गꣳ रवे॑ण। बृह॒स्पति॑रु॒स्रिया॑ हव्य॒सूदः॒ कनि॑क्रद॒द्वाव॑शती॒रुदा॑जत्। मरु॑तो॒ यद्ध॑ वो दि॒वो या वः॒ शर्म॑। अ॒र्य॒मा या॑ति वृष॒भस्तुवि॑ष्मान्दा॒ता वसू॑नां पुरुहू॒तो अर्\mbox{}हन्न्॑। स॒ह॒स्रा॒क्षो गो᳚त्र॒भिद्वज्र॑बाहुर॒स्मासु॑ दे॒वो द्रवि॑णं दधातु। ये ते᳚\-ऽर्यमन्ब॒हवो॑ देव॒यानाः॒ पन्था॑नः॥५४॥

%2.3.14.5
रा॒ज॒न्दि॒व आ॒चर॑न्ति। तेभि॑र्नो देव॒ महि॒ शर्म॑ यच्छ॒ शं न॑ एधि द्वि॒पदे॒ शं चतु॑ष्पदे। बु॒ध्नादग्र॒मङ्गि॑रोभिर्गृणा॒नो वि पर्व॑तस्य दृꣳहि॒तान्यै॑रत्। रु॒जद्रोधाꣳ॑सि कृ॒त्रिमा᳚ण्येषा॒ꣳ॒ सोम॑स्य॒ ता मद॒ इन्द्र॑श्चकार। बु॒ध्नादग्रे॑ण॒ वि मि॑माय॒ मानै॒र्वज्रे॑ण॒ खान्य॑तृणन्न॒दीना᳚म्। वृथा॑सृजत्प॒थिभि॑र्दीर्घया॒थैः सोम॑स्य॒ ता मद॒ इन्द्र॑श्चकार॥५॥

%2.3.14.6
प्र यो ज॒ज्ञे वि॒द्वाꣳ अ॒स्य बन्धुं॒ विश्वा॑नि दे॒वो जनि॑मा विवक्ति। ब्रह्म॒ ब्रह्म॑ण॒ उज्ज॑भार॒ मध्या᳚न्नी॒चादु॒च्चा स्व॒धया॒भि प्र त॑स्थौ। म॒हान्म॒ही अ॑स्तभाय॒द्वि जा॒तो द्याꣳ सद्म॒ पार्थि॑वं च॒ रजः॑। स बु॒ध्नादा᳚ष्ट ज॒नुषा॒भ्यग्र॒म्बृह॒स्पति॑र्दे॒वता॒ यस्य॑ स॒म्राट्। बु॒ध्नाद्यो अग्र॑म॒भ्यर्त्योज॑सा॒ बृह॒स्पति॒मा वि॑वासन्ति दे॒वाः। भि॒नद्व॒लं वि पुरो॑ दर्दरीति॒ कनि॑क्रद॒थ्सुव॑र॒पो जि॑गाय॥५६॥

%2.4.0.0
{\anuvakamend[{यु॒वं वी॒तमा॒ विवा॑सति॒ पन्था॑नो दीर्घया॒थैः सोम॑स्य॒ ता मद॒ इन्द्र॑श्चकार दे॒वा नव॑ च}]}%॥14॥

%2.4.0.0

{\anuvakamend[{दे॒वा म॑नु॒ष्या॑ देवासु॒रा अ॑ब्रुवन्देवासु॒रास्तेषा᳚ङ्गाय॒त्री प्र॒जाप॑ति॒स्ता यत्राग्ने॒ गोभि॑श्चि॒त्रया॑ मारु॒तन्देवा॑ वसव्या॒ अग्ने॑ मारु॒तमिति॒ देवा॑ वसव्या॒ देवाः᳚ शर्मण्या॒स्त्वष्टा॑ ह॒तपु॑त्रो दे॒वा वै रा॑ज॒न्या᳚न्नवो॑नव॒श्चतु॑र्दश}]%॥14॥
} 

\prashnaend[{दे॒वा म॑नु॒ष्याः᳚ प्र॒जां प॒शून्देवा॑ वसव्याः परिद॒ध्यादि॒दमस्म्य॒ष्टाच॑त्वारिꣳशत्॥48॥ दे॒वा म॑नु॒ष्या॑ मादयध्वम्॥}]
%%% END PRASHNA

\sect{चतुर्थः प्रश्नः}\setcounter{anuvakam}{0}
\dnsub{तैत्तिरीयसंहितायां द्वितीयकाण्डे चतुर्थः प्रश्नः}
%2.4.1.0
%2.4.1.1
दे॒वा म॑नु॒ष्याः᳚ पि॒तर॒स्ते᳚\-ऽन्यत॑ आस॒न्नसु॑रा॒ रक्षाꣳ॑सि पिशा॒चास्ते᳚\-ऽन्यत॒स्तेषां᳚ दे॒वाना॑मु॒त यदल्पं॒ लोहि॑त॒मकु॑र्व॒न्तद्रक्षाꣳ॑सि॒ रात्री॑भिरसुभ्न॒न्तान्थ्सु॒ब्धान्मृ॒तान॒भि व्यौ᳚च्छ॒त्ते दे॒वा अ॑विदु॒र्यो वै नो॒\-ऽयम्म्रि॒यते॒ रक्षाꣳ॑सि॒ वा इ॒मं घ्न॒न्तीति॒ ते रक्षा॒ꣳ॒स्युपा॑मन्त्रयन्त॒ तान्य॑ब्रुव॒न्वरं॑ वृणामहै॒ यत्॥१॥

%2.4.1.2
असु॑रा॒ञ्जया॑म॒ तन्नः॑ स॒हास॒दिति॒ ततो॒ वै दे॒वा असु॑रानजय॒न्ते\-ऽसु॑राञ्जि॒त्वा रक्षा॒ꣳ॒स्यपा॑नुदन्त॒ तानि॒ रक्षा॒ꣳ॒स्यनृ॑तमक॒र्तेति॑ सम॒न्तं दे॒वान्पर्य॑विश॒न्ते दे॒वा अ॒ग्नाव॑नाथन्त॒ ते᳚\-ऽग्नये॒ प्रव॑ते पुरो॒डाश॑म॒ष्टाक॑पालं॒ निर॑वपन्न॒ग्नये॑ विबा॒धव॑ते॒\-ऽग्नये॒ प्रती॑कवते॒ यद॒ग्नये॒ प्रव॑ते नि॒रव॑प॒न् यान्ये॒व पु॒रस्ता॒द्रक्षाꣳ॑सि॥२॥

%2.4.1.3
आस॒न्तानि॒ तेन॒ प्राणु॑दन्त॒ यद॒ग्नये॑ विबा॒धव॑ते॒ यान्ये॒वाभितो॒ रक्षा॒ꣳ॒स्यास॒न्तानि॒ तेन॒ व्य॑बाधन्त॒ यद॒ग्नये॒ प्रती॑कवते॒ यान्ये॒व प॒श्चाद्रक्षा॒ꣳ॒स्यास॒न्तानि॒ तेनापा॑नुदन्त॒ ततो॑ दे॒वा अभ॑व॒न्परासु॑रा॒ यो भ्रातृ॑व्यवा॒न्थ्स्याथ्स स्पर्ध॑मान ए॒तयेष्ट्या॑ यजेता॒ग्नये॒ प्रव॑ते पुरो॒डाश॑म॒ष्टाक॑पालं॒ निर्व॑पेद॒ग्नये॑ विबा॒धव॑ते॥३॥

%2.4.1.4
अ॒ग्नये॒ प्रती॑कवते॒ यद॒ग्नये॒ प्रव॑ते नि॒र्वप॑ति॒ य ए॒वास्मा॒च्छ्रेया॒न्भ्रातृ॑व्य॒स्तं तेन॒ प्र णु॑दते॒ यद॒ग्नये॑ विबा॒धव॑ते॒ य ए॒वैने॑न स॒दृङ्तं तेन॒ वि बा॑धते॒ यद॒ग्नये॒ प्रती॑कवते॒ य ए॒वास्मा॒त्पापी॑या॒न्तं तेनाप॑ नुदते॒ प्र श्रेयाꣳ॑स॒म्भ्रातृ॑व्यं नुद॒ते\-ऽति॑ स॒दृशं॑ क्रामति॒ नैन॒म्पापी॑यानाप्नोति॒ य ए॒वं वि॒द्वाने॒तयेष्ट्या॒ यज॑ते॥४॥

%2.4.2.0
{\anuvakamend[{वृ॒णा॒म॒है॒ यत्पु॒रस्ता॒द्रक्षाꣳ॑सि वपेद॒ग्नये॑ विबा॒धव॑त ए॒वं च॒त्वारि॑ च}]}%॥१॥

%2.4.2.1
दे॒वा॒सु॒राः संय॑त्ता आस॒न्ते दे॒वा अ॑ब्रुव॒न् यो नो॑ वी॒र्या॑वत्तम॒स्तमनु॑ स॒मार॑भामहा॒ इति॒ त इन्द्र॑मब्रुव॒न्त्वं वै नो॑ वी॒र्या॑वत्तमो\-ऽसि॒ त्वामनु॑ स॒मार॑भामहा॒ इति॒ सो᳚\-ऽब्रवीत्ति॒स्रो म॑ इ॒मास्त॒नुवो॑ वी॒र्या॑वती॒स्ताः प्री॑णी॒ताथासु॑रान॒भि भ॑विष्य॒थेति॒ ता वै ब्रू॒हीत्य॑ब्रुवन्नि॒यमꣳ॑हो॒मुगि॒यं वि॑मृ॒धेयमि॑न्द्रि॒याव॑ती॥५॥

%2.4.2.2
इत्य॑ब्रवी॒त्त इन्द्रा॑याꣳहो॒मुचे॑ पुरो॒डाश॒मेका॑दशकपालं॒ निर॑वप॒न्निन्द्रा॑य वैमृ॒धायेन्द्रा॑येन्द्रि॒याव॑ते॒ यदिन्द्रा॑याꣳहो॒मुचे॑ नि॒रव॑प॒न्नꣳह॑स ए॒व तेना॑मुच्यन्त॒ यदिन्द्रा॑य वैमृ॒धाय॒ मृध॑ ए॒व तेनापा᳚घ्नत॒ यदिन्द्रा॑येन्द्रि॒याव॑त इन्द्रि॒यमे॒व तेना॒त्मन्न॑दधत॒ त्रय॑स्त्रिꣳशत्कपालं पुरो॒डाशं॒ निर॑वप॒न्त्रय॑स्त्रिꣳश॒द्वै दे॒वता॒स्ता इन्द्र॑ आ॒त्मन्ननु॑ स॒मार॑म्भयत॒ भूत्यै᳚॥६॥

%2.4.2.3
तां वाव दे॒वा विजि॑तिमुत्त॒मामसु॑रै॒र्व्य॑जयन्त॒ यो भ्रातृ॑व्यवा॒न्थ्स्याथ्स स्पर्ध॑मान ए॒तयेष्ट्या॑ यजे॒तेन्द्रा॑याꣳहो॒मुचे॑ पुरो॒डाश॒मेका॑दशकपालं॒ निर्व॑पे॒दिन्द्रा॑य वैमृ॒धायेन्द्रा॑येन्द्रि॒याव॒ते\-ऽꣳह॑सा॒ वा ए॒ष गृ॑ही॒तो यस्मा॒च्छ्रेया॒न्भ्रातृ॑व्यो॒ यदिन्द्रा॑याꣳहो॒मुचे॑ नि॒र्वप॒त्यꣳह॑स ए॒व तेन॑ मुच्यते मृ॒धा वा ए॒षो॑\-ऽभिष॑ण्णो॒ यस्मा᳚थ्समा॒नेष्व॒न्यः श्रेया॑नु॒त॥७॥

%2.4.2.4
अभ्रा॑तृव्यो॒ यदिन्द्रा॑य वैमृ॒धाय॒ मृध॑ ए॒व तेनाप॑ हते॒ यदिन्द्रा॑येन्द्रि॒याव॑त इन्द्रि॒यमे॒व तेना॒त्मन्ध॑त्ते॒ त्रय॑स्त्रिꣳशत्कपालं पुरो॒डाशं॒ निर्व॑पति॒ त्रय॑स्त्रिꣳश॒द्वै दे॒वता॒स्ता ए॒व यज॑मान आ॒त्मन्ननु॑ स॒मार॑म्भयते॒ भूत्यै॒ सा वा ए॒षा विजि॑ति॒र्नामेष्टि॒र्य ए॒वं वि॒द्वाने॒तयेष्ट्या॒ यज॑त उत्त॒मामे॒व विजि॑ति॒म्भ्रातृ॑व्येण॒ वि ज॑यते॥८॥

%2.4.3.0
{\anuvakamend[{इ॒न्द्रि॒याव॑ती॒ भूत्या॑ उ॒तैका॒न्नप॑ञ्चा॒शच्च॑}]}%॥२॥

%2.4.3.1
दे॒वा॒सु॒राः संय॑त्ता आस॒न्तेषां᳚ गाय॒त्र्योजो॒ बल॑मिन्द्रि॒यं वी॒र्यं॑ प्र॒जां प॒शून्थ्सं॒गृह्या॒दाया॑प॒क्रम्या॑तिष्ठ॒त्ते॑\-ऽमन्यन्त यत॒रान् वा इ॒यमु॑पाव॒र्थ्स्यति॒ त इ॒दम्भ॑विष्य॒न्तीति॒ तां व्य॑ह्वयन्त॒ विश्व॑कर्म॒न्निति॑ दे॒वा दाभीत्यसु॑राः॒ सा नान्य॑त॒राꣴश्च॒ नोपाव॑र्तत॒ ते दे॒वा ए॒तद्यजु॑रपश्य॒न्नोजो॑\-ऽसि॒ सहो॑\-ऽसि॒ बल॑मसि॥९॥

%2.4.3.2
भ्राजो॑\-ऽसि दे॒वानां॒ धाम॒ नामा॑सि॒ विश्व॑मसि वि॒श्वायुः॒ सर्व॑मसि स॒र्वायु॑रभि॒भूरिति॒ वाव दे॒वा असु॑राणा॒मोजो॒ बल॑मिन्द्रि॒यं वी॒र्यं॑ प्र॒जां प॒शून॑वृञ्जत॒ यद्गा॑य॒त्र्य॑प॒क्रम्याति॑ष्ठ॒त्तस्मा॑दे॒तां गा॑य॒त्रीतीष्टि॑माहुः सं वथ्स॒रो वै गा॑य॒त्री सं॑वथ्स॒रो वै तद॑प॒क्रम्या॑तिष्ठ॒द्यदे॒तया॑ दे॒वा असु॑राणा॒मोजो॒ बल॑मिन्द्रि॒यं वी॒र्यम्᳚॥१०॥

%2.4.3.3
प्र॒जां प॒शूनवृ॑ञ्जत॒ तस्मा॑दे॒ताꣳ सं॑व॒र्ग इतीष्टि॑माहु॒र्यो भ्रातृ॑व्यवा॒न्थ्स्याथ्स स्पर्ध॑मान ए॒तयेष्ट्या॑ यजेता॒ग्नये॑ संव॒र्गाय॑ पुरो॒डाश॑म॒ष्टाक॑पालं॒ निर्व॑पे॒त्तꣳ शृ॒तमास॑न्नमे॒तेन॒ यजु॑षा॒भि मृ॑शे॒दोज॑ ए॒व बल॑मिन्द्रि॒यं वी॒र्यं॑ प्र॒जां प॒शून्भ्रातृ॑व्यस्य वृङ्क्ते॒ भव॑त्या॒त्मना॒ परा᳚स्य॒ भ्रातृ॑व्यो भवति॥११॥

%2.4.4.0
{\anuvakamend[{बल॑मस्ये॒तया॑ दे॒वा असु॑राणा॒मोजो॒ बल॑मिन्द्रि॒यं वी॒र्यं॑ पञ्च॑चत्वारिꣳशच्च}]}%॥३॥

%2.4.4.1
प्र॒जाप॑तिः प्र॒जा अ॑सृजत॒ ता अ॑स्माथ्सृ॒ष्टाः परा॑चीराय॒न्ता यत्राव॑स॒न्ततो॑ ग॒र्मुदुद॑तिष्ठ॒त्ता बृह॒स्पति॑श्चा॒न्ववै॑ता॒ꣳ॒ सो᳚\-ऽब्रवी॒द्बृह॒स्पति॑र॒नया᳚ त्वा॒ प्र ति॑ष्ठा॒न्यथ॑ त्वा प्र॒जा उ॒पाव॑र्थ्स्य॒न्तीति॒ तम्प्राति॑ष्ठ॒त्ततो॒ वै प्र॒जाप॑तिं प्र॒जा उ॒पाव॑र्तन्त॒ यः प्र॒जाका॑मः॒ स्यात्तस्मा॑ ए॒तम्प्रा॑जाप॒त्यं गा᳚र्मु॒तं च॒रुं निर्व॑पेत्प्र॒जाप॑तिम्॥१२॥

%2.4.4.2
ए॒व स्वेन॑ भाग॒धेये॒नोप॑ धावति॒ स ए॒वास्मै᳚ प्र॒जाम्प्र ज॑नयति प्र॒जाप॑तिः प॒शून॑सृजत॒ ते᳚\-ऽस्माथ्सृ॒ष्टाः परा᳚ञ्च आय॒न्ते यत्राव॑स॒न्ततो॑ ग॒र्मुदुद॑तिष्ठ॒त्तान्पू॒षा चा॒न्ववै॑ता॒ꣳ॒ सो᳚\-ऽब्रवीत्पू॒षानया॑ मा॒ प्र ति॒ष्ठाथ॑ त्वा प॒शव॑ उ॒पाव॑र्थ्स्य॒न्तीति॒ माम्प्र ति॒ष्ठेति॒ सोमो᳚\-ऽब्रवी॒न्मम॒ वै॥१३॥

%2.4.4.3
अ॒कृ॒ष्ट॒प॒च्यमित्यु॒भौ वा॒म्प्र ति॑ष्ठा॒नीत्य॑ब्रवी॒त्तौ प्राति॑ष्ठ॒त्ततो॒ वै प्र॒जाप॑तिम्प॒शव॑ उ॒पाव॑र्तन्त॒ यः प॒शुका॑मः॒ स्यात्तस्मा॑ ए॒तꣳ सो॑मापौ॒ष्णं गा᳚र्मु॒तं च॒रुं निर्व॑पेथ्सोमापू॒षणा॑वे॒व स्वेन॑ भाग॒धेये॒नोप॑ धावति॒ तावे॒वास्मै॑ प॒शून्प्र ज॑नयतः॒ सोमो॒ वै रे॑तो॒धाः पू॒षा प॑शू॒नाम्प्र॑जनयि॒ता सोम॑ ए॒वास्मै॒ रेतो॒ दधा॑ति पू॒षा प॒शून्प्र ज॑नयति॥१४॥

%2.4.5.0
{\anuvakamend[{व॒पे॒त्प्र॒जाप॑तिं॒ वै दधा॑ति पू॒षा त्रीणि॑ च}]}%॥४॥

%2.4.5.1
अग्ने॒ गोभि॑र्न॒ आ ग॒हीन्दो॑ पु॒ष्ट्या जु॑षस्व नः। इन्द्रो॑ ध॒र्ता गृ॒हेषु॑ नः॥ स॒वि॒ता यः स॑ह॒स्रियः॒ स नो॑ गृ॒हेषु॑ रारणत्। आ पू॒षा ए॒त्वा वसु॑॥ धा॒ता द॑दातु नो र॒यिमीशा॑नो॒ जग॑त॒स्पतिः॑। स नः॑ पू॒र्णेन॑ वावनत्॥ त्वष्टा॒ यो वृ॑ष॒भो वृषा॒ स नो॑ गृ॒हेषु॑ रारणत्। स॒हस्रे॑णा॒युते॑न च॥ येन॑ दे॒वा अ॒मृतम्᳚॥१५॥

%2.4.5.2
दी॒र्घꣴ श्रवो॑ दि॒व्यैर॑यन्त। राय॑स्पोष॒ त्वम॒स्मभ्यं॒ गवां᳚ कु॒ल्मिं जी॒वस॒ आ यु॑वस्व। अ॒ग्निर्गृ॒हप॑तिः॒ सोमो॑ विश्व॒वनिः॑ सवि॒ता सु॑मे॒धाः स्वाहा᳚। अग्ने॑ गृहपते॒ यस्ते॒ घृत्यो॑ भा॒गस्तेन॒ सह॒ ओज॑ आ॒क्रम॑माणाय धेहि॒ श्रैष्ठ्या᳚त्प॒थो मा यो॑षं मू॒र्धा भू॑यास॒ꣴ॒ स्वाहा᳚॥१६॥

%2.4.6.0
{\anuvakamend[{अ॒मृत॑म॒ष्टात्रिꣳ॑शच्च}]}%॥५॥

%2.4.6.1
चि॒त्रया॑ यजेत प॒शुका॑म इ॒यं वै चि॒त्रा यद्वा अ॒स्यां विश्व॑म्भू॒तमधि॑ प्र॒जाय॑ते॒ तेने॒यं चि॒त्रा य ए॒वं वि॒द्वाꣴश्चि॒त्रया॑ प॒शुका॑मो॒ यज॑ते॒ प्र प्र॒जया॑ प॒शुभि॑र्मिथु॒नैर्जा॑यते॒ प्रैवाग्ने॒येन॑ वापयति॒ रेतः॑ सौ॒म्येन॑ दधाति॒ रेत॑ ए॒व हि॒तं त्वष्टा॑ रू॒पाणि॒ वि क॑रोति सारस्व॒तौ भ॑वत ए॒तद्वै दैव्य॑म्मिथु॒नं दैव्य॑मे॒वास्मै᳚॥१७॥

%2.4.6.2
मि॒थु॒नम्म॑ध्य॒तो द॑धाति॒ पुष्ट्यै᳚ प्र॒जन॑नाय सिनीवा॒ल्यै च॒रुर्भ॑वति॒ वाग्वै सि॑नीवा॒ली पुष्टिः॒ खलु॒ वै वाक्पुष्टि॑मे॒व वाच॒मुपै᳚त्यै॒न्द्र उ॑त्त॒मो भ॑वति॒ तेनै॒व तन्मि॑थु॒नꣳ स॒प्तैतानि॑ ह॒वीꣳषि॑ भवन्ति स॒प्त ग्रा॒म्याः प॒शवः॑ स॒प्तार॒ण्याः स॒प्त छन्दाꣳ॑स्यु॒भय॒स्याव॑रुद्ध्या॒ अथै॒ता आहु॑तीर्जुहोत्ये॒ते वै दे॒वाः पुष्टि॑पतय॒स्त ए॒वास्मि॒न्पुष्टिं॑ दधति॒ पुष्य॑ति प्र॒जया॑ प॒शुभि॒रथो॒ यदे॒ता आहु॑तीर्जु॒होति॒ प्रति॑ष्ठित्यै॥१८॥

%2.4.7.0
{\anuvakamend[{अ॒स्मै॒ त ए॒व द्वाद॑श च}]}%॥६॥

%2.4.7.1
मा॒रु॒तम॑सि म॒रुता॒मोजो॒\-ऽपां धारां᳚ भिन्द्धि र॒मय॑त मरुतः श्ये॒नमा॒यिन॒म्मनो॑जवसं॒ वृष॑णꣳ सुवृ॒क्तिम्। येन॒ शर्ध॑ उ॒ग्रमव॑सृष्ट॒मेति॒ तद॑श्विना॒ परि॑ धत्तꣴ स्व॒स्ति। पु॒रो॒वा॒तो वर्\mbox{}ष॑ञ्जि॒न्वरा॒वृथ्स्वाहा॑ वा॒ताव॒द्वर्\mbox{}ष॑न्नु॒ग्ररा॒वृथ्स्वाहा᳚ स्त॒नय॒न्वर्\mbox{}ष॑न्भी॒मरा॒वृथ्स्वाहा॑नश॒न्य॑व॒स्फूर्ज॑न्दि॒द्युद्वर्\mbox{}ष॑न्त्वे॒षरा॒वृथ्स्वाहा॑तिरा॒त्रं वर्\mbox{}ष॑न्पू॒र्तिरा॒वृत्॥१९॥

%2.4.7.2
स्वाहा॑ ब॒हु हा॒यम॑वृषा॒दिति॑ श्रु॒तरा॒वृथ्स्वाहा॒तप॑ति॒ वर्\mbox{}ष॑न्वि॒राडा॒वृथ्स्वाहा॑व॒स्फूर्ज॑न्दि॒द्युद्वर्\mbox{}ष॑न्भू॒तरा॒वृथ्स्वाहा॒ मान्दा॒ वाशाः॒ शुन्ध्यू॒रजि॑राः। ज्योति॑ष्मती॒स्तम॑स्वरी॒रुन्द॑तीः॒ सुफे॑नाः। मित्र॑भृतः॒ क्षत्र॑भृतः॒ सुरा᳚ष्ट्रा इ॒ह मा॑\-ऽवत। वृष्णो॒ अश्व॑स्य सं॒दान॑मसि॒ वृष्ट्यै॒ त्वोप॑ नह्यामि॥२०॥

%2.4.8.0
{\anuvakamend[{पू॒र्तिरा॒वृद्द्विच॑त्वारिꣳशच्च}]}%॥७॥

%2.4.8.1
देवा॑ वसव्या॒ अग्ने॑ सोम सूर्य। देवाः᳚ शर्मण्या॒ मित्रा॑वरुणार्यमन्न्। देवाः᳚ सपीत॒यो\-ऽपां᳚ नपादाशुहेमन्न्। उ॒द्नो द॑त्तो\-ऽद॒धिम्भि॑न्त दि॒वः प॒र्जन्या॑द॒न्तरि॑क्षात्पृथि॒व्यास्ततो॑ नो॒ वृष्ट्या॑\-ऽवत। दिवा॑ चि॒त्तमः॑ कृण्वन्ति प॒र्जन्ये॑नोदवा॒हेन॑। पृ॒थि॒वीं यद्व्यु॒न्दन्ति॑। आ यं नरः॑ सु॒दान॑वो ददा॒शुषे॑ दि॒वः कोश॒मचु॑च्यवुः। वि प॒र्जन्याः᳚ सृजन्ति॒ रोद॑सी॒ अनु॒ धन्व॑ना यन्ति॥२१॥

%2.4.8.2
वृ॒ष्टयः॑। उदी॑रयथा मरुतः समुद्र॒तो यू॒यं वृ॒ष्टिं व॑र्\mbox{}षयथा पुरीषिणः। न वो॑ दस्रा॒ उप॑ दस्यन्ति धे॒नवः॒ शुभं॑ या॒तामनु॒ रथा॑ अवृथ्सत। सृ॒जा वृ॒ष्टिं दि॒व आद्भिः स॑मु॒द्रं पृ॑ण। अ॒ब्जा अ॑सि प्रथम॒जा बल॑मसि समु॒द्रियम्᳚। उन्न॑म्भय पृथि॒वीम्भि॒न्द्धीदं दि॒व्यं नभः॑। उ॒द्नो दि॒व्यस्य॑ नो दे॒हीशा॑नो॒ वि सृ॑जा॒ दृतिम्᳚। ये दे॒वा दि॒विभा॑गा॒ ये᳚\-ऽन्तरि॑क्षभागा॒ ये पृ॑थि॒विभा॑गाः। त इ॒मं य॒ज्ञम॑वन्तु॒ त इ॒दं क्षेत्र॒मा वि॑शन्तु॒ त इ॒दं क्षेत्र॒मनु॒ वि वि॑शन्तु॥२२॥

%2.4.9.0
{\anuvakamend[{य॒न्ति॒ दे॒वा विꣳ॑शति॒श्च॑}]}%॥८॥

%2.4.9.1
मा॒रु॒तम॑सि म॒रुता॒मोज॒ इति॑ कृ॒ष्णं वासः॑ कृ॒ष्णतू॑षं॒ परि॑ धत्त ए॒तद्वै वृष्ट्यै॑ रू॒पꣳ सरू॑प ए॒व भू॒त्वा प॒र्जन्यं॑ वर्\mbox{}षयति र॒मय॑त मरुतः श्ये॒नमा॒यिन॒मिति॑ पश्चाद्वा॒तं प्रति॑ मीवति पुरोवा॒तमे॒व ज॑नयति व॒र्\mbox{}षस्याव॑रुद्ध्यै वातना॒मानि॑ जुहोति वा॒युर्वै वृष्ट्या॑ ईशे वा॒युमे॒व स्वेन॑ भाग॒धेये॒नोप॑ धावति॒ स ए॒वास्मै॑ प॒र्जन्यं॑ वर्\mbox{}षयत्य॒ष्टौ॥२३॥

%2.4.9.2
जु॒हो॒ति॒ चत॑स्रो॒ वै दिश॒श्चत॑स्रो\-ऽवान्तरदि॒शा दि॒ग्भ्य ए॒व वृष्टि॒ꣳ॒ सम्प्र च्या॑वयति कृष्णाजि॒ने सं यौ॑ति ह॒विरे॒वाक॑रन्तर्वे॒दि सं यौ॒त्यव॑रुद्ध्यै॒ यती॑नाम॒द्यमा॑नानाꣳ शी॒र्\mbox{}षाणि॒ परा॑पत॒न्ते ख॒र्जूरा॑ अभव॒न्तेषा॒ꣳ॒ रस॑ ऊ॒र्ध्वो॑\-ऽपत॒त्तानि॑ क॒रीरा᳚ण्यभवन्थ्सौ॒म्यानि॒ वै क॒रीरा॑णि सौ॒म्या खलु॒ वा आहु॑तिर्दि॒वो वृष्टिं॑ च्यावयति॒ यत्क॒रीरा॑णि॒ भव॑न्ति॥२४॥

%2.4.9.3
सौ॒म्ययै॒वाहु॑त्या दि॒वो वृष्टि॒मव॑ रुन्द्धे॒ मधु॑षा॒ सं यौ᳚त्य॒पां वा ए॒ष ओष॑धीना॒ꣳ॒ रसो॒ यन्मध्व॒द्भ्य ए॒वौष॑धीभ्यो वर्\mbox{}ष॒त्यथो॑ अ॒द्भ्य ए॒वौष॑धीभ्यो॒ वृष्टिं॒ नि न॑यति॒ मान्दा॒ वाशा॒ इति॒ सं यौ॑ति नाम॒धेयै॑रे॒वैना॒ अच्छै॒त्यथो॒ यथा᳚ ब्रू॒यादसा॒वेहीत्ये॒वमे॒वैना॑ नाम॒धेयै॒रा॥२५॥

%2.4.9.4
च्या॒व॒य॒ति॒ वृष्णो॒ अश्व॑स्य सं॒दान॑मसि॒ वृष्ट्यै॒ त्वोप॑ नह्या॒मीत्या॑ह॒ वृषा॒ वा अश्वो॒ वृषा॑ प॒र्जन्यः॑ कृ॒ष्ण इ॑व॒ खलु॒ वै भू॒त्वा व॑र्\mbox{}षति रू॒पेणै॒वैन॒ꣳ॒ सम॑र्धयति व॒र्\mbox{}षस्याव॑रुद्ध्यै॥२६॥

%2.4.10.0
{\anuvakamend[{अ॒ष्टौ भव॑न्ति नाम॒धेयै॒रैका॒न्नत्रि॒ꣳ॒शच्च॑}]}%॥९॥

%2.4.10.1
देवा॑ वसव्या॒ देवाः᳚ शर्मण्या॒ देवाः᳚ सपीतय॒ इत्या ब॑ध्नाति दे॒वता॑भिरे॒वान्व॒हं वृष्टि॑मिच्छति॒ यदि॒ वर्\mbox{}षे॒त्ताव॑त्ये॒व हो॑त॒व्यं॑ यदि॒ न वर्\mbox{}षे॒च्छ्वो भू॒ते ह॒विर्निर्व॑पेदहोरा॒त्रे वै मि॒त्रावरु॑णावहोरा॒त्राभ्यां॒ खलु॒ वै प॒र्जन्यो॑ वर्\mbox{}षति॒ नक्तं॑ वा॒ हि दिवा॑ वा॒ वर्\mbox{}ष॑ति मि॒त्रावरु॑णावे॒व स्वेन॑ भाग॒धेये॒नोप॑ धावति॒ तावे॒वास्मै᳚॥२७॥

%2.4.10.2
अ॒हो॒रा॒त्रा\-भ्यां᳚ प॒र्जन्यं॑ वर्\mbox{}षयतो॒\-ऽग्नये॑ धाम॒च्छदे॑ पुरो॒डाश॑म॒ष्टाक॑पालं॒ निर्व॑पेन्मारु॒तꣳ स॒प्तक॑पालꣳ सौ॒र्यमेक॑कपालम॒ग्निर्वा इ॒तो वृष्टि॒मुदी॑रयति म॒रुतः॑ सृ॒ष्टां न॑यन्ति य॒दा खलु॒ वा अ॒सावा॑दि॒त्यो न्य॑ङ्र॒श्मिभिः॑ पर्या॒वर्त॒ते\-ऽथ॑ वर्\mbox{}षति धाम॒च्छदि॑व॒ खलु॒ वै भू॒त्वा व॑र्\mbox{}षत्ये॒ता वै दे॒वता॒ वृष्ट्या॑ ईशते॒ ता ए॒व स्वेन॑ भाग॒धेये॒नोप॑ धावति॒ ताः॥२८॥

%2.4.10.3
ए॒वास्मै॑ प॒र्जन्यं॑ वर्\mbox{}षयन्त्यु॒ताव॑र्\mbox{}षिष्य॒न्वर्\mbox{}ष॑त्ये॒व सृ॒जा वृ॒ष्टिं दि॒व आद्भिः स॑मु॒द्रं पृ॒णेत्या॑हे॒माश्चै॒वामूश्चा॒पः सम॑र्धय॒त्यथो॑ आ॒भिरे॒वामूरच्छै᳚त्य॒ब्जा अ॑सि प्रथम॒जा बल॑मसि समु॒द्रिय॒मित्या॑ह यथाय॒जुरे॒वैतदुन्न॑म्भय पृथि॒वीमिति॑ वर्\mbox{}षा॒ह्वां जु॑होत्ये॒षा वा ओष॑धीनां वृष्टि॒वनि॒स्तयै॒व वृष्टि॒मा च्या॑वयति॒ ये दे॒वा दि॒विभा॑गा॒ इति॑ कृष्णाजि॒नमव॑ धूनोती॒म ए॒वास्मै॑ लो॒काः प्री॒ता अ॒भीष्टा॑ भवन्ति॥२९॥

%2.4.11.0
{\anuvakamend[{अ॒स्मै॒ धा॒व॒ति॒ ता वा एक॑विꣳशतिश्च}]}%॥10॥

%2.4.11.1
सर्वा॑णि॒ छन्दाꣳ॑स्ये॒तस्या॒मिष्ट्या॑म॒नूच्या॒नीत्या॑हुस्त्रि॒ष्टुभो॒ वा ए॒तद्वी॒र्यं॑ यत्क॒कुदु॒ष्णिहा॒ जग॑त्यै॒ यदु॑ष्णिहक॒कुभा॑व॒न्वाह॒ तेनै॒व सर्वा॑णि॒ छन्दा॒ꣳ॒स्यव॑ रुन्द्धे गाय॒त्री वा ए॒षा यदु॒ष्णिहा॒ यानि॑ च॒त्वार्यध्य॒क्षरा॑णि॒ चतु॑ष्पाद ए॒व ते प॒शवो॒ यथा॑ पुरो॒डाशे॑ पुरो॒डाशो\-ऽध्ये॒वमे॒व तद्यदृ॒च्यध्य॒क्षरा॑णि॒ यज्जग॑त्या॥३०॥

%2.4.11.2
प॒रि॒द॒ध्यादन्तं॑ य॒ज्ञं ग॑मयेत्त्रि॒ष्टुभा॒ परि॑ दधातीन्द्रि॒यं वै वी॒र्यं॑ त्रि॒ष्टुगि॑न्द्रि॒य ए॒व वी॒र्ये॑ य॒ज्ञं प्रति॑ ष्ठापयति॒ नान्तं॑ गमय॒त्यग्ने॒ त्री ते॒ वाजि॑ना॒ त्री ष॒धस्थेति॒ त्रिव॑त्या॒ परि॑ दधाति सरूप॒त्वाय॒ सर्वो॒ वा ए॒ष य॒ज्ञो यत्त्रै॑धात॒वीय॒ङ्कामा॑यकामाय॒ प्र यु॑ज्यते॒ सर्वे᳚भ्यो॒ हि कामे᳚भ्यो य॒ज्ञः प्र॑यु॒ज्यते᳚ त्रैधात॒वीये॑न यजेताभि॒चर॒न्थ्सर्वो॒ वै॥३१॥

%2.4.11.3
ए॒ष य॒ज्ञो यत्त्रै॑धात॒वीय॒ꣳ॒ सर्वे॑णै॒वैनं॑ य॒ज्ञेना॒भि च॑रति स्तृणु॒त ए॒वैन॑मे॒तयै॒व य॑जेताभिच॒र्यमा॑णः॒ सर्वो॒ वा ए॒ष य॒ज्ञो यत्त्रै॑धात॒वीय॒ꣳ॒ सर्वे॑णै॒व य॒ज्ञेन॑ यजते॒ नैन॑मभि॒चर᳚न्थ्स्तृणुत ए॒तयै॒व य॑जेत स॒हस्रे॑ण य॒क्ष्यमा॑णः॒ प्रजा॑तमे॒वैन॑द्ददात्ये॒तयै॒व य॑जेत स॒हस्रे॑णेजा॒नो\-ऽन्तं॒ वा ए॒ष प॑शू॒नां ग॑च्छति॥३२॥

%2.4.11.4
यः स॒हस्रे॑ण॒ यज॑ते प्र॒जाप॑तिः॒ खलु॒ वै प॒शून॑सृजत॒ ताꣴ स्त्रै॑धात॒वीये॑नै॒वासृ॑जत॒ य ए॒वं वि॒द्वाꣴस्त्रै॑धात॒वीये॑न प॒शुका॑मो॒ यज॑ते॒ यस्मा॑दे॒व योनेः᳚ प्र॒जाप॑तिः प॒शूनसृ॑जत॒ तस्मा॑दे॒वैना᳚न्थ्सृजत॒ उपै॑न॒मुत्त॑रꣳ स॒हस्रं॑ नमति दे॒वता᳚भ्यो॒ वा ए॒ष आ वृ॑श्च्यते॒ यो य॒क्ष्य इत्यु॒क्त्वा न यज॑ते त्रैधात॒वीये॑न यजेत॒ सर्वो॒ वा ए॒ष य॒ज्ञः॥३३॥

%2.4.11.5
यत्त्रै॑धात॒वीय॒ꣳ॒ सर्वे॑णै॒व य॒ज्ञेन॑ यजते॒ न दे॒वता᳚भ्य॒ आ वृ॑श्च्यते॒ द्वाद॑शकपालः पुरो॒डाशो॑ भवति॒ ते त्रय॒श्चतु॑ष्कपालास्त्रिष्षमृद्ध॒त्वाय॒ त्रयः॑ पुरो॒डाशा॑ भवन्ति॒ त्रय॑ इ॒मे लो॒का ए॒षां लो॒काना॒माप्त्या॒ उत्त॑रउत्तरो॒ ज्याया᳚न्भवत्ये॒वमि॑व॒ हीमे लो॒का य॑व॒मयो॒ मध्य॑ ए॒तद्वा अ॒न्तरि॑क्षस्य रू॒पꣳ समृ॑द्ध्यै॒ सर्वे॑षामभिग॒मय॒न्नव॑ द्य॒त्यछ॑म्बट्कार॒ꣳ॒ हिर॑ण्यं ददाति॒ तेज॑ ए॒व॥३४॥

%2.4.11.6
अव॑ रुन्द्धे ता॒र्प्यं द॑दाति प॒शूने॒वाव॑ रुन्द्धे धे॒नुं द॑दात्या॒शिष॑ ए॒वाव॑ रुन्द्धे॒ साम्नो॒ वा ए॒ष वर्णो॒ यद्धिर॑ण्यं॒ यजु॑षां ता॒र्प्यमु॑क्थाम॒दानां᳚ धे॒नुरे॒ताने॒व सर्वा॒न् वर्णा॒नव॑ रुन्द्धे॥३५॥

%2.4.12.0
{\anuvakamend[{जग॑त्या\-ऽभि॒चर॒न्थ्सर्वो॒ वै ग॑च्छति य॒ज्ञस्तेज॑ ए॒व त्रि॒ꣳ॒शच्च॑}]}%॥11॥

%2.4.12.1
त्वष्टा॑ ह॒तपु॑त्रो॒ वीन्द्र॒ꣳ॒ सोम॒माह॑र॒त्तस्मि॒न्निन्द्र॑ उपह॒वमै᳚च्छत॒ तं नोपा᳚ह्वयत पु॒त्रम्मे॑\-ऽवधी॒रिति॒ स य॑ज्ञवेश॒सं कृ॒त्वा प्रा॒सहा॒ सोम॑मपिब॒त्तस्य॒ यद॒त्यशि॑ष्यत॒ तत्त्वष्टा॑हव॒नीय॒मुप॒ प्राव॑र्तय॒थ्स्वाहेन्द्र॑शत्रुर्वर्ध॒स्वेति॒ स याव॑दू॒र्ध्वः प॑रा॒विध्य॑ति॒ ताव॑ति स्व॒यमे॒व व्य॑रमत॒ यदि॑ वा॒ ताव॑त्प्रव॒णम्॥३६॥

%2.4.12.2
आसी॒द्यदि॑ वा॒ ताव॒दध्य॒ग्नेरासी॒थ्स स॒म्भव॑न्न॒ग्नीषोमा॑व॒भि सम॑भव॒थ्स इ॑षुमा॒त्रमि॑षुमात्रं॒ विष्व॑ङ्ङवर्धत॒ स इ॒माल्लोँ॒कान॑वृणो॒द्यदि॒माल्लोँ॒कानवृ॑णो॒त्तद्वृ॒त्रस्य॑ वृत्र॒त्वन्तस्मा॒दिन्द्रो॑\-ऽबिभे॒दपि॒ त्वष्टा॒ तस्मै॒ त्वष्टा॒ वज्र॑मसिञ्च॒त्तपो॒ वै स वज्र॑ आसी॒त्तमुद्य॑न्तुं॒ नाश॑क्नो॒दथ॒ वै तर्\mbox{}हि॒ विष्णुः॑॥३७॥

%2.4.12.3
अ॒न्या दे॒वता॑सी॒थ्सो᳚\-ऽब्रवी॒द्विष्ण॒वेही॒दमा ह॑रिष्यावो॒ येना॒यमि॒दमिति॒ स विष्णु॑स्त्रे॒धात्मानं॒ वि न्य॑धत्त पृथि॒व्यां तृती॑यम॒न्तरि॑क्षे॒ तृती॑यं दि॒वि तृती॑यमभिपर्याव॒र्ताद्ध्यबि॑भे॒द्यत्पृ॑थि॒व्यां तृती॑य॒मासी॒त्तेनेन्द्रो॒ वज्र॒मुद॑यच्छ॒द्विष्ण्व॑नुस्थितः॒ सो᳚\-ऽब्रवी॒न्मा मे॒ प्र हा॒रस्ति॒ वा इ॒दम्॥३८॥

%2.4.12.4
मयि॑ वी॒र्यं॑ तत्ते॒ प्र दा᳚स्या॒मीति॒ तद॑स्मै॒ प्राय॑च्छ॒त्तत्प्रत्य॑गृह्णा॒दधा॒ मेति॒ तद्विष्ण॒वेति॒ प्राय॑च्छ॒त्तद्विष्णुः॒ प्रत्य॑गृह्णाद॒स्मास्विन्द्र॑ इन्द्रि॒यं द॑धा॒त्विति॒ यद॒न्तरि॑क्षे॒ तृती॑य॒मासी॒त्तेनेन्द्रो॒ वज्र॒मुद॑यच्छ॒द्विष्ण्व॑नुस्थितः॒ सो᳚\-ऽब्रवी॒न्मा मे॒ प्र हा॒रस्ति॒ वा इ॒दम्॥३९॥

%2.4.12.5
मयि॑ वी॒र्यं॑ तत्ते॒ प्र दा᳚स्या॒मीति॒ तद॑स्मै॒ प्राय॑च्छ॒त्तत्प्रत्य॑गृह्णा॒द्द्विर्मा॑धा॒ इति॒ तद्विष्ण॒वेति॒ प्राय॑च्छ॒त्तद्विष्णुः॒ प्रत्य॑गृह्णाद॒स्मास्विन्द्र॑ इन्द्रि॒यं द॑धा॒त्विति॒ यद्दि॒वि तृती॑य॒मासी॒त्तेनेन्द्रो॒ वज्र॒मुद॑यच्छ॒द्विष्ण्व॑नुस्थितः॒ सो᳚\-ऽब्रवी॒न्मा मे॒ प्र हा॒र्येना॒हम्॥४०॥

%2.4.12.6
इ॒दमस्मि॒ तत्ते॒ प्र दा᳚स्या॒मीति॒ त्वी (३) इत्य॑ब्रवीथ्स॒न्धान्तु सं द॑धावहै॒ त्वामे॒व प्र वि॑शा॒नीति॒ यन्माम्प्र॑वि॒शेः किम्मा॑ भुञ्ज्या॒ इत्य॑ब्रवी॒त्त्वामे॒वेन्धी॑य॒ तव॒ भोगा॑य॒ त्वाम्प्र वि॑शेय॒मित्य॑ब्रवी॒त्तं वृ॒त्रः प्रावि॑शदु॒दरं॒ वै वृ॒त्रः क्षुत्खलु॒ वै म॑नु॒ष्य॑स्य॒ भ्रातृ॑व्यो॒ यः॥४१॥

%2.4.12.7
ए॒वं वेद॒ हन्ति॒ क्षुध॒म्भ्रातृ॑व्य॒न्तद॑स्मै॒ प्राय॑च्छ॒त्तत्प्रत्य॑गृह्णा॒त्त्रिर्मा॑धा॒ इति॒ तद्विष्ण॒वेति॒ प्राय॑च्छ॒त्तद्विष्णुः॒ प्रत्य॑गृह्णाद॒स्मास्विन्द्र॑ इन्द्रि॒यं द॑धा॒त्विति॒ यत्त्रिः प्राय॑च्छ॒त्त्रिः प्र॒त्यगृ॑ह्णा॒त्तत्त्रि॒धातो᳚स्त्रिधातु॒त्वं यद्विष्णु॑र॒न्वति॑ष्ठत॒ विष्ण॒वेति॒ प्राय॑च्छ॒त्तस्मा॑दैन्द्रावैष्ण॒वꣳ ह॒विर्भ॑वति॒ यद्वा इ॒दं किं च॒ तद॑स्मै॒ तत्प्राय॑च्छ॒दृचः॒ सामा॑नि॒ यजूꣳ॑षि स॒हस्रं॒ वा अ॑स्मै॒ तत्प्राय॑च्छ॒त्तस्मा᳚थ्स॒हस्र॑दक्षिणम्॥४२॥

%2.4.13.0
{\anuvakamend[{प्र॒व॒णं विष्णु॒र्वा इ॒दमि॒दम॒हं यो भ॑व॒त्येक॑विꣳशतिश्च}]}%॥12॥

%2.4.13.1
दे॒वा वै रा॑ज॒न्या᳚ज्जाय॑मानादबिभयु॒स्तम॒न्तरे॒व सन्तं॒ दाम्नापौ᳚म्भ॒न्थ्स वा ए॒षो\-ऽपो᳚ब्धो जायते॒ यद्रा॑ज॒न्यो॑ यद्वा ए॒षो\-ऽन॑पोब्धो॒ जाये॑त वृ॒त्रान्घ्नꣴश्च॑रे॒द्यं का॒मये॑त राज॒न्य॑मन॑पोब्धो जायेत वृ॒त्रान्घ्नꣴश्च॑रे॒दिति॒ तस्मा॑ ए॒तमै᳚न्द्राबार्\mbox{}हस्प॒त्यं च॒रुं निर्व॑पेदै॒न्द्रो वै रा॑ज॒न्यो᳚ ब्रह्म॒ बृह॒स्पति॒र्ब्रह्म॑णै॒वैनं॒ दाम्नो॒\-ऽपोम्भ॑नान्मुञ्चति हिर॒ण्मयं॒ दाम॒ दक्षि॑णा सा॒क्षादे॒वैनं॒ दाम्नो॒\-ऽपोम्भ॑नान्मुञ्चति॥४३॥

%2.4.14.0
{\anuvakamend[{ए॒न॒न्द्वाद॑श च}]}%॥13॥

%2.4.14.1
नवो॑नवो भवति॒ जाय॑मा॒नो\-ऽह्नां᳚ के॒तुरु॒षसा॑मे॒त्यग्रे᳚। भा॒गं दे॒वेभ्यो॒ वि द॑धात्या॒यन्प्र च॒न्द्रमा᳚स्तिरति दी॒र्घमायुः॑। यमा॑दि॒त्या अ॒ꣳ॒शुमा᳚प्या॒यय॑न्ति॒ यमक्षि॑त॒मक्षि॑तयः॒ पिब॑न्ति। तेन॑ नो॒ राजा॒ वरु॑णो॒ बृह॒स्पति॒रा प्या॑ययन्तु॒ भुव॑नस्य गो॒पाः। प्राच्यां᳚ दि॒शि त्वमि॑न्द्रासि॒ राजो॒तोदी᳚च्यां वृत्रहन्वृत्र॒हासि॑। यत्र॒ यन्ति॑ स्रो॒त्यास्तत्॥४४॥

%2.4.14.2
जि॒तं ते॑ दक्षिण॒तो वृ॑ष॒भ ए॑धि॒ हव्यः॑। इन्द्रो॑ जयाति॒ न परा॑ जयाता अधिरा॒जो राज॑सु राजयाति। विश्वा॒ हि भू॒याः पृत॑ना अभि॒ष्टीरु॑प॒सद्यो॑ नम॒स्यो॑ यथास॑त्। अ॒स्येदे॒व प्र रि॑रिचे महि॒त्वं दि॒वः पृ॑थि॒व्याः पर्य॒न्तरि॑क्षात्। स्व॒राडिन्द्रो॒ दम॒ आ वि॒श्वगू᳚र्तः स्व॒रिरम॑त्रो ववक्षे॒ रणा॑य। अ॒भि त्वा॑ शूर नोनु॒मो\-ऽदु॑ग्धा इव धे॒नवः॑। ईशा॑नम्॥४५॥

%2.4.14.3
अ॒स्य जग॑तः सुव॒र्दृश॒मीशा॑नमिन्द्र त॒स्थुषः॑। त्वामिद्धि हवा॑महे सा॒ता वाज॑स्य का॒रवः॑। त्वां वृ॒त्रेष्वि॑न्द्र॒ सत्प॑तिं॒ नर॒स्त्वां काष्ठा॒स्वर्व॑तः। यद्द्याव॑ इन्द्र ते श॒तꣳ श॒तम्भूमी॑रु॒त स्युः। न त्वा॑ वज्रिन्थ्स॒हस्र॒ꣳ॒ सूर्या॒ अनु॒ न जा॒तम॑ष्ट॒ रोद॑सी। पिबा॒ सोम॑मिन्द्र॒ मन्द॑तु त्वा॒ यं ते॑ सु॒षाव॑ हर्य॒श्वाद्रिः॑।॥४६॥

%2.4.14.4
सो॒तुर्बा॒हुभ्या॒ꣳ॒ सुय॑तो॒ नार्वा᳚। रे॒वती᳚र्नः सध॒माद॒ इन्द्रे॑ सन्तु तु॒विवा॑जाः। क्षु॒मन्तो॒ याभि॒र्मदे॑म। उद॑ग्ने॒ शुच॑य॒स्तव॒ वि ज्योति॒षोदु॒ त्यं जा॒तवे॑दसꣳ स॒प्त त्वा॑ ह॒रितो॒ रथे॒ वह॑न्ति देव सूर्य। शो॒चिष्के॑शं विचक्षण। चि॒त्रं दे॒वाना॒मुद॑गा॒दनी॑कं॒ चक्षु॑र्मि॒त्रस्य॒ वरु॑णस्या॒ग्नेः। आ\-ऽप्रा॒ द्यावा॑पृथि॒वी अ॒न्तरि॑क्ष॒ꣳ॒ सूर्य॑ आ॒त्मा जग॑तस्त॒स्थुषः॑॥४७॥

%2.4.14.5
च॒। विश्वे॑ दे॒वा ऋ॑ता॒वृध॑ ऋ॒तुभि॑र्\mbox{}हवन॒श्रुतः॑। जु॒षन्तां॒ युज्य॒म्पयः॑। विश्वे॑ देवाः शृणु॒तेमꣳ हव॑म्मे॒ ये अ॒न्तरि॑क्षे॒ य उप॒ द्यवि॒ ष्ठ। ये अ॑ग्निजि॒ह्वा उ॒त वा॒ यज॑त्रा आ॒सद्या॒स्मिन्ब॒र्\mbox{}हिषि॑ मादयध्वम्॥४८॥

%2.5.0.0

%2.5.0.0
{\anuvakamend[{तदीशा॑न॒मद्रि॑स्त॒स्थुष॑स्त्रि॒ꣳ॒शच्च॑}]}%॥14॥

{\anuvakamend[{वि॒श्वरू॑प॒स्त्वष्टेन्द्रं॑ वृ॒त्रम्ब्र॑ह्मवा॒दिनः॒ स त्वै नासो॑मयाज्ये॒ष वै दे॑वर॒थो दे॒वा वै नर्चि नाय॒ज्ञो\-ऽग्ने॑ म॒हान्त्रीन्निवी॑त॒मायु॑ष्टे॒ द्वाद॑श}]}%॥12॥ 
\prashnaend[{वि॒श्वरू॑पो॒ नैनꣳ॑ शीतरू॒राव॒द्य वसु॑ पूर्वे॒द्युर्वाजा॒ इत्यग्ने॑ म॒हान्निवी॑तम॒न्या यन्ति॒ चतुः॑सप्ततिः॥74॥ वि॒श्वरू॒पो\-ऽनु॑ ते दायि॥}]
%%% END PRASHNA

\sect{पञ्चमः प्रश्नः}\setcounter{anuvakam}{0}
\dnsub{तैत्तिरीयसंहितायां द्वितीयकाण्डे पञ्चमः प्रश्नः}
%2.5.1.0
%2.5.1.1
वि॒श्वरू॑पो॒ वै त्वा॒ष्ट्रः पु॒रोहि॑तो दे॒वाना॑मासीथ्स्व॒स्रीयो\-ऽसु॑राणा॒न्तस्य॒ त्रीणि॑ शी॒र्\mbox{}षाण्या॑सन्थ्सोम॒पानꣳ॑ सुरा॒पान॑म॒न्नाद॑न॒ꣳ॒ स प्र॒त्यक्षं॑ दे॒वेभ्यो॑ भा॒गम॑वदत्प॒रोक्ष॒मसु॑रेभ्यः॒ सर्व॑स्मै॒ वै प्र॒त्यक्षं॑ भा॒गं व॑दन्ति॒ यस्मा॑ ए॒व प॒रोक्षं॒ वद॑न्ति॒ तस्य॑ भा॒ग उ॑दि॒तस्तस्मा॒दिन्द्रो॑\-ऽबिभेदी॒दृङ्वै रा॒ष्ट्रं वि प॒र्याव॑र्तय॒तीति॒ तस्य॒ वज्र॑मा॒दाय॑ शी॒र्\mbox{}षाण्य॑च्छिन॒द्यथ्सो॑म॒पानम्᳚॥१॥

%2.5.1.2
आसी॒थ्स क॒पिञ्ज॑लो\-ऽभव॒द्यथ्सु॑रा॒पान॒ꣳ॒ स क॑ल॒विङ्को॒ यद॒न्नाद॑न॒ꣳ॒ स ति॑त्ति॒रिस्तस्या᳚ञ्ज॒लिना᳚ ब्रह्मह॒त्यामुपा॑गृह्णा॒त्ताꣳ सं॑वथ्स॒रम॑बिभ॒स्तम्भू॒तान्य॒भ्य॑क्रोश॒न्ब्रह्म॑ह॒न्निति॒ स पृ॑थि॒वीमुपा॑सीदद॒स्यै ब्र॑ह्मह॒त्यायै॒ तृती॑यं॒ प्रति॑ गृहा॒णेति॒ साब्र॑वी॒द्वरं॑ वृणै खा॒तात्प॑राभवि॒ष्यन्ती॑ मन्ये॒ ततो॒ मा परा॑ भूव॒मिति॑ पु॒रा ते᳚॥२॥

%2.5.1.3
सं॒व॒थ्स॒रादपि॑ रोहा॒दित्य॑ब्रवी॒त्तस्मा᳚त्पु॒रा सं॑वथ्स॒रात्पृ॑थि॒व्यै खा॒तमपि॑ रोहति॒ वारे॑वृत॒ꣴ॒ ह्य॑स्यै॒ तृती॑यं ब्रह्मह॒त्यायै॒ प्रत्य॑गृह्णा॒त्तथ्स्वकृ॑त॒मिरि॑णमभव॒त्तस्मा॒दाहि॑ताग्निः श्र॒द्धादे॑वः॒ स्वकृ॑त॒ इरि॑णे॒ नाव॑ स्येद्ब्रह्मह॒त्यायै॒ ह्ये॑ष वर्णः॒ स वन॒स्पती॒नुपा॑सीदद॒स्यै ब्र॑ह्मह॒त्यायै॒ तृती॑यं॒ प्रति॑ गृह्णी॒तेति॒ ते᳚\-ऽब्रुव॒न्वरं॑ वृणामहै वृ॒क्णात्॥३॥

%2.5.1.4
प॒रा॒भ॒वि॒ष्यन्तो॑ मन्यामहे॒ ततो॒ मा परा॑ भू॒मेत्या॒व्रश्च॑नाद्वो॒ भूयाꣳ॑स॒ उत्ति॑ष्ठा॒नित्य॑ब्रवी॒त्तस्मा॑दा॒व्रश्च॑नाद्वृ॒क्षाणा॒म्भूयाꣳ॑स॒ उत्ति॑ष्ठन्ति॒ वारे॑वृत॒ꣴ॒ ह्ये॑षा॒न्तृती॑यं ब्रह्मह॒त्यायै॒ प्रत्य॑गृह्ण॒न्थ्स नि॑र्या॒सो॑\-ऽभव॒त्तस्मा᳚न्निर्या॒सस्य॒ नाश्यं॑ ब्रह्मह॒त्यायै॒ ह्ये॑ष वर्णो\-ऽथो॒ खलु॒ य ए॒व लोहि॑तो॒ यो वा॒\-ऽ\-ऽव्रश्च॑नान्नि॒र्येष॑ति॒ तस्य॒ नाश्यम्᳚॥४॥

%2.5.1.5
काम॑म॒न्यस्य॒ स स्त्री॑षꣳसा॒दमुपा॑सीदद॒स्यै ब्र॑ह्मह॒त्यायै॒ तृती॑यं॒ प्रति॑ गृह्णी॒तेति॒ ता अ॑ब्रुव॒न्वरं॑ वृणामहा॒ ऋत्वि॑यात्प्र॒जां वि॑न्दामहै॒ काम॒मा विज॑नितोः॒ सम्भ॑वा॒मेति॒ तस्मा॒दृत्वि॑या॒थ्स्त्रियः॑ प्र॒जां वि॑न्दन्ते॒ काम॒मा विज॑नितोः॒ सम्भ॑वन्ति॒ वारे॑वृत॒ꣴ॒ ह्या॑सा॒न्तृती॑यं ब्रह्मह॒त्यायै॒ प्रत्य॑गृह्ण॒न्थ्सा मल॑वद्वासा अभव॒त्तस्मा॒न्मल॑वद्वाससा॒ न सं व॑देत॥५॥

%2.5.1.6
न स॒हासी॑त॒ नास्या॒ अन्न॑मद्याद्ब्रह्मह॒त्यायै॒ ह्ये॑षा वर्णं॑ प्रति॒मुच्यास्ते\-ऽथो॒ खल्वा॑हुर॒भ्यञ्ज॑नं॒ वाव स्त्रि॒या अन्न॑म॒भ्यञ्ज॑नमे॒व न प्र॑ति॒गृह्यं॒ काम॑म॒न्यदिति॒ याम्मल॑वद्वाससꣳ स॒म्भव॑न्ति॒ यस्ततो॒ जाय॑ते॒ सो॑\-ऽभिश॒स्तो यामर॑ण्ये॒ तस्यै᳚ स्ते॒नो यां परा॑चीं॒ तस्यै᳚ ह्रीतमु॒ख्य॑पग॒ल्भो या स्नाति॒ तस्या॑ अ॒फ्सु मारु॑को॒ या॥६॥

%2.5.1.7
अ॒भ्य॒ङ्क्ते तस्यै॑ दु॒श्चर्मा॒ या प्र॑लि॒खते॒ तस्यै॑ खल॒तिर॑पमा॒री या\-ऽ\-ऽङ्क्ते तस्यै॑ का॒णो या द॒तो धाव॑ते॒ तस्यै᳚ श्या॒वद॒न् या न॒खानि॑ निकृ॒न्तते॒ तस्यै॑ कुन॒खी या कृ॒णत्ति॒ तस्यै᳚ क्ली॒बो या रज्जुꣳ॑ सृ॒जति॒ तस्या॑ उ॒द्बन्धु॑को॒ या प॒र्णेन॒ पिब॑ति॒ तस्या॑ उ॒न्मादु॑को॒ या ख॒र्वेण॒ पिब॑ति॒ तस्यै॑ ख॒र्वस्ति॒स्रो रात्री᳚र्व्र॒तं च॑रेदञ्ज॒लिना॑ वा॒ पिबे॒दख॑र्वेण वा॒ पात्रे॑ण प्र॒जायै॑ गोपी॒थाय॑॥७॥

%2.5.2.0
{\anuvakamend[{यथ्सो॑म॒पान॑न्ते वृ॒क्णात्तस्य॒ नाश्यं॑ वदेत॒ मारु॑को॒ या\-ऽख॑र्वेण वा॒ त्रीणि॑ च}]}%॥१॥

%2.5.2.1
त्वष्टा॑ ह॒तपु॑त्रो॒ वीन्द्र॒ꣳ॒ सोम॒माह॑र॒त्तस्मि॒न्निन्द्र॑ उपह॒वमै᳚च्छत॒ तं नोपा᳚ह्वयत पु॒त्रम्मे॑\-ऽवधी॒रिति॒ स य॑ज्ञवेश॒सं कृ॒त्वा प्रा॒सहा॒ सोम॑मपिब॒त्तस्य॒ यद॒त्यशि॑ष्यत॒ तत्त्वष्टा॑हव॒नीय॒मुप॒ प्राव॑र्तय॒थ्स्वाहेन्द्र॑शत्रुर्वर्ध॒स्वेति॒ यदव॑र्तय॒त्तद्वृ॒त्रस्य॑ वृत्र॒त्वं यदब्र॑वी॒थ्स्वाहेन्द्र॑शत्रुर्वर्ध॒स्वेति॒ तस्मा॑दस्य॥८॥

%2.5.2.2
इन्द्रः॒ शत्रु॑रभव॒थ्स स॒म्भव॑न्न॒ग्नीषोमा॑व॒भि सम॑भव॒थ्स इ॑षुमा॒त्रमि॑षुमात्रं॒ विष्व॑ङ्ङवर्धत॒ स इ॒माल्लोँ॒कान॑वृणो॒द् यदि॒माल्लोँ॒कानवृ॑णो॒त्तद्वृ॒त्रस्य॑ वृत्र॒त्वन्तस्मा॒दिन्द्रो॑\-ऽबिभे॒थ्स प्र॒जाप॑ति॒मुपा॑धाव॒च्छत्रु॑र्मे\-ऽज॒नीति॒ तस्मै॒ वज्रꣳ॑ सि॒क्त्वा प्राय॑च्छदे॒तेन॑ ज॒हीति॒ तेना॒भ्या॑यत॒ ताव॑ब्रूताम॒ग्नीषोमौ॒ मा॥९॥

%2.5.2.3
प्र हा॑रा॒वम॒न्तः स्व॒ इति॒ मम॒ वै यु॒वꣴ स्थ॒ इत्य॑ब्रवी॒न्माम॒भ्येत॒मिति॒ तौ भा॑ग॒धेय॑मैच्छेता॒न्ताभ्या॑मे॒तम॑ग्नीषो॒मीय॒\-मेका॑दशकपालम्पू॒र्णमा॑से॒ प्राय॑च्छ॒त्ताव॑ब्रूताम॒भि सन्द॑ष्टौ॒ वै स्वो॒ न श॑क्नुव॒ ऐतु॒मिति॒ स इन्द्र॑ आ॒त्मनः॑ शीतरू॒राव॑जनय॒त्तच्छी॑तरू॒रयो॒र्जन्म॒ य ए॒वꣳ शी॑तरू॒रयो॒र्जन्म॒ वेद॑॥१०॥


%2.5.2.4
नैनꣳ॑ शीतरू॒रौ ह॑त॒स्ताभ्या॑मेनम॒भ्य॑नय॒त्तस्मा᳚ज्जञ्ज॒भ्यमा॑नाद॒ग्नीषोमौ॒ निर॑क्रामतां प्राणापा॒नौ वा ए॑नं॒ तद॑जहिताम् प्रा॒णो वै दक्षो॑\-ऽपा॒नः क्रतु॒स्तस्मा᳚ज्जञ्ज॒भ्यमा॑नो ब्रूया॒न्मयि॑ दक्षक्र॒तू इति॑ प्राणापा॒नावे॒वात्मन्ध॑त्ते॒ सर्व॒मायु॑रेति॒ स दे॒वता॑ वृ॒त्रान्नि॒र्\mbox{}हूय॒ वार्त्र॑घ्नꣳ ह॒विः पू॒र्णमा॑से॒ निर॑वप॒द्घ्नन्ति॒ वा ए॑नम्पू॒र्णमा॑स॒ आ॥११॥

%2.5.2.5
अ॒मा॒वा॒स्या॑याम्प्याययन्ति॒ तस्मा॒द्वार्त्र॑घ्नी पू॒र्णमा॒से\-ऽनू᳚च्येते॒ वृध॑न्वती अमावा॒स्या॑या॒न्तथ्स॒ꣴ॒स्थाप्य॒ वार्त्र॑घ्नꣳ ह॒विर्वज्र॑मा॒दाय॒ पुन॑र॒भ्या॑यत॒ ते अ॑ब्रूता॒न्द्यावा॑पृथि॒वी मा प्र हा॑रा॒वयो॒र्वै श्रि॒त इति॒ ते अ॑ब्रूतां॒ वरं॑ वृणावहै॒ नक्ष॑त्रविहिता॒\-ऽहमसा॒नीत्य॒साव॑ब्रवीच्चि॒त्रवि॑हिता॒\-ऽहमिती॒यन्तस्मा॒न्नक्ष॑त्रविहिता॒\-ऽसौ चि॒त्रवि॑हिते॒यं य ए॒वं द्यावा॑पृथि॒व्योः॥१२॥

%2.5.2.6
वरं॒ वेदैनं॒ वरो॑ गच्छति॒ स आ॒भ्यामे॒व प्रसू॑त॒ इन्द्रो॑ वृ॒त्रम॑ह॒न्ते दे॒वा वृ॒त्रꣳ ह॒त्वा\-ऽग्नीषोमा॑वब्रुवऩ्ह॒व्यं नो॑ वहत॒मिति॒ ताव॑ब्रूता॒मप॑तेजसौ॒ वै त्यौ वृ॒त्रे वै त्ययो॒स्तेज॒ इति॒ ते᳚\-ऽब्रुव॒न्क इ॒दमच्छै॒तीति॒ गौरित्य॑ब्रुव॒न्गौर्वाव सर्व॑स्य मि॒त्रमिति॒ सा\-ऽब्र॑वीत्॥१३॥

%2.5.2.7
वरं॑ वृणै॒ मय्ये॒व स॒तोभये॑न भुनजाध्वा॒ इति॒ तद्गौराह॑र॒त्तस्मा॒द्गवि॑ स॒तोभये॑न भुञ्जत ए॒तद्वा अ॒ग्नेस्तेजो॒ यद्घृ॒तमे॒तथ्सोम॑स्य॒ यत्पयो॒ य ए॒वम॒ग्नीषोम॑यो॒स्तेजो॒ वेद॑ तेज॒स्व्ये॑व भ॑वति ब्रह्मवा॒दिनो॑ वदन्ति किन्देव॒त्य॑म्पौर्णमा॒समिति॑ प्राजाप॒त्यमिति॑ ब्रूया॒त्तेनेन्द्रं॑ ज्ये॒ष्ठम्पु॒त्रं नि॒रवा॑सायय॒दिति॒ तस्मा᳚ज्ज्ये॒ष्ठम्पु॒त्रं धने॑न नि॒रव॑साययन्ति॥१४॥

%2.5.3.0
{\anuvakamend[{अ॒स्य॒ मा वेदा द्यावा॑पृथि॒व्योर॑ब्रवी॒दिति॒ तस्मा᳚च्च॒त्वारि॑ च}]}%॥२॥

%2.5.3.1
इन्द्रं॑ वृ॒त्रं ज॑घ्नि॒वाꣳस॒म्मृधो॒\-ऽभि प्रावे॑पन्त॒ स ए॒तं वै॑मृ॒धम्पू॒णमा॑से\-ऽनुनिर्वा॒प्य॑मपश्य॒त्तं निर॑वप॒त्तेन॒ वै स मृधो\-ऽपा॑हत॒ यद्वै॑मृ॒धः पू॒र्णमा॑से\-ऽनुनिर्वा॒प्यो॑ भव॑ति॒ मृध॑ ए॒व तेन॒ यज॑मा॒नो\-ऽप॑ हत॒ इन्द्रो॑ वृ॒त्रꣳ ह॒त्वा दे॒वता॑भिश्चेन्द्रि॒येण॑ च॒ व्या᳚र्ध्यत॒ स ए॒तमा᳚ग्ने॒यम॒ष्टाक॑पालममावा॒स्या॑यामपश्यदै॒न्द्रं दधि॑॥१५॥

%2.5.3.2
तं निर॑वप॒त्तेन॒ वै स दे॒वता᳚श्चेन्द्रि॒यं चावा॑रुन्द्ध॒ यदा᳚ग्ने॒यो᳚\-ऽष्टाक॑पालो\-ऽमावा॒स्या॑या॒म्भव॑त्यै॒न्द्रं दधि॑ दे॒वता᳚श्चै॒व तेने᳚न्द्रि॒यं च॒ यज॑मा॒नो\-ऽव॑ रुन्द्ध॒ इन्द्र॑स्य वृ॒त्रं ज॒घ्नुष॑ इन्द्रि॒यं वी॒र्यं॑ पृथि॒वीमनु॒ व्या᳚र्च्छ॒त्तदोष॑धयो वी॒रुधो॑\-ऽभव॒न्थ्स प्र॒जाप॑ति॒मुपा॑धावद्वृ॒त्रं मे॑ ज॒घ्नुष॑ इन्द्रि॒यं वी॒र्यम्᳚॥१६॥

%2.5.3.3
पृ॒थि॒वीमनु॒ व्या॑र॒त्तदोष॑धयो वी॒रुधो॑\-ऽभूव॒न्निति॒ स प्र॒जाप॑तिः प॒शून॑ब्रवीदे॒तद॑स्मै॒ सं न॑य॒तेति॒ तत्प॒शव॒ ओष॑धी॒भ्यो\-ऽध्या॒त्मन्थ्सम॑नय॒न्तत्प्रत्य॑दुह॒न् यथ्स॒मन॑य॒न्तथ्सा᳚न्ना॒य्यस्य॑ सान्नाय्य॒त्वं यत्प्र॒त्यदु॑ह॒न्तत्प्र॑ति॒धुषः॑ प्रतिधु॒क्त्वꣳ सम॑नैषुः॒ प्रत्य॑धुक्ष॒न्न तु मयि॑ श्रयत॒ इत्य॑ब्रवीदे॒तद॑स्मै॥१७॥

%2.5.3.4
शृ॒तं कु॑रु॒तेत्य॑ब्रवी॒त्तद॑स्मै शृ॒तम॑कुर्वन्निन्द्रि॒यं वावास्मि॑न्वी॒र्यं॑ तद॑श्रय॒न्तच्छृ॒तस्य॑ शृत॒त्वꣳ सम॑नैषुः॒ प्रत्य॑धुक्षञ्छृ॒तम॑क्र॒न्न तु मा॑ धिनो॒तीत्य॑ब्रवीदे॒तद॑स्मै॒ दधि॑ कुरु॒तेत्य॑ब्रवी॒त्तद॑स्मै॒ दध्य॑कुर्व॒न्तदे॑नमधिनो॒त्तद्द॒ध्नो द॑धि॒त्वं ब्र॑ह्मवा॒दिनो॑ वदन्ति द॒ध्नः पूर्व॑स्याव॒देयम्᳚॥१८॥

%2.5.3.5
दधि॒ हि पूर्वं॑ क्रि॒यत॒ इत्यना॑दृत्य॒ तच्छृ॒तस्यै॒व पूर्व॒स्याव॑ द्येदिन्द्रि॒यमे॒वास्मि॑न्वी॒र्यꣴ॑ श्रि॒त्वा द॒ध्नोपरि॑ष्टाद्धिनोति यथापू॒र्वमुपै॑ति॒ यत्पू॒तीकै᳚र्वा पर्णव॒ल्कैर्वा॑त॒ञ्च्याथ्सौ॒म्यं तद्यत्क्व॑लै राक्ष॒सं तद्यत्त॑ण्डु॒लैर्वै᳚श्वदे॒वं तद्यदा॒तञ्च॑नेन मानु॒षं तद्यद्द॒ध्ना तथ्सेन्द्रं॑ द॒ध्ना त॑नक्ति॥१९॥

%2.5.3.6
से॒न्द्र॒त्वाया᳚ग्निहोत्रोच्छेष॒णम॒भ्यात॑नक्ति य॒ज्ञस्य॒ सन्त॑त्या॒ इन्द्रो॑ वृ॒त्रꣳ ह॒त्वा परां᳚ परा॒वत॑मगच्छ॒दपा॑राध॒मिति॒ मन्य॑मान॒स्तं दे॒वताः॒ प्रैष॑मैच्छ॒न्थ्सो᳚\-ऽब्रवीत्प्र॒जाप॑ति॒र्यः प्र॑थ॒मो॑\-ऽनुवि॒न्दति॒ तस्य॑ प्रथ॒मम्भा॑ग॒धेय॒मिति॒ तम्पि॒तरो\-ऽन्व॑विन्द॒न्तस्मा᳚त्पि॒तृभ्यः॑ पूर्वे॒द्युः क्रि॑यते॒ सो॑\-ऽमावा॒स्यां᳚ प्रत्याग॑च्छ॒त्तं दे॒वा अ॒भि सम॑गच्छ॒न्तामा वै नः॑॥२०॥

%2.5.3.7
अ॒द्य वसु॑ वस॒तीतीन्द्रो॒ हि दे॒वानां॒ वसु॒ तद॑मावा॒स्या॑या अमावास्य॒त्वं ब्र॑ह्मवा॒दिनो॑ वदन्ति किन्देव॒त्यꣳ॑ सान्ना॒य्यमिति॑ वैश्वदे॒वमिति॑ ब्रूया॒द्विश्वे॒ हि तद्दे॒वा भा॑ग॒धेय॑म॒भि स॒मग॑च्छ॒न्तेत्यथो॒ खल्वै॒न्द्रमित्ये॒व ब्रू॑या॒दिन्द्रं॒ वाव ते तद्भि॑ष॒ज्यन्तो॒\-ऽभि सम॑गच्छ॒न्तेति॑॥२१॥

%2.5.4.0
{\anuvakamend[{दधि॑ मे ज॒घ्नुष॑ इन्द्रि॒यं वी॒र्य॑मित्य॑ब्रवीदे॒तद॑स्मा अव॒देय॑न्तनक्ति नो॒ द्विच॑त्वारिꣳशच्च}]}%॥३॥

%2.5.4.1
ब्र॒ह्म॒वा॒दिनो॑ वदन्ति॒ स त्वै द॑र्शपूर्णमा॒सौ य॑जेत॒ य ए॑नौ॒ सेन्द्रौ॒ यजे॒तेति॑ वैमृ॒धः पू॒र्णमा॑से\-ऽनुनिर्वा॒प्यो॑ भवति॒ तेन॑ पू॒र्णमा॑सः॒ सेन्द्र॑ ऐ॒न्द्रं दध्य॑मावा॒स्या॑यां॒ तेना॑मावा॒स्या॑ सेन्द्रा॒ य ए॒वं वि॒द्वान्द॑र्शपूर्णमा॒सौ यज॑ते॒ सेन्द्रा॑वे॒वैनौ॑ यजते॒ श्वःश्वो᳚\-ऽस्मा ईजा॒नाय॒ वसी॑यो भवति दे॒वा वै यद्य॒ज्ञे\-ऽकु॑र्वत॒ तदसु॑रा अकुर्वत॒ ते दे॒वा ए॒ताम्॥२२॥

%2.5.4.2
इष्टि॑मपश्यन्नाग्नावैष्ण॒वमेका॑दशकपाल॒ꣳ॒ सर॑स्वत्यै च॒रुꣳ सर॑स्वते च॒रुं ताम्पौ᳚र्णमा॒सꣳ स॒ꣴ॒स्थाप्यानु॒ निर॑वप॒न्ततो॑ दे॒वा अभ॑व॒न्परासु॑रा॒ यो भ्रातृ॑व्यवा॒न्थ्स्याथ्स पौ᳚र्णमा॒सꣳ स॒ꣴ॒स्थाप्यै॒तामिष्टि॒मनु॒ निर्व॑पेत्पौर्णमा॒सेनै॒व वज्र॒म्भ्रातृ॑व्याय प्र॒हृत्या᳚ग्नावैष्ण॒वेन॑ दे॒वता᳚श्च य॒ज्ञं च॒ भ्रातृ॑व्यस्य वृङ्क्ते मिथु॒नान्प॒शून्थ्सा॑रस्व॒ताभ्यां॒ याव॑दे॒वास्यास्ति॒ तत्॥२३॥

%2.5.4.3
सर्वं॑ वृङ्क्ते पौर्णमा॒सीमे॒व य॑जेत॒ भ्रातृ॑व्यवा॒न्नामा॑वा॒स्याꣳ॑ ह॒त्वा भ्रातृ॑व्यं॒ ना प्या॑ययति साकम्प्रस्था॒यीये॑न यजेत प॒शुका॑मो॒ यस्मै॒ वा अल्पे॑ना॒हर॑न्ति॒ नात्मना॒ तृप्य॑ति॒ नान्यस्मै॑ ददाति॒ यस्मै॑ मह॒ता तृप्य॑त्या॒त्मना॒ ददा᳚त्य॒न्यस्मै॑ मह॒ता पू॒र्णꣳ हो॑त॒व्य॑न्तृ॒प्त ए॒वैन॒मिन्द्रः॑ प्र॒जया॑ प॒शुभि॑स्तर्पयति दारुपा॒त्रेण॑ जुहोति॒ न हि मृ॒न्मय॒माहु॑तिमान॒श औदु॑म्बरम्॥२४॥

%2.5.4.4
भ॒व॒त्यूर्ग्वा उ॑दु॒म्बर॒ ऊर्क्प॒शव॑ ऊ॒र्जैवास्मा॒ ऊर्जं॑ प॒शूनव॑ रुन्द्धे॒ नाग॑तश्रीर्महे॒न्द्रं य॑जेत॒ त्रयो॒ वै ग॒तश्रि॑यः शुश्रु॒वान्ग्रा॑म॒णी रा॑ज॒न्य॑स्तेषा᳚म्महे॒न्द्रो दे॒वता॒ यो वै स्वां दे॒वता॑मति॒यज॑ते॒ प्र स्वायै॑ दे॒वता॑यै च्यवते॒ न परा॒म्प्राप्नो॑ति॒ पापी॑यान्भवति संवथ्स॒रमिन्द्रं॑ यजेत संवथ्स॒रꣳ हि व्र॒तं नाति॒ स्वा॥२५॥

%2.5.4.5
ए॒वैनं॑ दे॒वते॒ज्यमा॑ना॒ भूत्या॑ इन्द्धे॒ वसी॑यान्भवति संवथ्स॒रस्य॑ प॒रस्ता॑द॒ग्नये᳚ व्र॒तप॑तये पुरो॒डाश॑म॒ष्टाक॑पालं॒ निर्व॑पेथ्संवथ्स॒रमे॒वैनं॑ वृ॒त्रं ज॑घ्नि॒वाꣳस॑म॒ग्निर्व्र॒तप॑तिर्व्र॒तमा ल॑म्भयति॒ ततो\-ऽधि॒ कामं॑ यजेत॥२६॥

%2.5.5.0
{\anuvakamend[{ए॒तान्तदौदु॑म्बर॒ꣴ॒ स्वा त्रि॒ꣳ॒शच्च॑}]}%॥४॥

%2.5.5.1
नासो॑मयाजी॒ सं न॑ये॒दना॑गतं॒ वा ए॒तस्य॒ पयो॒ यो\-ऽसो॑मयाजी॒ यदसो॑मयाजी सं॒नये᳚त्परिमो॒ष ए॒व सो\-ऽनृ॑तं करो॒त्यथो॒ परै॒व सि॑च्यते सोमया॒ज्ये॑व सं न॑ये॒त्पयो॒ वै सोमः॒ पयः॑ सान्ना॒य्यम्पय॑सै॒व पय॑ आ॒त्मन्ध॑त्ते॒ वि वा ए॒तम्प्र॒जया॑ प॒शुभि॑रर्धयति व॒र्धय॑त्यस्य॒ भ्रातृ॑व्यं॒ यस्य॑ ह॒विर्निरु॑प्तम्पु॒रस्ता᳚च्च॒न्द्रमाः᳚॥२७॥

%2.5.5.2
अ॒भ्यु॑देति॑ त्रे॒धा त॑ण्डु॒लान् वि भ॑जे॒द्ये म॑ध्य॒माः स्युस्तान॒ग्नये॑ दा॒त्रे पु॑रो॒डाश॑म॒ष्टाक॑पालं कुर्या॒द्ये स्थवि॑ष्ठा॒स्तानिन्द्रा॑य प्रदा॒त्रे द॒धꣴश्च॒रुं ये\-ऽणि॑ष्ठा॒स्तान् विष्ण॑वे शिपिवि॒ष्टाय॑ शृ॒ते च॒रुम॒ग्निरे॒वास्मै᳚ प्र॒जाम्प्र॑ज॒नय॑ति वृ॒द्धामिन्द्रः॒ प्र य॑च्छति य॒ज्ञो वै विष्णुः॑ प॒शवः॒ शिपि॑र्य॒ज्ञ ए॒व प॒शुषु॒ प्रति॑ तिष्ठति॒ न द्वे॥२८॥

%2.5.5.3
य॒जे॒त॒ यत्पूर्व॑या सम्प्र॒ति यजे॒तोत्त॑रया छ॒म्बट्कु॑र्या॒द्यदुत्त॑रया सम्प्र॒ति यजे॑त॒ पूर्व॑या छ॒म्बट्कु॑र्या॒न्नेष्टि॒र्भव॑ति॒ न य॒ज्ञस्तदनु॑ ह्रीतमु॒ख्य॑पग॒ल्भो जा॑यत॒ एका॑मे॒व य॑जेत प्रग॒ल्भो᳚\-ऽस्य जाय॒ते\-ऽना॑दृत्य॒ तद्द्वे ए॒व य॑जेत यज्ञमु॒खमे॒व पूर्व॑या॒लभ॑ते॒ यज॑त॒ उत्त॑रया दे॒वता॑ ए॒व पूर्व॑यावरु॒न्द्ध इ॑न्द्रि॒यमुत्त॑रया देवलो॒कमे॒व॥२९॥

%2.5.5.4
पूर्व॑याभि॒जय॑ति मनुष्यलो॒कमुत्त॑रया॒ भूय॑सो यज्ञक्र॒तूनुपै᳚त्ये॒षा वै सु॒मना॒ नामेष्टि॒र्यम॒द्येजा॒नम्प॒श्चाच्च॒न्द्रमा॑ अ॒भ्यु॑देत्य॒स्मिन्ने॒वास्मै॑ लो॒के\-ऽर्धु॑कम्भवति दाक्षायणय॒ज्ञेन॑ सुव॒र्गका॑मो यजेत पू॒र्णमा॑से॒ सं न॑येन्मैत्रावरु॒ण्या\-ऽ\-ऽ\-मिक्ष॑यामावा॒स्या॑यां यजेत पू॒र्णमा॑से॒ वै दे॒वानाꣳ॑ सु॒तस्तेषा॑मे॒तम॑र्धमा॒सम्प्रसु॑त॒स्तेषा᳚म्मैत्रावरु॒णी व॒शामा॑वा॒स्या॑यामनूब॒न्ध्या॑ यत्॥३०॥

%2.5.5.5
पू॒र्वे॒द्युर्यज॑ते॒ वेदि॑मे॒व तत्क॑रोति॒ यद्व॒थ्सान॑पाक॒रोति॑ सदोहविर्धा॒ने ए॒व सम्मि॑नोति॒ यद्यज॑ते दे॒वैरे॒व सु॒त्याꣳ सम्पा॑दयति॒ स ए॒तम॑र्धमा॒सꣳ स॑ध॒मादं॑ दे॒वैः सोम॑म्पिबति॒ यन्मै᳚त्रावरु॒ण्यामि॑क्षयामावा॒स्या॑यां॒ यज॑ते॒ यैवासौ दे॒वानां᳚ व॒शानू॑ब॒न्ध्या॑ सो ए॒वैषैतस्य॑ सा॒क्षाद्वा ए॒ष दे॒वान॒भ्यारो॑हति॒ य ए॑षां य॒ज्ञम्॥३१॥

%2.5.5.6
अ॒भ्या॒रोह॑ति॒ यथा॒ खलु॒ वै श्रेया॑न॒भ्यारू॑ढः का॒मय॑ते॒ तथा॑ करोति॒ यद्य॑व॒विध्य॑ति॒ पापी॑यान्भवति॒ यदि॒ नाव॒विध्य॑ति स॒दृङ्व्या॒वृत्का॑म ए॒तेन॑ य॒ज्ञेन॑ यजेत क्षु॒रप॑वि॒र्\mbox{}ह्ये॑ष य॒ज्ञस्ता॒जक्पुण्यो॑ वा॒ भव॑ति॒ प्र वा॑ मीयते॒ तस्यै॒तद्व्र॒तं नानृ॑तं वदे॒न्न मा॒ꣳ॒सम॑श्नीया॒न्न स्त्रिय॒मुपे॑या॒न्नास्य॒ पल्पू॑लनेन॒ वासः॑ पल्पूलयेयुरे॒तद्धि दे॒वाः सर्वं॒ न कु॒र्वन्ति॑॥३२॥

%2.5.6.0
{\anuvakamend[{च॒न्द्रमा॒ द्वे दे॑वलो॒कमे॒व यद्य॒ज्ञं प॑ल्पूलयेयु॒ष्षट्च॑}]}%॥५॥

%2.5.6.1
ए॒ष वै दे॑वर॒थो यद्द॑र्\mbox{}शपूर्णमा॒सौ यो द॑र्\mbox{}शपूर्णमा॒सावि॒ष्ट्वा सोमे॑न॒ यज॑ते॒ रथ॑स्पष्ट ए॒वाव॒साने॒ वरे॑ दे॒वाना॒मव॑ स्यत्ये॒तानि॒ वा अङ्गा॒परूꣳ॑षि संवथ्स॒रस्य॒ यद्द॑र्\mbox{}शपूर्णमा॒सौ य ए॒वं वि॒द्वान्द॑र्\mbox{}शपूर्णमा॒सौ यज॒ते\-ऽङ्गा॒परूꣴ॑ष्ये॒व सं॑वथ्स॒रस्य॒ प्रति॑ दधात्ये॒ते वै सं॑वथ्स॒रस्य॒ चक्षु॑षी॒ यद्द॑र्\mbox{}शपूर्णमा॒सौ य ए॒वं वि॒द्वान्द॑र्\mbox{}शपूर्णमा॒सौ यज॑ते॒ ताभ्या॑मे॒व सु॑व॒र्गं लो॒कमनु॑ पश्यति॥३३॥

%2.5.6.2
ए॒षा वै दे॒वानां॒ विक्रा᳚न्ति॒र्यद्द॑र्\mbox{}शपूर्णमा॒सौ य ए॒वं वि॒द्वान्द॑र्\mbox{}शपूर्णमा॒सौ यज॑ते दे॒वाना॑मे॒व विक्रा᳚न्ति॒मनु॒ वि क्र॑मत ए॒ष वै दे॑व॒यानः॒ पन्था॒ यद्द॑र्\mbox{}शपूर्णमा॒सौ य ए॒वं वि॒द्वान्द॑र्\mbox{}शपूर्णमा॒सौ यज॑ते॒ य ए॒व दे॑व॒यानः॒ पन्था॒स्तꣳ स॒मारो॑हत्ये॒तौ वै दे॒वाना॒ꣳ॒ हरी॒ यद्द॑र्\mbox{}शपूर्णमा॒सौ य ए॒वं वि॒द्वान्द॑र्\mbox{}शपूर्णमा॒सौ यज॑ते॒ यावे॒व दे॒वाना॒ꣳ॒ हरी॒ ताभ्या᳚म्॥३४॥

%2.5.6.3
ए॒वैभ्यो॑ ह॒व्यं व॑हत्ये॒तद्वै दे॒वाना॑मा॒स्यं॑ यद्द॑र्\mbox{}शपूर्णमा॒सौ य ए॒वं वि॒द्वान्द॑र्\mbox{}शपूर्णमा॒सौ यज॑ते सा॒क्षादे॒व दे॒वाना॑मा॒स्ये॑ जुहोत्ये॒ष वै ह॑विर्धा॒नी यो द॑र्\mbox{}शपूर्णमासया॒जी सा॒यम्प्रा॑तरग्निहो॒त्रं जु॑होति॒ यज॑ते दर्\mbox{}शपूर्णमा॒सावह॑रहर्\mbox{}हविर्धा॒निनाꣳ॑ सु॒तो य ए॒वं वि॒द्वान्द॑र्\mbox{}शपूर्णमा॒सौ यज॑ते हविर्धा॒न्य॑स्मीति॒ सर्व॑मे॒वास्य॑ बर्\mbox{}हि॒ष्यं॑ द॒त्तम्भ॑वति दे॒वा वा अहः॑॥३५॥

%2.5.6.4
य॒ज्ञियं॒ नावि॑न्द॒न्ते द॑र्\mbox{}शपूर्णमा॒साव॑पुन॒न्तौ वा ए॒तौ पू॒तौ मेध्यौ॒ यद्द॑र्\mbox{}शपूर्णमा॒सौ य ए॒वं वि॒द्वान्द॑र्\mbox{}शपूर्णमा॒सौ यज॑ते पू॒तावे॒वैनौ॒ मेध्यौ॑ यजते॒ नामा॑वा॒स्या॑यां च पौर्णमा॒स्यां च॒ स्त्रिय॒मुपे॑या॒द्यदु॑पे॒यान्निरि॑न्द्रियः स्या॒थ्सोम॑स्य॒ वै राज्ञो᳚\-ऽर्धमा॒सस्य॒ रात्र॑यः॒ पत्न॑य आस॒न्तासा॑ममावा॒स्यां᳚ च पौर्णमा॒सीं च॒ नोपै᳚त्॥३६॥

%2.5.6.5
ते ए॑नम॒भि सम॑नह्येता॒न्तं यक्ष्म॑ आर्च्छ॒द्राजा॑नं॒ यक्ष्म॑ आर॒दिति॒ तद्रा॑जय॒क्ष्मस्य॒ जन्म॒ यत्पापी॑या॒नभ॑व॒त्त\-त्पा॑पय॒क्ष्मस्य॒ यज्जा॒याभ्या॒मवि॑न्द॒त्तज्जा॒येन्य॑स्य॒ य ए॒वमे॒तेषां॒ यक्ष्मा॑णां॒ जन्म॒ वेद॒ नैन॑मे॒ते यक्ष्मा॑ विन्दन्ति॒ स ए॒ते ए॒व न॑म॒स्यन्नुपा॑धाव॒त्ते अ॑ब्रूतां॒ वरं॑ वृणावहा आ॒वं दे॒वानां᳚ भाग॒धे अ॑साव॥३७॥

%2.5.6.6
आ॒वदधि॑ दे॒वा इ॑ज्यान्ता॒ इति॒ तस्मा᳚थ्स॒दृशी॑ना॒ꣳ॒ रात्री॑णाममावा॒स्या॑यां च पौर्णमा॒स्यां च॑ दे॒वा इ॑ज्यन्त ए॒ते हि दे॒वानां᳚ भाग॒धे भा॑ग॒धा अ॑स्मै मनु॒ष्या॑ भवन्ति॒ य ए॒वं वेद॑ भू॒तानि॒ क्षुध॑मघ्नन्थ्स॒द्यो म॑नु॒ष्या॑ अर्धमा॒से दे॒वा मा॒सि पि॒तरः॑ संवथ्स॒रे वन॒स्पत॑य॒स्तस्मा॒दह॑रहर्मनु॒ष्या॑ अश॑नमिच्छन्ते\-ऽर्धमा॒से दे॒वा इ॑ज्यन्ते मा॒सि पि॒तृभ्यः॑ क्रियते संवथ्स॒रे वन॒स्पत॑यः॒ फलं॑ गृह्णन्ति॒ य ए॒वं वेद॒ हन्ति॒ क्षुध॒म्भ्रातृ॑व्यम्॥३८॥

%2.5.7.0
{\anuvakamend[{प॒श्य॒ति॒ ताभ्या॒मह॑रैदसाव॒ फलꣳ॑ स॒प्त च॑}]}%॥६॥

%2.5.7.1
दे॒वा वै नर्चि न यजु॑ष्यश्रयन्त॒ ते साम॑न्ने॒वाश्र॑यन्त॒ हिं क॑रोति॒ सामै॒वाक॒र्\mbox{}हिं क॑रोति॒ यत्रै॒व दे॒वा अश्र॑यन्त॒ तत॑ ए॒वैना॒न्प्र यु॑ङ्क्ते॒ हिं क॑रोति वा॒च ए॒वैष योगो॒ हिं क॑रोति प्र॒जा ए॒व तद्यज॑मानः सृजते॒ त्रिः प्र॑थ॒मामन्वा॑ह॒ त्रिरु॑त्त॒मां य॒ज्ञस्यै॒व तद्ब॒र्\mbox{}सम्॥३९॥

%2.5.7.2
न॒ह्य॒त्यप्र॑स्रꣳसाय॒ सन्त॑त॒मन्वा॑ह प्रा॒णाना॑म॒न्नाद्य॑स्य॒ सन्त॑त्या॒ अथो॒ रक्ष॑सा॒मप॑हत्यै॒ राथं॑तरीम्प्रथ॒मामन्वा॑ह॒ राथं॑तरो॒ वा अ॒यं लो॒क इ॒ममे॒व लो॒कम॒भि ज॑यति॒ त्रिर्वि गृ॑ह्णाति॒ त्रय॑ इ॒मे लो॒का इ॒माने॒व लो॒कान॒भि ज॑यति॒ बार्\mbox{}ह॑तीमुत्त॒मामन्वा॑ह॒ बार्\mbox{}ह॑तो॒ वा अ॒सौ लो॒को॑\-ऽमुमे॒व लो॒कम॒भि ज॑यति॒ प्र वः॑॥४०॥

%2.5.7.3
वाजा॒ इत्यनि॑रुक्ताम्प्राजाप॒त्यामन्वा॑ह य॒ज्ञो वै प्र॒जाप॑तिर्य॒ज्ञमे॒व प्र॒जाप॑ति॒मा र॑भते॒ प्र वो॒ वाजा॒ इत्यन्वा॒हान्नं॒ वै वाजो\-ऽन्न॑मे॒वाव॑ रुन्द्धे॒ प्र वो॒ वाजा॒ इत्या॑न्वाह॒ तस्मा᳚त्प्रा॒चीन॒ꣳ॒ रेतो॑ धीय॒ते\-ऽग्न॒ आ या॑हि वी॒तय॒ इत्या॑ह॒ तस्मा᳚त्प्र॒तीचीः᳚ प्र॒जा जा॑यन्ते॒ प्र वो॒ वाजाः᳚॥४१॥

%2.5.7.4
इत्यन्वा॑ह॒ मासा॒ वै वाजा॑ अर्धमा॒सा अ॒भिद्य॑वो दे॒वा ह॒विष्म॑न्तो॒ गौर्घृ॒ताची॑ य॒ज्ञो दे॒वाञ्जि॑गाति॒ यज॑मानः सुम्न॒युरि॒दम॑सी॒दम॒सीत्ये॒व य॒ज्ञस्य॑ प्रि॒यं धामाव॑ रुन्द्धे॒ यं का॒मये॑त॒ सर्व॒मायु॑रिया॒दिति॒ प्र वो॒ वाजा॒ इति॒ तस्या॒नूच्याग्न॒ आ या॑हि वी॒तय॒ इति॒ सन्त॑त॒मुत्त॑रमर्ध॒र्चमा ल॑भेत॥४२॥

%2.5.7.5
प्रा॒णेनै॒वास्या॑पा॒नं दा॑धार॒ सर्व॒मायु॑रेति॒ यो वा अ॑र॒त्निꣳ सा॑मिधे॒नीनां॒ वेदा॑र॒त्नावे॒व भ्रातृ॑व्यं कुरुते\-ऽर्ध॒र्चौ सं द॑धात्ये॒ष वा अ॑र॒त्निः सा॑मिधे॒नीनां॒ य ए॒वं वेदा॑र॒त्नावे॒व भ्रातृ॑व्यं कुरुत॒ ऋषेर्\mbox{}॑ऋषे॒र्वा ए॒ता निर्मि॑ता॒ यथ्सा॑मिधे॒न्य॑स्ता यदसं॑युक्ताः॒ स्युः प्र॒जया॑ प॒शुभि॒र्यज॑मानस्य॒ वि ति॑ष्ठेरन्नर्ध॒र्चौ सन्द॑धाति॒ सं यु॑नक्त्ये॒वैना॒स्ता अ॑स्मै॒ संयु॑क्ता॒ अव॑रुद्धाः॒ सर्वा॑मा॒शिषं॑ दुह्रे॥४३॥


%2.5.8.0
{\anuvakamend[{ब॒र्\mbox{}सं वो॑ जायन्ते॒ प्र वो॒ वाजा॑ लभेत दधाति॒ सन्दश॑ च}]}%॥७॥

%2.5.8.1
अय॑ज्ञो॒ वा ए॒ष यो॑\-ऽसा॒मा\-ऽग्न॒ आ या॑हि वी॒तय॒ इत्या॑ह रथन्त॒रस्यै॒ष वर्ण॒स्तं त्वा॑ स॒मिद्भि॑रङ्गिर॒ इत्या॑ह वामदे॒व्यस्यै॒ष वर्णो॑ बृ॒हद॑ग्ने सु॒वीर्य॒मित्या॑ह बृह॒त ए॒ष वर्णो॒ यदे॒तं तृ॒चम॒न्वाह॑ य॒ज्ञमे॒व तथ्साम॑न्वन्तं करोत्य॒ग्निर॒मुष्मि॑ल्लोँ॒क आसी॑दादि॒त्यो᳚\-ऽस्मिन्तावि॒मौ लो॒कावशा᳚न्तौ॥४४॥

%2.5.8.2
आ॒स्ता॒न्ते दे॒वा अ॑ब्रुव॒न्नेते॒मौ वि पर्यू॑हा॒मेत्यग्न॒ आ या॑हि वी॒तय॒ इत्य॒स्मिल्लोँ॒के᳚\-ऽग्निम॑दधुर्बृ॒हद॑ग्ने सु॒वीर्य॒मित्य॒मुष्मि॑ल्लोँ॒क आ॑दि॒त्यन्ततो॒ वा इ॒मौ लो॒काव॑शाम्यतां॒ यदे॒वम॒न्वाहा॒नयो᳚र्लो॒कयोः॒ शान्त्यै॒ शाम्य॑तो\-ऽस्मा इ॒मौ लो॒कौ य ए॒वं वेद॒ पञ्च॑दश सामिधे॒नीरन्वा॑ह॒ पञ्च॑दश॥४५॥

%2.5.8.3
वा अ॑र्धमा॒सस्य॒ रात्र॑यो\-ऽर्धमास॒शः सं॑वथ्स॒र आ᳚प्यते॒ तासां॒ त्रीणि॑ च श॒तानि॑ ष॒ष्टिश्चा॒क्षरा॑णि॒ ताव॑तीः संवथ्स॒रस्य॒ रात्र॑यो\-ऽक्षर॒श ए॒व सं॑वथ्स॒रमा᳚प्नोति नृ॒मेध॑श्च॒ परु॑च्छेपश्च ब्रह्म॒वाद्य॑मवदेताम॒स्मिन्दारा॑वा॒र्द्रे᳚\-ऽग्निं ज॑नयाव यत॒रो नौ॒ ब्रह्मी॑या॒निति॑ नृ॒मेधो॒\-ऽभ्य॑वद॒थ्स धू॒मम॑जनय॒त्परु॑च्छेपो॒\-ऽभ्य॑वद॒थ्सो᳚\-ऽग्निम॑जनय॒दृष॒ इत्य॑ब्रवीत्॥४६॥

%2.5.8.4
यथ्स॒माव॑द्वि॒द्व क॒था त्वम॒ग्निमजी॑जनो॒ नाहमिति॑ सामिधे॒नीना॑मे॒वाहं वर्णं॑ वे॒देत्य॑ब्रवी॒द्यद्घृ॒तव॑त्प॒दम॑नू॒च्यते॒ स आ॑सां॒ वर्ण॒स्तं त्वा॑ स॒मिद्भि॑रङ्गिर॒ इत्या॑ह सामिधे॒नीष्वे॒व तज्ज्योति॑र्जनयति॒ स्त्रिय॒स्तेन॒ यदृचः॒ स्त्रिय॒स्तेन॒ यद्गा॑य॒त्रियः॒ स्त्रिय॒स्तेन॒ यथ्सा॑मिधे॒न्यो॑ वृष॑ण्वती॒मन्वा॑ह॥४७॥

%2.5.8.5
तेन॒ पुꣴस्व॑ती॒स्तेन॒ सेन्द्रा॒स्तेन॑ मिथु॒ना अ॒ग्निर्दे॒वानां᳚ दू॒त आसी॑दु॒शना॑ का॒व्यो\-ऽसु॑राणा॒न्तौ प्र॒जाप॑तिम्प्र॒श्ञमै॑ता॒ꣳ॒ स प्र॒जाप॑तिर॒ग्निं दू॒तं वृ॑णीमह॒ इत्य॒भि प॒र्याव॑र्तत॒ ततो॑ दे॒वा अभ॑वन्प॒रासु॑रा॒ यस्यै॒वं वि॒दुषो॒\-ऽग्निं दू॒तं वृ॑णीमह॒ इत्य॒न्वाह॒ भव॑त्या॒त्मना॒ परा᳚स्य॒ भ्रातृ॑व्यो भवत्यध्व॒रव॑ती॒मन्वा॑ह॒ भ्रातृ॑व्यमे॒वैतया᳚॥४८॥

%2.5.8.6
ध्व॒र॒ति॒ शो॒चिष्के॑श॒स्तमी॑मह॒ इत्या॑ह प॒वित्र॑मे॒वैतद्यज॑मानमे॒वैतया॑ पवयति॒ समि॑द्धो अग्न आहु॒तेत्या॑ह परि॒धिमे॒वैतं परि॑ दधा॒त्यस्क॑न्दाय॒ यदत॑ ऊ॒र्ध्वम॑भ्याद॒ध्याद्यथा॑ बहिःपरि॒धि स्कन्द॑ति ता॒दृगे॒व तत्त्रयो॒ वा अ॒ग्नयो॑ हव्य॒वाह॑नो दे॒वानां᳚ कव्य॒वाह॑नः पितृ॒णाꣳ स॒हर॑क्षा॒ असु॑राणा॒न्त ए॒तर्\mbox{}ह्या शꣳ॑सन्ते॒ मां व॑रिष्यते॒ माम्॥४९॥

%2.5.8.7
इति॑ वृणी॒ध्वꣳ ह॑व्य॒वाह॑न॒मित्या॑ह॒ य ए॒व दे॒वानां॒ तं वृ॑णीत आर्\mbox{}षे॒यं वृ॑णीते॒ बन्धो॑रे॒व नैत्यथो॒ सन्त॑त्यै प॒रस्ता॑द॒र्वाचो॑ वृणीते॒ तस्मा᳚त्प॒रस्ता॑द॒र्वाञ्चो॑ मनु॒ष्या᳚न्पित॒रो\-ऽनु॒ प्र पि॑पते॥५०॥

%2.5.9.0
{\anuvakamend[{अशा᳚न्तावाह॒ पञ्च॑दशाब्रवी॒दन्वा॑है॒तया॑ वरिष्यते॒ मामेका॒न्नत्रि॒ꣳ॒शच्च॑}]}%॥८॥

%2.5.9.1
अग्ने॑ म॒हाꣳ अ॒सीत्या॑ह म॒हान् ह्ये॑ष यद॒ग्निर्ब्रा᳚ह्म॒णेत्या॑ह ब्राह्म॒णो ह्ये॑ष भा॑र॒तेत्या॑है॒ष हि दे॒वेभ्यो॑ ह॒व्यम्भर॑ति दे॒वेद्ध॒ इत्या॑ह दे॒वा ह्ये॑तमैन्ध॑त॒ मन्वि॑द्ध॒ इत्या॑ह॒ मनु॒र्\mbox{}ह्ये॑तमुत्त॑रो दे॒वेभ्य॒ ऐन्द्धर्\mbox{}षि॑ष्टुत॒ इत्या॒हर्\mbox{}ष॑यो॒ ह्ये॑तमस्तु॑व॒न्विप्रा॑नुमदित॒ इत्या॑ह॥५१॥

%2.5.9.2
विप्रा॒ ह्ये॑ते यच्छु॑श्रु॒वाꣳसः॑ कविश॒स्त इत्या॑ह क॒वयो॒ ह्ये॑ते यच्छु॑श्रु॒वाꣳसो॒ ब्रह्म॑सꣳशित॒ इत्या॑ह॒ ब्रह्म॑सꣳशितो॒ ह्ये॑ष घृ॒ताह॑वन॒ इत्या॑ह घृताहु॒तिर्\mbox{}ह्य॑स्य प्रि॒यत॑मा प्र॒णीर्य॒ज्ञाना॒मित्या॑ह प्र॒णीर्\mbox{}ह्ये॑ष य॒ज्ञानाꣳ॑ र॒थीर॑ध्व॒राणा॒मित्या॑है॒ष हि दे॑वर॒थो॑\-ऽतूर्तो॒ होतेत्या॑ह॒ न ह्ये॑तं कश्च॒न॥५२॥

%2.5.9.3
तर॑ति॒ तूर्णि॑र्\mbox{}हव्य॒वाडित्या॑ह॒ सर्व॒ꣴ॒ ह्ये॑ष तर॒त्यास्पात्रं॑ जु॒हूर्दे॒वाना॒मित्या॑ह जु॒हूर्\mbox{}ह्ये॑ष दे॒वाना᳚ञ्चम॒सो दे॑व॒पान॒ इत्या॑ह चम॒सो ह्ये॑ष दे॑व॒पानो॒\-ऽराꣳ इ॑वाग्ने ने॒मिर्दे॒वाꣴस्त्वं प॑रि॒भूर॒सीत्या॑ह दे॒वान् ह्ये॑ष प॑रि॒भूर्यद्ब्रू॒यादा व॑ह दे॒वान्दे॑वय॒ते यज॑माना॒येति॒ भ्रातृ॑व्यमस्मै॥५३॥

%2.5.9.4
ज॒न॒ये॒दा व॑ह दे॒वान् यज॑माना॒येत्या॑ह॒ यज॑मानमे॒वैतेन॑ वर्धयत्य॒ग्निम॑ग्न॒ आ व॑ह॒ सोम॒मा व॒हेत्या॑ह दे॒वता॑ ए॒व तद्य॑थापू॒र्वमुप॑ ह्वयत॒ आ चा᳚ग्ने दे॒वान् वह॑ सु॒यजा॑ च यज जातवेद॒ इत्या॑हा॒ग्निमे॒व तथ्सꣴ श्य॑ति॒ सो᳚\-ऽस्य॒ सꣳशि॑तो दे॒वेभ्यो॑ ह॒व्यं व॑हत्य॒ग्निर्\mbox{}होता᳚॥५४॥

%2.5.9.5
इत्या॑हा॒ग्निर्वै दे॒वाना॒ꣳ॒ होता॒ य ए॒व दे॒वाना॒ꣳ॒ होता॒ तं वृ॑णीते॒ स्मो व॒यमित्या॑हा॒त्मान॑मे॒व स॒त्त्वं ग॑मयति सा॒धु ते॑ यजमान दे॒वतेत्या॑हा॒शिष॑मे॒वैतामा शा᳚स्ते॒ यद्ब्रू॒याद्यो᳚\-ऽग्निꣳ होता॑र॒मवृ॑था॒ इत्य॒ग्निनो॑भ॒यतो॒ यज॑मानं॒ परि॑ गृह्णीयात् प्र॒मायु॑कः स्याद्यजमानदेव॒त्या॑ वै जु॒हूर्भ्रा॑तृव्यदेव॒त्यो॑प॒भृत्॥५॥

%2.5.9.6
यद्द्वे इ॑व ब्रू॒याद्भ्रातृ॑व्यमस्मै जनयेद्घृ॒तव॑तीमध्वर्यो॒ स्रुच॒मास्य॒स्वेत्या॑ह॒ यज॑मानमे॒वैतेन॑ वर्धयति देवा॒युव॒मित्या॑ह दे॒वान् ह्ये॑षाव॑ति वि॒श्ववा॑रा॒मित्या॑ह॒ विश्व॒ꣴ॒ ह्ये॑षाव॒तीडा॑महै दे॒वाꣳ ई॒डेन्या᳚न्नम॒स्याम॑ नम॒स्यान्॑ यजा॑म य॒ज्ञिया॒नित्या॑ह मनु॒ष्या॑ वा ई॒डेन्याः᳚ पि॒तरो॑ नम॒स्या॑ दे॒वा य॒ज्ञिया॑ दे॒वता॑ ए॒व तद्य॑थाभा॒गं य॑जति॥५६॥

%2.5.10.0
{\anuvakamend[{विप्रा॑नुमदित॒ इत्या॑ह च॒नास्मै॒ होतो॑प॒भृद्दे॒वता॑ ए॒व त्रीणि॑ च}]}%॥९॥

%2.5.10.1
त्रीꣴ स्तृ॒चाननु॑ ब्रूयाद्राज॒न्य॑स्य॒ त्रयो॒ वा अ॒न्ये रा॑ज॒न्या᳚त्पुरु॑षा ब्राह्म॒णो वैश्यः॑ शू॒द्रस्ताने॒वास्मा॒ अनु॑कान्करोति॒ पञ्च॑द॒शानु॑ ब्रूयाद्राज॒न्य॑स्य पञ्चद॒शो वै रा॑ज॒न्यः॑ स्व ए॒वैन॒ꣴ॒ स्तोमे॒ प्रति॑ ष्ठापयति त्रि॒ष्टुभा॒ परि॑ दध्यादिन्द्रि॒यं वै त्रि॒ष्टुगि॑न्द्रि॒यका॑मः॒ खलु॒ वै रा॑ज॒न्यो॑ यजते त्रि॒ष्टुभै॒वास्मा॑ इन्द्रि॒यं परि॑ गृह्णाति॒ यदि॑ का॒मये॑त॥५७॥

%2.5.10.2
ब्र॒ह्म॒व॒र्च॒सम॒स्त्विति॑ गायत्रि॒या परि॑ दध्याद्ब्रह्मवर्च॒सं वै गा॑य॒त्री ब्र॑ह्मवर्च॒समे॒व भ॑वति स॒प्तद॒शानु॑ ब्रूया॒द्वैश्य॑स्य सप्तद॒शो वै वैश्यः॒ स्व ए॒वैन॒ꣴ॒ स्तोमे॒ प्रति॑ ष्ठापयति॒ जग॑त्या॒ परि॑ दध्या॒ज्जाग॑ता॒ वै प॒शवः॑ प॒शुका॑मः॒ खलु॒ वै वैश्यो॑ यजते॒ जग॑त्यै॒वास्मै॑ प॒शून्परि॑ गृह्णा॒त्येक॑विꣳशति॒मनु॑ ब्रूयात्प्रति॒ष्ठाका॑मस्यैकवि॒ꣳ॒शः स्तोमा॑नां प्रति॒ष्ठा प्रति॑ष्ठित्यै॥५८॥

%2.5.10.3
चतु॑र्विꣳशति॒मनु॑ ब्रूयाद्ब्रह्मवर्च॒सका॑मस्य॒ चतु॑र्विꣳशत्यक्षरा गाय॒त्री गा॑य॒त्री ब्र॑ह्मवर्च॒सङ्गायत्रि॒यैवास्मै᳚ ब्रह्मवर्च॒समव॑ रुन्द्धे त्रि॒ꣳ॒शत॒मनु॑ ब्रूया॒दन्न॑कामस्य त्रि॒ꣳ॒शद॑क्षरा वि॒राडन्नं॑ वि॒राड्वि॒राजै॒वास्मा॑ अ॒न्नाद्य॒मव॑ रुन्द्धे॒ द्वात्रिꣳ॑शत॒मनु॑ब्रूयात्प्रति॒ष्ठाका॑मस्य॒ द्वात्रिꣳ॑शदक्षरानु॒ष्टु॑गनु॒ष्टुप्छन्द॑सां प्रति॒ष्ठा प्रति॑ष्ठित्यै॒ षट्त्रिꣳ॑शत॒मनु॑ ब्रूयात्प॒शुका॑मस्य॒ षट्त्रिꣳ॑शदक्षरा बृह॒ती बार्\mbox{}ह॑ताः प॒शवो॑ बृह॒त्यैवास्मै॑ प॒शून्॥५९॥

%2.5.10.4
अव॑ रुन्द्धे॒ चतु॑श्चत्वारिꣳशत॒मनु॑ ब्रूयादिन्द्रि॒यका॑मस्य॒ चतु॑श्चत्वारिꣳशदक्षरा त्रि॒ष्टुगि॑न्द्रि॒यं त्रि॒ष्टुप्त्रि॒ष्टुभै॒वास्मा॑ इन्द्रि॒यमव॑ रुन्द्धे॒\-ऽष्टाच॑त्वारिꣳशत॒मनु॑ ब्रूयात्प॒शुका॑मस्या॒ष्टाच॑त्वारिꣳशदक्षरा॒ जग॑ती॒ जाग॑ताः प॒शवो॒ जग॑त्यै॒वास्मै॑ प॒शूनव॑ रुन्द्धे॒ सर्वा॑णि॒ छन्दा॒ꣳ॒स्यनु॑ ब्रूयाद्बहुया॒जिनः॒ सर्वा॑णि॒ वा ए॒तस्य॒ छन्दा॒ꣳ॒स्यव॑रुन्द्धानि॒ यो ब॑हुया॒ज्यप॑रिमित॒मनु॑ ब्रूया॒दप॑रिमित॒स्याव॑रुद्ध्यै॥६०॥

%2.5.11.0
{\anuvakamend[{का॒मये॑त॒ प्रति॑ष्ठित्यै प॒शून्थ्स॒प्तच॑त्वारिꣳशच्च}]}%॥10॥

%2.5.11.1
निवी॑तम्मनु॒ष्या॑णाम्प्राचीनावी॒तम्पि॑तृ॒णामुप॑वीतं दे॒वाना॒मुप॑ व्ययते देवल॒क्ष्ममे॒व तत्कु॑रुते॒ तिष्ठ॒न्नन्वा॑ह॒ तिष्ठ॒न् ह्याश्रु॑ततरं॒ वद॑ति॒ तिष्ठ॒न्नन्वा॑ह सुव॒र्गस्य॑ लो॒कस्या॒भिजि॑त्या॒ आसी॑नो यजत्य॒स्मिन्ने॒व लो॒के प्रति॑ तिष्ठति॒ यत्क्रौ॒ञ्चम॒न्वाहा॑सु॒रं तद्यन्म॒न्द्रम्मा॑नु॒षं तद्यद॑न्त॒रा तथ्सदे॑वमन्त॒रानूच्यꣳ॑ सदेव॒त्वाय॑ वि॒द्वाꣳसो॒ वै॥६१॥

%2.5.11.2
पु॒रा होता॑रो\-ऽभूव॒न्तस्मा॒द्विधृ॑ता॒ अध्वा॒नो\-ऽभू॑व॒न्न पन्था॑नः॒ सम॑रुक्षन्नन्तर्वे॒द्य॑न्यः पादो॒ भव॑ति बहिर्वे॒द्य॑न्यो\-ऽथान्वा॒हाध्व॑नां॒ विधृ॑त्यै प॒थामसꣳ॑रोहा॒याथो॑ भू॒तं चै॒व भ॑वि॒ष्यच्चाव॑ रु॒न्द्धे\-ऽथो॒ परि॑मितं चै॒वाप॑रिमितं॒ चाव॑ रु॒न्द्धे\-ऽथो᳚ ग्रा॒म्याꣴश्चै॒व प॒शूना॑र॒ण्याꣴश्चाव॑ रु॒न्द्धे\-ऽथो᳚॥६२॥

%2.5.11.3
दे॒व॒लो॒कं चै॒व म॑नुष्यलो॒कं चा॒भि ज॑यति दे॒वा वै सा॑मिधे॒नीर॒नूच्य॑ य॒ज्ञं नान्व॑पश्य॒न्थ्स प्र॒जाप॑तिस्तू॒ष्णीमा॑घा॒रमाघा॑रय॒त्ततो॒ वै दे॒वा य॒ज्ञमन्व॑पश्य॒न् यत्तू॒ष्णीमा॑घा॒रमा॑घा॒रय॑ति य॒ज्ञस्यानु॑ख्यात्या॒ अथो॑ सामिधे॒नीरे॒वाभ्य॑न॒क्त्यलू᳚क्षो भवति॒ य ए॒वं वेदाथो॑ त॒र्पय॑त्ये॒वैना॒स्तृप्य॑ति प्र॒जया॑ प॒शुभिः॑॥६३॥

%2.5.11.4
य ए॒वं वेद॒ यदेक॑याघा॒रये॒देकां᳚ प्रीणीया॒द्यद्द्वाभ्यां॒ द्वे प्री॑णीया॒द्यत्ति॒सृभि॒रति॒ तद्रे॑चये॒न्मन॒सा घा॑रयति॒ मन॑सा॒ ह्यना᳚प्तमा॒प्यते॑ ति॒र्यञ्च॒मा घा॑रय॒त्यछ॑म्बट्कारं॒ वाक्च॒ मन॑श्चार्तीयेताम॒हं दे॒वेभ्यो॑ ह॒व्यं व॑हा॒मीति॒ वाग॑ब्रवीद॒हं दे॒वेभ्य॒ इति॒ मन॒स्तौ प्र॒जाप॑तिम्प्र॒श्ञमै॑ता॒ꣳ॒ सो᳚\-ऽब्रवीत्॥६४॥

%2.5.11.5
प्र॒जाप॑तिर्दू॒तीरे॒व त्वं मन॑सो\-ऽसि॒ यद्धि मन॑सा॒ ध्याय॑ति॒ तद्वा॒चा वद॒तीति॒ तत्खलु॒ तुभ्यं॒ न वा॒चा जु॑हव॒न्नित्य॑ब्रवी॒त् तस्मा॒न्मन॑सा प्र॒जाप॑तये जुह्वति॒ मन॑ इव॒ हि प्र॒जाप॑तिः प्र॒जाप॑ते॒राप्त्यै॑ परि॒धीन्थ्सम्मा᳚र्ष्टि पु॒नात्ये॒वैना॒न्त्रिर्म॑ध्य॒मं त्रयो॒ वै प्रा॒णाः प्रा॒णाने॒वाभि ज॑यति॒ त्रिर्द॑क्षिणा॒र्ध्यं॑ त्रयः॑॥६५॥

%2.5.11.6
इ॒मे लो॒का इ॒माने॒व लो॒कान॒भि ज॑यति॒ त्रिरु॑त्तरा॒र्ध्यं॑ त्रयो॒ वै दे॑व॒यानाः॒ पन्था॑न॒स्ताने॒वाभि ज॑यति॒ त्रिरुप॑ वाजयति॒ त्रयो॒ वै दे॑वलो॒का दे॑वलो॒काने॒वाभि ज॑यति॒ द्वाद॑श॒ सम्प॑द्यन्ते॒ द्वाद॑श॒ मासाः᳚ संवथ्स॒रः सं॑वथ्स॒रमे॒व प्री॑णा॒त्यथो॑ संवथ्स॒रमे॒वास्मा॒ उप॑ दधाति सुव॒र्गस्य॑ लो॒कस्य॒ सम॑ष्ट्या आघा॒रमा घा॑रयति ति॒र इ॑व॥६६॥

%2.5.11.7
वै सु॑व॒र्गो लो॒कः सु॑व॒र्गमे॒वास्मै॑ लो॒कम्प्र रो॑यत्यृ॒जुमा घा॑रयत्यृ॒जुरि॑व॒ हि प्रा॒णः सन्त॑त॒मा घा॑रयति प्रा॒णाना॑म॒न्नाद्य॑स्य॒ सन्त॑त्या॒ अथो॒ रक्ष॑सा॒मप॑हत्यै॒ यं का॒मये॑त प्र॒मायु॑कः स्या॒दिति॑ जि॒ह्मं तस्या घा॑रयेत्प्रा॒णमे॒वास्मा᳚ज्जि॒ह्मं न॑यति ता॒जक्प्र मी॑यते॒ शिरो॒ वा ए॒तद्य॒ज्ञस्य॒ यदा॑घा॒र आ॒त्मा ध्रु॒वा॥६७॥

%2.5.11.8
आ॒घा॒रमा॒घार्य॑ ध्रु॒वाꣳ सम॑नक्त्या॒त्मन्ने॒व य॒ज्ञस्य॒ शिरः॒ प्रति॑ दधात्यग्निर्दे॒वानां᳚ दू॒त आसी॒द्दैव्यो\-ऽसु॑राणा॒न्तौ प्र॒जाप॑तिम्प्र॒श्ञमैता॒ꣳ॒ स प्र॒जाप॑तिर्ब्राह्म॒णम॑ब्रवीदे॒तद्वि ब्रू॒हीत्या श्रा॑व॒येती॒दं दे॑वाः शृणु॒तेति॒ वाव तद॑ब्रवीद॒ग्निर्दे॒वो होतेति॒ य ए॒व दे॒वानां॒ तम॑वृणीत॒ ततो॑ दे॒वाः॥६८॥

%2.5.11.9
अभ॑व॒न्परा॑सुरा॒ यस्यै॒वं वि॒दुषः॑ प्रव॒रम्प्र॑वृ॒णते॒ भव॑त्या॒त्मना॒ परा᳚स्य॒ भ्रातृ॑व्यो भवति॒ यद्ब्रा᳚ह्म॒णश्चाब्रा᳚ह्मणश्च प्र॒श्ञमे॒यातां᳚ ब्राह्म॒णायाधि॑ ब्रूया॒द्यद्ब्रा᳚ह्म॒णाया॒ध्याहा॒त्मने\-ऽध्या॑ह॒ यद्ब्रा᳚ह्म॒णम्प॒राहा॒त्मनं॒ परा॑ह॒ तस्मा᳚द्ब्राह्म॒णो न प॒रोच्यः॑॥६९॥

%2.5.12.0
{\anuvakamend[{वा आ॑र॒ण्याꣴश्चाव॑ रु॒न्धे\-ऽथो॑ प॒शुभिः॒ सो᳚\-ऽब्रवीद्दक्षिणा॒र्ध्य॑न्त्रय॑ इव ध्रु॒वा दे॒वाश्च॑त्वारि॒ꣳ॒शच्च॑}]}%॥11॥

%2.5.12.1
आयु॑ष्ट आयु॒र्दा अ॑ग्न॒ आ प्या॑यस्व॒ सं ते\-ऽव॑ ते॒ हेड॒ उदु॑त्त॒मम्प्र णो॑ दे॒व्या नो॑ दि॒वो\-ऽग्ना॑विष्णू॒ अग्ना॑विष्णू इ॒मं मे॑ वरुण॒ तत्त्वा॑ या॒म्युदु॒ त्यं चि॒त्रम्। अ॒पां नपा॒दा ह्यस्था॑दु॒पस्थं॑ जि॒ह्माना॑मू॒र्ध्वो वि॒द्युतं॒ वसा॑नः। तस्य॒ ज्येष्ठ॑म्महि॒मानं॒ वह॑न्ती॒र्\mbox{}हिर॑ण्यवर्णाः॒ परि॑ यन्ति य॒ह्वीः। सम्॥७०॥

%2.5.12.2
अ॒न्या यन्त्युप॑ यन्त्य॒न्याः स॑मा॒नमू॒र्वं न॒द्यः॑ पृणन्ति। तमू॒ शुचि॒ꣳ॒ शुच॑यो दीदि॒वाꣳस॑म॒पां नपा॑तं॒ परि॑ तस्थु॒रापः॑। तमस्मे॑रा युव॒तयो॒ युवा॑नम्मर्मृ॒ज्यमा॑नाः॒ परि॑ य॒न्त्यापः॑। स शु॒क्रेण॒ शिक्व॑ना रे॒वद॒ग्निर्दी॒दाया॑नि॒ध्मो घृ॒तनि॑र्णिग॒फ्सु। इन्द्रा॒वरु॑णयोर॒हꣳ स॒म्राजो॒रव॒ आ वृ॑णे। ता नो॑ मृडात ई॒दृशे᳚। इन्द्रा॑वरुणा यु॒वम॑ध्व॒राय॑ नः॥७१॥

%2.5.12.3
वि॒शे जना॑य॒ महि॒ शर्म॑ यच्छतम्। दी॒र्घप्र॑यज्यु॒मति॒ यो व॑नु॒ष्यति॑ व॒यं ज॑येम॒ पृत॑नासु दू॒ढ्यः॑। आ नो॑ मित्रावरुणा॒ प्र बा॒हवा᳚। त्वं नो॑ अग्ने॒ वरु॑णस्य वि॒द्वां दे॒वस्य॒ हेडो\-ऽव॑ यासिसीष्ठाः। यजि॑ष्ठो॒ वह्नि॑तमः॒ शोशु॑चानो॒ विश्वा॒ द्वेषाꣳ॑सि॒ प्र मु॑मुग्ध्य॒स्मत्। स त्वं नो॑ अग्ने\-ऽव॒मो भ॑वो॒ती नेदि॑ष्ठो अ॒स्या उ॒षसो॒ व्यु॑ष्टौ। अव॑ यक्ष्व नो॒ वरु॑णम्॥७२॥

%2.5.12.4
ररा॑णो वी॒हि मृ॑डी॒कꣳ सु॒हवो॑ न एधि। प्रप्रा॒यम॒ग्निर्भ॑र॒तस्य॑ शृण्वे॒ वि यथ्सूर्यो॒ न रोच॑ते बृ॒हद्भाः। अ॒भि यः पू॒रुं पृ॑तनासु त॒स्थौ दी॒दाय॒ दैव्यो॒ अति॑थिः शि॒वो नः॑। प्र ते॑ यक्षि॒ प्र त॑ इयर्मि॒ मन्म॒ भुवो॒ यथा॒ वन्द्यो॑ नो॒ हवे॑षु। धन्व॑न्निव प्र॒पा अ॑सि॒ त्वम॑ग्न इय॒क्षवे॑ पू॒रवे᳚ प्रत्न राजन्न्।॥७३॥

%2.5.12.5
वि पाज॑सा॒ वि ज्योति॑षा। स त्वम॑ग्ने॒ प्रती॑केन॒ प्रत्यो॑ष यातुधा॒न्यः॑। उ॒रु॒क्षये॑षु॒ दीद्य॑त्। तꣳ सु॒प्रती॑कꣳ सु॒दृश॒ꣴ॒ स्वञ्च॒मवि॑द्वाꣳसो वि॒दुष्ट॑रꣳ सपेम। स य॑क्ष॒द्विश्वा॑ व॒युना॑नि वि॒द्वान्प्र ह॒व्यम॒ग्निर॒मृते॑षु वोचत्। अ॒ꣳ॒हो॒मुचे॑ वि॒वेष॒ यन्मा॒ वि न॑ इ॒न्द्रेन्द्र॑ क्ष॒त्रमि॑न्द्रि॒याणि॑ शतक्र॒तो\-ऽनु॑ ते दायि॥७४॥

%2.6.0.0

%2.6.0.0
{\anuvakamend[{य॒ह्वीः सम॑ध्व॒राय॑ नो॒ वरु॑णꣳ राज॒ꣴ॒ श्चतु॑श्चत्वारिꣳशच्च}]}%॥12॥

{\prashnaend[{स॒मिध॒श्चक्षु॑षी प्र॒जाप॑ति॒राज्यं॑ दे॒वस्य॒ स्फ्यम्ब्र॑ह्मवा॒दिनो॒\-ऽद्भिर॒ग्नेस्त्रयो॒ मनुः॑ पृथि॒व्याः प॒शवो॒\-ऽग्नीधे॑ दे॒वा वै य॒ज्ञस्य॑ यु॒क्ष्वोशन्त॑स्त्वा॒ द्वाद॑श}]%॥12॥
}
%%% END PRASHNA

\sect{षष्ठमः प्रश्नः}\setcounter{anuvakam}{0}
\dnsub{तैत्तिरीयसंहितायां द्वितीयकाण्डे षष्ठमः प्रश्नः}
%2.6.1.0
%2.6.1.1
स॒मिधो॑ यजति वस॒न्तमे॒वर्तू॒नामव॑ रुन्द्धे॒ तनू॒नपा॑तं यजति ग्री॒ष्ममे॒वाव॑ रुन्द्ध इ॒डो य॑जति व॒र्\mbox{}षा ए॒वाव॑ रुन्द्धे ब॒र्\mbox{}हिर्य॑जति श॒रद॑मे॒वाव॑ रुन्द्धे स्वाहाका॒रं य॑जति हेम॒न्तमे॒वाव॑ रुन्द्धे॒ तस्मा॒थ्स्वाहा॑कृता॒ हेम॑न्प॒शवो\-ऽव॑ सीदन्ति स॒मिधो॑ यजत्यु॒षस॑ ए॒व दे॒वता॑ना॒मव॑ रुन्द्धे॒ तनू॒नपा॑तं यजति य॒ज्ञमे॒वाव॑ रुन्द्धे॥१॥

%2.6.1.2
इ॒डो य॑जति प॒शूने॒वाव॑ रुन्द्धे ब॒र्\mbox{}हिर्य॑जति प्र॒जामे॒वाव॑ रुन्द्धे स॒मान॑यत उप॒भृत॒स्तेजो॒ वा आज्यं॑ प्र॒जा ब॒र्\mbox{}हिः प्र॒जास्वे॒व तेजो॑ दधाति स्वाहाका॒रं य॑जति॒ वाच॑मे॒वाव॑ रुन्द्धे॒ दश॒ सम्प॑द्यन्ते॒ दशा᳚क्षरा वि॒राड्वि॒राजै॒वान्नाद्य॒मव॑ रुन्द्धे स॒मिधो॑ यजत्य॒स्मिन्ने॒व लो॒के प्रति॑ तिष्ठति॒ तनू॒नपा॑तं यजति॥२॥

%2.6.1.3
य॒ज्ञ ए॒वान्तरि॑क्षे॒ प्रति॑ तिष्ठती॒डो य॑जति प॒शुष्वे॒व प्रति॑ तिष्ठति ब॒र्\mbox{}हिर्य॑जति॒ य ए॒व दे॑व॒यानाः॒ पन्था॑न॒स्तेष्वे॒व प्रति॑ तिष्ठति स्वाहाका॒रं य॑जति सुव॒र्ग ए॒व लो॒के प्रति॑ तिष्ठत्ये॒ताव॑न्तो॒ वै दे॑वलो॒कास्तेष्वे॒व य॑थापू॒र्वं प्रति॑ तिष्ठति देवासु॒रा ए॒षु लो॒केष्व॑स्पर्धन्त॒ ते दे॒वाः प्र॑या॒जैरे॒भ्यो लो॒केभ्यो\-ऽसु॑रा॒न्प्राणु॑दन्त॒ तत्प्र॑या॒जाना᳚म्॥३॥

%2.6.1.4
प्र॒या॒ज॒त्वं यस्यै॒वं वि॒दुषः॑ प्रया॒जा इ॒ज्यन्ते॒ प्रैभ्यो लो॒केभ्यो॒ भ्रातृ॑व्यान्नुदते\-ऽभि॒क्रामं॑ जुहोत्य॒भिजि॑त्यै॒ यो वै प्र॑या॒जाना᳚म्मिथु॒नं वेद॒ प्र प्र॒जया॑ प॒शुभि॑र्मिथु॒नैर्जा॑यते स॒मिधो॑ ब॒ह्वीरि॑व यजति॒ तनू॒नपा॑त॒मेक॑मिव मिथु॒नं तदि॒डो ब॒ह्वीरि॑व यजति ब॒र्\mbox{}हिरेक॑मिव मिथु॒नं तदे॒तद्वै प्र॑या॒जाना᳚म्मिथु॒नम् य ए॒वं वेद॒ प्र॥४॥

%2.6.1.5
प्र॒जया॑ प॒शुभि॑र्मिथु॒नैर्जा॑यते दे॒वानां॒ वा अनि॑ष्टा दे॒वता॒ आस॒न्नथासु॑रा य॒ज्ञम॑जिघाꣳस॒न्ते दे॒वा गा॑य॒त्रीं व्यौ॑ह॒न् पञ्चा॒क्षरा॑णि प्रा॒चीना॑नि॒ त्रीणि॑ प्रती॒चीना॑नि॒ ततो॒ वर्म॑ य॒ज्ञायाभ॑व॒द्वर्म॒ यज॑मानाय॒ यत्प्र॑याजानूया॒जा इ॒ज्यन्ते॒ वर्मै॒व तद्य॒ज्ञाय॑ क्रियते॒ वर्म॒ यज॑मानाय॒ भ्रातृ॑व्याभिभूत्यै॒ तस्मा॒द्वरू॑थम्पु॒रस्ता॒द्वर्\mbox{}षी॑यः प॒श्चाद्ध्रसी॑यो दे॒वा वै पु॒रा रक्षो᳚भ्यः॥५॥

%2.6.1.6
इति॑ स्वाहाका॒रेण॑ प्रया॒जेषु॑ य॒ज्ञꣳ स॒ꣴ॒स्थाप्य॑मपश्य॒न्तꣴ स्वा॑हाका॒रेण॑ प्रया॒जेषु॒ सम॑स्थापय॒न्वि वा ए॒तद्य॒ज्ञं छि॑न्दन्ति॒ यथ्स्वा॑हाका॒रेण॑ प्रया॒जेषु॑ सꣴस्था॒पय॑न्ति प्रया॒जानि॒ष्ट्वा ह॒वीꣳष्य॒भि घा॑रयति य॒ज्ञस्य॒ सन्त॑त्या॒ अथो॑ ह॒विरे॒वाक॒रथो॑ यथापू॒र्वमुपै॑ति पि॒ता वै प्र॒याजाः प्र॒जानू॑या॒जा यत्प्र॑या॒जानि॒ष्ट्वा ह॒वीꣳष्य॑भिघा॒रय॑ति पि॒तैव तत्पु॒त्रेण॒ साधा॑रणम्॥६॥

%2.6.1.7
कु॒रु॒ते॒ तस्मा॑दाहु॒र्यश्चै॒वं वेद॒ यश्च॒ न क॒था पु॒त्रस्य॒ केव॑लं क॒था साधा॑रणम्पि॒तुरित्यस्क॑न्नमे॒व तद्यत्प्र॑या॒जेष्वि॒ष्टेषु॒ स्कन्द॑ति गाय॒त्र्ये॑व तेन॒ गर्भं॑ धत्ते॒ सा प्र॒जां प॒शून् यज॑मानाय॒ प्र ज॑नयति॥७॥

%2.6.2.0
{\anuvakamend[{य॒ज॒ति॒ य॒ज्ञमेवा॒व॑रुन्धे॒ तनू॒नपा॑तं यजति प्रया॒जाना॑मे॒वं वेद॒ प्र रक्षो᳚भ्यः॒ साधा॑रणं॒ पञ्च॑त्रिꣳशच्च}]}%॥१॥

%2.6.2.1
चक्षु॑षी॒ वा ए॒ते य॒ज्ञस्य॒ यदाज्य॑भागौ॒ यदाज्य॑भागौ॒ यज॑ति॒ चक्षु॑षी ए॒व तद्य॒ज्ञस्य॒ प्रति॑ दधाति पूर्वा॒र्धे जु॑होति॒ तस्मा᳚त्पूर्वा॒र्धे चक्षु॑षी प्र॒बाहु॑ग्जुहोति॒ तस्मा᳚त्प्र॒बाहु॒क्चक्षु॑षी देवलो॒कं वा अ॒ग्निना॒ यज॑मा॒नो\-ऽनु॑ पश्यति पितृलो॒कꣳ सोमे॑नोत्तरा॒र्धे᳚\-ऽग्नये॑ जुहोति दक्षिणा॒र्धे सोमा॑यै॒वमि॑व॒ हीमौ लो॒काव॒नयो᳚र्लो॒कयो॒रनु॑ख्यात्यै॒ राजा॑नौ॒ वा ए॒तौ दे॒वता॑नाम्॥८॥

%2.6.2.2
यद॒ग्नीषोमा॑वन्त॒रा दे॒वता॑ इज्येते दे॒वता॑नां॒ विधृ॑त्यै॒ तस्मा॒द्राज्ञा॑ मनु॒ष्या॑ विधृ॑ता ब्रह्मवा॒दिनो॑ वदन्ति॒ किं तद्य॒ज्ञे यज॑मानः कुरुते॒ येना॒न्यतो॑दतश्च प॒शून्दा॒धारो॑भ॒यतो॑दत॒श्चेत्यृच॑म॒नूच्याज्य॑भागस्य जुषा॒णेन॑ यजति॒ तेना॒न्यतो॑दतो दाधा॒रर्च॑म॒नूच्य॑ ह॒विष॑ ऋ॒चा य॑जति॒ तेनो॑भ॒यतो॑दतो दाधार मूर्ध॒न्वती॑ पुरोनुवा॒क्या॑ भवति मू॒र्धान॑मे॒वैनꣳ॑ समा॒नानां᳚ करोति॥९॥

%2.6.2.3
नि॒युत्व॑त्या यजति॒ भ्रातृ॑व्यस्यै॒व प॒शून्नि यु॑वते के॒शिनꣳ॑ ह दा॒र्भ्यं के॒शी सात्य॑कामिरुवाच स॒प्तप॑दां ते॒ शक्व॑री॒ꣴ॒ श्वो य॒ज्ञे प्र॑यो॒क्तासे॒ यस्यै॑ वी॒र्ये॑ण॒ प्र जा॒तान्भ्रातृ॑व्यान्नु॒दते॒ प्रति॑ जनि॒ष्यमा॑णा॒न् यस्यै॑ वी॒र्ये॑णो॒भयो᳚र्लो॒कयो॒र्ज्योति॑र्ध॒त्ते यस्यै॑ वी॒र्ये॑ण पूर्वा॒र्धेना॑न॒ड्वान्भु॒नक्ति॑ जघना॒र्धेन॑ धे॒नुरिति॑ पु॒रस्ता᳚ल्लक्ष्मा पुरोनुवा॒क्या॑ भवति जा॒ताने॒व भ्रातृ॑व्या॒न्प्र णु॑दत उ॒परि॑ष्टाल्लक्ष्मा॥१०॥

%2.6.2.4
या॒ज्या॑ जनि॒ष्यमा॑णाने॒व प्रति॑ नुदते पु॒रस्ता᳚ल्लक्ष्मा पुरोनुवा॒क्या॑ भवत्य॒स्मिन्ने॒व लो॒के ज्योति॑र्धत्त उ॒परि॑ष्टाल्लक्ष्मा या॒ज्या॑मुष्मि॑न्ने॒व लो॒के ज्योति॑र्धत्ते॒ ज्योति॑ष्मन्तावस्मा इ॒मौ लो॒कौ भ॑वतो॒ य ए॒वं वेद॑ पु॒रस्ता᳚ल्लक्ष्मा पुरोनुवा॒क्या॑ भवति॒ तस्मा᳚त्पूर्व॒र्धेना॑न॒ड्वान्भु॑नक्त्यु॒परि॑ष्टाल्लक्ष्मा या॒ज्या॑ तस्मा᳚ज्जघना॒र्धेन॑ धे॒नुर्य ए॒वं वेद॑ भु॒ङ्क्त ए॑नमे॒तौ वज्र॒ आज्यं॒ वज्र॒ आज्य॑भागौ॥११॥

%2.6.2.5
वज्रो॑ वषट्का॒रस्त्रि॒वृत॑मे॒व वज्रꣳ॑ स॒म्भृत्य॒ भ्रातृ॑व्याय॒ प्र ह॑र॒त्यछ॑म्बट्कारमप॒गूर्य॒ वष॑ट्करोति॒ स्तृत्यै॑ गाय॒त्री पु॑रोनुवा॒क्या॑ भवति त्रि॒ष्टुग्या॒ज्या᳚ ब्रह्म॑न्ने॒व क्ष॒त्रम॒न्वार॑म्भयति॒ तस्मा᳚द्ब्राह्म॒णो मुख्यो॒ मुख्यो॑ भवति॒ य ए॒वं वेद॒ प्रैवैनं॑ पुरोनुवा॒क्य॑याह॒ प्र ण॑यति या॒ज्य॑या ग॒मय॑ति वषट्का॒रेणैवैनं॑ पुरोनुवा॒क्य॑या दत्ते॒ प्र य॑च्छति या॒ज्य॑या॒ प्रति॑॥१२॥

%2.6.2.6
व॒ष॒ट्का॒रेण॑ स्थापयति त्रि॒पदा॑ पुरोनुवा॒क्या॑ भवति॒ त्रय॑ इ॒मे लो॒का ए॒ष्वे॑व लो॒केषु॒ प्रति॑ तिष्ठति॒ चतु॑ष्पदा या॒ज्या॑ चतु॑ष्पद ए॒व प॒शूनव॑ रुन्द्धे द्व्यक्ष॒रो व॑षट्का॒रो द्वि॒पाद्यज॑मानः प॒शुष्वे॒वोपरि॑ष्टा॒त्प्रति॑ तिष्ठति गाय॒त्री पु॑रोनुवा॒क्या॑ भवति त्रि॒ष्टुग्या॒ज्यै॑षा वै स॒प्तप॑दा॒ शक्व॑री॒ यद्वा ए॒तया॑ दे॒वा अशि॑क्ष॒न्तद॑शक्नुव॒न् य ए॒वं वेद॑ श॒क्नोत्ये॒व यच्छिक्ष॑ति॥१३॥

%2.6.3.0
{\anuvakamend[{दे॒वता॑नाङ्करोत्यु॒परि॑ष्टाल्ल॒क्ष्मा\-ऽ\-ऽज्य॑भागौ॒ प्रति॑ श॒क्नोत्ये॒व द्वे च॑}]}%॥२॥

%2.6.3.1
प्र॒जाप॑तिर्दे॒वेभ्यो॑ य॒ज्ञान्व्यादि॑श॒थ्स आ॒त्मन्नाज्य॑मधत्त॒ तं दे॒वा अ॑ब्रुवन्ने॒ष वाव य॒ज्ञो यदाज्य॒मप्ये॒व नोत्रा॒स्त्विति॒ सो᳚\-ऽब्रवी॒द्यजान्॑ व॒ आज्य॑भागा॒वुप॑ स्तृणान॒भि घा॑रया॒निति॒ तस्मा॒द्यज॒न्त्याज्य॑भागा॒वुप॑ स्तृणन्त्य॒भि घा॑रयन्ति ब्रह्मवा॒दिनो॑ वदन्ति॒ कस्मा᳚थ्स॒त्याद्या॒तया॑मान्य॒न्यानि॑ ह॒वीꣳष्यया॑तयाम॒माज्य॒मिति॑ प्राजाप॒त्यम्॥१४॥

%2.6.3.2
इति॑ ब्रूया॒दया॑तयामा॒ हि दे॒वानां᳚ प्र॒जाप॑ति॒रिति॒ छन्दाꣳ॑सि दे॒वेभ्यो\-ऽपा᳚क्राम॒न्न वो॑\-ऽभा॒गानि॑ ह॒व्यं व॑क्ष्याम॒ इति॒ तेभ्य॑ ए॒तच्च॑तुरव॒त्तम॑धारयन्पुरोनुवा॒क्या॑यै या॒ज्या॑यै दे॒वता॑यै वषट्का॒राय॒ यच्च॑तुरव॒त्तं जु॒होति॒ छन्दाꣳ॑स्ये॒व तत्प्री॑णाति॒ तान्य॑स्य प्री॒तानि॑ दे॒वेभ्यो॑ ह॒व्यं व॑ह॒न्त्यङ्गि॑रसो॒ वा इ॒त उ॑त्त॒माः सु॑व॒र्गं लो॒कमा॑य॒न्तदृष॑यो यज्ञवा॒स्त्व॑भ्य॒वाय॒न्ते॥१५॥

%2.6.3.3
अ॒प॒श्य॒न्पु॒रो॒डाशं॑ कू॒र्मम्भू॒तꣳ सर्प॑न्तं॒ तम॑ब्रुव॒न्निन्द्रा॑य ध्रियस्व॒ बृह॒स्पत॑ये ध्रियस्व॒ विश्वे᳚भ्यो दे॒वेभ्यो᳚ ध्रिय॒स्वेति॒ स नाध्रि॑यत॒ तम॑ब्रुवन्न॒ग्नये᳚ ध्रिय॒स्वेति॒ सो᳚\-ऽग्नये᳚\-ऽध्रियत॒ यदा᳚ग्ने॒यो᳚\-ऽष्टाक॑पालो\-ऽमावा॒स्या॑यां च पौर्णमा॒स्यां चा᳚च्यु॒तो भव॑ति सुव॒र्गस्य॑ लो॒कस्या॒भिजि॑त्यै॒ तम॑ब्रुवन्क॒थाहा᳚स्था॒ इत्यनु॑पाक्तो\-ऽभूव॒मित्य॑ब्रवी॒द्यथाक्षो\-ऽनु॑पाक्तः॥१६॥

%2.6.3.4
अ॒वार्च्छ॑त्ये॒वमवा॑र॒मित्यु॒परि॑ष्टाद॒भ्यज्या॒धस्ता॒दुपा॑नक्ति सुव॒र्गस्य॑ लो॒कस्य॒ सम॑ष्ट्यै॒ सर्वा॑णि क॒पाला᳚न्य॒भि प्र॑थयति॒ ताव॑तः पुरो॒डाशा॑न॒मुष्मि॑ल्लोँ॒के॑\-ऽभि ज॑यति॒ यो विद॑ग्धः॒ स नैर्\mbox{}॑ऋ॒तो यो\-ऽशृ॑तः॒ स रौ॒द्रो यः शृ॒तः स सदे॑व॒स्तस्मा॒दवि॑दहता शृतं॒कृत्यः॑ सदेव॒त्वाय॒ भस्म॑ना॒भि वा॑सयति॒ तस्मा᳚न्मा॒ꣳ॒सेनास्थि॑ छ॒न्नं वे॒देना॒भि वा॑सयति॒ तस्मा᳚त्॥१७॥

%2.6.3.5
केशैः॒ शिर॑श्छ॒न्नं प्रच्यु॑तं॒ वा ए॒तद॒स्माल्लो॒कादग॑तं देवलो॒कं यच्छृ॒तꣳ ह॒विरन॑भिघारितमभि॒घार्योद्वा॑सयति देव॒त्रैवैन॑द्गमयति॒ यद्येकं॑ क॒पालं॒ नश्ये॒देको॒ मासः॑ संवथ्स॒रस्यान॑वेतः॒ स्यादथ॒ यज॑मानः॒ प्र मी॑येत॒ यद्द्वे नश्ये॑तां॒ द्वौ मासौ॑ संवथ्स॒रस्यान॑वेतौ॒ स्याता॒मथ॒ यज॑मानः॒ प्र मी॑येत सं॒ख्यायोद्वा॑सयति॒ यज॑मानस्य॥१८॥

%2.6.3.6
गो॒पी॒थाय॒ यदि॒ नश्ये॑दाश्वि॒नं द्वि॑कपा॒लं निर्व॑पेद्द्यावापृथि॒व्य॑मेक॑कपालम॒श्विनौ॒ वै दे॒वानां᳚ भि॒षजौ॒ ताभ्या॑मे॒वास्मै॑ भेष॒जं क॑रोति द्यावापृथि॒व्य॑ एक॑कपालो भवत्य॒नयो॒र्वा ए॒तन्न॑श्यति॒ यन्नश्य॑त्य॒नयो॑रे॒वैन॑द्विन्दति॒ प्रति॑ष्ठित्यै॥१९॥

%2.6.4.0
{\anuvakamend[{प्रा॒जा॒प॒त्यन्ते\-ऽक्षो\-ऽनु॑पाक्तो वे॒देना॒भि वा॑सयति॒ तस्मा॒द्यज॑मानस्य॒ द्वात्रिꣳ॑शच्च}]}%॥३॥

%2.6.4.1
दे॒वस्य॑ त्वा सवि॒तुः प्र॑स॒व इति॒ स्फ्यमा द॑त्ते॒ प्रसू᳚त्या अ॒श्विनो᳚र्बा॒हुभ्या॒मित्या॑हा॒श्विनौ॒ हि दे॒वाना॑मध्व॒र्यू आस्ता᳚म् पू॒ष्णो हस्ता᳚भ्या॒मित्या॑ह॒ यत्यै॑ श॒तभृ॑ष्टिरसि वानस्प॒त्यो द्वि॑ष॒तो व॒ध इत्या॑ह॒ वज्र॑मे॒व तथ्सꣴ श्य॑ति॒ भ्रातृ॑व्याय प्रहरि॒ष्यन्थ्स्त॑म्बय॒जुर्\mbox{}ह॑रत्ये॒ताव॑ती॒ वै पृ॑थि॒वी याव॑ती॒ वेदि॒स्तस्या॑ ए॒ताव॑त ए॒व भ्रातृ॑व्यं॒ निर्भ॑जति॥२०॥

%2.6.4.2
तस्मा॒न्नाभा॒गं निर्भ॑जन्ति॒ त्रिर्\mbox{}ह॑रति॒ त्रय॑ इ॒मे लो॒का ए॒भ्य ए॒वैनं॑ लो॒केभ्यो॒ निर्भ॑जति तू॒ष्णीं च॑तु॒र्थꣳ ह॑र॒त्यप॑रिमितादे॒वैनं॒ निर्भ॑ज॒त्युद्ध॑न्ति॒ यदे॒वास्या॑ अमे॒ध्यं तदप॑ ह॒न्त्युद्ध॑न्ति॒ तस्मा॒दोष॑धयः॒ परा॑ भवन्ति॒ मूलं॑ छिनत्ति॒ भ्रातृ॑व्यस्यै॒व मूलं॑ छिनत्ति पितृदेव॒त्याति॑खा॒तेय॑तीं खनति प्र॒जाप॑तिना॥२१॥

%2.6.4.3
य॒ज्ञ॒मु॒खेन॒ सम्मि॑ता॒मा प्र॑ति॒ष्ठायै॑ खनति॒ यज॑मानमे॒व प्र॑ति॒ष्ठां ग॑मयति दक्षिण॒तो वर्\mbox{}षी॑यसीं करोति देव॒यज॑नस्यै॒व रू॒पम॑कः॒ पुरी॑षवतीं करोति प्र॒जा वै प॒शवः॒ पुरी॑षम्प्र॒जयै॒वैन॑म्प॒शुभिः॒ पुरी॑षवन्तं करो॒त्युत्त॑रं परिग्रा॒हं परि॑ गृह्णात्ये॒ताव॑ती॒ वै पृ॑थि॒वी याव॑ती॒ वेदि॒स्तस्या॑ ए॒ताव॑त ए॒व भ्रातृ॑व्यं नि॒र्भज्या॒त्मन॒ उत्त॑रं परिग्रा॒हं परि॑ गृह्णाति क्रू॒रमि॑व॒ वै॥२२॥

%2.6.4.4
ए॒तत्क॑रोति॒ यद्वेदिं॑ क॒रोति॒ धा अ॑सि स्व॒धा अ॒सीति॑ योयुप्यते॒ शान्त्यै॒ प्रोक्ष॑णी॒रा सा॑दय॒त्यापो॒ वै र॑क्षो॒घ्नी रक्ष॑सा॒मप॑हत्यै॒ स्फ्यस्य॒ वर्त्म᳚न्थ्सादयति य॒ज्ञस्य॒ सन्त॑त्यै॒ यं द्वि॒ष्यात्तं ध्या॑येच्छु॒चैवैन॑मर्पयति॥२३॥

%2.6.5.0
{\anuvakamend[{भ॒ज॒ति॒ प्र॒जाप॑तिनेव॒ वै त्रय॑स्त्रिꣳशच्च}]}%॥४॥

%2.6.5.1
ब्र॒ह्म॒वा॒दिनो॑ वदन्त्य॒द्भिर्\mbox{}ह॒वीꣳषि॒ प्रौक्षीः॒ केना॒प इति॒ ब्रह्म॒णेति॑ ब्रूयाद॒द्भिर्\mbox{}ह्ये॑व ह॒वीꣳषि॑ प्रो॒क्षति॒ ब्रह्म॑णा॒प इ॒ध्माब॒र्\mbox{}हिः प्रोक्ष॑ति॒ मेध्य॑मे॒वैन॑त्करोति॒ वेदिं॒ प्रोक्ष॑त्यृ॒क्षा वा ए॒षा\-ऽलो॒मका॑\-ऽमे॒ध्या यद्वेदि॒र्मेध्या॑मे॒वैनां᳚ करोति दि॒वे त्वा॒न्तरि॑क्षाय त्वा पृथि॒व्यै त्वेति॑ ब॒र्\mbox{}हिरा॒साद्य॒ प्र॥२४॥

%2.6.5.2
उ॒क्ष॒त्ये॒भ्य ए॒वैन॑ल्लो॒केभ्यः॒ प्रोक्ष॑ति क्रू॒रमि॑व॒ वा ए॒तत्क॑रोति॒ यत्खन॑त्य॒पो नि न॑यति॒ शान्त्यै॑ पु॒रस्ता᳚त्प्रस्त॒रं गृ॑ह्णाति॒ मुख्य॑मे॒वैनं॑ करो॒तीय॑न्तं गृह्णाति प्र॒जाप॑तिना यज्ञमु॒खेन॒ सम्मि॑तम्ब॒र्\mbox{}हिः स्तृ॑णाति प्र॒जा वै ब॒र्\mbox{}हिः पृ॑थि॒वी वेदिः॑ प्र॒जा ए॒व पृ॑थि॒व्यां प्रति॑ ष्ठापय॒त्यन॑तिदृश्नꣴ स्तृणाति प्र॒जयै॒वैन॑म्प॒शुभि॒रन॑तिदृश्ञं करोति॥२५॥

%2.6.5.3
उत्त॑रम्ब॒र्\mbox{}हिषः॑ प्रस्त॒रꣳ सा॑दयति प्र॒जा वै ब॒र्\mbox{}हिर्यज॑मानः प्रस्त॒रो यज॑मानमे॒वाय॑जमाना॒दुत्त॑रं करोति॒ तस्मा॒द्यज॑मा॒नो\-ऽय॑जमाना॒दुत्त॑रो॒\-ऽन्तर्द॑धाति॒ व्यावृ॑त्त्या अ॒नक्ति॑ ह॒विष्कृ॑तमे॒वैनꣳ॑ सुव॒र्गं लो॒कं ग॑मयति त्रे॒धान॑क्ति॒ त्रय॑ इ॒मे लो॒का ए॒भ्य ए॒वैनं॑ लो॒केभ्यो॑\-ऽनक्ति॒ न प्रति॑ शृणाति॒ यत्प्र॑तिशृणी॒यादनू᳚र्ध्वम्भावुकं॒ यज॑मानस्य स्यादु॒परी॑व॒ प्र ह॑रति॥२६॥

%2.6.5.4
उ॒परी॑व॒ हि सु॑व॒र्गो लो॒को नि य॑च्छति॒ वृष्टि॑मे॒वास्मै॒ नि य॑च्छति॒ नात्य॑ग्र॒म्प्र ह॑रे॒द्यदत्य॑ग्रम्प्र॒हरे॑दत्यासा॒रिण्य॑ध्व॒र्यो\-र्नाशु॑का स्या॒न्न पु॒रस्ता॒त्प्रत्य॑स्ये॒द्यत्पु॒रस्ता᳚त्प्र॒त्यस्ये᳚थ्सुव॒र्गाल्लो॒काद्यज॑मानं॒ प्रति॑ नुदे॒त्प्राञ्च॒म्प्र ह॑रति॒ यज॑मानमे॒व सु॑व॒र्गं लो॒कं ग॑मयति॒ न विष्व॑ञ्चं॒ वि यु॑या॒द्यद्विष्व॑ञ्चं वियु॒यात्॥२७॥

%2.6.5.5
स्त्र्य॑स्य जायेतो॒र्ध्वमुद्यौ᳚त्यू॒र्ध्वमि॑व॒ हि पु॒ꣳ॒सः पुमा॑ने॒वास्य॑ जायते॒ यथ्स्फ्येन॑ वोपवे॒षेण॑ वा योयु॒प्येत॒ स्तृति॑रे॒वास्य॒ सा हस्ते॑न योयुप्यते॒ यज॑मानस्य गोपी॒थाय॑ ब्रह्मवा॒दिनो॑ वदन्ति॒ किं य॒ज्ञस्य॒ यज॑मान॒ इति॑ प्रस्त॒र इति॒ तस्य॒ क्व॑ सुव॒र्गो लो॒क इत्या॑हव॒नीय॒ इति॑ ब्रूया॒द्यत्प्र॑स्त॒रमा॑हव॒नीये᳚ प्र॒हर॑ति॒ यज॑मानमे॒व॥२८॥

%2.6.5.6
सु॒व॒र्गं लो॒कं ग॑मयति॒ वि वा ए॒तद्यज॑मानो लिशते॒ यत्प्र॑स्त॒रं यो॑यु॒प्यन्ते॑ ब॒र्\mbox{}हिरनु॒ प्रह॑रति॒ शान्त्या॑ अनारम्भ॒ण इ॑व॒ वा ए॒तर्\mbox{}ह्य॑ध्व॒र्युः स ई᳚श्व॒रो वे॑प॒नो भवि॑तोर्ध्रु॒वासीती॒माम॒भि मृ॑शती॒यं वै ध्रु॒वा\-ऽस्यामे॒व प्रति॑ तिष्ठति॒ न वे॑प॒नो भ॑व॒त्यगा(३)न॑ग्नी॒दित्या॑ह॒ यद्ब्रू॒यादग॑न्न॒ग्निरित्य॒ग्नाव॒ग्निं ग॑मये॒न्निर्यज॑मानꣳ सुव॒र्गाल्लो॒काद्भ॑जे॒दग॒न्नित्ये॒व ब्रू॑या॒द्यज॑मानमे॒व सु॑व॒र्गं लो॒कं ग॑मयति॥२९॥

%2.6.6.0
{\anuvakamend[{आ॒साद्य॒ प्रान॑तिदृश्ञं करोति हरति वियु॒याद्यज॑मानमे॒वाग्निरिति॑ स॒प्तद॑श च}]}%॥५॥

%2.6.6.1
अ॒ग्नेस्त्रयो॒ ज्यायाꣳ॑सो॒ भ्रात॑र आस॒न्ते दे॒वेभ्यो॑ ह॒व्यं वह॑न्तः॒ प्रामी॑यन्त॒ सो᳚\-ऽग्निर॑बिभेदि॒त्थं वाव स्य आर्ति॒मारि॑ष्य॒तीति॒ स निला॑यत॒ सो॑\-ऽपः प्रावि॑श॒त्तं दे॒वताः॒ प्रैष॑मैच्छ॒न्तम्मथ्स्यः॒ प्राब्र॑वी॒त्तम॑शपद्धि॒याधि॑या त्वा वध्यासु॒र्यो मा॒ प्रावो॑च॒ इति॒ तस्मा॒न्मथ्स्यं॑ धि॒याधि॑या घ्नन्ति श॒प्तः॥३०॥

%2.6.6.2
हि तमन्व॑विन्द॒न्तम॑ब्रुव॒न्नुप॑ न॒ आ व॑र्तस्व ह॒व्यं नो॑ व॒हेति॒ सो᳚\-ऽब्रवी॒द्वरं॑ वृणै॒ यदे॒व गृ॑ही॒तस्याहु॑तस्य बहिःपरि॒धि स्कन्दा॒त्तन्मे॒ भ्रातृ॑णाम्भाग॒धेय॑मस॒दिति॒ तस्मा॒द्यद्गृ॑ही॒तस्याहु॑तस्य बहिःपरि॒धि स्कन्द॑ति॒ तेषां॒ तद्भा॑ग॒धेयं॒ ताने॒व तेन॑ प्रीणाति परि॒धीन्परि॑ दधाति॒ रक्ष॑सा॒मप॑हत्यै॒ सꣴ स्प॑र्शयति॥३१॥

%2.6.6.3
रक्ष॑सा॒मन॑न्ववचाराय॒ न पु॒रस्ता॒त्परि॑ दधात्यादि॒त्यो ह्ये॑वोद्यन्पु॒रस्ता॒द्रक्षाꣳ॑स्यप॒हन्त्यू॒र्ध्वे स॒मिधा॒वा द॑धात्यु॒परि॑ष्टादे॒व रक्षा॒ꣳ॒स्यप॑ हन्ति॒ यजु॑षा॒न्यां तू॒ष्णीम॒न्याम्मि॑थुन॒त्वाय॒ द्वे आ द॑धाति द्वि॒पाद्यज॑मानः॒ प्रति॑ष्ठित्यै ब्रह्मवा॒दिनो॑ वदन्ति॒ स त्वै य॑जेत॒ यो य॒ज्ञस्यार्त्या॒ वसी॑या॒न्थ्स्यादिति॒ भूप॑तये॒ स्वाहा॒ भुव॑नपतये॒ स्वाहा॑ भू॒ताना᳚म्॥३२॥

%2.6.6.4
पत॑ये॒ स्वाहेति॑ स्क॒न्नमनु॑ मन्त्रयेत य॒ज्ञस्यै॒व तदार्त्या॒ यज॑मानो॒ वसी॑यान्भवति॒ भूय॑सी॒र्\mbox{}हि दे॒वताः᳚ प्री॒णाति॑ जा॒मि वा ए॒तद्य॒ज्ञस्य॑ क्रियते॒ यद॒न्वञ्चौ॑ पुरो॒डाशा॑वुपाꣳशुया॒जम॑न्त॒रा य॑ज॒त्यजा॑मित्वा॒याथो॑ मिथुन॒त्वाया॒ग्निर॒मुष्मि॑ल्लोँ॒क आसी᳚द्य॒मो᳚\-ऽस्मिन्ते दे॒वा अ॑ब्रुव॒न्नेते॒मौ वि पर्यू॑हा॒मेत्य॒न्नाद्ये॑न दे॒वा अ॒ग्निम्॥३३॥

%2.6.6.5
उ॒पाम॑न्त्रयन्त रा॒ज्येन॑ पि॒तरो॑ य॒मं तस्मा॑द॒ग्निर्दे॒वाना॑मन्ना॒दो य॒मः पि॑तृ॒णाꣳ राजा॒ य ए॒वं वेद॒ प्र रा॒ज्यम॒न्नाद्य॑माप्नोति॒ तस्मा॑ ए॒तद्भा॑ग॒धेय॒म्प्राय॑च्छ॒न् यद॒ग्नये᳚ स्विष्ट॒कृते॑\-ऽव॒द्यन्ति॒ यद॒ग्नये᳚ स्विष्ट॒कृते॑\-ऽव॒द्यति॑ भाग॒धेये॑नै॒व तद्रु॒द्रꣳ सम॑र्धयति स॒कृथ्स॑कृ॒दव॑ द्यति स॒कृदि॑व॒ हि रु॒द्र उ॑त्तरा॒र्धादव॑ द्यत्ये॒षा वै रु॒द्रस्य॑॥३४॥

%2.6.6.6
दिख्स्वाया॑मे॒व दि॒शि रु॒द्रं नि॒रव॑दयते॒ द्विर॒भि घा॑रयति चतुरव॒त्तस्याप्त्यै॑ प॒शवो॒ वै पूर्वा॒ आहु॑तय ए॒ष रु॒द्रो यद॒ग्निर्यत्पूर्वा॒ आहु॑तीर॒भि जु॑हु॒याद्रु॒द्राय॑ प॒शूनपि॑ दध्यादप॒शुर्यज॑मानः स्यादति॒हाय॒ पूर्वा॒ आहु॑तीर्जुहोति पशू॒नां गो॑पी॒थाय॑॥३५॥

%2.6.7.0
{\anuvakamend[{श॒प्तः स्प॑र्शयति भू॒ताना॑म॒ग्निꣳ रु॒द्रस्य॑ स॒प्तत्रिꣳ॑शच्च}]}%॥६॥

%2.6.7.1
मनुः॑ पृथि॒व्या य॒ज्ञिय॑मैच्छ॒थ्स घृ॒तं निषि॑क्तमविन्द॒थ्सो᳚\-ऽब्रवी॒त्को᳚\-ऽस्येश्व॒रो य॒ज्ञे\-ऽपि॒ कर्तो॒रिति॒ ताव॑ब्रूताम्मि॒त्रावरु॑णौ॒ गोरे॒वावमी᳚श्व॒रौ कर्तोः᳚ स्व॒ इति॒ तौ ततो॒ गाꣳ समै॑रयता॒ꣳ॒ सा यत्र॑यत्र॒ न्यक्रा॑म॒त्ततो॑ घृ॒तम॑पीड्यत॒ तस्मा᳚द्घृ॒तप॑द्युच्यते॒ तद॑स्यै॒ जन्मोप॑हूतꣳ रथन्त॒रꣳ स॒ह पृ॑थि॒व्येत्या॑ह॥३६॥

%2.6.7.2
इ॒यं वै र॑थन्त॒रमि॒मामे॒व स॒हान्नाद्ये॒नोप॑ ह्वयत॒ उप॑हूतं वामदे॒व्यꣳ स॒हान्तरि॑क्षे॒णेत्या॑ह प॒शवो॒ वै वा॑मदे॒व्यं प॒शूने॒व स॒हान्तरि॑क्षे॒णोप॑ ह्वयत॒ उप॑हूतम्बृ॒हथ्स॒ह दि॒वेत्या॑है॒रं वै बृ॒हदिरा॑मे॒व स॒ह दि॒वोप॑ ह्वयत॒ उप॑हूताः स॒प्त होत्रा॒ इत्या॑ह॒ होत्रा॑ ए॒वोप॑ ह्वयत॒ उप॑हूता धे॒नुः॥३७॥

%2.6.7.3
स॒हर्\mbox{}ष॒भेत्या॑ह मिथु॒नमे॒वोप॑ ह्वयत॒ उप॑हूतो भ॒क्षः सखेत्या॑ह सोमपी॒थमे॒वोप॑ ह्वयत॒ उप॑हू॒ताँ (4) हो इत्या॑हा॒त्मान॑मे॒वोप॑ ह्वयत आ॒त्मा ह्युप॑हूतानां॒ वसि॑ष्ठ॒ इडा॒मुप॑ ह्वयते प॒शवो॒ वा इडा॑ प॒शूने॒वोप॑ ह्वयते च॒तुरुप॑ ह्वयते॒ चतु॑ष्पादो॒ हि प॒शवो॑ मान॒वीत्या॑ह॒ मनु॒र्\mbox{}ह्ये॑ताम्॥३८॥

%2.6.7.4
अग्रे\-ऽप॑श्यद्घृ॒तप॒दीत्या॑ह॒ यदे॒वास्यै॑ प॒दाद्घृ॒तमपी᳚ड्यत॒ तस्मा॑दे॒वमा॑ह मैत्रावरु॒णीत्या॑ह मि॒त्रावरु॑णौ॒ ह्ये॑नाꣳ स॒मैर॑यतां॒ ब्रह्म॑ दे॒वकृ॑त॒मुप॑हूत॒मित्या॑ह॒ ब्रह्मै॒वोप॑ ह्वयते॒ दैव्या॑ अध्व॒र्यव॒ उप॑हूता॒ उप॑हूता मनु॒ष्या॑ इत्या॑ह देवमनु॒ष्याने॒वोप॑ ह्वयते॒ य इ॒मं य॒ज्ञमवा॒न् ये य॒ज्ञप॑तिं॒ वर्धा॒नित्या॑ह॥३९॥

%2.6.7.5
य॒ज्ञाय॑ चै॒व यज॑मानाय चा॒शिष॒मा शा᳚स्त॒ उप॑हूते॒ द्यावा॑पृथि॒वी इत्या॑ह॒ द्यावा॑पृथि॒वी ए॒वोप॑ ह्वयते पूर्व॒जे ऋ॒ताव॑री॒ इत्या॑ह पूर्व॒जे ह्ये॑ते ऋ॒ताव॑री दे॒वी दे॒वपु॑त्रे॒ इत्या॑ह दे॒वी ह्ये॑ते दे॒वपु॑त्रे॒ उप॑हूतो॒\-ऽयं यज॑मान॒ इत्या॑ह॒ यज॑मानमे॒वोप॑ ह्वयत॒ उत्त॑रस्यां देवय॒ज्याया॒मुप॑हूतो॒ भूय॑सि हवि॒ष्कर॑ण॒ उप॑हूतो दि॒व्ये धाम॒न्नुप॑हूतः॥४०॥

%2.6.7.6
इत्या॑ह प्र॒जा वा उत्त॑रा देवय॒ज्या प॒शवो॒ भूयो॑ हवि॒ष्कर॑णꣳ सुव॒र्गो लो॒को दि॒व्यं धामे॒दम॑सी॒दम॒सीत्ये॒व य॒ज्ञस्य॑ प्रि॒यं धामोप॑ ह्वयते॒ विश्व॑मस्य प्रि॒यमुप॑हूत॒मित्या॒हाछ॑म्बट्कारमे॒वोप॑ ह्वयते॥४१॥

%2.6.8.0
{\anuvakamend[{आ॒ह॒ धे॒नुरे॒तां वर्धा॒नित्या॑ह॒ धाम॒न्नुप॑हूत॒श्चतु॑स्त्रिꣳशच्च}]}%॥७॥

%2.6.8.1
प॒शवो॒ वा इडा᳚ स्व॒यमा द॑त्ते॒ काम॑मे॒वात्मना॑ पशू॒नामा द॑त्ते॒ न ह्य॑न्यः काम॑म्पशू॒नाम्प्र॒यच्छ॑ति वा॒चस्पत॑ये त्वा हु॒तम्प्राश्ना॒मीत्या॑ह॒ वाच॑मे॒व भा॑ग॒धेये॑न प्रीणाति॒ सद॑स॒स्पत॑ये त्वा हु॒तम्प्राश्ना॒मीत्या॑ह स्व॒गाकृ॑त्यै चतुरव॒त्तम्भ॑वति ह॒विर्वै च॑तुरव॒त्तम्प॒शव॑श्चतुरव॒त्तं यद्धोता᳚ प्राश्नी॒याद्धोता᳚॥४२॥

%2.6.8.2
आर्ति॒मार्च्छे॒द्यद॒ग्नौ जु॑हु॒याद्रु॒द्राय॑ प॒शूनपि॑ दध्यादप॒शुर्यज॑मानः स्याद्वा॒चस्पत॑ये त्वा हु॒तम्प्राश्ना॒मीत्या॑ह प॒रोक्ष॑मे॒वैन॑ज्जुहोति॒ सद॑स॒स्पत॑ये त्वा हु॒तम्प्राश्ना॒मीत्या॑ह स्व॒गाकृ॑त्यै॒ प्राश्न॑न्ति ती॒र्थ ए॒व प्राश्न॑न्ति॒ दक्षि॑णां ददाति ती॒र्थ ए॒व दक्षि॑णां ददाति॒ वि वा ए॒तद्य॒ज्ञम्॥४३॥

%2.6.8.3
छि॒न्द॒न्ति॒ यन्म॑ध्य॒तः प्रा॒श्नन्त्य॒द्भिर्मा᳚र्जयन्त॒ आपो॒ वै सर्वा॑ दे॒वता॑ दे॒वता॑भिरे॒व य॒ज्ञꣳ सं त॑न्वन्ति दे॒वा वै य॒ज्ञाद्रु॒द्रम॒न्तरा॑य॒न्थ्स य॒ज्ञम॑विध्य॒त्तं दे॒वा अ॒भि सम॑गच्छन्त॒ कल्प॑तां न इ॒दमिति॒ ते᳚\-ऽब्रुव॒न्थ्स्वि॑ष्टं॒ वै न॑ इ॒दम्भ॑विष्यति॒ यदि॒मꣳ रा॑धयि॒ष्याम॒ इति॒ तथ्स्वि॑ष्ट॒कृतः॑ स्विष्टकृ॒त्त्वन्तस्यावि॑द्धं॒ निः॥४४॥

%2.6.8.4
अ॒कृ॒न्त॒न् यवे॑न॒ सम्मि॑तं॒ तस्मा᳚द्यवमा॒त्रमव॑ द्ये॒द्यज्ज्यायो॑\-ऽव॒द्येद्रो॒पये॒त्तद्य॒ज्ञस्य॒ यदुप॑ च स्तृणी॒याद॒भि च॑ घा॒रये॑दुभयतःसꣴश्वा॒यि कु॑र्यादव॒दाया॒भि घा॑रयति॒ द्विः सम्प॑द्यते द्वि॒पाद्यज॑मानः॒ प्रति॑ष्ठित्यै॒ यत्ति॑र॒श्चीन॑मति॒हरे॒दन॑भिविद्धं य॒ज्ञस्या॒भि वि॑ध्ये॒दग्रे॑ण॒ परि॑ हरति ती॒र्थेनै॒व परि॑ हरति॒ तत्पू॒ष्णे पर्य॑हर॒न्तत्॥४५॥

%2.6.8.5
पू॒षा प्राश्य॑ द॒तो॑\-ऽरुण॒त्तस्मा᳚त्पू॒षा प्र॑पि॒ष्टभा॑गो\-ऽद॒न्तको॒ हि तं दे॒वा अ॑ब्रुव॒न्वि वा अ॒यमा᳚र्ध्यप्राशित्रि॒यो वा अ॒यम॑भू॒दिति॒ तद्बृह॒स्पत॑ये॒ पर्य॑हर॒न्थ्सो॑\-ऽबिभे॒द्बृह॒स्पति॑रि॒त्थं वाव स्य आर्ति॒मारि॑ष्य॒तीति॒ स ए॒तम्मन्त्र॑मपश्य॒थ्सूर्य॑स्य त्वा॒ चक्षु॑षा॒ प्रति॑ पश्या॒मीत्य॑ब्रवी॒न्न हि सूर्य॑स्य॒ चक्षुः॑॥४६॥

%2.6.8.6
किं च॒न हि॒नस्ति॒ सो॑\-ऽबिभेत्प्रतिगृ॒ह्णन्तं॑ मा हिꣳसिष्य॒तीति॑ दे॒वस्य॑ त्वा सवि॒तुः प्र॑स॒वे᳚\-ऽश्विनो᳚र्बा॒हु\-भ्यां᳚ पू॒ष्णो हस्ता᳚भ्यां॒ प्रति॑ गृह्णा॒मीत्य॑ब्रवीथ्सवि॒तृप्र॑सूत ए॒वैन॒द्ब्रह्म॑णा दे॒वता॑भिः॒ प्रत्य॑गृह्णा॒थ्सो॑\-ऽबिभेत्प्रा॒श्नन्तं॑ मा हिꣳसिष्य॒तीत्य॒ग्नेस्त्वा॒स्ये॑न॒ प्राश्ना॒मीत्य॑ब्रवी॒न्न ह्य॑ग्नेरा॒स्यं॑ किं च॒न हि॒नस्ति॒ सो॑\-ऽबिभेत्॥४७॥

%2.6.8.7
प्राशि॑तं मा हिꣳसिष्य॒तीति॑ ब्राह्म॒णस्यो॒दरे॒णेत्य॑ब्रवी॒न्न हि ब्रा᳚ह्म॒णस्यो॒दरं॒ किं च॒न हि॒नस्ति॒ बृह॒स्पते॒र्ब्रह्म॒णेति॒ स हि ब्रह्मि॒ष्ठो\-ऽप॒ वा ए॒तस्मा᳚त्प्रा॒णाः क्रा॑मन्ति॒ यः प्रा॑शि॒त्रम्प्रा॒श्नात्य॒द्भिर्मा᳚र्जयि॒त्वा प्रा॒णान्थ्सम्मृ॑शते॒\-ऽमृतं॒ वै प्रा॒णा अ॒मृत॒मापः॑ प्रा॒णाने॒व य॑थास्था॒नमुप॑ ह्वयते॥४८॥

%2.6.9.0
{\anuvakamend[{प्रा॒श्ञी॒याद्धोता॑ य॒ज्ञन्निर॑हर॒न्तच्चक्षु॑रा॒स्य॑ङ्किं च॒न हि॒नस्ति॒ सो॑\-ऽबिभे॒च्चतु॑श्चत्वारिꣳशच्च}]}%॥८॥

%2.6.9.1
अ॒ग्नीध॒ आ द॑धात्य॒ग्निमु॑खाने॒वर्तून्प्री॑णाति स॒मिध॒मा द॑धा॒त्युत्त॑रासा॒माहु॑तीनां॒ प्रति॑ष्ठित्या॒ अथो॑ स॒मिद्व॑त्ये॒व जु॑होति परि॒धीन्थ्सम्मा᳚र्ष्टि पु॒नात्ये॒वैना᳚न्थ्स॒कृथ्स॑कृ॒थ्सम्मा᳚र्ष्टि॒ परा॑ङिव॒ ह्ये॑तर्\mbox{}हि॑ य॒ज्ञश्च॒तुः सम्प॑द्यते॒ चतु॑ष्पादः प॒शवः॑ प॒शूने॒वाव॑ रुन्द्धे॒ ब्रह्म॒न्प्र स्था᳚स्याम॒ इत्या॒हात्र॒ वा ए॒तर्\mbox{}हि॑ य॒ज्ञः श्रि॒तः॥४९॥

%2.6.9.2
यत्र॑ ब्र॒ह्मा यत्रै॒व य॒ज्ञः श्रि॒तस्तत॑ ए॒वैन॒मा र॑भते॒ यद्धस्ते॑न प्र॒मीवे᳚द्वेप॒नः स्या॒द्यच्छी॒र्ष्णा शी॑र्\mbox{}षक्ति॒मान्थ्स्या॒द्यत्तू॒ष्णीमासी॒तास॑म्प्रत्तो य॒ज्ञः स्या॒त्प्र ति॒ष्ठेत्ये॒व ब्रू॑याद्वा॒चि वै य॒ज्ञः श्रि॒तो यत्रै॒व य॒ज्ञः श्रि॒तस्तत॑ ए॒वैन॒ꣳ॒ सम्प्र य॑च्छति॒ देव॑ सवितरे॒तत्ते॒ प्र॥५०॥

%2.6.9.3
आ॒हेत्या॑ह॒ प्रसू᳚त्यै॒ बृह॒स्पति॑र्ब्र॒ह्मेत्या॑ह॒ स हि ब्रह्मि॑ष्ठः॒ स य॒ज्ञम्पा॑हि॒ स य॒ज्ञप॑तिम्पाहि॒ स माम्पा॒हीत्या॑ह य॒ज्ञाय॒ यज॑मानाया॒त्मने॒ तेभ्य॑ ए॒वाशिष॒मा शा॒स्ते\-ऽना᳚र्त्या आ॒श्राव्या॑ह दे॒वान् य॒जेति॑ ब्रह्मवा॒दिनो॑ वदन्ती॒ष्टा दे॒वता॒ अथ॑ कत॒म ए॒ते दे॒वा इति॒ छन्दा॒ꣳ॒सीति॑ ब्रूयाद्गाय॒त्रीं त्रि॒ष्टुभम्᳚॥५१॥

%2.6.9.4
जग॑ती॒मित्यथो॒ खल्वा॑हुर्ब्राह्म॒णा वै छन्दा॒ꣳ॒सीति॒ ताने॒व तद्य॑जति दे॒वानां॒ वा इ॒ष्टा दे॒वता॒ आस॒न्नथा॒ग्निर्नोद॑ज्वल॒त्तं दे॒वा आहु॑तीभिरनूया॒जेष्वन्व॑विन्द॒न् यद॑नूया॒जान् यज॑त्य॒ग्निमे॒व तथ्समि॑न्द्ध ए॒तदु॒र्वै नामा॑सु॒र आ॑सी॒थ्स ए॒तर्\mbox{}हि॑ य॒ज्ञस्या॒शिष॑मवृङ्क्त॒ यद्ब्रू॒यादे॒तत्॥५२॥

%2.6.9.5
उ॒ द्या॒वा॒पृ॒थि॒वी॒ भ॒द्रम॑भू॒दित्ये॒तदु॑मे॒वासु॒रं य॒ज्ञस्या॒शिषं॑ गमयेदि॒दं द्या॑वापृथिवी भ॒द्रम॑भू॒दित्ये॒व ब्रू॑या॒द्यज॑मानमे॒व य॒ज्ञस्या॒शिष॑म्गमय॒त्यार्ध्म॑ सूक्तवा॒कमु॒त न॑मोवा॒कमित्या॑हे॒दम॑रा॒थ्स्मेति॒ वावैतदा॒होप॑श्रितो दि॒वः पृ॑थि॒व्योरित्या॑ह॒ द्यावा॑पृथि॒व्योर्\mbox{}हि य॒ज्ञ उप॑श्रित॒ ओम॑न्वती ते॒\-ऽस्मिन् य॒ज्ञे य॑जमान॒ द्यावा॑पृथि॒वी॥५३॥

%2.6.9.6
स्ता॒मित्या॑हा॒शिष॑मे॒वैतामा शा᳚स्ते॒ यद्ब्रू॒याथ्सू॑पावसा॒ना च॑ स्वध्यवसा॒ना चेति॑ प्र॒मायु॑को॒ यज॑मानः स्याद्य॒दा हि प्र॒मीय॒ते\-ऽथे॒मामु॑पाव॒स्यति॑ सूपचर॒णा च॑ स्वधिचर॒णा चेत्ये॒व ब्रू॑या॒द्वरी॑यसीमे॒वास्मै॒ गव्यू॑ति॒मा शा᳚स्ते॒ न प्र॒मायु॑को भवति॒ तयो॑रा॒विद्य॒ग्निरि॒दꣳ ह॒विर॑जुष॒तेत्या॑ह॒ या अया᳚क्ष्म॥५४॥

%2.6.9.7
दे॒वता॒स्ता अ॑रीरधा॒मेति॒ वावैतदा॑ह॒ यन्न नि॑र्दि॒शेत्प्रति॑वेशं य॒ज्ञस्या॒शीर्ग॑च्छे॒दा शा᳚स्ते॒\-ऽयं यज॑मानो॒\-ऽसावित्या॑ह नि॒र्दिश्यै॒वैनꣳ॑ सुव॒र्गं लो॒कं ग॑मय॒त्यायु॒रा शा᳚स्ते सुप्रजा॒स्त्वमा शा᳚स्त॒ इत्या॑हा॒शिष॑मे॒वैतामा शा᳚स्ते सजातवन॒स्यामा शा᳚स्त॒ इत्या॑ह प्रा॒णा वै स॑जा॒ताः प्रा॒णाने॒व॥५॥

%2.6.9.8
नान्तरे॑ति॒ तद॒ग्निर्दे॒वो दे॒वेभ्यो॒ वन॑ते व॒यम॒ग्नेर्मानु॑षा॒ इत्या॑हा॒ग्निर्दे॒वेभ्यो॑ वनु॒ते व॒यं म॑नु॒ष्ये᳚भ्य॒ इति॒ वावैतदा॑हे॒ह गति॑र्वा॒मस्ये॒दं च॒ नमो॑ दे॒वेभ्य॒ इत्या॑ह॒ याश्चै॒व दे॒वता॒ यज॑ति॒ याश्च॒ न ताभ्य॑ ए॒वोभयी᳚भ्यो॒ नम॑स्करोत्या॒त्मनो\-ऽना᳚र्त्यै॥५६॥

%2.6.10.0
{\anuvakamend[{श्रि॒तस्ते॒ प्र त्रि॒ष्टुभ॑मे॒तद्द्यावा॑पृथि॒वी या अया᳚क्ष्म प्रा॒णाने॒व षट्च॑त्वारिꣳशच्च}]}%॥९॥

%2.6.10.1
दे॒वा वै य॒ज्ञस्य॑ स्वगाक॒र्तारं॒ नावि॑न्द॒न्ते शं॒ युम्बा॑र्\mbox{}हस्प॒त्यम॑ब्रुवन्नि॒मं नो॑ य॒ज्ञꣴ स्व॒गा कु॒र्विति॒ सो᳚\-ऽब्रवी॒द्वरं॑ वृणै॒ यदे॒वाब्रा᳚ह्मणो॒क्तो\-ऽश्र॑द्दधानो॒ यजा॑तै॒ सा मे॑ य॒ज्ञस्या॒शीर॑स॒दिति॒ तस्मा॒द्यदब्रा᳚ह्मणो॒क्तो\-ऽश्र॑द्दधानो॒ यज॑ते शं॒ युमे॒व तस्य॑ बार्\mbox{}हस्प॒त्यं य॒ज्ञस्या॒शीर्ग॑च्छत्ये॒तन्ममेत्य॑ब्रवी॒त्किम्मे᳚ प्र॒जायाः᳚॥५७॥

%2.6.10.2
इति॒ यो॑\-ऽपगु॒रातै॑ श॒तेन॑ यातया॒द्यो नि॒हन॑थ्स॒हस्रे॑ण यातया॒द्यो लोहि॑तं क॒रव॒द्याव॑तः प्र॒स्कद्य॑ पाꣳ॒सून्थ्सं॑गृ॒ह्णात्ताव॑तः संवथ्स॒रान्पि॑तृलो॒कं न प्र जा॑ना॒दिति॒ तस्मा᳚द्ब्राह्म॒णाय॒ नाप॑ गुरेत॒ न नि ह॑न्या॒न्न लोहि॑तं कुर्यादे॒ताव॑ता॒ हैन॑सा भवति॒ तच्छं॒ योरा वृ॑णीमह॒ इत्या॑ह य॒ज्ञमे॒व तथ्स्व॒गा क॑रोति॒ तत्॥५८॥

%2.6.10.3
शं॒ योरा वृ॑णीमह॒ इत्या॑ह शं॒ युमे॒व बा॑र्\mbox{}हस्प॒त्यम्भा॑ग॒धेये॑न॒ सम॑र्धयति गा॒तुं य॒ज्ञाय॑ गा॒तुं य॒ज्ञप॑तय॒ इत्या॑हा॒शिष॑मे॒वैतामा शा᳚स्ते॒ सोमं॑ यजति॒ रेत॑ ए॒व तद्द॑धाति॒ त्वष्टा॑रं यजति॒ रेत॑ ए॒व हि॒तं त्वष्टा॑ रू॒पाणि॒ वि क॑रोति दे॒वाना॒म्पत्नी᳚र्यजति मिथुन॒त्वाया॒ग्निं गृ॒हप॑तिं यजति॒ प्रति॑ष्ठित्यै जा॒मि वा ए॒तद्य॒ज्ञस्य॑ क्रियते॥५९॥

%2.6.10.4
यदाज्ये॑न प्रया॒जा इ॒ज्यन्त॒ आज्ये॑न पत्नीसंया॒जा ऋच॑म॒नूच्य॑ पत्नीसंया॒जाना॑मृ॒चा य॑ज॒त्यजा॑मित्वा॒याथो॑ मिथुन॒त्वाय॑ प॒ङ्क्तिप्रा॑यणो॒ वै य॒ज्ञः प॒ङ्क्त्यु॑दयनः॒ पञ्च॑ प्रया॒जा इ॑ज्यन्ते च॒त्वारः॑ पत्नीसंया॒जाः स॑मिष्टय॒जुः प॑ञ्च॒मम्प॒ङ्क्तिमे॒वानु॑ प्र॒ यन्ति॑ प॒ङ्क्तिमनूद्य॑न्ति॥६०॥

%2.6.11.0
{\anuvakamend[{प्र॒जायाः᳚ करोति॒ तत्क्रि॑यते॒ त्रय॑स्त्रिꣳशच्च}]}%॥10॥

%2.6.11.1
यु॒क्ष्वा हि दे॑व॒हूत॑मा॒ꣳ॒ अश्वाꣳ॑ अग्ने र॒थीरि॑व। नि होता॑ पू॒र्व्यः स॑दः। उ॒त नो॑ देव दे॒वाꣳ अच्छा॑ वोचो वि॒दुष्ट॑रः। श्रद्विश्वा॒ वार्या॑ कृधि। त्वꣳ ह॒ यद्य॑विष्ठ्य॒ सह॑सः सूनवाहुत। ऋ॒तावा॑ य॒ज्ञियो॒ भुवः॑। अ॒यम॒ग्निः स॑ह॒स्रिणो॒ वाज॑स्य श॒तिन॒स्पतिः॑। मू॒र्धा क॒वी र॑यी॒णाम्। तं ने॒मिमृ॒भवो॑ य॒था न॑मस्व॒ सहू॑तिभिः। नेदी॑यो य॒ज्ञम्॥६१॥

%2.6.11.2
अ॒ङ्गि॒रः॒। तस्मै॑ नू॒नम॒भिद्य॑वे वा॒चा वि॑रूप॒ नित्य॑या। वृष्णे॑ चोदस्व सुष्टु॒तिम्। कमु॑ ष्विदस्य॒ सेन॑या॒ग्नेरपा॑कचक्षसः। प॒णिं गोषु॑ स्तरामहे। मा नो॑ दे॒वानां॒ विशः॑ प्रस्ना॒तीरि॑वो॒स्राः। कृ॒शं न हा॑सु॒रघ्नि॑याः। मा नः॑ समस्य दू॒ढ्यः॑ परि॑द्वेषसो अꣳह॒तिः। ऊ॒र्मिर्न नाव॒मा व॑धीत्। नम॑स्ते अग्न॒ ओज॑से गृ॒णन्ति॑ देव कृ॒ष्टयः॑। अमैः᳚॥६२॥

%2.6.11.3
अ॒मित्र॑मर्दय। कु॒विथ्सु नो॒ गवि॑ष्ट॒ये\-ऽग्ने॑ सं॒वेषि॑षो र॒यिम्। उरु॑कृदु॒रु ण॑स्कृधि। मा नो॑ अ॒स्मिन्म॑हाध॒ने परा॑ वर्ग्भार॒भृद्य॑था। सं॒वर्ग॒ꣳ॒ सꣳ र॒यिञ्ज॑य। अ॒न्यम॒स्मद्भि॒या इ॒यमग्ने॒ सिष॑क्तु दु॒च्छुना᳚। वर्धा॑ नो॒ अम॑व॒च्छवः॑। यस्याजु॑षन्नम॒स्विनः॒ शमी॒मदु॑र्मखस्य वा। तं घेद॒ग्निर्वृ॒धाव॑ति। पर॑स्या॒ अधि॑॥६३॥

%2.6.11.4
सं॒वतो\-ऽव॑राꣳ अ॒भ्या त॑र। यत्रा॒हमस्मि॒ ताꣳ अ॑व। वि॒द्मा हि ते॑ पु॒रा व॒यमग्ने॑ पि॒तुर्यथाव॑सः। अधा॑ ते सु॒म्नमी॑महे। य उ॒ग्र इ॑व शर्य॒हा ति॒ग्मशृ॑ङ्गो॒ न वꣳस॑गः। अग्ने॒ पुरो॑ रु॒रोजि॑थ। सखा॑यः॒ सं वः॑ स॒म्यञ्च॒मिष॒ꣴ॒ स्तोमं॑ चा॒ग्नये᳚। वर्\mbox{}षि॑ष्ठाय क्षिती॒नामू॒र्जो नप्त्रे॒ सह॑स्वते। सꣳस॒मिद्यु॑वसे वृष॒न्नग्ने॒ विश्वा᳚न्य॒र्य आ। इ॒डस्प॒दे समि॑ध्यसे॒ स नो॒ वसू॒न्या भ॑र। प्रजा॑पते॒ स वे॑द॒ सोमा॑पूषणे॒मौ दे॒वौ॥६४॥

%2.6.12.0
{\anuvakamend[{य॒ज्ञममै॒रधि॑ वृष॒न्नेका॒न्नविꣳ॑श॒तिश्च॑}]}%॥11॥

%2.6.12.1
उ॒शन्त॑स्त्वा हवामह उ॒शन्तः॒ समि॑धीमहि। उ॒शन्नु॑श॒त आ व॑ह॒ पि॒तॄन् ह॒विषे॒ अत्त॑वे। त्वꣳ सो॑म॒ प्रचि॑कितो मनी॒षा त्वꣳ रजि॑ष्ठ॒मनु॑ नेषि॒ पन्था᳚म्। तव॒ प्रणी॑ती पि॒तरो॑ न इन्दो दे॒वेषु॒ रत्न॑मभजन्त॒ धीराः᳚। त्वया॒ हि नः॑ पि॒तरः॑ सोम॒ पूर्वे॒ कर्मा॑णि च॒क्रुः प॑वमान॒ धीराः᳚। व॒न्वन्नवा॑तः परि॒धीꣳ रपो᳚र्णु वी॒रेभि॒रश्वै᳚र्म॒घवा॑ भव॥६५॥

%2.6.12.2
नः॒। त्वꣳ सो॑म पि॒तृभिः॑ संविदा॒नो\-ऽनु॒ द्यावा॑पृथि॒वी आ त॑तन्थ। तस्मै॑ त इन्दो ह॒विषा॑ विधेम व॒यꣴ स्या॑म॒ पत॑यो रयी॒णाम्। अग्नि॑ष्वात्ताः पितर॒ एह ग॑च्छत॒ सदः॑सदः सदत सुप्रणीतयः। अ॒त्ता ह॒वीꣳषि॒ प्रय॑तानि ब॒र्\mbox{}हिष्यथा॑ र॒यिꣳ सर्व॑वीरं दधातन। बर्\mbox{}हि॑षदः पितर ऊ॒त्य॑र्वागि॒मा वो॑ ह॒व्या च॑कृमा जु॒षध्वम्᳚। त आ ग॒ताव॑सा॒ शन्त॑मे॒नाथा॒स्मभ्यम्᳚॥६६॥

%2.6.12.3
शं योर॑र॒पो द॑धात। आहं पि॒त़ॄन्थ्सु॑वि॒दत्राꣳ॑ अविथ्सि॒ नपा॑तञ्च वि॒क्रम॑णं च॒ विष्णोः᳚। ब॒र्\mbox{}हि॒षदो॒ ये स्व॒धया॑ सु॒तस्य॒ भज॑न्त पि॒त्वस्त इ॒हाग॑मिष्ठाः। उप॑हूताः पि॒तरो॑ बर्\mbox{}हि॒ष्ये॑षु नि॒धिषु॑ प्रि॒येषु॑। त आग॑मन्तु॒ त इ॒ह श्रु॑व॒न्त्वधि॑ ब्रुवन्तु॒ ते अ॑वन्त्व॒स्मान्। उदी॑रता॒मव॑र॒ उत्परा॑स॒ उन्म॑ध्य॒माः पि॒तरः॑ सो॒म्यासः॑। असुम्᳚॥६७॥

%2.6.12.4
य ई॒युर॑वृ॒का ऋ॑त॒ज्ञास्ते नो॑\-ऽवन्तु पि॒तरो॒ हवे॑षु। इ॒दम्पि॒तृभ्यो॒ नमो॑ अस्त्व॒द्य ये पूर्वा॑सो॒ य उप॑रास ई॒युः। ये पार्थि॑वे॒ रज॒स्या निष॑त्ता॒ ये वा॑ नू॒नꣳ सु॑वृ॒जना॑सु वि॒क्षु। अधा॒ यथा॑ नः पि॒तरः॒ परा॑सः प्र॒त्नासो॑ अग्न ऋ॒तमा॑शुषा॒णाः। शुचीद॑य॒न्दीधि॑तिमुक्थ॒शासः॒ क्षामा॑ भि॒न्दन्तो॑ अरु॒णीरप॑ व्रन्न्। यद॑ग्ने॥६८॥

%2.6.12.5
क॒व्य॒वा॒ह॒न॒ पि॒तॄन् यक्ष्यृ॑ता॒वृधः॑। प्र च॑ ह॒व्यानि॑ वक्ष्यसि दे॒वेभ्य॑श्च पि॒तृभ्य॒ आ। त्वम॑ग्न ईडि॒तो जा॑तवे॒दो\-ऽवा᳚ड्ढ॒व्यानि॑ सुर॒भीणि॑ कृ॒त्वा। प्रादाः᳚ पि॒तृभ्यः॑ स्व॒धया॒ ते अ॑क्षन्न॒द्धि त्वं दे॑व॒ प्रय॑ता ह॒वीꣳषि॑। मात॑ली क॒व्यैर्य॒मो अङ्गि॑रोभि॒र्बृह॒स्पति॒र्\mbox{}ऋक्व॑भिर्वावृधा॒नः। याꣴश्च॑ दे॒वा वा॑वृ॒धुर्ये च॑ दे॒वान्थ्स्वाहा॒न्ये स्व॒धया॒न्ये म॑दन्ति।॥६९॥

%2.6.12.6
इ॒मं य॑म प्रस्त॒रमा हि सीदाङ्गि॑रोभिः पि॒तृभिः॑ संविदा॒नः। आ त्वा॒ मन्त्राः᳚ कविश॒स्ता व॑हन्त्वे॒ना रा॑जन् ह॒विषा॑ मादयस्व। अङ्गि॑रोभि॒रा ग॑हि य॒ज्ञिये॑भि॒र्यम॑ वैरू॒पैरि॒ह मा॑दयस्व। विव॑स्वन्तꣳ हुवे॒ यः पि॒ता ते॒\-ऽस्मिन् य॒ज्ञे ब॒र्\mbox{}हिष्या नि॒षद्य॑। अङ्गि॑रसो नः पि॒तरो॒ नव॑ग्वा॒ अथ॑र्वाणो॒ भृग॑वः सो॒म्यासः॑। तेषां᳚ व॒यꣳ सु॑म॒तौ य॒ज्ञिया॑ना॒मपि॑ भ॒द्रे सौ॑मन॒से स्या॑म॥७०॥

\prashnaend[{स॒मिधो॑ या॒ज्या॑ तस्मा॒न्नाभा॒गꣳ हि तमन्वित्या॑ह प्र॒जा वा आ॒हेत्या॑ह यु॒क्ष्वा हि स॑प्त॒तिः॥70॥ स॒मिधः॑ सौमन॒से स्या॑म॥}]
%%% END PRASHNA

%3.1.0.0
{\anuvakamend[{भ॒वा॒स्मभ्य॒मसुं॒ यद॑ग्ने मदन्ति सौमन॒स एक॑ञ्च}]}%॥12॥

%3.1.0.0

{\anuvakamend[{प्र॒जाप॑तिरकामयतै॒ष ते॑ य॒ज्ञं वै प्र॒जाप॑ते॒र्जाय॑मानाः प्राजाप॒त्या यो वा अय॑थादेवतमि॒ष्टर्गो॑ निग्रा॒भ्याः᳚ स्थ॒ यो वै दे॒वां जुष्टो॒\-ऽग्निना॑ र॒यिमेका॑दश}]}%॥11॥ प्र॒जाप॑तिरकामयत प्र॒जाप॑ते॒र्जाय॑माना॒ व्याय॑च्छन्ते॒ मह्य॑मि॒मान्मा॒या मा॒यिनां॒ द्विच॑त्वारिꣳशत्॥42॥ प्र॒जाप॑तिरकामयता॒ग्निꣳ स॑मु॒द्रवा॑ससम्॥
%%% END KANDAM

\chapt{काण्डम् ३}
\sect{प्रथमः प्रश्नः}\setcounter{anuvakam}{0}
\dnsub{तैत्तिरीयसंहितायां तृतीयकाण्डे प्रथमः प्रश्नः}
%3.1.1.0
%3.1.1.1
प्र॒जाप॑तिरकामयत प्र॒जाः सृ॑जे॒येति॒ स तपो॑\-ऽतप्यत॒ स स॒र्पान॑सृजत॒ सो॑\-ऽकामयत प्र॒जाः सृ॑जे॒येति॒ स द्वि॒तीय॑म\-तप्यत॒ स वयाꣳ॑स्यसृजत॒ सो॑\-ऽकामयत प्र॒जाः सृ॑जे॒येति॒ स तृ॒तीय॑मतप्यत॒ स ए॒तं दी᳚क्षितवा॒दम॑पश्य॒त्तम॑वद॒त्ततो॒ वै स प्र॒जा अ॑सृजत॒ यत्तप॑स्त॒प्त्वा दी᳚क्षितवा॒दं वद॑ति प्र॒जा ए॒व तद्यज॑मानः॥१॥

%3.1.1.2
सृ॒ज॒ते॒ यद्वै दी᳚क्षि॒तो॑\-ऽमे॒ध्यम्पश्य॒त्यपा᳚स्माद्दी॒क्षा क्रा॑मति॒ नील॑मस्य॒ हरो॒ व्ये᳚त्यब॑द्ध॒म्मनो॑ द॒रिद्रं॒ चक्षुः॒ सूर्यो॒ ज्योति॑षा॒ꣴ॒ श्रेष्ठो॒ दीक्षे॒ मा मा॑ हासी॒रित्या॑ह॒ नास्मा᳚द्दी॒क्षाप॑ क्रामति॒ नास्य॒ नीलं॒ न हरो॒ व्ये॑ति॒ यद्वै दी᳚क्षि॒तम॑भि॒वर्\mbox{}ष॑ति दि॒व्या आपो\-ऽशा᳚न्ता॒ ओजो॒ बलं॑ दी॒क्षाम्॥२॥

%3.1.1.3
तपो᳚\-ऽस्य॒ निर्घ्न॑न्त्युन्द॒तीर्बलं॑ ध॒त्तौजो॑ धत्त॒ बलं॑ धत्त॒ मा मे॑ दी॒क्षां मा तपो॒ निर्व॑धि॒ष्टेत्या॑है॒तदे॒व सर्व॑मा॒त्मन्ध॑त्ते॒ नास्यौजो॒ बलं॒ न दी॒क्षां न तपो॒ निर्घ्न॑न्त्य॒ग्निर्वै दी᳚क्षि॒तस्य॑ दे॒वता॒ सो᳚\-ऽस्मादे॒तर्\mbox{}हि॑ ति॒र इ॑व॒ यर्\mbox{}हि॒ याति॒ तमी᳚श्व॒रꣳ रक्षाꣳ॑सि॒ हन्तोः᳚॥३॥

%3.1.1.4
भ॒द्राद॒भि श्रेयः॒ प्रेहि॒ बृह॒स्पतिः॑ पुरए॒ता ते॑ अ॒स्त्वित्या॑ह॒ ब्रह्म॒ वै दे॒वाना॒म्बृह॒स्पति॒स्तमे॒वान्वार॑भते॒ स ए॑न॒ꣳ॒ सम्पा॑रय॒त्येदम॑गन्म देव॒यज॑नं पृथि॒व्या इत्या॑ह देव॒यज॑न॒ꣴ॒ ह्ये॑ष पृ॑थि॒व्या आ॒गच्छ॑ति॒ यो यज॑ते॒ विश्वे॑ दे॒वा यदजु॑षन्त॒ पूर्व॒ इत्या॑ह॒ विश्वे॒ ह्ये॑तद्दे॒वा जो॒षय॑न्ते॒ यद्ब्रा᳚ह्म॒णा ऋ॑ख्सा॒माभ्यां॒ यजु॑षा स॒न्तर॑न्त॒ इत्या॑हर्ख्सा॒माभ्या॒ꣳ॒ ह्ये॑ष यजु॑षा स॒न्तर॑ति॒ यो यज॑ते रा॒यस्पोषे॑ण॒ समि॒षा म॑दे॒मेत्या॑हा॒शिष॑मे॒वैतामा शा᳚स्ते॥४॥

%3.1.2.0
{\anuvakamend[{यज॑मानो दी॒क्षाꣳ हन्तो᳚र्ब्राह्म॒णाश्चतु॑र्विꣳशतिश्च}]}%॥१॥

%3.1.2.1
ए॒ष ते॑ गाय॒त्रो भा॒ग इति॑ मे॒ सोमा॑य ब्रूतादे॒ष ते॒ँतँरैष्टु॑भो॒ जाग॑तो भा॒ग इति॑ मे॒ सोमा॑य ब्रूताच्छन्दो॒माना॒ꣳ॒ साम्रा᳚ज्यं ग॒च्छेति॑ मे॒ सोमा॑य ब्रूता॒द्यो वै सोम॒ꣳ॒ राजा॑न॒ꣳ॒ साम्रा᳚ज्यं लो॒कं ग॑मयि॒त्वा क्री॒णाति॒ गच्छ॑ति॒ स्वाना॒ꣳ॒ साम्रा᳚ज्यं॒ छन्दाꣳ॑सि॒ खलु॒ वै सोम॑स्य॒ राज्ञः॒ साम्रा᳚ज्यो लो॒कः पु॒रस्ता॒थ्सोम॑स्य क्र॒यादे॒वम॒भि म॑न्त्रयेत॒ साम्रा᳚ज्यमे॒व॥५॥

%3.1.2.2
ए॒नं॒ लो॒कं ग॑मयि॒त्वा क्री॑णाति॒ गच्छ॑ति॒ स्वाना॒ꣳ॒ साम्रा᳚ज्यं॒ यो वै ता॑नून॒प्त्रस्य॑ प्रति॒ष्ठां वेद॒ प्रत्ये॒व ति॑ष्ठति ब्रह्मवा॒दिनो॑ वदन्ति॒ न प्रा॒श्नन्ति॒ न जु॑ह्व॒त्यथ॒ क्व॑ तानून॒प्त्रं प्रति॑ तिष्ठ॒तीति॑ प्र॒जाप॑तौ॒ मन॒सीति॑ ब्रूया॒त्त्रिरव॑ जिघ्रेत्प्र॒जाप॑तौ त्वा॒ मन॑सि जुहो॒मीत्ये॒षा वै ता॑नून॒प्त्रस्य॑ प्रति॒ष्ठा य ए॒वं वेद॒ प्रत्ये॒व ति॑ष्ठति॒ यः॥६॥

%3.1.2.3
वा अ॑ध्व॒र्योः प्र॑ति॒ष्ठां वेद॒ प्रत्ये॒व ति॑ष्ठति॒ यतो॒ मन्ये॒तान॑भिक्रम्य होष्या॒मीति॒ तत्तिष्ठ॒न्ना श्रा॑वयेदे॒षा वा अ॑ध्व॒र्योः प्र॑ति॒ष्ठा य ए॒वं वेद॒ प्रत्ये॒व ति॑ष्ठति॒ यद॑भि॒क्रम्य॑ जुहु॒यात्प्र॑ति॒ष्ठाया॑ इया॒त्तस्मा᳚थ्समा॒नत्र॒ तिष्ठ॑ता होत॒व्यं॑ प्रति॑ष्ठित्यै॒ यो वा अ॑ध्व॒र्योः स्वं वेद॒ स्ववा॑ने॒व भ॑वति॒ स्रुग्वा अ॑स्य॒ स्वं वा॑य॒व्य॑मस्य॥७॥

%3.1.2.4
स्वं च॑म॒सो᳚\-ऽस्य॒ स्वं यद्वा॑य॒व्यं॑ वा चम॒सं वा\-ऽन॑न्वारभ्याश्रा॒वये॒थ्स्वादि॑या॒त्तस्मा॑दन्वा॒रभ्या॒श्राव्य॒ꣴ॒ स्वादे॒व नैति॒ यो वै सोम॒मप्र॑तिष्ठाप्य स्तो॒त्रमु॑पाक॒रोत्यप्र॑तिष्ठितः॒ सोमो॒ भव॒त्यप्र॑तिष्ठितः॒ स्तोमो\-ऽप्र॑तिष्ठितान्यु॒क्थान्यप्र॑तिष्ठितो॒ यज॑मा॒नो\-ऽप्र॑तिष्ठितो\-ऽध्व॒र्युर्वा॑य॒व्यं॑ वै सोम॑स्य प्रति॒ष्ठा च॑म॒सो᳚\-ऽस्य प्रति॒ष्ठा सोमः॒ स्तोम॑स्य॒ स्तोम॑ उ॒क्थानां॒ ग्रहं॑ वा गृही॒त्वा च॑म॒सं वो॒न्नीय॑ स्तो॒त्रमु॒पाकु॑र्या॒त्प्रत्ये॒व सोमꣴ॑ स्था॒पय॑ति॒ प्रति॒ स्तोम॒म्प्रत्यु॒क्थानि॒ प्रति॒ यज॑मान॒स्तिष्ठ॑ति॒ प्रत्य॑ध्व॒र्युः॥८॥

%3.1.3.0
{\anuvakamend[{ए॒व ति॑ष्ठति॒ यो वा॑य॒व्य॑मस्य॒ ग्रहं॒ वैका॒न्नविꣳ॑श॒तिश्च॑}]}%॥२॥

%3.1.3.1
य॒ज्ञं वा ए॒तथ्सम्भ॑रन्ति॒ यथ्सो॑म॒क्रय॑ण्यै प॒दं य॑ज्ञमु॒खꣳ ह॑वि॒र्धाने॒ यर्\mbox{}हि॑ हवि॒र्धाने॒ प्राची᳚ प्रव॒र्तये॑यु॒स्तर्\mbox{}हि॒ तेनाक्ष॒मुपा᳚ञ्ज्याद्यज्ञमु॒ख ए॒व य॒ज्ञमनु॒ सं त॑नोति॒ प्राञ्च॑म॒ग्निम्प्र ह॑र॒न्त्युत्पत्नी॒मा न॑य॒न्त्यन्वनाꣳ॑सि॒ प्र व॑र्तय॒न्त्यथ॒ वा अ॑स्यै॒ष धिष्णि॑यो हीयते॒ सो\-ऽनु॑ ध्यायति॒ स ई᳚श्व॒रो रु॒द्रो भू॒त्वा॥९॥

%3.1.3.2
प्र॒जां प॒शून् यज॑मानस्य॒ शम॑यितो॒र्यर्\mbox{}हि॑ प॒शुमाप्री॑त॒मुद॑ञ्चं॒ नय॑न्ति॒ तर्\mbox{}हि॒ तस्य॑ पशु॒श्रप॑णꣳ हरे॒त्तेनै॒वैन॑म्भा॒गिनं॑ करोति॒ यज॑मानो॒ वा आ॑हव॒नीयो॒ यज॑मानं॒ वा ए॒तद्वि क॑र्\mbox{}षन्ते॒ यदा॑हव॒नीया᳚त्पशु॒श्रप॑ण॒ꣳ॒ हर॑न्ति॒ स वै॒व स्यान्नि॑र्म॒न्थ्यं॑ वा कुर्या॒द्यज॑मानस्य सात्म॒त्वाय॒ यदि॑ प॒शोर॑व॒दानं॒ नश्ये॒दाज्य॑स्य प्रत्या॒ख्याय॒मव॑ द्ये॒थ्सैव ततः॒ प्राय॑श्चित्ति॒र्ये प॒शुं वि॑मथ्नी॒रन् यस्तान्का॒मये॒तार्ति॒मार्च्छे॑यु॒रिति॑ कु॒विद॒ङ्गेति॒ नमो॑वृक्तिवत्य॒र्चाग्नी᳚ध्रे जुहुया॒न्नमो॑वृक्तिमे॒वैषां᳚ वृङ्क्ते ता॒जगार्ति॒मार्च्छ॑न्ति॥१०॥

%3.1.4.0
{\anuvakamend[{भू॒त्वा तत॒ष्षड्विꣳ॑शतिश्च}]}%॥३॥

%3.1.4.1
प्र॒जाप॑ते॒र्जाय॑मानाः प्र॒जा जा॒ताश्च॒ या इ॒माः। तस्मै॒ प्रति॒ प्र वे॑दय चिकि॒त्वाꣳ अनु॑ मन्यताम्। इ॒मम्प॒शुम्प॑शुपते ते अ॒द्य ब॒ध्नाम्य॑ग्ने सुकृ॒तस्य॒ मध्ये᳚। अनु॑ मन्यस्व सु॒यजा॑ यजाम॒ जुष्टं॑ दे॒वाना॑मि॒दम॑स्तु ह॒व्यम्। प्र॒जा॒नन्तः॒ प्रति॑ गृह्णन्ति॒ पूर्वे᳚ प्रा॒णमङ्गे᳚भ्यः॒ पर्या॒चर॑न्तम्। सु॒व॒र्गं या॑हि प॒थिभि॑र्देव॒यानै॒रोष॑धीषु॒ प्रति॑ तिष्ठा॒ शरी॑रैः। येषा॒मीशे᳚॥११॥

%3.1.4.2
प॒शु॒पतिः॑ पशू॒नां चतु॑ष्पदामु॒त च॑ द्वि॒पदा᳚म्। निष्क्री॑तो॒\-ऽयं य॒ज्ञिय॑म्भा॒गमे॑तु रा॒यस्पोषा॒ यज॑मानस्य सन्तु। ये ब॒ध्यमा॑न॒मनु॑ ब॒ध्यमा॑ना अ॒भ्यैक्ष॑न्त॒ मन॑सा॒ चक्षु॑षा च। अ॒ग्निस्ताꣳ अग्रे॒ प्र मु॑मोक्तु दे॒वः प्र॒जाप॑तिः प्र॒जया॑ संविदा॒नः। य आ॑र॒ण्याः प॒शवो॑ वि॒श्वरू॑पा॒ विरू॑पाः॒ सन्तो॑ बहु॒धैक॑रूपाः। वा॒युस्ताꣳ अग्रे॒ प्र मु॑मोक्तु दे॒वः प्र॒जाप॑तिः प्र॒जया॑ संविदा॒नः। प्र॒मु॒ञ्चमा॑नाः॥१२॥

%3.1.4.3
भुव॑नस्य॒ रेतो॑ गा॒तुं ध॑त्त॒ यज॑मानाय देवाः। उ॒पाकृ॑तꣳ शशमा॒नं यदस्था᳚ज्जी॒वं दे॒वाना॒मप्ये॑तु॒ पाथः॑। नाना᳚ प्रा॒णो यज॑मानस्य प॒शुना॑ य॒ज्ञो दे॒वेभिः॑ स॒ह दे॑व॒यानः॑। जी॒वं दे॒वाना॒मप्ये॑तु॒ पाथः॑ स॒त्याः स॑न्तु॒ यज॑मानस्य॒ कामाः᳚। यत्प॒शुर्मा॒युमकृ॒तोरो॑ वा प॒द्भिरा॑ह॒ते। अ॒ग्निर्मा॒ तस्मा॒देन॑सो॒ विश्वा᳚न्मुञ्च॒त्वꣳह॑सः। शमि॑तार उ॒पेत॑न य॒ज्ञम्॥१३॥

%3.1.4.4
दे॒वेभि॑रिन्वि॒तम्। पाशा᳚त्प॒शुम्प्र मु॑ञ्चत ब॒न्धाद्य॒ज्ञप॑तिं॒ परि॑। अदि॑तिः॒ पाश॒म्प्र मु॑मोक्त्वे॒तं नमः॑ प॒शुभ्यः॑ पशु॒पत॑ये करोमि। अ॒रा॒ती॒यन्त॒मध॑रं कृणोमि॒ यं द्वि॒ष्मस्तस्मि॒न्प्रति॑ मुञ्चामि॒ पाशम्᳚। त्वामु॒ ते द॑धिरे हव्य॒वाहꣳ॑ शृतङ्क॒र्तार॑मु॒त य॒ज्ञियं॑ च। अग्ने॒ सद॑क्षः॒ सत॑नु॒र्\mbox{}हि भू॒त्वा\-ऽथ॑ ह॒व्या जा॑तवेदो जुषस्व। जात॑वेदो व॒पया॑ गच्छ दे॒वान्त्वꣳ हि होता᳚ प्रथ॒मो ब॒भूथ॑। घृ॒तेन॒ त्वं त॒नुवो॑ वर्धयस्व॒ स्वाहा॑कृतꣳ ह॒विर॑दन्तु दे॒वाः। स्वाहा॑ दे॒वेभ्यो॑ दे॒वेभ्यः॒ स्वाहा᳚॥१४॥

%3.1.5.0
{\anuvakamend[{ईशे᳚ प्रमु॒ञ्चमा॑ना य॒ज्ञन्त्वꣳ षोड॑श च}]}%॥४॥

%3.1.5.1
प्रा॒जा॒प॒त्या वै प॒शव॒स्तेषाꣳ॑ रु॒द्रो\-ऽधि॑पति॒र्यदे॒ताभ्या॑मुपाक॒रोति॒ ताभ्या॑मे॒वैनं॑ प्रति॒प्रोच्या ल॑भत आ॒त्मनो\-ऽना᳚व्रस्काय॒ द्वाभ्या॑मु॒पाक॑रोति द्वि॒पाद्यज॑मानः॒ प्रति॑ष्ठित्या उपा॒कृत्य॒ पञ्च॑ जुहोति॒ पाङ्क्ताः᳚ प॒शवः॑ प॒शूने॒वाव॑ रुन्द्धे मृ॒त्यवे॒ वा ए॒ष नी॑यते॒ यत्प॒शुस्तं यद॑न्वा॒रभे॑त प्र॒मायु॑को॒ यज॑मानः स्या॒न्नाना᳚ प्रा॒णो यज॑मानस्य प॒शुनेत्या॑ह॒ व्यावृ॑त्त्यै॥१५॥

%3.1.5.2
यत्प॒शुर्मा॒युमकृ॒तेति॑ जुहोति॒ शान्त्यै॒ शमि॑तार उ॒पेत॒नेत्या॑ह यथाय॒जुरे॒वैतद्व॒पायां॒ वा आ᳚ह्रि॒यमा॑णायाम॒ग्नेर्मेधो\-ऽप॑ क्रामति॒ त्वामु॒ ते द॑धिरे हव्य॒वाह॒मिति॑ व॒पाम॒भि जु॑होत्य॒ग्नेरे॒व मेध॒मव॑ रु॒न्द्धे\-ऽथो॑ शृत॒त्वाय॑ पु॒रस्ता᳚थ्स्वाहाकृतयो॒ वा अ॒न्ये दे॒वा उ॒परि॑ष्टाथ्स्वाहाकृतयो॒\-ऽन्ये स्वाहा॑ दे॒वेभ्यो॑ दे॒वेभ्यः॒ स्वाहेत्य॒भितो॑ व॒पां जु॑होति॒ ताने॒वोभया᳚न्प्रीणाति॥१६॥

%3.1.6.0
{\anuvakamend[{व्यावृ॑त्त्या अ॒भितो॑ व॒पां पञ्च॑ च}]}%॥५॥

%3.1.6.1
यो वा अय॑थादेवतं य॒ज्ञमु॑प॒चर॒त्या दे॒वता᳚भ्यो वृश्च्यते॒ पापी॑यान्भवति॒ यो य॑थादेव॒तं न दे॒वता᳚भ्य॒ आ वृ॑श्च्यते॒ वसी॑यान्भवत्याग्ने॒य्यर्चाग्नी᳚ध्रम॒भि मृ॑शेद्वैष्ण॒व्या ह॑वि॒र्धान॑माग्ने॒य्या स्रुचो॑ वाय॒व्य॑या वाय॒व्या᳚न्यैन्द्रि॒या सदो॑ यथादेव॒तमे॒व य॒ज्ञमुप॑ चरति॒ न दे॒वता᳚भ्य॒ आ वृ॑श्च्यते॒ वसी॑यान्भवति यु॒नज्मि॑ ते पृथि॒वीं ज्योति॑षा स॒ह यु॒नज्मि॑ वा॒युम॒न्तरि॑क्षेण॥१७॥

%3.1.6.2
ते॒ स॒ह यु॒नज्मि॒ वाचꣳ॑ स॒ह सूर्ये॑ण ते यु॒नज्मि॑ ति॒स्रो वि॒पृचः॒ सूर्य॑स्य ते। अ॒ग्निर्दे॒वता॑ गाय॒त्री छन्द॑ उपा॒ꣳ॒शोः पात्र॑मसि॒ सोमो॑ दे॒वता᳚ त्रि॒ष्टुप्छन्दो᳚\-ऽन्तर्या॒मस्य॒ पात्र॑म॒सीन्द्रो॑ दे॒वता॒ जग॑ती॒ छन्द॑ इन्द्रवायु॒वोः पात्र॑मसि॒ बृह॒स्पति॑र्दे॒वता॑\-ऽनु॒ष्टुप्छन्दो॑ मि॒त्रावरु॑णयोः॒ पात्र॑मस्य॒श्विनौ॑ दे॒वता॑ प॒ङ्क्तिश्छन्दो॒\-ऽश्विनोः॒ पात्र॑मसि॒ सूर्यो॑ दे॒वता॑ बृह॒ती॥१८॥

%3.1.6.3
छन्दः॑ शु॒क्रस्य॒ पात्र॑मसि च॒न्द्रमा॑ दे॒वता॑ स॒तोबृ॑हती॒ छन्दो॑ म॒न्थिनः॒ पात्र॑मसि॒ विश्वे॑ दे॒वा दे॒वतो॒ष्णिहा॒ छन्द॑ आग्रय॒णस्य॒ पात्र॑म॒सीन्द्रो॑ दे॒वता॑ क॒कुच्छन्द॑ उ॒क्थाना॒म्पात्र॑मसि पृथि॒वी दे॒वता॑ वि॒राट्छन्दो᳚ ध्रु॒वस्य॒ पात्र॑मसि॥१९॥

%3.1.7.0
{\anuvakamend[{अ॒न्तरि॑क्षेण बृह॒ती त्रय॑स्त्रिꣳशच्च}]}%॥६॥

%3.1.7.1
इ॒ष्टर्गो॒ वा अ॑ध्व॒र्युर्यज॑मानस्ये॒ष्टर्गः॒ खलु॒ वै पूर्वो॒\-ऽर्ष्टुः क्षी॑यत आस॒न्या᳚न्मा॒ मन्त्रा᳚त्पाहि॒ कस्या᳚श्चिद॒भिश॑स्त्या॒ इति॑ पु॒रा प्रा॑तरनुवा॒काज्जु॑हुयादा॒त्मन॑ एव॒ तद॑ध्व॒र्युः पु॒रस्ता॒च्छर्म॑ नह्य॒ते\-ऽना᳚र्त्यै संवे॒शाय॑ त्वोपवे॒शाय॑ त्वा गायत्रि॒यास्त्रि॒ष्टुभो॒ जग॑त्या अ॒भिभू᳚त्यै॒ स्वाहा॒ प्राणा॑पानौ मृ॒त्योर्मा॑ पातं॒ प्राणा॑पानौ॒ मा मा॑ हासिष्टं दे॒वता॑सु॒ वा ए॒ते प्रा॑णापा॒नयोः᳚॥२०॥

%3.1.7.2
व्याय॑च्छन्ते॒ येषा॒ꣳ॒ सोमः॑ समृ॒च्छते॑ संवे॒शाय॑ त्वोपवे॒शाय॒ त्वेत्या॑ह॒ छन्दाꣳ॑सि॒ वै सं॑वे॒श उ॑पवे॒शश्छन्दो॑भिरे॒वास्य॒ छन्दाꣳ॑सि वृङ्क्ते॒ प्रेति॑व॒न्त्याज्या॑नि भवन्त्य॒भिजि॑त्यै म॒रुत्व॑तीः प्रति॒पदो॒ विजि॑त्या उ॒भे बृ॑हद्रथन्त॒रे भ॑वत इ॒यं वाव र॑थन्त॒रम॒सौ बृ॒हदा॒भ्यामे॒वैन॑म॒न्तरे᳚त्य॒द्य वाव र॑थन्त॒रꣴ श्वो बृ॒हद॑द्या॒श्वादे॒वैन॑म॒न्तरे॑ति भू॒तम्॥२१॥

%3.1.7.3
वाव र॑थन्त॒रम्भ॑वि॒ष्यद्बृ॒हद्भू॒ताच्चै॒वैन॑म्भविष्य॒तश्चा॒न्तरे॑ति॒ परि॑मितं॒ वाव र॑थन्त॒रमप॑रिमितम्बृ॒हत्परि॑मिताच्चै॒वैन॒मप॑रि\-मिताच्चा॒न्तरे॑ति विश्वामित्रजमद॒ग्नी वसि॑ष्ठेनास्पर्धेता॒ꣳ॒ स ए॒तज्ज॒मद॑ग्निर्विह॒व्य॑मपश्य॒त्तेन॒ वै स वसि॑ष्ठस्येन्द्रि॒यं वी॒र्य॑मवृङ्क्त॒ यद्वि॑ह॒व्यꣳ॑ श॒स्यत॑ इन्द्रि॒यमे॒व तद्वी॒र्यं॑ यज॑मानो॒ भ्रातृ॑व्यस्य वृङ्क्ते॒ यस्य॒ भूयाꣳ॑सो यज्ञक्र॒तव॒ इत्या॑हुः॒ स दे॒वता॑ वृङ्क्त॒ इति॒ यद्य॑ग्निष्टो॒मः सोमः॑ प॒रस्ता॒थ्स्यादु॒क्थ्यं॑ कुर्वीत॒ यद्यु॒क्थ्यः॑ स्याद॑तिरा॒त्रं कु॑र्वीत यज्ञक्र॒तुभि॑रे॒वास्य॑ दे॒वता॑ वृङ्क्ते॒ वसी॑यान्भवति॥२२॥

%3.1.8.0
{\anuvakamend[{प्रा॒णा॒पा॒नयो᳚र्भू॒तं वृ॑ङ्क्ते॒\-ऽष्टाविꣳ॑शतिश्च}]}%॥७॥

%3.1.8.1
नि॒ग्रा॒भ्याः᳚ स्थ देव॒श्रुत॒ आयु॑र्मे तर्पयत प्रा॒णं मे॑ तर्पयतापा॒नं मे॑ तर्पयत व्या॒नं मे॑ तर्पयत॒ चक्षु॑र्मे तर्पयत॒ श्रोत्रं॑ मे तर्पयत॒ मनो॑ मे तर्पयत॒ वाचं॑ मे तर्पयता॒त्मानं॑ मे तर्पय॒ताङ्गा॑नि मे तर्पयत प्र॒जां मे॑ तर्पयत प॒शून्मे॑ तर्पयत गृ॒हान्मे॑ तर्पयत ग॒णान्मे॑ तर्पयत स॒र्वग॑णं मा तर्पयत त॒र्पय॑त मा॥२३॥

%3.1.8.2
ग॒णा मे॒ मा वि तृ॑ष॒न्नोष॑धयो॒ वै सोम॑स्य॒ विशो॒ विशः॒ खलु॒ वै राज्ञः॒ प्रदा॑तोरीश्व॒रा ऐ॒न्द्रः सोमो\-ऽवी॑वृधं वो॒ मन॑सा सुजाता॒ ऋत॑प्रजाता॒ भग॒ इद्वः॑ स्याम। इन्द्रे॑ण दे॒वीर्वी॒रुधः॑ संविदा॒ना अनु॑ मन्यन्ता॒ꣳ॒ सव॑नाय॒ सोम॒मित्या॒हौष॑धीभ्य ए॒वैन॒ꣴ॒ स्वायै॑ वि॒शः स्वायै॑ दे॒वता॑यै नि॒र्याच्या॒भि षु॑णोति॒ यो वै सोम॑स्याभिषू॒यमा॑णस्य॥२४॥

%3.1.8.3
प्र॒थ॒मो\-ऽꣳ॑शुः स्कन्द॑ति॒ स ई᳚श्व॒र इ॑न्द्रि॒यं वी॒र्यं॑ प्र॒जां प॒शून् यज॑मानस्य॒ निर्\mbox{}ह॑न्तो॒स्तम॒भि म॑न्त्रये॒ता मा᳚स्कान्थ्स॒ह प्र॒जया॑ स॒ह रा॒यस्पोषे॑णेन्द्रि॒यं मे॑ वी॒र्यं॑ मा निर्व॑धी॒रित्या॒शिष॑मे॒वैतामा शा᳚स्त इन्द्रि॒यस्य॑ वी॒र्य॑स्य प्र॒जायै॑ पशू॒नामनि॑र्घाताय द्र॒फ्सश्च॑स्कन्द पृथि॒वीमनु॒ द्यामि॒मं च॒ योनि॒मनु॒ यश्च॒ पूर्वः॑। तृ॒तीयं॒ योनि॒मनु॑ सं॒चर॑न्तं द्र॒फ्सं जु॑हो॒म्यनु॑ स॒प्त होत्राः᳚॥२५॥

%3.1.9.0
{\anuvakamend[{त॒र्पय॑त मा\-ऽभिषू॒यमा॑णस्य॒ यश्च॒ दश॑ च}]}%॥८॥

%3.1.9.1
यो वै दे॒वान्दे॑वयश॒सेना॒र्पय॑ति मनु॒ष्या᳚न्मनुष्ययश॒सेन॑ देवयश॒स्ये॑व दे॒वेषु॒ भव॑ति मनुष्ययश॒सी म॑नु॒ष्ये॑षु॒ यान्प्रा॒चीन॑माग्रय॒णाद्ग्रहा᳚न्गृह्णी॒यात्तानु॑पा॒ꣳ॒शु गृ॑ह्णीया॒द्यानू॒र्ध्वाꣴस्तानु॑पब्दि॒मतो॑ दे॒वाने॒व तद्दे॑वयश॒सेना᳚र्पयति मनु॒ष्या᳚न्मनुष्ययश॒सेन॑ देवयश॒स्ये॑व दे॒वेषु॑ भवति मनुष्ययश॒सी म॑नु॒ष्ये᳚ष्व॒ग्निः प्रा॑तःसव॒ने पा᳚त्व॒स्मान् वै᳚श्वान॒रो म॑हि॒ना वि॒श्वश॑म्भूः। स नः॑ पाव॒को द्रवि॑णं दधातु॥२६॥

%3.1.9.2
आयु॑ष्मन्तः स॒हभ॑क्षाः स्याम। विश्वे॑ दे॒वा म॒रुत॒ इन्द्रो॑ अ॒स्मान॒स्मिन्द्वि॒तीये॒ सव॑ने॒ न ज॑ह्युः। आयु॑ष्मन्तः प्रि॒यमे॑षां॒ वद॑न्तो व॒यं दे॒वानाꣳ॑ सुम॒तौ स्या॑म। इ॒दं तृ॒तीय॒ꣳ॒ सव॑नं कवी॒नामृ॒तेन॒ ये च॑म॒समैर॑यन्त। ते सौ॑धन्व॒नाः सुव॑रानशा॒नाः स्वि॑ष्टिं नो अ॒भि वसी॑यो नयन्तु। आ॒यत॑नवती॒र्वा अ॒न्या आहु॑तयो हू॒यन्ते॑\-ऽनायत॒ना अ॒न्या या आ॑घा॒रव॑ती॒स्ता आ॒यत॑नवती॒र्याः॥२७॥

%3.1.9.3
सौ॒म्यास्ता अ॑नायत॒ना ऐ᳚न्द्रवाय॒वमा॒दाया॑घा॒रमा घा॑रयेदध्व॒रो य॒ज्ञो॑\-ऽयम॑स्तु देवा॒ ओष॑धीभ्यः प॒शवे॑ नो॒ जना॑य॒ विश्व॑स्मै भू॒ताया᳚ध्व॒रो॑\-ऽसि॒ स पि॑न्वस्व घृत॒व॑द्देव सो॒मेति॑ सौ॒म्या ए॒व तदाहु॑तीरा॒यत॑नवतीः करोत्या॒यत॑नवान्भवति॒ य ए॒वं वेदाथो॒ द्यावा॑पृथि॒वी ए॒व घृ॒तेन॒ व्यु॑नत्ति॒ ते व्यु॑त्ते उपजीव॒नीये॑ भवत उपजीव॒नीयो॑ भवति॥२८॥

%3.1.9.4
य ए॒वं वेदै॒ष ते॑ रुद्र भा॒गो यं नि॒रया॑चथा॒स्तं जु॑षस्व वि॒देर्गौ॑प॒त्यꣳ रा॒यस्पोषꣳ॑ सु॒वीर्यꣳ॑ संवथ्स॒रीणाꣴ॑ स्व॒स्तिम्। मनुः॑ पु॒त्रेभ्यो॑ दा॒यं व्य॑भज॒थ्स नाभा॒नेदि॑ष्ठं ब्रह्म॒चर्यं॒ वस॑न्तं॒ निर॑भज॒थ्स आग॑च्छ॒थ्सो᳚\-ऽब्रवीत्क॒था मा॒ निर॑भा॒गिति॒ न त्वा॒ निर॑भाक्ष॒मित्य॑ब्रवी॒दङ्गि॑रस इ॒मे स॒त्तमा॑सते॒ ते॥२९॥

%3.1.9.5
सु॒व॒र्गं लो॒कं न प्र जा॑नन्ति॒ तेभ्य॑ इ॒दम्ब्राह्म॑णम्ब्रूहि॒ ते सु॑व॒र्गं लो॒कं यन्तो॒ य ए॑षाम्प॒शव॒स्ताꣴस्ते॑ दास्य॒न्तीति॒ तदे᳚भ्यो\-ऽब्रवी॒त्ते सु॑व॒र्गं लो॒कं यन्तो॒ य ए॑षाम्प॒शव॒ आस॒न्तान॑स्मा अददु॒स्तम्प॒शुभि॒श्चर॑न्तं यज्ञवा॒स्तौ रु॒द्र आग॑च्छ॒थ्सो᳚\-ऽब्रवी॒न्मम॒ वा इ॒मे प॒शव॒ इत्यदु॒र्वै॥३०॥

%3.1.9.6
मह्य॑मि॒मानित्य॑ब्रवी॒न्न वै तस्य॒ त ई॑शत॒ इत्य॑ब्रवी॒द्यद्य॑ज्ञवा॒स्तौ हीय॑ते॒ मम॒ वै तदिति॒ तस्मा᳚द्यज्ञवा॒स्तु नाभ्य॒वेत्य॒ꣳ॒ सो᳚\-ऽब्रवीद्य॒ज्ञे मा भ॒जाथ॑ ते प॒शून्नाभि मꣴ॑स्य॒ इति॒ तस्मा॑ ए॒तम्म॒न्थिनः॑ सꣴस्रा॒वम॑जुहो॒त्ततो॒ वै तस्य॑ रु॒द्रः प॒शून्नाभ्य॑मन्यत॒ यत्रै॒तमे॒वं वि॒द्वान्म॒न्थिनः॑ सꣴस्रा॒वं जु॒होति॒ न तत्र॑ रु॒द्रः प॒शून॒भि म॑न्यते॥३१॥

%3.1.10.0
{\anuvakamend[{द॒धा॒त्वा॒यत॑नवती॒र्या उ॑पजीव॒नीयो॑ भवति॒ ते\-ऽदु॒र्वै यत्रै॒तमेका॑दश च}]}%॥९॥

%3.1.10.1
जुष्टो॑ वा॒चो भू॑यासं॒ जुष्टो॑ वा॒चस्पत॑ये॒ देवि॑ वाक्। यद्वा॒चो मधु॑म॒त्तस्मि॑न्मा धाः॒ स्वाहा॒ सर॑स्वत्यै। ऋ॒चा स्तोम॒ꣳ॒ सम॑र्धय गाय॒त्रेण॑ रथन्त॒रम्। बृ॒हद्गा॑य॒त्रव॑र्तनि। यस्ते᳚ द्र॒फ्सः स्कन्द॑ति॒ यस्ते॑ अ॒ꣳ॒शुर्बा॒हुच्यु॑तो धि॒षण॑योरु॒पस्था᳚त्। अ॒ध्व॒र्योर्वा॒ परि॒ यस्ते॑ प॒वित्रा॒थ्स्वाहा॑कृत॒मिन्द्रा॑य॒ तं जु॑होमि। यो द्र॒फ्सो अ॒ꣳ॒शुः प॑ति॒तः पृ॑थि॒व्यां प॑रिवा॒पात्॥३२॥

%3.1.10.2
पु॒रो॒डाशा᳚त्कर॒म्भात्। धा॒ना॒सो॒मान्म॒न्थिन॑ इन्द्र शु॒क्राथ्स्वाहा॑कृत॒मिन्द्रा॑य॒ तं जु॑होमि। यस्ते᳚ द्र॒फ्सो मधु॑माꣳ इन्द्रि॒यावा॒न्थ्स्वाहा॑कृतः॒ पुन॑र॒प्येति॑ दे॒वान्। दि॒वः पृ॑थि॒व्याः पर्य॒न्तरि॑क्षा॒थ्स्वाहा॑कृत॒मिन्द्रा॑य॒ तं जु॑होमि। अ॒ध्व॒र्युर्वा ऋ॒त्विजां᳚ प्रथ॒मो यु॑ज्यते॒ तेन॒ स्तोमो॑ योक्त॒व्य॑ इत्या॑हु॒र्वाग॑ग्रे॒गा अग्र॑ एत्वृजु॒गा दे॒वेभ्यो॒ यशो॒ मयि॒ दध॑ती प्रा॒णान्प॒शुषु॑ प्र॒जाम्मयि॑॥३३॥

%3.1.10.3
च॒ यज॑माने॒ चेत्या॑ह॒ वाच॑मे॒व तद्य॑ज्ञमु॒खे यु॑नक्ति॒ वास्तु॒ वा ए॒तद्य॒ज्ञस्य॑ क्रियते॒ यद्ग्र॒हा᳚न्गृही॒त्वा ब॑हिष्पवमा॒नꣳ सर्प॑न्ति॒ परा᳚ञ्चो॒ हि यन्ति॒ परा॑चीभिः स्तु॒वते॑ वैष्ण॒व्यर्चा पुन॒रेत्योप॑ तिष्ठते य॒ज्ञो वै विष्णु॑र्य॒ज्ञमे॒वाक॒र्विष्णो॒ त्वं नो॒ अन्त॑मः॒ शर्म॑ यच्छ सहन्त्य। प्र ते॒ धारा॑ मधु॒श्चुत॒ उथ्सं॑ दुह्रते॒ अक्षि॑त॒मित्या॑ह॒ यदे॒वास्य॒ शया॑नस्योप॒शुष्य॑ति॒ तदे॒वास्यै॒तेना प्या॑ययति॥३४॥

%3.1.11.0
{\anuvakamend[{प॒रि॒वा॒पात्प्र॒जां मयि॑ दुह्रते॒ चतु॑र्दश च}]}%॥10॥

%3.1.11.1
अ॒ग्निना॑ र॒यिम॑श्नव॒त्पोष॑मे॒व दि॒वेदि॑वे। य॒शसं॑ वी॒रव॑त्तमम्॥ गोमाꣳ॑ अ॒ग्ने\-ऽवि॑माꣳ अ॒श्वी य॒ज्ञो नृ॒वथ्स॑खा॒ सद॒मिद॑प्रमृ॒ष्यः। इडा॑वाꣳ ए॒षो अ॑सुर प्र॒जावा᳚न्दी॒र्घो र॒यिः पृ॑थुबु॒ध्नः स॒भावान्॑॥ आ प्या॑यस्व॒ सं ते᳚॥ इ॒ह त्वष्टा॑रमग्रि॒यं वि॒श्वरू॑प॒मुप॑ ह्वये। अ॒स्माक॑मस्तु॒ केव॑लः॥ तन्न॑स्तु॒रीप॒मध॑ पोषयि॒त्नु देव॑ त्वष्ट॒र्वि र॑रा॒णः स्य॑स्व। यतो॑ वी॒रः॥३५॥

%3.1.11.2
क॒र्म॒ण्यः॑ सु॒दक्षो॑ यु॒क्तग्रा॑वा॒ जाय॑ते दे॒वका॑मः। शि॒वस्त्व॑ष्टरि॒हा ग॑हि वि॒भुः पोष॑ उ॒त त्मना᳚। य॒ज्ञेय॑ज्ञे न॒ उद॑व। पि॒शङ्ग॑रूपः सु॒भरो॑ वयो॒धाः श्रु॒ष्टी वी॒रो जा॑यते दे॒वका॑मः। प्र॒जां त्वष्टा॒ वि ष्य॑तु॒ नाभि॑म॒स्मे अथा॑ दे॒वाना॒मप्ये॑तु॒ पाथः॑। प्र णो॑ दे॒व्या नो॑ दि॒वः। पी॒पि॒वाꣳस॒ꣳ॒ सर॑स्वतः॒ स्तनं॒ यो वि॒श्वद॑र्शतः। धु॒क्षी॒महि॑ प्र॒जामिषम्᳚॥३६॥

%3.1.11.3
ये ते॑ सरस्व ऊ॒र्मयो॒ मधु॑मन्तो घृत॒श्चुतः॑। तेषां᳚ ते सु॒म्नमी॑महे। यस्य॑ व्र॒तम्प॒शवो॒ यन्ति॒ सर्वे॒ यस्य॑ व्र॒तमु॑प॒तिष्ठ॑न्त॒ आपः॒। यस्य॑ व्र॒ते पु॑ष्टि॒पति॒र्निवि॑ष्ट॒स्तꣳ सर॑स्वन्त॒मव॑से हुवेम। दि॒व्यꣳ सु॑प॒र्णं व॑य॒सम्बृ॒हन्त॑म॒पां गर्भं॑ वृष॒भमोष॑धीनाम्। अ॒भी॒प॒तो वृ॒ष्ट्या त॒र्पय॑न्तं॒ तꣳ सर॑स्वन्त॒मव॑से हुवेम। सिनी॑वालि॒ पृथु॑ष्टुके॒ या दे॒वाना॒मसि॒ स्वसा᳚। जु॒षस्व॑ ह॒व्यम्॥३७॥

%3.1.11.4
आहु॑तं प्र॒जां दे॑वि दिदिड्ढि नः। या सु॑पा॒णिः स्व॑ङ्गु॒रिः सु॒षूमा॑ बहु॒सूव॑री। तस्यै॑ वि॒श्पत्नि॑यै ह॒विः सि॑नीवा॒ल्यै जु॑होतन। इन्द्रं॑ वो वि॒श्वत॒स्परीन्द्रं॒ नरः॑। असि॑तवर्णा॒ हर॑यः सुप॒र्णा मिहो॒ वसा॑ना॒ दिव॒मुत्प॑तन्ति। त आ\-ऽव॑वृत्र॒न्थ्सद॑नानि कृ॒त्वादित्पृ॑थि॒वी घृ॒तैर्व्यु॑द्यते। हिर॑ण्यकेशो॒ रज॑सो विसा॒रे\-ऽहि॒र्धुनि॒र्वात॑ इव॒ ध्रजी॑मान्। शुचि॑भ्राजा उ॒षसः॑॥३८॥

%3.1.11.5
नवे॑दा॒ यश॑स्वतीरप॒स्युवो॒ न स॒त्याः। आ ते॑ सुप॒र्णा अ॑मिनन्त॒ एवैः᳚ कृ॒ष्णो नो॑नाव वृष॒भो यदी॒दम्। शि॒वाभि॒र्न स्मय॑मानाभि॒रागा॒त्पत॑न्ति॒ मिहः॑ स्त॒नय॑न्त्य॒भ्रा। वा॒श्रेव॑ वि॒द्युन्मि॑माति व॒थ्सं न मा॒ता सि॑षक्ति। यदे॑षां वृ॒ष्टिरस॑र्जि। पर्व॑तश्चि॒न्महि॑ वृ॒द्धो बि॑भाय दि॒वश्चि॒थ्सानु॑ रेजत स्व॒ने वः॑। यत्क्रीड॑थ मरुतः॥३९॥

%3.1.11.6
ऋ॒ष्टि॒मन्त॒ आप॑ इव स॒ध्रिय॑ञ्चो धवध्वे। अ॒भि क्र॑न्द स्त॒नय॒ गर्भ॒मा धा॑ उद॒न्वता॒ परि॑ दीया॒ रथे॑न। दृति॒ꣳ॒ सु क॑र्\mbox{}ष॒ विषि॑तं॒ न्य॑ञ्चꣳ स॒मा भ॑वन्तू॒द्वता॑ निपा॒दाः। त्वं त्या चि॒दच्यु॒ताग्ने॑ प॒शुर्न यव॑से। धामा॑ ह॒ यत्ते॑ अजर॒ वना॑ वृ॒श्चन्ति॒ शिक्व॑सः। अग्ने॒ भूरी॑णि॒ तव॑ जातवेदो॒ देव॑ स्वधावो॒\-ऽमृत॑स्य॒ धाम॑। याश्च॑॥४०॥

%3.1.11.7
मा॒या मा॒यिनां᳚ विश्वमिन्व॒ त्वे पू॒र्वीः सं॑द॒धुः पृ॑ष्टबन्धो। दि॒वो नो॑ वृ॒ष्टिम्म॑रुतो ररीध्व॒म्प्र पि॑न्वत॒ वृष्णो॒ अश्व॑स्य॒ धाराः᳚। अ॒र्वाङे॒तेन॑ स्तनयि॒त्नुतेह्य॒पो नि॑षि॒ञ्चन्नसु॑रः पि॒ता नः॑। पिन्व॑न्त्य॒पो म॒रुतः॑ सु॒दान॑वः॒ पयो॑ घृ॒तव॑द्वि॒दथे᳚ष्वा॒भुवः॑। अत्यं॒ न मि॒हे वि न॑यन्ति वा॒जिन॒मुथ्सं॑ दुहन्ति स्त॒नय॑न्त॒मक्षि॑तम्। उ॒द॒प्रुतो॑ मरुत॒स्ताꣳ इ॑यर्त॒ वृष्टिम्᳚॥४१॥

%3.1.11.8
ये विश्वे॑ म॒रुतो॑ जु॒नन्ति॑। क्रोशा॑ति॒ गर्दा॑ क॒न्ये॑व तु॒न्ना पेरुं॑ तुञ्जा॒ना पत्ये॑व जा॒या। घृ॒तेन॒ द्यावा॑पृथि॒वी मधु॑ना॒ समु॑क्षत॒ पय॑स्वतीः कृणु॒ताप॒ ओष॑धीः। ऊर्जं॑ च॒ तत्र॑ सुम॒तिं च॑ पिन्वथ॒ यत्रा॑ नरो मरुतः सि॒ञ्चथा॒ मधु॑। उदु॒ त्यञ्चि॒त्रम्। औ॒र्व॒भृ॒गु॒वच्छुचि॑मप्नवान॒वदा हु॑वे। अ॒ग्निꣳ स॑मु॒द्रवा॑ससम्। आ स॒वꣳ स॑वि॒तुर्य॑था॒ भग॑स्येव भु॒जिꣳ हु॑वे। अ॒ग्निꣳ स॑मु॒द्रवा॑ससम्। हु॒वे वात॑स्वनं क॒विम्प॒र्जन्य॑क्रन्द्य॒ꣳ॒ सहः॑। अ॒ग्निꣳ स॑मु॒द्रवा॑ससम्॥४२॥

%3.2.0.0
{\anuvakamend[{वी॒र इषꣳ॑ ह॒व्यमु॒षसो॑ मरुतश्च॒ वृष्टिं॒ भग॑स्य॒ द्वाद॑श च}]}%॥11॥

%3.2.0.0

{\anuvakamend[{यो वै पव॑मानाना॒न्त्रीणि॑ परि॒भूः स्फ्यः स्व॒स्तिर्भक्षेहि॑ मही॒नां पयो॑\-ऽसि॒ देव॑ सवितरे॒तत्ते᳚ श्ये॒नाय॒ यद्वै होतो॑पयाम॒गृ॑हीतो\-ऽसि वाक्ष॒सत्प्र सो अ॑ग्न॒ एका॑दश}]}%॥11॥ 

\prashnaend[{यो वै स्फ्यः स्व॒स्तिः स्व॒धायै॒ नमः॒ प्र मु़॑ञ्च॒ तिष्ठ॑तीव॒ षट्च॑त्वारिꣳशत्॥46॥ यो वै पव॑मानानां॒ वि क्र॑मस्व॥}]
%%% END PRASHNA

\sect{द्वितीयः प्रश्नः}\setcounter{anuvakam}{0}
\dnsub{तैत्तिरीयसंहितायां तृतीयकाण्डे द्वितीयः प्रश्नः}
%3.2.1.0
%3.2.1.1
यो वै पव॑मानानामन्वारो॒हान् वि॒द्वान् यज॒ते\-ऽनु॒ पव॑माना॒ना रो॑हति॒ न पव॑माने॒भ्यो\-ऽव॑च्छिद्यते श्ये॒नो॑\-ऽसि गाय॒त्रछ॑न्दा॒ अनु॒ त्वा र॑भे स्व॒स्ति मा॒ सम्पा॑रय सुप॒र्णो॑\-ऽसि त्रि॒ष्टुप्छ॑न्दा॒ अनु॒ त्वा र॑भे स्व॒स्ति मा॒ सम्पा॑रय॒ सघा॑सि॒ जग॑तीछन्दा॒ अनु॒ त्वा र॑भे स्व॒स्ति मा॒ सम्पा॑र॒येत्या॑है॒ते॥१॥

%3.2.1.2
वै पव॑मानानामन्वारो॒हास्तान् य ए॒वं वि॒द्वान् यज॒ते\-ऽनु॒ पव॑माना॒ना रो॑हति॒ न पव॑माने॒भ्यो\-ऽव॑च्छिद्यते॒ यो वै पव॑मानस्य॒ सन्त॑तिं॒ वेद॒ सर्व॒मायु॑रेति॒ न पु॒रायु॑षः॒ प्र मी॑यते पशु॒मान्भ॑वति वि॒न्दते᳚ प्र॒जाम्पव॑मानस्य॒ ग्रहा॑ गृह्य॒न्ते\-ऽथ॒ वा अ॑स्यै॒ते\-ऽगृ॑हीता द्रोणकल॒श आ॑धव॒नीयः॑ पूत॒भृत्तान् यदगृ॑हीत्वोपाकु॒र्यात्पव॑मानं॒ वि॥२॥

%3.2.1.3
छि॒न्द्या॒त्तं वि॒च्छिद्य॑मानमध्व॒र्योः प्रा॒णो\-ऽनु॒ विच्छि॑द्येतोपया॒मगृ॑हीतो\-ऽसि प्र॒जाप॑तये॒ त्वेति॑ द्रोणकल॒शम॒भि मृ॑शे॒दिन्द्रा॑य॒ त्वेत्या॑धव॒नीयं॒ विश्वे᳚भ्यस्त्वा दे॒वेभ्य॒ इति॑ पूत॒भृत॒म्पव॑मानमे॒व तथ्सं त॑नोति॒ सर्व॒मायु॑रेति॒ न पु॒रायु॑षः॒ प्र मी॑यते पशु॒मान्भ॑वति वि॒न्दते᳚ प्र॒जाम्॥३॥

%3.2.2.0
{\anuvakamend[{ए॒ते वि द्विच॑त्वारिꣳशच्च}]}%॥१॥

%3.2.2.1
त्रीणि॒ वाव सव॑ना॒न्यथ॑ तृ॒तीय॒ꣳ॒ सव॑न॒मव॑ लुम्पन्त्यन॒ꣳ॒शु कु॒र्वन्त॑ उपा॒ꣳ॒शुꣳ हु॒त्वोपाꣳ॑शुपा॒त्रे\-ऽꣳ॑शुम॒वास्य॒ तं तृ॑तीयसव॒ने॑\-ऽपि॒सृज्या॒भि षु॑णुया॒द्यदा᳚प्या॒यय॑ति॒ तेनाꣳ॑शु॒मद्यद॑भिषु॒णोति॒ तेन॑र्जी॒षि सर्वा᳚ण्ये॒व तथ्सव॑नान्यꣳशु॒मन्ति॑ शु॒क्रव॑न्ति स॒माव॑द्वीर्याणि करोति॒ द्वौ स॑मु॒द्रौ वित॑तावजू॒र्यौ प॒र्याव॑र्तेते ज॒ठरे॑व॒ पादाः᳚। तयोः॒ पश्य॑न्तो॒ अति॑ यन्त्य॒न्यमप॑श्यन्तः॥४॥

%3.2.2.2
सेतु॒नाति॑ यन्त्य॒न्यम्। द्वे द्रध॑सी स॒तती॑ वस्त॒ एकः॑ के॒शी विश्वा॒ भुव॑नानि वि॒द्वान्। ति॒रो॒धायै॒त्यसि॑तं॒ वसा॑नः शु॒क्रमा द॑त्ते अनु॒हाय॑ जा॒र्यै। दे॒वा वै यद्य॒ज्ञे\-ऽकु॑र्वत॒ तदसु॑रा अकुर्वत॒ ते दे॒वा ए॒तम्म॑हाय॒ज्ञम॑पश्य॒न्तम॑तन्वता\-ऽ\-ग्निहो॒त्रं व्र॒तम॑कुर्वत॒ तस्मा॒द्द्विव्र॑तः स्या॒द्द्विर्\mbox{}ह्य॑ग्निहो॒त्रं जुह्व॑ति पौर्णमा॒सं य॒ज्ञम॑ग्नीषो॒मीयम्᳚॥५॥

%3.2.2.3
प॒शुम॑कुर्वत दा॒र्श्यं य॒ज्ञमा᳚ग्ने॒यम्प॒शुम॑कुर्वत वैश्वदे॒वम्प्रा॑तःसव॒नम॑कुर्वत वरुणप्रघा॒सान्माध्यं॑दिन॒ꣳ॒ सव॑नꣳ साकमे॒धान्पि॑तृय॒ज्ञं त्र्य॑म्बकाꣴस्तृतीयसव॒नम॑कुर्वत॒ तमे॑षा॒मसु॑रा य॒ज्ञम॒न्ववा॑जिगाꣳस॒न्तं नान्ववा॑य॒न्ते᳚\-ऽब्रुवन्नध्वर्त॒व्या वा इ॒मे दे॒वा अ॑भूव॒न्निति॒ तद॑ध्व॒रस्या᳚ध्वर॒त्वन्ततो॑ दे॒वा अभ॑व॒न्परासु॑रा॒ य ए॒वं वि॒द्वान्थ्सोमे॑न॒ यज॑ते॒ भव॑त्या॒त्मना॒ परा᳚स्य॒ भ्रातृ॑व्यो भवति॥६॥

%3.2.3.0
{\anuvakamend[{अप॑श्यन्तो\-ऽग्नीषो॒मीय॑मा॒त्मना॒ परा॒ त्रीणि॑ च}]}%॥२॥

%3.2.3.1
प॒रि॒भूर॒ग्निं प॑रि॒भूरिन्द्रं॑ परि॒भूर्विश्वां᳚ दे॒वान्प॑रि॒भूर्माꣳ स॒ह ब्र॑ह्मवर्च॒सेन॒ स नः॑ पवस्व॒ शं गवे॒ शं जना॑य॒ शमर्व॑ते॒ शꣳ रा॑ज॒न्नोष॑धी॒भ्यो\-ऽच्छि॑न्नस्य ते रयिपते सु॒वीर्य॑स्य रा॒यस्पोष॑स्य ददि॒तारः॑ स्याम। तस्य॑ मे रास्व॒ तस्य॑ ते भक्षीय॒ तस्य॑ त इ॒दमुन्मृ॑जे। प्रा॒णाय॑ मे वर्चो॒दा वर्च॑से पवस्वापा॒नाय॑ व्या॒नाय॑ वा॒चे॥७॥

%3.2.3.2
द॒क्ष॒क्र॒तुभ्यां॒ चक्षु॑र्भ्यां मे वर्चो॒दौ वर्च॑से पवेथा॒ꣴ॒ श्रोत्रा॑या॒त्मने\-ऽङ्गे᳚भ्य॒ आयु॑षे वी॒र्या॑य॒ विष्णो॒रिन्द्र॑स्य॒ विश्वे॑षां दे॒वानां᳚ ज॒ठर॑मसि वर्चो॒दा मे॒ वर्च॑से पवस्व॒ को॑\-ऽसि॒ को नाम॒ कस्मै᳚ त्वा॒ काय॑ त्वा॒ यं त्वा॒ सोमे॒नाती॑तृपं॒ यं त्वा॒ सोमे॒नामी॑मदꣳ सुप्र॒जाः प्र॒जया॑ भूयासꣳ सु॒वीरो॑ वी॒रैः सु॒वर्चा॒ वर्च॑सा सु॒पोषः॒ पोषै॒र्विश्वे᳚भ्यो मे रू॒पेभ्यो॑ वर्चो॒दाः॥८॥

%3.2.3.3
वर्च॑से पवस्व॒ तस्य॑ मे रास्व॒ तस्य॑ ते भक्षीय॒ तस्य॑ त इ॒दमुन्मृ॑जे। बुभू॑ष॒न्नवे᳚क्षेतै॒ष वै पात्रि॑यः प्र॒जाप॑तिर्य॒ज्ञः प्र॒जाप॑ति॒स्तमे॒व त॑र्पयति॒ स ए॑नं तृ॒प्तो भूत्या॒\-ऽभि प॑वते ब्रह्मवर्च॒सका॒मो\-ऽवे᳚क्षेतै॒ष वै पात्रि॑यः प्र॒जाप॑तिर्य॒ज्ञः प्र॒जाप॑तिस्तमे॒व त॑र्पयति॒ स ए॑नं तृ॒प्तो ब्र॑ह्मवर्च॒सेना॒भि प॑वत आमया॒वी॥९॥

%3.2.3.4
अवे᳚क्षेतै॒ष वै पात्रि॑यः प्र॒जाप॑तिर्य॒ज्ञः प्र॒जाप॑ति॒स्तमे॒व त॑र्पयति॒ स ए॑नं तृ॒प्त आयु॑षा॒भि प॑वते\-ऽभि॒चर॒न्नवे᳚क्षेतै॒ष वै पात्रि॑यः प्र॒जाप॑तिर्य॒ज्ञः प्र॒जाप॑ति॒स्तमे॒व त॑र्पयति॒ स ए॑नं तृ॒प्तः प्रा॑णापा॒ना\-भ्यां᳚ वा॒चो द॑क्षक्र॒तुभ्यां॒ चक्षु॑र्भ्या॒ꣴ॒ श्रोत्रा᳚भ्यामा॒त्मनो\-ऽङ्गे᳚भ्य॒ आयु॑षो॒\-ऽन्तरे॑ति ता॒जक्प्र ध॑न्वति॥१०॥

%3.2.4.0
{\anuvakamend[{वा॒चे रू॒पेभ्यो॑ वर्चो॒दा आ॑मया॒वी पञ्च॑चत्वारिꣳशच्च}]}%॥३॥

%3.2.4.1
स्फ्यः स्व॒स्तिर्वि॑घ॒नः स्व॒स्तिः पर्\mbox{}शु॒र्वेदिः॑ पर॒शुर्नः॑ स्व॒स्तिः। य॒ज्ञिया॑ यज्ञ॒कृतः॑ स्थ॒ ते मा॒स्मिन् य॒ज्ञ उप॑ ह्वयध्व॒मुप॑ मा॒ द्यावा॑पृथि॒वी ह्व॑येता॒मुपा᳚स्ता॒वः क॒लशः॒ सोमो॑ अ॒ग्निरुप॑ दे॒वा उप॑ य॒ज्ञ उप॑ मा॒ होत्रा॑ उपह॒वे ह्व॑यन्ता॒न्नमो॒\-ऽग्नये॑ मख॒घ्ने म॒खस्य॑ मा॒ यशो᳚\-ऽर्या॒दित्या॑हव॒नीय॒मुप॑ तिष्ठते य॒ज्ञो वै म॒खः॥११॥

%3.2.4.2
य॒ज्ञं वाव स तद॑ह॒न्तस्मा॑ ए॒व न॑म॒स्कृत्य॒ सदः॒ प्र स॑र्पत्या॒त्मनो\-ऽना᳚र्त्यै॒ नमो॑ रु॒द्राय॑ मख॒घ्ने नम॑स्कृत्या मा पा॒हीत्याग्नी᳚ध्रं॒ तस्मा॑ ए॒व न॑म†ङस्कृत्य॒ सदः॒ प्र स॑र्पत्या॒त्मनो\-ऽना᳚र्त्यै नम॒ इन्द्रा॑य मख॒घ्न इ॑न्द्रि॒यं मे॑ वी॒र्य॑म्मा॒ निर्व॑धी॒रिति॑ हो॒त्रीय॑मा॒शिष॑मे॒वैतामा शा᳚स्त इन्द्रि॒यस्य॑ वी॒र्य॑स्यानि॑र्घाताय॒ या वै॥१२॥

%3.2.4.3
दे॒वताः॒ सद॒स्यार्ति॑मा॒र्पय॑न्ति॒ यस्ता वि॒द्वान्प्र॒सर्प॑ति॒ न सद॒स्यार्ति॒मार्च्छ॑ति॒ नमो॒\-ऽग्नये॑ मख॒घ्न इत्या॑है॒ता वै दे॒वताः॒ सद॒स्यार्ति॒मार्प॑यन्ति॒ ता य ए॒वं वि॒द्वान्प्र॒सर्प॑ति॒ न सद॒स्यार्ति॒मार्च्छ॑ति दृ॒ढे स्थः॑ शिथि॒रे स॒मीची॒ माꣳह॑सस्पात॒ꣳ॒ सूर्यो॑ मा दे॒वो दि॒व्यादꣳह॑सस्पातु वा॒युर॒न्तरि॑क्षात्॥१३॥

%3.2.4.4
अ॒ग्निः पृ॑थि॒व्या य॒मः पि॒तृभ्यः॒ सर॑स्वती मनु॒ष्ये᳚भ्यो॒ देवी᳚ द्वारौ॒ मा मा॒ सं ता॑प्त॒म् नमः॒ सद॑से॒ नमः॒ सद॑स॒स्पत॑ये॒ नमः॒ सखी॑नां पुरो॒गाणां॒ चक्षु॑षे॒ नमो॑ दि॒वे नमः॑ पृथि॒व्या अहे॑ दैधिष॒व्योदत॑स्तिष्ठा॒न्यस्य॒ सद॑ने सीद॒ यो᳚\-ऽस्मत्पाक॑तर॒ उन्नि॒वत॒ उदु॒द्वत॑श्च गेषम्पा॒तम्मा᳚ द्यावापृथिवी अ॒द्याह्नः॒ सदो॒ वै प्र॒सर्प॑न्तम्॥१४॥

%3.2.4.5
पि॒तरो\-ऽनु॒ प्र स॑र्पन्ति॒ त ए॑नमीश्व॒रा हिꣳसि॑तोः॒ सदः॑ प्र॒सृप्य॑ दक्षिणा॒र्धं परे᳚क्षे॒ताग॑न्त पितरः पितृ॒मान॒हं यु॒ष्माभि॑र्भूयासꣳ सुप्र॒जसो॒ मया॑ यू॒यम्भू॑या॒स्तेति॒ तेभ्य॑ ए॒व न॑म॒स्कृत्य॒ सदः॒ प्र स॑र्पत्या॒त्मनो\-ऽना᳚र्त्यै॥१५॥

%3.2.5.0
{\anuvakamend[{म॒खो वा अ॒न्तरि॑क्षात्प्र॒सर्प॑न्त॒न्त्रय॑स्त्रिꣳशच्च}]}%॥४॥

%3.2.5.1
भक्षेहि॒ मा वि॑श दीर्घायु॒त्वाय॑ शन्तनु॒त्वाय॑ रा॒यस्पोषा॑य॒ वर्च॑से सुप्रजा॒स्त्वायेहि॑ वसो पुरोवसो प्रि॒यो मे॑ हृ॒दो᳚\-ऽस्य॒श्विनो᳚स्त्वा बा॒हुभ्याꣳ॑ सघ्यासम् नृ॒चक्ष॑सं त्वा देव सोम सु॒चक्षा॒ अव॑ ख्येषम् म॒न्द्राभिभू॑तिः के॒तुर्य॒ज्ञानां॒ वाग्जु॑षा॒णा सोम॑स्य तृप्यतु म॒न्द्रा स्व॑र्वा॒च्यदि॑ति॒रना॑हतशीर्ष्णी॒ वाग्जु॑षा॒णा सोम॑स्य तृप्य॒त्वेहि॑ विश्वचर्\mbox{}षणे॥१६॥

%3.2.5.2
श॒म्भूर्म॑यो॒भूः स्व॒स्ति मा॑ हरिवर्ण॒ प्र च॑र॒ क्रत्वे॒ दक्षा॑य रा॒यस्पोषा॑य सुवी॒रता॑यै॒ मा मा॑ राज॒न्वि बी॑भिषो॒ मा मे॒ हार्दि॑ त्वि॒षा व॑धीः। वृष॑णे॒ शुष्मा॒यायु॑षे॒ वर्च॑से॥ वसु॑मद्गणस्य सोम देव ते मति॒विदः॑ प्रातःसव॒नस्य॑ गाय॒त्रछ॑न्दस॒ इन्द्र॑पीतस्य॒ नरा॒शꣳस॑पीतस्य पि॒तृपी॑तस्य॒ मधु॑मत॒ उप॑हूत॒स्योप॑हूतो भक्षयामि रु॒द्रव॑द्गणस्य सोम देव ते मति॒विदो॒ माध्यं॑दिनस्य॒ सव॑नस्य त्रि॒ष्टुप्छ॑न्दस॒ इन्द्र॑पीतस्य॒ नरा॒शꣳस॑पीतस्य॥१७॥

%3.2.5.3
पि॒तृपी॑तस्य॒ मधु॑मत॒ उप॑हूत॒स्योप॑हूतो भक्षयाम्यादि॒त्यव॑द्गणस्य सोम देव ते मति॒विद॑स्तृ॒तीय॑स्य॒ सव॑नस्य॒ जग॑तीछन्दस॒ इन्द्र॑पीतस्य॒ नरा॒शꣳस॑पीतस्य पि॒तृपी॑तस्य॒ मधु॑मत॒ उप॑हूत॒स्योप॑हूतो भक्षयामि। आ प्या॑यस्व॒ समे॑तु ते वि॒श्वतः॑ सोम॒ वृष्णि॑यम्। भवा॒ वाज॑स्य सङ्ग॒थे। हिन्व॑ मे॒ गात्रा॑ हरिवो ग॒णान्मे॒ मा वि ती॑तृषः। शि॒वो मे॑ सप्त॒र्\mbox{}षीनुप॑ तिष्ठस्व॒ मा मे\-ऽवा॒ङ्नाभि॒मति॑॥१८॥

%3.2.5.4
गाः॒। अपा॑म॒ सोम॑म॒मृता॑ अभू॒माद॑र्श्म॒ ज्योति॒रवि॑दाम दे॒वान्। किम॒स्मान्कृ॑णव॒दरा॑तिः॒ किमु॑ धू॒र्तिर॑मृत॒ मर्त्य॑स्य। यन्म॑ आ॒त्मनो॑ मि॒न्दाभू॑द॒ग्निस्तत्पुन॒राहा᳚र्जा॒तवे॑दा॒ विच॑र्\mbox{}षणिः। पुन॑र॒ग्निश्चक्षु॑रदा॒त्पुन॒रिन्द्रो॒ बृह॒स्पतिः॑। पुन॑र्मे अश्विना यु॒वं चक्षु॒रा ध॑त्तम॒क्ष्योः। इ॒ष्टय॑जुषस्ते देव सोम स्तु॒तस्तो॑मस्य॥१९॥

%3.2.5.5
श॒स्तोक्थ॑स्य॒ हरि॑वत॒ इन्द्र॑पीतस्य॒ मधु॑मत॒ उप॑हूत॒स्योप॑हूतो भक्षयामि। आ॒पूर्याः॒ स्था मा॑ पूरयत प्र॒जया॑ च॒ धने॑न च। ए॒तत्ते॑ तत॒ ये च॒ त्वामन्वे॒तत्ते॑ पितामह प्रपितामह॒ ये च॒ त्वामन्वत्र॑ पितरो यथाभा॒गम्म॑न्दध्व॒म् नमो॑ वः पितरो॒ रसा॑य॒ नमो॑ वः पितरः॒ शुष्मा॑य॒ नमो॑ वः पितरो जी॒वाय॒ नमो॑ वः पितरः॥२०॥

%3.2.5.6
स्व॒धायै॒ नमो॑ वः पितरो म॒न्यवे॒ नमो॑ वः पितरो घो॒राय॒ पित॑रो॒ नमो॑ वो॒ य ए॒तस्मि॑ल्लोँ॒के स्थ यु॒ष्माꣴस्ते\-ऽनु॒ ये᳚\-ऽस्मिल्लोँ॒के मां ते\-ऽनु॒ य ए॒तस्मि॑ल्लोँ॒के स्थ यू॒यं तेषां॒ वसि॑ष्ठा भूयास्त॒ ये᳚\-ऽस्मिल्लोँ॒के॑\-ऽहं तेषां॒ वसि॑ष्ठो भूयास॒म् प्रजा॑पते॒ न त्वदे॒तान्य॒न्यो विश्वा॑ जा॒तानि॒ परि॒ ता ब॑भूव॥२१॥

%3.2.5.7
यत्का॑मास्ते जुहु॒मस्तन्नो॑ अस्तु व॒यꣴ स्या॑म॒ पत॑यो रयी॒णाम्। दे॒वकृ॑त॒स्यैन॑सो\-ऽव॒यज॑नमसि मनु॒ष्य॑कृत॒स्यैन॑सो\-ऽ\-व॒यज॑नमसि पि॒तृकृ॑त॒स्यैन॑सो\-ऽव॒यज॑नमस्य॒फ्सु धौ॒तस्य॑ सोम देव ते॒ नृभिः॑ सु॒तस्ये॒ष्टय॑जुषः स्तु॒तस्तो॑मस्य श॒स्तोक्थ॑स्य॒ यो भ॒क्षो अ॑श्व॒सनि॒र्यो गो॒सनि॒स्तस्य॑ ते पि॒तृभि॑र्भ॒क्षं कृ॑त॒स्योप॑हूत॒स्योप॑हूतो भक्षयामि॥२२॥

%3.2.6.0
{\anuvakamend[{वि॒श्व॒च॒र्\mbox{}ष॒णे॒ त्रि॒ष्टुफ्छ॑न्दस॒ इन्द्र॑पीतस्य॒ नरा॒शꣳस॑पीत॒स्याति॑ स्तु॒तस्तो॑मस्य जी॒वाय॒ नमो॑ वः पितरो बभूव॒ चतु॑श्चत्वारिꣳशच्च}]}%॥५॥

%3.2.6.1
म॒ही॒नाम्पयो॑\-ऽसि॒ विश्वे॑षां दे॒वानां᳚ त॒नूर्\mbox{}ऋ॒ध्यास॑म॒द्य पृ॑षतीनां॒ ग्रह॒म्पृष॑तीनां॒ ग्रहो॑\-ऽसि॒ विष्णो॒र्\mbox{}हृद॑यम॒स्येक॑मिष॒ विष्णु॒स्त्वानु॒ वि च॑क्रमे भू॒तिर्द॒ध्ना घृ॒तेन॑ वर्धतां॒ तस्य॑ मे॒ष्टस्य॑ वी॒तस्य॒ द्रवि॑ण॒मा ग॑म्या॒ज्ज्योति॑रसि वैश्वान॒रं पृश्नि॑यै दु॒ग्धम् याव॑ती॒ द्यावा॑पृथि॒वी म॑हि॒त्वा याव॑च्च स॒प्त सिन्ध॑वो वित॒स्थुः। ताव॑न्तमिन्द्र ते॥२३॥

%3.2.6.2
ग्रहꣳ॑ स॒होर्जा गृ॑ह्णा॒म्यस्तृ॑तम्। यत्कृ॑ष्णशकु॒नः पृ॑षदा॒ज्यम॑वमृ॒शेच्छू॒द्रा अ॑स्य प्र॒मायु॑काः स्यु॒र्यच्छ्वा\-ऽव॑मृ॒शेच्चतु॑ष्पादो\-ऽस्य प॒शवः॑ प्र॒मायु॑काः स्यु॒र्यथ्स्कन्दे॒द्यज॑मानः प्र॒मायु॑कः स्यात्प॒शवो॒ वै पृ॑षदा॒ज्यम्प॒शवो॒ वा ए॒तस्य॑ स्कन्दन्ति॒ यस्य॑ पृषदा॒ज्यꣴ स्कन्द॑ति॒ यत्पृ॑षदा॒ज्यम्पुन॑र्गृ॒ह्णाति॑ प॒शूने॒वास्मै॒ पुन॑र्गृह्णाति प्रा॒णो वै पृ॑षदा॒ज्यं प्रा॒णो वै॥२४॥

%3.2.6.3
ए॒तस्य॑ स्कन्दति॒ यस्य॑ पृषदा॒ज्यꣴ स्कन्द॑ति॒ यत्पृ॑षदा॒ज्यम्पुन॑र्गृ॒ह्णाति॑ प्रा॒णमे॒वास्मै॒ पुन॑र्गृह्णाति॒ हिर॑ण्यमव॒धाय॑ गृह्णात्य॒मृतं॒ वै हिर॑ण्यं प्रा॒णः पृ॑षदा॒ज्यम॒मृत॑मे॒वास्य॑ प्रा॒णे द॑धाति श॒तमा॑नम्भवति श॒तायुः॒ पुरु॑षः श॒तेन्द्रि॑य॒ आयु॑ष्ये॒वेन्द्रि॒ये प्रति॑ तिष्ठ॒त्यश्व॒मव॑ घ्रापयति प्राजाप॒त्यो वा अश्वः॑ प्राजाप॒त्यः प्रा॒णः स्वादे॒वास्मै॒ योनेः᳚ प्रा॒णं निर्मि॑मीते॒ वि वा ए॒तस्य॑ य॒ज्ञश्छि॑द्यते॒ यस्य॑ पृषदा॒ज्यꣴ स्कन्द॑ति वैष्ण॒व्यर्चा पुन॑र्गृह्णाति य॒ज्ञो वै विष्णु॑र्य॒ज्ञेनै॒व य॒ज्ञꣳ सं त॑नोति॥२५॥

%3.2.7.0
{\anuvakamend[{ते॒ पृ॒ष॒दा॒ज्यं प्रा॒णो वै योनेः᳚ प्रा॒णन्द्वाविꣳ॑शतिश्च}]}%॥६॥

%3.2.7.1
देव॑ सवितरे॒तत्ते॒ प्राह॒ तत्प्र च॑ सु॒व प्र च॑ यज॒ बृह॒स्पति॑र्ब्र॒ह्मायु॑ष्मत्या ऋ॒चो मा गा॑त तनू॒पाथ्साम्नः॑ स॒त्या व॑ आ॒शिषः॑ सन्तु स॒त्या आकू॑तय ऋ॒तं च॑ स॒त्यं च॑ वदत स्तु॒त दे॒वस्य॑ सवि॒तुः प्र॑स॒वे स्तु॒तस्य॑ स्तु॒तम॒स्यूर्ज॒म्मह्यꣴ॑ स्तु॒तं दु॑हा॒मा मा᳚ स्तु॒तस्य॑ स्तु॒तं ग॑म्याच्छ॒स्त्रस्य॑ श॒स्त्रम्॥२६॥

%3.2.7.2
अ॒स्यूर्ज॒म्मह्यꣳ॑ श॒स्त्रं दु॑हा॒मा मा॑ श॒स्त्रस्य॑ श॒स्त्रं ग॑म्यादिन्द्रि॒याव॑न्तो वनामहे धुक्षी॒महि॑ प्र॒जामिषम्᳚। सा मे॑ स॒त्याशीर्दे॒वेषु॑ भूयात् ब्रह्मवर्च॒सं मा ग॑म्यात्। य॒ज्ञो ब॑भूव॒ स आ ब॑भूव॒ स प्र ज॑ज्ञे॒ स वा॑वृधे। स दे॒वाना॒मधि॑पतिर्बभूव॒ सो अ॒स्माꣳ अधि॑पतीन्करोतु व॒यꣴ स्या॑म॒ पत॑यो रयी॒णाम्। य॒ज्ञो वा॒ वै॥२७॥

%3.2.7.3
य॒ज्ञप॑तिं दु॒हे य॒ज्ञप॑तिर्वा य॒ज्ञं दु॑हे॒ स यः स्तु॑तश॒स्त्रयो॒र्दोह॒मवि॑द्वा॒न् यज॑ते॒ तं य॒ज्ञो दु॑हे स इ॒ष्ट्वा पापी॑यान्भवति॒ य ए॑नयो॒र्दोहं॑ वि॒द्वान् यज॑ते॒ स य॒ज्ञं दु॑हे॒ स इ॒ष्ट्वा वसी॑यान्भवति स्तु॒तस्य॑ स्तु॒तम॒स्यूर्ज॒म्मह्यꣴ॑ स्तु॒तं दु॑हा॒मा मा᳚ स्तु॒तस्य॑ स्तु॒तं ग॑म्याच्छ॒स्त्रस्य॑ श॒स्त्रम॒स्यूर्ज॒म्मह्यꣳ॑ श॒स्त्रं दु॑हा॒मा मा॑ श॒स्त्रस्य॑ श॒स्त्रं ग॑म्या॒दित्या॑है॒ष वै स्तु॑तश॒स्त्रयो॒र्दोह॒स्तं॒ य ए॒वं वि॒द्वान् यज॑ते दु॒ह ए॒व य॒ज्ञमि॒ष्ट्वा वसी॑यान्भवति॥२८॥

%3.2.8.0
{\anuvakamend[{श॒स्त्रं वै श॒स्त्रन्दु॑हा॒न्द्वाविꣳ॑शतिश्च}]}%॥७॥

%3.2.8.1
श्ये॒नाय॒ पत्व॑ने॒ स्वाहा॒ वट्थ्स्व॒यम॑भिगूर्ताय॒ नमो॑ विष्ट॒म्भाय॒ धर्म॑णे॒ स्वाहा॒ वट्थ्स्व॒यम॑भिगूर्ताय॒ नमः॑ परि॒धये॑ जन॒प्रथ॑नाय॒ स्वाहा॒ वट्थ्स्व॒यम॑भिगूर्ताय॒ नम॑ ऊ॒र्जे होत्रा॑णा॒ꣴ॒ स्वाहा॒ वट्थ्स्व॒यम॑भिगूर्ताय॒ नमः॒ पय॑से॒ होत्रा॑णा॒ꣴ॒ स्वाहा॒ वट्थ्स्व॒यम॑भिगूर्ताय॒ नमः॑ प्र॒जाप॑तये॒ मन॑वे॒ स्वाहा॒ वट्थ्स्व॒यम॑भिगूर्ताय॒ नम॑ ऋ॒तमृ॑तपाः सुवर्वा॒ट्थ्स्वाहा॒ वट्थ्स्व॒यम॑भिगूर्ताय॒ नम॑स्तृ॒म्पन्ता॒ꣳ॒ होत्रा॒ मधो᳚र्घृ॒तस्य॑ य॒ज्ञप॑ति॒मृष॑य॒ एन॑सा॥२९॥

%3.2.8.2
आ॒हुः॒। प्र॒जा निर्भ॑क्ता अनुत॒प्यमा॑ना मध॒व्यौ᳚ स्तो॒कावप॒ तौ र॑राध। सं न॒स्ताभ्याꣳ॑ सृजतु वि॒श्वक॑र्मा घो॒रा ऋष॑यो॒ नमो॑ अस्त्वेभ्यः। चक्षु॑ष एषा॒म्मन॑सश्च सं॒धौ बृह॒स्पत॑ये॒ महि॒ षद्द्यु॒मन्नमः॑। नमो॑ वि॒श्वक॑र्मणे॒ स उ॑ पात्व॒स्मान॑न॒न्यान्थ्सो॑म॒पान्मन्य॑मानः। प्रा॒णस्य॑ वि॒द्वान्थ्स॑म॒रे न धीर॒ एन॑श्चकृ॒वान्महि॑ ब॒द्ध ए॑षाम्। तं वि॑श्वकर्मन्न्॥३०॥

%3.2.8.3
प्र मु॑ञ्चा स्व॒स्तये॒ ये भ॒क्षय॑न्तो॒ न वसू᳚न्यानृ॒हुः। यान॒ग्नयो॒\-ऽन्वत॑प्यन्त॒ धिष्णि॑या इ॒यं तेषा॑मव॒या दुरि॑ष्ट्यै॒ स्विष्टिं न॒स्तां कृ॑णोतु वि॒श्वक॑र्मा। नमः॑ पि॒तृभ्यो॑ अ॒भि ये नो॒ अख्य॑न् यज्ञ॒कृतो॑ य॒ज्ञका॑माः सुदे॒वा अ॑का॒मा वो॒ दक्षि॑णां॒ न नी॑निम॒ मा न॒स्तस्मा॒देन॑सः पापयिष्ट। याव॑न्तो॒ वै स॑द॒स्या᳚स्ते॒ सर्वे॑ दक्षि॒ण्या᳚स्तेभ्यो॒ यो दक्षि॑णां॒ न॥३१॥

%3.2.8.4
न॒ये॒दैभ्यो॑ वृश्च्येत॒ यद्वै᳚श्वकर्म॒णानि॑ जु॒होति॑ सद॒स्या॑ने॒व तत्प्री॑णात्य॒स्मे दे॑वासो॒ वपु॑षे चिकिथ्सत॒ यमा॒शिरा॒ दम्प॑ती वा॒मम॑श्नु॒तः। पुमा᳚न्पु॒त्रो जा॑यते वि॒न्दते॒ वस्वथ॒ विश्वे॑ अर॒पा ए॑धते गृ॒हः। आ॒शी॒र्दा॒या दम्प॑ती वा॒मम॑श्नुता॒मरि॑ष्टो॒ रायः॑ सचता॒ꣳ॒ समो॑कसा। य आसि॑च॒थ्सन्दु॑ग्धं कु॒म्भ्या स॒हेष्टेन॒ याम॒न्नम॑तिं जहातु॒ सः। स॒र्पि॒र्ग्री॒वी॥३२॥


%3.2.8.5
पीव॑र्यस्य जा॒या पीवा॑नः पु॒त्रा अकृ॑शासो अस्य। स॒हजा॑नि॒र्यः सु॑मख॒स्यमा॑न॒ इन्द्रा॑या॒शिरꣳ॑ स॒ह कु॒म्भ्यादा᳚त्। आ॒शीर्म॒ ऊर्ज॑मु॒त सु॑प्रजा॒स्त्वमिषं॑ दधातु॒ द्रवि॑ण॒ꣳ॒ सव॑र्चसम्। सं॒जय॒न्क्षेत्रा॑णि॒ सह॑सा॒हमि॑न्द्र कृण्वा॒नो अ॒न्याꣳ अध॑रान्थ्स॒पत्नान्॑। भू॒तम॑सि भू॒ते मा॑ धा॒ मुख॑मसि॒ मुख॑म्भूयास॒म् द्यावा॑पृथि॒वी\-भ्यां᳚ त्वा॒ परि॑ गृह्णामि॒ विश्वे᳚ त्वा दे॒वा वै᳚श्वान॒राः॥३३॥

%3.2.8.6
प्र च्या॑वयन्तु दि॒वि दे॒वां दृꣳ॑हा॒न्तरि॑क्षे॒ वयाꣳ॑सि पृथि॒व्याम्पार्थि॑वान्ध्रु॒वं ध्रु॒वेण॑ ह॒विषाव॒ सोमं॑ नयामसि। यथा॑ नः॒ सर्व॒मिज्जग॑दय॒क्ष्मꣳ सु॒मना॒ अस॑त्। यथा॑ न॒ इन्द्र॒ इद्विशः॒ केव॑लीः॒ सर्वाः॒ सम॑नसः॒ कर॑त्। यथा॑ नः॒ सर्वा॒ इद्दिशो॒\-ऽस्माकं॒ केव॑लीरसन्न्॑॥३४॥

%3.2.9.0
{\anuvakamend[{एन॑सा विश्वकर्म॒न् यो दक्षि॑णां॒ न स॑र्पिर्ग्री॒वी वै᳚श्वान॒राश्च॑त्वारि॒ꣳ॒शच्च॑}]}%॥८॥

%3.2.9.1
यद्वै होता᳚ध्व॒र्युम॑भ्या॒ह्वय॑ते॒ वज्र॑मेनम॒भि प्र व॑र्तय॒त्युक्थ॑शा॒ इत्या॑ह प्रातःसव॒नम्प्र॑ति॒गीर्य॒ त्रीण्ये॒तान्य॒क्षरा॑णि त्रि॒पदा॑ गाय॒त्री गा॑य॒त्रम्प्रा॑तःसव॒नं गा॑यत्रि॒यैव प्रा॑तःसव॒ने वज्र॑म॒न्तर्ध॑त्त उ॒क्थं वा॒चीत्या॑ह॒ माध्यं॑दिन॒ꣳ॒ सव॑नं प्रति॒गीर्य॑ च॒त्वार्ये॒तान्य॒क्षरा॑णि॒ चतु॑ष्पदा त्रि॒ष्टुप्त्रैष्टु॑भ॒म्माध्यं॑दिन॒ꣳ॒ सव॑नं त्रि॒ष्टुभै॒व माध्यं॑दिने॒ सव॑ने॒ वज्र॑म॒न्तर्ध॑त्ते॥३५॥

%3.2.9.2
उ॒क्थं वा॒चीन्द्रा॒येत्या॑ह तृतीयसव॒नम्प्र॑ति॒गीर्य॑ स॒प्तैतान्य॒क्षरा॑णि स॒प्तप॑दा॒ शक्व॑री शाक्व॒रो वज्रो॒ वज्रे॑णै॒व तृ॑तीयसव॒ने वज्र॑म॒न्तर्ध॑त्ते ब्रह्मवा॒दिनो॑ वदन्ति स त्वा अ॑ध्व॒र्युः स्या॒द्यो य॑थासव॒नम्प्र॑तिग॒रे छन्दाꣳ॑सि सम्पा॒दये॒त्तेजः॑ प्रातःसव॒न आ॒त्मन्दधी॑तेन्द्रि॒यम्माध्यं॑दिने॒ सव॑ने प॒शूꣴस्तृ॑तीयसव॒न इत्युक्थ॑शा॒ इत्या॑ह प्रातःसव॒नम्प्र॑ति॒गीर्य॒ त्रीण्ये॒तान्य॒क्षरा॑णि॥३६॥

%3.2.9.3
त्रि॒पदा॑ गाय॒त्री गा॑य॒त्रम्प्रा॑तःसव॒नम्प्रा॑तःसव॒न ए॒व प्र॑तिग॒रे छन्दाꣳ॑सि॒ सम्पा॑दय॒त्यथो॒ तेजो॒ वै गा॑य॒त्री तेजः॑ प्रातःसव॒नं तेज॑ ए॒व प्रा॑तःसव॒न आ॒त्मन्ध॑त्त उ॒क्थं वा॒चीत्या॑ह॒ माध्यं॑दिन॒ꣳ॒ सव॑नं प्रति॒गीर्य॑ च॒त्वार्ये॒तान्य॒क्षरा॑णि॒ चतु॑ष्पदा त्रि॒ष्टुप्त्रैष्टु॑भ॒म्माध्यं॑दिन॒ꣳ॒ सव॑न॒म्माध्यं॑दिन ए॒व सव॑ने प्रतिग॒रे छन्दाꣳ॑सि॒ सम्पा॑दय॒त्यथो॑ इन्द्रि॒यं वै त्रि॒ष्टुगि॑न्द्रि॒यम्माध्यं॑दिन॒ꣳ॒ सव॑नम्॥३७॥

%3.2.9.4
इ॒न्द्रि॒यमे॒व माध्यं॑दिने॒ सव॑न आ॒त्मन्ध॑त्त उ॒क्थं वा॒चीन्द्रा॒येत्या॑ह तृतीयसव॒नम्प्र॑ति॒गीर्य॑ स॒प्तैतान्य॒क्षरा॑णि स॒प्तप॑दा॒ शक्व॑री शाक्व॒राः प॒शवो॒ जाग॑तं तृतीयसव॒नं तृ॑तीयसव॒न ए॒व प्र॑तिग॒रे छन्दाꣳ॑सि॒ सम्पा॑दय॒त्यथो॑ प॒शवो॒ वै जग॑ती प॒शव॑स्तृतीयसव॒नं प॒शूने॒व तृ॑तीयसव॒न आ॒त्मन्ध॑त्ते॒ यद्वै होता᳚ध्व॒र्युम॑भ्या॒ह्वय॑त आ॒व्य॑मस्मिन्दधाति॒ तद्यन्न॥३८॥

%3.2.9.5
अ॒प॒हनी॑त पु॒रास्य॑ संवथ्स॒राद्गृ॒ह आ वे॑वीर॒ञ्छोꣳसा॒ मोद॑ इ॒वेति॑ प्र॒त्याह्व॑यते॒ तेनै॒व तदप॑ हते॒ यथा॒ वा आय॑ताम्प्र॒तीक्ष॑त ए॒वम॑ध्व॒र्युः प्र॑तिग॒रम्प्रती᳚क्षते॒ यद॑भिप्रतिगृणी॒याद्यथाय॑तया समृ॒च्छते॑ ता॒दृगे॒व तद्यद॑र्ध॒र्चाल्लुप्ये॑त॒ यथा॒ धाव॑द्भ्यो॒ हीय॑ते ता॒दृगे॒व तत्प्र॒बाहु॒ग्वा ऋ॒त्विजा॑मुद्गी॒था उ॑द्गी॒थ ए॒वोद्गा॑तृ॒णाम्॥३९॥

%3.2.9.6
ऋ॒चः प्र॑ण॒व उ॑क्थश॒ꣳ॒सिनां᳚ प्रतिग॒रो॑\-ऽध्वर्यू॒णाम् य ए॒वं वि॒द्वान्प्र॑तिगृ॒णात्य॑न्ना॒द ए॒व भ॑व॒त्यास्य॑ प्र॒जायां᳚ वा॒जी जा॑यत इ॒यम्वै होता॒साव॑ध्व॒र्युर्यदासी॑नः॒ शꣳस॑त्य॒स्या ए॒व तद्धोता॒ नैत्यास्त॑ इव॒ हीयमथो॑ इ॒मामे॒व तेन॒ यज॑मानो दुहे॒ यत्तिष्ठ॑न्प्रतिगृ॒णात्य॒मुष्या॑ ए॒व तद॑ध्व॒र्युर्नैति॑॥४०॥

%3.2.9.7
तिष्ठ॑तीव॒ ह्य॑सावथो॑ अ॒मूमे॒व तेन॒ यज॑मानो दुहे॒ यदासी॑नः॒ शꣳस॑ति॒ तस्मा॑दि॒तःप्र॑दानं दे॒वा उप॑ जीवन्ति॒ यत्तिष्ठ॑न्प्रतिगृ॒णाति॒ तस्मा॑द॒मुतः॑प्रदानम्मनु॒ष्या॑ उप॑ जीवन्ति॒ यत्प्राङासी॑नः॒ शꣳस॑ति प्र॒त्यङ्तिष्ठ॑न्प्रतिगृ॒णाति॒ तस्मा᳚त्प्रा॒चीन॒ꣳ॒ रेतो॑ धीयते प्र॒तीचीः᳚ प्र॒जा जा॑यन्ते॒ यद्वै होता᳚ध्व॒र्युम॑भ्या॒ह्वय॑ते॒ वज्र॑मेनम॒भि प्र व॑र्तयति॒ परा॒ङा व॑र्तते॒ वज्र॑मे॒व तन्नि क॑रोति॥४१॥

%3.2.10.0
{\anuvakamend[{सव॑ने॒ वज्र॑म॒न्तर्ध॑त्ते॒ त्रीण्ये॒तान्य॒क्षरा॑णीन्द्रि॒यम्माध्य॑न्दिन॒ꣳ॒ सव॑न॒न्नोद्गा॑तृ॒णाम॑ध्व॒र्युर्नैति॑ वर्तयत्य॒ष्टौ च॑}]}%॥९॥

%3.2.10.1
उ॒प॒या॒मगृ॑हीतो\-ऽसि वाक्ष॒सद॑सि वा॒क्पा\-भ्यां᳚ त्वा क्रतु॒पाभ्या॑म॒स्य य॒ज्ञस्य॑ ध्रु॒वस्याध्य॑क्षाभ्यां गृह्णाम्युपया॒मगृ॑हीतो\-ऽस्यृत॒सद॑सि चक्षु॒ष्पा\-भ्यां᳚ त्वा क्रतु॒पाभ्या॑म॒स्य य॒ज्ञस्य॑ ध्रु॒वस्याध्य॑क्षाभ्यां गृह्णाम्युपया॒मगृ॑हीतो\-ऽसि श्रुत॒सद॑सि श्रोत्र॒पा\-भ्यां᳚ त्वा क्रतु॒पाभ्या॑म॒स्य य॒ज्ञस्य॑ ध्रु॒वस्याध्य॑क्षाभ्यां गृह्णामि दे॒वेभ्य॑स्त्वा वि॒श्वदे॑वेभ्यस्त्वा॒ विश्वे᳚भ्यस्त्वा दे॒वेभ्यो॒ विष्ण॑वुरुक्रमै॒ष ते॒ सोम॒स्तꣳ र॑क्षस्व॥४२॥

%3.2.10.2
तं ते॑ दु॒श्चक्षा॒ माव॑ ख्य॒त् मयि॒ वसुः॑ पुरो॒वसु॑र्वा॒क्पा वाचं॑ मे पाहि॒ मयि॒ वसु॑र्वि॒दद्व॑सुश्चक्षु॒ष्पाश्चक्षु॑र्मे पाहि॒ मयि॒ वसुः॑ सं॒यद्व॑सुः श्रोत्र॒पाः श्रोत्रं॑ मे पाहि॒ भूर॑सि॒ श्रेष्ठो॑ रश्मी॒नाम्प्रा॑ण॒पाः प्रा॒णं मे॑ पाहि॒ धूर॑सि॒ श्रेष्ठो॑ रश्मी॒नाम॑पान॒पा अ॑पा॒नं मे॑ पाहि॒ यो न॑ इन्द्रवायू मित्रावरुणावश्विनावभि॒दास॑ति॒ भ्रातृ॑व्य उ॒त्पिपी॑ते शुभस्पती इ॒दम॒हं तमध॑रम्पादयामि॒ यथे᳚न्द्रा॒हमु॑त्त॒मश्चे॒तया॑नि॥४३॥

%3.2.11.0
{\anuvakamend[{र॒क्ष॒स्व॒ भ्रातृ॑व्य॒स्त्रयो॑दश च}]}%॥10॥

%3.2.11.1
प्र सो अ॑ग्ने॒ तवो॒तिभिः॑ सु॒वीरा॑भिस्तरति॒ वाज॑कर्मभिः। यस्य॒ त्वꣳ स॒ख्यमावि॑थ। प्र होत्रे॑ पू॒र्व्यं वचो॒\-ऽग्नये॑ भरता बृ॒हत्। वि॒पां ज्योतीꣳ॑षि॒ बिभ्र॑ते॒ न वे॒धसे᳚। अग्ने॒ त्री ते॒ वाजि॑ना॒ त्री ष॒धस्था॑ ति॒स्रस्ते॑ जि॒ह्वा ऋ॑तजात पू॒र्वीः। ति॒स्र उ॑ ते त॒नुवो॑ दे॒ववा॑ता॒स्ताभि॑र्नः पाहि॒ गिरो॒ अप्र॑युच्छन्न्। सं वां॒ कर्म॑णा॒ समि॒षा॥४४॥

%3.2.11.2
हि॒नो॒मीन्द्रा॑विष्णू॒ अप॑सस्पा॒रे अ॒स्य। जु॒षेथां᳚ य॒ज्ञं द्रवि॑णं च धत्त॒मरि॑ष्टैर्नः प॒थिभिः॑ पा॒रय॑न्ता। उ॒भा जि॑ग्यथु॒र्न परा॑ जयेथे॒ न परा॑ जिग्ये कत॒रश्च॒नैनोः᳚। इन्द्र॑श्च विष्णो॒ यदप॑स्पृधेथां त्रे॒धा स॒हस्रं॒ वि तदै॑रयेथाम्। त्रीण्यायूꣳ॑षि॒ तव॑ जातवेदस्ति॒स्र आ॒जानी॑रु॒षस॑स्ते अग्ने। ताभि॑र्दे॒वाना॒मवो॑ यक्षि वि॒द्वानथ॑॥४५॥

%3.2.11.3
भ॒व॒ यज॑मानाय॒ शं योः। अ॒ग्निस्त्रीणि॑ त्रि॒धातू॒न्या क्षे॑ति वि॒दथा॑ क॒विः। स त्रीꣳ॑रेकाद॒शाꣳ इ॒ह। यक्ष॑च्च पि॒प्रय॑च्च नो॒ विप्रो॑ दू॒तः परि॑ष्कृतः। नभ॑न्तामन्य॒के स॑मे। इन्द्रा॑विष्णू दृꣳहि॒ताः शम्ब॑रस्य॒ नव॒ पुरो॑ नव॒तिं च॑ श्नथिष्टम्। श॒तं व॒र्चिनः॑ स॒हस्रं॑ च सा॒कꣳ ह॒थो अ॑प्र॒त्यसु॑रस्य वी॒रान्। उ॒त मा॒ता म॑हि॒षमन्व॑वेनद॒मी त्वा॑ जहति पुत्र दे॒वाः। अथा᳚ब्रवीद्वृ॒त्रमिन्द्रो॑ हनि॒ष्यन्थ्सखे॑ विष्णो वित॒रं वि क्र॑मस्व॥४६॥

%3.3.0.0

%3.3.0.0
{\anuvakamend[{इ॒षा\-ऽथ॑ त्वा॒ त्रयो॑दश च}]}%॥11॥

{\anuvakamend[{अग्ने॑ तेजस्विन्वा॒युर्वस॑वस्त्वै॒तद्वा अ॒पां वा॒युर॑सि प्रा॒णो नाम॑ दे॒वा वै यद्य॒ज्ञेन॒ न प्र॒जाप॑तिर्देवासु॒राना॑यु॒र्दा ए॒तं युवा॑न॒ꣳ॒ सूर्यो॑ दे॒व इ॒दं वा॒मेका॑दश}]}%॥11॥ अग्ने॑ तेजस्विन्वा॒युर॑सि॒ छन्द॑सां वी॒र्यं॑ मा॒तरं॑ च॒ षट्त्रिꣳ॑शत्॥36॥ अग्ने॑ तेजस्विꣴश्चिकि॒तुषे॑ दधातु॥
%%% END PRASHNA

\sect{तृतीयः प्रश्नः}\setcounter{anuvakam}{0}
\dnsub{तैत्तिरीयसंहितायां तृतीयकाण्डे तृतीयः प्रश्नः}
%3.3.1.0
%3.3.1.1
अग्ने॑ तेजस्विन्तेज॒स्वी त्वं दे॒वेषु॑ भूया॒स्तेज॑स्वन्त॒म्मामायु॑ष्मन्तं॒ वर्च॑स्वन्तम्मनु॒ष्ये॑षु कुरु दी॒क्षायै॑ च त्वा॒ तप॑सश्च॒ तेज॑से जुहोमि तेजो॒विद॑सि॒ तेजो॑ मा॒ मा हा॑सी॒न्मा\-ऽहं तेजो॑ हासिषं॒ मा मां तेजो॑ हासी॒दिन्द्रौ॑जस्विन्नोज॒स्वी त्वं दे॒वेषु॑ भूया॒ ओज॑स्वन्त॒म्मामायु॑ष्मन्तं॒ वर्च॑स्वन्तम्मनु॒ष्ये॑षु कुरु॒ ब्रह्म॑णश्च त्वा क्ष॒त्रस्य॑ च॥१॥

%3.3.1.2
ओज॑से जुहोम्योजो॒विद॒स्योजो॑ मा॒ मा हा॑सी॒न्माहमोजो॑ हासिषं॒ मा मामोजो॑ हासी॒थ्सूर्य॑ भ्राजस्विन्भ्राज॒स्वी त्वं दे॒वेषु॑ भूया॒ भ्राज॑स्वन्त॒म्मामायु॑ष्मन्तं॒ वर्च॑स्वन्तम्मनु॒ष्ये॑षु कुरु वा॒योश्च॑ त्वा॒\-ऽपां च॒ भ्राज॑से जुहोमि सुव॒र्विद॑सि॒ सुव॑र्मा॒ मा हा॑सी॒न्माहꣳ सुव॑र्\mbox{}हासिषं॒ मा माꣳ सुव॑र्\mbox{}हासी॒न्मयि॑ मे॒धाम्मयि॑ प्र॒जाम्मय्य॒ग्निस्तेजो॑ दधातु॒ मयि॑ मे॒धाम्मयि॑ प्र॒जाम्मयीन्द्र॑ इन्द्रि॒यं द॑धातु॒ मयि॑ मे॒धाम्मयि॑ प्र॒जाम्मयि॒ सूर्यो॒ भ्राजो॑ दधातु॥२॥

%3.3.2.0
{\anuvakamend[{क्ष॒त्रस्य॑ च॒ मयि॒ त्रयो॑विꣳशतिश्च}]}%॥१॥

%3.3.2.1
वा॒युर्\mbox{}हिं॑क॒र्ता\-ऽग्निः प्र॑स्तो॒ता प्र॒जाप॑तिः॒ साम॒ बृह॒स्पति॑रुद्गा॒ता विश्वे॑ दे॒वा उ॑पगा॒तारो॑ म॒रुतः॑ प्रतिह॒र्तार॒ इन्द्रो॑ नि॒धनं॒ ते दे॒वाः प्रा॑ण॒भृतः॑ प्रा॒णम्मयि॑ दधत्वे॒तद्वै सर्व॑मध्व॒र्युरु॑पाकु॒र्वन्नु॑द्गा॒तृभ्य॑ उ॒पाक॑रोति॒ ते दे॒वाः प्रा॑ण॒भृतः॑ प्रा॒णम्मयि॑ दध॒त्वित्या॑है॒तदे॒व सर्व॑मा॒त्मन्ध॑त्त॒ इडा॑ देव॒हूर्मनु॑र्यज्ञ॒नीर्बृह॒स्पति॑रुक्थाम॒दानि॑ शꣳसिष॒द्विश्वे॑ दे॒वाः॥३॥

%3.3.2.2
सू॒क्त॒वाच॒ः पृथि॑वि मात॒र्मा मा॑ हिꣳसी॒र्मधु॑ मनिष्ये॒ मधु॑ जनिष्ये॒ मधु॑ वक्ष्यामि॒ मधु॑ वदिष्यामि॒ मधु॑मतीं दे॒वेभ्यो॒ वाच॑मुद्यासꣳ शुश्रू॒षेण्या᳚म्मनु॒ष्ये᳚भ्य॒स्तम्मा॑ दे॒वा अ॑वन्तु शो॒भायै॑ पि॒तरो\-ऽनु॑ मदन्तु॥४॥

%3.3.3.0
{\anuvakamend[{श॒ꣳ॒सि॒ष॒द्विश्वे॑ दे॒वा अ॒ष्टाविꣳ॑शतिश्च}]}%॥२॥

%3.3.3.1
वस॑वस्त्वा॒ प्र वृ॑हन्तु गाय॒त्रेण॒ छन्द॑सा॒\-ऽग्नेः प्रि॒यम्पाथ॒ उपे॑हि रु॒द्रास्त्वा॒ प्र वृ॑हन्तु॒ त्रैष्टु॑भेन॒ छन्द॒सेन्द्र॑स्य प्रि॒यम्पाथ॒ उपे᳚ह्यादि॒त्यास्त्वा॒ प्र वृ॑हन्तु॒ जाग॑तेन॒ छन्द॑सा॒ विश्वे॑षां दे॒वानां᳚ प्रि॒यम्पाथ॒ उपे॑हि॒ मान्दा॑सु ते शुक्र शु॒क्रमा धू॑नोमि भ॒न्दना॑सु॒ कोत॑नासु॒ नूत॑नासु॒ रेशी॑षु॒ मेषी॑षु॒ वाशी॑षु विश्व॒भृथ्सु॒ माध्वी॑षु ककु॒हासु॒ शक्व॑रीषु॥५॥

%3.3.3.2
शु॒क्रासु॑ ते शुक्र शु॒क्रमा धू॑नोमि शु॒क्रं ते॑ शु॒क्रेण॑ गृह्णा॒म्यह्नो॑ रू॒पेण॒ सूर्य॑स्य र॒श्मिभिः॑। आ\-ऽस्मि॑न्नु॒ग्रा अ॑चुच्यवुर्दि॒वो धारा॑ असश्चत। क॒कु॒हꣳ रू॒पं वृ॑ष॒भस्य॑ रोचते बृ॒हथ्सोमः॒ सोम॑स्य पुरो॒गाः शु॒क्रः शु॒क्रस्य॑ पुरो॒गाः। यत्ते॑ सो॒मादा᳚भ्यं॒ नाम॒ जागृ॑वि॒ तस्मै॑ ते सोम॒ सोमा॑य॒ स्वाहो॒शिक्त्वं दे॑व सोम गाय॒त्रेण॒ छन्द॑सा॒\-ऽग्नेः॥६॥

%3.3.3.3
प्रि॒यम्पाथो॒ अपी॑हि व॒शी त्वं दे॑व सोम॒ त्रैष्टु॑भेन॒ छन्द॒सेन्द्र॑स्य प्रि॒यम्पाथो॒ अपी᳚ह्य॒स्मथ्स॑खा॒ त्वं दे॑व सोम॒ जाग॑तेन॒ छन्द॑सा॒ विश्वे॑षां दे॒वानां᳚ प्रि॒यम्पाथो॒ अपी॒ह्या नः॑ प्रा॒ण ए॑तु परा॒वत॒ आन्तरि॑क्षाद्दि॒वस्परि॑। आयुः॑ पृथि॒व्या अध्य॒मृत॑मसि प्रा॒णाय॑ त्वा। इ॒न्द्रा॒ग्नी मे॒ वर्चः॑ कृणुतां॒ वर्चः॒ सोमो॒ बृह॒स्पतिः॑। वर्चो॑ मे॒ विश्वे॑ दे॒वा वर्चो॑ मे धत्तमश्विना। द॒ध॒न्वे वा॒ यदी॒मनु॒ वोच॒द्ब्रह्मा॑णि॒ वेरु॒ तत्। परि॒ विश्वा॑नि॒ काव्या॑ ने॒मिश्च॒क्रमि॑वाभवत्॥७॥

%3.3.4.0
{\anuvakamend[{शक्व॑रीष्व॒ग्नेर्बृह॒स्पतिः॒ पञ्च॑विꣳशतिश्च}]}%॥३॥

%3.3.4.1
ए॒तद्वा अ॒पां ना॑म॒धेयं॒ गुह्यं॒ यदा॑धा॒वा मान्दा॑सु ते शुक्र शु॒क्रमा धू॑नो॒मीत्या॑हा॒पामे॒व ना॑म॒धेये॑न॒ गुह्ये॑न दि॒वो वृष्टि॒मव॑ रुन्द्धे शु॒क्रं ते॑ शु॒क्रेण॑ गृह्णा॒मीत्या॑है॒तद्वा अह्नो॑ रू॒पं यद्रात्रिः॒ सूर्य॑स्य र॒श्मयो॒ वृष्ट्या॑ ईश॒ते\-ऽह्न॑ ए॒व रू॒पेण॒ सूर्य॑स्य र॒श्मिभि॑र्दि॒वो वृ॑ष्टिं च्यावय॒त्या\-ऽस्मि॑न्नु॒ग्राः॥८॥

%3.3.4.2
अ॒चु॒च्य॒वु॒रित्या॑ह यथाय॒जुरे॒वैतत्क॑कु॒हꣳ रू॒पं वृ॑ष॒भस्य॑ रोचते बृ॒हदित्या॑है॒तद्वा अ॑स्य ककु॒हꣳ रू॒पं यद्वृष्टी॑ रू॒पेणै॒व वृष्टि॒मव॑ रुन्द्धे॒ यत्ते॑ सो॒मादा᳚भ्यं॒ नाम॒ जागृ॒वीत्या॑है॒ष ह॒ वै ह॒विषा॑ ह॒विर्य॑जति॒ यो\-ऽदा᳚भ्यं गृही॒त्वा सोमा॑य जु॒होति॒ परा॒ वा ए॒तस्यायुः॑ प्रा॒ण ए॑ति॥९॥

%3.3.4.3
यो\-ऽꣳ॑शुं गृ॒ह्णात्या नः॑ प्रा॒ण ए॑तु परा॒वत॒ इत्या॒हायु॑रे॒व प्रा॒णमा॒त्मन्ध॑त्ते॒\-ऽमृत॑मसि प्रा॒णाय॒ त्वेति॒ हिर॑ण्यम॒भि व्य॑नित्य॒मृतं॒ वै हिर॑ण्य॒मायुः॑ प्रा॒णो॑\-ऽमृते॑नै॒वायु॑रा॒त्मन्ध॑त्ते श॒तमा॑नम्भवति श॒तायुः॒ पुरु॑षः श॒तेन्द्रि॑य॒ आयु॑ष्ये॒वेन्द्रि॒ये प्रति॑ तिष्ठत्य॒प उप॑ स्पृशति भेष॒जं वा आपो॑ भेष॒जमे॒व कु॑रुते॥१०॥

%3.3.5.0
{\anuvakamend[{उ॒ग्रा ए॒त्याप॒स्त्रीणि॑ च}]}%॥४॥

%3.3.5.1
वा॒युर॑सि प्रा॒णो नाम॑ सवि॒तुराधि॑पत्ये\-ऽपा॒नं मे॑ दा॒श्चक्षु॑रसि॒ श्रोत्रं॒ नाम॑ धा॒तुराधि॑पत्य॒ आयु॑र्मे दा रू॒पम॑सि॒ वर्णो॒ नाम॒ बृह॒स्पते॒राधि॑पत्ये प्र॒जां मे॑ दा ऋ॒तम॑सि स॒त्यं नामेन्द्र॒स्याधि॑पत्ये क्ष॒त्रं मे॑ दा भू॒तम॑सि॒ भव्यं॒ नाम॑ पितृ॒णामाधि॑पत्ये॒\-ऽपामोष॑धीनां॒ गर्भं॑ धा ऋ॒तस्य॑ त्वा॒ व्यो॑मन ऋ॒तस्य॑॥११॥

%3.3.5.2
त्वा॒ विभू॑मन ऋ॒तस्य॑ त्वा॒ विध॑र्मण ऋ॒तस्य॑ त्वा स॒त्याय॒र्तस्य॑ त्वा॒ ज्योति॑षे प्र॒जाप॑तिर्वि॒राज॑मपश्य॒त्तया॑ भू॒तं च॒ भव्यं॑ चासृजत॒ तामृषि॑भ्यस्ति॒रो॑\-ऽदधा॒त्तां ज॒मद॑ग्नि॒स्तप॑सा\-ऽपश्य॒त्तया॒ वै स पृश्नी॒न्कामा॑नसृजत॒ तत्पृ॑श्नीनां पृश्नि॒त्वम् यत्पृश्न॑यो गृ॒ह्यन्ते॒ पृश्नी॑ने॒व तैः कामा॒न् यज॑मा॒नो\-ऽव॑ रुन्द्धे वा॒युर॑सि प्रा॒णः॥१२॥

%3.3.5.3
नामेत्या॑ह प्राणापा॒नावे॒वाव॑ रुन्द्धे॒ चक्षु॑रसि॒ श्रोत्रं॒ नामेत्या॒हायु॑रे॒वाव॑ रुन्द्धे रू॒पम॑सि॒ वर्णो॒ नामेत्या॑ह प्र॒जामे॒वाव॑ रुन्द्ध ऋ॒तम॑सि स॒त्यं नामेत्या॑ह क्ष॒त्रमे॒वाव॑ रुन्द्धे भू॒तम॑सि॒ भव्यं॒ नामेत्या॑ह प॒शवो॒ वा अ॒पामोष॑धीनां॒ गर्भः॑ प॒शूने॒व॥१३॥

%3.3.5.4
अव॑ रुन्द्ध ए॒ताव॒द्वै पुरु॑षम्प॒रित॒स्तदे॒वाव॑ रुन्द्ध ऋ॒तस्य॑ त्वा॒ व्यो॑मन॒ इत्या॑हे॒यं वा ऋ॒तस्य॒ व्यो॑मे॒मामे॒वाभि ज॑यत्यृ॒तस्य॑ त्वा॒ विभू॑मन॒ इत्या॑हा॒न्तरि॑क्षं॒ वा ऋ॒तस्य॒ विभू॑मा॒न्तरि॑क्षमे॒वाभि ज॑यत्यृ॒तस्य॑ त्वा॒ विध॑र्मण॒ इत्या॑ह॒ द्यौर्वा ऋ॒तस्य॒ विध॑र्म॒ दिव॑मे॒वाभि ज॑यत्यृ॒तस्य॑॥१४॥

%3.3.5.5
त्वा॒ स॒त्यायेत्या॑ह॒ दिशो॒ वा ऋ॒तस्य॑ स॒त्यं दिश॑ ए॒वाभि ज॑यत्यृ॒तस्य॑ त्वा॒ ज्योति॑ष॒ इत्या॑ह सुव॒र्गो वै लो॒क ऋ॒तस्य॒ ज्योतिः॑ सुव॒र्गमे॒व लो॒कम॒भि ज॑यत्ये॒ताव॑न्तो॒ वै दे॑वलो॒कास्ताने॒वाभि ज॑यति दश॒ सम्प॑द्यन्ते॒ दशा᳚क्षरा वि॒राडन्नं॑ वि॒राड्वि॒राज्ये॒वान्नाद्ये॒ प्रति॑ तिष्ठति॥१५॥

%3.3.6.0
{\anuvakamend[{व्यो॑मन ऋ॒तस्य॑ प्रा॒णः प॒शूने॒व विध॑र्म॒ दिव॑मे॒वाभि ज॑यत्यृ॒तस्य॒ षट्च॑त्वारिꣳशच्च}]}%॥५॥

%3.3.6.1
दे॒वा वै यद्य॒ज्ञेन॒ नावारु॑न्धत॒ तत्परै॒रवा॑रुन्धत॒ तत्परा॑णां पर॒त्वम् यत्परे॑ गृ॒ह्यन्ते॒ यदे॒व य॒ज्ञेन॒ नाव॑रु॒न्द्धे तस्याव॑रुद्ध्यै॒ यम्प्र॑थ॒मं गृ॒ह्णाती॒ममे॒व तेन॑ लो॒कम॒भि ज॑यति॒ यं द्वि॒तीय॑म॒न्तरि॑क्षं॒ तेन॒ यं तृ॒तीय॑म॒मुमे॒व तेन॑ लो॒कम॒भि ज॑यति यदे॒ते गृ॒ह्यन्त॑ ए॒षां लो॒काना॑म॒भिजि॑त्यै॥१६॥

%3.3.6.2
उत्त॑रे॒ष्वहः॑स्व॒मुतो॒\-ऽर्वाञ्चो॑ गृह्यन्ते\-ऽभि॒जित्यै॒वेमाल्लोँ॒कान्पुन॑रि॒मं लो॒कम्प्र॒त्यव॑रोहन्ति॒ यत्पूर्वे॒ष्वहः॑स्वि॒तः परा᳚ञ्चो गृ॒ह्यन्ते॒ तस्मा॑दि॒तः परा᳚ञ्च इ॒मे लो॒का यदुत्त॑रे॒ष्वहः॑स्व॒मुतो॒\-ऽर्वाञ्चो॑ गृ॒ह्यन्ते॒ तस्मा॑द॒मुतो॒\-ऽर्वां च॑ इ॒मे लो॒कास्तस्मा॒द\-या॑तयाम्नो लो॒कान्म॑नु॒ष्या॑ उप॑ जीवन्ति ब्रह्मवा॒दिनो॑ वदन्ति॒ कस्मा᳚थ्स॒त्याद॒द्भ्य ओष॑धयः॒ सम्भ॑व॒न्त्योष॑धयः॥१७॥

%3.3.6.3
म॒नु॒ष्या॑णा॒मन्नं॑ प्र॒जाप॑तिं प्र॒जा अनु॒ प्र जा॑यन्त॒ इति॒ परा॒नन्विति॑ ब्रूया॒द्यद्गृ॒ह्णात्य॒द्भ्यस्त्वौष॑धीभ्यो गृह्णा॒मीति॒ तस्मा॑द॒द्भ्य ओष॑धयः॒ सम्भ॑वन्ति॒ यद्गृ॒ह्णात्योष॑धीभ्यस्त्वा प्र॒जाभ्यो॑ गृह्णा॒मीति॒ तस्मा॒दोष॑धयो मनु॒ष्या॑णा॒मन्न॒म् यद्गृ॒ह्णाति॑ प्र॒जाभ्य॑स्त्वा प्र॒जाप॑तये गृह्णा॒मीति॒ तस्मा᳚त्प्र॒जाप॑तिं प्र॒जा अनु॒ प्र जा॑यन्ते॥१८॥

%3.3.7.0
{\anuvakamend[{अ॒भिजि॑त्या॒ ओष॑धयो॒\-ऽष्टाच॑त्वारिꣳशच्च}]}%॥६॥

%3.3.7.1
प्र॒जाप॑तिर्देवासु॒रान॑सृजत॒ तदनु॑ य॒ज्ञो॑\-ऽसृज्यत य॒ज्ञं छन्दाꣳ॑सि॒ ते विष्व॑ञ्चो॒ व्य॑क्राम॒न्थ्सो\-ऽसु॑रा॒ननु॑ य॒ज्ञो\-ऽपा᳚क्रामद्य॒ज्ञं छन्दाꣳ॑सि॒ ते दे॒वा अ॑मन्यन्ता॒मी वा इ॒दम॑भूव॒न् यद्व॒यꣴ स्म इति॒ ते प्र॒जाप॑ति॒मुपा॑धाव॒न्थ्सो᳚\-ऽब्रवीत्प्र॒जाप॑ति॒श्छन्द॑सां वी॒र्य॑मा॒दाय॒ तद्वः॒ प्र दा᳚स्या॒मीति॒ स छन्द॑सां वी॒र्यम्᳚॥१९॥

%3.3.7.2
आ॒दाय॒ तदे᳚भ्यः॒ प्राय॑च्छ॒त्तदनु॒ छन्दा॒ꣳ॒स्यपा᳚क्राम॒ञ्छन्दाꣳ॑सि य॒ज्ञस्ततो॑ दे॒वा अभ॑व॒न्परासु॑रा॒ य ए॒वं छन्द॑सां वी॒र्यं॑ वेदा श्रा॑व॒यास्तु॒ श्रौष॒ड्यज॒ ये यजा॑महे वषट्का॒रो भव॑त्या॒त्मना॒ परा᳚\-ऽस्य॒ भ्रातृ॑व्यो भवति ब्रह्मवा॒दिनो॑ वदन्ति॒ कस्मै॒ कम॑ध्व॒र्युरा श्रा॑वय॒तीति॒ छन्द॑सां वी॒र्या॑येति॑ ब्रूयादे॒तद्वै॥२०॥

%3.3.7.3
छन्द॑सां वी॒र्य॑मा श्रा॑व॒यास्तु॒ श्रौष॒ड्यज॒ ये यजा॑महे वषट्का॒रो य ए॒वं वेद॒ सवी᳚र्यैरे॒व छन्दो॑भिरर्चति॒ यत्किं चार्च॑ति॒ यदिन्द्रो॑ वृ॒त्रमह॑न्नमे॒ध्यं तद्यद्यती॑न॒पाव॑पदमे॒ध्यं तदथ॒ कस्मा॑दै॒न्द्रो य॒ज्ञ आ सꣴस्था॑तो॒रित्या॑हु॒रिन्द्र॑स्य॒ वा ए॒षा य॒ज्ञिया॑ त॒नूर्यद्य॒ज्ञस्तामे॒व तद्य॑जन्ति॒ य ए॒वं वेदोपै॑नं य॒ज्ञो न॑मति॥२१॥

%3.3.8.0
{\anuvakamend[{स छन्द॑सां वी॒र्यं॑ वा ए॒व तद॒ष्टौ च॑}]}%॥७॥

%3.3.8.1
आ॒यु॒र्दा अ॑ग्ने ह॒विषो॑ जुषा॒णो घृ॒तप्र॑तीको घृ॒तयो॑निरेधि। घृ॒तम्पी॒त्वा मधु॒ चारु॒ गव्यं॑ पि॒तेव॑ पु॒त्रम॒भि र॑क्षतादि॒मम्। आ वृ॑श्च्यते॒ वा ए॒तद्यज॑मानो॒\-ऽग्निभ्यां॒ यदे॑नयोः शृतं॒कृत्याथा॒न्यत्रा॑वभृ॒थम॒वैत्या॑यु॒र्दा अ॑ग्ने ह॒विषो॑ जुषा॒ण इत्य॑वभृ॒थम॑वै॒ष्यञ्जु॑हुया॒दाहु॑त्यै॒वैनौ॑ शमयति॒ नार्ति॒मार्च्छ॑ति॒ यज॑मानो॒ यत्कुसी॑दम्॥२२॥

%3.3.8.2
अप्र॑तीत्त॒म्मयि॒ येन॑ य॒मस्य॑ ब॒लिना॒ चरा॑मि। इ॒हैव सन्नि॒रव॑दये॒ तदे॒तत्तद॑ग्ने अनृ॒णो भ॑वामि। विश्व॑लोप विश्वदा॒वस्य॑ त्वा॒सञ्जु॑होम्य॒ग्धादेको॑\-ऽहु॒तादेकः॑ समस॒नादेकः॑। ते नः॑ कृण्वन्तु भेष॒जꣳ सदः॒ सहो॒ वरे᳚ण्यम्। अ॒यं नो॒ नभ॑सा पु॒रः स॒ꣴ॒स्फानो॑ अ॒भि र॑क्षतु। गृ॒हाणा॒मस॑मर्त्यै ब॒हवो॑ नो गृ॒हा अ॑सन्न्। स त्वं नः॑॥२३॥

%3.3.8.3
न॒भ॒स॒स्प॒त॒ ऊर्जं॑ नो धेहि भ॒द्रया᳚। पुन॑र्नो न॒ष्टमा कृ॑धि॒ पुन॑र्नो र॒यिमा कृ॑धि। देव॑ सꣴस्फान सहस्रपो॒षस्ये॑शिषे॒ स नो॑ रा॒स्वाज्या॑निꣳ रा॒यस्पोषꣳ॑ सु॒वीर्यꣳ॑ संवथ्स॒रीणाꣴ॑ स्व॒स्तिम्। अ॒ग्निर्वाव य॒म इ॒यं य॒मी कुसी॑दं॒ वा ए॒तद्य॒मस्य॒ यज॑मान॒ आ द॑त्ते॒ यदोष॑धीभि॒र्वेदिꣴ॑ स्तृ॒णाति॒ यदनु॑पौष्य प्रया॒याद्ग्री॑वब॒द्धमे॑नम्॥२४॥

%3.3.8.4
अ॒मुष्मि॑ल्लोँ॒के ने॑नीयेर॒न् यत्कुसी॑द॒मप्र॑तीत्त॒म्मयीत्युपौ॑षती॒हैव सन् य॒मं कुसी॑दं निरव॒दाया॑नृ॒णः सु॑व॒र्गं लो॒कमे॑ति॒ यदि॑ मि॒श्रमि॑व॒ चरे॑दञ्ज॒लिना॒ सक्तू᳚न्प्रदा॒व्ये॑ जुहुयादे॒ष वा अ॒ग्निर्वै᳚श्वान॒रो यत्प्र॑दा॒व्यः॑ स ए॒वैनꣴ॑ स्वदय॒त्यह्नां᳚ वि॒धान्या॑मेकाष्ट॒काया॑मपू॒पं चतुः॑शरावम्प॒क्त्वा प्रा॒तरे॒तेन॒ कक्ष॒मुपौ॑षे॒द्यदि॑॥२५॥

%3.3.8.5
दह॑ति पुण्य॒सम॑म्भवति॒ यदि॒ न दह॑ति पाप॒सम॑मे॒तेन॑ ह स्म॒ वा ऋ॑षयः पु॒रा वि॒ज्ञाने॑न दीर्घस॒त्त्रमुप॑ यन्ति॒ यो वा उ॑पद्र॒ष्टार॑मुपश्रो॒तार॑मनुख्या॒तारं॑ वि॒द्वान् यज॑ते॒ सम॒मुष्मि॑ल्लोँ॒क इ॑ष्टापू॒र्तेन॑ गच्छते॒\-ऽग्निर्वा उ॑पद्र॒ष्टा वा॒युरु॑पश्रो॒ता\-ऽ\-ऽ\-दि॒त्यो॑\-ऽनुख्या॒ता तान् य ए॒वं वि॒द्वान् यज॑ते॒ सम॒मुष्मि॑ल्लोँ॒क इ॑ष्टापू॒र्तेन॑ गच्छते॒\-ऽयं नो॒ नभ॑सा पु॒रः॥२६॥

%3.3.8.6
इत्या॑हा॒\-ऽग्निर्वै नभ॑सा पु॒रो᳚\-ऽग्निमे॒व तदा॑है॒तन्मे॑ गोपा॒येति॒ स त्वं नो॑ नभसस्पत॒ इत्या॑ह वा॒युर्वै नभ॑स॒स्पति॑र्वा॒युमे॒व तदा॑है॒तन्मे॑ गोपा॒येति॒ देव॑ सꣴस्फा॒नेत्या॑हा॒सौ वा आ॑दि॒त्यो दे॒वः स॒ꣴ॒स्फान॑ आदि॒त्यमे॒व तदा॑है॒तन्मे॑ गोपा॒येति॑॥२७॥

%3.3.9.0
{\anuvakamend[{कुसी॑द॒न्त्वन्न॑ एनमोषे॒द्यदि॑ पु॒र आ॑दि॒त्यमे॒व तदा॑है॒तन्मे॑ गोपा॒येति॑}]}%॥८॥

%3.3.9.1
ए॒तं युवा॑नं॒ परि॑ वो ददामि॒ तेन॒ क्रीड॑न्तीश्चरत प्रि॒येण॑। मा नः॑ शाप्त ज॒नुषा॑ सुभागा रा॒यस्पोषे॑ण॒ समि॒षा म॑देम। नमो॑ महि॒म्न उ॒त चक्षु॑षे ते॒ मरु॑ताम्पित॒स्तद॒हं गृ॑णामि। अनु॑ मन्यस्व सु॒यजा॑ यजाम॒ जुष्टं॑ दे॒वाना॑मि॒दम॑स्तु ह॒व्यम्। दे॒वाना॑मे॒ष उ॑पना॒ह आ॑सीद॒पां गर्भ॒ ओष॑धीषु॒ न्य॑क्तः। सोम॑स्य द्र॒फ्सम॑वृणीत पू॒षा॥२८॥

%3.3.9.2
बृ॒हन्नद्रि॑रभव॒त्तदे॑षाम्। पि॒ता व॒थ्साना॒म्पति॑रघ्नि॒याना॒मथो॑ पि॒ता म॑ह॒तां गर्ग॑राणाम्। व॒थ्सो ज॒रायु॑ प्रति॒धुक्पी॒यूष॑ आ॒मिक्षा॒ मस्तु॑ घृ॒तम॑स्य॒ रेतः॑। त्वां गावो॑\-ऽवृणत रा॒ज्याय॒ त्वाꣳ ह॑वन्त म॒रुतः॑ स्व॒र्काः। वर्ष्म॑न्क्ष॒त्रस्य॑ क॒कुभि॑ शिश्रिया॒णस्ततो॑ न उ॒ग्रो वि भ॑जा॒ वसू॑नि। व्यृ॑द्धेन॒ वा ए॒ष प॒शुना॑ यजते॒ यस्यै॒तानि॒ न क्रि॒यन्त॑ ए॒ष ह॒ त्वै समृ॑द्धेन यजते॒ यस्यै॒तानि॑ क्रि॒यन्ते᳚॥२९॥

%3.3.10.0
{\anuvakamend[{पू॒षा क्रि॒यन्त॑ ए॒षो᳚\-ऽष्टौ च॑}]}%॥९॥

%3.3.10.1
सूर्यो॑ दे॒वो दि॑वि॒षद्भ्यो॑ धा॒ता क्ष॒त्राय॑ वा॒युः प्र॒जाभ्यः॑। बृह॒स्पति॑स्त्वा प्र॒जाप॑तये॒ ज्योति॑ष्मतीं जुहोतु। यस्या᳚स्ते॒ हरि॑तो॒ गर्भो\-ऽथो॒ योनि॑र्\mbox{}हिर॒ण्ययी᳚। अङ्गा॒न्यह्रु॑ता॒ यस्यै॒ तां दे॒वैः सम॑जीगमम्। आ व॑र्तन वर्तय॒ नि नि॑वर्तन वर्त॒येन्द्र॑ नर्दबुद। भूम्या॒श्चत॑स्रः प्र॒दिश॒स्ताभि॒रा व॑र्तया॒ पुनः॑। वि ते॑ भिनद्मि तक॒रीं वि योनिं॒ वि ग॑वी॒न्यौ᳚। वि॥३०॥

%3.3.10.2
मा॒तरं॑ च पु॒त्रं च॒ वि गर्भं॑ च ज॒रायु॑ च। ब॒हिस्ते॑ अस्तु॒ बालिति॑। उ॒रु॒द्र॒फ्सो वि॒श्वरू॑प॒ इन्दुः॒ पव॑मानो॒ धीर॑ आनञ्ज॒ गर्भम्᳚। एक॑पदी द्वि॒पदी᳚ त्रि॒पदी॒ चतु॑ष्पदी॒ पञ्च॑पदी॒ षट्प॑दी स॒प्तप॑द्य॒ष्टाप॑दी॒ भुव॒नानु॑ प्रथता॒ꣴ॒ स्वाहा᳚। म॒ही द्यौः पृ॑थि॒वी च॑ न इ॒मं य॒ज्ञम्मि॑मिक्षताम्। पि॒पृ॒तां नो॒ भरी॑मभिः॥३१॥

%3.3.11.0
{\anuvakamend[{ग॒वी॒न्यौ॑ वि चतु॑श्चत्वारिꣳशच्च}]}%॥10॥

%3.3.11.1
इ॒दं वा॑मा॒स्ये॑ ह॒विः प्रि॒यमि॑न्द्राबृहस्पती। उ॒क्थम्मद॑श्च शस्यते। अ॒यं वां॒ परि॑ षिच्यते॒ सोम॑ इन्द्राबृहस्पती। चारु॒र्मदा॑य पी॒तये᳚। अ॒स्मे इ॑न्द्राबृहस्पती र॒यिं ध॑त्तꣳ शत॒ग्विनम्᳚। अश्वा॑वन्तꣳ सह॒स्रिणम्᳚। बृह॒स्पति॑र्नः॒ परि॑ पातु प॒श्चादु॒तोत्त॑रस्मा॒दध॑रादघा॒योः। इन्द्रः॑ पु॒रस्ता॑दु॒त म॑ध्य॒तो नः॒ सखा॒ सखि॑भ्यो॒ वरि॑वः कृणोतु। वि ते॒ विष्व॒ग्वात॑जूतासो अग्ने॒ भामा॑सः॥३२॥

%3.3.11.2
शु॒चे॒ शुच॑यश्चरन्ति। तु॒वि॒म्र॒क्षासो॑ दि॒व्या नव॑ग्वा॒ वना॑ वनन्ति धृष॒ता रु॒जन्तः॑। त्वाम॑ग्ने॒ मानु॑षीरीडते॒ विशो॑ होत्रा॒विदं॒ विवि॑चिꣳ रत्न॒धात॑मम्। गुहा॒ सन्तꣳ॑ सुभग वि॒श्वद॑र्शतं तुविष्म॒णसꣳ॑ सु॒यजं॑ घृत॒श्रियम्᳚। धा॒ता द॑दातु नो र॒यिमीशा॑नो॒ जग॑त॒स्पतिः॑। स नः॑ पू॒र्णेन॑ वावनत्। धा॒ता प्र॒जाया॑ उ॒त रा॒य ई॑शे धा॒तेदं विश्व॒म्भुव॑नं जजान। धा॒ता पु॒त्रं यज॑मानाय॒ दाता᳚॥३३॥

%3.3.11.3
तस्मा॑ उ ह॒व्यं घृ॒तव॑द्विधेम। धा॒ता द॑दातु नो र॒यिम्प्राचीं᳚ जी॒वातु॒मक्षि॑ताम्। व॒यं दे॒वस्य॑ धीमहि सुम॒तिꣳ स॒त्यरा॑धसः। धा॒ता द॑दातु दा॒शुषे॒ वसू॑नि प्र॒जाका॑माय मी॒ढुषे॑ दुरो॒णे। तस्मै॑ दे॒वा अ॒मृताः॒ सं व्य॑यन्तां॒ विश्वे॑ दे॒वासो॒ अदि॑तिः स॒जोषाः᳚। अनु॑ नो॒\-ऽद्यानु॑मतिर्य॒ज्ञं दे॒वेषु॑ मन्यताम्। अ॒ग्निश्च॑ हव्य॒वाह॑नो॒ भव॑तां दा॒शुषे॒ मयः॑। अन्विद॑नुमते॒ त्वम्॥३४॥

%3.3.11.4
मन्या॑सै॒ शं च॑ नः कृधि। क्रत्वे॒ दक्षा॑य नो हिनु॒ प्र ण॒ आयूꣳ॑षि तारिषः। अनु॑ मन्यतामनु॒मन्य॑माना प्र॒जाव॑न्तꣳ र॒यिमक्षी॑यमाणम्। तस्यै॑ व॒यꣳ हेड॑सि॒ मापि॑ भूम॒ सा नो॑ दे॒वी सु॒हवा॒ शर्म॑ यच्छतु। यस्या॑मि॒दम्प्र॒दिशि॒ यद्वि॒रोच॒ते\-ऽनु॑मतिं॒ प्रति॑ भूषन्त्या॒यवः॑। यस्या॑ उ॒पस्थ॑ उ॒र्व॑न्तरि॑क्ष॒ꣳ॒ सा नो॑ दे॒वी सु॒हवा॒ शर्म॑ यच्छतु॥३५॥

%3.3.11.5
रा॒काम॒हꣳ सु॒हवाꣳ॑ सुष्टु॒ती हु॑वे शृ॒णोतु॑ नः सु॒भगा॒ बोध॑तु॒ त्मना᳚। सीव्य॒त्वपः॑ सू॒च्या\-ऽच्छि॑द्यमानया॒ ददा॑तु वी॒रꣳ श॒तदा॑यमु॒क्थ्यम्᳚। यास्ते॑ राके सुम॒तयः॑ सु॒पेश॑सो॒ याभि॒र्ददा॑सि दा॒शुषे॒ वसू॑नि। ताभि॑र्नो अ॒द्य सु॒मना॑ उ॒पाग॑हि सहस्रपो॒षꣳ सु॑भगे॒ ररा॑णा। सिनी॑वालि॒ या सु॑पा॒णिः। कु॒हूम॒हꣳ सु॒भगां᳚ विद्म॒नाप॑सम॒स्मिन् य॒ज्ञे सु॒हवां᳚ जोहवीमि। सा नो॑ ददातु॒ श्रव॑णम्पितृ॒णां तस्या᳚स्ते देवि ह॒विषा॑ विधेम। कु॒हूर्दे॒वाना॑म॒मृत॑स्य॒ पत्नी॒ हव्या॑ नो अ॒स्य ह॒विष॑श्चिकेतु। सं दा॒शुषे॑ कि॒रतु॒ भूरि॑ वा॒मꣳ रा॒यस्पोषं॑ चिकि॒तुषे॑ दधातु॥३६॥

%3.4.0.0
{\anuvakamend[{भामा॑सो॒ दाता॒ त्वम॒न्तरि॑क्ष॒ꣳ॒ सा नो॑ दे॒वी सु॒हवा॒ शर्म॑ यच्छतु॒ श्रव॑णं॒ चतु॑र्विꣳशतिश्च}]}%॥11॥

%3.4.0.0

{\anuvakamend[{वि वा ए॒तस्या वा॑यो इ॒मे वै चि॒त्तञ्चा॒ग्निर्भू॒तानां᳚ दे॒वा वा अ॑भ्याता॒नानृ॑ता॒षाड्रा॒ष्ट्रका॑माय॒ देवि॑का॒ वास्तो᳚ष्पते॒ त्वम॑ग्ने बृ॒हदेका॑दश}]}%॥11॥
\prashnaend{ वि वा ए॒तस्येत्या॑ह मृ॒त्युर्ग॑न्ध॒र्वो\-ऽव॑ रुन्धे मध्य॒तस्त्वम॑ग्ने बृ॒हथ्षट्च॑त्वारिꣳशत्॥46॥ वि वा ए॒तस्य॑ प्रि॒यासः॑॥}
%%% END PRASHNA

\sect{चतुर्थः प्रश्नः}\setcounter{anuvakam}{0}
\dnsub{तैत्तिरीयसंहितायां तृतीयकाण्डे चतुर्थः प्रश्नः}
%3.4.1.0
%3.4.1.1
वि वा ए॒तस्य॑ य॒ज्ञ ऋ॑ध्यते॒ यस्य॑ ह॒विर॑ति॒रिच्य॑ते॒ सूर्यो॑ दे॒वो दि॑वि॒षद्भ्य॒ इत्या॑ह॒ बृह॒स्पति॑ना चै॒वास्य॑ प्र॒जाप॑तिना च य॒ज्ञस्य॒ व्यृ॑द्ध॒मपि॑ वपति॒ रक्षाꣳ॑सि॒ वा ए॒तत्प॒शुꣳ स॑चन्ते॒ यदे॑कदेव॒त्य॑ आल॑ब्धो॒ भूया॒न्भव॑ति॒ यस्या᳚स्ते॒ हरि॑तो॒ गर्भ॒ इत्या॑ह देव॒त्रैवैनां᳚ गमयति॒ रक्ष॑सा॒मप॑हत्या॒ आ व॑र्तन वर्त॒येत्या॑ह॥१॥

%3.4.1.2
ब्रह्म॑णै॒वैन॒मा व॑र्तयति॒ वि ते॑ भिनद्मि तक॒रीमित्या॑ह यथाय॒जुरे॒वैतदु॑रुद्र॒फ्सो वि॒श्वरू॑प॒ इन्दु॒रित्या॑ह प्र॒जा वै प॒शव॒ इन्दुः॑ प्र॒जयै॒वैन॑म्प॒शुभिः॒ सम॑र्धयति॒ दिवं॒ वै य॒ज्ञस्य॒ व्यृ॑द्धं गच्छति पृथि॒वीमति॑रिक्त॒न्तद्यन्न श॒मये॒दार्ति॒मार्च्छे॒द्यज॑मानो म॒ही द्यौः पृ॑थि॒वी च॑ न॒ इति॑॥२॥

%3.4.1.3
आ॒ह॒ द्यावा॑पृथि॒वीभ्या॑मे॒व य॒ज्ञस्य॒ व्यृ॑द्धं॒ चाति॑रिक्तं च शमयति॒ नार्ति॒मार्च्छ॑ति॒ यज॑मानो॒ भस्म॑ना॒भि समू॑हति स्व॒गाकृ॑त्या॒ अथो॑ अ॒नयो॒र्वा ए॒ष गर्भो॒\-ऽनयो॑रे॒वैनं॑ दधाति॒ यद॑व॒द्येदति॒ तद्रे॑चये॒द्यन्नाव॒द्येत्प॒शोराल॑ब्धस्य॒ नाव॑ द्येत् पु॒रस्ता॒न्नाभ्या॑ अ॒न्यद॑व॒द्येदु॒परि॑ष्टाद॒न्यत्पु॒रस्ता॒द्वै नाभ्यै᳚॥३॥

%3.4.1.4
प्रा॒ण उ॒परि॑ष्टादपा॒नो यावा॑ने॒व प॒शुस्तस्याव॑ द्यति॒ विष्ण॑वे शिपिवि॒ष्टाय॑ जुहोति॒ यद्वै य॒ज्ञस्या॑ति॒रिच्य॑ते॒ यः प॒शोर्भू॒मा या पुष्टि॒स्तद्विष्णुः॑ शिपिवि॒ष्टो\-ऽति॑रिक्त ए॒वाति॑रिक्तं दधा॒त्यति॑रिक्तस्य॒ शान्त्या॑ अ॒ष्टाप्रू॒ड्ढिर॑ण्यं॒ दक्षि॑णा॒\-ऽष्टाप॑दी॒ ह्ये॑षात्मा न॑व॒मः प॒शोराप्त्या॑ अन्तरको॒श उ॒ष्णीषे॒णावि॑ष्टितम्भवत्ये॒वमि॑व॒ हि प॒शुरुल्ब॑मिव॒ चर्मे॑व मा॒ꣳ॒समि॒वास्थी॑व॒ यावा॑ने॒व प॒शुस्तमा॒प्त्वाव॑ रुन्द्धे॒ यस्यै॒षा य॒ज्ञे प्राय॑श्चित्तिः क्रि॒यत॑ इ॒ष्ट्वा वसी॑यान्भवति॥४॥

%3.4.2.0
{\anuvakamend[{व॒र्त॒येत्या॑ह न॒ इति॒ वै नाभ्या॒ उल्ब॑मि॒वैक॑विꣳशतिश्च}]}%॥१॥

%3.4.2.1
आ वा॑यो भूष शुचिपा॒ उप॑ नः स॒हस्रं॑ ते नि॒युतो॑ विश्ववार। उपो॑ ते॒ अन्धो॒ मद्य॑मयामि॒ यस्य॑ देव दधि॒षे पू᳚र्व॒पेयम्᳚। आकू᳚त्यै त्वा॒ कामा॑य त्वा स॒मृधे᳚ त्वा किक्कि॒टा ते॒ मनः॑ प्र॒जाप॑तये॒ स्वाहा॑ किक्कि॒टा ते᳚ प्रा॒णं वा॒यवे॒ स्वाहा॑ किक्कि॒टा ते॒ चक्षुः॒ सूर्या॑य॒ स्वाहा॑ किक्कि॒टा ते॒ श्रोत्रं॒ द्यावा॑पृथि॒वीभ्या॒ꣴ॒ स्वाहा॑ किक्कि॒टा ते॒ वाच॒ꣳ॒ सर॑स्वत्यै॒ स्वाहा᳚॥५॥

%3.4.2.2
त्वं तु॒रीया॑ व॒शिनी॑ व॒शासि॑ स॒कृद्यत्त्वा॒ मन॑सा॒ गर्भ॒ आश॑यत्। व॒शा त्वं व॒शिनी॑ गच्छ दे॒वान्थ्स॒त्याः स॑न्तु॒ यज॑मानस्य॒ कामाः᳚। अ॒जासि॑ रयि॒ष्ठा पृ॑थि॒व्याꣳ सी॑दो॒र्ध्वान्तरि॑क्ष॒मुप॑ तिष्ठस्व दि॒वि ते॑ बृ॒हद्भाः। तन्तुं॑ त॒न्वन्रज॑सो भा॒नुमन्वि॑हि॒ ज्योति॑ष्मतः प॒थो र॑क्ष धि॒या कृ॒तान्। अ॒नु॒ल्ब॒णं व॑यत॒ जोगु॑वा॒मपो॒ मनु॑र्भव ज॒नया॒ दैव्यं॒ जनम्᳚। मन॑सो ह॒विर॑सि प्र॒जाप॑ते॒र्वर्णो॒ गात्रा॑णां ते गात्र॒भाजो॑ भूयास्म॥६॥

%3.4.3.0
{\anuvakamend[{सर॑स्वत्यै॒ स्वाहा॒ मनु॒स्त्रयो॑दश च}]}%॥२॥

%3.4.3.1
इ॒मे वै स॒हास्ता॒न्ते वा॒युर्व्य॑वा॒त्ते गर्भ॑मदधाता॒न्तꣳ सोमः॒ प्राज॑नयद॒ग्निर॑ग्रसत॒ स ए॒तं प्र॒जाप॑तिराग्ने॒यम॒ष्टाक॑पाल\-मपश्य॒त्तं निर॑वप॒त्तेनै॒वैना॑म॒ग्नेरधि॒ निर॑क्रीणा॒त्तस्मा॒दप्य॑न्यदेव॒त्या॑मा॒लभ॑मान आग्ने॒यम॒ष्टाक॑पालम्पु॒रस्ता॒न्निर्व॑पेद॒ग्ने\-रे॒वैना॒मधि॑ नि॒ष्क्रीया ल॑भते॒ यत्॥७॥

%3.4.3.2
वा॒युर्व्यवा॒त्तस्मा᳚द्वाय॒व्या॑ यदि॒मे गर्भ॒मद॑धातां॒ तस्मा᳚द्द्यावापृथि॒व्या॑ यथ्सोमः॒ प्राज॑नयद॒ग्निरग्र॑सत॒ तस्मा॑दग्नीषो॒मीया॒ यद॒नयो᳚र्विय॒त्योर्वागव॑द॒त्तस्मा᳚थ्सारस्व॒ती यत्प्र॒जाप॑तिर॒ग्नेरधि॑ नि॒रक्री॑णा॒त्तस्मा᳚त्प्राजाप॒त्या सा वा ए॒षा स॑र्वदेव॒त्या॑ यद॒जा व॒शा वा॑य॒व्या॑मा ल॑भेत॒ भूति॑कामो वा॒युर्वै क्षेपि॑ष्ठा दे॒वता॑ वा॒युमे॒व स्वेन॑॥८॥

%3.4.3.3
भा॒ग॒धेये॒नोप॑ धावति॒ स ए॒वैन॒म्भूतिं॑ गमयति द्यावापृथि॒व्या॑मा ल॑भेत कृ॒षमा॑णः प्रति॒ष्ठाका॑मो दि॒व ए॒वास्मै॑ प॒र्जन्यो॑ वर्\mbox{}षति॒ व्य॑स्यामोष॑धयो रोहन्ति स॒मर्धु॑कमस्य स॒स्यम्भ॑वत्यग्नीषो॒मीया॒मा ल॑भेत॒ यः का॒मये॒तान्न॑वानन्ना॒दः स्या॒मित्य॒ग्निनै॒वान्न॒मव॑ रुन्द्धे॒ सोमे॑ना॒न्नाद्य॒मन्न॑वाने॒वान्ना॒दो भ॑वति सारस्व॒तीमा ल॑भेत॒ यः॥९॥

%3.4.3.4
ई॒श्व॒रो वा॒चो वदि॑तोः॒ सन्वाचं॒ न वदे॒द्वाग्वै सर॑स्वती॒ सर॑स्वतीमे॒व स्वेन॑ भाग॒धेये॒नोप॑ धावति॒ सैवास्मि॒न्वाचं॑ दधाति प्राजाप॒त्यामा ल॑भेत॒ यः का॒मये॒तान॑भिजितम॒भि ज॑येय॒मिति॑ प्र॒जाप॑तिः॒ सर्वा॑ दे॒वता॑ दे॒वता॑भिरे॒वान॑भि\-जितम॒भि ज॑यति वाय॒व्य॑यो॒पाक॑रोति वा॒योरे॒वैना॑मव॒रुध्या ल॑भत॒ आकू᳚त्यै त्वा॒ कामा॑य त्वा॥१०॥

%3.4.3.5
इत्या॑ह यथाय॒जुरे॒वैतत्कि॑क्किटा॒कारं॑ जुहोति किक्किटाका॒रेण॒ वै ग्रा॒म्याः प॒शवो॑ रमन्ते॒ प्रार॒ण्याः प॑तन्ति॒ यत्कि॑क्किटा॒कारं॑ जु॒होति॑ ग्रा॒म्याणां᳚ पशू॒नां धृत्यै॒ पर्य॑ग्नौ क्रि॒यमा॑णे जुहोति॒ जीव॑न्तीमे॒वैनाꣳ॑ सुव॒र्गं लो॒कङ्ग॑मयति॒ त्वं तु॒रीया॑ व॒शिनी॑ व॒शासीत्या॑ह देव॒त्रैवैनां᳚ गमयति स॒त्याः स॑न्तु॒ यज॑मानस्य॒ कामा॒ इत्या॑है॒ष वै कामः॑॥११॥

%3.4.3.6
यज॑मानस्य॒ यदना᳚र्त उ॒दृचं॒ गच्छ॑ति॒ तस्मा॑दे॒वमा॑हा॒जासि॑ रयि॒ष्ठेत्या॑है॒ष्वे॑वैनां᳚ लो॒केषु॒ प्रति॑ ष्ठापयति दि॒वि ते॑ बृ॒हद्भा इत्या॑ह सुव॒र्ग ए॒वास्मै॑ लो॒के ज्योति॑र्दधाति॒ तन्तुं॑ त॒न्वन्रज॑सो भा॒नुमन्वि॒हीत्या॑हे॒माने॒वास्मै॑ लो॒कां ज्योति॑ष्मतः करोत्यनुल्ब॒णं व॑यत॒ जोगु॑वा॒मप॒ इति॑॥१२॥

%3.4.3.7
आ॒ह॒ यदे॒व य॒ज्ञ उ॒ल्बणं॑ क्रि॒यते॒ तस्यै॒वैषा शान्ति॒र्मनु॑र्भव ज॒नया॒ दैव्यं॒ जन॒मित्या॑ह मान॒व्यो॑ वै प्र॒जास्ता ए॒वाद्याः᳚ कुरुते॒ मन॑सो ह॒विर॒सीत्या॑ह स्व॒गाकृ॑त्यै॒ गात्रा॑णां ते गात्र॒भाजो॑ भूया॒स्मेत्या॑हा॒शिष॑मे॒वैतामा शा᳚स्ते॒ तस्यै॒ वा ए॒तस्या॒ एक॑मे॒वादे॑वयजनं॒ यदाल॑ब्धायाम॒भ्रः॥१३॥

%3.4.3.8
भव॑ति॒ यदाल॑ब्धायाम॒भ्रः स्याद॒फ्सु वा᳚ प्रवे॒शये॒थ्सर्वां᳚ वा॒ प्राश्नी॑या॒द्यद॒फ्सु प्र॑वे॒शये᳚द्यज्ञवेश॒सं कु॑र्या॒थ्सर्वा॑मे॒व प्राश्नी॑यादिन्द्रि॒यमे॒वात्मन्ध॑त्ते॒ सा वा ए॒षा त्र॑या॒णामे॒वाव॑रुद्धा संवथ्सर॒सदः॑ सहस्रया॒जिनो॑ गृहमे॒धिन॒स्त ए॒वैतया॑ यजेर॒न्तेषा॑मे॒वैषाप्ता॥14॥

%3.4.4.0
{\anuvakamend[{यथ्स्वेन॑ सारस्व॒तीमा ल॑भेत॒ यः कामा॑य त्वा॒ कामो\-ऽप॒ इत्य॒भ्रो द्विच॑त्वारिꣳशच्च}]}%॥३॥

%3.4.4.1
चि॒त्तं च॒ चित्ति॒श्चाकू॑तं॒ चाकू॑तिश्च॒ विज्ञा॑तं च वि॒ज्ञानं॑ च॒ मन॑श्च॒ शक्व॑रीश्च॒ दर्\mbox{}श॑श्च पू॒र्णमा॑सश्च बृ॒हच्च॑ रथन्त॒रं च॑ प्र॒जाप॑ति॒र्जया॒निन्द्रा॑य॒ वृष्णे॒ प्राय॑च्छदु॒ग्रः पृ॑त॒नाज्ये॑षु॒ तस्मै॒ विशः॒ सम॑नमन्त॒ सर्वाः॒ स उ॒ग्रः स हि हव्यो॑ ब॒भूव॑ देवासु॒राः संय॑त्ता आस॒न्थ्स इन्द्रः॑ प्र॒जाप॑ति॒मुपा॑धाव॒त्तस्मा॑ ए॒ताञ्जया॒न्प्राय॑च्छ॒त्तान॑जुहो॒त्ततो॒ वै दे॒वा असु॑रानजय॒न् य\-दज॑य॒न्तज्जया॑नां जय॒त्वꣴ स्पर्ध॑मानेनै॒ते हो॑त॒व्या॑ जय॑त्ये॒व तां पृत॑नाम्॥१५॥

%3.4.5.0
{\anuvakamend[{उप॒ पञ्च॑विꣳशतिश्च}]}%॥४॥

%3.4.5.1
अ॒ग्निर्भू॒ताना॒मधि॑पतिः॒ स मा॑व॒त्विन्द्रो᳚ ज्ये॒ष्ठानां᳚ य॒मः पृ॑थि॒व्या वा॒युर॒न्तरि॑क्षस्य॒ सूर्यो॑ दि॒वश्च॒न्द्रमा॒ नक्ष॑त्राणा॒म्बृह॒स्पति॒र्ब्रह्म॑णो मि॒त्रः स॒त्यानां॒ वरु॑णो॒\-ऽपाꣳ स॑मु॒द्रः स्रो॒त्याना॒मन्न॒ꣳ॒ साम्रा᳚ज्याना॒मधि॑पति॒ तन्मा॑वतु॒ सोम॒ ओष॑धीनाꣳ सवि॒ता प्र॑स॒वानाꣳ॑ रु॒द्रः प॑शू॒नां त्वष्टा॑ रू॒पाणां॒ विष्णुः॒ पर्व॑तानाम्म॒रुतो॑ ग॒णाना॒मधि॑पतय॒स्ते मा॑वन्तु॒ पित॑रः पितामहाः परे\-ऽवरे॒ तता᳚स्ततामहा इ॒ह मा॑वत। अ॒स्मिन्ब्रह्म॑न्न॒स्मिन्क्ष॒त्रे᳚\-ऽस्यामा॒शिष्य॒स्याम्पु॑रो॒धाया॑\-म॒स्मिन्कर्म॑न्न॒स्यां दे॒वहू᳚त्याम्॥१६॥

%3.4.6.0
{\anuvakamend[{अ॒व॒रे॒ स॒प्तद॑श च}]}%॥५॥

%3.4.6.1
दे॒वा वै यद्य॒ज्ञे\-ऽकु॑र्वत॒ तदसु॑रा अकुर्वत॒ ते दे॒वा ए॒तान॑भ्याता॒नान॑पश्य॒न्तान॒भ्यात॑न्वत॒ यद्दे॒वानां॒ कर्मासी॒दार्ध्य॑त॒ तद्यदसु॑राणां॒ न तदा᳚र्ध्यत॒ येन॒ कर्म॒णेर्थ्से॒त्तत्र॑ होत॒व्या॑ ऋ॒ध्नोत्ये॒व तेन॒ कर्म॑णा॒ यद्विश्वे॑ दे॒वाः स॒मभ॑र॒न्तस्मा॑दभ्याता॒ना वै᳚श्वदे॒वा यत्प्र॒जाप॑ति॒र्जया॒न्प्राय॑च्छ॒त्तस्मा॒ज्जयाः᳚ प्राजाप॒त्याः॥१७॥

%3.4.6.2
यद्रा᳚ष्ट्र॒भृद्भी॑ रा॒ष्ट्रमाद॑दत॒ तद्रा᳚ष्ट्र॒भृताꣳ॑ राष्ट्रभृ॒त्त्वन्ते दे॒वा अ॑भ्याता॒नैरसु॑रान॒भ्यात॑न्वत॒ जयै॑रजयन्राष्ट्र॒भृद्भी॑ रा॒ष्ट्रमाद॑दत॒ यद्दे॒वा अ॑भ्याता॒नैरसु॑रान॒भ्यात॑न्वत॒ तद॑भ्याता॒नाना॑मभ्यातान॒त्वं यज्जयै॒रज॑य॒न्तज्जया॑नां जय॒त्वं यद्रा᳚ष्ट्र॒भृद्भी॑ रा॒ष्ट्रमाद॑दत॒ तद्रा᳚ष्ट्र॒भृताꣳ॑ राष्ट्रभृ॒त्त्वन्ततो॑ दे॒वा अभ॑व॒न्परासु॑रा॒ यो भ्रातृ॑व्यवा॒न्थ्स्याथ्स ए॒ताञ्जु॑हुयादभ्याता॒नैरे॒व भ्रातृ॑व्यान॒भ्यात॑नुते॒ जयै᳚र्जयति राष्ट्र॒भृद्भी॑ रा॒ष्ट्रमा द॑त्ते॒ भव॑त्या॒त्मना॒ परा᳚स्य॒ भ्रातृ॑व्यो भवति॥१८॥

%3.4.7.0
{\anuvakamend[{प्रा॒जा॒प॒त्याः सो᳚\-ऽष्टाद॑श च}]}%॥६॥

%3.4.7.1
ऋ॒ता॒षाडृ॒तधा॑मा॒\-ऽग्निर्ग॑न्ध॒र्वस्तस्यौष॑धयो\-ऽफ्स॒रस॒ ऊर्जो॒ नाम॒ स इ॒दं ब्रह्म॑ क्ष॒त्रम्पा॑तु॒ ता इ॒दं ब्रह्म॑ क्ष॒त्रम्पा᳚न्तु॒ तस्मै॒ स्वाहा॒ ताभ्यः॒ स्वाहा॑ सꣳहि॒तो वि॒श्वसा॑मा॒ सूर्यो॑ गन्ध॒र्वस्तस्य॒ मरी॑चयो\-ऽफ्स॒रस॑ आ॒युवः॑ सुषु॒म्नः सूर्य॑रश्मिश्च॒न्द्रमा॑ गन्ध॒र्वस्तस्य॒ नक्ष॑त्राण्यफ्स॒रसो॑ बे॒कुर॑यो भु॒ज्युः सु॑प॒र्णो य॒ज्ञो ग॑न्ध॒र्वस्तस्य॒ दक्षि॑णा अफ्स॒रसः॑ स्त॒वाः प्र॒जाप॑तिर्वि॒श्वक॑र्मा॒ मनः॑॥१९॥

%3.4.7.2
ग॒न्ध॒र्वस्तस्य॑र्ख्सा॒मान्य॑फ्स॒रसो॒ वह्न॑य इषि॒रो वि॒श्वव्य॑चा॒ वातो॑ गन्ध॒र्वस्तस्यापो᳚\-ऽफ्स॒रसो॑ मु॒दा भुव॑नस्य पते॒ यस्य॑ त उ॒परि॑ गृ॒हा इ॒ह च॑। स नो॑ रा॒स्वाज्या॑निꣳ रा॒यस्पोषꣳ॑ सु॒वीर्यꣳ॑ संवथ्स॒रीणाꣴ॑ स्व॒स्तिम्। प॒र॒मे॒ष्ठ्यधि॑पति\-र्मृ॒त्युर्ग॑न्ध॒र्वस्तस्य॒ विश्व॑मफ्स॒रसो॒ भुवः॑ सुक्षि॒तिः सुभू॑तिर्भद्र॒कृथ्सुव॑र्वान्प॒र्जन्यो॑ गन्ध॒र्वस्तस्य॑ वि॒द्युतो᳚\-ऽफ्स॒रसो॒ रुचो॑ दू॒रेहे॑तिरमृड॒यः॥२०॥

%3.4.7.3
मृ॒त्युर्ग॑न्ध॒र्वस्तस्य॑ प्र॒जा अ॑फ्स॒रसो॑ भी॒रुव॒श्चारुः॑ कृपणका॒शी कामो॑ गन्ध॒र्वस्तस्या॒धयो᳚\-ऽफ्स॒रसः॑ शो॒चय॑न्ती॒र्नाम॒ स इ॒दं ब्रह्म॑ क्ष॒त्रम्पा॑तु॒ ता इ॒दं ब्रह्म॑ क्ष॒त्रम्पा᳚न्तु॒ तस्मै॒ स्वाहा॒ ताभ्यः॒ स्वाहा॒ स नो॑ भुवनस्य पते॒ यस्य॑ त उ॒परि॑ गृ॒हा इ॒ह च॑। उ॒रु ब्रह्म॑णे॒\-ऽस्मै क्ष॒त्राय॒ महि॒ शर्म॑ यच्छ॥२१॥

%3.4.8.0
{\anuvakamend[{मनो॑\-ऽमृड॒यष्षट्च॑त्वारिꣳशच्च}]}%॥७॥

%3.4.8.1
रा॒ष्ट्रका॑माय होत॒व्या॑ रा॒ष्ट्रं वै रा᳚ष्ट्र॒भृतो॑ रा॒ष्ट्रेणै॒वास्मै॑ रा॒ष्ट्रमव॑ रुन्द्धे रा॒ष्ट्रमे॒व भ॑वत्या॒त्मने॑ होत॒व्या॑ रा॒ष्ट्रं वै रा᳚ष्ट्र॒भृतो॑ रा॒ष्ट्रं प्र॒जा रा॒ष्ट्रम्प॒शवो॑ रा॒ष्ट्रं यच्छ्रेष्ठो॒ भव॑ति रा॒ष्ट्रेणै॒व रा॒ष्ट्रमव॑ रुन्द्धे॒ वसि॑ष्ठः समा॒नानां᳚ भवति॒ ग्राम॑कामाय होत॒व्या॑ रा॒ष्ट्रं वै रा᳚ष्ट्र॒भृतो॑ रा॒ष्ट्रꣳ स॑जा॒ता रा॒ष्ट्रेणै॒वास्मै॑ रा॒ष्ट्रꣳ स॑जा॒तानव॑ रुन्द्धे ग्रा॒मी॥२२॥

%3.4.8.2
ए॒व भ॑वत्यधि॒देव॑ने जुहोत्यधि॒देव॑न ए॒वास्मै॑ सजा॒तानव॑ रुन्द्धे॒ त ए॑न॒मव॑रुद्धा॒ उप॑ तिष्ठन्ते रथमु॒ख ओज॑स्कामस्य होत॒व्या॑ ओजो॒ वै रा᳚ष्ट्र॒भृत॒ ओजो॒ रथ॒ ओज॑सै॒वास्मा॒ ओजो\-ऽव॑ रुन्द्ध ओज॒स्व्ये॑व भ॑वति॒ यो रा॒ष्ट्रादप॑भूतः॒ स्यात्तस्मै॑ होत॒व्या॑ याव॑न्तो\-ऽस्य॒ रथाः॒ स्युस्तान्ब्रू॑याद्यु॒ङ्ग्ध्वमिति॑ रा॒ष्ट्रमे॒वास्मै॑ युनक्ति॥२३॥

%3.4.8.3
आहु॑तयो॒ वा ए॒तस्याकॢ॑प्ता॒ यस्य॑ रा॒ष्ट्रं न कल्प॑ते स्वर॒थस्य॒ दक्षि॑णं च॒क्रम्प्र॒वृह्य॑ ना॒डीम॒भि जु॑हुया॒दाहु॑तीरे॒वास्य॑ कल्पयति॒ ता अ॑स्य॒ कल्प॑माना रा॒ष्ट्रमनु॑ कल्पते सङ्ग्रा॒मे संय॑त्ते होत॒व्या॑ रा॒ष्ट्रं वै रा᳚ष्ट्र॒भृतो॑ रा॒ष्ट्रे खलु॒ वा ए॒ते व्याय॑च्छन्ते॒ ये सं॑ग्रा॒मꣳ सं॒यन्ति॒ यस्य॒ पूर्व॑स्य॒ जुह्व॑ति॒ स ए॒व भ॑वति॒ जय॑ति॒ तं सं॑ग्रा॒मं मा᳚न्धु॒क इ॒ध्मः॥२४॥

%3.4.8.4
भ॒व॒त्यङ्गा॑रा ए॒व प्र॑ति॒वेष्ट॑माना अ॒मित्रा॑णामस्य॒ सेनां॒ प्रति॑ वेष्टयन्ति॒ य उ॒न्माद्ये॒त्तस्मै॑ होत॒व्या॑ गन्धर्वाफ्स॒रसो॒ वा ए॒तमुन्मा॑दयन्ति॒ य उ॒न्माद्य॑त्ये॒ते खलु॒ वै ग॑न्धर्वाफ्स॒रसो॒ यद्रा᳚ष्ट्र॒भृत॒स्तस्मै॒ स्वाहा॒ ताभ्यः॒ स्वाहेति॑ जुहोति॒ तेनै॒वैना᳚ञ्छमयति॒ नैय॑ग्रोध॒ औदु॑म्बर॒ आश्व॑त्थः॒ प्लाक्ष॒ इती॒ध्मो भ॑वत्ये॒ते वै ग॑न्धर्वाफ्स॒रसां᳚ गृ॒हाः स्व ए॒वैनान्॑॥२५॥

%3.4.8.5
आ॒यत॑ने शमयत्यभि॒चर॑ता प्रतिलो॒मꣳ हो॑त॒व्याः᳚ प्रा॒णाने॒वास्य॑ प्र॒तीचः॒ प्रति॑ यौति॒ तं ततो॒ येन॒ केन॑ च स्तृणुते॒ स्वकृ॑त॒ इरि॑णे जुहोति प्रद॒रे वै॒तद्वा अ॒स्यै निर्\mbox{}ऋ॑तिगृहीतं॒ निर्\mbox{}ऋ॑तिगृहीत ए॒वैनं॒ निर्\mbox{}ऋ॑त्या ग्राहयति॒ यद्वा॒चः क्रू॒रन्तेन॒ वष॑ट्करोति वा॒च ए॒वैनं॑ क्रू॒रेण॒ प्र वृ॑श्चति ता॒जगार्ति॒मार्च्छ॑ति॒ यस्य॑ का॒मये॑ता॒न्नाद्यम्᳚॥२६॥

%3.4.8.6
आ द॑दी॒येति॒ तस्य॑ स॒भाया॑मुत्ता॒नो नि॒पद्य॒ भुव॑नस्य पत॒ इति॒ तृणा॑नि॒ सं गृ॑ह्णीयात्प्र॒जाप॑तिर्वै भुव॑नस्य॒ पतिः॑ प्र॒जाप॑तिनै॒वास्या॒न्नाद्य॒मा द॑त्त इ॒दम॒हम॒मुष्या॑मुष्याय॒णस्या॒न्नाद्यꣳ॑ हरा॒मीत्या॑हा॒न्नाद्य॑मे॒वास्य॑ हरति ष॒ड्भिर्\mbox{}ह॑रति॒ षड्वा ऋ॒तवः॑ प्र॒जाप॑तिनै॒वास्या॒न्नाद्य॑मा॒दाय॒र्तवो᳚\-ऽस्मा॒ अनु॒ प्र य॑च्छन्ति॥२७॥

%3.4.8.7
यो ज्ये॒ष्ठब॑न्धु॒रप॑भूतः॒ स्यात्तꣴ स्थले॑\-ऽव॒साय्य॑ ब्रह्मौद॒नं चतुः॑शरावम्प॒क्त्वा तस्मै॑ होत॒व्या॑ वर्ष्म॒ वै रा᳚ष्ट्र॒भृतो॒ वर्ष्म॒ स्थलं॒ वर्ष्म॑णै॒वैनं॒ वर्ष्म॑ समा॒नानां᳚ गमयति॒ चतुः॑शरावो भवति दि॒क्ष्वे॑व प्रति॑ तिष्ठति क्षी॒रे भ॑वति॒ रुच॑मे॒वास्मि॑\-न्दधा॒त्युद्ध॑रति शृत॒त्वाय॑ स॒र्पिष्वा᳚न्भवति मेध्य॒त्वाय॑ च॒त्वार॑ आर्\mbox{}षे॒याः प्राश्न॑न्ति दि॒शामे॒व ज्योति॑षि जुहोति॥२८॥

%3.4.9.0
{\anuvakamend[{ग्रा॒मी यु॑नक्ती॒ध्मः स्व ए॒वैना॑न॒न्नाद्यं॑ यच्छ॒न्त्येका॒न्नप॑ञ्चा॒शच्च॑}]}%॥८॥

%3.4.9.1
देवि॑का॒ निर्व॑पेत्प्र॒जाका॑म॒श्छन्दाꣳ॑सि॒ वै देवि॑का॒श्छन्दाꣳ॑सीव॒ खलु॒ वै प्र॒जाश्छन्दो॑भिरे॒वास्मै᳚ प्र॒जाः प्र ज॑नयति प्रथ॒मं धा॒तारं॑ करोति मिथु॒नी ए॒व तेन॑ करो॒त्यन्वे॒वास्मा॒ अनु॑मतिर्मन्यते रा॒ते रा॒का प्र सि॑नीवा॒ली ज॑नयति प्र॒जास्वे॒व प्रजा॑तासु कु॒ह्वा॑ वाचं॑ दधात्ये॒ता ए॒व निर्व॑पेत्प॒शुका॑म॒श्छन्दाꣳ॑सि॒ वै देवि॑का॒श्छन्दाꣳ॑सि॥२९॥

%3.4.9.2
इ॒व॒ खलु॒ वै प॒शव॒श्छन्दो॑भिरे॒वास्मै॑ प॒शून्प्र ज॑नयति प्रथ॒मं धा॒तारं॑ करोति॒ प्रैव तेन॑ वापय॒त्यन्वे॒वास्मा॒ अनु॑मतिर्मन्यते रा॒ते रा॒का प्र सि॑नीवा॒ली ज॑नयति प॒शूने॒व प्रजा॑तान्कु॒ह्वा᳚ प्रति॑ ष्ठापयत्ये॒ता ए॒व निर्व॑पे॒द्ग्राम॑काम॒श्छन्दाꣳ॑सि॒ वै देवि॑का॒श्छन्दाꣳ॑सीव॒ खलु॒ वै ग्राम॒श्छन्दो॑भिरे॒वास्मै॒ ग्रामम्᳚॥३०॥

%3.4.9.3
अव॑ रुन्द्धे मध्य॒तो धा॒तारं॑ करोति मध्य॒त ए॒वैनं॒ ग्राम॑स्य दधात्ये॒ता ए॒व निर्व॑पे॒ज्ज्योगा॑मयावी॒ छन्दाꣳ॑सि॒ वै देवि॑का॒श्छन्दाꣳ॑सि॒ खलु॒ वा ए॒तम॒भि म॑न्यन्ते॒ यस्य॒ ज्योगा॒मय॑ति॒ छन्दो॑भिरे॒वैन॑मग॒दं क॑रोति मध्य॒तो धा॒तारं॑ करोति मध्य॒तो वा ए॒तस्याकॢ॑प्तं॒ यस्य॒ ज्योगा॒मय॑ति मध्य॒त ए॒वास्य॒ तेन॑ कल्पयत्ये॒ता ए॒व निः॥३१॥

%3.4.9.4
व॒पे॒द्यं य॒ज्ञो नोप॒नमे॒च्छन्दाꣳ॑सि॒ वै देवि॑का॒श्छन्दाꣳ॑सि॒ खलु॒ वा ए॒तं नोप॑ नमन्ति॒ यं य॒ज्ञो नोप॒नम॑ति प्रथ॒मं धा॒तारं॑ करोति मुख॒त ए॒वास्मै॒ छन्दाꣳ॑सि दधा॒त्युपै॑नं य॒ज्ञो न॑मत्ये॒ता ए॒व निर्व॑पेदीजा॒नश्छन्दाꣳ॑सि॒ वै देवि॑का या॒तया॑मानीव॒ खलु॒ वा ए॒तस्य॒ छन्दाꣳ॑सि॒ य ई॑जा॒न उ॑त्त॒मं धा॒तारं॑ करोति॥३२॥

%3.4.9.5
उ॒परि॑ष्टादे॒वास्मै॒ छन्दा॒ꣳ॒स्यया॑तयामा॒न्यव॑ रुन्द्ध॒ उपै॑न॒मुत्त॑रो य॒ज्ञो न॑मत्ये॒ता ए॒व निर्व॑पे॒द्यम्मे॒धा नोप॒नमे॒च्छन्दाꣳ॑सि॒ वै देवि॑का॒श्छन्दाꣳ॑सि॒ खलु॒ वा ए॒तं नोप॑ नमन्ति॒ यम्मे॒धा नोप॒नम॑ति प्रथ॒मं धा॒तारं॑ करोति मुख॒त ए॒वास्मै॒ छन्दाꣳ॑सि दधा॒त्युपै॑नम्मे॒धा न॑मत्ये॒ता ए॒व निर्व॑पेत्॥३३॥

%3.4.9.6
रुक्का॑म॒श्छन्दाꣳ॑सि॒ वै देवि॑का॒श्छन्दाꣳ॑सीव॒ खलु॒ वै रुक्छन्दो॑भिरे॒वास्मि॒न्रुचं॑ दधाति क्षी॒रे भ॑वन्ति॒ रुच॑मे॒वास्मि॑न्दधति मध्य॒तो धा॒तारं॑ करोति मध्य॒त ए॒वैनꣳ॑ रु॒चो द॑धाति गाय॒त्री वा अनु॑मतिस्त्रि॒ष्टुग्रा॒का जग॑ती सिनीवाल्य॑नु॒ष्टुप्कु॒हूर्धा॒ता व॑षट्का॒रः पू᳚र्वप॒क्षो रा॒काप॑रप॒क्षः कु॒हूर॑मावा॒स्या॑ सिनीवा॒ली पौ᳚र्णमा॒स्यनु॑मतिश्च॒न्द्रमा॑ धा॒ता\-ऽष्टौ॥३४॥

%3.4.9.7
वस॑वो॒\-ऽष्टाक्ष॑रा गाय॒त्र्येका॑दश रु॒द्रा एका॑दशाक्षरा त्रि॒ष्टुब्द्वाद॑शादि॒त्या द्वाद॑शाक्षरा॒ जग॑ती प्र॒जाप॑तिरनु॒ष्टुब्धा॒ता व॑षट्का॒र ए॒तद्वै देवि॑काः॒ सर्वा॑णि च॒ छन्दाꣳ॑सि॒ सर्वा᳚श्च दे॒वता॑ वषट्का॒रस्ता यथ्स॒ह सर्वा॑ नि॒र्वपे॑दीश्व॒रा ए॑नम्प्र॒दहो॒ द्वे प्र॑थ॒मे नि॒रुप्य॑ धा॒तुस्तृ॒तीयं॒ निर्व॑पे॒त्तथो॑ ए॒वोत्त॑रे॒ निर्व॑पे॒त्तथै॑नं॒ न प्र द॑ह॒न्त्यथो॒ यस्मै॒ कामा॑य निरु॒प्यन्ते॒ तमे॒वाभि॒रुपा᳚प्नोति॥३५॥

%3.4.10.0
{\anuvakamend[{प॒शुका॑म॒श्छन्दाꣳ॑सि॒ वै देवि॑का॒श्छन्दाꣳ॑सि॒ ग्राम॑ङ्कल्पयत्ये॒ता ए॒व निरु॑त्त॒मन्धा॒तारं॑ करोति मे॒धा न॑मत्ये॒ता ए॒व निर्व॑पेद॒ष्टौ द॑हन्ति॒ नव॑ च॥९॥ देविकाः प्रजाकामो मिथुनी पशुकाम}]}

%3.4.10.1
वास्तो᳚ष्पते॒ प्रति॑ जानीह्य॒स्मान्थ्स्वा॑वे॒शो अ॑नमी॒वो भ॑वा नः। यत्त्वेम॑हे॒ प्रति॒ तन्नो॑ जुषस्व॒ शं न॑ एधि द्वि॒पदे॒ शं चतु॑ष्पदे। वास्तो᳚ष्पते श॒ग्मया॑ स॒ꣳ॒सदा॑ ते सक्षी॒महि॑ र॒ण्वया॑ गातु॒मत्या᳚। आवः॒ क्षेम॑ उ॒त योगे॒ वरं॑ नो यू॒यम्पा॑त स्व॒स्तिभिः॒ सदा॑ नः। यथ्सा॒यम्प्रा॑तरग्निहो॒त्रं जु॒होत्या॑हुतीष्ट॒का ए॒व ता उप॑ धत्ते॥३६॥

%3.4.10.2
यज॑मानो\-ऽहोरा॒त्राणि॒ वा ए॒तस्येष्ट॑का॒ य आहि॑ताग्नि॒र्यथ्सा॒यम्प्रा॑तर्जु॒होत्य॑होरा॒त्राण्ये॒वाप्त्वेष्ट॑काः कृ॒त्वोप॑ धत्ते॒ दश॑ समा॒नत्र॑ जुहोति॒ दशा᳚क्षरा वि॒राड्वि॒राज॑मे॒वाप्त्वेष्ट॑कां कृ॒त्वोप॑ ध॒त्ते\-ऽथो॑ वि॒राज्ये॒व य॒ज्ञमा᳚प्नोति॒ चित्य॑श्चित्यो\-ऽस्य भवति॒ तस्मा॒द्यत्र॒ दशो॑षि॒त्वा प्र॒याति॒ तद्य॑ज्ञवा॒स्त्ववा᳚स्त्वे॒व तद्यत्ततो᳚\-ऽर्वा॒चीनम्᳚॥३७॥

%3.4.10.3
रु॒द्रः खलु॒ वै वा᳚स्तोष्प॒तिर्यदहु॑त्वा वास्तोष्प॒तीय॑म्प्रया॒याद्रु॒द्र ए॑नम्भू॒त्वाग्निर॑नू॒त्थाय॑ हन्याद्वास्तोष्प॒तीयं॑ जुहोति भाग॒धेये॑नै॒वैनꣳ॑ शमयति॒ नार्ति॒मार्च्छ॑ति॒ यज॑मानो॒ यद्यु॒क्ते जु॑हु॒याद्यथा॒ प्रया॑ते॒ वास्ता॒वाहु॑तिं जु॒होति॑ ता॒दृगे॒व तद्यदयु॑क्ते जुहु॒याद्यथा॒ क्षेम॒ आहु॑तिं जु॒होति॑ ता॒दृगे॒व तदहु॑तमस्य वास्तोष्प॒तीयꣴ॑ स्यात्॥३८॥

%3.4.10.4
दक्षि॑णो यु॒क्तो भव॑ति स॒व्यो\-ऽयु॒क्तो\-ऽथ॑ वास्तोष्प॒तीयं॑ जुहोत्यु॒भय॑मे॒वाक॒रप॑रिवर्गमे॒वैनꣳ॑ शमयति॒ यदेक॑या जुहु॒याद्द॑र्विहो॒मं कु॑र्यात्पुरोनुवा॒क्या॑म॒नूच्य॑ या॒ज्य॑या जुहोति सदेव॒त्वाय॒ यद्धु॒त आ॑द॒ध्याद्रु॒द्रं गृ॒हान॒न्वारो॑हये॒द्यद॑व॒\-क्षाणा॒न्यस॑म्प्रक्षाप्य प्रया॒याद्यथा॑ यज्ञवेश॒सं वा॒दह॑नं वा ता॒दृगे॒व तद॒यं ते॒ योनि॑र्\mbox{}ऋ॒त्विय॒ इत्य॒रण्योः᳚ स॒मारो॑हयति॥३९॥

%3.4.10.5
ए॒ष वा अ॒ग्नेर्योनिः॒ स्व ए॒वैनं॒ योनौ॑ स॒मारो॑हय॒त्यथो॒ खल्वा॑हु॒र्यद॒रण्योः᳚ स॒मारू॑ढो॒ नश्ये॒दुद॑स्या॒ग्निः सी॑देत्पुनरा॒धेयः॑ स्या॒दिति॒ या ते॑ अग्ने य॒ज्ञिया॑ त॒नूस्तयेह्या रो॒हेत्या॒त्मन्थ्स॒मारो॑हयते॒ यज॑मानो॒ वा अ॒ग्नेर्योनिः॒ स्वाया॑मे॒वैनं॒ योन्याꣳ॑ स॒मारो॑हयते॥४०॥

%3.4.11.0
{\anuvakamend[{ध॒त्ते॒\-ऽर्वा॒चीनꣴ॑ स्याथ्स॒मारो॑हयति॒ पञ्च॑चत्वारिꣳशच्च}]}%॥10॥

%3.4.11.1
त्वम॑ग्ने बृ॒हद्वयो॒ दधा॑सि देव दा॒शुषे᳚। क॒विर्गृ॒हप॑ति॒र्युवा᳚॥ ह॒व्य॒वाड॒ग्निर॒जरः॑ पि॒ता नो॑ वि॒भुर्वि॒भावा॑ सु॒दृशी॑को अ॒स्मे। सु॒गा॒र्\mbox{}ह॒प॒त्याः समिषो॑ दिदीह्यस्म॒द्रिय॒ख्सम्मि॑मीहि॒ श्रवाꣳ॑सि। त्वं च॑ सोम नो॒ वशो॑ जी॒वातुं॒ न म॑रामहे। प्रि॒यस्तो᳚त्रो॒ वन॒स्पतिः॑। ब्र॒ह्मा दे॒वानां᳚ पद॒वीः क॑वी॒नामृषि॒र्विप्रा॑णाम्महि॒षो मृ॒गाणा᳚म्। श्ये॒नो गृ॑ध्राणा॒ꣴ॒ स्वधि॑ति॒र्वना॑ना॒ꣳ॒ सोमः॑॥४१॥

%3.4.11.2
प॒वित्र॒मत्ये॑ति॒ रेभन्न्॑। आ वि॒श्वदे॑व॒ꣳ॒ सत्प॑तिꣳ सू॒क्तैर॒द्या वृ॑णीमहे। स॒त्यस॑वꣳ सवि॒तारम्᳚॥ आ स॒त्येन॒ रज॑सा॒ वर्त॑मानो निवे॒शय॑न्न॒मृत॒म्मर्त्यं॑ च। हि॒र॒ण्यये॑न सवि॒ता रथे॒ना दे॒वो या॑ति॒ भुव॑ना वि॒पश्यन्न्॑। यथा॑ नो॒ अदि॑तिः॒ कर॒त्पश्वे॒ नृभ्यो॒ यथा॒ गवे᳚। यथा॑ तो॒काय॑ रु॒द्रियम्᳚। मा न॑स्तो॒के तन॑ये॒ मा न॒ आयु॑षि॒ मा नो॒ गोषु॒ मा॥४२॥

%3.4.11.3
नो॒ अश्वे॑षु रीरिषः। वी॒रान्मा नो॑ रुद्र भामि॒तो व॑धीर्\mbox{}ह॒विष्म॑न्तो॒ नम॑सा विधेम ते। उ॒द॒प्रुतो॒ न वयो॒ रक्ष॑माणा॒ वाव॑दतो अ॒भ्रिय॑स्येव॒ घोषाः᳚। गि॒रि॒भ्रजो॒ नोर्मयो॒ मद॑न्तो॒ बृह॒स्पति॑म॒भ्य॑र्का अ॑नावन्न्। ह॒ꣳ॒सैरि॑व॒ सखि॑भि॒र्वाव॑दद्भिरश्म॒न्मया॑नि॒ नह॑ना॒ व्यस्यन्न्॑। बृह॒स्पति॑रभि॒ कनि॑क्रद॒द्गा उ॒त प्रास्तौ॒दुच्च॑ वि॒द्वाꣳ अ॑गायत्। एन्द्र॑ सान॒सिꣳ र॒यिम्॥४३॥

%3.4.11.4
स॒जित्वा॑नꣳ सदा॒सहम्᳚। वर्\mbox{}षि॑ष्ठमू॒तये॑ भर। प्र स॑साहिषे पुरुहूत॒ शत्रू॒ञ्ज्येष्ठ॑स्ते॒ शुष्म॑ इ॒ह रा॒तिर॑स्तु। इन्द्रा भ॑र॒ दक्षि॑णेना॒ वसू॑नि॒ पतिः॒ सिन्धू॑नामसि रे॒वती॑नाम्। त्वꣳ सु॒तस्य॑ पी॒तये॑ स॒द्यो वृ॒द्धो अ॑जायथाः। इन्द्र॒ ज्यैष्ठ्या॑य सुक्रतो। भु॒वस्त्वमि॑न्द्र॒ ब्रह्म॑णा म॒हान्भुवो॒ विश्वे॑षु॒ सव॑नेषु य॒ज्ञियः॑। भुवो॒ नॄꣴश्च्यौ॒त्नो विश्व॑स्मि॒न्भरे॒ ज्येष्ठ॑श्च॒ मन्त्रः॑॥४४॥

%3.4.11.5
वि॒श्व॒च॒र्\mbox{}ष॒णे॒। मि॒त्रस्य॑ चर्\mbox{}षणी॒धृतः॒ श्रवो॑ दे॒वस्य॑ सान॒सिम्। स॒त्यं चि॒त्रश्र॑वस्तमम्। मि॒त्रो जनान्॑ यातयति प्रजा॒नन्मि॒त्रो दा॑धार पृथि॒वीमु॒त द्याम्। मि॒त्रः कृ॒ष्टीरनि॑मिषा॒भि च॑ष्टे स॒त्याय॑ ह॒व्यं घृ॒तव॑द्विधेम। प्र स मि॑त्र॒ मर्तो॑ अस्तु॒ प्रय॑स्वा॒न् यस्त॑ आदित्य॒ शिक्ष॑ति व्र॒तेन॑। न ह॑न्यते॒ न जी॑यते॒ त्वोतो॒ नैन॒मꣳहो॑ अश्नो॒त्यन्ति॑तो॒ न दू॒रात्। यत्॥४५॥

%3.4.11.6
चि॒द्धि ते॒ विशो॑ यथा॒ प्र दे॑व वरुण व्र॒तम्। मि॒नी॒मसि॒ द्यवि॑द्यवि। यत्किं चे॒दं व॑रुण॒ दैव्ये॒ जने॑\-ऽभिद्रो॒ह\-म्म॑नु॒ष्या᳚श्चरा॑मसि। अचि॑त्ती॒ यत्तव॒ धर्मा॑ युयोपि॒म मा न॒स्तस्मा॒देन॑सो देव रीरिषः। कि॒त॒वासो॒ यद्रि॑रि॒पुर्न दी॒वि यद्वा॑ घा स॒त्यमु॒त यन्न वि॒द्म। सर्वा॒ ता वि ष्य॑ शिथि॒रेव॑ दे॒वाथा॑ ते स्याम वरुण प्रि॒यासः॑॥४६॥

%3.5.0.0
{\anuvakamend[{सोमो॒ गोषु॒ मा र॒यिं मन्त्रो॒ यच्छि॑थि॒रा स॒प्त च॑}]}%॥11॥

%3.5.0.0

{\anuvakamend[{पू॒र्णर्\mbox{}ष॑यो॒\-ऽग्निना॒ ये दे॒वाः सूर्यो॑ मा॒ सन्त्वा॑ नह्यामि वषट्का॒रः स ख॑दि॒र उ॑पया॒मगृ॑हीतो\-ऽसि॒ यां वै त्वे क्रतु॒म्प्र दे॒वमेका॑दश}]}%॥11॥
\prashnaend{ पू॒र्णा स॑ह॒जान्तवा᳚ग्ने प्रा॒णैरे॒व षट्त्रिꣳ॑शत्॥36॥ पू॒र्णा सन्ति॑ दे॒वाः॥}
%%% END PRASHNA

\sect{पञ्चमः प्रश्नः}\setcounter{anuvakam}{0}
\dnsub{तैत्तिरीयसंहितायां तृतीयकाण्डे पञ्चमः प्रश्नः}
%3.5.1.0
%3.5.1.1
पू॒र्णा प॒श्चादु॒त पू॒र्णा पु॒रस्ता॒दुन्म॑ध्य॒तः पौ᳚र्णमा॒सी जि॑गाय। तस्यां᳚ दे॒वा अधि॑ सं॒वस॑न्त उत्त॒मे नाक॑ इ॒ह मा॑दयन्ताम्। यत्ते॑ दे॒वा अद॑धुर्भाग॒धेय॒ममा॑वास्ये सं॒वस॑न्तो महि॒त्वा। सा नो॑ य॒ज्ञम्पि॑पृहि विश्ववारे र॒यिं नो॑ धेहि सुभगे सु॒वीरम्᳚। नि॒वेश॑नी सं॒गम॑नी॒ वसू॑नां॒ विश्वा॑ रू॒पाणि॒ वसू᳚न्यावे॒शय॑न्ती। स॒ह॒स्र॒पो॒षꣳ सु॒भगा॒ ररा॑णा॒ सा न॒ आ ग॒न्वर्च॑सा॥१॥

%3.5.1.2
सं॒वि॒दा॒ना। अग्नी॑षोमौ प्रथ॒मौ वी॒र्ये॑ण॒ वसू᳚न्रु॒द्राना॑दि॒त्यानि॒ह जि॑न्वतम्। मा॒ध्यꣳ हि पौ᳚र्णमा॒सं जु॒षेथां॒ ब्रह्म॑णा वृ॒द्धौ सु॑कृ॒तेन॑ सा॒तावथा॒स्मभ्यꣳ॑ स॒हवी॑राꣳ र॒यिं नि य॑च्छतम्। आ॒दि॒त्याश्चाङ्गि॑रसश्चा॒ग्नीनाद॑धत॒ ते द॑र्\mbox{}शपूर्णमा॒सौ प्रैफ्स॒न्तेषा॒मङ्गि॑रसां॒ निरु॑प्तꣳ ह॒विरासी॒दथा॑दि॒त्या ए॒तौ होमा॑वपश्य॒न्ताव॑जुहवु॒स्ततो॒ वै ते द॑र्\mbox{}शपूर्णमा॒सौ॥२॥

%3.5.1.3
पूर्व॒ आल॑भन्त दर्\mbox{}शपूर्णमा॒सावा॒लभ॑मान ए॒तौ होमौ॑ पु॒रस्ता᳚ज्जुहुयाथ्सा॒क्षादे॒व द॑र्\mbox{}शपूर्णमा॒सावा ल॑भते ब्रह्मवा॒दिनो॑ वदन्ति॒ स त्वै द॑र्\mbox{}शपूर्णमा॒सावाल॑भेत॒ य ए॑नयोरनुलो॒मं च॑ प्रतिलो॒मं च॑ वि॒द्यादित्य॑मावा॒स्या॑या ऊ॒र्ध्वं तद॑नुलो॒म\-म्पौ᳚र्णमा॒स्यै प्र॑ती॒चीनं॒ तत्प्र॑तिलो॒मं यत्पौ᳚र्णमा॒सीम्पूर्वा॑मा॒लभे॑त प्रतिलो॒ममे॑ना॒वा ल॑भेता॒मुम॑प॒क्षीय॑माण॒मन्वप॑॥३॥

%3.5.1.4
क्षी॒ये॒त॒ सा॒र॒स्व॒तौ होमौ॑ पु॒रस्ता᳚ज्जुहुयादमावा॒स्या॑ वै सर॑स्वत्यनुलो॒ममे॒वैना॒वा ल॑भते॒\-ऽमुमा॒प्याय॑मान॒मन्वा प्या॑यत आग्नावैष्ण॒वमेका॑दशकपालम्पु॒रस्ता॒न्निर्व॑पे॒थ्सर॑स्वत्यै च॒रुꣳ सर॑स्वते॒ द्वाद॑शकपालं॒ यदा᳚ग्ने॒यो भव॑त्य॒ग्निर्वै य॑ज्ञमु॒खं य॑ज्ञमु॒खमे॒वर्द्धि॑म्पु॒रस्ता᳚द्धत्ते॒ यद्वै᳚ष्ण॒वो भव॑ति य॒ज्ञो वै विष्णु॑र्य॒ज्ञमे॒वारभ्य॒ प्र त॑नुते॒ सर॑स्वत्यै च॒रुर्भ॑वति॒ सर॑स्वते॒ द्वाद॑शकपालो\-ऽमावा॒स्या॑ वै सर॑स्वती पू॒र्णमा॑सः॒ सर॑स्वा॒न्तावे॒व सा॒क्षादा र॑भत ऋ॒ध्नोत्या᳚भ्या॒न्द्वाद॑शकपालः॒ सर॑स्वते भवति मिथुन॒त्वाय॒ प्रजा᳚त्यै मिथु॒नौ गावौ॒ दक्षि॑णा॒ समृ॑द्ध्यै॥४॥

%3.5.2.0
{\anuvakamend[{वर्च॑सा॒ वै ते द॑र्\mbox{}शपूर्णमा॒सावप॑ तनुते॒ सर॑स्वत्यै॒ पञ्च॑विꣳशतिश्च}]}%॥१॥

%3.5.2.1
ऋष॑यो॒ वा इन्द्र॑म्प्र॒त्यक्षं॒ नाप॑श्य॒न्तं वसि॑ष्ठः प्र॒त्यक्ष॑म्पश्य॒थ्सो᳚\-ऽब्रवी॒द्ब्राह्म॑णं ते वक्ष्यामि॒ यथा॒ त्वत्पु॑रोहिताः प्र॒जाः प्र॑जनि॒ष्यन्ते\-ऽथ॒ मेत॑रेभ्य॒ ऋषि॑भ्यो॒ मा प्र वो॑च॒ इति॒ तस्मा॑ ए॒तान्थ्स्तोम॑भागानब्रवी॒त्ततो॒ वसि॑ष्ठपुरोहिताः प्र॒जाः प्राजा॑यन्त॒ तस्मा᳚द्वासि॒ष्ठो ब्र॒ह्मा का॒र्यः॑ प्रैव जा॑यते र॒श्मिर॑सि॒ क्षया॑य त्वा॒ क्षयं॑ जि॒न्वेति॑॥५॥

%3.5.2.2
आ॒ह॒ दे॒वा वै क्षयो॑ दे॒वेभ्य॑ ए॒व य॒ज्ञम्प्राह॒ प्रेति॑रसि॒ धर्मा॑य त्वा॒ धर्मं॑ जि॒न्वेत्या॑ह मनु॒ष्या॑ वै धर्मो॑ मनु॒ष्ये᳚भ्य ए॒व य॒ज्ञम्प्राहान्वि॑तिरसि दि॒वे त्वा॒ दिवं॑ जि॒न्वेत्या॑है॒भ्य ए॒व लो॒केभ्यो॑ य॒ज्ञम्प्राह॑ विष्ट॒म्भो॑\-ऽसि॒ वृष्ट्यै᳚ त्वा॒ वृष्टिं॑ जि॒न्वेत्या॑ह॒ वृष्टि॑मे॒वाव॑॥६॥

%3.5.2.3
रु॒न्द्धे॒ प्र॒वास्य॑नु॒वासीत्या॑ह मिथुन॒त्वायो॒शिग॑सि॒ वसु॑भ्यस्त्वा॒ वसू᳚ञ्जि॒न्वेत्या॑हा॒ष्टौ वस॑व॒ एका॑दश रु॒द्रा द्वाद॑शादि॒त्या ए॒ताव॑न्तो॒ वै दे॒वास्तेभ्य॑ ए॒व य॒ज्ञम्प्राहौजो॑\-ऽसि पि॒तृभ्य॑स्त्वा पि॒तॄञ्जि॒न्वेत्या॑ह दे॒वाने॒व पि॒तॄननु॒ सं त॑नोति॒ तन्तु॑रसि प्र॒जाभ्य॑स्त्वा प्र॒जा जि॑न्व॥७॥

%3.5.2.4
इत्या॑ह पि॒तॄने॒व प्र॒जा अनु॒ सं त॑नोति पृतना॒षाड॑सि प॒शुभ्य॑स्त्वा प॒शूञ्जि॒न्वेत्या॑ह प्र॒जा ए॒व प॒शूननु॒ सं त॑नोति रे॒वद॒स्योष॑धीभ्य॒स्त्वौष॑धीर्जि॒न्वेत्या॒हौष॑धीष्वे॒व प॒शून्प्रति॑ ष्ठापयत्यभि॒जिद॑सि यु॒क्तग्रा॒वेन्द्रा॑य॒ त्वेन्द्रं॑ जि॒न्वेत्या॑हा॒भिजि॑त्या॒ अधि॑पतिरसि प्रा॒णाय॑ त्वा प्रा॒णम्॥८॥

%3.5.2.5
जि॒न्वेत्या॑ह प्र॒जास्वे॒व प्रा॒णान्द॑धाति त्रि॒वृद॑सि प्र॒वृद॒सीत्या॑ह मिथुन॒त्वाय॑ सꣳरो॒हो॑\-ऽसि नीरो॒हो॑\-ऽसीत्या॑ह॒ प्रजा᳚त्यै वसु॒को॑\-ऽसि॒ वेष॑श्रिरसि॒ वस्य॑ष्टिर॒सीत्या॑ह॒ प्रति॑ष्ठित्यै॥९॥

%3.5.3.0
{\anuvakamend[{जि॒न्वेत्यव॑ प्र॒जा जि॑न्व प्रा॒णन्त्रि॒ꣳ॒शच्च॑}]}%॥२॥

%3.5.3.1
अ॒ग्निना॑ दे॒वेन॒ पृत॑ना जयामि गाय॒त्रेण॒ छन्द॑सा त्रि॒वृता॒ स्तोमे॑न रथन्त॒रेण॒ साम्ना॑ वषट्का॒रेण॒ वज्रे॑ण पूर्व॒जान्भ्रातृ॑व्या॒नध॑रान्पादया॒म्यवै॑नान्बाधे॒ प्रत्ये॑नान्नुदे॒\-ऽस्मिन्क्षये॒\-ऽस्मिन्भू॑मिलो॒के यो᳚\-ऽस्मान्द्वेष्टि॒ यं च॑ व॒यं द्वि॒ष्मो विष्णोः॒ क्रमे॒णात्ये॑नान्क्रामा॒मीन्द्रे॑ण दे॒वेन॒ पृत॑ना जयामि॒ त्रैष्टु॑भेन॒ छन्द॑सा पञ्चद॒शेन॒ स्तोमे॑न बृह॒ता साम्ना॑ वषट्का॒रेण॒ वज्रे॑ण॥१०॥

%3.5.3.2
स॒ह॒जान् विश्वे॑भिर्दे॒वेभिः॒ पृत॑ना जयामि॒ जाग॑तेन॒ छन्द॑सा सप्तद॒शेन॒ स्तोमे॑न वामदे॒व्येन॒ साम्ना॑ वषट्का॒रेण॒ वज्रे॑णापर॒जानिन्द्रे॑ण स॒युजो॑ व॒यꣳ सा॑स॒ह्याम॑ पृतन्य॒तः। घ्नन्तो॑ वृ॒त्राण्य॑प्र॒ति। यत्ते॑ अग्ने॒ तेज॒स्तेना॒हं ते॑ज॒स्वी भू॑यासं॒ यत्ते॑ अग्ने॒ वर्च॒स्तेना॒हं व॑च॒स्वी भू॑यासं॒ यत्ते॑ अग्ने॒ हर॒स्तेना॒हꣳ ह॑र॒स्वी भू॑यासम्॥११॥

%3.5.4.0
{\anuvakamend[{बृ॒ह॒ता साम्ना॑ वषट्का॒रेण॒ वज्रे॑ण॒ षट्च॑त्वारिꣳशच्च}]}%॥३॥

%3.5.4.1
ये दे॒वा य॑ज्ञ॒हनो॑ यज्ञ॒मुषः॑ पृथि॒व्यामध्यास॑ते। अ॒ग्निर्मा॒ तेभ्यो॑ रक्षतु॒ गच्छे॑म सु॒कृतो॑ व॒यम्। आग॑न्म मित्रावरुणा वरेण्या॒ रात्री॑णाम्भा॒गो यु॒वयो॒र्यो अस्ति॑। नाकं॑ गृह्णा॒नाः सु॑कृ॒तस्य॑ लो॒के तृ॒तीये॑ पृ॒ष्ठे अधि॑ रोच॒ने दि॒वः। ये दे॒वा य॑ज्ञ॒हनो॑ यज्ञ॒मुषो॒\-ऽन्तरि॒क्षे\-ऽध्यास॑ते। वा॒युर्मा॒ तेभ्यो॑ रक्षतु॒ गच्छे॑म सु॒कृतो॑ व॒यम्। यास्ते॒ रात्रीः᳚ सवितः॥१२॥

%3.5.4.2
दे॒व॒यानी॑रन्त॒रा द्यावा॑पृथि॒वी वि॒यन्ति॑। गृ॒हैश्च॒ सर्वैः᳚ प्र॒जया॒ न्वग्रे॒ सुवो॒ रुहा॑णास्तरता॒ रजाꣳ॑सि। ये दे॒वा य॑ज्ञ॒हनो॑ यज्ञ॒मुषो॑ दि॒व्यध्यास॑ते। सूर्यो॑ मा॒ तेभ्यो॑ रक्षतु॒ गच्छे॑म सु॒कृतो॑ व॒यम्। येनेन्द्रा॑य स॒मभ॑रः॒ पयाꣳ॑स्युत्त॒\-मेन॑ ह॒विषा॑ जातवेदः। तेना᳚ग्ने॒ त्वमु॒त व॑र्धये॒मꣳ स॑जा॒ताना॒ꣴ॒ श्रैष्ठ्य॒ आ धे᳚ह्येनम्। य॒ज्ञ॒हनो॒ वै दे॒वा य॑ज्ञ॒मुषः॑॥१३॥

%3.5.4.3
स॒न्ति॒ त ए॒षु लो॒केष्वा॑सत आ॒ददा॑ना विमथ्ना॒ना यो ददा॑ति॒ यो यज॑ते॒ तस्य॑। ये दे॒वा य॑ज्ञ॒हनः॑ पृथि॒व्यामध्यास॑ते॒ ये अ॒न्तरि॑क्षे॒ ये दि॒वीत्या॑हे॒माने॒व लो॒काꣴस्ती॒र्त्वा सगृ॑हः॒ सप॑शुः सुव॒र्गं लो॒कमे॒त्यप॒ वै सोमे॑नेजा॒नाद्दे॒वता᳚श्च य॒ज्ञश्च॑ क्रामन्त्याग्ने॒यं पञ्च॑कपालमुदवसा॒नीयं॒ निर्व॑पेद॒ग्निः सर्वा॑ दे॒वताः᳚॥१४॥

%3.5.4.4
पाङ्क्तो॑ य॒ज्ञो दे॒वता᳚श्चै॒व य॒ज्ञं चाव॑ रुन्द्धे गाय॒त्रो वा अ॒ग्निर्गा॑य॒त्रछ॑न्दा॒स्तं छन्द॑सा॒ व्य॑र्धयति॒ यत्पञ्च॑कपालं क॒रोत्य॒ष्टाक॑पालः का॒र्यो᳚\-ऽष्टाक्ष॑रा गाय॒त्री गा॑य॒त्रो᳚\-ऽग्निर्गा॑य॒त्रछ॑न्दाः॒ स्वेनै॒वैनं॒ छन्द॑सा॒ सम॑र्धयति प॒ङ्क्त्यौ॑ याज्यानुवा॒क्ये॑ भवतः॒ पाङ्क्तो॑ य॒ज्ञस्तेनै॒व य॒ज्ञान्नैति॑॥१५॥

%3.5.5.0
{\anuvakamend[{स॒वि॒त॒र्दे॒वा य॑ज्ञ॒मुषः॒ सर्वा॑ दे॒वता॒स्त्रिच॑त्वारिꣳशच्च}]}%॥४॥

%3.5.5.1
सूर्यो॑ मा दे॒वो दे॒वेभ्यः॑ पातु वा॒युर॒न्तरि॑क्षा॒द्यज॑मानो॒\-ऽग्निर्मा॑ पातु॒ चक्षु॑षः। सक्ष॒ शूष॒ सवि॑त॒र्विश्व॑चर्\mbox{}षण ए॒तेभिः॑ सोम॒ नाम॑भिर्विधेम ते॒ तेभिः॑ सोम॒ नाम॑भिर्विधेम ते। अ॒हम्प॒रस्ता॑द॒हम॒वस्ता॑द॒हं ज्योति॑षा॒ वि तमो॑ ववार। यद॒न्तरि॑क्षं॒ तदु॑ मे पि॒ताभू॑द॒हꣳ सूर्य॑मुभ॒यतो॑ ददर्\mbox{}शा॒हम्भू॑यासमुत्त॒मः स॑मा॒नाना᳚म्॥१६॥

%3.5.5.2
आ स॑मु॒द्रादा\-ऽन्तरि॑क्षात्प्र॒जाप॑तिरुद॒धिं च्या॑वया॒तीन्द्रः॒ प्र स्नौ॑तु म॒रुतो॑ वर्\mbox{}षय॒न्तून्न॑म्भय पृथि॒वीम्भि॒न्द्धीदं दि॒व्यं नभः॑। उ॒द्नो दि॒व्यस्य॑ नो दे॒हीशा॑नो॒ वि सृ॑जा॒ दृतिम्᳚। प॒शवो॒ वा ए॒ते यदा॑दि॒त्य ए॒ष रु॒द्रो यद॒ग्निरोष॑धीः॒ प्रास्या॒ग्नावा॑दि॒त्यं जु॑होति रु॒द्रादे॒व प॒शून॒न्तर्द॑धा॒त्यथो॒ ओष॑धीष्वे॒व प॒शून्॥१७॥

%3.5.5.3
प्रति॑ ष्ठापयति क॒विर्य॒ज्ञस्य॒ वि त॑नोति॒ पन्थां॒ नाक॑स्य पृ॒ष्ठे अधि॑ रोच॒ने दि॒वः। येन॑ ह॒व्यं वह॑सि॒ यासि॑ दू॒त इ॒तः प्रचे॑ता अ॒मुतः॒ सनी॑यान्। यास्ते॒ विश्वाः᳚ स॒मिधः॒ सन्त्य॑ग्ने॒ याः पृ॑थि॒व्याम्ब॒र्\mbox{}हिषि॒ सूर्ये॒ याः। तास्ते॑ गच्छ॒न्त्वाहु॑तिं घृ॒तस्य॑ देवाय॒ते यज॑मानाय॒ शर्म॑। आ॒शासा॑नः सु॒वीर्यꣳ॑ रा॒यस्पोष॒ꣴ॒ स्वश्वि॑यम्। बृह॒स्पति॑ना रा॒या स्व॒गाकृ॑तो॒ मह्यं॒ यज॑मानाय तिष्ठ॥१८॥

%3.5.6.0
{\anuvakamend[{स॒मा॒नाना॒मोष॑धीष्वे॒व प॒शून्मह्यं॒ यज॑माना॒यैक॑ञ्च}]}%॥५॥

%3.5.6.1
सं त्वा॑ नह्यामि॒ पय॑सा घृ॒तेन॒ सं त्वा॑ नह्याम्य॒प ओष॑धीभिः। सं त्वा॑ नह्यामि प्र॒जया॒हम॒द्य सा दी᳚क्षि॒ता स॑नवो॒ वाज॑म॒स्मे। प्रैतु॒ ब्रह्म॑ण॒स्पत्नी॒ वेदिं॒ वर्णे॑न सीदतु। अथा॒हम॑नुका॒मिनी॒ स्वे लो॒के वि॒शा इ॒ह। सु॒प्र॒जस॑स्त्वा व॒यꣳ सु॒पत्नी॒रुप॑ सेदिम। अग्ने॑ सपत्न॒दम्भ॑न॒मद॑ब्धासो॒ अदा᳚भ्यम्। इ॒मं वि ष्या॑मि॒ वरु॑णस्य॒ पाशम्᳚॥१९॥

%3.5.6.2
यमब॑ध्नीत सवि॒ता सु॒केतः॑। धा॒तुश्च॒ योनौ॑ सुकृ॒तस्य॑ लो॒के स्यो॒नं मे॑ स॒ह पत्या॑ करोमि। प्रेह्यु॒देह्यृ॒तस्य॑ वा॒मीरन्व॒ग्निस्ते\-ऽग्रं॑ नय॒त्वदि॑ति॒र्मध्यं॑ ददताꣳ रु॒द्राव॑सृष्टासि यु॒वा नाम॒ मा मा॑ हिꣳसी॒र्वसु॑भ्यो रु॒द्रेभ्य॑ आदि॒त्येभ्यो॒ विश्वे᳚भ्यो वो दे॒वेभ्यः॑ प॒न्नेज॑नीर्गृह्णामि य॒ज्ञाय॑ वः प॒न्नेज॑नीः सादयामि॒ विश्व॑स्य ते॒ विश्वा॑वतो॒ वृष्णि॑यावतः॥२०॥

%3.5.6.3
तवा᳚ग्ने वा॒मीरनु॑ सं॒दृशि॒ विश्वा॒ रेताꣳ॑सि धिषी॒यागं॑ दे॒वान् य॒ज्ञो नि दे॒वीर्दे॒वेभ्यो॑ य॒ज्ञम॑शिषन्न॒स्मिन्थ्सु॑न्व॒ति यज॑मान आ॒शिषः॒ स्वाहा॑कृताः समुद्रे॒ष्ठा ग॑न्ध॒र्वमा ति॑ष्ठ॒ता\-ऽनु॑। वात॑स्य॒ पत्म॑न्नि॒ड ई॑डि॒ताः॥२१॥

%3.5.7.0
{\anuvakamend[{पाशं॒ वृष्णि॑यावतस्त्रि॒ꣳ॒शच्च॑}]}%॥६॥

%3.5.7.1
व॒ष॒ट्का॒रो वै गा॑यत्रि॒यै शिरो᳚\-ऽच्छिन॒त्तस्यै॒ रसः॒ परा॑पत॒थ्स पृ॑थि॒वीम्प्रावि॑श॒थ्स ख॑दि॒रो॑\-ऽभव॒द्यस्य॑ खादि॒रः स्रु॒वो भव॑ति॒ छन्द॑सामे॒व रसे॒नाव॑ द्यति॒ सर॑सा अ॒स्याहु॑तयो भवन्ति तृ॒तीय॑स्यामि॒तो दि॒वि सोम॑ आसी॒त्तं गा॑य॒त्र्याह॑र॒त्तस्य॑ प॒र्णम॑च्छिद्यत॒ तत्प॒र्णो॑\-ऽभव॒त्तत्प॒र्णस्य॑ पर्ण॒त्वं यस्य॑ पर्ण॒मयी॑ जु॒हूः॥२२॥

%3.5.7.2
भव॑ति सौ॒म्या अ॒स्याहु॑तयो भवन्ति जु॒षन्ते᳚\-ऽस्य दे॒वा आहु॑तीर्दे॒वा वै ब्रह्म॑न्नवदन्त॒ तत्प॒र्ण उपा॑शृणोथ्सु॒श्रवा॒ वै नाम॒ यस्य॑ पर्ण॒मयी॑ जु॒हूर्भव॑ति॒ न पा॒पꣴ श्लोकꣳ॑ शृणोति॒ ब्रह्म॒ वै प॒र्णो विण्म॒रुतो\-ऽन्नं॒ विण्मा॑रु॒तो᳚\-ऽश्व॒त्थो यस्य॑ पर्ण॒मयी॑ जु॒हूर्भव॒त्याश्व॑त्थ्युप॒भृद्ब्रह्म॑णै॒वान्न॒मव॑ रु॒न्द्धे\-ऽथो॒ ब्रह्म॑॥२३॥

%3.5.7.3
ए॒व वि॒श्यध्यू॑हति रा॒ष्ट्रं वै प॒र्णो विड॑श्व॒त्थो यत्प॑र्ण॒मयी॑ जु॒हूर्भव॒त्याश्व॑त्थ्युप॒भृद्रा॒ष्ट्रमे॒व वि॒श्यध्यू॑हति प्र॒जाप॑ति॒र्वा अ॑जुहो॒थ्सा यत्राहु॑तिः प्र॒त्यति॑ष्ठ॒त्ततो॒ विक॑ङ्कत॒ उद॑तिष्ठ॒त्ततः॑ प्र॒जा अ॑सृजत॒ यस्य॒ वैक॑ङ्कती ध्रु॒वा भव॑ति॒ प्रत्ये॒वास्याहु॑तयस्तिष्ठ॒न्त्यथो॒ प्रैव जा॑यत ए॒तद्वै स्रु॒चाꣳ रू॒पं यस्यै॒वꣳरू॑पाः॒ स्रुचो॒ भव॑न्ति॒ सर्वा᳚ण्ये॒वैनꣳ॑ रू॒पाणि॑ पशू॒नामुप॑ तिष्ठन्ते॒ नास्याप॑रूपमा॒त्मञ्जा॑यते॥२४॥

%3.5.8.0
{\anuvakamend[{जु॒हूरथो॒ ब्रह्म॑ स्रु॒चाꣳ स॒प्तद॑श च}]}%॥७॥

%3.5.8.1
उ॒प॒या॒मगृ॑हीतो\-ऽसि प्र॒जाप॑तये त्वा॒ ज्योति॑ष्मते॒ ज्योति॑ष्मन्तं गृह्णामि॒ दक्षा॑य दक्ष॒वृधे॑ रा॒तं दे॒वेभ्यो᳚\-ऽग्निजि॒ह्वेभ्य॑\-स्त्वर्ता॒युभ्य॒ इन्द्र॑ज्येष्ठेभ्यो॒ वरु॑णराजभ्यो॒ वाता॑पिभ्यः प॒र्जन्या᳚त्मभ्यो दि॒वे त्वा॒न्तरि॑क्षाय त्वा पृथि॒व्यै त्वापे᳚न्द्र द्विष॒तो मनो\-ऽप॒ जिज्या॑सतो ज॒ह्यप॒ यो नो॑\-ऽराती॒यति॒ तं ज॑हि प्रा॒णाय॑ त्वापा॒नाय॑ त्वा व्या॒नाय॑ त्वा स॒ते त्वास॑ते त्वा॒द्भ्यस्त्वौष॑धीभ्यो॒ विश्वे᳚भ्यस्त्वा भू॒तेभ्यो॒ यतः॑ प्र॒जा अक्खि॑द्रा॒ अजा॑यन्त॒ तस्मै᳚ त्वा प्र॒जाप॑तये विभू॒दाव्ने॒ ज्योति॑ष्मते॒ ज्योति॑ष्मन्तं जुहोमि॥२५॥

%3.5.9.0
{\anuvakamend[{ओष॑धीभ्य॒श्चतु॑र्दश च}]}%॥८॥

%3.5.9.1
यां वा अ॑ध्व॒र्युश्च॒ यज॑मानश्च दे॒वता॑मन्तरि॒तस्तस्या॒ आ वृ॑श्च्येते प्राजाप॒त्यं द॑धिग्र॒हं गृ॑ह्णीयात्प्र॒जाप॑तिः॒ सर्वा॑ दे॒वता॑ दे॒वता᳚भ्य ए॒व नि ह्नु॑वाते ज्ये॒ष्ठो वा ए॒ष ग्रहा॑णां॒ यस्यै॒ष गृ॒ह्यते॒ ज्यैष्ठ्य॑मे॒व ग॑च्छति॒ सर्वा॑सां॒ वा ए॒तद्दे॒वता॑नाꣳ रू॒पं यदे॒ष ग्रहो॒ यस्यै॒ष गृ॒ह्यते॒ सर्वा᳚ण्ये॒वैनꣳ॑ रू॒पाणि॑ पशू॒नामुप॑ तिष्ठन्त उपया॒मगृ॑हीतः॥२६॥

%3.5.9.2
अ॒सि॒ प्र॒जाप॑तये त्वा॒ ज्योति॑ष्मते॒ ज्योति॑ष्मन्तं गृह्णा॒मीत्या॑ह॒ ज्योति॑रे॒वैनꣳ॑ समा॒नानां᳚ करोत्यग्निजि॒ह्वेभ्य॑स्त्वर्ता॒युभ्य॒ इत्या॑है॒ताव॑ती॒र्वै दे॒वता॒स्ताभ्य॑ ए॒वैन॒ꣳ॒ सर्वा᳚भ्यो गृह्णा॒त्यपे᳚न्द्र द्विष॒तो मन॒ इत्या॑ह॒ भ्रातृ॑व्यापनुत्त्यै प्रा॒णाय॑ त्वापा॒नाय॒ त्वेत्या॑ह प्रा॒णाने॒व यज॑माने दधाति॒ तस्मै᳚ त्वा प्र॒जाप॑तये विभू॒दाव्ने॒ ज्योति॑ष्मते॒ ज्योति॑ष्मन्तं जुहोमि॥२७॥

%3.5.9.3
इत्या॑ह प्र॒जाप॑तिः॒ सर्वा॑ दे॒वताः॒ सर्वा᳚भ्य ए॒वैनं॑ दे॒वता᳚भ्यो जुहोत्याज्यग्र॒हं गृ॑ह्णीया॒त्तेज॑स्कामस्य॒ तेजो॒ वा आज्य॑न्तेज॒स्व्ये॑व भ॑वति सोमग्र॒हं गृ॑ह्णीयाद्ब्रह्मवर्च॒सका॑मस्य ब्रह्मवर्च॒सं वै सोमो᳚ ब्रह्मवर्च॒स्ये॑व भ॑वति दधिग्र॒हं गृ॑ह्णीयात्प॒शुका॑म॒स्योर्ग्वै दध्यूर्क्प॒शव॑ ऊ॒र्जैवास्मा॒ ऊर्जं॑ प॒शूनव॑ रुन्द्धे॥२८॥

%3.5.10.0
{\anuvakamend[{उ॒प॒या॒मगृ॑हीतो जुहोमि॒ त्रिच॑त्वारिꣳशच्च}]}%॥९॥

%3.5.10.1
त्वे क्रतु॒मपि॑ वृञ्जन्ति॒ विश्वे॒ द्विर्यदे॒ते त्रिर्भव॒न्त्यूमाः᳚। स्वा॒दोः स्वादी॑यः स्वा॒दुना॑ सृजा॒ समत॑ ऊ॒ षु मधु॒ मधु॑ना॒भि यो॑धि। उ॒प॒या॒मगृ॑हीतो\-ऽसि प्र॒जाप॑तये त्वा॒ जुष्टं॑ गृह्णाम्ये॒ष ते॒ योनिः॑ प्र॒जाप॑तये त्वा। प्रा॒ण॒ग्र॒हान्गृ॑ह्णात्ये॒ताव॒द्वा अ॑स्ति॒ याव॑दे॒ते ग्रहाः॒ स्तोमा॒श्छन्दाꣳ॑सि पृ॒ष्ठानि॒ दिशो॒ याव॑दे॒वास्ति॒ तत्॥२९॥

%3.5.10.2
अव॑ रुन्द्धे ज्ये॒ष्ठा वा ए॒तान्ब्रा᳚ह्म॒णाः पु॒रा वि॒द्वाम॑क्र॒न्तस्मा॒त्तेषा॒ꣳ॒ सर्वा॒ दिशो॒\-ऽभिजि॑ता अभूव॒न् यस्यै॒ते गृ॒ह्यन्ते॒ ज्यैष्ठ्य॑मे॒व ग॑च्छत्य॒भि दिशो॑ जयति॒ पञ्च॑ गृह्यन्ते॒ पञ्च॒ दिशः॒ सर्वा᳚स्वे॒व दि॒क्ष्वृ॑ध्नुवन्ति॒ नव॑नव गृह्यन्ते॒ नव॒ वै पुरु॑षे प्रा॒णाः प्रा॒णाने॒व यज॑मानेषु दधति प्राय॒णीये॑ चोदय॒नीये॑ च गृह्यन्ते प्रा॒णा वै प्रा॑णग्र॒हाः॥३०॥

%3.5.10.3
प्रा॒णैरे॒व प्र॒यन्ति॑ प्रा॒णैरुद्य॑न्ति दश॒मे\-ऽह॑न्गृह्यन्ते प्रा॒णा वै प्रा॑णग्र॒हाः प्रा॒णेभ्यः॒ खलु॒ वा ए॒तत्प्र॒जा य॑न्ति॒ यद्वा॑मदे॒व्यं योने॒श्च्यव॑ते दश॒मे\-ऽह॑न्वामदे॒व्यं योने᳚श्च्यवते॒ यद्द॑श॒मे\-ऽह॑न्गृ॒ह्यन्ते᳚ प्रा॒णेभ्य॑ ए॒व तत्प्र॒जा न य॑न्ति॥३१॥

%3.5.11.0
{\anuvakamend[{तत्प्रा॑णग्र॒हाः स॒प्तत्रिꣳ॑शच्च}]}%॥10॥

%3.5.11.1
प्र दे॒वं दे॒व्या धि॒या भर॑ता जा॒तवे॑दसम्। ह॒व्या नो॑ वक्षदानु॒षक्। अ॒यमु॒ ष्य प्र दे॑व॒युर्\mbox{}होता॑ य॒ज्ञाय॑ नीयते। रथो॒ न योर॒भीवृ॑तो॒ घृणी॑वाञ्चेतति॒ त्मना᳚। अ॒यम॒ग्निरु॑रुष्यत्य॒मृता॑दिव॒ जन्म॑नः। सह॑सश्चि॒थ्सही॑यां दे॒वो जी॒वात॑वे कृ॒तः। इडा॑यास्त्वा प॒दे व॒यं नाभा॑ पृथि॒व्या अधि॑। जात॑वेदो॒ नि धी॑म॒ह्यग्ने॑ ह॒व्याय॒ वोढ॑वे।॥३२॥

%3.5.11.2
अग्ने॒ विश्वे॑भिः स्वनीक दे॒वैरूर्णा॑वन्तम्प्रथ॒मः सी॑द॒ योनिम्᳚। कु॒ला॒यिनं॑ घृ॒तव॑न्तꣳ सवि॒त्रे य॒ज्ञं न॑य॒ यज॑मानाय सा॒धु। सीद॑ होतः॒ स्व उ॑ लो॒के चि॑कि॒त्वान्थ्सा॒दया॑ य॒ज्ञꣳ सु॑कृ॒तस्य॒ योनौ᳚। दे॒वा॒वीर्दे॒वान् ह॒विषा॑ यजा॒स्यग्ने॑ बृ॒हद्यज॑माने॒ वयो॑ धाः। नि होता॑ होतृ॒षद॑ने॒ विदा॑नस्त्वे॒षो दी॑दि॒वाꣳ अ॑सदथ्सु॒दक्षः॑। अद॑ब्धव्रतप्रमति॒र्वसि॑ष्ठः सहस्रम्भ॒रः शुचि॑जिह्वो अ॒ग्निः। त्वं दू॒तस्त्वम्॥३३॥

%3.5.11.3
उ॒ नः॒ प॒र॒स्पास्त्वं वस्य॒ आ वृ॑षभ प्रणे॒ता। अग्ने॑ तो॒कस्य॑ न॒स्तने॑ त॒नूना॒मप्र॑युच्छ॒न्दीद्य॑द्बोधि गो॒पाः। अ॒भि त्वा॑ देव सवित॒रीशा॑नं॒ वार्या॑णाम्। सदा॑वन्भा॒गमी॑महे। म॒ही द्यौः पृ॑थि॒वी च॑ न इ॒मं य॒ज्ञम्मि॑मिक्षताम्। पि॒पृ॒तां नो॒ भरी॑मभिः। त्वाम॑ग्ने॒ पुष्क॑रा॒दध्यथ॑र्वा॒ निर॑मन्थत। मू॒र्ध्नो विश्व॑स्य वा॒घतः॑। तमु॑॥३४॥

%3.5.11.4
त्वा॒ द॒ध्यङ्ङृषिः॑ पु॒त्र ई॑धे॒ अथ॑र्वणः। वृ॒त्र॒हणं॑ पुरन्द॒रम्। तमु॑ त्वा पा॒थ्यो वृषा॒ समी॑धे दस्यु॒हन्त॑मम्। ध॒नं॒ज॒यꣳ रणे॑रणे। उ॒त ब्रु॑वन्तु ज॒न्तव॒ उद॒ग्निर्वृ॑त्र॒हाज॑नि। ध॒नं॒ज॒यो रणे॑रणे। आ यꣳ हस्ते॒ न खा॒दिन॒ꣳ॒ शिशुं॑ जा॒तं न बिभ्र॑ति। वि॒शाम॒ग्निꣴ स्व॑ध्व॒रम्। प्र दे॒वं दे॒ववी॑तये॒ भर॑ता वसु॒वित्त॑मम्। आ स्वे योनौ॒ नि षी॑दतु। आ॥३५॥


%3.5.11.5
जा॒तं जा॒तवे॑दसि प्रि॒यꣳ शि॑शी॒ताति॑थिम्। स्यो॒न आ गृ॒हप॑तिम्। अ॒ग्निना॒ऽग्निः समि॑ध्यते क॒विर्गृ॒हप॑ति॒र्युवा᳚। ह॒व्य॒वाड्जु॒ह्वा᳚स्यः। त्वꣴ ह्य॑ग्ने अ॒ग्निना॒ विप्रो॒ विप्रे॑ण॒ सन्थ्स॒ता। सखा॒ सख्या॑ समि॒ध्यसे᳚। तम्म॑र्जयन्त सु॒क्रतुं॑ पुरो॒यावा॑नमा॒जिषु॑। स्वेषु॒ क्षये॑षु वा॒जिनम्᳚। य॒ज्ञेन॑ य॒ज्ञम॑यजन्त दे॒वास्तानि॒ धर्मा॑णि प्रथ॒मान्या॑सन्न्। ते ह॒ नाक॑म्महि॒मानः॑ सचन्ते॒ यत्र॒ पूर्वे॑ सा॒ध्याः सन्ति॑ दे॒वाः॥३६॥

%4.1.0.0

%4.1.0.0
{\anuvakamend[{वोढ॑वे दू॒तस्त्वन्तमु॑ सीद॒त्वा यत्र॑ च॒त्वारि॑ च}]}%॥11॥

{\anuvakamend[यु॒ञ्जा॒न इ॒माम॑गृभ्णं दे॒वस्य॒ सन्ते॒ वि पाज॑सा॒ वस॑वस्त्वा॒ समा᳚स्त्वो॒र्ध्वा अ॒स्याकू॑तिं॒ यद॑ग्ने॒ यान्यग्ने॒ यं य॒ज्ञमेका॑दश॥11॥ यु॒ञ्जा॒नो वर्म॑ च स्थ आदि॒त्यास्त्वा॒ भार॑ती॒ स्वाꣳ अहꣳ षट्च॑त्वारिꣳशत्॥46॥ यु॒ञ्जा॒नो वाजे॑वाजे॥]}
%%% END KANDAM

\chapt{काण्डम् ४}
\sect{प्रथमः प्रश्नः}\setcounter{anuvakam}{0}
\dnsub{तैत्तिरीयसंहितायां चतुर्थकाण्डे प्रथमः प्रश्नः}
%4.1.1.0
%4.1.1.1
यु॒ञ्जा॒नः प्र॑थ॒मम्मन॑स्त॒त्वाय॑ सवि॒ता धियः॑। अ॒ग्निं ज्योति॑र्नि॒चाय्य॑ पृथि॒व्या अध्याभ॑रत्। यु॒क्त्वाय॒ मन॑सा दे॒वान्थ्सुव॑र्य॒तो धि॒या दिवम्᳚। बृ॒हज्ज्योतिः॑ करिष्य॒तः स॑वि॒ता प्र सु॑वति॒ तान्। यु॒क्तेन॒ मन॑सा व॒यं दे॒वस्य॑ सवि॒तुः स॒वे। सु॒व॒र्गेया॑य॒ शक्त्यै᳚। यु॒ञ्जते॒ मन॑ उ॒त यु॑ञ्जते॒ धियो॒ विप्रा॒ विप्र॑स्य बृह॒तो वि॑प॒श्चितः॑। वि होत्रा॑ दधे वयुना॒विदेक॒ इत्॥१॥

%4.1.1.2
म॒ही दे॒वस्य॑ सवि॒तुः परि॑ष्टुतिः। यु॒जे वां॒ ब्रह्म॑ पू॒र्व्यं नमो॑भि॒र्वि श्लोका॑ यन्ति प॒थ्ये॑व॒ सूराः᳚। शृ॒ण्वन्ति॒ विश्वे॑ अ॒मृत॑स्य पु॒त्रा आ ये धामा॑नि दि॒व्यानि॑ त॒स्थुः। यस्य॑ प्र॒याण॒मन्व॒न्य इद्य॒युर्दे॒वा दे॒वस्य॑ महि॒मान॒मर्च॑तः। यः पार्थि॑वानि विम॒मे स एत॑शो॒ रजाꣳ॑सि दे॒वः स॑वि॒ता म॑हित्व॒ना। देव॑ सवितः॒ प्र सु॑व य॒ज्ञम्प्र सु॑व॥२॥

%4.1.1.3
य॒ज्ञप॑ति॒म्भगा॑य दि॒व्यो ग॑न्ध॒र्वः। के॒त॒पूः केतं॑ नः पुनातु वा॒चस्पति॒र्वाच॑म॒द्य स्व॑दाति नः। इ॒मं नो॑ देव सवितर्य॒ज्ञं प्र सु॑व देवा॒युवꣳ॑ सखि॒विदꣳ॑ सत्रा॒जितं॑ धन॒जितꣳ॑ सुव॒र्जितम्᳚। ऋ॒चा स्तोम॒ꣳ॒ सम॑र्धय गाय॒त्रेण॑ रथन्त॒रम्। बृ॒हद्गा॑य॒त्रव॑र्तनि। दे॒वस्य॑ त्वा सवि॒तुः प्र॑स॒वे᳚\-ऽश्विनो᳚र्बा॒हु\-भ्यां᳚ पू॒ष्णो हस्ता᳚भ्याम्गाय॒त्रेण॒ छन्द॒सा\-ऽ\-ऽद॑दे\-ऽङ्गिर॒स्वदभ्रि॑रसि॒ नारिः॑॥३॥

%4.1.1.4
अ॒सि॒ पृ॒थि॒व्याः स॒धस्था॑द॒ग्निम्पु॑री॒ष्य॑मङ्गिर॒स्वदा भ॑र॒ त्रैष्टु॑भेन त्वा॒ छन्द॒सा\-ऽ\-ऽद॑दे\-ऽङ्गिर॒स्वद्बभ्रि॑रसि॒ नारि॑रसि॒ त्वया॑ व॒यꣳ स॒धस्थ॒ आग्निꣳ श॑केम॒ खनि॑तुं पुरी॒ष्यं॑ जाग॑तेन त्वा॒ छन्द॒सा\-ऽ\-ऽद॑दे\-ऽङ्गिर॒स्वद्धस्त॑ आ॒धाय॑ सवि॒ता बिभ्र॒दभ्रिꣳ॑ हिर॒ण्ययी᳚म्। तया॒ ज्योति॒रज॑स्र॒मिद॒ग्निं खा॒त्वी न॒ आ भ॒रानु॑ष्टुभेन त्वा॒ छन्द॒सा\-ऽ\-ऽद॑दे\-ऽङ्गिर॒स्वत्॥

%4.1.2.0
{\anuvakamend[{इद्य॒ज्ञं प्र सु॑व॒ नारि॒रानु॑ष्टुभेन त्वा॒ छन्द॑सा॒ त्रीणि॑ च}]}%॥१॥

%4.1.2.1
इ॒माम॑गृभ्णन्रश॒नामृ॒तस्य॒ पूर्व॒ आयु॑षि वि॒दथे॑षु क॒व्या। तया॑ दे॒वाः सु॒तमा ब॑भूवुर्\mbox{}ऋ॒तस्य॒ साम᳚न्थ्स॒रमा॒रप॑न्ती। प्रतू᳚र्तं वाजि॒न्ना द्र॑व॒ वरि॑ष्ठा॒मनु॑ सं॒वतम्᳚। दि॒वि ते॒ जन्म॑ पर॒मम॒न्तरि॑क्षे॒ नाभिः॑ पृथि॒व्यामधि॒ योनिः॑। यु॒ञ्जाथा॒ꣳ॒ रास॑भं यु॒वम॒स्मिन् यामे॑ वृषण्वसू। अ॒ग्निम्भर॑न्तमस्म॒युम्। योगे॑योगे त॒वस्त॑रं॒ वाजे॑वाजे हवामहे। सखा॑य॒ इन्द्र॑मू॒तये᳚। प्र॒तूर्वन्न्॑॥५॥

%4.1.2.2
एह्य॑व॒क्राम॒न्नश॑स्ती रु॒द्रस्य॒ गाण॑पत्यान्मयो॒भूरेहि॑। उ॒र्व॑न्तरि॑क्ष॒मन्वि॑हि स्व॒स्तिग॑व्यूति॒रभ॑यानि कृ॒ण्वन्न्। पू॒ष्णा स॒युजा॑ स॒ह। पृ॒थि॒व्याः स॒धस्था॑द॒ग्निम्पु॑री॒ष्य॑मङ्गिर॒स्वदच्छे᳚ह्य॒ग्निम्पु॑री॒ष्य॑मङ्गिर॒स्वदच्छे॑मो॒\-ऽग्निम्पु॑री॒ष्य॑मङ्गिर॒\-स्वद्भ॑रिष्यामो॒\-ऽग्निम्पु॑री॒ष्य॑मङ्गिर॒स्वद्भ॑रामः। अन्व॒ग्निरु॒षसा॒मग्र॑मख्य॒दन्वहा॑नि प्रथ॒मो जा॒तवे॑दाः। अनु॒ सूर्य॑स्य॥६॥

%4.1.2.3
पु॒रु॒त्रा च॑ र॒श्मीननु॒ द्यावा॑पृथि॒वी आ त॑तान। आ॒गत्य॑ वा॒ज्यध्व॑नः॒ सर्वा॒ मृधो॒ वि धू॑नुते। अ॒ग्निꣳ स॒धस्थे॑ मह॒ति चक्षु॑षा॒ नि चि॑कीषते। आ॒क्रम्य॑ वाजिन्पृथि॒वीम॒ग्निमि॑च्छ रु॒चा त्वम्। भूम्या॑ वृ॒त्वाय॑ नो ब्रूहि॒ यतः॒ खना॑म॒ तं व॒यम्। द्यौस्ते॑ पृ॒ष्ठं पृ॑थि॒वी स॒धस्थ॑मा॒त्मान्तरि॑क्षꣳ समु॒द्रस्ते॒ योनिः॑। वि॒ख्याय॒ चक्षु॑षा॒ त्वम॒भि ति॑ष्ठ॥७॥

%4.1.2.4
पृ॒त॒न्य॒तः। उत्क्रा॑म मह॒ते सौभ॑गाया॒स्मादा॒स्थाना᳚द्द्रविणो॒दा वा॑जिन्न्। व॒यꣴ स्या॑म सुम॒तौ पृ॑थि॒व्या अ॒ग्निं ख॑नि॒ष्यन्त॑ उ॒पस्थे॑ अस्याः। उद॑क्रमीद्द्रविणो॒दा वा॒ज्यर्वाकः॒ स लो॒कꣳ सुकृ॑तं पृथि॒व्याः। ततः॑ खनेम सु॒प्रती॑कम॒ग्निꣳ सुवो॒ रुहा॑णा॒ अधि॒ नाक॑ उत्त॒मे। अ॒पो दे॒वीरुप॑ सृज॒ मधु॑मतीरय॒क्ष्माय॑ प्र॒जाभ्यः॑। तासा॒ꣴ॒ स्थाना॒दुज्जि॑हता॒मोष॑धयः सुपिप्प॒लाः। जिघ॑र्मि॥८॥

%4.1.2.5
अ॒ग्निम्मन॑सा घृ॒तेन॑ प्रति॒क्ष्यन्त॒म्भुव॑नानि॒ विश्वा᳚। पृ॒थुं ति॑र॒श्चा वय॑सा बृ॒हन्तं॒ व्यचि॑ष्ठ॒मन्नꣳ॑ रभ॒सं विदा॑नम्। आ त्वा॑ जिघर्मि॒ वच॑सा घृ॒तेना॑र॒क्षसा॒ मन॑सा॒ तज्जु॑षस्व। मर्य॑श्रीः स्पृह॒यद्व॑र्णो अ॒ग्निर्नाभि॒मृशे॑ त॒नुवा॒ जर्\mbox{}हृ॑षाणः। परि॒ वाज॑पतिः क॒विर॒ग्निर्\mbox{}ह॒व्या न्य॑क्रमीत्। दध॒द्रत्ना॑नि दा॒शुषे᳚। परि॑ त्वाऽग्ने॒ पुरं॑ व॒यं विप्रꣳ॑ सहस्य धीमहि। धृ॒षद्व॑र्णं दि॒वेदि॑वे भे॒त्तार॑म्भङ्गु॒॒राव॑तः। त्वम॑ग्ने॒ द्युभि॒स्त्वमा॑शुशु॒क्षणि॒स्त्वम॒द्भ्यस्त्वमश्म॑न॒स्परि॑। त्वं वने᳚भ्य॒स्त्वमोष॑धीभ्य॒स्त्वं नृ॒णां नृ॑पते जायसे॒ शुचिः॑॥९॥

%4.1.3.0
{\anuvakamend[{प्र॒तूर्व॒न्थ्सूर्य॑स्य तिष्ठ॒ जिघ॑र्मि भे॒त्तारं॑ विꣳश॒तिश्च॑}]}%॥२॥

%4.1.3.1
दे॒वस्य॑ त्वा सवि॒तुः प्र॑स॒वे᳚\-ऽश्विनो᳚र्बा॒हु\-भ्यां᳚ पू॒ष्णो हस्ता᳚भ्यां पृथि॒व्याः स॒धस्थे॒\-ऽग्निम्पु॑री॒ष्य॑मङ्गिर॒स्व\-त्ख॑नामि। ज्योति॑ष्मन्तं त्वाग्ने सु॒प्रती॑क॒मज॑स्रेण भा॒नुना॒ दीद्या॑नम्। शि॒वं प्र॒जाभ्यो\-ऽहिꣳ॑सन्तं पृथि॒व्याः स॒धस्थे॒\-ऽग्निं पु॑री॒ष्य॑मङ्गिर॒स्वत्ख॑नामि। अ॒पां पृ॒ष्ठम॑सि स॒प्रथा॑ उ॒र्व॑ग्निम्भ॑रि॒ष्यदप॑रावपिष्ठम्। वर्ध॑मानम्म॒ह आ च॒ पुष्क॑रं दि॒वो मात्र॑या वरि॒णा प्र॑थस्व। शर्म॑ च स्थः॥१०॥

%4.1.3.2
वर्म॑ च स्थो॒ अच्छि॑द्रे बहु॒ले उ॒भे। व्यच॑स्वती॒ सं व॑साथाम्भ॒र्तम॒ग्निम्पु॑री॒ष्यम्᳚। सं व॑साथाꣳ सुव॒र्विदा॑ स॒मीची॒ उर॑सा॒ त्मना᳚। अ॒ग्निम॒न्तर्भ॑रि॒ष्यन्ती॒ ज्योति॑ष्मन्त॒मज॑स्र॒मित्। पु॒री॒ष्यो॑\-ऽसि वि॒श्वभ॑राः। अथ॑र्वा त्वा प्रथ॒मो निर॑मन्थदग्ने। त्वाम॑ग्ने॒ पुष्क॑रा॒दध्यथ॑र्वा॒ निर॑मन्थत। मू॒र्ध्नो विश्व॑स्य वा॒घतः॑। तमु॑ त्वा द॒ध्यङ्ङृषिः॑ पु॒त्र ई॑धे॥११॥

%4.1.3.3
अथ॑र्वणः। वृ॒त्र॒हणं॑ पुरन्द॒रम्। तमु॑ त्वा पा॒थ्यो वृषा॒ समी॑धे दस्यु॒हन्त॑मम्। ध॒नं॒ज॒यꣳ रणे॑रणे। सीद॑ होतः॒ स्व उ॑ लो॒के चि॑कि॒त्वान्थ्सा॒दया॑ य॒ज्ञꣳ सु॑कृ॒तस्य॒ योनौ᳚। दे॒वा॒वीर्दे॒वान् ह॒विषा॑ यजा॒स्यग्ने॑ बृ॒हद्यज॑माने॒ वयो॑ धाः। नि होता॑ होतृ॒षद॑ने॒ विदा॑नस्त्वे॒षो दी॑दि॒वाꣳ अ॑सदथ्सु॒दक्षः॑। अद॑ब्धव्रतप्रमति॒र्वसि॑ष्ठः सहस्रम्भ॒रः शुचि॑जिह्वो अ॒ग्निः। सꣳ सी॑दस्व म॒हाꣳ अ॑सि॒ शोच॑स्व॥१२॥

%4.1.3.4
दे॒व॒वीत॑मः। वि धू॒मम॑ग्ने अरु॒षम्मि॑येध्य सृ॒ज प्र॑शस्त दर्\mbox{}श॒तम्। जनि॑ष्वा॒ हि जेन्यो॒ अग्रे॒ अह्नाꣳ॑ हि॒तो हि॒तेष्व॑रु॒षो वने॑षु। दमे॑दमे स॒प्त रत्ना॒ दधा॑नो॒\-ऽग्निर्\mbox{}होता॒ नि ष॑सादा॒ यजी॑यान्॥१३॥

%4.1.4.0
{\anuvakamend[{स्थ॒ ई॒धे॒ शोच॑स्व स॒प्तविꣳ॑शतिश्च}]}%॥३॥

%4.1.4.1
सं ते॑ वा॒युर्मा॑त॒रिश्वा॑ दधातूत्ता॒नायै॒ हृद॑यं॒ यद्विलि॑ष्टम्। दे॒वानां॒ यश्चर॑ति प्रा॒णथे॑न॒ तस्मै॑ च देवि॒ वष॑डस्तु॒ तुभ्यम्᳚। सुजा॑तो॒ ज्योति॑षा स॒ह शर्म॒ वरू॑थ॒मास॑दः॒ सुवः॑। वासो॑ अग्ने वि॒श्वरू॑प॒ꣳ॒ सं व्य॑यस्व विभावसो। उदु॑ तिष्ठ स्वध्व॒रावा॑ नो दे॒व्या कृ॒पा। दृ॒शे च॑ भा॒सा बृ॑ह॒ता सु॑शु॒क्वनि॒राग्ने॑ याहि सुश॒स्तिभिः॑।॥१४॥

%4.1.4.2
ऊ॒र्ध्व ऊ॒ षु ण॑ ऊ॒तये॒ तिष्ठा॑ दे॒वो न स॑वि॒ता। ऊ॒र्ध्वो वाज॑स्य॒ सनि॑ता॒ यद॒ञ्जिभि॑र्वा॒घद्भि॑र्वि॒ह्वया॑महे। स जा॒तो गर्भो॑ असि॒ रोद॑स्यो॒रग्ने॒ चारु॒र्विभृ॑त॒ ओष॑धीषु। चि॒त्रः शिशुः॒ परि॒ तमाꣳ॑स्य॒क्तः प्र मा॒तृभ्यो॒ अधि॒ कनि॑क्रदद्गाः। स्थि॒रो भ॑व वी॒ड्व॑ङ्ग आ॒शुर्भ॑व वा॒ज्य॑र्वन्न्। पृ॒थुर्भ॑व सु॒षद॒स्त्वम॒ग्नेः पु॑रीष॒वाह॑नः। शि॒वो भ॑व॥१५॥

%4.1.4.3
प्र॒जाभ्यो॒ मानु॑षीभ्य॒स्त्वम॑ङ्गिरः। मा द्यावा॑पृथि॒वी अ॒भि शू॑शुचो॒ मान्तरि॑क्षं॒ मा वन॒स्पतीन्॑। प्रैतु॑ वा॒जी कनि॑क्रद॒न्नान॑द॒द्रास॑भः॒ पत्वा᳚। भर॑न्न॒ग्निम्पु॑री॒ष्यं॑ मा पा॒द्यायु॑षः पु॒रा। रास॑भो वां॒ कनि॑क्रद॒थ्सुयु॑क्तो वृषणा॒ रथे᳚। स वा॑म॒ग्निम्पु॑री॒ष्य॑मा॒शुर्दू॒तो व॑हादि॒तः। वृषा॒ग्निं वृ॑षण॒म्भर॑न्न॒पां गर्भꣳ॑ समु॒द्रियम्᳚। अग्न॒ आ या॑हि॥१६॥

%4.1.4.4
वी॒तय॑ ऋ॒तꣳ स॒त्यम्। ओष॑धयः॒ प्रति॑ गृह्णीता॒ग्निमे॒तꣳ शि॒वमा॒यन्त॑म॒भ्यत्र॑ यु॒ष्मान्। व्यस्य॒न्विश्वा॒ अम॑ती॒ररा॑तीर्नि॒षीद॑न्नो॒ अप॑ दुर्म॒तिꣳ ह॑नत्। ओष॑धयः॒ प्रति॑ मोदध्वमेन॒म्पुष्पा॑वतीः सुपिप्प॒लाः। अ॒यं वो॒ गर्भ॑ ऋ॒त्वियः॑ प्र॒त्नꣳ स॒धस्थ॒मास॑दत्॥१७॥

%4.1.5.0
{\anuvakamend[{सु॒श॒स्तिभिः॑ शि॒वो भ॑व याहि॒ षट्त्रिꣳ॑शच्च}]}%॥४॥

%4.1.5.1
वि पाज॑सा पृ॒थुना॒ शोशु॑चानो॒ बाध॑स्व द्वि॒षो र॒क्षसो॒ अमी॑वाः। सु॒शर्म॑णो बृह॒तः शर्म॑णि स्याम॒ग्नेर॒हꣳ सु॒हव॑स्य॒ प्रणी॑तौ। आपो॒ हि ष्ठा म॑यो॒भुव॒स्ता न॑ ऊ॒र्जे द॑धातन। म॒हे रणा॑य॒ चक्ष॑से। यो वः॑ शि॒वत॑मो॒ रस॒स्तस्य॑ भाजयते॒ह नः॑। उ॒श॒तीरि॑व मा॒तरः॑। तस्मा॒ अरं॑ गमाम वो॒ यस्य॒ क्षया॑य॒ जिन्व॑थ। आपो॑ ज॒नय॑था च नः। मि॒त्रः॥१८॥

%4.1.5.2
स॒ꣳ॒सृज्य॑ पृथि॒वीम्भूमिं॑ च॒ ज्योति॑षा स॒ह। सुजा॑तं जा॒तवे॑दसम॒ग्निं वै᳚श्वान॒रं वि॒भुम्। अ॒य॒क्ष्माय॑ त्वा॒ सꣳ सृ॑जामि प्र॒जाभ्यः॑। विश्वे᳚ त्वा दे॒वा वै᳚श्वान॒राः सꣳ सृ॑ज॒न्त्वानु॑ष्टुभेन॒ छन्द॑साङ्गिर॒स्वत्। रु॒द्राः स॒म्भृत्य॑ पृथि॒वीम्बृ॒हज्ज्योतिः॒ समी॑धिरे। तेषां᳚ भा॒नुरज॑स्र॒ इच्छु॒क्रो दे॒वेषु॑ रोचते। सꣳसृ॑ष्टां॒ वसु॑भी रु॒द्रैर्धीरैः᳚ कर्म॒ण्या᳚म्मृदम्᳚। हस्ता᳚भ्याम्मृ॒द्वीं कृ॒त्वा सि॑नीवा॒ली क॑रोतु॥१९॥

%4.1.5.3
ताम्। सि॒नी॒वा॒ली सु॑कप॒र्दा सु॑कुरी॒रा स्वौ॑प॒शा। सा तुभ्य॑मदिते मह॒ ओखां द॑धातु॒ हस्त॑योः। उ॒खां क॑रोतु॒ शक्त्या॑ बा॒हुभ्या॒मदि॑तिर्धि॒या। मा॒ता पु॒त्रं यथो॒पस्थे॒ साग्निम्बि॑भर्तु॒ गर्भ॒ आ। म॒खस्य॒ शिरो॑\-ऽसि य॒ज्ञस्य॑ प॒दे स्थः॑। वस॑वस्त्वा कृण्वन्तु गाय॒त्रेण॒ छन्द॑साङ्गिर॒स्वत्पृ॑थि॒व्य॑सि रु॒द्रास्त्वा॑ कृण्वन्तु॒ त्रैष्टु॑भेन॒ छन्द॑साङ्गिर॒स्वद॒न्तरि॑क्षमसि॥२०॥

%4.1.5.4
आ॒दि॒त्यास्त्वा॑ कृण्वन्तु॒ जाग॑तेन॒ छन्द॑साङ्गिर॒स्वद्द्यौर॑सि॒ विश्वे᳚ त्वा दे॒वा वै᳚श्वान॒राः कृ॑ण्व॒न्त्वानु॑ष्टुभेन॒ छन्द॑साङ्गिर॒स्वद्दिशो॑\-ऽसि ध्रु॒वासि॑ धा॒रया॒ मयि॑ प्र॒जाꣳ रा॒यस्पोषं॑ गौप॒त्यꣳ सु॒वीर्यꣳ॑ सजा॒तान् यज॑माना॒यादि॑त्यै॒ रास्ना॒स्यदि॑तिस्ते॒ बिलं॑ गृह्णातु॒ पाङ्क्ते॑न॒ छन्द॑साङ्गिर॒स्वत्। कृ॒त्वाय॒ सा म॒हीमु॒खाम्मृ॒न्मयीं॒ योनि॑म॒ग्नये᳚। ताम्पु॒त्रेभ्यः॒ सम्प्राय॑च्छ॒ददि॑तिः श्र॒पया॒निति॑॥२१॥

%4.1.6.0
{\anuvakamend[{मि॒त्रः क॑रोत्व॒न्तरि॑क्षमसि॒ प्र च॒त्वारि॑ च}]}%॥५॥

%4.1.6.1
वस॑वस्त्वा धूपयन्तु गाय॒त्रेण॒ छन्द॑साङ्गिर॒स्वद्रु॒द्रास्त्वा॑ धूपयन्तु॒ त्रैष्टु॑भेन॒ छन्द॑साङ्गिर॒स्वदा॑दि॒त्यास्त्वा॑ धूपयन्तु॒ जाग॑तेन॒ छन्द॑साङ्गिर॒स्वद्विश्वे᳚ त्वा दे॒वा वै᳚श्वान॒रा धू॑पय॒न्त्वानु॑ष्टुभेन॒ छन्द॑साङ्गिर॒स्वदिन्द्र॑स्त्वा धूपयत्वङ्गिर॒स्वद्विष्णु॑स्त्वा धूपयत्वङ्गिर॒स्वद्वरु॑णस्त्वा धूपयत्वङ्गिर॒स्वददि॑तिस्त्वा दे॒वी वि॒श्वदे᳚व्यावती पृथि॒व्याः स॒धस्थे᳚\-ऽङ्गिर॒स्वत्ख॑नत्ववट दे॒वानां᳚ त्वा॒ पत्नीः᳚॥२२॥

%4.1.6.2
दे॒वीर्वि॒श्वदे᳚व्यावतीः पृथि॒व्याः स॒धस्थे᳚\-ऽङ्गिर॒स्वद्द॑धतूखे धि॒षणा᳚स्त्वा दे॒वीर्वि॒श्वदे᳚व्यावतीः पृथि॒व्याः स॒धस्थे᳚\-ऽङ्गिर॒स्वद॒भीन्ध॑तामुखे॒ ग्नास्त्वा॑ दे॒वीर्वि॒श्वदे᳚व्यावतीः पृथि॒व्याः स॒धस्थे᳚\-ऽङ्गिर॒स्वच्छ्र॑पयन्तूखे॒ वरू᳚त्रयो॒ जन॑यस्त्वा दे॒वीर्वि॒श्वदे᳚व्यावतीः पृथि॒व्याः स॒धस्थे᳚\-ऽङ्गिर॒स्वत्प॑चन्तूखे। मित्रै॒तामु॒खाम्प॑चै॒षा मा भे॑दि। एातां ते॒ परि॑ ददा॒म्यभि॑त्त्यै। अ॒भीमाम्॥२३॥

%4.1.6.3
म॒हि॒ना दिव॑म्मि॒त्रो ब॑भूव स॒प्रथाः᳚। उ॒त श्रव॑सा पृथि॒वीम्। मि॒त्रस्य॑ चर्\mbox{}षणी॒धृतः॒ श्रवो॑ दे॒वस्य॑ सान॒सिम्। द्यु॒म्नं चि॒त्रश्र॑वस्तमम्। दे॒वस्त्वा॑ सवि॒तोद्व॑पतु सुपा॒णिः स्व॑ङ्गु॒॒रिः। सु॒बा॒हुरु॒त शक्त्या᳚। अप॑द्यमाना पृथि॒व्याशा॒ दिश॒ आ पृ॑ण। उत्ति॑ष्ठ बृह॒ती भ॑वो॒र्ध्वा ति॑ष्ठ ध्रु॒वा त्वम्। वस॑व॒स्त्वाच्छृ॑न्दन्तु गाय॒त्रेण॒ छन्द॑साङ्गिर॒स्वद्रु॒द्रास्त्वा च्छृ॑न्दन्तु॒ त्रैष्टु॑भेन॒ छन्द॑साङ्गिर॒स्वदा॑दि॒त्यास्त्वाच्छृ॑न्दन्तु॒ जाग॑तेन॒ छन्द॑साङ्गिर॒स्वद्विश्वे᳚ त्वा दे॒वा वै᳚श्वान॒रा आच्छृ॑न्द॒न्त्वानु॑ष्टुभेन॒ छन्द॑साङ्गिर॒स्वत्॥२४॥

%4.1.7.0
{\anuvakamend[{पत्नी॑रि॒माꣳ रु॒द्रास्त्वाच्छृ॑न्द॒न्त्वेका॒न्नविꣳ॑श॒तिश्च॑}]}%॥६॥

%4.1.7.1
समा᳚स्त्वाग्न ऋ॒तवो॑ वर्धयन्तु संवथ्स॒रा ऋष॑यो॒ यानि॑ स॒त्या। सं दि॒व्येन॑ दीदिहि रोच॒नेन॒ विश्वा॒ आ भा॑हि प्र॒दिशः॑ पृथि॒व्याः। सं चे॒ध्यस्वा᳚ग्ने॒ प्र च॑ बोधयैन॒मुच्च॑ तिष्ठ मह॒ते सौभ॑गाय। मा च॑ रिषदुपस॒त्ता ते॑ अग्ने ब्र॒ह्माण॑स्ते य॒शसः॑ सन्तु॒ मान्ये। त्वाम॑ग्ने वृणते ब्राह्म॒णा इ॒मे शि॒वो अ॑ग्ने॥२५॥

%4.1.7.2
सं॒वर॑णे भवा नः। स॒प॒त्न॒हा नो॑ अभिमाति॒जिच्च॒ स्वे गये॑ जागृ॒ह्यप्र॑युच्छन्न्। इ॒हैवाग्ने॒ अधि॑ धारया र॒यिं मा त्वा॒ नि क्र॑न्पूर्व॒चितो॑ निका॒रिणः॑। क्ष॒त्रम॑ग्ने सु॒यम॑मस्तु॒ तुभ्य॑मुपस॒त्ता व॑र्धतां ते॒ अनि॑ष्टृतः। क्ष॒त्रेणा᳚ग्ने॒ स्वायुः॒ सꣳ र॑भस्व मि॒त्रेणा᳚ग्ने मित्र॒धेये॑ यतस्व। स॒जा॒ताना᳚म्मध्यम॒स्था ए॑धि॒ राज्ञा॑मग्ने विह॒व्यो॑ दीदिही॒ह। अति॑॥२६॥

%4.1.7.3
निहो॒ अति॒ स्रिधो\-ऽत्यचि॑त्ति॒मत्यरा॑तिमग्ने। विश्वा॒ ह्य॑ग्ने दुरि॒ता सह॒स्वाथा॒स्मभ्यꣳ॑ स॒हवी॑राꣳ र॒यिं दाः᳚। अ॒ना॒धृ॒ष्यो जा॒तवे॑दा॒ अनि॑ष्टृतो वि॒राड॑ग्ने क्षत्र॒भृद्दी॑दिही॒ह। विश्वा॒ आशाः᳚ प्रमु॒ञ्चन्मानु॑षीर्भि॒यः शि॒वाभि॑र॒द्य परि॑ पाहि नो वृ॒धे। बृह॑स्पते सवितर्बो॒धयै॑न॒ꣳ॒ सꣳशि॑तं चिथ्सन्त॒राꣳ सꣳ शि॑शाधि। व॒र्धयै॑नम्मह॒ते सौभ॑गाय॥२७॥

%4.1.7.4
विश्व॑ एन॒मनु॑ मदन्तु दे॒वाः। अ॒मु॒त्र॒भूया॒दध॒ यद्य॒मस्य॒ बृह॑स्पते अ॒भिश॑स्ते॒रमु॑ञ्चः। प्रत्यौ॑हताम॒श्विना॑ मृ॒त्युम॑स्माद्दे॒वाना॑मग्ने भि॒षजा॒ शची॑भिः। उद्व॒यं तम॑स॒स्परि॒ पश्य॑न्तो॒ ज्योति॒रुत्त॑रम्। दे॒वं दे॑व॒त्रा सूर्य॒मग॑न्म॒ ज्योति॑रुत्त॒मम्॥२८॥

%4.1.8.0
{\anuvakamend[{इ॒मे शि॒वो अ॒ग्ने\-ऽति॒ सौभ॑गाय॒ चतु॑स्त्रिꣳशच्च}]}%॥७॥

%4.1.8.1
ऊ॒र्ध्वा अ॑स्य स॒मिधो॑ भवन्त्यू॒र्ध्वा शु॒क्रा शो॒चीꣳष्य॒ग्नेः। द्यु॒मत्त॑मा सु॒प्रती॑कस्य सू॒नोः। तनू॒नपा॒दसु॑रो वि॒श्ववे॑दा दे॒वो दे॒वेषु॑ दे॒वः। प॒थ आन॑क्ति॒ मध्वा॑ घृ॒तेन॑। मध्वा॑ य॒ज्ञं न॑क्षसे प्रीणा॒नो नरा॒शꣳसो॑ अग्ने। सु॒कृद्दे॒वः स॑वि॒ता वि॒श्ववा॑रः। अच्छा॒यमे॑ति॒ शव॑सा घृ॒तेने॑डा॒नो वह्नि॒र्नम॑सा। अ॒ग्निꣴ स्रुचो॑ अध्व॒रेषु॑ प्र॒यथ्सु॑। स य॑क्षदस्य महि॒मान॑म॒ग्नेः सः॥२९॥

%4.1.8.2
ई॒ म॒न्द्रासु॑ प्र॒यसः॑। वसु॒श्चेति॑ष्ठो वसु॒धात॑मश्च। द्वारो॑ दे॒वीरन्व॑स्य॒ विश्वे᳚ व्र॒ता द॑दन्ते अ॒ग्नेः। उ॒रु॒व्यच॑सो॒ धाम्ना॒ पत्य॑मानाः। ते अ॑स्य॒ योष॑णे दि॒व्ये न योना॑वु॒षासा॒नक्ता᳚। इ॒मं य॒ज्ञम॑वतामध्व॒रं नः॑। दैव्या॑ होतारावू॒र्ध्वम॑ध्व॒रं नो॒\-ऽग्नेर्जि॒ह्वाम॒भि गृ॑णीतम्। कृ॒णु॒तं नः॒ स्वि॑ष्टिम्। ति॒स्रो दे॒वीर्ब॒र्हिरेदꣳ स॑द॒न्त्विडा॒ सर॑स्वती॥३०॥

%4.1.8.3
भार॑ती। म॒ही गृ॑णा॒ना। तन्न॑स्तु॒रीप॒मद्भु॑तं पुरु॒क्षु त्वष्टा॑ सु॒वीरम्᳚। रा॒यस्पोषं॒ वि ष्य॑तु॒ नाभि॑म॒स्मे। वन॑स्प॒ते\-ऽव॑ सृजा॒ ररा॑ण॒स्त्मना॑ दे॒वेषु॑। अ॒ग्निर्\mbox{}ह॒व्यꣳ श॑मि॒ता सू॑दयाति। अग्ने॒ स्वाहा॑ कृणुहि जातवेद॒ इन्द्रा॑य ह॒व्यम्। विश्वे॑ दे॒वा ह॒विरि॒दं जु॑षन्ताम्। हि॒र॒ण्य॒ग॒र्भः सम॑वर्त॒ताग्रे॑ भू॒तस्य॑ जा॒तः पति॒रेक॑ आसीत्। स दा॑धार पृथि॒वीं द्याम्॥३१॥

%4.1.8.4
उ॒तेमां कस्मै॑ दे॒वाय॑ ह॒विषा॑ विधेम। यः प्रा॑ण॒तो नि॑मिष॒तो म॑हि॒त्वैक॒ इद्राजा॒ जग॑तो ब॒भूव॑। य ईशे॑ अ॒स्य द्वि॒पद॒श्चतु॑ष्पदः॒ कस्मै॑ दे॒वाय॑ ह॒विषा॑ विधेम। य आ᳚त्म॒दा ब॑ल॒दा यस्य॒ विश्व॑ उ॒पास॑ते प्र॒शिषं॒ यस्य॑ दे॒वाः। यस्य॑ छा॒यामृतं॒ यस्य॑ मृ॒त्युः कस्मै॑ दे॒वाय॑ ह॒विषा॑ विधेम। यस्ये॒मे हि॒मव॑न्तो महि॒त्वा यस्य॑ समु॒द्रꣳ र॒सया॑ स॒ह॥३२॥

%4.1.8.5
आ॒हुः। यस्ये॒माः प्र॒दिशो॒ यस्य॑ बा॒हू कस्मै॑ दे॒वाय॑ ह॒विषा॑ विधेम। यं क्रन्द॑सी॒ अव॑सा तस्तभा॒ने अ॒भ्यैक्षे॑ता॒म्मन॑सा॒ रेज॑माने। यत्राधि॒ सूर॒ उदि॑तौ॒ व्येति॒ कस्मै॑ दे॒वाय॑ ह॒विषा॑ विधेम। येन॒ द्यौरु॒ग्रा पृ॑थि॒वी च॑ दृ॒ढे येन॒ सुवः॑ स्तभि॒तं येन॒ नाकः॑। यो अ॒न्तरि॑क्षे॒ रज॑सो वि॒मानः॒ कस्मै॑ दे॒वाय॑ ह॒विषा॑ विधेम। आपो॑ ह॒ यन्म॑ह॒तीर्विश्वम्᳚॥३३॥

%4.1.8.6
आय॒न्दक्षं॒ दधा॑ना ज॒नय॑न्तीर॒ग्निम्। ततो॑ दे॒वानां॒ निर॑वर्त॒तासु॒रेकः॒ कस्मै॑ दे॒वाय॑ ह॒विषा॑ विधेम। यश्चि॒दापो॑ महि॒ना प॒र्यप॑श्य॒द्दक्षं॒ दधा॑ना ज॒नय॑न्तीर॒ग्निम्। यो दे॒वेष्वधि॑ दे॒व एक॒ आसी॒त्कस्मै॑ दे॒वाय॑ ह॒विषा॑ विधेम॥३४॥

%4.1.9.0
{\anuvakamend[{अ॒ग्नेः स सर॑स्वती॒ द्याꣳ स॒ह विश्व॒ञ्चतु॑स्त्रिHꣳशश्च}]}%॥८॥

%4.1.9.1
आकू॑तिम॒ग्निम्प्र॒युज॒ꣴ॒ स्वाहा॒ मनो॑ मे॒धाम॒ग्निम्प्र॒युज॒ꣴ॒ स्वाहा॑ चि॒त्तं विज्ञा॑तम॒ग्निम्प्र॒युज॒ꣴ॒ स्वाहा॑ वा॒चो विधृ॑तिम॒ग्निम्प्र॒युज॒ꣴ॒ स्वाहा᳚ प्र॒जाप॑तये॒ मन॑वे॒ स्वाहा॒ग्नये॑ वैश्वान॒राय॒ स्वाहा॒ विश्वे॑ दे॒वस्य॑ ने॒तुर्मर्तो॑ वृणीत स॒ख्यं विश्वे॑ रा॒य इ॑षुध्यसि द्यु॒म्नं वृ॑णीत पु॒ष्यसे॒ स्वाहा॒ मा सु भि॑त्था॒ मा सु रि॑षो॒ दृꣳह॑स्व वी॒डय॑स्व॒ सु। अम्ब॑ धृष्णु वी॒रय॑स्व॥३५॥

%4.1.9.2
अ॒ग्निश्चे॒दं क॑रिष्यथः। दृꣳह॑स्व देवि पृथिवि स्व॒स्तय॑ आसु॒री मा॒या स्व॒धया॑ कृ॒तासि॑। जुष्टं॑ दे॒वाना॑मि॒दम॑स्तु ह॒व्यमरि॑ष्टा॒ त्वमुदि॑हि य॒ज्ञे अ॒स्मिन्न्। मित्रै॒तामु॒खां त॑पै॒षा मा भे॑दि। ए॒तान्ते॒ परि॑ ददा॒म्यभि॑त्त्यै। द्र्व॑न्नः स॒र्पिरा॑सुतिः प्र॒त्नो होता॒ वरे᳚ण्यः। सह॑सस्पु॒त्रो अद्भु॑तः। पर॑स्या॒ अधि॑ सं॒वतो\-ऽव॑राꣳ अभ्या॥३६॥

%4.1.9.3
त॒र॒। यत्रा॒हमस्मि॒ ताꣳ अ॑व। प॒र॒मस्याः᳚ परा॒वतो॑ रो॒हिद॑श्व इ॒हा ग॑हि। पु॒री॒ष्यः॑ पुरुप्रि॒यो\-ऽग्ने॒ त्वं त॑रा॒ मृधः॑। सीद॒ त्वं मा॒तुर॒स्या उ॒पस्थे॒ विश्वा᳚न्यग्ने व॒युना॑नि वि॒द्वान्। मैना॑म॒र्चिषा॒ मा तप॑सा॒भि शू॑शुचो॒\-ऽन्तर॑स्याꣳ शु॒क्रज्यो॑ति॒र्वि भा॑हि। अ॒न्तर॑ग्ने रु॒चा त्वमु॒खायै॒ सद॑ने॒ स्वे। तस्या॒स्त्वꣳ हर॑सा॒ तप॒ञ्जात॑वेदः शि॒वो भ॑व। शि॒वो भू॒त्वा मह्य॑म॒ग्ने\-ऽथो॑ सीद शि॒वस्त्वम्। शि॒वाः कृ॒त्वा दिशः॒ सर्वाः॒ स्वां योनि॑मि॒हास॑दः॥३७॥

%4.1.10.0
{\anuvakamend[{वी॒रय॒स्वा तप॑न्विꣳश॒तिश्च॑}]}%॥९॥

%4.1.10.1
यद॑ग्ने॒ यानि॒ कानि॒ चा ते॒ दारू॑णि द॒ध्मसि॑। तद॑स्तु॒ तुभ्य॒मिद्घृ॒तं तज्जु॑षस्व यविष्ठ्य। यदत्त्यु॑प॒जिह्वि॑का॒ यद्व॒म्रो अ॑ति॒सर्प॑ति। सर्वं॒ तद॑स्तु ते घृ॒तं तज्जु॑षस्व यविष्ठ्य। रात्रि॑ꣳरात्रि॒मप्र॑याव॒म्भर॒न्तो\-ऽश्वा॑येव॒ तिष्ठ॑ते घा॒समस्मै। रा॒यस्पोषे॑ण॒ समि॒षा मद॒न्तो\-ऽग्ने॒ मा ते॒ प्रति॑वेशा रिषाम। नाभा᳚॥३८॥

%4.1.10.2
पृ॒थि॒व्याः स॑मिधा॒नम॒ग्निꣳ रा॒यस्पोषा॑य बृह॒ते ह॑वामहे। इ॒र॒म्म॒दम्बृ॒हदु॑क्थं॒ यज॑त्रं॒ जेता॑रम॒ग्निं पृ॑तनासु सास॒हिम्। याः सेना॑ अ॒भीत्व॑रीराव्या॒धिनी॒रुग॑णा उ॒त। ये स्ते॒ना ये च॒ तस्क॑रा॒स्ताꣴस्ते॑ अ॒ग्ने\-ऽपि॑ दधाम्या॒स्ये᳚। दꣴष्ट्रा᳚भ्याम्म॒लिम्लू॒ञ्जम्भ्यै॒स्तस्क॑राꣳ उ॒त। हनू᳚भ्याꣴस्ते॒नान्भ॑गव॒स्ताꣴस्त्वं खा॑द॒ सुखा॑दितान्। ये जने॑षु म॒लिम्ल॑वः स्ते॒नास॒स्तस्क॑रा॒ वने᳚। ये॥३९॥

%4.1.10.3
कक्षे᳚ष्वघा॒यव॒स्ताꣴस्ते॑ दधामि॒ जम्भ॑योः। यो अ॒स्मभ्य॑मराती॒याद्यश्च॑ नो॒ द्वेष॑ते॒ जनः॑। निन्दा॒द्यो अ॒स्मान् दिफ्सा᳚च्च॒ सर्वं॒ तम्म॑स्म॒सा कु॑रु। सꣳशि॑तं मे॒ ब्रह्म॒ सꣳशि॑तं वीर्यं॑ बलम्᳚। सꣳशि॑तं क्ष॒त्रं जि॒ष्णु यस्या॒हमस्मि॑ पु॒रोहि॑तः। उदे॑षाम्बा॒हू अ॑तिर॒मुद्वर्च॒ उदू॒ बलम्᳚। क्षि॒णोमि॒ ब्रह्म॑णा॒मित्रा॒नुन्न॑यामि॥४०॥

%4.1.10.4
स्वाꣳ अ॒हम्। दृ॒शा॒नो रु॒क्म उ॒र्व्या व्य॑द्यौद्दु॒र्मर्\mbox{}ष॒मायुः॑ श्रि॒ये रु॑चा॒नः। अ॒ग्निर॒मृतो॑ अभव॒द्वयो॑भि॒र्यदे॑नं॒ द्यौरज॑नयथ्सु॒रेताः᳚। विश्वा॑ रू॒पाणि॒ प्रति॑ मुञ्चते क॒विः प्रासा॑वीद्भ॒द्रं द्वि॒पदे॒ चतु॑ष्पदे। वि नाक॑मख्यथ्सवि॒ता वरे॒ण्यो\-ऽनु॑ प्र॒याण॑मु॒षसो॒ वि रा॑जति। नक्तो॒षासा॒ सम॑नसा॒ विरू॑पे धा॒पये॑ते॒ शिशु॒मेकꣳ॑ समी॒ची। द्यावा॒ क्षामा॑ रु॒क्मः॥४१॥

%4.1.10.5
अ॒न्तर्वि भा॑ति दे॒वा अ॒ग्निं धा॑रयन्द्रविणो॒दाः। सु॒प॒र्णो\-ऽसि ग॒रुत्मा᳚न्त्रि॒वृत्ते॒ शिरो॑ गाय॒त्रं चक्षुः॒ स्तोम॑ आ॒त्मा साम॑ ते त॒नूर्वा॑मदे॒व्यम्बृ॑हद्रथन्त॒रे प॒क्षौ य॑ज्ञाय॒ज्ञिय॒म्पुच्छं॒ छन्दा॒ꣳ॒स्यङ्गा॑नि॒ धिष्णि॑याः श॒फा॒ यजूꣳ॑षि॒ नाम॑। सु॒प॒र्णो॑\-ऽसि ग॒रुत्मा॒न्दिवं॑ गच्छ॒ सुवः॑ पत॥४२॥

%4.1.11.0
{\anuvakamend[{नाभा॒ वने॒ येन॑ यामि॒ क्षामा॑ रु॒क्मो᳚\-ऽष्टात्रिꣳ॑शच्च}]}%॥10॥

%4.1.11.1
अग्ने॒ यं य॒ज्ञम॑ध्व॒रं वि॒श्वतः॑ परि॒भूरसि॑। स इद्दे॒वेषु॑ गच्छति। सोम॒ यास्ते॑ मयो॒भुव॑ ऊ॒तयः॒ सन्ति॑ दा॒शुषे᳚। ताभि॑र्नो\-ऽवि॒ता भ॑व। अ॒ग्निर्मू॒र्धा भुवः॑। त्वं नः॑ सोम॒ या ते॒ धामा॑नि। तथ्स॑वि॒तुर्वरे᳚ण्य॒म्भर्गो॑ दे॒वस्य॑ धीमहि। धियो॒ यो नः॑ प्रचो॒दया᳚त्। अचि॑त्ती॒ यच्च॑कृ॒मा दैव्ये॒ जने॑ दी॒नैर्दक्षैः॒ प्रभू॑ती पूरुष॒त्वता᳚।॥४३॥

%4.1.11.2
दे॒वेषु॑ च सवित॒र्मानु॑षेषु च॒ त्वं नो॒ अत्र॑ सुवता॒दना॑गसः। चो॒द॒यि॒त्री सू॒नृता॑नां॒ चेत॑न्ती सुमती॒नाम्। य॒ज्ञं द॑धे॒ सर॑स्वती। पावी॑रवी क॒न्या॑ चि॒त्रायुः॒ सर॑स्वती वी॒रप॑त्नी॒ धियं॑ धात्। ग्नाभि॒रच्छि॑द्रꣳ शर॒णꣳ स॒जोषा॑ दुरा॒धर्\mbox{}षं॑ गृण॒ते शर्म॑ यꣳसत्। पू॒षा गा अन्वे॑तु नः पू॒षा र॑क्ष॒त्वर्व॑तः। पू॒षा वाजꣳ॑ सनोतु नः। शु॒क्रं ते॑ अ॒न्यद्य॑ज॒तं ते॑ अ॒न्यत्॥४४॥

%4.1.11.3
विषु॑रूपे॒ अह॑नी॒ द्यौरि॑वासि। विश्वा॒ हि मा॒या अव॑सि स्वधावो भ॒द्रा ते॑ पूषन्नि॒ह रा॒तिर॑स्तु। ते॑\-ऽवर्धन्त॒ स्वत॑वसो महित्व॒ना नाकं॑ त॒स्थुरु॒रु च॑क्रिरे॒ सदः॑। विष्णु॒र्यद्धाव॒द्वृष॑णम्मद॒च्युतं॒ वयो॒ न सी॑द॒न्नधि॑ ब॒र्\mbox{}हिषि॑ प्रि॒ये। प्र चि॒त्रम॒र्कं गृ॑ण॒ते तु॒राय॒ मारु॑ताय॒ स्वत॑वसे भरध्वम्। ये सहाꣳ॑सि॒ सह॑सा॒ सह॑न्ते॥४५॥

%4.1.11.4
रेज॑ते अग्ने पृथि॒वी म॒खेभ्यः॑। विश्वे॑ दे॒वा विश्वे॑ देवाः। द्यावा॑ नः पृथि॒वी इ॒मꣳ सि॒ध्रम॒द्य दि॑वि॒स्पृशम्᳚। य॒ज्ञं दे॒वेषु॑ यच्छताम्। प्र पू᳚र्व॒जे पि॒तरा॒ नव्य॑सीभिर्गी॒र्भिः कृ॑णुध्व॒ꣳ॒ सद॑ने ऋ॒तस्य॑। आ नो᳚ द्यावापृथिवी॒ दैव्ये॑न॒ जने॑न यात॒म्महि॑ वां॒ वरू॑थम्। अ॒ग्निꣴ स्तोमे॑न बोधय समिधा॒नो अम॑र्त्यम्। ह॒व्या दे॒वेषु॑ नो दधत्। स ह॑व्य॒वाडम॑र्त्य उ॒शिग्दू॒तश्चनो॑हितः। अ॒ग्निर्धि॒या समृ॑ण्वति। शं नो॑ भवन्तु॒ वाजे॑वाजे॥४६॥

%4.2.0.0
{\anuvakamend[{पू॒रु॒ष॒त्वता॑ यज॒तन्ते॑ अ॒न्यथ्सह॑न्ते॒ चनो॑हितो॒\-ऽष्टौ च॑}]}%॥11॥

%4.2.0.0

{\anuvakamend[{विष्णोः॒ क्रमो॑\-ऽसि दि॒वस्पर्यन्न॑प॒ते\-ऽपे॑त॒ समि॑तं॒ या जा॒ता मा नो॑ हिꣳसीद्ध्रु॒वा\-ऽस्या॑दि॒त्यङ्गर्भ॒मिन्द्रा᳚ग्नी रोच॒नैका॑दश}]}%॥11॥

\prashnaend{विष्णो॑रस्मिन् ह॒व्येति॑ त्वा॒\-ऽहं धी॒तिभि॒र्\mbox{}होत्रा॑ अ॒ष्टाच॑त्वारिꣳशत्॥48॥ विष्णोः॒ क्रमो॑\-ऽसि॒ स त्वन्नो॑ अग्ने॥}
%%% END PRASHNA

\sect{द्वितीयः प्रश्नः}\setcounter{anuvakam}{0}
\dnsub{तैत्तिरीयसंहितायां चतुर्थकाण्डे द्वितीयः प्रश्नः}
%4.2.1.0
%4.2.1.1
विष्णोः॒ क्रमो᳚\-ऽस्यभिमाति॒हा गा॑य॒त्रं छन्द॒ आ रो॑ह पृथि॒वीमनु॒ वि क्र॑मस्व॒ निर्भ॑क्तः॒ स यं द्वि॒ष्मो विष्णोः॒ क्रमो᳚\-ऽस्यभिशस्ति॒हा त्रैष्टु॑भं॒ छन्द॒ आ रो॑हा॒न्तरि॑क्ष॒मनु॒ वि क्र॑मस्व॒ निर्भ॑क्तः॒ स यं द्वि॒ष्मो विष्णोः॒ क्रमो᳚\-ऽ स्यरातीय॒तो ह॒न्ता जाग॑तं॒ छन्द॒ आ रो॑ह॒ दिव॒मनु॒ वि क्र॑मस्व॒ निर्भ॑क्तः॒ स यं द्वि॒ष्मो विष्णोः᳚॥१॥

%4.2.1.2
क्रमो॑\-ऽसि शत्रूय॒तो ह॒न्तानु॑ष्टुभं॒ छन्द॒ आ रो॑ह॒ दिशो\-ऽनु॒ वि क्र॑मस्व॒ निर्भ॑क्तः॒ स यं द्वि॒ष्मः। अक्र॑न्दद॒ग्निः स्त॒नय॑न्निव॒ द्यौः क्षामा॒ रेरि॑हद्वी॒रुधः॑ सम॒ञ्जन्न्। स॒द्यो ज॑ज्ञा॒नो वि हीमि॒द्धो अख्य॒दा रोद॑सी भा॒नुना॑ भात्य॒न्तः। अग्ने᳚\-ऽभ्यावर्तिन्न॒भि न॒ आ व॑र्त॒स्वायु॑षा॒ वर्च॑सा स॒न्या मे॒धया᳚ प्र॒जया॒ धने॑न। अग्ने᳚॥२॥

%4.2.1.3
अ॒ङ्गि॒रः॒ श॒तं ते॑ सन्त्वा॒वृतः॑ स॒हस्रं॑ त उपा॒वृतः॑। तासा॒म्पोष॑स्य॒ पोषे॑ण॒ पुन॑र्नो न॒ष्टमा कृ॑धि॒ पुन॑र्नो र॒यिमा कृ॑धि। पुन॑रू॒र्जा नि व॑र्तस्व॒ पुन॑रग्न इ॒षायु॑षा। पुन॑र्नः पाहि वि॒श्वतः॑। स॒ह र॒य्या नि व॑र्त॒स्वाग्ने॒ पिन्व॑स्व॒ धार॑या। वि॒श्वफ्स्नि॑या वि॒श्वत॒स्परि॑। उदु॑त्त॒मं व॑रुण॒ पाश॑म॒स्मदवा॑ध॒मम्॥३॥

%4.2.1.4
वि म॑ध्य॒मꣴ श्र॑थाय। अथा॑ व॒यमा॑दित्य व्र॒ते तवाना॑गसो॒ अदि॑तये स्याम। आ त्वा॑हार्\mbox{}षम॒न्तर॑भूर्ध्रु॒वस्ति॒ष्ठा\-वि॑चाचलिः। विश॑स्त्वा॒ सर्वा॑ वाञ्छन्त्व॒स्मिन्रा॒ष्ट्रमधि॑ श्रय। अग्रे॑ बृ॒हन्नु॒षसा॑मू॒र्ध्वो अ॑स्थान्निर्जग्मि॒वान्तम॑सो॒ ज्योति॒षागा᳚त्। अ॒ग्निर्भा॒नुना॒ रुश॑ता॒ स्वङ्ग॒ आ जा॒तो विश्वा॒ सद्मा᳚न्यप्राः। सीद॒ त्वं मा॒तुर॒स्याः॥४॥

%4.2.1.5
उ॒पस्थे॒ विश्वा᳚न्यग्ने व॒युना॑नि वि॒द्वान्। मैना॑म॒र्चिषा॒ मा तप॑सा॒भि शू॑शुचो॒\-ऽन्तर॑स्याꣳ शु॒क्रज्यो॑ति॒र्वि भा॑हि। अ॒न्तर॑ग्ने रु॒चा त्वमु॒खायै॒ सद॑ने॒ स्वे। तस्या॒स्त्वꣳ हर॑सा॒ तप॒ञ्जात॑वेदः शि॒वो भ॑व। शि॒वो भू॒त्वा मह्य॑म॒ग्ने\-ऽथो॑ सीद शि॒वस्त्वम्। शि॒वाः कृ॒त्वा दिशः॒ सर्वाः॒ स्वं योनि॑मि॒हास॑दः। ह॒ꣳ॒सः शु॑चि॒षद्वसु॑रन्तरिक्ष॒सद्धोता॑ वेदि॒षदति॑थिर्दुरोण॒सत्। नृ॒षद्व॑र॒सदृ॑त॒सद्व्यो॑म॒सद॒ब्जा गो॒जा ऋ॑त॒जा अ॑द्रि॒जा ऋ॒तम्बृ॒हत्॥५॥

%4.2.2.0
{\anuvakamend[{दिव॒मनु॒ वि क्र॑मस्व॒ निर्भ॑क्तः॒ स यं द्वि॒ष्मो विष्णो॒र्धने॒नाग्ने॑\-ऽध॒मम॒स्याः शु॑चि॒षथ्षोड॑श च}]}%॥१॥

%4.2.2.1
दि॒वस्परि॑ प्रथ॒मं ज॑ज्ञे अ॒ग्निर॒स्मद्द्वि॒तीयं॒ परि॑ जा॒तवे॑दाः। तृ॒तीय॑म॒फ्सु नृ॒मणा॒ अज॑स्र॒मिन्धा॑न एनं जरते स्वा॒धीः। वि॒द्मा ते॑ अग्ने त्रे॒धा त्र॒याणि॑ वि॒द्मा ते॒ सद्म॒ विभृ॑तं पुरु॒त्रा। वि॒द्मा ते॒ नाम॑ पर॒मं गुहा॒ यद्वि॒द्मा तमुथ्सं॒ यत॑ आज॒गन्थ॑। स॒मु॒द्रे त्वा॑ नृ॒मणा॑ अ॒फ्स्व॑न्तर्नृ॒चक्षा॑ ईधे दि॒वो अ॑ग्न॒ ऊधन्न्॑। तृ॒तीये᳚ त्वा॥६॥

%4.2.2.2
रज॑सि तस्थि॒वाꣳस॑मृ॒तस्य॒ योनौ॑ महि॒षा अ॑हिन्वन्न्। अक्र॑न्दद॒ग्निः स्त॒नय॑न्निव॒ द्यौः क्षामा॒ रेरि॑हद्वी॒रुधः॑ सम॒ञ्जन्न्। स॒द्यो ज॑ज्ञा॒नो वि हीमि॒द्धो अख्य॒दा रोद॑सी भा॒नुना॑ भात्य॒न्तः। उ॒शिक्पा॑व॒को अ॑र॒तिः सु॑मे॒धा मर्ते᳚ष्व॒ग्निर॒मृतो॒ निधा॑यि। इय॑र्ति धू॒मम॑रु॒षम्भरि॑भ्र॒दुच्छु॒क्रेण॑ शो॒चिषा॒ द्यामिन॑क्षत्। विश्व॑स्य के॒तुर्भुव॑नस्य॒ गर्भ॒ आ॥७॥

%4.2.2.3
रोद॑सी अपृणा॒ज्जाय॑मानः। वी॒डुं चि॒दद्रि॑मभिनत्परा॒यञ्जना॒ यद॒ग्निमय॑जन्त॒ पञ्च॑। श्री॒णामु॑दा॒रो ध॒रुणो॑ रयी॒णाम्म॑नी॒षाणा॒म्प्रार्प॑णः॒ सोम॑गोपाः। वसोः᳚ सू॒नुः सह॑सो अ॒फ्सु राजा॒ वि भा॒त्यग्र॑ उ॒षसा॑मिधा॒नः। यस्ते॑ अ॒द्य कृ॒णव॑द्भद्रशोचे\-ऽपू॒पं दे॑व घृ॒तव॑न्तमग्ने। प्र तं न॑य प्रत॒रां वस्यो॒ अच्छा॒भि द्यु॒म्नं दे॒वभ॑क्तं यविष्ठ। आ॥८॥

%4.2.2.4
तम्भ॑ज सौश्रव॒सेष्व॑ग्न उ॒क्थउ॑क्थ॒ आ भ॑ज श॒स्यमा॑ने। प्रि॒यः सूर्ये᳚ प्रि॒यो अ॒ग्ना भ॑वा॒त्युज्जा॒तेन॑ भि॒नद॒दुज्जनि॑त्वैः। त्वाम॑ग्ने॒ यज॑माना॒ अनु॒ द्यून् विश्वा॒ वसू॑नि दधिरे॒ वार्या॑णि। त्वया॑ स॒ह द्रवि॑णमि॒च्छमा॑ना व्र॒जं गोम॑न्तमु॒शिजो॒ वि व॑व्रुः। दृ॒शा॒नो रु॒क्म उ॒र्व्या व्य॑द्यौद्दु॒र्मर्\mbox{}ष॒मायुः॑ श्रि॒ये रु॑चा॒नः। अ॒ग्निर॒मृतो॑ अभव॒द्वयो॑भि॒र्यदे॑नं॒ द्यौरज॑नयथ्सु॒रेताः᳚॥९॥

%4.2.3.0
{\anuvakamend[{तृ॒तीये᳚ त्वा॒ गर्भ॒ आ य॑वि॒ष्ठा यच्च॒त्वारि॑ च}]}%॥२॥

%4.2.3.1
अन्न॑प॒ते\-ऽन्न॑स्य नो देह्यनमी॒वस्य॑ शु॒ष्मिणः॑। प्रप्र॑दा॒तारं॑ तारिष॒ ऊर्जं॑ नो धेहि द्वि॒पदे॒ चतु॑ष्पदे। उदु॑ त्वा॒ विश्वे॑ दे॒वा अग्ने॒ भर॑न्तु॒ चित्ति॑भिः। स नो॑ भव शि॒वत॑मः सु॒प्रती॑को वि॒भाव॑सुः। प्रेद॑ग्ने॒ ज्योति॑ष्मान् याहि शि॒वेभि॑र॒र्चिभि॒स्त्वम्। बृ॒हद्भि॑र्भा॒नुभि॒र्भास॒न्मा हिꣳ॑सीस्त॒नुवा᳚ प्र॒जाः। स॒मिधा॒ग्निं दु॑वस्यत घृ॒तैर्बो॑धय॒ताति॑थिम्। आ॥१०॥

%4.2.3.2
अ॒स्मि॒न् ह॒व्या जु॑होतन। प्रप्रा॒यम॒ग्निर्भ॑र॒तस्य॑ शृण्वे॒ वि यथ्सूर्यो॒ न रोच॑ते बृ॒हद्भाः। अ॒भि यः पू॒रुम् पृत॑नासु त॒स्थौ दी॒दाय॒ दैव्यो॒ अति॑थिः शि॒वो नः॑। आपो॑ देवीः॒ प्रति॑ गृह्णीत॒ भस्मै॒तथ्स्यो॒ने कृ॑णुध्वꣳ सुर॒भावु॑ लो॒के। तस्मै॑ नमन्तां॒ जन॑यः सु॒पत्नी᳚र्मा॒तेव॑ पु॒त्रम्बि॑भृ॒ता स्वे॑नम्। अ॒फ्स्व॑ग्ने॒ सधि॒ष्टव॑॥११॥

%4.2.3.3
सौष॑धी॒रनु॑ रुध्यसे। गर्भे॒ सञ्जा॑यसे॒ पुनः॑। गर्भो॑ अ॒स्योष॑धीनां॒ गर्भो॒ वन॒स्पती॑नाम्। गर्भो॒ विश्व॑स्य भू॒तस्याग्ने॒ गर्भो॑ अ॒पाम॑सि। प्र॒सद्य॒ भस्म॑ना॒ योनि॑म॒पश्च॑ पृथि॒वीम॑ग्ने। स॒ꣳ॒सृज्य॑ मा॒तृभि॒स्त्वं ज्योति॑ष्मा॒न्पुन॒रास॑दः। पुन॑रा॒सद्य॒ सद॑नम॒पश्च॑ पृथि॒वीम॑ग्ने। शेषे॑ मा॒तुर्यथो॒पस्थे॒\-ऽन्तर॒स्याꣳ शि॒वत॑मः। पुन॑रू॒र्जा॥१२॥

%4.2.3.4
नि व॑र्तस्व॒ पुन॑रग्न इ॒षायु॑षा। पुन॑र्नः पाहि वि॒श्वतः॑। स॒ह र॒य्या नि व॑र्त॒स्वाग्ने॒ पिन्व॑स्व॒ धार॑या। वि॒श्वफ्स्नि॑या वि॒श्वत॒स्परि॑। पुन॑स्त्वादि॒त्या रु॒द्रा वस॑वः॒ समि॑न्धता॒म्पुन॑र्ब्र॒ह्माणो॑ वसुनीथ य॒ज्ञैः। घृ॒तेन॒ त्वं त॒नुवो॑ वर्धयस्व स॒त्याः स॑न्तु॒ यज॑मानस्य॒ कामाः᳚। बोधा॑ नो अ॒स्य वच॑सो यविष्ठ॒ मꣳहि॑ष्ठस्य॒ प्रभृ॑तस्य स्वधावः। पीय॑ति त्वो॒ अनु॑ त्वो गृणाति व॒न्दारु॑स्ते त॒नुवं॑ वन्दे अग्ने। स बो॑धि सू॒रिर्म॒घवा॑ वसु॒दावा॒ वसु॑पतिः। यु॒यो॒ध्य॑स्मद्द्वेषाꣳ॑सि॥१३॥

%4.2.4.0
{\anuvakamend[{आ तवो॒र्जा\-ऽनु॒ षोड॑श च}]}%॥३॥

%4.2.4.1
अपे॑त॒ वीत॒ वि च॑ सर्प॒तातो॒ ये\-ऽत्र॒ स्थ पु॑रा॒णा ये च॒ नूत॑नाः। अदा॑दि॒दं य॒मो॑\-ऽव॒सानं॑ पृथि॒व्या अक्र॑न्नि॒मम् पि॒तरो॑ लो॒कम॑स्मै। अ॒ग्नेर्भस्मा᳚स्य॒ग्नेः पुरी॑षमसि सं॒ज्ञान॑मसि काम॒धर॑ण॒म्मयि॑ ते काम॒धर॑णम्भूयात्। सं या वः॑ प्रि॒यास्त॒नुवः॒ सम्प्रि॒या हृ॑दयानि वः। आ॒त्मा वो॑ अस्तु॥१४॥

%4.2.4.2
सम्प्रि॑यः॒ सम्प्रि॑यास्त॒नुवो॒ मम॑। अ॒यꣳ सो अ॒ग्निर्यस्मि॒न्थ्सोम॒मिन्द्रः॑ सु॒तं द॒धे ज॒ठरे॑ वावशा॒नः। स॒ह॒स्रियं॒ वाज॒मत्यं॒ न सप्तिꣳ॑ सस॒वान्थ्सन्थ्स्तू॑यसे जातवेदः। अग्ने॑ दि॒वो अर्ण॒मच्छा॑ जिगा॒स्यच्छा॑ दे॒वाꣳ ऊ॑चिषे॒ धिष्णि॑या॒ ये। याः प॒रस्ता᳚द्रोच॒ने सूर्य॑स्य॒ याश्चा॒वस्ता॑दुप॒तिष्ठ॑न्त॒ आपः॑। अग्ने॒ यत्ते॑ दि॒वि वर्चः॑ पृथि॒व्यां यदोष॑धीषु॥१५॥

%4.2.4.3
अ॒फ्सु वा॑ यजत्र। येना॒न्तरि॑क्षमु॒र्वा॑त॒तन्थ॑ त्वे॒षः स भा॒नुर॑र्ण॒वो नृ॒चक्षाः᳚। पु॒री॒ष्या॑सो अ॒ग्नयः॑ प्राव॒णेभिः॑ स॒जोष॑सः। जु॒षन्ताꣳ॑ ह॒व्यमाहु॑तमनमी॒वा इषो॑ म॒हीः। इडा॑मग्ने पुरु॒दꣳसꣳ॑ स॒निं गोः श॑श्वत्त॒मꣳ हव॑मानाय साध। स्यान्नः॑ सु॒नुस्तन॑यो वि॒जावाग्ने॒ सा ते॑ सुम॒तिर्भू᳚त्व॒स्मे। अ॒यं ते॒ योनि॑र्\mbox{}ऋ॒त्वियो॒ यतो॑ जा॒तो अरो॑चथाः। तं जा॒नन्न्॥१६॥

%4.2.4.4
अ॒ग्न॒ आ रो॒हाथा॑ नो वर्धया र॒यिम्। चिद॑सि॒ तया॑ दे॒वत॑याङ्गिर॒स्वद्ध्रु॒वा सी॑द परि॒चिद॑सि॒ तया॑ दे॒वत॑या\-ऽ ङ्गिर॒स्वद्ध्रु॒वा सी॑द लो॒कं पृ॑ण छि॒द्रं पृ॒णाथो॑ सीद शि॒वा त्वम्। इ॒न्द्रा॒ग्नी त्वा॒ बृह॒स्पति॑र॒स्मिन् योना॑वसीषदन्न्। ता अ॑स्य॒ सूद॑दोहसः॒ सोमꣴ॑ श्रीणन्ति॒ पृश्न॑यः। जन्मं॑ दे॒वानां॒ विश॑स्त्रि॒ष्वा रो॑च॒ने दि॒वः॥१७॥

%4.2.5.0
{\anuvakamend[{अ॒स्त्वोष॑धीषु जा॒नन्न॒ष्टाच॑त्वारिꣳशच्च}]}%॥४॥

%4.2.5.1
समि॑त॒ꣳ॒ सं क॑ल्पेथा॒ꣳ॒ सम्प्रि॑यौ रोचि॒ष्णू सु॑मन॒स्यमा॑नौ। इष॒मूर्ज॑म॒भि सं॒वसा॑नौ॒ सं वा॒म्मनाꣳ॑सि॒ सं व्र॒ता समु॑ चि॒त्तान्याक॑रम्। अग्ने॑ पुरीष्याधि॒पा भ॑वा॒ त्वं नः॑। इष॒मूर्जं॒ यज॑मानाय धेहि। पु॒री॒ष्य॑स्त्वम॑ग्ने रयि॒मान्पु॑ष्टि॒माꣳ अ॑सि। शि॒वाः कृ॒त्वा दिशः॒ सर्वाः॒ स्वां योनि॑मि॒हास॑दः। भव॑तं नः॒ सम॑नसौ॒ समो॑कसौ॥१८॥

%4.2.5.2
अ॒रे॒पसौ᳚। मा य॒ज्ञꣳ हिꣳ॑सिष्टं॒ मा य॒ज्ञप॑तिं जातवेदसौ शि॒वौ भ॑वतम॒द्य नः॑। मा॒तेव॑ पु॒त्रं पृ॑थि॒वी पु॑री॒ष्य॑म॒ग्निꣴ स्वे योना॑वभारु॒खा। तां विश्वै᳚र्दे॒वैर्\mbox{}ऋ॒तुभिः॑ संविदा॒नः प्र॒जाप॑तिर्वि॒श्वक॑र्मा॒ वि मु॑ञ्चतु। यद॒स्य पा॒रे रज॑सः शु॒क्रं ज्योति॒रजा॑यत। तन्नः॑ पर्\mbox{}ष॒दति॒ द्विषो\-ऽग्ने॑ वैश्वानर॒ स्वाहा᳚। नमः॒ सु ते॑ निर्\mbox{}ऋते विश्वरूपे॥१९॥

%4.2.5.3
अ॒य॒स्मयं॒ वि चृ॑ता ब॒न्धमे॒तम्। य॒मेन॒ त्वं य॒म्या॑ संविदा॒नोत्त॒मं नाक॒मधि॑ रोहये॒मम्। यत्ते॑ दे॒वी निर्\mbox{}ऋ॑तिरा ब॒बन्ध॒ दाम॑ ग्री॒वास्व॑विच॒र्त्यम्। इ॒दं ते॒ तद्वि ष्या॒म्यायु॑षो॒ न मध्या॒दथा॑ जी॒वः पि॒तुम॑द्धि॒ प्रमु॑क्तः। यस्या᳚स्ते अ॒स्याः क्रू॒र आ॒सञ्जु॒होम्ये॒षाम्ब॒न्धाना॑मव॒सर्ज॑नाय। भूमि॒रिति॑ त्वा॒ जना॑ वि॒दुर्निर्\mbox{}ऋ॑तिः॥२०॥

%4.2.5.4
इति॑ त्वा॒हं परि॑ वेद वि॒श्वतः॑। असु॑न्वन्त॒मय॑जमानमिच्छ स्ते॒नस्ये॒त्यां तस्क॑र॒स्यान्वे॑षि। अ॒न्यम॒स्मदि॑च्छ॒ सा त॑ इ॒त्या नमो॑ देवि निर्\mbox{}ऋते॒ तुभ्य॑मस्तु। दे॒वीम॒हं निर्\mbox{}ऋ॑तिं॒ वन्द॑मानः पि॒तेव॑ पु॒त्रं द॑सये॒ वचो॑भिः। विश्व॑स्य॒ या जाय॑मानस्य॒ वेद॒ शिरः॑शिरः॒ प्रति॑ सू॒री वि च॑ष्टे। नि॒वेश॑नः सं॒गम॑नो॒ वसू॑नां॒ विश्वा॑ रू॒पाभि च॑ष्टे॥२१॥

%4.2.5.5
शची॑भिः। दे॒व इ॑व सवि॒ता स॒त्यध॒र्मेन्द्रो॒ न त॑स्थौ सम॒रे प॑थी॒नाम्। सं व॑र॒त्रा द॑धातन॒ निरा॑हा॒वान्कृ॑णोतन। सि॒ञ्चाम॑हा अव॒टमु॒द्रिणं॑ व॒यं विश्वाहाद॑स्त॒मक्षि॑तम्। निष्कृ॑ताहावमव॒टꣳ सु॑वर॒त्रꣳ सु॑षेच॒नम्। उ॒द्रिणꣳ॑ सिञ्चे॒ अक्षि॑तम्। सीरा॑ युञ्जन्ति क॒वयो॑ यु॒गा वि त॑न्वते॒ पृथ॑क्। धीरा॑ दे॒वेषु॑ सुम्न॒या। यु॒नक्त॒ सीरा॒ वि यु॒गा त॑नोत कृ॒ते योनौ॑ वपते॒ह॥२२॥

%4.2.5.6
बीजम्᳚। गि॒रा च॑ श्रु॒ष्टिः सभ॑रा॒ अस॑न्नो॒ नेदी॑य॒ इथ्सृ॒ण्या॑ प॒क्वमाय॑त्। लाङ्ग॑ल॒म्पवी॑रवꣳ सु॒शेवꣳ॑ सुम॒तिथ्स॑रु। उदित्कृ॑षति॒ गामवि॑म्प्रफ॒र्व्यं॑ च॒ पीव॑रीम्। प्र॒स्थाव॑द्रथ॒वाह॑नम्। शु॒नं नः॒ फाला॒ वि तु॑दन्तु॒ भूमिꣳ॑ शु॒नं की॒नाशा॑ अ॒भि य॑न्तु वा॒हान्। शु॒नम्प॒र्जन्यो॒ मधु॑ना॒ पयो॑भिः॒ शुना॑सीरा शु॒नम॒स्मासु॑ धत्तम्। कामं॑ कामदुघे धुक्ष्व मि॒त्राय॒ वरु॑णाय च। इन्द्रा॑या॒ग्नये॑ पू॒ष्ण ओष॑धीभ्यः प्र॒जाभ्यः॑। घृ॒तेन॒ सीता॒ मधु॑ना॒ सम॑क्ता॒ विश्वै᳚र्दे॒वैरनु॑मता म॒रुद्भिः॑। ऊर्ज॑स्वती॒ पय॑सा॒ पिन्व॑माना॒स्मान्थ्सी॑ते॒ पय॑सा॒भ्याव॑वृथ्स्व॥२३॥

%4.2.6.0
{\anuvakamend[{समो॑कसौ विश्वरूपे वि॒दुर्निर्\mbox{}ऋ॑तिर॒भि च॑ष्ट इ॒ह मि॒त्राय॒ द्वाविꣳ॑शतिश्च}]}%॥५॥

%4.2.6.1
या जा॒ता ओष॑धयो दे॒वेभ्य॑स्त्रियु॒गम्पु॒रा। मन्दा॑मि ब॒भ्रूणा॑महꣳ श॒तं धामा॑नि स॒प्त च॑। श॒तं वो॑ अम्ब॒ धामा॑नि स॒हस्र॑मुत वो॒ रुहः॑। अथा॑ शतक्रत्वो यू॒यमि॒मं मे॑ अग॒दं कृ॑त। पुष्पा॑वतीः प्र॒सूव॑तीः फ॒लिनी॑रफ॒ला उ॒त। अश्वा॑ इव स॒जित्व॑रीर्वी॒रुधः॑ पारयि॒ष्णवः॑। ओष॑धी॒रिति॑ मातर॒स्तद्वो॑ देवी॒रुप॑ ब्रुवे। रपाꣳ॑सि विघ्न॒तीरि॑त॒ रपः॑॥२४॥

%4.2.6.2
चा॒तय॑मानाः। अ॒श्व॒त्थे वो॑ नि॒षद॑नम्प॒र्णे वो॑ वस॒तिः कृ॒ता। गो॒भाज॒ इत्किला॑सथ॒ यथ्स॒नव॑थ॒ पूरु॑षम्। यद॒हं वा॒जय॑न्नि॒मा ओष॑धी॒र्\mbox{}हस्त॑ आद॒धे। आ॒त्मा यक्ष्म॑स्य नश्यति पु॒रा जी॑व॒गृभो॑ यथा। यदोष॑धयः सं॒गच्छ॑न्ते॒ राजा॑नः॒ समि॑ताविव। विप्रः॒ स उ॑च्यते भि॒षग्र॑क्षो॒हामी॑व॒चात॑नः। निष्कृ॑ति॒र्नाम॑ वो मा॒ताथा॑ यू॒यꣴ स्थ॒ सङ्कृ॑तीः। स॒राः प॑त॒त्रिणीः᳚॥२५॥

%4.2.6.3
स्थ॒न॒ यदा॒मय॑ति॒ निष्कृ॑त। अ॒न्या वो॑ अ॒न्याम॑वत्व॒न्यान्यस्या॒ उपा॑वत। ताः सर्वा॒ ओष॑धयः संविदा॒ना इ॒दम्मे॒ प्राव॑ता॒ वचः॑। उच्छुष्मा॒ ओष॑धीनां॒ गावो॑ गो॒ष्ठादि॑वेरते। धनꣳ॑ सनि॒ष्यन्ती॑नामा॒त्मानं॒ तव॑ पूरुष। अति॒ विश्वाः᳚ परि॒ष्ठाः स्ते॒न इ॑व व्र॒जम॑क्रमुः। ओष॑धयः॒ प्राचु॑च्यवु॒र्यत् किं च॑ त॒नुवा॒ꣳ॒ रपः॑। याः॥२६॥

%4.2.6.4
त॒ आ॒त॒स्थुरा॒त्मानं॒ या आ॑विवि॒शुः परुः॑परुः। तास्ते॒ यक्ष्मं॒ वि बा॑धन्तामु॒ग्रो म॑ध्यम॒शीरि॑व। सा॒कं य॑क्ष्म॒ प्र प॑त श्ये॒नेन॑ किकिदी॒विना᳚। सा॒कं वात॑स्य॒ ध्राज्या॑ सा॒कं न॑श्य नि॒हाक॑या। अ॒श्वा॒व॒तीꣳ सो॑मव॒तीमू॒र्जय॑न्ती॒\-मुदो॑जसम्। आ वि॑थ्सि॒ सर्वा॒ ओष॑धीर॒स्मा अ॑रि॒ष्टता॑तये। याः फ॒लिनी॒र्या अ॑फ॒ला अ॑पु॒ष्पा याश्च॑ पु॒ष्पिणीः᳚। बृह॒स्पति॑प्रसूता॒स्ता नो॑ मुञ्च॒न्त्वꣳह॑सः। याः॥२७॥

%4.2.6.5
ओष॑धयः॒ सोम॑राज्ञीः॒ प्रवि॑ष्टाः पृथि॒वीमनु॑। तासां॒ त्वम॑स्युत्त॒मा प्र णो॑ जी॒वात॑वे सुव। अ॒व॒पत॑न्तीरवदन्दि॒व ओष॑धयः॒ परि॑। यं जी॒वम॒श्नवा॑महै॒ न स रि॑ष्याति॒ पूरु॑षः। याश्चे॒दमु॑पशृ॒ण्वन्ति॒ याश्च॑ दू॒रं परा॑गताः। इ॒ह सं॒गत्य॒ ताः सर्वा॑ अस्मै॒ सं द॑त्त भेष॒जम्। मा वो॑ रिषत्खनि॒ता यस्मै॑ चा॒हं खना॑मि वः। द्वि॒पच्चतु॑ष्पद॒स्माक॒ꣳ॒ सर्व॑म॒स्त्वना॑तुरम्। ओष॑धयः॒ सं व॑दन्ते॒ सोमे॑न स॒ह राज्ञा᳚। यस्मै॑ क॒रोति॑ ब्राह्म॒णस्तꣳ रा॑जन्पारयामसि॥२८॥

%4.2.7.0
{\anuvakamend[{रपः॑ पत॒त्रिणी॒र्या अꣳह॑सो॒ याः खना॑मि वो॒\-ऽष्टाद॑श च}]}%॥६॥

%4.2.7.1
मा नो॑ हिꣳसीज्जनि॒ता यः पृ॑थि॒व्या यो वा॒ दिवꣳ॑ स॒त्यध॑र्मा ज॒जान॑। यश्चा॒पश्च॒न्द्रा बृ॑ह॒तीर्ज॒जान॒ कस्मै॑ दे॒वाय॑ ह॒विषा॑ विधेम। अ॒भ्याव॑र्तस्व पृथिवि य॒ज्ञेन॒ पय॑सा स॒ह। व॒पां ते॑ अ॒ग्निरि॑षि॒तो\-ऽव॑ सर्पतु। अग्ने॒ यत्ते॑ शु॒क्रं यच्च॒न्द्रं यत्पू॒तं यद्य॒ज्ञियम्᳚। तद्दे॒वेभ्यो॑ भरामसि। इष॒मूर्ज॑म॒हमि॒त आ॥२९॥

%4.2.7.2
द॒द॒ ऋ॒तस्य॒ धाम्नो॑ अ॒मृत॑स्य॒ योनेः᳚। आ नो॒ गोषु॑ विश॒त्वौष॑धीषु॒ जहा॑मि से॒दिमनि॑रा॒ममी॑वाम्। अग्ने॒ तव॒ श्रवो॒ वयो॒ महि॑ भ्राजन्त्य॒र्चयो॑ विभावसो। बृह॑द्भानो॒ शव॑सा॒ वाज॑मु॒क्थ्यं॑ दधा॑सि दा॒शुषे॑ कवे। इ॒र॒ज्यन्न॑ग्ने प्रथयस्व ज॒न्तुभि॑र॒स्मे रायो॑ अमर्त्य। स द॑र्\mbox{}श॒तस्य॒ वपु॑षो॒ वि रा॑जसि पृ॒णक्षि॑ सान॒सिꣳ र॒यिम्। ऊर्जो॑ नपा॒ज्जात॑वेदः सुश॒स्तिभि॒र्मन्द॑स्व॥३०॥

%4.2.7.3
धी॒तिभि॑र्हि॒तः। त्वे इषः॒ सं द॑धु॒र्भूरि॑रेतसश्चि॒त्रोत॑यो वा॒मजा॑ताः। पा॒व॒कव॑र्चाः शु॒क्रव॑र्चा॒ अनू॑नवर्चा॒ उदि॑यर्\mbox{}षि भा॒नुना᳚। पु॒त्रः पि॒तरा॑ वि॒चर॒न्नुपा॑वस्यु॒भे पृ॑णक्षि॒ रोद॑सी। ऋ॒तावा॑नम्महि॒षं वि॒श्वच॑र्\mbox{}षणिम॒ग्निꣳ सु॒म्नाय॑ दधिरे पु॒रो जनाः᳚। श्रुत्क॑र्णꣳ स॒प्रथ॑स्तं᳚ त्वा गि॒रा दैव्य॒म्मानु॑षा यु॒गा। नि॒ष्क॒र्तार॑मध्व॒रस्य॒ प्रचे॑तसं॒ क्षय॑न्त॒ꣳ॒ राध॑से म॒हे। रा॒तिम्भृगू॑णामु॒शिजं॑ क॒विक्र॑तुं पृ॒णक्षि॑ सान॒सिम्॥३१॥

%4.2.7.4
र॒यिम्। चितः॑ स्थ परि॒चित॑ ऊर्ध्व॒चितः॑ श्रयध्वं॒ तया॑ दे॒वत॑याङ्गिर॒स्वद् ध्रु॒वाः सी॑दत। आ प्या॑यस्व॒ समे॑तु ते वि॒श्वतः॑ सोम॒ वृष्णि॑यम्। भवा॒ वाज॑स्य सङ्ग॒थे। सं ते॒ पयाꣳ॑सि॒ समु॑ यन्तु॒ वाजाः॒ सं वृष्णि॑यान्यभिमाति॒षाहः॑। आ॒प्याय॑मानो अ॒मृता॑य सोम दि॒वि श्रवाꣳ॑स्युत्त॒मानि॑ धिष्व॥३२॥

%4.2.8.0
{\anuvakamend[{आ मन्द॑स्व सान॒सिमेका॒न्नच॑त्वारि॒ꣳ॒शच्च॑}]}%॥७॥

%4.2.8.1
अ॒भ्य॑स्था॒द्विश्वाः॒ पृत॑ना॒ अरा॑ती॒स्तद॒ग्निरा॑ह॒ तदु॒ सोम॑ आह। बृह॒स्पतिः॑ सवि॒ता तन्म॑ आह पु॒षा मा॑धाथ्सुकृ॒तस्य॑ लो॒के। यदक्र॑न्दः प्रथ॒मं जाय॑मान उ॒द्यन्थ्स॑मु॒द्रादुत वा॒ पुरी॑षात्। श्ये॒नस्य॑ प॒क्षा ह॑रि॒णस्य॑ बा॒हू उप॑स्तुतं॒ जनि॑म॒ तत्ते॑ अर्वन्न्। अ॒पां पृ॒ष्ठम॑सि॒ योनि॑र॒ग्नेः स॑मु॒द्रम॒भितः॒ पिन्व॑मानम्। वर्ध॑मानम्म॒हः॥३३॥

%4.2.8.2
आ च॒ पुष्क॑रं दि॒वो मात्र॑या व॒रिणा प्र॑थस्व। ब्रह्म॑ जज्ञा॒नम्प्र॑थ॒मम्पु॒रस्ता॒द्वि सी॑म॒तः सु॒रुचो॑ वे॒न आ॑वः। स बु॒ध्निया॑ उप॒मा अ॑स्य वि॒ष्ठाः स॒तश्च॒ योनि॒मस॑तश्च॒ विवः॑। हि॒र॒ण्य॒ग॒र्भः सम॑वर्त॒ताग्रे॑ भू॒तस्य॑ जा॒तः पति॒रेक॑ आसीत्। स दा॑धार पृथि॒वीं द्यामु॒तेमां कस्मै॑ दे॒वाय॑ ह॒विषा॑ विधेम। द्र॒फ्सश्च॑स्कन्द पृथि॒वीमनु॑॥३४॥

%4.2.8.3
द्यामि॒मं च॒ योनि॒मनु॒ यश्च॒ पूर्वः॑। तृ॒तीयं॒ योनि॒मनु॑ सं॒चर॑न्तं द्र॒फ्सं जु॑हो॒म्यनु॑ स॒प्त होत्राः᳚। नमो॑ अस्तु स॒र्पेभ्यो॒ ये के च॑ पृथि॒वीमनु॑। ये अ॒न्तरि॑क्षे॒ ये दि॒वि तेभ्यः॑ स॒र्पेभ्यो॒ नमः॑। ये॑\-ऽदो रो॑च॒ने दि॒वो ये वा॒ सूर्य॑स्य र॒श्मिषु॑। येषा॑म॒फ्सु सदः॑ कृ॒तं तेभ्यः॑ स॒र्पेभ्यो॒ नमः॑। या इष॑वो यातु॒धाना॑नां॒ ये वा॒ वन॒स्पती॒ꣳ॒ रनु॑। ये वा॑व॒टेषु॒ शेर॑ते॒ तेभ्यः॑ स॒र्पेभ्यो॒ नमः॑॥३५॥

%4.2.9.0
{\anuvakamend[{म॒हो\-ऽनु॑ यातु॒धाना॑ना॒मेका॑दश च}]}%॥८॥

%4.2.9.1
ध्रु॒वासि॑ ध॒रुणास्तृ॑ता वि॒श्वक॑र्मणा॒ सुकृ॑ता। मा त्वा॑ समु॒द्र उद्व॑धी॒न्मा सु॑प॒र्णो\-ऽव्य॑थमाना पृथि॒वीं दृꣳ॑ह। प्र॒जाप॑तिस्त्वा सादयतु पृथि॒व्याः पृ॒ष्ठे व्यच॑स्वती॒म्प्रथ॑स्वती॒म्प्रथो॑\-ऽसि पृथि॒व्य॑सि॒ भूर॑सि॒ भूमि॑र॒स्यदि॑तिरसि वि॒श्वधा॑या॒ विश्व॑स्य॒ भुव॑नस्य ध॒र्त्री पृ॑थि॒वीं य॑च्छ पृथि॒वीं दृꣳ॑ह पृथि॒वीं मा हिꣳ॑सी॒र्विश्व॑स्मै प्रा॒णाया॑पा॒नाय॑ व्या॒नायो॑दा॒नाय॑ प्रति॒ष्ठायै᳚॥३६॥

%4.2.9.2
च॒रित्रा॑या॒ग्निस्त्वा॒भि पा॑तु म॒ह्या स्व॒स्त्या छ॒र्दिषा॒ शन्त॑मेन॒ तया॑ दे॒वत॑याङ्गिर॒स्वद्ध्रु॒वा सी॑द। काण्डा᳚त्काण्डात् प्र॒रोह॑न्ती॒ परु॑षःपरुषः॒ परि॑। ए॒वा नो॑ दूर्वे॒ प्र त॑नु स॒हस्रे॑ण श॒तेन॑ च। या श॒तेन॑ प्रत॒नोषि॑ स॒हस्रे॑ण वि॒रोह॑सि। तस्या᳚स्ते देवीष्टके वि॒धेम॑ ह॒विषा॑ व॒यम्। अषा॑ढासि॒ सह॑माना॒ सह॒स्वारा॑तीः॒ सह॑स्वारातीय॒तः सह॑स्व॒ पृत॑नाः॒ सह॑स्व पृतन्य॒तः। स॒हस्र॑वीर्या॥३७॥

%4.2.9.3
अ॒सि॒ सा मा॑ जिन्व। मधु॒ वाता॑ ऋताय॒ते मधु॑ क्षरन्ति॒ सिन्ध॑वः। माध्वी᳚र्नः स॒न्त्वोष॑धीः। मधु॒ नक्त॑मु॒तोषसि॒ मधु॑म॒त्पार्थि॑व॒ꣳ॒ रजः। मधु॒ द्यौर॑स्तु नः पि॒ता। मधु॑मान्नो॒ वन॒स्पति॒र्मधु॑माꣳ अस्तु॒ सूर्यः॑। माध्वी॒र्गावो॑ भवन्तु नः। म॒ही द्यौः पृ॑थि॒वी च॑ न इ॒मं य॒ज्ञम्मि॑मिक्षताम्। पि॒पृ॒तां नो॒ भरी॑मभिः। तद्विष्णोः᳚ पर॒मम्॥३८॥

%4.2.9.4
प॒दꣳ सदा॑ पश्यन्ति सू॒रयः॑। दि॒वीव॒ चक्षु॒रात॑तम्। ध्रु॒वासि॑ पृथिवि॒ सह॑स्व पृतन्य॒तः। स्यू॒ता दे॒वेभि॑र॒मृते॒नागाः᳚। यास्ते॑ अग्ने॒ सूर्ये॒ रुच॑ उद्य॒तो दिव॑मात॒न्वन्ति॑ र॒श्मिभिः॑। ताभिः॒ सर्वा॑भी रु॒चे जना॑य नस्कृधि। या वो॑ देवाः॒ सूर्ये॒ रुचो॒ गोष्वश्वे॑षु॒ या रुचः॑। इन्द्रा᳚ग्नी॒ ताभिः॒ सर्वा॑भी॒ रुचं॑ नो धत्त बृहस्पते। वि॒राट्॥३९॥

%4.2.9.5
ज्योति॑रधारयथ्स॒म्राड्ज्योति॑रधारयथ्स्व॒राड्ज्योति॑रधारयत्। अग्ने॑ यु॒क्ष्वा हि ये तवाश्वा॑सो देव सा॒धवः॑। अरं॒ वह॑न्त्या॒शवः॑। यु॒क्ष्वा हि दे॑व॒हूत॑मा॒ꣳ॒ अश्वाꣳ॑ अग्ने र॒थीरि॑व। नि होता॑ पू॒र्व्यः स॑दः। द्र॒फ्सश्च॑स्कन्द पृथि॒वीमनु॒ द्यामि॒मं च॒ योनि॒मनु॒ यश्च॒ पूर्वः॑। तृ॒तीयं॒ योनि॒मनु॑ सं॒चर॑न्तं द्र॒फ्सं जु॑हो॒म्यनु॑ स॒प्त॥४०॥

%4.2.9.6
होत्राः᳚। अभू॑दि॒दं विश्व॑स्य॒ भुव॑नस्य॒ वाजि॑नम॒ग्नेर्वै᳚श्वान॒रस्य॑ च। अ॒ग्निर्ज्योति॑षा॒ ज्योति॑ष्मान्रु॒क्मो वर्च॑सा॒ वर्च॑स्वान्। ऋ॒चे त्वा॑ रु॒चे त्वा॒ समिथ्स्र॑वन्ति स॒रितो॒ न धेनाः᳚। अ॒न्तर्\mbox{}हृ॒दा मन॑सा पू॒यमा॑नाः। घृ॒तस्य॒ धारा॑ अ॒भि चा॑कशीमि। हि॒र॒ण्ययो॑ वेत॒सो मध्य॑ आसाम्। तस्मि᳚न्थ्सुप॒र्णो म॑धु॒कृत्कु॑ला॒यी भज॑न्नास्ते॒ मधु॑ दे॒वता᳚भ्यः। तस्या॑सते॒ हर॑यः स॒प्त तीरे᳚ स्व॒धां दुहा॑ना अ॒मृत॑स्य॒ धारा᳚म्॥४१॥

%4.2.10.0
{\anuvakamend[{प्र॒ति॒ष्ठायै॑ स॒हस्र॑वीर्या पर॒मं वि॒राट्थ्स॒प्त तीरे॑ च॒त्वारि॑ च}]}%॥९॥

%4.2.10.1
आ॒दि॒त्यं गर्भ॒म्पय॑सा सम॒ञ्जन्थ्स॒हस्र॑स्य प्रति॒मां वि॒श्वरू॑पम्। परि॑ वृङ्ग्धि॒ हर॑सा॒ माभि मृ॑क्षः श॒तायु॑षं कृणुहि ची॒यमा॑नः। इ॒मं मा हिꣳ॑सीर्द्वि॒पाद॑म्पशू॒नाꣳ सह॑स्राक्ष॒ मेध॒ आ ची॒यमा॑नः। म॒युमा॑र॒ण्यमनु॑ ते दिशामि॒ तेन॑ चिन्वा॒नस्त॒नुवो॒ नि षी॑द। वात॑स्य॒ ध्राजिं॒ वरु॑णस्य॒ नाभि॒मश्वं॑ जज्ञा॒नꣳ स॑रि॒रस्य॒ मध्ये᳚। शिशुं॑ न॒दीना॒ꣳ॒ हरि॒मद्रि॑बुद्ध॒मग्ने॒ मा हिꣳ॑सीः॥४२॥

%4.2.10.2
प॒र॒मे व्यो॑मन्न्। इ॒मं मा हिꣳ॑सी॒रेक॑शफम्पशू॒नां क॑निक्र॒दं वा॒जिनं॒ वाजि॑नेषु। गौ॒रमा॑र॒ण्यमनु॑ ते दिशामि॒ तेन॑ चिन्वा॒नस्त॒नुवो॒ नि षी॑द। अज॑स्र॒मिन्दु॑मरु॒षम्भु॑र॒ण्युम॒ग्निमी॑डे पू॒र्वचि॑त्तौ॒ नमो॑भिः। स पर्व॑भिर्\mbox{}ऋतु॒शः कल्प॑मानो॒ गां मा हिꣳ॑सी॒रदि॑तिं वि॒राजम्᳚। इ॒मꣳ स॑मु॒द्रꣳ श॒तधा॑र॒मुथ्सं॑ व्य॒च्यमा॑न॒म्भुव॑नस्य॒ मध्ये᳚। घृ॒तं दुहा॑ना॒मदि॑तिं॒ जना॒याग्ने॒ मा॥४३॥

%4.2.10.3
हि॒ꣳ॒सीः॒ प॒र॒मे व्यो॑मन्न्। ग॒व॒यमा॑र॒ण्यमनु॑ ते दिशामि॒ तेन॑ चिन्वा॒नस्त॒नुवो॒ नि षी॑द। वरू᳚त्रिं॒ त्वष्टु॒र्वरु॑णस्य॒ नाभि॒मविं॑ जज्ञा॒नाꣳ रज॑सः॒ पर॑स्मात्। म॒हीꣳ सा॑ह॒स्रीमसु॑रस्य मा॒यामग्ने॒ मा हिꣳ॑सीः॒ पर॒मे व्यो॑मन्न्। इ॒मामू᳚र्णा॒युं वरु॑णस्य मा॒यां त्वच॑म्पशू॒नां द्वि॒पदां॒ चतु॑ष्पदाम्। त्वष्टुः॑ प्र॒जानां᳚ प्रथ॒मं ज॒नित्र॒मग्ने॒ मा हिꣳ॑सीः पर॒मे व्यो॑मन्न्। उष्ट्र॑मार॒ण्यमनु॑॥४४॥

%4.2.10.4
ते॒ दि॒शा॒मि॒ तेन॑ चिन्वा॒नस्त॒नुवो॒ नि षी॑द। यो अ॒ग्निर॒ग्नेस्तप॒सो\-ऽधि॑ जा॒तः शोचा᳚त्पृथि॒व्या उ॒त वा॑ दि॒वस्परि॑। येन॑ प्र॒जा वि॒श्वक॑र्मा॒ व्यान॒ट्तम॑ग्ने॒ हेडः॒ परि॑ ते वृणक्तु। अ॒जा ह्य॑ग्नेरज॑निष्ट॒ गर्भा॒थ्सा वा अ॑पश्यज्जनि॒तार॒मग्रे᳚। तया॒ रोह॑माय॒न्नुप॒ मेध्या॑स॒स्तया॑ दे॒वा दे॒वता॒मग्र॑ आयन्न्। श॒र॒भमा॑र॒ण्यमनु॑ ते दिशामि॒ तेन॑ चिन्वा॒नस्त॒नुवो॒ नि षी॑द॥४५॥

%4.2.11.0
{\anuvakamend[{अग्ने॒ मा हिꣳ॑सी॒रग्ने॒ मोष्ट्र॑मार॒ण्यमनु॑ शर॒भं नव॑ च}]}%॥10॥ आ॒दि॒त्यमि॒मं द्वि॒पाद॑म्म॒युं वात॒स्याश्व॑मि॒ममेक॑शफङ्गौ॒रमज॑स्रङ्गव॒यं वरू᳚त्रि॒मवि॑मि॒मामू᳚र्णा॒युमुष्ट्रं॒ यो अ॒ग्निर॒ग्नेः श॑र॒भम्॥

%4.2.11.1
इन्द्रा᳚ग्नी रोच॒ना दि॒वः परि॒ वाजे॑षु भूषथः। तद्वां᳚ चेति॒ प्र वी॒र्यम्᳚। श्नथ॑द्वृ॒त्रमु॒त स॑नोति॒ वाज॒मिन्द्रा॒ यो अ॒ग्नी सहु॑री सप॒र्यात्। इ॒र॒ज्यन्ता॑ वस॒व्य॑स्य॒ भूरेः॒ सह॑स्तमा॒ सह॑सा वाज॒यन्ता᳚। प्र च॑र्\mbox{}ष॒णिभ्यः॑ पृतना॒ हवे॑षु॒ प्र पृ॑थि॒व्या रि॑रिचाथे दि॒वश्च॑। प्र सिन्धु॑भ्यः॒ प्र गि॒रिभ्यो॑ महि॒त्वा प्रेन्द्रा᳚ग्नी॒ विश्वा॒ भुव॒नात्य॒न्या। मरु॑तो॒ यस्य॒ हि॥४६॥

%4.2.11.2
क्षये॑ पा॒था दि॒वो वि॑महसः। स सु॑गो॒पात॑मो॒ जनः॑। य॒ज्ञैर्वा॑ यज्ञवाहसो॒ विप्र॑स्य वा मती॒नाम्। मरु॑तः शृणु॒ता हवम्᳚। श्रि॒यसे॒ कम्भा॒नुभिः॒ सम्मि॑मिक्षिरे॒ ते र॒श्मिभि॒स्त ऋक्व॑भिः सुखा॒दयः॑। ते वाशी॑मन्त इ॒ष्मिणो॒ अभी॑रवो वि॒द्रे प्रि॒यस्य॒ मारु॑तस्य॒ धाम्नः॑। अव॑ ते॒ हेड॒ उदु॑त्त॒मम्। कया॑ नश्चि॒त्र आ भु॑वदू॒ती स॒दावृ॑धः॒ सखा᳚। कया॒ शचि॑ष्ठया वृ॒ता।॥४७॥

%4.2.11.3
को अ॒द्य यु॑ङ्क्ते धु॒रि गा ऋ॒तस्य॒ शिमी॑वतो भा॒मिनो॑ दुर्\mbox{}हृणा॒यून्। आ॒सन्नि॑षून् हृ॒थ्स्वसो॑ मयो॒भून् य ए॑षाम् भृ॒त्यामृ॒णध॒थ्स जी॑वात्। अग्ने॒ नया दे॒वाना॒ꣳ॒ शं नो॑ भवन्तु॒ वाजे॑वाजे। अ॒फ्स्व॑ग्ने॒ सधि॒ष्टव॒ सौष॑धी॒रनु॑ रुध्यसे। गर्भे॒ सञ्जा॑यसे॒ पुनः॑। वृषा॑ सोम द्यु॒माꣳ अ॑सि॒ वृषा॑ देव॒ वृष॑व्रतः। वृषा॒ धर्मा॑णि दधिषे। इ॒मं मे॑ वरुण तत्त्वा॑ यामि॒ त्वं नो॑ अग्ने॒ स त्वं नो॑ अग्ने॥४८॥

%4.3.0.0
{\anuvakamend[{हि वृ॒ता म॒ एका॑दश च}]}%॥11॥

%4.3.0.0

{\anuvakamend[{अ॒पां त्वेम॑न्न॒यं पु॒रो भुवः॒ प्राची᳚ ध्रु॒वक्षि॑ति॒स्त्र्यवि॒रिन्द्रा᳚ग्नी॒ मा छन्द॑ आ॒शुस्त्रि॒वृद॒ग्नेर्भा॒गो᳚\-ऽस्येक॑ये॒यमे॒व सा याग्ने॑ जा॒तान॒ग्निर्वृ॒त्राणि॒ त्रयो॑दश}]}%॥13॥
\prashnaend{ अ॒पां त्वेन्द्रा᳚ग्नी इ॒यमे॒व दे॒वता॑ता॒ षट्त्रिꣳ॑शत्॥36॥ अ॒पां त्वेम॑न् ह॒विषा॒ वर्ध॑नेन॥}
%%% END PRASHNA

\sect{तृतीयः प्रश्नः}\setcounter{anuvakam}{0}
\dnsub{तैत्तिरीयसंहितायां चतुर्थकाण्डे तृतीयः प्रश्नः}
%4.3.1.0
%4.3.1.1
अ॒पां त्वेम᳚न्थ्सादयाम्य॒पां त्वोद्म᳚न्थ्सादयाम्य॒पां त्वा॒ भस्म᳚न्थ्सादयाम्य॒पां त्वा॒ ज्योति॑षि सादयाम्य॒पां त्वाय॑ने सादयाम्यर्ण॒वे सद॑ने सीद समु॒द्रे सद॑ने सीद सलि॒ले सद॑ने सीदा॒पां क्षये॑ सीदा॒पाꣳ सधि॑षि सीदा॒पां त्वा॒ सद॑ने सादयाम्य॒पां त्वा॑ स॒धस्थे॑ सादयाम्य॒पां त्वा॒ पुरी॑षे सादयाम्य॒पां त्वा॒ योनौ॑ सादयाम्य॒पां त्वा॒ पाथ॑सि सादयामि गाय॒त्री छन्द॑स्त्रि॒ष्टुप्छन्दो॒ जग॑ती॒ छन्दो॑\-ऽनु॒ष्टुप्छन्दः॑ प॒ङ्क्तिश्छन्दः॑॥१॥

%4.3.2.0
{\anuvakamend[{योनौ॒ पञ्च॑दश च}]}%॥१॥

%4.3.2.1
अ॒यम्पु॒रो भुव॒स्तस्य॑ प्रा॒णो भौ॑वाय॒नो व॑स॒न्तः प्रा॑णाय॒नो गा॑य॒त्री वा॑स॒न्ती गा॑यत्रि॒यै गा॑य॒त्रं गा॑य॒त्रादु॑पा॒ꣳ॒शु\-रु॑पा॒ꣳ॒शोस्त्रि॒वृत्त्रि॒वृतो॑ रथन्त॒रꣳ र॑थन्त॒राद्वसि॑ष्ठ॒ ऋषिः॑ प्र॒जाप॑तिगृहीतया॒ त्वया᳚ प्रा॒णं गृ॑ह्णामि प्र॒जाभ्यो॒\-ऽयं द॑क्षि॒णा वि॒श्वक॑र्मा॒ तस्य॒ मनो॑ वैश्वकर्म॒णं ग्री॒ष्मो मा॑न॒सस्त्रि॒ष्टुग्ग्रै॒ष्मी त्रि॒ष्टुभ॑ ऐ॒डमै॒डाद॑न्तर्या॒मो᳚\-ऽन्तर्या॒मात् प॑ञ्चद॒शः प॑ञ्चद॒शाद्बृ॒हद्बृ॑ह॒तो भ॒रद्वा॑ज॒ ऋषिः॑ प्र॒जाप॑तिगृहीतया॒ त्वया॒ मनः॑॥२॥

%4.3.2.2
गृ॒ह्णा॒मि॒ प्र॒जाभ्यो॒\-ऽयम्प॒श्चाद्वि॒श्वव्य॑चा॒स्तस्य॒ चक्षु॑र्वैश्वव्यच॒सं व॒र्\mbox{}षाणि॑ चाक्षु॒षाणि॒ जग॑ती वा॒र्\mbox{}षी जग॑त्या॒ ऋक्ष॑म॒मृक्ष॑माच्छु॒क्रः शु॒क्राथ्स॑प्तद॒शः स॑प्तद॒शाद्वै॑रू॒पं वै॑रू॒पाद्वि॒श्वामि॑त्र॒ ऋषिः॑ प्र॒जाप॑तिगृहीतया॒ त्वया॒ चक्षु॑र्गृह्णामि प्र॒जाभ्य॑ इ॒दमु॑त्त॒राथ्सुव॒स्तस्य॒ श्रोत्रꣳ॑ सौ॒वꣳ श॒रच्छ्रौ॒त्र्य॑नु॒ष्टुप्छा॑र॒द्य॑नु॒ष्टुभः॑ स्वा॒रꣴ स्वा॒रान्म॒न्थी म॒न्थिन॑ एकवि॒ꣳ॒श ए॑कवि॒ꣳ॒शाद्वै॑रा॒जं वै॑रा॒जाज्ज॒मद॑ग्नि॒र्\mbox{}ऋषिः॑ प्र॒जाप॑तिगृहीतया॥३॥

%4.3.2.3
त्वया॒ श्रोत्रं॑ गृह्णामि प्र॒जाभ्य॑ इ॒यमु॒परि॑ म॒तिस्तस्यै॒ वाङ्मा॒ती हे॑म॒न्तो वा᳚च्याय॒नः प॒ङ्क्तिर्\mbox{}है॑म॒न्ती प॒ङ्क्त्यै नि॒धन॑वन्नि॒धन॑वत आग्रय॒ण आ᳚ग्रय॒णात्त्रि॑णवत्रयस्त्रिꣳ॒शौ त्रि॑णवत्रयस्त्रिꣳ॒शाभ्याꣳ॑ शाक्वररैव॒ते शा᳚क्वररैव॒ता\-भ्यां᳚ वि॒श्वक॒र्मर्\mbox{}षिः॑ प्र॒जाप॑तिगृहीतया॒ त्वया॒ वाचं॑ गृह्णामि प्रजाभ्यः॥४॥

%4.3.3.0
{\anuvakamend[{त्वया॒ मनो॑ ज॒मद॑ग्नि॒र्\mbox{}ऋषिः॑ प्र॒जाप॑तिगृहीतया त्रि॒ꣳ॒शच्च॑}]}%॥२॥

%4.3.3.1
प्राची॑ दि॒शां व॑स॒न्त ऋ॑तू॒नाम॒ग्निर्दे॒वता॒ ब्रह्म॒ द्रवि॑णं त्रि॒वृथ्स्तोमः॒ स उ॑ पञ्चद॒शव॑र्तनि॒स्त्र्यवि॒र्वयः॑ कृ॒तमया॑नां पुरोवा॒तो वातः॒ सान॑ग॒ ऋषि॑र्दक्षि॒णा दि॒शां ग्री॒ष्म ऋ॑तू॒नामिन्द्रो॑ दे॒वता᳚ क्ष॒त्रं द्रवि॑णं पञ्चद॒शः स्तोमः॒ स उ॑ सप्तद॒शव॑र्तनिर्दित्य॒वाड्वय॒स्त्रेताया॑नां दक्षिणाद्वा॒तो वातः॑ सना॒तन॒ ऋषिः॑ प्र॒तीची॑ दि॒शां व॒र्\mbox{}षा ऋ॑तू॒नां विश्वे॑ दे॒वा दे॒वता॒ विट्॥५॥

%4.3.3.2
द्रवि॑णꣳ सप्तद॒शः स्तोमः॒ स उ॑वेकवि॒ꣳ॒शव॑र्तनिस्त्रिव॒थ्सो वयो᳚ द्वाप॒रो\-ऽया॑नाम्पश्चाद्वा॒तो वातो॑\-ऽह॒भून॒ ऋषि॒रुदी॑ची दि॒शाꣳ श॒रदृ॑तू॒नाम्मि॒त्रावरु॑णौ दे॒वता॑ पु॒ष्टं द्रवि॑णमेकवि॒ꣳ॒शः स्तोमः॒ स उ॑ त्रिण॒वव॑र्तनिस्तुर्य॒वाड्वय॑ आस्क॒न्दो-\-ऽ या॑नामुत्तराद्वा॒तो वातः॑ प्र॒त्न ऋषि॑रू॒र्ध्वा दि॒शाꣳ हे॑मन्तशिशि॒रावृ॑तू॒नाम्बृह॒स्पति॑र्दे॒वता॒ वर्चो॒ द्रवि॑णं त्रिण॒वः स्तोमः॒ स उ॑ त्रयस्त्रि॒ꣳ॒शव॑र्तनिः पष्ठ॒वाद्वयो॑\-ऽभि॒भूरया॑नां विष्वग्वा॒तो वातः॑ सुप॒र्ण ऋषिः॑ पि॒तरः॑ पिताम॒हाः परे\-ऽव॑रे॒ ते नः॑ पान्तु॒ ते नो॑\-ऽवन्त्व॒स्मिन्ब्रह्म॑न्न॒स्मिन्क्ष॒त्रे᳚\-ऽस्यामा॒शिष्य॒स्याम्पु॑रो॒धाया॑म॒स्मिन्कर्म॑न्न॒स्यां दे॒वहू᳚त्याम्॥६॥

%4.3.4.0
{\anuvakamend[{विट्प॑ष्ठ॒वाड्वयो॒\-ऽष्टाविꣳ॑शतिश्च}]}%॥३॥

%4.3.4.1
ध्रु॒वक्षि॑तिर्ध्रु॒वयो॑निर्ध्रु॒वासि॑ ध्रु॒वं योनि॒मा सी॑द सा॒ध्या। उख्य॑स्य के॒तुम्प्र॑थ॒मम्पु॒रस्ता॑द॒श्विना᳚ध्व॒र्यू सा॑दयतामि॒ह त्वा᳚। स्वे दक्षे॒ दक्ष॑पिते॒ह सी॑द देव॒त्रा पृ॑थि॒वी बृ॑ह॒ती ररा॑णा। स्वा॒स॒स्था त॒नुवा॒ सं वि॑शस्व पि॒तेवै॑धि सू॒नव॒ आ सु॒शेवा॒श्विना᳚ध्व॒र्यू सा॑दयतामि॒ह त्वा᳚। कु॒ला॒यिनी॒ वसु॑मती वयो॒धा र॒यिं नो॑ वर्ध बहु॒लꣳ सु॒वीरम्᳚।॥७॥

%4.3.4.2
अपा॑मतिं दुर्म॒तिम्बाध॑माना रा॒यस्पोषे॑ य॒ज्ञप॑तिमा॒भज॑न्ती॒ सुव॑र्धेहि॒ यज॑मानाय॒ पोष॑म॒श्विना᳚ध्व॒र्यू सा॑दयतामि॒ह त्वा᳚। अ॒ग्नेः पुरी॑षमसि देव॒यानी॒ तां त्वा॒ विश्वे॑ अ॒भि गृ॑णन्तु दे॒वाः। स्तोम॑पृष्ठा घृ॒तव॑ती॒ह सी॑द प्र॒जाव॑द॒स्मे द्रवि॒णा य॑जस्वा॒श्विना᳚ध्व॒र्यू सा॑दयतामि॒ह त्वा᳚। दि॒वो मू॒र्धासि॑ पृथि॒व्या नाभि॑र्वि॒ष्टम्भ॑नी दि॒शामधि॑पत्नी॒ भुव॑नानाम्।॥८॥

%4.3.4.3
ऊ॒र्मिर्द्र॒फ्सो अ॒पाम॑सि वि॒श्वक॑र्मा त॒ ऋषि॑र॒श्विना᳚ध्व॒र्यू सा॑दयतामि॒ह त्वा᳚। स॒जूर्\mbox{}ऋ॒तुभिः॑ स॒जूर्वि॒धाभिः॑ स॒जूर्वसु॑भिः स॒जू रु॒द्रैः स॒जूरा॑दि॒त्यैः स॒जूर्विश्वै᳚र्दे॒वैः स॒जूर्दे॒वैः स॒जूर्दे॒वैर्व॑योना॒धैर॒ग्नये᳚ त्वा वैश्वान॒राया॒श्विना᳚ध्व॒र्यू सा॑दयतामि॒ह त्वा᳚। प्रा॒णं मे॑ पाह्यपा॒नं मे॑ पाहि व्या॒नं मे॑ पाहि॒ चक्षु॑र्म उ॒र्व्या वि भा॑हि॒ श्रोत्रं॑ मे श्लोकया॒पस्पि॒न्वौष॑धीर्जिन्व द्वि॒पात्पा॑हि॒ चतु॑ष्पादव दि॒वो वृष्टि॒मेर॑य॥९॥

%4.3.5.0
{\anuvakamend[{सु॒वीरं॒ भुव॑नानामु॒र्व्या स॒प्तद॑श च}]}%॥४॥

%4.3.5.1
त्र्यवि॒र्वय॑स्त्रि॒ष्टुप्छन्दो॑ दित्य॒वाड्वयो॑ वि॒राट्छन्दः॒ पञ्चा॑वि॒र्वयो॑ गाय॒त्री छन्द॑स्त्रिव॒थ्सो वय॑ उ॒ष्णिहा॒ छन्द॑स्तुर्य॒वाड्वयो॑\-ऽ\-नु॒ष्टुप्छन्दः॑ पष्ठ॒वाद्वयो॑ बृह॒ती छन्द॑ उ॒क्षा वयः॑ स॒तोबृ॑हती॒ छन्द॑ ऋष॒भो वयः॑ क॒कुच्छन्दो॑ धे॒नुर्वयो॒ जग॑ती॒ छन्दो॑\-ऽ\-न॒ड्वान् वयः॑ प॒ङ्क्तिश्छन्दो॑ ब॒स्तो वयो॑ विव॒लं छन्दो॑ वृ॒ष्णिर्वयो॑ विशा॒लं छन्दः॒ पुरु॑षो॒ वय॑स्त॒न्द्रं छन्दो᳚ व्या॒घ्रो वयो\-ऽ\-ना॑धृष्टं॒ छन्दः॑ सि॒ꣳ॒हो वय॑श्छ॒दिश्छन्दो॑ विष्ट॒म्भो वयो\-ऽधि॑पति॒श्छन्दः॑ क्ष॒त्रं वयो॒ मयं॑दं॒ छन्दो॑ वि॒श्वक॑र्मा॒ वयः॑ परमे॒ष्ठी छन्दो मू॒र्धा वयः॑ प्र॒जाप॑ति॒श्छन्दः॑॥१०॥

%4.3.6.0
{\anuvakamend[{पुरु॑षो॒ वय॒ष्षड्विꣳ॑शतिश्च}]}%॥५॥

%4.3.6.1
इन्द्रा᳚ग्नी॒ अव्य॑थमाना॒मिष्ट॑कां दृꣳहतं यु॒वम्। पृ॒ष्ठेन॒ द्यावा॑पृथि॒वी अ॒न्तरि॑क्षं च॒ वि बा॑धताम्॥ वि॒श्वक॑र्मा त्वा सादयत्व॒न्तरि॑क्षस्य पृ॒ष्ठे व्यच॑स्वती॒म्प्रथ॑स्वती॒म्भास्व॑तीꣳ सूरि॒मती॒मा या द्याम्भास्या पृ॑थि॒वीमोर्व॑न्तरि॑क्षम॒न्तरि॑क्षं यच्छा॒न्तरि॑क्षं दृꣳहा॒न्तरि॑क्षं॒ मा हिꣳ॑सी॒र्विश्व॑स्मै प्रा॒णाया॑पा॒नाय॑ व्या॒नायो॑दा॒नाय॑ प्रति॒ष्ठायै॑ च॒रित्रा॑य वा॒युस्त्वा॒भि पा॑तु म॒ह्या स्व॒स्त्या छ॒र्दिषा᳚॥११॥

%4.3.6.2
शन्त॑मेन॒ तया॑ दे॒वत॑याङ्गिर॒स्वद्ध्रु॒वा सी॑द। राज्ञ्य॑सि॒ प्राची॒ दिग्वि॒राड॑सि दक्षि॒णा दिख्स॒म्राड॑सि प्र॒तीची॒ दिख्स्व॒राड॒स्युदी॑ची॒ दिगधि॑पत्न्यसि बृह॒ती दिगायु॑र्मे पाहि प्रा॒णं मे॑ पाह्यपा॒नं मे॑ पाहि व्या॒नं मे॑ पाहि॒ चक्षु॑र्मे पाहि॒ श्रोत्रं॑ मे पाहि॒ मनो॑ मे जिन्व॒ वाचं॑ मे पिन्वा॒त्मानं॑ मे पाहि॒ ज्योति॑र्मे यच्छ॥१२॥

%4.3.7.0
{\anuvakamend[{छ॒र्दिषा॑ पिन्व॒ षट्च॑}]}%॥६॥

%4.3.7.1
मा छन्दः॑ प्र॒मा छन्दः॑ प्रति॒मा छन्दो᳚\-ऽस्री॒विश्छन्दः॑ प॒ङ्क्तिश्छन्द॑ उ॒ष्णिहा॒ छन्दो॑ बृह॒ती छन्दो॑\-ऽनु॒ष्टुप्छन्दो॑ वि॒राट्छन्दो॑ गाय॒त्री छन्द॑स्त्रि॒ष्टुप्छन्दो॒ जग॑ती॒ छन्दः॑ पृथि॒वी छन्दो॒\-ऽन्तरि॑क्षं॒ छन्दो॒ द्यौश्छन्दः॒ समा॒श्छन्दो॒ नक्ष॑त्राणि॒ छन्दो॒ मन॒श्छन्दो॒ वाक्छन्दः॑ कृ॒षिश्छन्दो॒ हिर॑ण्यं॒ छन्दो॒ गौश्छन्दो॒\-ऽजा छन्दो\-ऽश्व॒श्छन्दः॑। अ॒ग्निर्दे॒वता᳚॥१३॥

%4.3.7.2
वातो॑ दे॒वता॒ सूर्यो॑ दे॒वता॑ च॒न्द्रमा॑ दे॒वता॒ वस॑वो दे॒वता॑ रु॒द्रा दे॒वता॑दि॒त्या दे॒वता॒ विश्वे॑ दे॒वा दे॒वता॑ म॒रुतो॑ दे॒वता॒ बृह॒स्पति॑र्दे॒वतेन्द्रो॑ दे॒वता॒ वरु॑णो दे॒वता॑ मू॒र्धासि॒ राड्ध्रु॒वासि॑ ध॒रुणा॑ य॒न्त्र्य॑सि॒ यमि॑त्री॒षे त्वो॒र्जे त्वा॑ कृ॒ष्यै त्वा॒ क्षेमा॑य त्वा॒ यन्त्री॒ राड्ध्रु॒वासि॒ धर॑णी ध॒र्त्र्य॑सि॒ धरि॒त्र्यायु॑षे त्वा॒ वर्च॑से॒ त्वौज॑से त्वा॒ बला॑य त्वा॥१४॥

%4.3.8.0
{\anuvakamend[{दे॒वता\-ऽ\-ऽयु॑षे त्वा॒ षट्च॑}]}%॥७॥

%4.3.8.1
आ॒शुस्त्रि॒वृद्भा॒न्तः प॑ञ्चद॒शो व्यो॑म सप्तद॒शः प्रतू᳚र्तिरष्टाद॒शस्तपो॑ नवद॒शो॑\-ऽभिव॒र्तः स॑वि॒ꣳ॒शो ध॒रुण॑ एकवि॒ꣳ॒शो वर्चो᳚ द्वावि॒ꣳ॒शः स॒म्भर॑णस्त्रयोवि॒ꣳ॒शो योनि॑श्चतुर्वि॒ꣳ॒शो गर्भाः᳚ पञ्चवि॒ꣳ॒श ओज॑स्त्रिण॒वः क्रतु॑रेकत्रि॒ꣳ॒शः प्र॑ति॒ष्ठा त्र॑यस्त्रि॒ꣳ॒शो ब्र॒ध्नस्य॑ वि॒ष्टपं॑ चतुस्त्रि॒ꣳ॒शो नाकः॑ षट्त्रि॒ꣳ॒शो वि॑व॒र्तो᳚\-ऽष्टाचत्वारि॒ꣳ॒शो ध॒र्त्रश्च॑तुष्टो॒मः॥१५॥

%4.3.9.0
{\anuvakamend[{आ॒शुः स॒प्तत्रिꣳ॑शत्}]}%॥८॥

%4.3.9.1
अ॒ग्नेर्भा॒गो॑\-ऽसि दी॒क्षाया॒ आधि॑पत्यं॒ ब्रह्म॑ स्पृ॒तं त्रि॒वृथ्स्तोम॒ इन्द्र॑स्य भा॒गो॑\-ऽसि॒ विष्णो॒राधि॑पत्यं क्ष॒त्रꣴ स्पृ॒तम्प॑ञ्चद॒शः स्तोमो॑ नृ॒चक्ष॑साम्भा॒गो॑\-ऽसि धा॒तुराधि॑पत्यं ज॒नित्रꣴ॑ स्पृ॒तꣳ स॑प्तद॒शः स्तोमो॑ मि॒त्रस्य॑ भा॒गो॑\-ऽसि॒ वरु॑ण॒स्याधि॑पत्यं दि॒वो वृ॒ष्टिर्वाताः᳚ स्पृ॒ता ए॑कवि॒ꣳ॒शः स्तोमो\-ऽदि॑त्यै भा॒गो॑\-ऽसि पू॒ष्ण आधि॑पत्य॒मोजः॑ स्पृ॒तं त्रि॑ण॒वः स्तोमो॒ वसू॑नाम्भा॒गो॑\-ऽसि॥१६॥

%4.3.9.2
रु॒द्राणा॒माधि॑पत्यं॒ चतु॑ष्पाथ्स्पृ॒तं च॑तुर्वि॒ꣳ॒शः स्तोम॑ आदि॒त्यानां᳚ भा॒गो॑\-ऽसि म॒रुता॒माधि॑पत्यं॒ गर्भाः᳚ स्पृ॒ताः प॑ञ्चवि॒ꣳ॒शः स्तोमो॑ दे॒वस्य॑ सवि॒तुर्भा॒गो॑\-ऽसि॒ बृह॒स्पते॒राधि॑पत्यꣳ स॒मीची॒र्दिशः॑ स्पृ॒ताश्च॑तुष्टो॒मः स्तोमो॒ यावा॑नाम्भा॒गो᳚\-ऽस्यया॑वाना॒माधि॑पत्यं प्र॒जाः स्पृ॒ताश्च॑तुश्चत्वारि॒ꣳ॒शः स्तोम॑ ऋभू॒णाम्भा॒गो॑\-ऽसि॒ विश्वे॑षां दे॒वाना॒माधि॑पत्यम्भू॒तं निशा᳚न्तꣴ स्पृ॒तं त्र॑यस्त्रि॒ꣳ॒शः स्तोमः॑॥१७॥

%4.3.10.0
{\anuvakamend[{वसू॑नां भा॒गो॑\-ऽसि॒ षट्च॑त्वारिꣳशच्च}]}%॥९॥

%4.3.10.1
एक॑यास्तुवत प्र॒जा अ॑धीयन्त प्र॒जाप॑ति॒रधि॑पतिरासीत्ति॒सृभि॑रस्तुवत॒ ब्रह्मा॑सृज्यत॒ ब्रह्म॑ण॒स्पति॒रधि॑पतिरासीत् प॒ञ्चभि॑रस्तुवत भू॒तान्य॑सृज्यन्त भू॒ताना॒म्पति॒रधि॑पतिरासीथ्स॒प्तभि॑रस्तुवत सप्त॒र्\mbox{}षयो॑\-ऽसृज्यन्त धा॒ताधि॑पतिरा\-सीन्न॒वभि॑रस्तुवत पि॒तरो॑\-ऽसृज्य॒न्तादि॑ति॒रधि॑पत्न्यासीदेकाद॒शभि॑रस्तुवत॒र्तवो॑\-ऽसृज्यन्तार्त॒वो\-ऽधि॑पतिरासीत् त्रयोद॒शभि॑रस्तुवत॒ मासा॑ असृज्यन्त संवथ्स॒रो\-ऽधि॑पतिः॥१८॥

%4.3.10.2
आ॒सी॒त्प॒ञ्च॒द॒शभि॑रस्तुवत क्ष॒त्रम॑सृज्य॒तेन्द्रो\-ऽधि॑पतिरासीथ्सप्तद॒शभि॑रस्तुवत प॒शवो॑\-ऽसृज्यन्त॒ बृह॒स्पति॒रधि॑पतिरासी\-न्नवद॒शभि॑रस्तुवत शूद्रा॒र्याव॑सृज्येतामहोरा॒त्रे अधि॑पत्नी आस्ता॒मेक॑विꣳशत्यास्तुव॒तैक॑शफाः प॒शवो॑\-ऽसृज्यन्त॒ वरु॒णो\-ऽधि॑पतिरासी॒त्त्रयो॑विꣳशत्यास्तुवत क्षु॒द्राः प॒शवो॑\-ऽसृज्यन्त पू॒षाधि॑पतिरासी॒त्पञ्च॑विꣳशत्यास्तुवतार॒ण्याः प॒शवो॑\-ऽसृज्यन्त वा॒युरधि॑पतिरासीथ्स॒प्तविꣳ॑शत्यास्तुवत॒ द्यावा॑पृथि॒वी वि॥१९॥

%4.3.10.3
ऐ॒तां॒ वस॑वो रु॒द्रा आ॑दि॒त्या अनु॒ व्या॑य॒न्तेषा॒माधि॑पत्यमासी॒न्नव॑विꣳशत्यास्तुवत॒ वन॒स्पत॑यो\-ऽसृज्यन्त॒ सोमो\-ऽ\-धि॑पतिरासी॒देक॑त्रिꣳशतास्तुवत प्र॒जा अ॑सृज्यन्त॒ यावा॑नां॒ चाया॑वानां॒ चाधि॑पत्यमासी॒त्त्रय॑स्त्रिꣳशतास्तुवत भू॒तान्य॑शाम्यन्प्र॒जाप॑तिः परमे॒ष्ठ्यधि॑पतिरासीत्॥२०॥

%4.3.11.0
{\anuvakamend[{सं॒ व॒थ्स॒रो\-ऽधि॑पति॒र्वि पञ्च॑त्रिꣳशच्च}]}%॥10॥

%4.3.11.1
इ॒यमे॒व सा या प्र॑थ॒मा व्यौच्छ॑द॒न्तर॒स्यां च॑रति॒ प्रवि॑ष्टा। व॒धूर्ज॑जान नव॒गज्जनि॑त्री॒ त्रय॑ एनाम्महि॒मानः॑ सचन्ते॥ छन्द॑स्वती उ॒षसा॒ पेपि॑शाने समा॒नं योनि॒मनु॑ सं॒चर॑न्ती। सूर्य॑पत्नी॒ वि च॑रतः प्रजान॒ती के॒तुं कृ॑ण्वा॒ने अ॒जरे॒ भूरि॑रेतसा॥ ऋ॒तस्य॒ पन्था॒मनु॑ ति॒स्र आगु॒स्त्रयो॑ घ॒र्मासो॒ अनु॒ ज्योति॒षागुः॑। प्र॒जामेका॒ रक्ष॒त्यूर्ज॒मेका᳚॥२१॥

%4.3.11.2
व्र॒तमेका॑ रक्षति देवयू॒नाम्॥ च॒तु॒ष्टो॒मो अ॑भव॒द्या तु॒रीया॑ य॒ज्ञस्य॑ प॒क्षावृ॑षयो॒ भव॑न्ती। गा॒य॒त्रीं त्रि॒ष्टुभं॒ जग॑तीमनु॒ष्टुभ॑म्बृ॒हद॒र्कं यु॑ञ्जा॒नाः सुव॒राभ॑रन्नि॒दम्॥ प॒ञ्चभि॑र्धा॒ता वि द॑धावि॒दं यत्तासा॒ꣴ॒ स्वसॄ॑रजनय॒त्पञ्च॑पञ्च। तासा॑मु यन्ति प्रय॒वेण॒ पञ्च॒ नाना॑ रू॒पाणि॒ क्रत॑वो॒ वसा॑नाः॥ त्रि॒ꣳ॒शथ्स्वसा॑र॒ उप॑ यन्ति निष्कृ॒तꣳ स॑मा॒नं के॒तुम्प्र॑तिमु॒ञ्चमा॑नाः।॥२२॥

%4.3.11.3
ऋ॒तूꣴस्त॑न्वते क॒वयः॑ प्रजान॒तीर्मध्ये॑छन्दसः॒ परि॑ यन्ति॒ भास्व॑तीः। ज्योति॑ष्मती॒ प्रति॑ मुञ्चते॒ नभो॒ रात्री॑ दे॒वी सूर्य॑स्य व्र॒तानि॑। वि प॑श्यन्ति प॒शवो॒ जाय॑माना॒ नाना॑रूपा मा॒तुर॒स्या उ॒पस्थे᳚। ए॒का॒ष्ट॒का तप॑सा॒ तप्य॑माना ज॒जान॒ गर्भ॑म्महि॒मान॒मिन्द्रम्᳚। तेन॒ दस्यू॒न्व्य॑सहन्त दे॒वा ह॒न्तासु॑राणामभव॒च्छची॑भिः। अना॑नुजामनु॒जाम्माम॑कर्त स॒त्यं वद॒न्त्यन्वि॑च्छ ए॒तत्। भू॒यासम्᳚॥२३॥

%4.3.11.4
अ॒स्य॒ सु॒म॒तौ यथा॑ यू॒यम॒न्या वो॑ अ॒न्यामति॒ मा प्र यु॑क्त। अभू॒न्मम॑ सुम॒तौ वि॒श्ववे॑दा॒ आष्ट॑ प्रति॒ष्ठामवि॑द॒द्धि गा॒धम्। भू॒यास॑मस्य सुम॒तौ यथा॑ यू॒यम॒न्या वो॑ अ॒न्यामति॒ मा प्र यु॑क्त। पञ्च॒ व्यु॑ष्टी॒रनु॒ पञ्च॒ दोहा॒ गां पञ्च॑नाम्नीमृ॒तवो\-ऽनु॒ पञ्च॑। पञ्च॒ दिशः॑ पञ्चद॒शेन॒ कॢ॒प्ताः स॑मा॒नमू᳚र्ध्नीर॒भि लो॒कमेकम्᳚॥२४॥

%4.3.11.5
ऋ॒तस्य॒ गर्भः॑ प्रथ॒मा व्यू॒षुष्य॒पामेका॑ महि॒मान॑म्बिभर्ति। सूर्य॒स्यैका॒ चर॑ति निष्कृ॒तेषु॑ घ॒र्मस्यैका॑ सवि॒तैकां॒ नि य॑च्छति। या प्र॑थ॒मा व्यौच्छ॒थ्सा धे॒नुर॑भवद्य॒मे। सा नः॒ पय॑स्वती धु॒क्ष्वोत्त॑रामुत्तरा॒ꣳ॒ समा᳚म्। शु॒क्रर्\mbox{}ष॑भा॒ नभ॑सा॒ ज्योति॒षागा᳚द्वि॒श्वरू॑पा शब॒लीर॒ग्निके॑तुः। स॒मा॒नमर्थꣴ॑ स्वप॒स्यमा॑ना॒ बिभ्र॑ती ज॒राम॑जर उष॒ आगाः᳚। ऋ॒तू॒नाम्पत्नी᳚ प्रथ॒मेयमागा॒दह्नां᳚ ने॒त्री ज॑नि॒त्री प्र॒जाना᳚म्। एका॑ स॒ती ब॑हु॒धोषो॒ व्यु॑च्छ॒स्यजी᳚र्णा॒ त्वं ज॑रयसि॒ सर्व॑म॒न्यत्॥२५॥

%4.3.12.0
{\anuvakamend[{ऊर्ज॒मेका᳚ प्रतिमु॒ञ्चमा॑ना भू॒यास॒मेकं॒ पत्न्येका॒न्नविꣳ॑श॒तिश्च॑}]}%॥11॥

%4.3.12.1
अग्ने॑ जा॒तान्प्र णु॑दा नः स॒पत्ना॒न्प्रत्यजा॑ताञ्जातवेदो नुदस्व। अ॒स्मे दी॑दिहि सु॒मना॒ अहे॑ड॒न्तव॑ स्या॒ꣳ॒ शर्म॑न्त्रि॒वरू॑थ उ॒द्भित्। सह॑सा जा॒तान्प्र णु॑दा नः स॒पत्ना॒न्प्रत्यजा॑ताञ्जातवेदो नुदस्व। अधि॑ नो ब्रूहि सुमन॒स्यमा॑नो व॒यꣴ स्या॑म॒ प्र णु॑दा नः स॒पत्नान्॑। च॒तु॒श्च॒त्वा॒रि॒ꣳ॒शः स्तोमो॒ वर्चो॒ द्रवि॑णꣳ षोड॒शः स्तोम॒ ओजो॒ द्रवि॑णं पृथि॒व्याः पुरी॑षमसि॥२६॥

%4.3.12.2
अफ्सो॒ नाम॑। एव॒श्छन्दो॒ वरि॑व॒श्छन्दः॑ श॒म्भूश्छन्दः॑ परि॒भूश्छन्द॑ आ॒च्छच्छन्दो॒ मन॒श्छन्दो॒ व्यच॒श्छन्दः॒ सिन्धु॒श्छन्दः॑ समु॒द्रं छन्दः॑ सलि॒लं छन्दः॑ सं॒यच्छन्दो॑ वि॒यच्छन्दो॑ बृ॒हच्छन्दो॑ रथन्त॒रं छन्दो॑ निका॒यश्छन्दो॑ विव॒धश्छन्दो॒ गिर॒श्छन्दो॒ भ्रज॒श्छन्दः॑ स॒ष्टुप्छन्दो॑\-ऽनु॒ष्टुप्छन्दः॑ क॒कुच्छन्द॑स्त्रिक॒कुच्छन्दः॑ का॒व्यं छन्दो᳚\-ऽङ्कु॒पं छन्दः॑॥२७॥

%4.3.12.3
प॒दप॑ङ्क्ति॒श्छन्दो॒\-ऽक्षर॑पङ्क्ति॒श्छन्दो॑ विष्टा॒रप॑ङ्क्ति॒श्छन्दः॑ क्षु॒रो भृज्वा॒ञ्छन्दः॑ प्र॒च्छच्छन्दः॑ प॒क्षश्छन्द॒ एव॒श्छन्दो॒ वरि॑व॒श्छन्दो॒ वय॒श्छन्दो॑ वय॒स्कृच्छन्दो॑ विशा॒लं छन्दो॒ विष्प॑र्धा॒श्छन्द॑श्छ॒दिश्छन्दो॑ दूरोह॒णं छन्द॑स्त॒न्द्रं छन्दो᳚\-ऽङ्का॒ङ्कं छन्दः॑॥२८॥

%4.3.13.0
{\anuvakamend[{अ॒स्य॒ङ्कु॒पञ्छन्द॒स्त्रय॑स्त्रिꣳशच्च}]}%॥12॥

%4.3.13.1
अ॒ग्निर्वृ॒त्राणि॑ जङ्घनद्द्रविण॒स्युर्वि॑प॒न्यया᳚। समि॑द्धः शु॒क्र आहु॑तः॥ त्वꣳ सो॑मासि॒ सत्प॑ति॒स्त्वꣳ राजो॒त वृ॑त्र॒हा। त्वं भ॒द्रो अ॑सि॒ क्रतुः॑॥ भ॒द्रा ते॑ अग्ने स्वनीक सं॒दृग्घो॒रस्य॑ स॒तो विषु॑णस्य॒ चारुः॑। न यत्ते॑ शो॒चिस्तम॑सा॒ वर॑न्त॒ न ध्व॒स्मान॑स्त॒नुवि॒ रेप॒ आ धुः॑॥ भ॒द्रं ते॑ अग्ने सहसि॒न्ननी॑कमुपा॒क आ रो॑चते॒ सूर्य॑स्य।॥२९॥

%4.3.13.2
रुश॑द्दृ॒शे द॑दृशे नक्त॒या चि॒दरू᳚क्षितं दृ॒श आ रू॒पे अन्नम्᳚। सैनानी॑केन सुवि॒दत्रो॑ अ॒स्मे यष्टा॑ दे॒वाꣳ आय॑जिष्ठः स्व॒स्ति। अद॑ब्धो गो॒पा उ॒त नः॑ पर॒स्पा अग्ने᳚ द्यु॒मदु॒त रे॒वद्दि॑दीहि। स्व॒स्ति नो॑ दि॒वो अ॑ग्ने पृथि॒व्या वि॒श्वायु॑र्धेहि य॒जथा॑य देव। यथ्सी॒महि॑ दिविजात॒ प्रश॑स्तं॒ तद॒स्मासु॒ द्रवि॑णं धेहि चि॒त्रम्। यथा॑ होत॒र्मनु॑षः॥३०॥

%4.3.13.3
दे॒वता॑ता य॒ज्ञेभिः॑ सूनो सहसो॒ यजा॑सि। ए॒वानो॑ अ॒द्य स॑म॒ना स॑मा॒नानु॒शन्न॑ग्न उश॒तो य॑क्षि दे॒वान्॥ अ॒ग्निमी॑डे पु॒रोहि॑तं य॒ज्ञस्य॑ दे॒वमृ॒त्विजम्᳚। होता॑रꣳ रत्न॒धात॑मम्॥ वृषा॑ सोम द्यु॒माꣳ अ॑सि॒ वृषा॑ देव॒ वृष॑व्रतः। वृषा॒ धर्मा॑णि दधिषे॥ सान्त॑पना इ॒दꣳ ह॒विर्मरु॑त॒स्तज्जु॑जुष्टन। यु॒ष्माको॒ती रि॑शादसः॥ यो नो॒ मर्तो॑ वसवो दुर्\mbox{}हृणा॒युस्ति॒रः स॒त्यानि॑ मरुतः॥३१॥

%4.3.13.4
जिघाꣳ॑सात्। द्रु॒हः पाशं॒ प्रति॒ स मु॑चीष्ट॒ तपि॑ष्ठेन॒ तप॑सा हन्तना॒ तम्। सं॒व॒थ्स॒रीणा॑ म॒रुतः॑ स्व॒र्का उ॑रु॒क्षयाः॒ सग॑णा॒ मानु॑षेषु। ते᳚\-ऽस्मत्पाशा॒न्प्र मु॑ञ्च॒न्त्वꣳह॑सः सान्तप॒ना म॑दि॒रा मा॑दयि॒ष्णवः॑। पि॒प्री॒हि दे॒वाꣳ उ॑श॒तो य॑विष्ठ वि॒द्वाꣳ ऋ॒तूꣳर्\mbox{}ऋ॑तुपते यजे॒ह। ये दैव्या॑ ऋ॒त्विज॒स्तेभि॑रग्ने॒ त्वꣳ होतॄ॑णाम॒स्याय॑जिष्ठः। अग्ने॒ यद॒द्य वि॒शो अ॑ध्वरस्य होतः॒ पाव॑क॥३२॥

%4.3.13.5
शो॒चे॒ वेष्ट्वꣳ हि यज्वा᳚। ऋ॒ता य॑जासि महि॒ना वि यद्भूर्\mbox{}ह॒व्या व॑ह यविष्ठ॒ या ते॑ अ॒द्य। अ॒ग्निना॑ र॒यिम॑श्नव॒त्पोष॑मे॒व दि॒वेदि॑वे। य॒शसं॑ वी॒रव॑त्तमम्॥ ग॒य॒स्फानो॑ अमीव॒हा व॑सु॒वित्पु॑ष्टि॒वर्ध॑नः। सु॒मि॒त्रः सो॑म नो भव। गृह॑मेधास॒ आ ग॑त॒ मरु॑तो॒ माप॑ भूतन। प्र॒मु॒ञ्चन्तो॑ नो॒ अꣳह॑सः। पू॒र्वीभि॒र्\mbox{}हि द॑दाशि॒म श॒रद्भि॑र्मरुतो व॒यम्। महो॑भिः॥३३॥

%4.3.13.6
च॒र्\mbox{}ष॒णी॒नाम्। प्र बु॒ध्निया॑ ईरते वो॒ महाꣳ॑सि॒ प्र णामा॑नि प्रयज्यवस्तिरध्वम्। स॒ह॒स्रियं॒ दम्य॑म्भा॒गमे॒तं गृ॑हमे॒धीय॑म्मरुतो जुषध्वम्। उप॒ यमेति॑ युव॒तिः सु॒दक्षं॑ दो॒षा वस्तोर्\mbox{}॑ह॒विष्म॑ती घृ॒ताची᳚। उप॒ स्वैन॑म॒रम॑तिर्वसू॒युः। इ॒मो अ॑ग्ने वी॒तत॑मानि ह॒व्याज॑स्रो वक्षि दे॒वता॑ति॒मच्छ॑। प्रति॑ न ईꣳ सुर॒भीणि॑ वियन्तु। क्री॒डं वः॒ शर्धो॒ मारु॑तमन॒र्वाणꣳ॑ रथे॒शुभम्᳚।॥३४॥

%4.3.13.7
कण्वा॑ अ॒भि प्र गा॑यत। अत्या॑सो॒ न ये म॒रुतः॒ स्वञ्चो॑ यक्ष॒दृशो॒ न शु॒भय॑न्त॒ मर्याः᳚। ते ह॑र्म्ये॒ष्ठाः शिश॑वो॒ न शु॒भ्रा व॒थ्सासो॒ न प्र॑क्री॒डिनः॑ पयो॒धाः। प्रैषा॒मज्मे॑षु विथु॒रेव॑ रेजते॒ भूमि॒र्यामे॑षु॒ यद्ध॑ यु॒ञ्जते॑ शु॒भे। ते क्री॒डयो॒ धुन॑यो॒ भ्राज॑दृष्टयः स्व॒यं म॑हि॒त्वं प॑नयन्त॒ धूत॑यः। उ॒प॒ह्व॒रेषु॒ यदचि॑ध्वं य॒यिं वय॑ इव मरुतः॒ केन॑॥३५॥

%4.3.13.8
चि॒त्प॒था। श्चोत॑न्ति॒ कोशा॒ उप॑ वो॒ रथे॒ष्वा घृ॒तमु॑क्षता॒ मधु॑वर्ण॒मर्च॑ते। अ॒ग्निम॑ग्नि॒ꣳ॒ हवी॑मभिः॒ सदा॑ हवन्त वि॒श्पतिम्᳚। ह॒व्य॒वाहं॑ पुरुप्रि॒यम्। तꣳ हि शश्व॑न्त॒ ईड॑ते स्रु॒चा दे॒वं घृ॑त॒श्चुता᳚। अ॒ग्निꣳ ह॒व्याय॒ वोढ॑वे। इन्द्रा᳚ग्नी रोच॒ना दि॒वः श्नथ॑द्वृ॒त्रमिन्द्रं॑ वो वि॒श्वत॒स्परीन्द्रं॒ नरो॒ विश्व॑कर्मन् ह॒विषा॑ वावृधा॒नो विश्व॑कर्मन् ह॒विषा॒ वर्ध॑नेन॥३६॥

%4.4.0.0
{\anuvakamend[{सूर्य॑स्य॒ मनु॑षो मरुतः॒ पाव॑क॒ महो॑भी रथे॒शुभं॒ केन॒ षट्च॑त्वारिꣳशच्च}]}%॥13॥

%4.4.0.0

{\anuvakamend[{र॒श्मिर॑सि॒ राज्ञ्य॑स्य॒यं पु॒रो हरि॑केशो॒\-ऽग्निर्मू॒र्धेन्द्रा॒ग्निभ्यां॒ बृह॒स्पति॑र्भूय॒स्कृद॑स्य॒ग्निना॑ विश्वा॒षाट्प्र॒जाप॑ति॒र्मन॑सा॒ कृत्ति॑का॒ मधु॑श्च स॒मिद्दि॒शान्द्वाद॑श}]}%॥12॥
\prashnaend{ र॒श्मिर॑सि॒ प्रति॑ धे॒नुम॑सि स्तनयित्नु॒सनि॑रस्यादि॒त्यानाꣳ॑ स॒प्तत्रिꣳ॑शत्॥37॥ र॒श्मिर॑सि॒ को अ॒द्य यु॑ङ्क्ते॥}
%%% END PRASHNA

\sect{चतुर्थः प्रश्नः}\setcounter{anuvakam}{0}
\dnsub{तैत्तिरीयसंहितायां चतुर्थकाण्डे चतुर्थः प्रश्नः}
%4.4.1.0
%4.4.1.1
र॒श्मिर॑सि॒ क्षया॑य त्वा॒ क्षयं॑ जिन्व॒ प्रेति॑रसि॒ धर्मा॑य त्वा॒ धर्मं॑ जि॒न्वान्वि॑तिरसि दि॒वे त्वा॒ दिवं॑ जिन्व सं॒धिर॑स्य॒न्तरि॑क्षाय त्वा॒न्तरि॑क्षं जिन्व प्रति॒धिर॑सि पृथि॒व्यै त्वा॑ पृथि॒वीं जि॑न्व विष्ट॒म्भो॑\-ऽसि॒ वृष्ट्यै᳚ त्वा॒ वृष्टिं॑ जिन्व प्र॒वास्यह्ने॒ त्वाह॑र्जिन्वानु॒वासि॒ रात्रि॑यै त्वा॒ रात्रिं॑ जिन्वो॒शिग॑सि॥१॥

%4.4.1.2
वसु॑भ्यस्त्वा॒ वसू᳚ञ्जिन्व प्रके॒तो॑\-ऽसि रु॒द्रेभ्य॑स्त्वा रु॒द्राञ्जि॑न्व सुदी॒तिर॑स्यादि॒त्येभ्य॑स्त्वा\-ऽ\-ऽदि॒त्याञ्जि॒न्वौजो॑\-ऽसि पि॒तृभ्य॑स्त्वा पि॒तॄञ्जि॑न्व॒ तन्तु॑रसि प्र॒जाभ्य॑स्त्वा प्र॒जा जि॑न्व पृतना॒षाड॑सि प॒शुभ्य॑स्त्वा प॒शूञ्जि॑न्व रे॒वद॒स्योष॑धीभ्य॒स्त्वौष॑धीर्जिन्वाभि॒जिद॑सि यु॒क्तग्रा॒वेन्द्रा॑य॒ त्वेन्द्रं॑ जि॒न्वाधि॑पतिरसि प्रा॒णाय॑॥२॥

%4.4.1.3
त्वा॒ प्रा॒णं जि॑न्व य॒न्तास्य॑पा॒नाय॑ त्वापा॒नं जि॑न्व स॒ꣳ॒सर्पो॑\-ऽसि॒ चक्षु॑षे त्वा॒ चक्षु॑र्जिन्व वयो॒धा अ॑सि॒ श्रोत्रा॑य त्वा॒ श्रोत्रं॑ जिन्व त्रि॒वृद॑सि प्र॒वृद॑सि सं॒वृद॑सि वि॒वृद॑सि सꣳरो॒हो॑\-ऽसि नीरो॒हो॑\-ऽसि प्ररो॒हो᳚\-ऽस्यनुरो॒हो॑\-ऽसि वसु॒को॑\-ऽसि॒ वेष॑श्रिरसि॒ वस्य॑ष्टिरसि॥३॥

%4.4.2.0
{\anuvakamend[{उ॒शिग॑सि प्रा॒णाय॒ त्रिच॑त्वारिꣳशच्च}]}%॥१॥

%4.4.2.1
राज्ञ्य॑सि॒ प्राची॒ दिग्वस॑वस्ते दे॒वा अधि॑पतयो॒\-ऽग्निर्\mbox{}हे॑ती॒नाम्प्र॑तिध॒र्ता त्रि॒वृत्त्वा॒ स्तोमः॑ पृथि॒व्याꣴ श्र॑य॒त्वाज्य॑मु॒क्थ\-मव्य॑थयथ्स्तभ्नातु रथन्त॒रꣳ साम॒ प्रति॑ष्ठित्यै वि॒राड॑सि दक्षि॒णा दिग्रु॒द्रास्ते॑ दे॒वा अधि॑पतय॒ इन्द्रो॑ हेती॒नाम्प्र॑तिध॒र्ता प॑ञ्चद॒शस्त्वा॒ स्तोमः॑ पृथि॒व्याꣴ श्र॑यतु॒ प्रउ॑गमु॒क्थमव्य॑थयथ्स्तभ्नातु बृ॒हथ्साम॒ प्रति॑ष्ठित्यै स॒म्राड॑सि प्र॒तीची॒ दिक्॥४॥

%4.4.2.2
आ॒दि॒त्यास्ते॑ दे॒वा अधि॑पतयः॒ सोमो॑ हेती॒नाम्प्र॑तिध॒र्ता स॑प्तद॒शस्त्वा॒ स्तोमः॑ पृथि॒व्याꣴ श्र॑यतु मरुत्व॒तीय॑मु॒क्थ\-मव्य॑थयथ्स्तभ्नातु वैरू॒पꣳ साम॒ प्रति॑ष्ठित्यै स्व॒राड॒स्युदी॑ची॒ दिग्विश्वे॑ ते दे॒वा अधि॑पतयो॒ वरु॑णो हेती॒नाम्प्र॑तिध॒र्तैक॑\-वि॒ꣳ॒शस्त्वा॒ स्तोमः॑ पृथि॒व्याꣴ श्र॑यतु॒ निष्के॑वल्यमु॒क्थमव्य॑थयथ्स्तभ्नातु वैरा॒जꣳ साम॒ प्रति॑ष्ठित्या॒ अधि॑पत्न्यसि बृह॒ती दिङ्म॒रुत॑स्ते दे॒वा अधि॑पतयः॥५॥

%4.4.2.3
बृह॒स्पति॑र्\mbox{}हेती॒नाम्प्र॑तिध॒र्ता त्रि॑णवत्रयस्त्रि॒ꣳ॒शौ त्वा॒ स्तोमौ॑ पृथि॒व्याꣴ श्र॑यतां वैश्वदेवाग्निमारु॒ते उ॒क्थे अव्य॑थयन्ती स्तभ्नीताꣳ शाक्वररैव॒ते साम॑नी॒ प्रति॑ष्ठित्या अ॒न्तरि॑क्षा॒यर्\mbox{}ष॑यस्त्वा प्रथम॒जा दे॒वेषु॑ दि॒वो मात्र॑या वरि॒णा प्र॑थन्तु विध॒र्ता चा॒यमधि॑पतिश्च॒ ते त्वा॒ सर्वे॑ संविदा॒ना नाक॑स्य पृ॒ष्ठे सु॑व॒र्गे लो॒के यज॑मानं च सादयन्तु॥६॥

%4.4.3.0
{\anuvakamend[{प्र॒तीची॒ दिङ्म॒रुत॑स्ते दे॒वा अधि॑पतयश्चत्वारि॒ꣳ॒शच्च॑}]}%॥२॥

%4.4.3.1
अ॒यम्पु॒रो हरि॑केशः॒ सूर्य॑रश्मि॒स्तस्य॑ रथगृ॒थ्सश्च॒ रथौ॑जाश्च सेनानिग्राम॒ण्यौ॑ पुञ्जिकस्थ॒ला च॑ कृतस्थ॒ला चा᳚फ्स॒रसौ॑ यातु॒धाना॑ हे॒ती रक्षाꣳ॑सि॒ प्रहे॑तिर॒यं द॑क्षि॒णा वि॒श्वक॑र्मा॒ तस्य॑ रथस्व॒नश्च॒ रथे॑चित्रश्च सेनानिग्राम॒ण्यौ॑ मेन॒का च॑ सहज॒न्या चा᳚फ्स॒रसौ॑ द॒ङ्क्ष्णवः॑ प॒शवो॑ हे॒तिः पौरु॑षेयो व॒धः प्रहे॑तिर॒यम्प॒श्चाद्वि॒श्वव्य॑चा॒स्तस्य॒ रथ॑प्रोत॒श्चास॑मरथश्च सेनानिग्राम॒ण्यौ᳚ प्र॒म्लोच॑न्ती च॥७॥

%4.4.3.2
अ॒नु॒म्लोच॑न्ती चाफ्स॒रसौ॑ स॒र्पा हे॒तिर्व्या॒घ्राः प्रहे॑तिर॒यमु॑त्त॒राथ्सं॒यद्व॑सु॒स्तस्य॑ सेन॒जिच्च॑ सु॒षेण॑श्च सेनानिग्राम॒ण्यौ॑ वि॒श्वाची॑ च घृ॒ताची॑ चाफ्स॒रसा॒वापो॑ हे॒तिर्वातः॒ प्रहे॑तिर॒यमु॒पर्य॒र्वाग्व॑सु॒स्तस्य॒ तार्क्ष्य॒श्चारि॑ष्टनेमिश्च सेनानिग्राम॒ण्या॑\-वु॒र्वशी॑ च पू॒र्वचि॑त्तिश्चाफ्स॒रसौ॑ वि॒द्युद्धे॒तिर॑व॒स्फूर्ज॒न्प्रहे॑ति॒स्तेभ्यो॒ नम॒स्ते नो॑ मृडयन्तु॒ ते यम्॥८॥

%4.4.3.3
द्वि॒ष्मो यश्च॑ नो॒ द्वेष्टि॒ तं वो॒ जम्भे॑ दधाम्या॒योस्त्वा॒ सद॑ने सादया॒म्यव॑तश्छा॒यायां॒ नमः॑ समु॒द्राय॒ नमः॑ समु॒द्रस्य॒ चक्ष॑से परमे॒ष्ठी त्वा॑ सादयतु दि॒वः पृ॒ष्ठे व्यच॑स्वती॒म्प्रथ॑स्वतीं वि॒भूम॑तीम्प्र॒भूम॑तीं परि॒भूम॑तीं॒ दिवं॑ यच्छ॒ दिवं॑ दृꣳह॒ दिवं॒ मा हिꣳ॑सी॒र्विश्व॑स्मै प्रा॒णाया॑पा॒नाय॑ व्या॒नायो॑दा॒नाय॑ प्रति॒ष्ठायै॑ च॒रित्रा॑य॒ सूर्य॑स्त्वा॒भि पा॑तु म॒ह्या स्व॒स्त्या छ॒र्दिषा॒ शन्त॑मेन॒ तया॑ दे॒वत॑याङ्गिर॒स्वद्ध्रु॒वा सी॑द। प्रोथ॒दश्वो॒ न यव॑से अवि॒ष्यन् य॒दा म॒हः सं॒वर॑णा॒द्व्यस्था᳚त्। आद॑स्य॒ वातो॒ अनु॑ वाति शो॒चिरध॑ स्म ते॒ व्रज॑नं कृ॒ष्णम॑स्ति॥९॥

%4.4.4.0
{\anuvakamend[{प्र॒म्लोच॑न्ती च॒ यꣴ स्व॒स्त्याष्टाविꣳ॑शतिश्च}]}%॥३॥

%4.4.4.1
अ॒ग्निर्मू॒र्धा दि॒वः क॒कुत्पतिः॑ पृथि॒व्या अ॒यम्। अ॒पाꣳ रेताꣳ॑सि जिन्वति॥ त्वाम॑ग्ने॒ पुष्क॑रा॒दध्यथ॑र्वा॒ निर॑मन्थत। मू॒र्ध्नो विश्व॑स्य वा॒घतः॑॥ अ॒यम॒ग्निः स॑ह॒स्रिणो॒ वाज॑स्य श॒तिन॒स्पतिः॑। मू॒र्धा क॒वी र॑यी॒णाम्॥ भुवो॑ य॒ज्ञस्य॒ रज॑सश्च ने॒ता यत्रा॑ नि॒युद्भिः॒ सच॑से शि॒वाभिः॑। दि॒वि मू॒र्धानं॑ दधिषे सुव॒र्\mbox{}षां जि॒ह्वाम॑ग्ने चकृषे हव्य॒वाहम्᳚॥ अबो᳚ध्य॒ग्निः स॒मिधा॒ जना॑नाम्॥१०॥

%4.4.4.2
प्रति॑ धे॒नुमि॑वाय॒तीमु॒षासम्᳚। य॒ह्वा इ॑व॒ प्र व॒यामु॒ज्जिहा॑नाः॒ प्र भा॒नवः॑ सिस्रते॒ नाक॒मच्छ॑। अवो॑चाम क॒वये॒ मेध्या॑य॒ वचो॑ व॒न्दारु॑ वृष॒भाय॒ वृष्णे᳚। गवि॑ष्ठिरो॒ नम॑सा॒ स्तोम॑म॒ग्नौ दि॒वीव॑ रु॒क्ममु॒र्व्यञ्च॑मश्रेत्। जन॑स्य गो॒पा अ॑जनिष्ट॒ जागृ॑विर॒ग्निः सु॒दक्षः॑ सुवि॒ताय॒ नव्य॑से। घृ॒तप्र॑तीको बृह॒ता दि॑वि॒स्पृशा᳚ द्यु॒मद्वि भा॑ति भर॒तेभ्यः॒ शुचिः॑। त्वाम॑ग्ने॒ अङ्गि॑रसः॥११॥

%4.4.4.3
गुहा॑ हि॒तमन्व॑विन्दञ्छिश्रिया॒णं वने॑वने। स जा॑यसे म॒थ्यमा॑नः॒ सहो॑ म॒हत्त्वामा॑हुः॒ सह॑सस्पु॒त्रम॑ङ्गिरः। य॒ज्ञस्य॑ के॒तुम्प्र॑थ॒मम्पु॒रोहि॑तम॒ग्निं नर॑स्त्रिषध॒स्थे समि॑न्धते। इन्द्रे॑ण दे॒वैः स॒रथ॒ꣳ॒ स ब॒र्\mbox{}हिषि॒ सीद॒न्नि होता॑ य॒जथा॑य सु॒क्रतुः॑। त्वं चि॑त्रश्रवस्तम॒ हव॑न्ते वि॒क्षु ज॒न्तवः॑। शो॒चिष्के॑शं पुरुप्रि॒याग्ने॑ ह॒व्याय॒ वोढ॑वे। सखा॑यः॒ सं वः॑ स॒म्यञ्च॒मिषम्᳚॥१२॥

%4.4.4.4
स्तोमं॑ चा॒ग्नये᳚। वर्\mbox{}षि॑ष्ठाय क्षिती॒नामू॒र्जो नप्त्रे॒ सह॑स्वते। सꣳस॒मिद्यु॑वसे वृष॒न्नग्ने॒ विश्वा᳚न्य॒र्य आ। इ॒डस्प॒दे समि॑ध्यसे॒ स नो॒ वसू॒न्या भ॑र। ए॒ना वो॑ अ॒ग्निं नम॑सो॒र्जो नपा॑त॒मा हु॑वे। प्रि॒यं चेति॑ष्ठमर॒तिꣴ स्व॑ध्व॒रं विश्व॑स्य दू॒तम॒मृतम्᳚। स यो॑जते अरु॒षो वि॒श्वभो॑जसा॒ स दु॑द्रव॒थ्स्वा॑हुतः। सु॒ब्रह्मा॑ य॒ज्ञः सु॒शमी᳚॥१३॥

%4.4.4.5
वसू॑नां दे॒वꣳ राधो॒ जना॑नाम्। उद॑स्य शो॒चिर॑स्थादा॒जुह्वा॑नस्य मी॒ढुषः॑। उद्धू॒मासो॑ अरु॒षासो॑ दिवि॒स्पृशः॒ सम॒ग्निमि॑न्धते॒ नरः॑। अग्ने॒ वाज॑स्य॒ गोम॑त॒ ईशा॑नः सहसो यहो। अ॒स्मे धे॑हि जातवेदो॒ महि॒ श्रवः॑। स इ॑धा॒नो वसु॑ष्क॒विर॒ग्निरी॒डेन्यो॑ गि॒रा। रे॒वद॒स्मभ्यं॑ पुर्वणीक दीदिहि। क्ष॒पो रा॑जन्नु॒त त्मनाग्ने॒ वस्तो॑रु॒तोषसः॑। स ति॑ग्मजम्भ॥१४॥

%4.4.4.6
र॒क्षसो॑ दह॒ प्रति॑। आ ते॑ अग्न इधीमहि द्यु॒मन्तं॑ देवा॒जरम्᳚। यद्ध॒ स्या ते॒ पनी॑यसी स॒मिद्दी॒दय॑ति॒ द्यवीषꣴ॑ स्तो॒तृभ्य॒ आ भ॑र। आ ते॑ अग्न ऋ॒चा ह॒विः शु॒क्रस्य॑ ज्योतिषस्पते। सुश्च॑न्द्र॒ दस्म॒ विश्प॑ते॒ हव्य॑वा॒ट्तुभ्यꣳ॑ हूयत॒ इषꣴ॑ स्तो॒तृभ्य॒ आ भ॑र। उ॒भे सु॑श्चन्द्र स॒र्पिषो॒ दर्वी᳚ श्रीणीष आ॒सनि॑। उ॒तो न॒ उत्पु॑पूर्याः॥१५॥

%4.4.4.7
उ॒क्थेषु॑ शवसस्पत॒ इषꣴ॑ स्तो॒तृभ्य॒ आ भ॑र। अग्ने॒ तम॒द्याश्वं॒ न स्तोमैः॒ क्रतुं॒ न भ॒द्रꣳ हृ॑दि॒स्पृशम्᳚। ऋ॒ध्यामा॑ त॒ ओहैः᳚। अधा॒ ह्य॑ग्ने॒ क्रतो᳚र्भ॒द्रस्य॒ दक्ष॑स्य सा॒धोः। र॒थीर्\mbox{}ऋ॒तस्य॑ बृह॒तो ब॒भूथ॑। आ॒भिष्टे॑ अ॒द्य गी॒र्भिर्गृ॒णन्तो\-ऽग्ने॒ दाशे॑म। प्र ते॑ दि॒वो न स्त॑नयन्ति॒ शुष्माः᳚। ए॒भिर्नो॑ अ॒र्कैर्भवा॑ नो अ॒र्वाङ्॥१६॥

%4.4.4.8
सुव॒र्न ज्योतिः॑। अग्ने॒ विश्वे॑भिः सु॒मना॒ अनी॑कैः। अ॒ग्निꣳ होता॑रम्मन्ये॒ दास्व॑न्तं॒ वसोः᳚ सू॒नुꣳ सह॑सो जा॒तवे॑दसम्। विप्रं॒ न जा॒तवे॑दसम्। य ऊ॒र्ध्वया᳚ स्वध्व॒रो दे॒वो दे॒वाच्या॑ कृ॒पा। घृ॒तस्य॒ विभ्रा᳚ष्टि॒मनु॑ शु॒क्रशो॑चिष आ॒जुह्वा॑नस्य स॒र्पिषः॑। अग्ने॒ त्वन्नो॒ अन्त॑मः। उ॒त त्रा॒ता शि॒वो भ॑व वरू॒थ्यः॑। तं त्वा॑ शोचिष्ठ दीदिवः। सु॒म्नाय॑ नू॒नमी॑महे॒ सखि॑भ्यः। वसु॑र॒ग्निर्वसु॑श्रवाः। अच्छा॑ नक्षि द्यु॒मत्त॑मो र॒यिं दाः᳚॥१७॥

%4.4.5.0
{\anuvakamend[{जना॑ना॒मङ्गि॑रस॒ इषꣳ॑ सु॒शमी॑ तिग्मजम्भ पुपूर्या अ॒र्वाङ्वसु॑श्रवाः॒ पञ़्च॑ च}]}%॥४॥

%4.4.5.1
इ॒न्द्रा॒ग्नि\-भ्यां᳚ त्वा स॒युजा॑ यु॒जा यु॑नज्म्याघा॒राभ्यां॒ तेज॑सा॒ वर्च॑सो॒क्थेभिः॒ स्तोमे॑भि॒श्छन्दो॑भी र॒य्यै पोषा॑य सजा॒ताना᳚म्मध्यम॒स्थेया॑य॒ मया᳚ त्वा स॒युजा॑ यु॒जा यु॑नज्म्य॒म्बा दु॒ला नि॑त॒त्निर॒भ्रय॑न्ती मे॒घय॑न्ती व॒र्\mbox{}षय॑न्ती चुपु॒णीका॒ नामा॑सि प्र॒जाप॑तिना त्वा॒ विश्वा॑भिर्धी॒भिरुप॑ दधामि पृथि॒व्यु॑दपु॒रमन्ने॑न वि॒ष्टा म॑नु॒ष्या᳚स्ते गो॒प्तारो॒\-ऽग्निर्विय॑त्तो\-ऽस्यां॒ ताम॒हम्प्र प॑द्ये॒ सा॥१८॥

%4.4.5.2
मे॒ शर्म॑ च॒ वर्म॑ चा॒स्त्वधि॑द्यौर॒न्तरि॑क्षं॒ ब्रह्म॑णा वि॒ष्टा म॒रुत॑स्ते गो॒प्तारो॑ वा॒युर्विय॑त्तो\-ऽस्यां॒ ताम॒हम्प्र प॑द्ये॒ सा मे॒ शर्म॑ च॒ वर्म॑ चास्तु॒ द्यौरप॑राजिता॒मृते॑न वि॒ष्टादि॒त्यास्ते॑ गो॒प्तारः॒ सूर्यो॒ विय॑त्तो\-ऽस्यां॒ ताम॒हम्प्र प॑द्ये॒ सा मे॒ शर्म॑ च॒ वर्म॑ चास्तु॥१९॥

%4.4.6.0
{\anuvakamend[{सा\-ऽष्टाच॑त्वारिꣳशच्च}]}%॥५॥

%4.4.6.1
बृह॒स्पति॑स्त्वा सादयतु पृथि॒व्याः पृ॒ष्ठे ज्योति॑ष्मतीं॒ विश्व॑स्मै प्रा॒णाया॑पा॒नाय॒ विश्वं॒ ज्योति॑र्यच्छा॒ग्निस्ते\-ऽधि॑पतिर्\-वि॒श्वक॑र्मा त्वा सादयत्व॒न्तरि॑क्षस्य पृ॒ष्ठे ज्योति॑ष्मतीं॒ विश्व॑स्मै प्रा॒णाया॑पा॒नाय॒ विश्वं॒ ज्योति॑र्यच्छ वा॒युस्ते\-ऽधि॑पतिः प्र॒जाप॑तिस्त्वा सादयतु दि॒वः पृ॒ष्ठे ज्योति॑ष्मतीं॒ विश्व॑स्मै प्रा॒णाया॑पा॒नाय॒ विश्वं॒ ज्योति॑र्यच्छ परमे॒ष्ठी ते\-ऽधि॑पतिः पुरोवात॒सनि॑रस्यभ्र॒सनि॑रसि विद्यु॒थ्सनिः॑॥२०॥

%4.4.6.2
अ॒सि॒ स्त॒न॒यि॒त्नु॒सनि॑रसि वृष्टि॒सनि॑रस्य॒ग्नेर्यान्य॑सि दे॒वाना॑मग्ने॒यान्य॑सि वा॒योर्यान्य॑सि दे॒वानां᳚ वायो॒यान्य॑स्य॒न्तरि॑क्षस्य॒ यान्य॑सि दे॒वाना॑मन्तरिक्ष॒यान्य॑स्य॒न्तरि॑क्षमस्य॒न्तरि॑क्षाय त्वा सलि॒लाय॑ त्वा॒ सर्णी॑काय त्वा॒ सती॑काय त्वा॒ केता॑य त्वा॒ प्रचे॑तसे त्वा॒ विव॑स्वते त्वा दि॒वस्त्वा॒ ज्योति॑ष आदि॒त्येभ्य॑स्त्व॒र्चे त्वा॑ रु॒चे त्वा᳚ द्यु॒ते त्वा॑ भा॒से त्वा॒ ज्योति॑षे त्वा यशो॒दां त्वा॒ यश॑सि तेजो॒दां त्वा॒ तेज॑सि पयो॒दां त्वा॒ पय॑सि वर्चो॒दां त्वा॒ वर्च॑सि द्रविणो॒दां त्वा॒ द्रवि॑णे सादयामि॒ तेनर्\mbox{}षि॑णा॒ तेन॒ ब्रह्म॑णा॒ तया॑ दे॒वत॑याङ्गिर॒स्वद्ध्रु॒वा सी॑द॥२१॥

%4.4.7.0
{\anuvakamend[{वि॒द्यु॒थ्सनि॑र्द्यु॒त्वैका॒न्नत्रि॒ꣳ॒शच्च॑}]}%॥६॥

%4.4.7.1
भू॒य॒स्कृद॑सि वरिव॒स्कृद॑सि॒ प्राच्य॑स्यू॒र्ध्वास्य॑न्तरिक्ष॒सद॑स्य॒न्तरि॑क्षे सीदाफ्सु॒षद॑सि श्येन॒सद॑सि गृध्र॒सद॑सि सुपर्ण॒सद॑सि नाक॒सद॑सि पृथि॒व्यास्त्वा॒ द्रवि॑णे सादयाम्य॒न्तरि॑क्षस्य त्वा॒ द्रवि॑णे सादयामि दि॒वस्त्वा॒ द्रवि॑णे सादयामि दि॒शां त्वा॒ द्रवि॑णे सादयामि द्रविणो॒दां त्वा॒ द्रवि॑णे सादयामि प्रा॒णं मे॑ पाह्यपा॒नं मे॑ पाहि व्या॒नम्मे᳚॥२२॥

%4.4.7.2
पा॒ह्यायु॑र्मे पाहि वि॒श्वायु॑र्मे पाहि स॒र्वायु॑र्मे पा॒ह्यग्ने॒ यत्ते॒ पर॒ꣳ॒ हृन्नाम॒ तावेहि॒ सꣳ र॑भावहै॒ पाञ्च॑जन्ये॒ष्वप्ये᳚ध्यग्ने॒ यावा॒ अया॑वा॒ एवा॒ ऊमाः॒ सब्दः॒ सग॑रः सु॒मेकः॑॥२३॥

%4.4.8.0
{\anuvakamend[{व्या॒नम्मे॒ द्वात्रिꣳ॑शच्च}]}%॥७॥

%4.4.8.1
अ॒ग्निना॑ विश्वा॒षाट्थ्सूर्ये॑ण स्व॒राट्क्रत्वा॒ शची॒पति॑र्\mbox{}ऋष॒भेण॒ त्वष्टा॑ य॒ज्ञेन॑ म॒घवा॒न्दक्षि॑णया सुव॒र्गो म॒न्युना॑ वृत्र॒हा सौहा᳚र्द्येन तनू॒धा अन्ने॑न॒ गयः॑ पृथि॒व्यास॑नोदृ॒ग्भिर॑न्ना॒दो व॑षट्का॒रेण॒र्द्धः साम्ना॑ तनू॒पा वि॒राजा॒ ज्योति॑ष्मा॒न् ब्रह्म॑णा सोम॒पा गोभि॑र्य॒ज्ञं दा॑धार क्ष॒त्रेण॑ मनु॒ष्या॑नश्वे॑न च॒ रथे॑न च व॒ज्र्यृ॑तुभिः॑ प्र॒भुः सं॑वथ्स॒रेण॑ परि॒भूस्तप॒साना॑धृष्टः॒ सूर्यः॒ सन्त॒नूभिः॑॥२४॥

%4.4.9.0
{\anuvakamend[{अ॒ग्निनैका॒न्नप॑ञ्चा॒शत्}]}%॥८॥

%4.4.9.1
प्र॒जाप॑ति॒र्मन॒सान्धो\-ऽच्छे॑तो धा॒ता दी॒क्षायाꣳ॑ सवि॒ता भृ॒त्यां पू॒षा सो॑म॒क्रय॑ण्यां॒ वरु॑ण॒ उप॑न॒द्धो\-ऽसु॑रः क्री॒यमा॑णो मि॒त्रः क्री॒तः शि॑पिवि॒ष्ट आसा॑दितो न॒रन्धि॑षः प्रो॒ह्यमा॒णो\-ऽधि॑पति॒राग॑तः प्र॒जाप॑तिः प्रणी॒यमा॑नो॒\-ऽग्निराग्नी᳚ध्रे॒ बृह॒स्पति॒राग्नी᳚ध्रात्प्रणी॒यमा॑न॒ इन्द्रो॑ हवि॒र्धाने\-ऽदि॑ति॒रासा॑दितो॒ विष्णु॑रुपावह्रि॒यमा॒णो\-ऽथ॒र्वोपो᳚त्तो य॒मो॑\-ऽभिषु॑तो\-ऽपूत॒पा आ॑धू॒यमा॑नो वा॒युः पू॒यमा॑नो मि॒त्रः क्षी॑र॒श्रीर्म॒न्थी स॑क्तु॒श्रीर्वै᳚श्वदे॒व उन्नी॑तो रुद्र॒ आहु॑तो वा॒युरावृ॑त्तो नृ॒चक्षाः॒ प्रति॑ख्यातो भ॒क्ष आग॑तः पितृ॒णां ना॑राश॒ꣳ॒सो\-ऽसु॒रात्तः॒ सिन्धु॑रवभृ॒थम॑वप्र॒यन्थ्स॑मु॒द्रो\-ऽव॑गतः सलि॒लः प्रप्लु॑तः सुव॑रु॒दृचं॑ ग॒तः॥२५॥

%4.4.10.0
{\anuvakamend[{रु॒द्र एक॑विꣳशतिश्च}]}%॥८॥

%4.4.10.1
कृत्ति॑का॒ नक्ष॑त्रम॒ग्निर्दे॒वता॒ग्ने रुचः॑ स्थ प्र॒जाप॑तेर्धा॒तुः सोम॑स्य॒र्चे त्वा॑ रु॒चे त्वा᳚ द्यु॒ते त्वा॑ भा॒से त्वा॒ ज्योति॑षे त्वा रोहि॒णी नक्ष॑त्रं प्र॒जाप॑तिर्दे॒वता॑ मृगशी॒र्\mbox{}षं नक्ष॑त्र॒ꣳ॒ सोमो॑ दे॒वता॒र्द्रा नक्ष॑त्रꣳ रु॒द्रो दे॒वता॒ पुन॑र्वसू॒ नक्ष॑त्र॒मदि॑ति\-र्दे॒वता॑ ति॒ष्यो॑ नक्ष॑त्र॒म्बृह॒स्पति॑र्दे॒वता᳚श्रे॒षा नक्ष॑त्रꣳ स॒र्पा दे॒वता॑ म॒घा नक्ष॑त्रम्पि॒तरो॑ दे॒वता॒ फल्गु॑नी॒ नक्ष॑त्रम्॥२६॥

%4.4.10.2
अ॒र्य॒मा दे॒वता॒ फल्गु॑नी॒ नक्ष॑त्र॒म्भगो॑ दे॒वता॒ हस्तो॒ नक्ष॑त्रꣳ सवि॒ता दे॒वता॑ चि॒त्रा नक्ष॑त्र॒मिन्द्रो॑ दे॒वता᳚ स्वा॒ती नक्ष॑त्रं वा॒युर्दे॒वता॒ विशा॑खे॒ नक्ष॑त्रमिन्द्रा॒ग्नी दे॒वता॑\-ऽनूरा॒धा नक्ष॑त्रम्मि॒त्रो दे॒वता॑ रोहि॒णी नक्ष॑त्र॒मिन्द्रो॑ दे॒वता॑ वि॒चृतौ॒ नक्ष॑त्रम्पि॒तरो॑ दे॒वता॑षा॒ढा नक्ष॑त्र॒मापो॑ दे॒वता॑षा॒ढा नक्ष॑त्रं॒ विश्वे॑ दे॒वा दे॒वता᳚ श्रो॒णा नक्ष॑त्त्रं॒ विष्णु॑र्दे॒वता॒ श्रवि॑ष्ठा॒ नक्ष॑त्रं॒ वस॑वः॥२७॥

%4.4.10.3
दे॒वता॑ श॒तभि॑ष॒ङ्नक्ष॑त्र॒मिन्द्रो॑ दे॒वता᳚ प्रोष्ठप॒दा नक्ष॑त्रम॒ज एक॑पाद्दे॒वता᳚ प्रोष्ठप॒दा नक्ष॑त्र॒महि॑र्बु॒ध्नियो॑ दे॒वता॑ रे॒वती॒ नक्ष॑त्रं पू॒षा दे॒वता᳚श्व॒युजौ॒ नक्ष॑त्रम॒श्विनौ॑ दे॒वता॑प॒भर॑णी॒र्नक्ष॑त्रं य॒मो दे॒वता॑ पू॒र्णा प॒श्चाद्यत्ते॑ दे॒वा अद॑धुः॥२८॥

%4.4.11.0
{\anuvakamend[{फल्गु॑नी॒ नक्ष॑त्रं॒ वस॑व॒स्त्रय॑स्त्रिꣳशच्च}]}%॥10॥

%4.4.11.1
मधु॑श्च॒ माध॑वश्च॒ वास॑न्तिकावृ॒तू शु॒क्रश्च॒ शुचि॑श्च॒ ग्रैष्मा॑वृ॒तू नभ॑श्च नभ॒स्य॑श्च॒ वार्\mbox{}षि॑कावृ॒तू इ॒षश्चो॒र्जश्च॑ शार॒दावृ॒तू सह॑श्च सह॒स्य॑श्च॒ हैम॑न्तिकावृ॒तू तप॑श्च तप॒स्य॑श्च शैशि॒रावृ॒तू अ॒ग्नेर॑न्तःश्ले॒षो॑\-ऽसि॒ कल्पे॑तां॒ द्यावा॑पृथि॒वी कल्प॑न्ता॒माप॒ ओष॑धीः॒ कल्प॑न्ताम॒ग्नयः॒ पृथ॒ङ्मम॒ ज्यैष्ठ्या॑य॒ सव्र॑ताः॥२९॥

%4.4.11.2
ये᳚\-ऽग्नयः॒ सम॑नसो\-ऽन्त॒रा द्यावा॑पृथि॒वी शै॑शि॒रावृ॒तू अ॒भि कल्प॑माना॒ इन्द्र॑मिव दे॒वा अ॒भि सं वि॑शन्तु सं॒यच्च॒ प्रचे॑ताश्चा॒ग्नेः सोम॑स्य॒ सूर्य॑स्यो॒ग्रा च॑ भी॒मा च॑ पितृ॒णां य॒मस्येन्द्र॑स्य ध्रु॒वा च॑ पृथि॒वी च॑ दे॒वस्य॑ सवि॒तुर्म॒रुतां॒ वरु॑णस्य ध॒र्त्री च॒ धरि॑त्री च मि॒त्रावरु॑णयोर्मि॒त्रस्य॑ धा॒तुः प्राची॑ च प्र॒तीची॑ च॒ वसू॑नाꣳ रु॒द्राणा᳚म्॥३०॥

%4.4.11.3
आ॒दि॒त्याना॒न्ते ते\-ऽधि॑पतय॒स्तेभ्यो॒ नम॒स्ते नो॑ मृडयन्तु॒ ते यं द्वि॒ष्मो यश्च॑ नो॒ द्वेष्टि॒ तं वो॒ जम्भे॑ दधामि स॒हस्र॑स्य प्र॒मा अ॑सि स॒हस्र॑स्य प्रति॒मा अ॑सि स॒हस्र॑स्य वि॒मा अ॑सि स॒हस्र॑स्यो॒न्मा अ॑सि साह॒स्रो॑\-ऽसि स॒हस्रा॑य त्वे॒मा मे॑ अग्न॒ इष्ट॑का धे॒नवः॑ स॒न्त्वेका॑ च श॒तं च॑ स॒हस्रं॑ चा॒युतं॑ च॥३१॥

%4.4.11.4
नि॒युतं॑ च प्र॒युतं॒ चार्बु॑दं च॒ न्य॑र्बुदं च समु॒द्रश्च॒ मध्यं॒ चान्त॑श्च परा॒र्धश्चे॒मा मे॑ अग्न॒ इष्ट॑का धे॒नवः॑ सन्तु ष॒ष्टिः स॒हस्र॑म॒युत॒मक्षी॑यमाणा ऋत॒स्थाः स्थ॑र्ता॒वृधो॑ घृत॒श्चुतो॑ मधु॒श्चुत॒ ऊर्ज॑स्वतीः स्वधा॒विनी॒स्ता मे॑ अग्न॒ इष्ट॑का धे॒नवः॑ सन्तु वि॒राजो॒ नाम॑ काम॒दुघा॑ अ॒मुत्रा॒मुष्मि॑ल्लोँ॒के॥३२॥

%4.4.12.0
{\anuvakamend[{सव्र॑ता रु॒द्राणा॑म॒युतं॑ च॒ पञ्च॑चत्वारिꣳशच्च}]}%॥11॥

%4.4.12.1
स॒मिद्दि॒शामा॒शया॑ नः सुव॒र्विन्मधो॒रतो॒ माध॑वः पात्व॒स्मान्। अ॒ग्निर्दे॒वो दु॒ष्टरी॑तु॒रदा᳚भ्य इ॒दं क्ष॒त्रꣳ र॑क्षतु॒ पात्व॒स्मान्। र॒थं॒त॒रꣳ साम॑भिः पात्व॒स्मान्गा॑य॒त्री छन्द॑सां वि॒श्वरू॑पा। त्रि॒वृन्नो॑ वि॒ष्ठया॒ स्तोमो॒ अह्नाꣳ॑ समु॒द्रो वात॑ इ॒दमोजः॑ पिपर्तु। उ॒ग्रा दि॒शाम॒भिभू॑तिर्वयो॒धाः शुचिः॑ शु॒क्रे अह॑न्योज॒सीना᳚। इन्द्राधि॑पतिः पिपृता॒दतो॑ नो॒ महि॑॥३३॥

%4.4.12.2
क्ष॒त्रं वि॒श्वतो॑ धारये॒दम्। बृ॒हथ्साम॑ क्षत्र॒भृद्वृ॒द्धवृ॑ष्णियं त्रि॒ष्टुभौजः॑ शुभि॒तमु॒ग्रवी॑रम्। इन्द्र॒ स्तोमे॑न पञ्चद॒शेन॒ मध्य॑मि॒दं वाते॑न॒ सग॑रेण रक्ष। प्राची॑ दि॒शाꣳ स॒हय॑शा॒ यश॑स्वती॒ विश्वे॑ देवाः प्रा॒वृषाह्ना॒ꣳ॒ सुव॑र्वती। इ॒दं क्ष॒त्रं दु॒ष्टर॑म॒स्त्वोजो\-ऽना॑धृष्टꣳ सह॒स्रिय॒ꣳ॒ सह॑स्वत्। वै॒रू॒पे साम॑न्नि॒ह तच्छ॑केम॒ जग॑त्यैनं वि॒क्ष्वा वे॑शयामः। विश्वे॑ देवाः सप्तद॒शेन॑॥३४॥

%4.4.12.3
वर्च॑ इ॒दं क्ष॒त्रꣳ स॑लि॒लवा॑तमु॒ग्रम्। ध॒र्त्री दि॒शां क्ष॒त्रमि॒दं दा॑धारोप॒स्थाशा॑नाम्मि॒त्रव॑द॒स्त्वोजः॑। मित्रा॑वरुणा श॒रदाह्नां᳚ चिकित्नू अ॒स्मै रा॒ष्ट्राय॒ महि॒ शर्म॑ यच्छतम्। वै॒रा॒जे साम॒न्नधि॑ मे मनी॒षानु॒ष्टुभा॒ सम्भृ॑तं वी॒र्यꣳ॑ सहः॑। इ॒दं क्ष॒त्रम्मि॒त्रव॑दा॒र्द्रदा॑नु॒ मित्रा॑वरुणा॒ रक्ष॑त॒माधि॑पत्यैः। स॒म्राड्दि॒शाꣳ स॒हसा᳚म्नी॒ सह॑स्वत्यृ॒तुर्\mbox{}हे॑म॒न्तो वि॒ष्ठया॑ नः पिपर्तु। अ॒व॒स्युवा॑ताः॥३५॥

%4.4.12.4
बृ॒ह॒तीर्नु शक्व॑रीरि॒मं य॒ज्ञम॑वन्तु नो घृ॒ताचीः᳚। सुव॑र्वती सु॒दुघा॑ नः॒ पय॑स्वती दि॒शां दे॒व्य॑वतु नो घृ॒ताची᳚। त्वं गो॒पाः पु॑रए॒तोत प॒श्चाद्बृह॑स्पते॒ याम्यां᳚ युङ्ग्धि॒ वाचम्᳚। ऊ॒र्ध्वा दि॒शाꣳ रन्ति॒राशौष॑धीनाꣳ संवथ्स॒रेण॑ सवि॒ता नो॒ अह्ना᳚म्। रे॒वथ्सामाति॑च्छन्दा उ॒ छन्दोजा॑तशत्रुः स्यो॒ना नो॑ अस्तु। स्तोम॑त्रयस्त्रिꣳशे॒ भुव॑नस्य पत्नि॒ विव॑स्वद्वाते अ॒भि नः॑॥३६॥

%4.4.12.5
गृ॒णा॒हि॒। घृ॒तव॑ती सवित॒राधि॑पत्यैः॒ पय॑स्वती॒ रन्ति॒राशा॑ नो अस्तु। ध्रु॒वा दि॒शां वि॑ष्णुप॒त्न्यघो॑रा॒स्येशा॑ना॒ सह॑सो॒ या म॒नोता᳚। बृह॒स्पति॑र्मात॒रिश्वो॒त वा॒युः सं॑धुवा॒ना वाता॑ अ॒भि नो॑ गृणन्तु। वि॒ष्ट॒म्भो दि॒वो ध॒रुणः॑ पृथि॒व्या अ॒स्येशा॑ना॒ जग॑तो॒ विष्णु॑पत्नी। वि॒श्वव्य॑चा इ॒षय॑न्ती॒ सुभू॑तिः शि॒वा नो॑ अ॒स्त्वदि॑तिरु॒पस्थे᳚। वै॒श्वा॒न॒रो न॑ ऊ॒त्या पृ॒ष्टो दि॒व्यनु॑ नो॒\-ऽद्यानु॑मति॒रन्विद॑नुमते॒ त्वङ्कया॑ नश्चि॒त्र आ भु॑व॒त्को अ॒द्य यु॑ङ्क्ते॥३७॥

%4.5.0.0
{\anuvakamend[{महि॑ सप्तद॒शेना॑व॒स्युवा॑ता अ॒भि नो\-ऽनु॑ न॒श्चतु॑र्दश च}]}%॥12॥

%4.5.0.0

{\anuvakamend[{नम॑स्ते रुद्र॒ नमो॒ हिर॑ण्यबाहवे॒ नमः॒ सह॑मानाय॒ नम॑ आव्या॒धिनी᳚भ्यो॒ नमो॑ भ॒वाय॒ नमो᳚ ज्ये॒ष्ठाय॒ नमो॑ दुन्दु॒भ्या॑य॒ नमः॒ सोमा॑य॒ नम॑ इरि॒ण्या॑य॒ द्रापे॑ स॒हस्रा॒ण्येका॑दश}]}%॥11॥
\prashnaend{ नम॑स्ते रुद्र॒ नमो॑ भ॒वाय॒ द्रापे॑ स॒प्तविꣳ॑शतिः॥27॥ नम॑स्ते रुद्र॒ तं वो॒ जम्भे॑ दधामि॥}
%%% END PRASHNA

\sect{पञ्चमः प्रश्नः}\setcounter{anuvakam}{0}
\dnsub{तैत्तिरीयसंहितायां चतुर्थकाण्डे पञ्चमः प्रश्नः}
%4.5.1.0
%4.5.1.1
नम॑स्ते रुद्र म॒न्यव॑ उ॒तो त॒ इष॑वे॒ नमः॑। नम॑स्ते अस्तु॒ धन्व॑ने बा॒हुभ्या॑मु॒त ते॒ नमः॑। या त॒ इषुः॑ शि॒वत॑मा शि॒वम्ब॒भूव॑ ते॒ धनुः॑। शि॒वा श॑र॒व्या॑ या तव॒ तया॑ नो रुद्र मृडय। या ते॑ रुद्र शि॒वा त॒नूरघो॒रापा॑पकाशिनी। तया॑ नस्त॒नुवा॒ शन्त॑मया॒ गिरि॑शन्ता॒भि चा॑कशीहि। यामिषुं॑ गिरिशन्त॒ हस्ते᳚॥१॥

%4.5.1.2
बिभ॒र्ष्यस्त॑वे। शि॒वां गि॑रित्र॒ तां कु॑रु॒ मा हिꣳ॑सीः॒ पुरु॑षं॒ जग॑त्। शि॒वेन॒ वच॑सा त्वा॒ गिरि॒शाच्छा॑ वदामसि। यथा॑ नः॒ सर्व॒मिज्जग॑दय॒क्ष्मꣳ सु॒मना॒ अस॑त्। अध्य॑वोचदधिव॒क्ता प्र॑थ॒मो दैव्यो॑ भि॒षक्। अहीꣳ॑श्च॒ सर्वा᳚ञ्ज॒म्भय॒न्थ्सर्वा᳚श्च यातुधा॒न्यः॑। अ॒सौ यस्ता॒म्रो अ॑रु॒ण उ॒त ब॒भ्रुः सु॑म॒ङ्गलः॑। ये चे॒माꣳ रु॒द्रा अ॒भितो॑ दि॒क्षु॥२॥

%4.5.1.3
श्रि॒ताः स॑हस्र॒शो\-ऽवै॑षा॒ꣳ॒ हेड॑ ईमहे। अ॒सौ यो॑\-ऽव॒सर्प॑ति॒ नील॑ग्रीवो॒ विलो॑हितः। उ॒तैनं॑ गो॒पा अ॑दृश॒न्नदृ॑शन्नुदहा॒र्यः॑। उ॒तैनं॒ विश्वा॑ भू॒तानि॒ स दृ॒ष्टो मृ॑डयाति नः। नमो॑ अस्तु॒ नील॑ग्रीवाय सहस्रा॒क्षाय॑ मी॒ढुषे᳚। अथो॒ ये अ॑स्य॒ सत्वा॑नो॒\-ऽहं तेभ्यो॑\-ऽकरं॒ नमः॑। प्र मु॑ञ्च॒ धन्व॑न॒स्त्वमु॒भयो॒रार्त्नि॑यो॒र्ज्याम्। याश्च॑ ते॒ हस्त॒ इष॑वः॥३॥

%4.5.1.4
परा॒ ता भ॑गवो वप। अ॒व॒तत्य॒ धनु॒स्त्वꣳ सह॑स्राक्ष॒ शते॑षुधे। नि॒शीर्य॑ श॒ल्याना॒म्मुखा॑ शि॒वो नः॑ सु॒मना॑ भव। विज्यं॒ धनुः॑ कप॒र्दिनो॒ विश॑ल्यो॒ बाण॑वाꣳ उ॒त। अने॑शन्न॒स्येष॑व आ॒भुर॑स्य निष॒ङ्गथिः॑। या ते॑ हे॒तिर्मी॑ढुष्टम॒ हस्ते॑ ब॒भूव॑ ते॒ धनुः॑। तया॒स्मान् वि॒श्वत॒स्त्वम॑यक्ष॒मया॒ परि॑ ब्भुज। नम॑स्ते अ॒स्त्वायु॑धा॒याना॑तताय धृ॒ष्णवे᳚। उ॒भाभ्या॑मु॒त ते॒ नमो॑ बा॒हुभ्यां॒ तव॒ धन्व॑ने। परि॑ ते॒ धन्व॑नो हे॒तिर॒स्मान्वृ॑णक्तु वि॒श्वतः॑। अथो॒ य इ॑षु॒धिस्तवा॒रे अ॒स्मन्नि धे॑हि॒ तम्॥४॥

%4.5.2.0
{\anuvakamend[{हस्ते॑ दि॒क्ष्विष॑व उ॒भाभ्यां॒ द्वाविꣳ॑शतिश्च}]}%॥१॥

%4.5.2.1
नमो॒ हिर॑ण्यबाहवे सेना॒न्ये॑ दि॒शां च॒ पत॑ये॒ नमो॒ नमो॑ वृ॒क्षेभ्यो॒ हरि॑केशेभ्यः पशू॒नाम्पत॑ये॒ नमो॒ नमः॑ स॒स्पिञ्ज॑राय॒ त्विषी॑मते पथी॒नाम्पत॑ये॒ नमो॒ नमो॑ बभ्लु॒शाय॑ विव्या॒धिने\-ऽन्ना॑ना॒म्पत॑ये॒ नमो॒ नमो॒ हरि॑केशायोपवी॒तिने॑ पु॒ष्टाना॒म्पत॑ये॒ नमो॒ नमो॑ भ॒वस्य॑ हे॒त्यै जग॑ता॒म्पत॑ये॒ नमो॒ नमो॑ रु॒द्राया॑तता॒विने॒ क्षेत्रा॑णा॒म्पत॑ये॒ नमो॒ नमः॑ सू॒तायाह॑न्त्याय॒ वना॑ना॒म्पत॑ये॒ नमो॒ नमः॑॥५॥

%4.5.2.2
रोहि॑ताय स्थ॒पत॑ये वृ॒क्षाणा॒म्पत॑ये॒ नमो॒ नमो॑ म॒न्त्रिणे॑ वाणि॒जाय॒ कक्षा॑णा॒म्पत॑ये॒ नमो॒ नमो॑ भुव॒न्तये॑ वारिवस्कृ॒तायौष॑धीना॒म्पत॑ये॒ नमो॒ नम॑ उ॒च्चैर्घो॑षायाक्र॒न्दय॑ते पत्ती॒नाम्पत॑ये॒ नमो॒ नमः॑ कृथ्स्नवी॒ताय॒ धाव॑ते॒ सत्व॑ना॒म्पत॑ये॒ नमः॑॥६॥

%4.5.3.0
{\anuvakamend[{वना॑ना॒म्पत॑ये॒ नमो॒ नम॒ एका॒न्नत्रि॒ꣳ॒शच्च॑}]}%॥२॥

%4.5.3.1
नमः॒ सह॑मानाय निव्या॒धिन॑ आव्या॒धिनी॑ना॒म्पत॑ये॒ नमो॒ नमः॑ ककु॒भाय॑ निष॒ङ्गिणे᳚ स्ते॒नाना॒म्पत॑ये॒ नमो॒ नमो॑ निष॒ङ्गिण॑ इषुधि॒मते॒ तस्क॑राणा॒म्पत॑ये॒ नमो॒ नमो॒ वञ्च॑ते परि॒वञ्च॑ते स्तायू॒नाम्पत॑ये॒ नमो॒ नमो॑ निचे॒रवे॑ परिच॒रायार॑ण्याना॒म्पत॑ये॒ नमो॒ नमः॑ सृका॒विभ्यो॒ जिघाꣳ॑सद्भ्यो मुष्ण॒ताम्पत॑ये॒ नमो॒ नमो॑\-ऽसि॒मद्भ्यो॒ नक्तं॒ चर॑द्भ्यः प्रकृ॒न्ताना॒म्पत॑ये॒ नमो॒ नम॑ उष्णी॒षिणे॑ गिरिच॒राय॑ कुलु॒ञ्चाना॒म्पत॑ये॒ नमो॒ नमः॑॥७॥

%4.5.3.2
इषु॑मद्भ्यो धन्वा॒विभ्य॑श्च वो॒ नमो॒ नम॑ आतन्वा॒नेभ्यः॑ प्रति॒दधा॑नेभ्यश्च वो॒ नमो॒ नम॑ आ॒यच्छ॑द्भ्यो विसृ॒जद्भ्य॑श्च वो॒ नमो॒ नमो\-ऽस्य॑द्भ्यो॒ विध्य॑द्भ्यश्च वो॒ नमो॒ नम॒ आसी॑नेभ्यः॒ शया॑नेभ्यश्च वो॒ नमो॒ नमः॑ स्व॒पद्भ्यो॒ जाग्र॑द्भ्यश्च वो॒ नमो॒ नम॒स्तिष्ठ॑द्भ्यो॒ धाव॑द्भ्यश्च वो॒ नमो॒ नमः॑ स॒भाभ्यः॑ स॒भाप॑तिभ्यश्च वो॒ नमो॒ नमो॒ अश्वे॒भ्यो\-ऽश्व॑पतिभ्यश्च वो॒ नमः॑॥८॥

%4.5.4.0
{\anuvakamend[{कु॒लु॒ञ्चाना॒म्पत॑ये॒ नमो॒ नमो\-ऽश्व॑पतिभ्य॒स्त्रीणि॑ च}]}%॥३॥

%4.5.4.1
नम॑ आव्या॒धिनी᳚भ्यो वि॒विध्य॑न्तीभ्यश्च वो॒ नमो॒ नम॒ उग॑णाभ्यस्तृꣳह॒तीभ्य॑श्च वो॒ नमो॒ नमो॑ गृ॒थ्सेभ्यो॑ गृ॒थ्सप॑तिभ्यश्च वो॒ नमो॒ नमो॒ व्राते᳚भ्यो॒ व्रात॑पतिभ्यश्च वो॒ नमो॒ नमो॑ ग॒णेभ्यो॑ ग॒णप॑तिभ्यश्च वो॒ नमो॒ नमो॒ विरू॑पेभ्यो वि॒श्वरू॑पेभ्यश्च वो॒ नमो॒ नमो॑ म॒हद्भ्यः॑ क्षुल्ल॒केभ्य॑श्च वो॒ नमो॒ नमो॑ र॒थिभ्यो॑\-ऽर॒थेभ्य॑श्च वो॒ नमो॒ नमो॒ रथे᳚भ्यः॥९॥

%4.5.4.2
रथ॑पतिभ्यश्च वो॒ नमो॒ नमः॒ सेना᳚भ्यः सेना॒निभ्य॑श्च वो॒ नमो॒ नमः॑ क्ष॒त्तृभ्यः॑ सङ्ग्रही॒तृभ्य॑श्च वो॒ नमो॒ नम॒स्तक्ष॑भ्यो रथका॒रेभ्य॑श्च वो॒ नमो॒ नमः॒ कुला॑लेभ्यः क॒र्मारे᳚भ्यश्च वो॒ नमो॒ नमः॑ पु॒ञ्जिष्टे᳚भ्यो निषा॒देभ्य॑श्च वो॒ नमो॒ नम॑ इषु॒कृद्भ्यो॑ धन्व॒कृद्भ्य॑श्च वो॒ नमो॒ नमो॑ मृग॒युभ्यः॑ श्व॒निभ्य॑श्च वो॒ नमो॒ नमः॒ श्वभ्यः॒ श्वप॑तिभ्यश्च वो॒ नमः॑॥१०॥

%4.5.5.0
{\anuvakamend[{रथे᳚भ्यः॒ श्वप॑तिभ्यश्च॒ द्वे च॑}]}%॥४॥

%4.5.5.1
नमो॑ भ॒वाय॑ च रु॒द्राय॑ च॒ नमः॑ श॒र्वाय॑ च पशु॒पत॑ये च॒ नमो॒ नील॑ग्रीवाय च शिति॒कण्ठा॑य च॒ नमः॑ कप॒र्दिने॑ च॒ व्यु॑प्तकेशाय च॒ नमः॑ सहस्रा॒क्षाय॑ च श॒तध॑न्वने च॒ नमो॑ गिरि॒शाय॑ च शिपिवि॒ष्टाय॑ च॒ नमो॑ मी॒ढुष्ट॑माय॒ चेषु॑मते च॒ नमो᳚ ह्र॒स्वाय॑ च वाम॒नाय॑ च॒ नमो॑ बृह॒ते च॒ वर्\mbox{}षी॑यसे च॒ नमो॑ वृ॒द्धाय॑ च सं॒वृध्व॑ने च॥११॥

%4.5.5.2
नमो॒ अग्रि॑याय च प्रथ॒माय॑ च॒ नम॑ आ॒शवे॑ चाजि॒राय॑ च॒ नमः॒ शीघ्रि॑याय च॒ शीभ्या॑य च॒ नम॑ ऊ॒र्म्या॑य चावस्व॒न्या॑य च॒ नमः॑ स्रोत॒स्या॑य च॒ द्वीप्या॑य च॥१२॥

%4.5.6.0
{\anuvakamend[{सं॒ वृध्व॑ने च॒ पञ्च॑विꣳशतिश्च}]}%॥५॥

%4.5.6.1
नमो᳚ ज्ये॒ष्ठाय॑ च कनि॒ष्ठाय॑ च॒ नमः॑ पूर्व॒जाय॑ चापर॒जाय॑ च॒ नमो॑ मध्य॒माय॑ चापग॒ल्भाय॑ च॒ नमो॑ जघ॒न्या॑य च॒ बुध्नि॑याय च॒ नमः॑ सो॒भ्या॑य च प्रतिस॒र्या॑य च॒ नमो॒ याम्या॑य च॒ क्षेम्या॑य च॒ नम॑ उर्व॒र्या॑य च॒ खल्या॑य च॒ नमः॒ श्लोक्या॑य चावसा॒न्या॑य च॒ नमो॒ वन्या॑य च॒ कक्ष्या॑य च॒ नमः॑ श्र॒वाय॑ च प्रतिश्र॒वाय॑ च॥१३॥

%4.5.6.2
नम॑ आ॒शुषे॑णाय चा॒शुर॑थाय च॒ नमः॒ शूरा॑य चावभिन्द॒ते च॒ नमो॑ व॒र्मिणे॑ च वरू॒थिने॑ च॒ नमो॑ बि॒ल्मिने॑ च कव॒चिने॑ च॒ नमः॑ श्रु॒ताय॑ च श्रुतसे॒नाय॑ च॥१४॥

%4.5.7.0
{\anuvakamend[{प्र॒ति॒श्र॒वाय॑ च॒ पञ्च॑विꣳशतिश्च}]}%॥६॥

%4.5.7.1
नमो॑ दुन्दु॒भ्या॑य चाहन॒न्या॑य च॒ नमो॑ धृ॒ष्णवे॑ च प्रमृ॒शाय॑ च॒ नमो॑ दू॒ताय॑ च॒ प्रहि॑ताय च॒ नमो॑ निष॒ङ्गिणे॑ चेषुधि॒मते॑ च॒ नम॑स्ती॒क्ष्णेष॑वे चायु॒धिने॑ च॒ नमः॑ स्वायु॒धाय॑ च सु॒धन्व॑ने च॒ नमः॒ स्रुत्या॑य च॒ पथ्या॑य च॒ नमः॑ का॒ट्या॑य च नी॒प्या॑य च॒ नमः॒ सूद्या॑य च सर॒स्या॑य च॒ नमो॑ ना॒द्याय॑ च वैश॒न्ताय॑ च॥१५॥

%4.5.7.2
नमः॒ कूप्या॑य चाव॒ट्या॑य च॒ नमो॒ वर्ष्या॑य चाव॒र्ष्याय॑ च॒ नमो॑ मे॒घ्या॑य च विद्यु॒त्या॑य च॒ नम॑ ई॒ध्रिया॑य चात॒प्या॑य च॒ नमो॒ वात्या॑य च॒ रेष्मि॑याय च॒ नमो॑ वास्त॒व्या॑य च वास्तु॒पाय॑ च॥१६॥

%4.5.8.0
{\anuvakamend[{वै॒श॒न्ताय॑ च त्रि॒ꣳ॒शच्च॑}]}%॥७॥

%4.5.8.1
नमः॒ सोमा॑य च रु॒द्राय॑ च॒ नम॑स्ता॒म्राय॑ चारु॒णाय॑ च॒ नमः॑ शं॒गाय॑ च पशु॒पत॑ये च॒ नम॑ उ॒ग्राय॑ च भी॒माय॑ च॒ नमो॑ अग्रेव॒धाय॑ च दूरेव॒धाय॑ च॒ नमो॑ ह॒न्त्रे च॒ हनी॑यसे च॒ नमो॑ वृ॒क्षेभ्यो॒ हरि॑केशेभ्यो॒ नम॑स्ता॒राय॒ नमः॑ श॒म्भवे॑ च मयो॒भवे॑ च॒ नमः॑ शङ्क॒राय॑ च मयस्क॒राय॑ च॒ नमः॑ शि॒वाय॑ च शि॒वत॑राय च॥१७॥

%4.5.8.2
नम॒स्तीर्थ्या॑य च॒ कूल्या॑य च॒ नमः॑ पा॒र्या॑य चावा॒र्या॑य च॒ नमः॑ प्र॒तर॑णाय चो॒त्तर॑णाय च॒ नम॑ आता॒र्या॑य चाला॒द्या॑य च॒ नमः॒ शष्प्या॑य च॒ फेन्या॑य च॒ नमः॑ सिक॒त्या॑य च प्रवा॒ह्या॑य च॥१८॥

%4.5.9.0
{\anuvakamend[{शि॒वत॑राय च त्रि॒ꣳ॒शच्च॑}]}%॥८॥

%4.5.9.1
नम॑ इरि॒ण्या॑य च प्रप॒थ्या॑य च॒ नमः॑ किꣳशि॒लाय॑ च॒ क्षय॑णाय च॒ नमः॑ कप॒र्दिने॑ च पुल॒स्तये॑ च॒ नमो॒ गोष्ठ्या॑य च॒ गृह्या॑य च॒ नम॒स्तल्प्या॑य च॒ गेह्या॑य च॒ नमः॑ का॒ट्या॑य च गह्वरे॒ष्ठाय॑ च॒ नमो᳚ ह्रद॒य्या॑य च निवे॒ष्प्या॑य च॒ नमः॑ पाꣳस॒व्या॑य च रज॒स्या॑य च॒ नमः॒ शुष्क्या॑य च हरि॒त्या॑य च॒ नमो॒ लोप्या॑य चोल॒प्या॑य च॥१९॥

%4.5.9.2
नम॑ ऊ॒र्व्या॑य च सू॒र्म्या॑य च॒ नमः॑ प॒र्ण्या॑य च पर्णश॒द्या॑य च॒ नमो॑\-ऽपगु॒रमा॑णाय चाभिघ्न॒ते च॒ नम॑ आक्खिद॒ते च॑ प्रक्खिद॒ते च॒ नमो॑ वः किरि॒केभ्यो॑ दे॒वाना॒ꣳ॒ हृद॑येभ्यो॒ नमो॑ विक्षीण॒केभ्यो॒ नमो॑ विचिन्व॒त्केभ्यो॒ नम॑ आनिर्\mbox{}ह॒तेभ्यो॒ नम॑ आमीव॒त्केभ्यः॑॥२०॥

%4.5.10.0
{\anuvakamend[{उ॒ल॒प्या॑य च॒ त्रय॑स्त्रिꣳशच्च}]}%॥९॥

%4.5.10.1
द्रापे॒ अन्ध॑सस्पते॒ दरि॑द्र॒न्नील॑लोहित। ए॒षां पुरु॑षाणामे॒षाम्प॑शू॒नां मा भेर्मारो॒ मो ए॑षां॒ किं च॒नाम॑मत्। या ते॑ रुद्र शि॒वा त॒नूः शि॒वा वि॒श्वाह॑भेषजी। शि॒वा रु॒द्रस्य॑ भेष॒जी तया॑ नो मृड जी॒वसे᳚। इ॒माꣳ रु॒द्राय॑ त॒वसे॑ कप॒र्दिने᳚ क्ष॒यद्वी॑राय॒ प्र भ॑रामहे म॒तिम्। यथा॑ नः॒ शमस॑द्द्वि॒पदे॒ चतु॑ष्पदे॒ विश्व॑म्पु॒ष्टम्ग्रा॒मे॑ अ॒स्मिन्न्॥२१॥

%4.5.10.2
अना॑तुरम्। मृ॒डा नो॑ रु॒द्रोत नो॒ मय॑स्कृधि क्ष॒यद्वी॑राय॒ नम॑सा विधेम ते। यच्छं च॒ योश्च॒ मनु॑राय॒जे पि॒ता तद॑श्याम॒ तव॑ रुद्र॒ प्रणी॑तौ। मा नो॑ म॒हान्त॑मु॒त मा नो॑ अर्भ॒कं मा न॒ उक्ष॑न्तमु॒त मा न॑ उक्षि॒तम्। मा नो॑ वधीः पि॒तर॒म्मोत मा॒तर॑म्प्रि॒या मा न॑स्त॒नुवः॑॥२२॥

%4.5.10.3
रु॒द्र॒ री॒रि॒षः॒। मा न॑स्तो॒के तन॑ये॒ मा न॒ आयु॑षि॒ मा नो॒ गोषु॒ मा नो॒ अश्वे॑षु रीरिषः। वी॒रान्मा नो॑ रुद्र भामि॒तो व॑धीर्\mbox{}ह॒विष्म॑न्तो॒ नम॑सा विधेम ते। आ॒रात्ते॑ गो॒घ्न उ॒त पू॑रुष॒घ्ने क्ष॒यद्वी॑राय सु॒म्नम॒स्मे ते॑ अस्तु। रक्षा॑ च नो॒ अधि॑ च देव ब्रू॒ह्यधा॑ च नः॒ शर्म॑ यच्छ द्वि॒बर्\mbox{}हाः᳚। स्तु॒हि॥२३॥

%4.5.10.4
श्रु॒तं ग॑र्त॒सदं॒ युवा॑नम्मृ॒गं न भी॒ममु॑पह॒त्नुमु॒ग्रम्। मृ॒डा ज॑रि॒त्रे रु॑द्र॒ स्तवा॑नो अ॒न्यं ते॑ अ॒स्मन्नि व॑पन्तु॒ सेनाः᳚। परि॑ णो रु॒द्रस्य॑ हे॒तिर्वृ॑णक्तु॒ परि॑ त्वे॒षस्य॑ दुर्म॒तिर॑घा॒योः। अव॑ स्थि॒रा म॒घव॑द्भ्यस्तनुष्व॒ मीढ्व॑स्तो॒काय॒ तन॑याय मृडय। मीढु॑ष्टम॒ शिव॑तम शि॒वो नः॑ सु॒मना॑ भव। प॒र॒मे वृ॒क्ष आयु॑धं नि॒धाय॒ कृत्तिं॒ वसा॑न॒ आ च॑र॒ पिना॑कम्॥२४॥

%4.5.10.5
बिभ्र॒दा ग॑हि। विकि॑रिद॒ विलो॑हित॒ नम॑स्ते अस्तु भगवः। यास्ते॑ स॒हस्रꣳ॑ हे॒तयो॒\-ऽन्यम॒स्मन्नि व॑पन्तु॒ ताः। स॒हस्रा॑णि सहस्र॒धा बा॑हु॒वोस्तव॑ हे॒तयः॑। तासा॒मीशा॑नो भगवः परा॒चीना॒ मुखा॑ कृधि॥२५॥

%4.5.11.0
{\anuvakamend[{अ॒स्मिꣴ स्त॒नुवः॑ स्तु॒हि पिना॑क॒मेका॒न्नत्रि॒ꣳ॒शच्च॑}]}%॥10॥

%4.5.11.1
स॒हस्रा॑णि सहस्र॒शो ये रु॒द्रा अधि॒ भूम्या᳚म्। तेषाꣳ॑ सहस्रयोज॒ने\-ऽव॒ धन्वा॑नि तन्मसि। अ॒स्मिन्म॑ह॒त्य॑र्ण॒वे᳚\-ऽ\-न्तरि॑क्षे भ॒वा अधि॑। नील॑ग्रीवाः शिति॒कण्ठाः᳚ श॒र्वा अ॒धः क्ष॑माच॒राः। नील॑ग्रीवाः शिति॒कण्ठा॒ दिवꣳ॑ रु॒द्रा उप॑श्रिताः। ये वृ॒क्षेषु॑ स॒स्पिञ्ज॑रा॒ नील॑ग्रीवा॒ विलो॑हिताः। ये भू॒ताना॒मधि॑पतयो विशि॒खासः॑ कप॒र्दिनः॑। ये अन्ने॑षु वि॒विध्य॑न्ति॒ पात्रे॑षु॒ पिब॑तो॒ जनान्॑। ये प॒थाम्प॑थि॒रक्ष॑य ऐलबृ॒दा य॒व्युधः॑। ये ती॒र्थानि॑॥२६॥

%4.5.11.2
प्र॒चर॑न्ति सृ॒काव॑न्तो निष॒ङ्गिणः॑। य ए॒ताव॑न्तश्च॒ भूयाꣳ॑सश्च॒ दिशो॑ रु॒द्रा वि॑तस्थि॒रे। तेषाꣳ॑ सहस्रयोज॒ने\-ऽव॒ धन्वा॑नि तन्मसि। नमो॑ रु॒द्रेभ्यो॒ ये पृ॑थि॒व्यां ये᳚\-ऽन्तरि॑क्षे॒ ये दि॒वि येषा॒मन्नं॒ वातो॑ व॒र्\mbox{}षमिष॑व॒स्तेभ्यो॒ दश॒ प्राची॒र्दश॑ दक्षि॒णा दश॑ प्र॒तीची॒र्दशोदी॑ची॒र्दशो॒र्ध्वास्तेभ्यो॒ नम॒स्ते नो॑ मृडयन्तु॒ ते यं द्वि॒ष्मो यश्च॑ नो॒ द्वेष्टि॒ तं वो॒ जम्भे॑ दधामि॥२७॥

%4.6.0.0

%4.6.0.0
{\anuvakamend[{ती॒र्थानि॒ यश्च॒ षट्च॑}]}%॥11॥

%4.6.0.0
{\prashnaend[{अश्म॒न् य इ॒मोदे॑नमा॒शुः प्राचीं᳚ जी॒मूत॑स्य॒ यदक्र॑न्दो॒ मा नो॑ मि॒त्रो ये वा॒जिनं॒ नव॑॥९॥ अश्म॑न्मनो॒युजं॒ प्राची॒मनु॒ शर्म॑ यच्छतु॒ तेषा॑म॒भिगू᳚र्तिः॒ षट्च॑त्वारिꣳशत्। अश्म॑न् ह॒विष्मान्॑॥}]}

%%% END PRASHNA

\sect{षष्ठमः प्रश्नः}\setcounter{anuvakam}{0}
\dnsub{तैत्तिरीयसंहितायां चतुर्थकाण्डे षष्ठमः प्रश्नः}
%4.6.1.0
%4.6.1.1
अश्म॒न्नूर्जं॒ पर्व॑ते शिश्रिया॒णां वाते॑ प॒र्जन्ये॒ वरु॑णस्य॒ शुष्मे᳚। अ॒द्भ्य ओष॑धीभ्यो॒ वन॒स्पति॒भ्यो\-ऽधि॒ सम्भृ॑तां॒ तां न॒ इष॒मूर्जं॑ धत्त मरुतः सꣳररा॒णाः। अश्मꣴ॑स्ते॒ क्षुद॒मुं ते॒ शुगृ॑च्छतु॒ यं द्वि॒ष्मः। स॒मु॒द्रस्य॑ त्वा॒\-ऽवाक॒याग्ने॒ परि॑ व्ययामसि। पाव॒को अ॒स्मभ्यꣳ॑ शि॒वो भ॑व। हि॒मस्य॑ त्वा ज॒रायु॒णाग्ने॒ परि॑ व्ययामसि। पा॒व॒को अ॒स्मभ्यꣳ॑ शि॒वो भ॑व। उप॑॥१॥

%4.6.1.2
ज्मन्नुप॑ वेत॒से\-ऽव॑त्तरं न॒दीष्वा। अग्ने॑ पि॒त्तम॒पाम॑सि। मण्डू॑कि॒ ताभि॒रा ग॑हि॒ सेमं नो॑ य॒ज्ञम्। पा॒व॒कव॑र्णꣳ शि॒वं कृ॑धि। पा॒व॒क आ चि॒तय॑न्त्या कृ॒पा। क्षाम॑न्रुरु॒च उ॒षसो॒ न भा॒नुना᳚। तूर्व॒न्न याम॒न्नेत॑शस्य॒ नू रण॒ आ यो घृ॒णे। न त॑तृषा॒णो अ॒जरः॑। अग्ने॑ पावक रो॒चिषा॑ म॒न्द्रया॑ देव जि॒ह्वया᳚। आ दे॒वान्॥२॥

%4.6.1.3
व॒क्षि॒ यक्षि॑ च। स नः॑ पावक दीदि॒वो\-ऽग्ने॑ दे॒वाꣳ इ॒हा व॑ह। उप॑ य॒ज्ञꣳ ह॒विश्च॑ नः। अ॒पामि॒दं न्यय॑नꣳ समु॒द्रस्य॑ नि॒वेश॑नम्। अ॒न्यं ते॑ अ॒स्मत्त॑पन्तु हे॒तयः॑ पाव॒को अ॒स्मभ्यꣳ॑ शि॒वो भ॑व। नम॑स्ते॒ हर॑से शो॒चिषे॒ नम॑स्ते अस्त्व॒र्चिषे᳚। अ॒न्यं ते॑ अ॒स्मत्त॑पन्तु हे॒तयः॑ पाव॒को अ॒स्मभ्यꣳ॑ शि॒वो भ॑व। नृ॒षदे॒ वट्॥३॥

%4.6.1.4
अ॒फ्सु॒षदे॒ वड्व॑न॒सदे॒ वड्ब॑र्\mbox{}हि॒षदे॒ वट्थ्सु॑व॒र्विदे॒ वट्। ये दे॒वा दे॒वानां᳚ य॒ज्ञिया॑ य॒ज्ञिया॑नाꣳ संवथ्स॒रीण॒मुप॑ भा॒गमास॑ते। अ॒हु॒तादो॑ ह॒विषो॑ य॒ज्ञे अ॒स्मिन्थ्स्व॒यं जु॑हुध्व॒म्मधु॑नो घृ॒तस्य॑। ये दे॒वा दे॒वेष्वधि॑ देव॒त्वमाय॒न् ये ब्रह्म॑णः पुरए॒तारो॑ अ॒स्य। येभ्यो॒ नर्ते पव॑ते॒ धाम॒ किं च॒न न ते दि॒वो न पृ॑थि॒व्या अधि॒ स्नुषु॑। प्रा॒ण॒दाः॥४॥

%4.6.1.5
अ॒पा॒न॒दा व्या॑न॒दाश्च॑क्षु॒र्दा व॑र्चो॒दा व॑रिवो॒दाः। अ॒न्यं ते॑ अ॒स्मत्त॑पन्तु हे॒तयः॑ पाव॒को अ॒स्मभ्यꣳ॑ शि॒वो भ॑व। अ॒ग्निस्ति॒ग्मेन॑ शो॒चिषा॒ यꣳस॒द्विश्वं॒ न्य॑त्रिणम्᳚। अ॒ग्निर्नो॑ वꣳसते र॒यिम्। सैनानी॑केन सुवि॒दत्रो॑ अ॒स्मे यष्टा॑ दे॒वाꣳ आय॑जिष्ठः स्व॒स्ति। अद॑ब्धो गो॒पा उ॒त नः॑ पर॒स्पा अग्ने᳚ द्यु॒मदु॒त रे॒वद्दि॑दीहि॥५॥

%4.6.2.0
{\anuvakamend[{उप॑ दे॒वान् वट्प्रा॑ण॒दाश्चतु॑श्चत्वारिꣳशच्च}]}%॥१॥

%4.6.2.1
य इ॒मा विश्वा॒ भुव॑नानि॒ जुह्व॒दृषि॒र्\mbox{}होता॑ निष॒सादा॑ पि॒ता नः॑। स आ॒शिषा॒ द्रवि॑णमि॒च्छमा॑नः परम॒च्छदो॒ वर॒ आ वि॑वेश। वि॒श्वक॑र्मा॒ मन॑सा॒ यद्विहा॑या धा॒ता वि॑धा॒ता प॑र॒मोत सं॒दृक्। तेषा॑मि॒ष्टानि॒ समि॒षा म॑दन्ति॒ यत्र॑ सप्त॒र्\mbox{}षीन्प॒र एक॑मा॒हुः। यो नः॑ पि॒ता ज॑नि॒ता यो वि॑धा॒ता यो नः॑ स॒तो अ॒भ्या सज्ज॒जान॑।॥६॥

%4.6.2.2
यो दे॒वानां᳚ नाम॒धा एक॑ ए॒व तꣳ स॑म्प्र॒श्ञम्भुव॑ना यन्त्य॒न्या। त आय॑जन्त॒ द्रवि॑ण॒ꣳ॒ सम॑स्मा॒ ऋष॑यः॒ पूर्वे॑ जरि॒तारो॒ न भू॒ना। अ॒सूर्ता॒ सूर्ता॒ रज॑सो वि॒माने॒ ये भू॒तानि॑ स॒मकृ॑ण्वन्नि॒मानि॑। न तं वि॑दाथ॒ य इ॒दं ज॒जाना॒न्यद्यु॒ष्माक॒मन्त॑रम्भवाति। नी॒हा॒रेण॒ प्रावृ॑ता॒ जल्प्या॑ चासु॒तृप॑ उक्थ॒शास॑श्चरन्ति। प॒रो दि॒वा प॒र ए॒ना॥७॥

%4.6.2.3
पृ॒थि॒व्या प॒रो दे॒वेभि॒रसु॑रै॒र्गुहा॒ यत्। कꣴ स्वि॒द्गर्भं॑ प्रथ॒मं द॑ध्र॒ आपो॒ यत्र॑ दे॒वाः स॒मग॑च्छन्त॒ विश्वे᳚। तमिद्गर्भ॑म्प्रथ॒मं द॑ध्र॒ आपो॒ यत्र॑ दे॒वाः स॒मग॑च्छन्त॒ विश्वे᳚। अ॒जस्य॒ नाभा॒वध्येक॒मर्पि॑तं॒ यस्मि॑न्नि॒दं विश्व॒म्भुव॑न॒\-मधि॑ श्रि॒तम्। वि॒श्वक॑र्मा॒ ह्यज॑निष्ट दे॒व आदिद्ग॑न्ध॒र्वो अ॑भवद्द्वि॒तीयः॑। तृ॒तीयः॑ पि॒ता ज॑नि॒तौष॑धीनाम्॥८॥

%4.6.2.4
अ॒पां गर्भं॒ व्य॑दधात्पुरु॒त्रा। चक्षु॑षः पि॒ता मन॑सा॒ हि धीरो॑ घृ॒तमे॑ने अजन॒न्नन्न॑माने। य॒देदन्ता॒ अद॑दृꣳहन्त॒ पूर्व॒ आदिद्द्यावा॑पृथि॒वी अ॑प्रथेताम्। वि॒श्वत॑श्चक्षुरु॒त वि॒श्वतो॑मुखो वि॒श्वतो॑हस्त उ॒त वि॒श्वत॑स्पात्। सं बा॒हुभ्यां॒ नम॑ति॒ सम्पत॑त्रै॒र्द्यावा॑पृथि॒वी ज॒नयं॑ दे॒व एकः॑। किꣴ स्वि॑दासीदधि॒ष्ठान॑मा॒रम्भ॑णं कत॒मथ्स्वि॒त्किमा॑सीत्। यदी॒ भूमिं॑ ज॒नयन्न्॑॥९॥

%4.6.2.5
वि॒श्वक॑र्मा॒ वि द्यामौर्णो᳚न्महि॒ना वि॒श्वच॑क्षाः। किꣴ स्वि॒द्वनं॒ क उ॒ स वृ॒क्ष आ॑सी॒द्यतो॒ द्यावा॑पृथि॒वी नि॑ष्टत॒क्षुः। मनी॑षिणो॒ मन॑सा पृ॒च्छतेदु॒ तद्यद॒ध्यति॑ष्ठ॒द्भुव॑नानि धा॒रयन्न्॑। या ते॒ धामा॑नि पर॒माणि॒ याव॒मा या म॑ध्य॒मा वि॑श्वकर्मन्नु॒तेमा। शिक्षा॒ सखि॑भ्यो ह॒विषि॑ स्वधावः स्व॒यं य॑जस्व त॒नुवं॑ जुषा॒णः। वा॒चस्पतिं॑ वि॒श्वक॑र्माणमू॒तये᳚॥१०॥

%4.6.2.6
म॒नो॒युजं॒ वाजे॑ अ॒द्या हु॑वेम। स नो॒ नेदि॑ष्ठा॒ हव॑नानि जोषते वि॒श्वश॑म्भू॒रव॑से सा॒धुक॑र्मा। विश्व॑कर्मन् ह॒विषा॑ वावृधा॒नः स्व॒यं य॑जस्व त॒नुवं॑ जुषा॒णः। मुह्य॑न्त्व॒न्ये अ॒भितः॑ स॒पत्ना॑ इ॒हास्माक॑म्म॒घवा॑ सू॒रिर॑स्तु। विश्व॑कर्मन् ह॒विषा॒ वर्ध॑नेन त्रा॒तार॒मिन्द्र॑मकृणोरव॒ध्यम्। तस्मै॒ विशः॒ सम॑नमन्त पू॒र्वीर॒यमु॒ग्रो वि॑ह॒व्यो॑ यथास॑त्। स॒मु॒द्राय॑ व॒युना॑य॒ सिन्धू॑ना॒म्पत॑ये॒ नमः॑। न॒दीना॒ꣳ॒ सर्वा॑साम्पि॒त्रे जु॑हु॒ता वि॒श्वक॑र्मणे॒ विश्वाहाम॑र्त्यꣳ ह॒विः॥११॥

%4.6.3.0
{\anuvakamend[{ज॒जानै॒नौष॑धीनां॒ भूमिं॑ ज॒नय॑न्नू॒तये॒ नमो॒ नव॑ च}]}%॥२॥

%4.6.3.1
उदे॑नमुत्त॒रां न॒याग्ने॑ घृतेनाहुत। रा॒यस्पोषे॑ण॒ सꣳ सृ॑ज प्र॒जया॑ च॒ धने॑न च। इन्द्रे॒मम्प्र॑त॒रां कृ॑धि सजा॒ताना॑मसद्व॒शी। समे॑नं॒ वर्च॑सा सृज दे॒वेभ्यो॑ भाग॒धा अ॑सत्। यस्य॑ कु॒र्मो ह॒विर्गृ॒हे तम॑ग्ने वर्धया॒ त्वम्। तस्मै॑ दे॒वा अधि॑ ब्रवन्न॒यं च॒ ब्रह्म॑ण॒स्पतिः॑। उदु॑ त्वा॒ विश्वे॑ दे॒वाः॥१२॥

%4.6.3.2
अग्ने॒ भर॑न्तु॒ चित्ति॑भिः। स नो॑ भव शि॒वत॑मः सु॒प्रती॑को वि॒भाव॑सुः। पञ्च॒ दिशो॒ दैवी᳚र्य॒ज्ञम॑वन्तु दे॒वीरपाम॑तिं दुर्म॒तिम्बाध॑मानाः। रा॒यस्पोषे॑ य॒ज्ञप॑तिमा॒भज॑न्तीः। रा॒यस्पोषे॒ अधि॑ य॒ज्ञो अ॑स्था॒थ्समि॑द्धे अ॒ग्नावधि॑ मामहा॒नः। उ॒क्थप॑त्त्र॒ ईड्यो॑ गृभी॒तस्त॒प्तं घ॒र्मं प॑रि॒गृह्या॑यजन्त। ऊ॒र्जा यद्य॒ज्ञमश॑मन्त दे॒वा दैव्या॑य ध॒र्त्रे जोष्ट्रे᳚। दे॒व॒श्रीः श्रीम॑णाः श॒तप॑याः॥१३॥

%4.6.3.3
प॒रि॒गृह्य॑ दे॒वा य॒ज्ञमा॑यन्न्। सूर्य॑रश्मि॒र्\mbox{}हरि॑केशः पु॒रस्ता᳚थ्सवि॒ता ज्योति॒रुद॑या॒ꣳ॒ अज॑स्रम्। तस्य॑ पू॒षा प्र॑स॒वं या॑ति दे॒वः स॒म्पश्य॒न्विश्वा॒ भुव॑नानि गो॒पाः। दे॒वा दे॒वेभ्यो॑ अध्व॒र्यन्तो॑ अस्थुर्वी॒तꣳ श॑मि॒त्रे श॑मि॒ता य॒जध्यै᳚। तु॒रीयो॑ य॒ज्ञो यत्र॑ ह॒व्यमेति॒ ततः॑ पाव॒का आ॒शिषो॑ नो जुषन्ताम्। वि॒मान॑ ए॒ष दि॒वो मध्य॑ आस्त आपप्रि॒वान्रोद॑सी अ॒न्तरि॑क्षम्। स वि॒श्वाची॑र॒भि॥१४॥

%4.6.3.4
च॒ष्टे॒ घृ॒ताची॑रन्त॒रा पूर्व॒मप॑रं च के॒तुम्। उ॒क्षा स॑मु॒द्रो अ॑रु॒णः सु॑प॒र्णः पूर्व॑स्य॒ योनि॑म्पि॒तुरा वि॑वेश। मध्ये॑ दि॒वो निहि॑तः॒ पृश्ञि॒रश्मा॒ वि च॑क्रमे॒ रज॑सः पा॒त्यन्तौ᳚। इन्द्रं॒ विश्वा॑ अवीवृधन्थ्समु॒द्रव्य॑चसं॒ गिरः॑। र॒थीत॑मꣳ रथी॒नां वाजा॑ना॒ꣳ॒ सत्प॑ति॒म्पतिम्᳚। सु॒म्न॒हूर्य॒ज्ञो दे॒वाꣳ आ च॑ वक्ष॒द्यक्ष॑द॒ग्निर्दे॒वो दे॒वाꣳ आ च॑ वक्षत्। वाज॑स्य मा प्रस॒वेनो᳚द्ग्रा॒भेणोद॑ग्रभीत्। अथा॑ स॒पत्ना॒ꣳ॒ इन्द्रो॑ मे निग्रा॒भेणाध॑राꣳ अकः। उ॒द्ग्रा॒भं च॑ निग्रा॒भं च॒ ब्रह्म॑ दे॒वा अ॑वीवृधन्न्। अथा॑ स॒पत्ना॑निन्द्रा॒ग्नी मे॑ विषू॒चीना॒न्व्य॑स्यताम्॥१५॥

%4.6.4.0
{\anuvakamend[{दे॒वाः श॒तप॑या अ॒भि वाज॑स्य॒ षड्विꣳ॑शतिश्च}]}%॥३॥

%4.6.4.1
आ॒शुः शिशा॑नो वृष॒भो न यु॒ध्मो घ॑नाघ॒नः क्षोभ॑णश्चर्\mbox{}षणी॒नाम्। स॒ङ्क्रन्द॑नो\-ऽनिमि॒ष ए॑कवी॒रः श॒तꣳ सेना॑ अजयथ्सा॒कमिन्द्रः॑। सं॒क्रन्द॑नेनानिमि॒षेण॑ जि॒ष्णुना॑ युत्का॒रेण॑ दुश्च्यव॒नेन॑ धृ॒ष्णुना᳚। तदिन्द्रे॑ण जयत॒ तथ्स॑हध्वं॒ युधो॑ नर॒ इषु॑हस्तेन॒ वृष्णा᳚। स इषु॑हस्तैः॒ स नि॑ष॒ङ्गिभि॑र्व॒शी सꣴस्र॑ष्टा॒ स युध॒ इन्द्रो॑ ग॒णेन। स॒ꣳ॒सृ॒ष्ट॒जिथ्सो॑म॒पा बा॑हुश॒र्ध्यू᳚र्ध्वध॑न्वा॒ प्रति॑हिताभि॒रस्ता᳚। बृह॑स्पते॒ परि॑ दीय॥१६॥

%4.6.4.2
रथे॑न रक्षो॒हामित्राꣳ॑ अप॒बाध॑मानः। प्र॒भ॒ञ्जन्थ्सेनाः᳚ प्रमृ॒णो यु॒धा जय॑न्न॒स्माक॑मेध्यवि॒ता रथा॑नाम्। गो॒त्र॒भिदं॑ गो॒विदं॒ वज्र॑बाहुं॒ जय॑न्त॒मज्म॑ प्रमृ॒णन्त॒मोज॑सा। इ॒मꣳ स॑जाता॒ अनु॑ वीरयध्व॒मिन्द्रꣳ॑ सखा॒यो\-ऽनु॒ सꣳ र॑भध्वम्। ब॒ल॒वि॒ज्ञा॒यः स्थवि॑रः॒ प्रवी॑रः॒ सह॑स्वान् वा॒जी सह॑मान उ॒ग्रः। अ॒भिवी॑रो अ॒भिस॑त्वा सहो॒जा जैत्र॑मिन्द्र॒ रथ॒मा ति॑ष्ठ गो॒वित्। अ॒भि गो॒त्राणि॒ सह॑सा॒ गाह॑मानो\-ऽदा॒यः॥१७॥

%4.6.4.3
वी॒रः श॒तम॑न्यु॒रिन्द्रः॑। दु॒श्च्य॒व॒नः पृ॑तना॒षाड॑यु॒ध्यो᳚\-ऽस्माक॒ꣳ॒ सेना॑ अवतु॒ प्र यु॒थ्सु। इन्द्र॑ आसां ने॒ता बृह॒स्पति॒र्दक्षि॑णा य॒ज्ञः पु॒र ए॑तु॒ सोमः॑। दे॒व॒से॒नाना॑मभिभञ्जती॒नां जय॑न्तीनाम्म॒रुतो॑ य॒न्त्वग्रे᳚। इन्द्र॑स्य॒ वृष्णो॒ वरु॑णस्य॒ राज्ञ॑ आदि॒त्याना᳚म्म॒रुता॒ꣳ॒ शर्ध॑ उ॒ग्रम्। म॒हाम॑नसाम्भुवनच्य॒वानां॒ घोषो॑ दे॒वानां॒ जय॑ता॒मुद॑स्थात्। अ॒स्माक॒मिन्द्रः॒ समृ॑तेषु ध्व॒जेष्व॒स्माकं॒ या इष॑व॒स्ता ज॑यन्तु।॥१८॥

%4.6.4.4
अ॒स्माकं॑ वी॒रा उत्त॑रे भवन्त्व॒स्मानु॑ देवा अवता॒ हवे॑षु। उद्ध॑र्\mbox{}षय मघव॒न्नायु॑धा॒न्युथ्सत्व॑नाम्माम॒काना॒म्महाꣳ॑सि। उद्वृ॑त्रहन्वा॒जिनां॒ वाजि॑ना॒न्युद्रथा॑नां॒ जय॑तामेतु॒ घोषः॑। उप॒ प्रेत॒ जय॑ता नरः स्थि॒रा वः॑ सन्तु बा॒हवः॑। इन्द्रो॑ वः॒ शर्म॑ यच्छत्वनाधृ॒ष्या यथास॑थ। अव॑सृष्टा॒ परा॑ पत॒ शर॑व्ये॒ ब्रह्म॑सꣳशिता। गच्छा॒मित्रा॒न्प्र॥१९॥

%4.6.4.5
वि॒श॒ मैषां॒ कं च॒नोच्छि॑षः। मर्मा॑णि ते॒ वर्म॑भिश्छादयामि॒ सोम॑स्त्वा॒ राजा॒मृते॑ना॒भि व॑स्ताम्। उ॒रोर्वरी॑यो॒ वरि॑वस्ते अस्तु॒ जय॑न्तं॒ त्वामनु॑ मदन्तु दे॒वाः। यत्र॑ बा॒णाः स॒म्पत॑न्ति कुमा॒रा वि॑शि॒खा इ॑व। इन्द्रो॑ न॒स्तत्र॑ वृत्र॒हा वि॑श्वा॒हा शर्म॑ यच्छतु॥२०॥

%4.6.5.0
{\anuvakamend[{दी॒या॒ दा॒यो ज॑यन्त्व॒मित्रा॒न्प्र च॑त्वारि॒ꣳ॒शच्च॑}]}%॥४॥

%4.6.5.1
प्राची॒मनु॑ प्र॒दिश॒म्प्रेहि॑ वि॒द्वान॒ग्नेर॑ग्ने पु॒रो अ॑ग्निर्भवे॒ह। विश्वा॒ आशा॒ दीद्या॑नो॒ वि भा॒ह्यूर्जं॑ नो धेहि द्वि॒पदे॒ चतु॑ष्पदे। क्रम॑ध्वम॒ग्निना॒ नाक॒मुख्य॒ꣳ॒ हस्ते॑षु॒ बिभ्र॑तः। दि॒वः पृ॒ष्ठꣳ सुव॑र्ग॒त्वा मि॒श्रा दे॒वेभि॑राद्ध्वम्। पृ॒थि॒व्या अ॒हमुद॒न्तरि॑क्ष॒मारु॑हम॒न्तरि॑क्षा॒द्दिव॒मारु॑हम्। दि॒वो नाक॑स्य पृ॒ष्ठाथ्सुव॒र्ज्योति॑रगाम्॥२१॥

%4.6.5.2
अ॒हम्। सुव॒र्यन्तो॒ नापे᳚क्षन्त॒ आ द्याꣳ रो॑हन्ति॒ रोद॑सी। य॒ज्ञं ये वि॒श्वतो॑धार॒ꣳ॒ सुवि॑द्वाꣳसो वितेनि॒रे। अग्ने॒ प्रेहि॑ प्रथ॒मो दे॑वय॒तां चक्षु॑र्दे॒वाना॑मु॒त मर्त्या॑नाम्। इय॑क्षमाणा॒ भृगु॑भिः स॒जोषाः॒ सुव॑र्यन्तु॒ यज॑मानाः स्व॒स्ति। नक्तो॒षासा॒ सम॑नसा॒ विरू॑पे धा॒पये॑ते॒ शिशु॒मेकꣳ॑ समी॒ची। द्यावा॒ क्षामा॑ रु॒क्मो अ॒न्तर्विभा॑ति दे॒वा अ॒ग्निं धा॑रयन्द्रविणो॒दाः। अग्ने॑ सहस्राक्ष॥२२॥

%4.6.5.3
श॒त॒मू॒र्ध॒ञ्छ॒तं ते᳚ प्रा॒णाः स॒हस्र॑मपा॒नाः। त्वꣳ सा॑ह॒स्रस्य॑ रा॒य ई॑शिषे॒ तस्मै॑ ते विधेम॒ वाजा॑य॒ स्वाहा᳚। सु॒प॒र्णो॑\-ऽसि ग॒रुत्मा᳚न्पृथि॒व्याꣳ सी॑द पृ॒ष्ठे पृ॑थि॒व्याः सी॑द भा॒सान्तरि॑क्ष॒मा पृ॑ण॒ ज्योति॑षा॒ दिव॒मुत्त॑भान॒ तेज॑सा॒ दिश॒ उद्दृꣳ॑ह। आ॒जुह्वा॑नः सु॒प्रती॑कः पु॒रस्ता॒दग्ने॒ स्वां योनि॒मा सी॑द सा॒ध्या। अ॒स्मिन्थ्स॒धस्थे॒ अध्युत्त॑रस्मि॒न्विश्वे॑ देवाः॥२३॥

%4.6.5.4
यज॑मानश्च सीदत। प्रेद्धो॑ अग्ने दीदिहि पु॒रो नो\-ऽज॑स्रया सू॒र्म्या॑ यविष्ठ। त्वाꣳ शश्व॑न्त॒ उप॑ यन्ति॒ वाजाः᳚। वि॒धेम॑ ते पर॒मे जन्म॑न्नग्ने वि॒धेम॒ स्तोमै॒रव॑रे स॒धस्थे᳚। यस्मा॒द्योने॑रु॒दारि॑था॒ यजे॒ तम्प्र त्वे ह॒वीꣳषि॑ जुहुरे॒ समि॑द्धे। ताꣳ स॑वि॒तुर्वरे᳚ण्यस्य चि॒त्रामाहं वृ॑णे सुम॒तिं वि॒श्वज॑न्याम्। याम॑स्य॒ कण्वो॒ अदु॑ह॒त्प्रपी॑नाꣳ स॒हस्र॑धाराम्॥२४॥

%4.6.5.5
पय॑सा म॒हीं गाम्। स॒प्त ते॑ अग्ने स॒मिधः॑ स॒प्त जि॒ह्वाः स॒प्तर्\mbox{}ष॑यः स॒प्त धाम॑ प्रि॒याणि॑। स॒प्त होत्राः᳚ सप्त॒धा त्वा॑ यजन्ति स॒प्त योनी॒रा पृ॑णस्वा घृ॒तेन॑। ई॒दृङ्चा᳚न्या॒दृङ्चै॑ता॒दृङ्च॑ प्रति॒दृङ्च॑ मि॒तश्च॒ सम्मि॑तश्च॒ सभ॑राः। शु॒क्रज्यो॑तिश्च चि॒त्रज्यो॑तिश्च स॒त्यज्यो॑तिश्च॒ ज्योति॑ष्माꣴश्च स॒त्यश्च॑र्त॒पाश्चात्यꣳ॑हाः।॥२५॥

%4.6.5.6
ऋ॒त॒जिच्च॑ सत्य॒जिच्च॑ सेन॒जिच्च॑ सु॒षेण॒श्चान्त्य॑मित्रश्च दू॒रेअ॑मित्रश्च ग॒णः। ऋ॒तश्च॑ स॒त्यश्च॑ ध्रु॒वश्च॑ ध॒रुण॑श्च ध॒र्ता च॑ विध॒र्ता च॑ विधार॒यः। ई॒दृक्षा॑स एता॒दृक्षा॑स ऊ॒ षु णः॑ स॒दृक्षा॑सः॒ प्रति॑सदृक्षास॒ एत॑न। मि॒तास॑श्च॒ सम्मि॑तासश्च न ऊ॒तये॒ सभ॑रसो मरुतो य॒ज्ञे अ॒स्मिन्निन्द्रं॒ दैवी॒र्विशो॑ म॒रुतो\-ऽनु॑वर्त्मानो॒ यथेन्द्रं॒ दैवी॒र्विशो॑ म॒रुतो\-ऽनु॑वर्त्मान ए॒वमि॒मं यज॑मानं॒ दैवी᳚श्च॒ विशो॒ मानु॑षी॒श्चानु॑वर्त्मानो भवन्तु॥२६॥

%4.6.6.0
{\anuvakamend[{अ॒गा॒ꣳ स॒ह॒स्रा॒क्ष॒ दे॒वाः॒ स॒हस्र॑धारा॒मत्यꣳ॑हा॒ अनु॑वर्त्मानः॒ षोड॑श च}]}%॥५॥

%4.6.6.1
जी॒मूत॑स्येव भवति॒ प्रती॑कं॒ यद्व॒र्मी याति॑ स॒मदा॑मु॒पस्थे᳚। अना॑विद्धया त॒नुवा॑ जय॒ त्वꣳ स त्वा॒ वर्म॑णो महि॒मा पि॑पर्तु। धन्व॑ना॒ गा धन्व॑ना॒जिं ज॑येम॒ धन्व॑ना ती॒व्राः स॒मदो॑ जयेम। धनुः॒ शत्रो॑रपका॒मं कृ॑णोति॒ धन्व॑ना॒ सर्वाः᳚ प्र॒दिशो॑ जयेम। व॒क्ष्यन्ती॒वेदा ग॑नीगन्ति॒ कर्ण॑म्प्रि॒यꣳ सखा॑यं परिषस्वजा॒ना। योषे॑व शिङ्क्ते॒ वित॒ताधि॒ धन्वन्न्॑॥२७॥

%4.6.6.2
ज्या इ॒यꣳ सम॑ने पा॒रय॑न्ती। ते आ॒चर॑न्ती॒ सम॑नेव॒ योषा॑ मा॒तेव॑ पु॒त्रम्बि॑भृतामु॒पस्थे᳚। अप॒ शत्र न्॑विध्यताꣳ संविदा॒ने आर्त्नी॑ इ॒मे वि॑ष्फु॒रन्ती॑ अ॒मित्रान्॑। ब॒ह्वी॒नाम्पि॒ता ब॒हुर॑स्य पु॒त्रश्चि॒श्चा कृ॑णोति॒ सम॑नाव॒गत्य॑। इ॒षु॒धिः सङ्काः॒ पृत॑नाश्च॒ सर्वाः᳚ पृ॒ष्ठे निन॑द्धो जयति॒ प्रसू॑तः। रथे॒ तिष्ठ॑न्नयति वा॒जिनः॑ पु॒रो यत्र॑यत्र का॒मय॑ते सुषार॒थिः। अ॒भीशू॑नाम्महि॒मानम्᳚॥२८॥

%4.6.6.3
प॒ना॒य॒त॒ मनः॑ प॒श्चादनु॑ यच्छन्ति र॒श्मयः॑। ती॒व्रान्घोषा᳚न्कृण्वते॒ वृष॑पाण॒यो\-ऽश्वा॒ रथे॑भिः स॒ह वा॒जय॑न्तः। अ॒व॒क्राम॑न्तः॒ प्रप॑दैर॒मित्रा᳚न्क्षि॒णन्ति॒ शत्रू॒ꣳ॒रन॑पव्ययन्तः। र॒थ॒वाह॑नꣳ ह॒विर॑स्य॒ नाम॒ यत्रायु॑धं॒ निहि॑तमस्य॒ वर्म॑। तत्रा॒ रथ॒मुप॑ श॒ग्मꣳ स॑देम वि॒श्वाहा॑ व॒यꣳ सु॑मन॒स्यमा॑नाः। स्वा॒दु॒ष॒ꣳ॒सदः॑ पि॒तरो॑ वयो॒धाः कृ॑च्छ्रे॒श्रितः॒ शक्ती॑वन्तो गभी॒राः। चि॒त्रसे॑ना॒ इषु॑बला॒ अमृ॑ध्राः स॒तोवी॑रा उ॒रवो᳚ व्रातसा॒हाः। ब्राह्म॑णासः॥२९॥

%4.6.6.4
पित॑रः॒ सोम्या॑सः शि॒वे नो॒ द्यावा॑पृथि॒वी अ॑ने॒हसा᳚। पू॒षा नः॑ पातु दुरि॒तादृ॑तावृधो॒ रक्षा॒ माकि॑र्नो अ॒घशꣳ॑स ईशत। सु॒प॒र्णं व॑स्ते मृ॒गो अ॑स्या॒ दन्तो॒ गोभिः॒ सन्न॑द्धा पतति॒ प्रसू॑ता। यत्रा॒ नरः॒ सं च॒ वि च॒ द्रव॑न्ति॒ तत्रा॒स्मभ्य॒मिष॑वः॒ शर्म॑ यꣳसन्न्। ऋजी॑ते॒ परि॑ वृङ्ग्धि॒ नो\-ऽश्मा॑ भवतु नस्त॒नूः। सोमो॒ अधि॑ ब्रवीतु॒ नो\-ऽदि॑तिः॥३०॥

%4.6.6.5
शर्म॑ यच्छतु। आ ज॑ङ्घन्ति॒ सान्वे॑षां ज॒घना॒ꣳ॒ उप॑ जिघ्नते। अश्वा॑जनि॒ प्रचे॑त॒सो\-ऽश्वा᳚न्थ्स॒मथ्सु॑ चोदय। अहि॑रिव भो॒गैः पर्ये॑ति बा॒हुं ज्याया॑ हे॒तिं प॑रि॒बाध॑मानः। ह॒स्त॒घ्नो विश्वा॑ व॒युना॑नि वि॒द्वान्पुमा॒न्पुमाꣳ॑सं॒ परि॑ पातु वि॒श्वतः॑। वन॑स्पते वी॒ड्व॑ङ्गो॒ हि भू॒या अ॒स्मथ्स॑खा प्र॒तर॑णः सु॒वीरः॑। गोभिः॒ सन्न॑द्धो असि वी॒डय॑स्वास्था॒ता ते॑ जयतु॒ जेत्वा॑नि। दि॒वः पृ॑थि॒व्याः परि॑॥३१॥

%4.6.6.6
ओज॒ उद्भृ॑तं॒ वन॒स्पति॑भ्यः॒ पर्याभृ॑त॒ꣳ॒ सहः॑। अ॒पामो॒ज्मानं॒ परि॒ गोभि॒रावृ॑त॒मिन्द्र॑स्य॒ वज्रꣳ॑ ह॒विषा॒ रथं॑ यज। इन्द्र॑स्य॒ वज्रो॑ म॒रुता॒मनी॑कम्मि॒त्रस्य॒ गर्भो॒ वरु॑णस्य॒ नाभिः॑। सेमां नो॑ ह॒व्यदा॑तिं जुषा॒णो देव॑ रथ॒ प्रति॑ ह॒व्या गृ॑भाय। उप॑ श्वासय पृथि॒वीमु॒त द्याम्पु॑रु॒त्रा ते॑ मनुतां॒ विष्ठि॑तं॒ जग॑त्। स दु॑न्दुभे स॒जूरिन्द्रे॑ण दे॒वैर्दू॒रात्॥३२॥

%4.6.6.7
दवी॑यो॒ अप॑ सेध॒ शत्रून्॑। आ क्र॑न्दय॒ बल॒मोजो॑ न॒ आ धा॒ नि ष्ट॑निहि दुरि॒ता बाध॑मानः। अप॑ प्रोथ दुन्दुभे दु॒च्छुनाꣳ॑ इ॒त इन्द्र॑स्य मु॒ष्टिर॑सि वी॒डय॑स्व। आमूर॑ज प्र॒त्याव॑र्तये॒माः के॑तु॒मद्दु॑न्दु॒भिर्वा॑वदीति। समश्व॑पर्णा॒श्चर॑न्ति नो॒ नरो॒\-ऽस्माक॑मिन्द्र र॒थिनो॑ जयन्तु॥३३॥

%4.6.7.0
{\anuvakamend[{धन्व॑न्महि॒मानं॒ ब्राह्म॑णा॒सो\-ऽदि॑तिः पृथि॒व्याः परि॑ दू॒रादेक॑चत्वारिꣳशच्च}]}%॥६॥

%4.6.7.1
यदक्र॑न्दः प्रथ॒मं जाय॑मान उ॒द्यन्थ्स॑मु॒द्रादु॒त वा॒ पुरी॑षात्। श्ये॒नस्य॑ प॒क्षा ह॑रि॒णस्य॑ बा॒हू उ॑प॒स्तुत्य॒म्महि॑ जा॒तं ते॑ अर्वन्न्। य॒मेन॑ द॒त्तं त्रि॒त ए॑नमायुन॒गिन्द्र॑ एणम्प्रथ॒मो अध्य॑तिष्ठत्। ग॒न्ध॒र्वो अ॑स्य रश॒नाम॑गृभ्णा॒थ्सूरा॒दश्वं॑ वसवो॒ निर॑तष्ट। असि॑ य॒मो अस्या॑दि॒त्यो अ॑र्व॒न्नसि॑ त्रि॒तो गुह्ये॑न व्र॒तेन॑। असि॒ सोमे॑न स॒मया॒ विपृ॑क्तः॥३४॥

%4.6.7.2
आ॒हुस्ते॒ त्रीणि॑ दि॒वि बन्ध॑नानि। त्रीणि॑ त आहुर्दि॒वि बन्ध॑नानि॒ त्रीण्य॒फ्सु त्रीण्य॒न्तः स॑मु॒द्रे। उ॒तेव॑ मे॒ वरु॑णश्छन्थ्स्यर्व॒न् यत्रा॑ त आ॒हुः प॑र॒मं ज॒नित्रम्᳚। इ॒मा ते॑ वाजिन्नव॒मार्ज॑नानी॒मा श॒फानाꣳ॑ सनि॒तुर्नि॒धाना᳚। अत्रा॑ ते भ॒द्रा र॑श॒ना अ॑पश्यमृ॒तस्य॒ या अ॑भि॒रक्ष॑न्ति गो॒पाः। आ॒त्मानं॑ ते॒ मन॑सा॒राद॑जानाम॒वो दि॒वा॥३५॥

%4.6.7.3
प॒तय॑न्तम्पतं॒गम्। शिरो॑ अपश्यम्प॒थिभिः॑ सु॒गेभि॑ररे॒णुभि॒र्जेह॑मानम्पत॒त्रि। अत्रा॑ ते रू॒पमु॑त्त॒मम॑पश्यं॒ जिगी॑षमाणमि॒ष आ प॒दे गोः। य॒दा ते॒ मर्तो॒ अनु॒ भोग॒मान॒डादिद्ग्रसि॑ष्ठ॒ ओष॑धीरजीगः। अनु॑ त्वा॒ रथो॒ अनु॒ मर्यो॑ अर्व॒न्ननु॒ गावो\-ऽनु॒ भगः॑ क॒नीना᳚म्। अनु॒ व्राता॑स॒स्तव॑ स॒ख्यमी॑यु॒रनु॑ दे॒वा म॑मिरे वी॒र्यम्᳚॥३६॥

%4.6.7.4
ते॒। हिर॑ण्यशृ॒ङ्गो\-ऽयो॑ अस्य॒ पादा॒ मनो॑जवा॒ अव॑र॒ इन्द्र॑ आसीत्। दे॒वा इद॑स्य हवि॒रद्य॑माय॒न् यो अर्व॑न्तम्प्रथ॒मो अ॒ध्यति॑ष्ठत्। ई॒र्मान्ता॑सः॒ सिलि॑कमध्यमासः॒ सꣳ शूर॑णासो दि॒व्यासो॒ अत्याः᳚। ह॒ꣳ॒सा इ॑व श्रेणि॒शो य॑तन्ते॒ यदाक्षि॑षुर्दि॒व्यमज्म॒मश्वाः᳚। तव॒ शरी॑रम्पतयि॒ष्ण्व॑र्व॒न्तव॑ चि॒त्तं वात॑ इव॒ ध्रजी॑मान्। तव॒ शृङ्गा॑णि॒ विष्ठि॑ता पुरु॒त्रार॑ण्येषु॒ जर्भु॑राणा चरन्ति। उप॑॥३७॥

%4.6.7.5
प्रागा॒च्छस॑नं वा॒ज्यर्वा॑ देव॒द्रीचा॒ मन॑सा॒ दीध्या॑नः। अ॒जः पु॒रो नी॑यते॒ नाभि॑र॒स्यानु॑ प॒श्चात्क॒वयो॑ यन्ति रे॒भाः। उप॒ प्रागा᳚त्पर॒मं यथ्स॒धस्थ॒मर्वा॒ꣳ॒ अच्छा॑ पि॒तर॑म्मा॒तरं॑ च। अ॒द्या दे॒वां जुष्ट॑तमो॒ हि ग॒म्या अथा शा᳚स्ते दा॒शुषे॒ वार्या॑णि॥३८॥

%4.6.8.0
{\anuvakamend[{विपृ॑क्तो दि॒वा वी॒र्य॑मुपैका॒न्नच॑त्वारि॒ꣳ॒शच्च॑}]}%॥७॥

%4.6.8.1
मा नो॑ मि॒त्रो वरु॑णो अर्य॒मायुरिन्द्र॑ ऋभु॒क्षा म॒रुतः॒ परि॑ ख्यन्न्। यद्वा॒जिनो॑ दे॒वजा॑तस्य॒ सप्तेः᳚ प्रव॒क्ष्यामो॑ वि॒दथे॑ वी॒र्या॑णि। यन्नि॒र्णिजा॒ रेक्ण॑सा॒ प्रावृ॑तस्य रा॒तिं गृ॑भी॒ताम्मु॑ख॒तो नय॑न्ति। सुप्रा॑ङ॒जो मेम्य॑द्वि॒श्वरू॑प इन्द्रापू॒ष्णोः प्रि॒यमप्ये॑ति॒ पाथः॑। ए॒ष च्छागः॑ पु॒रो अश्वे॑न वा॒जिना॑ पू॒ष्णो भा॒गो नी॑यते वि॒श्वदे᳚व्यः। अ॒भि॒प्रियं॒ यत्पु॑रो॒डाश॒मर्व॑ता॒ त्वष्टेत्॥३९॥

%4.6.8.2
ए॒न॒ꣳ॒ सौ॒श्र॒व॒साय॑ जिन्वति। यद्ध॒विष्य॑मृतु॒शो दे॑व॒यानं॒ त्रिर्मानु॑षाः॒ पर्यश्वं॒ नय॑न्ति। अत्रा॑ पू॒ष्णः प्र॑थ॒मो भा॒ग ए॑ति य॒ज्ञं दे॒वेभ्यः॑ प्रतिवे॒दय॑न्न॒जः। होता᳚ध्व॒र्युराव॑या अग्निमि॒न्धो ग्रा॑वग्रा॒भ उ॒त शꣴस्ता॒ सुवि॑प्रः। तेन॑ य॒ज्ञेन॑ स्व॑रं कृतेन॒ स्वि॑ष्टेन व॒क्षणा॒ आ पृ॑णध्वम्। यू॒प॒व्र॒स्का उ॒त ये यू॑पवा॒हाश्च॒षालं॒ ये अ॑श्वयू॒पाय॒ तक्ष॑ति। ये चार्व॑ते॒ पच॑नꣳ स॒म्भर॑न्त्यु॒तो॥४०॥

%4.6.8.3
तेषा॑म॒भिगू᳚र्तिर्न इन्वतु। उप॒ प्रागा᳚थ्सु॒मन्मे॑\-ऽधायि॒ मन्म॑ दे॒वाना॒माशा॒ उप॑ वी॒तपृ॑ष्ठः। अन्वे॑नं॒ विप्रा॒ ऋष॑यो मदन्ति दे॒वानां᳚ पु॒ष्टे च॑कृमा सु॒बन्धुम्᳚। यद्वा॒जिनो॒ दाम॑ सं॒दान॒मर्व॑तो॒ या शी॑र्\mbox{}ष॒ण्या॑ रश॒ना रज्जु॑रस्य। यद्वा॑ घास्य॒ प्रभृ॑तमा॒स्ये॑ तृण॒ꣳ॒ सर्वा॒ ता ते॒ अपि॑ दे॒वेष्व॑स्तु। यदश्व॑स्य क्र॒विषः॑॥४१॥

%4.6.8.4
मक्षि॒काश॒ यद्वा॒ स्वरौ॒ स्वधि॑तौ रि॒प्तमस्ति॑। यद्धस्त॑योः शमि॒तुर्यन्न॒खेषु॒ सर्वा॒ ता ते॒ अपि॑ दे॒वेष्व॑स्तु। यदूव॑ध्यमु॒दर॑स्याप॒वाति॒ य आ॒मस्य॑ क्र॒विषो॑ ग॒न्धो अस्ति॑। सु॒कृ॒ता तच्छ॑मि॒तारः॑ कृण्वन्तू॒त मेधꣳ॑ शृत॒पाकं॑ पचन्तु। यत्ते॒ गात्रा॑द॒ग्निना॑ प॒च्यमा॑नाद॒भि शूलं॒ निह॑तस्याव॒धाव॑ति। मा तद्भूम्या॒मा श्रि॑ष॒न्मा तृणे॑षु दे॒वेभ्य॒स्तदु॒शद्भ्यो॑ रा॒तम॑स्तु॥४२॥

%4.6.9.0
{\anuvakamend[{इदु॒तो क्र॒विषः॑ श्रिषथ्स॒प्त च॑}]}%॥८॥

%4.6.9.1
ये वा॒जिनं॑ परि॒पश्य॑न्ति प॒क्वं य ई॑मा॒हुः सु॑र॒भिर्निर्\mbox{}ह॒रेति॑। ये चार्व॑तो माꣳसभि॒क्षामु॒पास॑त उ॒तो तेषा॑म॒भिगू᳚र्तिर्न इन्वतु। यन्नीक्ष॑णम्मा॒ꣳ॒स्पच॑न्या उ॒खाया॒ या पात्रा॑णि यू॒ष्ण आ॒सेच॑नानि। ऊ॒ष्म॒ण्या॑पि॒धाना॑ चरू॒णाम॒ङ्काः सू॒नाः परि॑ भूष॒न्त्यश्वम्᳚। नि॒क्रम॑णं नि॒षद॑नं वि॒वर्त॑नं॒ यच्च॒ पड्बी॑श॒मर्व॑तः। यच्च॑ प॒पौ यच्च॑ घा॒सिम्॥४३॥

%4.6.9.2
ज॒घास॒ सर्वा॒ ता ते॒ अपि॑ दे॒वेष्व॑स्तु। मा त्वा॒ग्निर्ध्व॑नयिद्धू॒मग॑न्धि॒र्मोखा भ्राज॑न्त्य॒भि वि॑क्त॒ जघ्रिः॑। इ॒ष्टं वी॒तम॒भिगू᳚र्तं॒ वष॑ट्कृतं॒ तं दे॒वासः॒ प्रति॑ गृभ्ण॒न्त्यश्वम्᳚। यदश्वा॑य॒ वास॑ उपस्तृ॒णन्त्य॑धीवा॒सं या हिर॑ण्यान्यस्मै। सं॒दान॒मर्व॑न्त॒म्पड्बी॑शम्प्रि॒या दे॒वेष्वा या॑मयन्ति। यत्ते॑ सा॒दे मह॑सा॒ शूकृ॑तस्य॒ पार्ष्णि॑या वा॒ कश॑या॥४४॥

%4.6.9.3
वा॒ तु॒तोद॑। स्रु॒चेव॒ ता ह॒विषो॑ अध्व॒रेषु॒ सर्वा॒ ता ते॒ ब्रह्म॑णा सूदयामि। चतु॑स्त्रिꣳशद्वा॒जिनो॑ दे॒वब॑न्धो॒र्वङ्क्री॒रश्व॑स्य॒ स्वधि॑तिः॒ समे॑ति। अच्छि॑द्रा॒ गात्रा॑ व॒युना॑ कृणोत॒ परु॑ष्परुरनु॒घुष्या॒ वि श॑स्त। एक॒स्त्वष्टु॒रश्व॑स्या विश॒स्ता द्वा य॒न्तारा॑ भवत॒स्तथ॒र्तुः। या ते॒ गात्रा॑णामृतु॒था कृ॒णोमि॒ ताता॒ पिण्डा॑ना॒म्प्र जु॑होम्य॒ग्नौ। मा त्वा॑ तपत्॥४५॥

%4.6.9.4
प्रि॒य आ॒त्मापि॒यन्तं॒ मा स्वधि॑तिस्त॒नुव॒ आ ति॑ष्ठिपत्ते। मा ते॑ गृ॒ध्नुर॑विश॒स्ताति॒हाय॑ छि॒द्रा गात्रा॑ण्य॒सिना॒ मिथू॑ कः। न वा उ॑ वे॒तन्म्रि॑यसे॒ न रि॑ष्यसि दे॒वाꣳ इदे॑षि प॒थिभिः॑ सु॒गेभिः॑। हरी॑ ते॒ युञ्जा॒ पृष॑ती अभूता॒मुपा᳚स्थाद्वा॒जी धु॒रि रास॑भस्य। सु॒गव्यं॑ नो वा॒जी स्वश्वि॑यम्पु॒ꣳ॒सः पु॒त्राꣳ उ॒त वि॑श्वा॒पुषꣳ॑ र॒यिम्। अ॒ना॒गा॒स्त्वं नो॒ अदि॑तिः कृणोतु क्ष॒त्रं नो॒ अश्वो॑ वनताꣳ ह॒विष्मान्॑॥४६॥

%4.7.0.0

{\anuvakamend[{घा॒सिं कश॑या तपद्र॒यिं नव॑ च}]}%॥९॥
%%% END PRASHNA

\sect{सप्तमः प्रश्नः}\setcounter{anuvakam}{0}
\dnsub{तैत्तिरीयसंहितायां चतुर्थकाण्डे सप्तमः प्रश्नः}
%4.7.1.0
%4.7.1.1
अग्ना॑विष्णू स॒जोष॑से॒मा व॑र्धन्तु वां॒ गिरः॑। द्यु॒म्नैर्वाजे॑भि॒राग॑तम्। वाज॑श्च मे प्रस॒वश्च॑ मे॒ प्रय॑तिश्च मे॒ प्रसि॑तिश्च मे धी॒तिश्च॑ मे॒ क्रतु॑श्च मे॒ स्वर॑श्च मे॒ श्लोक॑श्च मे श्रा॒वश्च॑ मे॒ श्रुति॑श्च मे॒ ज्योति॑श्च मे॒ सुव॑श्च मे प्रा॒णश्च॑ मे\-ऽपा॒नः॥१॥

%4.7.1.2
च॒ मे॒ व्या॒नश्च॒ मे\-ऽसु॑श्च मे चि॒त्तं च॑ म॒ आधी॑तं च मे॒ वाक्च॑ मे॒ मन॑श्च मे॒ चक्षु॑श्च मे॒ श्रोत्रं॑ च मे॒ दक्ष॑श्च मे॒ बलं॑ च म॒ ओज॑श्च मे॒ सह॑श्च म॒ आयु॑श्च मे ज॒रा च॑ म आ॒त्मा च॑ मे त॒नूश्च॑ मे॒ शर्म॑ च मे॒ वर्म॑ च॒ मे\-ऽङ्गा॑नि च मे॒\-ऽस्थानि॑ च मे॒ परूꣳ॑षि च मे॒ शरी॑राणि च मे॥२॥

%4.7.2.0
{\anuvakamend[{अ॒पा॒नस्त॒नूश्च॑ मे॒\-ऽष्टाद॑श च}]}%॥१॥

%4.7.2.1
ज्यैष्ठ्यं॑ च म॒ आधि॑पत्यं च मे म॒न्युश्च॑ मे॒ भाम॑श्च॒ मे\-ऽम॑श्च॒ मे\-ऽम्भ॑श्च मे जे॒मा च॑ मे महि॒मा च॑ मे वरि॒मा च॑ मे प्रथि॒मा च॑ मे व॒र्ष्मा च॑ मे द्राघु॒या च॑ मे वृ॒द्धं च॑ मे॒ वृद्धि॑श्च मे स॒त्यं च॑ मे श्र॒द्धा च॑ मे॒ जग॑च्च॥३॥

%4.7.2.2
मे॒ धनं॑ च मे॒ वश॑श्च मे॒ त्विषि॑श्च मे क्री॒डा च॑ मे॒ मोद॑श्च मे जा॒तं च॑ मे जनि॒ष्यमा॑णं च मे सू॒क्तं च॑ मे सुकृ॒तं च॑ मे वि॒त्तं च॑ मे॒ वेद्यं॑ च मे भू॒तं च॑ मे भवि॒ष्यच्च॑ मे सु॒गं च॑ मे सु॒पथं॑ च म ऋ॒द्धं च॑ म॒ ऋद्धि॑श्च मे कॢ॒प्तं च॑ मे॒ कॢप्ति॑श्च मे म॒तिश्च॑ मे सुम॒तिश्च॑ मे॥४॥

%4.7.3.0
{\anuvakamend[{जग॒च्चर्द्धि॒श्चतु॑र्दश च}]}%॥२॥

%4.7.3.1
शं च॑ मे॒ मय॑श्च मे प्रि॒यं च॑ मे\-ऽनुका॒मश्च॑ मे॒ काम॑श्च मे सौमन॒सश्च॑ मे भ॒द्रं च॑ मे॒ श्रेय॑श्च मे॒ वस्य॑श्च मे॒ यश॑श्च मे॒ भग॑श्च मे॒ द्रवि॑णं च मे य॒न्ता च॑ मे ध॒र्ता च॑ मे॒ क्षेम॑श्च मे॒ धृति॑श्च मे॒ विश्वं॑ च॥५॥

%4.7.3.2
मे॒ मह॑श्च मे सं॒विच्च॑ मे॒ ज्ञात्रं॑ च मे॒ सूश्च॑ मे प्र॒सूश्च॑ मे॒ सीरं॑ च मे ल॒यश्च॑ म ऋ॒तं च॑ मे॒\-ऽमृतं॑ च मे\-ऽय॒क्ष्मं च॒ मे\-ऽना॑मयच्च मे जी॒वातु॑श्च मे दीर्घायु॒त्वं च॑ मे\-ऽनमि॒त्रं च॒ मे\-ऽभ॑यं च मे सु॒गं च॑ मे॒ शय॑नं च मे सू॒षा च॑ मे सु॒दिनं॑ च मे॥६॥

%4.7.4.0
{\anuvakamend[{विश्वं॑ च॒ शय॑नम॒ष्टौ च॑}]}%॥३॥

%4.7.4.1
ऊर्क्च॑ मे सू॒नृता॑ च मे॒ पय॑श्च मे॒ रस॑श्च मे घृ॒तं च॑ मे॒ मधु॑ च मे॒ सग्धि॑श्च मे॒ सपी॑तिश्च मे कृ॒षिश्च॑ मे॒ वृष्टि॑श्च मे॒ जैत्रं॑ च म॒ औद्भि॑द्यं च मे र॒यिश्च॑ मे॒ राय॑श्च मे पु॒ष्टं च॑ मे॒ पुष्टि॑श्च मे वि॒भु च॑॥७॥

%4.7.4.2
मे॒ प्र॒भु च॑ मे ब॒हु च॑ मे॒ भूय॑श्च मे पू॒र्णं च॑ मे पू॒र्णत॑रं च॒ मे\-ऽक्षि॑तिश्च मे॒ कूय॑वाश्च॒ मे\-ऽन्नं॑ च॒ मे\-ऽक्षु॑च्च मे व्री॒हय॑श्च मे॒ यवा᳚श्च मे॒ माषा᳚श्च मे॒ तिला᳚श्च मे मु॒द्गाश्च॑ मे ख॒ल्वा᳚श्च मे गो॒धूमा᳚श्च मे म॒सुरा᳚श्च मे प्रि॒यङ्ग॑वश्च॒ मे\-ऽण॑वश्च मे श्या॒माका᳚श्च मे नी॒वारा᳚श्च मे॥८॥

%4.7.5.0
{\anuvakamend[{वि॒भु च॑ म॒सुरा॒श्चतु॑र्दश च}]}%॥४॥

%4.7.5.1
अश्मा॑ च मे॒ मृत्ति॑का च मे गि॒रय॑श्च मे॒ पर्व॑ताश्च मे॒ सिक॑ताश्च मे॒ वन॒स्पत॑यश्च मे॒ हिर॑ण्यं च॒ मे\-ऽय॑श्च मे॒ सीसं॑ च मे॒ त्रपु॑श्च मे श्या॒मं च॑ मे लो॒हं च॑ मे॒\-ऽग्निश्च॑ म॒ आप॑श्च मे वी॒रुध॑श्च म॒ ओष॑धयश्च मे कृष्टप॒च्यं च॑॥९॥

%4.7.5.2
मे॒\-ऽकृ॒ष्ट॒प॒च्यं च॑ मे ग्रा॒म्याश्च॑ मे प॒शव॑ आर॒ण्याश्च॑ य॒ज्ञेन॑ कल्पन्तां वि॒त्तं च॑ मे॒ वित्ति॑श्च मे भू॒तं च॑ मे॒ भूति॑श्च मे॒ वसु॑ च मे वस॒तिश्च॑ मे॒ कर्म॑ च मे॒ शक्ति॑श्च॒ मे\-ऽर्थ॑श्च म॒ एम॑श्च म॒ इति॑श्च मे॒ गति॑श्च मे॥१०॥

%4.7.6.0
{\anuvakamend[{कृ॒ष्ट॒प॒च्यञ्चा॒ष्टाच॑त्वारिꣳशच्च}]}%॥५॥

%4.7.6.1
अ॒ग्निश्च॑ म॒ इन्द्र॑श्च मे॒ सोम॑श्च म॒ इन्द्र॑श्च मे सवि॒ता च॑ म॒ इन्द्र॑श्च मे॒ सर॑स्वती च म॒ इन्द्र॑श्च मे पू॒षा च॑ म॒ इन्द्र॑श्च मे॒ बृह॒स्पति॑श्च म॒ इन्द्र॑श्च मे मि॒त्रश्च॑ म॒ इन्द्र॑श्च मे॒ वरु॑णश्च म॒ इन्द्र॑श्च मे॒ त्वष्टा॑ च॥११॥

%4.7.6.2
म॒ इन्द्र॑श्च मे धा॒ता च॑ म॒ इन्द्र॑श्च मे॒ विष्णु॑श्च म॒ इन्द्र॑श्च मे॒\-ऽश्विनौ॑ च म॒ इन्द्र॑श्च मे म॒रुत॑श्च म॒ इन्द्र॑श्च मे॒ विश्वे॑ च मे दे॒वा इन्द्र॑श्च मे पृथि॒वी च॑ म॒ इन्द्र॑श्च मे॒\-ऽन्तरि॑क्षञ्च म॒ इन्द्र॑श्च मे॒ द्यौश्च॑ म॒ इन्द्र॑श्च मे॒ दिश॑श्च म॒ इन्द्र॑श्च मे मू॒र्धा च॑ म॒ इन्द्र॑श्च मे प्र॒जाप॑तिश्च म॒ इन्द्र॑श्च मे॥१२॥

%4.7.7.0
{\anuvakamend[{त्वष्टा॑ च॒ द्यौश्च॑ म॒ एक॑विꣳशतिश्च}]}%॥६॥

%4.7.7.1
अ॒ꣳ॒शुश्च॑ मे र॒श्मिश्च॒ मे\-ऽदा᳚भ्यश्च॒ मे\-ऽधि॑पतिश्च म उपा॒ꣳ॒शुश्च॑ मे\-ऽन्तर्या॒मश्च॑ म ऐन्द्रवाय॒वश्च॑ मे मैत्रावरु॒णश्च॑ म आश्वि॒नश्च॑ मे प्रतिप्र॒स्थान॑श्च मे शु॒क्रश्च॑ मे म॒न्थी च॑ म आग्रय॒णश्च॑ मे वैश्वदे॒वश्च॑ मे ध्रु॒वश्च॑ मे वैश्वान॒रश्च॑ म ऋतुग्र॒हाश्च॑॥१३॥

%4.7.7.2
मे॒\-ऽति॒ग्रा॒ह्या᳚श्च म ऐन्द्रा॒ग्नश्च॑ मे वैश्वदे॒वश्च॑ मे मरुत्व॒तीया᳚श्च मे माहे॒न्द्रश्च॑ म आदि॒त्यश्च॑ मे सावि॒त्रश्च॑ मे सारस्व॒तश्च॑ मे पौ॒ष्णश्च॑ मे पात्नीव॒तश्च॑ मे हारियोज॒नश्च॑ मे॥१४॥

%4.7.8.0
{\anuvakamend[{ऋ॒तु॒ग्र॒हाश्च॒ चतु॑स्त्रिꣳशच्च}]}%॥७॥

%4.7.8.1
इ॒ध्मश्च॑ मे ब॒र्\mbox{}हिश्च॑ मे॒ वेदि॑श्च मे॒ धिष्णि॑याश्च मे॒ स्रुच॑श्च मे चम॒साश्च॑ मे॒ ग्रावा॑णश्च मे॒ स्वर॑वश्च म उपर॒वाश्च॑ मे\-ऽधि॒षव॑णे च मे द्रोणकल॒शश्च॑ मे वाय॒व्या॑नि च मे पूत॒भृच्च॑ म आधव॒नीय॑श्च म॒ आग्नी᳚ध्रं च मे हवि॒र्धानं॑ च मे गृ॒हाश्च॑ मे॒ सद॑श्च मे पुरो॒डाशा᳚श्च मे पच॒ताश्च॑ मे\-ऽवभृ॒थश्च॑ मे स्वगाका॒रश्च॑ मे॥१५॥

%4.7.9.0
{\anuvakamend[{गृ॒हाश्च॒ षोड॑श च}]}%॥८॥

%4.7.9.1
अ॒ग्निश्च॑ मे घ॒र्मश्च॑ मे॒\-ऽर्कश्च॑ मे॒ सूर्य॑श्च मे प्रा॒णश्च॑ मे\-ऽश्वमे॒धश्च॑ मे पृथि॒वी च॒ मे\-ऽदि॑तिश्च मे॒ दिति॑श्च मे॒ द्यौश्च॑ मे॒ शक्व॑रीर॒ङ्गुल॑यो॒ दिश॑श्च मे य॒ज्ञेन॑ कल्पन्ता॒मृक्च॑ मे॒ साम॑ च मे॒ स्तोम॑श्च मे॒ यजु॑श्च मे दी॒क्षा च॑ मे॒ तप॑श्च म ऋ॒तुश्च॑ मे व्र॒तं च॑ मे\-ऽहोरा॒त्रयो᳚र्वृ॒ष्ट्या बृ॑हद्रथन्त॒रे च॑ मे य॒ज्ञेन॑ कल्पेताम्॥१६॥

%4.7.10.0
{\anuvakamend[{दी॒क्षा\-ऽष्टाद॑श च}]}%॥९॥

%4.7.10.1
गर्भा᳚श्च मे व॒थ्साश्च॑ मे॒ त्र्यवि॑श्च मे त्र्य॒वी च॑ मे दित्य॒वाट्च॑ मे दित्यौ॒ही च॑ मे॒ पञ्चा॑विश्च मे पञ्चा॒वी च॑ मे त्रिव॒थ्सश्च॑ मे त्रिव॒थ्सा च॑ मे तुर्य॒वाट्च॑ मे तुर्यौ॒ही च॑ मे पष्ठ॒वाच्च॑ मे पष्ठौ॒ही च॑ म उ॒क्षा च॑ मे व॒शा च॑ म ऋष॒भश्च॑॥१७॥

%4.7.10.2
मे॒ वे॒हच्चमे\-ऽन॒ड्वां च॑ मे धे॒नुश्च॑ म॒ आयु॑र्य॒ज्ञेन॑ कल्पतां प्रा॒णो य॒ज्ञेन॑ कल्पतामपा॒नो य॒ज्ञेन॑ कल्पताव्व्याँ॒नो य॒ज्ञेन॑ कल्पतां॒ चक्षु॑र्य॒ज्ञेन॑ कल्पता॒ꣴ॒ श्रोत्रं॑ य॒ज्ञेन॑ कल्पता॒म्मनो॑ य॒ज्ञेन॑ कल्पतां॒ वाग्य॒ज्ञेन॑ कल्पतामा॒त्मा य॒ज्ञेन॑ कल्पतां य॒ज्ञो य॒ज्ञेन॑ कल्पताम्॥१८॥

%4.7.11.0
{\anuvakamend[{ऋ॒ष॒भश्च॑ चत्वारि॒ꣳ॒शच्च॑}]}%॥10॥

%4.7.11.1
एका॑ च मे ति॒स्रश्च॑ मे॒ पञ्च॑ च मे स॒प्त च॑ मे॒ नव॑ च म॒ एका॑दश च मे॒ त्रयो॑दश च मे॒ पञ्च॑दश च मे स॒प्तद॑श च मे॒ नव॑दश च म॒ एक॑विꣳशतिश्च मे॒ त्रयो॑विꣳशतिश्च मे॒ पञ्च॑विꣳशतिश्च मे स॒प्तविꣳ॑शतिश्च मे॒ नव॑विꣳशतिश्च म॒ एक॑त्रिꣳशच्च मे॒ त्रय॑स्त्रिꣳशच्च॥१९॥

%4.7.11.2
मे॒ चत॑स्रश्च मे॒\-ऽष्टौ च॑ मे॒ द्वाद॑श च मे॒ षोड॑श च मे विꣳश॒तिश्च॑ मे॒ चतु॑र्विꣳशतिश्च मे॒\-ऽष्टाविꣳ॑शतिश्च मे॒ द्वात्रिꣳ॑शच्च मे॒ षट्त्रिꣳ॑शच्च मे चत्वारि॒ꣳ॒शच्च॑ मे॒ चतु॑श्चत्वारिꣳशच्च मे॒\-ऽष्टाच॑त्वारिꣳशच्च मे॒ वाज॑श्च प्रस॒वश्चा॑पि॒जश्च॒ क्रतु॑श्च॒ सुव॑श्च मू॒र्धा च॒ व्यश्नि॑यश्चान्त्याय॒नश्चान्त्य॑श्च भौव॒नश्च॒ भुव॑न॒श्चाधि॑पतिश्च॥२०॥

%4.7.12.0
{\anuvakamend[{त्रय॑स्त्रिꣳशच्च॒ व्यश्ञि॑य॒ एका॑दश च}]}%॥11॥

%4.7.12.1
वाजो॑ नः स॒प्त प्र॒दिश॒श्चत॑स्रो वा परा॒वतः॑। वाजो॑ नो॒ विश्वै᳚र्दे॒वैर्धन॑सातावि॒हाव॑तु। विश्वे॑ अ॒द्य म॒रुतो॒ विश्व॑ ऊ॒ती विश्वे॑ भवन्त्व॒ग्नयः॒ समि॑द्धाः। विश्वे॑ नो दे॒वा अव॒सा ग॑मन्तु॒ विश्व॑मस्तु॒ द्रवि॑णं॒ वाजो॑ अ॒स्मे। वाज॑स्य प्रस॒वं दे॑वा॒ रथै᳚र्याता हिर॒ण्ययैः᳚। अ॒ग्निरिन्द्रो॒ बृह॒स्पति॑र्म॒रुतः॒ सोम॑पीतये। वाजे॑वाजे\-ऽवत वाजिनो नो॒ धने॑षु॥२१॥

%4.7.12.2
वि॒प्रा॒ अ॒मृ॒ता॒ ऋ॒त॒ज्ञाः॒। अ॒स्य मध्वः॑ पिबत मा॒दय॑ध्वं तृ॒प्ता या॑त प॒थिभि॑र्देव॒यानैः᳚। वाजः॑ पु॒रस्ता॑दु॒त म॑ध्य॒तो नो॒ वाजो॑ दे॒वाꣳ ऋ॒तुभिः॑ कल्पयाति। वाज॑स्य॒ हि प्र॑स॒वो नन्न॑मीति॒ विश्वा॒ आशा॒ वाज॑पतिर्भवेयम्। पयः॑ पृथि॒व्याम्पय॒ ओष॑धीषु॒ पयो॑ दिव्य॒न्तरि॑क्षे॒ पयो॑ धाम्। पय॑स्वतीः प्र॒दिशः॑ सन्तु॒ मह्यम्᳚। सम्मा॑ सृजामि॒ पय॑सा घृ॒तेन॒ सम्मा॑ सृजाम्य॒पः॥२२॥

%4.7.12.3
ओष॑धीभिः। सो॑\-ऽहं वाजꣳ॑ सनेयमग्ने। नक्तो॒षासा॒ सम॑नसा॒ विरू॑पे धा॒पये॑ते॒ शिशु॒मेकꣳ॑ समी॒ची। द्यावा॒ क्षामा॑ रु॒क्मो अ॒न्तर्वि भा॑ति दे॒वा अ॒ग्निं धा॑रयन्द्रविणो॒दाः। स॒मु॒द्रो॑\-ऽसि॒ नभ॑स्वाना॒र्द्रदा॑नुः श॒म्भूर्म॑यो॒भूर॒भि मा॑ वाहि॒ स्वाहा॑ मारु॒तो॑\-ऽसि म॒रुतां᳚ ग॒णः श॒म्भूर्म॑यो॒भूर॒भि मा॑ वाहि॒ स्वाहा॑व॒स्युर॑सि॒ दुव॑स्वाञ्छ॒म्भूर्म॑यो॒भूर॒भि मा॑ वाहि॒ स्वाहा᳚॥२३॥

%4.7.13.0
{\anuvakamend[{धने᳚ष्व॒पो दुव॑स्वाञ्छ॒म्भूर्म॑यो॒भूर॒भि मा॒ द्वे च॑}]}%॥12॥

%4.7.13.1
अ॒ग्निं यु॑नज्मि॒ शव॑सा घृ॒तेन॑ दि॒व्यꣳ सु॑प॒र्णं वय॑सा बृ॒हन्तम्᳚। तेन॑ व॒यं प॑तेम ब्र॒ध्नस्य॑ वि॒ष्टप॒ꣳ॒ सुवो॒ रुहा॑णा॒ अधि॒ नाक॑ उत्त॒मे। इ॒मौ ते॑ प॒क्षाव॒जरौ॑ पत॒त्रिणो॒ याभ्या॒ꣳ॒ रक्षाꣳ॑स्यप॒हꣴस्य॑ग्ने। ता\-भ्यां᳚ पतेम सु॒कृता॑मु लो॒कं यत्रर्\mbox{}ष॑यः प्रथम॒जा ये पु॑रा॒णाः। चिद॑सि समु॒द्रयो॑नि॒रिन्दु॒र्दक्षः॑ श्ये॒न ऋ॒तावा᳚। हिर॑ण्यपक्षः शकु॒नो भु॑र॒ण्युर्म॒हान्थ्स॒धस्थे᳚ ध्रु॒वः॥२४॥

%4.7.13.2
आ निष॑त्तः। नम॑स्ते अस्तु॒ मा मा॑ हिꣳसी॒र्विश्व॑स्य मू॒र्धन्नधि॑ तिष्ठसि श्रि॒तः। स॒मु॒द्रे ते॒ हृद॑यम॒न्तरायु॒र्द्यावा॑पृथि॒वी भुव॑ने॒ष्वर्पि॑ते। उ॒द्नो द॑त्तोद॒धिम्भि॑न्त दि॒वः प॒र्जन्या॑द॒न्तरि॑क्षात्पृथि॒व्यास्ततो॑ नो॒ वृष्ट्या॑वत। दि॒वो मू॒र्धासि॑ पृथि॒व्या नाभि॒रूर्ग॒पामोष॑धीनाम्। वि॒श्वायुः॒ शर्म॑ स॒प्रथा॒ नम॑स्प॒थे। येनर्\mbox{}ष॑य॒स्तप॑सा स॒त्त्रम्॥२५॥

%4.7.13.3
आस॒तेन्धा॑ना अ॒ग्निꣳ सुव॑रा॒भर॑न्तः। तस्मि॑न्न॒हं नि द॑धे॒ नाके॑ अ॒ग्निमे॒तं यमा॒हुर्मन॑वः स्ती॒र्णब॑र्\mbox{}हिषम्। तम्पत्नी॑भि॒रनु॑ गच्छेम देवाः पु॒त्रैर्भ्रातृ॑भिरु॒त वा॒ हिर॑ण्यैः। नाकं॑ गृह्णा॒नाः सु॑कृ॒तस्य॑ लो॒के तृ॒तीये॑ पृ॒ष्ठे अधि॑ रोच॒ने दि॒वः। आ वा॒चो मध्य॑मरुहद्भुर॒ण्युर॒यम॒ग्निः सत्प॑ति॒श्चेकि॑तानः। पृ॒ष्ठे पृ॑थि॒व्या निहि॑तो॒ दवि॑द्युतदधस्प॒दं कृ॑णुते॥२६॥

%4.7.13.4
ये पृ॑त॒न्यवः॑। अ॒यम॒ग्निर्वी॒रत॑मो वयो॒धाः स॑ह॒स्रियो॑ दीप्यता॒मप्र॑युच्छन्न्। वि॒भ्राज॑मानः सरि॒रस्य॒ मध्य॒ उप॒ प्र या॑त दि॒व्यानि॒ धाम॑। सम्प्र च्य॑वध्व॒मनु॒ सम्प्र या॒ताग्ने॑ प॒थो दे॑व॒याना᳚न्कृणुध्वम्। अ॒स्मिन्थ्स॒धस्थे॒ अध्युत्त॑रस्मि॒न्विश्वे॑ देवा॒ यज॑मानश्च सीदत। येना॑ स॒हस्रं॒ वह॑सि॒ येना᳚ग्ने सर्ववेद॒सम्। तेने॒मं य॒ज्ञं नो॑ वह देव॒यानो॒ यः॥२७॥

%4.7.13.5
उ॒त्त॒मः। उद्बु॑ध्यस्वाग्ने॒ प्रति॑ जागृह्येनमिष्टापू॒र्ते सꣳसृ॑जेथाम॒यं च॑। पुनः॑ कृ॒ण्वꣴस्त्वा॑ पि॒तरं॒ युवा॑नम॒न्वाताꣳ॑सी॒त् त्वयि॒ तन्तु॑मे॒तम्। अ॒यं ते॒ योनि॑र्\mbox{}ऋ॒त्वियो॒ यतो॑ जा॒तो अरो॑चथाः। तं जा॒नन्न॑ग्न॒ आ रो॒हाथा॑ नो वर्धया र॒यिम्॥२८॥

%4.7.14.0
{\anuvakamend[{ध्रु॒वः स॒त्रं कृ॑णुते॒ यः स॒प्तत्रिꣳ॑शच्च}]}%॥13॥

%4.7.14.1
ममा᳚ग्ने॒ वर्चो॑ विह॒वेष्व॑स्तु व॒यं त्वेन्धा॑नास्त॒नुव॑म्पुषेम। मह्यं॑ नमन्ताम्प्र॒दिश॒श्चत॑स्र॒स्त्वयाध्य॑क्षेण॒ पृत॑ना जयेम। मम॑ दे॒वा वि॑ह॒वे स॑न्तु॒ सर्व॒ इन्द्रा॑वन्तो म॒रुतो॒ विष्णु॑र॒ग्निः। ममा॒न्तरि॑क्षमु॒रु गो॒पम॑स्तु॒ मह्यं॒ वातः॑ पवतां॒ कामे॑ अ॒स्मिन्न्। मयि॑ दे॒वा द्रवि॑ण॒मा य॑जन्ता॒म्मय्या॒शीर॑स्तु॒ मयि॑ दे॒वहू॑तिः। दैव्या॒ होता॑रा वनिषन्त॥२९॥

%4.7.14.2
पूर्वे\-ऽरि॑ष्टाः स्याम त॒नुवा॑ सु॒वीराः᳚। मह्यं॑ यजन्तु॒ मम॒ यानि॑ ह॒व्याकू॑तिः स॒त्या मन॑सो मे अस्तु। एनो॒ मा नि गां᳚ कत॒मच्च॒नाहं विश्वे॑ देवासो॒ अधि॑ वोचता मे। देवीः᳚ षडुर्वीरु॒रु णः॑ कृणोत॒ विश्वे॑ देवास इ॒ह वी॑रयध्वम्। मा हा᳚स्महि प्र॒जया॒ मा त॒नूभि॒र्मा र॑धाम द्विष॒ते सो॑म राजन्न्। अ॒ग्निर्म॒न्युम्प्र॑तिनु॒दन्पु॒रस्ता᳚त्॥३०॥

%4.7.14.3
अद॑ब्धो गो॒पाः परि॑ पाहि न॒स्त्वम्। प्र॒त्यञ्चो॑ यन्तु नि॒गुतः॒ पुन॒स्ते॑\-ऽमैषां᳚ चि॒त्तम्प्र॒बुधा॒ वि ने॑शत्। धा॒ता धा॑तृ॒णाम्भुव॑नस्य॒ यस्पति॑र्दे॒वꣳ स॑वि॒तार॑मभिमाति॒षाहम्᳚। इ॒मं य॒ज्ञम॒श्विनो॒भा बृह॒स्पति॑र्दे॒वाः पा᳚न्तु॒ यज॑मानं न्य॒र्थात्। उ॒रु॒व्यचा॑ नो महि॒षः शर्म॑ यꣳसद॒स्मिन् हवे॑ पुरुहू॒तः पु॑रु॒क्षु। स नः॑ प्र॒जायै॑ हर्यश्व मृड॒येन्द्र॒ मा॥३१॥

%4.7.14.4
नो॒ री॒रि॒षो॒ मा परा॑ दाः। ये नः॑ स॒पत्ना॒ अप॒ ते भ॑वन्त्विन्द्रा॒ग्निभ्या॒मव॑ बाधामहे॒ तान्। वस॑वो रु॒द्रा आ॑दि॒त्या उ॑परि॒स्पृश॑म्मो॒ग्रं चेत्ता॑रमधिरा॒जम॑क्रन्न्। अ॒र्वाञ्च॒मिन्द्र॑म॒मुतो॑ हवामहे॒ यो गो॒जिद्ध॑न॒जिद॑श्व॒जिद्यः। इ॒मं नो॑ य॒ज्ञं वि॑ह॒वे जु॑षस्वा॒स्य कु॑र्मो हरिवो मे॒दिनं॑ त्वा॥३२॥

%4.7.15.0
{\anuvakamend[{व॒नि॒ष॒न्त॒ पु॒रस्ता॒न्मा त्रिच॑त्वारिꣳशच्च}]}%॥14॥

%4.7.15.1
अ॒ग्नेर्म॑न्वे प्रथ॒मस्य॒ प्रचे॑तसो॒ यम्पाञ्च॑जन्यम्ब॒हवः॑ समि॒न्धते᳚। विश्व॑स्यां वि॒शि प्र॑विविशि॒वाꣳस॑मीमहे॒ स नो॑ मुञ्च॒त्वꣳह॑सः। यस्ये॒दं प्रा॒णन्नि॑मि॒षद्यदेज॑ति॒ यस्य॑ जा॒तं जन॑मानं च॒ केव॑लम्। स्तौम्य॒ग्निं ना॑थि॒तो जो॑हवीमि॒ स नो॑ मुञ्च॒त्वꣳह॑सः। इन्द्र॑स्य मन्ये प्रथ॒मस्य॒ प्रचे॑तसो वृत्र॒घ्नः स्तोमा॒ उप॒ मामु॒पागुः॑। यो दा॒शुषः॑ सु॒कृतो॒ हव॒मुप॒ गन्ता᳚॥३३॥

%4.7.15.2
स नो॑ मुञ्च॒त्वꣳह॑सः। यः सं॑ग्रा॒मं नय॑ति॒ सं व॒शी यु॒धे यः पु॒ष्टानि॑ सꣳसृ॒जति॑ त्र॒याणि॑। स्तौमीन्द्रं॑ नाथि॒तो जो॑हवीमि॒ स नो॑ मुञ्च॒त्वꣳह॑सः। म॒न्वे वा᳚म्मित्रावरुणा॒ तस्य॑ वित्त॒ꣳ॒ सत्यौ॑जसा दृꣳहणा॒ यं नु॒देथे᳚। या राजा॑नꣳ स॒रथं॑ या॒थ उ॑ग्रा॒ ता नो॑ मुञ्चत॒माग॑सः। यो वा॒ꣳ॒ रथ॑ ऋ॒जुर॑श्मिः स॒त्यध॑र्मा॒ मिथु॒श्चर॑न्तमुप॒याति॑ दू॒षयन्न्॑। स्तौमि॑॥३४॥

%4.7.15.3
मि॒त्रावरु॑णा नाथि॒तो जो॑हवीमि॒ तौ नो॑ मुञ्चत॒माग॑सः। वा॒योः स॑वि॒तुर्वि॒दथा॑नि मन्महे॒ यावा᳚त्म॒न्वद्बि॑भृ॒तो यौ च॒ रक्ष॑तः। यौ विश्व॑स्य परि॒भू ब॑भू॒वतु॒स्तौ नो॑ मुञ्चत॒माग॑सः। उप॒ श्रेष्ठा॑ न आ॒शिषो॑ दे॒वयो॒र्धर्मे॑ अस्थिरन्न्। स्तौमि॑ वा॒युꣳ स॑वि॒तारं॑ नाथि॒तो जो॑हवीमि॒ तौ नो॑ मुञ्चत॒माग॑सः। र॒थीत॑मौ रथी॒नाम॑ह्व ऊ॒तये॒ शुभं॒ गमि॑ष्ठौ सु॒यमे॑भि॒रश्वैः᳚। ययोः᳚॥३५॥

%4.7.15.4
वां॒ दे॒वौ॒ दे॒वेष्वनि॑शित॒मोज॒स्तौ नो॑ मुञ्चत॒माग॑सः। यदया॑तं वह॒तुꣳ सू॒र्याया᳚स्त्रिच॒क्रेण॑ स॒ꣳ॒सद॑मि॒च्छमा॑नौ। स्तौमि॑ दे॒वाव॒श्विनौ॑ नाथि॒तो जो॑हवीमि॒ तौ नो॑ मुञ्चत॒माग॑सः। म॒रुता᳚म्मन्वे॒ अधि॑ नो ब्रुवन्तु॒ प्रेमां वाचं॒ विश्वा॑मवन्तु॒ विश्वे᳚। आ॒शून् हु॑वे सु॒यमा॑नू॒तये॒ ते नो॑ मुञ्च॒न्त्वेन॑सः। ति॒ग्ममायु॑धं वीडि॒तꣳ सह॑स्वद्दि॒व्यꣳ शर्धः॑॥३६॥

%4.7.15.5
पृत॑नासु जि॒ष्णु। स्तौमि॑ दे॒वान्म॒रुतो॑ नाथि॒तो जो॑हवीमि॒ ते नो॑ मुञ्च॒न्त्वेन॑सः। दे॒वाना᳚म्मन्वे॒ अधि॑ नो ब्रुवन्तु॒ प्रेमां वाचं॒ विश्वा॑मवन्तु॒ विश्वे᳚। आ॒शून् हु॑वे सु॒यमा॑नू॒तये॒ ते नो॑ मुञ्च॒न्त्वेन॑सः। यदि॒दम्मा॑भि॒शोच॑ति॒ पौरु॑षेयेण॒ दैव्ये॑न। स्तौमि॒ विश्वां᳚ दे॒वान्ना॑थि॒तो जो॑हवीमि॒ ते नो॑ मुञ्च॒न्त्वेन॑सः। अनु॑ नो॒\-ऽद्यानु॑मति॒रनु॑॥३७॥

%4.7.15.6
इद॑नुमते॒ त्वं वै᳚श्वान॒रो न॑ ऊ॒त्या पृ॒ष्टो दि॒वि। ये अप्र॑थेता॒ममि॑तेभि॒रोजो॑भि॒र्ये प्र॑ति॒ष्ठे अभ॑वतां॒ वसू॑नाम्। स्तौमि॒ द्यावा॑पृथि॒वी ना॑थि॒तो जो॑हवीमि॒ ते नो॑ मुञ्चत॒मꣳह॑सः। उर्वी॑ रोदसी॒ वरि॑वः कृणोतं॒ क्षेत्र॑स्य पत्नी॒ अधि॑ नो ब्रूयातम्। स्तौमि॒ द्यावा॑पृथि॒वी ना॑थि॒तो जो॑हवीमि॒ ते नो॑ मुञ्चत॒मꣳह॑सः। यत्ते॑ व॒यं पु॑रुष॒त्रा य॑वि॒ष्ठावि॑द्वाꣳसश्चकृ॒मा कच्च॒न॥३८॥

%4.7.15.7
आगः॑। कृ॒धी स्व॑स्माꣳ अदि॑ते॒रना॑गा॒ व्येनाꣳ॑सि शिश्रथो॒ विष्व॑गग्ने। यथा॑ ह॒ तद्व॑सवो गौ॒र्यं॑ चित्प॒दि षि॒ताममु॑ञ्चता यजत्राः। ए॒वा त्वम॒स्मत्प्र मु॑ञ्चा॒ व्यꣳहः॒ प्राता᳚र्यग्ने प्रत॒रां न॒ आयुः॑॥३९॥

%5.1.0.0

%5.1.0.0
{\anuvakamend[{गन्ता॑ दू॒षय॒न्थ्स्तौमि॒ ययोः॒ शर्धो\-ऽनु॑मति॒रनु॑ च॒न चतु॑स्त्रिꣳशच्च}]}%॥15॥

{\anuvakamend[{अ॒ग्निष्ट्वा॑ वा॒मश्वो॒ द्विच॑त्वारिꣳशच्च}]}%॥11॥
%%% END KANDAM

\chapt{काण्डम् ५}
\sect{प्रथमः प्रश्नः}\setcounter{anuvakam}{0}
\dnsub{तैत्तिरीयसंहितायां पञ्चमकाण्डे प्रथमः प्रश्नः}
%5.1.1.0
%5.1.1.1
सा॒वि॒त्राणि॑ जुहोति॒ प्रसू᳚त्यै चतुर्गृही॒तेन॑ जुहोति॒ चतु॑ष्पादः प॒शवः॑ प॒शूने॒वाव॑ रुन्द्धे॒ चत॑स्रो॒ दिशो॑ दि॒क्ष्वे॑व प्रति॑ तिष्ठति॒ छन्दाꣳ॑सि दे॒वेभ्यो\-ऽपा᳚क्राम॒न्न वो॑\-ऽभा॒गानि॑ ह॒व्यं व॑क्ष्याम॒ इति॒ तेभ्य॑ ए॒तच्च॑तुर्गृही॒तम॑धारयन् पुरोनुवा॒क्या॑यै या॒ज्या॑यै दे॒वता॑यै वषट्का॒राय॒ यच्च॑तुर्गृही॒तं जु॒होति॒ छन्दाꣳ॑स्ये॒व तत्प्री॑णाति॒ तान्य॑स्य प्री॒तानि॑ दे॒वेभ्यो॑ ह॒व्यं व॑हन्ति॒ यं का॒मये॑त॥१॥

%5.1.1.2
पापी॑यान्थ्स्या॒दित्येकै॑कं॒ तस्य॑ जुहुया॒दाहु॑तीभिरे॒वैन॒मप॑ गृह्णाति॒ पापी॑यान्भवति॒ यं का॒मये॑त॒ वसी॑यान्थ्स्या॒दिति॒ सर्वा॑णि॒ तस्या॑नु॒द्रुत्य॑ जुहुया॒दाहु॑त्यै॒वैन॑म॒भि क्र॑मयति॒ वसी॑यान्भव॒त्यथो॑ य॒ज्ञस्यै॒वैषाभिक्रा᳚न्ति॒रेति॒ वा ए॒ष य॑ज्ञमु॒खादृद्ध्या॒ यो᳚\-ऽग्नेर्दे॒वता॑या॒ एत्य॒ष्टावे॒तानि॑ सावि॒त्राणि॑ भवन्त्य॒ष्टाक्ष॑रा गाय॒त्री गा॑य॒त्रः॥२॥

%5.1.1.3
अ॒ग्निस्तेनै॒व य॑ज्ञमु॒खादृद्ध्या॑ अ॒ग्नेर्दे॒वता॑यै॒ नैत्य॒ष्टौ सा॑वि॒त्राणि॑ भव॒न्त्याहु॑तिर्नव॒मी त्रि॒वृत॑मे॒व य॑ज्ञमु॒खे वि या॑तयति॒ यदि॑ का॒मये॑त॒ छन्दाꣳ॑सि यज्ञयश॒सेना᳚र्पयेय॒मित्यृच॑मन्त॒मां कु॑र्या॒च्छन्दाꣳ॑स्ये॒व य॑ज्ञयश॒सेना᳚र्पयति॒ यदि॑ का॒मये॑त॒ यज॑मानं यज्ञयश॒सेना᳚र्पयेय॒मिति॒ यजु॑रन्त॒मं कु॑र्या॒द्यज॑मानमे॒व य॑ज्ञयश॒सेना᳚र्पयत्यृ॒चा स्तोम॒ꣳ॒ सम॑र्ध॒येति॑॥३॥

%5.1.1.4
आ॒ह॒ समृ॑द्ध्यै च॒तुर्भि॒रभ्रि॒मा द॑त्ते च॒त्वारि॒ छन्दाꣳ॑सि॒ छन्दो॑भिरे॒व दे॒वस्य॑ त्वा सवि॒तुः प्र॑स॒व इत्या॑ह॒ प्रसू᳚त्या अ॒ग्निर्दे॒वेभ्यो॒ निला॑यत॒ स वेणु॒म्प्रावि॑श॒थ्स ए॒तामू॒तिमनु॒ सम॑चर॒द्यद्वेणोः᳚ सुषि॒रꣳ सु॑षि॒राभ्रि॑र्भवति सयोनि॒त्वाय॒ स यत्र॑य॒त्राव॑स॒त्तत्कृ॒ष्णम॑भवत्कल्मा॒षी भ॑वति रू॒पस॑मृद्ध्या उभयतः॒क्ष्णूर्भ॑वती॒तश्चा॒मुत॑श्चा॒र्कस्याव॑रुद्ध्यै व्याममा॒त्री भ॑वत्ये॒ताव॒द्वै पुरु॑षे वी॒र्यं॑ वी॒र्य॑सम्मि॒ता\-ऽप॑रिमिता भव॒त्यप॑रिमित॒स्याव॑रुद्ध्यै॒ यो वन॒स्पती॑नाम्फल॒ग्रहिः॒ स ए॑षां वी॒र्या॑वान्फल॒ग्रहि॒र्वेणु॑र्वैण॒वी भ॑वति वी॒र्य॑स्याव॑रुद्ध्यै॥४॥

%5.1.2.0
{\anuvakamend[{का॒मये॑त गाय॒त्रो᳚\-ऽर्ध॒येति॑ च स॒प्तविꣳ॑शतिश्च}]}%॥१॥

%5.1.2.1
व्यृ॑द्धं॒ वा ए॒तद्य॒ज्ञस्य॒ यद॑य॒जुष्के॑ण क्रि॒यत॑ इ॒माम॑गृभ्णन्रश॒नामृ॒तस्येत्य॑श्वाभि॒धानी॒मा द॑त्ते॒ यजु॑ष्कृत्यै य॒ज्ञस्य॒ समृ॑द्ध्यै॒ प्रतू᳚र्तं वाजि॒न्ना द्र॒वेत्यश्व॑म॒भि द॑धाति रू॒पमे॒वास्यै॒तन्म॑हि॒मानं॒ व्याच॑ष्टे यु॒ञ्जाथा॒ꣳ॒ रास॑भं यु॒वमिति॑ गर्द॒भमस॑त्ये॒व ग॑र्द॒भं प्रति॑ ष्ठापयति॒ तस्मा॒दश्वा᳚द्गर्द॒भो\-ऽस॑त्तरो॒ योगे॑योगे त॒वस्त॑र॒मित्या॑ह॥५॥

%5.1.2.2
योगे॑योग ए॒वैनं॑ युङ्क्ते॒ वाजे॑वाजे हवामह॒ इत्या॒हान्नं॒ वै वाजो\-ऽन्न॑मे॒वाव॑ रुन्द्धे॒ सखा॑य॒ इन्द्र॑मू॒तय॒ इत्या॑हेन्द्रि॒यमे॒वाव॑ रुन्द्धे॒\-ऽग्निर्दे॒वेभ्यो॒ निला॑यत॒ तं प्र॒जाप॑ति॒रन्व॑विन्दत्प्राजाप॒त्यो\-ऽश्वो\-ऽश्वे॑न॒ सम्भ॑र॒त्यनु॑वित्त्यै पापवस्य॒सं वा ए॒तत्क्रि॑यते॒ यच्छ्रेय॑सा च॒ पापी॑यसा च समा॒नं कर्म॑ कु॒र्वन्ति॒ पापी॑यान्॥६॥

%5.1.2.3
ह्यश्वा᳚द्गर्द॒भो\-ऽश्व॒म्पूर्वं॑ नयन्ति पापवस्य॒सस्य॒ व्यावृ॑त्त्यै॒ तस्मा॒च्छ्रेयाꣳ॑स॒म्पापी॑यान्प॒श्चादन्वे॑ति ब॒हुर्वै भव॑तो॒ भ्रातृ॑व्यो॒ भव॑तीव॒ खलु॒ वा ए॒ष यो᳚\-ऽग्निञ्चि॑नु॒ते व॒ज्र्यश्वः॑ प्र॒तूर्व॒न्नेह्य॑व॒क्राम॒न्नश॑स्ती॒रित्या॑ह॒ वज्रे॑णै॒व पा॒प्मान॒म्भ्रातृ॑व्य॒मव॑ क्रामति रु॒द्रस्य॒ गाण॑पत्या॒दित्या॑ह रौ॒द्रा वै प॒शवो॑ रु॒द्रादे॒व॥७॥

%5.1.2.4
प॒शून्नि॒र्याच्या॒त्मने॒ कर्म॑ कुरुते पू॒ष्णा स॒युजा॑ स॒हेत्या॑ह पू॒षा वा अध्व॑नाꣳ सन्ने॒ता सम॑ष्ट्यै॒ पुरी॑षायतनो॒ वा ए॒ष यद॒ग्निरङ्गि॑रसो॒ वा ए॒तमग्रे॑ दे॒वता॑ना॒ꣳ॒ सम॑भरन्पृथि॒व्याः स॒धस्था॑द॒ग्निम्पु॑री॒ष्य॑मङ्गिर॒स्वदच्छे॒हीत्या॑ह॒ साय॑तनमे॒वैनं॑ दे॒वता॑भिः॒ सम्भ॑रत्य॒ग्निम्पु॑री॒ष्य॑मङ्गिर॒स्वदच्छे॑म॒ इत्या॑ह येन॑॥८॥

%5.1.2.5
सं॒गच्छ॑ते॒ वाज॑मे॒वास्य॑ वृङ्क्ते प्र॒जाप॑तये प्रति॒प्रोच्या॒ग्निः स॒म्भृत्य॒ इत्या॑हुरि॒यं वै प्र॒जाप॑ति॒स्तस्या॑ ए॒तच्छ्रोत्रं॒ यद्व॒ल्मीको॒\-ऽग्निम्पु॑री॒ष्य॑मङ्गिर॒स्वद्भ॑रिष्याम॒ इति॑ वल्मीकव॒पामुप॑ तिष्ठते सा॒क्षादे॒व प्र॒जाप॑तये प्रति॒प्रोच्या॒ग्निꣳ सम्भ॑रत्य॒ग्निम्पु॑री॒ष्य॑मङ्गिर॒स्वद्भ॑राम॒ इत्या॑ह॒ येन॑ सं॒गच्छ॑ते॒ वाज॑मे॒वास्य॑ वृ॒ङ्क्ते\-ऽन्व॒ग्निरु॒षसा॒मग्रम्᳚॥९॥

%5.1.2.6
अ॒ख्य॒दित्या॒हानु॑ख्यात्या आ॒गत्य॑ वा॒ज्यध्व॑न आ॒क्रम्य॑ वाजिन्पृथि॒वीमित्या॑हे॒च्छत्ये॒वैन॒म्पूर्व॑या वि॒न्दत्युत्त॑रया॒ द्वाभ्या॒मा क्र॑मयति॒ प्रति॑ष्ठित्या॒ अनु॑रूपाभ्या॒न्तस्मा॒दनु॑रूपाः प॒शवः॒ प्र जा॑यन्ते॒ द्यौस्ते॑ पृ॒ष्ठं पृ॑थि॒वी स॒धस्थ॒मित्या॑है॒भ्यो वा ए॒तं लो॒केभ्यः॑ प्र॒जाप॑तिः॒ समै॑रयद्रू॒पमे॒वास्यै॒तन्म॑हि॒मानं॒ व्याच॑ष्टे व॒ज्री वा ए॒ष यदश्वो॑ द॒द्भिर॒न्यतो॑दद्भ्यो॒ भूया॒ल्लोँम॑भिरुभ॒याद॑द्भ्यो॒ यं द्वि॒ष्यात्तम॑धस्प॒दं ध्या॑ये॒द्वज्रे॑णै॒वैनꣴ॑ स्तृणुते॥10॥

%5.1.3.0
{\anuvakamend[{आ॒ह॒ पापी॑यान्रु॒द्रादे॒व येनाग्रं॑ व॒ज्री वै स॒प्तद॑श च}]}%॥२॥

%5.1.3.1
उत्क्रा॒मोद॑क्रमी॒दिति॒ द्वाभ्या॒मुत्क्र॑मयति॒ प्रति॑ष्ठित्या॒ अनु॑रूपाभ्या॒न्तस्मा॒दनु॑रूपाः प॒शवः॒ प्र जा॑यन्ते॒\-ऽप उप॑ सृजति॒ यत्र॒ वा आप॑ उप॒गच्छ॑न्ति॒ तदोष॑धयः॒ प्रति॑ तिष्ठ॒न्त्योष॑धीः प्रति॒तिष्ठ॑न्तीः प॒शवो\-ऽनु॒ प्रति॑ तिष्ठन्ति प॒शून् य॒ज्ञो य॒ज्ञं यज॑मानो॒ यज॑मानं प्र॒जास्तस्मा॑द॒प उप॑ सृजति॒ प्रति॑ष्ठित्यै॒ यद॑ध्व॒र्युर॑न॒ग्नावाहु॑तिं जुहु॒याद॒न्धो᳚\-ऽध्व॒र्युः॥११॥

%5.1.3.2
स्या॒द्रक्षाꣳ॑सि य॒ज्ञꣳ ह॑न्यु॒र्\mbox{}हिर॑ण्यमु॒पास्य॑ जुहोत्यग्नि॒वत्ये॒व जु॑होति॒ नान्धो᳚\-ऽध्व॒र्युर्भव॑ति॒ न य॒ज्ञꣳ रक्षाꣳ॑सि घ्नन्ति॒ जिघ॑र्म्य॒ग्निम्मन॑सा घृ॒तेनेत्या॑ह॒ मन॑सा॒ हि पुरु॑षो य॒ज्ञम॑भि॒गच्छ॑ति प्रति॒क्ष्यन्त॒म्भुव॑नानि॒ विश्वेत्या॑ह॒ सर्व॒ꣴ॒ ह्ये॑ष प्र॒त्यङ्क्षेति॑ पृ॒थुं ति॑र॒श्चा वय॑सा बृ॒हन्त॒मित्या॒हाल्पो॒ ह्ये॑ष जा॒तो म॒हान्॥१२॥

%5.1.3.3
भव॑ति॒ व्यचि॑ष्ठ॒मन्नꣳ॑ रभ॒सं विदा॑न॒मित्या॒हान्न॑मे॒वास्मै᳚ स्वदयति॒ सर्व॑मस्मै स्वदते॒ य ए॒वं वेदा त्वा॑ जिघर्मि॒ वच॑सा घृ॒तेनेत्या॑ह॒ तस्मा॒द्यत्पुरु॑षो॒ मन॑साभि॒गच्छ॑ति॒ तद्वा॒चा व॑दत्यर॒क्षसेत्या॑ह॒ रक्ष॑सा॒मप॑हत्यै॒ मर्य॑श्रीः स्पृह॒यद्व॑र्णो अ॒ग्निरित्या॒हाप॑चितिमे॒वास्मि॑न्दधा॒त्यप॑चितिमान्भवति॒ य ए॒वं॥१३॥

%5.1.3.4
वेद॒ मन॑सा॒ त्वै तामाप्तु॑मर्\mbox{}हति॒ याम॑ध्व॒र्युर॑न॒ग्नावाहु॑तिं जु॒होति॒ मन॑स्वतीभ्यां जुहो॒त्याहु॑त्यो॒राप्त्यै॒ द्वाभ्यां॒ प्रति॑ष्ठित्यै यज्ञमु॒खेय॑ज्ञमुखे॒ वै क्रि॒यमा॑णे य॒ज्ञꣳ रक्षाꣳ॑सि जिघाꣳसन्त्ये॒तर्\mbox{}हि॒ खलु॒ वा ए॒तद्य॑ज्ञमु॒खं यर्\mbox{}ह्ये॑न॒दाहु॑तिरश्ञु॒ते परि॑ लिखति॒ रक्ष॑सा॒मप॑हत्यै ति॒सृभिः॒ परि॑ लिखति त्रि॒वृद्वा अ॒ग्निर्यावा॑ने॒वाग्निस्तस्मा॒द्रक्षा॒ꣳ॒स्यप॑ हन्ति॥१४॥

%5.1.3.5
गा॒य॒त्रि॒या परि॑ लिखति॒ तेजो॒ वै गा॑य॒त्री तेज॑सै॒वैनं॒ परि॑ गृह्णाति त्रि॒ष्टुभा॒ परि॑ लिखतीन्द्रि॒यं वै त्रि॒ष्टुगि॑न्द्रि॒येणै॒वैन॒म् परि॑ गृह्णात्यनु॒ष्टुभा॒ परि॑ लिखत्यनु॒ष्टुफ्सर्वा॑णि॒ छन्दाꣳ॑सि परि॒भूः पर्या᳚प्त्यै मध्य॒तो॑\-ऽनु॒ष्टुभा॒ वाग्वा अ॑नु॒ष्टुप्तस्मा᳚न्मध्य॒तो वा॒चा व॑दामो गायत्रि॒या प्र॑थ॒मया॒ परि॑ लिख॒त्यथा॑नु॒ष्टुभाथ॑ त्रि॒ष्टुभा॒ तेजो॒ वै गा॑य॒त्री य॒ज्ञो॑\-ऽनु॒ष्टुगि॑न्द्रि॒यं त्रि॒ष्टुप्तेज॑सा चै॒वेन्द्रि॒येण॑ चोभ॒यतो॑ य॒ज्ञं परि॑ गृह्णाति॥१५॥

%5.1.4.0
{\anuvakamend[{अ॒न्धो᳚\-ऽध्व॒र्युर्म॒हान्भ॑वति त्रि॒ष्टुभा॒ तेजो॒ वै गा॑य॒त्री त्रयो॑दश च}]}%॥३॥

%5.1.4.1
दे॒वस्य॑ त्वा सवि॒तुः प्र॑स॒व इति॑ खनति॒ प्रसू᳚त्या॒ अथो॑ धू॒ममे॒वैतेन॑ जनयति॒ ज्योति॑ष्मन्तं त्वाग्ने सु॒प्रती॑क॒मित्या॑ह॒ ज्योति॑रे॒वैतेन॑ जनयति॒ सो᳚\-ऽग्निर्जा॒तः प्र॒जाः शु॒चार्प॑य॒त्तं दे॒वा अ॑र्ध॒र्चेना॑शमयञ्छि॒वं प्र॒जाभ्यो\-ऽहिꣳ॑सन्त॒मित्या॑ह प्र॒जाभ्य॑ ए॒वैनꣳ॑ शमयति॒ द्वा\-भ्यां᳚ खनति॒ प्रति॑ष्ठित्या अ॒पां पृ॒ष्ठम॒सीति॑ पुष्करप॒र्णमा॥१६॥

%5.1.4.2
ह॒र॒त्य॒पां वा ए॒तत्पृ॒ष्ठं यत्पु॑ष्करप॒र्णꣳ रू॒पेणै॒वैन॒दा ह॑रति पुष्करप॒र्णेन॒ सम्भ॑रति॒ योनि॒र्वा अ॒ग्नेः पु॑ष्करप॒र्णꣳ सयो॑निमे॒वाग्निꣳ सम्भ॑रति कृष्णाजि॒नेन॒ सम्भ॑रति य॒ज्ञो वै कृ॑ष्णाजि॒नं य॒ज्ञेनै॒व य॒ज्ञꣳ सम्भ॑रति॒ यद्ग्रा॒म्याणां᳚ पशू॒नां चर्म॑णा स॒म्भरे᳚द्ग्रा॒म्यान्प॒शूञ्छु॒चार्प॑येत्कृष्णाजि॒नेन॒ सम्भ॑रत्यार॒ण्याने॒व प॒शून्॥१७॥

%5.1.4.3
शु॒चार्प॑यति॒ तस्मा᳚थ्स॒माव॑त्पशू॒नां प्र॒जाय॑मानानामार॒ण्याः प॒शवः॒ कनी॑याꣳसः शु॒चा ह्यृ॑ता लो॑म॒तः सम्भ॑र॒त्यतो॒ ह्य॑स्य॒ मेध्यं॑ कृष्णाजि॒नं च॑ पुष्करप॒र्णं च॒ सꣴ स्तृ॑णाती॒यं वै कृ॑ष्णाजि॒नम॒सौ पु॑ष्करप॒र्णमा॒भ्यामे॒वैन॑मुभ॒यतः॒ परि॑ गृह्णात्य॒ग्निर्दे॒वेभ्यो॒ निला॑यत॒ तमथ॒र्वान्व॑पश्य॒दथ॑र्वा त्वा प्रथ॒मो निर॑मन्थदग्न॒ इति॑॥१८॥

%5.1.4.4
आ॒ह॒ य ए॒वैन॑म॒न्वप॑श्य॒त्तेनै॒वैन॒ꣳ॒ सम्भ॑रति॒ त्वाम॑ग्ने॒ पुष्क॑रा॒दधीत्या॑ह पुष्करप॒र्णे ह्ये॑न॒मुप॑श्रित॒मवि॑न्द॒त्तमु॑ त्वा द॒ध्यङ्ङृषि॒रित्या॑ह द॒ध्यङ्वा आ॑थर्व॒णस्ते॑ज॒स्व्या॑सी॒त्तेज॑ ए॒वास्मि॑न्दधाति॒ तमु॑ त्वा पा॒थ्यो वृषेत्या॑ह॒ पूर्व॑मे॒वोदि॒तमुत्त॑रेणा॒भि गृ॑णाति॥१९॥

%5.1.4.5
च॒त॒सृभिः॒ सम्भ॑रति च॒त्वारि॒ छन्दाꣳ॑सि॒ छन्दो॑भिरे॒व गा॑य॒त्रीभि॑र्ब्राह्म॒णस्य॑ गाय॒त्रो हि ब्रा᳚ह्म॒णस्त्रि॒॒ष्टुग्भी॑ राज॒न्य॑स्य॒ त्रैष्टु॑भो॒ हि रा॑ज॒न्यो॑ यं का॒मये॑त॒ वसी॑यान्थ्स्या॒दित्यु॒भयी॑भि॒स्तस्य॒ सम्भ॑रे॒त्तेज॑श्चै॒वास्मा॑ इन्द्रि॒यं च॑ स॒मीची॑ दधात्यष्टा॒भिः सम्भ॑रत्य॒ष्टाक्ष॑रा गाय॒त्री गा॑य॒त्रो᳚\-ऽग्निर्यावा॑ने॒वाग्निस्तꣳ सम्भ॑रति॒ सीद॑ होत॒रित्या॑ह दे॒वता॑ ए॒वास्मै॒ सꣳ सा॑दयति॒ नि होतेति॑ मनु॒ष्या᳚न्थ्सꣳ सी॑द॒स्वेति॒ वयाꣳ॑सि॒ जनि॑ष्वा॒ हि जेन्यो॒ अग्रे॒ अह्ना॒मित्या॑ह देवमनु॒ष्याने॒वास्मै॒ सꣳस॑न्ना॒न्प्र ज॑नयति॥२०॥

%5.1.5.0
{\anuvakamend[{ऐव प॒शूनिति॑ गृणाति होत॒रिति॑ स॒प्तविꣳ॑शतिश्च}]}%॥४॥

%5.1.5.1
क्रू॒रमि॑व॒ वा अ॑स्या ए॒तत्क॑रोति॒ यत्खन॑त्य॒प उप॑ सृज॒त्यापो॒ वै शा॒न्ताः शा॒न्ताभि॑रे॒वास्यै॒ शुचꣳ॑ शमयति॒ सं ते॑ वा॒युर्मा॑त॒रिश्वा॑ दधा॒त्वित्या॑ह प्रा॒णो वै वा॒युः प्रा॒णेनै॒वास्यै᳚ प्रा॒णꣳ सं द॑धाति॒ सं ते॑ वा॒युरित्या॑ह॒ तस्मा᳚द्वा॒युप्र॑च्युता दि॒वो वृ॑ष्टिरीर्ते॒ तस्मै॑ च देवि॒ वष॑डस्तु॥२१॥

%5.1.5.2
तुभ्य॒मित्या॑ह॒ षड्वा ऋ॒तव॑ ऋ॒तुष्वे॒व वृष्टिं॑ दधाति॒ तस्मा॒थ्सर्वा॑नृ॒तून् व॑र्\mbox{}षति॒ यद्व॑षट्कु॒र्याद्रक्षाꣳ॑सि य॒ज्ञꣳ ह॑न्यु॒र्वडित्या॑ह प॒रोक्ष॑मे॒व वष॑ट्करोति॒ नास्य॑ या॒तया॑मा वषट्का॒रो भव॑ति॒ न य॒ज्ञꣳ रक्षाꣳ॑सि घ्नन्ति॒ सुजा॑तो॒ ज्योति॑षा स॒हेत्य॑नु॒ष्टुभोप॑ नह्यत्यनु॒ष्टुप्॥२२॥

%5.1.5.3
सर्वा॑णि॒ छन्दाꣳ॑सि॒ छन्दाꣳ॑सि॒ खलु॒ वा अ॒ग्नेः प्रि॒या त॒नूः प्रि॒ययै॒वैनं॑ त॒नुवा॒ परि॑ दधाति॒ वेदु॑को॒ वासो॑ भवति॒ य ए॒वं वेद॑ वारु॒णो वा अ॒ग्निरुप॑नद्ध॒ उदु॑ तिष्ठ स्वध्वरो॒र्ध्व ऊ॒ षु ण॑ ऊ॒तय॒ इति॑ सावि॒त्रीभ्या॒मुत्ति॑ष्ठति सवि॒तृप्र॑सूत ए॒वास्यो॒र्ध्वां व॑रुणमे॒निमुथ्सृ॑जति॒ द्वाभ्यां॒ प्रति॑ष्ठित्यै॒ स जा॒तो गर्भो॑ असि॥२३॥

%5.1.5.4
रोद॑स्यो॒रित्या॑हे॒मे वै रोद॑सी॒ तयो॑रे॒ष गर्भो॒ यद॒ग्निस्तस्मा॑दे॒वमा॒हाग्ने॒ चारु॒र्विभृ॑त॒ ओष॑धी॒ष्वित्या॑ह य॒दा ह्ये॑तं वि॒भर॒न्त्यथ॒ चारु॑तरो॒ भव॑ति॒ प्र मा॒तृभ्यो॒ अधि॒ कनि॑क्रदद्गा॒ इत्या॒हौष॑धयो॒ वा अ॑स्य मा॒तर॒स्ताभ्य॑ ए॒वैन॒म्प्र च्या॑वयति स्थि॒रो भ॑व वी॒ड्व॑ङ्ग॒ इति॑ गर्द॒भ आ सा॑दयति॥२४॥

%5.1.5.5
सं न॑ह्यत्ये॒वैन॑मे॒तया᳚ स्थे॒म्ने ग॑र्द॒भेन॒ सम्भ॑रति॒ तस्मा᳚द्गर्द॒भः प॑शू॒नाम्भा॑रभा॒रित॑मो गर्द॒भेन॒ सम्भ॑रति॒ तस्मा᳚द्गर्द॒भो\-ऽप्य॑नाले॒शेत्य॒न्यान्प॒शून्मे᳚द्य॒त्यन्न॒ꣴ॒ ह्ये॑नेना॒र्कꣳ स॒म्भर॑न्ति गर्द॒भेन॒ सम्भ॑रति॒ तस्मा᳚द्गर्द॒भो द्वि॒रेताः॒ सन्कनि॑ष्ठम्पशू॒नाम्प्र जा॑यते॒\-ऽग्निर्ह्य॑स्य॒ योनिं॑ नि॒र्दह॑ति प्र॒जासु॒ वा ए॒ष ए॒तर्\mbox{}ह्यारू॑ढः॥२५॥

%5.1.5.6
स ई᳚श्व॒रः प्र॒जाः शु॒चा प्र॒दहः॑ शि॒वो भ॑व प्र॒जाभ्य॒ इत्या॑ह प्र॒जाभ्य॑ ए॒वैनꣳ॑ शमयति॒ मानु॑षीभ्य॒स्त्वम॑ङ्गिर॒ इत्या॑ह मान॒व्यो॑ हि प्र॒जा मा द्यावा॑पृथि॒वी अ॒भि शू॑शुचो॒ मान्तरि॑क्षं॒ मा वन॒स्पती॒नित्या॑है॒भ्य ए॒वैनं॑ लो॒केभ्यः॑ शमयति॒ प्रैतु॑ वा॒जी कनि॑क्रद॒दित्या॑ह वा॒जी ह्ये॑ष नान॑द॒द्रास॑भः॒ पत्वेति॑॥२६॥

%5.1.5.7
आ॒ह॒ रास॑भ॒ इति॒ ह्ये॑तमृष॒यो\-ऽव॑द॒न्भर॑न्न॒ग्निम्पु॑री॒ष्य॑मित्या॑हा॒ग्निꣴ ह्ये॑ष भर॑ति॒ मा पा॒द्यायु॑षः पु॒रेत्या॒हायु॑रे॒वास्मि॑न्दधाति॒ तस्मा᳚द्गर्द॒भः सर्व॒मायु॑रेति॒ तस्मा᳚द्गर्द॒भे पु॒रायु॑षः॒ प्रमी॑ते बिभ्यति॒ वृषा॒ग्निं वृष॑ण॒म्भर॒न्नित्या॑ह वृषा॒ ह्ये॑ष वृषा॒ग्निर॒पां गर्भम्᳚॥२७॥

%5.1.5.8
स॒मु॒द्रिय॒मित्या॑हा॒पाꣳ ह्ये॑ष गर्भो॒ यद॒ग्निरग्न॒ आ या॑हि वी॒तय॒ इति॒ वा इ॒मौ लो॒कौ व्यै॑ता॒मग्न॒ आ या॑हि वी॒तय॒ इति॒ यदाहा॒नयो᳚र्लो॒कयो॒र्वीत्यै॒ प्रच्यु॑तो॒ वा ए॒ष आ॒यत॑ना॒दग॑तः प्रति॒ष्ठाꣳ स ए॒तर्\mbox{}ह्य॑ध्व॒र्युं च॒ यज॑मानं च ध्यायत्यृ॒तꣳ स॒त्यमित्या॑हे॒यं वा ऋ॒तम॒सौ॥२८॥

%5.1.5.9
स॒त्यम॒नयो॑रे॒वैनं॒ प्रति॑ ष्ठापयति॒ नार्ति॒मार्च्छ॑त्यध्व॒र्युर्न यज॑मानो॒ वरु॑णो॒ वा ए॒ष यज॑मानम॒भ्यैति॒ यद॒ग्निरुप॑नद्ध॒ ओष॑धयः॒ प्रति॑ गृह्णीता॒ग्निमे॒तमित्या॑ह॒ शान्त्यै॒ व्यस्य॒न्विश्वा॒ अम॑ती॒ररा॑ती॒रित्या॑ह॒ रक्ष॑सा॒मप॑हत्यै नि॒षीद॑न्नो॒ अप॑ दुर्म॒तिꣳ ह॑न॒दित्या॑ह॒ प्रति॑ष्ठित्या॒ ओष॑धयः॒ प्रति॑ मोदध्वम्॥२९॥

%5.1.5.10
{\anuvakamend[{अ॒स्त्व॒नु॒ष्टुब॑सि सादय॒त्यारू॑ढः॒ पत्वेति॒ गर्भ॑म॒सौ मो॑दध्वं॒ द्विच॑त्वारिꣳशच्च}]}%॥५॥

%5.1.6.0
{{\anuvakamend[ए॒न॒मित्या॒हौष॑धयो॒ वा अ॒ग्नेर्भा॑ग॒धेय॒न्ताभि॑रे॒वैन॒ꣳ॒ सम॑र्धयति॒ पुष्पा॑वतीः सुपिप्प॒ला इत्या॑ह॒ तस्मा॒दोष॑धयः॒ फलं॑ गृह्णन्त्य॒यं वो॒ गर्भ॑ ऋ॒त्वियः॑ प्र॒त्नꣳ स॒धस्थ॒मास॑द॒दित्या॑ह॒ याभ्य॑ ए॒वैन॑म्प्रच्या॒वय॑ति॒ तास्वे॒वैनं॒ प्रति॑ ष्ठापयति॒ द्वाभ्या॑मु॒पाव॑हरति॒ प्रति॑ष्ठित्यै॥३०॥]}}

%5.1.6.1
वा॒रु॒णो वा अ॒ग्निरुप॑नद्धो॒ वि पाज॒सेति॒ वि स्रꣳ॑सयति सवि॒तृप्र॑सूत ए॒वास्य॒ विषू॑चीं वरुणमे॒निं वि सृ॑जत्य॒प उप॑ सृज॒त्यापो॒ वै शा॒न्ताः शा॒न्ताभि॑रे॒वास्य॒ शुचꣳ॑ शमयति ति॒सृभि॒रुप॑ सृजति त्रि॒वृद्वा अ॒ग्निर्यावा॑ने॒वाग्निस्तस्य॒ शुचꣳ॑ शमयति मि॒त्रः स॒ꣳ॒सृज्य॑ पृथि॒वीमित्या॑ह मि॒त्रो वै शि॒वो दे॒वाना॒न्तेनै॒व॥३१॥

%5.1.6.2
ए॒न॒ꣳ॒ सꣳ सृ॑जति॒ शान्त्यै॒ यद्ग्रा॒म्याणा॒म्पात्रा॑णां क॒पालैः᳚ सꣳसृ॒जेद्ग्रा॒म्याणि॒ पात्रा॑णि शु॒चार्प॑येदर्मकपा॒लैः सꣳ सृ॑जत्ये॒तानि॒ वा अ॑नुपजीवनी॒यानि॒ तान्ये॒व शु॒चार्प॑यति॒ शर्क॑राभिः॒ सꣳ सृ॑जति॒ धृत्या॒ अथो॑ शं॒त्वाया॑जलो॒मैः सꣳ सृ॑जत्ये॒षा वा अ॒ग्नेः प्रि॒या त॒नूर्यद॒जा प्रि॒ययै॒वैनं॑ त॒नुवा॒ सꣳ सृ॑ज॒त्यथो॒ तेज॑सा कृष्णाजि॒नस्य॒ लोम॑भिः॒ सम्॥३२॥

%5.1.6.3
सृ॒ज॒ति॒ य॒ज्ञो वै कृ॑ष्णाजि॒नं य॒ज्ञेनै॒व य॒ज्ञꣳ सꣳ सृ॑जति रु॒द्राः स॒म्भृत्य॑ पृथि॒वीमित्या॑है॒ता वा ए॒तं दे॒वता॒ अग्रे॒ सम॑भर॒न्ताभि॑रे॒वैन॒ꣳ॒ सम्भ॑रति म॒खस्य॒ शिरो॒\-ऽसीत्या॑ह य॒ज्ञो वै म॒खस्तस्यै॒तच्छिरो॒ यदु॒खा तस्मा॑दे॒वमा॑ह य॒ज्ञस्य॑ प॒दे स्थ॒ इत्या॑ह य॒ज्ञस्य॒ ह्ये॑ते॥३३॥

%5.1.6.4
प॒दे अथो॒ प्रति॑ष्ठित्यै॒ प्रान्याभि॒र्यच्छ॒त्यन्व॒न्यैर्म॑न्त्रयते मिथुन॒त्वाय॒ त्र्यु॑द्धिं करोति॒ त्रय॑ इ॒मे लो॒का ए॒षां लो॒काना॒माप्त्यै॒ छन्दो॑भिः करोति वी॒र्यं॑ वै छन्दाꣳ॑सि वी॒र्ये॑णै॒वैनां᳚ करोति॒ यजु॑षा॒ बिलं॑ करोति॒ व्यावृ॑त्त्या॒ इय॑तीं करोति प्र॒जाप॑तिना यज्ञमु॒खेन॒ सम्मि॑तां द्विस्त॒नां क॑रोति॒ द्यावा॑पृथि॒व्योर्दोहा॑य॒ चतुः॑ स्तनां करोति पशू॒नां दोहा॑या॒ष्टास्त॑नां करोति॒ छन्द॑सां॒ दोहा॑य॒ नवा᳚श्रिमभि॒चर॑तः कुर्यात्त्रि॒वृत॑मे॒व वज्रꣳ॑ स॒म्भृत्य॒ भ्रातृ॑व्याय॒ प्र ह॑रति॒ स्तृत्यै॑ कृ॒त्वाय॒ सा म॒हीमु॒खामिति॒ नि द॑धाति दे॒वता᳚स्वे॒वैनां॒ प्रति॑ ष्ठापयति॥३४॥

%5.1.7.0
{\anuvakamend[{तेनै॒व लोम॑भिः॒ समे॒ते अ॑भि॒चर॑त॒ एक॑विꣳशतिश्च}]}%॥६॥

%5.1.7.1
स॒प्तभि॑र्धूपयति स॒प्त वै शी॑र्\mbox{}ष॒ण्याः᳚ प्रा॒णाः शिर॑ ए॒तद्य॒ज्ञस्य॒ यदु॒खा शी॒र्\mbox{}षन्ने॒व य॒ज्ञस्य॑ प्रा॒णान्द॑धाति॒ तस्मा᳚थ्स॒प्त शी॒र्\mbox{}षन्प्रा॒णा अ॑श्वश॒केन॑ धूपयति प्राजाप॒त्यो वा अश्वः॑ सयोनि॒त्वायादि॑ति॒स्त्वेत्या॑हे॒यं वा अदि॑ति॒रदि॑त्यै॒वादि॑त्यां खनत्य॒स्या अक्रू॑रङ्काराय॒ न हि स्वः स्वꣳ हि॒नस्ति॑ दे॒वानां᳚ त्वा॒ पत्नी॒रित्या॑ह दे॒वाना᳚म्॥३५॥

%5.1.7.2
वा ए॒ताम्पत्न॒यो\-ऽग्रे॑\-ऽकुर्व॒न्ताभि॑रे॒वैनां᳚ दधाति धि॒षणा॒स्त्वेत्या॑ह वि॒द्या वै धि॒षणा॑ वि॒द्याभि॑रे॒वैना॑म॒भीन्द्धे॒ ग्नास्त्वेत्या॑ह॒ छन्दाꣳ॑सि॒ वै ग्नाश्छन्दो॑भिरे॒वैनाꣴ॑ श्रपयति॒ वरू᳚त्रय॒स्त्वेत्या॑ह॒ होत्रा॒ वै वरू᳚त्रयो॒ होत्रा॑भिरे॒वैनां᳚ पचति॒ जन॑य॒स्त्वेत्या॑ह दे॒वानां॒ वै पत्नीः᳚॥३६॥

%5.1.7.3
जन॑य॒स्ताभि॑रे॒वैनां᳚ पचति ष॒ड्भिः प॑चति॒ षड्वा ऋ॒तव॑ ऋ॒तुभि॑रे॒वैनां᳚ पचति॒ द्विः पच॒न्त्वित्या॑ह॒ तस्मा॒द्द्विः सं॑वथ्स॒रस्य॑ स॒स्यम्प॑च्यते वारु॒ण्यु॑खाभीद्धा॑ मै॒त्रियोपै॑ति॒ शान्त्यै॑ दे॒वस्त्वा॑ सवि॒तोद्व॑प॒त्वित्या॑ह सवि॒तृप्र॑सूत ए॒वैनां॒ ब्रह्म॑णा दे॒वता॑भि॒रुद्व॑प॒त्यप॑द्यमाना पृथि॒व्याशा॒ दिश॒ आ पृ॑ण॥३७॥

%5.1.7.4
इत्या॑ह॒ तस्मा॑द॒ग्निः सर्वा॒ दिशो\-ऽनु॒ वि भा॒त्युत्ति॑ष्ठ बृह॒ती भ॑वो॒र्ध्वा ति॑ष्ठ ध्रु॒वा त्वमित्या॑ह॒ प्रति॑ष्ठित्या असु॒र्य॑म्पात्र॒\-मना᳚च्छृण्ण॒मा च्छृ॑णत्ति देव॒त्राक॑रजक्षी॒रेणा च्छृ॑णत्ति पर॒मं वा ए॒तत्पयो॒ यद॑जक्षी॒रं प॑र॒मेणै॒वैना॒म्पय॒सा च्छृ॑णत्ति॒ यजु॑षा॒ व्यावृ॑त्त्यै॒ छन्दो॑भि॒रा च्छृ॑णत्ति॒ छन्दो॑भि॒र्वा ए॒षा क्रि॑यते॒ छन्दो॑भिरे॒व छन्दा॒ꣳ॒स्या च्छृ॑णत्ति॥३८॥

%5.1.8.0
{\anuvakamend[{आ॒ह॒ दे॒वानां॒ वै पत्नीः᳚ पृणै॒षा षट्च॑}]}%॥७॥

%5.1.8.1
एक॑विꣳशत्या॒ माषैः᳚ पुरुषशी॒र्\mbox{}षमच्छै᳚त्यमे॒ध्या वै माषा॑ अमे॒ध्यम्पु॑रुषशी॒र्\mbox{}षम॑मे॒ध्यैरे॒वास्या॑मे॒ध्यं नि॑रव॒दाय॒ मेध्यं॑ कृ॒त्वा ह॑र॒त्येक॑विꣳशतिर्भवन्त्येकवि॒ꣳ॒शो वै पुरु॑षः॒ पुरु॑ष॒स्याप्त्यै॒ व्यृ॑द्धं॒ वा ए॒तत्प्रा॒णैर॑मे॒ध्यं यत्पु॑रुषशी॒र्\mbox{}षꣳ स॑प्त॒धा वितृ॑ण्णां वल्मीकव॒पां प्रति॒ नि द॑धाति स॒प्त वै शी॑र्\mbox{}ष॒ण्याः᳚ प्रा॒णाः प्रा॒णैरे॒वैन॒थ्सम॑र्धयति मेध्य॒त्वाय॒ याव॑न्तः॥३९॥

%5.1.8.2
वै मृ॒त्युब॑न्धव॒स्तेषां᳚ य॒म आधि॑पत्यं॒ परी॑याय यमगा॒थाभिः॒ परि॑ गायति य॒मादे॒वैन॑द्वृङ्क्ते ति॒सृभिः॒ परि॑ गायति॒ त्रय॑ इ॒मे लो॒का ए॒भ्य ए॒वैन॑ल्लो॒केभ्यो॑ वृङ्क्ते॒ तस्मा॒द्गाय॑ते॒ न देय॒ङ्गाथा॒ हि तद्वृ॒ङ्क्ते᳚\-ऽग्निभ्यः॑ प॒शूना ल॑भते॒ कामा॒ वा अ॒ग्नयः॒ कामा॑ने॒वाव॑ रुन्द्धे॒ यत्प॒शून्नालभे॒तान॑वरुद्धा अस्य॥४०॥

%5.1.8.3
प॒शवः॑ स्यु॒र्यत्पर्य॑ग्निकृतानुथ्सृ॒जेद्य॑ज्ञवेश॒सं कु॑र्या॒द्यथ्सꣴ॑स्था॒पये᳚द्या॒तया॑मानि शी॒र्\mbox{}षाणि॑ स्यु॒र्यत्प॒शूना॒लभ॑ते॒ तेनै॒व प॒शूनव॑ रुन्द्धे॒ यत्पर्य॑ग्निकृतानुथ्सृ॒जति॑ शी॒र्ष्णामया॑तयामत्वाय प्राजाप॒त्येन॒ सꣴ स्था॑पयति य॒ज्ञो वै प्र॒जाप॑तिर्य॒ज्ञ ए॒व य॒ज्ञं प्रति॑ ष्ठापयति प्र॒जाप॑तिः प्र॒जा अ॑सृजत॒ स रि॑रिचा॒नो॑\-ऽमन्यत॒ स ए॒ता आ॒प्रीर॑पश्य॒त्ताभि॒र्वै स मु॑ख॒तः॥४१॥

%5.1.8.4
आ॒त्मान॒माप्री॑णीत॒ यदे॒ता आ॒प्रियो॒ भव॑न्ति य॒ज्ञो वै प्र॒जाप॑तिर्य॒ज्ञमे॒वैताभि॑र्मुख॒त आ प्री॑णा॒त्यप॑रिमितछन्दसो भव॒न्त्यप॑रिमितः प्र॒जाप॑तिः प्र॒जाप॑ते॒राप्त्या॑ ऊनातिरि॒क्ता मि॑थु॒नाः प्रजा᳚त्यै लोम॒शं वै नामै॒तच्छन्दः॑ प्र॒जाप॑तेः प॒शवो॑ लोम॒शाः प॒शूने॒वाव॑ रुन्द्धे॒ सर्वा॑णि॒ वा ए॒ता रू॒पाणि॒ सर्वा॑णि रू॒पाण्य॒ग्नौ चित्ये᳚ क्रियन्ते॒ तस्मा॑दे॒ता अ॒ग्नेश्चित्य॑स्य॥४२॥

%5.1.8.5
भ॒व॒न्त्येक॑विꣳशतिꣳ सामिधे॒नीरन्वा॑ह॒ रुग्वा ए॑कवि॒ꣳ॒शो रुच॑मे॒व ग॑च्छ॒त्यथो᳚ प्रति॒ष्ठामे॒व प्र॑ति॒ष्ठा ह्ये॑कवि॒ꣳ॒शश्चतु॑र्विꣳशति॒मन्वा॑ह॒ चतु॑र्विꣳशतिरर्धमा॒साः सं॑वथ्स॒रः सं॑वथ्स॒रो᳚\-ऽग्निर्वै᳚श्वान॒रः सा॒क्षादे॒व वै᳚श्वान॒रमव॑ रुन्द्धे॒ परा॑ची॒रन्वा॑ह॒ परा॑ङिव॒ हि सु॑व॒र्गो लो॒कः समा᳚स्त्वाग्न ऋ॒तवो॑ वर्धय॒न्त्वित्या॑ह॒ समा॑भिरे॒वाग्निं व॑र्धयति॥४३॥

%5.1.8.6
ऋ॒तुभिः॑ संवथ्स॒रं विश्वा॒ आ भा॑हि प्र॒दिशः॑ पृथि॒व्या इत्या॑ह॒ तस्मा॑द॒ग्निः सर्वा॒ दिशो\-ऽनु॒ वि भा॑ति॒ प्रत्यौ॑हताम॒श्विना॑ मृ॒त्युम॑स्मा॒दित्या॑ह मृ॒त्युमे॒वास्मा॒दप॑ नुद॒त्युद्व॒यं तम॑स॒स्परीत्या॑ह पा॒प्मा वै तमः॑ पा॒प्मान॑मे॒वास्मा॒दप॑ ह॒न्त्यग॑न्म॒ ज्योति॑रुत्त॒ममित्या॑हा॒सौ वा आ॑दि॒त्यो ज्योति॑रुत्त॒ममा॑दि॒त्यस्यै॒व सायु॑ज्यं गच्छति॒ न सं॑वथ्स॒रस्ति॑ष्ठति॒ नास्य॒ श्रीस्ति॑ष्ठति॒ यस्यै॒ताः क्रि॒यन्ते॒ ज्योति॑ष्मतीमुत्त॒मामन्वा॑ह॒ ज्योति॑रे॒वास्मा॑ उ॒परि॑ष्टाद्दधाति सुव॒र्गस्य॑ लो॒कस्यानु॑ख्यात्यै॥४४॥

%5.1.9.0
{\anuvakamend[{याव॑न्तो\-ऽस्य मुख॒तश्चित्य॑स्य वर्धयत्यादि॒त्यो᳚\-ऽष्टाविꣳ॑शतिश्च}]}%॥८॥

%5.1.9.1
ष॒ड्भिर्दी᳚क्षयति॒ षड्वा ऋ॒तव॑ ऋ॒तुभि॑रे॒वैनं॑ दीक्षयति स॒प्तभि॑र्दीक्षयति स॒प्त छन्दाꣳ॑सि॒ छन्दो॑भिरे॒वैनं॑ दीक्षयति॒ विश्वे॑ दे॒वस्य॑ ने॒तुरित्य॑नु॒ष्टुभो᳚त्त॒मया॑ जुहोति॒ वाग्वा अ॑नु॒ष्टुप्तस्मा᳚त्प्रा॒णानां॒ वागु॑त्त॒मैक॑स्माद॒क्षरा॒दना᳚प्तम्प्रथ॒मम्प॒दम् तस्मा॒द्यद्वा॒चो\-ऽना᳚प्तं॒ तन्म॑नु॒ष्या॑ उप॑ जीवन्ति पू॒र्णया॑ जुहोति पू॒र्ण इ॑व॒ हि प्र॒जाप॑तिः॥४५॥

%5.1.9.2
प्र॒जाप॑ते॒राप्त्यै॒ न्यू॑नया जुहोति॒ न्यू॑ना॒द्धि प्र॒जाप॑तिः प्र॒जा असृ॑जत प्र॒जाना॒ꣳ॒ सृष्ट्यै॒ यद॒र्चिषि॑ प्रवृ॒ञ्ज्याद्भू॒तमव॑ रुन्धीत॒ यदङ्गा॑रेषु भवि॒ष्यदङ्गा॑रेषु॒ प्र वृ॑णक्ति भवि॒ष्यदे॒वाव॑ रुन्द्धे भवि॒ष्यद्धि भूयो॑ भू॒ताद्द्वाभ्या॒म्प्र वृ॑णक्ति द्वि॒पाद्यज॑मानः॒ प्रति॑ष्ठित्यै॒ ब्रह्म॑णा॒ वा ए॒षा यजु॑षा॒ सम्भृ॑ता॒ यदु॒खा सा यद्भिद्ये॒तार्ति॒मार्च्छे᳚त्॥४६॥

%5.1.9.3
यज॑मानो ह॒न्येता᳚स्य य॒ज्ञो मित्रै॒तामु॒खां त॒पेत्या॑ह॒ ब्रह्म॒ वै मि॒त्रो ब्रह्म॑न्ने॒वैनां॒ प्रति॑ ष्ठापयति॒ नार्ति॒मार्च्छ॑ति॒ यज॑मानो॒ नास्य॑ य॒ज्ञो ह॑न्यते॒ यदि॒ भिद्ये॑त॒ तैरे॒व क॒पालैः॒ सꣳ सृ॑जे॒थ्सैव ततः॒ प्राय॑श्चित्ति॒र्यो ग॒तश्रीः॒ स्यान्म॑थि॒त्वा तस्याव॑ दध्याद्भू॒तो वा ए॒ष स स्वां॥४७॥

%5.1.9.4
दे॒वता॒मुपै॑ति॒ यो भूति॑कामः॒ स्याद्य उ॒खायै॑ स॒म्भवे॒थ्स ए॒व तस्य॑ स्या॒दतो॒ ह्ये॑ष स॒म्भव॑त्ये॒ष वै स्व॑य॒म्भूर्नाम॒ भव॑त्ये॒व यं का॒मये॑त॒ भ्रातृ॑व्यमस्मै जनयेय॒मित्य॒न्यत॒स्तस्या॒हृत्याव॑ दध्याथ्सा॒क्षादे॒वास्मै॒ भ्रातृ॑व्यं जनयत्यम्ब॒रीषा॒\-दन्न॑काम॒स्याव॑ दध्यादम्ब॒रीषे॒ वा अन्न॑म्भ्रियते॒ सयो᳚न्ये॒वान्नम्᳚॥४८॥

%5.1.9.5
अव॑ रुन्द्धे॒ मुञ्जा॒नव॑ दधा॒त्यूर्ग्वै मुञ्जा॒ ऊर्ज॑मे॒वास्मा॒ अपि॑ दधात्य॒ग्निर्दे॒वेभ्यो॒ निला॑यत॒ स क्रु॑मु॒कम्प्रावि॑शत् क्रुमु॒कमव॑ दधाति॒ यदे॒वास्य॒ तत्र॒ न्य॑क्तं॒ तदे॒वाव॑ रुन्द्ध॒ आज्ये॑न॒ सं यौ᳚त्ये॒तद्वा अ॒ग्नेः प्रि॒यं धाम॒ यदाज्य॑म् प्रि॒येणै॒वैनं॒ धाम्ना॒ सम॑र्धय॒त्यथो॒ तेज॑सा॥४९॥

%5.1.9.6
वैकं॑कती॒मा द॑धाति॒ भा ए॒वाव॑ रुन्द्धे शमी॒मयी॒मा द॑धाति॒ शान्त्यै॒ सीद॒ त्वं मा॒तुर॒स्या उ॒पस्थ॒ इति॑ ति॒सृभि॑र्जा॒तमुप॑ तिष्ठते॒ त्रय॑ इ॒मे लो॒का ए॒ष्वे॑व लो॒केष्वा॒विदं॑ गच्छ॒त्यथो᳚ प्रा॒णाने॒वात्मन्ध॑त्ते॥५०॥

%5.1.10.0
{\anuvakamend[{प्र॒जाप॑तिर्\mbox{}ऋच्छे॒थ्स्वामे॒वान्नं॒ तेज॑सा॒ चतु॑स्त्रिꣳशच्च}]}%॥९॥

%5.1.10.1
न ह॑ स्म॒ वै पु॒राग्निरप॑रशुवृक्णं दहति॒ तद॑स्मै प्रयो॒ग ए॒वर्\mbox{}षि॑रस्वदय॒द्यद॑ग्ने॒ यानि॒ कानि॒ चेति॑ स॒मिध॒मा द॑धा॒त्यप॑रशुवृक्णमे॒वास्मै᳚ स्वदयति॒ सर्व॑मस्मै स्वदते॒ य ए॒वं वेदौदु॑म्बरी॒मा द॑धा॒त्यूर्ग्वा उ॑दु॒म्बर॒ ऊर्ज॑मे॒वास्मा॒ अपि॑ दधाति प्र॒जाप॑तिर॒ग्निम॑सृजत॒ तꣳ सृ॒ष्टꣳ रक्षाꣳ॑सि॥५१॥

%5.1.10.2
अ॒जि॒घा॒ꣳ॒स॒न्थ्स ए॒तद्रा᳚क्षो॒घ्नम॑पश्य॒त्तेन॒ वै स रक्षा॒ꣳ॒स्यपा॑हत॒ यद्रा᳚क्षो॒घ्नम्भव॑त्य॒ग्नेरे॒व तेन॑ जा॒ताद्रक्षा॒ꣳ॒स्यप॑ ह॒न्त्याश्व॑त्थी॒मा द॑धात्यश्व॒त्थो वै वन॒स्पती॑नाꣳ सपत्नसा॒हो विजि॑त्यै॒ वैक॑ङ्कती॒मा द॑धाति॒ भा ए॒वाव॑ रुन्द्धे शमी॒मयी॒मा द॑धाति॒ शान्त्यै॒ सꣳ॑शितम्मे॒ ब्रह्मोदे॑षाम्बा॒हू अ॑तिर॒मित्यु॑त्त॒मे औदु॑म्बरी॥५२॥

%5.1.10.3
वा॒च॒य॒ति॒ ब्रह्म॑णै॒व क्ष॒त्रꣳ सꣴ श्य॑ति क्ष॒त्रेण॒ ब्रह्म॒ तस्मा᳚द्ब्राह्म॒णो रा॑ज॒न्य॑वा॒नत्य॒न्यम्ब्रा᳚ह्म॒णं तस्मा᳚द्राज॒न्यो᳚ ब्राह्म॒णवा॒नत्य॒न्यꣳ रा॑ज॒न्य॑म्मृ॒त्युर्वा ए॒ष यद॒ग्निर॒मृत॒ꣳ॒ हिर॑ण्यꣳ रु॒क्ममन्त॑रं॒ प्रति॑ मुञ्चते॒\-ऽमृत॑मे॒व मृ॒त्योर॒न्तर्ध॑त्त॒ एक॑विꣳशतिनिर्बाधो भव॒त्येक॑विꣳशति॒र्वै दे॑वलो॒का द्वाद॑श॒ मासाः॒ पञ्च॒र्तव॒स्त्रय॑ इ॒मे लो॒का अ॒सावा॑दि॒त्यः॥५३॥

%5.1.10.4
ए॒क॒वि॒ꣳ॒श ए॒ताव॑न्तो॒ वै दे॑वलो॒कास्तेभ्य॑ ए॒व भ्रातृ॑व्यम॒न्तरे॑ति निर्बा॒धैर्वै दे॒वा असु॑रान्निर्बा॒धे॑\-ऽकुर्वत॒ तन्नि॑र्बा॒धानां᳚ निर्बाध॒त्वन्नि॑र्बा॒धी भ॑वति॒ भ्रातृ॑व्याने॒व नि॑र्बा॒धे कु॑रुते सावित्रि॒या प्रति॑ मुञ्चते॒ प्रसू᳚त्यै॒ नक्तो॒षासेत्युत्त॑रयाहोरा॒त्राभ्या॑\-मे॒वैन॒मुद्य॑च्छते दे॒वा अ॒ग्निं धा॑रयन्द्रविणो॒दा इत्या॑ह प्रा॒णा वै दे॒वा द्र॑विणो॒दा अ॑होरा॒त्राभ्या॑मे॒वैन॑मु॒द्यत्य॑॥५४॥

%5.1.10.5
प्रा॒णैर्दा॑धा॒रासी॑नः॒ प्रति॑ मुञ्चते॒ तस्मा॒दासी॑नाः प्र॒जाः प्र जा॑यन्ते कृष्णाजि॒नमुत्त॑र॒न्तेजो॒ वै हिर॑ण्यं॒ ब्रह्म॑ कृष्णाजि॒नन्तेज॑सा चै॒वैनं॒ ब्रह्म॑णा चोभ॒यतः॒ परि॑ गृह्णाति॒ षडु॑द्यामꣳ शि॒क्य॑म्भवति॒ षड्वा ऋ॒तव॑ ऋ॒तुभि॑रे॒वैन॒मुद्य॑च्छते॒ यद्द्वाद॑शोद्यामꣳ संवथ्स॒रेणै॒व मौ॒ञ्जम्भ॑व॒त्यूर्ग्वै मुञ्जा॑ ऊ॒र्जैवैन॒ꣳ॒ सम॑र्धयति सुप॒र्णो॑\-ऽसि ग॒रुत्मा॒नित्यवे᳚क्षते रू॒पमे॒वास्यै॒तन्म॑हि॒मानं॒ व्याच॑ष्टे॒ दिवं॑ गच्छ॒ सुवः॑ प॒तेत्या॑ह सुव॒र्गमे॒वैनं॑ लो॒कं ग॑मयति॥५॥

%5.1.11.0
{\anuvakamend[{रक्षा॒ꣳ॒स्यौदु॑म्बरी आदि॒त्य उ॒द्यत्य॒ स़ञ्चतु॑र्विꣳशतिश्च}]}%॥10॥

%5.1.11.1
समि॑द्धो अ॒ञ्जन्कृद॑रम्मती॒नां घृ॒तम॑ग्ने॒ मधु॑म॒त्पिन्व॑मानः। वा॒जी वह॑न्वा॒जिनं॑ जातवेदो दे॒वानां᳚ वक्षि प्रि॒यमा स॒धस्थम्᳚। घृ॒तेना॒ञ्जन्थ्सम्प॒थो दे॑व॒याना᳚न्प्रजा॒नन्वा॒ज्यप्ये॑तु दे॒वान्। अनु॑ त्वा सप्ते प्र॒दिशः॑ सचन्ताꣴ स्व॒धाम॒स्मै यज॑मानाय धेहि। ईड्य॒श्चासि॒ वन्द्य॑श्च वाजिन्ना॒शुश्चासि॒ मेध्य॑श्च सप्ते। अ॒ग्निष्ट्वा᳚॥५६॥

%5.1.11.2
दे॒वैर्वसु॑भिः स॒जोषाः᳚ प्री॒तं वह्निं॑ वहतु जा॒तवे॑दाः। स्ती॒र्णम्ब॒र्\mbox{}हिः सु॒ष्टरी॑मा जुषा॒णोरु पृ॒थु प्रथ॑मानं पृथि॒व्याम्। दे॒वेभि॑र्यु॒क्तमदि॑तिः स॒जोषाः᳚ स्यो॒नं कृ॑ण्वा॒ना सु॑वि॒ते द॑धातु। ए॒ता उ॑ वः सु॒भगा॑ वि॒श्वरू॑पा॒ वि पक्षो॑भिः॒ श्रय॑माणा॒ उदातैः᳚। ऋ॒ष्वाः स॒तीः क॒वषः॒ शुम्भ॑माना॒ द्वारो॑ दे॒वीः सु॑प्राय॒णा भ॑वन्तु। अ॒न्त॒रा मि॒त्रावरु॑णा॒ चर॑न्ती॒ मुखं॑ य॒ज्ञाना॑म॒भि सं॑विदा॒ने। उ॒षासा॑ वाम्॥५७॥

%5.1.11.3
सु॒हि॒र॒ण्ये सु॑शि॒ल्पे ऋ॒तस्य॒ योना॑वि॒ह सा॑दयामि। प्र॒थ॒मा वाꣳ॑ सर॒थिना॑ सु॒वर्णा॑ दे॒वौ पश्य॑न्तौ॒ भुव॑नानि॒ विश्वा᳚। अपि॑प्रयं॒ चोद॑ना वा॒म्मिमा॑ना॒ होता॑रा॒ ज्योतिः॑ प्र॒दिशा॑ दि॒शन्ता᳚। आ॒दि॒त्यैर्नो॒ भार॑ती वष्टु य॒ज्ञꣳ सर॑स्वती स॒ह रु॒द्रैर्न॑ आवीत्। इडोप॑हूता॒ वसु॑भिः स॒जोषा॑ य॒ज्ञं नो॑ देवीर॒मृते॑षु धत्त। त्वष्टा॑ वी॒रं दे॒वका॑मं जजान॒ त्वष्टु॒रर्वा॑ जायत आ॒शुरश्वः॑।॥५८॥

%5.1.11.4
त्वष्टे॒दं विश्व॒म्भुव॑नं जजान ब॒होः क॒र्तार॑मि॒ह य॑क्षि होतः। अश्वो॑ घृ॒तेन॒ त्मन्या॒ सम॑क्त॒ उप॑ दे॒वाꣳ ऋ॑तु॒शः पाथ॑ एतु। वन॒स्पति॑र्देवलो॒कम्प्र॑जा॒नन्न॒ग्निना॑ ह॒व्या स्व॑दि॒तानि॑ वक्षत्। प्र॒जाप॑ते॒स्तप॑सा वावृधा॒नः स॒द्यो जा॒तो द॑धिषे य॒ज्ञम॑ग्ने। स्वाहा॑कृतेन ह॒विषा॑ पुरोगा या॒हि सा॒ध्या ह॒विर॑दन्तु दे॒वाः॥५९॥

%5.2.0.0
{\anuvakamend[{अ॒ग्निष्ट्वा॑ वा॒मश्वो॒ द्विच॑त्वारिꣳशच्च}]}%॥11॥

%5.2.0.0

{\anuvakamend[{विष्णु॑मुखा॒ अन्न॑पते॒ याव॑ती॒ वि वै पु॑रुषमा॒त्रेणाग्ने॒ तव॒ श्रवो॒ ब्रह्म॑ जज्ञा॒नꣴ स्व॑यमातृ॒ण्णामे॒षां वै प॒शुर्गा॑य॒त्री कस्त्वा॒ द्वाद॑श}]}%॥12॥ 
\prashnaend{विष्णु॑मुखा॒ अप॑चितिमा॒न् वि वा ए॒तावग्ने॒ तव॑ स्वयमातृ॒ण्णां वि॑षू॒चीना॑नि गाय॒त्री चतु॑ष्षष्टिः॥64॥ विष्णु॑मुखास्त॒नुवे॑ भुवत्॥}
%%% END PRASHNA

\sect{द्वितीयः प्रश्नः}\setcounter{anuvakam}{0}
\dnsub{तैत्तिरीयसंहितायां पञ्चमकाण्डे द्वितीयः प्रश्नः}
%5.2.1.0
%5.2.1.1
विष्णु॑मुखा॒ वै दे॒वाश्छन्दो॑भिरि॒माल्लोँ॒कान॑नपज॒य्यम॒भ्य॑जय॒न्॒ यद्वि॑ष्णुक्र॒मान्क्रम॑ते॒ विष्णु॑रे॒व भू॒त्वा यज॑मान॒श्छन्दो॑भिरि॒माल्लोँ॒कान॑नपज॒य्यम॒भि ज॑यति॒ विष्णोः॒ क्रमो᳚\-ऽस्यभिमाति॒हेत्या॑ह गाय॒त्री वै पृ॑थि॒वी त्रैष्ठु॑भम॒न्तरि॑क्ष॒म् जाग॑ती॒ द्यौरानु॑ष्टुभी॒र्दिश॒श्छन्दो॑भिरे॒वेमाल्लोँ॒कान् य॑थापू॒र्वम॒भि ज॑यति प्र॒जाप॑तिर॒ग्निम॑सृजत॒ सो᳚\-ऽस्माथ्सृ॒ष्टः॥१॥

%5.2.1.2
परा॑ङै॒त्तमे॒तयान्वै॒दक्र॑न्द॒दिति॒ तया॒ वै सो᳚\-ऽग्नेः प्रि॒यं धामावा॑रुन्द्ध॒ यदे॒ताम॒न्वाहा॒ग्नेरे॒वैतया᳚ प्रि॒यं धामाव॑ रुन्द्ध ईश्व॒रो वा ए॒ष परा᳚ङ्प्र॒दघो॒ यो वि॑ष्णुक्र॒मान्क्रम॑ते चत॒सृभि॒रा व॑र्तते च॒त्वारि॒ छन्दाꣳ॑सि॒ छन्दाꣳ॑सि॒ खलु॒ वा अ॒ग्नेः प्रि॒या त॒नूः प्रि॒यामे॒वास्य॑ त॒नुव॑म॒भि॥२॥

%5.2.1.3
प॒र्याव॑र्तते दक्षि॒णा प॒र्याव॑र्तते॒ स्वमे॒व वी॒र्य॑मनु॑ प॒र्याव॑र्तते॒ तस्मा॒द्दक्षि॒णो\-ऽर्ध॑ आ॒त्मनो॑ वी॒र्या॑वत्त॒रो\-ऽथो॑ आदि॒त्यस्यै॒वावृत॒मनु॑ प॒र्याव॑र्तते॒ शुनः॒शेप॒माजी॑गर्तिं॒ वरु॑णो\-ऽगृह्णा॒थ्स ए॒तां वा॑रु॒णीम॑पश्य॒त्तया॒ वै स आ॒त्मानं॑ वरुणपा॒शाद॑मुञ्च॒द्वरु॑णो॒ वा ए॒तं गृ॑ह्णाति॒ य उ॒खाम्प्र॑तिमु॒ञ्चत॒ उदु॑त्त॒मं व॑रुण॒ पाश॑म॒स्मदित्या॑हा॒त्मान॑मे॒वैतया᳚॥३॥

%5.2.1.4
व॒रु॒ण॒पा॒शान्मु॑ञ्च॒त्या त्वा॑हार्\mbox{}ष॒मित्या॒हा ह्ये॑न॒ꣳ॒ हर॑ति ध्रु॒वस्ति॒ष्ठावि॑चाचलि॒रित्या॑ह॒ प्रति॑ष्ठित्यै॒ विश॑स्त्वा॒ सर्वा॑ वाञ्छ॒न्त्वित्या॑ह वि॒शैवैन॒ꣳ॒ सम॑र्धयत्य॒स्मिन्रा॒ष्ट्रमधि॑ श्र॒येत्या॑ह रा॒ष्ट्रमे॒वास्मि॑न्ध्रु॒वम॑क॒र्यं का॒मये॑त रा॒ष्ट्रꣴ स्या॒दिति॒ तम्मन॑सा ध्यायेद्रा॒ष्ट्रमे॒व भ॑वति॥४॥

%5.2.1.5
अग्रे॑ बृ॒हन्नु॒षसा॑मू॒र्ध्वो अ॑स्था॒दित्या॒हाग्र॑मे॒वैनꣳ॑ समा॒नानां᳚ करोति निर्जग्मि॒वान्तम॑स॒ इत्या॑ह॒ तम॑ ए॒वास्मा॒दप॑ हन्ति॒ ज्योति॒षागा॒दित्या॑ह॒ ज्योति॑रे॒वास्मि॑न्दधाति चत॒सृभिः॑ सादयति च॒त्वारि॒ छन्दाꣳ॑सि॒ छन्दो॑भिरे॒वाति॑छन्दसोत्त॒मया॒ वर्ष्म॒ वा ए॒षा छन्द॑सां॒ यदति॑च्छन्दा॒ वर्ष्मै॒वैनꣳ॑ समा॒नानां᳚ करोति॒ सद्व॑ती॥५॥

%5.2.1.6
भ॒व॒ति॒ स॒त्त्वमे॒वैनं॑ गमयति वाथ्स॒प्रेणोप॑ तिष्ठत ए॒तेन॒ वै व॑थ्स॒प्रीर्भा॑लन्द॒नो᳚\-ऽग्नेः प्रि॒यं धामावा॑रुन्द्धा॒ग्नेरे॒वैतेन॑ प्रि॒यं धामाव॑ रुन्द्ध एकाद॒शम्भ॑वत्येक॒धैव यज॑माने वी॒र्यं॑ दधाति॒ स्तोमे॑न॒ वै दे॒वा अ॒स्मिल्लोँ॒क आ᳚र्ध्नुव॒ञ्छन्दो॑भिर॒मुष्मि॒न्स्तोम॑स्येव॒ खलु॒ वा ए॒तद्रू॒पं यद्वा᳚थ्स॒प्रम्यद्वा᳚थ्स॒प्रेणो॑प॒तिष्ठ॑ते॥६॥

%5.2.1.7
इ॒ममे॒व तेन॑ लो॒कम॒भि ज॑य॒ति यद्वि॑ष्णुक्र॒मान्क्रम॑ते॒\-ऽमुमे॒व तैर्लो॒कम॒भि ज॑यति पूर्वे॒द्युः प्र क्रा॑मत्युत्तरे॒द्युरुप॑ तिष्ठते॒ तस्मा॒द्योगे॒\-ऽन्यासां᳚ प्र॒जाना॒म्मनः॒ क्षेमे॒\-ऽन्यासा॒न्तस्मा᳚द्यायाव॒रः क्षे॒म्यस्ये॑शे॒ तस्मा᳚द्यायाव॒रः क्षे॒म्यम॒ध्यव॑स्यति मु॒ष्टी क॑रोति॒ वाचं॑ यच्छति य॒ज्ञस्य॒ धृत्यै᳚॥७॥

%5.2.2.0
{\anuvakamend[{सृ॒ष्टो\-ऽभ्ये॑तया॑ भवति॒ सद्व॑त्युप॒तिष्ठ॑ते॒ द्विच॑त्वारिꣳशच्च}]}%॥१॥

%5.2.2.1
अन्न॑प॒ते\-ऽन्न॑स्य नो दे॒हीत्या॑हा॒ग्निर्वा अन्न॑पतिः॒ स ए॒वास्मा॒ अन्न॒म्प्र य॑च्छत्यनमी॒वस्य॑ शु॒ष्मिण॒ इत्या॑हाय॒क्ष्मस्येति॒ वावैतदा॑ह॒ प्र प्र॑दा॒तारं॑ तारिष॒ ऊर्जं॑ नो धेहि द्वि॒पदे॒ चतु॑ष्पद॒ इत्या॑हा॒शिष॑मे॒वैतामा शा᳚स्त॒ उदु॑ त्वा॒ विश्वे॑ दे॒वा इत्या॑ह प्रा॒णा वै विश्वे॑ दे॒वाः॥८॥

%5.2.2.2
प्रा॒णैरे॒वैन॒मुद्य॑च्छ॒ते\-ऽग्ने॒ भर॑न्तु॒ चित्ति॑भि॒रित्या॑ह॒ यस्मा॑ ए॒वैनं॑ चि॒त्तायो॒द्यच्छ॑ते॒ तेनै॒वैन॒ꣳ॒ सम॑र्धयति चत॒सृभि॒रा सा॑दयति च॒त्वारि॒ छन्दाꣳ॑सि॒ छन्दो॑भिरे॒वाति॑च्छन्दसोत्त॒मया॒ वर्ष्म॒ वा ए॒षा छन्द॑सां॒ यदति॑च्छन्दा॒ वर्ष्मै॒वैनꣳ॑ समा॒नानां᳚ करोति॒ सद्व॑ती भवति स॒त्त्वमे॒वैनं॑ गमयति॒ प्रेद॑ग्ने॒ ज्योति॑ष्मान्॥९॥

%5.2.2.3
या॒हीत्या॑ह॒ ज्योति॑रे॒वास्मि॑न्दधाति त॒नुवा॒ वा ए॒ष हि॑नस्ति॒ यꣳ हि॒नस्ति॒ मा हिꣳ॑सीस्त॒नुवा᳚ प्र॒जा इत्या॑ह प्र॒जाभ्य॑ ए॒वैनꣳ॑ शमयति॒ रक्षाꣳ॑सि॒ वा ए॒तद्य॒ज्ञꣳ स॑चन्ते॒ यदन॑ उ॒थ्सर्ज॒त्यक्र॑न्द॒दित्यन्वा॑ह॒ रक्ष॑सा॒मप॑हत्या॒ अन॑सा वह॒न्त्यप॑चितिमे॒वास्मि॑न्दधाति॒ तस्मा॑दन॒स्वी च॑ र॒थी चाति॑थीना॒मप॑चिततमौ॥१०॥

%5.2.2.4
अप॑चितिमान्भवति॒ य ए॒वं वेद॑ स॒मिधा॒\-ऽग्निं दु॑वस्य॒तेति॑ घृतानुषि॒क्तामव॑सिते स॒मिध॒मा द॑धाति॒ यथाति॑थय॒ आग॑ताय स॒र्पिष्व॑दाति॒थ्यं क्रि॒यते॑ ता॒दृगे॒व तद्गा॑यत्रि॒या ब्रा᳚ह्म॒णस्य॑ गाय॒त्रो हि ब्रा᳚ह्म॒णस्त्रि॒ष्टुभा॑ राज॒न्य॑स्य॒ त्रैष्टु॑भो॒ हि रा॑ज॒न्यो᳚\-ऽफ्सु भस्म॒ प्र वे॑शयत्य॒फ्सुयो॑नि॒र्वा अ॒ग्निः स्वामे॒वैनं॒ योनिं॑ गमयति ति॒सृभिः॒ प्र वे॑शयति त्रि॒वृद्वै॥११॥

%5.2.2.5
अ॒ग्निर्यावा॑ने॒वाग्निस्तम्प्र॑ति॒ष्ठां ग॑मयति॒ परा॒ वा ए॒षो᳚\-ऽग्निं व॑पति॒ यो᳚\-ऽफ्सु भस्म॑ प्रवे॒शय॑ति॒ ज्योति॑ष्मतीभ्या॒मव॑ दधाति॒ ज्योति॑रे॒वास्मि॑न्दधाति॒ द्वाभ्यां॒ प्रति॑ष्ठित्यै॒ परा॒ वा ए॒ष प्र॒जां प॒शून् व॑पति॒ यो᳚\-ऽफ्सु भस्म॑ प्रवे॒शय॑ति॒ पुन॑रू॒र्जा स॒ह र॒य्येति॒ पुन॑रु॒दैति॑ प्र॒जामे॒व प॒शूना॒त्मन्ध॑त्ते॒ पुन॑स्त्वादि॒त्याः॥१२॥

%5.2.2.6
रु॒द्रा वस॑वः॒ समि॑न्धता॒मित्या॑है॒ता वा ए॒तं दे॒वता॒ अग्रे॒ समै᳚न्धत॒ ताभि॑रे॒वैन॒ꣳ॒ समि॑न्द्धे॒ बोधा॒ स बो॒धीत्युप॑ तिष्ठते बो॒धय॑त्ये॒वैन॒न्तस्मा᳚थ्सु॒प्त्वा प्र॒जाः प्र बु॑ध्यन्ते यथास्था॒नमुप॑ तिष्ठते॒ तस्मा᳚द्यथास्था॒नम्प॒शवः॒ पुन॒रेत्योप॑ तिष्ठन्ते॥१३॥

%5.2.3.0
{\anuvakamend[{वै विश्वे॑ दे॒वा ज्योति॑ष्मा॒नप॑चिततमौ त्रि॒वृद्वा आ॑दि॒त्या द्विच॑त्वारिꣳशच्च}]}%॥२॥

%5.2.3.1
याव॑ती॒ वै पृ॑थि॒वी तस्यै॑ य॒म आधि॑पत्यं॒ परी॑याय॒ यो वै य॒मं दे॑व॒यज॑नम॒स्या अनि॑र्याच्या॒ग्निं चि॑नु॒ते य॒मायै॑न॒ꣳ॒ स चि॑नु॒ते\-ऽपे॒तेत्य॒ध्यव॑साययति य॒ममे॒व दे॑व॒यज॑नम॒स्यै नि॒र्याच्या॒त्मने॒\-ऽग्निं चि॑नुत इष्व॒ग्रेण॒ वा अ॒स्या अना॑मृतमि॒च्छन्तो॒ नावि॑न्द॒न्ते दे॒वा ए॒तद्यजु॑रपश्य॒न्नपे॒तेति॒ यदे॒तेना᳚ध्यवसा॒यय॑ति॥१४॥

%5.2.3.2
अना॑मृत ए॒वाग्निं चि॑नुत॒ उद्ध॑न्ति॒ यदे॒वास्या॑ अमे॒ध्यं तदप॑ हन्त्य॒पो\-ऽवो᳚क्षति॒ शान्त्यै॒ सिक॑ता॒ नि व॑पत्ये॒तद्वा अ॒ग्नेर्वै᳚श्वान॒रस्य॑ रू॒पꣳ रू॒पेणै॒व वै᳚श्वान॒रमव॑ रुन्द्ध॒ ऊषा॒न्नि व॑पति॒ पुष्टि॒र्वा ए॒षा प्र॒जन॑नं॒ यदूषाः॒ पुष्ट्या॑मे॒व प्र॒जन॑ने॒\-ऽग्निं चि॑नु॒ते\-ऽथो॑ सं॒ज्ञान॑ ए॒व सं॒ज्ञान॒ꣴ॒ ह्ये॑तत्॥१५॥

%5.2.3.3
प॒शू॒नां यदूषा॒ द्यावा॑पृथि॒वी स॒हास्ता॒न्ते वि॑य॒ती अ॑ब्रूता॒मस्त्वे॒व नौ॑ स॒ह य॒ज्ञिय॒मिति॒ यद॒मुष्या॑ य॒ज्ञिय॒मासी॒त्तद॒स्याम॑दधा॒त्त ऊषा॑ अभव॒न् यद॒स्या य॒ज्ञिय॒मासी॒त्तद॒मुष्या॑मदधा॒त्तद॒दश्च॒न्द्रम॑सि कृ॒ष्णमूषा᳚न्नि॒वप॑न्न॒दो ध्या॑ये॒द्द्यावा॑पृथि॒व्योरे॒व य॒ज्ञिये॒\-ऽग्निं चि॑नुते॒\-ऽयꣳ सो अ॒ग्निरिति॑ वि॒श्वामि॑त्रस्य॥१६॥

%5.2.3.4
सू॒क्तम्भ॑वत्ये॒तेन॒ वै वि॒श्वामि॑त्रो॒\-ऽग्नेः प्रि॒यं धामावा॑रुन्द्धा॒ग्नेरे॒वैतेन॑ प्रि॒यं धामाव॑ रुन्द्धे॒ छन्दो॑भि॒र्वै दे॒वाः सु॑व॒र्गं लो॒कमा॑य॒ञ्चत॑स्रः॒ प्राची॒रुप॑ दधाति च॒त्वारि॒ छन्दाꣳ॑सि॒ छन्दो॑भिरे॒व तद्यज॑मानः सुव॒र्गं लो॒कमे॑ति॒ तेषाꣳ॑ सुव॒र्गं लो॒कं य॒तां दिशः॒ सम॑व्लीयन्त॒ ते द्वे पु॒रस्ता᳚थ्स॒मीची॒ उपा॑दधत॒ द्वे॥१७॥

%5.2.3.5
प॒श्चाथ्स॒मीची॒ ताभि॒र्वै ते दिशो॑\-ऽदृꣳह॒न्॒ यद्द्वे पु॒रस्ता᳚थ्स॒मीची॑ उप॒दधा॑ति॒ द्वे प॒श्चाथ्स॒मीची॑ दि॒शां विधृ॑त्या॒ अथो॑ प॒शवो॒ वै छन्दाꣳ॑सि पशूने॒वास्मै॑ स॒मीचो॑ दधात्य॒ष्टावुप॑ दधात्य॒ष्टाक्ष॑रा गाय॒त्री गा॑य॒त्रो᳚\-ऽग्निर्यावा॑ने॒वाग्निस्तं चि॑नुते॒\-ऽष्टावुप॑ दधात्य॒ष्टाक्ष॑रा गाय॒त्री गा॑य॒त्री सु॑व॒र्गं लो॒कमञ्ज॑सा वेद सुव॒र्गस्य॑ लो॒कस्य॑॥१८॥

%5.2.3.6
प्रज्ञा᳚त्यै॒ त्रयो॑दश लोकं पृ॒णा उप॑ दधा॒त्येक॑विꣳशतिः॒ सम्प॑द्यन्ते प्रति॒ष्ठा वा ए॑कवि॒ꣳ॒शः प्र॑ति॒ष्ठा गार्\mbox{}ह॑पत्य एकवि॒ꣳ॒शस्यै॒व प्र॑ति॒ष्ठां गार्\mbox{}ह॑पत्य॒मनु॒ प्रति॑ तिष्ठति॒ प्रत्य॒ग्निं चि॑क्या॒नस्ति॑ष्ठति॒ य ए॒वं वेद॒ पञ्च॑चितीकं चिन्वीत प्रथ॒मं चि॑न्वा॒नः पाङ्क्तो॑ य॒ज्ञः पाङ्क्ताः᳚ प॒शवो॑ य॒ज्ञमे॒व प॒शूनव॑ रुन्द्धे॒ त्रिचि॑तीकं चिन्वीत द्वि॒तीयं॑ चिन्वा॒नस्त्रय॑ इ॒मे लो॒का ए॒ष्वे॑व लो॒केषु॑॥१९॥

%5.2.3.7
प्रति॑ तिष्ठ॒त्येक॑चितीकं चिन्वीत तृ॒तीयं॑ चिन्वा॒न ए॑क॒धा वै सु॑व॒र्गो लो॒क ए॑क॒वृतै॒व सु॑व॒र्गं लो॒कमे॑ति॒ पुरी॑षेणा॒भ्यू॑हति॒ तस्मा᳚न्मा॒ꣳ॒सेनास्थि॑ छ॒न्नन्न दु॒श्चर्मा॑ भवति॒ य ए॒वं वेद॒ पञ्च॒ चित॑यो भवन्ति प॒ञ्चभिः॒ पुरी॑षैर॒भ्यू॑हति॒ दश॒ सम्प॑द्यन्ते॒ दशा᳚क्षरा वि॒राडन्नं॑ वि॒राड्वि॒राज्ये॒वान्नाद्ये॒ प्रति॑ तिष्ठति॥२०॥

%5.2.4.0
{\anuvakamend[{अ॒द्ध्य॒व॒सा॒यय॑ति॒ ह्ये॑तद्वि॒श्वामि॑त्रस्यादधत॒ द्वे लो॒कस्य॑ लो॒केषु॑ स॒प्तच॑त्वारिꣳशच्च}]}%॥३॥

%5.2.4.1
वि वा ए॒तौ द्वि॑षाते॒ यश्च॑ पु॒राग्निर्यश्चो॒खाया॒ꣳ॒ समि॑त॒मिति॑ चत॒सृभिः॒ सं नि व॑पति च॒त्वारि॒ छन्दाꣳ॑सि॒ छन्दाꣳ॑सि॒ खलु॒ वा अ॒ग्नेः प्रि॒या त॒नूः प्रि॒ययै॒वैनौ॑ त॒नुवा॒ सꣳ शा᳚स्ति॒ समि॑त॒मित्या॑ह॒ तस्मा॒द्ब्रह्म॑णा क्ष॒त्रꣳ समे॑ति॒ यथ्सं॒न्युप्य॑ वि॒हर॑ति॒ तस्मा॒द्ब्रह्म॑णा क्ष॒त्रं व्ये᳚त्यृ॒तुभिः॑॥२१॥

%5.2.4.2
वा ए॒तं दी᳚क्षयन्ति॒ स ऋ॒तुभि॑रे॒व वि॒मुच्यो॑ मा॒तेव॑ पु॒त्रं पृ॑थि॒वी पु॑री॒ष्य॑मित्या॑ह॒र्तुभि॑रे॒वैनं॑ दीक्षयि॒त्वर्तुभि॒र्वि मु॑ञ्चति वैश्वान॒र्या शि॒क्य॑मा द॑त्ते स्व॒दय॑त्ये॒वैन॑न्नैर्\mbox{}ऋ॒तीः कृ॒ष्णास्ति॒स्रस्तुष॑पक्वा भवन्ति॒ निर्\mbox{}ऋ॑त्यै॒ वा ए॒तद्भा॑ग॒धेयं॒ यत्तुषा॒ निर्\mbox{}ऋ॑त्यै रू॒पं कृ॒ष्णꣳ रू॒पेणै॒व निर्\mbox{}ऋ॑तिं नि॒रव॑दयत इ॒मां दिशं॑ यन्त्ये॒षा॥२२॥

%5.2.4.3
वै निर्\mbox{}ऋ॑त्यै॒ दिख्स्वाया॑मे॒व दि॒शि निर्\mbox{}ऋ॑तिं नि॒रव॑दयते॒ स्वकृ॑त॒ इरि॑ण॒ उप॑ दधाति प्रद॒रे वै॒तद्वै निर्\mbox{}ऋ॑त्या आ॒यत॑न॒ꣴ॒ स्व ए॒वायत॑ने॒ निर्\mbox{}ऋ॑तिं नि॒रव॑दयते शि॒क्य॑म॒भ्युप॑ दधाति नैर्\mbox{}ऋ॒तो वै पाशः॑ सा॒क्षादे॒वैनं॑ निर्\mbox{}ऋतिपा॒शान्मु॑ञ्चति ति॒स्र उप॑ दधाति त्रेधाविहि॒तो वै पुरु॑षो॒ यावा॑ने॒व पुरु॑ष॒स्तस्मा॒न्निर्\mbox{}ऋ॑ति॒मव॑ यजते॒ परा॑ची॒रुप॑॥२३॥

%5.2.4.4
द॒धा॒ति॒ परा॑चीमे॒वास्मा॒न्निर्\mbox{}ऋ॑ति॒म्प्र णु॑द॒ते\-ऽप्र॑तीक्ष॒मा य॑न्ति॒ निर्\mbox{}ऋ॑त्या अ॒न्तर्\mbox{}हि॑त्यै मार्जयि॒त्वोप॑ तिष्ठन्ते मेध्य॒त्वाय॒ गार्\mbox{}ह॑पत्य॒मुप॑ तिष्ठन्ते निर्\mbox{}ऋतिलो॒क ए॒व च॑रि॒त्वा पू॒ता दे॑वलो॒कमु॒पाव॑र्तन्त॒ एक॒योप॑ तिष्ठन्त एक॒धैव यज॑माने वी॒र्यं॑ दधति नि॒वेश॑नः सं॒गम॑नो॒ वसू॑ना॒मित्या॑ह प्र॒जा वै प॒शवो॒ वसु॑ प्र॒जयै॒वैन॑म्प॒शुभिः॒ सम॑र्धयन्ति॥२४॥

%5.2.5.0
{\anuvakamend[{ऋ॒तुभि॑रे॒षा परा॑ची॒रुपा॒ष्टाच॑त्वारिꣳशच्च}]}%॥४॥

%5.2.5.1
पु॒रु॒ष॒मा॒त्रेण॒ वि मि॑मीते य॒ज्ञेन॒ वै पुरु॑षः॒ सम्मि॑तो यज्ञप॒रुषै॒वैनं॒ वि मि॑मीते॒ यावा॒न्पुरु॑ष ऊ॒र्ध्वबा॑हु॒स्तावा᳚न्भव\-त्ये॒ताव॒द्वै पुरु॑षे वी॒र्यं॑ वी॒र्ये॑णै॒वैनं॒ वि मि॑मीते प॒क्षी भ॑वति॒ न ह्य॑प॒क्षः पति॑तु॒मर्\mbox{}ह॑त्यर॒त्निना॑ प॒क्षौ द्राघी॑याꣳसौ भवत॒स्तस्मा᳚त्प॒क्षप्र॑वयाꣳसि॒ वयाꣳ॑सि व्याममा॒त्रौ प॒क्षौ च॒ पुच्छं॑ च भवत्ये॒ताव॒द्वै पुरु॑षे वी॒र्यम्᳚॥२५॥

%5.2.5.2
वी॒र्य॑सम्मितो॒ वेणु॑ना॒ वि मि॑मीत आग्ने॒यो वै वेणुः॑ सयोनि॒त्वाय॒ यजु॑षा युनक्ति॒ यजु॑षा कृषति॒ व्यावृ॑त्त्यै षड्ग॒वेन॑ कृषति॒ षड्वा ऋ॒तव॑ ऋ॒तुभि॑रे॒वैनं॑ कृषति॒ यद्द्वा॑दशग॒वेन॑ संवथ्स॒रेणै॒वेयं वा अ॒ग्नेर॑तिदा॒हाद॑बिभे॒थ्सैतद्द्वि॑गु॒णम॑पश्यत्कृ॒ष्टं चाकृ॑ष्टं च॒ ततो॒ वा इ॒मां नात्य॑दह॒द्यत्कृ॒ष्टं चाकृ॑ष्टं च॥२६॥

%5.2.5.3
भव॑त्य॒स्या अन॑तिदाहाय द्विगु॒णं त्वा अ॒ग्निमुद्य॑न्तुमर्\mbox{}ह॒तीत्या॑हु॒र्यत्कृ॒ष्टं चाकृ॑ष्टं च॒ भव॑त्य॒ग्नेरुद्य॑त्या ए॒ताव॑न्तो॒ वै प॒शवो᳚ द्वि॒पाद॑श्च॒ चतु॑ष्पादश्च॒ तान् यत्प्राच॑ उथ्सृ॒जेद्रु॒द्रायापि॑ दध्या॒द्यद्द॑क्षि॒णा पि॒तृभ्यो॒ नि धु॑वे॒द्यत्प्र॒तीचो॒ रक्षाꣳ॑सि हन्यु॒रुदी॑च॒ उथ्सृ॑जत्ये॒षा वै दे॑वमनु॒ष्याणाꣳ॑ शा॒न्ता दिक्॥२७॥

%5.2.5.4
तामे॒वैना॒ननूथ्सृ॑ज॒त्यथो॒ खल्वि॒मां दिश॒मुथ्सृ॑जत्य॒सौ वा आ॑दि॒त्यः प्रा॒णः प्रा॒णमे॒वैना॒ननूथ्सृ॑जति दक्षि॒णा प॒र्याव॑र्तन्ते॒ स्वमे॒व वी॒र्य॑मनु॑ प॒र्याव॑र्तन्ते॒ तस्मा॒द्दक्षि॒णो\-ऽर्ध॑ आ॒त्मनो॑ वी॒र्या॑वत्त॒रो\-ऽथो॑ आदि॒त्यस्यै॒वावृत॒मनु॑ प॒र्याव॑र्तन्ते॒ तस्मा॒त्परा᳚ञ्चः प॒शवो॒ वि ति॑ष्ठन्ते प्र॒त्यं च॒ आ व॑र्तन्ते ति॒स्रस्ति॑स्रः॒ सीताः᳚॥२८॥

%5.2.5.5
कृ॒ष॒ति॒ त्रि॒वृत॑मे॒व य॑ज्ञमु॒खे वि या॑तय॒त्योष॑धीर्वपति॒ ब्रह्म॒णान्न॒मव॑ रुन्द्धे॒\-ऽर्के᳚\-ऽर्कश्ची॑यते चतुर्द॒शभि॑र्वपति स॒प्त ग्रा॒म्या ओष॑धयः स॒प्तार॒ण्या उ॒भयी॑षा॒मव॑रुद्ध्या॒ अन्न॑स्यान्नस्य वप॒त्यन्न॑स्यान्न॒स्याव॑रुद्ध्यै कृ॒ष्टे व॑पति कृ॒ष्टे ह्योष॑धयः प्रति॒तिष्ठ॑न्त्यनुसी॒तं व॑पति॒ प्रजा᳚त्यै द्वाद॒शसु॒ सीता॑सु वपति॒ द्वाद॑श॒ मासाः᳚ संवथ्स॒रः सं॑वथ्स॒रेणै॒वास्मा॒ अन्न॑म्पचति॒ यद॑ग्नि॒चित्॥२९॥

%5.2.5.6
अन॑वरुद्धस्याश्ञी॒यादव॑रुद्धेन॒ व्यृ॑द्ध्येत॒ ये वन॒स्पती॑नाम्फल॒ग्रह॑य॒स्तानि॒ध्मे\-ऽपि॒ प्रोक्षे॒दन॑वरुद्ध॒स्याव॑रुद्ध्यै दि॒ग्भ्यो लो॒ष्टान्थ्सम॑स्यति दि॒शामे॒व वी॒र्य॑मव॒रुध्य॑ दि॒शां वी॒र्ये᳚\-ऽग्निं चि॑नुते॒ यं द्वि॒ष्याद्यत्र॒ स स्यात्तस्यै॑ दि॒शो लो॒ष्टमा ह॑रे॒दिष॒मूर्ज॑म॒हमि॒त आ द॑द॒ इतीष॑मे॒वोर्जं॒ तस्यै॑ दि॒शो\-ऽव॑ रुन्द्धे॒ क्षोधु॑को भवति॒ यस्तस्यां᳚ दि॒शि भव॑त्युत्तरवे॒दिमुप॑ वपत्युत्तरवे॒द्याꣳ ह्य॑ग्निश्ची॒यते\-ऽथो॑ प॒शवो॒ वा उ॑त्तरवे॒दिः प॒शूने॒वाव॑ रु॒न्द्धे\-ऽथो॑ यज्ञप॒रुषो\-ऽन॑न्तरित्यै॥३०॥

%5.2.6.0
{\anuvakamend[{च॒ भ॒व॒त्ये॒ताव॒द्वै पुरु॑षे वी॒र्यं॑ यत्कृ॒ष्टञ्चाकृ॑ष्टं च॒ दिख्सीता॑ अग्नि॒चिदव॒ पञ्च॑विꣳशतिश्च}]}%॥५॥

%5.2.6.1
अग्ने॒ तव॒ श्रवो॒ वय॒ इति॒ सिक॑ता॒ नि व॑पत्ये॒तद्वा अ॒ग्नेर्वै᳚श्वान॒रस्य॑ सू॒क्तꣳ सू॒क्तेनै॒व वै᳚श्वान॒रमव॑ रुन्द्धे ष॒ड्भिर्नि व॑पति॒ षड्वा ऋ॒तवः॑ सं वथ्स॒रः सं॑वथ्स॒रो᳚\-ऽग्निर्वै᳚श्वान॒रः सा॒क्षादे॒व वै᳚श्वान॒रमव॑ रुन्द्धे समु॒द्रं वै नामै॒तच्छन्दः॑ समु॒द्रमनु॑ प्र॒जाः प्र जा॑यन्ते॒ यदे॒तेन॒ सिक॑ता नि॒वप॑ति प्र॒जानां᳚ प्र॒जन॑ना॒येन्द्रः॑॥३१॥

%5.2.6.2
वृ॒त्राय॒ वज्र॒म्प्राह॑र॒थ्स त्रे॒धा व्य॑भव॒थ्स्फ्यस्तृती॑य॒ꣳ॒ रथ॒स्तृती॑यं॒ यूप॒स्तृती॑यं॒ ये᳚\-ऽन्तःश॒रा अशी᳚र्यन्त॒ ताः शर्क॑रा अभव॒न्तच्छर्क॑राणाꣳ शर्कर॒त्वं वज्रो॒ वै शर्क॑राः प॒शुर॒ग्निर्यच्छर्क॑राभिर॒ग्निं प॑रिमि॒नोति॒ वज्रे॑णै॒वास्मै॑ प॒शून्परि॑ गृह्णाति॒ तस्मा॒द्वज्रे॑ण प॒शवः॒ परि॑गृहीता॒स्तस्मा॒थ्स्थेया॒नस्थे॑यसो॒ नोप॑ हरते त्रिस॒प्ताभिः॑॥३२॥

%5.2.6.3
प॒शुका॑मस्य॒ परि॑ मिनुयाथ्स॒प्त वै शी॑र्\mbox{}ष॒ण्याः᳚ प्रा॒णाः प्रा॒णाः प॒शवः॑ प्रा॒णैरे॒वास्मै॑ प॒शूनव॑ रुन्द्धे त्रिण॒वाभि॒\-र्भ्रातृ॑व्यवतस्त्रि॒वृत॑मे॒व वज्रꣳ॑ स॒म्भृत्य॒ भ्रातृ॑व्याय॒ प्र ह॑रति॒ स्तृत्या॒ अप॑रिमिताभिः॒ परि॑ मिनुया॒दप॑रिमित॒स्याव॑रुद्ध्यै॒ यं का॒मये॑ताप॒शुः स्या॒दित्यप॑रिमित्य॒ तस्य॒ शर्क॑राः॒ सिक॑ता॒ व्यू॑हे॒दप॑रिगृहीत ए॒वास्य॑ विषू॒चीन॒ꣳ॒ रेतः॒ परा॒ सिञ्चत्यप॒शुरे॒व भ॑वति॥३३॥

%5.2.6.4
यं का॒मये॑त पशु॒मान्थ्स्या॒दिति॑ परि॒मित्य॒ तस्य॒ शर्क॑राः॒ सिक॑ता॒ व्यू॑हे॒त्परि॑गृहीत ए॒वास्मै॑ समी॒चीन॒ꣳ॒ रेतः॑ सिञ्चति पशु॒माने॒व भ॑वति सौ॒म्या व्यू॑हति॒ सोमो॒ वै रे॑तो॒धा रेत॑ ए॒व तद्द॑धाति गायत्रि॒या ब्रा᳚ह्म॒णस्य॑ गाय॒त्रो हि ब्रा᳚ह्म॒णस्त्रि॒ष्टुभा॑ राज॒न्य॑स्य॒ त्रैष्टु॑भो॒ हि रा॑ज॒न्यः॑ शं॒ युम्बा॑र्\mbox{}हस्प॒त्यम्मेधो॒ नोपा॑नम॒थ्सो᳚\-ऽग्निम्प्रावि॑शत्॥३४॥

%5.2.6.5
सो᳚\-ऽग्नेः कृष्णो॑ रू॒पं कृ॒त्वोदा॑यत॒ सो\-ऽश्व॒म्प्रावि॑श॒थ्सो\-ऽश्व॑स्यावान्तरश॒फो॑\-ऽभव॒द्यदश्व॑माक्र॒मय॑ति॒ य ए॒व मेधो\-ऽश्व॒म्प्रावि॑श॒त्तमे॒वाव॑ रुन्द्धे प्र॒जाप॑तिना॒ग्निश्चे॑त॒व्य॑ इत्या॑हुः प्राजाप॒त्यो\-ऽश्वो॒ यदश्व॑माक्र॒मय॑ति प्र॒जाप॑तिनै॒वाग्निं चि॑नुते पुष्करप॒र्णमुप॑ दधाति॒ योनि॒र्वा अ॒ग्नेः पु॑ष्करप॒र्णꣳ सयो॑निमे॒वाग्निं चि॑नुते॒\-ऽपां पृ॒ष्ठम॒सीत्युप॑ दधात्य॒पां वा ए॒तत्पृ॒ष्ठं यत्पु॑ष्करप॒र्णꣳ रू॒पेणै॒वैन॒दुप॑ दधाति॥३५॥

%5.2.7.0
{\anuvakamend[{इन्द्रः॑ प॒शुका॑मस्य भवत्यविश॒थ्सयो॑निं विꣳश॒तिश्च॑}]}%॥६॥

%5.2.7.1
ब्रह्म॑ जज्ञा॒नमिति॑ रु॒क्ममुप॑ दधाति॒ ब्रह्म॑मुखा॒ वै प्र॒जाप॑तिः प्र॒जा अ॑सृजत॒ ब्रह्म॑मुखा ए॒व तत्प्र॒जा यज॑मानः सृजते॒ ब्रह्म॑ जज्ञा॒नमित्या॑ह॒ तस्मा᳚द्ब्राह्म॒णो मुख्यो॒ मुख्यो॑ भवति॒ य ए॒वं वेद॑ ब्रह्मवा॒दिनो॑ वदन्ति॒ न पृ॑थि॒व्यां नान्तरि॑क्षे॒ न दि॒व्य॑ग्निश्चे॑त॒व्य॑ इति॒ यत्पृ॑थि॒व्यां चि॑न्वी॒त पृ॑थि॒वीꣳ शु॒चार्प॑ये॒न्नौष॑धयो॒ न वन॒स्पत॑यः॥३६॥

%5.2.7.2
प्र जा॑येर॒न् यद॒न्तरि॑क्षे चिन्वी॒तान्तरि॑क्षꣳ शु॒चार्प॑ये॒न्न वयाꣳ॑सि॒ प्र जा॑येर॒न् यद्दि॒वि चि॑न्वी॒त दिवꣳ॑ शु॒चार्प॑ये॒न्न प॒र्जन्यो॑ वर्\mbox{}षेद्रु॒क्ममुप॑ दधात्य॒मृतं॒ वै हिर॑ण्यम॒मृत॑ ए॒वाग्निं चि॑नुते॒ प्रजा᳚त्यै हिर॒ण्मयं॒ पुरु॑ष॒मुप॑ दधाति यजमानलो॒कस्य॒ विधृ॑त्यै॒ यदिष्ट॑काया॒ आतृ॑ण्णमनूपद॒ध्यात्प॑शू॒नां च॒ यज॑मानस्य च प्रा॒णमपि॑ दध्याद्दक्षिण॒तः॥३७॥

%5.2.7.3
प्राञ्च॒मुप॑ दधाति दा॒धार॑ यजमानलो॒कन्न प॑शू॒नां च॒ यज॑मानस्य च प्रा॒णमपि॑ दधा॒त्यथो॒ खल्विष्ट॑काया॒ आतृ॑ण्ण॒मनूप॑ दधाति प्रा॒णाना॒मुथ्सृ॑ष्ट्यै द्र॒फ्सश्च॑स्क॒न्देत्य॒भि मृ॑शति॒ होत्रा᳚स्वे॒वैनं॒ प्रति॑ ष्ठापयति॒ स्रुचा॒वुप॑ दधा॒त्याज्य॑स्य पू॒र्णां का᳚र्ष्मर्य॒मयीं᳚ द॒ध्नः पू॒र्णामौदु॑म्बरीमि॒यं वै का᳚र्ष्मर्य॒मय्य॒सावौदु॑म्बरी॒मे ए॒वोप॑ धत्ते॥३८॥

%5.2.7.4
तू॒ष्णीमुप॑ दधाति॒ न हीमे यजु॒षाप्तु॒मर्\mbox{}ह॑ति॒ दक्षि॑णां कार्ष्मर्य॒मयी॒मुत्त॑रा॒मौदु॑म्बरी॒न्तस्मा॑द॒स्या अ॒सावुत्त॒राज्य॑स्य पू॒र्णां का᳚र्ष्मर्य॒मयीं॒ वज्रो॒ वा आज्यं॒ वज्रः॑ कार्ष्म॒र्यो॑ वज्रे॑णै॒व य॒ज्ञस्य॑ दक्षिण॒तो रक्षा॒ꣳ॒स्यप॑ हन्ति द॒ध्नः पू॒र्णामौदु॑म्बरीम्प॒शवो॒ वै दध्यूर्गु॑दु॒म्बरः॑ प॒शुष्वे॒वोर्जं॑ दधाति पू॒र्णे उप॑ दधाति पू॒र्णे ए॒वैनम्᳚॥३९॥

%5.2.7.5
अ॒मुष्मि॑ल्लोँ॒क उप॑ तिष्ठेते वि॒राज्य॒ग्निश्चे॑त॒व्य॑ इत्या॒॑हुः स्रुग्वै वि॒राड्यथ्स्रुचा॑वुप॒दधा॑ति वि॒राज्ये॒वाग्निं चि॑नुते यज्ञमु॒खेय॑ज्ञमुखे॒ वै क्रि॒यमा॑णे य॒ज्ञꣳ रक्षाꣳ॑सि जिघाꣳसन्ति यज्ञमु॒खꣳ रु॒क्मो यद्रु॒क्मं व्या॑घा॒रय॑ति यज्ञमु॒खादे॒व रक्षा॒ꣳ॒स्यप॑ हन्ति प॒ञ्चभि॒र्व्याघा॑रयति॒ पाङ्क्तो॑ य॒ज्ञो यावा॑ने॒व य॒ज्ञस्तस्मा॒द्रक्षा॒ꣳ॒स्यप॑ हन्त्यक्ष्ण॒या व्याघा॑रयति॒ तस्मा॑दक्ष्ण॒या प॒शवो\-ऽङ्गा॑नि॒ प्र ह॑रन्ति॒ प्रति॑ष्ठित्यै॥४०॥

%5.2.8.0
{\anuvakamend[{वन॒स्पत॑यो दक्षिण॒तो ध॑त्त एन॒न्तस्मा॑दक्ष्ण॒या पञ्च॑ च}]}%॥७॥

%5.2.8.1
स्व॒य॒मा॒तृ॒ण्णामुप॑ दधाती॒यं वै स्व॑यमातृ॒ण्णेमामे॒वोप॑ ध॒त्ते\-ऽश्व॒मुप॑ घ्रापयति प्रा॒णमे॒वास्यां᳚ दधा॒त्यथो᳚ प्राजाप॒त्यो वा अश्वः॑ प्र॒जाप॑तिनै॒वाग्निं चि॑नुते प्रथ॒मेष्ट॑कोपधी॒यमा॑ना पशू॒नां च॒ यज॑मानस्य च प्रा॒णमपि॑ दधाति स्वयमातृ॒ण्णा भ॑वति प्रा॒णाना॒मुथ्सृ॑ष्ट्या॒ अथो॑ सुव॒र्गस्य॑ लो॒कस्यानु॑ख्यात्या अ॒ग्नाव॒ग्निश्चे॑त॒व्य॑ इत्या॑हुरे॒ष वै॥४१॥

%5.2.8.2
अ॒ग्निर्वै᳚श्वान॒रो यद्ब्रा᳚ह्म॒णस्तस्मै᳚ प्रथ॒मामिष्ट॑कां॒ यजु॑ष्कृता॒म्प्र य॑च्छे॒त्ताम्ब्रा᳚ह्म॒णश्चोप॑ दध्याताम॒ग्नावे॒व तद॒ग्निं चि॑नुत ईश्व॒रो वा ए॒ष आर्ति॒मार्तो॒र्यो\-ऽवि॑द्वा॒निष्ट॑कामुप॒दधा॑ति॒ त्रीन् वरा᳚न्दद्या॒त्त्रयो॒ वै प्रा॒णाः प्रा॒णाना॒ꣴ॒ स्पृत्यै॒ द्वावे॒व देयौ॒ द्वौ हि प्रा॒णावेक॑ ए॒व देय॒ एको॒ हि प्रा॒णः प॒शुः॥४२॥

%5.2.8.3
वा ए॒ष यद॒ग्निर्न खलु॒ वै प॒शव॒ आय॑वसे रमन्ते दूर्वेष्ट॒कामुप॑ दधाति पशू॒नां धृत्यै॒ द्वाभ्यां॒ प्रति॑ष्ठित्यै॒ काण्डा᳚त्काण्डात्प्र॒रोह॒न्तीत्या॑ह॒ काण्डे॑नकाण्डेन॒ ह्ये॑षा प्र॑ति॒तिष्ठ॑त्ये॒वा नो॑ दूर्वे॒ प्र त॑नु स॒हस्रे॑ण श॒तेन॒ चेत्या॑ह साह॒स्रः प्र॒जाप॑तिः प्र॒जाप॑ते॒राप्त्यै॑ देवल॒क्ष्मं वै त्र्या॑लिखि॒ता तामुत्त॑रलक्ष्माणं दे॒वा उपा॑दध॒ताध॑रलक्ष्माण॒मसु॑रा॒ यम्॥४३॥

%5.2.8.4
का॒मये॑त॒ वसी॑यान्थ्स्या॒दित्युत्त॑रलक्ष्माणं॒ तस्योप॑ दध्या॒द्वसी॑याने॒व भ॑वति॒ यं का॒मये॑त॒ पापी॑यान्थ्स्या॒दित्यध॑र\-लक्ष्माणं॒ तस्योप॑ दध्यादसुरयो॒निमे॒वैन॒मनु॒ परा॑ भावयति॒ पापी॑यान्भवति त्र्यालिखि॒ता भ॑वती॒मे वै लो॒का\-स्त्र्या॑लिखि॒तैभ्य ए॒व लो॒केभ्यो॒ भ्रातृ॑व्यम॒न्तरे॒त्यङ्गि॑रसः सुव॒र्गं लो॒कं य॒तः पु॑रो॒डाशः॑ कू॒र्मो भू॒त्वानु॒ प्रास॑र्पत्॥४४॥

%5.2.8.5
यत्कू॒र्ममु॑प॒दधा॑ति॒ यथा᳚ क्षेत्र॒विदञ्ज॑सा॒ नय॑त्ये॒वमे॒वैनं॑ कू॒र्मः सु॑व॒र्गं लो॒कमञ्ज॑सा नयति॒ मेधो॒ वा ए॒ष प॑शू॒नां यत्कू॒र्मो यत्कू॒र्ममु॑प॒दधा॑ति॒ स्वमे॒व मेध॒म्पश्य॑न्तः प॒शव॒ उप॑ तिष्ठन्ते श्मशा॒नं वा ए॒तत्क्रि॑यते॒ यन्मृ॒तानां᳚ पशू॒नाꣳ शी॒र्\mbox{}षाण्यु॑पधी॒यन्ते॒ यज्जीव॑न्तं कू॒र्ममु॑प॒दधा॑ति॒ तेनाश्म॑शानचिद्वास्त॒व्यो॑ वा ए॒ष यत्॥४५॥

%5.2.8.6
कू॒र्मो मधु॒ वाता॑ ऋताय॒त इति॑ द॒ध्ना म॑धुमि॒श्रेणा॒भ्य॑नक्ति स्व॒दय॑त्ये॒वैन॑ङ्ग्रा॒म्यं वा ए॒तदन्नं॒ यद्दध्या॑र॒ण्यम्मधु॒ यद्द॒ध्ना म॑धुमि॒श्रेणा᳚भ्य॒नक्त्यु॒भय॒स्याव॑रुद्ध्यै म॒ही द्यौः पृ॑थि॒वी च॑ न॒ इत्या॑हा॒भ्यामे॒वैन॑मुभ॒यतः॒ परि॑ गृह्णाति॒ प्राञ्च॒मुप॑ दधाति॒ सुव॒र्गस्य॑ लो॒कस्य॒ सम॑ष्ट्यै पु॒रस्ता᳚त्प्र॒त्यञ्च॒मुप॑ दधाति॒ तस्मा᳚त्॥४६॥

%5.2.8.7
पु॒रस्ता᳚त्प्र॒त्यञ्चः॑ प॒शवो॒ मेध॒मुप॑ तिष्ठन्ते॒ यो वा अप॑नाभिम॒ग्निं चि॑नु॒ते यज॑मानस्य॒ नाभि॒मनु॒ प्र वि॑शति॒ स ए॑नमीश्व॒रो हिꣳसि॑तोरु॒लूख॑ल॒मुप॑ दधात्ये॒षा वा अ॒ग्नेर्नाभिः॒ सना॑भिमे॒वाग्निं चि॑नु॒ते\-ऽहिꣳ॑साया॒ औ॑दुम्बरम्भव॒त्यूर्ग्वा उ॑दु॒म्बर॒ ऊर्ज॑मे॒वाव॑ रुन्द्धे मध्य॒त उप॑ दधाति मध्य॒त ए॒वास्मा॒ ऊर्जं॑ दधाति॒ तस्मा᳚न्मध्य॒त ऊ॒र्जा भु॑ञ्जत॒ इय॑द्भवति प्र॒जाप॑तिना यज्ञमु॒खेन॒ सम्मि॑त॒मव॑ ह॒न्त्यन्न॑मे॒वाक॑र्वैष्ण॒व्यर्चोप॑ दधाति॒ विष्णु॒र्वै य॒ज्ञो वै᳚ष्ण॒वा वन॒स्पत॑यो य॒ज्ञ ए॒व य॒ज्ञं प्रति॑ ष्ठापयति॥४७॥

%5.2.9.0
{\anuvakamend[{ए॒ष वै प॒शुर्यम॑सर्पदे॒ष यत्तस्मा॒त्तस्मा᳚थ्स॒प्तविꣳ॑शतिश्च}]}%॥८॥

%5.2.9.1
ए॒षां वा ए॒तल्लो॒कानां॒ ज्योतिः॒ सम्भृ॑तं॒ यदु॒खा यदु॒खामु॑प॒दधा᳚त्ये॒भ्य ए॒व लो॒केभ्यो॒ ज्योति॒रव॑ रुन्द्धे मध्य॒त उप॑ दधाति मध्य॒त ए॒वास्मै॒ ज्योति॑र्दधाति॒ तस्मा᳚न्मध्य॒तो ज्योति॒रुपा᳚स्महे॒ सिक॑ताभिः पूरयत्ये॒तद्वा अ॒ग्नेर्वै᳚श्वान॒रस्य॑ रू॒पꣳ रू॒पेणै॒व वै᳚श्वान॒रमव॑ रुन्द्धे॒ यं का॒मये॑त॒ क्षोधु॑कः स्या॒दित्यू॒नां तस्योप॑॥४८॥

%5.2.9.2
द॒ध्या॒त्क्षोधु॑क ए॒व भ॑वति॒ यं का॒मये॒तानु॑पदस्य॒दन्न॑मद्या॒दिति॑ पू॒र्णां तस्योप॑ दध्या॒दनु॑पदस्यदे॒वान्न॑मत्ति स॒हस्रं॒ वै प्रति॒ पुरु॑षः पशू॒नां य॑च्छति स॒हस्र॑म॒न्ये प॒शवो॒ मध्ये॑ पुरुषशी॒र्\mbox{}षमुप॑ दधाति सवीर्य॒त्वायो॒खाया॒मपि॑ दधाति प्रति॒ष्ठामे॒वैन॑द्गमयति॒ व्यृ॑द्धं॒ वा ए॒तत्प्रा॒णैर॑मे॒ध्यं यत्पु॑रुषशी॒र्\mbox{}षम॒मृतं॒ खलु॒ वै प्रा॒णाः॥४९॥

%5.2.9.3
अ॒मृत॒ꣳ॒ हिर॑ण्यं प्रा॒णेषु॑ हिरण्यश॒ल्कान्प्रत्य॑स्यति प्रति॒ष्ठामे॒वैन॑द्गमयि॒त्वा प्रा॒णैः सम॑र्धयति द॒ध्ना म॑धुमि॒श्रेण॑ पूरयति मध॒व्यो॑\-ऽसा॒नीति॑ शृतात॒ङ्क्ये॑न मेध्य॒त्वाय॑ ग्रा॒म्यं वा ए॒तदन्नं॒ यद्दध्या॑र॒ण्यम्मधु॒ यद्द॒ध्ना म॑धुमि॒श्रेण॑ पू॒रय॑त्यु॒भय॒स्याव॑रुद्ध्यै पशुशी॒र्\mbox{}षाण्युप॑ दधाति प॒शवो॒ वै प॑शुशी॒र्\mbox{}षाणि॑ प॒शूने॒वाव॑ रुन्द्धे॒ यं का॒मये॑ताप॒शुः स्या॒दिति॑॥५०॥

%5.2.9.4
वि॒षू॒चीना॑नि॒ तस्योप॑ दध्या॒द्विषू॑च ए॒वास्मा᳚त्प॒शून्द॑धात्यप॒शुरे॒व भ॑वति॒ यं का॒मये॑त पशु॒मान्थ्स्या॒दिति॑ समी॒चीना॑नि॒ तस्योप॑ दध्याथ्स॒मीच॑ ए॒वास्मै॑ प॒शून्द॑धाति पशु॒माने॒व भ॑वति पु॒रस्ता᳚त्प्रती॒चीन॒मश्व॒स्योप॑ दधाति प॒श्चात्प्रा॒चीन॑मृष॒भस्याप॑शवो॒ वा अ॒न्ये गो॑अ॒श्वेभ्यः॑ प॒शवो॑ गोअ॒श्वाने॒वास्मै॑ स॒मीचो॑ दधात्ये॒ताव॑न्तो॒ वै प॒शवः॑॥५१॥

%5.2.9.5
द्वि॒पाद॑श्च॒ चतु॑ष्पादश्च॒ तान् वा ए॒तद॒ग्नौ प्र द॑धाति॒ यत्प॑शुशी॒र्\mbox{}षाण्यु॑प॒दधा᳚त्य॒मुमा॑र॒ण्यमनु॑ ते दिशा॒मीत्या॑ह ग्रा॒म्येभ्य॑ ए॒व प॒शुभ्य॑ आर॒ण्यान्प॒शूञ्छुच॒मनूथ्सृ॑जति॒ तस्मा᳚थ्स॒माव॑त्पशू॒नां प्र॒जाय॑मानानामार॒ण्याः प॒शवः॒ कनी॑याꣳसः शु॒चा ह्यृ॑ताः स॑र्पशी॒र्\mbox{}षमुप॑ दधाति॒ यैव स॒र्पे त्विषि॒स्तामे॒वाव॑ रुन्द्धे॥५२॥

%5.2.9.6
यथ्स॑मी॒चीन॑म्पशुशी॒र्\mbox{}षैरु॑पद॒ध्याद्ग्रा॒म्यान्प॒शून्दꣳशु॑काः स्यु॒र्यद्वि॑षू॒चीन॑मार॒ण्यान् यजु॑रे॒व व॑दे॒दव॒ तां त्विषिꣳ॑ रुन्द्धे॒ या स॒र्पे न ग्रा॒म्यान्प॒शून् हि॒नस्ति॒ नार॒ण्यानथो॒ खलू॑प॒धेय॑मे॒व यदु॑प॒दधा॑ति॒ तेन॒ तां त्विषि॒मव॑ रुन्द्धे॒ या स॒र्पे यद्यजु॒र्वद॑ति॒ तेन॑ शा॒न्तम्॥५३॥

%5.2.10.0
{\anuvakamend[{ऊ॒नान्तस्योप॑ प्रा॒णाः स्या॒दिति॒ वै प॒शवो॑ रुन्धे॒ चतु॑श्चत्वारिꣳशच्च}]}%॥९॥

%5.2.10.1
प॒शुर्वा ए॒ष यद॒ग्निर्योनिः॒ खलु॒ वा ए॒षा प॒शोर्वि क्रि॑यते॒ यत्प्रा॒चीन॑मैष्ट॒काद्यजुः॑ क्रि॒यते॒ रेतो॑\-ऽप॒स्या॑ अप॒स्या॑ उप॑ दधाति॒ योना॑वे॒व रेतो॑ दधाति॒ पञ्चोप॑ दधाति॒ पाङ्क्ताः᳚ प॒शवः॑ प॒शूने॒वास्मै॒ प्र ज॑नयति॒ पञ्च॑ दक्षिण॒तो वज्रो॒ वा अ॑प॒स्या॑ वज्रे॑णै॒व य॒ज्ञस्य॑ दक्षिण॒तो रक्षा॒ꣳ॒स्यप॑ हन्ति॒ पञ्च॑ प॒श्चात्॥५४॥

%5.2.10.2
प्राची॒रुप॑ दधाति प॒श्चाद्वै प्रा॒चीन॒ꣳ॒ रेतो॑ धीयते प॒श्चादे॒वास्मै᳚ प्रा॒चीन॒ꣳ॒ रेतो॑ दधाति॒ पञ्च॑ पु॒रस्ता᳚त्प्र॒तीची॒रुप॑ दधाति॒ पञ्च॑ प॒श्चात्प्राची॒स्तस्मा᳚त्प्रा॒चीन॒ꣳ॒ रेतो॑ धीयते प्र॒तीचीः᳚ प्र॒जा जा॑यन्ते॒ पञ्चो᳚त्तर॒तश्छ॑न्द॒स्याः᳚ प॒शवो॒ वै छ॑न्द॒स्याः᳚ प॒शूने॒व प्रजा॑ता॒न्थ्स्वमा॒यत॑नम॒भि पर्यू॑हत इ॒यं वा अ॒ग्नेर॑तिदा॒हाद॑बिभे॒थ्सैताः॥५॥

%5.2.10.3
अ॒प॒स्या॑ अपश्य॒त्ता उपा॑धत्त॒ ततो॒ वा इ॒मां नात्य॑दह॒द्यद॑प॒स्या॑ उप॒दधा᳚त्य॒स्या अन॑तिदाहायो॒वाच॑ हे॒यमद॒दिथ्स ब्रह्म॒णान्नं॒ यस्यै॒ता उ॑पधी॒यान्तै॒ य उ॑ चैना ए॒वं वेद॒दिति॑ प्राण॒भृत॒ उप॑ दधाति॒ रेत॑स्ये॒व प्रा॒णान्द॑धाति॒ तस्मा॒द्वद॑न्प्रा॒णन्पश्य॑ञ्छृ॒ण्वन्प॒शुर्जा॑यते॒\-ऽयम्पु॒रः॥५६॥

%5.2.10.4
भुव॒ इति॑ पु॒रस्ता॒दुप॑ दधाति प्रा॒णमे॒वैताभि॑र्दाधारा॒यं द॑क्षि॒णा वि॒श्वक॒र्मेति॑ दक्षिण॒तो मन॑ ए॒वैताभि॑र्दाधारा॒यम्प॒श्चाद्वि॒श्वव्य॑चा॒ इति॑ प॒श्चाच्चक्षु॑रे॒वैताभि॑र्दाधारे॒दमु॑त्त॒राथ्सुव॒रित्यु॑त्तर॒तः श्रोत्र॑मे॒वैताभि॑र्दाधारे॒यमु॒परि॑ म॒तिरित्यु॒परि॑ष्टा॒द्वाच॑मे॒वैताभि॑र्दाधार॒ दश॑द॒शोप॑ दधाति सवीर्य॒त्वाया᳚क्ष्ण॒या॥५७॥

%5.2.10.5
उप॑ दधाति॒ तस्मा॑दक्ष्ण॒या प॒शवो\-ऽङ्गा॑नि॒ प्र ह॑रन्ति॒ प्रति॑ष्ठित्यै॒ याः प्राची॒स्ताभि॒र्वसि॑ष्ठ आर्ध्नो॒द्या द॑क्षि॒णा ताभि॑र्भ॒रद्वा॑जो॒ याः प्र॒तीची॒स्ताभि॑र्वि॒श्वामि॑त्रो॒ या उदी॑ची॒स्ताभि॑र्ज॒मद॑ग्नि॒र्या ऊ॒र्ध्वास्ताभि॑र्वि॒श्वक॑र्मा॒ य ए॒वमे॒तासा॒मृद्धिं॒ वेद॒र्ध्नोत्ये॒व य आ॑सामे॒वम्ब॒न्धुतां॒ वेद॒ बन्धु॑मान्भवति॒ य आ॑सामे॒वं कॢप्तिं॒ वेद॒ कल्प॑ते॥५८॥

%5.2.10.6
अ॒स्मै॒ य आ॑सामे॒वमा॒यत॑नं॒ वेदा॒यत॑नवान्भवति॒ य आ॑सामे॒वम्प्र॑ति॒ष्ठां वेद॒ प्रत्ये॒व ति॑ष्ठति प्राण॒भृत॑ उप॒धाय॑ सं॒यत॒ उप॑ दधाति प्रा॒णाने॒वास्मि॑न्धि॒त्वा सं॒यद्भिः॒ सं य॑च्छति॒ तथ्सं॒यताꣳ॑ संय॒त्त्वमथो᳚ प्रा॒ण ए॒वापा॒नं द॑धाति॒ तस्मा᳚त्प्राणापा॒नौ सं च॑रतो॒ विषू॑ची॒रुप॑ दधाति॒ तस्मा॒द्विष्व॑ञ्चौ प्राणापा॒नौ यद्वा अ॒ग्नेरसं॑ यतम्॥५९॥

%5.2.10.7
असु॑वर्ग्यमस्य॒ तथ्सु॑व॒र्ग्यो᳚\-ऽग्निर्यथ्सं॒ यत॑ उप॒दधा॑ति॒ समे॒वैनं॑ यच्छति सुव॒र्ग्य॑मे॒वाक॒स्त्र्यवि॒र्वयः॑ कृ॒तमया॑ना॒मित्या॑ह॒ वयो॑भिरे॒वाया॒नव॑ रु॒न्द्धे\-ऽयै॒र्वयाꣳ॑सि स॒र्वतो॑ वायु॒मती᳚र्भवन्ति॒ तस्मा॑द॒यꣳ स॒र्वतः॑ पवते॥६०॥

%5.2.11.0
{\anuvakamend[{प॒श्चादे॒ताः पु॒रो᳚\-ऽक्ष्ण॒या कल्प॒ते\-ऽसं॑ यतं॒ पञ्च॑त्रिꣳशच्च}]}%॥10॥

%5.2.11.1
गा॒य॒त्री त्रि॒ष्टुब्जग॑त्यनु॒ष्टुक्प॒ङ्क्त्या॑ स॒ह। बृ॒ह॒त्यु॑ष्णिहा॑ क॒कुथ्सू॒चीभिः॑ शिम्यन्तु त्वा। द्वि॒पदा॒ या चतु॑ष्पदा त्रि॒पदा॒ या च॒ षट्प॑दा। सछ॑न्दा॒ या च॒ विच्छ॑न्दाः सू॒चीभिः॑ शिम्यन्तु त्वा। म॒हाना᳚म्नी रे॒वत॑यो॒ विश्वा॒ आशाः᳚ प्र॒सूव॑रीः। मेघ्या॑ वि॒द्युतो॒ वाचः॑ सू॒चीभिः॑ शिम्यन्तु त्वा। र॒ज॒ता हरि॑णीः॒ सीसा॒ युजो॑ युज्यन्ते॒ कर्म॑भिः। अश्व॑स्य वा॒जिन॑स्त्व॒चि सू॒चीभिः॑ शिम्यन्तु त्वा। नारीः᳚॥६१॥

%5.2.11.2
ते॒ पत्न॑यो॒ लोम॒ वि चि॑न्वन्तु मनी॒षया᳚। दे॒वाना॒म्पत्नी॒र्दिशः॑ सू॒चीभिः॑ शिम्यन्तु त्वा। कु॒विद॒ङ्ग यव॑मन्तो॒ यवं॑ चि॒द्यथा॒ दान्त्य॑नुपू॒र्वं वि॒यूय॑। इ॒हेहै॑षां कृणुत॒ भोज॑नानि॒ ये ब॒र्\mbox{}हिषो॒ नमो॑वृक्तिं॒ न ज॒ग्मुः॥६२॥

%5.2.12.0
{\anuvakamend[{नारी᳚स्त्रि॒ꣳ॒शच्च॑}]}%॥11॥

%5.2.12.1
कस्त्वा᳚ छ्यति॒ कस्त्वा॒ वि शा᳚स्ति॒ कस्ते॒ गात्रा॑णि शिम्यति। क उ॑ ते शमि॒ता क॒विः। ऋ॒तव॑स्त ऋतु॒धा परुः॑ शमि॒तारो॒ वि शा॑सतु। सं॒व॒थ्स॒रस्य॒ धाय॑सा॒ शिमी॑भिः शिम्यन्तु त्वा। दैव्या॑ अध्व॒र्यव॑स्त्वा॒ छ्यन्तु॒ वि च॑ शासतु। गात्रा॑णि पर्व॒शस्ते॒ शिमाः᳚ कृण्वन्तु॒ शिम्य॑न्तः। अ॒र्ध॒मा॒साः परूꣳ॑षि ते॒ मासा᳚श्छ्यन्तु॒ शिम्य॑न्तः। अ॒हो॒रा॒त्राणि॑ म॒रुतो॒ विलि॑ष्टं॥६३॥

%5.2.12.2
सू॒द॒य॒न्तु॒ ते॒। पृ॒थि॒वी ते॒\-ऽन्तरि॑क्षेण वा॒युश्छि॒द्रम्भि॑षज्यतु। द्यौस्ते॒ नक्ष॑त्रैः स॒ह रू॒पं कृ॑णोतु साधु॒या। शं ते॒ परे᳚भ्यो॒ गात्रे᳚भ्यः॒ शम॒स्त्वव॑रेभ्यः। शम॒स्थभ्यो॑ म॒ज्जभ्यः॒ शमु॑ ते त॒नुवे॑ भुवत्॥६४॥

%5.3.0.0

%5.3.0.0
{\anuvakamend[{विलि॑ष्टन्त्रि॒ꣳ॒शच्च॑}]}%॥12॥

{\anuvakamend[{उ॒थ्स॒न्न॒य॒ज्ञ इन्द्रा᳚ग्नी दे॒वा वा अ॑क्षणयास्तो॒मीया॑ अ॒ग्नेर्भा॒गो᳚\-ऽस्यग्ने॑ जा॒तान्र॒श्मिरिति॑ नाक॒सद्भि॒श्छन्दाꣳ॑सि॒ सर्वा᳚भ्यो वृष्टि॒सनी᳚र्देवासु॒राः कनी॑याꣳसः प्र॒जाप॑ते॒रक्षि॒ द्वाद॑श}]}%॥12॥ 
\prashnaend{उ॒थ्स॒न्न॒य॒ज्ञो दे॒वा वै यस्य॒ मुख्य॑वतीर्नाक॒सद्भि॑रे॒वैताभि॑र॒ष्टाच॑त्वारिꣳशत्॥48॥ उ॒थ्स॒न्न॒य॒ज्ञः स॑र्व॒त्वाय॑॥}
%%% END PRASHNA

\sect{तृतीयः प्रश्नः}\setcounter{anuvakam}{0}
\dnsub{तैत्तिरीयसंहितायां पञ्चमकाण्डे तृतीयः प्रश्नः}
%5.3.1.0
%5.3.1.1
उ॒थ्स॒न्न॒य॒ज्ञो वा ए॒ष यद॒ग्निः किं वाहै॒तस्य॑ क्रि॒यते॒ किं वा॒ न यद्वै य॒ज्ञस्य॑ क्रि॒यमा॑णस्यान्त॒र्यन्ति॒ पूय॑ति॒ वा अ॑स्य॒ तदा᳚श्वि॒नीरुप॑ दधात्य॒श्विनौ॒ वै दे॒वानां᳚ भि॒षजौ॒ ताभ्या॑मे॒वास्मै॑ भेष॒जं क॑रोति॒ पञ्चोप॑ दधाति॒ पाङ्क्तो॑ य॒ज्ञो यावा॑ने॒व य॒ज्ञस्तस्मै॑ भेष॒जं क॑रोत्यृत॒व्या॑ उप॑ दधात्यृतू॒नां कॢप्त्यै᳚॥१॥

%5.3.1.2
पञ्चोप॑ दधाति पञ्च॒ वा ऋ॒तवो॒ याव॑न्त ए॒वर्तव॒स्तान्क॑ल्पयति समा॒नप्र॑भृतयो भवन्ति समा॒नोद॑र्का॒स्तस्मा᳚थ्समा॒ना ऋ॒तव॒ एके॑न प॒देन॒ व्याव॑र्तन्ते॒ तस्मा॑दृ॒तवो॒ व्याव॑र्तन्ते प्राण॒भृत॒ उप॑ दधात्यृ॒तुष्वे॒व प्रा॒णान्द॑धाति॒ तस्मा᳚थ्समा॒नाः सन्त॑ ऋ॒तवो॒ न जी᳚र्य॒न्त्यथो॒ प्र ज॑नयत्ये॒वैना॑ने॒ष वै वा॒युर्यत्प्रा॒णो यदृ॑त॒व्या॑ उप॒धाय॑ प्राण॒भृतः॑॥२॥

%5.3.1.3
उ॒प॒दधा॑ति॒ तस्मा॒थ्सर्वा॑नृ॒तूननु॑ वा॒युरा व॑रीवर्त्ति वृष्टि॒सनी॒रुप॑ दधाति॒ वृष्टि॑मे॒वाव॑ रुन्द्धे॒ यदे॑क॒धोप॑द॒ध्यादेक॑मृ॒तुं व॑र्\mbox{}षेदनुपरि॒हारꣳ॑ सादयति॒ तस्मा॒थ्सर्वा॑नृ॒तून् व॑र्\mbox{}षति॒ यत्प्रा॑ण॒भृत॑ उप॒धाय॑ वृष्टि॒सनी॑रुप॒दधा॑ति॒ तस्मा᳚द्वा॒युप्र॑च्युता दि॒वो वृ॑ष्टिरीर्ते प॒शवो॒ वै व॑य॒स्या॑ नाना॑मनसः॒ खलु॒ वै प॒शवो॒ नाना᳚व्रता॒स्ते॑\-ऽप ए॒वाभि सम॑नसः॥३॥

%5.3.1.4
यं का॒मये॑ताप॒शुः स्या॒दिति॑ वय॒स्या᳚स्तस्यो॑प॒धाया॑प॒स्या॑ उप॑ दध्या॒दसं᳚ज्ञानमे॒वास्मै॑ प॒शुभिः॑ करोत्यप॒शुरे॒व भ॑वति॒ यं का॒मये॑त पशु॒मान्थ्स्या॒दित्य॑प॒स्या᳚स्तस्यो॑प॒धाय॑ वय॒स्या॑ उप॑ दध्याथ्सं॒ज्ञान॑मे॒वास्मै॑ प॒शुभिः॑ करोति पशु॒माने॒व भ॑वति॒ चत॑स्रः पु॒रस्ता॒दुप॑ दधाति॒ तस्मा᳚च्च॒त्वारि॒ चक्षु॑षो रू॒पाणि॒ द्वे शु॒क्ले द्वे कृ॒ष्णे॥४॥

%5.3.1.5
मू॒र्ध॒न्वती᳚र्भवन्ति॒ तस्मा᳚त्पु॒रस्ता᳚न्मू॒र्धा पञ्च॒ दक्षि॑णाया॒ꣴ॒ श्रोण्या॒मुप॑ दधाति॒ पञ्चोत्त॑रस्यां॒ तस्मा᳚त्प॒श्चाद्वर्\mbox{}षी॑यान् पु॒रस्ता᳚त्प्रवणः प॒शुर्ब॒स्तो वय॒ इति॒ दक्षि॒णे\-ऽꣳस॒ उप॑ दधाति वृ॒ष्णिर्वय॒ इत्युत्त॒रे\-ऽꣳसा॑वे॒व प्रति॑ दधाति व्या॒घ्रो वय॒ इति॒ दक्षि॑णे प॒क्ष उप॑ दधाति सि॒ꣳ॒हो वय॒ इत्युत्त॑रे प॒क्षयो॑रे॒व वी॒र्यं॑ दधाति॒ पुरु॑षो॒ वय॒ इति॒ मध्ये॒ तस्मा॒त्पुरु॑षः पशू॒नामधि॑पतिः॥५॥

%5.3.2.0
{\anuvakamend[{कॢप्त्या॑ उप॒धाय॑ प्राण॒भृतः॒ सम॑नसः कृ॒ष्णे पुरु॑षो॒ वय॒ इति॒ पञ्च॑ च}]}%॥१॥

%5.3.2.1
इन्द्रा᳚ग्नी॒ अव्य॑थमाना॒मिति॑ स्वयमातृ॒ण्णामुप॑ दधातीन्द्रा॒ग्निभ्यां॒ वा इ॒मौ लो॒कौ विधृ॑ताव॒नयो᳚र्लो॒कयो॒र्विधृ॑त्या॒ अधृ॑तेव॒ वा ए॒षा यन्म॑ध्य॒मा चिति॑र॒न्तरि॑क्षमिव॒ वा ए॒षेन्द्रा᳚ग्नी॒ इत्या॑हेन्द्रा॒ग्नी वै दे॒वाना॑मोजो॒भृता॒वोज॑सै॒वैना॑\-म॒न्तरि॑क्षे चिनुते॒ धृत्यै᳚ स्वयमातृ॒ण्णामुप॑ दधात्य॒न्तरि॑क्षं॒ वै स्व॑यमातृ॒ण्णान्तरि॑क्षमे॒वोप॑ ध॒त्ते\-ऽश्व॒मुप॑॥६॥

%5.3.2.2
घ्रा॒प॒य॒ति॒ प्रा॒णमे॒वास्यां᳚ दधा॒त्यथो᳚ प्राजाप॒त्यो वा अश्वः॑ प्र॒जाप॑तिनै॒वाग्निं चि॑नुते स्वयमातृ॒ण्णा भ॑वति प्रा॒णाना॒मुथ्सृ॑ष्ट्या॒ अथो॑ सुव॒र्गस्य॑ लो॒कस्यानु॑ख्यात्यै दे॒वानां॒ वै सु॑व॒र्गं लो॒कं य॒तां दिशः॒ सम॑व्लीयन्त॒ त ए॒ता दिश्या॑ अपश्य॒न्ता उपा॑दधत॒ ताभि॒र्वै ते दिशो॑\-ऽदृꣳह॒न्॒यद्दिश्या॑ उप॒दधा॑ति दि॒शां विधृ॑त्यै॒ दश॑ प्राण॒भृतः॑ पु॒रस्ता॒दुप॑॥७॥

%5.3.2.3
द॒धा॒ति॒ नव॒ वै पुरु॑षे प्रा॒णा नाभि॑र्दश॒मी प्रा॒णाने॒व पु॒रस्ता᳚द्धत्ते॒ तस्मा᳚त्पु॒रस्ता᳚त्प्रा॒णा ज्योति॑ष्मतीमुत्त॒मामुप॑ दधाति॒ तस्मा᳚त्प्रा॒णानां॒ वाग्ज्योति॑रुत्त॒मा दशोप॑ दधाति॒ दशा᳚क्षरा वि॒राड्वि॒राट्छन्द॑सां॒ ज्योति॒र्ज्योति॑रे॒व पु॒रस्ता᳚द्धत्ते॒ तस्मा᳚त्पु॒रस्ता॒ज्ज्योति॒रुपा᳚स्महे॒ छन्दाꣳ॑सि प॒शुष्वा॒जिम॑यु॒स्तान्बृ॑ह॒त्युद॑जय॒त्तस्मा॒द्बार्\mbox{}ह॑ताः॥८॥

%5.3.2.4
प॒शव॑ उच्यन्ते॒ मा छन्द॒ इति॑ दक्षिण॒त उप॑ दधाति॒ तस्मा᳚द्दक्षि॒णावृ॑तो॒ मासाः᳚ पृथि॒वी छन्द॒ इति॑ प॒श्चात्प्रति॑ष्ठित्या अ॒ग्निर्दे॒वतेत्यु॑त्तर॒त ओजो॒ वा अ॒ग्निरोज॑ ए॒वोत्त॑र॒तो ध॑त्ते॒ तस्मा॑दुत्तरतोभिप्रया॒यी ज॑यति॒ षट्त्रिꣳ॑श॒थ्सम्प॑द्यन्ते॒ षट्त्रिꣳ॑शदक्षरा बृह॒ती बार्\mbox{}ह॑ताः प॒शवो॑ बृह॒त्यैवास्मै॑ प॒शूनव॑ रुन्द्धे बृह॒ती छन्द॑सा॒ꣴ॒ स्वारा᳚ज्यं॒ परी॑याय॒ यस्यै॒ताः॥९॥

%5.3.2.5
उ॒प॒धी॒यन्ते॒ गच्छ॑ति॒ स्वारा᳚ज्यꣳ स॒प्त वाल॑खिल्याः पु॒रस्ता॒दुप॑ दधाति स॒प्त प॒श्चाथ्स॒प्त वै शी॑र्\mbox{}ष॒ण्याः᳚ प्रा॒णा द्वाववा᳚ञ्चौ प्रा॒णानाꣳ॑ सवीर्य॒त्वाय॑ मू॒र्धासि॒ राडिति॑ पु॒रस्ता॒दुप॑ दधाति॒ यन्त्री॒ राडिति॑ प॒श्चात्प्रा॒णाने॒वास्मै॑ स॒मीचो॑ दधाति॥१०॥

%5.3.3.0
{\anuvakamend[{अश्व॒मुप॑ पु॒रस्ता॒दुप॒ बार्\mbox{}ह॑ता ए॒ताश्चतु॑स्त्रिꣳशच्च}]}%॥२॥

%5.3.3.1
दे॒वा वै यद्य॒ज्ञे\-ऽकु॑र्वत॒ तदसु॑रा अकुर्वत॒ ते दे॒वा ए॒ता अ॑क्ष्णयास्तो॒मीया॑ अपश्य॒न्ता अ॒न्यथा॒नूच्या॒न्यथोपा॑दधत॒ तदसु॑रा॒ नान्ववा॑य॒न्ततो॑ दे॒वा अभ॑व॒न्परासु॑रा॒ यद॑क्ष्णयास्तो॒मीया॑ अ॒न्यथा॒नूच्या॒न्यथो॑प॒दधा॑ति॒ भ्रातृ॑व्याभिभूत्यै॒ भव॑त्या॒त्मना॒ परा᳚स्य॒ भ्रातृ॑व्यो भवत्या॒शुस्त्रि॒वृदिति॑ पु॒रस्ता॒दुप॑ दधाति य॒ज्ञमु॒खं वै त्रि॒वृत्॥११॥

%5.3.3.2
य॒ज्ञ॒मु॒खमे॒व पु॒रस्ता॒द्वि या॑तयति॒ व्यो॑म सप्तद॒श इति॑ दक्षिण॒तो\-ऽन्नं॒ वै व्यो॑मान्नꣳ॑ सप्तद॒शो\-ऽन्न॑मे॒व द॑क्षिण॒तो ध॑त्ते॒ तस्मा॒द्दक्षि॑णे॒नान्न॑मद्यते ध॒रुण॑ एकवि॒ꣳ॒श इति॑ प॒श्चात्प्र॑ति॒ष्ठा वा ए॑कवि॒ꣳ॒शः प्रति॑ष्ठित्यै भा॒न्तः प॑ञ्चद॒श इत्यु॑त्तर॒त ओजो॒ वै भा॒न्त ओजः॑ पञ्चद॒श ओज॑ ए॒वोत्त॑र॒तो ध॑त्ते॒ तस्मा॑दुत्तरतोभिप्रया॒यी ज॑यति॒ प्रतू᳚र्तिरष्टाद॒श इति॑ पु॒रस्ता᳚त्॥१२॥

%5.3.3.3
उप॑ दधाति॒ द्वौ त्रि॒वृता॑वभिपू॒र्वं य॑ज्ञमु॒खे वि या॑तयत्यभिव॒र्तः स॑वि॒ꣳ॒श इति॑ दक्षिण॒तो\-ऽन्नं॒ वा अ॑भिव॒र्तो\-ऽन्नꣳ॑ सवि॒ꣳ॒शो\-ऽन्न॑मे॒व द॑क्षिण॒तो ध॑त्ते॒ तस्मा॒द्दक्षि॑णे॒नान्न॑मद्यते॒ वर्चो᳚ द्वावि॒ꣳ॒श इति॑ प॒श्चाद्यद्विꣳ॑श॒तिर्द्वे तेन॑ वि॒राजौ॒ यद्द्वे प्र॑ति॒ष्ठा तेन॑ वि॒राजो॑रे॒वाभि॑पू॒र्वम॒न्नाद्ये॒ प्रति॑ तिष्ठति॒ तपो॑ नवद॒श इत्यु॑त्तर॒तस्तस्मा᳚थ्स॒व्यः॥१३॥

%5.3.3.4
हस्त॑योस्तप॒स्वित॑रो॒ योनि॑श्चतुर्वि॒ꣳ॒श इति॑ पु॒रस्ता॒दुप॑ दधाति॒ चतु॑र्विꣳशत्यक्षरा गाय॒त्री गा॑य॒त्री य॑ज्ञमु॒खम् य॑ज्ञमु॒खमे॒व पु॒रस्ता॒द्वि या॑तयति॒ गर्भाः᳚ पञ्चवि॒ꣳ॒श इति॑ दक्षिण॒तो\-ऽन्नं॒ वै गर्भा॒ अन्नं॑ पञ्चवि॒ꣳ॒शो\-ऽन्न॑मे॒व द॑क्षिण॒तो ध॑त्ते॒ तस्मा॒द्दक्षि॑णे॒नान्न॑मद्यत॒ ओज॑स्त्रिण॒व इति॑ प॒श्चादि॒मे वै लो॒कास्त्रि॑ण॒व ए॒ष्वे॑व लो॒केषु॒ प्रति॑ तिष्ठति स॒म्भर॑णस्त्रयोवि॒ꣳ॒श इति॑॥१४॥

%5.3.3.5
उ॒त्त॒र॒तस्तस्मा᳚थ्स॒व्यो हस्त॑योः सम्भा॒र्य॑तरः॒ क्रतु॑रेकत्रि॒ꣳ॒श इति॑ पु॒रस्ता॒दुप॑ दधाति॒ वाग्वै क्रतु॑र्यज्ञमु॒खं वाग्य॑ज्ञमु॒खमे॒व पु॒रस्ता॒द्वि या॑तयति ब्र॒ध्नस्य॑ वि॒ष्टपं॑ चतुस्त्रि॒ꣳ॒श इति॑ दक्षिण॒तो॑\-ऽसौ वा आ॑दि॒त्यो ब्र॒ध्नस्य॑ वि॒ष्टप॑म् ब्रह्मवर्च॒समे॒व द॑क्षिण॒तो ध॑त्ते॒ तस्मा॒द्दक्षि॒णो\-ऽर्धो᳚ ब्रह्मवर्च॒सित॑रः प्रति॒ष्ठा त्र॑यस्त्रि॒ꣳ॒श इति॑ प॒श्चात्प्रति॑ष्ठित्यै॒ नाकः॑ षट्त्रि॒ꣳ॒श इत्यु॑त्तर॒तः सु॑व॒र्गो वै लो॒को नाकः॑ सुव॒र्गस्य॑ लो॒कस्य॒ सम॑ष्ट्यै॥१५॥

%5.3.4.0
{\anuvakamend[{वै त्रि॒वृदिति॑ पु॒रस्ता᳚थ्स॒व्यस्त्र॑योवि॒ꣳ॒श इति॑ सुव॒र्गो वै पञ्च॑ च}]}%॥३॥ आ॒शुर्व्यो॑म ध॒रुणो॑ भा॒न्तः प्रतू᳚र्तिरभिव॒र्तो वर्च॒स्तपो॒ योनि॒र्गर्भा॒ ओजः॑ स॒म्भर॑णः॒ क्रतु॑र्ब्र॒ध्नस्य॑ प्रति॒ष्ठा नाक॒ष्षोड॑श॥

%5.3.4.1
अ॒ग्नेर्भा॒गो॑\-ऽसीति॑ पु॒रस्ता॒दुप॑ दधाति यज्ञमु॒खं वा अ॒ग्निर्य॑ज्ञमु॒खं दी॒क्षा य॑ज्ञमु॒खं ब्रह्म॑ यज्ञमु॒खं त्रि॒वृद्य॑ज्ञमु॒खमे॒व पु॒रस्ता॒द्वि या॑तयति नृ॒चक्ष॑साम्भा॒गो॑\-ऽसीति॑ दक्षिण॒तः शु॑श्रु॒वाꣳसो॒ वै नृ॒चक्ष॒सो\-ऽन्नं॑ धा॒ता जा॒तायै॒वास्मा॒ अन्न॒मपि॑ दधाति॒ तस्मा᳚ज्जा॒तो\-ऽन्न॑मत्ति ज॒नित्रꣴ॑ स्पृ॒तꣳ स॑प्तद॒शः स्तोम॒ इत्या॒हान्नं॒ वै ज॒नित्रम्᳚॥१६॥

%5.3.4.2
अन्नꣳ॑ सप्तद॒शो\-ऽन्न॑मे॒व द॑क्षिण॒तो ध॑त्ते॒ तस्मा॒द्दक्षि॑णे॒नान्न॑मद्यते मि॒त्रस्य॑ भा॒गो॑\-ऽसीति॑ प॒श्चात्प्रा॒णो वै मि॒त्रो॑\-ऽपा॒नो वरु॑णः प्राणापा॒नावे॒वास्मि॑न्दधाति दि॒वो वृ॒ष्टिर्वाताः᳚ स्पृ॒ता ए॑कवि॒ꣳ॒शः स्तोम॒ इत्या॑ह प्रति॒ष्ठा वा ए॑कवि॒ꣳ॒शः प्रति॑ष्ठित्या॒ इन्द्र॑स्य भा॒गो॑\-ऽसीत्यु॑त्तर॒त ओजो॒ वा इन्द्र॒ ओजो॒ विष्णु॒रोजः॑ क्ष॒त्रमोजः॑ पञ्चद॒शः॥१७॥

%5.3.4.3
ओज॑ ए॒वोत्त॑र॒तो ध॑त्ते॒ तस्मा॑दुत्तरतोभिप्रया॒यी ज॑यति॒ वसू॑नाम्भा॒गो॑\-ऽसीति॑ पु॒रस्ता॒दुप॑ दधाति यज्ञमु॒खं वै वस॑वो ॑यज्ञमु॒खꣳ रु॒द्रा य॑ज्ञमु॒खं च॑तुर्वि॒ꣳ॒शो य॑ज्ञमु॒खमे॒व पु॒रस्ता॒द्वि या॑तयत्यादि॒त्यानां᳚ भा॒गो॑\-ऽसीति॑ दक्षिण॒तो\-ऽन्नं॒ वा आ॑दि॒त्या अन्न॑म्म॒रुतो\-ऽन्नं॒ गर्भा॒ अन्नं॑ पञ्चवि॒ꣳ॒शो\-ऽन्न॑मे॒व द॑क्षिण॒तो ध॑त्ते॒ तस्मा॒द्दक्षि॑णे॒नान्न॑मद्य॒ते\-ऽदि॑त्यै भा॒गः॥१८॥

%5.3.4.4
अ॒सीति॑ प॒श्चात्प्र॑ति॒ष्ठा वा अदि॑तिः प्रति॒ष्ठा पू॒षा प्र॑ति॒ष्ठा त्रि॑ण॒वः प्रति॑ष्ठित्यै दे॒वस्य॑ सवि॒तुर्भा॒गो॑\-ऽसीत्यु॑त्तर॒तो ब्रह्म॒ वै दे॒वः स॑वि॒ता ब्रह्म॒ बृह॒स्पति॒र्ब्रह्म॑ चतुष्टो॒मो ब्र॑ह्मवर्च॒समे॒वोत्त॑र॒तो ध॑त्ते॒ तस्मा॒दुत्त॒रो\-ऽर्धो᳚ ब्रह्मवर्च॒सित॑रः सावि॒त्रव॑ती भवति॒ प्रसू᳚त्यै॒ तस्मा᳚द्ब्राह्म॒णाना॒मुदी॑ची स॒निः प्रसू॑ता ध॒र्त्रश्च॑तुष्टो॒म इति॑ पु॒रस्ता॒दुप॑ दधाति यज्ञमु॒खं वै ध॒र्त्रः॥१९॥

%5.3.4.5
य॒ज्ञ॒मु॒खं च॑तुष्टो॒मो य॑ज्ञमु॒खमे॒व पु॒रस्ता॒द्वि या॑तयति॒ यावा॑नाम्भा॒गो॑\-ऽसीति॑ दक्षिण॒तो मासा॒ वै यावा॑ अर्धमा॒सा अया॑वा॒स्तस्मा᳚द्दक्षि॒णावृ॑तो॒ मासा॒ अन्नं॒ वै यावा॒ अन्नं॑ प्र॒जा अन्न॑मे॒व द॑क्षिण॒तो ध॑त्ते॒ तस्मा॒द्दक्षि॑णे॒नान्न॑मद्यत ऋभू॒णाम्भा॒गो॑\-ऽसीति॑ प॒श्चात् प्रति॑ष्ठित्यै विव॒र्तो᳚\-ऽष्टाचत्वारि॒ꣳ॒श इत्यु॑त्तर॒तो॑\-ऽनयो᳚र्लो॒कयोः᳚ सवीर्य॒त्वाय॒ तस्मा॑दि॒मौ लो॒कौ स॒माव॑द्वीर्यौ॥२०॥

%5.3.4.6
यस्य॒ मुख्य॑वतीः पु॒रस्ता॑दुपधी॒यन्ते॒ मुख्य॑ ए॒व भ॑व॒त्यास्य॒ मुख्यो॑ जायते॒ यस्यान्न॑वतीर्दक्षिण॒तो\-ऽत्त्यन्न॒मास्या᳚न्ना॒दो जा॑यते॒ यस्य॑ प्रति॒ष्ठाव॑तीः प॒श्चात्प्रत्ये॒व ति॑ष्ठति॒ यस्यौज॑स्वतीरुत्तर॒त ओ॑ज॒स्व्ये॑व भ॑व॒त्यास्यौ॑ज॒स्वी जा॑यते॒\-ऽर्को वा ए॒ष यद॒ग्निस्तस्यै॒तदे॒व स्तो॒त्रमे॒तच्छ॒स्त्रं यदे॒षा वि॒धा॥२१॥

%5.3.4.7
वि॒धी॒यते॒\-ऽर्क ए॒व तद॒र्क्य॑मनु॒ वि धी॑य॒ते\-ऽत्त्यन्न॒मास्या᳚न्ना॒दो जा॑यते॒ यस्यै॒षा वि॒धा वि॑धी॒यते॒ य उ॑ चैनामे॒वं वेद॒ सृष्टी॒रुप॑ दधाति यथासृ॒ष्टमे॒वाव॑ रुन्द्धे॒ न वा इ॒दं दिवा॒ न नक्त॑मासी॒दव्या॑वृत्त॒न्ते दे॒वा ए॒ता व्यु॑ष्टीरपश्य॒न्ता उपा॑दधत॒ ततो॒ वा इ॒दं व्यौ᳚च्छ॒द्यस्यै॒ता उ॑पधी॒यन्ते॒ व्ये॑वास्मा॑ उच्छ॒त्यथो॒ तम॑ ए॒वाप॑ हते॥२२॥

%5.3.5.0
{\anuvakamend[{वै ज॒नित्रं॑ पञ्चद॒शो\-ऽदि॑त्यै भा॒गो वै ध॒र्त्रः स॒माव॑द्वीर्यौ वि॒धा ततो॒ वा इ॒दं चतु॑र्दश च}]}%॥४॥ अ॒ग्नेर्नृ॒चक्ष॑साञ्ज॒नित्रं॑ मि॒त्रस्येन्द्र॑स्य॒ वसू॑नामादि॒त्याना॒मदि॑त्यै दे॒वस्य॑ सवि॒तुः सा॑वि॒त्रव॑ती ध॒र्त्रो यावा॑नामृभू॒णां वि॑व॒र्तश्चतु॑र्दश॥

%5.3.5.1
अग्ने॑ जा॒तान्प्र णु॑दा नः स॒पत्ना॒निति॑ पु॒रस्ता॒दुप॑ दधाति जा॒ताने॒व भ्रातृ॑व्या॒न्प्र णु॑दते॒ सह॑सा जा॒तानिति॑ प॒श्चाज्ज॑नि॒ष्यमा॑णाने॒व प्रति॑ नुदते चतुश्चत्वारि॒ꣳ॒शः स्तोम॒ इति॑ दक्षिण॒तो ब्र॑ह्मवर्च॒सं वै च॑तुश्चत्वारि॒ꣳ॒शो ब्र॑ह्मवर्च॒समे॒व द॑क्षिण॒तो ध॑त्ते॒ तस्मा॒द्दक्षि॒णो\-ऽर्धो᳚ ब्रह्मवर्च॒सित॑रः षोड॒शः स्तोम॒ इत्यु॑त्तर॒त ओजो॒ वै षो॑ड॒श ओज॑ ए॒वोत्त॑र॒तो ध॑त्ते॒ तस्मा᳚त्॥२३॥

%5.3.5.2
उ॒त्त॒र॒तो॒भि॒प्र॒या॒यी ज॑यति॒ वज्रो॒ वै च॑तुश्चत्वारि॒ꣳ॒शो वज्रः॑ षोड॒शो यदे॒ते इष्ट॑के उप॒दधा॑ति जा॒ताꣴश्चै॒व ज॑नि॒ष्यमा॑णाꣴश्च॒ भ्रातृ॑व्यान्प्र॒णुद्य॒ वज्र॒मनु॒ प्र ह॑रति॒ स्तृत्यै॒ पुरी॑षवती॒म्मध्य॒ उप॑ दधाति॒ पुरी॑षं॒ वै मध्य॑मा॒त्मनः॒ सात्मा॑नमे॒वाग्निं चि॑नुते॒ सात्मा॒मुष्मि॑ल्लोँ॒के भ॑वति॒ य ए॒वं वेदै॒ता वा अ॑सप॒त्ना नामेष्ट॑का॒ यस्यै॒ता उ॑पधी॒यन्ते᳚॥२४॥

%5.3.5.3
नास्य॑ स॒पत्नो॑ भवति प॒शुर्वा ए॒ष यद॒ग्निर्वि॒राज॑ उत्त॒मायां॒ चित्या॒मुप॑ दधाति वि॒राज॑मे॒वोत्त॒माम्प॒शुषु॑ दधाति॒ तस्मा᳚त्पशु॒मानु॑त्त॒मां वाचं॑ वदति॒ दश॑द॒शोप॑ दधाति सवीर्य॒त्वाया᳚क्ष्ण॒योप॑ दधाति॒ तस्मा॑दक्ष्ण॒या प॒शवो\-ऽङ्गा॑नि॒ प्र ह॑रन्ति॒ प्रति॑ष्ठित्यै॒ यानि॒ वै छन्दाꣳ॑सि सुव॒र्ग्या᳚ण्यास॒न्तैर्दे॒वाः सु॑व॒र्गं लो॒कमा॑य॒न्तेनर्\mbox{}ष॑यः॥२५॥

%5.3.5.4
अ॒श्रा॒म्य॒न्ते तपो॑\-ऽतप्यन्त॒ तानि॒ तप॑सापश्य॒न्तेभ्य॑ ए॒ता इष्ट॑का॒ निर॑मिम॒तेव॒श्छन्दो॒ वरि॑व॒श्छन्द॒ इति॒ ता उपा॑दधत॒ ताभि॒र्वै ते सु॑व॒र्गं लो॒कमा॑य॒न्॒ यदे॒ता इष्ट॑का उप॒दधा॑ति॒ यान्ये॒व छन्दाꣳ॑सि सुव॒र्ग्या॑णि॒ तैरे॒व यज॑मानः सुव॒र्गं लो॒कमे॑ति य॒ज्ञेन॒ वै प्र॒जाप॑तिः प्र॒जा अ॑सृजत॒ ताः स्तोम॑भागैरे॒वासृ॑जत॒ यत्॥२६॥

%5.3.5.5
स्तोम॑भागा उप॒दधा॑ति प्र॒जा ए॒व तद्यज॑मानः सृजते॒ बृह॒स्पति॒र्वा ए॒तद्य॒ज्ञस्य॒ तेजः॒ सम॑भर॒द्यथ्स्तोम॑भागा॒ यथ्स्तोम॑भागा उप॒दधा॑ति॒ सते॑जसमे॒वाग्निं चि॑नुते॒ बृह॒स्पति॒र्वा ए॒तां य॒ज्ञस्य॑ प्रति॒ष्ठाम॑पश्य॒द्यथ्स्तोम॑भागा॒ यथ्स्तोम॑भागा उप॒दधा॑ति य॒ज्ञस्य॒ प्रति॑ष्ठित्यै स॒प्तस॒प्तोप॑ दधाति सवीर्य॒त्वाय॑ ति॒स्रो मध्ये॒ प्रति॑ष्ठित्यै॥२७॥

%5.3.6.0
{\anuvakamend[{उ॒त्त॒र॒तो ध॑त्ते॒ तस्मा॑दुपधी॒यन्त॒ ऋष॑यो\-ऽसृजत॒ यत्त्रिच॑त्वारिꣳशच्च}]}%॥५॥

%5.3.6.1
र॒श्मिरित्ये॒वादि॒त्यम॑सृजत॒ प्रेति॒रिति॒ धर्म॒मन्वि॑ति॒रिति॒ दिवꣳ॑ सं॒धिरित्य॒न्तरि॑क्षं प्रति॒धिरिति॑ पृथि॒वीं वि॑ष्ट॒म्भ इति॒ वृष्टि॑म्प्र॒वेत्यह॑रनु॒वेति॒ रात्रि॑मु॒शिगिति॒ वसू᳚न्प्रके॒त इति॑ रु॒द्रान्थ्सु॑दी॒तिरित्या॑दि॒त्यानोज॒ इति॑ पि॒तॄꣴस्तन्तु॒रिति॑ प्र॒जाः पृ॑तना॒षाडिति॑ प॒शून्रे॒वदित्योष॑धीरभि॒जिद॑सि यु॒क्तग्रा॑वा॥२८॥

%5.3.6.2
इन्द्रा॑य॒ त्वेन्द्रं॑ जि॒न्वेत्ये॒व द॑क्षिण॒तो वज्रं॒ पर्यौ॑हद॒भिजि॑त्यै॒ ताः प्र॒जा अप॑प्राणा असृजत॒ तास्वधि॑पतिर॒सीत्ये॒व प्रा॒णम॑दधाद्य॒न्तेत्य॑पा॒नꣳ स॒ꣳ॒सर्प॒ इति॒ चक्षु॑र्वयो॒धा इति॒ श्रोत्र॒न्ताः प्र॒जाः प्रा॑ण॒तीर॑पान॒तीः पश्य॑न्तीः शृण्व॒तीर्न मि॑थु॒नी अ॑भव॒न्तासु॑ त्रि॒वृद॒सीत्ये॒व मि॑थु॒नम॑दधा॒त्ताः प्र॒जा मि॑थु॒नी॥२९॥

%5.3.6.3
भव॑न्ती॒र्न प्राजा॑यन्त॒ ताः सꣳ॑रो॒हो॑\-ऽसि नीरो॒हो॑\-ऽसीत्ये॒व प्राज॑नय॒त्ताः प्र॒जाः प्रजा॑ता॒ न प्रत्य॑तिष्ठ॒न्ता व॑सु॒को॑\-ऽसि॒ वेष॑श्रिरसि॒ वस्य॑ष्टिर॒सीत्ये॒वैषु लो॒केषु॒ प्रत्य॑स्थापय॒द्यदाह॑ वसु॒को॑\-ऽसि॒ वेष॑श्रिरसि॒ वस्य॑ष्टिर॒सीति॑ प्र॒जा ए॒व प्रजा॑ता ए॒षु लो॒केषु॒ प्रति॑ ष्ठापयति॒ सात्मा॒न्तरि॑क्षꣳ रोहति॒ सप्रा॑णो॒\-ऽमुष्मि॑ल्लोँ॒के प्रति॑ तिष्ठ॒त्यव्य॑र्धुकः प्राणापा॒ना\-भ्यां᳚ भवति॒ य ए॒वं वेद॑॥३०॥

%5.3.7.0
{\anuvakamend[{यु॒क्तग्रा॑वा प्र॒जा मि॑थु॒न्य॑न्तरि॑क्ष॒न्द्वाद॑श च}]}%॥६॥

%5.3.7.1
ना॒क॒सद्भि॒र्वै दे॒वाः सु॑व॒र्गं लो॒कमा॑य॒न्तन्ना॑क॒सदां᳚ नाकस॒त्त्वं यन्ना॑क॒सद॑ उप॒दधा॑ति नाक॒सद्भि॑रे॒व तद्यज॑मानः सुव॒र्गं लो॒कमे॑ति सुव॒र्गो वै लो॒को नाको॒ यस्यै॒ता उ॑पधी॒यन्ते॒ नास्मा॒ अक॑म्भवति यजमानायत॒नं वै ना॑क॒सदो॒ यन्ना॑क॒सद॑ उप॒दधा᳚त्या॒यत॑नमे॒व तद्यज॑मानः कुरुते पृ॒ष्ठानां॒ वा ए॒तत्तेजः॒ सम्भृ॑तं॒ यन्ना॑क॒सदो॒ यन्ना॑क॒सदः॑॥३१॥

%5.3.7.2
उ॒प॒दधा॑ति पृ॒ष्ठाना॑मे॒व तेजो\-ऽव॑ रुन्द्धे पञ्च॒चोडा॒ उप॑ दधात्यफ्स॒रस॑ ए॒वैन॑मे॒ता भू॒ता अ॒मुष्मि॑ल्लोँ॒क उप॑ शे॒रे\-ऽथो॑ तनू॒पानी॑रे॒वैता यज॑मानस्य॒ यं द्वि॒ष्यात्तमु॑प॒दध॑द्ध्यायेदे॒ताभ्य॑ ए॒वैनं॑ दे॒वता᳚भ्य॒ आ वृ॑श्चति ता॒जगार्ति॒मार्च्छ॒त्युत्त॑रा नाक॒सद्भ्य॒ उप॑ दधाति॒ यथा॑ जा॒यामा॒नीय॑ गृ॒हेषु॑ निषा॒दय॑ति ता॒दृगे॒व तत्॥३२॥

%5.3.7.3
प॒श्चात्प्राची॑मुत्त॒मामुप॑ दधाति॒ तस्मा᳚त्प॒श्चात्प्राची॒ पत्न्यन्वा᳚स्ते स्वयमातृ॒ण्णां च॑ विक॒र्णीं चो᳚त्त॒मे उप॑ दधाति प्रा॒णो वै स्व॑यमातृ॒ण्णायु॑र्विक॒र्णी प्रा॒णं चै॒वायु॑श्च प्रा॒णाना॑मुत्त॒मौ ध॑त्ते॒ तस्मा᳚त्प्रा॒णश्चायु॑श्च प्रा॒णाना॑मुत्त॒मौ नान्यामुत्त॑रा॒मिष्ट॑का॒मुप॑ दध्या॒द्यद॒न्यामुत्त॑रा॒मिष्ट॑कामुपद॒ध्यात्प॑शू॒नाम्॥३३॥

%5.3.7.4
च॒ यज॑मानस्य च प्रा॒णं चायु॒श्चापि॑ दध्या॒त्तस्मा॒न्नान्योत्त॒रेष्ट॑कोप॒धेया᳚ स्वयमातृ॒ण्णामुप॑ दधात्य॒सौ वै स्व॑यमातृ॒ण्णामूमे॒वोप॑ ध॒त्ते\-ऽश्व॒मुप॑ घ्रापयति प्रा॒णमे॒वास्यां᳚ दधा॒त्यथो᳚ प्राजाप॒त्यो वा अश्वः॑ प्र॒जाप॑तिनै॒वाग्निं चि॑नुते स्वयमातृ॒ण्णा भ॑वति प्रा॒णाना॒मुथ्सृ॑ष्ट्या॒ अथो॑ सुव॒र्गस्य॑ लो॒कस्यानु॑ख्यात्या ए॒षा वै दे॒वानां॒ विक्रा᳚न्ति॒र्यद्वि॑क॒र्णी यद्वि॑क॒र्णीमु॑प॒दधा॑ति दे॒वाना॑मे॒व विक्रा᳚न्ति॒मनु॒ वि क्र॑मत उत्तर॒त उप॑ दधाति॒ तस्मा॑दुत्तर॒तउ॑पचारो॒\-ऽग्निर्वा॑यु॒मती॑ भवति॒ समि॑द्ध्यै॥३४॥

%5.3.8.0
{\anuvakamend[{सम्भृ॑तं॒ यन्ना॑क॒सदो॒ यन्ना॑क॒सद॒स्तत्प॑शू॒नामे॒षां वै द्वाविꣳ॑शतिश्च}]}%॥७॥

%5.3.8.1
छन्दा॒ꣳ॒स्युप॑ दधाति प॒शवो॒ वै छन्दाꣳ॑सि प॒शूने॒वाव॑ रुन्द्धे॒ छन्दाꣳ॑सि॒ वै दे॒वानां᳚ वा॒मम्प॒शवो॑ वा॒ममे॒व प॒शूनव॑ रुन्द्ध ए॒ताꣳ ह॒ वै य॒ज्ञसे॑नश्चैत्रियाय॒णश्चितिं॑ वि॒दां च॑कार॒ तया॒ वै स प॒शूनवा॑रुन्द्ध॒ यदे॒तामु॑प॒दधा॑ति प॒शूने॒वाव॑ रुन्द्धे गाय॒त्रीः पु॒रस्ता॒दुप॑ दधाति॒ तेजो॒ वै गा॑य॒त्री तेज॑ ए॒व॥३५॥

%5.3.8.2
मु॒ख॒तो ध॑त्ते मूर्ध॒न्वती᳚र्भवन्ति मू॒र्धान॑मे॒वैनꣳ॑ समा॒नानां᳚ करोति त्रि॒ष्टुभ॒ उप॑ दधातीन्द्रि॒यं वै त्रि॒ष्टुगि॑न्द्रि॒यमे॒व म॑ध्य॒तो ध॑त्ते॒ जग॑ती॒रुप॑ दधाति॒ जाग॑ता॒ वै प॒शवः॑ प॒शूने॒वाव॑ रुन्द्धे\-ऽनु॒ष्टुभ॒ उप॑ दधाति प्रा॒णा वा अ॑नु॒ष्टुप्प्रा॒णाना॒मुथ्सृ॑ष्ट्यै बृह॒तीरु॒ष्णिहाः᳚ प॒ङ्क्तीर॒क्षर॑पङ्क्ती॒रिति॒ विषु॑रूपाणि॒ छन्दा॒ꣳ॒स्युप॑ दधाति॒ विषु॑रूपा॒ वै प॒शवः॑ प॒शवः॑॥३६॥

%5.3.8.3
छन्दाꣳ॑सि॒ विषु॑रूपाने॒व प॒शूनव॑ रुन्द्धे॒ विषु॑रूपमस्य गृ॒हे दृ॑श्यते॒ यस्यै॒ता उ॑पधी॒यन्ते॒ य उ॑ चैना ए॒वं वेदाति॑च्छन्दस॒मुप॑ दधा॒त्यति॑च्छन्दा॒ वै सर्वा॑णि॒ छन्दाꣳ॑सि॒ सर्वे॑भिरे॒वैनं॒ छन्दो॑भिश्चिनुते॒ वर्ष्म॒ वा ए॒षा छन्द॑सां॒ यदति॑च्छन्दा॒ यदति॑च्छन्दसमुप॒दधा॑ति॒ वर्ष्मै॒वैनꣳ॑ समा॒नानां᳚ करोति द्वि॒पदा॒ उप॑ दधाति द्वि॒पाद्यज॑मानः॒ प्रति॑ष्ठित्यै॥३७॥

%5.3.9.0
{\anuvakamend[{तेज॑ ए॒व प॒शवः॑ प॒शवो॒ यज॑मान॒ एक॑ञ्च}]}%॥८॥

%5.3.9.1
सर्वा᳚भ्यो॒ वै दे॒वता᳚भ्यो॒\-ऽग्निश्ची॑यते॒ यथ्स॒युजो॒ नोप॑द॒ध्याद्दे॒वता॑ अस्या॒ग्निं वृ॑ञ्जीर॒न्॒ यथ्स॒युज॑ उप॒दधा᳚त्या॒त्मनै॒वैनꣳ॑ स॒युजं॑ चिनुते॒ नाग्निना॒ व्यृ॑ध्य॒ते\-ऽथो॒ यथा॒ पुरु॑षः॒ स्नाव॑भिः॒ सन्त॑त ए॒वमे॒वैताभि॑र॒ग्निः सन्त॑तो॒\-ऽग्निना॒ वै दे॒वाः सु॑व॒र्गं लो॒कमा॑य॒न्ता अ॒मूः कृ॑त्तिका अभव॒न् यस्यै॒ता उ॑पधी॒यन्ते॑ सुव॒र्गमे॒व॥३८॥

%5.3.9.2
लो॒कमे॑ति॒ गच्छ॑ति प्रका॒शं चि॒त्रमे॒व भ॑वति मण्डलेष्ट॒का उप॑ दधाती॒मे वै लो॒का म॑ण्डलेष्ट॒का इ॒मे खलु॒ वै लो॒का दे॑वपु॒रा दे॑वपु॒रा ए॒व प्र वि॑शति॒ नार्ति॒मार्च्छ॑त्य॒ग्निं चि॑क्या॒नो वि॒श्वज्यो॑तिष॒ उप॑ दधाती॒माने॒वैताभि॑र्लो॒कां ज्योति॑ष्मतः कुरु॒ते\-ऽथो᳚ प्रा॒णाने॒वैता यज॑मानस्य दाध्रत्ये॒ता वै दे॒वताः᳚ सुव॒र्ग्या᳚स्ता ए॒वान्वा॒रभ्य॑ सुव॒र्गं लो॒कमे॑ति॥३९॥

%5.3.10.0
{\anuvakamend[{सु॒व॒र्गमे॒व ता ए॒व च॒त्वारि॑ च}]}%॥९॥

%5.3.10.1
वृ॒ष्टि॒सनी॒रुप॑ दधाति॒ वृष्टि॑मे॒वाव॑ रुन्द्धे॒ यदे॑क॒धोप॑द॒ध्यादेक॑मृ॒तुं व॑र्\mbox{}षेदनुपरि॒हारꣳ॑ सादयति॒ तस्मा॒थ्सर्वा॑नृ॒तून् व॑र्\mbox{}षति पुरोवात॒सनि॑र॒सीत्या॑है॒तद्वै वृष्ट्यै॑ रू॒पꣳ रू॒पेणै॒व वृष्टि॒मव॑ रुन्द्धे सं॒यानी॑भि॒र्वै दे॒वा इ॒माल्लोँ॒कान्थ्सम॑यु॒स्तथ्सं॒यानी॑नाꣳ संयानि॒त्वं यथ्सं॒यानी॑रुप॒दधा॑ति॒ यथा॒फ्सु ना॒वा सं॒यात्ये॒वम्॥४०॥

%5.3.10.2
ए॒वैताभि॒र्यज॑मान इ॒माल्लोँ॒कान्थ्सं या॑ति प्ल॒वो वा ए॒षो᳚\-ऽग्नेर्यथ्सं॒यानी॒र्यथ्सं॒यानी॑रुप॒दधा॑ति प्ल॒वमे॒वैतम॒ग्नय॒ उप॑ दधात्यु॒त यस्यै॒तासूप॑हिता॒स्वापो॒\-ऽग्निꣳ हर॒न्त्यहृ॑त ए॒वास्या॒ग्निरा॑दित्येष्ट॒का उप॑ दधात्यादि॒त्या वा ए॒तम्भूत्यै॒ प्रति॑ नुदन्ते॒ यो\-ऽल॒म्भूत्यै॒ सन्भूतिं॒ न प्रा॒प्नोत्या॑दि॒त्याः॥४१॥

%5.3.10.3
ए॒वैन॒म्भूतिं॑ गमयन्त्य॒सौ वा ए॒तस्या॑दि॒त्यो रुच॒मा द॑त्ते॒ यो᳚\-ऽग्निं चि॒त्वा न रोच॑ते॒ यदा॑दित्येष्ट॒का उ॑प॒दधा᳚त्य॒सावे॒वास्मि॑न्नादि॒त्यो रुचं॑ दधाति॒ यथा॒सौ दे॒वाना॒ꣳ॒ रोच॑त ए॒वमे॒वैष म॑नु॒ष्या॑णाꣳ रोचते घृतेष्ट॒का उप॑ दधात्ये॒तद्वा अ॒ग्नेः प्रि॒यं धाम॒ यद्घृ॒तम्प्रि॒येणै॒वैनं॒ धाम्ना॒ सम॑र्धयति॥४२॥

%5.3.10.4
अथो॒ तेज॑सानुपरि॒हारꣳ॑ सादय॒त्यप॑रिवर्गमे॒वास्मि॒न्तेजो॑ दधाति प्र॒जाप॑तिर॒ग्निम॑चिनुत॒ स यश॑सा॒ व्या᳚र्ध्यत॒ स ए॒ता य॑शो॒दा अ॑पश्य॒त्ता उपा॑धत्त॒ ताभि॒र्वै स यश॑ आ॒त्मन्न॑धत्त॒ यद्य॑शो॒दा उ॑प॒दधा॑ति॒ यश॑ ए॒व ताभि॒र्यज॑मान आ॒त्मन्ध॑त्ते॒ पञ्चोप॑ दधाति॒ पाङ्क्तः॒ पुरु॑षो॒ यावा॑ने॒व पुरु॑ष॒स्तस्मि॒न् यशो॑ दधाति॥४३॥

%5.3.11.0
{\anuvakamend[{ए॒वं प्रा॒प्नोत्या॑दि॒त्या अ॑र्धय॒त्येका॒न्नप॑ञ्चा॒शच्च॑}]}%॥10॥

%5.3.11.1
दे॒वा॒सु॒राः संय॑त्ता आस॒न्कनी॑याꣳसो दे॒वा आस॒न्भूया॒ꣳ॒सो\-ऽसु॑रा॒स्ते दे॒वा ए॒ता इष्ट॑का अपश्य॒न्ता उपा॑दधत भूय॒स्कृद॒सीत्ये॒व भूयाꣳ॑सो\-ऽभव॒न्वन॒स्पति॑भि॒रोष॑धीभिर्वरिव॒स्कृद॒सीती॒माम॑जय॒न्प्राच्य॒सीति॒ प्राचीं॒ दिश॑मजयन्नू॒र्ध्वासीत्य॒मूम॑जयन्नन्तरिक्ष॒सद॑स्य॒न्तरि॑क्षे सी॒देत्य॒न्तरि॑क्षमजय॒न्ततो॑ दे॒वा अभ॑वन्न्॥४४॥

%5.3.11.2
परासु॑रा॒ यस्यै॒ता उ॑पधी॒यन्ते॒ भूया॑ने॒व भ॑वत्य॒भीमाल्लोँ॒काञ्ज॑यति॒ भव॑त्या॒त्मना॒ परा᳚स्य॒ भ्रातृ॑व्यो भवत्यफ्सु॒षद॑सि श्येन॒सद॒सीत्या॑है॒तद्वा अ॒ग्ने रू॒पꣳ रू॒पेणै॒वाग्निमव॑ रुन्द्धे पृथि॒व्यास्त्वा॒ द्रवि॑णे सादया॒मीत्या॑हे॒माने॒वैताभि॑र्लो॒कान् द्रवि॑णावतः कुरुत आयु॒ष्या॑ उप॑ दधा॒त्यायु॑रे॒व॥४५॥

%5.3.11.3
अ॒स्मि॒न्द॒धा॒त्यग्ने॒ यत्ते॒ पर॒ꣳ॒ हृन्नामेत्या॑है॒तद्वा अ॒ग्नेः प्रि॒यं धाम॑ प्रि॒यमे॒वास्य॒ धामोपा᳚प्नोति॒ तावेहि॒ सꣳ र॑भावहा॒ इत्या॑ह॒ व्ये॑वैने॑न॒ परि॑ धत्ते॒ पाञ्च॑जन्ये॒ष्वप्ये᳚ध्यग्न॒ इत्या॑है॒ष वा अ॒ग्निः पाञ्च॑जन्यो॒ यः पञ्च॑चितीक॒स्तस्मा॑दे॒वमा॑हर्त॒व्या॑ उप॑ दधात्ये॒तद्वा ऋ॑तू॒नाम्प्रि॒यं धाम॒ यदृ॑त॒व्या॑ ऋतू॒नामे॒व प्रि॒यं धामाव॑ रुन्द्धे सु॒मेक॒ इत्या॑ह संवथ्स॒रो वै सु॒मेकः॑ संवथ्स॒रस्यै॒व प्रि॒यं धामोपा᳚प्नोति॥४६॥

%5.3.12.0
{\anuvakamend[{अभ॑व॒न्नायु॑रे॒वर्त॒व्या॑ उप॒ षड्विꣳ॑शतिश्च}]}%॥11॥

%5.3.12.1
प्र॒जाप॑ते॒रक्ष्य॑श्वय॒त्तत्परा॑पत॒त्तदश्वो॑\-ऽभव॒द्यदश्व॑य॒त्तदश्व॑स्याश्व॒त्वन्तद्दे॒वा अ॑श्वमे॒धेनै॒व प्रत्य॑दधुरे॒ष वै प्र॒जाप॑ति॒ꣳ॒ सर्वं॑ करोति॒ यो᳚\-ऽश्वमे॒धेन॒ यज॑ते॒ सर्व॑ ए॒व भ॑वति॒ सर्व॑स्य॒ वा ए॒षा प्राय॑श्चित्तिः॒ सर्व॑स्य भेष॒जꣳ सर्वं॒ वा ए॒तेन॑ पा॒प्मानं॑ दे॒वा अ॑तर॒न्नपि॒ वा ए॒तेन॑ ब्रह्मह॒त्याम॑तर॒न्थ्सर्व॑म्पा॒प्मानम्᳚॥४७॥

%5.3.12.2
त॒र॒ति॒ तर॑ति ब्रह्मह॒त्यां यो᳚\-ऽश्वमे॒धेन॒ यज॑ते॒ य उ॑ चैनमे॒वं वेदोत्त॑रं॒ वै तत्प्र॒जाप॑ते॒रक्ष्य॑श्वय॒त्तस्मा॒दश्व॑स्योत्तर॒तो\-ऽव॑ द्यन्ति दक्षिण॒तो᳚\-ऽन्येषां᳚ पशू॒नाम्वै॑त॒सः कटो॑ भवत्य॒फ्सुयो॑नि॒र्वा अश्वो᳚\-ऽफ्सु॒जो वे॑त॒सः स्व ए॒वैनं॒ योनौ॒ प्रति॑ ष्ठापयति चतुष्टो॒मः स्तोमो॑ भवति स॒रड्ढ॒ वा अश्व॑स्य॒ सक्थ्यावृ॑ह॒त्तद्दे॒वाश्च॑तुष्टो॒मेनै॒व प्रत्य॑दधु॒र्यच्च॑तुष्टो॒मः स्तोमो॒ भव॒त्यश्व॑स्य सर्व॒त्वाय॑॥४८॥

%5.4.0.0
{\anuvakamend[{सर्वं॑ पा॒प्मान॑मवृह॒द्द्वाद॑श च}]}%॥12॥

%5.4.0.0

{\anuvakamend[{दे॒वा॒सु॒राः तेनर्त॒व्या॑ रु॒द्रो\-ऽश्म॑न्नृ॒षदे॒ वडुदे॑नं॒ प्राची॒मिति॒ वसो॒र्धारा॑म॒ग्निर्दे॒वेभ्यः॑ सुव॒र्गाय॑ यत्राकू॒ताय॑ छन्द॒श्चितं॒ पव॑स्व॒ द्वाद॑श}]}%॥12॥ 
\prashnaend{दे॒वा॒सु॒रा अ॒जायां॒ वै ग्रु॑मु॒ष्टिः प्र॑थ॒मो दे॑वय॒तामे॒तद्वै छन्द॑सामृ॒ध्नोत्य॒ष्टौ प॑ञ्चाशचत्॥58॥ दे॒वा॒सु॒राः सर्वं॑ जयति॥}
%%% END PRASHNA

\sect{चतुर्थः प्रश्नः}\setcounter{anuvakam}{0}
\dnsub{तैत्तिरीयसंहितायां पञ्चमकाण्डे चतुर्थः प्रश्नः}
%5.4.1.0
%5.4.1.1
दे॒वा॒सु॒राः संय॑त्ता आस॒न्ते न व्य॑जयन्त॒ स ए॒ता इन्द्र॑स्त॒नूर॑पश्य॒त्ता उपा॑धत्त॒ ताभि॒र्वै स त॒नुव॑मिन्द्रि॒यं वी॑र्यमा॒त्मन्न॑धत्त॒ ततो॑ दे॒वा अभ॑व॒न्परासु॑रा॒ यदि॑न्द्रत॒नूरु॑प॒दधा॑ति त॒नुव॑मे॒व ताभि॑रिन्द्रि॒यं वी॒र्यं॑ यज॑मान आ॒त्मन्ध॒त्ते\-ऽथो॒ सेन्द्र॑मे॒वाग्निꣳ सत॑नुं चिनुते॒ भव॑त्या॒त्मना॒ परा᳚स्य॒ भ्रातृ॑व्यः॥१॥

%5.4.1.2
भ॒व॒ति॒ य॒ज्ञो दे॒वेभ्यो\-ऽपा᳚क्राम॒त्तम॑व॒रुधं॒ नाश॑क्नुव॒न्त ए॒ता य॑ज्ञत॒नूर॑पश्य॒न्ता उपा॑दधत॒ ताभि॒र्वै ते य॒ज्ञमवा॑रुन्धत॒ यद्य॑ज्ञत॒नूरु॑प॒दधा॑ति य॒ज्ञमे॒व ताभि॒र्यज॑मा॒नो\-ऽव॑ रुन्द्धे॒ त्रय॑स्त्रिꣳशत॒मुप॑ दधाति॒ त्रय॑स्त्रिꣳश॒द्वै दे॒वता॑ दे॒वता॑ ए॒वाव॑ रु॒न्द्धे\-ऽथो॒ सात्मा॑नमे॒वाग्निꣳ सत॑नुं चिनुते॒ सात्मा॒मुष्मि॑ल्लोँ॒के॥२॥

%5.4.1.3
भ॒व॒ति॒ य ए॒वं वेद॒ ज्योति॑ष्मती॒रुप॑ दधाति॒ ज्योति॑रे॒वास्मि॑न्दधात्ये॒ताभि॒र्वा अ॒ग्निश्चि॒तो ज्व॑लति॒ ताभि॑रे॒वैन॒ꣳ॒ समि॑न्द्ध उ॒भयो॑रस्मै लो॒कयो॒र्ज्योति॑र्भवति नक्षत्रेष्ट॒का उप॑ दधात्ये॒तानि॒ वै दि॒वो ज्योतीꣳ॑षि॒ तान्ये॒वाव॑ रुन्द्धे सु॒कृतां॒ वा ए॒तानि॒ ज्योतीꣳ॑षि॒ यन्नक्ष॑त्राणि॒ तान्ये॒वाप्नो॒त्यथो॑ अनूका॒शमे॒वैतानि॑॥३॥

%5.4.1.4
ज्योतीꣳ॑षि कुरुते सुव॒र्गस्य॑ लो॒कस्यानु॑ख्यात्यै॒ यथ्सꣴस्पृ॑ष्टा उपद॒ध्याद्वृष्ट्यै॑ लो॒कमपि॑ दध्या॒दव॑र्\mbox{}षुकः प॒र्जन्यः॑ स्या॒दसꣴ॑स्पृष्टा॒ उप॑ दधाति॒ वृष्ट्या॑ ए॒व लो॒कं क॑रोति॒ वर्\mbox{}षु॑कः प॒र्जन्यो॑ भवति पु॒रस्ता॑द॒न्याः प्र॒तीची॒रुप॑ दधाति प॒श्चाद॒न्याः प्राची॒स्तस्मा᳚त्प्रा॒चीना॑नि च प्रती॒चीना॑नि च॒ नक्ष॑त्रा॒ण्या व॑र्तन्ते॥४॥

%5.4.2.0
{\anuvakamend[{भ्रातृ॑व्यो लो॒क ए॒वैतान्येक॑चत्वारिꣳशच्च}]}%॥१॥

%5.4.2.1
ऋ॒त॒व्या॑ उप॑ दधात्यृतू॒नां कॢप्त्यै᳚ द्वं॒द्वमुप॑ दधाति॒ तस्मा᳚द्द्वं॒द्वमृ॒तवो\-ऽधृ॑तेव॒ वा ए॒षा यन्म॑ध्य॒मा चिति॑र॒न्तरि॑क्षमिव॒ वा ए॒षा द्वं॒द्वम॒न्यासु॒ चिती॒षूप॑ दधाति॒ चत॑स्रो॒ मध्ये॒ धृत्या॑ अन्तः॒श्लेष॑णं॒ वा ए॒ताश्चिती॑नां॒ यदृ॑त॒व्या॑ यदृ॑त॒व्या॑ उप॒दधा॑ति॒ चितीनां॒ विधृ॑त्या॒ अव॑का॒मनूप॑ दधात्ये॒षा वा अ॒ग्नेर्योनिः॒ सयो॑निम्॥५॥

%5.4.2.2
ए॒वाग्निं चि॑नुत उ॒वाच॑ ह वि॒श्वामि॒त्रो\-ऽद॒दिथ्स ब्रह्म॒णान्नं॒ यस्यै॒ता उ॑पधी॒यान्तै॒ य उ॑ चैना ए॒वं वेद॒दिति॑ संवथ्स॒रो वा ए॒तम्प्र॑ति॒ष्ठायै॑ नुदते॒ यो᳚\-ऽग्निं चि॒त्वा न प्र॑ति॒तिष्ठ॑ति॒ पञ्च॒ पूर्वा॒श्चित॑यो भव॒न्त्यथ॑ ष॒ष्ठीं चितिं॑ चिनुते॒ षड्वा ऋ॒तवः॑ संवथ्स॒र ऋ॒तुष्वे॒व सं॑वथ्स॒रे प्रति॑ तिष्ठत्ये॒ता वै॥६॥

%5.4.2.3
अधि॑पत्नी॒र्नामेष्ट॑का॒ यस्यै॒ता उ॑पधी॒यन्ते\-ऽधि॑पतिरे॒व स॑मा॒नानां᳚ भवति॒ यं द्वि॒ष्यात्तमु॑प॒दध॑द्ध्यायेदे॒ताभ्य॑ ए॒वैनं॑ दे॒वता᳚भ्य॒ आ वृ॑श्चति ता॒जगार्ति॒मार्च्छ॒त्यङ्गि॑रसः सुव॒र्गं लो॒कं यन्तो॒ या य॒ज्ञस्य॒ निष्कृ॑ति॒रासी॒त्तामृषि॑भ्यः॒ प्रत्यौ॑ह॒न् तद्धिर॑ण्यमभव॒द्यद्धि॑रण्यश॒ल्कैः प्रो॒क्षति॑ य॒ज्ञस्य॒ निष्कृ॑त्या॒ अथो॑ भेष॒जमे॒वास्मै॑ करोति॥७॥

%5.4.2.4
अथो॑ रू॒पेणै॒वैन॒ꣳ॒ सम॑र्धय॒त्यथो॒ हिर॑ण्यज्योतिषै॒व सु॑व॒र्गं लो॒कमे॑ति साह॒स्रव॑ता॒ प्रोक्ष॑ति साह॒स्रः प्र॒जाप॑तिः प्र॒जाप॑ते॒राप्त्या॑ इ॒मा मे॑ अग्न॒ इष्ट॑का धे॒नवः॑ स॒न्त्वित्या॑ह धे॒नूरे॒वैनाः᳚ कुरुते॒ ता ए॑नं काम॒दुघा॑ अ॒मुत्रा॒मुष्मि॑ल्लोँ॒क उप॑ तिष्ठन्ते॥८॥

%5.4.3.0
{\anuvakamend[{सयो॑निमे॒ता वै क॑रो॒त्येका॒न्नच॑त्वारि॒ꣳ॒शच्च॑}]}%॥२॥

%5.4.3.1
रु॒द्रो वा ए॒ष यद॒ग्निः स ए॒तर्\mbox{}हि॑ जा॒तो यर्\mbox{}हि॒ सर्व॑श्चि॒तः स यथा॑ व॒थ्सो जा॒तः स्तन॑म्प्रे॒फ्सत्ये॒वं वा ए॒ष ए॒तर्\mbox{}हि॑ भाग॒धेय॒म्प्रेफ्स॑ति॒ तस्मै॒ यदाहु॑तिं॒ न जु॑हु॒याद॑ध्व॒र्युं च॒ यज॑मानं च ध्यायेच्छतरु॒द्रीयं॑ जुहोति भाग॒धेये॑नै॒वैनꣳ॑ शमयति॒ नार्ति॒मार्च्छ॑त्यध्व॒र्युर्न यज॑मानो॒ यद्ग्रा॒म्याणां᳚ पशू॒नाम्॥९॥

%5.4.3.2
पय॑सा जुहु॒याद्ग्रा॒म्यान्प॒शूञ्छु॒चार्प॑ये॒द्यदा॑र॒ण्याना॑मार॒ण्याञ्ज॑र्तिलयवा॒ग्वा॑ वा जुहु॒याद्ग॑वीधुकयवा॒ग्वा॑ वा॒ न ग्रा॒म्यान्प॒शून् हि॒नस्ति॒ नार॒ण्यानथो॒ खल्वा॑हु॒रना॑हुति॒र्वै ज॒र्तिला᳚श्च ग॒वीधु॑का॒श्चेत्य॑जक्षी॒रेण॑ जुहोत्याग्ने॒यी वा ए॒षा यद॒जाहु॑त्यै॒व जु॑होति॒ न ग्रा॒म्यान्प॒शून् हि॒नस्ति॒ नार॒ण्यानङ्गि॑रसः सुव॒र्गं लो॒कं यन्तः॑॥१०॥

%5.4.3.3
अ॒जायां᳚ घ॒र्मम्प्रासि॑ञ्च॒न्थ्सा शोच॑न्ती प॒र्णं परा॑जिहीत॒ सो \-ऽर्को॑\-ऽभव॒त्तद॒र्कस्या᳚र्क॒त्वम॑र्कप॒र्णेन॑ जुहोति सयोनि॒त्वायोद॒ङ्तिष्ठ॑ञ्जुहोत्ये॒षा वै रु॒द्रस्य॒ दिख्स्वाया॑मे॒व दि॒शि रु॒द्रं नि॒रव॑दयते चर॒माया॒मिष्ट॑कायां जुहोत्यन्त॒त ए॒व रु॒द्रं नि॒रव॑दयते त्रेधाविभ॒क्तं जु॑होति॒ त्रय॑ इ॒मे लो॒का इ॒माने॒व लो॒कान्थ्स॒माव॑द्वीर्यान्करो॒तीय॒त्यग्रे॑ जुहोति॥११॥

%5.4.3.4
अथेय॒त्यथेय॑ति॒ त्रय॑ इ॒मे लो॒का ए॒भ्य ए॒वैनं॑ लो॒केभ्यः॑ शमयति ति॒स्र उत्त॑रा॒ आहु॑तीर्जुहोति॒ षट्थ्सम्प॑द्य॒न्ते षड्वा ऋ॒तव॑ ऋ॒तुभि॑रे॒वैनꣳ॑ शमयति॒ यद॑नुपरि॒क्रामं॑ जुहु॒याद॑न्तरवचा॒रिणꣳ॑ रु॒द्रं कु॑र्या॒दथो॒ खल्वा॑हुः॒ कस्यां॒ वाह॑ दि॒शि रु॒द्रः कस्यां॒ वेत्य॑नुपरि॒क्राम॑मे॒व हो॑त॒व्य॑मप॑रिवर्गमे॒वैनꣳ॑ शमयति॥१२॥

%5.4.3.5
ए॒ता वै दे॒वताः᳚ सुव॒र्ग्या॑ या उ॑त्त॒मास्ता यज॑मानं वाचयति॒ ताभि॑रे॒वैनꣳ॑ सुव॒र्गं लो॒कं ग॑मयति॒ यं द्वि॒ष्यात्तस्य॑ सञ्च॒रे प॑शू॒नां न्य॑स्ये॒द्यः प्र॑थ॒मः प॒शुर॑भि॒तिष्ठ॑ति॒ स आर्ति॒मार्च्छ॑ति॥१३॥

%5.4.4.0
{\anuvakamend[{प॒शू॒नां यन्तो\-ऽग्रे॑ जुहो॒त्यप॑रिवर्गमे॒वैनꣳ॑ शमयति त्रि॒ꣳ॒शच्च॑}]}%॥३॥

%5.4.4.1
अश्म॒न्नूर्ज॒मिति॒ परि॑ षिञ्चति मा॒र्जय॑त्ये॒वैन॒मथो॑ त॒र्पय॑त्ये॒व स ए॑नं तृ॒प्तो\-ऽक्षु॑ध्य॒न्नशो॑चन्न॒मुष्मि॑ल्लोँ॒क उप॑ तिष्ठते॒ तृप्य॑ति प्र॒जया॑ प॒शुभि॒र्य ए॒वं वेद॒ तां न॒ इष॒मूर्जं॑ धत्त मरुतः सꣳररा॒णा इत्या॒हान्नं॒ वा ऊर्गन्न॑म्म॒रुतो\-ऽन्न॑मे॒वाव॑ रु॒न्द्धे\-ऽश्मꣴ॑स्ते॒ क्षुद॒मुं ते॒ शुक्॥१४॥

%5.4.4.2
ऋ॒च्छ॒तु॒ यं द्वि॒ष्म इत्या॑ह॒ यमे॒व द्वेष्टि॒ तम॑स्य क्षु॒धा च॑ शु॒चा चा᳚र्पयति॒ त्रिः प॑रिषि॒ञ्चन्पर्ये॑ति त्रि॒वृद्वा अ॒ग्निर्यावा॑ने॒वाग्निस्तस्य॒ शुचꣳ॑ शमयति॒ त्रिः पुनः॒ पर्ये॑ति॒ षट्थ्सम्प॑द्यन्ते॒ षड्वा ऋ॒तव॑ ऋ॒तुभि॑रे॒वास्य॒ शुचꣳ॑ शमयत्य॒पां वा ए॒तत्पुष्पं॒ यद्वे॑त॒सो॑\-ऽपाम्॥१५॥

%5.4.4.3
शरो\-ऽव॑का वेतसशा॒खया॒ चाव॑काभिश्च॒ वि क॑र्\mbox{}ष॒त्यापो॒ वै शा॒न्ताः शा॒न्ताभि॑रे॒वास्य॒ शुचꣳ॑ शमयति॒ यो वा अ॒ग्निं चि॒तम्प्र॑थ॒मः प॒शुर॑धि॒क्राम॑तीश्व॒रो वै तꣳ शु॒चा प्र॒दहो म॒ण्डूके॑न॒ वि क॑र्\mbox{}षत्ये॒ष वै प॑शू॒नाम॑नुपजीवनी॒यो न वा ए॒ष ग्रा॒म्येषु॑ प॒शुषु॑ हि॒तो नार॒ण्येषु॒ तमे॒व शु॒चार्प॑यत्यष्टा॒भिर्वि क॑र्\mbox{}षति॥१६॥

%5.4.4.4
अ॒ष्टाक्ष॑रा गाय॒त्री गा॑य॒त्रो᳚\-ऽग्निर्यावा॑ने॒वाग्निस्तस्य॒ शुचꣳ॑ शमयति पाव॒कव॑तीभि॒रन्नं॒ वै पा॑व॒को\-ऽन्ने॑नै॒वास्य॒ शुचꣳ॑ शमयति मृ॒त्युर्वा ए॒ष यद॒ग्निर्ब्रह्म॑ण ए॒तद्रू॒पं यत्कृ॑ष्णाजि॒नम् कार्\mbox{}ष्णी॑ उपा॒नहा॒वुप॑ मुञ्चते॒ ब्रह्म॑णै॒व मृ॒त्योर॒न्तर्ध॑त्ते॒\-ऽन्तर्मृ॒त्योर्ध॑त्ते॒\-ऽन्तर॒न्नाद्या॒दित्या॑हुर॒न्यामु॑पमु॒ञ्चते॒\-ऽन्यां नान्तः॥१७॥

%5.4.4.5
ए॒व मृ॒त्योर्ध॒त्ते\-ऽवा॒न्नाद्यꣳ॑ रुन्द्धे॒ नम॑स्ते॒ हर॑से शो॒चिष॒ इत्या॑ह नम॒स्कृत्य॒ हि वसी॑याꣳसमुप॒चर॑न्त्य॒न्यं ते॑ अ॒स्मत्त॑पन्तु हे॒तय॒ इत्या॑ह॒ यमे॒व द्वेष्टि॒ तम॑स्य शु॒चार्प॑यति पाव॒को अ॒स्मभ्यꣳ॑ शि॒वो भ॒वेत्या॒हान्नं॒ वै पा॑व॒को\-ऽन्न॑मे॒वाव॑ रुन्द्धे॒ द्वाभ्या॒मधि॑ क्रामति॒ प्रति॑ष्ठित्या अप॒स्य॑वतीभ्या॒ꣳ॒ शान्त्यै᳚॥१८॥

%5.4.5.0
{\anuvakamend[{शुग्वे॑त॒सो॑\-ऽपाम॑ष्टा॒भिर्विक॑र्\mbox{}षति॒ नान्तरेका॒न्नप॑ञ्चा॒शच्च॑}]}%॥४॥

%5.4.5.1
नृ॒षदे॒ वडिति॒ व्याघा॑रयति प॒ङ्क्त्याहु॑त्या यज्ञमु॒खमा र॑भते\-ऽक्ष्ण॒या व्याघा॑रयति॒ तस्मा॑दक्ष्ण॒या प॒शवो\-ऽङ्गा॑नि॒ प्र ह॑रन्ति॒ प्रति॑ष्ठित्यै॒ यद्व॑षट्कु॒र्याद्या॒तया॑मास्य वषट्का॒रः स्या॒द्यन्न व॑षट्कु॒॒र्याद्रक्षाꣳ॑सि य॒ज्ञꣳ ह॑न्यु॒र्वडित्या॑ह प॒रोक्ष॑मे॒व वष॑ट्करोति॒ नास्य॑ या॒तया॑मा वषट्का॒रो भव॑ति॒ न य॒ज्ञꣳ रक्षाꣳ॑सि घ्नन्ति हु॒तादो॒ वा अ॒न्ये दे॒वाः॥१९॥

%5.4.5.2
अ॒हु॒तादो॒\-ऽन्ये तान॑ग्नि॒चिदे॒वोभया᳚न्प्रीणाति॒ ये दे॒वा दे॒वाना॒मिति॑ द॒ध्ना म॑धुमि॒श्रेणावो᳚क्षति हु॒ताद॑श्चै॒व दे॒वान॑हु॒ताद॑श्च॒ यज॑मानः प्रीणाति॒ ते यज॑मानम्प्रीणन्ति द॒ध्नैव हु॒तादः॑ प्री॒णाति॒ मधु॑षाहु॒तादो᳚ ग्रा॒म्यं वा ए॒तदन्नं॒ यद्दध्या॑र॒ण्यम्मधु॒ यद्द॒ध्ना म॑धुमि॒श्रेणा॒वोक्ष॑त्यु॒भय॒स्याव॑रुद्ध्यै ग्रुमु॒ष्टिनावो᳚क्षति प्राजाप॒त्यः॥२०॥

%5.4.5.3
वै ग्रु॑मु॒ष्टिः स॑योनि॒त्वाय॒ द्वाभ्यां॒ प्रति॑ष्ठित्या अनुपरि॒चार॒मवो᳚क्ष॒त्यप॑रिवर्गमे॒वैना᳚न्प्रीणाति॒ वि वा ए॒ष प्रा॒णैः प्र॒जया॑ प॒शुभि॑र्\mbox{}ऋध्यते॒ यो᳚\-ऽग्निं चि॒न्वन्न॑धि॒क्राम॑ति प्राण॒दा अ॑पान॒दा इत्या॑ह प्रा॒णाने॒वात्मन्ध॑त्ते वर्चो॒दा व॑रिवो॒दा इत्या॑ह प्र॒जा वै वर्चः॑ प॒शवो॒ वरि॑वः प्र॒जामे॒व प॒शूना॒त्मन्ध॑त्त॒ इन्द्रो॑ वृ॒त्रम॑ह॒न्तं वृ॒त्रः॥२१॥

%5.4.5.4
ह॒तः षो॑ड॒शभि॑र्भो॒गैर॑सिना॒थ्स ए॒ताम॒ग्नये\-ऽनी॑कवत॒ आहु॑तिमपश्य॒त्ताम॑जुहो॒त्तस्या॒ग्निरनी॑कवा॒न्थ्स्वेन॑ भाग॒धेये॑न प्री॒तः षो॑डश॒धा वृ॒त्रस्य॑ भो॒गानप्य॑दहद्वैश्वकर्म॒णेन॑ पा॒प्मनो॒ निर॑मुच्यत॒ यद॒ग्नये\-ऽनी॑कवत॒ आहु॑तिं जु॒होत्य॒ग्निरे॒वास्यानी॑कवा॒न्थ्स्वेन॑ भाग॒धेये॑न प्री॒तः पा॒प्मान॒मपि॑ दहति वैश्वकर्म॒णेन॑ पा॒प्मनो॒ निर्मु॑च्यते॒ यं का॒मये॑त चि॒रम्पा॒प्मनः॑॥२२॥

%5.4.5.5
निर्मु॑च्ये॒तेत्येकै॑कं॒ तस्य॑ जुहुयाच्चि॒रमे॒व पा॒प्मनो॒ निर्मु॑च्यते॒ यं का॒मये॑त ता॒जक्पा॒प्मनो॒ निर्मु॑च्ये॒तेति॒ सर्वा॑णि॒ तस्या॑नु॒द्रुत्य॑ जुहुयात्ता॒जगे॒व पा॒प्मनो॒ निर्मु॑च्य॒ते\-ऽथो॒ खलु॒ नानै॒व सू॒क्ता\-भ्यां᳚ जुहोति॒ नानै॒व सू॒क्तयो᳚र्वी॒र्यं॑ दधा॒त्यथो॒ प्रति॑ष्ठित्यै॥२३॥

%5.4.6.0
{\anuvakamend[{दे॒वाः प्रा॑जाप॒त्यो वृ॒त्रश्चि॒रं पा॒प्मन॑श्चत्वारि॒ꣳ॒शच्च॑}]}%॥५॥

%5.4.6.1
उदे॑नमुत्त॒रां न॒येति॑ स॒मिध॒ आ द॑धाति॒ यथा॒ जनं॑ य॒ते॑\-ऽव॒सं क॒रोति॑ ता॒दृगे॒व तत्ति॒स्र आ द॑धाति त्रि॒वृद्वा अ॒ग्निर्यावा॑ने॒वाग्निस्तस्मै॑ भाग॒धेयं॑ करो॒त्यौदु॑म्बरीर्भव॒न्त्यूर्ग्वा उ॑दु॒म्बर॒ ऊर्ज॑मे॒वास्मा॒ अपि॑ दधा॒त्युदु॑ त्वा॒ विश्वे॑ दे॒वा इत्या॑ह प्रा॒णा वै विश्वे॑ दे॒वाः प्रा॒णैः॥२४॥

%5.4.6.2
ए॒वैन॒मुद्य॑च्छ॒ते\-ऽग्ने॒ भर॑न्तु॒ चित्ति॑भि॒रित्या॑ह॒ यस्मा॑ ए॒वैनं॑ चि॒त्तायो॒द्यच्छ॑ते॒ तेनै॒वैन॒ꣳ॒ सम॑र्धयति॒ पञ्च॒ दिशो॒ दैवी᳚र्य॒ज्ञम॑वन्तु दे॒वीरित्या॑ह॒ दिशो॒ ह्ये॑षो\-ऽनु॑ प्र॒च्यव॒ते\-ऽपाम॑तिं दुर्म॒तिम्बाध॑माना॒ इत्या॑ह॒ रक्ष॑सा॒मप॑हत्यै रा॒यस्पोषे॑ य॒ज्ञप॑तिमा॒भज॑न्ती॒रित्या॑ह प॒शवो॒ वै रा॒यस्पोषः॑॥२५॥

%5.4.6.3
प॒शूने॒वाव॑ रुन्द्धे ष॒ड्भिर्\mbox{}ह॑रति॒ षड्वा ऋ॒तव॑ ऋ॒तुभि॑रे॒वैनꣳ॑ हरति॒ द्वे प॑रि॒गृह्य॑वती भवतो॒ रक्ष॑सा॒मप॑हत्यै॒ सूर्य॑रश्मि॒र्\mbox{}हरि॑केशः पु॒रस्ता॒दित्या॑ह॒ प्रसू᳚त्यै॒ ततः॑ पाव॒का आ॒शिषो॑ नो जुषन्ता॒मित्या॒हान्नं॒ वै पा॑व॒को\-ऽन्न॑मे॒वाव॑ रुन्द्धे देवासु॒राः संय॑त्ता आस॒न्ते दे॒वा ए॒तदप्र॑तिरथमपश्य॒न्तेन॒ वै ते᳚\-ऽप्र॒ति॥२६॥

%5.4.6.4
असु॑रानजय॒न्तदप्र॑तिरथस्याप्रतिरथ॒त्वं यदप्र॑तिरथं द्वि॒तीयो॒ होता॒न्वाहा᳚प्र॒त्ये॑व तेन॒ यज॑मानो॒ भ्रातृ॑व्याञ्जय॒त्यथो॒ अन॑भिजितमे॒वाभि ज॑यति दश॒र्चम्भ॑वति॒ दशा᳚क्षरा वि॒राड्वि॒राजे॒मौ लो॒कौ विधृ॑ताव॒नयो᳚र्लो॒कयो॒र्विधृ॑त्या॒ अथो॒ दशा᳚क्षरा वि॒राडन्नं॑ वि॒राड्वि॒राज्ये॒वान्नाद्ये॒ प्रति॑ तिष्ठ॒त्यस॑दिव॒ वा अ॒न्तरि॑क्षम॒न्तरि॑क्षमि॒वाग्नी᳚ध्र॒माग्नी᳚ध्रे॥२७॥

%5.4.6.5
अश्मा॑नं॒ नि द॑धाति स॒त्त्वाय॒ द्वाभ्यां॒ प्रति॑ष्ठित्यै वि॒मान॑ ए॒ष दि॒वो मध्य॑ आस्त॒ इत्या॑ह॒ व्ये॑वैतया॑ मिमीते॒ मध्ये॑ दि॒वो निहि॑तः पृश्नि॒रश्मेत्या॒हान्नं॒ वै पृश्न्यन्न॑मे॒वाव॑ रुन्द्धे चत॒सृभि॒रा पुच्छा॑देति च॒त्वारि॒ छन्दाꣳ॑सि॒ छन्दो॑भिरे॒वेन्द्रं॒ विश्वा॑ अवीवृध॒न्नित्या॑ह॒ वृद्धि॑मे॒वोपाव॑र्तते॒ वाजा॑ना॒ꣳ॒ सत्प॑ति॒म्पतिम्᳚॥२८॥

%5.4.6.6
इत्या॒हान्नं॒ वै वाजो\-ऽन्न॑मे॒वाव॑ रुन्द्धे सुम्न॒हूर्य॒ज्ञो दे॒वाꣳ आ च॑ वक्ष॒दित्या॑ह प्र॒जा वै प॒शवः॑ सु॒म्नं प्र॒जामे॒व प॒शूना॒त्मन्ध॑त्ते॒ यक्ष॑द॒ग्निर्दे॒वो दे॒वाꣳ आ च॑ वक्ष॒दित्या॑ह स्व॒गाकृ॑त्यै॒ वाज॑स्य मा प्रस॒वेनो᳚द्ग्रा॒भेणोद॑ग्रभी॒दित्या॑हा॒सौ वा आ॑दि॒त्य उ॒द्यन्नु॑द्ग्रा॒भ ए॒ष नि॒म्रोच॑न्निग्रा॒भो ब्रह्म॑णै॒वात्मान॑मुद्गृ॒ह्णाति॒ ब्रह्म॑णा॒ भ्रातृ॑व्यं॒ नि गृ॑ह्णाति॥२९॥

%5.4.7.0
{\anuvakamend[{प्रा॒णैः पोषो᳚\-ऽप्र॒त्याग्नी᳚ध्रे॒ पति॑मे॒ष दश॑ च}]}%॥६॥

%5.4.7.1
प्राची॒मनु॑ प्र॒दिश॒म्प्रेहि॑ वि॒द्वानित्या॑ह देवलो॒कमे॒वैतयो॒पाव॑र्तते॒ क्रम॑ध्वम॒ग्निना॒ नाक॒मित्या॑हे॒माने॒वैतया॑ लो॒कान्क्र॑मते पृथि॒व्या अ॒हमुद॒न्तरि॑क्ष॒मारु॑ह॒मित्या॑हे॒माने॒वैतया॑ लो॒कान्थ्स॒मारो॑हति॒ सुव॒र्यन्तो॒ नापे᳚क्षन्त॒ इत्या॑ह सुव॒र्गमे॒वैतया॑ लो॒कमे॒त्यग्ने॒ प्रेहि॑॥३०॥

%5.4.7.2
प्र॒थ॒मो दे॑वय॒तामित्या॑हो॒भये᳚ष्वे॒वैतया॑ देवमनु॒ष्येषु॒ चक्षु॑र्दधाति प॒ञ्चभि॒रधि॑ क्रामति॒ पाङ्क्तो॑ य॒ज्ञो यावा॑ने॒व य॒ज्ञस्तेन॑ स॒ह सु॑व॒र्गं लो॒कमे॑ति॒ नक्तो॒षासेति॑ पुरोनुवा॒क्या॑मन्वा॑ह॒ प्रत्त्या॒ अग्ने॑ सहस्रा॒क्षेत्या॑ह साह॒स्रः प्र॒जाप॑तिः प्र॒जाप॑ते॒राप्त्यै॒ तस्मै॑ ते विधेम॒ वाजा॑य॒ स्वाहेत्या॒हान्नं॒ वै वाजो\-ऽन्न॑मे॒वाव॑॥३१॥

%5.4.7.3
रु॒न्द्धे॒ द॒ध्नः पू॒र्णामौदु॑म्बरीꣴ स्वयमातृ॒ण्णायां᳚ जुहो॒त्यूर्ग्वै दध्यूर्गु॑दु॒म्बरो॒\-ऽसौ स्व॑यमातृ॒ण्णामुष्या॑मे॒वोर्जं॑ दधाति॒ तस्मा॑द॒मुतो॒\-ऽर्वाची॒मूर्ज॒मुप॑ जीवामस्ति॒सृभिः॑ सादयति त्रि॒वृद्वा अ॒ग्निर्यावा॑ने॒वाग्निस्तम्प्र॑ति॒ष्ठां ग॑मयति॒ प्रेद्धो॑ अग्ने दीदिहि पु॒रो न॒ इत्यौदुम्ब॑री॒मा द॑धात्ये॒षा वै सू॒र्मी कर्ण॑कावत्ये॒तया॑ ह स्म॥३२॥

%5.4.7.4
वै दे॒वा असु॑राणाꣳ शतत॒र्\mbox{}हाꣴस्तृꣳ॑हन्ति॒ यदे॒तया॑ स॒मिध॑मा॒दधा॑ति॒ वज्र॑मे॒वैतच्छ॑त॒घ्नीं यज॑मानो॒ भ्रातृ॑व्याय॒ प्र ह॑रति॒ स्तृत्या॒ अछ॑म्बट्कारं वि॒धेम॑ ते पर॒मे जन्म॑न्नग्न॒ इति॒ वैक॑ङ्कती॒मा द॑धाति॒ भा ए॒वाव॑ रुन्द्धे॒ ताꣳ स॑वि॒तुर्वरे᳚ण्यस्य चि॒त्रामिति॑ शमी॒मयी॒ꣳ॒ शान्त्या॑ अ॒ग्निर्वा॑ ह॒ वा अ॑ग्नि॒चितं॑ दु॒हे᳚\-ऽग्नि॒चिद्वा॒ग्निं दु॑हे॒ ताम्॥३३॥

%5.4.7.5
स॒वि॒तुर्वरे᳚ण्यस्य चि॒त्रामित्या॑है॒ष वा अ॒ग्नेर्दोह॒स्तम॑स्य॒ कण्व॑ ए॒व श्रा॑य॒सो॑\-ऽवे॒त्तेन॑ ह स्मैन॒ꣳ॒ स दु॑हे॒ यदे॒तया॑ स॒मिध॑मा॒दधा᳚त्यग्नि॒चिदे॒व तद॒ग्निं दु॑हे स॒प्त ते॑ अग्ने स॒मिधः॑ स॒प्त जि॒ह्वा इत्या॑ह स॒प्तैवास्य॒ साप्ता॑नि प्रीणाति पू॒र्णया॑ जुहोति पू॒र्ण इ॑व॒ हि प्र॒जाप॑तिः प्र॒जाप॑तेः॥३४॥

%5.4.7.6
आप्त्यै॒ न्यू॑नया जुहोति॒ न्यू॑ना॒द्धि प्र॒जाप॑तिः प्र॒जा असृ॑जत प्र॒जाना॒ꣳ॒ सृष्ट्या॑ अ॒ग्निर्दे॒वेभ्यो॒ निला॑यत॒ स दिशो\-ऽनु॒ प्रावि॑श॒ज्जुह्व॒न्मन॑सा॒ दिशो᳚ ध्यायेद्दि॒ग्भ्य ए॒वैन॒मव॑ रुन्द्धे द॒ध्ना पु॒रस्ता᳚ज्जुहो॒त्याज्ये॑नो॒परि॑ष्टा॒त्तेज॑श्चै॒वास्मा॑ इन्द्रि॒यं च॑ स॒मीची॑ दधाति॒ द्वाद॑शकपालो वैश्वान॒रो भ॑वति॒ द्वाद॑श॒ मासाः᳚ संवथ्स॒रः सं॑वथ्स॒रो᳚\-ऽग्निर्वै᳚श्वान॒रः सा॒क्षात्॥३५॥

%5.4.7.7
ए॒व वै᳚श्वान॒रमव॑ रुन्द्धे॒ यत्प्र॑याजानूया॒जान्कु॒र्याद्विक॑स्तिः॒ सा य॒ज्ञस्य॑ दर्विहो॒मं क॑रोति य॒ज्ञस्य॒ प्रति॑ष्ठित्यै रा॒ष्ट्रं वै वै᳚श्वान॒रो विण्म॒रुतो वैश्वान॒रꣳ हु॒त्वा मा॑रु॒ताञ्जु॑होति रा॒ष्ट्र ए॒व विश॒मनु॑ बध्नात्यु॒च्चैर्वै᳚श्वान॒रस्या श्रा॑वयत्युपा॒ꣳ॒शु मा॑रु॒ताञ्जु॑होति॒ तस्मा᳚द्रा॒ष्ट्रं विश॒मति॑ वदति मारु॒ता भ॑वन्ति म॒रुतो॒ वै दे॒वानां॒ विशो॑ देववि॒शेनै॒वास्मै॑ मनुष्यवि॒शमव॑ रुन्द्धे स॒प्त भ॑वन्ति स॒प्तग॑णा॒ वै म॒रुतो॑ गण॒श ए॒व विश॒मव॑ रुन्द्धे ग॒णेन॑ ग॒णम॑नु॒द्रुत्य॑ जुहोति॒ विश॑मे॒वास्मा॒ अनु॑वर्त्मानं करोति॥३६॥

%5.4.8.0
{\anuvakamend[{अग्ने॒ प्रेह्यव॑ स्म दुहे॒ तां प्र॒जाप॑तेः सा॒क्षान्म॑नुष्यवि॒शमेक॑विꣳशतिश्च}]}%॥७॥

%5.4.8.1
वसो॒र्धारां᳚ जुहोति॒ वसो᳚र्मे॒ धारा॑स॒दिति॒ वा ए॒षा हू॑यते घृ॒तस्य॒ वा ए॑नमे॒षा धारा॒मुष्मि॑ल्लोँ॒के पिन्व॑मा॒नोप॑ तिष्ठत॒ आज्ये॑न जुहोति॒ तेजो॒ वा आज्यं॒ तेजो॒ वसो॒र्धारा॒ तेज॑सै॒वास्मै॒ तेजो\-ऽव॑ रु॒न्द्धे\-ऽथो॒ कामा॒ वै वसो॒र्धारा॒ कामा॑ने॒वाव॑ रुन्द्धे॒ यं का॒मये॑त प्रा॒णान॑स्या॒न्नाद्यं॒ वि॥३७॥

%5.4.8.2
छि॒न्द्या॒मिति॑ वि॒ग्राहं॒ तस्य॑ जुहुयात्प्रा॒णाने॒वास्या॒न्नाद्यं॒ विच्छि॑नत्ति॒ यं का॒मये॑त प्रा॒णान॑स्या॒न्नाद्य॒ꣳ॒ सं त॑नुया॒मिति॒ सन्त॑तां॒ तस्य॑ जुहुयात्प्रा॒णाने॒वास्या॒न्नाद्य॒ꣳ॒ सं त॑नोति॒ द्वाद॑श द्वाद॒शानि॑ जुहोति॒ द्वाद॑श॒ मासाः᳚ संवथ्स॒रः संवथ्स॒रेणै॒वास्मा॒ अन्न॒मव॑ रु॒न्द्धे\-ऽन्नं॑ च॒ मे\-ऽक्षु॑च्च म॒ इत्या॑है॒तद्वै॥३८॥

%5.4.8.3
अन्न॑स्य रू॒पꣳ रू॒पेणै॒वान्न॒मव॑ रुन्द्धे॒\-ऽग्निश्च॑ म॒ आप॑श्च म॒ इत्या॑है॒षा वा अन्न॑स्य॒ योनिः॒ सयो᳚न्ये॒वान्न॒मव॑ रुन्द्धे\-ऽर्धे॒न्द्राणि॑ जुहोति दे॒वता॑ ए॒वाव॑ रुन्द्धे॒ यथ्सर्वे॑षाम॒र्धमिन्द्रः॒ प्रति॒ तस्मा॒दिन्द्रो॑ दे॒वता॑नाम्भूयिष्ठ॒भाक्त॑म॒ इन्द्र॒मुत्त॑रमाहेन्द्रि॒यमे॒वास्मि॑न्नु॒परि॑ष्टाद्दधाति यज्ञायु॒धानि॑ जुहोति य॒ज्ञः॥३९॥

%5.4.8.4
वै य॑ज्ञायु॒धानि॑ य॒ज्ञमे॒वाव॑ रु॒न्द्धे\-ऽथो॑ ए॒तद्वै य॒ज्ञस्य॑ रू॒पꣳ रू॒पेणै॒व य॒ज्ञमव॑ रुन्द्धे\-ऽवभृ॒थश्च॑ मे स्वगाका॒रश्च॑ म॒ इत्या॑ह स्व॒गाकृ॑त्या अ॒ग्निश्च॑ मे घ॒र्मश्च॑ म॒ इत्या॑है॒तद्वै ब्र॑ह्मवर्च॒सस्य॑ रू॒पꣳ रू॒पेणै॒व ब्र॑ह्मवर्च॒समव॑ रुन्द्ध॒ ऋक्च॑ मे॒ साम॑ च म॒ इत्या॑ह॥४०॥

%5.4.8.5
ए॒तद्वै छन्द॑साꣳ रू॒पꣳ रू॒पेणै॒व छन्दा॒ꣳ॒स्यव॑ रुन्द्धे॒ गर्भा᳚श्च मे व॒थ्साश्च॑ म॒ इत्या॑है॒तद्वै प॑शू॒नाꣳ रू॒पꣳ रू॒पेणै॒व प॒शूनव॑ रुन्द्धे॒ कल्पा᳚ञ्जुहो॒त्यकॢ॑प्तस्य॒ कॢप्त्यै॑ युग्मदयु॒जे जु॑होति मिथुन॒त्वायो᳚त्त॒राव॑ती भवतो॒\-ऽभिक्रा᳚न्त्या॒ एका॑ च मे ति॒स्रश्च॑ म॒ इत्या॑ह देवछन्द॒सं वा एका॑ च ति॒स्रश्च॑॥४१॥

%5.4.8.6
म॒नु॒ष्य॒छ॒न्द॒सं चत॑स्रश्चा॒ष्टौ च॑ देवछन्द॒सं चै॒व म॑नुष्यछन्द॒सं चाव॑ रुन्द्ध॒ आ त्रय॑स्त्रिꣳशतो जुहोति॒ त्रय॑स्त्रिꣳश॒द्वै दे॒वता॑ दे॒वता॑ ए॒वाव॑ रुन्द्ध॒ आष्टाच॑त्वारिꣳशतो जुहोत्य॒ष्टाच॑त्वारिꣳशदक्षरा॒ जग॑ती॒ जाग॑ताः प॒शवो॒ जग॑त्यै॒वास्मै॑ प॒शूनव॑ रुन्द्धे॒ वाज॑श्च प्रस॒वश्चेति॑ द्वाद॒शं जु॑होति॒ द्वाद॑श॒ मासाः᳚ संवथ्स॒रः सं॑वथ्स॒र ए॒व प्रति॑ तिष्ठति॥४२॥

%5.4.9.0
{\anuvakamend[{वि वै य॒ज्ञः साम॑ च म॒ इत्या॑ह च ति॒स्रश्चैका॒न्नप॑ञ्चा॒शच्च॑}]}%॥८॥

%5.4.9.1
अ॒ग्निर्दे॒वेभ्यो\-ऽपा᳚क्रामद्भाग॒धेय॑मि॒च्छमा॑न॒स्तं दे॒वा अ॑ब्रुव॒न्नुप॑ न॒ आ व॑र्तस्व ह॒व्यं नो॑ व॒हेति॒ सो᳚\-ऽब्रवी॒द्वरं॑ वृणै॒ मह्य॑मे॒व वा॑जप्रस॒वीयं॑ जुहव॒न्निति॒ तस्मा॑द॒ग्नये॑ वाजप्रस॒वीयं॑ जुह्वति॒ यद्वा॑जप्रस॒वीयं॑ जु॒होत्य॒ग्निमे॒व तद्भा॑ग॒धेये॑न॒ सम॑र्धय॒त्यथो॑ अभिषे॒क ए॒वास्य॒ स च॑तुर्द॒शभि॑र्जुहोति स॒प्त ग्रा॒म्या ओष॑धयः स॒प्त॥४३॥

%5.4.9.2
आ॒र॒ण्या उ॒भयी॑षा॒मव॑रुद्ध्या॒ अन्न॑स्यान्नस्य जुहो॒त्यन्न॑स्यान्न॒स्याव॑रुद्ध्या॒ औदु॑म्बरेण स्रु॒वेण॑ जुहो॒त्यूर्ग्वा उ॑दु॒म्बर॒ ऊर्गन्न॑मू॒र्जैवास्मा॒ ऊर्ज॒मन्न॒मव॑ रुन्द्धे॒\-ऽग्निर्वै दे॒वाना॑म॒भिषि॑क्तो\-ऽग्नि॒चिन्म॑नु॒ष्या॑णा॒न्तस्मा॑दग्नि॒चिद्वर्\mbox{}ष॑ति॒ न धा॑वे॒दव॑रुद्ध॒ꣴ॒ ह्य॑स्यान्न॒मन्न॑मिव॒ खलु॒ वै व॒र्\mbox{}षं यद्धावे॑द॒न्नाद्या᳚द्धावेदु॒पाव॑र्तेता॒न्नाद्य॑मे॒वाभि॥४४॥

%5.4.9.3
उ॒पाव॑र्तते॒ नक्तो॒षासेति॑ कृ॒ष्णायै᳚ श्वे॒तव॑थ्सायै॒ पय॑सा जुहो॒त्यह्नै॒वास्मै॒ रात्रि॒म्प्र दा॑पयति॒ रात्रि॒याह॑रहोरा॒त्रे ए॒वास्मै॒ प्रत्ते॒ काम॑म॒न्नाद्यं॑ दुहाते राष्ट्र॒भृतो॑ जुहोति रा॒ष्ट्रमे॒वाव॑ रुन्द्धे ष॒ड्भिर्जु॑होति॒ षड्वा ऋ॒तव॑ ऋ॒तुष्वे॒व प्रति॑ तिष्ठति॒ भुव॑नस्य पत॒ इति॑ रथमु॒खे पञ्चाहु॑तीर्जुहोति॒ वज्रो॒ वै रथो॒ वज्रे॑णै॒व दिशः॑॥४५॥

%5.4.9.4
अ॒भि ज॑यत्यग्नि॒चितꣳ॑ ह॒ वा अ॒मुष्मि॑ल्लोँ॒के वातो॒\-ऽभि प॑वते वातना॒मानि॑ जुहोत्य॒भ्ये॑वैन॑म॒मुष्मि॑ल्लोँ॒के वातः॑ पवते॒ त्रीणि॑ जुहोति॒ त्रय॑ इ॒मे लो॒का ए॒भ्य ए॒व लो॒केभ्यो॒ वात॒मव॑ रुन्द्धे समु॒द्रो॑\-ऽसि॒ नभ॑स्वा॒नित्या॑है॒तद्वै वात॑स्य रू॒पꣳ रू॒पेणै॒व वात॒मव॑ रुन्द्धे\-ऽञ्ज॒लिना॑ जुहोति॒ न ह्ये॑तेषा॑म॒न्यथाहु॑तिरव॒कल्प॑ते॥४६॥

%5.4.10.0
{\anuvakamend[{ओष॑धयः स॒प्ताभि दिशो॒\-ऽन्यथा॒ द्वे च॑}]}%॥९॥

%5.4.10.1
सुव॒र्गाय॒ वै लो॒काय॑ देवर॒थो यु॑ज्यते यत्राकू॒ताय॑ मनुष्यर॒थ ए॒ष खलु॒ वै दे॑वर॒थो यद॒ग्निर॒ग्निं यु॑नज्मि॒ शव॑सा घृ॒तेनेत्या॑ह यु॒नक्त्ये॒वैन॒ꣳ॒ स ए॑नं यु॒क्तः सु॑व॒र्गं लो॒कम॒भि व॑हति॒ यथ्सर्वा॑भिः प॒ञ्चभि॑र्यु॒ञ्ज्याद्यु॒क्तो᳚\-ऽस्या॒ग्निः प्रच्यु॑तः स्या॒दप्र॑तिष्ठिता॒ आहु॑तयः॒ स्युरप्र॑तिष्ठिताः॒ स्तोमा॒ अप्र॑तिष्ठितान्यु॒क्थानि॑ ति॒सृभिः॑ प्रातःसव॒ने॑\-ऽभि मृ॑शति त्रि॒वृत्॥४७॥

%5.4.10.2
वा अ॒ग्निर्यावा॑ने॒वाग्निस्तं यु॑नक्ति॒ यथान॑सि यु॒क्त आ॑धी॒यत॑ ए॒वमे॒व तत्प्रत्याहु॑तय॒स्तिष्ठ॑न्ति॒ प्रति॒ स्तोमाः॒ प्रत्यु॒क्थानि॑ यज्ञाय॒ज्ञिय॑स्य स्तो॒त्रे द्वाभ्या॑म॒भि मृ॑शत्ये॒तावा॒न् वै य॒ज्ञो यावा॑नग्निष्टो॒मो भू॒मा त्वा अ॒स्यात॑ ऊ॒र्ध्वः क्रि॑यते॒ यावा॑ने॒व य॒ज्ञस्तम॑न्त॒तो᳚\-ऽन्वारो॑हति॒ द्वाभ्यां॒ प्रति॑ष्ठित्या॒ एक॒याप्र॑स्तुत॒म्भव॒त्यथ॑॥४८॥

%5.4.10.3
अ॒भि मृ॑श॒त्युपै॑न॒मुत्त॑रो य॒ज्ञो न॑म॒त्यथो॒ सन्त॑त्यै॒ प्र वा ए॒षो᳚\-ऽस्माल्लो॒काच्च्य॑वते॒ यो᳚\-ऽग्निं चि॑नु॒ते न वा ए॒तस्या॑निष्ट॒क आहु॑ति॒रव॑ कल्पते॒ यां वा ए॒षो॑\-ऽनिष्ट॒क आहु॑तिं जु॒होति॒ स्रव॑ति॒ वै सा ताꣴ स्रव॑न्तीं य॒ज्ञो\-ऽनु॒ परा॑ भवति य॒ज्ञं यज॑मानो॒ यत्पु॑नश्चि॒तिं चि॑नु॒त आहु॑तीनां॒ प्रति॑ष्ठित्यै॒ प्रत्याहु॑तय॒स्तिष्ठ॑न्ति॥४९॥

%5.4.10.4
न य॒ज्ञः प॑रा॒भव॑ति॒ न यज॑मानो॒\-ऽष्टावुप॑ दधात्य॒ष्टाक्ष॑रा गाय॒त्री गा॑य॒त्रेणै॒वैनं॒ छन्द॑सा चिनुते॒ यदेका॑दश॒ त्रैष्टु॑भेन॒ यद्द्वाद॑श॒ जाग॑तेन॒ छन्दो॑भिरे॒वैनं॑ चिनुते नपा॒त्को वै नामै॒षो᳚\-ऽग्निर्यत्पु॑नश्चि॒तिर्य ए॒वं वि॒द्वान्पु॑नश्चि॒तिं चि॑नु॒त आ तृ॒तीया॒त्पुरु॑षा॒दन्न॑मत्ति॒ यथा॒ वै पु॑नरा॒धेय॑ ए॒वम्पु॑नश्चि॒तिर्यो᳚\-ऽग्न्या॒धेये॑न॒ न॥५०॥

%5.4.10.5
ऋ॒ध्नोति॒ स पु॑नरा॒धेय॒मा ध॑त्ते॒ यो᳚\-ऽग्निं चि॒त्वा नर्ध्नोति॒ स पु॑नश्चि॒तिं चि॑नुते॒ यत्पु॑नश्चि॒तिं चि॑नु॒त ऋद्ध्या॒ अथो॒ खल्वा॑हु॒र्न चे॑त॒व्येति॑ रु॒द्रो वा ए॒ष यद॒ग्निर्यथा᳚ व्या॒घ्रꣳ सु॒प्तम्बो॒धय॑ति ता॒दृगे॒व तदथो॒ खल्वा॑हुश्चेत॒व्येति॒ यथा॒ वसी॑याꣳसम्भाग॒धेये॑न बो॒धय॑ति ता॒दृगे॒व तन्मनु॑र॒ग्निम॑चिनुत॒ तेन॒ नार्ध्नो॒थ्स ए॒ताम्पु॑नश्चि॒तिम॑पश्य॒त्ताम॑चिनुत॒ तया॒ वै स आ᳚र्ध्नो॒द्यत्पु॑नश्चि॒तिं चि॑नु॒त ऋद्ध्यै᳚॥५१॥

%5.4.11.0
{\anuvakamend[{त्रि॒वृदथ॒ तिष्ठ॑न्त्यग्न्या॒धेये॑न॒ नाचि॑नुत स॒प्तद॑श च}]}%॥10॥

%5.4.11.1
छ॒न्द॒श्चितं॑ चिन्वीत प॒शुका॑मः प॒शवो॒ वै छन्दाꣳ॑सि पशु॒माने॒व भ॑वति श्येन॒चितं॑ चिन्वीत सुव॒र्गका॑मः श्ये॒नो वै वय॑सा॒म्पति॑ष्ठः श्ये॒न ए॒व भू॒त्वा सु॑व॒र्गं लो॒कम्प॑तति कङ्क॒चितं॑ चिन्वीत॒ यः का॒मये॑त शीर्\mbox{}ष॒ण्वान॒मुष्मि॑ल्लोँ॒के स्या॒मिति॑ शीर्\mbox{}ष॒ण्वाने॒वामुष्मि॑ल्लोँ॒के भ॑वत्यलज॒चितं॑ चिन्वीत॒ चतुः॑सीतं प्रति॒ष्ठाका॑म॒श्चत॑स्रो॒ दिशो॑ दि॒क्ष्वे॑व प्रति॑ तिष्ठति प्रउग॒चितं॑ चिन्वीत॒ भ्रातृ॑व्यवा॒न्प्र॥५२॥

%5.4.11.2
ए॒व भ्रातृ॑व्यान्नुदत उभ॒यतः॑प्रउगं चिन्वीत॒ यः का॒मये॑त॒ प्र जा॒तान्भ्रातृ॑व्यान्नु॒देय॒ प्रति॑जनि॒ष्यमा॑णा॒निति॒ प्रैव जा॒तान्भ्रातृ॑व्यान्नु॒दते॒ प्रति॑ जनि॒ष्यमा॑णान्रथचक्र॒चितं॑ चिन्वीत॒ भ्रातृ॑व्य॒वान् वज्रो॒ वै रथो॒ वज्र॑मे॒व भ्रातृ॑व्येभ्यः॒ प्र ह॑रति द्रोण॒चितं॑ चिन्वी॒तान्न॑कामो॒ द्रोणे॒ वा अन्न॑म्भ्रियते॒ सयो᳚न्ये॒वान्न॒मव॑ रुन्द्धे समू॒ह्यं॑ चिन्वीत प॒शुका॑मः पशु॒माने॒व भ॑वति॥५३॥

%5.4.11.3
प॒रि॒चा॒य्यं॑ चिन्वीत॒ ग्राम॑कामो ग्रा॒म्ये॑व भ॑वति श्मशान॒चितं॑ चिन्वीत॒ यः का॒मये॑त पितृलो॒क ऋ॑ध्नुया॒मिति॑ पितृलो॒क ए॒वर्ध्नो॑ति विश्वामित्रजमद॒ग्नी वसि॑ष्ठेनास्पर्धेता॒ꣳ॒ स ए॒ता ज॒मद॑ग्निर्विह॒व्या॑ अपश्य॒त्ता उपा॑धत्त॒ ताभि॒र्वै स वसि॑ष्ठस्येन्द्रि॒यं वी॒र्य॑मवृङ्क्त॒ यद्वि॑ह॒व्या॑ उप॒दधा॑तीन्द्रि॒यमे॒व ताभि॑र्वी॒र्यं॑ यज॑मानो॒ भ्रातृ॑व्यस्य वृङ्क्ते॒ होतु॒र्धिष्णि॑य॒ उप॑ दधाति यजमानायत॒नं वै॥५४॥

%5.4.11.4
होता॒ स्व ए॒वास्मा॑ आ॒यत॑न इन्द्रि॒यं वी॒र्य॑मव॑ रुन्द्धे॒ द्वाद॒शोप॑ दधाति॒ द्वाद॑शाक्षरा॒ जग॑ती॒ जाग॑ताः प॒शवो॒ जग॑त्यै॒वास्मै॑ प॒शूनव॑ रुन्द्धे॒\-ऽष्टाव॑ष्टाव॒न्येषु॒ धिष्णि॑ये॒षूप॑ दधात्य॒ष्टाश॑फाः प॒शवः॑ प॒शूने॒वाव॑ रुन्द्धे॒ षण्मा᳚र्जा॒लीये॒ षड्वा ऋ॒तव॑ ऋ॒तवः॒ खलु॒ वै दे॒वाः पि॒तर॑ ऋ॒तूने॒व दे॒वान्पि॒तॄन्प्री॑णाति॥५॥

%5.4.12.0
{\anuvakamend[{प्र भ॑वति यजमानायत॒नं वा अ॒ष्टाच॑त्वारिꣳशच्च}]}%॥11॥

%5.4.12.1
पव॑स्व॒ वाज॑सातय॒ इत्य॑नु॒ष्टुक्प्र॑ति॒पद्भ॑वति ति॒स्रो॑\-ऽनु॒ष्टुभ॒श्चत॑स्रो गाय॒त्रियो॒ यत्ति॒स्रो॑\-ऽनु॒ष्टुभ॒स्तस्मा॒दः श्व॑स्त्रि॒भिस्तिष्ठꣴ॑ स्तिष्ठति॒ यच्चत॑स्रो गाय॒त्रिय॒स्तस्मा॒थ्सर्वाꣳ॑श्च॒तुरः॑ प॒दः प्र॑ति॒दध॒त्पला॑यते पर॒मा वा ए॒षा छन्द॑सां॒ यद॑नु॒ष्टुक्प॑र॒मश्च॑तुष्टो॒मः स्तोमा॑नां पर॒मस्त्रि॑रा॒त्रो य॒ज्ञानां᳚ पर॒मो\-ऽश्वः॑ पशू॒नां पर॑मेणै॒वैनं॑ पर॒मतां᳚ गमयत्येकवि॒ꣳ॒शमह॑र्भवति॥५६॥

%5.4.12.2
यस्मि॒न्नश्व॑ आल॒भ्यते॒ द्वाद॑श॒ मासाः॒ पञ्च॒र्तव॒स्त्रय॑ इ॒मे लो॒का अ॒सावा॑दि॒त्य ए॑कवि॒ꣳ॒श ए॒ष प्र॒जाप॑तिः प्राजाप॒त्यो\-ऽश्व॒स्तमे॒व सा॒क्षादृ॑ध्नोति॒ शक्व॑रयः पृ॒ष्ठम्भ॑वन्त्य॒न्यद॑न्य॒च्छन्दो॒\-ऽन्ये᳚न्ये॒ वा ए॒ते प॒शव॒ आ ल॑भ्यन्त उ॒तेव॑ ग्रा॒म्या उ॒तेवा॑र॒ण्या यच्छक्व॑रयः पृ॒ष्ठम्भव॒न्त्यश्व॑स्य सर्व॒त्वाय॑ पार्थुर॒श्मम्ब्र॑ह्मसा॒मम्भ॑वति र॒श्मिना॒ वा अश्वः॑॥५७॥

%5.4.12.3
य॒त ई᳚श्व॒रो वा अश्वो\-ऽय॒तो\-ऽप्र॑तिष्ठितः॒ परां᳚ परा॒वतं॒ गन्तो॒र्त्पा᳚र्थुर॒श्मम्ब्र॑ह्मसा॒मम्भव॒त्यश्व॑स्य॒ यत्यै॒ धृत्यै॒ सङ्कृ॑त्यच्छावाकसा॒मम्भ॑वत्युथ्सन्नय॒ज्ञो वा ए॒ष यद॑श्वमे॒धः कस्तद्वे॒देत्या॑हु॒र्यदि॒ सर्वो॑ वा क्रि॒यते॒ न वा॒ सर्व॒ इति॒ यथ्सङ्कृ॑त्यच्छावाकसा॒मम्भव॒त्यश्व॑स्य सर्व॒त्वाय॒ पर्या᳚प्त्या॒ अन॑न्तरायाय॒ सर्व॑स्तोमो\-ऽतिरा॒त्र उ॑त्त॒ममह॑र्भवति॒ सर्व॒स्याप्त्यै॒ सर्व॑स्य॒ जित्यै॒ सर्व॑मे॒व तेना᳚प्नोति॒ सर्वं॑ जयति॥५८॥

%5.5.0.0
{\anuvakamend[{अह॑र्भवति॒ वा अश्वो\-ऽह॑र्भवति॒ दश॑ च}]}%॥12॥

%5.5.0.0

{\anuvakamend[{यदेके॑न प्र॒जाप॑तिः प्रे॒णानु॒ यजु॒षापो॑ वि॒श्वक॒र्माग्न॒ आ या॑हि सुव॒र्गाय॒ वज्रो॑ गाय॒त्रेणाग्न॑ उदधे स॒मीचीन्द्रा॑य म॒युर॒पां बला॑य पुरुषमृ॒गः सौ॒री पृ॑ष॒तः शका॒ रुरु॑रल॒जः सु॑प॒र्ण आ᳚ग्ने॒यो\-ऽश्वो॒\-ऽग्नये\-ऽनी॑कवते॒ चतु॑र्विꣳशतिः}]}%॥24॥ 
\prashnaend{यदेके॑न॒ स पापी॑याने॒तद्वा अ॒ग्नेर्धनु॒स्तद्दे॒वास्त्वेन्द्र॑ज्येष्ठा अ॒पां नप्रे\-ऽश्व॑स्तूप॒रो द्विष॑ष्टिः॥62॥ यदेके॒नैक॑शितिपा॒त्पेत्वः॑॥}
%%% END PRASHNA

\sect{पञ्चमः प्रश्नः}\setcounter{anuvakam}{0}
\dnsub{तैत्तिरीयसंहितायां पञ्चमकाण्डे पञ्चमः प्रश्नः}
%5.5.1.0
%5.5.1.1
यदेके॑न सꣴस्था॒पय॑ति य॒ज्ञस्य॒ सन्त॑त्या॒ अवि॑च्छेदायै॒न्द्राः प॒शवो॒ ये मु॑ष्क॒रा यदै॒न्द्राः सन्तो॒\-ऽग्निभ्य॑ आल॒भ्यन्ते॑ दे॒वता᳚भ्यः स॒मदं॑ दधात्याग्ने॒यीस्त्रि॒ष्टुभो॑ याज्यानुवा॒क्याः᳚ कुर्या॒द्यदा᳚ग्ने॒यीस्तेना᳚ग्ने॒या यत्त्रि॒ष्टुभ॒स्तेनै॒न्द्राः समृ॑द्ध्यै॒ न दे॒वता᳚भ्यः स॒मदं॑ दधाति वा॒यवे॑ नि॒युत्व॑ते तूप॒रमा ल॑भते॒ तेजो॒\-ऽग्नेर्वा॒युस्तेज॑स ए॒ष आ ल॑भ्यते॒ तस्मा᳚द्य॒द्रिय॑ङ्वा॒युः॥१॥

%5.5.1.2
वाति॑ त॒द्रिय॑ङ्ङ॒ग्निर्द॑हति॒ स्वमे॒व तत्तेजो\-ऽन्वे॑ति॒ यन्न नि॒युत्व॑ते॒ स्यादुन्मा᳚द्ये॒द्यज॑मानो नि॒युत्व॑ते भवति॒ यज॑मान॒स्यानु॑न्मादाय वायु॒मती᳚ श्वे॒तव॑ती याज्यानुवा॒क्ये॑ भवतः सतेज॒स्त्वाय॑ हिरण्यग॒र्भः सम॑वर्त॒ताग्र॒ इत्या॑घा॒रमा घा॑रयति प्र॒जाप॑ति॒र्वै हि॑रण्यग॒र्भः प्र॒जाप॑तेरनुरूप॒त्वाय॒ सर्वा॑णि॒ वा ए॒ष रू॒पाणि॑ पशू॒नाम्प्रत्या ल॑भ्यते॒ यच्छ्म॑श्रु॒णस्तत्॥२॥

%5.5.1.3
पुरु॑षाणाꣳ रू॒पम् यत्तू॑प॒रस्तदश्वा॑नां॒ यद॒न्यतो॑द॒न्तद्गवां॒ यदव्या॑ इव श॒फास्तदवी॑नां॒ यद॒जस्तद॒जाना᳚ं वा॒युर्वै प॑शू॒नाम्प्रि॒यं धाम॒ यद्वा॑य॒व्यो॑ भव॑त्ये॒तमे॒वैन॑म॒भि सं॑जाना॒नाः प॒शव॒ उप॑ तिष्ठन्ते वाय॒व्यः॑ का॒र्या(३)ः प्रा॑जाप॒त्या(३) इत्या॑हु॒र्यद्वा॑य॒व्यं॑ कु॒र्यात्प्र॒जाप॑तेरिया॒द्यत्प्रा॑जाप॒त्यं कु॒र्याद्वा॒योः॥३॥

%5.5.1.4
इ॒या॒द्यद्वा॑य॒व्यः॑ प॒शुर्भव॑ति॒ तेन॑ वा॒योर्नैति॒ यत्प्रा॑जाप॒त्यः पु॑रो॒डाशो॒ भव॑ति॒ तेन॑ प्रा॒जाप॑ते॒र्नैति॒ यद्द्वाद॑शकपाल॒स्तेन॑ वैश्वान॒रान्नैत्या᳚ग्नावैष्ण॒वमेका॑दशकपालं॒ निर्व॑पति दीक्षि॒ष्यमा॑णो॒\-ऽग्निः सर्वा॑ दे॒वता॒ विष्णु॑र्य॒ज्ञो दे॒वता᳚श्चै॒व य॒ज्ञं चा र॑भते॒\-ऽग्निर॑व॒मो दे॒वता॑नां॒ विष्णुः॑ पर॒मो यदा᳚ग्नावैष्ण॒वमेका॑दशकपालं नि॒र्वप॑ति दे॒वताः᳚॥४॥

%5.5.1.5
ए॒वोभ॒यतः॑ परि॒गृह्य॒ यज॑मा॒नो\-ऽव॑ रुन्द्धे पुरो॒डाशे॑न॒ वै दे॒वा अ॒मुष्मि॑ल्लोँ॒क आ᳚र्धुवं च॒रुणा॒स्मिन् यः का॒मये॑ता॒मुष्मि॑ल्लोँ॒क ऋ॑ध्नुया॒मिति॒ स पु॑रो॒डाशं॑ कुर्वीता॒मुष्मि॑न्ने॒व लो॒क ऋ॑ध्नोति॒ यद॒ष्टाक॑पाल॒स्तेना᳚ग्ने॒यो यत्त्रि॑कपा॒लस्तेन॑ वैष्ण॒वः समृ॑द्ध्यै॒ यः का॒मये॑ता॒स्मिल्लोँ॒क ऋ॑ध्नुया॒मिति॒ स च॒रुं कु॑र्वीता॒ग्नेर्घृ॒तं विष्णो᳚स्तण्डु॒लास्तस्मा᳚त्॥५॥

%5.5.1.6
च॒रुः का॒र्यो᳚\-ऽस्मिन्ने॒व लो॒क ऋ॑ध्नोत्यादि॒त्यो भ॑वती॒यं वा अदि॑तिर॒स्यामे॒व प्रति॑ तिष्ठ॒त्यथो॑ अ॒स्यामे॒वाधि॑ य॒ज्ञं त॑नुते॒ यो वै सं॑वथ्स॒रमुख्य॒मभृ॑त्वा॒ग्निं चि॑नु॒ते यथा॑ सा॒मि गर्भो॑\-ऽव॒पद्य॑ते ता॒दृगे॒व तदार्ति॒मार्च्छे᳚द्वैश्वान॒रं द्वाद॑शकपालम् पु॒रस्ता॒न्निर्व॑पेथ्संवथ्स॒रो वा अ॒ग्निर्वै᳚श्वान॒रो यथा॑ संवथ्स॒रमा॒प्त्वा॥६॥

%5.5.1.7
का॒ल आग॑ते वि॒जाय॑त ए॒वमे॒व सं॑वथ्स॒रमा॒प्त्वा का॒ल आग॑ते॒\-ऽग्निं चि॑नुते॒ नार्ति॒मार्च्छ॑त्ये॒षा वा अ॒ग्नेः प्रि॒या त॒नूर्यद्वै᳚श्वान॒रः प्रि॒यामे॒वास्य॑ त॒नुव॒मव॑ रुन्द्धे॒ त्रीण्ये॒तानि॑ ह॒वीꣳषि॑ भवन्ति॒ त्रय॑ इ॒मे लो॒का ए॒षां लो॒काना॒ꣳ॒ रोहा॑य॥७॥

%5.5.2.0
{\anuvakamend[{य॒द्रिय॑ङ्वा॒युर्यच्छ्म॑श्रु॒णस्तद्वा॒योर्नि॒र्वप॑ति दे॒वता॒स्तस्मा॑दा॒प्त्वाष्टात्रिꣳ॑शच्च}]}%॥१॥

%5.5.2.1
प्र॒जाप॑तिः प्र॒जाः सृ॒ष्ट्वा प्रे॒णानु॒ प्रावि॑श॒त्ताभ्यः॒ पुनः॒ सम्भ॑वितुं॒ नाश॑क्नो॒थ्सो᳚\-ऽब्रवीदृ॒ध्नव॒दिथ्स यो मे॒तः पुनः॑ सञ्चि॒नव॒दिति॒ तं दे॒वाः सम॑चिन्व॒न्ततो॒ वै त आ᳚र्ध्नुव॒न् यथ्स॒मचि॑न्व॒न्तच्चित्य॑स्य चित्य॒त्वम् य ए॒वं वि॒द्वान॒ग्निं चि॑नु॒त ऋ॒ध्नोत्ये॒व कस्मै॒ कम॒ग्निश्ची॑यत॒ इत्या॑हुरग्नि॒वान्॥८॥

%5.5.2.2
अ॒सा॒नीति॒ वा अ॒ग्निश्ची॑यते\-ऽग्नि॒वाने॒व भ॑वति॒ कस्मै॒ कम॒ग्निश्ची॑यत॒ इत्या॑हुर्दे॒वा मा॑ वेद॒न्निति॒ वा अ॒ग्निश्ची॑यते वि॒दुरे॑नं दे॒वाः कस्मै॒ कम॒ग्निश्ची॑यत॒ इत्या॑हुर्गृ॒ह्य॑सा॒नीति॒ वा अ॒ग्निश्ची॑यते गृ॒ह्ये॑व भ॑वति॒ कस्मै॒ कम॒ग्निश्ची॑यत॒ इत्या॑हुः पशु॒मान॑सा॒नीति॒ वा अ॒ग्निः॥९॥

%5.5.2.3
ची॒य॒ते॒ प॒शु॒माने॒व भ॑वति॒ कस्मै॒ कम॒ग्निश्ची॑यत॒ इत्या॑हुः स॒प्त मा॒ पुरु॑षा॒ उप॑ जीवा॒निति॒ वा अ॒ग्निश्ची॑यते॒ त्रयः॒ प्राञ्च॒स्त्रयः॑ प्र॒त्यं च॑ आ॒त्मा स॑प्त॒म ए॒ताव॑न्त ए॒वैन॑म॒मुष्मि॑ल्लोँ॒क उप॑ जीवन्ति प्र॒जाप॑तिर॒ग्निम॑चिकीषत॒ तं पृ॑थि॒व्य॑ब्रवी॒न्न मय्य॒ग्निं चे᳚ष्य॒सेति॑ मा धक्ष्यति॒ सा त्वा॑तिद॒ह्यमा॑ना॒ वि ध॑विष्ये॥१०॥

%5.5.2.4
स पापी॑यान्भविष्य॒सीति॒ सो᳚\-ऽब्रवी॒त्तथा॒ वा अ॒हं क॑रिष्यामि॒ यथा᳚ त्वा॒ नाति॑ध॒क्ष्यतीति॒ स इ॒माम॒भ्य॑मृशत् प्र॒जाप॑तिस्त्वा सादयतु॒ तया॑ दे॒वत॑याङ्गिर॒स्वद्ध्रु॒वा सी॒देती॒मामे॒वेष्ट॑कां कृ॒त्वोपा॑ध॒त्तान॑तिदाहाय॒ यत्प्रत्य॒ग्निं चि॑न्वी॒त तद॒भि मृ॑शेत्प्र॒जाप॑तिस्त्वा सादयतु॒ तया॑ दे॒वत॑याङ्गिर॒स्वद्ध्रु॒वा सी॑द॥११॥

%5.5.2.5
इती॒मामे॒वेष्ट॑कां कृ॒त्वोप॑ ध॒त्ते\-ऽन॑तिदाहाय प्र॒जाप॑तिरकामयत॒ प्र जा॑ये॒येति॒ स ए॒तमुख्य॑मपश्य॒त्तꣳ सं॑वथ्स॒रम॑बिभ॒स्ततो॒ वै स प्राजा॑यत॒ तस्मा᳚थ्संवथ्स॒रम्भा॒र्यः॑ प्रैव जा॑यते॒ तं वस॑वो\-ऽब्रुव॒न्प्र त्वम॑जनिष्ठा व॒यं प्र जा॑यामहा॒ इति॒ तं वसु॑भ्यः॒ प्राय॑च्छ॒त्तं त्रीण्यहा᳚न्यबिभरु॒स्तेन॑॥१२॥

%5.5.2.6
त्रीणि॑ च श॒तान्यसृ॑जन्त॒ त्रय॑स्त्रिꣳशतं च॒ तस्मा᳚त्त्र्य॒हम्भा॒र्यः॑ प्रैव जा॑यते॒ तान्रु॒द्रा अ॑ब्रुव॒न्प्र यू॒यम॑जनिढ्वं व॒यं प्र जा॑यामहा॒ इति॒ तꣳ रु॒द्रेभ्यः॒ प्राय॑च्छ॒न्तꣳ षडहा᳚न्यबिभरु॒स्तेन॒ त्रीणि॑ च श॒तान्यसृ॑जन्त॒ त्रय॑स्त्रिꣳशतं च॒ तस्मा᳚त्षड॒हम्भा॒र्यः॑ प्रैव जा॑यते॒ ताना॑दि॒त्या अ॑ब्रुव॒न्प्र यू॒यम॑जनिढ्वं व॒यं ॥१३॥

%5.5.2.7
प्र जा॑यामहा॒ इति॒ तमा॑दि॒त्येभ्यः॒ प्राय॑च्छ॒न्तं द्वाद॒शाहा᳚न्यबिभरु॒स्तेन॒ त्रीणि॑ च श॒तान्यसृ॑जन्त॒ त्रय॑स्त्रिꣳशतं च॒ तस्मा᳚द्द्वादशा॒हम्भा॒र्यः॑ प्रैव जा॑यते॒ तेनै॒व ते स॒हस्र॑मसृजन्तो॒खाꣳ स॑हस्रत॒मीं य ए॒वमुख्यꣳ॑ साह॒स्रं वेद॒ प्र स॒हस्रं॑ प॒शूना᳚प्नोति॥१४॥

%5.5.3.0
{\anuvakamend[{अ॒ग्नि॒वान्प॑शु॒मान॑सा॒नीति॒ वा अ॒ग्निर्ध॑विष्ये मृशेत्प्र॒जाप॑तिस्त्वा सादयतु॒ तया॑ दे॒वत॑याङ्गिर॒स्वद्ध्रु॒वा सी॑द॒ तेन॒ ताना॑दि॒त्या अ॑ब्रुव॒न्प्र यू॒यम॑जनिढ्वं व॒यञ्च॑त्वारि॒ꣳ॒शच्च॑}]}%॥२॥

%5.5.3.1
यजु॑षा॒ वा ए॒षा क्रि॑यते॒ यजु॑षा पच्यते॒ यजु॑षा॒ वि मु॑च्यते॒ यदु॒खा सा वा ए॒षैतर्\mbox{}हि॑ या॒तया᳚म्नी॒ सा न पुनः॑ प्र॒युज्येत्या॑हु॒रग्ने॑ यु॒क्ष्वा हि ये तव॑ यु॒क्ष्वा हि दे॑व॒हूत॑मा॒ꣳ॒ इत्यु॒खायां᳚ जुहोति॒ तेनै॒वैना॒म्पुनः॒ प्र यु॑ङ्क्ते॒ तेनाया॑तयाम्नी॒ यो वा अ॒ग्निं योग॒ आग॑ते यु॒नक्ति॑ यु॒ङ्क्ते यु॑ञ्जा॒नेष्वग्ने᳚॥१५॥

%5.5.3.2
यु॒क्ष्वा हि ये तव॑ यु॒क्ष्वा हि दे॑व॒हूत॑मा॒ꣳ॒ इत्या॑है॒ष वा अ॒ग्नेर्योग॒स्तेनै॒वैनं॑ युनक्ति यु॒ङ्क्ते यु॑ञ्जा॒नेषु॑ ब्रह्मवा॒दिनो॑ वदन्ति न्य॑ङ्ङ॒ग्निश्चे॑त॒व्या(३) उ॑त्ता॒ना(३) इति॒ वय॑सां॒ वा ए॒ष प्र॑ति॒मया॑ चीयते॒ यद॒ग्निर्यन्न्य॑ञ्चं चिनु॒यात्पृ॑ष्टि॒त ए॑न॒माहु॑तय ऋच्छेयु॒र्यदु॑त्ता॒नं न पति॑तुꣳ शक्नुया॒दसु॑वर्ग्यो\-ऽस्य स्यात्प्रा॒चीन॑मुत्ता॒नम्॥१६॥

%5.5.3.3
पु॒रु॒ष॒शी॒र्\mbox{}षमुप॑ दधाति मुख॒त ए॒वैन॒माहु॑तय ऋच्छन्ति॒ नोत्ता॒नं चि॑नुते सुव॒र्ग्यो᳚\-ऽस्य भवति सौ॒र्या जु॑होति॒ चक्षु॑रे॒वास्मि॒न्प्रति॑ दधाति॒ द्विर्जु॑होति॒ द्वे हि चक्षु॑षी समा॒न्या जु॑होति समा॒नꣳ हि चक्षुः॒ समृ॑द्ध्यै देवासु॒राः संय॑त्ता आस॒न्ते वा॒मं वसु॒ सं न्य॑दधत॒ तद्दे॒वा वा॑म॒भृता॑वृञ्जत॒ तद्वा॑म॒भृतो॑ वामभृ॒त्त्वं यद्वा॑म॒भृत॑मुप॒दधा॑ति वा॒ममे॒व तया॒ वसु॒ यज॑मानो॒ भ्रातृ॑व्यस्य वृङ्क्ते॒ हिर॑ण्यमूर्ध्नी भवति॒ ज्योति॒र्वै हिर॑ण्यं॒ ज्योति॑र्वा॒मं ज्योति॑षै॒वास्य॒ ज्योति॑र्वा॒मं वृ॑ङ्क्ते द्विय॒जुर्भ॑वति॒ प्रति॑ष्ठित्यै॥१७॥

%5.5.4.0
{\anuvakamend[{यु॒ञ्जा॒नेष्वग्ने᳚ प्रा॒चीन॑मुत्ता॒नं वा॑म॒भृत॒ञ्चतु॑र्विꣳशतिश्च}]}%॥३॥

%5.5.4.1
आपो॒ वरु॑णस्य॒ पत्न॑य आस॒न्ता अ॒ग्निर॒भ्य॑ध्याय॒त्ताः सम॑भव॒त्तस्य॒ रेतः॒ परा॑पत॒त्तदि॒यम॑भव॒द्यद्द्वि॒तीय॑म्प॒राप॑त॒त्तद॒सा\-व॑भवदि॒यं वै वि॒राड॒सौ स्व॒राड्यद्वि॒राजा॑वुप॒दधा॑ती॒मे ए॒वोप॑ धत्ते॒ यद्वा अ॒सौ रेतः॑ सि॒ञ्चति॒ तद॒स्यां प्रति॑ तिष्ठति॒ तत्प्र जा॑यते॒ ता ओष॑धयः॥१८॥

%5.5.4.2
वी॒रुधो॑ भवन्ति॒ ता अ॒ग्निर॑त्ति॒ य ए॒वं वेद॒ प्रैव जा॑यते\-ऽन्ना॒दो भ॑वति॒ यो रे॑त॒स्वी स्यात्प्र॑थ॒मायां॒ तस्य॒ चित्या॑मु॒भे उप॑ दध्यादि॒मे ए॒वास्मै॑ स॒मीची॒ रेतः॑ सिञ्चतो॒ यः सि॒क्तरे॑ताः॒ स्यात्प्र॑थ॒मायां॒ तस्य॒ चित्या॑म॒न्यामुप॑ दध्यादुत्त॒माया॑\-म॒न्याꣳ रेत॑ ए॒वास्य॑ सि॒क्तमा॒भ्यामु॑भ॒यतः॒ परि॑ गृह्णाति संवथ्स॒रं न कम्॥१९॥

%5.5.4.3
च॒न प्र॒त्यव॑रोहे॒न्न हीमे कं च॒न प्र॑त्यव॒रोह॑त॒स्तदे॑नयोर्व्र॒तं यो वा अप॑शीर्\mbox{}षाणम॒ग्निं चि॑नु॒ते\-ऽप॑शीर्\mbox{}षा॒मुष्मि॑ल्लोँ॒के भ॑वति॒ यः सशी॑र्\mbox{}षाणं चिनु॒ते सशी॑र्\mbox{}षा॒मुष्मि॑ल्लोँ॒के भ॑वति॒ चित्तिं॑ जुहोमि॒ मन॑सा घृ॒तेन॒ यथा॑ दे॒वा इ॒हागम॑न्वी॒तिहो᳚त्रा ऋता॒वृधः॑ समु॒द्रस्य॑ व॒युन॑स्य॒ पत्म॑ञ्जु॒होमि॑ वि॒श्वक॑र्मणे॒ विश्वाहाम॑र्त्यꣳ ह॒विरिति॑ स्वयमातृ॒ण्णामु॑प॒धाय॑ जुहोति॥२०॥

%5.5.4.4
ए॒तद्वा अ॒ग्नेः शिरः॒ सशी॑र्\mbox{}षाणमे॒वाग्निं चि॑नुते॒ सशी॑र्\mbox{}षा॒मुष्मि॑ल्लोँ॒के भ॑वति॒ य ए॒वं वेद॑ सुव॒र्गाय॒ वा ए॒ष लो॒काय॑ चीयते॒ यद॒ग्निस्तस्य॒ यदय॑थापूर्वं क्रि॒यते\-ऽसु॑वर्ग्यमस्य॒ तथ्सु॑व॒र्ग्यो᳚\-ऽग्निश्चिति॑मुप॒धाया॒भि मृ॑शे॒च्चित्ति॒मचि॑त्तिं चिनव॒द्वि वि॒द्वान्पृ॒ष्ठेव॑ वी॒ता वृ॑जि॒ना च॒ मर्ता᳚न्रा॒ये च॑ नः स्वप॒त्याय॑ देव॒ दितिं॑ च॒ रास्वादि॑तिमुरु॒ष्येति॑ यथापू॒र्वमे॒वैना॒मुप॑ धत्ते॒ प्राञ्च॑मेनं चिनुते सुव॒र्ग्यो᳚\-ऽस्य भवति॥२१॥

%5.5.5.0
{\anuvakamend[{ओष॑धयः॒ कञ्जु॑होति स्वप॒त्याया॒ष्टाद॑श च}]}%॥४॥

%5.5.5.1
वि॒श्वक॑र्मा दि॒शाम्पतिः॒ स नः॑ प॒शून्पा॑तु॒ सो᳚\-ऽस्मान्पा॑तु॒ तस्मै॒ नमः॑ प्र॒जाप॑ती रु॒द्रो वरु॑णो॒\-ऽग्निर्दि॒शाम्पतिः॒ स नः॑ प॒शून्पा॑तु॒ सो᳚\-ऽस्मान्पा॑तु॒ तस्मै॒ नम॑ ए॒ता वै दे॒वता॑ ए॒तेषां᳚ पशू॒नामधि॑पतय॒स्ताभ्यो॒ वा ए॒ष आ वृ॑श्च्यते॒ यः प॑शुशी॒र्\mbox{}षाण्यु॑प॒दधा॑ति हिरण्येष्ट॒का उप॑ दधात्ये॒ताभ्य॑ ए॒व दे॒वता᳚भ्यो॒ नम॑स्करोति ब्रह्मवा॒दिनः॑॥२२॥

%5.5.5.2
व॒द॒न्त्य॒ग्नौ ग्रा॒म्यान्प॒शून्प्र द॑धाति शु॒चार॒ण्यान॑र्पयति॒ किं तत॒ उच्छिꣳ॑ष॒तीति॒ यद्धि॑रण्येष्ट॒का उ॑प॒दधा᳚त्य॒मृतं॒ वै हिर॑ण्यम॒मृते॑नै॒व ग्रा॒म्येभ्यः॑ प॒शुभ्यो॑ भेष॒जं क॑रोति॒ नैनान्॑ हिनस्ति प्रा॒णो वै प्र॑थ॒मा स्व॑यमातृ॒ण्णा व्या॒नो द्वि॒तीया॑पा॒नस्तृ॒तीयानु॒ प्राण्या᳚त्प्रथ॒माꣴ स्व॑यमातृ॒ण्णामु॑प॒धाय॑ प्रा॒णेनै॒व प्रा॒णꣳ सम॑र्धयति॒ व्य॑न्यात्॥२३॥

%5.5.5.3
द्वि॒तीया॑मुप॒धाय॑ व्या॒नेनै॒व व्या॒नꣳ सम॑र्धय॒त्यपा᳚न्यात्तृ॒तीया॑मुप॒धाया॑पा॒नेनै॒वापा॒नꣳ सम॑र्धय॒त्यथो᳚ प्रा॒णैरे॒वैन॒ꣳ॒ समि॑न्द्धे॒ भूर्भुवः॒ सुव॒रिति॑ स्वयमातृ॒ण्णा उप॑ दधाती॒मे वै लो॒काः स्व॑यमातृ॒ण्णा ए॒ताभिः॒ खलु॒ वै व्याहृ॑तीभिः प्र॒जाप॑तिः॒ प्राजा॑यत॒ यदे॒ताभि॒र्व्याहृ॑तीभिः स्वयमातृ॒ण्णा उ॑प॒दधा॑ती॒माने॒व लो॒कानु॑प॒धायै॒षु॥२४॥

%5.5.5.4
लो॒केष्वधि॒ प्र जा॑यते प्रा॒णाय॑ व्या॒नाया॑पा॒नाय॑ वा॒चे त्वा॒ चक्षु॑षे त्वा॒ तया॑ दे॒वत॑याङ्गिर॒स्वद्ध्रु॒वा सी॑दा॒ग्निना॒ वै दे॒वाः सु॑व॒र्गं लो॒कम॑जिगाꣳस॒न्तेन॒ पति॑तुं॒ नाश॑क्नुव॒न्त ए॒ताश्चत॑स्रः स्वयमातृ॒ण्णा अ॑पश्य॒न्ता दि॒क्षूपा॑दधत॒ तेन॑ स॒र्वत॑श्चक्षुषा सुव॒र्गं लो॒कमा॑य॒न्॒यच्चत॑स्रः स्वयमातृ॒ण्णा दि॒क्षू॑प॒दधा॑ति स॒र्वत॑श्चक्षुषै॒व तद॒ग्निना॒ यज॑मानः सुव॒र्गं लो॒कमे॑ति॥२५॥

%5.5.6.0
{\anuvakamend[{ब्र॒ह्म॒वा॒दिनो॒ व्य॑न्यादे॒षु यज॑मान॒स्त्रीणि॑ च}]}%॥५॥

%5.5.6.1
अग्न॒ आ या॑हि वी॒तय॒ इत्या॒हाह्व॑तै॒वैन॑म॒ग्निं दू॒तं वृ॑णीमह॒ इत्या॑ह हू॒त्वैवैनं॑ वृणीते॒\-ऽग्निना॒ग्निः समि॑ध्यत॒ इत्या॑ह॒ समि॑न्द्ध ए॒वैन॑म॒ग्निर्वृ॒त्राणि॑ जङ्घन॒दित्या॑ह॒ समि॑द्ध ए॒वास्मि॑न्निन्द्रि॒यं द॑धात्य॒ग्नेः स्तोम॑म्मनामह॒ इत्या॑ह मनु॒त ए॒वैन॑मे॒तानि॒ वा अह्नाꣳ॑ रू॒पाणि॑॥२६॥

%5.5.6.2
अ॒न्व॒हमे॒वैनं॑ चिनु॒ते\-ऽवाह्नाꣳ॑ रू॒पाणि॑ रुन्द्धे ब्रह्मवा॒दिनो॑ वदन्ति॒ कस्मा᳚थ्स॒त्याद्या॒तया᳚म्नीर॒न्या इष्ट॑का॒ अया॑तयाम्नी लोकं पृ॒णेत्यै᳚न्द्रा॒ग्नी हि बा॑र्\mbox{}हस्प॒त्येति॑ ब्रूयादिन्द्रा॒ग्नी च॒ हि दे॒वाना॒म्बृह॒स्पति॒श्चाया॑तयामानो\-ऽनुच॒रव॑ती भव॒त्यजा॑मित्वायानु॒ष्टुभानु॑ चरत्या॒त्मा वै लो॑कं पृ॒णा प्रा॒णो॑\-ऽनु॒ष्टुप्तस्मा᳚त्प्रा॒णः सर्वा॒ण्यङ्गा॒न्यनु॑ चरति॒ ता अ॑स्य॒ सूद॑दोहसः॥२७॥

%5.5.6.3
इत्या॑ह॒ तस्मा॒त्परु॑षिपरुषि॒ रसः॒ सोमꣴ॑ श्रीणन्ति॒ पृश्ञ॑य॒ इत्या॒हान्नं॒ वै पृश्न्यन्न॑मे॒वाव॑ रुन्द्धे॒\-ऽर्को वा अ॒ग्निर॒र्को\-ऽन्न॒मन्न॑मे॒वाव॑ रुन्द्धे॒ जन्मं॑ दे॒वानां॒ विश॑स्त्रि॒ष्वा रो॑च॒ने दि॒व इत्या॑हे॒माने॒वास्मै॑ लो॒कां ज्योति॑ष्मतः करोति॒ यो वा इष्ट॑कानां प्रति॒ष्ठां वेद॒ प्रत्ये॒व ति॑ष्ठति॒ तया॑ दे॒वत॑याङ्गिर॒स्वद्ध्रु॒वा सी॒देत्या॑है॒षा वा इष्ट॑कानां प्रति॒ष्ठा य ए॒वं वेद॒ प्रत्ये॒व ति॑ष्ठति॥२८॥

%5.5.7.0
{\anuvakamend[{रू॒पाणि॒ सूद॑दोहस॒स्तया॒ षोड॑श च}]}%॥६॥

%5.5.7.1
सु॒व॒र्गाय॒ वा ए॒ष लो॒काय॑ चीयते॒ यद॒ग्निर्वज्र॑ एकाद॒शिनी॒ यद॒ग्नावे॑काद॒शिनी᳚म्मिनु॒याद्वज्रे॑णैनꣳ सुव॒र्गाल्लो॒का\-द॒न्तर्द॑ध्या॒द्यन्न मि॑नु॒याथ्स्वरु॑भिः प॒शून्व्य॑र्धयेदेकयू॒पम्मि॑नोति॒ नैनं॒ वज्रे॑ण सुव॒र्गाल्लो॒काद॑न्त॒र्दधा॑ति॒ न स्वरु॑भिः प॒शून्व्य॑र्धयति॒ वि वा ए॒ष इ॑न्द्रि॒येण॑ वी॒र्ये॑णर्ध्यते॒ यो᳚\-ऽग्निं चि॒न्वन्न॑धि॒क्राम॑त्यैन्द्रि॒या॥२९॥

%5.5.7.2
ऋ॒चाक्रम॑ण॒म्प्रतीष्ट॑का॒मुप॑ दध्या॒न्नेन्द्रि॒येण॑ वी॒र्ये॑ण॒ व्यृ॑ध्यते रु॒द्रो वा ए॒ष यद॒ग्निस्तस्य॑ ति॒स्रः श॑र॒व्याः᳚ प्र॒तीची॑ ति॒रश्च्य॒नूची॒ ताभ्यो॒ वा ए॒ष आ वृ॑श्च्यते॒ यो᳚\-ऽग्निं चि॑नु॒ते᳚\-ऽग्निं चि॒त्वा ति॑सृध॒न्वमया॑चितम्ब्राह्म॒णाय॑ दद्या॒त्ताभ्य॑ ए॒व नम॑स्करो॒त्यथो॒ ताभ्य॑ ए॒वात्मानं॒ निष्क्री॑णीते॒ यत्ते॑ रुद्र पु॒रः॥३०॥

%5.5.7.3
धनु॒स्तद्वातो॒ अनु॑ वातु ते॒ तस्मै॑ ते रुद्र संवथ्स॒रेण॒ नम॑स्करोमि॒ यत्ते॑ रुद्र दक्षि॒णा धनु॒स्तद्वातो॒ अनु॑ वातु ते॒ तस्मै॑ ते रुद्र परिवथ्स॒रेण॒ नम॑स्करोमि॒ यत्ते॑ रुद्र प॒श्चाद्धनु॒स्तद्वातो॒ अनु॑ वातु ते॒ तस्मै॑ ते रुद्रेदावथ्स॒रेण॒ नम॑स्करोमि॒ यत्ते॑ रुद्रोत्त॒राद्धनु॒स्तत्॥३१॥

%5.5.7.4
वातो॒ अनु॑ वातु ते॒ तस्मै॑ ते रुद्रेदुवथ्स॒रेण॒ नम॑स्करोमि॒ यत्ते॑ रुद्रो॒परि॒ धनु॒स्तद्वातो॒ अनु॑ वातु ते॒ तस्मै॑ ते रुद्र वथ्स॒रेण॒ नम॑स्करोमि रु॒द्रो वा ए॒ष यद॒ग्निः स यथा᳚ व्या॒घ्रः क्रु॒द्धस्तिष्ठ॑त्ये॒वं वा ए॒ष ए॒तर्\mbox{}हि॒ सञ्चि॑तमे॒तैरुप॑ तिष्ठते नमस्का॒रैरे॒वैनꣳ॑ शमयति॒ ये᳚\-ऽग्नयः॑॥३२॥

%5.5.7.5
पु॒री॒ष्याः᳚ प्रवि॑ष्टाः पृथि॒वीमनु॑। तेषां॒ त्वम॑स्युत्त॒मः प्र णो॑ जी॒वात॑वे सुव। आपं॑ त्वाऽग्ने॒ मन॒सापं॑ त्वाऽग्ने॒ तप॒सापं॑ त्वाग्ने दी॒क्षयापं॑ त्वाग्न उप॒सद्भि॒रापं॑ त्वाग्ने सु॒त्ययापं॑ त्वाऽग्ने॒ दक्षि॑णाभि॒रापं॑ त्वाग्ने\-ऽवभृ॒थेनापं॑ त्वाग्ने व॒शयापं॑ त्वाग्ने स्वगाका॒रेणेत्या॑है॒षा वा अ॒ग्नेराप्ति॒स्तयै॒वैन॑माप्नोति॥३३॥

%5.5.8.0
{\anuvakamend[{ऐ॒न्द्रि॒या पु॒र उ॑त्त॒राद्धनु॒स्तद॒ग्नय॑ आहा॒ष्टौ च॑}]}%॥७॥

%5.5.8.1
गा॒य॒त्रेण॑ पु॒रस्ता॒दुप॑ तिष्ठते प्रा॒णमे॒वास्मि॑न्दधाति बृहद्रथन्त॒रा\-भ्यां᳚ प॒क्षावोज॑ ए॒वास्मि॑न्दधात्यृतु॒स्थाय॑ज्ञाय॒ज्ञिये॑न॒ पुच्छ॑मृ॒तुष्वे॒व प्रति॑ तिष्ठति पृ॒ष्ठैरुप॑ तिष्ठते॒ तेजो॒ वै पृ॒ष्ठानि॒ तेज॑ ए॒वास्मि॑न्दधाति प्र॒जाप॑तिर॒ग्निम॑सृजत॒ सो᳚\-ऽस्माथ्सृ॒ष्टः परा॑ङै॒त्तं वा॑रव॒न्तीये॑नावारयत॒ तद्वा॑रव॒न्तीय॑स्य वारवन्तीय॒त्वꣴ श्यै॒तेन॑ श्ये॒ती अ॑कुरुत॒ तच्छ्यै॒तस्य॑ श्यैत॒त्वम्॥३४॥

%5.5.8.2
यद्वा॑रव॒न्तीये॑नोप॒तिष्ठ॑ते वा॒रय॑त ए॒वैनꣴ॑ श्यै॒तेन॑ श्ये॒ती कु॑रुते प्र॒जाप॑ते॒र्\mbox{}हृद॑येनापिप॒क्षम्प्रत्युप॑ तिष्ठते प्रे॒माण॑मे॒वास्य॑ गच्छति॒ प्राच्या᳚ त्वा दि॒शा सा॑दयामि गाय॒त्रेण॒ छन्द॑सा॒ग्निना॑ दे॒वत॑या॒ग्नेः शी॒र्ष्णाग्नेः शिर॒ उप॑ दधामि॒ दक्षि॑णया त्वा दि॒शा सा॑दयामि॒ त्रैष्टु॑भेन॒ छन्द॒सेन्द्रे॑ण दे॒वत॑या॒ग्नेः प॒क्षेणा॒ग्नेः प॒क्षमुप॑ दधामि प्र॒तीच्या᳚ त्वा दि॒शा सा॑दयामि॥३५॥

%5.5.8.3
जाग॑तेन॒ छन्द॑सा सवि॒त्रा दे॒वत॑या॒ग्नेः पुच्छे॑ना॒ग्नेः पुच्छ॒मुप॑ दधा॒म्युदी᳚च्या त्वा दि॒शा सा॑दया॒म्यानु॑ष्टुभेन॒ छन्द॑सा मि॒त्रावरु॑णाभ्यां दे॒वत॑या॒ग्नेः प॒क्षेणा॒ग्नेः प॒क्षमुप॑ दधाम्यू॒र्ध्वया᳚ त्वा दि॒शा सा॑दयामि॒ पाङ्क्ते॑न॒ छन्द॑सा॒ बृह॒स्पति॑ना दे॒वत॑या॒ग्नेः पृ॒ष्ठेना॒ग्नेः पृ॒ष्ठमुप॑ दधामि॒ यो वा अपा᳚त्मानम॒ग्निं चि॑नु॒ते\-ऽपा᳚त्मा॒मुष्मि॑ल्लोँ॒के भ॑वति॒ यः सात्मा॑नं चिनु॒ते सात्मा॒मुष्मि॑ल्लोँ॒के भ॑वत्यात्मेष्ट॒का उप॑ दधात्ये॒ष वा अ॒ग्नेरा॒त्मा सात्मा॑नमे॒वाग्निं चि॑नुते॒ सात्मा॒मुष्मि॑ल्लोँ॒के भ॑वति॒ य ए॒वं वेद॑॥३६॥

%5.5.9.0
{\anuvakamend[{श्यै॒त॒त्वं प्र॒तीच्या᳚ त्वा दि॒शा सा॑दयामि॒ यः सात्मा॑नञ्चिनु॒ते द्वाविꣳ॑शतिश्च}]}%॥८॥

%5.5.9.1
अग्न॑ उदधे॒ या त॒ इषु॑र्यु॒वा नाम॒ तया॑ नो मृड॒ तस्या᳚स्ते॒ नम॒स्तस्या᳚स्त॒ उप॒ जीव॑न्तो भूया॒स्माग्ने॑ दुध्र गह्य किꣳशिल वन्य॒ या त॒ इषु॑र्यु॒वा नाम॒ तया॑ नो मृड॒ तस्या᳚स्ते॒ नम॒स्तस्या᳚स्त॒ उप॒ जीव॑न्तो भूयास्म॒ पञ्च॒ वा ए॒ते᳚\-ऽग्नयो॒ यच्चित॑य उद॒धिरे॒व नाम॑ प्रथ॒मो दु॒ध्रः॥३७॥

%5.5.9.2
द्वि॒तीयो॒ गह्य॑स्तृ॒तीयः॑ किꣳशि॒लश्च॑तु॒र्थो वन्यः॑ पञ्च॒मस्तेभ्यो॒ यदाहु॑ती॒र्न जु॑हु॒याद॑ध्व॒र्युं च॒ यज॑मानं च॒ प्र द॑हेयु॒र्यदे॒ता आहु॑तीर्जु॒होति॑ भाग॒धेये॑नै॒वैना᳚ञ्छमयति॒ नार्ति॒मार्च्छ॑त्यध्व॒र्युर्न यज॑मानो॒ वाङ्म॑ आ॒सन्न॒सोः प्रा॒णो᳚\-ऽक्ष्योश्चक्षुः॒ कर्ण॑योः॒ श्रोत्र॑म्बाहु॒वोर्बल॑मूरु॒वोरोजो\-ऽरि॑ष्टा॒ विश्वा॒न्यङ्गा॑नि त॒नूः॥३८॥

%5.5.9.3
त॒नुवा॑ मे स॒ह नम॑स्ते अस्तु॒ मा मा॑ हिꣳसी॒रप॒ वा ए॒तस्मा᳚त्प्रा॒णाः क्रा॑मन्ति॒ यो᳚\-ऽग्निं चि॒न्वन्न॑धि॒क्राम॑ति॒ वाङ्म॑ आ॒सन्न॒सोः प्रा॒ण इत्या॑ह प्रा॒णाने॒वात्मन्ध॑त्ते॒ यो रु॒द्रो अ॒ग्नौ यो अ॒फ्सु य ओष॑धीषु॒ यो रु॒द्रो विश्वा॒ भुव॑नावि॒वेश॒ तस्मै॑ रु॒द्राय॒ नमो॑ अ॒स्त्वाहु॑तिभागा॒ वा अ॒न्ये रु॒द्रा ह॒विर्भा॑गाः॥३९॥

%5.5.9.4
अ॒न्ये श॑तरु॒द्रीयꣳ॑ हु॒त्वा गा॑वीधु॒कं च॒रुमे॒तेन॒ यजु॑षा चर॒माया॒मिष्ट॑कायां॒ नि द॑ध्याद्भाग॒धेये॑नै॒वैनꣳ॑ शमयति॒ तस्य॒ त्वै श॑तरु॒द्रीयꣳ॑ हु॒तमित्या॑हु॒र्यस्यै॒तद॒ग्नौ क्रि॒यत॒ इति॒ वस॑वस्त्वा रु॒द्रैः पु॒रस्ता᳚त्पान्तु पि॒तर॑स्त्वा य॒मरा॑जानः पि॒तृभि॑र्दक्षिण॒तः पा᳚न्त्वादि॒त्यास्त्वा॒ विश्वै᳚र्दे॒वैः प॒श्चात्पा᳚न्तु द्युता॒नस्त्वा॑ मारु॒तो म॒रुद्भि॑रुत्तर॒तः पा॑तु॥४०॥

%5.5.9.5
दे॒वास्त्वेन्द्र॑ज्येष्ठा॒ वरु॑णराजानो॒\-ऽधस्ता᳚च्चो॒परि॑ष्टाच्च पान्तु॒ न वा ए॒तेन॑ पू॒तो न मेध्यो॒ न प्रोक्षि॑तो॒ यदे॑न॒मतः॑ प्रा॒चीनं॑ प्रो॒क्षति॒ यथ्सञ्चि॑त॒माज्ये॑न प्रो॒क्षति॒ तेन॑ पू॒तस्तेन॒ मेध्य॒स्तेन॒ प्रोक्षि॑तः॥४१॥

%5.5.10.0
{\anuvakamend[{दु॒ध्रस्त॒नूर्\mbox{}ह॒विर्भा॑गाः पातु॒ द्वात्रिꣳ॑शच्च}]}%॥९॥

%5.5.10.1
स॒मीची॒ नामा॑सि॒ प्राची॒ दिक्तस्या᳚स्ते॒\-ऽग्निरधि॑पतिरसि॒तो र॑क्षि॒ता यश्चाधि॑पति॒र्यश्च॑ गो॒प्ता ताभ्यां॒ नम॒स्तौ नो॑ मृडयता॒न्ते यं द्वि॒ष्मो यश्च॑ नो॒ द्वेष्टि॒ तं वां॒ जम्भे॑ दधाम्योज॒स्विनी॒ नामा॑सि दक्षि॒णा दिक्तस्या᳚स्त॒ इन्द्रो\-ऽधि॑पतिः॒ पृदा॑कुः॒ प्राची॒ नामा॑सि प्र॒तीची॒ दिक्तस्या᳚स्ते॥४२॥

%5.5.10.2
सोमो\-ऽधि॑पतिः स्व॒जो॑\-ऽव॒स्थावा॒ नामा॒स्युदी॑ची॒ दिक्तस्या᳚स्ते॒ वरु॒णो\-ऽधि॑पतिस्ति॒रश्च॑राजि॒रधि॑पत्नी॒ नामा॑सि बृह॒ती दिक्तस्या᳚स्ते॒ बृह॒स्पति॒रधि॑पतिः श्वि॒त्रो व॒शिनी॒ नामा॑सी॒यं दिक्तस्या᳚स्ते य॒मो\-ऽधि॑पतिः क॒ल्माष॑ग्रीवो रक्षि॒ता यश्चाधि॑पति॒र्यश्च॑ गो॒प्ता ताभ्यां॒ नम॒स्तौ नो॑ मृडयता॒न्ते यं द्वि॒ष्मो यश्च॑॥४३॥

%5.5.10.3
नो॒ द्वेष्टि॒ तं वां॒ जम्भे॑ दधाम्ये॒ता वै दे॒वता॑ अ॒ग्निं चि॒तꣳ र॑क्षन्ति॒ ताभ्यो॒ यदाहु॑ती॒र्न जु॑हु॒याद॑ध्व॒र्युं च॒ यज॑मानं च ध्यायेयु॒र्यदे॒ता आहु॑तीर्जु॒होति॑ भाग॒धेये॑नै॒वैना᳚ञ्छमयति॒ नार्ति॒मार्च्छ॑त्यध्व॒र्युर्न यज॑मानो हे॒तयो॒ नाम॑ स्थ॒ तेषां᳚ वः पु॒रो गृ॒हा अ॒ग्निर्व॒ इष॑वः सलि॒लो निलि॒म्पा नाम॑॥४४॥

%5.5.10.4
स्थ॒ तेषां᳚ वो दक्षि॒णा गृ॒हाः पि॒तरो॑ व॒ इष॑वः॒ सग॑रो व॒ज्रिणो॒ नाम॑ स्थ॒ तेषां᳚ वः प॒श्चाद्गृ॒हाः स्वप्नो॑ व॒ इष॑वो॒ गह्व॑रो\-ऽव॒स्थावा॑नो॒ नाम॑ स्थ॒ तेषां᳚ व उत्त॒राद्गृ॒हा आपो॑ व॒ इष॑वः समु॒द्रो\-ऽधि॑पतयो॒ नाम॑ स्थ॒ तेषां᳚ व उ॒परि॑ गृ॒हा व॒र्\mbox{}षं व॒ इष॒वो\-ऽव॑स्वान्क्र॒व्या नाम॑ स्थ॒ पार्थि॑वा॒स्तेषां᳚ व इ॒ह गृ॒हाः॥४५॥

%5.5.10.5
अन्नं॑ व॒ इष॑वो निमि॒षो वा॑तना॒मन्तेभ्यो॑ वो॒ नम॒स्ते नो॑ मृडयत॒ ते यं द्वि॒ष्मो यश्च॑ नो॒ द्वेष्टि॒ तं वो॒ जम्भे॑ दधामि हु॒तादो॒ वा अ॒न्ये दे॒वा अ॑हु॒तादो॒\-ऽन्ये तान॑ग्नि॒चिदे॒वोभया᳚न्प्रीणाति द॒ध्ना म॑धुमि॒श्रेणै॒ता आहु॑तीर्जुहोति भाग॒धेये॑नै॒वैना᳚न्प्रीणा॒त्यथो॒ खल्वा॑हु॒रिष्ट॑का॒ वै दे॒वा अ॑हु॒ताद॒ इति॑॥४६॥

%5.5.10.6
अ॒नु॒प॒रि॒क्रामं॑ जुहो॒त्यप॑रिवर्गमे॒वैना᳚न्प्रीणाती॒मꣴ स्तन॒मूर्ज॑स्वन्तं धया॒पाम्प्रप्या॑तमग्ने सरि॒रस्य॒ मध्ये᳚। उथ्सं॑ जुषस्व॒ मधु॑मन्तमूर्व समु॒द्रिय॒ꣳ॒ सद॑न॒मा वि॑शस्व। यो वा अ॒ग्निम्प्र॒युज्य॒ न वि॑मु॒ञ्चति॒ यथाश्वो॑ यु॒क्तो\-ऽवि॑मुच्यमानः॒ क्षुध्य॑न्परा॒भव॑त्ये॒वम॑स्या॒ग्निः परा॑ भवति॒ तं प॑रा॒भव॑न्तं॒ यज॑मा॒नो\-ऽनु॒ परा॑ भवति॒ सो᳚\-ऽग्निं चि॒त्वा लू॒क्षः॥४७॥

%5.5.10.7
भ॒व॒ती॒मꣴ स्तन॒मूर्ज॑स्वन्तं धया॒पामित्याज्य॑स्य पू॒र्णाꣴ स्रुचं॑ जुहोत्ये॒ष वा अ॒ग्नेर्वि॑मो॒को वि॒मुच्यै॒वास्मा॒ अन्न॒मपि॑ दधाति॒ तस्मा॑दाहु॒र्यश्चै॒वं वेद॒ यश्च॒ न सु॒धायꣳ॑ ह॒ वै वा॒जी सुहि॑तो दधा॒तीत्य॒ग्निर्वाव वा॒जी तमे॒व तत्प्री॑णाति॒ स ए॑नम्प्री॒तः प्री॑णाति॒ वसी॑यान्भवति॥४८॥

%5.5.11.0
{\anuvakamend[{प्र॒तीची॒ दिक्तस्या᳚स्ते द्वि॒ष्मो यश्च॑ निलि॒म्पा नामे॒ह गृ॒हा इति॑ लू॒क्षो वसी॑यान्भवति}]}%॥10॥

%5.5.11.1
इन्द्रा॑य॒ राज्ञे॑ सूक॒रो वरु॑णाय॒ राज्ञे॒ कृष्णो॑ य॒माय॒ राज्ञ॒ ऋश्य॑ ऋष॒भाय॒ राज्ञे॑ गव॒यः शा᳚र्दू॒लाय॒ राज्ञे॑ गौ॒रः पु॑रुषरा॒जाय॑ म॒र्कटः॑ क्षिप्रश्ये॒नस्य॒ वर्ति॑का॒ नीलं॑गोः॒ क्रिमिः॒ सोम॑स्य॒ राज्ञः॑ कुलु॒ङ्गः सिन्धोः᳚ शिꣳशु॒मारो॑ हि॒मव॑तो ह॒स्ती॥४९॥

%5.5.12.0
{\anuvakamend[{इन्द्रा॑या॒ष्टाविꣳ॑शतिः}]}%॥11॥

%5.5.12.1
म॒युः प्रा॑जाप॒त्य ऊ॒लो हली᳚क्ष्णो वृषद॒ꣳ॒शस्ते धा॒तुः सर॑स्वत्यै॒ शारिः॑ श्ये॒ता पु॑रुष॒वाख्सर॑स्वते॒ शुकः॑ श्ये॒तः पु॑रुष॒वागा॑र॒ण्यो॑\-ऽजो न॑कु॒लः शका॒ ते पौ॒ष्णा वा॒चे क्रौ॒ञ्चः॥५०॥

%5.5.13.0
{\anuvakamend[{म॒युस्त्रयो॑विꣳशतिः}]}%॥12॥

%5.5.13.1
अ॒पां नप्त्रे॑ ज॒षो ना॒क्रो मक॑रः कुली॒कय॒स्ते\-ऽकू॑पारस्य वा॒चे पै᳚ङ्गरा॒जो भगा॑य कु॒षीत॑क आ॒ती वा॑ह॒सो दर्वि॑दा॒ ते वा॑य॒व्या॑ दि॒ग्भ्यश्च॑क्रवा॒कः॥५१॥

%5.5.14.0
{\anuvakamend[{अ॒पामेका॒न्नविꣳ॑शतिः}]}%॥13॥

%5.5.14.1
बला॑याजग॒र आ॒खुः सृ॑ज॒या श॒यण्ड॑क॒स्ते मै॒त्रा मृ॒त्यवे॑\-ऽसि॒तो म॒न्यवे᳚ स्व॒जः कु॑म्भी॒नसः॑ पुष्करसा॒दो लो॑हिता॒हिस्ते त्वा॒ष्ट्राः प्र॑ति॒श्रुत्का॑यै वाह॒सः॥५२॥

%5.5.15.0
{\anuvakamend[{}]}

%5.5.15.1
पु॒रु॒ष॒मृ॒गश्च॒न्द्रम॑से गो॒धा काल॑का दार्वाघा॒टस्ते वन॒स्पती॑नामे॒ण्यह्ने॒ कृष्णो॒ रात्रि॑यै पि॒कः क्ष्विङ्का॒ नील॑शीर्ष्णी॒ ते᳚\-ऽर्य॒म्णे धा॒तुः क॑त्क॒टः॥५३॥

%5.5.16.0
{\anuvakamend[{}]}

%5.5.16.1
सौ॒री ब॒लाकर्श्यो॑ म॒यूरः॑ श्ये॒नस्ते ग॑न्ध॒र्वाणां॒ वसू॑नां क॒पिञ्ज॑लो रु॒द्राणां᳚ तित्ति॒री रो॒हित्कु॑ण्डृ॒णाची॑ गो॒लत्ति॑का॒ ता अ॑फ्स॒रसा॒मर॑ण्याय सृम॒रः॥५४॥

%5.5.17.0
{\anuvakamend[{}]}

%5.5.17.1
पृ॒ष॒तो वै᳚श्वदे॒वः पि॒त्वो न्यङ्कुः॒ कश॒स्ते\-ऽनु॑मत्या अन्यवा॒पो᳚\-ऽर्धमा॒साना᳚म्मा॒सां क॒श्यपः॒ क्वयिः॑ कु॒टरु॑र्दात्यौ॒हस्ते सि॑नीवा॒ल्यै बृह॒स्पत॑ये शित्पु॒टः॥५॥

%5.5.18.0
{\anuvakamend[{}]}

%5.5.18.1
शका॑ भौ॒मी पा॒न्त्रः कशो॑ मान्थी॒लव॒स्ते पि॑तृ॒णामृ॑तू॒नां जह॑का संवथ्स॒राय॒ लोपा॑ क॒पोत॒ उलू॑कः श॒शस्ते नैर्\mbox{}॑ऋ॒ताः कृ॑क॒वाकुः॑ सावि॒त्रः॥५६॥

%5.5.19.0
{\anuvakamend[{बला॑य पुरुषमृ॒गः सौ॒री पृ॑ष॒तः शका॒ष्टाद॑शा॒ष्टाद॑श}]}%॥14-18॥

%5.5.19.1
रुरू॑ रौ॒द्रः कृ॑कला॒सः श॒कुनिः॒ पिप्प॑का॒ ते श॑र॒व्या॑यै हरि॒णो मा॑रु॒तो ब्रह्म॑णे शा॒र्गस्त॒रक्षुः॑ कृ॒ष्णः श्वा च॑तुर॒क्षो ग॑र्द॒भस्त इ॑तरज॒नाना॑म॒ग्नये॒ धूङ्क्ष्णा᳚॥५७॥

%5.5.20.0
{\anuvakamend[{रुरु॑र्विꣳश॒तिः}]}%॥19॥

%5.5.20.1
अ॒ल॒ज आ᳚न्तरि॒क्ष उ॒द्रो म॒द्गुः प्ल॒वस्ते॑\-ऽपामदि॑त्यै हꣳस॒साचि॑रिन्द्रा॒ण्यै कीर्\mbox{}शा॒ गृध्रः॑ शितिक॒क्षी वा᳚र्ध्राण॒सस्ते दि॒व्या द्या॑वापृथि॒व्या᳚ श्वा॒वित्॥५८॥

%5.5.21.0
{\anuvakamend[{}]}

%5.5.21.1
सु॒प॒र्णः पा᳚र्ज॒न्यो ह॒ꣳ॒सो वृको॑ वृषद॒ꣳ॒शस्त ऐ॒न्द्रा अ॒पामु॒द्रो᳚\-ऽर्य॒म्णे लो॑पा॒शः सि॒ꣳ॒हो न॑कु॒लो व्या॒घ्रस्ते म॑हे॒न्द्राय॒ कामा॑य॒ पर॑स्वान्॥५९॥

%5.5.22.0
{\anuvakamend[{अ॒ल॒जः सु॑प॒र्णो᳚\-ऽष्टाद॑शाष्टा॒द॑श}]}%॥21॥

%5.5.22.1
आ॒ग्ने॒यः कृ॒ष्णग्री॑वः सारस्व॒ती मे॒षी ब॒भ्रुः सौ॒म्यः पौ॒ष्णः श्या॒मः शि॑तिपृ॒ष्ठो बा॑र्\mbox{}हस्प॒त्यः शि॒ल्पो वै᳚श्वदे॒व ऐ॒न्द्रो॑\-ऽरु॒णो मा॑रु॒तः क॒ल्माष॑ ऐन्द्रा॒ग्नः सꣳ॑हि॒तो॑\-ऽधोरा॑मः सावि॒त्रो वा॑रु॒णः पेत्वः॑॥६०॥

%5.5.23.0
{\anuvakamend[{आ॒ग्ने॒यो द्वाविꣳ॑शतिः}]}%॥22॥

%5.5.23.1
अश्व॑स्तूप॒रो गो॑मृ॒गस्ते प्रा॑जाप॒त्या आ᳚ग्ने॒यौ कृ॒ष्णग्री॑वौ त्वा॒ष्ट्रौ लो॑मशस॒क्थौ शि॑तिपृ॒ष्ठौ बा॑र्\mbox{}हस्प॒त्यौ धा॒त्रे पृ॑षोद॒रः सौ॒र्यो ब॒लक्षः॒ पेत्वः॑॥६१॥

%5.5.24.0
{\anuvakamend[{अश्व॒ष्षोड॑श}]}%॥23॥

%5.5.24.1
अ॒ग्नये\-ऽनी॑कवते॒ रोहि॑ताञ्जिरन॒ड्वान॒धोरा॑मौ सावि॒त्रौ पौ॒ष्णौ र॑ज॒तना॑भी वैश्वदे॒वौ पि॒शङ्गौ॑ तूप॒रौ मा॑रु॒तः क॒ल्माष॑ आग्ने॒यः कृ॒ष्णो॑\-ऽजः सा॑रस्व॒ती मे॒षी वा॑रु॒णः कृ॒ष्ण एक॑शितिपा॒त्पेत्वः॑॥६२॥

%5.6.0.0

%5.6.0.0
{\anuvakamend[{अ॒ग्नयो\-ऽनी॑कवते॒ द्वाविꣳ॑शतिः}]}%॥24॥

{\anuvakamend[{हिर॑ण्यवर्णा अ॒पां ग्रहा᳚न्भूतेष्ट॒काः स॒जूः सं॑ वथ्स॒रं प्र॒जाप॑तिः॒ स क्षु॒रप॑विर॒ग्नेर्वै दी॒क्षया॑ सुव॒र्गाय॒ तं यन्न सू॒यते᳚ प्र॒जाप॑तिर्\mbox{}ऋ॒तुभी॒ रोहि॑तः॒ पृश्ञिः॑ शितिबा॒हुरु॑न्न॒तः क॒र्णाः शु॒ण्ठा इन्द्रा॒यादि॑त्यै सौ॒म्या वा॑रु॒णाः सोमा॒यैका॑दश पि॒शङ्गा॒स्त्रयो॑विꣳशतिः}]}%॥23॥
\prashnaend{ हिर॑ण्यवर्णा भूतेष्ट॒काश्छन्दो॒ यत्कनी॑याꣳसन्त्रि॒वृद्ध्य॑ग्निर्वा॑रु॒णाश्चतुः॑पञ्चाशत्॥54॥ हिर॑ण्यवर्णा॒ निव॑क्षसः॥}
%%% END PRASHNA

\sect{षष्ठमः प्रश्नः}\setcounter{anuvakam}{0}
\dnsub{तैत्तिरीयसंहितायां पञ्चमकाण्डे षष्ठमः प्रश्नः}
%5.6.1.0
%5.6.1.1
हिर॑ण्यवर्णाः॒ शुच॑यः पाव॒का यासु॑ जा॒तः क॒श्यपो॒ यास्विन्द्रः॑। अ॒ग्निं या गर्भं॑ दधि॒रे विरू॑पा॒स्ता न॒ आपः॒ शꣴ स्यो॒ना भ॑वन्तु। यासा॒ꣳ॒ राजा॒ वरु॑णो॒ याति॒ मध्ये॑ सत्यानृ॒ते अ॑व॒पश्य॒ञ्जना॑नाम्। म॒धु॒श्चुतः॒ शुच॑यो॒ याः पा॑व॒कास्ता न॒ आपः॒ शꣴ स्यो॒ना भ॑वन्तु। यासां᳚ दे॒वा दि॒वि कृ॒ण्वन्ति॑ भ॒क्षं या अ॒न्तरि॑क्षे बहु॒धा भव॑न्ति। याः पृ॑थि॒वीम्पय॑सो॒न्दन्ति॑॥१॥

%5.6.1.2
शु॒क्रास्ता न॒ आपः॒ शꣴ स्यो॒ना भ॑वन्तु। शि॒वेन॑ मा॒ चक्षु॑षा पश्यतापः शि॒वया॑ त॒नुवोप॑ स्पृशत॒ त्वच॑म्मे। सर्वाꣳ॑ अ॒ग्नीꣳ र॑फ्सु॒षदो॑ हुवे वो॒ मयि॒ वर्चो॒ बल॒मोजो॒ नि ध॑त्त। यद॒दः स॑म्प्रय॒तीरहा॒वन॑दता ह॒ते। तस्मा॒दा न॒द्यो॑ नाम॑ स्थ॒ ता वो॒ नामा॑नि सिन्धवः। यत्प्रेषि॑ता॒ वरु॑णेन॒ ताः शीभꣳ॑ स॒मव॑ल्गत।॥२॥

%5.6.1.3
तदा᳚प्नो॒दिन्द्रो॑ वो य॒तीस्तस्मा॒दापो॒ अनु॑ स्थन। अ॒प॒का॒मꣴ स्यन्द॑माना॒ अवी॑वरत वो॒ हिकम्᳚। इन्द्रो॑ वः॒ शक्ति॑भिर्देवी॒स्तस्मा॒द्वार्णाम॑ वो हि॒तम्। एको॑ दे॒वो अप्य॑तिष्ठ॒थ्स्यन्द॑माना यथाव॒शम्। उदा॑निषुर्म॒हीरिति॒ तस्मा॑दुद॒कमु॑च्यते। आपो॑ भ॒द्रा घृ॒तमिदाप॑ आसुर॒ग्नीषोमौ॑ बिभ्र॒त्याप॒ इत्ताः। ती॒व्रो रसो॑ मधु॒पृचा᳚म्॥३॥

%5.6.1.4
अ॒रं॒ग॒म आ मा᳚ प्रा॒णेन॑ स॒ह वर्च॑सा गन्न्। आदित्प॑श्याम्यु॒त वा॑ शृणो॒म्या मा॒ घोषो॑ गच्छति॒ वाङ्न॑ आसाम्। मन्ये॑ भेजा॒नो अ॒मृत॑स्य॒ तर्\mbox{}हि॒ हिर॑ण्यवर्णा॒ अतृ॑पं य॒दा वः॑। आपो॒ हि ष्ठा म॑यो॒भुव॒स्ता न॑ ऊ॒र्जे द॑धातन। म॒हे रणा॑य॒ चक्ष॑से। यो वः॑ शि॒वत॑मो॒ रस॒स्तस्य॑ भाजयते॒ह नः॑। उ॒श॒तीरि॑व मा॒तरः॑। तस्मा॒ अरं॑ गमाम वो॒ यस्य॒ क्षया॑य॒ जिन्व॑थ। आपो॑ ज॒नय॑था च नः। दि॒वि श्र॑यस्वा॒न्तरि॑क्षे यतस्व पृथि॒व्या सम्भ॑व ब्रह्मवर्च॒सम॑सि ब्रह्मवर्च॒साय॑ त्वा॥४॥

%5.6.2.0
{\anuvakamend[{उ॒न्दन्ति॑ स॒मव॑ल्गत मधु॒पृचा᳚म्मा॒तरो॒ द्वाविꣳ॑शतिश्च}]}%॥१॥

%5.6.2.1
अ॒पां ग्रहा᳚न्गृह्णात्ये॒तद्वाव रा॑ज॒सूयं॒ यदे॒ते ग्रहाः᳚ स॒वो᳚\-ऽग्निर्व॑रुणस॒वो रा॑ज॒सूय॑मग्निस॒वश्चित्य॒स्ताभ्या॑मे॒व सू॑य॒ते\-ऽथो॑ उ॒भावे॒व लो॒काव॒भि ज॑यति॒ यश्च॑ राज॒सूये॑नेजा॒नस्य॒ यश्चा᳚ग्नि॒चित॒ आपो॑ भव॒न्त्यापो॒ वा अ॒ग्नेर्भ्रातृ॑व्या॒ यद॒पो᳚\-ऽग्नेर॒धस्ता॑दुप॒दधा॑ति॒ भ्रातृ॑व्याभिभूत्यै॒ भव॑त्या॒त्मना॒ परा᳚स्य॒ भ्रातृ॑व्यो भवत्य॒मृतम्᳚॥५॥

%5.6.2.2
वा आप॒स्तस्मा॑द॒द्भिरव॑तान्तम॒भि षि॑ञ्चन्ति॒ नार्ति॒मार्च्छ॑ति॒ सर्व॒मायु॑रेति॒ यस्यै॒ता उ॑पधी॒यन्ते॒ य उ॑ चैना ए॒वं वेदान्नं॒ वा आपः॑ प॒शव॒ आपो\-ऽन्न॑म्प॒शवो᳚\-ऽन्ना॒दः प॑शु॒मान्भ॑वति॒ यस्यै॒ता उ॑पधी॒यन्ते॒ य उ॑ चैना ए॒वं वेद॒ द्वाद॑श भवन्ति॒ द्वाद॑श॒ मासाः᳚ संवथ्स॒रः सं॑वथ्स॒रेणै॒वास्मै᳚॥६॥

%5.6.2.3
अन्न॒मव॑ रुन्द्धे॒ पात्रा॑णि भवन्ति॒ पात्रे॒ वा अन्न॑मद्यते॒ सयो᳚न्ये॒वान्न॒मव॑ रुन्द्ध॒ आ द्वा॑द॒शात्पुरु॑षा॒दन्न॑म॒त्त्यथो॒ पात्रा॒न्न छि॑द्यते॒ यस्यै॒ता उ॑पधी॒यन्ते॒ य उ॑ चैना ए॒वं वेद॑ कु॒म्भाश्च॑ कु॒म्भीश्च॑ मिथु॒नानि॑ भवन्ति मिथु॒नस्य॒ प्रजा᳚त्यै॒ प्र प्र॒जया॑ प॒शुभि॑र्मिथु॒नैर्जा॑यते॒ यस्यै॒ता उ॑पधी॒यन्ते॒ य उ॑॥७॥

%5.6.2.4
चै॒ना॒ ए॒वं वेद॒ शुग्वा अ॒ग्निः सो᳚\-ऽध्व॒र्युं यज॑मानं प्र॒जाः शु॒चार्प॑यति॒ यद॒प उ॑प॒दधा॑ति॒ शुच॑मे॒वास्य॑ शमयति॒ नार्ति॒मार्च्छ॑त्यध्व॒र्युर्न यज॑मानः॒ शाम्य॑न्ति प्र॒जा यत्रै॒ता उ॑पधी॒यन्ते॒\-ऽपां वा ए॒तानि॒ हृद॑यानि॒ यदे॒ता आपो॒ यदे॒ता अ॒प उ॑प॒दधा॑ति दि॒व्याभि॑रे॒वैनाः॒ सꣳ सृ॑जति॒ वर्\mbox{}षु॑कः प॒र्जन्यः॑॥८॥

%5.6.2.5
भ॒व॒ति॒ यो वा ए॒तासा॑मा॒यत॑नं॒ कॢप्तिं॒ वेदा॒यत॑नवान्भवति॒ कल्प॑ते\-ऽस्मा अनुसी॒तमुप॑ दधात्ये॒तद्वा आ॑सामा॒यत॑नमे॒षा कॢप्ति॒र्य ए॒वं वेदा॒यत॑नवान्भवति॒ कल्प॑ते\-ऽस्मै द्व॒न्द्वम॒न्या उप॑ दधाति॒ चत॑स्रो॒ मध्ये॒ धृत्या॒ अन्नं॒ वा इष्ट॑का ए॒तत्खलु॒ वै सा॒क्षादन्नं॒ यदे॒ष च॒रुर्यदे॒तं च॒रुमु॑प॒दधा॑ति सा॒क्षात्॥९॥

%5.6.2.6
ए॒वास्मा॒ अन्न॒मव॑ रुन्द्धे मध्य॒त उप॑ दधाति मध्य॒त ए॒वास्मा॒ अन्नं॑ दधाति॒ तस्मा᳚न्मध्य॒तो\-ऽन्न॑मद्यते बार्\mbox{}हस्प॒त्यो भ॑वति॒ ब्रह्म॒ वै दे॒वाना॒म्बृह॒स्पति॒र्ब्रह्म॑णै॒वास्मा॒ अन्न॒मव॑ रुन्द्धे ब्रह्मवर्च॒सम॑सि ब्रह्मवर्च॒साय॒ त्वेत्या॑ह तेज॒स्वी ब्र॑ह्मवर्च॒सी भ॑वति॒ यस्यै॒ष उ॑पधी॒यते॒ य उ॑ चैनमे॒वं वेद॑॥१०॥

%5.6.3.0
{\anuvakamend[{अ॒मृत॑मस्मै जायते॒ यस्यै॒ता उ॑पधी॒यन्ते॒ य उ॑ प॒र्जन्य॑ उप॒दधा॑ति सा॒क्षाथ्स॒प्तच॑त्वारिꣳशच्च}]}%॥२॥

%5.6.3.1
भू॒ते॒ष्ट॒का उप॑ दधा॒त्यत्रा᳚त्र॒ वै मृ॒त्युर्जा॑यते॒ यत्र॑यत्रै॒व मृ॒त्युर्जाय॑ते॒ तत॑ ए॒वैन॒मव॑ यजते॒ तस्मा॑दग्नि॒चिथ्सर्व॒मायु॑रेति॒ सर्वे॒ ह्य॑स्य मृ॒त्यवो\-ऽवे᳚ष्टा॒स्तस्मा॑दग्नि॒चिन्नाभिच॑रित॒वै प्र॒त्यगे॑नमभिचा॒रः स्तृ॑णुते सू॒यते॒ वा ए॒ष यो᳚\-ऽग्निं चि॑नु॒ते दे॑वसु॒वामे॒तानि॑ ह॒वीꣳषि॑ भवन्त्ये॒ताव॑न्तो॒ वै दे॒वानाꣳ॑ स॒वास्त ए॒व॥११॥

%5.6.3.2
अ॒स्मै॒ स॒वान्प्र य॑च्छन्ति॒ त ए॑नꣳ सुवन्ते स॒वो᳚\-ऽग्निर्व॑रुणस॒वो रा॑ज॒सूयं॑ ब्रह्मस॒वश्चित्यो॑ दे॒वस्य॑ त्वा सवि॒तुः प्र॑स॒व इत्या॑ह सवि॒तृप्र॑सूत ए॒वैनं॒ ब्रह्म॑णा दे॒वता॑भिर॒भि षि॑ञ्च॒त्यन्न॑स्यान्नस्या॒भि षि॑ञ्च॒त्यन्न॑स्यान्न॒स्याव॑रुद्ध्यै पु॒रस्ता᳚त्प्र॒त्यञ्च॑म॒भि षि॑ञ्चति पु॒रस्ता॒द्धि प्र॑ती॒चीन॒मन्न॑म॒द्यते॑ शीर्\mbox{}ष॒तो॑\-ऽभि षि॑ञ्चति शीर्\mbox{}ष॒तो ह्यन्न॑म॒द्यत॒ आ मुखा॑द॒न्वव॑स्रावयति॥१२॥

%5.6.3.3
मु॒ख॒त ए॒वास्मा॑ अ॒न्नाद्यं॑ दधात्य॒ग्नेस्त्वा॒ साम्रा᳚ज्येना॒भि षि॑ञ्चा॒मीत्या॑है॒ष वा अ॒ग्नेः स॒वस्तेनै॒वैन॑म॒भि षि॑ञ्चति॒ बृह॒स्पते᳚स्त्वा॒ साम्रा᳚ज्येना॒भि षि॑ञ्चा॒मीत्या॑ह॒ ब्रह्म॒ वै दे॒वाना॒म्बृह॒स्पति॒र्ब्रह्म॑णै॒वैन॑म॒भि षि॑ञ्च॒तीन्द्र॑स्य त्वा॒ साम्रा᳚ज्येना॒भि षि॑ञ्चा॒मीत्या॑हेन्द्रि॒यमे॒वास्मि॑न्नु॒परि॑ष्टाद्दधात्ये॒तत्॥१३॥

%5.6.3.4
वै रा॑ज॒सूय॑स्य रू॒पं य ए॒वं वि॒द्वान॒ग्निं चि॑नु॒त उ॒भावे॒व लो॒काव॒भि ज॑यति॒ यश्च॑ राज॒सूये॑नेजा॒नस्य॒ यश्चा᳚ग्नि॒चित॒ इन्द्र॑स्य सुषुवा॒णस्य॑ दश॒धेन्द्रि॒यं वी॒र्यं॑ परा॑पत॒त्तद्दे॒वाः सौ᳚त्राम॒ण्या सम॑भरन्थ्सू॒यते॒ वा ए॒ष यो᳚\-ऽग्निं चि॑नु॒ते᳚\-ऽग्निं चि॒त्वा सौ᳚त्राम॒ण्या य॑जेतेन्द्रि॒यमे॒व वी॒र्यꣳ॑ स॒म्भृत्या॒त्मन्ध॑त्ते॥१४॥

%5.6.4.0
{\anuvakamend[{त ए॒वान्वव॑स्रावयत्ये॒तद॒ष्टाच॑त्वारिꣳशच्च}]}%॥३॥

%5.6.4.1
स॒जूरब्दो\-ऽया॑वभिः स॒जूरु॒षा अरु॑णीभिः स॒जूः सूर्य॒ एत॑शेन स॒जोषा॑व॒श्विना॒ दꣳसो॑भिः स॒जूर॒ग्निर्वै᳚श्वान॒र इडा॑भिर्घृ॒तेन॒ स्वाहा॑ संवथ्स॒रो वा अब्दो॒ मासा॒ अया॑वा उ॒षा अरु॑णी॒ सूर्य॒ एत॑श इ॒मे अ॒श्विना॑ संवथ्स॒रो᳚\-ऽग्निर्वै᳚श्वान॒रः प॒शव॒ इडा॑ प॒शवो॑ घृ॒तꣳ सं॑वथ्स॒रम्प॒शवो\-ऽनु॒ प्र जा॑यन्ते संवथ्स॒रेणै॒वास्मै॑ प॒शून्प्र ज॑नयति दर्भस्त॒म्बे जु॑होति॒ यत्॥१५॥

%5.6.4.2
वा अ॒स्या अ॒मृतं॒ यद्वी॒र्यं॑ तद्द॒र्भास्तस्मि॑ञ्जुहोति॒ प्रैव जा॑यते\-ऽन्ना॒दो भ॑वति॒ यस्यै॒वं जुह्व॑त्ये॒ता वै दे॒वता॑ अ॒ग्नेः पु॒रस्ता᳚द्भागा॒स्ता ए॒व प्री॑णा॒त्यथो॒ चक्षु॑रे॒वाग्नेः पु॒रस्ता॒त्प्रति॑ दधा॒त्यन॑न्धो भवति॒ य ए॒वं वेदापो॒ वा इ॒दमग्रे॑ सलि॒लमा॑सी॒थ्स प्र॒जाप॑तिः पुष्करप॒र्णे वातो॑ भू॒तो॑\-ऽलेलाय॒थ्सः॥१६॥

%5.6.4.3
प्र॒ति॒ष्ठां नावि॑न्द॒त स ए॒तद॒पां कु॒लाय॑मपश्य॒त्तस्मि॑न्न॒ग्निम॑चिनुत॒ तदि॒यम॑भव॒त्ततो॒ वै स प्रत्य॑तिष्ठ॒द्याम्पु॒रस्ता॑दु॒पा\-द॑धा॒त्तच्छिरो॑\-ऽभव॒थ्सा प्राची॒ दिग्यां द॑क्षिण॒त उ॒पाद॑धा॒थ्स दक्षि॑णः प॒क्षो॑\-ऽभव॒थ्सा द॑क्षि॒णा दिग्याम्प॒श्चादु॒पा\-द॑धा॒त्तत्पुच्छ॑मभव॒थ्सा प्र॒तीची॒ दिग्यामु॑त्तर॒त उ॒पाद॑धात्॥१७॥

%5.6.4.4
स उत्त॑रः प॒क्षो॑\-ऽभव॒थ्सोदी॑ची॒ दिग्यामु॒परि॑ष्टादु॒पाद॑धा॒त्तत्पृ॒ष्ठम॑भव॒थ्सोर्ध्वा दिगि॒यं वा अ॒ग्निः पञ्चे᳚ष्टक॒स्तस्मा॒द्यद॒स्यां खन॑न्त्य॒भीष्ट॑कां तृ॒न्दन्त्य॒भि शर्क॑रा॒ꣳ॒ सर्वा॒ वा इ॒यं वयो᳚भ्यो॒ नक्तं॑ दृ॒शे दी᳚प्यते॒ तस्मा॑दि॒मां वयाꣳ॑सि॒ नक्तं॒ नाध्या॑सते॒ य ए॒वं वि॒द्वान॒ग्निं चि॑नु॒ते प्रत्ये॒व॥१८॥

%5.6.4.5
ति॒ष्ठ॒त्य॒भि दिशो॑ जयत्याग्ने॒यो वै ब्रा᳚ह्म॒णस्तस्मा᳚द्ब्राह्म॒णाय॒ सर्वा॑सु दि॒क्ष्वर्धु॑क॒ꣴ॒ स्वामे॒व तद्दिश॒मन्वे᳚त्य॒पां वा अ॒ग्निः कु॒लाय॒न्तस्मा॒दापो॒\-ऽग्निꣳ हारु॑काः॒ स्वामे॒व तद्योनि॒म्प्र वि॑शन्ति॥१९॥

%5.6.5.0
{\anuvakamend[{यद॑लेलाय॒थ्स उ॑त्तर॒त उ॒पाद॑धादे॒व द्वात्रिꣳ॑शच्च}]}%॥४॥

%5.6.5.1
सं॒व॒थ्स॒रमुख्य॑म्भृ॒त्वा द्वि॒तीये॑ संवथ्स॒र आ᳚ग्ने॒यम॒ष्टाक॑पालं॒ निर्व॑पेदै॒न्द्रमेका॑दशकपालं वैश्वदे॒वं द्वाद॑शकपालम् बार्\mbox{}हस्प॒त्यं च॒रुं वै᳚ष्ण॒वं त्रि॑कपा॒लन्तृ॒तीये॑ संवथ्स॒रे॑\-ऽभि॒जिता॑ यजेत॒ यद॒ष्टाक॑पालो॒ भव॑त्य॒ष्टाक्ष॑रा गाय॒त्र्या᳚ग्ने॒यं गा॑य॒त्रम्प्रा॑तःसव॒नम् प्रा॑तःसव॒नमे॒व तेन॑ दाधार गाय॒त्रं छन्दो॒ यदेका॑दशकपालो॒ भव॒त्येका॑दशाक्षरा त्रि॒ष्टुगै॒न्द्रं त्रैष्टु॑भ॒म्माध्यं॑दिन॒ꣳ॒ सव॑न॒म्माध्यं॑दिनमे॒व सव॑नं॒ तेन॑ दाधार त्रि॒ष्टुभम्᳚॥२०॥

%5.6.5.2
छन्दो॒ यद्द्वाद॑शकपालो॒ भव॑ति॒ द्वाद॑शाक्षरा॒ जग॑ती वैश्वदे॒वं जाग॑तं तृतीयसव॒नन्तृ॑तीयसव॒नमे॒व तेन॑ दाधार॒ जग॑तीं॒ छन्दो॒ यद्बा॑र्\mbox{}हस्प॒त्यश्च॒रुर्भव॑ति॒ ब्रह्म॒ वै दे॒वाना॒म्बृह॒स्पति॒र्ब्रह्मै॒व तेन॑ दाधार॒ यद्वै᳚ष्ण॒वस्त्रि॑कपा॒लो भव॑ति य॒ज्ञो वै विष्णु॑र्य॒ज्ञमे॒व तेन॑ दाधार॒ यत्तृ॒तीये॑ संवथ्स॒रे॑\-ऽभि॒जिता॒ यज॑ते॒\-ऽभिजि॑त्यै॒ यथ्सं॑वथ्स॒रमुख्य॑म्बि॒भर्ती॒ममे॒व॥२१॥

%5.6.5.3
तेन॑ लो॒कꣴ स्पृ॑णोति॒ यद्द्वि॒तीये॑ संवथ्स॒रे᳚\-ऽग्निं चि॑नु॒ते᳚\-ऽन्तरि॑क्षमे॒व तेन॑ स्पृणोति॒ यत्तृ॒तीये॑ संवथ्स॒रे यज॑ते॒\-ऽमुमे॒व तेन॑ लो॒कꣴ स्पृ॑णोत्ये॒तं वै पर॑ आट्णा॒रः क॒क्षीवाꣳ॑ औशि॒जो वी॒तह॑व्यः श्राय॒सस्त्र॒सद॑स्युः पौरुकु॒थ्स्यः प्र॒जाका॑मा अचिन्वत॒ ततो॒ वै ते स॒हस्रꣳ॑सहस्रम्पु॒त्रान॑विन्दन्त॒ प्रथ॑ते प्र॒जया॑ प॒शुभि॒स्ताम्मात्रा॑माप्नोति॒ यां ते\-ऽग॑च्छ॒न् य ए॒वं वि॒द्वाने॒तम॒ग्निं चि॑नु॒ते॥२२॥

%5.6.6.0
{\anuvakamend[{दा॒धा॒र॒ त्रि॒ष्टुभ॑मि॒ममे॒वैवं च॒त्वारि॑ च}]}%॥५॥

%5.6.6.1
प्र॒जाप॑तिर॒ग्निम॑चिनुत॒ स क्षु॒रप॑विर्भू॒त्वाति॑ष्ठ॒त्तं दे॒वा बिभ्य॑तो॒ नोपा॑य॒न्ते छन्दो॑भिरा॒त्मानं॑ छादयि॒त्वोपा॑य॒न्तच्छन्द॑सां छन्द॒स्त्वं ब्रह्म॒ वै छन्दाꣳ॑सि॒ ब्रह्म॑ण ए॒तद्रू॒पं यत्कृ॑ष्णाजि॒नङ्कार्ष्णी॑ उपा॒नहा॒वुप॑ मुञ्चते॒ छन्दो॑भिरे॒वात्मानं॑ छादयि॒त्वाग्निमुप॑ चरत्या॒त्मनो\-ऽहिꣳ॑सायै देवनि॒धिर्वा ए॒ष नि धी॑यते॒ यद॒ग्निः॥२३॥

%5.6.6.2
अ॒न्ये वा॒ वै नि॒धिमगु॑प्तं वि॒न्दन्ति॒ न वा॒ प्रति॒ प्र जा॑नात्यु॒खामा क्रा॑मत्या॒त्मान॑मे॒वाधि॒पां कु॑रुते॒ गुप्त्या॒ अथो॒ खल्वा॑हु॒र्नाक्रम्येति॑ नैर्\mbox{}ऋ॒त्यु॑खा यदा॒क्रामे॒न्निर्\mbox{}ऋ॑त्या आ॒त्मान॒मपि॑ दध्या॒त्तस्मा॒न्नाक्रम्या॑ पुरुषशी॒र्\mbox{}षमुप॑ दधाति॒ गुप्त्या॒ अथो॒ यथा᳚ ब्रू॒यादे॒तन्मे॑ गोपा॒येति॑ ता॒दृगे॒व तत्॥२४॥

%5.6.6.3
प्र॒जाप॑ति॒र्वा अथ॑र्वा॒ग्निरे॒व द॒ध्यङ्ङा॑थर्व॒णस्तस्येष्ट॑का अ॒स्थान्ये॒तꣳ ह॒ वाव तदृषि॑र॒भ्यनू॑वा॒चेन्द्रो॑ दधी॒चो अ॒स्थभि॒रिति॒ यदिष्ट॑काभिर॒ग्निं चि॒नोति॒ सात्मा॑नमे॒वाग्निं चि॑नुते॒ सात्मा॒मुष्मि॑ल्लोँ॒के भ॑वति॒ य ए॒वं वेद॒ शरी॑रं॒ वा ए॒तद॒ग्नेर्यच्चित्य॑ आ॒त्मा वै᳚श्वान॒रो यच्चि॒ते वै᳚श्वान॒रं जु॒होति॒ शरी॑रमे॒व स॒ꣴ॒स्कृत्य॑॥२५॥

%5.6.6.4
अ॒भ्यारो॑हति॒ शरी॑रं॒ वा ए॒तद्यज॑मानः॒ सꣴस्कु॑रुते॒ यद॒ग्निं चि॑नु॒ते यच्चि॒ते वै᳚श्वान॒रं जु॒होति॒ शरी॑रमे॒व स॒ꣴ॒स्कृत्या॒त्मना॒भ्यारो॑हति॒ तस्मा॒त्तस्य॒ नाव॑ द्यन्ति॒ जीव॑न्ने॒व दे॒वानप्ये॑ति वैश्वान॒र्यर्चा पुरी॑ष॒मुप॑ दधाती॒यं वा अ॒ग्निर्वै᳚श्वान॒रस्तस्यै॒षा चिति॒र्यत्पुरी॑षम॒ग्निमे॒व वै᳚श्वान॒रं चि॑नुत ए॒षा वा अ॒ग्नेः प्रि॒या त॒नूर्यद्वै᳚श्वान॒रः प्रि॒यामे॒वास्य॑ त॒नुव॒मव॑ रुन्द्धे॥२६॥

%5.6.7.0
{\anuvakamend[{अ॒ग्निस्तथ्स॒ꣴ॒स्कृत्या॒ग्नेर्दश॑ च}]}%॥६॥

%5.6.7.1
अ॒ग्नेर्वै दी॒क्षया॑ दे॒वा वि॒राज॑माप्नुवन्ति॒स्रो रात्री᳚र्दीक्षि॒तः स्या᳚त्त्रि॒पदा॑ वि॒राड्वि॒राज॑माप्नोति॒ षड्रात्री᳚र्दीक्षि॒तः स्या॒त् षड्वा ऋ॒तवः॑ संवथ्स॒रः सं॑वथ्स॒रो वि॒राड्वि॒राज॑माप्नोति॒ दश॒ रात्री᳚र्दीक्षि॒तः स्या॒द्दशा᳚क्षरा वि॒राड्वि॒राज॑माप्नोति॒ द्वाद॑श॒ रात्री᳚र्दीक्षि॒तः स्या॒द्द्वाद॑श॒ मासाः᳚ संवथ्स॒रः सं॑वथ्स॒रो वि॒राड्वि॒राज॑माप्नोति॒ त्रयो॑दश॒ रात्री᳚र्दीक्षि॒तः स्या॒त्त्रयो॑दश॥२७॥

%5.6.7.2
मासाः᳚ संवथ्स॒रः सं॑वथ्स॒रो वि॒राड्वि॒राज॑माप्नोति॒ पञ्च॑दश॒ रात्री᳚र्दीक्षि॒तः स्या॒त्पञ्च॑दश॒ वा अ॑र्धमा॒सस्य॒ रात्र॑यो\-ऽर्धमास॒शः सं॑वथ्स॒र आ᳚प्यते संवथ्स॒रो वि॒राड्वि॒राज॑माप्नोति स॒प्तद॑श॒ रात्री᳚र्दीक्षि॒तः स्या॒द्द्वाद॑श॒ मासाः॒ पञ्च॒र्तवः॒ स सं॑वथ्स॒रः सं॑वथ्स॒रो वि॒राड्वि॒राज॑माप्नोति॒ चतु॑र्विꣳशति॒ꣳ॒ रात्री᳚र्दीक्षि॒तः स्या॒च्चतु॑र्विꣳशतिरर्धमा॒साः सं॑वथ्स॒रः सं॑वथ्स॒रो वि॒राड्वि॒राज॑माप्नोति त्रि॒ꣳ॒शत॒ꣳ॒ रात्री᳚र्दीक्षि॒तः स्या᳚त्॥२८॥

%5.6.7.3
त्रि॒ꣳ॒शद॑क्षरा वि॒राड्वि॒राज॑माप्नोति॒ मासं॑ दीक्षि॒तः स्या॒द्यो मासः॒ स सं॑वथ्स॒रः सं॑वथ्स॒रो वि॒राड्वि॒राज॑माप्नोति च॒तुरो॑ मा॒सो दी᳚क्षि॒तः स्या᳚च्च॒तुरो॒ वा ए॒तम्मा॒सो वस॑वो\-ऽबिभरु॒स्ते पृ॑थि॒वीमाज॑यन्गाय॒त्रीं छन्दो॒\-ऽष्टौ रु॒द्रास्ते᳚\-ऽन्तरि॑क्ष॒माज॑यन्त्रि॒ष्टुभं॒ छन्दो॒ द्वाद॑शादि॒त्यास्ते दिव॒माज॑य॒ञ्जग॑तीं॒ छन्द॒स्ततो॒ वै ते व्या॒वृत॑मगच्छ॒ञ्छ्रैष्ठ्यं॑ दे॒वाना॒म् तस्मा॒द्द्वाद॑श मा॒सो भृ॒त्वाग्निं चि॑न्वीत॒ द्वाद॑श॒ मासाः᳚ संवथ्स॒रः सं॑वथ्स॒रो᳚\-ऽग्निश्चित्य॒स्तस्या॑होरा॒त्राणीष्ट॑का आ॒प्तेष्ट॑कमेनं चिनु॒ते\-ऽथो᳚ व्या॒वृत॑मे॒व ग॑च्छति॒ श्रैष्ठ्यꣳ॑ समा॒नाना᳚म्॥२९॥

%5.6.8.0
{\anuvakamend[{स्या॒त्त्रयो॑दश त्रि॒ꣳ॒शत॒ꣳ॒ रात्री᳚र्दीक्षि॒तः स्या॒द्वै ते᳚\-ऽष्टाविꣳ॑शतिश्च}]}%॥७॥

%5.6.8.1
सु॒व॒र्गाय॒ वा ए॒ष लो॒काय॑ चीयते॒ यद॒ग्निस्तं यन्नान्वा॒रोहे᳚थ्सुव॒र्गाल्लो॒काद्यज॑मानो हीयेत पृथि॒वीमाक्र॑मिषं प्रा॒णो मा॒ मा हा॑सीद॒न्तरि॑क्ष॒माक्र॑मिषं प्र॒जा मा॒ मा हा॑सी॒द्दिव॒माक्र॑मिष॒ꣳ॒ सुव॑रग॒न्मेत्या॑है॒ष वा अ॒ग्नेर॑न्वारो॒हस्तेनै॒वैन॑\-म॒न्वारो॑हति सुव॒र्गस्य॑ लो॒कस्य॒ सम॑ष्ट्यै॒ यत्प॒क्षस॑म्मिताम्मिनु॒यात्॥३०॥

%5.6.8.2
कनी॑याꣳसं यज्ञक्र॒तुमुपे॑या॒त्पापी॑यस्यस्या॒त्मनः॑ प्र॒जा स्या॒द्वेदि॑सम्मिताम्मिनोति॒ ज्यायाꣳ॑समे॒व य॑ज्ञक्र॒तुमुपै॑ति॒ नास्या॒त्मनः॒ पापी॑यसी प्र॒जा भ॑वति साह॒स्रं चि॑न्वीत प्रथ॒मं चि॑न्वा॒नः स॒हस्र॑सम्मितो॒ वा अ॒यं लो॒क इ॒ममे॒व लो॒कम॒भि ज॑यति॒ द्विषा॑हस्रं चिन्वीत द्वि॒तीयं॑ चिन्वा॒नो द्विषा॑हस्रं॒ वा अ॒न्तरि॑क्षम॒न्तरि॑क्षमे॒वाभि ज॑यति॒ त्रिषा॑हस्रं चिन्वीत तृ॒तीयं॑ चिन्वा॒नः॥३१॥

%5.6.8.3
त्रिषा॑हस्रो॒ वा अ॒सौ लो॒को॑\-ऽमुमे॒व लो॒कम॒भि ज॑यति जानुद॒घ्नं चि॑न्वीत प्रथ॒मं चि॑न्वा॒नो गा॑यत्रि॒यैवेमं लो॒कम॒भ्यारो॑हति नाभिद॒घ्नं चि॑न्वीत द्वि॒तीयं॑ चिन्वा॒नस्त्रि॒ष्टुभै॒वान्तरि॑क्षम॒भ्यारो॑हति ग्रीवद॒घ्नं चि॑न्वीत तृ॒तीयं॑ चिन्वा॒नो जग॑त्यै॒वामुं लो॒कम॒भ्यारो॑हति॒ नाग्निं चि॒त्वा रा॒मामुपे॑यादयो॒नौ रेतो॑ धास्या॒मीति॒ न द्वि॒तीयं॑ चि॒त्वान्यस्य॒ स्त्रियम्᳚॥३२॥

%5.6.8.4
उपे॑या॒न्न तृ॒तीयं॑ चि॒त्वा कां च॒नोपे॑या॒द्रेतो॒ वा ए॒तन्नि ध॑त्ते॒ यद॒ग्निं चि॑नु॒ते यदु॑पे॒याद्रेत॑सा॒ व्यृ॑ध्ये॒ताथो॒ खल्वा॑हुरप्रज॒स्यं तद्यन्नोपे॒यादिति॒ यद्रे॑तः॒सिचा॑वुप॒दधा॑ति॒ ते ए॒व यज॑मानस्य॒ रेतो॑ बिभृत॒स्तस्मा॒दुपे॑या॒द्रेत॒सो\-ऽस्क॑न्दाय॒ त्रीणि॒ वाव रेताꣳ॑सि पि॒ता पु॒त्रः पौत्रः॑॥३३॥

%5.6.8.5
यद्द्वे रे॑तः॒सिचा॑वुपद॒ध्याद्रेतो᳚\-ऽस्य॒ विच्छि॑न्द्यात्ति॒स्र उप॑ दधाति॒ रेत॑सः॒ सन्त॑त्या इ॒यं वाव प्र॑थ॒मा रे॑तः॒सिग्वाग्वा इ॒यं तस्मा॒त्पश्य॑न्ती॒माम्पश्य॑न्ति॒ वाचं॒ वद॑न्तीम॒न्तरि॑क्षं द्वि॒तीया᳚ प्रा॒णो वा अ॒न्तरि॑क्षं॒ तस्मा॒न्नान्तरि॑क्ष॒म्पश्य॑न्ति॒ न प्रा॒णम॒सौ तृ॒तीया॒ चक्षु॒र्वा अ॒सौ तस्मा॒त्पश्य॑न्त्य॒मूम्पश्य॑न्ति॒ चक्षु॒र्यजु॑षे॒मां च॑॥३४॥

%5.6.8.6
अ॒मूं चोप॑ दधाति॒ मन॑सा मध्य॒मामे॒षां लो॒कानां॒ कॢप्त्या॒ अथो᳚ प्रा॒णाना॑मि॒ष्टो य॒ज्ञो भृगु॑भिराशी॒र्दा वसु॑भि॒स्तस्य॑ त इ॒ष्टस्य॑ वी॒तस्य॒ द्रवि॑णे॒ह भ॑क्षी॒येत्या॑ह स्तुतश॒स्त्रे ए॒वैतेन॑ दुहे पि॒ता मा॑त॒रिश्वाच्छि॑द्रा प॒दा धा॒ अच्छि॑द्रा उ॒शिजः॑ प॒दानु॑ तक्षुः॒ सोमो॑ विश्व॒विन्ने॒ता ने॑ष॒द्बृह॒स्पति॑रुक्थाम॒दानि॑ शꣳसिष॒दित्या॑है॒तद्वा अ॒ग्नेरु॒क्थन्तेनै॒वैन॒मनु॑ शꣳसति॥३५॥

%5.6.9.0
{\anuvakamend[{मि॒नु॒यात्तृ॒तीयं॑ चिन्वा॒नस्त्रियं॒ पौत्र॑श्च॒ वै स॒प्तद॑श च}]}%॥८॥

%5.6.9.1
सू॒यते॒ वा ए॒षो᳚\-ऽग्नी॒नां य उ॒खायां᳚ भ्रि॒यते॒ यद॒धः सा॒दये॒द्गर्भाः᳚ प्र॒पादु॑काः स्यु॒रथो॒ यथा॑ स॒वात्प्र॑त्यव॒रोह॑ति ता॒दृगे॒व तदा॑स॒न्दी सा॑दयति॒ गर्भा॑णां॒ धृत्या॒ अप्र॑पादा॒याथो॑ स॒वमे॒वैनं॑ करोति॒ गर्भो॒ वा ए॒ष यदुख्यो॒ योनिः॑ शि॒क्यं॑ यच्छि॒क्या॑दु॒खां नि॒रूहे॒द्योने॒र्गर्भं॒ निर्\mbox{}ह॑ण्या॒थ्षडु॑द्यामꣳ शि॒क्य॑म्भवति षोढाविहि॒तो वै॥३६॥

%5.6.9.2
पुरु॑ष आ॒त्मा च॒ शिर॑श्च च॒त्वार्यङ्गा᳚न्या॒त्मन्ने॒वैन॑म्बिभर्ति प्र॒जाप॑ति॒र्वा ए॒ष यद॒ग्निस्तस्यो॒खा चो॒लूख॑लं च॒ स्तनौ॒ ताव॑स्य प्र॒जा उप॑ जीवन्ति॒ यदु॒खां चो॒लूख॑लं चोप॒दधा॑ति॒ ताभ्या॑मे॒व यज॑मानो॒\-ऽमुष्मि॑ल्लोँ॒के᳚\-ऽग्निं दु॑हे संवथ्स॒रो वा ए॒ष यद॒ग्निस्तस्य॑ त्रेधाविहि॒ता इ॑ष्टकाः प्राजाप॒त्या वै᳚ष्ण॒वीः॥३७॥

%5.6.9.3
वै॒श्व॒क॒र्म॒णीर॑होरा॒त्राण्ये॒वास्य॑ प्राजाप॒त्या यदुख्य॑म्बि॒भर्ति॑ प्राजाप॒त्या ए॒व तदुप॑ धत्ते॒ यथ्स॒मिध॑ आ॒दधा॑ति वैष्ण॒वा वै वन॒स्पत॑यो वैष्ण॒वीरे॒व तदुप॑ धत्ते॒ यदिष्ट॑काभिर॒ग्निं चि॒नोती॒यं वै वि॒श्वक॑र्मा वैश्वकर्म॒णीरे॒व तदुप॑ धत्ते तस्मा॑दाहुस्त्रि॒वृद॒ग्निरिति॒ तं वा ए॒तं यज॑मान ए॒व चि॑न्वीत॒ यद॑स्या॒न्यश्चि॑नु॒याद्यत्तं दक्षि॑णाभि॒र्न रा॒धये॑द॒ग्निम॑स्य वृञ्जीत॒ यो᳚\-ऽस्या॒ग्निं चि॑नु॒यात्तं दक्षि॑णाभी राधयेद॒ग्निमे॒व तथ्स्पृ॑णोति॥३८॥

%5.6.10.0
{\anuvakamend[{षो॒ढा॒वि॒हि॒तो वै वै᳚ष्ण॒वीर॒न्यो विꣳ॑श॒तिश्च॑}]}%॥९॥

%5.6.10.1
प्र॒जाप॑तिर॒ग्निम॑चिनुत॒र्तुभिः॑ संवथ्स॒रं व॑स॒न्तेनै॒वास्य॑ पूर्वा॒र्धम॑चिनुत ग्री॒ष्मेण॒ दक्षि॑णम्प॒क्षं व॒र्\mbox{}षाभिः॒ पुच्छꣳ॑ श॒रदोत्त॑रम्प॒क्षꣳ हे॑म॒न्तेन॒ मध्यं॒ ब्रह्म॑णा॒ वा अ॑स्य॒ तत्पू᳚र्वा॒र्धम॑चिनुत क्ष॒त्रेण॒ दक्षि॑णम्प॒क्षम्प॒शुभिः॒ पुच्छं॑ वि॒शोत्त॑रम्प॒क्षमा॒शया॒ मध्यं॒ य ए॒वं वि॒द्वान॒ग्निं चि॑नु॒त ऋ॒तुभि॑रे॒वैनं॑ चिनु॒ते\-ऽथो॑ ए॒तदे॒व सर्व॒मव॑॥३९॥

%5.6.10.2
रु॒न्द्धे॒ शृ॒ण्वन्त्ये॑नम॒ग्निं चि॑क्या॒नमत्त्यन्न॒ꣳ॒ रोच॑त इ॒यं वाव प्र॑थ॒मा चिति॒रोष॑धयो॒ वन॒स्पत॑यः॒ पुरी॑षम॒न्तरि॑क्षं द्वि॒तीया॒ वयाꣳ॑सि॒ पुरी॑षम॒सौ तृ॒तीया॒ नक्ष॑त्राणि॒ पुरी॑षं य॒ज्ञश्च॑तु॒र्थी दक्षि॑णा॒ पुरी॑षं॒ यज॑मानः पञ्च॒मी प्र॒जा पुरी॑षं॒ यत्त्रिचि॑तीकं चिन्वी॒त य॒ज्ञं दक्षि॑णामा॒त्मानं॑ प्र॒जाम॒न्तरि॑या॒त्तस्मा॒त्पञ्च॑चितीकश्चेत॒व्य॑ ए॒तदे॒व सर्वꣴ॑ स्पृणोति॒ यत्ति॒स्रश्चित॑यः॥४०॥

%5.6.10.3
त्रि॒वृद्ध्य॑ग्निर्यद्द्वे द्वि॒पाद्यज॑मानः॒ प्रति॑ष्ठित्यै॒ पञ्च॒ चित॑यो भवन्ति॒ पाङ्क्तः॒ पुरु॑ष आ॒त्मान॑मे॒व स्पृ॑णोति॒ पञ्च॒ चित॑यो भवन्ति प॒ञ्चभिः॒ पुरी॑षैर॒भ्यू॑हति॒ दश॒ सम्प॑द्यन्ते॒ दशा᳚क्षरो॒ वै पुरु॑षो॒ यावा॑ने॒व पुरु॑ष॒स्तꣴ स्पृ॑णो॒त्यथो॒ दशा᳚क्षरा वि॒राडन्नं॑ वि॒राड्वि॒राज्ये॒वान्नाद्ये॒ प्रति॑ तिष्ठति संवथ्स॒रो वै ष॒ष्ठी चिति॑र्\mbox{}ऋ॒तवः॒ पुरी॑ष॒ꣳ॒ षट्चित॑यो भवन्ति॒ षट्पुरी॑षाणि॒ द्वाद॑श॒ सम्प॑द्यन्ते॒ द्वाद॑श॒ मासाः᳚ संवथ्स॒रः सं॑वथ्स॒र ए॒व प्रति॑ तिष्ठति॥४१॥

%5.6.11.0
{\anuvakamend[{अव॒ चित॑यः॒ पुरी॑षं॒ पञ्च॑दश च}]}%॥10॥

%5.6.11.1
रोहि॑तो धू॒म्ररो॑हितः क॒र्कन्धु॑रोहित॒स्ते प्रा॑जाप॒त्या ब॒भ्रुर॑रु॒णब॑भ्रुः॒ शुक॑बभ्रु॒स्ते रौ॒द्राः श्येतः॑ श्येता॒क्षः श्येत॑ग्रीव॒स्ते पि॑तृदेव॒त्या᳚स्ति॒स्रः कृ॒ष्णा व॒शा वा॑रु॒ण्य॑स्ति॒स्रः श्वे॒ता व॒शाः सौ॒र्यो॑ मैत्राबार्\mbox{}हस्प॒त्या धू॒म्रल॑लामास्तूप॒राः॥४२॥

%5.6.12.0
{\anuvakamend[{}]}%॥11॥

%5.6.12.1
पृश्ञि॑स्तिर॒श्चीन॑पृश्ञिरू॒र्ध्वपृ॑श्ञि॒स्ते मा॑रु॒ताः फ॒ल्गूर्लो॑हितो॒र्णी ब॑ल॒क्षी ताः सा॑रस्व॒त्यः॑ पृष॑ती स्थू॒लपृ॑षती क्षु॒द्रपृ॑षती॒ ता वै᳚श्वदे॒व्य॑स्ति॒स्रः श्या॒मा व॒शाः पौ॒ष्णिय॑स्ति॒स्रो रोहि॑णीर्व॒शा मै॒त्रिय॑ ऐन्द्राबार्\mbox{}हस्प॒त्या अ॑रु॒णल॑लामास्तूप॒राः॥४३॥

%5.6.13.0
{\anuvakamend[{रोहि॑तः॒ पृश्ञि॒ष्षड्विꣳ॑शति॒ष्षड्विꣳ॑शतिः}]}%॥12॥

%5.6.13.1
शि॒ति॒बा॒हुर॒न्यतः॑शितिबाहुः सम॒न्तशि॑तिबाहु॒स्त ऐ᳚न्द्रवाय॒वाः शि॑ति॒रन्ध्रो॒\-ऽन्यतः॑शितिरन्ध्रः सम॒न्तशि॑तिरन्ध्र॒स्ते मै᳚त्रावरु॒णाः शु॒द्धवा॑लः स॒र्वशु॑द्धवालो म॒णिवा॑ल॒स्त आ᳚श्वि॒नास्ति॒स्रः शि॒ल्पा व॒शा वै॑श्वदे॒व्य॑स्ति॒स्रः श्येनीः᳚ परमे॒ष्ठिने॑ सोमापौ॒ष्णाः श्या॒मल॑लामास्तूप॒राः॥४४॥

%5.6.14.0
{\anuvakamend[{}]}%॥13॥

%5.6.14.1
उ॒न्न॒त ऋ॑ष॒भो वा॑म॒नस्त ऐ᳚न्द्रावरु॒णाः शिति॑ककुच्छितिपृ॒ष्ठः शिति॑भस॒त्त ऐ᳚न्द्राबार्\mbox{}हस्प॒त्याः शिति॒पाच्छि॒त्योष्ठः॑ शिति॒भ्रुस्त ऐ᳚न्द्रावैष्ण॒वास्ति॒स्रः सि॒ध्मा व॒शा वै᳚श्वकर्म॒ण्य॑स्ति॒स्रो धा॒त्रे पृ॑षोद॒रा ऐ᳚न्द्रापौ॒ष्णाः श्येत॑ललामास्तूप॒राः॥४५॥

%5.6.15.0
{\anuvakamend[{शि॒ति॒बा॒हुरु॑न्न॒तः पञ्च॑विꣳशतिः॒ पञ्च॑विꣳशतिः}]}%॥14॥

%5.6.15.1
क॒र्णास्त्रयो॑ या॒माः सौ॒म्यास्त्रयः॑ श्विति॒ङ्गा अ॒ग्नये॒ यवि॑ष्ठाय॒ त्रयो॑ नकु॒लास्ति॒स्रो रोहि॑णी॒स्त्र्यव्य॒स्ता वसू॑नान्ति॒स्रो॑\-ऽरु॒णा दि॑त्यौ॒ह्य॑स्ता रु॒द्राणाꣳ॑ सोमै॒न्द्रा ब॒भ्रुल॑लामास्तूप॒राः॥४६॥

%5.6.16.0
{\anuvakamend[{क॒र्णास्त्रयो॑विꣳशतिः}]}%॥15॥

%5.6.16.1
शु॒ण्ठास्त्रयो॑ वैष्ण॒वा अ॑धीलोध॒कर्णा॒स्त्रयो॒ विष्ण॑व उरुक्र॒माय॑ लफ्सु॒दिन॒स्त्रयो॒ विष्ण॑व उरुगा॒याय॒ पञ्चा॑वीस्ति॒स्र आ॑दि॒त्याना᳚न्त्रिव॒थ्सास्ति॒स्रो\-ऽङ्गि॑रसामैन्द्रावैष्ण॒वा गौ॒रल॑लामास्तूप॒राः॥४७॥

%5.6.17.0
{\anuvakamend[{शु॒ण्ठा विꣳ॑श॒तिः}]}%॥16॥

%5.6.17.1
इन्द्रा॑य॒ राज्ञे॒ त्रयः॑ शितिपृ॒ष्ठा इन्द्रा॑याधिरा॒जाय॒ त्रयः॒ शिति॑ककुद॒ इन्द्रा॑य स्व॒राज्ञे॒ त्रयः॒ शिति॑भसदस्ति॒स्रस्तु॑र्यौ॒ह्यः॑ सा॒ध्याना᳚न्ति॒स्रः प॑ष्ठौ॒ह्यो॑ विश्वे॑षां दे॒वाना॑माग्ने॒न्द्राः कृ॒ष्णल॑लामास्तूप॒राः॥४८॥

%5.6.18.0
{\anuvakamend[{इन्द्रा॑य॒ राज्ञे॒ द्वाविꣳ॑शतिः}]}%॥17॥

%5.6.18.1
अदि॑त्यै॒ त्रयो॑ रोहितै॒ता इ॑न्द्रा॒ण्यै त्रयः॑ कृष्णै॒ताः कु॒ह्वै᳚ त्रयो॑\-ऽरुणै॒तास्ति॒स्रो धे॒नवो॑ रा॒कायै॒ त्रयो॑\-ऽन॒ड्वाहः॑ सिनीवा॒ल्या आ᳚ग्नावैष्ण॒वा रोहि॑तललामास्तूप॒राः॥४९॥

%5.6.19.0
{\anuvakamend[{अदि॑त्या अ॒ष्टाद॑श}]}%॥18॥

%5.6.19.1
सौ॒म्यास्त्रयः॑ पि॒शङ्गाः॒ सोमा॑य॒ राज्ञे॒ त्रयः॑ सा॒रङ्गाः᳚ पार्ज॒न्या नभो॑रूपास्ति॒स्रो॑\-ऽजा म॒ल्\mbox{}हा इ॑न्द्रा॒ण्यै ति॒स्रो मे॒ष्य॑ आदि॒त्या द्या॑वापृथि॒व्या॑ मा॒लङ्गा᳚स्तूप॒राः॥५०॥

%5.6.20.0
{\anuvakamend[{सौ॒म्या एका॒न्नविꣳ॑शतिः}]}%॥19॥

%5.6.20.1
वा॒रु॒णास्त्रयः॑ कृ॒ष्णल॑लामा॒ वरु॑णाय॒ राज्ञे॒ त्रयो॒ रोहि॑तोललामा॒ वरु॑णाय रि॒शाद॑से॒ त्रयो॑\-ऽरु॒णल॑लामाः शि॒ल्पास्त्रयो॑ वैश्वदे॒वास्त्रयः॒ पृश्ञ॑यः सर्वदेव॒त्या॑ ऐन्द्रासू॒राः श्येत॑ललामास्तूप॒राः॥५१॥

%5.6.21.0
{\anuvakamend[{वा॒रु॒णा विꣳ॑श॒तिः}]}%॥20॥

%5.6.21.1
सोमा॑य स्व॒राज्ञे॑\-ऽनोवा॒हाव॑न॒ड्वाहा॑विन्द्रा॒ग्निभ्या॑मोजो॒दाभ्या॒मुष्टा॑राविन्द्रा॒ग्नि\-भ्यां᳚ बल॒दाभ्याꣳ॑ सीरवा॒हाववी॒ द्वे धे॒नू भौ॒मी दि॒ग्भ्यो वड॑बे॒ द्वे धे॒नू भौ॒मी वै॑रा॒जी पु॑रु॒षी द्वे धे॒नू भौ॒मी वा॒यव॑ आरोहणवा॒हाव॑न॒ड्वाहौ॑ वारु॒णी कृ॒ष्णे व॒शे अ॑रा॒ड्यौ॑ दि॒व्यावृ॑ष॒भौ प॑रिम॒रौ॥५२॥

%5.6.22.0
{\anuvakamend[{सोमा॑य स्व॒राज्ञे॒ चतु॑स्त्रिꣳशत्}]}%॥21॥

%5.6.22.1
एका॑दश प्रा॒तर्ग॒व्याः प॒शव॒ आ ल॑भ्यन्ते छग॒लः क॒ल्माषः॑ किकिदी॒विर्वि॑दी॒गय॒स्ते त्वा॒ष्ट्राः सौ॒रीर्नव॑ श्वे॒ता व॒शा अ॑नूब॒न्ध्या॑ भवन्त्याग्ने॒य ऐ᳚न्द्रा॒ग्न आ᳚श्वि॒नस्ते वि॑शालयू॒प आ ल॑भ्यन्ते॥५३॥

%5.6.23.0
{\anuvakamend[{एका॑दश॒ पञ्च॑विꣳशतिः}]}%॥22॥

%5.6.23.1
पि॒शङ्गा॒स्त्रयो॑ वास॒न्ताः सा॒रङ्गा॒स्त्रयो॒ ग्रैष्माः॒ पृष॑न्त॒स्त्रयो॒ वार्\mbox{}षि॑काः॒ पृश्ञ॑य॒स्त्रयः॑ शार॒दाः पृ॑श्ञिस॒क्थास्त्रयो॒ हैम॑न्तिका अवलि॒प्तास्त्रयः॑ शैशि॒राः सं॑वथ्स॒राय॒ निव॑क्षसः॥५४॥

%5.7.0.0

%5.7.0.0
{\anuvakamend[{पि॒शङ्गा॑ विꣳश॒तिः}]}%॥23॥

{\anuvakamend[{यो वा अय॑थादेवत॒न्त्वाम॑ग्न॒ इन्द्र॑स्य॒ चित्तिं॒ यथा॒ वै वयो॒ वै यदाकू॑ता॒द्यास्ते॑ अग्ने॒ मयि॑ गृह्णामि प्र॒जाप॑तिः॒ सो᳚\-ऽस्माथ्स्ते॒गान् वाजं॑ कू॒र्मान् योक्त्रं॑ मि॒त्रावरु॑णा॒विन्द्र॑स्य पू॒ष्ण ओज॑ आन॒न्दमह॑र॒ग्नेर्वा॒योः पन्था॒ङ्क्रमै॒र्द्यौस्ते॒\-ऽग्निः प॒शुरा॑सी॒थ्षड्विꣳ॑शतिः}]}%॥26॥
\prashnaend{यो वा ए॒वाहु॑तिमभवन्प॒थिभि॑रव॒रुध्या॑न॒न्दम॒ष्टौप॑ञ्चा॒शत्॥58॥ यो वा अय॑थादेवतं॒ यद्य॑व॒जिघ्र॑सि॥}
%%% END PRASHNA

\sect{सप्तमः प्रश्नः}\setcounter{anuvakam}{0}
\dnsub{तैत्तिरीयसंहितायां पञ्चमकाण्डे सप्तमः प्रश्नः}
%5.7.1.0
%5.7.1.1
यो वा अय॑थादेवतम॒ग्निं चि॑नु॒त आ दे॒वता᳚भ्यो वृश्च्यते॒ पापी॑यान्भवति॒ यो य॑थादेव॒तं न दे॒वता᳚भ्य॒ आ वृ॑श्च्यते॒ वसी॑यान्भवत्याग्ने॒य्या गा॑यत्रि॒या प्र॑थ॒मां चिति॑म॒भि मृ॑शेत्त्रि॒ष्टुभा᳚ द्वि॒तीयां॒ जग॑त्या तृ॒तीया॑मनु॒ष्टुभा॑ चतु॒र्थीम्प॒ङ्क्त्या प॑ञ्च॒मीं य॑थादेव॒तमे॒वाग्निं चि॑नुते॒ न दे॒वता᳚भ्य॒ आ वृ॑श्च्यते॒ वसी॑यान्भव॒तीडा॑यै॒ वा ए॒षा विभ॑क्तिः प॒शव॒ इडा॑ प॒शुभि॑रेनम्॥१॥

%5.7.1.2
चि॒नु॒ते॒ यो वै प्र॒जाप॑तये प्रति॒प्रोच्या॒ग्निं चि॒नोति॒ नार्ति॒मार्च्छ॒त्यश्वा॑व॒भित॑स्तिष्ठेतां कृ॒ष्ण उ॑त्तर॒तः श्वे॒तो दक्षि॑ण॒\-स्तावा॒लभ्येष्ट॑का॒ उप॑ दध्यादे॒तद्वै प्र॒जाप॑ते रू॒पम्प्रा॑जाप॒त्यो\-ऽश्वः॑ सा॒क्षादे॒व प्र॒जाप॑तये प्रति॒प्रोच्या॒ग्निं चि॑नोति॒ नार्ति॒मार्च्छ॑त्ये॒तद्वा अह्नो॑ रू॒पं यच्छ्वे॒तो\-ऽश्वो॒ रात्रि॑यै कृ॒ष्ण ए॒तदह्नः॑॥२॥

%5.7.1.3
रू॒पं यदिष्ट॑का॒ रात्रि॑यै॒ पुरी॑ष॒मिष्ट॑का उपधा॒स्यञ्छ्वे॒तमश्व॑म॒भि मृ॑शे॒त्पुरी॑षमुपधा॒स्यन्कृ॒ष्णम॑होरा॒त्राभ्या॑मे॒वैनं॑ चिनुते हिरण्यपा॒त्रम्मधोः᳚ पू॒र्णं द॑दाति मध॒व्यो॑\-ऽसा॒नीति॑ सौ॒र्या चि॒त्रव॒त्यावे᳚क्षते चि॒त्रमे॒व भ॑वति म॒ध्यन्दि॒ने\-ऽश्व॒मव॑ घ्रापयत्य॒सौ वा आ॑दि॒त्य इन्द्र॑ ए॒ष प्र॒जाप॑तिः प्राजाप॒त्यो\-ऽश्व॒स्तमे॒व सा॒क्षादृ॑ध्नोति॥३॥

%5.7.2.0
{\anuvakamend[{ए॒नमे॒तदह्नो॒\-ऽष्टाच॑त्वारिꣳशच्च}]}%॥१॥

%5.7.2.1
त्वाम॑ग्ने वृष॒भं चेकि॑तान॒म्पुन॒र्युवा॑नञ्ज॒नय॑न्नु॒पागा᳚म्। अ॒स्थू॒रि णो॒ गार्\mbox{}ह॑पत्यानि सन्तु ति॒ग्मेन॑ नो॒ ब्रह्म॑णा॒ सꣳ शि॑शाधि। प॒शवो॒ वा ए॒ते यदिष्ट॑का॒श्चित्यां᳚चित्यामृष॒भमुप॑ दधाति मिथु॒नमे॒वास्य॒ तद्य॒ज्ञे क॑रोति प्र॒जन॑नाय॒ तस्मा᳚द्यू॒थेयू॑थ ऋष॒भः। सं॒व॒थ्स॒रस्य॑ प्रति॒मां यां त्वा॑ रात्र्यु॒पास॑ते। प्र॒जाꣳ सु॒वीरां᳚ कृ॒त्वा विश्व॒मायु॒र्व्य॑श्ञवत्। प्रा॒जा॒प॒त्याम्॥४॥

%5.7.2.2
ए॒तामुप॑ दधाती॒यं वावैषैका᳚ष्ट॒का यदे॒वैका᳚ष्ट॒काया॒मन्नं॑ क्रि॒यते॒ तदे॒वैतयाव॑ रुन्द्ध ए॒षा वै प्र॒जाप॑तेः काम॒दुघा॒ तयै॒व यज॑मानो॒\-ऽमुष्मि॑ल्लोँ॒के᳚\-ऽग्निं दु॑हे॒ येन॑ दे॒वा ज्योति॑षो॒र्ध्वा उ॒दाय॒न् येना॑दि॒त्या वस॑वो॒ येन॑ रु॒द्राः। येनाङ्गि॑रसो महि॒मान॑मान॒शुस्तेनै॑तु॒ यज॑मानः स्व॒स्ति। सु॒व॒र्गाय॒ वा ए॒ष लो॒काय॑॥५॥

%5.7.2.3
ची॒य॒ते॒ यद॒ग्निर्येन॑ दे॒वा ज्योति॑षो॒र्ध्वा उ॒दाय॒न्नित्युख्य॒ꣳ॒ समि॑न्द्ध॒ इष्ट॑का ए॒वैता उप॑ धत्ते वानस्प॒त्याः सु॑व॒र्गस्य॑ लो॒कस्य॒ सम॑ष्ट्यै श॒तायु॑धाय श॒तवी᳚र्याय श॒तोत॑ये\-ऽभिमाति॒षाहे᳚। श॒तं यो नः॑ श॒रदो॒ अजी॑ता॒निन्द्रो॑ नेष॒दति॑ दुरि॒तानि॒ विश्वा᳚। ये च॒त्वारः॑ प॒थयो॑ देव॒याना॑ अन्त॒रा द्यावा॑पृथि॒वी वि॒यन्ति॑। तेषां॒ यो अज्या॑नि॒मजी॑तिमा॒ वहा॒त्तस्मै॑ नो देवाः॥६॥

%5.7.2.4
परि॑ दत्ते॒ह सर्वे᳚। ग्री॒ष्मो हे॑म॒न्त उ॒त नो॑ वस॒न्तः श॒रद्व॒र्\mbox{}षाः सु॑वि॒तं नो॑ अस्तु। तेषा॑मृतू॒नाꣳ श॒तशा॑रदानां निवा॒त ए॑षा॒मभ॑ये स्याम। इ॒दु॒व॒थ्स॒राय॑ परिवथ्स॒राय॑ संवथ्स॒राय॑ कृणुता बृ॒हन्नमः॑। तेषां᳚ व॒यꣳ सु॑म॒तौ य॒ज्ञिया॑नां॒ ज्योगजी॑ता॒ अह॑ताः स्याम। भ॒द्रान्नः॒ श्रेयः॒ सम॑नैष्ट देवा॒स्त्वया॑व॒सेन॒ सम॑शीमहि त्वा। स नो॑ मयो॒भूः पि॑तो॥७॥

%5.7.2.5
आ वि॑शस्व॒ शं तो॒काय॑ त॒नुवे᳚ स्यो॒नः। अज्या॑नीरे॒ता उप॑ दधात्ये॒ता वै दे॒वता॒ अप॑राजिता॒स्ता ए॒व प्र वि॑शति॒ नैव जी॑यते ब्रह्मवा॒दिनो॑ वदन्ति॒ यद॑र्धमा॒सा मासा॑ ऋ॒तवः॑ संवथ्स॒र ओष॑धीः॒ पच॒न्त्यथ॒ कस्मा॑द॒न्याभ्यो॑ दे॒वता᳚भ्य आग्रय॒णं निरु॑प्यत॒ इत्ये॒ता हि तद्दे॒वता॑ उ॒दज॑य॒न् यदृ॒तुभ्यो॑ नि॒र्वपे᳚द्दे॒वता᳚भ्यः स॒मदं॑ दध्यादाग्रय॒णं नि॒रुप्यै॒ता आहु॑तीर्जुहोत्यर्धमा॒साने॒व मासा॑नृ॒तून्थ्सं॑वथ्स॒रम्प्री॑णाति॒ न दे॒वता᳚भ्यः स॒मद॑न्दधाति भ॒द्रान्नः॒ श्रेयः॒ सम॑नैष्ट देवा॒ इत्या॑ह हु॒ताद्या॑य॒ यज॑मान॒स्याप॑राभावाय॥८॥

%5.7.3.0
{\anuvakamend[{प्रा॒जा॒प॒त्याल्लोँ॒काय॑ देवाः पितो दध्यादाग्रय॒णं पञ्च॑विꣳशतिश्च}]}%॥२॥

%5.7.3.1
इन्द्र॑स्य॒ वज्रो॑\-ऽसि॒ वार्त्र॑घ्नस्तनू॒पा नः॑ प्रतिस्प॒शः। यो नः॑ पु॒रस्ता᳚द्दक्षिण॒तः प॒श्चादु॑त्तर॒तो॑\-ऽघा॒युर॑भि॒दास॑त्ये॒तꣳ सो\-ऽश्मा॑नमृच्छतु। दे॒वा॒सु॒राः संय॑त्ता आस॒न्ते\-ऽसु॑रा दि॒ग्भ्य आबा॑धन्त॒ तां दे॒वा इष्वा॑ च॒ वज्रे॑ण॒ चापा॑नुदन्त॒ यद्व॒ज्रिणी॑रुप॒दधा॒तीष्वा॑ चै॒व तद्वज्रे॑ण च॒ यज॑मानो॒ भ्रातृ॑व्या॒नप॑ नुदते दि॒क्षूप॑॥९॥

%5.7.3.2
द॒धा॒ति॒ दे॒व॒पु॒रा ए॒वैतास्त॑नू॒पानीः॒ पर्यू॑ह॒ते\-ऽग्ना॑विष्णू स॒जोष॑से॒मा व॑र्धन्तु वां॒ गिरः॑। द्यु॒म्नैर्वाजे॑भि॒रा ग॑तम्। ब्र॒ह्म॒वा॒दिनो॑ वदन्ति॒ यन्न दे॒वता॑यै॒ जुह्व॒त्यथ॑ किन्देव॒त्या॑ वसो॒र्धारेत्य॒ग्निर्वसु॒स्तस्यै॒षा धारा॒ विष्णु॒र्वसु॒स्तस्यै॒षा धारा᳚ग्नावैष्ण॒व्यर्चा वसो॒र्धारां᳚ जुहोति भाग॒धेये॑नै॒वैनौ॒ सम॑र्धय॒त्यथो॑ ए॒ताम्॥१०॥

%5.7.3.3
ए॒वाहु॑तिमा॒यत॑नवतीं करोति॒ यत्का॑म एनां जु॒होति॒ तदे॒वाव॑ रुन्द्धे रु॒द्रो वा ए॒ष यद॒ग्निस्तस्यै॒ते त॒नुवौ॑ घो॒रान्या शि॒वान्या यच्छ॑तरु॒द्रीयं॑ जु॒होति॒ यैवास्य॑ घो॒रा त॒नूस्तां तेन॑ शमयति॒ यद्वसो॒र्धारां᳚ जु॒होति॒ यैवास्य॑ शि॒वा त॒नूस्तां तेन॑ प्रीणाति॒ यो वै वसो॒र्धारा॑यै॥११॥

%5.7.3.4
प्र॒ति॒ष्ठां वेद॒ प्रत्ये॒व ति॑ष्ठति॒ यदाज्य॑मु॒च्छिष्ये॑त॒ तस्मि॑न्ब्रह्मौद॒नम्प॑चे॒त्तम्ब्रा᳚ह्म॒णाश्च॒त्वारः॒ प्राश्ञी॑युरे॒ष वा अ॒ग्निर्वै᳚श्वान॒रो यद्ब्रा᳚ह्म॒ण ए॒षा खलु॒ वा अ॒ग्नेः प्रि॒या त॒नूर्यद्वै᳚श्वान॒रः प्रि॒याया॑मे॒वैनां᳚ त॒नुवां॒ प्रति॑ ष्ठापयति॒ चत॑स्रो धे॒नूर्द॑द्या॒त्ताभि॑रे॒व यज॑मानो॒\-ऽमुष्मि॑ल्लोँ॒के᳚\-ऽग्निं दु॑हे॥१२॥

%5.7.4.0
{\anuvakamend[{उपै॒तान्धारा॑यै॒ षट्च॑त्वारिꣳशच्च}]}%॥३॥

%5.7.4.1
चित्ति॑ञ्जुहोमि॒ मन॑सा घृ॒तेनेत्या॒हादा᳚भ्या॒ वै नामै॒षाहु॑तिर्वैश्वकर्म॒णी नैनं॑ चिक्या॒नम्भ्रातृ॑व्यो दभ्नो॒त्यथो॑ दे॒वता॑ ए॒वाव॑ रु॒न्द्धे\-ऽग्ने॒ तम॒द्येति॑ प॒ङ्क्त्या जु॑होति प॒ङ्क्त्याहु॑त्या यज्ञमु॒खमार॑भते स॒प्त ते॑ अग्ने स॒मिधः॑ स॒प्त जि॒ह्वा इत्या॑ह॒ होत्रा॑ ए॒वाव॑ रुन्द्धे॒\-ऽग्निर्दे॒वेभ्यो\-ऽपा᳚क्रामद्भाग॒धेयम्᳚॥१३॥

%5.7.4.2
इ॒च्छमा॑न॒स्तस्मा॑ ए॒तद्भा॑ग॒धेय॒म्प्राय॑च्छन्ने॒तद्वा अ॒ग्नेर॑ग्निहो॒त्रमे॒तर्\mbox{}हि॒ खलु॒ वा ए॒ष जा॒तो यर्\mbox{}हि॒ सर्व॑श्चि॒तो जा॒तायै॒वास्मा॒ अन्न॒मपि॑ दधाति॒ स ए॑नम्प्री॒तः प्री॑णाति॒ वसी॑यान्भवति ब्रह्मवा॒दिनो॑ वदन्ति॒ यदे॒ष गार्\mbox{}ह॑पत्यश्ची॒यते\-ऽथ॒ क्वा᳚स्याहव॒नीय॒ इत्य॒सावा॑दि॒त्य इति॑ ब्रूयादे॒तस्मि॒न् हि सर्वा᳚भ्यो दे॒वता᳚भ्यो॒ जुह्व॑ति॥१४॥

%5.7.4.3
य ए॒वं वि॒द्वान॒ग्निं चि॑नु॒ते सा॒क्षादे॒व दे॒वता॑ ऋध्नो॒त्यग्ने॑ यशस्वि॒न् यश॑से॒मम॑र्प॒येन्द्रा॑वती॒मप॑चितीमि॒हा व॑ह। अ॒यम्मू॒र्धा प॑रमे॒ष्ठी सु॒वर्चाः᳚ समा॒नाना॑मुत्त॒मश्लो॑को अस्तु। भ॒द्रम्पश्य॑न्त॒ उप॑ सेदु॒रग्रे॒ तपो॑ दी॒क्षामृष॑यः सुव॒र्विदः॑। ततः॑ क्ष॒त्रम्बल॒मोज॑श्च जा॒तं तद॒स्मै दे॒वा अ॒भि सं न॑मन्तु। धा॒ता वि॑धा॒ता प॑र॒मा॥१५॥

%5.7.4.4
उ॒त सं॒दृक्प्र॒जाप॑तिः परमे॒ष्ठी वि॒राजा᳚। स्तोमा॒श्छन्दाꣳ॑सि नि॒विदो॑ म आहुरे॒तस्मै॑ रा॒ष्ट्रम॒भि सं न॑माम। अ॒भ्याव॑र्तध्व॒मुप॒ मेत॑ सा॒कम॒यꣳ शा॒स्ताधि॑पतिर्वो अस्तु। अ॒स्य वि॒ज्ञान॒मनु॒ सꣳ र॑भध्वमि॒मम्प॒श्चादनु॑ जीवाथ॒ सर्वे᳚। रा॒ष्ट्रभृत॑ ए॒ता उप॑ दधात्ये॒षा वा अ॒ग्नेश्चिती॑ राष्ट्र॒भृत्तयै॒वास्मि॑न्रा॒ष्ट्रं द॑धाति रा॒ष्ट्रमे॒व भ॑वति॒ नास्मा᳚द्रा॒ष्ट्रम्भ्रꣳ॑शते॥१६॥

%5.7.5.0
{\anuvakamend[{भा॒ग॒धेय॒ञ्जुह्व॑ति पर॒मा रा॒ष्ट्रन्द॑धाति स॒प्त च॑}]}%॥४॥

%5.7.5.1
यथा॒ वै पु॒त्रो जा॒तो म्रि॒यत॑ ए॒वं वा ए॒ष म्रि॑यते॒ यस्या॒ग्निरुख्य॑ उ॒द्वाय॑ति॒ यन्नि॑र्म॒न्थ्यं॑ कु॒र्याद्विच्छि॑न्द्या॒द्भ्रातृ॑व्यमस्मै जनये॒थ्स ए॒व पुनः॑ प॒रीध्यः॒ स्वादे॒वैनं॒ योने᳚र्जनयति॒ नास्मै॒ भ्रातृ॑व्यं जनयति॒ तमो॒ वा ए॒तं गृ॑ह्णाति॒ यस्या॒ग्निरुख्य॑ उ॒द्वाय॑ति मृ॒त्युस्तमः॑ कृ॒ष्णं वासः॑ कृ॒ष्णा धे॒नुर्दक्षि॑णा॒ तम॑सा॥१७॥

%5.7.5.2
ए॒व तमो॑ मृ॒त्युमप॑ हते॒ हिर॑ण्यं ददाति॒ ज्योति॒र्वै हिर॑ण्यं॒ ज्योति॑षै॒व तमो\-ऽप॑ ह॒ते\-ऽथो॒ तेजो॒ वै हिर॑ण्य॒न्तेज॑ ए॒वात्मन्ध॑त्ते॒ सुव॒र्न घ॒र्मः स्वाहा॒ सुव॒र्नार्कः स्वाहा॒ सुव॒र्न शु॒क्रः स्वाहा॒ सुव॒र्न ज्योतिः॒ स्वाहा॒ सुव॒र्न सूर्यः॒ स्वाहा॒र्को वा ए॒ष यद॒ग्निर॒सावा॑दि॒त्यः॥१८॥

%5.7.5.3
अ॒श्व॒मे॒धो यदे॒ता आहु॑तीर्जु॒होत्य॑र्काश्वमे॒धयो॑रे॒व ज्योतीꣳ॑षि॒ सं द॑धात्ये॒ष ह॒ त्वा अ॑र्काश्वमे॒धी यस्यै॒तद॒ग्नौ क्रि॒यत॒ आपो॒ वा इ॒दमग्रे॑ सलि॒लमा॑सी॒थ्स ए॒तां प्र॒जाप॑तिः प्रथ॒मां चिति॑मपश्य॒त्तामुपा॑धत्त॒ तदि॒यम॑भव॒त्तं वि॒श्वक॑र्माब्रवी॒दुप॒ त्वाया॒नीति॒ नेह लो॒को᳚\-ऽस्तीति॑॥१९॥

%5.7.5.4
अ॒ब्र॒वी॒थ्स ए॒तां द्वि॒तीयां॒ चिति॑मपश्य॒त्तामुपा॑धत्त॒ तद॒न्तरि॑क्षमभव॒थ्स य॒ज्ञः प्र॒जाप॑तिमब्रवी॒दुप॒ त्वाया॒नीति॒ नेह लो॒को᳚\-ऽस्तीत्य॑ब्रवी॒थ्स वि॒श्वक॑र्माणमब्रवी॒दुप॒ त्वाया॒नीति॒ केन॑ मो॒पैष्य॒सीति॒ दिश्या॑भि॒रित्य॑ब्रवी॒त्तन्दिश्या॑भिरु॒पैत्ता उपा॑धत्त॒ ता दिशः॑॥२०॥

%5.7.5.5
अ॒भ॒व॒न्थ्स प॑रमे॒ष्ठी प्र॒जाप॑तिमब्रवी॒दुप॒ त्वाया॒नीति॒ नेह लो॒को᳚\-ऽस्तीत्य॑ब्रवी॒थ्स वि॒श्वक॑र्माणं च य॒ज्ञं चा᳚ब्रवी॒दुप॑ वा॒माया॒नीति॒ नेह लो॒को᳚\-ऽस्तीत्य॑ब्रूता॒ꣳ॒ स ए॒तां तृ॒तीयां॒ चिति॑मपश्य॒त्तामुपा॑धत्त॒ तद॒साव॑भव॒थ्स आ॑दि॒त्यः प्र॒जाप॑तिमब्रवी॒दुप॑ त्वा॥२१॥

%5.7.5.6
आ॒या॒नीति॒ नेह लो॒को᳚\-ऽस्तीत्य॑ब्रवी॒थ्स वि॒श्वक॑र्माणं च य॒ज्ञं चा᳚ब्रवी॒दुप॑ वा॒माया॒नीति॒ नेह लो॒को᳚\-ऽस्तीत्य॑ब्रूता॒ꣳ॒ स प॑रमे॒ष्ठिन॑मब्रवी॒दुप॒ त्वाया॒नीति॒ केन॑ मो॒पैष्य॒सीति॑ लोकं पृ॒णयेत्य॑ब्रवी॒त्तं लो॑कं पृ॒णयो॒पैत्तस्मा॒दया॑तयाम्नी लोकं पृ॒णा\-ऽया॑तयामा॒ ह्य॑सौ॥२२॥

%5.7.5.7
आ॒दि॒त्यस्तानृष॑यो\-ऽब्रुव॒न्नुप॑ व॒ आया॒मेति॒ केन॑ न उ॒पैष्य॒थेति॑ भू॒म्नेत्य॑ब्रुव॒न्तान्द्वाभ्यां॒ चिती᳚भ्यामु॒पाय॒न्थ्स पञ्च॑चितीकः॒ सम॑पद्यत॒ य ए॒वं वि॒द्वान॒ग्निं चि॑नु॒ते भूया॑ने॒व भ॑वत्य॒भीमाल्लोँ॒काञ्ज॑यति वि॒दुरे॑नं दे॒वा अथो॑ ए॒तासा॑मे॒व दे॒वता॑ना॒ꣳ॒ सायु॑ज्यं गच्छति॥२३॥

%5.7.6.0
{\anuvakamend[{तम॑सा\-ऽ\-ऽदि॒त्यो᳚\-ऽस्तीति॒ दिश॑ आदि॒त्यः प्र॒जाप॑तिमब्रवी॒दुप॑ त्वा॒\-ऽसौ पञ्च॑चत्वारिꣳशच्च}]}%॥६॥

%5.7.6.1
वयो॒ वा अ॒ग्निर्यद॑ग्नि॒चित्प॒क्षिणो᳚\-ऽश्ञी॒यात्तमे॒वाग्निम॑द्या॒दार्ति॒मार्च्छे᳚थ्संवथ्स॒रं व्र॒तं च॑रेथ्संवथ्स॒रꣳ हि व्र॒तं नाति॑ प॒शुर्वा ए॒ष यद॒ग्निर्\mbox{}हि॒नस्ति॒ खलु॒ वै तम्प॒शुर्य ए॑नम्पु॒रस्ता᳚त्प्र॒त्यञ्च॑मुप॒चर॑ति॒ तस्मा᳚त्प॒श्चात्प्राङु॑प॒चर्य॑ आ॒त्मनो\-ऽहिꣳ॑सायै॒ तेजो॑\-ऽसि॒ तेजो॑ मे यच्छ पृथि॒वीं य॑च्छ॥२४॥

%5.7.6.2
पृ॒थि॒व्यै मा॑ पाहि॒ ज्योति॑रसि॒ ज्योति॑र्मे यच्छा॒न्तरि॑क्षं यच्छा॒न्तरि॑क्षान्मा पाहि॒ सुव॑रसि॒ सुव॑र्मे यच्छ॒ दिवं॑ यच्छ दि॒वो मा॑ पा॒हीत्या॑है॒ताभि॒र्वा इ॒मे लो॒का विधृ॑ता॒ यदे॒ता उ॑प॒दधा᳚त्ये॒षां लो॒कानां॒ विधृ॑त्यै स्वयमातृ॒ण्णा उ॑प॒धाय॑ हिरण्येष्ट॒का उप॑ दधाती॒मे वै लो॒काः स्व॑यमातृ॒ण्णा ज्योति॒र्\mbox{}हिर॑ण्यं॒ यथ्स्व॑यमातृ॒ण्णा उ॑प॒धाय॑॥२५॥

%5.7.6.3
हि॒र॒ण्ये॒ष्ट॒का उ॑प॒दधा॑ती॒माने॒वैताभि॑र्लो॒कां ज्योति॑ष्मतः कुरु॒ते\-ऽथो॑ ए॒ताभि॑रे॒वास्मा॑ इ॒मे लो॒काः प्र भा᳚न्ति॒ यास्ते॑ अग्ने॒ सूर्ये॒ रुच॑ उद्य॒तो दिव॑मात॒न्वन्ति॑ र॒श्मिभिः॑। ताभिः॒ सर्वा॑भी रु॒चे जना॑य नस्कृधि। या वो॑ देवाः॒ सूर्ये॒ रुचो॒ गोष्वश्वे॑षु॒ या रुचः॑। इन्द्रा᳚ग्नी॒ ताभिः॒ सर्वा॑भी॒ रुचं॑ नो धत्त बृहस्पते। रुचं॑ नो धेहि॥२६॥

%5.7.6.4
ब्रा॒ह्म॒णेषु॒ रुच॒ꣳ॒ राज॑सु नस्कृधि। रुचं॑ वि॒श्ये॑षु शू॒द्रेषु॒ मयि॑ धेहि रु॒चा रुचम्᳚। द्वे॒धा वा अ॒ग्निं चि॑क्या॒नस्य॒ यश॑ इन्द्रि॒यं ग॑च्छत्य॒ग्निं वा॑ चि॒तमी॑जा॒नं वा॒ यदे॒ता आहु॑तीर्जु॒होत्या॒त्मन्ने॒व यश॑ इन्द्रि॒यं ध॑त्त ईश्व॒रो वा ए॒ष आर्ति॒मार्तो॒र्यो᳚\-ऽग्निं चि॒न्वन्न॑धि॒क्राम॑ति॒ तत्त्वा॑ यामि॒ ब्रह्म॑णा॒ वन्द॑मान॒ इति॑ वारु॒ण्यर्चा॥२७॥

%5.7.6.5
जु॒हु॒या॒च्छान्ति॑रे॒वैषाग्नेर्गुप्ति॑रा॒त्मनो॑ ह॒विष्कृ॑तो॒ वा ए॒ष यो᳚\-ऽग्निं चि॑नु॒ते यथा॒ वै ह॒विः स्कन्द॑त्ये॒वं वा ए॒ष स्क॑न्दति॒ यो᳚\-ऽग्निं चि॒त्वा स्त्रिय॑मु॒पैति॑ मैत्रावरु॒ण्यामिक्ष॑या यजेत मैत्रावरु॒णता॑मे॒वोपै᳚त्या॒त्मनो\-ऽस्क॑न्दाय॒ यो वा अ॒ग्निमृ॑तु॒स्थां वेद॒र्तुर्\mbox{}ऋ॑तुरस्मै॒ कल्प॑मान एति॒ प्रत्ये॒व ति॑ष्ठति संवथ्स॒रो वा अ॒ग्निः॥२८॥

%5.7.6.6
ऋ॒तु॒स्थास्तस्य॑ वस॑न्तः॒ शिरो᳚ ग्री॒ष्मो दक्षि॑णः प॒क्षो व॒र्\mbox{}षाः पुच्छꣳ॑ श॒रदुत्त॑रः प॒क्षो हे॑म॒न्तो मध्य॑म्पूर्वप॒क्षाश्चित॑यो\-ऽपरप॒क्षाः पुरी॑षमहोरा॒त्राणीष्ट॑का ए॒ष वा अ॒ग्निर्\mbox{}ऋ॑तु॒स्था य ए॒वं वेद॒र्तुर्\mbox{}ऋ॑तुरस्मै॒ कल्प॑मान एति॒ प्रत्ये॒व ति॑ष्ठति प्र॒जाप॑ति॒र्वा ए॒तं ज्यैष्ठ्य॑कामो॒ न्य॑धत्त॒ ततो॒ वै स ज्यैष्ठ्य॑मगच्छ॒द्य ए॒वं वि॒द्वान॒ग्निं चि॑नु॒ते ज्यैष्ठ्य॑मे॒व ग॑च्छति॥२९॥

%5.7.7.0
{\anuvakamend[{पृ॒थि॒वीं य॑च्छ॒ यथ्स्व॑यमातृ॒ण्णा उ॑प॒धाय॑ धेह्यृ॒चाग्निश्चि॑नु॒ते त्रीणि॑ च}]}%॥७॥

%5.7.7.1
यदाकू॑ताथ्स॒मसु॑स्रोद्धृ॒दो वा॒ मन॑सो वा॒ सम्भृ॑तं॒ चक्षु॑षो वा। तमनु॒ प्रेहि॑ सुकृ॒तस्य॑ लो॒कं यत्रर्\mbox{}ष॑यः प्रथम॒जा ये पु॑रा॒णाः। ए॒तꣳ स॑धस्थ॒ परि॑ ते ददामि॒ यमा॒वहा᳚च्छेव॒धिं जा॒तवे॑दाः। अ॒न्वा॒ग॒न्ता य॒ज्ञप॑तिर्वो॒ अत्र॒ तꣴ स्म॑ जानीत पर॒मे व्यो॑मन्न्। जा॒नी॒तादे॑नं पर॒मे व्यो॑म॒न्देवाः᳚ सधस्था वि॒द रू॒पम॑स्य। यदा॒गच्छा᳚त्॥३०॥

%5.7.7.2
प॒थिभि॑र्देव॒यानै॑रिष्टापू॒र्ते कृ॑णुतादा॒विर॑स्मै। सम्प्र च्य॑वध्व॒मनु॒ सम्प्र या॒ताग्ने॑ प॒थो दे॑व॒याना᳚न्कृणुध्वम्। अ॒स्मिन्थ्स॒धस्थे॒ अध्युत्त॑रस्मि॒न्विश्वे॑ देवा॒ यज॑मानश्च सीदत। प्र॒स्त॒रेण॑ परि॒धिना᳚ स्रु॒चा वेद्या॑ च ब॒र्\mbox{}हिषा᳚। ऋ॒चेमं य॒ज्ञं नो॑ वह॒ सुव॑र्दे॒वेषु॒ गन्त॑वे। यदि॒ष्टं यत्प॑रा॒दानं॒ यद्द॒त्तं या च॒ दक्षि॑णा। तत्॥३१॥

%5.7.7.3
अ॒ग्निर्वै᳚श्वकर्म॒णः सुव॑र्दे॒वेषु॑ नो दधत्। येना॑ स॒हस्रं॒ वह॑सि॒ येना᳚ग्ने सर्ववेद॒सम्। तेने॒मं य॒ज्ञं नो॑ वह॒ सुव॑र्दे॒वेषु॒ गन्त॑वे। येना᳚ग्ने॒ दक्षि॑णा यु॒क्ता य॒ज्ञं वह॑न्त्यृ॒त्विजः॑। तेने॒मं य॒ज्ञं नो॑ वह॒ सुव॑र्दे॒वेषु॒ गन्त॑वे। येना᳚ग्ने सु॒कृतः॑ प॒था मधो॒र्धारा᳚ व्यान॒शुः। तेने॒मं य॒ज्ञं नो॑ वह॒ सुव॑र्दे॒वेषु॒ गन्त॑वे। यत्र॒ धारा॒ अन॑पेता॒ मधो᳚र्घृ॒तस्य॑ च॒ याः। तद॒ग्निर्वै᳚श्वकर्म॒णः सुव॑र्दे॒वेषु॑ नो दधत्॥३२॥

%5.7.8.0
{\anuvakamend[{आ॒गच्छा॒त्तद्व्या॑न॒शुस्तेने॒मं य॒ज्ञं नो॑ वह॒ सुव॑र्दे॒वेषु॒ गन्त॑वे॒ चतु॑र्दश च}]}%॥७॥

%5.7.8.1
यास्ते॑ अग्ने स॒मिधो॒ यानि॒ धाम॒ या जि॒ह्वा जा॑तवेदो॒ यो अ॒र्चिः। ये ते॑ अग्ने मे॒डयो॒ य इन्द॑व॒स्तेभि॑रा॒त्मानं॑ चिनुहि प्रजा॒नन्न्। उ॒थ्स॒न्न॒य॒ज्ञो वा ए॒ष यद॒ग्निः किं वाहै॒तस्य॑ क्रि॒यते॒ किं वा॒ न यद्वा अ॑ध्व॒र्युर॒ग्नेश्चि॒न्वन्न॑न्त॒रेत्या॒त्मनो॒ वै तद॒न्तरे॑ति॒ यास्ते॑ अग्ने स॒मिधो॒ यानि॑॥३३॥

%5.7.8.2
धामेत्या॑है॒षा वा अ॒ग्नेः स्व॑यञ्चि॒तिर॒ग्निरे॒व तद॒ग्निं चि॑नोति॒ नाध्व॒र्युरा॒त्मनो॒\-ऽन्तरे॑ति॒ चत॑स्र॒ आशाः॒ प्र च॑रन्त्व॒ग्नय॑ इ॒मं नो॑ य॒ज्ञं न॑यतु प्रजा॒नन्न्। घृ॒तम्पिन्व॑न्न॒जरꣳ॑ सु॒वीरं॒ ब्रह्म॑ स॒मिद्भ॑व॒त्याहु॑तीनाम्। सु॒व॒र्गाय॒ वा ए॒ष लो॒कायोप॑ धीयते॒ यत्कू॒र्मश्चत॑स्र॒ आशाः॒ प्र च॑रन्त्व॒ग्नय॒ इत्या॑ह॥३४॥

%5.7.8.3
दिश॑ ए॒वैतेन॒ प्र जा॑नाती॒मं नो॑ य॒ज्ञं न॑यतु प्रजा॒नन्नित्या॑ह सुव॒र्गस्य॑ लो॒कस्या॒भ᳚नी॑त्यै॒ ब्रह्म॑ स॒मिद्भ॑व॒त्याहु॑तीना॒मित्या॑ह॒ ब्रह्म॑णा॒ वै दे॒वाः सु॑व॒र्गं लो॒कमा॑य॒न् यद्ब्रह्म॑ण्वत्योप॒दधा॑ति॒ ब्रह्म॑णै॒व तद्यज॑मानः सुव॒र्गं लो॒कमे॑ति प्र॒जाप॑ति॒र्वा ए॒ष यद॒ग्निस्तस्य॑ प्र॒जाः प॒शव॒श्छन्दाꣳ॑सि रू॒पꣳ सर्वा॒न् वर्णा॒निष्ट॑कानां कुर्याद्रू॒पेणै॒व प्र॒जां प॒शूञ्छन्दा॒ꣳ॒स्यव॑ रु॒न्द्धे\-ऽथो᳚ प्र॒जाभ्य॑ ए॒वैन॑म्प॒शुभ्य॒श्छन्दो᳚भ्यो\-ऽव॒रुद्ध्य॑ चिनुते॥३५॥

%5.7.9.0
{\anuvakamend[{यान्य॒ग्नय॒ इत्या॒हेष्ट॑काना॒ꣳ॒ षोड॑श च}]}%॥८॥

%5.7.9.1
मयि॑ गृह्णा॒म्यग्रे॑ अ॒ग्निꣳ रा॒यस्पोषा॑य सुप्रजा॒स्त्वाय॑ सु॒वीर्या॑य। मयि॑ प्र॒जाम्मयि॒ वर्चो॑ दधा॒म्यरि॑ष्टाः स्याम त॒नुवा॑ सु॒वीराः᳚। यो नो॑ अ॒ग्निः पि॑तरो हृ॒थ्स्व॑न्तरम॑र्त्यो॒ मर्त्याꣳ॑ आवि॒वेश॑। तमा॒त्मन्परि॑ गृह्णीमहे व॒यं मा सो अ॒स्माꣳ अ॑व॒हाय॒ परा॑ गात्। यद॑ध्व॒र्युरा॒त्मन्न॒ग्निमगृ॑हीत्वा॒ग्निं चि॑नु॒याद्यो᳚\-ऽस्य॒ स्वो᳚\-ऽग्निस्तमपि॑॥३६॥

%5.7.9.2
यज॑मानाय चिनुयाद॒ग्निं खलु॒ वै प॒शवो\-ऽनूप॑ तिष्ठन्ते\-ऽप॒क्रामु॑का अस्मात्प॒शवः॑ स्यु॒र्मयि॑ गृह्णाम्यग्रे॑ अ॒ग्निमित्या॑हा॒त्मन्ने॒व स्वम॒ग्निं दा॑धार॒ नास्मा᳚त्प॒शवो\-ऽप॑ क्रामन्ति ब्रह्मवा॒दिनो॑ वदन्ति॒ यन्मृच्चाप॑श्चा॒ग्नेर॑ना॒द्यमथ॒ कस्मा᳚न्मृ॒दा चा॒द्भिश्चा॒ग्निश्ची॑यत॒ इति॒ यद॒द्भिः सं॒यौति॑॥३७॥

%5.7.9.3
आपो॒ वै सर्वा॑ दे॒वता॑ दे॒वता॑भिरे॒वैन॒ꣳ॒ सꣳ सृ॑जति॒ यन्मृ॒दा चि॒नोती॒यं वा अ॒ग्निर्वै᳚श्वान॒रो᳚\-ऽग्निनै॒व तद॒ग्निं चि॑नोति ब्रह्मवा॒दिनो॑ वदन्ति॒ यन्मृ॒दा चा॒द्भिश्चा॒ग्निश्ची॒यते\-ऽथ॒ कस्मा॑द॒ग्निरु॑च्यत॒ इति॒ यच्छन्दो॑भिश्चि॒नोत्य॒ग्नयो॒ वै छन्दाꣳ॑सि॒ तस्मा॑द॒ग्निरु॑च्य॒ते\-ऽथो॑ इ॒यं वा अ॒ग्निर्वै᳚श्वान॒रो यत्॥३८॥

%5.7.9.4
मृ॒दा चि॒नोति॒ तस्मा॑द॒ग्निरु॑च्यते हिरण्येष्ट॒का उप॑ दधाति॒ ज्योति॒र्वै हिर॑ण्यं॒ ज्योति॑रे॒वास्मि॑न्दधा॒त्यथो॒ तेजो॒ वै हिर॑ण्यं॒ तेज॑ ए॒वात्मन्ध॑त्ते॒ यो वा अ॒ग्निꣳ स॒र्वतो॑मुखं चिनु॒ते सर्वा॑सु प्र॒जास्वन्न॑मत्ति॒ सर्वा॒ दिशो॒\-ऽभि ज॑यति गाय॒त्रीम्पु॒रस्ता॒दुप॑ दधाति त्रिष्टुभं॑ दक्षिण॒तो जग॑तीम्प॒श्चाद॑नु॒ष्टुभ॑मुत्तर॒तः प॒ङ्क्तिम्मध्य॑ ए॒ष वा अ॒ग्निः स॒र्वतो॑मुख॒स्तं य ए॒वं वि॒द्वाꣴश्चि॑नु॒ते सर्वा॑सु प्र॒जास्वन्न॑मत्ति॒ सर्वा॒ दिशो॒\-ऽभि ज॑य॒त्यथो॑ दि॒श्ये॑व दिश॒म्प्र व॑यति॒ तस्मा᳚द्दि॒शि दिक्प्रोता᳚॥३९॥

%5.7.10.0
{\anuvakamend[{अपि॑ सं॒ यौति॑ वैश्वान॒रो यदे॒ष वै पञ्च॑विꣳशतिश्च}]}%॥९॥

%5.7.10.1
प्र॒जाप॑तिर॒ग्निम॑सृजत॒ सो᳚\-ऽस्माथ्सृ॒ष्टः प्राङ्प्राद्र॑व॒त्तस्मा॒ अश्व॒म्प्रत्या᳚स्य॒थ्स द॑क्षि॒णाव॑र्तत॒ तस्मै॑ वृ॒ष्णिम्प्रत्या᳚स्य॒थ्स प्र॒त्यङ्ङाव॑र्तत॒ तस्मा॑ ऋष॒भम्प्रत्या᳚स्य॒थ्स उद॒ङ्ङाव॑र्तत॒ तस्मै॑ ब॒स्तम्प्रत्या᳚स्य॒थ्स ऊ॒र्ध्वो᳚\-ऽद्रव॒त्तस्मै॒ पुरु॑ष॒म्प्रत्या᳚स्य॒त् यत्प॑शुशी॒र्\mbox{}षाण्यु॑प॒दधा॑ति स॒र्वत॑ ए॒वैनम्᳚॥४०॥

%5.7.10.2
अ॒व॒रुध्य॑ चिनुत ए॒ता वै प्रा॑ण॒भृत॒श्चक्षु॑ष्मती॒रिष्ट॑का॒ यत्प॑शुशी॒र्\mbox{}षाणि॒ यत्प॑शुशी॒र्\mbox{}षाण्यु॑प॒दधा॑ति॒ ताभि॑रे॒व यज॑मानो॒\-ऽमुष्मि॑ल्लोँ॒के प्राणि॒त्यथो॒ ताभि॑रे॒वास्मा॑ इ॒मे लो॒काः प्र भा᳚न्ति मृ॒दाभि॒लिप्योप॑ दधाति मेध्य॒त्वाय॑ प॒शुर्वा ए॒ष यद॒ग्निरन्न॑म्प॒शव॑ ए॒ष खलु॒ वा अ॒ग्निर्यत्प॑शुशी॒र्\mbox{}षाणि॒ यं का॒मये॑त॒ कनी॑यो॒\-ऽस्यान्नम्᳚॥४१॥

%5.7.10.3
स्या॒दिति॑ सन्त॒रां तस्य॑ पशुशी॒र्\mbox{}षाण्युप॑ दध्या॒त्कनी॑य ए॒वास्यान्न॑म्भवति॒ यं का॒मये॑त स॒माव॑द॒स्यान्नꣴ॑ स्या॒दिति॑ मध्य॒तस्तस्योप॑ दध्याथ्स॒माव॑दे॒वास्यान्न॑म्भवति॒ यं का॒मये॑त॒ भूयो॒\-ऽस्यान्नꣴ॑ स्या॒दित्यन्ते॑षु॒ तस्य॑ व्यु॒दूह्योप॑ दध्यादन्त॒त ए॒वास्मा॒ अन्न॒मव॑ रुन्द्धे॒ भूयो॒\-ऽस्यान्न॑म्भवति॥४२॥

%5.7.11.0
{\anuvakamend[{ए॒न॒म॒स्यान्न॒म्भूयो॒स्यान्न॑म्भवति}]}%॥10॥

%5.7.11.1
स्ते॒गान्दꣴष्ट्रा᳚भ्याम्म॒ण्डूका॒ञ्जम्भ्ये॑भि॒राद॑कां खा॒देनोर्जꣳ॑ सꣳसू॒देनार॑ण्यं॒ जाम्बी॑लेन॒ मृद॑म्ब॒र्स्वे॑भिः॒ शर्क॑राभि॒रव॑का॒मव॑काभिः॒ शर्क॑रामुथ्सा॒देन॑ जि॒ह्वाम॑वक्र॒न्देन॒ तालु॒ꣳ॒ सर॑स्वतीं जिह्वा॒ग्रेण॑॥४३॥

%5.7.12.0
{\anuvakamend[{स्ते॒गान्द्वाविꣳ॑शतिः}]}%॥11॥

%5.7.12.1
वाज॒ꣳ॒ हनू᳚भ्याम॒प आ॒स्ये॑नादि॒त्याञ्छ्मश्रु॑भिरुपया॒ममध॑रे॒णोष्ठे॑न॒ सदुत्त॑रे॒णान्त॑रेणानूका॒शम्प्र॑का॒शेन॒ बाह्यꣴ॑ स्तनयि॒त्नुं नि॑र्बा॒धेन॑ सूर्या॒ग्नी चक्षु॑र्भ्यां वि॒द्युतौ॑ क॒नान॑काभ्याम॒शनि॑म्म॒स्तिष्के॑ण॒ बल॑म्म॒ज्जभिः॑॥४४॥

%5.7.13.0
{\anuvakamend[{वाजं॒ पञ्च॑विꣳशतिः}]}%॥12॥

%5.7.13.1
कू॒र्माञ्छ॒फैर॒च्छला॑भिः क॒पिञ्ज॑ला॒न्थ्साम॒ कुष्ठि॑काभिर्ज॒वं जङ्घा॑भिरग॒दं जानु॑भ्यां वी॒र्यं॑ कु॒हा\-भ्यां᳚ भ॒यम्प्र॑चा॒लाभ्या॒म् गुहो॑पप॒क्षाभ्या॑म॒श्विना॒वꣳसा᳚भ्या॒मदि॑तिꣳ शी॒र्ष्णा निर्\mbox{}ऋ॑तिं॒ निर्जा᳚ल्मकेन शी॒र्ष्णा॥४५॥

%5.7.14.0
{\anuvakamend[{कू॒र्मान्त्रयो॑विꣳशतिः}]}%॥13॥

%5.7.14.1
योक्त्रं॒ गृध्रा॑भिर्यु॒गमान॑तेन चि॒त्तम्मन्या॑भिः सङ्क्रो॒शान्प्रा॒णैः प्र॑का॒शेन॒ त्वचं॑ पराका॒शेनान्त॑राम्म॒शका॒न्केशै॒रिन्द्र॒ꣴ॒ स्वप॑सा॒ वहे॑न॒ बृह॒स्पतिꣳ॑ शकुनिसा॒देन॒ रथ॑मु॒ष्णिहा॑भिः॥४६॥

%5.7.15.0
{\anuvakamend[{योक्त्र॒मेक॑विꣳशतिः}]}%॥14॥

%5.7.15.1
मि॒त्रावरु॑णौ॒ श्रोणी᳚भ्यामिन्द्रा॒ग्नी शि॑ख॒ण्डाभ्या॒मिन्द्रा॒बृह॒स्पती॑ ऊ॒रुभ्या॒मिन्द्रा॒विष्णू॑ अष्ठी॒वद्भ्याꣳ॑ सवि॒तार॒म्पुच्छे॑न गन्ध॒र्वाञ्छेपे॑नाफ्स॒रसो॑ मु॒ष्काभ्या॒म्पव॑मानम्पा॒युना॑ प॒वित्र॒म्पोत्रा᳚भ्यामा॒क्रम॑णꣴ स्थू॒रा\-भ्यां᳚ प्रति॒क्रम॑णं॒ कुष्ठा᳚भ्याम्॥४७॥

%5.7.16.0
{\anuvakamend[{}]}%॥15॥

%5.7.16.1
इन्द्र॑स्य क्रो॒डो\-ऽदि॑त्यै पाज॒स्य॑न्दि॒शां ज॒त्रवो॑ जी॒मूता᳚न्हृदयौप॒शाभ्या॑म॒न्तरि॑क्षं पुरि॒तता॒ नभ॑ उद॒र्ये॑णेन्द्रा॒णीम्प्ली॒ह्ना व॒ल्मीका᳚न्क्लो॒म्ना गि॒रीन्प्ला॒शिभिः॑ समु॒द्रमु॒दरे॑ण वैश्वान॒रम्भस्म॑ना॥४८॥

%5.7.17.0
{\anuvakamend[{मि॒त्रावरु॑णा॒विन्द्र॑स्य॒ द्वाविꣳ॑शति॒र्द्वाविꣳ॑शतिः}]}%॥16॥

%5.7.17.1
पू॒ष्णो व॑नि॒ष्ठुर॑न्धा॒हेः स्थू॑रगु॒दा स॒र्पान्गुदा॑भिर्\mbox{}ऋ॒तून्पृ॒ष्टीभि॒र्दिवं॑ पृ॒ष्ठेन॒ वसू॑नाम्प्रथ॒मा कीक॑सा रु॒द्राणां᳚ द्वि॒तीया॑दि॒त्यानां᳚ तृ॒तीयाङ्गि॑रसां चतु॒र्थी सा॒ध्यानां᳚ पञ्च॒मी विश्वे॑षां दे॒वानाꣳ॑ ष॒ष्ठी॥४९॥

%5.7.18.0
{\anuvakamend[{पू॒ष्णश्चतु॑र्विꣳशतिः}]}%॥17॥

%5.7.18.1
ओजो᳚ ग्री॒वाभि॒र्निर्\mbox{}ऋ॑तिम॒स्थभि॒रिन्द्र॒ꣴ॒ स्वप॑सा॒ वहे॑न रु॒द्रस्य॑ विच॒लः स्क॒न्धो॑\-ऽहोरा॒त्रयो᳚र्द्वि॒तीयो᳚\-ऽर्धमा॒सानां᳚ तृ॒तीयो॑ मा॒सां च॑तु॒र्थ ऋ॑तू॒नाम्प॑ञ्च॒मः सं॑वथ्स॒रस्य॑ ष॒ष्ठः॥५०॥

%5.7.19.0
{\anuvakamend[{ओजो॑ विꣳश॒तिः}]}%॥18॥

%5.7.19.1
आ॒न॒न्दं न॒न्दथु॑ना॒ काम॑म्प्रत्या॒सा\-भ्यां᳚ भ॒यꣳ शि॑ती॒म\-भ्यां᳚ प्र॒शिष॑म्प्रशा॒साभ्याꣳ॑ सूर्याचन्द्र॒मसौ॒ वृक्या᳚भ्याꣴ श्यामशब॒लौ मत॑स्नाभ्या॒व्व्युँ॑ष्टिꣳ रू॒पेण॒ निम्रु॑क्ति॒मरू॑पेण॥५१॥

%5.7.20.0
{\anuvakamend[{आ॒न॒न्दꣳ षोड॑श}]}%॥19॥

%5.7.20.1
अह॑र्मा॒ꣳ॒सेन॒ रात्रि॒म्पीव॑सा॒पो यू॒षेण॑ घृ॒तꣳ रसे॑न॒ श्यां वस॑या दू॒षीका॑भिर्\mbox{}ह्रा॒दुनि॒मश्रु॑भिः॒ पृष्वा॒न्दिवꣳ॑ रू॒पेण॒ नक्ष॑त्राणि॒ प्रति॑रूपेण पृथि॒वीं चर्म॑णा छ॒वीं छ॒व्यो॑पाकृ॑ताय॒ स्वाहाल॑ब्धाय॒ स्वाहा॑ हु॒ताय॒ स्वाहा᳚॥५२॥

%5.7.21.0
{\anuvakamend[{अह॑र॒ष्टाविꣳ॑शतिः}]}%॥20॥

%5.7.21.1
अ॒ग्नेः प॑क्ष॒तिः सर॑स्वत्यै॒ निप॑क्षतिः॒ सोम॑स्य तृ॒तीया॒पां च॑तु॒र्थ्योष॑धीनां पञ्च॒मी सं॑वथ्स॒रस्य॑ ष॒ष्ठी म॒रुताꣳ॑ सप्त॒मी बृह॒स्पते॑रष्ट॒मी मि॒त्रस्य॑ नव॒मी वरु॑णस्य दश॒मीन्द्र॑स्यैकाद॒शी विश्वे॑षां दे॒वानां᳚ द्वाद॒शी द्यावा॑पृथि॒व्योः पा॒र्श्वं य॒मस्य॑ पाटू॒रः॥५३॥

%5.7.22.0
{\anuvakamend[{अ॒ग्नेरेका॒न्नत्रि॒ꣳ॒शत्}]}%॥21॥

%5.7.22.1
वा॒योः प॑क्ष॒तिः सर॑स्वतो॒ निप॑क्षतिश्च॒न्द्रम॑सस्तृ॒तीया॒ नक्ष॑त्राणां चतु॒र्थी स॑वि॒तुः प॑ञ्च॒मी रु॒द्रस्य॑ ष॒ष्ठी स॒र्पाणाꣳ॑ सप्त॒म्य॑र्य॒म्णो᳚\-ऽष्ट॒मी त्वष्टु॑र्नव॒मी धा॒तुर्द॑श॒मीन्द्रा॒ण्या ए॑काद॒श्यदि॑त्यै द्वाद॒शी द्यावा॑पृथि॒व्योः पा॒र्श्वं य॒म्यै॑ पाटू॒रः॥५४॥

%5.7.23.0
{\anuvakamend[{वा॒योर॒ष्टाविꣳ॑शतिः}]}%॥22॥

%5.7.23.1
पन्था॑मनू॒वृग्भ्या॒ꣳ॒ सन्त॑तिꣴ स्नाव॒न्या᳚भ्या॒ꣳ॒ शुका᳚न्पि॒त्तेन॑ हरि॒माणं॑ य॒क्ना हली᳚क्ष्णान्पापवा॒तेन॑ कू॒श्माञ्छक॑भिः शव॒र्तानूव॑ध्येन॒ शुनो॑ वि॒शस॑नेन स॒र्पाल्लोँ॑हितग॒न्धेन॒ वयाꣳ॑सि पक्वग॒न्धेन॑ पि॒पीलि॑काः प्रशा॒देन॑॥५॥

%5.7.24.0
{\anuvakamend[{पन्था॒न्द्वाविꣳ॑शतिः}]}%॥23॥

%5.7.24.1
क्रमै॒रत्य॑क्रमीद्वा॒जी विश्वै᳚र्दे॒वैर्य॒ज्ञियैः᳚ संविदा॒नः। स नो॑ नय सुकृ॒तस्य॑ लो॒कं तस्य॑ ते व॒यꣴ स्व॒धया॑ मदेम॥५६॥

%5.7.25.0
{\anuvakamend[{क्रमै॑र॒ष्टाद॑श}]}%॥24॥

%5.7.25.1
द्यौस्ते॑ पृ॒ष्ठं पृ॑थि॒वी स॒धस्थ॑मा॒त्मान्तरि॑क्षꣳ समु॒द्रो योनिः॒ सूर्य॑स्ते॒ चक्षु॒र्वातः॑ प्रा॒णश्च॒न्द्रमाः॒ श्रोत्र॒म्मासा᳚श्चार्धमा॒साश्च॒ पर्वा᳚ण्यृ॒तवोङ्गा॑नि संवथ्स॒रो म॑हि॒मा॥५७॥

%5.7.26.0
{\anuvakamend[{द्यौः पञ्च॑विꣳशतिः}]}%॥25॥

%5.7.26.1
अ॒ग्निः प॒शुरा॑सी॒त्तेना॑यजन्त॒ स ए॒तं लो॒कम॑जय॒द्यस्मि॑न्न॒ग्निः स ते॑ लो॒कस्तं जे᳚ष्य॒स्यथाव॑ जिघ्र वा॒युः प॒शुरा॑सी॒त्तेना॑यजन्त॒ स ए॒तं लो॒कम॑जय॒द्यस्मि॑न्वा॒युः स ते॑ लो॒कस्तस्मा᳚त्त्वा॒न्तरे᳚ष्यामि॒ यदि॒ नाव॒जिघ्र॑स्यादि॒त्यः प॒शुरा॑सी॒त्तेना॑यजन्त॒ स ए॒तं लो॒कम॑जय॒द्यस्मि॑न्नादि॒त्यः स ते॑ लो॒कस्तं जे᳚ष्यसि॒ यद्य॑व॒जिघ्र॑सि॥५८॥

%6.1.0.0
{\anuvakamend[{यस्मि॑न्न॒ष्टौ च॑}]}%॥26॥

%6.1.0.0

{\anuvakamend[{प्रा॒चीन॑वꣳशं॒ याव॑न्त ऋख्सा॒मे वाग्वै दे॒वेभ्यो॑ दे॒वा वै दे॑व॒यज॑नङ्क॒द्रूश्च॒ तद्धिर॑ण्य॒ꣳ॒ षट्प॒दानि॑ ब्रह्मवा॒दिनो॑ वि॒चित्यो॒ यत्क॒लया॑ ते वारु॒णो वै क्री॒तः सोम॒ एका॑दश}]}%॥11॥ प्रा॒चीन॑वꣳश॒ꣴ॒ स्वाहेत्या॑ह॒ ये᳚\-ऽन्तः श॒रा ह्ये॑ष सं तप॑सा च॒ यत्क॑र्णगृही॒तेति॑ लोम॒तो वा॑रु॒णष्षट्थ्स॑प्ततिः॥76॥ प्रा॒चीन॑वꣳशं॒ परि॑चरति॥
%%% END KANDAM

\chapt{काण्डम् ६}
\sect{प्रथमः प्रश्नः}\setcounter{anuvakam}{0}
\dnsub{तैत्तिरीयसंहितायां षष्ठमकाण्डे प्रथमः प्रश्नः}
%6.1.1.0
%6.1.1.1
प्रा॒चीन॑वꣳशं करोति देवमनु॒ष्या दिशो॒ व्य॑भजन्त॒ प्राचीं᳚ दे॒वा द॑क्षि॒णा पि॒तरः॑ प्र॒तीची᳚म्मनु॒ष्या॑ उदी॑चीꣳ रु॒द्रा यत्प्रा॒चीन॑वꣳशं क॒रोति॑ देवलो॒कमे॒व तद्यज॑मान उ॒पाव॑र्तते॒ परि॑ श्रयत्य॒न्तर्\mbox{}हि॑तो॒ हि दे॑वलो॒को म॑नुष्यलो॒का\-न्नास्माल्लो॒काथ्स्वे॑तव्यमि॒वेत्या॑हुः॒ को हि तद्वेद॒ यद्य॒मुष्मि॑ल्लोँ॒के\-ऽस्ति॑ वा॒ न वेति॑ दि॒क्ष्वती॑का॒शान्क॑रोति॥१॥

%6.1.1.2
उ॒भयो᳚र्लो॒कयो॑र॒भिजि॑त्यै केशश्म॒श्रु व॑पते न॒खानि॒ नि कृ॑न्तते मृ॒ता वा ए॒षा त्वग॑मे॒ध्या यत्के॑शश्म॒श्रु मृ॒तामे॒व त्वच॑ममे॒ध्याम॑प॒हत्य॑ य॒ज्ञियो॑ भू॒त्वा मेध॒मुपै॒त्यङ्गि॑रसः सुव॒र्गं लो॒कं यन्तो॒\-ऽफ्सु दी᳚क्षात॒पसी॒ प्रावे॑शयन्न॒फ्सु स्ना॑ति सा॒क्षादे॒व दी᳚क्षात॒पसी॒ अव॑ रुन्द्धे ती॒र्थे स्ना॑ति ती॒र्थे हि ते ताम्प्रावे॑शयन्ती॒र्थे स्ना॑ति॥२॥

%6.1.1.3
ती॒र्थमे॒व स॑मा॒नानां᳚ भवत्य॒पो᳚\-ऽश्ञात्यन्तर॒त ए॒व मेध्यो॑ भवति॒ वास॑सा दीक्षयति सौ॒म्यं वै क्षौमं॑ दे॒वत॑या॒ सोम॑मे॒ष दे॒वता॒मुपै॑ति॒ यो दीक्ष॑ते॒ सोम॑स्य त॒नूर॑सि त॒नुवं॑ मे पा॒हीत्या॑ह॒ स्वामे॒व दे॒वता॒मुपै॒त्यथो॑ आ॒शिष॑मे॒वैतामा शा᳚स्ते॒\-ऽग्नेस्तू॑षा॒धानं॑ वा॒योर्वा॑त॒पान॑म्पितृ॒णान्नी॒विरोष॑धीनाम्प्रघा॒तः॥३॥

%6.1.1.4
आ॒दि॒त्यानां᳚ प्राचीनता॒नो विश्वे॑षां दे॒वाना॒मोतु॒र्नक्ष॑त्राणामतीका॒शास्तद्वा ए॒तथ्स॑र्वदेव॒त्यं॑ यद्वासो॒ यद्वास॑सा दी॒क्षय॑ति॒ सर्वा॑भिरे॒वैनं॑ दे॒वता॑भिर्दीक्षयति ब॒हिःप्रा॑णो॒ वै म॑नु॒ष्य॑स्तस्याश॑नं प्रा॒णो᳚\-ऽश्ञाति॒ सप्रा॑ण ए॒व दी᳚क्षत॒ आशि॑तो भवति॒ यावा॑ने॒वास्य॑ प्रा॒णस्तेन॑ स॒ह मेध॒मुपै॑ति घृ॒तं दे॒वाना॒म्मस्तु॑ पितृ॒णान्निष्प॑क्वम्मनु॒ष्या॑णा॒न्तद्वै॥४॥

%6.1.1.5
ए॒तथ्स॑र्वदेव॒त्यं॑ यन्नव॑नीतं॒ यन्नव॑नीतेनाभ्य॒ङ्क्ते सर्वा॑ ए॒व दे॒वताः᳚ प्रीणाति॒ प्रच्यु॑तो॒ वा ए॒षो᳚\-ऽस्माल्लो॒कादग॑तो देवलो॒कं यो दी᳚क्षि॒तो᳚\-ऽन्त॒रेव॒ नव॑नीत॒न्तस्मा॒न्नव॑नीतेना॒भ्य॑ङ्क्ते\-ऽनुलो॒मं यजु॑षा॒ व्यावृ॑त्त्या॒ इन्द्रो॑ वृ॒त्रम॑ह॒न्तस्य॑ क॒नीनि॑का॒ परा॑पत॒त्तदाञ्ज॑नमभव॒द्यदा॒ङ्क्ते चक्षु॑रे॒व भ्रातृ॑व्यस्य वृङ्क्ते॒ दक्षि॑ण॒म्पूर्व॒माङ्क्ते᳚॥५॥

%6.1.1.6
स॒व्यꣳ हि पूर्व॑म्मनु॒ष्या॑ आ॒ञ्जते॒ न नि धा॑वते॒ नीव॒ हि म॑नु॒ष्या॑ धाव॑न्ते॒ पञ्च॒ कृत्व॒ आङ्क्ते॒ पञ्चा᳚क्षरा प॒ङ्क्तिः पाङ्क्तो॑ य॒ज्ञो य॒ज्ञमे॒वाव॑ रुन्द्धे॒ परि॑मित॒माङ्क्ते\-ऽप॑रिमित॒ꣳ॒ हि म॑नु॒ष्या॑ आ॒ञ्जते॒ सतू॑ल॒याङ्क्ते\-ऽप॑तूलया॒ हि म॑नु॒ष्या॑ आ॒ञ्जते॒ व्यावृ॑त्त्यै॒ यदप॑तूलयाञ्जी॒त वज्र॑ इव स्या॒थ्सतू॑ल॒याङ्क्ते॑ मित्र॒त्वाय॑॥६॥

%6.1.1.7
इन्द्रो॑ वृ॒त्रम॑ह॒न्थ्सो\-ऽ पो\-ऽ भ्य॑म्रियत॒ तासां॒ यन्मेध्यं॑ य॒ज्ञिय॒ꣳ॒ सदे॑व॒मासी॒त्तदपोद॑क्राम॒त्ते द॒र्भा अ॑भव॒न्॒ यद्द॑र्भपुञ्जी॒लैः प॒वय॑ति॒ या ए॒व मेध्या॑ य॒ज्ञियाः॒ सदे॑वा॒ आप॒स्ताभि॑रे॒वैन॑म्पवयति॒ द्वा\-भ्यां᳚ पवयत्यहोरा॒त्राभ्या॑मे॒वैन॑म्पवयति त्रि॒भिः प॑वयति॒ त्रय॑ इ॒मे लो॒का ए॒भिरे॒वैनं॑ लो॒कैः प॑वयति प॒ञ्चभिः॑॥७॥

%6.1.1.8
प॒व॒य॒ति॒ पञ्चा᳚क्षरा प॒ङ्क्तिः पाङ्क्तो॑ य॒ज्ञो य॒ज्ञायै॒वैन॑म्पवयति ष॒ड्भिः प॑वयति॒ षड्वा ऋ॒तव॑ ऋ॒तुभि॑रे॒वैन॑म्पवयति स॒प्तभिः॑ पवयति स॒प्त छन्दाꣳ॑सि॒ छन्दो॑भिरे॒वैन॑म्पवयति न॒वभिः॑ पवयति॒ नव॒ वै पुरु॑षे प्रा॒णाः सप्रा॑णमे॒वैन॑म्पव\-य॒त्येक॑विꣳशत्या पवयति॒ दश॒ हस्त्या॑ अ॒ङ्गुल॑यो॒ दश॒ पद्या॑ आ॒त्मैक॑वि॒ꣳ॒शो यावा॑ने॒व पुरु॑ष॒स्तमप॑रिवर्गम्॥८॥

%6.1.1.9
प॒व॒य॒ति॒ चि॒त्पति॑स्त्वा पुना॒त्वित्या॑ह॒ मनो॒ वै चि॒त्पति॒र्मन॑सै॒वैन॑म्पवयति वा॒क्पति॑स्त्वा पुना॒त्वित्या॑ह वा॒चैवैन॑म्पवयति दे॒वस्त्वा॑ सवि॒ता पु॑ना॒त्वित्या॑ह सवि॒तृप्र॑सूत ए॒वैन॑म्पवयति॒ तस्य॑ ते पवित्रपते प॒वित्रे॑ण॒ यस्मै॒ कम्पु॒ने तच्छ॑केय॒मित्या॑हा॒शिष॑मे॒वैतामा शा᳚स्ते॥९॥

%6.1.2.0
{\anuvakamend[{अ॒ती॒का॒शान्क॑रो॒त्यवे॑शयन्ती॒र्थे स्ना॑ति प्रघा॒तो म॑नु॒ष्या॑णा॒न्तद्वा आङ्क्ते॑ मित्र॒त्वाय॑ प॒ञ्चभि॒रप॑रिवर्गम॒ष्टाच॑त्वारिꣳशच्च}]}%॥१॥

%6.1.2.1
याव॑न्तो॒ वै दे॒वा य॒ज्ञायापु॑नत॒ त ए॒वाभ॑व॒न्॒ य ए॒वं वि॒द्वान् य॒ज्ञाय॑ पुनी॒ते भव॑त्ये॒व ब॒हिः प॑वयि॒त्वान्तः प्र पा॑दयति मनुष्यलो॒क ए॒वैन॑म्पवयि॒त्वा पू॒तन्दे॑वलो॒कम्प्र ण॑य॒त्यदी᳚क्षित॒ एक॒याहु॒त्येत्या॑हुः स्रु॒वेण॒ चत॑स्रो जुहोति दीक्षित॒त्वाय॑ स्रु॒चा प॑ञ्च॒मीं पञ्चा᳚क्षरा प॒ङ्क्तिः पाङ्क्तो॑ य॒ज्ञो य॒ज्ञमे॒वाव॑ रुन्द्ध॒ आकू᳚त्यै प्र॒युजे॒\-ऽग्नये᳚॥१०॥

%6.1.2.2
स्वाहेत्या॒हाकू᳚त्या॒ हि पुरु॑षो य॒ज्ञम॒भि प्र॑यु॒ङ्क्ते यजे॒येति॑ मे॒धायै॒ मन॑से॒\-ऽग्नये॒ स्वाहेत्या॑ह मे॒धया॒ हि मन॑सा॒ पुरु॑षो य॒ज्ञम॑भि॒गच्छ॑ति॒ सर॑स्वत्यै पू॒ष्णे᳚\-ऽग्नये॒ स्वाहेत्या॑ह॒ वाग्वै सर॑स्वती पृथि॒वी पू॒षा वा॒चैव पृ॑थि॒व्या य॒ज्ञम्प्र यु॑ङ्क्त॒ आपो॑ देवीर्बृहतीर्विश्वशम्भुव॒ इत्या॑ह॒ या वै वर्ष्या॒स्ताः॥११॥

%6.1.2.3
आपो॑ दे॒वीर्बृ॑ह॒तीर्वि॒श्वश॑म्भुवो॒ यदे॒तद्यजु॒र्न ब्रू॒याद्दि॒व्या आपो\-ऽशा᳚न्ता इ॒मल्लोँ॒कमा ग॑च्छेयु॒रापो॑ देवीर्बृहतीर्विश्वशम्भुव॒ इत्या॑हा॒स्मा ए॒वैना॑ लो॒काय॑ शमयति॒ तस्मा᳚च्छा॒न्ता इ॒मल्लोँ॒कमा ग॑च्छन्ति॒ द्यावा॑पृथि॒वी इत्या॑ह॒ द्यावा॑पृथि॒व्योर्\mbox{}हि य॒ज्ञ उ॒र्व॑न्तरि॑क्ष॒मित्या॑हा॒न्तरि॑क्षे॒ हि य॒ज्ञो बृह॒स्पति॑र्नो ह॒विषा॑ वृधातु॥१२॥

%6.1.2.4
इत्या॑ह॒ ब्रह्म॒ वै दे॒वाना॒म्बृह॒स्पति॒र्ब्रह्म॑णै॒वास्मै॑ य॒ज्ञमव॑ रुन्द्धे॒ यद्ब्रू॒याद्वि॑धे॒रिति॑ यज्ञस्था॒णुमृ॑च्छेद्वृधा॒त्वित्या॑ह यज्ञस्था॒णुमे॒व परि॑ वृणक्ति प्र॒जाप॑तिर्य॒ज्ञम॑सृजत॒ सो᳚\-ऽस्माथ्सृ॒ष्टः परा॑ङै॒थ्स प्र यजु॒रव्ली॑ना॒त्प्र साम॒ तमृगुद॑यच्छ॒द्यदृगु॒दय॑च्छ॒त्तदौ᳚द्ग्रह॒णस्यौ᳚द्ग्रहण॒त्वमृ॒चा॥१३॥

%6.1.2.5
जु॒हो॒ति॒ य॒ज्ञस्योद्य॑त्या अनु॒ष्टुप्छन्द॑सा॒मुद॑यच्छ॒दित्या॑हु॒स्तस्मा॑दनु॒ष्टुभा॑ जुहोति य॒ज्ञस्योद्य॑त्यै॒ द्वाद॑श वाथ्सब॒न्धान्युद॑यच्छ॒न्नित्या॑हु॒स्तस्मा᳚द्द्वाद॒शभि॑र्वाथ्सबन्ध॒विदो॑ दीक्षयन्ति॒ सा वा ए॒षर्ग॑नु॒ष्टुग्वाग॑नु॒ष्टुग्यदे॒तय॒र्चा दी॒क्षय॑ति वा॒चैवैन॒ꣳ॒ सर्व॑या दीक्षयति॒ विश्वे॑ दे॒वस्य॑ ने॒तुरित्या॑ह सावि॒त्र्ये॑तेन॒ मर्तो॑ वृणीत स॒ख्यम्॥१४॥

%6.1.2.6
इत्या॑ह पितृदेव॒त्यै॑तेन॒ विश्वे॑ रा॒य इ॑षुध्य॒सीत्या॑ह वैश्वदे॒व्ये॑तेन॑ द्यु॒म्नं वृ॑णीत पु॒ष्यस॒ इत्या॑ह पौ॒ष्ण्ये॑तेन॒ सा वा ए॒षर्ख्स॑र्वदेव॒त्या॑ यदे॒तय॒र्चा दी॒क्षय॑ति॒ सर्वा॑भिरे॒वैनं॑ दे॒वता॑भिर्दीक्षयति स॒प्ताक्ष॑रम्प्रथ॒मम्प॒दम॒ष्टाक्ष॑राणि॒ त्रीणि॒ यानि॒ त्रीणि॒ तान्य॒ष्टावुप॑ यन्ति॒ यानि॑ च॒त्वारि॒ तान्य॒ष्टौ यद॒ष्टाक्ष॑रा॒ तेन॑॥१५॥

%6.1.2.7
गा॒य॒त्री यदेका॑दशाक्षरा॒ तेन॑ त्रि॒ष्टुग्यद्द्वाद॑शाक्षरा॒ तेन॒ जग॑ती॒ सा वा ए॒षर्ख्सर्वा॑णि॒ छन्दाꣳ॑सि॒ यदे॒तय॒र्चा दी॒क्षय॑ति॒ सर्वे॑भिरे॒वैनं॒ छन्दो॑भिर्दीक्षयति स॒प्ताक्ष॑रम्प्रथ॒मम्प॒दꣳ स॒प्तप॑दा॒ शक्व॑री प॒शवः॒ शक्व॑री प॒शूने॒वाव॑ रुन्द्ध॒ एक॑स्माद॒क्षरा॒दना᳚प्तम्प्रथ॒मम्प॒दन्तस्मा॒द्यद्वा॒चो\-ऽना᳚प्त॒न्तन्म॑नु॒ष्या॑ उप॑ जीवन्ति पू॒र्णया॑ जुहोति पू॒र्ण इ॑व॒ हि प्र॒जाप॑तिः प्र॒जाप॑ते॒राप्त्यै॒ न्यू॑नया जुहोति॒ न्यू॑ना॒द्धि प्र॒जाप॑तिः प्र॒जा असृ॑जत प्र॒जाना॒ꣳ॒ सृष्ट्यै᳚॥१६॥

%6.1.3.0
{\anuvakamend[{अ॒ग्नये॒ ता वृ॑धात्वृ॒चा स॒ख्यन्तेन॑ जुहोति॒ पञ्च॑दश च}]}%॥२॥

%6.1.3.1
ऋ॒ख्सा॒मे वै दे॒वेभ्यो॑ य॒ज्ञायाति॑ष्ठमाने॒ कृष्णो॑ रू॒पं कृ॒त्वाप॒क्रम्या॑तिष्ठता॒न्ते॑\-ऽमन्यन्त॒ यं वा इ॒मे उ॑पाव॒र्थ्स्यतः॒ स इ॒दं भ॑विष्य॒तीति॒ ते उपा॑मन्त्रयन्त॒ ते अ॑होरा॒त्रयो᳚र्महि॒मान॑मपनि॒धाय॑ दे॒वानु॒पाव॑र्तेतामे॒ष वा ऋ॒चो वर्णो॒ यच्छु॒क्लं कृ॑ष्णाजि॒नस्यै॒ष साम्नो॒ यत्कृ॒ष्णमृ॑ख्सा॒मयोः॒ शिल्पे᳚ स्थ॒ इत्या॑हर्ख्सा॒मे ए॒वाव॑ रुन्ध ए॒षः॥१७॥

%6.1.3.2
वा अह्नो॒ वर्णो॒ यच्छु॒क्लं कृ॑ष्णाजि॒नस्यै॒ष रात्रि॑या॒ यत्कृ॒ष्णं यदे॒वैन॑यो॒स्तत्र॒ न्य॑क्तं॒ तदे॒वाव॑ रुन्द्धे कृष्णाजि॒नेन॑ दीक्षयति॒ ब्रह्म॑णो॒ वा ए॒तद्रू॒पं यत्कृ॑ष्णाजि॒नं ब्रह्म॑णै॒वैनं॑ दीक्षयती॒मान्धिय॒ꣳ॒ शिक्ष॑माणस्य दे॒वेत्या॑ह यथाय॒जुरे॒वैतद्गर्भो॒ वा ए॒ष यद्दी᳚क्षि॒त उल्बं॒ वासः॒ प्रोर्णु॑ते॒ तस्मा᳚त्॥१८॥

%6.1.3.3
गर्भाः॒ प्रावृ॑ता जायन्ते॒ न पु॒रा सोम॑स्य क्र॒यादपो᳚र्ण्वीत॒ यत्पु॒रा सोम॑स्य क्र॒याद॑पोर्ण्वी॒त गर्भाः᳚ प्र॒जानां᳚ परा॒पातु॑काः स्युः क्री॒ते सोमे\-ऽपो᳚र्णुते॒ जाय॑त ए॒व तदथो॒ यथा॒ वसी॑याꣳसम्प्रत्यपोर्णु॒ते ता॒दृगे॒व तदङ्गि॑रसः सुव॒र्गं लो॒कं यन्त॒ ऊर्जं॒ व्य॑भजन्त॒ ततो॒ यद॒त्यशि॑ष्यत॒ ते श॒रा अ॑भव॒न्नूर्ग्वै श॒रा यच्छ॑र॒मयी᳚॥१९॥

%6.1.3.4
मेख॑ला॒ भव॒त्यूर्ज॑मे॒वाव॑ रुन्द्धे मध्य॒तः सन्न॑ह्यति मध्य॒त ए॒वास्मा॒ ऊर्जं॑ दधाति॒ तस्मा᳚न्मध्य॒त ऊ॒र्जा भु॑ञ्जत ऊ॒र्ध्वं वै पुरु॑षस्य॒ नाभ्यै॒ मेध्य॑मवा॒चीन॑ममे॒ध्यं यन्म॑ध्य॒तः सं॒नह्य॑ति॒ मेध्यं॑ चै॒वास्या॑मे॒ध्यं च॒ व्याव॑र्तय॒तीन्द्रो॑ वृ॒त्राय॒ वज्र॒म्प्राह॑र॒थ्स त्रे॒धा व्य॑भव॒थ्स्फ्यस्तृती॑य॒ꣳ॒ रथ॒स्तृती॑यं॒ यूप॒स्तृती॑यम्॥२०॥

%6.1.3.5
ये᳚\-ऽन्तःश॒रा अशी᳚र्यन्त॒ ते श॒रा अ॑भव॒न्तच्छ॒राणाꣳ॑ शर॒त्वं वज्रो॒ वै श॒राः क्षुत्खलु॒ वै म॑नु॒ष्य॑स्य॒ भ्रातृ॑व्यो॒ यच्छ॑र॒मयी॒ मेख॑ला॒ भव॑ति॒ वज्रे॑णै॒व सा॒क्षात्क्षुध॒म्भ्रातृ॑व्यम्मध्य॒तो\-ऽप॑ हते त्रि॒वृद्भ॑वति त्रि॒वृद्वै प्रा॒णस्त्रि॒वृत॑मे॒व प्रा॒णम्म॑ध्य॒तो यज॑माने दधाति पृ॒थ्वी भ॑वति॒ रज्जू॑ना॒व्व्याँवृ॑त्त्यै॒ मेख॑लया॒ यज॑मानन्दीक्षयति॒ योक्त्रे॑ण॒ पत्नी᳚म्मिथुन॒त्वाय॑॥२१॥

%6.1.3.6
य॒ज्ञो दक्षि॑णाम॒भ्य॑ध्याय॒त्ताꣳ सम॑भव॒त्तदिन्द्रो॑\-ऽचाय॒थ्सो॑\-ऽमन्यत॒ यो वा इ॒तो ज॑नि॒ष्यते॒ स इ॒दम्भ॑विष्य॒तीति॒ ताम्प्रावि॑श॒त्तस्या॒ इन्द्र॑ ए॒वाजा॑यत॒ सो॑\-ऽमन्यत॒ यो वै मदि॒तो\-ऽप॑रो जनि॒ष्यते॒ स इ॒दम्भ॑विष्य॒तीति॒ तस्या॑ अनु॒मृश्य॒ योनि॒माच्छि॑न॒थ्सा सू॒तव॑शाभव॒त्तथ्सू॒तव॑शायै॒ जन्म॑॥२२॥

%6.1.3.7
ताꣳ हस्ते॒ न्य॑वेष्टयत॒ ताम्मृ॒गेषु॒ न्य॑दधा॒थ्सा कृ॑ष्णविषा॒णाभ॑व॒दिन्द्र॑स्य॒ योनि॑रसि॒ मा मा॑ हिꣳसी॒रिति॑ कृष्णविषा॒णाम्प्र य॑च्छति॒ सयो॑निमे॒व य॒ज्ञं क॑रोति॒ सयो॑नि॒न्दक्षि॑णा॒ꣳ॒ सयो॑नि॒मिन्द्रꣳ॑ सयोनि॒त्वाय॑ कृ॒ष्यै त्वा॑ सुस॒स्याया॒ इत्या॑ह॒ तस्मा॑दकृष्टप॒च्या ओष॑धयः पच्यन्ते सुपिप्प॒लाभ्य॒स्त्वौष॑धीभ्य॒ इत्या॑ह॒ तस्मा॒दोष॑धयः॒ फलं॑ गृह्णन्ति॒ यद्धस्ते॑न॥२३॥

%6.1.3.8
क॒ण्डू॒येत॑ पामन॒म्भावु॑काः प्र॒जाः स्यु॒र्यथ्स्मये॑त नग्न॒म्भावु॑काः कृष्णविषा॒णया॑ कण्डूयते\-ऽपि॒गृह्य॑ स्मयते प्र॒जानां᳚ गोपी॒थाय॒ न पु॒रा दक्षि॑णाभ्यो॒ नेतोः᳚ कृष्णविषा॒णामव॑ चृते॒द्यत्पु॒रा दक्षि॑णाभ्यो॒ नेतोः᳚ कृष्णविषा॒णामव॑चृ॒तेद्योनिः॑ प्र॒जानां᳚ परा॒पातु॑का स्यान्नी॒तासु॒ दक्षि॑णासु॒ चात्वा॑ले कृष्णविषा॒णाम्प्रास्य॑ति॒ योनि॒र्वै य॒ज्ञस्य॒ चात्वा॑लं॒ योनिः॑ कृष्णविषा॒णा योना॑वे॒व योनि॑न्दधाति य॒ज्ञस्य॑ सयोनि॒त्वाय॑॥२४॥

%6.1.4.0
{\anuvakamend[{रु॒न्ध॒ ए॒ष तस्मा᳚च्छर॒मयी॒ यूप॒स्तृती॑यम्मिथुन॒त्वाय॒ जन्म॒ हस्ते॑ना॒ष्टाच॑त्वारिꣳशच्च}]}%॥३॥

%6.1.4.1
वाग्वै दे॒वेभ्यो\-ऽपा᳚क्रामद्यज्ञा॒याति॑ष्ठमाना॒ सा वन॒स्पती॒न्प्रावि॑श॒थ्सैषा वाग्वन॒स्पति॑षु वदति॒ या दु॑न्दु॒भौ या तूण॑वे॒ या वीणा॑यां॒ यद्दी᳚क्षितद॒ण्डम्प्र॒यच्छ॑ति॒ वाच॑मे॒वाव॑ रुन्द्ध॒ औदु॑म्बरो भव॒त्यूर्ग्वा उ॑दु॒म्बर॒ ऊर्ज॑मे॒वाव॑ रुन्द्धे॒ मुखे॑न॒ सम्मि॑तो भवति मुख॒त ए॒वास्मा॒ ऊर्जं॑ दधाति॒ तस्मा᳚न्मुख॒त ऊ॒र्जा भु॑ञ्जते॥२५॥

%6.1.4.2
क्री॒ते सोमे॑ मैत्रावरु॒णाय॑ द॒ण्डम्प्र य॑च्छति मैत्रावरु॒णो हि पु॒रस्ता॑दृ॒त्विग्भ्यो॒ वाचं॑ वि॒भज॑ति॒ तामृ॒त्विजो॒ यज॑माने॒ प्रति॑ ष्ठापयन्ति॒ स्वाहा॑ य॒ज्ञम्मन॒सेत्या॑ह॒ मन॑सा॒ हि पुरु॑षो य॒ज्ञम॑भि॒गच्छ॑ति॒ स्वाहा॒ द्यावा॑पृथि॒वीभ्या॒मित्या॑ह॒ द्यावा॑पृथि॒व्योर्\mbox{}हि य॒ज्ञः स्वाहो॒रोर॒न्तरि॑क्षा॒दित्या॑हा॒न्तरि॑क्षे॒ हि य॒ज्ञः स्वाहा॑ य॒ज्ञं वाता॒दार॑भ॒ इत्या॑हा॒यम्॥२६॥

%6.1.4.3
वाव यः पव॑ते॒ स य॒ज्ञस्तमे॒व सा॒क्षादा र॑भते मु॒ष्टी क॑रोति॒ वाचं॑ यच्छति य॒ज्ञस्य॒ धृत्या॒ अदी᳚क्षिष्टा॒यम्ब्रा᳚ह्म॒ण इति॒ त्रिरु॑पा॒ꣳ॒श्वा॑ह दे॒वेभ्य॑ ए॒वैन॒म्प्राह॒ त्रिरु॒च्चैरु॒भये᳚भ्य ए॒वैनं॑ देवमनु॒ष्येभ्यः॒ प्राह॒ न पु॒रा नक्ष॑त्रेभ्यो॒ वाचं॒ वि सृ॑जे॒द्यत्पु॒रा नक्ष॑त्रेभ्यो॒ वाचं॑ विसृ॒जेद्य॒ज्ञं विच्छि॑न्द्यात्॥२७॥

%6.1.4.4
उदि॑तेषु॒ नक्ष॑त्रेषु व्र॒तं कृ॑णु॒तेति॒ वाचं॒ वि सृ॑जति य॒ज्ञव्र॑तो॒ वै दी᳚क्षि॒तो य॒ज्ञमे॒वाभि वाचं॒ वि सृ॑जति॒ यदि॑ विसृ॒जेद्वै᳚ष्ण॒वीमृच॒मनु॑ ब्रूयाद्य॒ज्ञो वै विष्णु॑र्य॒ज्ञेनै॒व य॒ज्ञꣳ सं त॑नोति॒ दैवी॒न्धिय॑म्मनामह॒ इत्या॑ह य॒ज्ञमे॒व तन्म्र॑दयति सुपा॒रा नो॑ अस॒द्वश॒ इत्या॑ह॒ व्यु॑ष्टिमे॒वाव॑ रुन्द्धे॥२८॥

%6.1.4.5
ब्र॒ह्म॒वा॒दिनो॑ वदन्ति होत॒व्यं॑ दीक्षि॒तस्य॑ गृ॒हा(३)इ न हो॑त॒व्या(३)मिति॑ ह॒विर्वै दी᳚क्षि॒तो यज्जु॑हु॒याद्यज॑मानस्याव॒दाय॑ जुहुया॒द्यन्न जु॑हु॒याद्य॑ज्ञप॒रुर॒न्तरि॑या॒द्ये दे॒वा मनो॑जाता मनो॒युज॒ इत्या॑ह प्रा॒णा वै दे॒वा मनो॑जाता मनो॒युज॒स्तेष्वे॒व प॒रोक्षं॑ जुहोति॒ तन्नेव॑ हु॒तं नेवाहु॑तꣴ स्व॒पन्तं॒ वै दी᳚क्षि॒तꣳ रक्षाꣳ॑सि जिघाꣳसन्त्य॒ग्निः॥२९॥

%6.1.4.6
खलु॒ वै र॑क्षो॒हाग्ने॒ त्वꣳ सु जा॑गृहि व॒यꣳ सु म॑न्दिषीम॒हीत्या॑हा॒ग्निमे॒वाधि॒पां कृ॒त्वा स्व॑पिति॒ रक्ष॑सा॒मप॑हत्या अव्र॒त्यमि॑व॒ वा ए॒ष क॑रोति॒ यो दी᳚क्षि॒तः स्वपि॑ति॒ त्वम॑ग्ने व्रत॒पा अ॒सीत्या॑हा॒ग्निर्वै दे॒वानां᳚ व्र॒तप॑तिः॒ स ए॒वैनं॑ व्र॒तमाल॑म्भयति दे॒व आ मर्त्ये॒ष्वेत्या॑ह दे॒वः॥३०॥

%6.1.4.7
ह्ये॑ष सन्मर्त्ये॑षु॒ त्वं य॒ज्ञेष्वीड्य॒ इत्या॑है॒तꣳ हि य॒ज्ञेष्वीड॒ते\-ऽप॒ वै दी᳚क्षि॒ताथ्सु॑षु॒पुष॑ इन्द्रि॒यं दे॒वताः᳚ क्रामन्ति॒ विश्वे॑ दे॒वा अ॒भि मामाव॑वृत्र॒न्नित्या॑हेन्द्रि॒येणै॒वैनं॑ दे॒वता॑भिः॒ सं न॑यति॒ यदे॒तद्यजु॒र्न ब्रू॒याद्याव॑त ए॒व प॒शून॒भि दीक्षे॑त॒ ताव॑न्तो\-ऽस्य प॒शवः॑ स्यू॒ रास्वेय॑त्॥३१॥

%6.1.4.8
सो॒मा भूयो॑ भ॒रेत्या॒हाप॑रिमिताने॒व प॒शूनव॑ रुन्द्धे च॒न्द्रम॑सि॒ मम॒ भोगा॑य भ॒वेत्या॑ह यथादेव॒तमे॒वैनाः॒ प्रति॑ गृह्णाति वा॒यवे᳚ त्वा॒ वरु॑णाय॒ त्वेति॒ यदे॒वमे॒ता नानु॑दि॒शेदय॑थादेवतं॒ दक्षि॑णा गमये॒दा दे॒वता᳚भ्यो वृश्च्येत॒ यदे॒वमे॒ता अ॑नुदि॒शति॑ यथादेव॒तमे॒व दक्षि॑णा गमयति॒ न दे॒वता᳚भ्य॒ आ॥३२॥

%6.1.4.9
वृ॒श्च्य॒ते॒ देवी॑रापो अपां नपा॒दित्या॑ह॒ यद्वो॒ मेध्यं॑ य॒ज्ञिय॒ꣳ॒ सदे॑वं॒ तद्वो॒ माव॑ क्रमिष॒मिति॒ वावैतदा॒हाच्छि॑न्नं॒ तन्तुं॑ पृथि॒व्या अनु॑ गेष॒मित्या॑ह॒ सेतु॑मे॒व कृ॒त्वात्ये॑ति॥३३॥

%6.1.5.0
{\anuvakamend[{भु॒ञ्ज॒ते॒\-ऽयञ्छि॑न्द्याद्रुन्धे॒\-ऽग्निरा॑ह दे॒व इय॑द्दे॒वता᳚भ्य॒ आ त्रय॑स्त्रिꣳशच्च}]}%॥४॥

%6.1.5.1
दे॒वा वै दे॑व॒यज॑नमध्यव॒साय॒ दिशो॒ न प्राजा॑न॒न्ते\-ऽ न्यो᳚न्यमुपा॑धाव॒न्त्वया॒ प्र जा॑नाम॒ त्वयेति॒ ते\-ऽदि॑त्या॒ꣳ॒ सम॑ध्रियन्त॒ त्वया॒ प्र जा॑ना॒मेति॒ साब्र॑वी॒द्वरं॑ वृणै॒ मत्प्रा॑यणा ए॒व वो॑ य॒ज्ञा मदु॑दयना अस॒न्निति॒ तस्मा॑दादि॒त्यः प्रा॑य॒णीयो॑ य॒ज्ञाना॑मादि॒त्य उ॑दय॒नीयः॒ पञ्च॑ दे॒वता॑ यजति॒ पञ्च॒ दिशो॑ दि॒शाम्प्रज्ञा᳚त्यै॥३४॥

%6.1.5.2
अथो॒ पञ्चा᳚क्षरा प॒ङ्क्तिः पाङ्क्तो॑ य॒ज्ञो य॒ज्ञमे॒वाव॑ रुन्द्धे॒ पथ्याꣴ॑ स्व॒स्तिम॑यज॒न्प्राची॑मे॒व तया॒ दिश॒म्प्राजा॑नन्न॒ग्निना॑ दक्षि॒णा सोमे॑न प्र॒तीचीꣳ॑ सवि॒त्रोदी॑ची॒मदि॑त्यो॒र्ध्वाम्पथ्याꣴ॑ स्व॒स्तिं य॑जति॒ प्राची॑मे॒व तया॒ दिश॒म्प्र जा॑नाति॒ पथ्याꣴ॑ स्व॒स्तिमि॒ष्ट्वाग्नीषोमौ॑ यजति॒ चक्षु॑षी॒ वा ए॒ते य॒ज्ञस्य॒ यद॒ग्नीषोमौ॒ ताभ्या॑मे॒वानु॑ पश्यति॥३५॥

%6.1.5.3
अ॒ग्नीषोमा॑वि॒ष्ट्वा स॑वि॒तारं॑ यजति सवि॒तृप्र॑सूत ए॒वानु॑ पश्यति सवि॒तार॑मि॒ष्ट्वादि॑तिं यजती॒यं वा अदि॑तिर॒स्यामे॒व प्र॑ति॒ष्ठायानु॑ पश्य॒त्यदि॑तिमि॒ष्ट्वा मा॑रु॒तीमृच॒मन्वा॑ह म॒रुतो॒ वै दे॒वानां॒ विशो॑ देववि॒शं खलु॒ वै कल्प॑मानम्मनुष्यवि॒श\-मनु॑ कल्पते॒ यन्मा॑रु॒तीमृच॑म॒न्वाह॑ वि॒शां कॢप्त्यै᳚ ब्रह्मवा॒दिनो॑ वदन्ति प्रया॒जव॑दननूया॒जम्प्रा॑य॒णीयं॑ का॒र्य॑मनूया॒जव॑त्॥३६॥

%6.1.5.4
अ॒प्र॒या॒जमु॑दय॒नीय॒मिती॒मे वै प्र॑या॒जा अ॒मी अ॑नूया॒जाः सैव सा य॒ज्ञस्य॒ सन्त॑ति॒स्तत्तथा॒ न का॒र्य॑मा॒त्मा वै प्र॑या॒जाः प्र॒जानू॑या॒जा यत्प्र॑या॒जान॑न्तरि॒यादा॒त्मान॑म॒न्तरि॑या॒द्यद॑नूया॒जान॑न्तरि॒यात्प्र॒जाम॒न्तरि॑या॒द्यतः॒ खलु॒ वै य॒ज्ञस्य॒ वित॑तस्य॒ न क्रि॒यते॒ तदनु॑ य॒ज्ञः परा॑ भवति य॒ज्ञं प॑रा॒भव॑न्तं॒ यज॑मा॒नो\-ऽनु॑॥३७॥

%6.1.5.5
परा॑ भवति प्रया॒जव॑दे॒वानू॑या॒जव॑त्प्राय॒णीयं॑ का॒र्य॑म्प्रया॒जव॑दनूया॒जव॑दुदय॒नीयं॒ नात्मान॑मन्त॒रेति॒ न प्र॒जां न य॒ज्ञः प॑रा॒भव॑ति॒ न यज॑मानः प्राय॒णीय॑स्य निष्का॒स उ॑दय॒नीय॑म॒भि निर्व॑पति॒ सैव सा य॒ज्ञस्य॒ सन्त॑ति॒र्याः प्रा॑य॒णीय॑स्य या॒ज्या॑ यत्ता उ॑दय॒नीय॑स्य या॒ज्याः᳚ कु॒र्यात्परा॑ङ॒मुं लो॒कमा रो॑हेत्प्र॒मायु॑कः स्या॒द्याः प्रा॑य॒णीय॑स्य पुरोनुवा॒क्या᳚स्ता उ॑दय॒नीय॑स्य या॒ज्याः᳚ करोत्य॒स्मिन्ने॒व लो॒के प्रति॑ तिष्ठति॥३८॥

%6.1.6.0
{\anuvakamend[{प्रज्ञा᳚त्यै पश्यत्यनूया॒जव॒द्यज॑मा॒नो\-ऽनु॑ पुरोनुवा॒क्या᳚स्ता अ॒ष्टौ च॑}]}%॥५॥

%6.1.6.1
क॒द्रूश्च॒ वै सु॑प॒र्णी चा᳚त्मरू॒पयो॑रस्पर्धेता॒ꣳ॒ सा क॒द्रूः सु॑प॒र्णीम॑जय॒थ्साब्र॑वीत्तृ॒तीय॑स्यामि॒तो दि॒वि सोम॒स्तमा ह॑र॒ तेना॒त्मानं॒ निष्क्री॑णी॒ष्वेती॒यं वै क॒द्रूर॒सौ सु॑प॒र्णी छन्दाꣳ॑सि सौपर्णे॒याः साब्र॑वीद॒स्मै वै पि॒तरौ॑ पु॒त्रान्बि॑भृतस्तृ॒तीय॑स्यामि॒तो दि॒वि सोम॒स्तमा ह॑र॒ तेना॒त्मानं॒ निष्क्री॑णीष्व॥३९॥

%6.1.6.2
इति॑ मा क॒द्रूर॑वोच॒दिति॒ जग॒त्युद॑पत॒च्चतु॑र्दशाक्षरा स॒ती साप्रा᳚प्य॒ न्य॑वर्तत॒ तस्यै॒ द्वे अ॒क्षरे॑ अमीयेता॒ꣳ॒ सा प॒शुभि॑श्च दी॒क्षया॒ चाग॑च्छ॒त्तस्मा॒ज्जग॑ती॒ छन्द॑साम्पश॒व्य॑तमा॒ तस्मा᳚त्पशु॒मन्तं॑ दी॒क्षोप॑ नमति त्रि॒ष्टुगुद॑पत॒त्त्रयो॑दशाक्षरा स॒ती साप्रा᳚प्य॒ न्य॑वर्तत॒ तस्यै॒ द्वे अ॒क्षरे॑ अमीयेता॒ꣳ॒ सा दक्षि॑णाभिश्च॥४०॥

%6.1.6.3
तप॑सा॒ चाग॑च्छ॒त्तस्मा᳚त्त्रि॒ष्टुभो॑ लो॒के माध्यं॑दिने॒ सव॑ने॒ दक्षि॑णा नीयन्त ए॒तत्खलु॒ वाव तप॒ इत्या॑हु॒र्यः स्वं ददा॒तीति॑ गाय॒त्र्युद॑पत॒च्चतु॑रक्षरा स॒त्य॑जया॒ ज्योति॑षा॒ तम॑स्या अ॒जाभ्य॑रुन्द्ध॒ तद॒जाया॑ अज॒त्वꣳ सा सोमं॒ चाह॑रच्च॒त्वारि॑ चा॒क्षरा॑णि साष्टाक्ष॑रा॒ सम॑पद्यत ब्रह्मवा॒दिनो॑ वदन्ति॥४१॥

%6.1.6.4
कस्मा᳚थ्स॒त्याद्गा॑य॒त्री कनि॑ष्ठा॒ छन्द॑साꣳ स॒ती य॑ज्ञमु॒खं परी॑या॒येति॒ यदे॒वादः सोम॒माह॑र॒त्तस्मा᳚द्यज्ञमु॒खं पर्यै॒त् तस्मा᳚त्तेज॒स्विनी॑तमा प॒द्भ्यां द्वे सव॑ने स॒मगृ॑ह्णा॒न्मुखे॒नैकं॒ यन्मुखे॑न स॒मगृ॑ह्णा॒त्तद॑धय॒त्तस्मा॒द्द्वे सव॑ने शु॒क्रव॑ती प्रातःसव॒नं च॒ माध्यं॑दिनं च॒ तस्मा᳚त्तृतीयसव॒न ऋ॑जी॒षम॒भि षु॑ण्वन्ति धी॒तमि॑व॒ हि मन्य॑न्ते॥४२॥

%6.1.6.5
आ॒शिर॒मव॑ नयति सशुक्र॒त्वायाथो॒ सम्भ॑रत्ये॒वैन॒त्तꣳ सोम॑माह्रि॒यमा॑णं गन्ध॒र्वो वि॒श्वाव॑सुः॒ पर्य॑मुष्णा॒थ्स ति॒स्रो रात्रीः॒ परि॑मुषितो\-ऽवस॒त्तस्मा᳚त्ति॒स्रो रात्रीः᳚ क्री॒तः सोमो॑ वसति॒ ते दे॒वा अ॑ब्रुव॒न्थ्स्त्रीका॑मा॒ वै ग॑न्ध॒र्वाः स्त्रि॒या निष्क्री॑णा॒मेति॒ ते वाच॒ꣴ॒ स्त्रिय॒मेक॑हायनीं कृ॒त्वा तया॒ निर॑क्रीण॒न्थ्सा रो॒हिद्रू॒पं कृ॒त्वा ग॑न्ध॒र्वेभ्यः॑॥४३॥

%6.1.6.6
अ॒प॒क्रम्या॑तिष्ठ॒त्तद्रो॒हितो॒ जन्म॒ ते दे॒वा अ॑ब्रुव॒न्नप॑ यु॒ष्मदक्र॑मी॒न्नास्मानु॒पाव॑र्तते॒ वि ह्व॑यामहा॒ इति॒ ब्रह्म॑ गन्ध॒र्वा अव॑द॒न्नगा॑यं दे॒वाः सा दे॒वान्गाय॑त उ॒पाव॑र्तत॒ तस्मा॒द्गाय॑न्त॒ꣴ॒ स्त्रियः॑ कामयन्ते॒ कामु॑का एन॒ꣴ॒ स्त्रियो॑ भवन्ति॒ य ए॒वं वेदाथो॒ य ए॒वं वि॒द्वानपि॒ जन्ये॑षु॒ भव॑ति॒ तेभ्य॑ ए॒व द॑दत्यु॒त यद्ब॒हुत॑याः॥४४॥

%6.1.6.7
भव॒न्त्येक॑हायन्या क्रीणाति वा॒चैवैन॒ꣳ॒ सर्व॑या क्रीणाति॒ तस्मा॒देक॑हायना मनु॒ष्या॑ वाचं॑ वद॒न्त्यकू॑ट॒या\-ऽक॑र्ण॒या\-ऽ का॑ण॒या\-ऽश्लो॑ण॒या\-ऽस॑प्तशफया क्रीणाति॒ सर्व॑यै॒वैनं॑ क्रीणाति॒ यच्छ्वे॒तया᳚ क्रीणी॒याद्दु॒श्चर्मा॒ यज॑मानः स्या॒द्यत्कृ॒ष्णया॑नु॒स्तर॑णी स्यात्प्र॒मायु॑को॒ यज॑मानः स्या॒द्यद्द्वि॑रू॒पया॒ वात्र॑घ्नी स्या॒थ्स वा॒न्यं जि॑नी॒यात्तं वा॒न्यो जि॑नीयादरु॒णया॑ पिङ्गा॒क्ष्या क्री॑णात्ये॒तद्वै सोम॑स्य रू॒पꣴ स्वयै॒वैनं॑ दे॒वत॑या क्रीणाति॥४५॥

%6.1.7.0
{\anuvakamend[{निष्क्री॑णीष्व॒ दक्षि॑णाभिश्च वदन्ति॒ मन्य॑न्ते गन्ध॒र्वेभ्यो॑ ब॒हुत॑याः पिङ्गा॒क्ष्या दश॑ च}]}%॥६॥

%6.1.7.1
तद्धिर॑ण्यमभव॒त्तस्मा॑द॒द्भ्यो हिर॑ण्यम्पुनन्ति ब्रह्मवा॒दिनो॑ वदन्ति॒ कस्मा᳚थ्स॒त्याद॑न॒स्थिके॑न प्र॒जाः प्र॒वीय॑न्ते\-ऽ\-स्थ॒न्वती᳚र्जायन्त॒ इति॒ यद्धिर॑ण्यं घृ॒ते॑\-ऽव॒धाय॑ जु॒होति॒ तस्मा॑दन॒स्थिके॑न प्र॒जाः प्र वी॑यन्ते\-ऽस्थ॒न्वती᳚र्जायन्त ए॒तद्वा अ॒ग्नेः प्रि॒यं धाम॒ यद्घृ॒तं तेजो॒ हिर॑ण्यमि॒यं ते॑ शुक्र त॒नूरि॒दं वर्च॒ इत्या॑ह॒ सते॑जसमे॒वैन॒ꣳ॒ सत॑नुम्॥४६॥

%6.1.7.2
क॒रो॒त्यथो॒ सम्भ॑रत्ये॒वैनं॒ यदब॑द्धमवद॒ध्याद्गर्भाः᳚ प्र॒जानां᳚ परा॒पातु॑काः स्युर्ब॒द्धमव॑ दधाति॒ गर्भा॑णां॒ धृत्यै॑ निष्ट॒र्क्य॑म्बध्नाति प्र॒जानां᳚ प्र॒जन॑नाय॒ वाग्वा ए॒षा यथ्सो॑म॒क्रय॑णी॒ जूर॒सीत्या॑ह॒ यद्धि मन॑सा॒ जव॑ते॒ तद्वा॒चा वद॑ति धृ॒ता मन॒सेत्या॑ह॒ मन॑सा॒ हि वाग्धृ॒ता जुष्टा॒ विष्ण॑व॒ इत्या॑ह॥४७॥

%6.1.7.3
य॒ज्ञो वै विष्णु॑र्य॒ज्ञायै॒वैनां॒ जुष्टां᳚ करोति॒ तस्या᳚स्ते स॒त्यस॑वसः प्रस॒व इत्या॑ह सवि॒तृप्र॑सूतामे॒व वाच॒मव॑ रुन्द्धे॒ काण्डे॑काण्डे॒ वै क्रि॒यमा॑णे य॒ज्ञꣳ रक्षाꣳ॑सि जिघाꣳसन्त्ये॒ष खलु॒ वा अर॑क्षोहतः॒ पन्था॒ यो᳚\-ऽग्नेश्च॒ सूर्य॑स्य च॒ सूर्य॑स्य॒ चक्षु॒रारु॑हम॒ग्नेर॒क्ष्णः क॒नीनि॑का॒मित्या॑ह॒ य ए॒वार॑क्षोहतः॒ पन्था॒स्तꣳ स॒मारो॑हति॥४८॥

%6.1.7.4
वाग्वा ए॒षा यथ्सो॑म॒क्रय॑णी॒ चिद॑सि म॒नासीत्या॑ह॒ शास्त्ये॒वैना॑मे॒तत्तस्मा᳚च्छि॒ष्टाः प्र॒जा जा॑यन्ते॒ चिद॒सीत्या॑ह॒ यद्धि मन॑सा चे॒तय॑ते॒ तद्वा॒चा वद॑ति म॒नासीत्या॑ह॒ यद्धि मन॑साभि॒गच्छ॑ति॒ तत्क॒रोति॒ धीर॒सीत्या॑ह॒ यद्धि मन॑सा॒ ध्याय॑ति॒ तद्वा॒चा॥४९॥

%6.1.7.5
वद॑ति॒ दक्षि॑णा॒सीत्या॑ह॒ दक्षि॑णा ह्ये॑षा य॒ज्ञिया॒सीत्या॑ह य॒ज्ञिया॑मे॒वैनां᳚ करोति क्ष॒त्रिया॒सीत्या॑ह क्ष॒त्रिया॒ ह्ये॑षादि॑तिरस्युभ॒यतः॑शी॒र्\mbox{}ष्णीत्या॑ह॒ यदे॒वादि॒त्यः प्रा॑य॒णीयो॑ य॒ज्ञाना॑मादि॒त्य उ॑दय॒नीय॒स्तस्मा॑दे॒वमा॑ह॒ यदब॑द्धा॒ स्यादय॑ता स्या॒द्यत्प॑दिब॒द्धानु॒स्तर॑णी स्यात्प्र॒मायु॑को॒ यज॑मानः स्यात्॥५०॥

%6.1.7.6
यत्क॑र्णगृही॒ता वार्त्र॑घ्नी स्या॒थ्स वा॒न्यं जि॑नी॒यात्तं वा॒न्यो जि॑नीयान्मि॒त्रस्त्वा॑ प॒दि ब॑ध्ना॒त्वित्या॑ह मि॒त्रो वै शि॒वो दे॒वाना॒न्तेनै॒वैनां᳚ प॒दि ब॑ध्नाति पू॒षाध्व॑नः पा॒त्वित्या॑हे॒यं वै पू॒षेमामे॒वास्या॑ अधि॒पाम॑कः॒ सम॑ष्ट्या॒ इन्द्रा॒याध्य॑क्षा॒येत्या॒हेन्द्र॑मे॒वास्या॒ अध्य॑क्षं करोति॥५१॥

%6.1.7.7
अनु॑ त्वा मा॒ता म॑न्यता॒मनु॑ पि॒तेत्या॒हानु॑मतयै॒वैन॑या क्रीणाति॒ सा दे॑वि दे॒वमच्छे॒हीत्या॑ह दे॒वी ह्ये॑षा दे॒वः सोम॒ इन्द्रा॑य॒ सोम॒मित्या॒हेन्द्रा॑य॒ हि सोम॑ आह्रि॒यते॒ यदे॒तद्यजु॒र्न ब्रू॒यात्परा᳚च्ये॒व सो॑म॒क्रय॑णीयाद्रु॒द्रस्त्वा व॑र्तय॒त्वित्या॑ह रु॒द्रो वै क्रू॒रः॥५२॥

%6.1.7.8
दे॒वाना॒न्तमे॒वास्यै॑ प॒रस्ता᳚द्दधा॒त्यावृ॑त्त्यै क्रू॒रमि॑व॒ वा ए॒तत्क॑रोति॒ यद्रु॒द्रस्य॑ की॒र्तय॑ति मि॒त्रस्य॑ प॒थेत्या॑ह॒ शान्त्यै॑ वा॒चा वा ए॒ष वि क्री॑णीते॒ यः सो॑म॒क्रय॑ण्या स्व॒स्ति सोम॑सखा॒ पुन॒रेहि॑ स॒ह र॒य्येत्या॑ह वा॒चैव वि॒क्रीय॒ पुन॑रा॒त्मन्वाचं॑ ध॒त्ते\-ऽनु॑पदासुकास्य॒ वाग्भ॑वति॒ य ए॒वं वेद॑॥५३॥

%6.1.8.0
{\anuvakamend[{सत॑नुं॒ विष्ण॑व॒ इत्या॑ह स॒मारो॑हति॒ ध्याय॑ति॒ तद्वा॒चा यज॑मानः स्यात्करोति क्रू॒रो वेद॑}]}%॥७॥

%6.1.8.1
षट्प॒दान्यनु॒ नि क्रा॑मति षड॒हं वाङ्नाति॑ वदत्यु॒त सं॑वथ्स॒रस्याय॑ने॒ याव॑त्ये॒व वाक्तामव॑ रुन्द्धे सप्त॒मे प॒दे जु॑होति स॒प्तप॑दा॒ शक्व॑री प॒शवः॒ शक्व॑री प॒शूने॒वाव॑ रुन्द्धे स॒प्त ग्रा॒म्याः प॒शवः॑ स॒प्तार॒ण्याः स॒प्त छन्दाꣳ॑स्यु॒भय॒स्याव॑रुद्ध्यै॒ वस्व्य॑सि रु॒द्रासीत्या॑ह रू॒पमे॒वास्या॑ ए॒तन्म॑हि॒मानम्᳚॥५४॥

%6.1.8.2
व्याच॑ष्टे॒ बृह॒स्पति॑स्त्वा सु॒म्ने र॑ण्व॒त्वित्या॑ह॒ ब्रह्म॒ वै दे॒वाना॒म्बृह॒स्पति॒र्ब्रह्म॑णै॒वास्मै॑ प॒शूनव॑ रुन्द्धे रु॒द्रो वसु॑भि॒रा चि॑के॒त्वित्या॒हावृ॑त्त्यै पृथि॒व्यास्त्वा॑ मू॒र्धन्ना जि॑घर्मि देव॒यज॑न॒ इत्या॑ह पृथि॒व्या ह्ये॑ष मू॒र्धा यद्दे॑व॒यज॑न॒मिडा॑याः प॒द इत्या॒हेडा॑यै॒ ह्ये॑तत्प॒दं यथ्सो॑म॒क्रय॑ण्यै घृ॒तव॑ति॒ स्वाहा᳚॥५॥

%6.1.8.3
इत्या॑ह॒ यदे॒वास्यै॑ प॒दाद्घृ॒तमपी᳚ड्यत॒ तस्मा॑दे॒वमा॑ह॒ यद॑ध्व॒र्युर॑न॒ग्नावाहु॑तिं जुहु॒याद॒न्धो᳚\-ऽध्व॒र्युः स्या॒द्रक्षाꣳ॑सि य॒ज्ञꣳ ह॑न्यु॒र्\mbox{}हिर॑ण्यमु॒पास्य॑ जुहोत्यग्नि॒वत्ये॒व जु॑होति॒ नान्धो᳚\-ऽध्व॒र्युर्भव॑ति॒ न य॒ज्ञꣳ रक्षाꣳ॑सि घ्नन्ति॒ काण्डे॑काण्डे॒ वै क्रि॒यमा॑णे य॒ज्ञꣳ रक्षाꣳ॑सि जिघाꣳसन्ति॒ परि॑लिखित॒ꣳ॒ रक्षः॒ परि॑लिखिता॒ अरा॑तय॒ इत्या॑ह॒ रक्ष॑सा॒मप॑हत्यै॥५६॥

%6.1.8.4
इ॒दम॒हꣳ रक्ष॑सो ग्री॒वा अपि॑ कृन्तामि॒ यो᳚\-ऽस्मान्द्वेष्टि॒ यं च॑ व॒यं द्वि॒ष्म इत्या॑ह॒ द्वौ वाव पुरु॑षौ॒ यं चै॒व द्वेष्टि॒ यश्चै॑नं॒ द्वेष्टि॒ तयो॑रे॒वान॑न्तरायं ग्री॒वाः कृ॑न्तति प॒शवो॒ वै सो॑म॒क्रय॑ण्यै प॒दं या॑वत्त्मू॒तꣳ सं व॑पति प॒शूने॒वाव॑ रुन्द्धे॒\-ऽस्मे राय॒ इति॒ सं व॑पत्या॒त्मान॑मे॒वाध्व॒र्युः॥५७॥

%6.1.8.5
प॒शुभ्यो॒ नान्तरे॑ति॒ त्वे राय॒ इति॒ यज॑मानाय॒ प्र य॑च्छति॒ यज॑मान ए॒व र॒यिन्द॑धाति॒ तोते॒ राय॒ इति॒ पत्नि॑या अ॒र्धो वा ए॒ष आ॒त्मनो॒ यत्पत्नी॒ यथा॑ गृ॒हेषु॑ निध॒त्ते ता॒दृगे॒व तत्त्वष्टी॑मती ते सपे॒येत्या॑ह॒ त्वष्टा॒ वै प॑शू॒नाम्मि॑थु॒नानाꣳ॑ रूप॒कृद्रू॒पमे॒व प॒शुषु॑ दधात्य॒स्मै वै लो॒काय॒ गार्\mbox{}ह॑पत्य॒ आ धी॑यते॒\-ऽमुष्मा॑ आहव॒नीयो॒ यद्गार्\mbox{}ह॑पत्य उप॒वपे॑द॒स्मिल्लोँ॒के प॑शु॒मान्थ्स्या॒द्यदा॑हव॒नीये॒\-ऽमुष्मि॑ल्लोँ॒के प॑शु॒मान्थ्स्या॑दु॒भयो॒रुप॑ वपत्यु॒भयो॑रे॒वैन॑ल्लोँ॒कयोः᳚ पशु॒मन्तं॑ करोति॥५८॥

%6.1.9.0
{\anuvakamend[{म॒हि॒मान॒ꣴ॒ स्वाहाप॑हत्या अध्व॒र्युर्धी॑यते॒ चतु॑र्विHꣳशतिश्च}]}%॥८॥

%6.1.9.1
ब्र॒ह्म॒वा॒दिनो॑ वदन्ति वि॒चित्यः॒ सोमा (३) न वि॒चित्या (३) इति॒ सोमो॒ वा ओष॑धीना॒ꣳ॒ राजा॒ तस्मि॒न् यदाप॑न्नं ग्रसि॒तमे॒वास्य॒ तद्यद्वि॑चिनु॒याद्यथा॒स्या᳚द्ग्रसि॒तं नि॑ष्खि॒दति॑ ता॒दृगे॒व तद्यन्न वि॑चिनु॒याद्यथा॒क्षन्नाप॑न्नं वि॒धाव॑ति ता॒दृगे॒व तत्क्षोधु॑को\-ऽध्व॒र्युः स्यात्क्षोधु॑को॒ यज॑मानः॒ सोम॑विक्रयि॒न्थ्सोमꣳ॑ शोध॒येत्ये॒व ब्रू॑या॒द्यदीत॑रम्॥५९॥

%6.1.9.2
यदीत॑रमु॒भये॑नै॒व सो॑मविक्र॒यिण॑मर्पयति॒ तस्मा᳚थ्सोमविक्र॒यी क्षोधु॑को\-ऽरु॒णो ह॑ स्मा॒हौप॑वेशिः सोम॒क्रय॑ण ए॒वाहं तृ॑तीयसव॒नमव॑ रुन्ध॒ इति॑ पशू॒नां चर्म॑न्मिमीते प॒शूने॒वाव॑ रुन्द्धे प॒शवो॒ हि तृ॒तीय॒ꣳ॒ सव॑नं॒ यङ्का॒मये॑ताप॒शुः स्या॒दित्यृ॑क्ष॒तस्तस्य॑ मिमीत॒र्क्षं वा अ॑पश॒व्यम॑प॒शुरे॒व भ॑वति॒ यं का॒मये॑त पशु॒मान्थ्स्या᳚त्॥६०॥

%6.1.9.3
इति॑ लोम॒तस्तस्य॑ मिमीतै॒तद्वै प॑शू॒नाꣳ रू॒पꣳ रू॒पेणै॒वास्मै॑ प॒शूनव॑ रुन्द्धे पशु॒माने॒व भ॑वत्य॒पामन्ते᳚ क्रीणाति॒ सर॑समे॒वैनं॑ क्रीणात्य॒मात्यो॒\-ऽसीत्या॑हा॒मैवैनं॑ कुरुते शु॒क्रस्ते॒ ग्रह॒ इत्या॑ह शु॒क्रो ह्य॑स्य॒ ग्रहो\-ऽन॒साच्छ॑ याति महि॒मान॑मे॒वास्याच्छ॑ या॒त्यन॑सा॥६१॥

%6.1.9.4
अच्छ॑ याति॒ तस्मा॑दनोवा॒ह्यꣳ॑ स॒मे जीव॑नं॒ यत्र॒ खलु॒ वा ए॒तꣳ शी॒र्\mbox{}ष्णा हर॑न्ति॒ तस्मा᳚च्छीर्\mbox{}षहा॒र्यं॑ गि॒रौ जीव॑नम॒भि त्यं दे॒वꣳ स॑वि॒तार॒मित्यति॑छन्दस॒र्चा मि॑मी॒ते\-ऽति॑च्छन्दा॒ वै सर्वा॑णि॒ छन्दाꣳ॑सि॒ सर्वे॑भिरे॒वैनं॒ छन्दो॑भिर्मिमीते॒ वर्\mbox{}ष्म॒ वा ए॒षा छन्द॑सां॒ यदति॑च्छन्दा॒ यदति॑च्छन्दस॒र्चा मिमी॑ते॒ वर्\mbox{}ष्मै॒वैनꣳ॑ समा॒नानां᳚ करो॒त्येक॑यैकयो॒थ्सर्गम्᳚॥६२॥

%6.1.9.5
मि॒मी॒ते\-ऽया॑तयाम्नियायातयाम्नियै॒वैन॑म्मिमीते॒ तस्मा॒न्नाना॑वीर्या अ॒ङ्गुल॑यः॒ सर्वा᳚स्वङ्गु॒ष्ठमुप॒ नि गृ॑ह्णाति॒ तस्मा᳚थ्स॒माव॑द्वीर्यो॒\-ऽन्याभि॑र॒ङ्गुलि॑भि॒स्तस्मा॒थ्सर्वा॒ अनु॒ सं च॑रति॒ यथ्स॒ह सर्वा॑भि॒र्मिमी॑त॒ सꣴश्लि॑ष्टा अ॒ङ्गुल॑यो जायेर॒न्नेक॑यैकयो॒थ्सर्ग॑म्मिमीते॒ तस्मा॒द्विभ॑क्ता जायन्ते॒ पञ्च॒ कृत्वो॒ यजु॑षा मिमीते॒ पञ्चा᳚क्षरा प॒ङ्क्तिः पाङ्क्तो॑ य॒ज्ञो य॒ज्ञमे॒वाव॑ रुन्द्धे॒ पञ्च॒ कृत्व॑स्तू॒ष्णीम्॥६३॥

%6.1.9.6
दश॒ सम्प॑द्यन्ते॒ दशा᳚क्षरा वि॒राडन्नं॑ वि॒राड्वि॒राजै॒वान्नाद्य॒मव॑ रुन्द्धे॒ यद्यजु॑षा॒ मिमी॑ते भू॒तमे॒वाव॑ रुन्द्धे॒ यत्तू॒ष्णीम्भ॑वि॒ष्यद्यद्वै तावा॑ने॒व सोमः॒ स्याद्याव॑न्त॒म्मिमी॑ते॒ यज॑मानस्यै॒व स्या॒न्नापि॑ सद॒स्या॑नां प्र॒जाभ्य॒स्त्वेत्युप॒ समू॑हति सद॒स्या॑ने॒वान्वाभ॑जति॒ वास॒सोप॑ नह्यति सर्वदेव॒त्यं॑ वै॥६४॥

%6.1.9.7
वासः॒ सर्वा॑भिरे॒वैनं॑ दे॒वता॑भिः॒ सम॑र्धयति प॒शवो॒ वै सोमः॑ प्रा॒णाय॒ त्वेत्युप॑ नह्यति प्रा॒णमे॒व प॒शुषु॑ दधाति व्या॒नाय॒ त्वेत्यनु॑ शृन्थति व्या॒नमे॒व प॒शुषु॑ दधाति॒ तस्मा᳚थ्स्व॒पन्तं॑ प्रा॒णा न ज॑हति॥६५॥

%6.1.10.0
{\anuvakamend[{इत॑रम्पशु॒मान्थ्स्या᳚द्या॒त्यन॑सो॒थ्सर्ग॑न्तू॒ष्णीꣳ स॑र्वदेव॒त्यं॑ वै त्रय॑स्त्रिꣳशच्च}]}%॥९॥

%6.1.10.1
यत्क॒लया॑ ते श॒फेन॑ ते क्रीणा॒नीति॒ पणे॒तागो॑अर्घ॒ꣳ॒ सोमं॑ कु॒र्यादगो॑अर्घं॒ यज॑मान॒मगो॑अर्घमध्व॒र्युङ्गोस्तु म॑हि॒मानं॒ नाव॑ तिरे॒द्गवा॑ ते क्रीणा॒नीत्ये॒व ब्रू॑याद्गोअ॒र्घमे॒व सोमं॑ क॒रोति॑ गोअ॒र्घं यज॑मानं गोअ॒र्घम॑ध्व॒र्युन्न गोर्म॑हि॒मान॒मव॑ तिरत्य॒जया᳚ क्रीणाति॒ सत॑पसमे॒वैनं॑ क्रीणाति॒ हिर॑ण्येन क्रीणाति॒ सशु॑क्रमे॒व॥६६॥

%6.1.10.2
ए॒नं॒ क्री॒णा॒ति॒ धे॒न्वा क्री॑णाति॒ साशि॑रमे॒वैनं॑ क्रीणात्यृष॒भेण॑ क्रीणाति॒ सेन्द्र॑मे॒वैनं॑ क्रीणात्यन॒डुहा᳚ क्रीणाति॒ वह्नि॒र्वा अ॑न॒ड्वान् वह्नि॑नै॒व वह्नि॑ य॒ज्ञस्य॑ क्रीणाति मिथु॒ना\-भ्यां᳚ क्रीणाति मिथु॒नस्याव॑रुद्ध्यै॒ वास॑सा क्रीणाति सर्वदेव॒त्यं॑ वै वासः॒ सर्वा᳚भ्य ए॒वैनं॑ दे॒वता᳚भ्यः क्रीणाति॒ दश॒ सम्प॑द्यन्ते॒ दशा᳚क्षरा वि॒राडन्नं॑ वि॒राड्वि॒राजै॒वान्नाद्य॒मव॑ रुन्द्धे॥६७॥

%6.1.10.3
तप॑सस्त॒नूर॑सि प्र॒जाप॑ते॒र्वर्ण॒ इत्या॑ह प॒शुभ्य॑ ए॒व तद॑ध्व॒र्युर्नि ह्नु॑त आ॒त्मनो\-ऽना᳚व्रस्काय॒ गच्छ॑ति॒ श्रिय॒म्प्र प॒शूना᳚प्नोति॒ य ए॒वं वेद॑ शु॒क्रं ते॑ शु॒क्रेण॑ क्रीणा॒मीत्या॑ह यथाय॒जुरे॒वैतद्दे॒वा वै येन॒ हिर॑ण्येन॒ सोम॒मक्री॑ण॒न्तद॑भी॒षहा॒ पुन॒राद॑दत॒ को हि तेज॑सा विक्रे॒ष्यत॒ इति॒ येन॒ हिर॑ण्येन॥६८॥

%6.1.10.4
सोमं॑ क्रीणी॒यात्तद॑भी॒षहा॒ पुन॒रा द॑दीत॒ तेज॑ ए॒वात्मन्ध॑त्ते॒\-ऽस्मे ज्योतिः॑ सोमविक्र॒यिणि॒ तम॒ इत्या॑ह॒ ज्योति॑रे॒व यज॑माने दधाति॒ तम॑सा॒ सोमविक्र॒यिण॑मर्पयति॒ यदनु॑पग्रथ्य ह॒न्याद्द॑न्द॒शूका॒स्ताꣳ समाꣳ॑ स॒र्पाः स्यु॑रि॒दम॒हꣳ स॒र्पाणां᳚ दन्द॒शूका॑नां ग्री॒वा उप॑ ग्रथ्ना॒मीत्या॒हाद॑न्दशूका॒स्ताꣳ समाꣳ॑ स॒र्पा भ॑वन्ति॒ तम॑सा सोमविक्र॒यिणं॑ विध्यति॒ स्वान॑॥६९॥

%6.1.10.5
भ्राजेत्या॑है॒ते वा अ॒मुष्मि॑ल्लोँ॒के सोम॑मरक्ष॒न्तेभ्यो\-ऽधि॒ सोम॒माह॑र॒न् यदे॒तेभ्यः॑ सोम॒क्रय॑णा॒न्नानु॑दि॒शेदक्री॑तो\-ऽस्य॒ सोमः॑ स्या॒न्नास्यै॒ते॑\-ऽमुष्मि॑ल्लोँ॒के सोमꣳ॑ रक्षेयु॒र्यदे॒तेभ्यः॑ सोम॒क्रय॑णाननुदि॒शति॑ क्री॒तो᳚\-ऽस्य॒ सोमो॑ भवत्ये॒ते᳚\-ऽस्या॒मुष्मि॑ल्लोँ॒के सोमꣳ॑ रक्षन्ति॥७०॥

%6.1.11.0
{\anuvakamend[{सशु॑क्रमे॒व रु॑न्ध॒ इति॒ येन॒ हिर॑ण्येन॒ स्वान॒ चतु॑श्चत्वारिꣳशच्च}]}%॥10॥

%6.1.11.1
वा॒रु॒णो वै क्री॒तः सोम॒ उप॑नद्धो मि॒त्रो न॒ एहि॒ सुमि॑त्रधा॒ इत्या॑ह॒ शान्त्या॒ इन्द्र॑स्यो॒रुमा वि॑श॒ दक्षि॑ण॒मित्या॑ह दे॒वा वै यꣳ सोम॒मक्री॑ण॒न्तमिन्द्र॑स्यो॒रौ दक्षि॑ण॒ आसा॑दयन्ने॒ष खलु॒ वा ए॒तर्\mbox{}हीन्द्रो॒ यो यज॑ते॒ तस्मा॑दे॒वमा॒होदायु॑षा स्वा॒युषेत्या॑ह दे॒वता॑ ए॒वान्वा॒रभ्योत्॥७१॥

%6.1.11.2
ति॒ष्ठ॒त्यु॒र्व॑न्तरि॑क्ष॒मन्वि॒हीत्या॑हान्तरिक्षदेव॒त्यो  ह्ये॑तर्\mbox{}हि॒ सोमो\-ऽदि॑त्याः॒ सदो॒\-ऽस्यदि॑त्याः॒ सद॒ आ सी॒देत्या॑ह यथाय॒जुरे॒वैतद्वि वा ए॑नमे॒तद॑र्धयति॒ यद्वा॑रु॒णꣳ सन्त॑म्मै॒त्रं क॒रोति॑ वारु॒ण्यर्चा सा॑दयति॒ स्वयै॒वैनं॑ दे॒वत॑या॒ सम॑र्धयति॒ वास॑सा प॒र्यान॑ह्यति सर्वदेव॒त्यं॑ वै वासः॒ सर्वा॑भिरे॒व॥७२॥

%6.1.11.3
ए॒नं॒ दे॒वता॑भिः॒ सम॑र्धय॒त्यथो॒ रक्ष॑सा॒मप॑हत्यै॒ वने॑षु॒ व्य॑न्तरि॑क्षं तता॒नेत्या॑ह॒ वने॑षु॒ हि व्य॑न्तरि॑क्षं त॒तान॒ वाज॒मर्व॒थ्स्वित्या॑ह॒ वाज॒ꣴ॒ ह्यर्व॑थ्सु॒ पयो॑ अघ्नि॒यास्वित्या॑ह॒ पयो॒ ह्य॑घ्नि॒यासु॑ हृ॒थ्सु क्रतु॒मित्या॑ह हृ॒थ्सु हि क्रतुं॒ वरु॑णो वि॒क्ष्व॑ग्निमित्या॑ह॒ वरु॑णो॒ हि वि॒क्ष्व॑ग्निन्दि॒वि सूर्यम्᳚॥७३॥

%6.1.11.4
इत्या॑ह दि॒वि हि सूर्य॒ꣳ॒ सोम॒मद्रा॒वित्या॑ह॒ ग्रावा॑णो॒ वा अद्र॑य॒स्तेषु॒ वा ए॒ष सोमं॑ दधाति॒ यो यज॑ते॒ तस्मा॑दे॒वमा॒होदु॒ त्यं जा॒तवे॑दस॒मिति॑ सौ॒र्यर्चा कृ॑ष्णाजि॒नम्प्र॒त्यान॑ह्यति॒ रक्ष॑सा॒मप॑हत्या॒ उस्रा॒वेतं॑ धूर्\mbox{}षाहा॒वित्या॑ह यथाय॒जुरे॒वैतत्प्र च्य॑वस्व भुवस्पत॒ इत्या॑ह भू॒ताना॒ꣳ॒ हि॥७४॥

%6.1.11.5
ए॒ष पति॒र्विश्वा᳚न्य॒भि धामा॒नीत्या॑ह॒ विश्वा॑नि॒ ह्ये  षो॑\-ऽभि धामा॑नि प्र॒च्यव॑ते॒ मा त्वा॑ परिप॒री वि॑द॒दित्या॑ह॒ यदे॒वादः सोम॑माह्रि॒यमा॑णं गन्ध॒र्वो वि॒श्वाव॑सुः प॒र्यमु॑ष्णा॒त्तस्मा॑दे॒वमा॒हाप॑रिमोषाय॒ यज॑मानस्य स्व॒स्त्यय॑न्य॒सीत्या॑ह॒ यज॑मानस्यै॒वैष य॒ज्ञस्या᳚न्वार॒म्भो\-ऽन॑वछित्त्यै॒ वरु॑णो॒ वा ए॒ष यज॑मानम॒भ्यैति॒ यत्॥७५॥

%6.1.11.6
क्री॒तः सोम॒ उप॑नद्धो॒ नमो॑ मि॒त्रस्य॒ वरु॑णस्य॒ चक्ष॑स॒ इत्या॑ह॒ शान्त्या॒ आ सोमं॒ वह॑न्त्य॒ग्निना॒ प्रति॑ तिष्ठते॒ तौ स॒म्भव॑न्तौ॒ यज॑मानम॒भि सम्भ॑वतः पु॒रा खलु॒ वावैष मेधा॑या॒त्मान॑मा॒रभ्य॑ चरति॒ यो दी᳚क्षि॒तो यद॑ग्नीषो॒मीय॑म्प॒शुमा॒लभ॑त आत्मनि॒ष्क्रय॑ण ए॒वास्य॒ स तस्मा॒त्तस्य॒ नाश्यं॑ पुरुषनि॒ष्क्रय॑ण इव॒ ह्यथो॒ खल्वा॑हुर॒ग्नीषोमा᳚भ्यां॒ वा इन्द्रो॑ वृ॒त्रम॑ह॒न्निति॒ यद॑ग्नीषो॒मीय॑म्प॒शुमा॒लभ॑ते॒ वार्त्र॑घ्न ए॒वास्य॒ स तस्मा᳚द्वा॒श्यं॑ वारु॒ण्यर्चा परि॑ चरति॒ स्वयै॒वैनं॑ दे॒वत॑या॒ परि॑ चरति॥७६॥

%6.2.0.0
{\anuvakamend[{अ॒न्वा॒रभ्योथ्सर्वा॑भिरे॒व सूर्यं॑ भू॒ताना॒ꣳ॒ ह्ये॑ति॒ यदा॑हुः स॒प्तविꣳ॑शतिश्च}]}%॥11॥

%6.2.0.0

{\anuvakamend[{यदु॒भौ दे॑वासु॒रा मि॒थस्तेषाꣳ॑ सुव॒र्गं यद्वा अनी॑शानः पु॒रोह॑विषि॒ तेभ्यः॒ सोत्त॑रवे॒दिर्ब॒द्धं दे॒वस्याभ्रि॒ꣳ॒ शिरो॒ वा एका॑दश}]}%॥11॥
\prashnaend{यदु॒भावित्या॑ह दे॒वाना᳚ं य॒ज्ञो दे॒वेभ्यो॒ न रथा॑य॒ यज॑मानाय प॒रस्ता॑द॒र्वाची॒न्नव॑पञ्चा॒शत्॥59॥ यदु॒भौ दु॒ह ए॒वैना᳚म्॥}
%%% END PRASHNA

\sect{द्वितीयः प्रश्नः}\setcounter{anuvakam}{0}
\dnsub{तैत्तिरीयसंहितायां षष्ठमकाण्डे द्वितीयः प्रश्नः}
%6.2.1.0
%6.2.1.1
यदु॒भौ वि॒मुच्या॑ति॒थ्यं गृ॑ह्णी॒याद्य॒ज्ञं विच्छि॑न्द्या॒द्यदु॒भाववि॑मुच्य॒ यथाना॑गतायाति॒थ्यं क्रि॒यते॑ ता॒दृगे॒व तद्विमु॑क्तो॒\-ऽन्यो॑\-ऽन॒ड्वान्भव॒त्यवि॑मुक्तो॒\-ऽन्यो\-ऽथा॑ति॒थ्यं गृ॑ह्णाति य॒ज्ञस्य॒ सन्त॑त्यै॒ पत्न्य॒न्वार॑भते॒ पत्नी॒ हि पारी॑णह्य॒स्येशे॒ पत्नि॑यै॒वानु॑मतं॒ निर्व॑पति॒ यद्वै पत्नी॑ य॒ज्ञस्य॑ क॒रोति॑ मिथु॒नं तदथो॒ पत्नि॑या ए॒व॥१॥

%6.2.1.2
ए॒ष य॒ज्ञस्या᳚न्वार॒म्भो\-ऽन॑वच्छित्त्यै॒ याव॑द्भि॒र्वै राजा॑नुच॒रैरा॒गच्छ॑ति॒ सर्वे᳚भ्यो॒ वै तेभ्य॑ आति॒थ्यं क्रि॑यते॒ छन्दाꣳ॑सि॒ खलु॒ वै सोम॑स्य॒ राज्ञो॑\-ऽनुच॒राण्य॒ग्नेरा॑ति॒थ्यम॑सि॒ विष्ण॑वे॒ त्वेत्या॑ह गायत्रि॒या ए॒वैतेन॑ करोति॒ सोम॑स्याति॒थ्यम॑सि॒ विष्ण॑वे॒ त्वेत्या॑ह त्रि॒ष्टुभ॑ ए॒वैतेन॑ करो॒त्यति॑थेराति॒थ्यम॑सि॒ विष्ण॑वे॒ त्वेत्या॑ह॒ जग॑त्यै॥२॥

%6.2.1.3
ए॒वैतेन॑ करोत्य॒ग्नये᳚ त्वा रायस्पोष॒दाव्ने॒ विष्ण॑वे॒ त्वेत्या॑हानु॒ष्टुभ॑ ए॒वैतेन॑ करोति श्ये॒नाय॑ त्वा सोम॒भृते॒ विष्ण॑वे॒ त्वेत्या॑ह गायत्रि॒या ए॒वैतेन॑ करोति॒ पञ्च॒ कृत्वो॑ गृह्णाति॒ पञ्चा᳚क्षरा प॒ङ्क्तिः पाङ्क्तो॑ य॒ज्ञो य॒ज्ञमे॒वाव॑ रुन्द्धे ब्रह्मवा॒दिनो॑ वदन्ति॒ कस्मा᳚थ्स॒त्याद्गा॑यत्रि॒या उ॑भ॒यत॑ आति॒थ्यस्य॑ क्रियत॒ इति॒ यदे॒वादः सोम॒मा॥३॥

%6.2.1.4
अह॑र॒त्तस्मा᳚द्गायत्रि॒या उ॑भ॒यत॑ आति॒थ्यस्य॑ क्रियते पु॒रस्ता᳚च्चो॒परि॑ष्टाच्च॒ शिरो॒ वा ए॒तद्य॒ज्ञस्य॒ यदा॑ति॒थ्यं नव॑कपालः पुरो॒डाशो॑ भवति॒ तस्मा᳚न्नव॒धा शिरो॒ विष्यू॑त॒न्नव॑कपालः पुरो॒डाशो॑ भवति॒ ते त्रय॑स्त्रिकपा॒लास्त्रि॒वृता॒ स्तोमे॑न॒ सम्मि॑ता॒स्तेज॑स्त्रि॒वृत्तेज॑ ए॒व य॒ज्ञस्य॑ शी॒र्\mbox{}षन्द॑धाति॒ नव॑कपालः पुरो॒डाशो॑ भवति॒ ते त्रय॑स्त्रिकपा॒लास्त्रि॒वृता᳚ प्रा॒णेन॒ सम्मि॑तास्त्रि॒वृद्वै॥४॥

%6.2.1.5
प्रा॒णस्त्रि॒वृत॑मे॒व प्रा॒णम॑भिपू॒र्वं य॒ज्ञस्य॑ शी॒र्\mbox{}षन्द॑धाति प्र॒जाप॑ते॒र्वा ए॒तानि॒ पक्ष्मा॑णि॒ यद॑श्ववा॒ला ऐ᳚क्ष॒वी ति॒रश्ची॒ यदाश्व॑वालः प्रस्त॒रो भव॑त्यैक्ष॒वी ति॒रश्ची᳚ प्र॒जाप॑तेरे॒व तच्चक्षुः॒ सम्भ॑रति दे॒वा वै या आहु॑ती॒रजु॑हवु॒स्ता असु॑रा नि॒ष्काव॑माद॒न्ते दे॒वाः का᳚र्\mbox{}ष्म॒र्य॑मपश्यन्कर्म॒ण्यो॑ वै कर्मै॑नेन कुर्वी॒तेति॒ ते का᳚र्ष्मर्य॒मया᳚न्परि॒धीन्॥५॥

%6.2.1.6
अ॒कु॒र्व॒त॒ तैर्वै ते रक्षा॒ꣳ॒स्यपा᳚घ्नत॒ यत्का᳚र्ष्मर्य॒मयाः᳚ परि॒धयो॒ भव॑न्ति॒ रक्ष॑सा॒मप॑हत्यै॒ सꣴस्प॑र्शयति॒ रक्ष॑सा॒मन॑न्ववचाराय॒ न पु॒रस्ता॒त्परि॑ दधात्यादि॒त्यो ह्ये॑वोद्यन्पु॒रस्ता॒द्रक्षाꣳ॑स्यप॒हन्त्यू॒र्ध्वे स॒मिधा॒वा द॑धात्यु॒परि॑ष्टादे॒व रक्षा॒ꣳ॒स्यप॑हन्ति॒ यजु॑षा॒न्यां तू॒ष्णीम॒न्याम्मि॑थुन॒त्वाय॒ द्वे आ द॑धाति द्वि॒पाद्यज॑मानः॒ प्रति॑ष्ठित्यै ब्रह्मवा॒दिनो॑ वदन्ति॥६॥

%6.2.1.7
अ॒ग्निश्च॒ वा ए॒तौ सोम॑श्च क॒था सोमा॑याति॒थ्यं क्रि॒यते॒ नाग्नय॒ इति॒ यद॒ग्नाव॒ग्निम्म॑थि॒त्वा प्र॒हर॑ति॒ तेनै॒वाग्नय॑ आति॒थ्यं क्रि॑य॒ते\-ऽथो॒ खल्वा॑हुर॒ग्निः सर्वा॑ दे॒वता॒ इति॒ यद्ध॒विरा॒साद्या॒ग्निम्मन्थ॑ति ह॒व्यायै॒वास॑न्नाय॒ सर्वा॑ दे॒वता॑ जनयति॥७॥

%6.2.2.0
{\anuvakamend[{पत्नि॑या ए॒व जग॑त्या॒ आ त्रि॒वृद्वै प॑रि॒धीन् व॑द॒न्त्येक॑चत्वारिꣳशच्च}]}%॥१॥

%6.2.2.1
दे॒वा॒सु॒राः संय॑त्ता आस॒न्ते दे॒वा मि॒थो विप्रि॑या आस॒न्ते\-ऽ  न्यो᳚न्यस्मै॒ ज्यैष्ठ्या॒याति॑ष्ठमानाः पञ्च॒धा व्य॑क्रामन्न॒ग्नि\-र्वसु॑भिः॒ सोमो॑ रु॒द्रैरिन्द्रो॑ म॒रुद्भि॒र्वरु॑ण आदि॒त्यैर्बृह॒स्पति॒र्विश्वै᳚र्दे॒वैस्ते॑\-ऽमन्य॒न्तासु॑रेभ्यो॒ वा इ॒दम्भ्रातृ॑व्येभ्यो रध्यामो॒ यन्मि॒थो विप्रि॑याः॒ स्मो या न॑ इ॒माः प्रि॒यास्त॒नुव॒स्ताः स॒मव॑द्यामहै॒ ताभ्यः॒ स निर्\mbox{}ऋ॑च्छा॒द्यः॥८॥

%6.2.2.2
नः॒ प्र॒थ॒मो\-ऽ  न्यो᳚न्यस्मै॒ द्रुह्या॒दिति॒ तस्मा॒द्यः सता॑नूनप्त्रिणाम्प्रथ॒मो द्रुह्य॑ति॒ स आर्ति॒मार्च्छ॑ति॒ यत्ता॑नून॒प्त्रꣳ स॑मव॒द्यति॒ भ्रातृ॑व्याभिभूत्यै॒ भव॑त्या॒त्मना॒ परा᳚स्य॒ भ्रातृ॑व्यो भवति॒ पञ्च॒ कृत्वो\-ऽव॑ द्यति पञ्च॒धा हि ते तथ्स॑म॒वाद्य॒न्ताथो॒ पञ्चा᳚क्षरा प॒ङ्क्तिः पाङ्क्तो॑ य॒ज्ञो य॒ज्ञमे॒वाव॑ रुन्द्ध॒ आप॑तये त्वा गृह्णा॒मीत्या॑ह प्रा॒णो वै॥९॥

%6.2.2.3
आप॑तिः प्रा॒णमे॒व प्री॑णाति॒ परि॑पतय॒ इत्या॑ह॒ मनो॒ वै परि॑पति॒र्मन॑ ए॒व प्री॑णाति॒ तनू॒नप्त्र॒ इत्या॑ह त॒नुवो॒ हि ते ताः स॑म॒वाद्य॑न्त शाक्व॒रायेत्या॑ह॒ शक्त्यै॒ हि ते ताः स॑म॒वाद्य॑न्त॒ शक्म॒न्नोजि॑ष्ठा॒येत्या॒हौजि॑ष्ठ॒ꣳ॒ हि ते तदा॒त्मनः॑ सम॒वाद्य॒न्ताना॑धृष्टमस्यनाधृ॒ष्यमित्या॒हाना॑धृष्ट॒ꣴ॒ ह्ये॑तद॑नाधृ॒ष्यं दे॒वाना॒मोजः॑॥१०॥

%6.2.2.4
इत्या॑ह दे॒वाना॒ꣳ॒ ह्ये॑तदोजो॑\-ऽभिशस्ति॒पा अ॑नभिशस्ते॒न्यमित्या॑हाभिशस्ति॒पा ह्ये॑तद॑नभिशस्ते॒न्यमनु॑ मे दी॒क्षां दी॒क्षाप॑तिर्मन्यता॒मित्या॑ह यथाय॒जुरे॒वैतद्घृ॒तं वै दे॒वा वज्रं॑ कृ॒त्वा सोम॑मघ्नन्नन्ति॒कमि॑व॒ खलु॒ वा अ॑स्यै॒तच्च॑रन्ति॒ यत्ता॑नून॒प्त्रेण॑ प्र॒चर॑न्त्य॒ꣳ॒शुरꣳ॑शुस्ते देव सो॒मा प्या॑यता॒मित्या॑ह॒ यत्॥११॥

%6.2.2.5
ए॒वास्या॑पुवा॒यते॒ यन्मीय॑ते॒ तदे॒वास्यै॒तेना प्या॑यय॒त्या तुभ्य॒मिन्द्रः॑ प्यायता॒मा त्वमिन्द्रा॑य प्याय॒स्वेत्या॑हो॒भावे॒वेन्द्रं॑ च॒ सोमं॒ चा प्या॑यय॒त्या प्या॑यय॒ सखी᳚न्थ्स॒न्या मे॒धयेत्या॑ह॒र्त्विजो॒ वा अ॑स्य॒ सखा॑य॒स्ताने॒वा प्या॑ययति स्व॒स्ति ते॑ देव सोम सु॒त्याम॑शीय॥१२॥

%6.2.2.6
इत्या॑हा॒शिष॑मे॒वैतामा शा᳚स्ते॒ प्र वा ए॒ते᳚\-ऽस्माल्लो॒काच्च्य॑वन्ते॒ ये सोम॑माप्या॒यय॑न्त्यन्तरिक्षदेव॒त्यो॑ हि सोम॒ आप्या॑यित॒ एष्टा॒ रायः॒ प्रेषे भगा॒येत्या॑ह॒ द्यावा॑पृथि॒वीभ्या॑मे॒व न॑म॒स्कृत्या॒स्मिल्लोँ॒के प्रति॑ तिष्ठन्ति देवासु॒राः संय॑त्ता आस॒न्ते दे॒वा बिभ्य॑तो॒\-ऽग्निम्प्रावि॑श॒न्तस्मा॑दाहुर॒ग्निः सर्वा॑ दे॒वता॒ इति॒ ते॥१३॥

%6.2.2.7
अ॒ग्निमे॒व वरू॑थं कृ॒त्वासु॑रान॒भ्य॑भवन्न॒ग्निमि॑व॒ खलु॒ वा ए॒ष प्र वि॑शति॒ यो॑\-ऽवान्तरदी॒क्षामु॒पैति॒ भ्रातृ॑व्याभिभूत्यै॒ भव॑त्या॒त्मना॒ परा᳚स्य॒ भ्रातृ॑व्यो भवत्या॒त्मान॑मे॒व दी॒क्षया॑ पाति प्र॒जाम॑वान्तरदी॒क्षया॑ सन्त॒राम्मेख॑लाꣳ स॒माय॑च्छते प्र॒जा ह्या᳚त्मनो\-ऽन्त॑रतरा त॒प्तव्र॑तो भवति॒ मद॑न्तीभिर्मार्जयते॒ निर्\mbox{}ह्य॑ग्निः शी॒तेन॒ वाय॑ति॒ समि॑द्ध्यै॒ या ते॑ अग्ने॒ रुद्रि॑या त॒नूरित्या॑ह॒ स्वयै॒वैन॑द्दे॒वत॑या व्रतयति सयोनि॒त्वाय॒ शान्त्यै᳚॥१४॥

%6.2.3.0
{\anuvakamend[{यो वा ओज॑ आह॒ यद॑शी॒येति॒ ते᳚\-ऽग्न॒ एका॑दश च}]}%॥२॥

%6.2.3.1
तेषा॒मसु॑राणान्ति॒स्रः पुर॑ आसन्नय॒स्मय्य॑व॒मा\-ऽथ॑ रज॒ता\-ऽथ॒ हरि॑णी॒ ता दे॒वा जेतु॒न्नाश॑क्नुव॒न्ता उ॑प॒सदै॒वाजि॑गीष॒न्तस्मा॑दाहु॒र्यश्चै॒वं वेद॒ यश्च॒ नोप॒सदा॒ वै म॑हापु॒रं ज॑य॒न्तीति॒ त इषु॒ꣳ॒ सम॑स्कुर्वता॒ग्निमनी॑क॒ꣳ॒ सोमꣳ॑ श॒ल्यं विष्णु॒न्तेज॑न॒न्ते᳚\-ऽब्रुव॒न्क इ॒माम॑सिष्य॒तीति॑॥१५॥

%6.2.3.2
रु॒द्र इत्य॑ब्रुवन्रु॒द्रो वै क्रू॒रः सो᳚\-ऽस्य॒त्विति॒ सो᳚\-ऽब्रवी॒द्वरं॑ वृणा अ॒हमे॒व प॑शू॒नामधि॑पतिरसा॒नीति॒ तस्मा᳚द्रु॒द्रः प॑शू॒नामधि॑पति॒स्ताꣳ रु॒द्रो\-ऽवा॑सृज॒थ्स ति॒स्रः पुरो॑ भि॒त्त्वैभ्यो लो॒केभ्यो\-ऽसु॑रा॒न्प्राणु॑दत॒ यदु॑प॒सद॑ उपस॒द्यन्ते॒ भ्रातृ॑व्यपराणुत्त्यै॒ नान्यामाहु॑तिम्पु॒रस्ता᳚ज्जुहुया॒द्यद॒न्यामाहु॑तिम्पु॒रस्ता᳚ज्जुहु॒यात्॥१६॥

%6.2.3.3
अ॒न्यन्मुखं॑ कुर्याथ्स्रु॒वेणा॑घा॒रमा घा॑रयति य॒ज्ञस्य॒ प्रज्ञा᳚त्यै॒ परा॑ङति॒क्रम्य॑ जुहोति॒ परा॑च ए॒वैभ्यो लो॒केभ्यो॒ यज॑मानो॒ भ्रातृ॑व्या॒न्प्र णु॑दते॒ पुन॑रत्या॒क्रम्यो॑प॒सदं॑ जुहोति प्र॒णुद्यै॒वैभ्यो लो॒केभ्यो॒ भ्रातृ॑व्याञ्जि॒त्वा भ्रा॑तृव्यलो॒कम॒भ्यारो॑हति दे॒वा वै याः प्रा॒तरु॑प॒सद॑ उ॒पासी॑द॒न्नह्न॒स्ताभि॒रसु॑रा॒न्प्राणु॑दन्त॒ याः सा॒यꣳ रात्रि॑यै॒ ताभि॒र्यथ्सा॒यम्प्रा॑तरुप॒सदः॑॥१७॥

%6.2.3.4
उ॒प॒स॒द्यन्ते॑\-ऽहोरा॒त्राभ्या॑मे॒व तद्यज॑मानो॒ भ्रातृ॑व्या॒न्प्र णु॑दते॒ याः प्रा॒तर्या॒ज्याः᳚ स्युस्ताः सा॒यम्पु॑रोनुवा॒क्याः᳚ कुर्या॒दया॑तयामत्वाय ति॒स्र उ॑प॒सद॒ उपै॑ति॒ त्रय॑ इ॒मे लो॒का इ॒माने॒व लो॒कान्प्री॑णाति॒ षट्थ्सम्प॑द्यन्ते॒ षड्वा ऋ॒तव॑ ऋ॒तूने॒व प्री॑णाति॒ द्वाद॑शा॒हीने॒ सोम॒ उपै॑ति॒ द्वाद॑श॒ मासाः᳚ संवथ्स॒रः सं॑वथ्स॒रमे॒व प्री॑णाति॒ चतु॑र्विꣳशतिः॒ सम्॥१८॥

%6.2.3.5
प॒द्य॒न्ते॒ चतु॑र्विꣳशतिरर्धमा॒सा अ॑र्धमा॒साने॒व प्री॑णा॒त्यारा᳚ग्रामवान्तरदी॒क्षामुपे॑या॒द्यः का॒मये॑ता॒स्मिन्मे॑ लो॒के\-ऽर्धु॑कꣴ स्या॒दित्येक॒मग्रे\-ऽथे॒ द्वावथ॒ त्रीनथ॑ च॒तुर॑ ए॒षा वा आरा᳚ग्रावान्तरदी॒क्षास्मिन्ने॒वास्मै॑ लो॒के\-ऽर्धु॑कम्भवति प॒रोव॑रीयसीमवान्तरदी॒क्षामुपे॑या॒द्यः का॒मये॑ता॒मुष्मि॑न्मे लो॒के\-ऽर्धु॑कꣴ स्या॒दिति॑ च॒तुरो\-ऽग्रे\-ऽथ॒ त्रीनथ॒ द्वावथैक॑मे॒षा वै प॒रोव॑रीयस्यवान्तरदी॒क्षामुष्मि॑न्ने॒वास्मै॑ लो॒के\-ऽर्धु॑कम्भवति॥१९॥

%6.2.4.0
{\anuvakamend[{अ॒सि॒ष्य॒तीति॑ जुहु॒याथ्सा॒यम्प्रा॑तरुप॒सद॒श्चतु॑र्विꣳशतिः॒ सञ्च॒तुरो\-ऽग्रे॒ षोड॑श च}]}%॥३॥

%6.2.4.1
सु॒व॒र्गं वा ए॒ते लो॒कं य॑न्ति॒ य उ॑प॒सद॑ उप॒यन्ति॒ तेषां॒ य उ॒न्नय॑ते॒ हीय॑त ए॒व स नोद॑ने॒षीति॒ सू᳚न्नीयमिव॒ यो वै स्वा॒र्थेता᳚ं य॒ताꣴ श्रा॒न्तो हीय॑त उ॒त स नि॒ष्ट्याय॑ स॒ह व॑सति॒ तस्मा᳚थ्स॒कृदु॒न्नीय॒ नाप॑र॒मुन्न॑येत द॒ध्नोन्न॑येतै॒तद्वै प॑शू॒नाꣳ रू॒पꣳ रू॒पेणै॒व प॒शूनव॑ रुन्द्धे॥२०॥

%6.2.4.2
य॒ज्ञो दे॒वेभ्यो॒ निला॑यत॒ विष्णू॑ रू॒पं कृ॒त्वा स पृ॑थि॒वीम्प्रावि॑श॒त्तं दे॒वा हस्ता᳚न्थ्स॒ꣳ॒रभ्यै᳚च्छ॒न्तमिन्द्र॑ उ॒पर्यु॑प॒र्यत्य॑क्राम॒थ्सो᳚\-ऽब्रवी॒त्को मा॒यमु॒पर्यु॑प॒र्यत्य॑क्रमी॒दित्य॒हं दु॒र्गे हन्तेत्यथ॒ कस्त्वमित्य॒हं दु॒र्गादाह॒र्तेति॒ सो᳚\-ऽब्रवीद्दु॒र्गे वै हन्ता॑वोचथा वरा॒हो॑\-ऽयं वा॑ममो॒षः॥२१॥

%6.2.4.3
स॒प्ता॒नां गि॑री॒णाम्प॒रस्ता᳚द्वि॒त्तं वेद्य॒मसु॑राणाम्बिभर्ति॒ तं ज॑हि॒ यदि॑ दु॒र्गे हन्तासीति॒ स द॑र्भपुञ्जी॒लमु॒द्वृह्य॑ स॒प्त गि॒रीन्भि॒त्त्वा तम॑ह॒न्थ्सो᳚\-ऽब्रवीद्दु॒र्गाद्वा आह॑र्तावोचथा ए॒तमा ह॒रेति॒ तमे᳚भ्यो य॒ज्ञ ए॒व य॒ज्ञमाह॑र॒द्यत्तद्वि॒त्तं वेद्य॒मसु॑राणा॒मवि॑न्दन्त॒ तदेकं॒ वेद्यै॑ वेदि॒त्वमसु॑राणाम्॥२२॥

%6.2.4.4
वा इ॒यमग्र॑ आसी॒द्याव॒दासी॑नः परा॒पश्य॑ति॒ ताव॑द्दे॒वाना॒न्ते दे॒वा अ॑ब्रुव॒न्नस्त्वे॒व नो॒\-ऽस्यामपीति॒ किय॑द्वो दास्याम॒ इति॒ याव॑दि॒यꣳ स॑लावृ॒की त्रिः प॑रि॒क्राम॑ति॒ ताव॑न्नो द॒त्तेति॒ स इन्द्रः॑ सलावृ॒की रू॒पं कृ॒त्वेमां त्रिः स॒र्वतः॒ पर्य॑क्राम॒त्तदि॒माम॑विन्दन्त॒ यदि॒मामवि॑न्दन्त॒ तद्वेद्यै॑ वेदि॒त्वम्॥२३॥

%6.2.4.5
सा वा इ॒यꣳ सर्वै॒व वेदि॒रिय॑ति शक्ष्या॒मीति॒ त्वा अ॑व॒माय॑ यजन्ते त्रि॒ꣳ॒शत्प॒दानि॑ प॒श्चात्ति॒रश्ची॑ भवति॒ षट्त्रिꣳ॑श॒त्प्राची॒ चतु॑र्विꣳशतिः पु॒रस्ता᳚त्ति॒रश्ची॒ दश॑दश॒ सम्प॑द्यन्ते॒ दशा᳚क्षरा वि॒राडन्नं॑ वि॒राड्वि॒राजै॒वान्नाद्य॒मव॑ रुन्द्ध॒ उद्ध॑न्ति॒ यदे॒वास्या॑ अमे॒ध्यं तदप॑ ह॒न्त्युद्ध॑न्ति॒ तस्मा॒दोष॑धयः॒ परा॑ भवन्ति ब॒र्\mbox{}हिः स्तृ॑णाति॒ तस्मा॒दोष॑धयः॒ पुन॒रा भ॑व॒न्त्युत्त॑रम्ब॒र्\mbox{}हिष॑ उत्तरब॒र्\mbox{}हिः स्तृ॑णाति प्र॒जा वै ब॒र्\mbox{}हिर्यज॑मान उत्तरब॒र्\mbox{}हिर्यज॑मानमे॒वाय॑जमाना॒दुत्त॑रं करोति॒ तस्मा॒द्यज॑मा॒नो\-ऽय॑जमाना॒दुत्त॑रः॥२४॥

%6.2.5.0
{\anuvakamend[{रु॒न्धे॒ वा॒म॒मो॒षो वे॑दि॒त्वमसु॑राणां वेदि॒त्वं भ॑वन्ति॒ पञ्च॑विꣳशतिश्च}]}%॥४॥

%6.2.5.1
यद्वा अनी॑शानो भा॒रमा॑द॒त्ते वि वै स लि॑शते॒ यद्द्वाद॑श सा॒ह्नस्यो॑प॒सदः॒ स्युस्ति॒स्रो॑\-ऽहीन॑स्य य॒ज्ञस्य॒ विलो॑म क्रियेत ति॒स्र ए॒व सा॒ह्नस्यो॑प॒सदो॒ द्वाद॑शा॒हीन॑स्य य॒ज्ञस्य॑ सवीर्य॒त्वायाथो॒ सलो॑म क्रियते व॒थ्सस्यैकः॒ स्तनो॑ भा॒गी हि सो\-ऽथैक॒ꣴ॒ स्तनं॑ व्र॒तमुपै॒त्यथ॒ द्वावथ॒ त्रीनथ॑ च॒तुर॑ ए॒तद्वै॥२५॥

%6.2.5.2
क्षु॒रप॑वि॒ नाम॑ व्र॒तं येन॒ प्र जा॒तान्भ्रातृ॑व्यान्नु॒दते॒ प्रति॑ जनि॒ष्यमा॑णा॒नथो॒ कनी॑यसै॒व भूय॒ उपै॑ति च॒तुरो\-ऽग्रे॒ स्तना᳚न्व्र॒तमुपै॒त्यथ॒ त्रीनथ॒ द्वावथैक॑मे॒तद्वै सु॑जघ॒नं नाम॑ व्र॒तं त॑प॒स्यꣳ॑ सुव॒र्ग्य॑मथो॒ प्रैव जा॑यते प्र॒जया॑ प॒शुभि॑र्यवा॒गू रा॑ज॒न्य॑स्य व्र॒तं क्रू॒रेव॒ वै य॑वा॒गूः क्रू॒र इ॑व॥२६॥

%6.2.5.3
रा॒ज॒न्यो॑ वज्र॑स्य रू॒पꣳ समृ॑द्ध्या आ॒मिक्षा॒ वैश्य॑स्य पाकय॒ज्ञस्य॑ रू॒पम्पुष्ट्यै॒ पयो᳚ ब्राह्म॒णस्य॒ तेजो॒ वै ब्रा᳚ह्म॒णस्तेजः॒ पय॒स्तेज॑सै॒व तेजः॒ पय॑ आ॒त्मन्ध॒त्ते\-ऽथो॒ पय॑सा॒ वै गर्भा॑ वर्धन्ते॒ गर्भ॑ इव॒ खलु॒ वा ए॒ष यद्दी᳚क्षि॒तो यद॑स्य॒ पयो᳚ व्र॒तम्भव॑त्या॒त्मान॑मे॒व तद्व॑र्धयति॒ त्रिव्र॑तो॒ वै मनु॑रासी॒द्द्विव्र॑ता॒ असु॑रा॒ एक॑व्रताः॥२७॥

%6.2.5.4
दे॒वाः प्रा॒तर्म॒ध्यन्दि॑ने सा॒यं तन्मनो᳚र्व्र॒तमा॑सीत्पाकय॒ज्ञस्य॑ रू॒पम्पुष्ट्यै᳚ प्रा॒तश्च॑ सा॒यं चासु॑राणां निर्म॒ध्यं क्षु॒धो रू॒पं तत॒स्ते परा॑भवन्म॒ध्यन्दि॑ने मध्यरा॒त्रे दे॒वानां॒ तत॒स्ते॑\-ऽभवन्थ्सुव॒र्गं लो॒कमा॑य॒न् यद॑स्य म॒ध्यन्दि॑ने मध्यरा॒त्रे व्र॒तम्भव॑ति मध्य॒तो वा अन्ने॑न भुञ्जते मध्य॒त ए॒व तदूर्जं॑ धत्ते॒ भ्रातृ॑व्याभिभूत्यै॒ भव॑त्या॒त्मना᳚॥२८॥




%6.2.5.5
परा᳚\-ऽस्य॒ भ्रातृ॑व्यो भवति॒ गर्भो॒ वा ए॒ष यद्दी᳚क्षि॒तो योनि॑र्दीक्षितविमि॒तं यद्दी᳚क्षि॒तो दी᳚क्षितविमि॒तात्प्र॒वसे॒द्यथा॒ योने॒र्गर्भः॒ स्कन्द॑ति ता॒दृगे॒व तन्न प्र॑वस्त॒व्य॑मा॒त्मनो॑ गोपी॒थायै॒ष वै व्या॒घ्रः कु॑लगो॒पो यद॒ग्निस्तस्मा॒द्यद्दी᳚क्षि॒तः प्र॒वसे॒थ्स ए॑नमीश्व॒रो॑\-ऽनू॒त्थाय॒ हन्तो॒र्न प्र॑वस्त॒व्य॑मा॒त्मनो॒ गुप्त्यै॑ दक्षिण॒तः श॑य ए॒तद्वै यज॑मानस्या॒यत॑न॒ꣴ॒ स्व ए॒वायत॑ने शये॒\-ऽग्निम॑भ्या॒वृत्य॑ शये दे॒वता॑ ए॒व य॒ज्ञम॑भ्या॒वृत्य॑ शये॥२९॥

%6.2.6.0
{\anuvakamend[{ए॒तद्वै क्रू॒र इ॒वैक॑व्रता आ॒त्मना॒ यज॑मानस्य॒ त्रयो॑दश च}]}%॥५॥

%6.2.6.1
पु॒रोह॑विषि देव॒यज॑ने याजये॒द्यं का॒मये॒तोपै॑न॒मुत्त॑रो य॒ज्ञो न॑मेद॒भि सु॑व॒र्गं लो॒कं ज॑ये॒दित्ये॒तद्वै पु॒रोह॑विर्देव॒यज॑नं॒ यस्य॒ होता᳚ प्रातरनुवा॒कम॑नुब्रु॒वन्न॒ग्निम॒प आ॑दि॒त्यम॒भि वि॒पश्य॒त्युपै॑न॒मुत्त॑रो य॒ज्ञो न॑मत्य॒भि सु॑व॒र्गं लो॒कं ज॑यत्या॒प्ते दे॑व॒यज॑ने याजये॒द्भ्रातृ॑व्यवन्त॒म्पन्थां᳚ वाधिस्प॒र्\mbox{}शये॑त्क॒र्तं वा॒ याव॒न्नान॑से॒ यात॒वै॥३०॥

%6.2.6.2
न रथा॑यै॒तद्वा आ॒प्तं दे॑व॒यज॑नमा॒प्नोत्ये॒व भ्रातृ॑व्यं॒ नैन॒म्भ्रातृ॑व्य आप्नो॒त्येको᳚न्नते देव॒यज॑ने याजयेत्प॒शुका॑म॒मेको᳚न्नता॒द्वै दे॑व॒यज॑ना॒दङ्गि॑रसः प॒शून॑सृजन्तान्त॒रा स॑दोहविर्धा॒ने उ॑न्न॒तꣴ स्या॑दे॒तद्वा एको᳚न्नतं देव॒यज॑नम्पशु॒माने॒व भ॑वति॒ त्र्यु॑न्नते देव॒यज॑ने याजयेथ्सुव॒र्गका॑म॒न्त्र्यु॑न्नता॒द्वै दे॑व॒यज॑ना॒दङ्गि॑रसः सुव॒र्गं लो॒कमा॑यन्नन्त॒राह॑व॒नीयं॑ च हवि॒र्धानं॑ च॥३१॥

%6.2.6.3
उ॒न्न॒तꣴ स्या॑दन्त॒रा ह॑वि॒र्धानं॑ च॒ सद॑श्चान्त॒रा सद॑श्च॒ गार्\mbox{}ह॑पत्यं चै॒तद्वै त्र्यु॑न्नतं देव॒यज॑नꣳ सुव॒र्गमे॒व लो॒कमे॑ति॒ प्रति॑ष्ठिते देव॒यज॑ने याजयेत्प्रति॒ष्ठाका॑ममे॒तद्वै प्रति॑ष्ठितं देव॒यज॑नं॒ यथ्स॒र्वतः॑ स॒मम्प्रत्ये॒व ति॑ष्ठति॒ यत्रा॒न्याअ॑न्या॒ ओष॑धयो॒ व्यति॑षक्ताः॒ स्युस्तद्या॑जयेत्प॒शुका॑ममे॒तद्वै प॑शू॒नाꣳ रू॒पꣳ रू॒पेणै॒वास्मै॑ प॒शून्॥३२॥

%6.2.6.4
अव॑ रुन्द्धे पशु॒माने॒व भ॑वति॒ निर्\mbox{}ऋ॑तिगृहीते देव॒यज॑ने याजये॒द्यं का॒मये॑त॒ निर्\mbox{}ऋ॑त्यास्य य॒ज्ञं ग्रा॑हयेय॒मित्ये॒तद्वै निर्\mbox{}ऋ॑तिगृहीतं देव॒यज॑नं॒ यथ्स॒दृश्यै॑ स॒त्या॑ ऋ॒क्षन्निर्\mbox{}ऋ॑त्यै॒वास्य॑ य॒ज्ञं ग्रा॑हयति॒ व्यावृ॑त्ते देव॒यज॑ने याजयेद्व्या॒वृत्का॑मं॒ यम्पात्रे॑ वा॒ तल्पे॑ वा॒ मीमाꣳ॑सेरन्प्रा॒चीन॑माहव॒नीया᳚त्प्रव॒णꣴ स्या᳚त्प्रती॒चीनं॒ गार्\mbox{}ह॑पत्यादे॒तद्वै व्यावृ॑त्तं देव॒यज॑नं॒ वि पा॒प्मना॒ भ्रातृ॑व्ये॒णा व॑र्तते॒ नैन॒म्पात्रे॒ न तल्पे॑ मीमाꣳसन्ते का॒र्ये॑ देव॒यज॑ने याजये॒द्भूति॑कामं का॒र्यो॑ वै पुरु॑षो॒ भव॑त्ये॒व॥३३॥

%6.2.7.0
{\anuvakamend[{यात॒वै ह॑वि॒र्धान॑ञ्च प॒शून्पा॒प्मना॒\-ऽष्टाद॑श च}]}%॥६॥

%6.2.7.1
तेभ्य॑ उत्तरवे॒दिः सि॒ꣳ॒ही रू॒पं कृ॒त्वोभया॑नन्त॒राप॒क्रम्या॑तिष्ठ॒त्ते दे॒वा अ॑मन्यन्त यत॒रान् वा इ॒यमु॑पाव॒र्थ्स्यति॒ त इ॒दम्भ॑विष्य॒न्तीति॒ तामुपा॑मन्त्रयन्त॒ साब्र॑वी॒द्वरं॑ वृणै॒ सर्वा॒न्मया॒ कामा॒न्व्य॑श्ञवथ॒ पूर्वां तु मा॒\-ऽग्नेराहु॑तिरश्ञवता॒ इति॒ तस्मा॑दुत्तरवे॒दिम्पूर्वा॑म॒ग्नेर्व्याघा॑रयन्ति॒ वारे॑वृत॒ꣴ॒ ह्य॑स्यै॒ शम्य॑या॒ परि॑ मिमीते॥३४॥

%6.2.7.2
मात्रै॒वास्यै॒ सा\-ऽथो॑ यु॒क्तेनै॒व यु॒क्तमव॑ रुन्द्धे वि॒त्ताय॑नी मे॒\-ऽसीत्या॑ह वि॒त्ता ह्ये॑ना॒नाव॑त्ति॒क्ताय॑नी मे॒\-ऽसीत्या॑ह ति॒क्तान् ह्ये॑ना॒नाव॒दव॑तान्मा नाथि॒तमित्या॑ह नाथि॒तान् ह्ये॑ना॒नाव॒दव॑तान्मा व्यथि॒तमित्या॑ह व्यथि॒तान् ह्ये॑ना॒नाव॑द्वि॒देर॒ग्निर्नभो॒ नाम॑॥३५॥

%6.2.7.3
अग्ने॑ अङ्गिर॒ इति॒ त्रिर्\mbox{}ह॑रति॒ य ए॒वैषु लो॒केष्व॒ग्नय॒स्ताने॒वाव॑ रुन्द्धे तू॒ष्णीं च॑तु॒र्थꣳ ह॑र॒त्यनि॑रुक्तमे॒वाव॑ रुन्द्धे सि॒ꣳ॒हीर॑सि महि॒षीर॒सीत्या॑ह सि॒ꣳ॒हीर्\mbox{}ह्ये॑षा रू॒पं कृ॒त्वोभया॑नन्त॒राप॒क्रम्याति॑ष्ठदु॒रु प्र॑थस्वो॒रु ते॑ य॒ज्ञप॑तिः प्रथता॒मित्या॑ह॒ यज॑मानमे॒व प्र॒जया॑ प॒शुभिः॑ प्रथयति ध्रु॒वा॥३६॥

%6.2.7.4
अ॒सीति॒ सꣳ ह॑न्ति॒ धृत्यै॑ दे॒वेभ्यः॑ शुन्धस्व दे॒वेभ्यः॑ शुम्भ॒स्वेत्यव॑ चो॒क्षति॒ प्र च॑ किरति॒ शुद्ध्या॑ इन्द्रघो॒षस्त्वा॒ वसु॑भिः पु॒रस्ता᳚त्पा॒त्वित्या॑ह दि॒ग्भ्य ए॒वैनां॒ प्रोक्ष॑ति दे॒वाꣴश्चेदु॑त्तरवे॒दिरु॒पाव॑वर्ती॒हैव वि ज॑यामहा॒ इत्यसु॑रा॒ वज्र॑मु॒द्यत्य॑ दे॒वान॒भ्या॑यन्त॒ तानि॑न्द्रघो॒षो वसु॑भिः पु॒रस्ता॒दप॑॥३७॥

%6.2.7.5
अ॒नु॒द॒त॒ मनो॑जवाः पि॒तृभि॑र्दक्षिण॒तः प्रचे॑ता रु॒द्रैः प॒श्चाद्वि॒श्वक॑र्मादि॒त्यैरु॑त्तर॒तो यदे॒वमु॑त्तरवे॒दिं प्रो॒क्षति॑ दि॒ग्भ्य ए॒व तद्यज॑मानो॒ भ्रातृ॑व्या॒न्प्रणु॑दत॒ इन्द्रो॒ यती᳚न्थ्सालावृ॒केभ्यः॒ प्राय॑च्छ॒त्तान्द॑क्षिण॒त उ॑त्तरवे॒द्या आ॑द॒न् यत्प्रोक्ष॑णीनामु॒च्छिष्ये॑त॒ तद्द॑क्षिण॒त उ॑त्तरवे॒द्यै नि न॑ये॒द्यदे॒व तत्र॑ क्रू॒रं तत्तेन॑ शमयति॒ यं द्वि॒ष्यात्तं ध्या॑येच्छु॒चैवैन॑मर्पयति॥३८॥

%6.2.8.0
{\anuvakamend[{मि॒मी॒ते॒ नाम॑ ध्रु॒वा\-ऽप॑ शु॒चा त्रीणि॑ च}]}%॥७॥

%6.2.8.1
सोत्त॑रवे॒दिर॑ब्रवी॒थ्सर्वा॒न्मया॒ कामा॒न्व्य॑श्ञव॒थेति॒ ते दे॒वा अ॑कामय॒न्तासु॑रा॒न्भ्रातृ॑व्यान॒भि भ॑वे॒मेति॒ ते॑\-ऽजुहवुः सि॒ꣳ॒हीर॑सि सपत्नसा॒ही स्वाहेति॒ ते\-ऽसु॑रा॒न्भ्रातृ॑व्यान॒भ्य॑भव॒न्ते\-ऽसु॑रा॒न्भ्रातृ॑व्यानभि॒भूया॑कामयन्त प्र॒जां वि॑न्देम॒हीति॒ ते॑\-ऽजुहवुः सि॒ꣳ॒हीर॑सि सुप्रजा॒वनिः॒ स्वाहेति॒ ते प्र॒जाम॑विन्दन्त॒ ते प्र॒जां वि॒त्त्वा॥३९॥

%6.2.8.2
अ॒का॒म॒य॒न्त॒ प॒शून् वि॑न्देम॒हीति॒ ते॑\-ऽजुहवुः सि॒ꣳ॒हीर॑सि रायस्पोष॒वनिः॒ स्वाहेति॒ ते प॒शून॑विन्दन्त॒ ते प॒शून् वि॒त्त्वा\-ऽ का॑मयन्त प्रति॒ष्ठां वि॑न्देम॒हीति॒ ते॑\-ऽजुहवुः सि॒ꣳ॒हीर॑स्यादित्य॒वनिः॒ स्वाहेति॒ त इ॒माम्प्र॑ति॒ष्ठाम॑विन्दन्त॒ त इ॒माम्प्र॑ति॒ष्ठां वि॒त्त्वाका॑मयन्त दे॒वता॑ आ॒शिष॒ उपे॑या॒मेति॒ ते॑\-ऽजुहवुः सि॒ꣳ॒हीर॒स्या व॑ह दे॒वान्दे॑वय॒ते॥४०॥

%6.2.8.3
यज॑मानाय॒ स्वाहेति॒ ते दे॒वता॑ आ॒शिष॒ उपा॑य॒न्पञ्च॒ कृत्वो॒ व्याघा॑रयति॒ पञ्चा᳚क्षरा प॒ङ्क्तिः पाङ्क्तो॑ य॒ज्ञो य॒ज्ञमे॒वाव॑ रुन्द्धे\-ऽक्ष्ण॒या व्याघा॑रयति॒ तस्मा॑दक्ष्ण॒या प॒शवो\-ऽङ्गा॑नि॒ प्र ह॑रन्ति॒ प्रति॑ष्ठित्यै भू॒तेभ्य॒स्त्वेति॒ स्रुच॒मुद्गृ॑ह्णाति॒ य ए॒व दे॒वा भू॒तास्तेषा॒न्तद्भा॑ग॒धेय॒न्ताने॒व तेन॑ प्रीणाति॒ पौतु॑द्रवान्परि॒धीन्परि॑ दधात्ये॒षाम्॥४१॥

%6.2.8.4
लो॒कानां॒ विधृ॑त्या अ॒ग्नेस्त्रयो॒ ज्यायाꣳ॑सो॒ भ्रात॑र आस॒न्ते दे॒वेभ्यो॑ ह॒व्यं वह॑न्तः॒ प्रामी॑यन्त॒ सो᳚\-ऽग्निर॑बिभेदि॒त्थं वाव स्य आर्ति॒मारि॑ष्य॒तीति॒ स निला॑यत॒ स यां वन॒स्पति॒ष्वव॑स॒त्ताम्पूतु॑द्रौ॒ यामोष॑धीषु॒ ताꣳ सु॑गन्धि॒तेज॑ने॒ याम्प॒शुषु॒ ताम्पेत्व॑स्यान्त॒रा शृङ्गे॒ तं दे॒वताः॒ प्रैष॑मैच्छ॒न्तमन्व॑विन्द॒न्तम॑ब्रुवन्न्॥४२॥

%6.2.8.5
उप॑ न॒ आ व॑र्तस्व ह॒व्यं नो॑ व॒हेति॒ सो᳚\-ऽब्रवी॒द्वरं॑ वृणै॒ यदे॒व गृ॑ही॒तस्याहु॑तस्य बहिःपरि॒धि स्कन्दा॒त्तन्मे॒ भ्रातृ॑णाम्भाग॒धेय॑मस॒दिति॒ तस्मा॒द्यद्गृ॑ही॒तस्याहु॑तस्य बहिःपरि॒धि स्कन्द॑ति॒ तेषा॒न्तद्भा॑ग॒धेयं॒ ताने॒व तेन॑ प्रीणाति॒ सो॑\-ऽमन्यतास्थ॒न्वन्तो॑ मे॒ पूर्वे॒ भ्रात॑रः॒ प्रामे॑षता॒स्थानि॑ शातया॒ इति॒ स यानि॑॥४३॥

%6.2.8.6
अ॒स्थान्यशा॑तयत॒ तत्पूतु॑द्र्वभव॒द्यन्मा॒ꣳ॒समुप॑मृतं॒ तद्गुल्गु॑लु॒ यदे॒तान्थ्स॑म्भा॒रान्थ्स॒म्भर॑त्य॒ग्निमे॒व तथ्सम्भ॑रत्य॒ग्नेः पुरी॑षम॒सीत्या॑हा॒ग्नेर्\mbox{}ह्ये॑तत्पुरी॑षं॒ यथ्सं॑भा॒रा अथो॒ खल्वा॑हुरे॒ते वावैनं॒ ते भ्रात॑रः॒ परि॑ शेरे॒ यत्पौतु॑द्रवाः परि॒धय॒ इति॑॥४४॥

%6.2.9.0
{\anuvakamend[{वि॒त्त्वा दे॑वय॒त ए॒षाम॑ब्रुव॒न् यानि॒ चतु॑श्चत्वारिꣳशच्च}]}%॥८॥

%6.2.9.1
ब॒द्धमव॑ स्यति वरुणपा॒शादे॒वैने॑ मुञ्चति॒ प्र णे॑नेक्ति॒ मेध्ये॑ ए॒वैने॑ करोति सावित्रि॒यर्चा हु॒त्वा ह॑वि॒र्धाने॒ प्र व॑र्तयति सवि॒तृप्र॑सूत ए॒वैने॒ प्र व॑र्तयति॒ वरु॑णो॒ वा ए॒ष दु॒र्वागु॑भ॒यतो॑ ब॒द्धो यदक्षः॒ स यदु॒थ्सर्जे॒द्यज॑मानस्य गृ॒हान॒भ्युथ्स॑र्जेथ्सु॒वाग्दे॑व॒ दुर्या॒ꣳ॒ आ व॒देत्या॑ह गृ॒हा वै दुर्याः॒ शान्त्यै॒ पत्नी᳚॥४५॥

%6.2.9.2
उपा॑नक्ति॒ पत्नी॒ हि सर्व॑स्य मि॒त्रम्मि॑त्र॒त्वाय॒ यद्वै पत्नी॑ य॒ज्ञस्य॑ क॒रोति॑ मिथु॒नं तदथो॒ पत्नि॑या ए॒वैष य॒ज्ञस्या᳚न्वार॒म्भो\-ऽन॑वच्छित्त्यै॒ वर्त्म॑ना॒ वा अ॒न्वित्य॑ य॒ज्ञꣳ रक्षाꣳ॑सि जिघाꣳसन्ति वैष्ण॒वीभ्या॑मृ॒ग्भ्यां वर्त्म॑नोर्जुहोति य॒ज्ञो वै विष्णु॑र्य॒ज्ञादे॒व रक्षा॒ꣳ॒स्यप॑ हन्ति॒ यद॑ध्व॒र्युर॑न॒ग्नावाहु॑तिञ्जुहु॒याद॒न्धो᳚\-ऽध्व॒र्युः स्या॒द्रक्षाꣳ॑सि य॒ज्ञꣳ ह॑न्युः॥४६॥

%6.2.9.3
हिर॑ण्यमु॒पास्य॑ जुहोत्यग्नि॒वत्ये॒व जु॑होति॒ नान्धो᳚\-ऽध्व॒र्युर्भव॑ति॒ न य॒ज्ञꣳ रक्षाꣳ॑सि घ्नन्ति॒ प्राची॒ प्रेत॑मध्व॒रं क॒ल्पय॑न्ती॒ इत्या॑ह सुव॒र्गमे॒वैने॑ लो॒कं ग॑मय॒त्यत्र॑ रमेथां॒ वर्ष्म॑न्पृथि॒व्या इत्या॑ह॒ वर्ष्म॒ ह्ये॑तत्पृ॑थि॒व्या यद्दे॑व॒यज॑न॒ꣳ॒ शिरो॒ वा ए॒तद्य॒ज्ञस्य॒ यद्ध॑वि॒र्धान॑न्दि॒वो वा॑ विष्णवु॒त वा॑ पृथि॒व्याः॥४७॥

%6.2.9.4
इत्या॒शीर्प॑दय॒र्चा दक्षि॑णस्य हवि॒र्धान॑स्य मे॒थीं नि ह॑न्ति शीर्\mbox{}ष॒त ए॒व य॒ज्ञस्य॒ यज॑मान आ॒शिषो\-ऽव॑ रुन्द्धे द॒ण्डो वा औ॑प॒रस्तृ॒तीय॑स्य हवि॒र्धान॑स्य वषट्का॒रेणाक्ष॑मच्छिन॒द्यत्तृ॒तीयं॑ छ॒दिर्\mbox{}ह॑वि॒र्धान॑योरुदाह्रि॒यते॑ तृ॒तीय॑स्य हवि॒र्धान॒स्याव॑रुद्ध्यै॒ शिरो॒ वा ए॒तद्य॒ज्ञस्य॒ यद्ध॑वि॒र्धानं॒ विष्णो॑ र॒राट॑मसि॒ विष्णोः᳚ पृ॒ष्ठम॒सीत्या॑ह॒ तस्मा॑देताव॒द्धा शिरो॒ विष्यू॑तं॒ विष्णोः॒ स्यूर॑सि॒ विष्णो᳚र्ध्रु॒वम॒सीत्या॑ह वैष्ण॒वꣳ हि दे॒वत॑या हवि॒र्धानं॒ यम्प्र॑थ॒मं ग्र॒न्थिं ग्र॑थ्नी॒याद्यत्तं न वि॑स्र॒ꣳ॒सये॒दमे॑हेनाध्व॒र्युः प्र मी॑येत॒ तस्मा॒थ्स वि॒स्रस्यः॑॥४८॥

%6.2.10.0
{\anuvakamend[{पत्नी॑ हन्युर्वा पृथि॒व्या विष्यू॑तं॒ विष्णोः॒ षड्विꣳ॑शतिश्च}]}%॥९॥

%6.2.10.1
दे॒वस्य॑ त्वा सवि॒तुः प्र॑स॒व इत्यभ्रि॒मा द॑त्ते॒ प्रसू᳚त्या अ॒श्विनो᳚र्बा॒हुभ्या॒मित्या॑हा॒श्विनौ॒ हि दे॒वाना॑मध्व॒र्यू आस्तां᳚ पू॒ष्णो हस्ता᳚भ्या॒मित्या॑ह॒ यत्यै॒ वज्र॑ इव॒ वा ए॒षा यदभ्रि॒रभ्रि॑रसि॒ नारि॑र॒सीत्या॑ह॒ शान्त्यै॒ काण्डे॑काण्डे॒ वै क्रि॒यमा॑णे य॒ज्ञꣳ रक्षाꣳ॑सि जिघाꣳसन्ति॒ परि॑लिखित॒ꣳ॒ रक्षः॒ परि॑लिखिता॒ अरा॑तय॒ इत्या॑ह॒ रक्ष॑सा॒मप॑हत्यै॥४९॥

%6.2.10.2
इ॒दम॒हꣳ रक्ष॑सो ग्री॒वा अपि॑ कृन्तामि॒ यो᳚\-ऽस्मान्द्वेष्टि॒ यं च॑ व॒यं द्वि॒ष्म इत्या॑ह॒ द्वौ वाव पुरु॑षौ॒ यं चै॒व द्वेष्टि॒ यश्चै॑नं॒ द्वेष्टि॒ तयो॑रे॒वान॑न्तरायं ग्री॒वाः कृ॑न्तति दि॒वे त्वा॒न्तरि॑क्षाय त्वा पृथि॒व्यै त्वेत्या॑है॒भ्य ए॒वैना᳚ल्लोँ॒केभ्यः॒ प्रोक्ष॑ति प॒रस्ता॑द॒र्वाचीं॒ प्रोक्ष॑ति॒ तस्मा᳚त्॥५०॥

%6.2.10.3
प॒रस्ता॑द॒र्वाची᳚म्मनु॒ष्या॑ ऊर्ज॒मुप॑ जीवन्ति क्रू॒रमि॑व॒ वा ए॒तत्क॑रोति॒ यत्खन॑त्य॒पो\-ऽव॑ नयति॒ शान्त्यै॒ यव॑मती॒रव॑ नय॒त्यूर्ग्वै यव॒ ऊर्गु॑दु॒म्बर॑ ऊ॒र्जैवोर्ज॒ꣳ॒ सम॑र्धयति॒ यज॑मानेन॒ सम्मि॒तौदु॑म्बरी भवति॒ यावा॑ने॒व यज॑मान॒स्ताव॑तीमे॒वास्मि॒न्नूर्जं॑ दधाति पितृ॒णाꣳ सद॑नम॒सीति॑ ब॒र्\mbox{}हिरव॑ स्तृणाति पितृदेव॒त्यम्᳚॥५१॥

%6.2.10.4
ह्ये॑तद्यन्निखा॑तं॒ यद्ब॒र्\mbox{}हिरन॑वस्तीर्य मिनु॒यात्पि॑तृदेव॒त्या॑ निखा॑ता स्याद्ब॒र्\mbox{}हिर॑व॒स्तीर्य॑ मिनोत्य॒स्यामे॒वैना᳚म्मिनो॒त्यथो᳚ स्वा॒रुह॑मे॒वैना᳚ङ्करो॒त्युद्दिवꣴ॑ स्तभा॒नान्तरि॑क्षं पृ॒णेत्या॑है॒षाल्लोँ॒कानां॒ विधृ॑त्यै द्युता॒नस्त्वा॑ मारु॒तो मि॑नो॒त्वित्या॑ह द्युता॒नो ह॑ स्म॒ वै मा॑रु॒तो दे॒वाना॒मौदु॑म्बरीम्मिनोति॒ तेनै॒व॥५२॥

%6.2.10.5
ए॒ना॒म्मि॒नो॒ति॒ ब्र॒ह्म॒वनिं᳚ त्वा क्षत्र॒वनि॒मित्या॑ह यथाय॒जुरे॒वैतद्घृ॒तेन॑ द्यावापृथिवी॒ आ पृ॑णेथा॒मित्यौदु॑म्बर्यां जुहोति॒ द्यावा॑पृथि॒वी ए॒व रसे॑नानक्त्या॒न्तम॒न्वव॑स्रावयत्या॒न्तमे॒व यज॑मानं॒ तेज॑सा\-ऽनक्त्यै॒न्द्रम॒सीति॑ छ॒दिरधि॒ नि द॑धात्यै॒न्द्रꣳ हि दे॒वत॑या॒ सदो॑ विश्वज॒नस्य॑ छा॒येत्या॑ह विश्वज॒नस्य॒ ह्ये॑षा छा॒या यथ्सदो॒ नव॑छदि॥५३॥

%6.2.10.6
तेज॑स्कामस्य मिनुयात्त्रि॒वृता॒ स्तोमे॑न॒ सम्मि॑त॒न्तेज॑स्त्रि॒वृत्ते॑ज॒स्व्ये॑व भ॑व॒त्येका॑दशछदीन्द्रि॒यका॑म॒स्यैका॑दशाक्षरा त्रि॒ष्टुगि॑न्द्रि॒यं त्रि॒ष्टुगि॑न्द्रिया॒व्ये॑व भ॑वति॒ पञ्च॑दशछदि॒ भ्रातृ॑व्यवतः पञ्चद॒शो वज्रो॒ भ्रातृ॑व्याभिभूत्यै स॒प्तद॑शछदि प्र॒जाका॑मस्य सप्तद॒शः प्र॒जाप॑तिः प्र॒जाप॑ते॒राप्त्या॒ एक॑विꣳशतिछदि प्रति॒ष्ठाका॑मस्यैकवि॒ꣳ॒शः स्तोमा॑नां प्रति॒ष्ठा प्रति॑ष्ठित्या उ॒दरं॒ वै सद॒ ऊर्गु॑दु॒म्बरो॑ मध्य॒त औदु॑म्बरीम्मिनोति मध्य॒त ए॒व प्र॒जाना॒मूर्जं॑ दधाति॒ तस्मा᳚त्॥५४॥

%6.2.10.7
म॒ध्य॒त ऊ॒र्जा भु॑ञ्जते यजमानलो॒के वै दक्षि॑णानि छ॒दीꣳषि॑ भ्रातृव्यलो॒क उत्त॑राणि॒ दक्षि॑णा॒न्युत्त॑राणि करोति॒ यज॑मानमे॒वाय॑जमाना॒दुत्त॑रं करोति॒ तस्मा॒द्यज॑मा॒नो\-ऽय॑जमाना॒दुत्त॑रो\-ऽन्तर्व॒र्तान्क॑रोति॒ व्यावृ॑त्त्यै॒ तस्मा॒दर॑ण्यं प्र॒जा उप॑ जीवन्ति॒ परि॑ त्वा गिर्वणो॒ गिर॒ इत्या॑ह यथाय॒जुरे॒वैतदिन्द्र॑स्य॒ स्यूर॒सीन्द्र॑स्य ध्रु॒वम॒सीत्या॑है॒न्द्रꣳ हि दे॒वत॑या॒ सदो॒ यम्प्र॑थ॒मं ग्र॒न्थिं ग्र॑थ्नी॒याद्यत्तं न वि॑स्र॒ꣳ॒सये॒दमे॑हेनाध्व॒र्युः प्र मी॑येत॒ तस्मा॒थ्स वि॒स्रस्यः॑॥५॥

%6.2.11.0
{\anuvakamend[{अप॑हत्यै॒ तस्मा᳚त्पितृदेव॒त्य॑न्तेनै॒व नव॑छदि॒ तस्मा॒थ्सदः॒ पञ्च॑दश च}]}%॥10॥

%6.2.11.1
शिरो॒ वा ए॒तद्य॒ज्ञस्य॒ यद्ध॑वि॒र्धानं॑ प्रा॒णा उ॑पर॒वा ह॑वि॒र्धाने॑ खायन्ते॒ तस्मा᳚च्छी॒र्\mbox{}षन्प्रा॒णा अ॒धस्ता᳚त्खायन्ते॒ तस्मा॑द॒धस्ता᳚च्छी॒र्ष्णः प्रा॒णा र॑क्षो॒हणो॑ वलग॒हनो॑ वैष्ण॒वान्ख॑ना॒मीत्या॑ह वैष्ण॒वा हि दे॒वत॑योपर॒वा असु॑रा॒ वै नि॒र्यन्तो॑ दे॒वानां᳚ प्रा॒णेषु॑ वल॒गान्न्य॑खन॒न्तान्बा॑हुमा॒त्रे\-ऽन्व॑विन्द॒न्तस्मा᳚द्बाहुमा॒त्राः खा॑यन्त इ॒दम॒हं तं व॑ल॒गमुद्व॑पामि॥५६॥

%6.2.11.2
यं नः॑ समा॒नो यमस॑मानो निच॒खानेत्या॑ह॒ द्वौ वाव पुरु॑षौ॒ यश्चै॒व स॑मा॒नो यश्चास॑मानो॒ यमे॒वास्मै॒ तौ व॑ल॒गं नि॒खन॑त॒स्तमे॒वोद्व॑पति॒ सं तृ॑णत्ति॒ तस्मा॒थ्सन्तृ॑ण्णा अन्तर॒तः प्रा॒णा न सम्भि॑नत्ति॒ तस्मा॒दस॑म्भिन्नाः प्रा॒णा अ॒पो\-ऽव॑ नयति॒ तस्मा॑दा॒र्द्रा अ॑न्तर॒तः प्रा॒णा यव॑मती॒रव॑ नयति॥५७॥

%6.2.11.3
ऊर्ग्वै यवः॑ प्रा॒णा उ॑पर॒वाः प्रा॒णेष्वे॒वोर्जं॑ दधाति ब॒र्\mbox{}हिरव॑ स्तृणाति॒ तस्मा᳚ल्लोम॒शा अ॑न्तर॒तः प्रा॒णा आज्ये॑न॒ व्याघा॑रयति॒ तेजो॒ वा आज्यं॑ प्रा॒णा उ॑पर॒वाः प्रा॒णेष्वे॒व तेजो॑ दधाति॒ हनू॒ वा ए॒ते य॒ज्ञस्य॒ यद॑धि॒षव॑णे॒ न सं तृ॑ण॒त्त्यसं॑तृण्णे॒ हि हनू॒ अथो॒ खलु॑ दीर्घसो॒मे स॒न्तृद्ये॒ धृत्यै॒ शिरो॒ वा ए॒तद्य॒ज्ञस्य॒ यद्ध॑वि॒र्धानम्᳚॥५८॥

%6.2.11.4
प्रा॒णा उ॑पर॒वा हनू॑ अधि॒षव॑णे जि॒ह्वा चर्म॒ ग्रावा॑णो॒ दन्ता॒ मुख॑माहव॒नीयो॒ नासि॑कोत्तरवे॒दिरु॒दर॒ꣳ॒ सदो॑ य॒दा खलु॒ वै जि॒ह्वया॑ द॒थ्स्वधि॒ खाद॒त्यथ॒ मुखं॑ गच्छति य॒दा मुखं॒ गच्छ॒त्यथो॒दरं॑ गच्छति॒ तस्मा᳚द्धवि॒र्धाने॒ चर्म॒न्नधि॒ ग्राव॑भिरभि॒षुत्या॑हव॒नीये॑ हु॒त्वा प्र॒त्यञ्चः॑ प॒रेत्य॒ सद॑सि भक्षयन्ति॒ यो वै वि॒राजो॑ यज्ञमु॒खे दोहं॒ वेद॑ दु॒ह ए॒वैना॑मि॒यं वै वि॒राट्तस्यैँ त्वक्चर्मोधो॑\-ऽधि॒षव॑णे॒ स्तना॑ उपर॒वा ग्रावा॑णो व॒थ्सा ऋ॒त्विजो॑ दुहन्ति॒ सोमः॒ पयो॒ य ए॒वं वेद॑ दु॒ह ए॒वैना᳚म्॥५९॥

%6.3.0.0
{\anuvakamend[{व॒पा॒मि॒ यव॑मती॒रव॑ नयति हवि॒र्धान॑मे॒व त्रयो॑विꣳशतिश्च}]}%॥11॥

%6.3.0.0

{\anuvakamend[{चात्वा॑लाथ्सुव॒र्गाय॒ यद्वै॑सर्ज॒नानि॑ वैष्ण॒व्यर्चा पृ॑थि॒व्यै सा॒ध्या इ॒षे त्वेत्य॒ग्निना॒ पर्य॑ग्नि प॒शोः प॒शुमा॒लभ्य॒ मेद॑सा॒ स्रुचा॒वेका॑दश}]}%॥11॥
\prashnaend{चात्वा॑लाद्दे॒वानु॒पैति॑ मुञ्चति प्रह्रि॒यमा॑णाय॒ पर्य॑ग्नि प॒शुमा॒लभ्य॒ चतु॑ष्पादो॒ द्विष॑ष्टिः॥62॥ चात्वा॑लात्प॒शुषु॑ दधाति॥}
%%% END PRASHNA

\sect{तृतीयः प्रश्नः}\setcounter{anuvakam}{0}
\dnsub{तैत्तिरीयसंहितायां षष्ठमकाण्डे तृतीयः प्रश्नः}
%6.3.1.0
%6.3.1.1
चात्वा॑ला॒द्धिष्णि॑या॒नुप॑ वपति॒ योनि॒र्वै य॒ज्ञस्य॒ चात्वा॑लं य॒ज्ञस्य॑ सयोनि॒त्वाय॑ दे॒वा वै य॒ज्ञं परा॑जयन्त॒ तमाग्नी᳚ध्रा॒त्पुन॒रपा॑जयन्ने॒तद्वै य॒ज्ञस्याप॑राजितं॒ यदाग्नी᳚ध्रं॒ यदाग्नी᳚ध्रा॒द्धिष्णि॑यान् वि॒हर॑ति॒ यदे॒व य॒ज्ञस्याप॑राजितं॒ तत॑ ए॒वैन॒म्पुन॑स्तनुते परा॒जित्ये॑व॒ खलु॒ वा ए॒ते य॑न्ति॒ ये ब॑हिष्पवमा॒नꣳ सर्प॑न्ति बहिष्पवमा॒ने स्तु॒ते॥१॥

%6.3.1.2
आ॒हाग्नी॑द॒ग्नीन् वि ह॑र ब॒र्\mbox{}हिः स्तृ॑णाहि पुरो॒डाशा॒ꣳ॒ अलं॑ कु॒र्विति॑ य॒ज्ञमे॒वाप॒जित्य॒ पुन॑स्तन्वा॒ना य॒न्त्यङ्गा॑रै॒र्द्वे सव॑ने॒ वि ह॑रति श॒लाका॑भिस्तृ॒तीयꣳ॑ सशुक्र॒त्वायाथो॒ सम्भ॑रत्ये॒वैन॒द्धिष्णि॑या॒ वा अ॒मुष्मि॑ल्लोँ॒के सोम॑मरक्ष॒न्तेभ्यो\-ऽधि॒ सोम॒माह॑र॒न्तम॑न्व॒वाय॒न्तं पर्य॑विश॒न् य ए॒वं वेद॑ वि॒न्दते᳚॥२॥

%6.3.1.3
प॒रि॒वे॒ष्टार॒न्ते सो॑मपी॒थेन॒ व्या᳚र्ध्यन्त॒ ते दे॒वेषु॑ सोमपी॒थमै᳚च्छन्त॒ तां दे॒वा अ॑ब्रुव॒न्द्वेद्वे॒ नाम॑नी कुरुध्व॒मथ॒ प्र वा॒फ्स्यथ॒ न वेत्य॒ग्नयो॒ वा अथ॒ धिष्णि॑या॒स्तस्मा᳚द्द्वि॒नामा᳚ ब्राह्म॒णो\-ऽर्धु॑क॒स्तेषां॒ ये नेदि॑ष्ठम्प॒र्यवि॑श॒न्ते सो॑मपी॒थं प्राप्नु॑वन्नाहव॒नीय॑ आग्नी॒ध्रीयो॑ हो॒त्रीयो॑ मार्जा॒लीय॒स्तस्मा॒त्तेषु॑ जुह्वत्यति॒हाय॒ वष॑ट्करोति॒ वि हि॥३॥

%6.3.1.4
ए॒ते सो॑मपी॒थेनार्ध्य॑न्त दे॒वा वै याः प्राची॒राहु॑ती॒रजु॑हवु॒र्ये पु॒रस्ता॒दसु॑रा॒ आस॒न्ताꣴस्ताभिः॒ प्राणु॑दन्त॒ याः प्र॒तीची॒र्ये प॒श्चादसु॑रा॒ आस॒न्ताꣴस्ताभि॒रपा॑नुदन्त॒ प्राची॑र॒न्या आहु॑तयो हू॒यन्ते᳚ प्र॒त्यङ्ङासी॑नो॒ धिष्णि॑या॒न्व्याघा॑रयति प॒श्चाच्चै॒व पु॒रस्ता᳚च्च॒ यज॑मानो॒ भ्रातृ॑व्या॒न्प्र णु॑दते॒ तस्मा॒त्परा॑चीः प्र॒जाः प्र वी॑यन्ते प्र॒तीचीः᳚॥४॥

%6.3.1.5
जा॒य॒न्ते॒ प्रा॒णा वा ए॒ते यद्धिष्णि॑या॒ यद॑ध्व॒र्युः प्र॒त्यङ्धिष्णि॑यानति॒सर्पे᳚त्प्रा॒णान्थ्सं क॑र्\mbox{}षेत्प्र॒मायु॑कः स्या॒न्नाभि॒र्वा ए॒षा य॒ज्ञस्य॒ यद्धोतो॒र्ध्वः खलु॒ वै नाभ्यै᳚ प्रा॒णो\-ऽवा॑ङपा॒नो यद॑ध्व॒र्युः प्र॒त्यङ्होता॑रमति॒सर्पे॑दपा॒ने प्रा॒णं द॑ध्यात् प्र॒मायु॑कः स्या॒न्नाध्व॒र्युरुप॑ गाये॒द्वाग्वी᳚र्यो॒ वा अ॑ध्व॒र्युर्यद॑ध्व॒र्युरु॑प॒गाये॑दुद्गा॒त्रे॥५॥

%6.3.1.6
वाच॒ꣳ॒ सम्प्र य॑च्छेदुप॒दासु॑कास्य॒ वाख्स्या᳚द्ब्रह्मवा॒दिनो॑ वदन्ति॒ नासꣴ॑स्थिते॒ सोमे᳚\-ऽध्व॒र्युः प्र॒त्यङ्ख्सदो\-ऽती॑या॒दथ॑ क॒था दा᳚क्षि॒णानि॒ होतु॑मेति॒ यामो॒ हि स तेषां॒ कस्मा॒ अह॑ दे॒वा यामं॒ वाया॑मं॒ वानु॑ ज्ञास्य॒न्तीत्युत्त॑रे॒णाग्नी᳚ध्रं प॒रीत्य॑ जुहोति दाक्षि॒णानि॒ न प्रा॒णान्थ्सं क॑र्\mbox{}षति॒ न्य॑न्ये धिष्णि॑या उ॒प्यन्ते॒ नान्ये यान्नि॒वप॑ति॒ तेन॒ तान्प्री॑णाति॒ यान्न नि॒वप॑ति॒ यद॑नुदि॒शति॒ तेन॒ तान्॥६॥

%6.3.2.0
{\anuvakamend[{स्तु॒ते वि॒न्दते॒ हि वी॑यन्ते प्र॒तीची॑रुद्गा॒त्र उ॒प्यन्ते॒ चतु॑र्दश च}]}%॥१॥

%6.3.2.1
सु॒व॒र्गाय॒ वा ए॒तानि॑ लो॒काय॑ हूयन्ते॒ यद्वै॑सर्ज॒नानि॒ द्वाभ्यां॒ गार्\mbox{}ह॑पत्ये जुहोति द्वि॒पाद्यज॑मानः॒ प्रति॑ष्ठित्या॒ आग्नी᳚ध्रे जुहोत्य॒न्तरि॑क्ष ए॒वा क्र॑मत आहव॒नीये॑ जुहोति सुव॒र्गमे॒वैनं॑ लो॒कं ग॑मयति दे॒वान् वै सु॑व॒र्गं लो॒कं य॒तो रक्षाꣳ॑स्यजिघाꣳस॒न्ते सोमे॑न॒ राज्ञा॒ रक्षाꣳ॑स्यप॒हत्या॒प्तुमा॒त्मानं॑ कृ॒त्वा सु॑व॒र्गं लो॒कमा॑य॒न्रक्ष॑सा॒मनु॑पलाभा॒यात्तः॒ सोमो॑ भव॒त्यथ॑॥७॥

%6.3.2.2
वै॒स॒र्ज॒नानि॑ जुहोति॒ रक्ष॑सा॒मप॑हत्यै॒ त्वꣳ सो॑म तनू॒कृद्भ्य॒ इत्या॑ह तनू॒कृद्ध्ये॑ष द्वेषो᳚भ्यो॒\-ऽन्यकृ॑तेभ्य॒ इत्या॑हा॒न्यकृ॑तानि॒ हि रक्षाꣳ॑स्यु॒रु य॒न्तासि॒ वरू॑थ॒मित्या॑हो॒रु ण॑स्कृ॒धीति॒ वावैतदा॑ह जुषा॒णो अ॒प्तुराज्य॑स्य वे॒त्वित्या॑हा॒प्तुमे॒व यज॑मानं कृ॒त्वा सु॑व॒र्गं लो॒कं ग॑मयति॒ रक्ष॑सा॒मनु॑पलाभा॒या सोमं॑ ददते॥८॥

%6.3.2.3
आ ग्राव्ण्ण॒ आ वा॑य॒व्या᳚न्या द्रो॑णकल॒शमुत्पत्नी॒मा न॑य॒न्त्यन्वनाꣳ॑सि॒ प्र व॑र्तयन्ति॒ याव॑दे॒वास्यास्ति॒ तेन॑ स॒ह सु॑व॒र्गं लो॒कमे॑ति॒ नय॑वत्य॒र्चाग्नी᳚ध्रे जुहोति सुव॒र्गस्य॑ लो॒कस्या॒भिनी᳚त्यै॒ ग्राव्ण्णो॑ वाय॒व्या॑नि द्रोणकल॒शमाग्नी᳚ध्र॒ उप॑ वासयति॒ वि ह्ये॑नं॒ तैर्गृ॒ह्णते॒ यथ्स॒होप॑वा॒सये॑दपुवा॒येत॑ सौ॒म्यर्चा प्र पा॑दयति॒ स्वया᳚॥९॥

%6.3.2.4
ए॒वैनं॑ दे॒वत॑या॒ प्र पा॑दय॒त्यदि॑त्याः॒ सदो॒\-ऽस्यदि॑त्याः॒ सद॒ आ सी॒देत्या॑ह यथाय॒जुरे॒वैतद्यज॑मानो॒ वा ए॒तस्य॑ पु॒रा गो॒प्ता भ॑वत्ये॒ष वो॑ देव सवितः॒ सोम॒ इत्या॑ह सवि॒तृप्र॑सूत ए॒वैनं॑ दे॒वता᳚भ्यः॒ सम्प्र य॑च्छत्ये॒तत्त्वꣳ सो॑म दे॒वो दे॒वानुपा॑गा॒ इत्या॑ह दे॒वो ह्ये॑ष सन्॥१०॥

%6.3.2.5
दे॒वानु॒पैती॒दम॒हम्म॑नु॒ष्यो॑ मनु॒ष्या॑नित्या॑ह मनु॒ष्यो  ह्ये॑ष सन्म॑नु॒ष्या॑नु॒पैति॒ यदे॒तद्यजु॒र्न ब्रू॒यादप्र॑जा अप॒शुर्यज॑मानः स्याथ्स॒ह प्र॒जया॑ स॒ह रा॒यस्पोषे॒णेत्या॑ह प्र॒जयै॒व प॒शुभिः॑ स॒हेमं लो॒कमु॒पाव॑र्तते॒ नमो॑ दे॒वेभ्य॒ इत्या॑ह नमस्का॒रो हि दे॒वानाꣴ॑ स्व॒धा पि॒तृभ्य॒ इत्या॑ह स्वधाका॒रो हि॥११॥

%6.3.2.6
पि॒तृ॒णामि॒दम॒हं निर्वरु॑णस्य॒ पाशा॒दित्या॑ह वरुणपा॒शादे॒व निर्मु॑च्य॒ते\-ऽग्ने᳚ व्रतपत आ॒त्मनः॒ पूर्वा॑ त॒नूरा॒देयेत्या॑हुः॒ को हि तद्वेद॒ यद्वसी॑या॒न्थ्स्वे वशे॑ भू॒ते पुन॑र्वा॒ ददा॑ति॒ न वेति॒ ग्रावा॑णो॒ वै सोम॑स्य॒ राज्ञो॑ मलिम्लुसे॒ना य ए॒वं वि॒द्वान्ग्राव्ण्ण॒ आग्नी᳚ध्र उपवा॒सय॑ति॒ नैन॑म्मलिम्लुसे॒ना वि॑न्दति॥१२॥

%6.3.3.0
{\anuvakamend[{अथ॑ ददते॒ स्वया॒ सन्थ्स्व॑धाका॒रो हि वि॑न्दति}]}%॥२॥

%6.3.3.1
वै॒ष्ण॒व्यर्चा हु॒त्वा यूप॒मच्छै॑ति वैष्ण॒वो वै दे॒वत॑या॒ यूपः॒ स्वयै॒वैनं॑ दे॒वत॒याच्छै॒त्यत्य॒न्यानगां॒ नान्यानुपा॑गा॒मित्या॒हाति॒ ह्य॑न्यानेति॒ नान्यानु॒पैत्य॒र्वाक्त्वा॒ परै॑रविदम्प॒रो\-ऽव॑रै॒रित्या॑हा॒र्वाग्घ्ये॑नं॒ परै᳚र्वि॒न्दति॑ प॒रोव॑रै॒स्तं त्वा॑ जुषे॥१३॥

%6.3.3.2
वै॒ष्ण॒वं दे॑वय॒ज्याया॒ इत्या॑ह देवय॒ज्यायै॒ ह्ये॑नं जु॒षते॑ दे॒वस्त्वा॑ सवि॒ता मध्वा॑न॒क्त्वित्या॑ह॒ तेज॑सै॒वैन॑मन॒क्त्योष॑धे॒ त्राय॑स्वैन॒ꣴ॒ स्वधि॑ते॒ मैनꣳ॑ हिꣳसी॒रित्या॑ह॒ वज्रो॒ वै स्वधि॑तिः॒ शान्त्यै॒ स्वधि॑तेर्वृ॒क्षस्य॒ बिभ्य॑तः प्रथ॒मेन॒ शक॑लेन स॒ह तेजः॒ परा॑ पतति॒ यः प्र॑थ॒मः शक॑लः परा॒पते॒त्तमप्या ह॑रे॒थ्सते॑जसम्॥१४॥

%6.3.3.3
ए॒वैन॒मा ह॑रती॒मे वै लो॒का यूपा᳚त्प्रय॒तो बि॑भ्यति॒ दिव॒मग्रे॑ण॒ मा ले॑खीर॒न्तरि॑क्ष॒म्मध्ये॑न॒ मा हिꣳ॑सी॒रित्या॑है॒भ्य ए॒वैनं॑ लो॒केभ्यः॑ शमयति॒ वन॑स्पते श॒तव॑ल्\mbox{}शो॒ वि रो॒हेत्या॒व्रश्च॑ने जुहोति॒ तस्मा॑दा॒व्रश्च॑नाद्वृ॒क्षाणा॒म्भूयाꣳ॑स॒ उत्ति॑ष्ठन्ति स॒हस्र॑वल्\mbox{}शा॒ वि व॒यꣳ रु॑हे॒मेत्या॑हा॒शिष॑मे॒वैतामा शा॒स्ते\-ऽन॑क्षसङ्गम्॥१५॥

%6.3.3.4
वृ॒श्चे॒द्यद॑क्षस॒ङ्गं वृ॒श्चेद॑धई॒षं यज॑मानस्य प्र॒मायु॑कꣴ स्या॒द्यं का॒मये॒ताप्र॑तिष्ठितः स्या॒दित्या॑रो॒हं तस्मै॑ वृश्चेदे॒ष वै वन॒स्पती॑ना॒मप्र॑तिष्ठि॒तो\-ऽप्र॑तिष्ठित ए॒व भ॑वति॒ यं का॒मये॑ताप॒शुः स्या॒दित्य॑प॒र्णं तस्मै॒ शुष्का᳚ग्रं वृश्चेदे॒ष वै वन॒स्पती॑नामपश॒व्यो॑\-ऽप॒शुरे॒व भ॑वति॒ यं का॒मये॑त पशु॒मान्थ्स्या॒दिति॑ बहुप॒र्णं तस्मै॑ बहुशा॒खं वृ॑श्चेदे॒ष वै॥१६॥

%6.3.3.5
वन॒स्पती॑नाम्पश॒व्यः॑ पशु॒माने॒व भ॑वति॒ प्रति॑ष्ठितं वृश्चेत्प्रति॒ष्ठाका॑मस्यै॒ष वै वन॒स्पती॑नां॒ प्रति॑ष्ठितो॒ यः स॒मे भूम्यै॒ स्वाद्योने॑ रू॒ढः प्रत्ये॒व ति॑ष्ठति॒ यः प्र॒त्यङ्ङुप॑नत॒स्तं वृ॑श्चे॒थ्स हि मेध॑म॒भ्युप॑नतः॒ पञ्चा॑रत्निं॒ तस्मै॑ वृश्चे॒द्यं का॒मये॒तोपै॑न॒मुत्त॑रो य॒ज्ञो न॑मे॒दिति॒ पञ्चा᳚क्षरा प॒ङ्क्तिः पाङ्क्तो॑ य॒ज्ञ उपै॑न॒मुत्त॑रो य॒ज्ञः॥१७॥

%6.3.3.6
न॒म॒ति॒ षड॑रत्निं प्रति॒ष्ठाका॑मस्य॒ षड्वा ऋ॒तव॑ ऋ॒तुष्वे॒व प्रति॑ तिष्ठति स॒प्तार॑त्निम्प॒शुका॑मस्य स॒प्तप॑दा॒ शक्व॑री प॒शवः॒ शक्व॑री प॒शूने॒वाव॑ रुन्द्धे॒ नवा॑रत्निं॒ तेज॑स्कामस्य त्रि॒वृता॒ स्तोमे॑न॒ सम्मि॑तं॒ तेज॑स्त्रि॒वृत्ते॑ज॒स्व्ये॑व भ॑व॒त्येका॑दशारत्नि\-मिन्द्रि॒यका॑म॒स्यैका॑दशाक्षरा त्रि॒ष्टुगि॑न्द्रि॒यं त्रि॒ष्टुगि॑न्द्रिया॒व्ये॑व भ॑वति॒ पञ्च॑दशारत्नि॒म्भ्रातृ॑व्यवतः पञ्चद॒शो वज्रो॒ भ्रातृ॑व्याभिभूत्यै स॒प्तद॑शारत्निं प्र॒जाका॑मस्य सप्तद॒शः प्र॒जाप॑तिः प्र॒जाप॑ते॒राप्त्या॒ एक॑विꣳशत्यरत्निं प्रति॒ष्ठाका॑मस्यैक\-वि॒ꣳ॒शः स्तोमा॑नां प्रति॒ष्ठा प्रति॑ष्ठित्या अ॒ष्टाश्रि॑र्भवत्य॒ष्टाक्ष॑रा गाय॒त्री तेजो॑ गाय॒त्री गा॑य॒त्री य॑ज्ञमु॒खं तेज॑सै॒व गा॑यत्रि॒या य॑ज्ञमु॒खेन॒ सम्मि॑तः॥१८॥

%6.3.4.0
{\anuvakamend[{जु॒षे॒ सते॑जस॒मन॑क्षसङ्गं बहुशा॒खं वृ॑श्चेदे॒ष वै य॒ज्ञ उपै॑न॒मुत्त॑रो य॒ज्ञ आप्त्या॒ एका॒न्नविꣳ॑श॒तिश्च॑}]}%॥३॥

%6.3.4.1
पृ॒थि॒व्यै त्वा॒न्तरि॑क्षाय त्वा दि॒वे त्वेत्या॑है॒भ्य ए॒वैनं॑ लो॒केभ्यः॒ प्रोक्ष॑ति॒ परा᳚ञ्चं॒ प्रोक्ष॑ति॒ परा॑ङिव॒ हि सु॑व॒र्गो लो॒कः क्रू॒रमि॑व॒ वा ए॒तत्क॑रोति॒ यत्खन॑त्य॒पोव॑ नयति॒ शान्त्यै॒ यव॑मती॒रव॑ नय॒त्यूर्ग्वै यवो॒ यज॑मानेन॒ यूपः॒ सम्मि॑तो॒ यावा॑ने॒व यज॑मान॒स्ताव॑तीमे॒वास्मि॒न्नूर्जं॑ दधाति॥१९॥

%6.3.4.2
पि॒तृ॒णाꣳ सद॑नम॒सीति॑ ब॒र्\mbox{}हिरव॑ स्तृणाति पितृदेव॒त्य ꣴ॒ ह्ये॑तद्यन्निखा॑तं॒ यद्ब॒र्\mbox{}हिरन॑वस्तीर्य मिनु॒यात्पि॑तृदेव॒त्यो॑ निखा॑तः स्याद्ब॒र्\mbox{}हिर॑व॒स्तीर्य॑ मिनोत्य॒स्यामे॒वैन॑म्मिनोति यूपशक॒लमवा᳚स्यति॒ सते॑जसमे॒वैन॑म्मिनोति दे॒वस्त्वा॑ सवि॒ता मध्वा॑न॒क्त्वित्या॑ह॒ तेज॑सै॒वैन॑मनक्ति सुपिप्प॒लाभ्य॒स्त्वौष॑धीभ्य॒ इति॑ च॒षालं॒ प्रति॑॥२०॥

%6.3.4.3
मु॒ञ्च॒ति॒ तस्मा᳚च्छीर्\mbox{}ष॒त ओष॑धयः॒ फलं॑ गृह्णन्त्य॒नक्ति॒ तेजो॒ वा आज्यं॒ यज॑मानेनाग्नि॒ष्ठाश्रिः॒ सम्मि॑ता॒ यद॑ग्नि॒ष्ठा\-मश्रि॑म॒नक्ति॒ यज॑मानमे॒व तेज॑सानक्त्या॒न्तम॑नक्त्या॒न्तमे॒व यज॑मानं॒ तेज॑सानक्ति स॒र्वतः॒ परि॑ मृश॒त्यप॑रिवर्गमे॒वा\-स्मि॒न्तेजो॑ दधा॒त्युद्दिवꣴ॑ स्तभा॒नान्तरि॑क्षं पृ॒णेत्या॑है॒षां लो॒कानां॒ विधृ॑त्यै वैष्ण॒व्यर्चा॥२१॥

%6.3.4.4
क॒ल्प॒य॒ति॒ वै॒ष्ण॒वो वै दे॒वत॑या॒ यूपः॒ स्वयै॒वैनं॑ दे॒वत॑या कल्पयति॒ द्वा\-भ्यां᳚ कल्पयति द्वि॒पाद्यज॑मानः॒ प्रति॑ष्ठित्यै॒ यं का॒मये॑त॒ तेज॑सैनं दे॒वता॑भिरिन्द्रि॒येण॒ व्य॑र्धयेय॒मित्य॑ग्नि॒ष्ठां तस्याश्रि॑माहव॒नीया॑दि॒त्थं वे॒त्थं वाति॑ नावये॒त्तेज॑सै॒वैनं॑ दे॒वता॑भिरिन्द्रि॒येण॒ व्य॑र्धयति॒ यं का॒मये॑त॒ तेज॑सैनं दे॒वता॑भिरिन्द्रि॒येण॒ सम॑र्धयेय॒मिति॑॥२२॥

%6.3.4.5
अ॒ग्नि॒ष्ठां तस्याश्रि॑माहव॒नीये॑न॒ सम्मि॑नुया॒त्तेज॑सै॒वैनं॑ दे॒वता॑भिरिन्द्रि॒येण॒ सम॑र्धयति ब्रह्म॒वनिं॑ त्वा क्षत्र॒वनि॒मित्या॑ह यथाय॒जुरे॒वैतत्परि॑ व्यय॒त्यूर्ग्वै र॑श॒ना यज॑मानेन॒ यूपः॒ सम्मि॑तो॒ यज॑मानमे॒वोर्जा सम॑र्धयति नाभिद॒घ्ने परि॑ व्ययति नाभिद॒घ्न ए॒वास्मा॒ ऊर्जं॑ दधाति॒ तस्मा᳚न्नाभिद॒घ्न ऊ॒र्जा भु॑ञ्जते॒ यं का॒मये॑तो॒र्जैनम्᳚॥२३॥

%6.3.4.6
व्य॑र्धयेय॒मित्यू॒र्ध्वां वा॒ तस्यावा॑चीं॒ वावो॑हेदू॒र्जैवैनं॒ व्य॑र्धयति॒ यदि॑ का॒मये॑त॒ वर्\mbox{}षु॑कः प॒र्जन्यः॑ स्या॒दित्यवा॑ची॒मवो॑हे॒\-द्वृष्टि॑मे॒व नि य॑च्छति॒ यदि॑ का॒मये॒ताव॑र्\mbox{}षुकः स्या॒दित्यू॒र्ध्वामुदू॑हे॒द्वृष्टि॑मे॒वोद्य॑च्छति पितृ॒णां निखा॑तम्मनु॒ष्या॑णामू॒र्ध्वं निखा॑ता॒दा र॑श॒नाया॒ ओष॑धीनाꣳ रश॒ना विश्वे॑षाम्॥२४॥

%6.3.4.7
दे॒वाना॑मू॒र्ध्वꣳ र॑श॒नाया॒ आ च॒षाला॒दिन्द्र॑स्य च॒षालꣳ॑ सा॒ध्याना॒मति॑रिक्त॒ꣳ॒ स वा ए॒ष स॑र्वदेव॒त्यो॑ यद्यूपो॒ यद्यूप॑\-म्मि॒नोति॒ सर्वा॑ ए॒व दे॒वताः᳚ प्रीणाति य॒ज्ञेन॒ वै दे॒वाः सु॑व॒र्गं लो॒कमा॑य॒न्ते॑\-ऽमन्यन्त मनु॒ष्या॑ नो॒\-ऽन्वाभ॑विष्य॒न्तीति॒ ते यूपे॑न योपयि॒त्वा सु॑व॒र्गं लो॒कमा॑य॒न्तमृष॑यो॒ यूपे॑नै॒वानु॒ प्राजा॑न॒न्तद्यूप॑स्य यूप॒त्वम्॥२५॥

%6.3.4.8
यद्यूप॑म्मि॒नोति॑ सुव॒र्गस्य॑ लो॒कस्य॒ प्रज्ञा᳚त्यै पु॒रस्ता᳚न्मिनोति पु॒रस्ता॒द्धि य॒ज्ञस्य॑ प्रज्ञा॒यते\-ऽप्र॑ज्ञात॒ꣳ॒ हि तद्यदति॑पन्न आ॒हुरि॒दं का॒र्य॑मासी॒दिति॑ सा॒ध्या वै दे॒वा य॒ज्ञमत्य॑मन्यन्त॒ तान् य॒ज्ञो नास्पृ॑श॒त्तान् यद्य॒ज्ञस्याति॑रिक्त॒मासी॒त्तद॑स्पृश॒\-दति॑रिक्तं॒ वा ए॒तद्य॒ज्ञस्य॒ यद॒ग्नाव॒ग्निम्म॑थि॒त्वा प्र॒हर॒त्यति॑रिक्तमे॒तत्॥२६॥

%6.3.4.9
यूप॑स्य॒ यदू॒र्ध्वं च॒षाला॒त्तेषां॒ तद्भा॑ग॒धेयं॒ ताने॒व तेन॑ प्रीणाति दे॒वा वै सꣴस्थि॑ते॒ सोमे॒ प्र स्रुचोह॑र॒न्प्र यूपं॒ ते॑\-ऽमन्यन्त यज्ञवेश॒सं वा इ॒दं कु॑र्म॒ इति॒ ते प्र॑स्त॒रꣴ स्रु॒चान्नि॒ष्क्रय॑णमपश्य॒न्थ्स्वरुं॒ यूप॑स्य॒ सꣴस्थि॑ते॒ सोमे॒ प्र प्र॑स्त॒रꣳ हर॑ति जु॒होति॒ स्व॒रुमय॑ज्ञवेशसाय॥२७॥

%6.3.5.0
{\anuvakamend[{द॒धा॒ति॒ प्रत्यृ॒चा सम॑र्धयेय॒मित्यू॒र्जैनं॒ विश्वे॑षां यूप॒त्वमति॑रिक्तमे॒तद्द्विच॑त्वारिꣳशच्च}]}%॥४॥

%6.3.5.1
सा॒ध्या वै दे॒वा अ॒स्मिल्लोँ॒क आ॑स॒न्नान्यत्किञ्च॒न मि॒षत्ते᳚\-ऽग्निमे॒वाग्नये॒ मेधा॒याल॑भन्त॒ न ह्य॑न्यदा॑ल॒म्भ्य॑मवि॑न्द॒न्ततो॒ वा इ॒माः प्र॒जाः प्राजा॑यन्त॒ यद॒ग्नाव॒ग्निम्म॑थि॒त्वा प्र॒हर॑ति प्र॒जानां᳚ प्र॒जन॑नाय रु॒द्रो वा ए॒ष यद॒ग्निर्यज॑मानः प॒शुर्यत्प॒शुमा॒लभ्या॒ग्निम्मन्थे᳚द्रु॒द्राय॒ यज॑मानम्॥२८॥

%6.3.5.2
अपि॑ दध्यात्प्र॒मायु॑कः स्या॒दथो॒ खल्वा॑हुर॒ग्निः सर्वा॑ दे॒वता॑ ह॒विरे॒तद्यत्प॒शुरिति॒ यत्प॒शुमा॒लभ्या॒ग्निम्मन्थ॑ति ह॒व्यायै॒वास॑न्नाय॒ सर्वा॑ दे॒वता॑ जनयत्युपा॒कृत्यै॒व मन्थ्य॒स्तन्नेवाल॑ब्धं॒ नेवाना॑लब्धम॒ग्नेर्ज॒नित्र॑म॒सीत्या॑हा॒ग्नेर्\mbox{}ह्ये॑तज्ज॒नित्रं॒ वृष॑णौ स्थ॒ इत्या॑ह॒ वृष॑णौ॥२९॥

%6.3.5.3
ह्ये॑तावु॒र्वश्य॑स्या॒युर॒सीत्या॑ह मिथुन॒त्वाय॑ घृ॒तेना॒क्ते वृष॑णं दधाथा॒मित्या॑ह॒ वृष॑ण॒ꣴ॒ ह्ये॑ते दधा॑ते॒ ये अ॒ग्निङ्गा॑य॒त्रं छन्दो\-ऽनु॒ प्र जा॑य॒स्वेत्या॑ह॒ छन्दो॑भिरे॒वैन॒म्प्र ज॑नयत्य॒ग्नये॑ म॒थ्यमा॑ना॒यानु॑ ब्रू॒हीत्या॑ह सावि॒त्रीमृच॒मन्वा॑ह सवि॒तृप्र॑सूत ए॒वैन॑म्मन्थति जा॒ताया॑नु ब्रूहि॥३०॥

%6.3.5.4
प्र॒ह्रि॒यमा॑णा॒यानु॑ ब्रू॒हीत्या॑ह॒ काण्डे॑काण्ड ए॒वैनं॑ क्रियमा॑णे॒ सम॑र्धयति गाय॒त्रीः सर्वा॒ अन्वा॑ह गाय॒त्रछ॑न्दा॒ वा अ॒ग्निः स्वेनै॒वैनं॒ छन्द॑सा॒ सम॑र्धयत्य॒ग्निः पु॒रा भव॑त्य॒ग्निम्म॑थि॒त्वा प्र ह॑रति॒ तौ स॒म्भव॑न्तौ॒ यज॑मानम॒भि सम्भ॑वतो॒ भव॑तं नः॒ सम॑नसा॒वित्या॑ह॒ शान्त्यै᳚ प्र॒हृत्य॑ जुहोति जा॒तायै॒वास्मा॒ अन्न॒मपि॑ दधा॒त्याज्ये॑न जुहोत्ये॒तद्वा अ॒ग्नेः प्रि॒यं धाम॒ यदाज्य॑म्प्रि॒येणै॒वैनं॒ धाम्ना॒ सम॑र्धय॒त्यथो॒ तेज॑सा॥३१॥

%6.3.6.0
{\anuvakamend[{यज॑मानमाह॒ वृष॑णौ जाता॒यानु॑ब्रू॒ह्यप्य॒ष्टाद॑श च}]}%॥५॥

%6.3.6.1
इ॒षे त्वेति॑ ब॒र्\mbox{}हिरा द॑त्त इ॒च्छत॑ इव॒ ह्ये॑ष यो यज॑त उप॒वीर॒सीत्या॒होप॒ ह्ये॑नानाक॒रोत्युपो॑ दे॒वान्दैवी॒र्विशः॒ प्रागु॒रित्या॑ह॒ दैवी॒र्\mbox{}ह्ये॑ता विशः॑ स॒तीर्दे॒वानु॑प॒यन्ति॒ वह्नी॑रु॒शिज॒ इत्या॑ह॒र्त्विजो॒ वै वह्न॑य उ॒शिज॒स्तस्मा॑दे॒वमा॑ह॒ बृह॑स्पते धा॒रया॒ वसू॒नीति॑॥३२॥

%6.3.6.2
आ॒ह॒ ब्रह्म॒ वै दे॒वाना॒म्बृह॒स्पति॒र्ब्रह्म॑णै॒वास्मै॑ प॒शूनव॑ रुन्द्धे ह॒व्या ते᳚ स्वदन्ता॒मित्या॑ह स्व॒दय॑त्ये॒वैना॒न्देव॑ त्वष्ट॒र्वसु॑ र॒ण्वेत्या॑ह॒ त्वष्टा॒ वै प॑शू॒नाम्मि॑थु॒नानाꣳ॑ रूप॒कृद्रू॒पमे॒व प॒शुषु॑ दधाति॒ रेव॑ती॒ रम॑ध्व॒मित्या॑ह प॒शवो॒ वै रे॒वतीः᳚ प॒शूने॒वास्मै॑ रमयति दे॒वस्य॑ त्वा सवि॒तुः प्र॑स॒व इति॑॥३३॥

%6.3.6.3
र॒श॒नामा द॑त्ते॒ प्रसू᳚त्या अ॒श्विनो᳚र्बा॒हुभ्या॒मित्या॑हा॒श्विनौ॒ हि दे॒वाना॑मध्व॒र्यू आस्तां᳚ पू॒ष्णो हस्ता᳚भ्या॒मित्या॑ह॒ यत्या॑ ऋ॒तस्य॑ त्वा देवहविः॒ पाशे॒ना र॑भ॒ इत्या॑ह स॒त्यं वा ऋ॒तꣳ स॒त्येनै॒वैन॑मृ॒तेना र॑भते\-ऽक्ष्ण॒या परि॑ हरति॒ वध्य॒ꣳ॒ हि प्र॒त्यञ्चं॑ प्रतिमु॒ञ्चन्ति॒ व्यावृ॑त्त्यै॒ धर्\mbox{}षा॒ मानु॑षा॒निति॒ नि यु॑नक्ति॒ धृत्या॑ अ॒द्भ्यः॥३४॥

%6.3.6.4
त्वौष॑धीभ्यः॒ प्रोक्षा॒मीत्या॑हा॒द्भ्यो ह्ये॑ष ओष॑धीभ्यः स॒म्भव॑ति॒ यत्प॒शुर॒पाम्पे॒रुर॒सीत्या॑है॒ष ह्य॑पाम्पा॒ता यो मेधा॑यार॒भ्यते᳚ स्वा॒त्तं चि॒थ्सदे॑वꣳ ह॒व्यमापो॑ देवीः॒ स्वद॑तैन॒मित्या॑ह स्व॒दय॑त्ये॒वैन॑मु॒परि॑ष्टा॒त्प्रोक्ष॑त्यु॒परि॑ष्टादे॒वैन॒म्मेध्यं॑ करोति पा॒यय॑त्यन्तर॒त ए॒वैन॒म्मेध्यं॑ करोत्य॒धस्ता॒दुपो᳚क्षति स॒र्वत॑ ए॒वैन॒म्मेध्यं॑ करोति॥३५॥

%6.3.7.0
{\anuvakamend[{वसू॒निति॑ प्रस॒व इत्य॒द्भ्यो᳚\-ऽन्तर॒त ए॒वैन॒न्दश॑ च}]}%॥६॥

%6.3.7.1
अ॒ग्निना॒ वै होत्रा॑ दे॒वा असु॑रान॒भ्य॑भवन्न॒ग्नये॑ समि॒ध्यमा॑ना॒यानु॑ ब्रू॒हीत्या॑ह॒ भ्रातृ॑व्याभिभूत्यै स॒प्तद॑श सामिधे॒नीरन्वा॑ह सप्तद॒शः प्र॒जाप॑तिः प्र॒जाप॑ते॒राप्त्यै॑ स॒प्तद॒शान्वा॑ह॒ द्वाद॑श॒ मासाः॒ पञ्च॒र्तवः॒ स सं॑वथ्स॒रः सं॑वथ्स॒रं प्र॒जा अनु॒ प्र जा॑यन्ते प्र॒जानां᳚ प्र॒जन॑नाय दे॒वा वै सा॑मिधे॒नीर॒नूच्य॑ य॒ज्ञं नान्व॑पश्य॒न्थ्स प्र॒जाप॑तिस्तू॒ष्णीमा॑घा॒रम्॥३६॥

%6.3.7.2
आघा॑रय॒त्ततो॒ वै दे॒वा य॒ज्ञमन्व॑पश्य॒न् यत्तू॒ष्णीमा॑घा॒रमा॑घा॒रय॑ति य॒ज्ञस्यानु॑ख्यात्या॒ असु॑रेषु॒ वै य॒ज्ञ आ॑सी॒त्तं दे॒वास्तू᳚ष्णीꣳहो॒मेना॑वृञ्जत॒ यत्तू॒ष्णीमा॑घा॒रमा॑घा॒रय॑ति॒ भ्रातृ॑व्यस्यै॒व तद्य॒ज्ञं वृ॑ङ्क्ते परि॒धीन्थ्सम्मा᳚र्ष्टि पु॒नात्ये॒वैना॒न्त्रिस्त्रिः॒ सम्मा᳚र्ष्टि॒ त्र्या॑वृ॒द्धि य॒ज्ञो\-ऽथो॒ रक्ष॑सा॒मप॑हत्यै॒ द्वाद॑श॒ सम्प॑द्यन्ते॒ द्वाद॑श॥३७॥

%6.3.7.3
मासाः᳚ संवथ्स॒रः सं॑वथ्स॒रमे॒व प्री॑णा॒त्यथो॑ संवथ्स॒रमे॒वास्मा॒ उप॑ दधाति सुव॒र्गस्य॑ लो॒कस्य॒ सम॑ष्ट्यै॒ शिरो॒ वा ए॒तद्य॒ज्ञस्य॒ यदा॑घा॒रो᳚\-ऽग्निः सर्वा॑ दे॒वता॒ यदा॑घा॒रमा॑घा॒रय॑ति शीर्\mbox{}ष॒त ए॒व य॒ज्ञस्य॒ यज॑मानः॒ सर्वा॑ दे॒वता॒ अव॑ रुन्द्धे॒ शिरो॒ वा ए॒तद्य॒ज्ञस्य॒ यदा॑घा॒र आ॒त्मा प॒शुरा॑घा॒रमा॒घार्य॑ प॒शुꣳ सम॑नक्त्या॒त्मन्ने॒व य॒ज्ञस्य॑॥३८॥

%6.3.7.4
शिरः॒ प्रति॑ दधाति॒ सं ते᳚ प्रा॒णो वा॒युना॑ गच्छता॒मित्या॑ह वायुदेव॒त्यो॑ वै प्रा॒णो वा॒यावे॒वास्य॑ प्रा॒णं जु॑होति॒ सं यज॑त्रै॒रङ्गा॑नि॒ सं य॒ज्ञप॑तिरा॒शिषेत्या॑ह य॒ज्ञप॑तिमे॒वास्या॒शिषं॑ गमयति वि॒श्वरू॑पो॒ वै त्वा॒ष्ट्र उ॒परि॑ष्टात्प॒शुम॒भ्य॑वमी॒त्तस्मा॑दु॒परि॑ष्टात्प॒शोर्नाव॑ द्यन्ति॒ यदु॒परि॑ष्टात्प॒शुꣳ स॑म॒नक्ति॒ मेध्य॑मे॒व॥३९॥

%6.3.7.5
ए॒नं॒ क॒रो॒त्यृ॒त्विजो॑ वृणीते॒ छन्दाꣳ॑स्ये॒व वृ॑णीते स॒प्त वृ॑णीते स॒प्त ग्रा॒म्याः प॒शवः॑ स॒प्तार॒ण्याः स॒प्त छन्दाꣳ॑स्यु॒भय॒स्याव॑रुद्ध्या॒ एका॑दश प्रया॒जान् य॑जति॒ दश॒ वै प॒शोः प्रा॒णा आ॒त्मैका॑द॒शो यावा॑ने॒व प॒शुस्तम्प्र य॑जति व॒पामेकः॒ परि॑ शय आ॒त्मैवात्मानं॒ परि॑ शये॒ वज्रो॒ वै स्वधि॑ति॒र्वज्रो॑ यूपशक॒लो घृ॒तं खलु॒ वै दे॒वा वज्रं॑ कृ॒त्वा सोम॑मघ्नन्घृ॒तेना॒क्तौ प॒शं त्रा॑येथा॒मित्या॑ह॒ वज्रे॑णै॒वैनं॒ वशे॑ कृ॒त्वा ल॑भते॥४०॥

%6.3.8.0
{\anuvakamend[{आ॒घा॒रम्प॑द्यन्ते॒ द्वाद॑शा॒त्मन्ने॒व य॒ज्ञस्य॒ मेध्य॑मे॒व खलु॒ वा अ॒ष्टाद॑श च}]}%॥७॥

%6.3.8.1
पर्य॑ग्नि करोति सर्व॒हुत॑मे॒वैनं॑ करो॒त्यस्क॑न्दा॒यास्क॑न्न॒ꣳ॒ हि तद्यद्धु॒तस्य॒ स्कन्द॑ति॒ त्रिः पर्य॑ग्नि करोति॒ त्र्या॑वृ॒द्धि य॒ज्ञो\-ऽथो॒ रक्ष॑सा॒मप॑हत्यै ब्रह्मवा॒दिनो॑ वदन्त्यन्वा॒रभ्यः॑ प॒शू (३) र्नान्वा॒रभ्या (३) इति॑ मृ॒त्यवे॒ वा ए॒ष नी॑यते॒ यत्प॒शुस्तं यद॑न्वा॒रभे॑त प्र॒मायु॑को॒ यज॑मानः स्या॒दथो॒ खल्वा॑हुः सुव॒र्गाय॒ वा ए॒ष लो॒काय॑ नीयते॒ यत्॥४१॥

%6.3.8.2
प॒शुरिति॒ यन्नान्वा॒रभे॑त सुव॒र्गाल्लो॒काद्यज॑मानो हीयेत वपा॒श्रप॑णीभ्याम॒न्वार॑भते॒ तन्नेवा॒न्वार॑ब्धं॒ नेवान॑न्वारब्ध॒मुप॒ प्रेष्य॑ होतर्\mbox{}ह॒व्या दे॒वेभ्य॒ इत्या॑हेषि॒तꣳ हि कर्म॑ क्रि॒यते॒ रेव॑तीर्य॒ज्ञप॑तिं प्रिय॒धा वि॑श॒तेत्या॑ह यथाय॒जुरे॒वैतद॒ग्निना॑ पु॒रस्ता॑देति॒ रक्ष॑सा॒मप॑हत्यै पृथि॒व्याः सं॒पृचः॑ पा॒हीति॑ ब॒र्\mbox{}हिः॥४२॥

%6.3.8.3
उपा᳚स्य॒त्यस्क॑न्दा॒यास्क॑न्न॒ꣳ॒ हि तद्यद्ब॒र्\mbox{}हिषि॒ स्कन्द॒त्यथो॑ बर्\mbox{}हि॒षद॑मे॒वैनं॑ करोति॒ परा॒ङा व॑र्तते\-ऽध्व॒र्युः प॒शोः सं᳚ज्ञ॒प्यमा॑नात्प॒शुभ्य॑ ए॒व तन्नि ह्नु॑त आ॒त्मनोना᳚व्रस्काय॒ गच्छ॑ति॒ श्रिय॒म्प्र प॒शूना᳚प्नोति॒ य ए॒वं वेद॑ प॒श्चाल्लो॑का॒ वा ए॒षा प्राच्यु॒दानी॑यते॒ यत्पत्नी॒ नम॑स्त आता॒नेत्या॑हादि॒त्यस्य॒ वै र॒श्मयः॑॥४३॥

%6.3.8.4
आ॒ता॒नास्तेभ्य॑ ए॒व नम॑स्करोत्यन॒र्वा प्रेहीत्या॑ह॒ भ्रातृ॑व्यो॒ वा अर्वा॒ भ्रातृ॑व्यापनुत्त्यै घृ॒तस्य॑ कु॒ल्यामनु॑ स॒ह प्र॒जया॑ स॒ह रा॒यस्पोषे॒णेत्या॑हा॒शिष॑मे॒वैतामा शा᳚स्त॒ आपो॑ देवीः शुद्धायुव॒ इत्या॑ह यथाय॒जुरे॒वैतत्॥४४॥

%6.3.9.0
{\anuvakamend[{लो॒काय॑ नीयते॒ यद्ब॒र्\mbox{}ही र॒श्मयः॑ स॒प्तत्रिꣳ॑शच्च}]}%॥८॥

%6.3.9.1
प॒शोर्वा आल॑ब्धस्य प्रा॒णाञ्छुगृ॑च्छति॒ वाक्त॒ आ प्या॑यतां प्रा॒णस्त॒ आ प्या॑यता॒मित्या॑ह प्रा॒णेभ्य॑ ए॒वास्य॒ शुचꣳ॑ शमयति॒ सा प्रा॒णेभ्यो\-ऽधि॑ पृथि॒वीꣳ शुक्प्र वि॑शति॒ शमहो᳚भ्या॒मिति॒ नि न॑यत्यहोरा॒त्राभ्या॑मे॒व पृ॑थि॒व्यै शुचꣳ॑ शमय॒त्योष॑धे॒ त्राय॑स्वैन॒ꣴ॒ स्वधि॑ते॒ मैनꣳ॑ हिꣳसी॒रित्या॑ह॒ वज्रो॒ वै स्वधि॑तिः॥४५॥

%6.3.9.2
शान्त्यै॑ पार्श्व॒त आच्छ्य॑ति मध्य॒तो हि म॑नु॒ष्या॑ आ॒च्छ्यन्ति॑ तिर॒श्चीन॒मा च्छ्य॑त्यनू॒चीन॒ꣳ॒ हि म॑नु॒ष्या॑ आ॒च्छ्यन्ति॒ व्यावृ॑त्त्यै॒ रक्ष॑साम्भा॒गो॑\-ऽसीति॑ स्थविम॒तो ब॒र्\mbox{}हिर॒क्त्वापा᳚स्यत्य॒स्नैव रक्षाꣳ॑सि नि॒रव॑दयत इ॒दम॒हꣳ रक्षो॑\-ऽध॒मं तमो॑ नयामि॒ यो᳚\-ऽस्मान्द्वेष्टि॒ यं च॑ व॒यं द्वि॒ष्म इत्या॑ह॒ द्वौ वाव पुरु॑षौ॒ यं चै॒व॥४६॥

%6.3.9.3
द्वे॒ष्टि॒ यश्चै॑नं॒ द्वेष्टि॒ तावु॒भाव॑ध॒मं तमो॑ नयती॒षे त्वेति॑ व॒पामुत्खि॑दती॒च्छत॑ इव॒ ह्ये॑ष यो यज॑ते॒ यदु॑पतृ॒न्द्याद्रु॒द्रो᳚\-ऽस्य प॒शून्घातु॑कः स्या॒द्यन्नोप॑तृ॒न्द्यादय॑ता स्याद॒न्ययो॑पतृ॒णत्त्य॒न्यया॒ न धृत्यै॑ घृ॒तेन॑ द्यावापृथिवी॒ प्रोर्ण्वा॑था॒मित्या॑ह॒ द्यावा॑पृथि॒वी ए॒व रसे॑नान॒क्त्यछि॑न्नः॥४७॥

%6.3.9.4
रायः॑ सु॒वीर॒ इत्या॑ह यथाय॒जुरे॒वैतत्क्रू॒रमि॑व॒ वा ए॒तत्क॑रोति॒ यद्व॒पामु॑त्खि॒दत्यु॒र्व॑न्तरि॑क्ष॒मन्वि॒हीत्या॑ह॒ शान्त्यै॒ प्र वा ए॒षो᳚\-ऽस्माल्लो॒काच्च्य॑वते॒ यः प॒शुम्मृ॒त्यवे॑ नी॒यमा॑नमन्वा॒रभ॑ते वपा॒श्रप॑णी॒ पुन॑र॒न्वार॑भते॒\-ऽस्मिन्ने॒व लो॒के प्रति॑ तिष्ठत्य॒ग्निना॑ पु॒रस्ता॑देति॒ रक्ष॑सा॒मप॑हत्या॒ अथो॑ दे॒वता॑ ए॒व ह॒व्येन॑॥४८॥

%6.3.9.5
अन्वे॑ति॒ नान्त॒ममङ्गा॑र॒मति॑ हरे॒द्यद॑न्त॒ममङ्गा॑रमति॒हरे᳚द्दे॒वता॒ अति॑ मन्येत॒ वायो॒ वीहि॑ स्तो॒काना॒मित्या॑ह॒ तस्मा॒द्विभ॑क्ताः स्तो॒का अव॑ पद्य॒न्ते\-ऽग्रं॒ वा ए॒तत्प॑शू॒नां यद्व॒पाग्र॒मोष॑धीनाम्ब॒र्\mbox{}हिरग्रे॑णै॒वाग्र॒ꣳ॒ सम॑र्धय॒त्यथो॒ ओष॑धीष्वे॒व प॒शून्प्रति॑ ष्ठापयति॒ स्वाहा॑कृतीभ्यः॒ प्रेष्येत्या॑ह॥४९॥

%6.3.9.6
य॒ज्ञस्य॒ समि॑ष्ट्यै प्राणापा॒नौ वा ए॒तौ प॑शू॒नां यत्पृ॑षदा॒ज्यमा॒त्मा व॒पा पृ॑षदा॒ज्यम॑भि॒घार्य॑ व॒पाम॒भि घा॑रयत्या॒त्मन्ने॒व प॑शू॒नाम्प्रा॑णापा॒नौ द॑धाति॒ स्वाहो॒र्ध्वन॑भसम्मारु॒तं ग॑च्छत॒मित्या॑हो॒र्ध्वन॑भा ह स्म॒ वै मा॑रु॒तो दे॒वानां᳚ वपा॒श्रप॑णी॒ प्रह॑रति॒ तेनै॒वैने॒ प्र ह॑रति॒ विषू॑ची॒ प्र ह॑रति॒ तस्मा॒द्विष्व॑ञ्चौ प्राणापा॒नौ॥५०॥

%6.3.10.0
{\anuvakamend[{स्वधि॑तिश्चै॒वाच्छि॑न्नो ह॒व्येने॒ष्येत्या॑ह॒ षट्च॑त्वारिꣳशच्च}]}%॥९॥

%6.3.10.1
प॒शुमा॒लभ्य॑ पुरो॒डाशं॒ निर्व॑पति॒ समे॑धमे॒वैन॒मा ल॑भते व॒पया᳚ प्र॒चर्य॑ पुरो॒डाशे॑न॒ प्र च॑र॒त्यूर्ग्वै पु॑रो॒डाश॒ ऊर्ज॑मे॒व प॑शू॒नाम्म॑ध्य॒तो द॑धा॒त्यथो॑ प॒शोरे॒व छि॒द्रमपि॑ दधाति पृषदा॒ज्यस्यो॑प॒हत्य॒ त्रिः पृ॑च्छति शृ॒तꣳ ह॒वीः (३) श॑मित॒रिति॒ त्रिष॑त्या॒ हि दे॒वा यो\-ऽशृ॑तꣳ शृ॒तमाह॒ स एन॑सा प्राणापा॒नौ वा ए॒तौ प॑शू॒नाम्॥५१॥

%6.3.10.2
यत्पृ॑षदा॒ज्यम्प॒शोः खलु॒ वा आल॑ब्धस्य॒ हृद॑यमा॒त्माभि समे॑ति॒ यत्पृ॑षदा॒ज्येन॒ हृद॑यमभिघा॒रय॑त्या॒त्मन्ने॒व प॑शू॒नाम्प्रा॑णापा॒नौ द॑धाति प॒शुना॒ वै दे॒वाः सु॑व॒र्गं लो॒कमा॑य॒न्ते॑\-ऽमन्यन्त मनु॒ष्या॑ नो॒\-ऽन्वाभ॑विष्य॒न्तीति॒ तस्य॒ शिर॑श्छि॒त्त्वा मेध॒म्प्राक्षा॑रय॒न्थ्स प्र॒क्षो॑\-ऽभव॒त्तत्प्र॒क्षस्य॑ प्रक्ष॒त्वं यत्प्ल॑क्षशा॒खोत्त॑रब॒र्\mbox{}हिर्भव॑ति॒ समे॑धस्यै॒व॥५२॥

%6.3.10.3
प॒शोरव॑ द्यति प॒शुं वै ह्रि॒यमा॑ण॒ꣳ॒ रक्षा॒ꣳ॒स्यनु॑ सचन्ते\-ऽन्त॒रा यूपं॑ चाहव॒नीयं॑ च हरति॒ रक्ष॑सा॒मप॑हत्यै प॒शोर्वा आल॑ब्धस्य॒ मनो\-ऽप॑ क्रामति म॒नोता॑यै ह॒विषो॑\-ऽवदी॒यमा॑न॒स्यानु॑ ब्रू॒हीत्या॑ह॒ मन॑ ए॒वास्याव॑ रुन्द्ध॒ एका॑दशाव॒दाना॒न्यव॑ द्यति॒ दश॒ वै प॒शोः प्रा॒णा आ॒त्मैका॑द॒शो यावा॑ने॒व प॒शुस्तस्याव॑॥५३॥

%6.3.10.4
द्य॒ति॒ हृद॑य॒स्याग्रे\-ऽव॑ द्य॒त्यथ॑ जि॒ह्वाया॒ अथ॒ वक्ष॑सो॒ यद्वै हृद॑येनाभि॒गच्छ॑ति॒ तज्जि॒ह्वया॑ वदति॒ यज्जि॒ह्वया॒ वद॑ति॒ तदुर॒सो\-ऽधि॒ निर्व॑दत्ये॒तद्वै प॒शोर्य॑थापू॒र्वं यस्यै॒वम॑व॒दाय॑ यथा॒काम॒मुत्त॑रेषामव॒द्यति॑ यथापू॒र्वमे॒वास्य॑ प॒शोरव॑त्तम्भवति मध्य॒तो गु॒दस्याव॑ द्यति मध्य॒तो हि प्रा॒ण उ॑त्त॒मस्याव॑ द्यति॥५४॥

%6.3.10.5
उ॒त्त॒मो हि प्रा॒णो यदीत॑रं॒ यदीत॑रमु॒भय॑मे॒वाजा॑मि॒ जाय॑मानो॒ वै ब्रा᳚ह्म॒णस्त्रि॒भिर्\mbox{}ऋ॑ण॒वा जा॑यते ब्रह्म॒चर्ये॒णर्\mbox{}षि॑भ्यो य॒ज्ञेन॑ दे॒वेभ्यः॑ प्र॒जया॑ पि॒तृभ्य॑ ए॒ष वा अ॑नृ॒णो यः पु॒त्री यज्वा᳚ ब्रह्मचारिवा॒सी तद॑व॒दानै॑रे॒वाव॑ दयते॒ तद॑व॒दाना॑नामवदान॒त्वन्दे॑वासु॒राः संय॑त्ता आस॒न्ते दे॒वा अ॒ग्निम॑ब्रुव॒न्त्वया॑ वी॒रेणासु॑रान॒भि भ॑वा॒मेति॑॥५॥

%6.3.10.6
सो᳚\-ऽब्रवी॒द्वरं॑ वृणै प॒शोरु॑द्धा॒रमुद्ध॑रा॒ इति॒ स ए॒तमु॑द्धा॒रमुद॑हरत॒ दोः पू᳚र्वा॒र्धस्य॑ गु॒दम्म॑ध्य॒तः श्रोणिं॑ जघना॒र्धस्य॒ ततो॑ दे॒वा अभ॑व॒न्परासु॑रा॒ यत्त्र्य॒ङ्गाणाꣳ॑ समव॒द्यति॒ भ्रातृ॑व्या॒भिभूत्यै॒ भव॑त्या॒त्मना॒ परा᳚स्य॒ भ्रातृ॑व्यो भवत्यक्ष्ण॒याव॑ द्यति॒ तस्मा॑दक्ष्ण॒या प॒शवो\-ऽङ्गा॑नि॒ प्र ह॑रन्ति॒ प्रति॑ष्ठित्यै॥५६॥

%6.3.11.0
{\anuvakamend[{ए॒तौ प॑शू॒नाꣳ समे॑धस्यै॒व तस्यावो᳚त्त॒मस्याव॑ द्य॒तीति॒ पञ्च॑चत्वारिꣳशच्च}]}%॥10॥

%6.3.11.1
मेद॑सा॒ स्रुचौ॒ प्रोर्णो॑ति॒ मेदो॑रूपा॒ वै प॒शवो॑ रू॒पमे॒व प॒शुषु॑ दधाति यू॒षन्न॑व॒धाय॒ प्रोर्णो॑ति॒ रसो॒ वा ए॒ष प॑शू॒नां यद्यू रस॑मे॒व प॒शुषु॑ दधाति पा॒र्श्वेन॑ वसाहो॒मम्प्र यौ॑ति॒ मध्यं॒ वा ए॒तत्प॑शू॒नां यत्पा॒र्श्वꣳ रस॑ ए॒ष प॑शू॒नां यद्वसा॒ यत्पा॒र्श्वेन॑ वसाहो॒मम्प्र॒यौति॑ मध्य॒त ए॒व प॑शू॒नाꣳ रसं॑ दधाति॒ घ्नन्ति॑॥५७॥

%6.3.11.2
वा ए॒तत्प॒शुं यथ्सं᳚ज्ञ॒पय॑न्त्यै॒न्द्रः खलु॒ वै दे॒वत॑या प्रा॒ण ऐ॒न्द्रो॑\-ऽपा॒न ऐ॒न्द्रः प्रा॒णो अङ्गे॑अङ्गे॒ नि दे᳚ध्य॒दित्या॑ह प्राणापा॒नावे॒व प॒शुषु॑ दधाति॒ देव॑ त्वष्ट॒र्भूरि॑ ते॒ सꣳस॑मे॒त्वित्या॑ह त्वा॒ष्ट्रा हि दे॒वत॑या प॒शवो॒ विषु॑रूपा॒ यथ्सल॑क्ष्माणो॒ भव॒थेत्या॑ह॒ विषु॑रूपा॒ ह्ये॑ते सन्तः॒ सल॑क्ष्माण ए॒तर्\mbox{}हि॒ भव॑न्ति देव॒त्रा यन्तम्᳚॥५८॥

%6.3.11.3
अव॑से॒ सखा॒यो\-ऽनु॑ त्वा मा॒ता पि॒तरो॑ मद॒न्त्वित्या॒हानु॑मतमे॒वैन॑म्मा॒त्रा पि॒त्रा सु॑व॒र्गं लो॒कं ग॑मयत्यर्ध॒र्चे व॑साहो॒मं जु॑होत्य॒सौ वा अ॑र्ध॒र्च इ॒यम॑र्ध॒र्च इ॒मे ए॒व रसे॑नानक्ति॒ दिशो॑ जुहोति॒ दिश॑ ए॒व रसे॑नान॒क्त्यथो॑ दि॒ग्भ्य ए॒वोर्ज॒ꣳ॒ रस॒मव॑ रुन्द्धे प्राणापा॒नौ वा ए॒तौ प॑शू॒नां यत्पृ॑षदा॒ज्यं वा॑नस्प॒त्याः खलु॑॥५९॥

%6.3.11.4
वै दे॒वत॑या प॒शवो॒ यत्पृ॑षदा॒ज्यस्यो॑प॒हत्याह॒ वन॒स्पत॒ये\-ऽनु॑ ब्रूहि॒ वन॒स्पत॑ये॒ प्रेष्येति॑ प्राणापा॒नावे॒व प॒शुषु॑ दधात्य॒न्यस्या᳚न्यस्य समव॒त्तꣳ स॒मव॑द्यति॒ तस्मा॒न्नाना॑रूपाः प॒शवो॑ यू॒ष्णोप॑ सिञ्चति॒ रसो॒ वा ए॒ष प॑शू॒नां यद्यू रस॑मे॒व प॒शुषु॑ दधा॒तीडा॒मुप॑ ह्वयते प॒शवो॒ वा इडा॑ प॒शूने॒वोप॑ ह्वयते च॒तुरुप॑ ह्वयते॥६०॥

%6.3.11.5
चतु॑ष्पादो॒ हि प॒शवो॒ यं का॒मये॑ताप॒शुः स्या॒दित्य॑मे॒दस्कं॒ तस्मा॒ आ द॑ध्या॒न्मेदो॑रूपा॒ वै प॒शवो॑ रू॒पेणै॒वैन॑म्प॒शुभ्यो॒ निर्भ॑जत्यप॒शुरे॒व भ॑वति॒ यं का॒मये॑त पशु॒मान्थ्स्या॒दिति॒ मेद॑स्व॒त्तस्मा॒ आ द॑ध्या॒न्मेदो॑रूपा॒ वै प॒शवो॑ रू॒पेणै॒वास्मै॑ प॒शूनव॑ रुन्द्धे पशु॒माने॒व भ॑वति प्र॒जाप॑तिर्य॒ज्ञम॑सृजत॒ स आज्यम्᳚॥६१॥

%6.3.11.6
पु॒रस्ता॑दसृजत प॒शुम्म॑ध्य॒तः पृ॑षदा॒ज्यम्प॒श्चात्तस्मा॒दाज्ये॑न प्रया॒जा इ॑ज्यन्ते प॒शुना॑ मध्य॒तः पृ॑षदा॒ज्येना॑नूया॒जास्तस्मा॑दे॒तन्मि॒श्रमि॑व पश्चाथ्सृ॒ष्टꣴ ह्येका॑दशानूया॒जान् य॑जति॒ दश॒ वै प॒शोः प्रा॒णा आ॒त्मैका॑द॒शो यावा॑ने॒व प॒शुस्तमनु॑ यजति॒ घ्नन्ति॒ वा ए॒तत्प॒शुं यथ्सं᳚ज्ञ॒पय॑न्ति प्राणापा॒नौ खलु॒ वा ए॒तौ प॑शू॒नां यत्पृ॑षदा॒ज्यं यत्पृ॑षदा॒ज्येना॑नूया॒जान् यज॑ति प्राणापा॒नावे॒व प॒शुषु॑ दधाति॥६२॥

%6.4.0.0
{\anuvakamend[{घ्नन्ति॒ यन्त॒ङ्खलु॑ च॒तुरुप॑ ह्वयत॒ आज्यं॒ यत्पृ॑षदा॒ज्येन॒ षट्च॑}]}%॥11॥

%6.4.0.0

{\anuvakamend[{य॒ज्ञेन॒ ता उ॑प॒यड्भि॑र्दे॒वा वै य॒ज्ञमाग्नी᳚ध्रे ब्रह्मवा॒दिनः॒ सत्वै दे॒वस्य॒ ग्रावा॑णं प्रा॒ण उ॑पा॒ꣳ॒श्व॑ग्रा दे॒वा वा उ॑पा॒ꣳ॒शौ वाग्वै मि॒त्रं य॒ज्ञस्य॒ बृह॒स्पति॑र्दे॒वा वा आ᳚ग्रय॒णाग्रा॒नेका॑दश}]}%॥11॥
\prashnaend{य॒ज्ञेन॑ लो॒के प॑शु॒मान्थ्स्या॒थ्सव॑न॒म्माध्य॑न्दिनं॒ वाग्वा अरि॑क्तानि॒ तत्प्र॒जा अ॒भ्येक॑पञ्चा॒शत्॥51॥ य॒ज्ञेन॒ गौर॒भि निव॑र्तते॥}
%%% END PRASHNA

\sect{चतुर्थः प्रश्नः}\setcounter{anuvakam}{0}
\dnsub{तैत्तिरीयसंहितायां षष्ठमकाण्डे चतुर्थः प्रश्नः}
%6.4.1.0
%6.4.1.1
य॒ज्ञेन॒ वै प्र॒जाप॑तिः प्र॒जा अ॑सृजत॒ ता उ॑प॒यड्भि॑रे॒वासृ॑जत॒ यदु॑प॒यज॑ उप॒यज॑ति प्र॒जा ए॒व तद्यज॑मानः सृजते जघना॒र्धादव॑ द्यति जघना॒र्धाद्धि प्र॒जाः प्र॒जाय॑न्ते स्थविम॒तो\-ऽव॑ द्यति स्थविम॒तो हि प्र॒जाः प्र॒जाय॒न्ते\-ऽस॑म्भिन्द॒न्नव॑ द्यति प्रा॒णाना॒मस॑म्भेदाय॒ न प॒र्याव॑र्तयति॒ यत्प॑र्याव॒र्तये॑दुदाव॒र्तः प्र॒जा ग्राहु॑कः स्याथ्समु॒द्रं ग॑च्छ॒ स्वाहेत्या॑ह रेतः॑॥१॥

%6.4.1.2
ए॒व तद्द॑धात्य॒न्तरि॑क्षं गच्छ॒ स्वाहेत्या॑हा॒न्तरि॑क्षेणै॒वास्मै᳚ प्र॒जाः प्र ज॑नयत्य॒न्तरि॑क्ष॒ꣴ॒ ह्यनु॑ प्र॒जाः प्र॒जाय॑न्ते दे॒वꣳ स॑वि॒तारं॑ गच्छ॒ स्वाहेत्या॑ह सवि॒तृप्र॑सूत ए॒वास्मै᳚ प्र॒जाः प्र ज॑नयत्यहोरा॒त्रे ग॑च्छ॒ स्वाहेत्या॑हाहोरा॒त्राभ्या॑मे॒वास्मै᳚ प्र॒जाः प्र ज॑नयत्यहोरा॒त्रे ह्यनु॑ प्र॒जाः प्र॒जाय॑न्ते मि॒त्रावरु॑णौ गच्छ॒ स्वाहा᳚॥२॥

%6.4.1.3
इत्या॑ह प्र॒जास्वे॒व प्रजा॑तासु प्राणापा॒नौ द॑धाति॒ सोमं॑ गच्छ॒ स्वाहेत्या॑ह सौ॒म्या हि दे॒वत॑या प्र॒जा य॒ज्ञं ग॑च्छ॒ स्वाहेत्या॑ह प्र॒जा ए॒व य॒ज्ञियाः᳚ करोति॒ छन्दाꣳ॑सि गच्छ॒ स्वाहेत्या॑ह प॒शवो॒ वै छन्दाꣳ॑सि प॒शूने॒वाव॑ रुन्द्धे॒ द्यावा॑पृथि॒वी ग॑च्छ॒ स्वाहेत्या॑ह प्र॒जा ए॒व प्रजा॑ता॒ द्यावा॑पृथि॒वीभ्या॑मुभ॒यतः॒ परि॑ गृह्णाति नभः॑॥३॥

%6.4.1.4
दि॒व्यं ग॑च्छ॒ स्वाहेत्या॑ह प्र॒जाभ्य॑ ए॒व प्रजा॑ताभ्यो॒ वृष्टिं॒ नि य॑च्छत्य॒ग्निं वै᳚श्वान॒रं ग॑च्छ॒ स्वाहेत्या॑ह प्र॒जा ए॒व प्रजा॑ता अ॒स्यां प्रति॑ ष्ठापयति प्रा॒णानां॒ वा ए॒षो\-ऽव॑ द्यति॒ यो॑\-ऽव॒द्यति॑ गु॒दस्य॒ मनो॑ मे॒ हार्दि॑ य॒च्छेत्या॑ह प्रा॒णाने॒व य॑थास्था॒नमुप॑ ह्वयते प॒शोर्वा आल॑ब्धस्य॒ हृद॑य॒ꣳ॒ शुगृ॑च्छति॒ सा हृ॑दयशू॒लम्॥४॥

%6.4.1.5
अ॒भि समे॑ति॒ यत्पृ॑थि॒व्याꣳ हृ॑दयशू॒लमु॑द्वा॒सये᳚त्पृथि॒वीꣳ शु॒चार्प॑ये॒द्यद॒फ्स्व॑पः शु॒चार्प॑ये॒च्छुष्क॑स्य चा॒र्द्रस्य॑ च सं॒धावुद्वा॑सयत्यु॒भय॑स्य॒ शान्त्यै॒ यं द्वि॒ष्यात्तं ध्या॑येच्छु॒चैवैन॑मर्पयति॥५॥

%6.4.2.0
{\anuvakamend[{रेतो॑ मि॒त्रावरु॑णौ गच्छ॒ स्वाहा॒ नभो॑ हृदयशू॒लन्द्वात्रिꣳ॑शच्च}]}%॥१॥

%6.4.2.1
दे॒वा वै य॒ज्ञमाग्नी᳚ध्रे॒ व्य॑भजन्त॒ ततो॒ यद॒त्यशि॑ष्यत॒ तद॑ब्रुव॒न्वस॑तु॒ नु न॑ इ॒दमिति॒ तद्व॑सती॒वरी॑णां वसतीवरि॒त्वम् तस्मि॑न्प्रा॒तर्न सम॑शक्नुव॒न्तद॒फ्सु प्रावे॑शय॒न्ता व॑सती॒वरी॑रभवन्वसती॒वरी᳚र्गृह्णाति य॒ज्ञो वै व॑सती॒वरी᳚र्य॒ज्ञमे॒वारभ्य॑ गृही॒त्वोप॑ वसति॒ यस्यागृ॑हीता अ॒भि नि॒म्रोचे॒दना॑रब्धो\-ऽस्य य॒ज्ञः स्या᳚त्॥६॥

%6.4.2.2
य॒ज्ञं वि च्छि॑न्द्याज्ज्योति॒ष्या॑ वा गृह्णी॒याद्धिर॑ण्यं वाव॒धाय॒ सशु॑क्राणामे॒व गृ॑ह्णाति॒ यो वा᳚ ब्राह्म॒णो ब॑हुया॒जी तस्य॒ कुम्भ्या॑नां गृह्णीया॒थ्स हि गृ॑ही॒तव॑सतीवरीको वसती॒वरी᳚र्गृह्णाति प॒शवो॒ वै व॑सती॒वरीः᳚ प॒शूने॒वारभ्य॑ गृही॒त्वोप॑ वसति॒ यद॑न्वी॒पं तिष्ठ॑न्गृह्णी॒यान्नि॒र्मार्गु॑का अस्मात्प॒शवः॑ स्युः प्रती॒पं तिष्ठ॑न्गृह्णाति प्रति॒रुध्यै॒वास्मै॑ प॒शून्गृ॑ह्णा॒तीन्द्रः॑॥७॥

%6.4.2.3
वृ॒त्रम॑ह॒न्थ्सो \-ऽपो \-ऽभ्य॑म्रियत॒ तासां॒ यन्मेध्यं॑ य॒ज्ञिय॒ꣳ॒ सदे॑व॒मासी॒त्तदत्य॑मुच्यत॒ ता वह॑न्तीरभव॒न्वह॑न्तीनां गृह्णाति॒ या ए॒व मेध्या॑ य॒ज्ञियाः॒ सदे॑वा॒ आप॒स्तासा॑मे॒व गृ॑ह्णाति॒ नान्त॒मा वह॑न्ती॒रती॑या॒द्यद॑न्त॒मा वह॑न्तीरती॒याद्य॒ज्ञमति॑ मन्येत॒ न स्था॑व॒राणां᳚ गृह्णीया॒द्वरु॑णगृहीता॒ वै स्था॑व॒रा यथ्स्था॑व॒राणां᳚ गृह्णी॒यात्॥८॥

%6.4.2.4
वरु॑णेनास्य य॒ज्ञं ग्रा॑हये॒द्यद्वै दिवा॒ भव॑त्य॒पो रात्रिः॒ प्र वि॑शति॒ तस्मा᳚त्ता॒म्रा आपो॒ दिवा॑ ददृश्रे॒ यन्नक्त॒म्भव॑त्य॒पो\-ऽहः॒ प्र वि॑शति॒ तस्मा᳚च्च॒न्द्रा आपो॒ नक्तं॑ ददृश्रे छा॒यायै॑ चा॒तप॑तश्च सं॒धौ गृ॑ह्णात्यहोरा॒त्रयो॑रे॒वास्मै॒ वर्णं॑ गृह्णाति ह॒विष्म॑तीरि॒मा आप॒ इत्या॑ह ह॒विष्कृ॑तानामे॒व गृ॑ह्णाति ह॒विष्माꣳ॑ अस्तु॥९॥

%6.4.2.5
सूर्य॒ इत्या॑ह॒ सशु॑क्राणामे॒व गृ॑ह्णात्यनु॒ष्टुभा॑ गृह्णाति॒ वाग्वा अ॑नु॒ष्टुग्वा॒चैवैनाः॒ सर्व॑या गृह्णाति॒ चतु॑ष्पदय॒र्चा गृ॑ह्णाति॒ त्रिः सा॑दयति स॒प्त सम्प॑द्यन्ते स॒प्तप॑दा॒ शक्व॑री प॒शवः॒ शक्व॑री प॒शूने॒वाव॑ रुन्द्धे॒\-ऽस्मै वै लो॒काय॒ गार्\mbox{}ह॑पत्य॒ आ धी॑यते॒\-ऽमुष्मा॑ आहव॒नीयो॒ यद्गार्\mbox{}ह॑पत्य उपसा॒दये॑द॒स्मिल्लोँ॒के प॑शु॒मान्थ्स्या॒द्ययदा॑हव॒नीये॒\-ऽमुष्मिन्न्॑॥१०॥

%6.4.2.6
लो॒के प॑शु॒मान्थ्स्या॑दु॒भयो॒रुप॑ सादयत्यु॒भयो॑रे॒वैनं॑ लो॒कयोः᳚ पशु॒मन्तं॑ करोति स॒र्वतः॒ परि॑ हरति॒ रक्ष॑सा॒मप॑हत्या इन्द्राग्नि॒योर्भा॑ग॒धेयीः॒ स्थेत्या॑ह यथाय॒जुरे॒वैतदाग्नी᳚ध्र॒ उप॑ वासयत्ये॒तद्वै य॒ज्ञस्याप॑राजितं॒ यदाग्नी᳚ध्रं॒ यदे॒व य॒ज्ञस्याप॑राजितं॒ तदे॒वैना॒ उप॑ वासयति॒ यतः॒ खलु॒ वै य॒ज्ञस्य॒ वित॑तस्य॒ न क्रि॒यते॒ तदनु॑ य॒ज्ञꣳ रक्षा॒ꣳ॒स्यव॑ चरन्ति॒ यद्वह॑न्तीनां गृ॒ह्णाति॑ क्रि॒यमा॑णमे॒व तद्य॒ज्ञस्य॑ शये॒ रक्ष॑सा॒मन॑न्ववचाराय॒ न ह्ये॑ता ई॒लय॒न्त्या तृ॑तीयसव॒नात्परि॑ शेरे य॒ज्ञस्य॒ सन्त॑त्यै॥११॥

%6.4.3.0
{\anuvakamend[{स्या॒दिन्द्रो॑ गृह्णी॒याद॑स्त्व॒मुष्मि॑न्क्रि॒यते॒ षड्विꣳ॑शतिश्च}]}%॥२॥

%6.4.3.1
ब्र॒ह्म॒वा॒दिनो॑ वदन्ति॒ स त्वा अ॑ध्व॒र्युः स्या॒द्यः सोम॑मुपाव॒हर॒न्थ्सर्वा᳚भ्यो दे॒वता᳚भ्य उपाव॒हरे॒दिति॑ हृ॒दे त्वेत्या॑ह मनु॒ष्ये᳚भ्य ए॒वैतेन॑ करोति॒ मन॑से॒ त्वेत्या॑ह पि॒तृभ्य॑ ए॒वैतेन॑ करोति दि॒वे त्वा॒ सूर्या॑य॒ त्वेत्या॑ह दे॒वेभ्य॑ ए॒वैतेन॑ करोत्ये॒ताव॑ती॒र्वै दे॒वता॒स्ताभ्य॑ ए॒वैन॒ꣳ॒ सर्वा᳚भ्य उ॒पाव॑हरति पु॒रा वा॒चः॥१२॥

%6.4.3.2
प्रव॑दितोः प्रातरनुवा॒कमु॒पाक॑रोति॒ याव॑त्ये॒व वाक्तामव॑ रुन्द्धे॒\-ऽपो\-ऽग्रे॑\-ऽभि॒व्याह॑रति य॒ज्ञो वा आपो॑ य॒ज्ञमे॒वाभि वाचं॒ वि सृ॑जति॒ सर्वा॑णि॒ छन्दा॒ꣳ॒स्यन्वा॑ह प॒शवो॒ वै छन्दाꣳ॑सि प॒शूने॒वाव॑ रुन्द्धे गायत्रि॒या तेज॑स्कामस्य॒ परि॑ दध्यात्त्रि॒ष्टुभे᳚न्द्रि॒यका॑मस्य॒ जग॑त्या प॒शुका॑मस्यानु॒ष्टुभा᳚ प्रति॒ष्ठाका॑मस्य प॒ङ्क्त्या य॒ज्ञका॑मस्य वि॒राजान्न॑कामस्य शृ॒णोत्व॒ग्निः स॒मिधा॒ हवम्᳚॥१३॥

%6.4.3.3
म॒ इत्या॑ह सवि॒तृप्र॑सूत ए॒व दे॒वता᳚भ्यो नि॒वेद्या॒पो\-ऽच्छै᳚त्य॒प इ॑ष्य होत॒रित्या॑हेषि॒तꣳ हि कर्म॑ क्रि॒यते॒ मैत्रा॑वरुणस्य चमसाध्वर्य॒वा द्र॒वेत्या॑ह मि॒त्रावरु॑णौ॒ वा अ॒पां ने॒तारौ॒ ताभ्या॑मे॒वैना॒ अच्छै॑ति॒ देवी॑रापो अपां नपा॒दित्या॒हाहु॑त्यै॒वैना॑ नि॒ष्क्रीय॑ गृह्णा॒त्यथो॑ ह॒विष्कृ॑तानामे॒वाभिघृ॑तानां गृह्णाति॥१४॥

%6.4.3.4
कार्\mbox{}षि॑र॒सीत्या॑ह॒ शम॑लमे॒वासा॒मप॑ प्लावयति समु॒द्रस्य॒ वोक्षि॑त्या॒ उन्न॑य॒ इत्या॑ह॒ तस्मा॑द॒द्यमा॑नाः पी॒यमा॑ना॒ आपो॒ न क्षी॑यन्ते॒ योनि॒र्वै य॒ज्ञस्य॒ चात्वा॑लं य॒ज्ञो व॑सती॒वरीर्\mbox{}॑होतृचम॒सं च॑ मैत्रावरुणचम॒सं च॑ स॒ꣴ॒स्पर्श्य॑ वसती॒वरी॒र्व्यान॑यति य॒ज्ञस्य॑ सयोनि॒त्वायाथो॒ स्वादे॒वैना॒ योनेः॒ प्र ज॑नय॒त्यध्व॒र्यो\-ऽवे॑र॒पा (३) इत्या॑हो॒तेम॑नन्नमुरु॒तेमाः प॒श्येति॒ वावैतदा॑ह॒ यद्य॑ग्निष्टो॒मो जु॒होति॒ यद्यु॒क्थ्यः॑ परि॒धौ नि मा᳚र्ष्टि॒ यद्य॑तिरा॒त्रो यजु॒र्वद॒न्प्र प॑द्यते यज्ञक्रतू॒नां व्यावृ॑त्त्यै॥१५॥

%6.4.4.0
{\anuvakamend[{वा॒चो हव॑म॒भिघृ॑तानां गृह्णात्यु॒त पञ्च॑विꣳशतिश्च}]}%॥३॥

%6.4.4.1
दे॒वस्य॑ त्वा सवि॒तुः प्र॑स॒व इति॒ ग्रावा॑ण॒मा द॑त्ते॒ प्रसू᳚त्या अ॒श्विनो᳚र्बा॒हुभ्या॒मित्या॑हा॒श्विनौ॒ हि दे॒वाना॑मध्व॒र्यू आस्ता᳚म् पू॒ष्णो हस्ता᳚भ्या॒मित्या॑ह॒ यत्यै॑ प॒शवो॒ वै सोमो᳚ व्या॒न उ॑पाꣳशु॒सव॑नो॒ यदु॑पाꣳशु॒सव॑नम॒भि मिमी॑ते व्या॒नमे॒व प॒शुषु॑ दधा॒तीन्द्रा॑य॒ त्वेन्द्रा॑य॒ त्वेति॑ मिमीत॒ इन्द्रा॑य॒ हि सोम॑ आह्रि॒यते॒ पञ्च॒ कृत्वो॒ यजु॑षा मिमीते॥१६॥

%6.4.4.2
पञ्चा᳚क्षरा प॒ङ्क्तिः पाङ्क्तो॑ य॒ज्ञो य॒ज्ञमे॒वाव॑ रुन्द्धे॒ पञ्च॒ कृत्व॑स्तू॒ष्णीन्दश॒ सम्प॑द्यन्ते॒ दशा᳚क्षरा वि॒राडन्नं॑ वि॒राड्वि॒राजै॒वान्नाद्य॒मव॑ रुन्द्धे श्वा॒त्राः स्थ॑ वृत्र॒तुर॒ इत्या॑है॒ष वा अ॒पाꣳ सो॑मपी॒थो य ए॒वं वेद॒ नाफ्स्वार्ति॒मार्च्छ॑ति॒ यत्ते॑ सोम दि॒वि ज्योति॒रित्या॑है॒भ्य ए॒वैनम्᳚॥१७॥

%6.4.4.3
लो॒केभ्यः॒ सम्भ॑रति॒ सोमो॒ वै राजा॒ दिशो॒\-ऽभ्य॑ध्याय॒थ्स दिशो\-ऽनु॒ प्रावि॑श॒त्प्रागपा॒गुद॑गध॒रागित्या॑ह दि॒ग्भ्य ए॒वैन॒ꣳ॒ सम्भ॑र॒त्यथो॒ दिश॑ ए॒वास्मा॒ अव॑ रु॒न्द्धे\-ऽम्ब॒ नि ष्व॒रेत्या॑ह॒ कामु॑का एन॒ꣴ॒ स्त्रियो॑ भवन्ति॒ य ए॒वं वेद॒ यत्ते॑ सो॒मादा᳚भ्यं॒ नाम॒ जागृ॒वीति॑॥१८॥

%6.4.4.4
आ॒है॒ष वै सोम॑स्य सोमपी॒थो य ए॒वं वेद॒ न सौ॒म्यामार्ति॒मार्च्छ॑ति॒ घ्नन्ति॒ वा ए॒तथ्सोमं॒ यद॑भिषु॒ण्वन्त्य॒ꣳ॒शूनप॑ गृह्णाति॒ त्राय॑त ए॒वैनं॑ प्रा॒णा वा अ॒ꣳ॒शवः॑ प॒शवः॒ सोमो॒\-ऽꣳ॒शून्पुन॒रपि॑ सृजति प्रा॒णाने॒व प॒शुषु॑ दधाति॒ द्वौद्वा॒वपि॑ सृजति॒ तस्मा॒द्द्वौद्वौ᳚ प्रा॒णाः॥१९॥

%6.4.5.0
{\anuvakamend[{यजु॑षा मिमीत एनं॒ जागृ॒वीति॒ चतु॑श्चत्वारिꣳशच्च}]}%॥४॥

%6.4.5.1
प्रा॒णो वा ए॒ष यदु॑पा॒ꣳ॒शुर्यदु॑पा॒ꣳ॒श्व॑ग्रा॒ ग्रहा॑ गृ॒ह्यन्ते᳚ प्रा॒णमे॒वानु॒ प्र य॑न्त्यरु॒णो ह॑ स्मा॒हौप॑वेशिः प्रातःसव॒न ए॒वाहं य॒ज्ञꣳ सꣴस्था॑पयामि॒ तेन॒ ततः॒ सꣴस्थि॑तेन चरा॒मीत्य॒ष्टौ कृत्वो\-ऽग्रे॒\-ऽभि षु॑णोत्य॒ष्टाक्ष॑रा गाय॒त्री गा॑य॒त्रम्प्रा॑तःसव॒नम् प्रा॑तःसव॒नमे॒व तेना᳚प्नो॒त्येका॑दश॒ कृत्वो᳚ द्वि॒तीय॒मेका॑दशाक्षरा त्रि॒ष्टुप्त्रैष्टु॑भ॒म्माध्यं॑दिनम्॥२०॥

%6.4.5.2
सव॑न॒म्माध्यं॑दिनमे॒व सव॑नं॒ तेना᳚प्नोति॒ द्वाद॑श॒ कृत्व॑स्तृ॒तीय॒न्द्वाद॑शाक्षरा॒ जग॑ती॒ जाग॑तं तृतीयसव॒नन्तृ॑तीयसव॒नमे॒व तेना᳚प्नोत्ये॒ताꣳ ह॒ वाव स य॒ज्ञस्य॒ सꣴस्थि॑तिमुवा॒चास्क॑न्दा॒यास्क॑न्न॒ꣳ॒ हि तद्यद्य॒ज्ञस्य॒ सꣴस्थि॑तस्य॒ स्कन्द॒त्यथो॒ खल्वा॑हुर्गाय॒त्री वाव प्रा॑तःसव॒ने नाति॒वाद॒ इत्यन॑तिवादुक एन॒म्भ्रातृ॑व्यो भवति॒ य ए॒वं वेद॒ तस्मा॑द॒ष्टाव॑ष्टौ॥२१॥

%6.4.5.3
कृत्वो॑\-ऽभि॒षुत्यं॑ ब्रह्मवा॒दिनो॑ वदन्ति प॒वित्र॑वन्तो॒\-ऽन्ये ग्रहा॑ गृ॒ह्यन्ते॒ किम्प॑वित्र उपा॒ꣳ॒शुरिति॒ वाक्प॑वित्र॒ इति॑ ब्रूयात् वा॒चस्पत॑ये पवस्व वाजि॒न्नित्या॑ह वा॒चैवैन॑म्पवयति॒ वृष्णो॑ अ॒ꣳ॒शुभ्या॒मित्या॑ह॒ वृष्णो॒ ह्ये॑ताव॒ꣳ॒शू यौ सोम॑स्य॒ गभ॑स्तिपूत॒ इत्या॑ह॒ गभ॑स्तिना॒ ह्ये॑नम्प॒वय॑ति दे॒वो दे॒वानां᳚ प॒वित्र॑म॒सीत्या॑ह दे॒वो ह्ये॑षः॥२२॥

%6.4.5.4
सं दे॒वानां᳚ प॒वित्रं॒ येषां᳚ भा॒गो\-ऽसि॒ तेभ्य॒स्त्वेत्या॑ह॒ येषा॒ꣳ॒ ह्ये॑ष भा॒गस्तेभ्य॑ एनं गृ॒ह्णाति॒ स्वां कृ॑तो॒\-ऽसीत्या॑ह प्रा॒णमे॒व स्वम॑कृत॒ मधु॑मतीर्न॒ इष॑स्कृ॒धीत्या॑ह॒ सर्व॑मे॒वास्मा॑ इ॒दꣴ स्व॑दयति॒ विश्वे᳚भ्यस्त्वेन्द्रि॒येभ्यो॑ दि॒व्येभ्यः॒ पार्थि॑वेभ्य॒ इत्या॑हो॒भये᳚ष्वे॒व दे॑वमनु॒ष्येषु॑ प्रा॒णान्द॑धाति॒ मन॑स्त्वा॥२३॥

%6.4.5.5
अ॒ष्ट्वित्या॑ह॒ मन॑ ए॒वाश्ञु॑त उ॒र्व॑न्तरि॑क्ष॒मन्वि॒हीत्या॑हान्तरिक्षदेव॒त्यो॑ हि प्रा॒णः स्वाहा᳚ त्वा सुभवः॒ सूर्या॒येत्या॑ह प्रा॒णा वै स्वभ॑वसो दे॒वास्तेष्वे॒व प॒रोक्षं॑ जुहोति दे॒वेभ्य॑स्त्वा मरीचि॒पेभ्य॒ इत्या॑हादि॒त्यस्य॒ वै र॒श्मयो॑ दे॒वा म॑रीचि॒पास्तेषां॒ तद्भा॑ग॒धेय॒न्ताने॒व तेन॑ प्रीणाति॒ यदि॑ का॒मये॑त॒ वर्\mbox{}षु॑कः प॒र्जन्यः॑॥२४॥

%6.4.5.6
स्या॒दिति॒ नीचा॒ हस्ते॑न॒ नि मृ॑ज्या॒द्वृष्टि॑मे॒व नि य॑च्छति॒ यदि॑ का॒मये॒ताव॑र्\mbox{}षुकः स्या॒दित्यु॑त्ता॒नेन॒ नि मृ॑ज्या॒द्वृष्टि॑मे॒वोद्य॑च्छति॒ यद्य॑भि॒चरे॑द॒मुं ज॒ह्यथ॑ त्वा होष्या॒मीति॑ ब्रूया॒दाहु॑तिमे॒वैन॑म्प्रे॒फ्सन् ह॑न्ति॒ यदि॑ दू॒रे स्यादा तमि॑तोस्तिष्ठेत्प्रा॒णमे॒वास्या॑नु॒गत्य॑ हन्ति॒ यद्य॑भि॒चरे॑द॒मुष्य॑॥२५॥

%6.4.5.7
त्वा॒ प्रा॒णे सा॑दया॒मीति॑ सादये॒दस॑न्नो॒ वै प्रा॒णः प्रा॒णमे॒वास्य॑ सादयति ष॒ड्भिर॒ꣳ॒शुभिः॑ पवयति॒ षड्वा ऋ॒तव॑ ऋ॒तुभि॑रे॒वैन॑म्पवयति॒ त्रिः प॑वयति॒ त्रय॑ इ॒मे लो॒का ए॒भिरे॒वैनं॑ लो॒कैः प॑वयति ब्रह्मवा॒दिनो॑ वदन्ति॒ कस्मा᳚थ्स॒त्यात्त्रयः॑ पशू॒नाꣳ हस्ता॑दाना॒ इति॒ यत्त्रिरु॑पा॒ꣳ॒शुꣳ हस्ते॑न विगृ॒ह्णाति॒ तस्मा॒त्त्रयः॑ पशू॒नाꣳ हस्ता॑दानाः॒ पुरु॑षो ह॒स्ती म॒र्कटः॑॥२६॥

%6.4.6.0
{\anuvakamend[{माध्य॑न्दिनम॒ष्टाव॑ष्टावे॒ष मन॑स्त्वा प॒र्जन्यो॒\-ऽमुष्य॒ पुरु॑षो॒ द्वे च॑}]}%॥५॥

%6.4.6.1
दे॒वा वै यद्य॒ज्ञे\-ऽकु॑र्वत॒ तदसु॑रा अकुर्वत॒ ते दे॒वा उ॑पा॒ꣳ॒शौ य॒ज्ञꣳ स॒ꣴ॒स्थाप्य॑मपश्य॒न्तमु॑पा॒ꣳ॒शौ सम॑स्थापय॒न्ते\-ऽसु॑रा॒ वज्र॑मु॒द्यत्य॑ दे॒वान॒भ्या॑यन्त॒ ते दे॒वा बिभ्य॑त॒ इन्द्र॒मुपा॑धाव॒न्तानिन्द्रो᳚\-ऽन्तर्या॒मेणा॒न्तर॑धत्त॒ तद॑न्तर्या॒मस्या᳚न्तर्याम॒त्वम् यद॑न्तर्या॒मो गृ॒ह्यते॒ भ्रातृ॑व्याने॒व तद्यज॑मानो॒\-ऽन्तर्ध॑त्ते॒\-ऽन्तस्ते᳚॥२७॥

%6.4.6.2
द॒धा॒मि॒ द्यावा॑पृथि॒वी अ॒न्तरु॒र्व॑न्तरि॑क्ष॒मित्या॑है॒भिरे॒व लो॒कैर्यज॑मानो॒ भ्रातृ॑व्यान॒न्तर्ध॑त्ते॒ ते दे॒वा अ॑मन्य॒न्तेन्द्रो॒ वा इ॒दम॑भू॒द्यद्व॒यꣴ स्म इति॒ ते᳚\-ऽब्रुव॒न्मघ॑व॒न्ननु॑ न॒ आ भ॒जेति॑ स॒जोषा॑ दे॒वैरव॑रैः॒ परै॒श्चेत्य॑ब्रवी॒द्ये चै॒व दे॒वाः परे॒ ये चाव॑रे॒ तानु॒भयान्॑॥२८॥

%6.4.6.3
अ॒न्वाभ॑जथ्स॒जोषा॑ दे॒वैरव॑रैः॒ परै॒श्चेत्या॑ह॒ ये चै॒व दे॒वाः परे॒ य चाव॑रे॒ तानु॒भया॑न॒न्वाभ॑जत्यन्तर्या॒मे म॑घवन्मादय॒स्वेत्या॑ह य॒ज्ञादे॒व यज॑मानं॒ नान्तरे᳚त्युपया॒मगृ॑हीतो॒\-ऽसीत्या॑हापा॒नस्य॒ धृत्यै॒ यदु॒भाव॑पवि॒त्रौ गृ॒ह्येया॑तां प्रा॒णम॑पा॒नो\-ऽनु॒ न्यृ॑च्छेत्प्र॒मायु॑कः स्यात्प॒वित्र॑वानन्तर्या॒मो गृ॑ह्यते॥२९॥

%6.4.6.4
प्रा॒णा॒पा॒नयो॒र्विधृ॑त्यै प्राणापा॒नौ वा ए॒तौ यदु॑पाꣳश्वन्तर्या॒मौ व्या॒न उ॑पाꣳशु॒सव॑नो॒ यं का॒मये॑त प्र॒मायु॑कः स्या॒दित्यसꣴ॑स्पृष्टौ॒ तस्य॑ सादयेद्व्या॒नेनै॒वास्य॑ प्राणापा॒नौ वि च्छि॑नत्ति ता॒जक्प्रमी॑यते॒ यं का॒मये॑त॒ सर्व॒मायु॑रिया॒दिति॒ सꣴस्पृ॑ष्टौ॒ तस्य॑ सादयेद्व्या॒नेनै॒वास्य॑ प्राणापा॒नौ सं त॑नोति॒ सर्व॒मायु॑रेति॥३०॥

%6.4.7.0
{\anuvakamend[{त॒ उ॒भया᳚न्गृह्यते॒ चतु॑श्चत्वारिꣳशच्च}]}%॥६॥

%6.4.7.1
वाग्वा ए॒षा यदै᳚न्द्रवाय॒वो यदै᳚न्द्रवाय॒वाग्रा॒ ग्रहा॑ गृ॒ह्यन्ते॒ वाच॑मे॒वानु॒ प्र य॑न्ति वा॒युं दे॒वा अ॑ब्रुव॒न्थ्सोम॒ꣳ॒ राजा॑नꣳ हना॒मेति॒ सो᳚\-ऽब्रवी॒द्वरं॑ वृणै॒ मद॑ग्रा ए॒व वो॒ ग्रहा॑ गृह्यान्ता॒ इति॒ तस्मा॑दैन्द्रवाय॒वाग्रा॒ ग्रहा॑ गृह्यन्ते॒ तम॑घ्न॒न्थ्सो॑\-ऽपूय॒त् तं दे॒वा नोपा॑धृष्णुव॒न्ते वा॒युम॑ब्रुवन्नि॒मं नः॑ स्वदय॥३१॥

%6.4.7.2
इति॒ सो᳚\-ऽब्रवी॒द्वरं॑ वृणै मद्देव॒त्या᳚न्ये॒व वः॒ पात्रा᳚ण्युच्यान्ता॒ इति॒ तस्मा᳚न्नानादेव॒त्या॑नि॒ सन्ति॑ वाय॒व्या᳚न्युच्यन्ते॒ तमे᳚भ्यो वा॒युरे॒वास्व॑दय॒त्तस्मा॒द्यत्पूय॑ति॒ तत्प्र॑वा॒ते वि ष॑जन्ति वा॒युर्\mbox{}हि तस्य॑ पवयि॒ता स्व॑दयि॒ता तस्य॑ वि॒ग्रह॑णं॒ नावि॑न्द॒न्थ्सा\-ऽदि॑तिरब्रवी॒द्वरं॑ वृणा॒ अथ॒ मया॒ वि गृ॑ह्णीध्वम्मद्देव॒त्या॑ ए॒व वः॒ सोमाः᳚॥३२॥

%6.4.7.3
स॒न्ना अ॑स॒न्नित्यु॑पया॒मगृ॑हीतो॒\-ऽसीत्या॑हादितिदेव॒त्या᳚स्तेन॒ यानि॒ हि दा॑रु॒मया॑णि॒ पात्रा᳚ण्य॒स्यै तानि॒ योनेः॒ सम्भू॑तानि॒ यानि॑ मृ॒न्मया॑नि सा॒क्षात्तान्य॒स्यै तस्मा॑दे॒वमा॑ह॒ वाग्वै परा॒च्यव्या॑कृतावद॒त्ते दे॒वा इन्द्र॑मब्रुवन्नि॒मां नो॒ वाचं॒ व्याकु॒र्विति॒ सो᳚\-ऽब्रवी॒द्वरं॑ वृणै॒ मह्यं॑ चै॒वैष वा॒यवे॑ च स॒ह गृ॑ह्याता॒ इति॒ तस्मा॑दैन्द्रवाय॒वः स॒ह गृ॑ह्यते॒ तामिन्द्रो॑ मध्य॒तो॑\-ऽव॒क्रम्य॒ व्याक॑रो॒त्तस्मा॑दि॒यं व्याकृ॑ता॒ वागु॑द्यते॒ तस्मा᳚थ्स॒कृदिन्द्रा॑य मध्य॒तो गृ॑ह्यते॒ द्विर्वा॒यवे॒ द्वौ हि स वरा॒ववृ॑णीत॥३३॥

%6.4.8.0
{\anuvakamend[{स्व॒द॒य॒ सोमाः᳚ स॒हाष्टाविꣳ॑शतिश्च}]}%॥७॥

%6.4.8.1
मि॒त्रं दे॒वा अ॑ब्रुव॒न्थ्सोम॒ꣳ॒ राजा॑नꣳ हना॒मेति॒ सो᳚\-ऽब्रवी॒न्नाहꣳ सर्व॑स्य॒ वा अ॒हम्मि॒त्रम॒स्मीति॒ तम॑ब्रुव॒न् हना॑मै॒वेति॒ सो᳚\-ऽब्रवी॒द्वरं॑ वृणै॒ पय॑सै॒व मे॒ सोमꣴ॑ श्रीण॒न्निति॒ तस्मा᳚न्मैत्रावरु॒णम्पय॑सा श्रीणन्ति॒ तस्मा᳚त्प॒शवो\-ऽपा᳚क्रामन् मि॒त्रः सन्क्रू॒रम॑क॒रिति॑ क्रू॒रमि॑व॒ खलु॒ वा ए॒षः॥३४॥

%6.4.8.2
क॒रो॒ति॒ यः सोमे॑न॒ यज॑ते॒ तस्मा᳚त्प॒शवो\-ऽप॑ क्रामन्ति॒ यन्मै᳚त्रावरु॒णम्पय॑सा श्री॒णाति॑ प॒शुभि॑रे॒व तन्मि॒त्रꣳ स॑म॒र्धय॑ति प॒शुभि॒र्यज॑मानम्पु॒रा खलु॒ वावैवम्मि॒त्रो॑\-ऽवे॒दप॒ मत्क्रू॒रं च॒क्रुषः॑ प॒शवः॑ क्रमिष्य॒न्तीति॒ तस्मा॑दे॒वम॑वृणीत॒ वरु॑णं दे॒वा अ॑ब्रुव॒न्त्वयाꣳ॑श॒भुवा॒ सोम॒ꣳ॒ राजा॑नꣳ हना॒मेति॒ सो᳚\-ऽब्रवी॒द्वरं॑ वृणै॒ मह्यं॑ च॥३५॥

%6.4.8.3
ए॒वैष मि॒त्राय॑ च स॒ह गृ॑ह्याता॒ इति॒ तस्मा᳚न्मैत्रावरु॒णः स॒ह गृ॑ह्यते॒ तस्मा॒द्राज्ञा॒ राजा॑नमꣳश॒भुवा᳚ घ्नन्ति॒ वैश्ये॑न॒ वैश्यꣳ॑ शू॒द्रेण॑ शू॒द्रन्न वा इ॒दं दिवा॒ न नक्त॑मासी॒दव्या॑वृत्त॒न्ते दे॒वा मि॒त्रावरु॑णावब्रुवन्नि॒दं नो॒ वि वा॑सयत॒मिति॒ ताव॑ब्रूतां॒ वरं॑ वृणावहा॒ एक॑ ए॒वावत्पूर्वो॒ ग्रहो॑ ग्रहो गृह्याता॒ इति॒ तस्मा॑दैन्द्रवाय॒वः पूर्वो॑ मैत्रावरु॒णाद्गृ॑ह्यते प्राणापा॒नौ ह्ये॑तौ यदु॑पाꣳश्वन्तर्या॒मौ मि॒त्रो\-ऽह॒रज॑नय॒द्वरु॑णो॒ रात्रि॒न्ततो॒ वा इ॒दं व्यौ᳚च्छ॒द्यन्मै᳚त्रावरु॒णो गृ॒ह्यते॒ व्यु॑ष्ट्यै॥३६॥

%6.4.9.0
{\anuvakamend[{ए॒ष चै᳚न्द्रवाय॒वो द्वाविꣳ॑शतिश्च}]}%॥८॥

%6.4.9.1
य॒ज्ञस्य॒ शिरो᳚\-ऽच्छिद्यत॒ ते दे॒वा अ॒श्विना॑वब्रुवन्भि॒षजौ॒ वै स्थ॑ इ॒दं य॒ज्ञस्य॒ शिरः॒ प्रति॑ धत्त॒मिति॒ ताव॑ब्रूतां॒ वरं॑ वृणावहै॒ ग्रह॑ ए॒व ना॒वत्रापि॑ गृह्यता॒मिति॒ ताभ्या॑मे॒तमा᳚श्वि॒नम॑गृह्ण॒न्ततो॒ वै तौ य॒ज्ञस्य॒ शिरः॒ प्रत्य॑धत्ता॒म् यदा᳚श्वि॒नो गृ॒ह्यते॑ य॒ज्ञस्य॒ निष्कृ॑त्यै॒ तौ दे॒वा अ॑ब्रुव॒न्नपू॑तौ॒ वा इ॒मौ म॑नुष्यच॒रौ॥३७॥

%6.4.9.2
भि॒षजा॒विति॒ तस्मा᳚द्ब्राह्म॒णेन॑ भेष॒जं न का॒र्य॑मपू॑तो॒ ह्ये  षो॑\-ऽमे॒ध्यो यो भि॒षक्तौ ब॑हिष्पवमा॒नेन॑ पवयि॒त्वा ताभ्या॑मे॒तमा᳚श्वि॒नम॑गृह्ण॒न्तस्मा᳚द्बहिष्पवमा॒ने स्तु॒त आ᳚श्वि॒नो गृ॑ह्यते॒ तस्मा॑दे॒वं वि॒दुषा॑ बहिष्पवमा॒न उ॑प॒सद्यः॑ प॒वित्रं॒ वै ब॑हिष्पवमा॒न आ॒त्मान॑मे॒व प॑वयते॒ तयो᳚स्त्रे॒धा भैष॑ज्यं॒ वि न्य॑दधुर॒ग्नौ तृ॑तीयम॒फ्सु तृती॑यम्ब्राह्म॒णे तृती॑य॒न्तस्मा॑दुदपा॒त्रम्॥३८॥

%6.4.9.3
उ॒प॒नि॒धाय॑ ब्राह्म॒णं द॑क्षिण॒तो नि॒षाद्य॑ भेष॒जं कु॑र्या॒द्याव॑दे॒व भे॑ष॒जं तेन॑ करोति स॒मर्धु॑कमस्य कृ॒तम्भ॑वति ब्रह्मवा॒दिनो॑ वदन्ति॒ कस्मा᳚थ्स॒त्यादेक॑पात्रा द्विदेव॒त्या॑ गृ॒ह्यन्ते᳚ द्वि॒पात्रा॑ हूयन्त॒ इति॒ यदेक॑पात्रा गृ॒ह्यन्ते॒ तस्मा॒देको᳚\-ऽन्तर॒तः प्रा॒णो द्वि॒पात्रा॑ हूयन्ते॒ तस्मा॒द्द्वौद्वौ॑ ब॒हिष्टा᳚त्प्रा॒णाः प्रा॒णा वा ए॒ते यद्द्वि॑देव॒त्याः᳚ प॒शव॒ इडा॒ यदिडा॒म्पूर्वां᳚ द्विदेव॒त्ये᳚भ्य उप॒ह्वये॑त॥३९॥

%6.4.9.4
प॒शुभिः॑ प्रा॒णान॒न्तर्द॑धीत प्र॒मायु॑कः स्याद्द्विदेव॒त्या᳚न्भक्षयि॒त्वेडा॒मुप॑ ह्वयते प्रा॒णाने॒वात्मन्धि॒त्वा प॒शूनुप॑ ह्वयते॒ वाग्वा ऐ᳚न्द्रवाय॒वश्चक्षु॑र्मैत्रावरु॒णः श्रोत्र॑माश्वि॒नः पु॒रस्ता॑दैन्द्रवाय॒वम्भ॑क्षयति॒ तस्मा᳚त्पु॒रस्ता᳚द्वा॒चा व॑दति पु॒रस्ता᳚न्मैत्रावरु॒णं तस्मा᳚त्पु॒रस्ता॒च्चक्षु॑षा पश्यति स॒र्वतः॑ परि॒हार॑माश्वि॒नं तस्मा᳚थ्स॒र्वतः॒ श्रोत्रे॑ण शृणोति प्रा॒णा वा ए॒ते यद्द्वि॑देव॒त्याः᳚॥४०॥

%6.4.9.5
अरि॑क्तानि॒ पात्रा॑णि सादयति॒ तस्मा॒दरि॑क्ता अन्तर॒तः प्रा॒णा यतः॒ खलु॒ वै य॒ज्ञस्य॒ वित॑तस्य॒ न क्रि॒यते॒ तदनु॑ य॒ज्ञꣳ रक्षा॒ꣳ॒स्यव॑ चरन्ति॒ यदरि॑क्तानि॒ पात्रा॑णि सा॒दय॑ति क्रि॒यमा॑णमे॒व तद्य॒ज्ञस्य॑ शये॒ रक्ष॑सा॒मन॑न्ववचाराय॒ दक्षि॑णस्य हवि॒र्धान॒स्योत्त॑रस्यां वर्त॒न्याꣳ सा॑दयति वा॒च्ये॑व वाचं॑ दधा॒त्या तृ॑तीयसव॒नात्परि॑ शेरे य॒ज्ञस्य॒ सन्त॑त्यै॥४१॥

%6.4.10.0
{\anuvakamend[{म॒नु॒ष्य॒च॒रावु॑दपा॒त्रमु॑प॒ह्वये॑त द्विदेव॒त्याः᳚ षट्च॑त्वारिꣳशच्च}]}%॥९॥

%6.4.10.1
बृह॒स्पति॑र्दे॒वानां᳚ पु॒रोहि॑त॒ आसी॒च्छण्डा॒मर्का॒वसु॑राणां॒ ब्रह्म॑ण्वन्तो दे॒वा आस॒न्ब्रह्म॑ण्व॒न्तो\-ऽसु॑रा॒स्ते \-ऽन्यो᳚न्यं नाश॑क्नुवन्न॒भिभ॑वितु॒न्ते दे॒वाः शण्डा॒मर्का॒वुपा॑मन्त्रयन्त॒ ताव॑ब्रूतां॒ वरं॑ वृणावहै॒ ग्रहा॑वे॒व ना॒वत्रापि॑ गृह्येता॒मिति॒ ताभ्या॑मे॒तौ शु॒क्राम॒न्थिना॑वगृह्ण॒न्ततो॑ दे॒वा अभ॑व॒न्परासु॑रा॒ यस्यै॒वं वि॒दुषः॑ शु॒क्राम॒न्थिनौ॑ गृ॒ह्येते॒ भव॑त्या॒त्मना॒ परा᳚॥४२॥

%6.4.10.2
अ॒स्य॒ भ्रातृ॑व्यो भवति॒ तौ दे॒वा अ॑प॒नुद्या॒त्मन॒ इन्द्रा॑याजुहवु॒रप॑नुत्तौ॒ शण्डा॒मर्कौ॑ स॒हामुनेति॑ ब्रूया॒द्यं द्वि॒ष्याद्यमे॒व द्वेष्टि॒ तेनै॑नौ स॒हाप॑ नुदते॒ स प्र॑थ॒मः सङ्कृ॑तिर्वि॒श्वक॒र्मेत्ये॒वैना॑वा॒त्मन॒ इन्द्रा॑याजुहवु॒रिन्द्रो॒ ह्ये॑तानि॑ रू॒पाणि॒ करि॑क्र॒दच॑रद॒सौ वा आ॑दि॒त्यः शु॒क्रश्च॒न्द्रमा॑ म॒न्थ्य॑पि॒गृह्य॒ प्राञ्चौ॒ निः॥४३॥

%6.4.10.3
क्रा॒म॒त॒स्तस्मा॒त्प्राञ्चौ॒ यन्तौ॒ न प॑श्यन्ति प्र॒त्यञ्चा॑वा॒वृत्य॑ जुहुत॒स्तस्मा᳚त्प्र॒त्यञ्चौ॒ यन्तौ॑ पश्यन्ति॒ चक्षु॑षी॒ वा ए॒ते य॒ज्ञस्य॒ यच्छु॒क्राम॒न्थिनौ॒ नासि॑कोत्तरवे॒दिर॒भितः॑ परि॒क्रम्य॑ जुहुत॒स्तस्मा॑द॒भितो॒ नासि॑कां॒ चक्षु॑षी॒ तस्मा॒न्नासि॑कया॒ चक्षु॑षी॒ विधृ॑ते स॒र्वतः॒ परि॑ क्रामतो॒ रक्ष॑सा॒मप॑हत्यै दे॒वा वै याः प्राची॒राहु॑ती॒रजु॑हवु॒र्ये पु॒रस्ता॒दसु॑रा॒ आस॒न्ताꣴ स्ताभिः॒ प्र॥४४॥

%6.4.10.4
अ॒नु॒द॒न्त॒ याः प्र॒तीची॒र्ये प॒श्चादसु॑रा॒ आस॒न्ताꣴस्ताभि॒रपा॑नुदन्त॒ प्राची॑र॒न्या आहु॑तयो हू॒यन्ते᳚ प्र॒त्यञ्चौ॑ शु॒क्राम॒न्थिनौ॑ प॒श्चाच्चै॒व पु॒रस्ता᳚च्च॒ यज॑मानो॒ भ्रातृ॑व्या॒न्प्र णु॑दते॒ तस्मा॒त्परा॑चीः प्र॒जाः प्र वी॑यन्ते प्र॒तीची᳚र्जायन्ते शु॒क्राम॒न्थिनौ॒ वा अनु॑ प्र॒जाः प्र जा॑यन्ते॒\-ऽत्त्रीश्चा॒द्या᳚श्च सु॒वीराः᳚ प्र॒जाः प्र॑ज॒नय॒न्परी॑हि शु॒क्रः शु॒क्रशो॑चिषा॥४५॥

%6.4.10.5
सु॒प्र॒जाः प्र॒जाः प्र॑ज॒नय॒न्परी॑हि म॒न्थी म॒न्थिशो॑चि॒षेत्या॑है॒ता वै सु॒वीरा॒ या अ॒त्त्रीरे॒ताः सु॑प्र॒जा या आ॒द्या॑ य ए॒वं वेदा॒त्त्र्य॑स्य प्र॒जा जा॑यते॒ नाद्या᳚ प्र॒जाप॑ते॒रक्ष्य॑श्वय॒त्तत्परा॑पत॒त्तद्विक॑ङ्कत॒म्प्रावि॑श॒त्तद्विक॑ङ्कते॒ नार॑मत॒ तद्यव॒म्प्रावि॑श॒त् तद्यवे॑\-ऽरमत॒ तद्यव॑स्य॥४६॥

%6.4.10.6
य॒व॒त्वं यद्वैक॑ङ्कतम्मन्थिपा॒त्रम्भव॑ति॒ सक्तु॑भिः श्री॒णाति॑ प्र॒जाप॑तेरे॒व तच्चक्षुः॒ सम्भ॑रति ब्रह्मवा॒दिनो॑ वदन्ति॒ कस्मा᳚थ्स॒त्यान्म॑न्थिपा॒त्रꣳ सदो॒ नाश्नु॑त॒ इत्या᳚र्तपा॒त्रꣳ हीति॑ ब्रूया॒द्यद॑श्नुवी॒तान्धो᳚\-ऽध्व॒र्युः स्या॒दार्ति॒मार्च्छे॒त्तस्मा॒न्नाश्नु॑ते॥४७॥

%6.4.11.0
{\anuvakamend[{आ॒त्मना॒ परा॒ निष्प्र शु॒क्रशो॑चिषा॒ यव॑स्य स॒प्तत्रिꣳ॑शच्च}]}%॥10॥

%6.4.11.1
दे॒वा वै यद्य॒ज्ञे\-ऽकु॑र्वत॒ तदसु॑रा अकुर्वत॒ ते दे॒वा आ᳚ग्रय॒णाग्रा॒न्ग्रहा॑नपश्य॒न्तान॑गृह्णत॒ ततो॒ वै ते\-ऽग्रं॒ पर्या॑य॒न्॒ यस्यै॒वं वि॒दुष॑ आग्रय॒णाग्रा॒ ग्रहा॑ गृ॒ह्यन्ते\-ऽग्र॑मे॒व स॑मा॒नानां॒ पर्ये॑ति रु॒ग्णव॑त्य॒र्चा भ्रातृ॑व्यवतो गृह्णीया॒द्भ्रातृ॑व्यस्यै॒व रु॒क्त्वाग्रꣳ॑ समा॒नानां॒ पर्ये॑ति॒ ये दे॑वा दि॒व्येका॑दश॒ स्थेत्या॑ह॥४८॥

%6.4.11.2
ए॒ताव॑ती॒र्वै दे॒वता॒स्ताभ्य॑ ए॒वैन॒ꣳ॒ सर्वा᳚भ्यो गृह्णात्ये॒ष ते॒ योनि॒र्विश्वे᳚भ्यस्त्वा दे॒वेभ्य॒ इत्या॑ह वैश्वदे॒वो ह्ये॑ष दे॒वत॑या॒ वाग्वै दे॒वेभ्यो\-ऽपा᳚क्रामद्य॒ज्ञायाति॑ष्ठमाना॒ ते दे॒वा वा॒च्यप॑क्रान्तायां तू॒ष्णीं ग्रहा॑नगृह्णत॒ सा\-ऽम॑न्यत॒ वाग॒न्तर्य॑न्ति॒ वै मेति॒ साग्र॑य॒णम्प्रत्याग॑च्छ॒त्तदा᳚ग्रय॒णस्या᳚ग्रयण॒त्वम्॥४९॥

%6.4.11.3
तस्मा॑दाग्रय॒णे वाग्वि सृ॑ज्यते॒ यत्तू॒ष्णीम्पूर्वे॒ ग्रहा॑ गृ॒ह्यन्ते॒ यथा᳚ थ्सा॒रीय॑ति म॒ आख॒ इय॑ति॒ नाप॑ राथ्स्या॒मीत्यु॑पावसृ॒जत्ये॒वमे॒व तद॑ध्व॒र्युरा᳚ग्रय॒णं गृ॑ही॒त्वा य॒ज्ञमा॒रभ्य॒ वाचं॒ वि सृ॑जते॒ त्रिर्\mbox{}हिं क॑रोत्युद्गा॒तॄने॒व तद्वृ॑णीते प्र॒जाप॑ति॒र्वा ए॒ष यदा᳚ग्रय॒णो यदा᳚ग्रय॒णं गृ॑ही॒त्वा हिं॑क॒रोति॑ प्र॒जाप॑तिरे॒व॥५०॥

%6.4.11.4
तत्प्र॒जा अ॒भि जि॑घ्रति॒ तस्मा᳚द्व॒थ्सं जा॒तं गौर॒भि जि॑घ्रत्या॒त्मा वा ए॒ष य॒ज्ञस्य॒ यदा᳚ग्रय॒णः सव॑नेसवने॒\-ऽभि गृ॑ह्णात्या॒त्मन्ने॒व य॒ज्ञꣳ सं त॑नोत्यु॒परि॑ष्टा॒दा न॑यति॒ रेत॑ ए॒व तद्द॑धात्य॒धस्ता॒दुप॑ गृह्णाति॒ प्र ज॑नयत्ये॒व तद्ब्र॑ह्मवा॒दिनो॑ वदन्ति॒ कस्मा᳚थ्स॒त्याद्गा॑य॒त्री कनि॑ष्ठा॒ छन्द॑साꣳ स॒ती सर्वा॑णि॒ सव॑नानि वह॒तीत्ये॒ष वै गा॑यत्रि॒यै व॒थ्सो यदा᳚ग्रय॒णस्तमे॒व तद॑भिनि॒वर्त॒ꣳ॒ सर्वा॑णि॒ सव॑नानि वहति॒ तस्मा᳚द्व॒थ्सम॒पाकृ॑तं॒ गौर॒भि नि व॑र्तते॥५१॥

%6.5.0.0
{\anuvakamend[{आ॒हा॒ग्र॒य॒ण॒त्वं प्र॒जाप॑तिरे॒वेति॑ विꣳश॒तिश्च॑}]}%॥11॥

%6.5.0.0

{\anuvakamend[{इन्द्रो॑ वृ॒त्रायायु॒र्वै य॒ज्ञेन॑ सुव॒र्गायेन्द्रो॑ म॒रुद्भि॒रदि॑तिरन्तर्यामपा॒त्रेण॑ प्रा॒ण उ॑पाꣳशुपा॒त्रेणेन्द्रो॑ वृ॒त्रम॑ह॒न्तस्य॒ ग्रहा॒न्॒ वै प्रान्यान्येका॑दश}]}%॥11॥ 
\prashnaend{इन्द्रो॑ वृ॒त्राय॒ पुन॑र्\mbox{}ऋ॒तुना॑ह मिथु॒नम्प॒शवो॒ नेष्टः॒ पत्नी॑मुपाꣳश्वन्तर्या॒मयो॒र्द्विच॑त्वारिꣳशत्॥42॥ इन्द्रो॑ वृ॒त्राय॑ पाङ्क्त॒त्वम्॥}
%%% END PRASHNA

\sect{पञ्चमः प्रश्नः}\setcounter{anuvakam}{0}
\dnsub{तैत्तिरीयसंहितायां षष्ठमकाण्डे पञ्चमः प्रश्नः}
%6.5.1.0
%6.5.1.1
इन्द्रो॑ वृ॒त्राय॒ वज्र॒मुद॑यच्छ॒थ्स वृ॒त्रो वज्रा॒दुद्य॑तादबिभे॒थ्सो᳚\-ऽब्रवी॒न्मा मे॒ प्र हा॒रस्ति॒ वा इ॒दम्मयि॑ वी॒र्यं॑ तत्ते॒ प्र दा᳚स्या॒मीति॒ तस्मा॑ उ॒क्थ्य॑म्प्राय॑च्छ॒त्तस्मै᳚ द्वि॒तीय॒मुद॑यच्छ॒थ्सो᳚\-ऽब्रवी॒न्मा मे॒ प्र हा॒रस्ति॒ वा इ॒दं मयि॑ वी॒र्यं॑ तत्ते॒ प्र दा᳚स्या॒मीति॑॥१॥

%6.5.1.2
तस्मा॑ उ॒क्थ्य॑मे॒व प्राय॑च्छ॒त्तस्मै॑ तृ॒तीय॒मुद॑यच्छ॒त्तं विष्णु॒रन्व॑तिष्ठत ज॒हीति॒ सो᳚\-ऽब्रवी॒न्मा मे॒ प्र हा॒रस्ति॒ वा इ॒दम्मयि॑ वी॒र्यं॑ तत्ते॒ प्र दा᳚स्या॒मीति॒ तस्मा॑ उ॒क्थ्य॑मे॒व प्राय॑च्छ॒त्तं निर्मा॑यम्भू॒तम॑हन् य॒ज्ञो हि तस्य॑ मा॒यासी॒द्यदु॒क्थ्यो॑ गृ॒ह्यत॑ इन्द्रि॒यमे॒व॥२॥

%6.5.1.3
तद्वी॒र्यं॑ यज॑मानो॒ भ्रातृ॑व्यस्य वृङ्क्त॒ इन्द्रा॑य त्वा बृ॒हद्व॑ते॒ वय॑स्वत॒ इत्या॒हेन्द्रा॑य॒ हि स तम्प्राय॑च्छ॒त्तस्मै᳚ त्वा॒ विष्ण॑वे॒ त्वेत्या॑ह॒ यदे॒व विष्णु॑र॒न्वति॑ष्ठत ज॒हीति॒ तस्मा॒द्विष्णु॑म॒न्वाभ॑जति॒ त्रिर्निर्गृ॑ह्णाति॒ त्रिर्\mbox{}हि स तं तस्मै॒ प्राय॑च्छदे॒ष ते॒ योनिः॒ पुन॑र्\mbox{}हविर॒सीत्या॑ह॒ पुनः॑पुनः॥३॥

%6.5.1.4
ह्य॑स्मान्निर्गृ॒ह्णाति॒ चक्षु॒र्वा ए॒तद्य॒ज्ञस्य॒ यदु॒क्थ्य॑स्तस्मा॑दु॒क्थ्यꣳ॑ हु॒तꣳ सोमा॑ अ॒न्वाय॑न्ति॒ तस्मा॑दा॒त्मा चक्षु॒रन्वे॑ति॒ तस्मा॒देकं॒ यन्त॑म्ब॒हवो\-ऽनु॑ यन्ति॒ तस्मा॒देको॑ बहू॒नाम्भ॒द्रो भ॑वति॒ तस्मा॒देको॑ ब॒ह्वीर्जा॒या वि॑न्दते॒ यदि॑ का॒मये॑ताध्व॒र्युरा॒त्मानं॑ यज्ञयश॒सेना᳚र्पयेय॒मित्य॑न्त॒राह॑व॒नीयं॑ च हवि॒र्धानं॑ च॒ तिष्ठ॒न्नव॑ नयेत्॥४॥

%6.5.1.5
आ॒त्मान॑मे॒व य॑ज्ञयश॒सेना᳚र्पयति॒ यदि॑ का॒मये॑त॒ यज॑मानं यज्ञयश॒सेना᳚र्पयेय॒मित्य॑न्त॒रा स॑दोहविर्धा॒ने तिष्ठ॒न्नव॑ नये॒द्यज॑मानमे॒व य॑ज्ञयश॒सेना᳚र्पयति॒ यदि॑ का॒मये॑त सद॒स्यान्॑ यज्ञयश॒सेना᳚र्पयेय॒मिति॒ सद॑ आ॒लभ्याव॑ नयेथ्सद॒स्या॑ने॒व य॑ज्ञयश॒सेना᳚र्पयति॥५॥

%6.5.2.0
{\anuvakamend[{इती᳚न्द्रि॒यमे॒व पुनः॑पुनर्नये॒त्त्रय॑स्त्रिꣳशच्च}]}%॥१॥

%6.5.2.1
आयु॒र्वा ए॒तद्य॒ज्ञस्य॒ यद्ध्रु॒व उ॑त्त॒मो ग्रहा॑णां गृह्यते॒ तस्मा॒दायुः॑ प्रा॒णाना॑मुत्त॒मम्मू॒र्धानं॑ दि॒वो अ॑र॒तिं पृ॑थि॒व्या इत्या॑ह मू॒र्धान॑मे॒वैनꣳ॑ समा॒नानां᳚ करोति वैश्वान॒रमृ॒ताय॑ जा॒तम॒ग्निमित्या॑ह वैश्वान॒रꣳ हि दे॒वत॒यायु॑रुभ॒यतो॑वैश्वानरो गृह्यते॒ तस्मा॑दुभ॒यतः॑ प्रा॒णा अ॒धस्ता᳚च्चो॒परि॑ष्टाच्चा॒र्धिनो॒\-ऽन्ये ग्रहा॑ गृ॒ह्यन्ते॒\-ऽर्धी ध्रु॒वस्तस्मा᳚त्॥६॥

%6.5.2.2
अ॒र्ध्यवा᳚ङ्प्रा॒णो᳚\-ऽन्येषां᳚ प्रा॒णाना॒मुपो᳚प्ते॒\-ऽन्ये ग्रहाः᳚ सा॒द्यन्ते\-ऽनु॑पोप्ते ध्रु॒वस्तस्मा॑द॒स्थ्नान्याः प्र॒जाः प्र॑ति॒तिष्ठ॑न्ति मा॒ꣳ॒सेना॒न्या असु॑रा॒ वा उ॑त्तर॒तः पृ॑थि॒वीम्प॒र्याचि॑कीर्\mbox{}ष॒न्तां दे॒वा ध्रु॒वेणा॑दृꣳह॒न्तद्ध्रु॒वस्य॑ ध्रुव॒त्वं यद्ध्रु॒व उ॑त्तर॒तः सा॒द्यते॒ धृत्या॒ आयु॒र्वा ए॒तद्य॒ज्ञस्य॒ यद्ध्रु॒व आ॒त्मा होता॒ यद्धो॑तृचम॒से ध्रु॒वम॑व॒नय॑त्या॒त्मन्ने॒व य॒ज्ञस्य॑॥७॥

%6.5.2.3
आयु॑र्दधाति पु॒रस्ता॑दु॒क्थस्या॑व॒नीय॒ इत्या॑हुः पु॒रस्ता॒द्ध्यायु॑षो भु॒ङ्क्ते म॑ध्य॒तो॑\-ऽव॒नीय॒ इत्या॑हुर्मध्य॒मेन॒ ह्यायु॑षो भु॒ङ्क्त उ॑त्तरा॒र्धे॑\-ऽव॒नीय॒ इत्या॑हुरुत्त॒मेन॒ ह्यायु॑षो भु॒ङ्क्ते वै᳚श्वदे॒व्यामृ॒चि श॒स्यमा॑नाया॒मव॑ नयति वैश्वदे॒व्यो॑ वै प्र॒जाः प्र॒जास्वे॒वायु॑र्दधाति॥८॥

%6.5.3.0
{\anuvakamend[{ध्रु॒वस्तस्मा॑दे॒व य॒ज्ञस्यैका॒न्नच॑त्वारि॒ꣳ॒शच्च॑}]}%॥२॥

%6.5.3.1
य॒ज्ञेन॒ वै दे॒वाः सु॑व॒र्गं लो॒कमा॑य॒न्ते॑\-ऽमन्यन्त मनु॒ष्या॑ नो॒\-ऽन्वाभ॑विष्य॒न्तीति॒ ते सं॑वथ्स॒रेण॑ योपयि॒त्वा सु॑व॒र्गं लो॒कमा॑य॒न्तमृष॑य ऋतुग्र॒हैरे॒वानु॒ प्राजा॑न॒न्॒यदृ॑तुग्र॒हा गृ॒ह्यन्ते॑ सुव॒र्गस्य॑ लो॒कस्य॒ प्रज्ञा᳚त्यै॒ द्वाद॑श गृह्यन्ते॒ द्वाद॑श॒ मासाः᳚ संवथ्स॒रः सं॑वथ्स॒रस्य॒ प्रज्ञा᳚त्यै स॒ह प्र॑थ॒मौ गृ॑ह्येते स॒होत्त॒मौ तस्मा॒द्द्वौद्वा॑वृ॒तू उ॑भ॒यतो॑मुखमृतुपा॒त्रम्भ॑वति॒ कः॥९॥

%6.5.3.2
हि तद्वेद॒ यत॑ ऋतू॒नाम्मुख॑मृ॒तुना॒ प्रेष्येति॒ षट्कृत्व॑ आह॒ षड्वा ऋ॒तव॑ ऋ॒तूने॒व प्री॑णात्यृ॒तुभि॒रिति॑ च॒तुश्चतु॑ष्पद ए॒व प॒शून्प्री॑णाति॒ द्विः पुन॑र्\mbox{}ऋ॒तुना॑ह द्वि॒पद॑ ए॒व प्री॑णात्यृ॒तुना॒ प्रेष्येति॒ षट्कृत्व॑ आह॒र्तुभि॒रिति॑ च॒तुस्तस्मा॒च्चतु॑ष्पादः प॒शव॑ ऋ॒तूनुप॑ जीवन्ति॒ द्विः॥१०॥

%6.5.3.3
पुन॑र्\mbox{}ऋ॒तुना॑ह॒ तस्मा᳚द्द्वि॒पाद॒श्चतु॑ष्पदः प॒शूनुप॑ जीवन्त्यृ॒तुना॒ प्रेष्येति॒ षट्कृत्व॑ आह॒र्तुभि॒रिति॑ च॒तुर्द्विः पुन॑र्\mbox{}ऋ॒तुना॑हा॒क्रम॑णमे॒व तथ्सेतुं॒ यज॑मानः कुरुते सुव॒र्गस्य॑ लो॒कस्य॒ सम॑ष्ट्यै॒ नान्यो᳚न्यमनु॒ प्र प॑द्येत॒ यद॒न्यो᳚\-ऽन्यम॑नु प्र॒पद्ये॑त॒र्तुर्\mbox{}ऋ॒तुमनु॒ प्र प॑द्येत॒र्तवो॒ मोहु॑काः स्युः॥११॥

%6.5.3.4
प्रसि॑द्धमे॒वाध्व॒र्युर्दक्षि॑णेन॒ प्र प॑द्यते॒ प्रसि॑द्धं प्रतिप्रस्था॒तोत्त॑रेण॒ तस्मा॑दादि॒त्यः षण्मा॒सो दक्षि॑णेनैति॒ षडुत्त॑रेणोपया॒मगृ॑हीतो\-ऽसि स॒ꣳ॒सर्पो᳚\-ऽस्यꣳहस्प॒त्याय॒ त्वेत्या॒हास्ति॑ त्रयोद॒शो मास॒ इत्या॑हु॒स्तमे॒व तत्प्री॑णाति॥१२॥

%6.5.4.0
{\anuvakamend[{को जी॑वन्ति॒ द्विः स्यु॒श्चतु॑स्त्रिꣳशच्च}]}%॥३॥

%6.5.4.1
सु॒व॒र्गाय॒ वा ए॒ते लो॒काय॑ गृह्यन्ते॒ यदृ॑तुग्र॒हा ज्योति॑रिन्द्रा॒ग्नी यदै᳚न्द्रा॒ग्नमृ॑तुपा॒त्रेण॑ गृ॒ह्णाति॒ ज्योति॑रे॒वास्मा॑ उ॒परि॑ष्टाद्दधाति सुव॒र्गस्य॑ लो॒कस्यानु॑ख्यात्या ओजो॒भृतौ॒ वा ए॒तौ दे॒वानां॒ यदि॑न्द्रा॒ग्नी यदै᳚न्द्रा॒ग्नो गृ॒ह्यत॒ ओज॑ ए॒वाव॑ रुन्द्धे वैश्वदे॒वꣳ शु॑क्रपा॒त्रेण॑ गृह्णाति वैश्वदे॒व्यो॑ वै प्र॒जा अ॒सावा॑दि॒त्यः शु॒क्रो यद्वै᳚श्वदे॒वꣳ शु॑क्रपा॒त्रेण॑ गृ॒ह्णाति॒ तस्मा॑द॒सावा॑दि॒त्यः॥१३॥

%6.5.4.2
सर्वाः᳚ प्र॒जाः प्र॒त्यङ्ङुदे॑ति॒ तस्मा॒थ्सर्व॑ ए॒व म॑न्यते॒ माम्प्रत्युद॑गा॒दिति॑ वैश्वदे॒वꣳ शु॑क्रपा॒त्रेण॑ गृह्णाति वैश्वदे॒व्यो॑ वै प्र॒जास्तेजः॑ शु॒क्रो यद्वै᳚श्वदे॒वꣳ शु॑क्रपा॒त्रेण॑ गृ॒ह्णाति॑ प्र॒जास्वे॒व तेजो॑ दधाति॥१४॥

%6.5.5.0
{\anuvakamend[{तस्मा॑द॒सावा॑दि॒त्यस्त्रि॒ꣳ॒शच्च॑}]}%॥४॥

%6.5.5.1
इन्द्रो॑ म॒रुद्भिः॒ सांवि॑द्येन॒ माध्यं॑दिने॒ सव॑ने वृ॒त्रम॑ह॒न्॒यन्माध्यं॑दिने॒ सव॑ने मरुत्व॒तीया॑ गृ॒ह्यन्ते॒ वार्त्र॑घ्ना ए॒व ते यज॑मानस्य गृह्यन्ते॒ तस्य॑ वृ॒त्रं ज॒घ्नुष॑ ऋ॒तवो॑\-ऽमुह्य॒न्थ्स ऋ॑तुपा॒त्रेण॑ मरुत्व॒तीया॑नगृह्णा॒त्ततो॒ वै स ऋ॒तून्प्राजा॑ना॒द्यदृ॑तुपा॒त्रेण॑ मरुत्व॒तीया॑ गृ॒ह्यन्त॑ ऋतू॒नाम्प्रज्ञा᳚त्यै॒ वज्रं॒ वा ए॒तं यज॑मानो॒ भ्रातृ॑व्याय॒ प्र ह॑रति॒ यन्म॑रुत्व॒तीया॒ उदे॒व प्र॑थ॒मेन॑॥१५॥

%6.5.5.2
य॒च्छ॒ति॒ प्र ह॑रति द्वि॒तीये॑न स्तृणु॒ते तृ॒तीये॒नायु॑धं॒ वा ए॒तद्यज॑मानः॒ सꣴस्कु॑रुते॒ यन्म॑रुत्व॒तीया॒ धनु॑रे॒व प्र॑थ॒मो ज्या द्वि॒तीय॒ इषु॑स्तृ॒तीयः॒ प्रत्ये॒व प्र॑थ॒मेन॑ धत्ते॒ वि सृ॑जति द्वि॒तीये॑न॒ विध्य॑ति तृ॒तीये॒नेन्द्रो॑ वृ॒त्रꣳ ह॒त्वा परां᳚ परा॒वत॑मगच्छ॒दपा॑राध॒मिति॒ मन्य॑मानः॒ स हरि॑तो\-ऽभव॒थ्स ए॒तान्म॑रुत्व॒तीया॑नात्म॒स्पर॑णानपश्य॒त्तान॑गृह्णीत॥१६॥

%6.5.5.3
प्रा॒णमे॒व प्र॑थ॒मेना᳚स्पृणुतापा॒नं द्वि॒तीये॑ना॒त्मानं॑ तृ॒तीये॑नात्म॒स्पर॑णा॒ वा ए॒ते यज॑मानस्य गृह्यन्ते॒ यन्म॑रुत्व॒तीयाः᳚ प्रा॒णमे॒व प्र॑थ॒मेन॑ स्पृणुते\-ऽपा॒नं द्वि॒तीये॑ना॒त्मानं॑ तृ॒तीये॒नेन्द्रो॑ वृ॒त्रम॑ह॒न्तं दे॒वा अ॑ब्रुवन्म॒हान् वा अ॒यम॑भू॒द्यो वृ॒त्रमव॑धी॒दिति॒ तन्म॑हे॒न्द्रस्य॑ महेन्द्र॒त्वꣳ स ए॒तम्मा॑हे॒न्द्रमु॑द्धा॒रमुद॑हरत वृ॒त्रꣳ ह॒त्वान्यासु॑ दे॒वता॒स्वधि॒ यन्मा॑हे॒न्द्रो गृ॒ह्यत॑ उद्धा॒रमे॒व तं यज॑मान॒ उद्ध॑रते॒\-ऽन्यासु॑ प्र॒जास्वधि॑ शुक्रपा॒त्रेण॑ गृह्णाति यजमानदेव॒त्यो॑ वै मा॑हे॒न्द्रस्तेजः॑ शु॒क्रो यन्मा॑हे॒न्द्रꣳ शु॑क्रपा॒त्रेण॑ गृ॒ह्णाति॒ यज॑मान ए॒व तेजो॑ दधाति॥१७॥

%6.5.6.0
{\anuvakamend[{प्र॒थ॒मेना॑गृह्णीत दे॒वता᳚स्व॒ष्टाविꣳ॑शतिश्च}]}%॥५॥

%6.5.6.1
अदि॑तिः पु॒त्रका॑मा सा॒ध्येभ्यो॑ दे॒वेभ्यो᳚ ब्रह्मौद॒नम॑पच॒त्तस्या॑ उ॒च्छेष॑णमददु॒स्तत्प्राश्ना॒थ्सा रेतो॑\-ऽधत्त॒ तस्यै॑ च॒त्वार॑ आदि॒त्या अ॑जायन्त॒ सा द्वि॒तीय॑मपच॒थ्साम॑न्यतो॒च्छेष॑णान्म इ॒मे᳚\-ऽज्ञत॒ यदग्रे᳚ प्राशि॒ष्यामी॒तो मे॒ वसी॑याꣳसो जनिष्यन्त॒ इति॒ साग्रे॒ प्राश्ना॒थ्सा रेतो॑\-ऽधत्त॒ तस्यै॒ व्यृ॑द्धमा॒ण्डम॑जायत॒ सादि॒त्येभ्य॑ ए॒व॥१८॥

%6.5.6.2
तृ॒तीय॑मपच॒द्भोगा॑य म इ॒दꣴ श्रा॒न्तम॒स्त्विति॒ ते᳚\-ऽब्रुव॒न्वरं॑ वृणामहै॒ यो\-ऽतो॒ जाया॑ता अ॒स्माक॒ꣳ॒ स एको॑\-ऽस॒द्यो᳚\-ऽस्य प्र॒जाया॒मृध्या॑ता अ॒स्माक॒म्भोगा॑य भवा॒दिति॒ ततो॒ विव॑स्वानादि॒त्यो॑\-ऽजायत॒ तस्य॒ वा इ॒यं प्र॒जा यन्म॑नु॒ष्या᳚स्तास्वेक॑ ए॒वर्द्धो यो यज॑ते॒ स दे॒वाना॒म्भोगा॑य भवति दे॒वा वै य॒ज्ञात्॥१९॥

%6.5.6.3
रु॒द्रम॒न्तरा॑य॒न्थ्स आ॑दि॒त्यान॒न्वाक्र॑मत॒ ते द्वि॑देव॒त्या᳚न्प्राप॑द्यन्त॒ तान्न प्रति॒ प्राय॑च्छ॒न्तस्मा॒दपि॒ वध्य॒म्प्रप॑न्नं॒ न प्रति॒ प्र य॑च्छन्ति॒ तस्मा᳚द्द्विदेव॒त्ये᳚भ्य आदि॒त्यो निर्गृ॑ह्यते॒ यदु॒च्छेष॑णा॒दजा॑यन्त॒ तस्मा॑दु॒च्छेष॑णाद्गृह्यते ति॒सृभि॑र्\mbox{}ऋ॒ग्भिर्गृ॑ह्णाति मा॒ता पि॒ता पु॒त्रस्तदे॒व तन्मि॑थु॒नमुल्बं॒ गर्भो॑ ज॒रायु॒ तदे॒व तत्॥२०॥

%6.5.6.4
मि॒थु॒नम्प॒शवो॒ वा ए॒ते यदा॑दि॒त्य ऊर्ग्दधि॑ द॒ध्ना म॑ध्य॒तः श्री॑णा॒त्यूर्ज॑मे॒व प॑शू॒नाम्म॑ध्य॒तो द॑धाति शृतात॒ङ्क्ये॑न मेध्य॒त्वाय॒ तस्मा॑दा॒मा प॒क्वं दु॑हे प॒शवो॒ वा ए॒ते यदा॑दि॒त्यः प॑रि॒श्रित्य॑ गृह्णाति प्रति॒रुध्यै॒वास्मै॑ प॒शून्गृ॑ह्णाति प॒शवो॒ वा ए॒ते यदा॑दि॒त्य ए॒ष रु॒द्रो यद॒ग्निः प॑रि॒श्रित्य॑ गृह्णाति रु॒द्रादे॒व प॒शून॒न्तर्द॑धाति॥२१॥

%6.5.6.5
ए॒ष वै विव॑स्वानादि॒त्यो यदु॑पाꣳशु॒सव॑नः॒ स ए॒तमे॒व सो॑मपी॒थं परि॑ शय॒ आ तृ॑तीयसव॒नाद्विव॑स्व आदित्यै॒ष ते॑ सोमपी॒थ इत्या॑ह॒ विव॑स्वन्तमे॒वादि॒त्यꣳ सो॑मपी॒थेन॒ सम॑र्धयति॒ या दि॒व्या वृष्टि॒स्तया᳚ त्वा श्रीणा॒मीति॒ वृष्टि॑कामस्य श्रीणीया॒द्वृष्टि॑मे॒वाव॑ रुन्द्धे॒ यदि॑ ता॒जक्प्र॒स्कन्दे॒द्वर्\mbox{}षु॑कः प॒र्जन्यः॑ स्या॒द्यदि॑ चि॒रमव॑र्\mbox{}षुको॒ न सा॑दय॒त्यस॑न्ना॒द्धि प्र॒जाः प्र॒जाय॑न्ते॒ नानु॒ वष॑ट्करोति॒ यद॑नुवषट्कु॒॒र्याद्रु॒द्रं प्र॒जा अ॒न्वव॑सृजे॒न्न हु॒त्वान्वी᳚क्षेत॒ यद॒न्वीक्षे॑त॒ चक्षु॑रस्य प्र॒मायु॑कꣴ स्या॒त्तस्मा॒न्नान्वीक्ष्यः॑॥२२॥

%6.5.7.0
{\anuvakamend[{ए॒व य॒ज्ञाज्ज॒रायु॒ तदे॒व तद॒न्तर्द॑धाति॒ न स॒प्तविꣳ॑शतिश्च}]}%॥६॥

%6.5.7.1
अ॒न्त॒र्या॒म॒पा॒त्रेण॑ सावि॒त्रमा᳚ग्रय॒णाद्गृ॑ह्णाति प्र॒जाप॑ति॒र्वा ए॒ष यदा᳚ग्रय॒णः प्र॒जानां᳚ प्र॒जन॑नाय॒ न सा॑दय॒त्यस॑न्ना॒द्धि प्र॒जाः प्र॒जाय॑न्ते॒ नानु॒ वष॑ट्करोति॒ यद॑नुवषट्कु॒र्याद्रु॒द्रं प्र॒जा अ॒न्वव॑सृजेदे॒ष वै गा॑य॒त्रो दे॒वानां॒ यथ्स॑वि॒तैष गा॑यत्रि॒यै लो॒के गृ॑ह्यते॒ यदा᳚ग्रय॒णो यद॑न्तर्यामपा॒त्रेण॑ सावि॒त्रमा᳚ग्रय॒णाद्गृ॒ह्णाति॒ स्वादे॒वैनं॒ योने॒र्निर्गृ॑ह्णाति॒ विश्वे᳚॥२३॥

%6.5.7.2
दे॒वास्तृ॒तीय॒ꣳ॒ सव॑नं॒ नोद॑यच्छ॒न्ते स॑वि॒तार॑म्प्रातःसव॒नभा॑ग॒ꣳ॒ सन्तं॑ तृतीयसव॒नम॒भि पर्य॑णय॒न्ततो॒ वै ते तृ॒तीय॒ꣳ॒ सव॑न॒मुद॑यच्छ॒न्॒यत्तृ॑तीयसव॒ने सा॑वि॒त्रो गृ॒ह्यते॑ तृ॒तीय॑स्य॒ सव॑न॒स्योद्य॑त्यै सवितृपा॒त्रेण॑ वैश्वदे॒वं क॒लशा᳚द्गृह्णाति वैश्वदे॒व्यो॑ वै प्र॒जा वै᳚श्वदे॒वः क॒लशः॑ सवि॒ता प्र॑स॒वाना॑मीशे॒ यथ्स॑वितृपा॒त्रेण॑ वैश्वदे॒वं क॒लशा᳚द्गृ॒ह्णाति॑ सवि॒तृप्र॑सूत ए॒वास्मै᳚ प्र॒जाः प्र॥२४॥

%6.5.7.3
ज॒न॒य॒ति॒ सोमे॒ सोम॑म॒भि गृ॑ह्णाति॒ रेत॑ ए॒व तद्द॑धाति सु॒शर्मा॑सि सुप्रतिष्ठा॒न इत्या॑ह॒ सोमे॒ हि सोम॑मभिगृ॒ह्णाति॒ प्रति॑ष्ठित्या ए॒तस्मि॒न्वा अपि॒ ग्रहे॑ मनु॒ष्ये᳚भ्यो दे॒वेभ्यः॑ पि॒तृभ्यः॑ क्रियते सु॒शर्मा॑सि सुप्रतिष्ठा॒न इत्या॑ह मनु॒ष्ये᳚भ्य ए॒वैतेन॑ करोति बृ॒हदित्या॑ह दे॒वेभ्य॑ ए॒वैतेन॑ करोति॒ नम॒ इत्या॑ह पि॒तृभ्य॑ ए॒वैतेन॑ करोत्ये॒ताव॑ती॒र्वै दे॒वता॒स्ताभ्य॑ ए॒वैन॒ꣳ॒ सर्वा᳚भ्यो गृह्णात्ये॒ष ते॒ योनि॒र्विश्वे᳚भ्यस्त्वा दे॒वेभ्य॒ इत्या॑ह वैश्वदे॒वो ह्ये॑षः॥२५॥

%6.5.8.0
{\anuvakamend[{विश्वे॒ प्र पि॒तृभ्य॑ ए॒वैतेन॑ करो॒त्येका॒न्नविꣳ॑श॒तिश्च॑}]}%॥७॥

%6.5.8.1
प्रा॒णो वा ए॒ष यदु॑पा॒ꣳ॒शुर्यदु॑पाꣳशुपा॒त्रेण॑ प्रथ॒मश्चो᳚त्त॒मश्च॒ ग्रहौ॑ गृ॒ह्येते᳚ प्रा॒णमे॒वानु॑ प्र॒यन्ति॑ प्रा॒णमनूद्य॑न्ति प्र॒जाप॑ति॒र्वा ए॒ष यदा᳚ग्रय॒णः प्रा॒ण उ॑पा॒ꣳ॒शुः पत्नीः᳚ प्र॒जाः प्र ज॑नयन्ति॒ यदु॑पाꣳशुपा॒त्रेण॑ पात्नीव॒तमा᳚ग्रय॒णाद्गृ॒ह्णाति॑ प्र॒जानां᳚ प्र॒जन॑नाय॒ तस्मा᳚त्प्रा॒णं प्र॒जा अनु॒ प्र जा॑यन्ते दे॒वा वा इ॒तइ॑तः॒ पत्नीः᳚ सुव॒र्गम्॥२६॥

%6.5.8.2
लो॒कम॑जिगाꣳस॒न्ते सु॑व॒र्गं लो॒कं न प्राजा॑न॒न्त ए॒तम्पा᳚त्नीव॒तम॑पश्य॒न्तम॑गृह्णत॒ ततो॒ वै ते सु॑व॒र्गं लो॒कम्प्राजा॑न॒न्॒ यत्पा᳚त्नीव॒तो गृ॒ह्यते॑ सुव॒र्गस्य॑ लो॒कस्य॒ प्रज्ञा᳚त्यै॒ स सोमो॒ नाति॑ष्ठत स्त्री॒भ्यो गृ॒ह्यमा॑ण॒स्तं घृ॒तं वज्रं॑ कृ॒त्वाघ्न॒न्तं निरि॑न्द्रियम्भू॒तम॑गृह्ण॒न्तस्मा॒थ्स्त्रियो॒ निरि॑न्द्रिया॒ अदा॑यादी॒रपि॑ पा॒पात्पु॒ꣳ॒स उप॑स्तितरम्॥२७॥

%6.5.8.3
व॒द॒न्ति॒ यद्घृ॒तेन॑ पात्नीव॒तꣴ श्री॒णाति॒ वज्रे॑णै॒वैनं॒ वशे॑ कृ॒त्वा गृ॑ह्णात्युपया॒मगृ॑हीतो॒\-ऽसीत्या॑हे॒यं वा उ॑पया॒मस्तस्मा॑दि॒मां प्र॒जा अनु॒ प्र जा॑यन्ते॒ बृह॒स्पति॑सुतस्य त॒ इत्या॑ह॒ ब्रह्म॒ वै दे॒वाना॒म्बृह॒स्पति॒र्ब्रह्म॑णै॒वास्मै᳚ प्र॒जाः प्र ज॑नयतीन्दो॒ इत्या॑ह॒ रेतो॒ वा इन्दू॒ रेत॑ ए॒व तद्द॑धातीन्द्रियाव॒ इति॑॥२८॥

%6.5.8.4
आ॒ह॒ प्र॒जा वा इ॑न्द्रि॒यं प्र॒जा ए॒वास्मै॒ प्र ज॑नय॒त्यग्ना(३) इत्या॑हा॒ग्निर्वै रे॑तो॒धाः पत्नी॑व॒ इत्या॑ह मिथुन॒त्वाय॑ स॒जूर्दे॒वेन॒ त्वष्ट्रा॒ सोम॑म्पि॒बेत्या॑ह॒ त्वष्टा॒ वै प॑शू॒नाम्मि॑थु॒नानाꣳ॑ रूप॒कृद्रू॒पमे॒व प॒शुषु॑ दधाति दे॒वा वै त्वष्टा॑रमजिघाꣳस॒न्थ्स पत्नीः॒ प्राप॑द्यत॒ तं न प्रति॒ प्राय॑च्छ॒न्तस्मा॒दपि॑॥२९॥

%6.5.8.5
वध्य॒म्प्रप॑न्नं॒ न प्रति॒ प्र य॑च्छन्ति॒ तस्मा᳚त्पात्नीव॒ते त्वष्ट्रे\-ऽपि॑ गृह्यते॒ न सा॑दय॒त्यस॑न्ना॒द्धि प्र॒जाः प्र॒जाय॑न्ते॒ नानु॒ वष॑ट्करोति॒ यद॑नुवषट्कु॒र्याद्रु॒द्रं प्र॒जा अ॒न्वव॑सृजे॒द्यन्नानु॑वषट्कु॒र्यादशा᳚न्तम॒ग्नीथ्सोम॑म्भक्षयेदुपा॒ꣳ॒श्वनु॒ वष॑ट्करोति॒ न रु॒द्रं प्र॒जा अ॑न्ववसृ॒जति॑ शा॒न्तम॒ग्नीथ्सोम॑म्भक्षय॒त्यग्नी॒न्नेष्टु॑रु॒पस्थ॒मा सी॑द॥३०॥

%6.5.8.6
नेष्टः॒ पत्नी॑मु॒दान॒येत्या॑हा॒ग्नीदे॒व नेष्ट॑रि॒ रेतो॒ दधा॑ति॒ नेष्टा॒ पत्नि॑यामुद्गा॒त्रा सं ख्या॑पयति प्र॒जाप॑ति॒र्वा ए॒ष यदु॑द्गा॒ता प्र॒जानां᳚ प्र॒जन॑नाया॒प उप॒ प्र व॑र्तयति॒ रेत॑ ए॒व तथ्सि॑ञ्चत्यू॒रुणोप॒ प्र व॑र्तयत्यू॒रुणा॒ हि रेतः॑ सि॒च्यते॑ नग्नं॒कृत्यो॒रुमुप॒ प्र व॑र्तयति य॒दा हि न॒ग्न ऊ॒रुर्भव॒त्यथ॑ मिथु॒नी भ॑व॒तो\-ऽथ॒ रेतः॑ सिच्य॒ते\-ऽथ॑ प्र॒जाः प्र जा॑यन्ते॥३१॥

%6.5.9.0
{\anuvakamend[{पत्नीः᳚ सुव॒र्गमुप॑स्तितरमिन्द्रियाव॒ इत्यपि॑ सीद मिथु॒न्य॑ष्टौ च॑}]}%॥८॥

%6.5.9.1
इन्द्रो॑ वृ॒त्रम॑ह॒न्तस्य॑ शीर्\mbox{}षकपा॒लमुदौ᳚ब्ज॒थ्स द्रो॑णकल॒शो॑\-ऽभव॒त्तस्मा॒थ्सोमः॒ सम॑स्रव॒थ्स हा॑रियोज॒नो॑\-ऽभव॒त्तं व्य॑चिकिथ्सज्जु॒हवा॒नी(३) मा हौ॒षा(३) मिति॒ सो॑\-ऽमन्यत॒ यद्धो॒ष्याम्या॒मꣳ हो᳚ष्यामि॒ यन्न हो॒ष्यामि॑ यज्ञवेश॒सं क॑रिष्या॒मीति॒ तम॑ध्रियत॒ होतु॒ꣳ॒ सो᳚\-ऽग्निर॑ब्रवी॒न्न मय्या॒मꣳ हो᳚ष्य॒सीति॒ तं धा॒नाभि॑रश्रीणात्॥३२॥

%6.5.9.2
तꣳ शृ॒तम्भू॒तम॑जुहो॒द्यद्धा॒नाभि॑र्\mbox{}हारियोज॒नꣴ श्री॒णाति॑ शृत॒त्वाय॑ शृ॒तमे॒वैन॑म्भू॒तं जु॑होति ब॒ह्वीभिः॑ श्रीणात्ये॒ताव॑ती\-रे॒वास्या॒मुष्मि॑ल्लोँ॒के का॑म॒दुघा॑ भव॒न्त्यथो॒ खल्वा॑हुरे॒ता वा इन्द्र॑स्य॒ पृश्न॑यः काम॒दुघा॒ यद्धा॑रियोज॒नीरिति॒ तस्मा᳚द्ब॒ह्वीभिः॑ श्रीणीयादृख्सा॒मे वा इन्द्र॑स्य॒ हरी॑ सोम॒पानौ॒ तयोः᳚ परि॒धय॑ आ॒धानं॒ यदप्र॑हृत्य परि॒धीञ्जु॑हु॒याद॒न्तरा॑धानाभ्याम्॥३३॥

%6.5.9.3
घा॒सम्प्र य॑च्छेत्प्र॒हृत्य॑ परि॒धीञ्जु॑होति॒ निरा॑धानाभ्यामे॒व घा॒सम्प्र य॑च्छत्युन्ने॒ता जु॑होति या॒तया॑मेव॒ ह्ये॑तर्\mbox{}ह्य॑ध्व॒र्युः स्व॒गाकृ॑तो॒ यद॑ध्व॒र्युर्जु॑हु॒याद्यथा॒ विमु॑क्त॒म्पुन॑र्यु॒नक्ति॑ ता॒दृगे॒व तच्छी॒र्\mbox{}षन्न॑धिनि॒धाय॑ जुहोति शीर्\mbox{}ष॒तो हि स स॒मभ॑वद्वि॒क्रम्य॑ जुहोति वि॒क्रम्य॒ हीन्द्रो॑ वृ॒त्रमह॒न्थ्समृ॑द्ध्यै प॒शवो॒ वै हा॑रियोज॒नीर्यथ्स॑म्भि॒न्द्यादल्पाः᳚॥३४॥

%6.5.9.4
ए॒न॒म्प॒शवो॑ भु॒ञ्जन्त॒ उप॑ तिष्ठेर॒न्॒यन्न स॑म्भि॒न्द्याद्ब॒हव॑ एनम्प॒शवो\-ऽभु॑ञ्जन्त॒ उप॑ तिष्ठेर॒न्मन॑सा॒ सम्बा॑धत उ॒भयं॑ करोति ब॒हव॑ ए॒वैन॑म्प॒शवो॑ भु॒ञ्जन्त॒ उप॑ तिष्ठन्त उन्ने॒तर्यु॑पह॒वमि॑च्छन्ते॒ य ए॒व तत्र॑ सोमपी॒थस्तमे॒वाव॑ रुन्धत उत्तरवे॒द्यां नि व॑पति प॒शवो॒ वा उ॑त्तरवे॒दिः प॒शवो॑ हारियोज॒नीः प॒शुष्वे॒व प॒शून्प्रति॑ ष्ठापयन्ति॥३५॥

%6.5.10.0
{\anuvakamend[{अ॒श्री॒णा॒द॒न्तरा॑धानाभ्या॒मल्पाः᳚ स्थापयन्ति}]}%॥९॥

%6.5.10.1
ग्रहा॒न् वा अनु॑ प्र॒जाः प॒शवः॒ प्र जा॑यन्त उपाꣳश्वन्तर्या॒माव॑जा॒वयः॑ शु॒क्राम॒न्थिनौ॒ पुरु॑षा ऋतुग्र॒हानेक॑शफा आदित्यग्र॒हं गाव॑ आदित्यग्र॒हो भूयि॑ष्ठाभिर्\mbox{}ऋ॒ग्भिर्गृ॑ह्यते॒ तस्मा॒द्गावः॑ पशू॒नाम्भूयि॑ष्ठा॒ यत्त्रिरु॑पा॒ꣳ॒शुꣳ हस्ते॑न विगृ॒ह्णाति॒ तस्मा॒द्द्वौ त्रीन॒जा ज॒नय॒त्यथाव॑यो॒ भूय॑सीः पि॒ता वा ए॒ष यदा᳚ग्रय॒णः पु॒त्रः क॒लशो॒ यदा᳚ग्रय॒ण उ॑प॒दस्ये᳚त्क॒लशा᳚द्गृह्णीया॒द्यथा॑ पि॒ता॥३६॥

%6.5.10.2
पु॒त्रं क्षि॒त उ॑प॒धाव॑ति ता॒दृगे॒व तद्यत्क॒लश॑ उप॒दस्ये॑दाग्रय॒णाद्गृ॑ह्णीया॒द्यथा॑ पु॒त्रः पि॒तरं॑ क्षि॒त उ॑प॒धाव॑ति ता॒दृगे॒व तदा॒त्मा वा ए॒ष य॒ज्ञस्य॒ यदा᳚ग्रय॒णो यद्ग्रहो॑ वा क॒लशो॑ वोप॒दस्ये॑दाग्रय॒णाद्गृ॑ह्णीयादा॒त्मन॑ ए॒वाधि॑ य॒ज्ञं निष्क॑रो॒त्यवि॑ज्ञातो॒ वा ए॒ष गृ॑ह्यते॒ यदा᳚ग्रय॒णः स्था॒ल्या गृ॒ह्णाति॑ वाय॒व्ये॑न जुहोति॒ तस्मा᳚त्॥३७॥

%6.5.10.3
गर्भे॒णावि॑ज्ञातेन ब्रह्म॒हाव॑भृ॒थमव॑ यन्ति॒ परा᳚ स्था॒लीरस्य॒न्त्युद्वा॑य॒व्या॑नि हरन्ति॒ तस्मा॒थ्स्त्रियं॑ जा॒तां परा᳚स्य॒न्त्यु\-त्पुमाꣳ॑सꣳ हरन्ति॒ यत्पु॑रो॒रुच॒माह॒ यथा॒ वस्य॑स आ॒हर॑ति ता॒दृगे॒व तद्यद्ग्रहं॑ गृ॒ह्णाति॒ यथा॒ वस्य॑स आ॒हृत्य॒ प्राह॑ ता॒दृगे॒व तद्यथ्सा॒दय॑ति॒ यथा॒ वस्य॑स उपनि॒धाया॑प॒क्राम॑ति ता॒दृगे॒व तद्यद्वै य॒ज्ञस्य॒ साम्ना॒ यजु॑षा क्रि॒यते॑ शिथि॒लं तद्यदृ॒चा तद्दृ॒ढम्पु॒रस्ता॑दुपयामा॒ यजु॑षा गृह्यन्त उ॒परि॑ष्टादुपयामा ऋ॒चा य॒ज्ञस्य॒ धृत्यै᳚॥३८॥

%6.5.11.0
{\anuvakamend[{यथा॑ पि॒ता तस्मा॑दप॒क्राम॑ति ता॒दृगे॒व तद्यद॒ष्टाद॑श च}]}%॥10॥

%6.5.11.1
प्रान्यानि॒ पात्रा॑णि यु॒ज्यन्ते॒ नान्यानि॒ यानि॑ परा॒चीना॑नि प्रयु॒ज्यन्ते॒\-ऽमुमे॒व तैर्लो॒कम॒भि ज॑यति॒ परा॑ङिव॒ ह्य॑सौ लो॒को यानि॒ पुनः॑ प्रयु॒ज्यन्त॑ इ॒ममे॒व तैर्लो॒कम॒भि ज॑यति॒ पुनः॑पुनरिव॒ ह्य॑यं लो॒कः प्रान्यानि॒ पात्रा॑णि यु॒ज्यन्ते॒ नान्यानि॒ यानि॑ परा॒चीना॑नि प्रयु॒ज्यन्ते॒ तान्यन्वोष॑धयः॒ परा॑ भवन्ति॒ यानि॒ पुनः॑॥३९॥

%6.5.11.2
प्र॒यु॒ज्यन्ते॒ तान्यन्वोष॑धयः॒ पुन॒रा भ॑वन्ति॒ प्रान्यानि॒ पात्रा॑णि यु॒ज्यन्ते॒ नान्यानि॒ यानि॑ परा॒चीना॑नि प्रयु॒ज्यन्ते॒ तान्यन्वा॑र॒ण्याः प॒शवो\-ऽर॑ण्य॒मप॑ यन्ति॒ यानि॒ पुनः॑ प्रयु॒ज्यन्ते॒ तान्यनु॑ ग्रा॒म्याः प॒शवो॒ ग्राम॑मु॒पाव॑यन्ति॒ यो वै ग्रहा॑णां नि॒दानं॒ वेद॑ नि॒दान॑वान्भव॒त्याज्य॒मित्यु॒क्थं तद्वै ग्रहा॑णां नि॒दानं॒ यदु॑पा॒ꣳ॒शु शꣳस॑ति॒ तत्॥४०॥

%6.5.11.3
उ॒पा॒ꣳ॒श्व॒न्त॒र्या॒मयो॒र्यदु॒च्चैस्तदित॑रेषां॒ ग्रहा॑णामे॒तद्वै ग्रहा॑णां नि॒दानं॒ य ए॒वं वेद॑ नि॒दान॑वान्भवति॒ यो वै ग्रहा॑णाम्मिथु॒नं वेद॒ प्र प्र॒जया॑ प॒शुभि॑र्मिथु॒नैर्जा॑यते स्था॒लीभि॑र॒न्ये ग्रहा॑ गृ॒ह्यन्ते॑ वाय॒व्यै॑र॒न्य ए॒तद्वै ग्रहा॑णाम्मिथु॒नं य ए॒वं वेद॒ प्र प्र॒जया॑ प॒शुभि॑र्मिथु॒नैर्जा॑यत॒ इन्द्र॒स्त्वष्टुः॒ सोम॑मभी॒षहा॑पिब॒थ्स विष्वङ्ङ्॑॥४१॥

%6.5.11.4
व्या᳚र्च्छ॒थ्स आ॒त्मन्ना॒रम॑णं॒ नावि॑न्द॒थ्स ए॒तान॑नुसव॒नम्पु॑रो॒डाशा॑नपश्य॒त्तान्निर॑वप॒त्तैर्वै स आ॒त्मन्ना॒रम॑णमकुरुत॒ तस्मा॑दनुसव॒नम्पु॑रो॒डाशा॒ निरु॑प्यन्ते॒ तस्मा॑दनुसव॒नम्पु॑रो॒डाशा॑ना॒म्प्राश्नी॑यादा॒त्मन्ने॒वारम॑णं कुरुते॒ नैन॒ꣳ॒ सोमो\-ऽति॑ पवते ब्रह्मवा॒दिनो॑ वदन्ति॒ नर्चा न यजु॑षा प॒ङ्क्तिरा᳚प्य॒ते\-ऽथ॒ किं य॒ज्ञस्य॑ पाङ्क्त॒त्वमिति॑ धा॒नाः क॑र॒म्भः प॑रिवा॒पः पु॑रो॒डाशः॑ पय॒स्या॑ तेन॑ प॒ङ्क्तिरा᳚प्यते॒ तद्य॒ज्ञस्य॑ पाङ्क्त॒त्वम्॥४२॥

%6.6.0.0
{\anuvakamend[{भ॒व॒न्ति॒ यानि॒ पुनः॒ शꣳस॑ति॒ तद्विष्व॒ङ्किञ्चतु॑र्दश च}]}%॥11॥

%6.6.0.0

{\anuvakamend[{सु॒व॒र्गाय॒ यद्दा᳚क्षि॒णानि॑ समिष्टय॒जूꣳष्य॑वभृथय॒जूꣳषि॒ स्फ्येन॑ प्र॒जाप॑तिरेकाद॒शिनी॒मिन्द्रः॒ पत्नि॑या॒ घ्नन्ति॑ दे॒वा वा इ॑न्द्रि॒यं दे॒वा वा अदा᳚भ्ये दे॒वा वै प्र॒बाहु॑क्प्र॒जाप॑तिर्दे॒वेभ्यः॒ स रि॑रिचा॒नष्षो॑डश॒धैका॑दश}]}%॥11॥ 
\prashnaend{सु॒व॒र्गाय॑ यजति प्र॒जाः सौ॒म्येन॑ गृह्णी॒यात्प्र॒त्यञ्चं॑ गृह्णी॒यात्प्र॒जां प॒शून्त्रिच॑त्वारिꣳशत्॥43॥ सु॒व॒र्गाय॒ वज्र॑स्य रू॒पꣳ समृ॑द्ध्यै॥}
%%% END PRASHNA

\sect{षष्ठमः प्रश्नः}\setcounter{anuvakam}{0}
\dnsub{तैत्तिरीयसंहितायां षष्ठमकाण्डे षष्ठमः प्रश्नः}
%6.6.1.0
%6.6.1.1
सु॒व॒र्गाय॒ वा ए॒तानि॑ लो॒काय॑ हूयन्ते॒ यद्दा᳚क्षि॒णानि॒ द्वाभ्यां॒ गार्\mbox{}ह॑पत्ये जुहोति द्वि॒पाद्यज॑मानः॒ प्रति॑ष्ठित्या॒ आग्नी᳚ध्रे जुहोत्य॒न्तरि॑क्ष ए॒वा क्र॑मते॒ सदो॒\-ऽभ्यैति॑ सुव॒र्गमे॒वैनं॑ लो॒कं ग॑मयति सौ॒रीभ्या॑मृ॒ग्भ्यां गार्\mbox{}ह॑पत्ये जुहोत्य॒मुमे॒वैनं॑ लो॒कꣳ स॒मारो॑हयति॒ नय॑वत्य॒र्चाग्नी᳚ध्रे जुहोति सुव॒र्गस्य॑ लो॒कस्या॒भिनी᳚त्यै॒ दिवं॑ गच्छ॒ सुवः॑ प॒तेति॒ हिर॑ण्यम्॥१॥

%6.6.1.2
हु॒त्वोद्गृ॑ह्णाति सुव॒र्गमे॒वैनं॑ लो॒कङ्ग॑मयति रू॒पेण॑ वो रू॒पम॒भ्यैमीत्या॑ह रू॒पेण॒ ह्या॑साꣳ रू॒पम॒भ्यैति॒ यद्धिर॑ण्येन तु॒थो वो॑ वि॒श्ववे॑दा॒ वि भ॑ज॒त्वित्या॑ह तु॒थो ह॑ स्म॒ वै वि॒श्ववे॑दा दे॒वानां॒ दक्षि॑णा॒ वि भ॑जति॒ तेनै॒वैना॒ वि भ॑जत्ये॒तत्ते॑ अग्ने॒ राधः॑॥२॥

%6.6.1.3
ऐति॒ सोम॑च्युत॒मित्या॑ह॒ सोम॑च्युत॒ꣴ॒ ह्य॑स्य॒ राध॒ ऐति॒ तन्मि॒त्रस्य॑ प॒था न॒येत्या॑ह॒ शान्त्या॑ ऋ॒तस्य॑ प॒था प्रेत॑ च॒न्द्रद॑क्षिणा॒ इत्या॑ह स॒त्यं वा ऋ॒तꣳ स॒त्येनै॒वैना॑ ऋ॒तेन॒ वि भ॑जति य॒ज्ञस्य॑ प॒था सु॑वि॒ता नय॑न्ती॒रित्या॑ह य॒ज्ञस्य॒ ह्ये॑ताः प॒था यन्ति॒ यद्दक्षि॑णा ब्राह्म॒णम॒द्य रा᳚ध्यासम्॥३॥

%6.6.1.4
ऋषि॑मार्\mbox{}षे॒यमित्या॑है॒ष वै ब्रा᳚ह्म॒ण ऋषि॑रार्\mbox{}षे॒यो यः शु॑श्रु॒वान्तस्मा॑दे॒वमा॑ह॒ वि सुवः॒ पश्य॒ व्य॑न्तरि॑क्ष॒मित्या॑ह सुव॒र्गमे॒वैनं॑ लो॒कं ग॑मयति॒ यत॑स्व सद॒स्यै॑रित्या॑ह मित्र॒त्वाया॒स्मद्दा᳚त्रा देव॒त्रा ग॑च्छत॒ मधु॑मतीः प्र दा॒तार॒मा वि॑श॒तेत्या॑ह व॒यमि॒ह प्र॑दा॒तारः॒ स्मो᳚\-ऽस्मान॒मुत्र॒ मधु॑मती॒रा वि॑श॒तेति॑॥४॥

%6.6.1.5
वावैतदा॑ह॒ हिर॑ण्यं ददाति॒ ज्योति॒र्वै हिर॑ण्यं॒ ज्योति॑रे॒व पु॒रस्ता᳚द्धत्ते सुव॒र्गस्य॑ लो॒कस्यानु॑ख्यात्या अ॒ग्नीधे॑ ददात्य॒ग्निमु॑खाने॒वर्तून्प्री॑णाति ब्र॒ह्मणे॑ ददाति॒ प्रसू᳚त्यै॒ होत्रे॑ ददात्या॒त्मा वा ए॒ष य॒ज्ञस्य॒ यद्धोता॒त्मान॑मे॒व य॒ज्ञस्य॒ दक्षि॑णाभिः॒ सम॑र्धयति॥५॥

%6.6.2.0
{\anuvakamend[{हिर॑ण्य॒ꣳ॒ राधो॑ राध्यासम॒मुत्र॒ मधु॑मती॒रा वि॑श॒तेत्य॒ष्टात्रिꣳ॑शच्च}]}%॥१॥

%6.6.2.1
स॒मि॒ष्ट॒य॒जूꣳषि॑ जुहोति य॒ज्ञस्य॒ समि॑ष्ट्यै॒ यद्वै य॒ज्ञस्य॑ क्रू॒रं यद्विलि॑ष्टं॒ यद॒त्येति॒ यन्नात्येति॒ यद॑तिक॒रोति॒ यन्नापि॑ क॒रोति॒ तदे॒व तैः प्री॑णाति॒ नव॑ जुहोति॒ नव॒ वै पुरु॑षे प्रा॒णाः पुरु॑षेण य॒ज्ञः सम्मि॑तो॒ यावा॑ने॒व य॒ज्ञस्तम्प्री॑णाति॒ षडृग्मि॑याणि जुहोति॒ षड्वा ऋ॒तव॑ ऋ॒तूने॒व प्री॑णाति॒ त्रीणि॒ यजूꣳ॑षि॥६॥

%6.6.2.2
त्रय॑ इ॒मे लो॒का इ॒माने॒व लो॒कान्प्री॑णाति॒ यज्ञ॑ य॒ज्ञं ग॑च्छ य॒ज्ञप॑तिं ग॒च्छेत्या॑ह य॒ज्ञप॑तिमे॒वैनं॑ गमयति॒ स्वां योनिं॑ ग॒च्छेत्या॑ह॒ स्वामे॒वैनं॒ योनिं॑ गमयत्ये॒ष ते॑ य॒ज्ञो य॑ज्ञपते स॒हसू᳚क्तवाकः सु॒वीर॒ इत्या॑ह॒ यज॑मान ए॒व वी॒र्यं॑ दधाति वासि॒ष्ठो ह॑ सात्यह॒व्यो दे॑वभा॒गम्प॑प्रच्छ॒ यथ्सृञ्ज॑यान्बहुया॒जिनो\-ऽयी॑यजो य॒ज्ञे॥७॥

%6.6.2.3
य॒ज्ञम्प्रत्य॑तिष्ठि॒पा(३)य॒ज्ञप॒ता(३)विति॒ स हो॑वाच य॒ज्ञप॑ता॒विति॑ स॒त्याद्वै सृञ्ज॑याः॒ परा॑ बभूवु॒रिति॑ होवाच य॒ज्ञे वाव य॒ज्ञः प्र॑ति॒ष्ठाप्य॑ आसी॒द्यज॑मान॒स्याप॑राभावा॒येति॒ देवा॑ गातुविदो गा॒तुं वि॒त्त्वा गा॒तुमि॒तेत्या॑ह य॒ज्ञ ए॒व य॒ज्ञं प्रति॑ ष्ठापयति॒ यज॑मान॒स्याप॑राभावाय॥८॥

%6.6.3.0
{\anuvakamend[{यजूꣳ॑षि य॒ज्ञ एक॑चत्वारिꣳशच्च}]}%॥२॥

%6.6.3.1
अ॒व॒भृ॒थ॒य॒जूꣳषि॑ जुहोति॒ यदे॒वार्वा॒चीन॒मेक॑हायना॒देनः॑ क॒रोति॒ तदे॒व तैरव॑ यजते॒\-ऽपो॑\-ऽवभृ॒थमवै᳚त्य॒फ्सु वै वरु॑णः सा॒क्षादे॒व वरु॑ण॒मव॑ यजते॒ वर्त्म॑ना॒ वा अ॒न्वित्य॑ य॒ज्ञꣳ रक्षाꣳ॑सि जिघाꣳसन्ति॒ साम्ना᳚ प्रस्तो॒तान्ववै॑ति॒ साम॒ वै र॑क्षो॒हा रक्ष॑सा॒मप॑हत्यै॒ त्रिर्नि॒धन॒मुपै॑ति॒ त्रय॑ इ॒मे लो॒का ए॒भ्य ए॒व लो॒केभ्यो॒ रक्षाꣳ॑सि॥९॥

%6.6.3.2
अप॑ हन्ति॒ पुरु॑षःपुरुषो नि॒धन॒मुपै॑ति॒ पुरु॑षःपुरुषो॒ हि र॑क्ष॒स्वी रक्ष॑सा॒मप॑हत्या उ॒रुꣳ हि राजा॒ वरु॑णश्च॒कारेत्या॑ह॒ प्रति॑ष्ठित्यै श॒तं ते॑ राजन्भि॒षजः॑ स॒हस्र॒मित्या॑ह भेष॒जमे॒वास्मै॑ करोत्य॒भिष्ठि॑तो॒ वरु॑णस्य॒ पाश॒ इत्या॑ह वरुणपा॒शमे॒वाभि ति॑ष्ठति ब॒र्\mbox{}हिर॒भि जु॑हो॒त्याहु॑तीनां॒ प्रति॑ष्ठित्या॒ अथो॑ अग्नि॒वत्ये॒व जु॑हो॒त्यप॑बर्\mbox{}हिषः प्रया॒जान्॥१०॥

%6.6.3.3
य॒ज॒ति॒ प्र॒जा वै ब॒र्\mbox{}हिः प्र॒जा ए॒व व॑रुणपा॒शान्मु॑ञ्च॒त्याज्य॑भागौ यजति य॒ज्ञस्यै॒व चक्षु॑षी॒ नान्तरे॑ति॒ वरु॑णं यजति वरुणपा॒शादे॒वैन॑म्मुञ्चत्य॒ग्नीवरु॑णौ यजति सा॒क्षादे॒वैनं॑ वरुणपा॒शान्मु॑ञ्च॒त्यप॑बर्\mbox{}हिषावनूया॒जौ य॑जति प्र॒जा वै ब॒र्\mbox{}हिः प्र॒जा ए॒व व॑रुणपा॒शान्मु॑ञ्चति च॒तुरः॑ प्रया॒जान् य॑जति॒ द्वाव॑नूया॒जौ षट्थ्सम्प॑द्यन्ते॒ षड्वा ऋ॒तवः॑॥११॥

%6.6.3.4
ऋ॒तुष्वे॒व प्रति॑ तिष्ठ॒त्यव॑भृथ निचङ्कु॒णेत्या॑ह यथोदि॒तमे॒व वरु॑ण॒मव॑ यजते समु॒द्रे ते॒ हृद॑यम॒फ्स्व॑न्तरित्या॑ह समु॒द्रे ह्य॑न्तर्वरु॑णः॒ सं त्वा॑ विश॒न्त्वोष॑धीरु॒ताप॒ इत्या॑हा॒द्भिरे॒वैन॒मोष॑धीभिः स॒म्यञ्चं॑ दधाति॒ देवी॑राप ए॒ष वो॒ गर्भ॒ इत्या॑ह यथाय॒जुरे॒वैतत्प॒शवो॒ वै॥१२॥

%6.6.3.5
सोमो॒ यद्भि॑न्दू॒नाम्भ॒क्षये᳚त्पशु॒मान्थ्स्या॒द्वरु॑ण॒स्त्वे॑नं गृह्णीया॒द्यन्न भ॒क्षये॑दप॒शुः स्या॒न्नैनं॒ वरु॑णो गृह्णीयादुप॒स्पृश्य॑मे॒व प॑शु॒मान्भ॑वति॒ नैनं॒ वरु॑णो गृह्णाति॒ प्रति॑युतो॒ वरु॑णस्य॒ पाश॒ इत्या॑ह वरुणपा॒शादे॒व निर्मु॑च्य॒ते\-ऽप्र॑तीक्ष॒मा य॑न्ति॒ वरु॑णस्या॒न्तर्\mbox{}हि॑त्या॒ एधो᳚\-ऽस्येधिषीम॒हीत्या॑ह स॒मिधै॒वाग्निं न॑म॒स्यन्त॑ उ॒पाय॑न्ति॒ तेजो॑\-ऽसि॒ तेजो॒ मयि॑ धे॒हीत्या॑ह॒ तेज॑ ए॒वात्मन्ध॑त्ते॥१३॥

%6.6.4.0
{\anuvakamend[{रक्षाꣳ॑सि प्रया॒जानृ॒तवो॒ वै न॑म॒स्यन्तो॒ द्वाद॑श च}]}%॥३॥

%6.6.4.1
स्फ्येन॒ वेदि॒मुद्ध॑न्ति रथा॒क्षेण॒ वि मि॑मीते॒ यूप॑म्मिनोति त्रि॒वृत॑मे॒व वज्रꣳ॑ स॒म्भृत्य॒ भ्रातृ॑व्याय॒ प्र ह॑रति॒ स्तृत्यै॒ यद॑न्तर्वे॒दि मि॑नु॒याद्दे॑वलो॒कम॒भि ज॑ये॒द्यद्ब॑हिर्वे॒दि म॑नुष्यलो॒कं वे᳚द्य॒न्तस्य॑ सं॒धौ मि॑नोत्यु॒भयो᳚र्लो॒कयो॑र॒भिजि॑त्या॒ उप॑रसम्मिताम्मिनुयात्पितृलो॒कका॑मस्य रश॒नस॑म्मिताम्मनुष्यलो॒कका॑मस्य च॒षाल॑सम्मितामिन्द्रि॒यका॑मस्य॒ सर्वा᳚न्थ्स॒मान्प्र॑ति॒ष्ठाका॑मस्य॒ ये त्रयो॑ मध्य॒मास्तान्थ्स॒मान्प॒शुका॑मस्यै॒तान् वै॥१४॥

%6.6.4.2
अनु॑ प॒शव॒ उप॑ तिष्ठन्ते पशु॒माने॒व भ॑वति॒ व्यति॑षजे॒दित॑रान्प्र॒जयै॒वैन॑म्प॒शुभि॒र्व्यति॑षजति॒ यं का॒मये॑त प्र॒मायु॑कः स्या॒दिति॑ गर्त॒मितं॒ तस्य॑ मिनुयादुत्तरा॒र्ध्यं॑ वर्\mbox{}षि॑ष्ठ॒मथ॒ ह्रसी॑याꣳसमे॒षा वै ग॑र्त॒मिद्यस्यै॒वम्मि॒नोति॑ ता॒जक्प्र मी॑यते दक्षिणा॒र्ध्यं॑ वर्\mbox{}षि॑ष्ठम्मिनुयाथ्सुव॒र्गका॑म॒स्याथ॒ ह्रसी॑याꣳसमा॒क्रम॑णमे॒व तथ्सेतुं॒ यज॑मानः कुरुते सुव॒र्गस्य॑ लो॒कस्य॒ सम॑ष्ट्यै॥१५॥

%6.6.4.3
यदेक॑स्मि॒न् यूपे॒ द्वे र॑श॒ने प॑रि॒व्यय॑ति॒ तस्मा॒देको॒ द्वे जा॒ये वि॑न्दते॒ यन्नैकाꣳ॑ रश॒नां द्वयो॒र्यूप॑योः परि॒व्यय॑ति॒ तस्मा॒न्नैका॒ द्वौ पती॑ विन्दते॒ यं का॒मये॑त॒ स्त्र्य॑स्य जाये॒तेत्यु॑पा॒न्ते तस्य॒ व्यति॑षजे॒थ्स्त्र्ये॑वास्य॑ जायते॒ यं का॒मये॑त॒ पुमा॑नस्य जाये॒तेत्या॒न्तं तस्य॒ प्र वे᳚ष्टये॒त्पुमा॑ने॒वास्य॑॥१६॥

%6.6.4.4
जा॒य॒ते\-ऽसु॑रा॒ वै दे॒वान्द॑क्षिण॒त उपा॑नय॒न्तां दे॒वा उ॑पश॒येनै॒वापा॑नुदन्त॒ तदु॑पश॒यस्यो॑पशय॒त्वं यद्द॑क्षिण॒त उ॑पश॒य उ॑प॒शये॒ भ्रातृ॑व्यापनुत्त्यै॒ सर्वे॒ वा अ॒न्ये यूपाः᳚ पशु॒मन्तो\-ऽथो॑पश॒य ए॒वाप॒शुस्तस्य॒ यज॑मानः प॒शुर्यन्न नि॑र्दि॒शेदार्ति॒\-मार्च्छे॒द्यज॑मानो॒\-ऽसौ ते॑ प॒शुरिति॒ निर्दि॑शे॒द्यं द्वि॒ष्याद्यमे॒व॥१७॥

%6.6.4.5
द्वेष्टि॒ तम॑स्मै प॒शुं निर्दि॑शति॒ यदि॒ न द्वि॒ष्यादा॒खुस्ते॑ प॒शुरिति॑ ब्रूया॒न्न ग्रा॒म्यान्प॒शून् हि॒नस्ति॒ नार॒ण्यान्प्र॒जाप॑तिः प्र॒जा अ॑सृजत॒ सो᳚\-ऽन्नाद्ये॑न॒ व्या᳚र्ध्यत॒ स ए॒तामे॑काद॒शिनी॑मपश्य॒त्तया॒ वै सो᳚\-ऽन्नाद्य॒मवा॑रुन्द्ध॒ यद्दश॒ यूपा॒ भव॑न्ति॒ दशा᳚क्षरा वि॒राडन्नं॑ वि॒राड्वि॒राजै॒वान्नाद्य॒मव॑ रुन्द्धे॥१८॥

%6.6.4.6
य ए॑काद॒शः स्तन॑ ए॒वास्यै॒ स दु॒ह ए॒वैनां॒ तेन॒ वज्रो॒ वा ए॒षा सम्मी॑यते॒ यदे॑काद॒शिनी॒ सेश्व॒रा पु॒रस्ता᳚त्प्र॒त्यञ्चं॑ य॒ज्ञꣳ सम्म॑र्दितो॒र्यत्पा᳚त्नीव॒तम्मि॒नोति॑ य॒ज्ञस्य॒ प्रत्युत्त॑ब्ध्यै सय॒त्वाय॑॥१९॥

%6.6.5.0
{\anuvakamend[{वै सम॑ष्ट्यै॒ पुमा॑ने॒वास्य॒ यमे॒व रु॑न्धे त्रि॒ꣳ॒शच्च॑}]}%॥४॥

%6.6.5.1
प्र॒जाप॑तिः प्र॒जा अ॑सृजत॒ स रि॑रिचा॒नो॑\-ऽमन्यत॒ स ए॒तामे॑काद॒शिनी॑मपश्य॒त्तया॒ वै स आयु॑रिन्द्रि॒यं वी॒र्य॑मा॒त्मन्न॑धत्त प्र॒जा इ॑व॒ खलु॒ वा ए॒ष सृ॑जते॒ यो यज॑ते॒ स ए॒तर्\mbox{}हि॑ रिरिचा॒न इ॑व॒ यदे॒षैका॑द॒शिनी॒ भव॒त्यायु॑रे॒व तये᳚न्द्रि॒यं वी॒र्यं॑ यज॑मान आ॒त्मन्ध॑त्ते॒ प्रैवाग्ने॒येन॑ वापयति मिथु॒नꣳ सा॑रस्व॒त्या क॑रोति॒ रेतः॑॥२०॥

%6.6.5.2
सौ॒म्येन॑ दधाति॒ प्र ज॑नयति पौ॒ष्णेन॑ बार्\mbox{}हस्प॒त्यो भ॑वति॒ ब्रह्म॒ वै दे॒वाना॒म्बृह॒स्पति॒र्ब्रह्म॑णै॒वास्मै᳚ प्र॒जाः प्र ज॑नयति वैश्वदे॒वो भ॑वति वैश्वदे॒व्यो॑ वै प्र॒जाः प्र॒जा ए॒वास्मै॒ प्र ज॑नयतीन्द्रि॒यमे॒वैन्द्रेणाव॑रुन्द्धे॒ विश॑म्मारु॒तेनौजो॒ बल॑मैन्द्रा॒ग्नेन॑ प्रस॒वाय॑ सावि॒त्रो नि॑र्वरुण॒त्वाय॑ वारु॒णो म॑ध्य॒त ऐ॒न्द्रमा ल॑भते मध्य॒त ए॒वेन्द्रि॒यं यज॑माने दधाति॥२१॥

%6.6.5.3
पु॒रस्ता॑दै॒न्द्रस्य॑ वैश्वदे॒वमाल॑भते वैश्वदे॒वं वा अन्न॒मन्न॑मे॒व पु॒रस्ता᳚द्धत्ते॒ तस्मा᳚त्पु॒रस्ता॒दन्न॑मद्यत ऐ॒न्द्रमा॒लभ्य॑ मारु॒तमा ल॑भते॒ विड्वै म॒रुतो॒ विश॑मे॒वास्मा॒ अनु॑ बध्नाति॒ यदि॑ का॒मये॑त॒ यो\-ऽव॑गतः॒ सो\-ऽप॑ रुध्यतां॒ यो\-ऽप॑रुद्धः॒ सो\-ऽव॑ गच्छ॒त्वित्यै॒न्द्रस्य॑ लो॒के वा॑रु॒णमा ल॑भेत वारु॒णस्य॑ लो॒क ऐ॒न्द्रम्॥२२॥

%6.6.5.4
य ए॒वाव॑गतः॒ सो\-ऽप॑ रुध्यते॒ यो\-ऽप॑रुद्धः॒ सो\-ऽव॑ गच्छति॒ यदि॑ का॒मये॑त प्र॒जा मु॑ह्येयु॒रिति॑ प॒शून्व्यति॑षजेत्प्र॒जा ए॒व मो॑हयति॒ यद॑भिवाह॒तो॑\-ऽपां वा॑रु॒णमा॒लभे॑त प्र॒जा वरु॑णो गृह्णीयाद्दक्षिण॒त उद॑ञ्च॒मा ल॑भते\-ऽपवाह॒तो॑\-ऽ\-पां प्र॒जाना॒मव॑रुणग्राहाय॥२३॥

%6.6.6.0
{\anuvakamend[{रेतो॒ यज॑माने दधाति लो॒क ऐ॒न्द्रꣳ स॒प्तत्रिꣳ॑शच्च}]}%॥५॥

%6.6.6.1
इन्द्रः॒ पत्नि॑या॒ मनु॑मयाजय॒त्तां पर्य॑ग्निकृता॒मुद॑सृज॒त्तया॒ मनु॑रार्ध्नो॒द्यत्पर्य॑ग्निकृतम्पात्नीव॒तमु॑थ्सृ॒जति॒ यामे॒व मनु॒र्\mbox{}ऋद्धि॒\-मार्ध्नो॒त्तामे॒व यज॑मान ऋध्नोति य॒ज्ञस्य॒ वा अप्र॑तिष्ठिताद्य॒ज्ञः परा॑ भवति य॒ज्ञं प॑रा॒भव॑न्तं॒ यज॑मा॒नो\-ऽनु॒ परा॑ भवति॒ यदाज्ये॑न पात्नीव॒तꣳ सꣴ॑स्था॒पय॑ति य॒ज्ञस्य॒ प्रति॑ष्ठित्यै य॒ज्ञम्प्र॑ति॒तिष्ठ॑न्तं॒ यज॑मा॒नो\-ऽनु॒ प्रति॑ तिष्ठती॒ष्टं व॒पया᳚॥२४॥

%6.6.6.2
भव॒त्यनि॑ष्टं व॒शयाथ॑ पात्नीव॒तेन॒ प्र च॑रति ती॒र्थ ए॒व प्र च॑र॒त्यथो॑ ए॒तर्\mbox{}ह्ये॒वास्य॒ याम॑स्त्वा॒ष्ट्रो भ॑वति॒ त्वष्टा॒ वै रेत॑सः सि॒क्तस्य॑ रू॒पाणि॒ वि क॑रोति॒ तमे॒व वृ॑षाण॒म्पत्नी॒ष्वपि॑ सृजति॒ सो᳚\-ऽस्मै रू॒पाणि॒ वि क॑रोति॥२५॥

%6.6.7.0
{\anuvakamend[{व॒पया॒ षट्त्रिꣳ॑शच्च}]}%॥६॥

%6.6.7.1
घ्नन्ति॒ वा ए॒तथ्सोमं॒ यद॑भिषु॒ण्वन्ति॒ यथ्सौ॒म्यो भव॑ति॒ यथा॑ मृ॒ताया॑नु॒स्तर॑णीं॒ घ्नन्ति॑ ता॒दृगे॒व तद्यदु॑त्तरा॒र्धे वा॒ मध्ये॑ वा जुहु॒याद्दे॒वता᳚भ्यः स॒मदं॑ दध्याद्दक्षिणा॒र्धे जु॑होत्ये॒षा वै पि॑तृ॒णां दिख्स्वाया॑मे॒व दि॒शि पि॒तॄन्नि॒रव॑दयत उद्गा॒तृभ्यो॑ हरन्ति सामदेव॒त्यो॑ वै सौ॒म्यो यदे॒व साम्न॑श्छम्बट्कु॒र्वन्ति॒ तस्यै॒व स शान्ति॒रव॑॥२६॥

%6.6.7.2
ई॒क्ष॒न्ते॒ प॒वित्रं॒ वै सौ॒म्य आ॒त्मान॑मे॒व प॑वयन्ते॒ य आ॒त्मानं॒ न प॑रि॒पश्ये॑दि॒तासुः॑ स्यादभिद॒दिं कृ॒त्वावे᳚क्षेत॒ तस्मि॒न् ह्या᳚त्मानं॑ परि॒पश्य॒त्यथो॑ आ॒त्मान॑मे॒व प॑वयते॒ यो ग॒तम॑नाः॒ स्याथ्सो\-ऽवे᳚क्षेत॒ यन्मे॒ मनः॒ परा॑गतं॒ यद्वा॑ मे॒ अप॑रागतम्। राज्ञा॒ सोमे॑न॒ तद्व॒यम॒स्मासु॑ धारयाम॒सीति॒ मन॑ ए॒वात्मन्दा॑धार॥२७॥

%6.6.7.3
न ग॒तम॑ना भव॒त्यप॒ वै तृ॑तीयसव॒ने य॒ज्ञः क्रा॑मतीजा॒नादनी॑जानम॒भ्या᳚ग्नावैष्ण॒व्यर्चा घृ॒तस्य॑ यजत्य॒ग्निः सर्वा॑ दे॒वता॒ विष्णु॑र्य॒ज्ञो दे॒वता᳚श्चै॒व य॒ज्ञं च॑ दाधारोपा॒ꣳ॒शु य॑जति मिथुन॒त्वाय॑ ब्रह्मवा॒दिनो॑ वदन्ति मि॒त्रो य॒ज्ञस्य॒ स्वि॑ष्टं युवते॒ वरु॑णो॒ दुरि॑ष्टं॒ क्व॑ तर्\mbox{}हि॑ य॒ज्ञः क्व॑ यज॑मानो भव॒तीति॒ यन्मै᳚त्रावरु॒णीं व॒शामा॒लभ॑ते मि॒त्रेणै॒व॥२८॥

%6.6.7.4
य॒ज्ञस्य॒ स्वि॑ष्टꣳ शमयति॒ वरु॑णेन॒ दुरि॑ष्टं॒ नार्ति॒मार्च्छ॑ति॒ यज॑मानो॒ यथा॒ वै लाङ्ग॑लेनो॒र्वरां᳚ प्रभि॒न्दन्त्ये॒वमृ॑ख्सा॒मे य॒ज्ञम्प्र भि॑न्तो॒ यन्मै᳚त्रावरु॒णीं व॒शामा॒लभ॑ते य॒ज्ञायै॒व प्रभि॑न्नाय म॒त्य॑म॒न्ववा᳚स्यति॒ शान्त्यै॑ या॒तया॑मानि॒ वा ए॒तस्य॒ छन्दाꣳ॑सि॒ य ई॑जा॒नश्छन्द॑सामे॒ष रसो॒ यद्व॒शा यन्मै᳚त्रावरु॒णीं व॒शामा॒लभ॑ते॒ छन्दाꣳ॑स्ये॒व पुन॒रा प्री॑णा॒त्यया॑तयामत्वा॒याथो॒ छन्दः॑स्वे॒व रसं॑ दधाति॥२९॥

%6.6.8.0
{\anuvakamend[{अव॑ दाधार मि॒त्रेणै॒व प्री॑णाति॒ षट्च॑}]}%॥७॥

%6.6.8.1
दे॒वा वा इ॑न्द्रि॒यं वी॒र्यं  व्य॑भजन्त॒ ततो॒ यद॒त्यशि॑ष्यत॒ तद॑तिग्रा॒ह्या॑ अभव॒न्तद॑तिग्रा॒ह्या॑णामतिग्राह्य॒त्वं यद॑तिग्रा॒ह्या॑ गृ॒ह्यन्त॑ इन्द्रि॒यमे॒व तद्वी॒र्यं॑ यज॑मान आ॒त्मन्ध॑त्ते॒ तेज॑ आग्ने॒येने᳚न्द्रि॒यमै॒न्द्रेण॑ ब्रह्मवर्च॒सꣳ सौ॒र्येणो॑प॒स्तम्भ॑नं॒ वा ए॒तद्य॒ज्ञस्य॒ यद॑तिग्रा॒ह्या᳚श्च॒क्रे पृ॒ष्ठानि॒ यत्पृष्ठ्ये॒ न गृ॑ह्णी॒यात्प्राञ्चं॑ य॒ज्ञं पृ॒ष्ठानि॒ सꣳ शृ॑णीयु॒र्यदु॒क्थ्ये᳚॥३०॥

%6.6.8.2
गृ॒ह्णी॒यात्प्र॒त्यञ्चं॑ य॒ज्ञम॑तिग्रा॒ह्याः᳚ सꣳ शृ॑णीयुर्विश्व॒जिति॒ सर्व॑पृष्ठे ग्रहीत॒व्या॑ य॒ज्ञस्य॑ सवीर्य॒त्वाय॑ प्र॒जाप॑तिर्दे॒वेभ्यो॑ य॒ज्ञान्व्यादि॑श॒थ्स प्रि॒यास्त॒नूरप॒ न्य॑धत्त॒ तद॑तिग्रा॒ह्या॑ अभव॒न्वित॑नु॒स्तस्य॑ य॒ज्ञ इत्या॑हु॒र्यस्या॑तिग्रा॒ह्या॑ न गृ॒ह्यन्त॒ इत्यप्य॑ग्निष्टो॒मे ग्र॑हीत॒व्या॑ य॒ज्ञस्य॑ सतनु॒त्वाय॑ दे॒वता॒ वै सर्वाः᳚ स॒दृशी॑रास॒न्ता न व्या॒वृत᳚मगच्छ॒न्ते दे॒वाः॥३१॥

%6.6.8.3
ए॒त ए॒तान्ग्रहा॑नपश्य॒न्तान॑गृह्णताग्ने॒यम॒ग्निरै॒न्द्रमिन्द्रः॑ सौ॒र्यꣳ सूर्य॒स्ततो॒ वै ते᳚\-ऽन्याभि॑र्दे॒वता॑भिर्व्या॒वृत॑मगच्छ॒न् यस्यै॒वं वि॒दुष॑ ए॒ते ग्रहा॑ गृ॒ह्यन्ते᳚ व्या॒वृत॑मे॒व पा॒प्मना॒ भ्रातृ॑व्येण गच्छती॒मे लो॒का ज्योति॑ष्मन्तः स॒माव॑द्वीर्याः का॒र्या॑ इत्या॑हुराग्ने॒येना॒स्मिल्लोँ॒के ज्योति॑र्धत्त ऐ॒न्द्रेणा॒न्तरि॑क्ष इन्द्रवा॒यू हि स॒युजौ॑ सौ॒र्येणा॒मुष्मि॑ल्लोँ॒के॥३२॥

%6.6.8.4
ज्योति॑र्धत्ते॒ ज्योति॑ष्मन्तो\-ऽस्मा इ॒मे लो॒का भ॑वन्ति स॒माव॑द्वीर्यानेनान्कुरुत ए॒तान् वै ग्रहा᳚न्ब॒म्बावि॒श्वव॑यसाववित्ता॒म् ताभ्या॑मि॒मे लो॒काः परा᳚ञ्चश्चा॒र्वाञ्च॑श्च॒ प्राभु॒र्यस्यै॒वं वि॒दुष॑ ए॒ते ग्रहा॑ गृ॒ह्यन्ते॒ प्रास्मा॑ इ॒मे लो॒काः परा᳚ञ्चश्चा॒र्वाञ्च॑श्च भान्ति॥३३॥

%6.6.9.0
{\anuvakamend[{उ॒क्थ्ये॑ दे॒वा अ॒मुष्मि॑ल्लोँ॒क एका॒न्नच॑त्वारि॒ꣳ॒शच्च॑}]}%॥८॥

%6.6.9.1
दे॒वा वै यद्य॒ज्ञे\-ऽकु॑र्वत॒ तदसु॑रा अकुर्वत॒ ते दे॒वा अदा᳚भ्ये॒ छन्दाꣳ॑सि॒ सव॑नानि॒ सम॑स्थापय॒न्ततो॑ दे॒वा अभ॑व॒न्परासु॑रा॒ यस्यै॒वं वि॒दुषो\-ऽदा᳚भ्यो गृ॒ह्यते॒ भव॑त्या॒त्मना॒ परा᳚स्य॒ भ्रातृ॑व्यो भवति॒ यद्वै दे॒वा असु॑रा॒नदा᳚भ्ये॒ना\-द॑भ्नुव॒न्तददा᳚भ्यस्यादाभ्य॒त्वं य ए॒वं वेद॑ द॒भ्नोत्ये॒व भ्रातृ॑व्यं॒ नैन॒म्भ्रातृ॑व्यो दभ्नोति॥३४॥

%6.6.9.2
ए॒षा वै प्र॒जाप॑तेरतिमो॒क्षिणी॒ नाम॑ त॒नूर्यददा᳚भ्य॒ उप॑नद्धस्य गृह्णा॒त्यति॑मुक्त्या॒ अति॑ पा॒प्मान॒म्भ्रातृ॑व्यम्मुच्यते॒ य ए॒वं वेद॒ घ्नन्ति॒ वा ए॒तथ्सोमं॒ यद॑भिषु॒ण्वन्ति॒ सोमे॑ ह॒न्यमा॑ने य॒ज्ञो ह॑न्यते य॒ज्ञे यज॑मानो ब्रह्मवा॒दिनो॑ वदन्ति॒ किं तद्य॒ज्ञे यज॑मानः कुरुते॒ येन॒ जीव᳚न्थ्सुव॒र्गं लो॒कमेतीति॑ जीवग्र॒हो वा ए॒ष यददा॒भ्यो\-ऽन॑भिषुतस्य गृह्णाति॒ जीव॑न्तमे॒वैनꣳ॑ सुव॒र्गं लो॒कं ग॑मयति॒ वि वा ए॒तद्य॒ज्ञं छि॑न्दन्ति॒ यददा᳚भ्ये सꣴस्था॒पय॑न्त्य॒ꣳ॒शूनपि॑ सृजति य॒ज्ञस्य॒ सन्त॑त्यै॥३५॥

%6.6.10.0
{\anuvakamend[{द॒भ्नो॒त्यन॑भिषुतस्य गृह्णा॒त्येका॒न्नविꣳ॑श॒तिश्च॑}]}%॥९॥

%6.6.10.1
दे॒वा वै प्र॒बाहु॒ग्ग्रहा॑नगृह्णत॒ स ए॒तं प्र॒जाप॑तिर॒ꣳ॒शुम॑पश्य॒त्तम॑गृह्णीत॒ तेन॒ वै स आ᳚र्ध्नो॒द्यस्यै॒वं वि॒दुषो॒\-ऽꣳ॒शुर्गृ॒ह्यत॑ ऋ॒ध्नोत्ये॒व स॒कृद॑भिषुतस्य गृह्णाति स॒कृद्धि स तेनार्ध्नो॒न्मन॑सा गृह्णाति॒ मन॑ इव॒ हि प्र॒जाप॑तिः प्र॒जाप॑ते॒राप्त्या॒ औदु॑म्बरेण गृह्णा॒त्यूर्ग्वा उ॑दु॒म्बर॒ ऊर्ज॑मे॒वाव॑ रुन्द्धे॒ चतुः॑स्रक्ति भवति दि॒क्षु॥३६॥

%6.6.10.2
ए॒व प्रति॑ तिष्ठति॒ यो वा अ॒ꣳ॒शोरा॒यत॑नं॒ वेदा॒यत॑नवान्भवति वामदे॒व्यमिति॒ साम॒ तद्वा अ॑स्या॒यत॑न॒म्मन॑सा॒ गाय॑मानो गृह्णात्या॒यत॑नवाने॒व भ॑वति॒ यद॑ध्व॒र्युर॒ꣳ॒शुं गृ॒ह्णन्नार्धये॑दु॒भाभ्यां॒ नर्ध्ये॑ताध्व॒र्यवे॑ च॒ यज॑मानाय च॒ यद॒र्धये॑दु॒भाभ्या॑मृध्ये॒तान॑वानं गृह्णाति॒ सैवास्यर्द्धि॒र्\mbox{}हिर॑ण्यम॒भि व्य॑नित्य॒मृतं॒ वै हिर॑ण्य॒मायुः॑ प्रा॒ण आयु॑षै॒वामृत॑म॒भि धि॑नोति श॒तमा॑नम्भवति श॒तायुः॒ पुरु॑षः श॒तेन्द्रि॑य॒ आयु॑ष्ये॒वेन्द्रि॒ये प्रति॑ तिष्ठति॥३७॥

%6.6.11.0
{\anuvakamend[{दि॒क्ष्व॑निति विꣳश॒तिश्च॑}]}%॥10॥

%6.6.11.1
प्र॒जाप॑तिर्दे॒वेभ्यो॑ य॒ज्ञान्व्यादि॑श॒थ्स रि॑रिचा॒नो॑\-ऽमन्यत॒ स य॒ज्ञानाꣳ॑ षोडश॒धेन्द्रि॒यं वी॒र्य॑मा॒त्मान॑म॒भि सम॑क्खिद॒त् तथ्षो॑ड॒श्य॑भव॒न्न वै षो॑ड॒शी नाम॑ य॒ज्ञो᳚\-ऽस्ति॒ यद्वाव षो॑ड॒शꣴ स्तो॒त्रꣳ षो॑ड॒शꣳ श॒स्त्रं तेन॑ षोड॒शी तथ्षो॑ड॒शिनः॑ षोडशि॒त्वं यथ्षो॑ड॒शी गृ॒ह्यत॑ इन्द्रि॒यमे॒व तद्वी॒र्यं॑ यज॑मान आ॒त्मन्ध॑त्ते दे॒वेभ्यो॒ वै सु॑व॒र्गो लो॒कः॥३८॥

%6.6.11.2
न प्राभ॑व॒त्त ए॒तꣳ षो॑ड॒शिन॑मपश्य॒न्तम॑गृह्णत॒ ततो॒ वै तेभ्यः॑ सुव॒र्गो लो॒कः प्राभ॑व॒द्यथ्षो॑ड॒शी गृ॒ह्यते॑ सुव॒र्गस्य॑ लो॒कस्या॒भिजि॑त्या॒ इन्द्रो॒ वै दे॒वाना॑मानुजाव॒र आ॑सी॒थ्स प्र॒जाप॑ति॒मुपा॑धाव॒त्तस्मा॑ ए॒तꣳ षो॑ड॒शिन॒म्प्राय॑च्छ॒त्तम॑गृह्णीत॒ ततो॒ वै सो\-ऽग्रं॑ दे॒वता॑नां॒ पर्यै॒द्यस्यै॒वं वि॒दुषः॑ षोड॒शी गृ॒ह्यते᳚॥३९॥

%6.6.11.3
अग्र॑मे॒व स॑मा॒नानां॒ पर्ये॑ति प्रातःसव॒ने गृ॑ह्णाति॒ वज्रो॒ वै षो॑ड॒शी वज्रः॑ प्रातःसव॒नꣴ स्वादे॒वैनं॒ योने॒र्निर्गृ॑ह्णाति॒ सव॑नेसवने॒\-ऽभि गृ॑ह्णाति॒ सव॑नाथ्सवनादे॒वैन॒म्प्र ज॑नयति तृतीयसव॒ने प॒शुका॑मस्य गृह्णीया॒द्वज्रो॒ वै षो॑ड॒शी प॒शव॑स्तृतीयसव॒नं वज्रे॑णै॒वास्मै॑ तृतीयसव॒नात्प॒शूनव॑ रुन्द्धे॒ नोक्थ्ये॑ गृह्णीयात्प्र॒जा वै प॒शव॑ उ॒क्थानि॒ यदु॒क्थ्ये᳚॥४०॥

%6.6.11.4
गृ॒ह्णी॒यात्प्र॒जां प॒शून॑स्य॒ निर्द॑हेदतिरा॒त्रे प॒शुका॑मस्य गृह्णीया॒द्वज्रो॒ वै षो॑ड॒शी वज्रे॑णै॒वास्मै॑ प॒शून॑व॒रुध्य॒ रात्रि॑यो॒परि॑ष्टाच्छमय॒त्यप्य॑ग्निष्टो॒मे रा॑ज॒न्य॑स्य गृह्णीयाद्व्या॒वृत्का॑मो॒ हि रा॑ज॒न्यो॑ यज॑ते सा॒ह्न ए॒वास्मै॒ वज्रं॑ गृह्णाति॒ स ए॑नं॒ वज्रो॒ भूत्या॑ इन्द्धे॒ निर्वा दहत्येकवि॒ꣳ॒शꣴ स्तो॒त्रम्भ॑वति॒ प्रति॑ष्ठित्यै॒ हरि॑वच्छस्यत॒ इन्द्र॑स्य प्रि॒यं धाम॑॥४१॥

%6.6.11.5
उपा᳚प्नोति॒ कनी॑याꣳसि॒ वै दे॒वेषु॒ छन्दा॒ꣳ॒स्यास॒ञ्ज्याया॒ꣳ॒स्यसु॑रेषु॒ ते दे॒वाः कनी॑यसा॒ छन्द॑सा॒ ज्याय॒श्छन्दो॒\-ऽभि व्य॑शꣳस॒न्ततो॒ वै ते\-ऽसु॑राणां लो॒कम॑वृञ्जत॒ यत्कनी॑यसा॒ छन्द॑सा॒ ज्याय॒श्छन्दो॒\-ऽभि वि॒शꣳस॑ति॒ भ्रातृ॑व्यस्यै॒व तल्लो॒कं वृ॑ङ्क्ते॒ षड॒क्षरा॒ण्यति॑ रेचयन्ति॒ षड्वा ऋ॒तव॑ ऋ॒तूने॒व प्री॑णाति च॒त्वारि॒ पूर्वा॒ण्यव॑ कल्पयन्ति॥४२॥

%6.6.11.6
चतु॑ष्पद ए॒व प॒शूनव॑ रुन्द्धे॒ द्वे उत्त॑रे द्वि॒पद॑ ए॒वाव॑ रुन्द्धे\-ऽनु॒ष्टुभ॑म॒भि सम्पा॑दयन्ति॒ वाग्वा अ॑नु॒ष्टुप्तस्मा᳚त्प्रा॒णानां॒ वागु॑त्त॒मा स॑मयाविषि॒ते सूर्ये॑ षोड॒शिनः॑ स्तो॒त्रमु॒पाक॑रोत्ये॒तस्मि॒न्वै लो॒क इन्द्रो॑ वृ॒त्रम॑हन्थ्सा॒क्षादे॒व वज्र॒म्भ्रातृ॑व्याय॒ प्र ह॑रत्यरुणपिशं॒गो\-ऽश्वो॒ दक्षि॑णै॒तद्वै वज्र॑स्य रू॒पꣳ समृ॑द्ध्यै॥४३॥

%7.1.0.0

%7.1.0.0
{\anuvakamend[{लो॒को वि॒दुषः॑ षोड॒शी गृ॒ह्यते॒ यदु॒क्थ्ये॑ धाम॑ कल्पयन्ति स॒प्तच॑त्वारिꣳशच्च}]}%॥11॥

%%% END KANDAM

\chapt{काण्डम् ७}
\sect{प्रथमः प्रश्नः}\setcounter{anuvakam}{0}
\dnsub{तैत्तिरीयसंहितायां सप्तमकाण्डे प्रथमः प्रश्नः}
%7.1.1.0
%7.1.1.1
प्र॒जन॑नं॒ ज्योति॑र॒ग्निर्दे॒वता॑नां॒ ज्योति॑र्वि॒राट्छन्द॑सां॒ ज्योति॑र्वि॒राड्वा॒चो᳚\-ऽग्नौ सं ति॑ष्ठते वि॒राज॑म॒भि सम्प॑द्यते॒ तस्मा॒\-त्तज्ज्योति॑रुच्यते॒ द्वौ स्तोमौ᳚ प्रातःसव॒नं व॑हतो॒ यथा᳚ प्रा॒णश्चा॑पा॒नश्च॒ द्वौ माध्यं॑दिन॒ꣳ॒ सव॑नं॒ यथा॒ चक्षु॑श्च॒ श्रोत्रं॑ च॒ द्वौ तृ॑तीयसव॒नं यथा॒ वाक्च॑ प्रति॒ष्ठा च॒ पुरु॑षसम्मितो॒ वा ए॒ष य॒ज्ञो\-ऽस्थू॑रिः॥१॥

%7.1.1.2
यं कामं॑ का॒मय॑ते॒ तमे॒तेना॒भ्य॑श्ञुते॒ सर्व॒ꣴ॒ ह्यस्थू॑रिणाभ्यश्ञु॒ते᳚\-ऽग्निष्टो॒मेन॒ वै प्र॒जाप॑तिः प्र॒जा अ॑सृजत॒ ता अ॑ग्निष्टो॒मेनै॒व पर्य॑गृह्णा॒त्तासां॒ परि॑गृहीतानामश्वत॒रो\-ऽत्य॑प्रवत॒ तस्या॑नु॒हाय॒ रेत॒ आद॑त्त॒ तद्ग॑र्द॒भे न्य॑मा॒र्ट्तस्मा᳚द्गर्द॒भो द्वि॒रेता॒ अथो॑ आहु॒र्वड॑बायां॒ न्य॑मा॒र्डिति॒ तस्मा॒द्वड॑बा द्वि॒रेता॒ अथो॑ आहु॒रोष॑धीषु॥२॥

%7.1.1.3
न्य॑मा॒र्डिति॒ तस्मा॒दोष॑ध॒यो\-ऽन॑भ्यक्ता रेभ॒न्त्यथो॑ आहुः प्र॒जासु॒ न्य॑मा॒र्डिति॒ तस्मा᳚द्य॒मौ जा॑येते॒ तस्मा॑दश्वत॒रो न प्र जा॑यत॒ आत्त॑रेता॒ हि तस्मा᳚द्ब॒र्\mbox{}हिष्यन॑वकॢप्तः सर्ववेद॒से वा॑ स॒हस्रे॒ वाव॑कॢ॒प्तो\-ऽति॒ ह्यप्र॑वत॒ य ए॒वं वि॒द्वान॑ग्निष्टो॒मेन॒ यज॑ते॒ प्राजा॑ताः प्र॒जा ज॒नय॑ति॒ परि॒ प्रजा॑ता गृह्णाति॒ तस्मा॑दाहुर्ज्येष्ठय॒ज्ञ इति॑॥३॥

%7.1.1.4
प्र॒जाप॑ति॒र्वाव ज्येष्ठः॒ स ह्ये॑तेनाग्रे\-ऽय॑जत प्र॒जाप॑तिरकामयत॒ प्र जा॑ये॒येति॒ स मु॑ख॒तस्त्रि॒वृतं॒ निर॑मिमीत॒ तम॒ग्नि\-र्दे॒वतान्व॑सृज्यत गाय॒त्री छन्दो॑ रथन्त॒रꣳ साम॑ ब्राह्म॒णो म॑नु॒ष्या॑णाम॒जः प॑शू॒नान्तस्मा॒त्ते मुख्या॑ मुख॒तो ह्यसृ॑ज्य॒न्तोर॑सो बा॒हु\-भ्यां᳚ पञ्चद॒शं निर॑मिमीत॒ तमिन्द्रो॑ दे॒वतान्व॑सृज्यत त्रि॒ष्टुप्छन्दो॑ बृ॒हत्॥४॥

%7.1.1.5
साम॑ राज॒न्यो॑ मनु॒ष्या॑णा॒मविः॑ पशू॒नान्तस्मा॒त्ते वी॒र्या॑वन्तो वी॒र्या᳚द्ध्यसृ॑ज्यन्त मध्य॒तः स॑प्तद॒शं निर॑मिमीत॒ तं विश्वे॑ दे॒वा दे॒वता॒ अन्व॑सृज्यन्त॒ जग॑ती॒ छन्दो॑ वैरू॒पꣳ साम॒ वैश्यो॑ मनु॒ष्या॑णां॒ गावः॑ पशू॒नान्तस्मा॒त्त आ॒द्या॑ अन्न॒धाना॒\-द्ध्यसृ॑ज्यन्त॒ तस्मा॒द्भूयाꣳ॑सो॒\-ऽन्येभ्यो॒ भूयि॑ष्ठा॒ हि दे॒वता॒ अन्वसृ॑ज्यन्त प॒त्त ए॑कवि॒ꣳ॒शं निर॑मिमीत॒ तम॑नु॒ष्टुप्छन्दः॑ [5[]

%7.1.1.6
अन्व॑सृज्यत वैरा॒जꣳ साम॑ शू॒द्रो म॑नु॒ष्या॑णा॒मश्वः॑ पशू॒नान्तस्मा॒त्तौ भू॑तसङ्क्रा॒मिणा॒वश्व॑श्च शू॒द्रश्च॒ तस्मा᳚च्छू॒द्रो य॒ज्ञे\-ऽन॑वकॢप्तो॒ न हि दे॒वता॒ अन्वसृ॑ज्यत॒ तस्मा॒त्पादा॒वुप॑ जीवतः प॒त्तो ह्यसृ॑ज्येतां प्रा॒णा वै त्रि॒वृद॑र्धमा॒साः प॑ञ्चद॒शः प्र॒जाप॑तिः सप्तद॒शस्त्रय॑ इ॒मे लो॒का अ॒सावा॑दि॒त्य ए॑कवि॒ꣳ॒श ए॒तस्मि॒न्वा ए॒ते श्रि॒ता ए॒तस्मि॒न्प्रति॑ष्ठिता॒ य ए॒वं वेदै॒तस्मि॑न्ने॒व श्र॑यत ए॒तस्मि॒न्प्रति॑ तिष्ठति॥६॥

%7.1.2.0
{\anuvakamend[{अस्थू॑रि॒रोष॑धीषु ज्येष्ठय॒ज्ञ इति॑ बृ॒हद॑नु॒ष्टुप्छन्दः॒ प्रति॑ष्ठिता॒ नव॑ च}]}%॥१॥

%7.1.2.1
प्रा॒तः॒स॒व॒ने वै गा॑य॒त्रेण॒ छन्द॑सा त्रि॒वृते॒ स्तोमा॑य॒ ज्योति॒र्दध॑देति त्रि॒वृता᳚ ब्रह्मवर्च॒सेन॑ पञ्चद॒शाय॒ ज्योति॒र्दध॑देति पञ्चद॒शेनौज॑सा वी॒र्ये॑ण सप्तद॒शाय॒ ज्योति॒र्दध॑देति सप्तद॒शेन॑ प्राजाप॒त्येन॑ प्र॒जन॑नेनैकवि॒ꣳ॒शाय॒ ज्योति॒र्दध॑देति॒ स्तोम॑ ए॒व तथ्स्तोमा॑य॒ ज्योति॒र्दध॑दे॒त्यथो॒ स्तोम॑ ए॒व स्तोम॑म॒भि प्र ण॑यति॒ याव॑न्तो॒ वै स्तोमा॒स्ताव॑न्तः॒ कामा॒स्ताव॑न्तो लो॒कास्ताव॑न्ति॒ ज्योतीꣴ॑ष्ये॒ताव॑त ए॒व स्तोमा॑ने॒ताव॑तः॒ कामा॑ने॒ताव॑तो लो॒काने॒ताव॑न्ति॒ ज्योती॒ꣳ॒ष्यव॑ रुन्द्धे॥७॥

%7.1.3.0
{\anuvakamend[{ताव॑न्तो लो॒कास्त्रयो॑दश च}]}%॥२॥

%7.1.3.1
ब्र॒ह्म॒वा॒दिनो॑ वदन्ति॒ स त्वै य॑जेत॒ यो᳚\-ऽग्निष्टो॒मेन॒ यज॑मा॒नो\-ऽथ॒ सर्व॑स्तोमेन॒ यजे॒तेति॒ यस्य॑ त्रि॒वृत॑मन्त॒र्यन्ति॑ प्रा॒णाꣴस्तस्या॒न्तर्य॑न्ति प्रा॒णेषु॒ मे\-ऽप्य॑स॒दिति॒ खलु॒ वै य॒ज्ञेन॒ यज॑मानो यजते॒ यस्य॑ पञ्चद॒शम॑न्त॒र्यन्ति॑ वी॒र्यं॑ तस्या॒न्तर्य॑न्ति वी॒र्ये॑ मे\-ऽप्य॑स॒दिति॒ खलु॒ वै य॒ज्ञेन॒ यज॑मानो यजते॒ यस्य॑ सप्तद॒शम॑न्त॒र्यन्ति॑॥८॥

%7.1.3.2
प्र॒जां तस्या॒न्तर्य॑न्ति प्र॒जाया॒म्मे\-ऽप्य॑स॒दिति॒ खलु॒ वै य॒ज्ञेन॒ यज॑मानो यजते॒ यस्यै॑कवि॒ꣳ॒शम॑न्त॒र्यन्ति॑ प्रति॒ष्ठां तस्या॒न्तर्य॑न्ति प्रति॒ष्ठाया॒म्मे\-ऽप्य॑स॒दिति॒ खलु॒ वै य॒ज्ञेन॒ यज॑मानो यजते॒ यस्य॑ त्रिण॒वम॑न्त॒र्यन्त्यृ॒तूꣴश्च॒ तस्य॑ नक्ष॒त्रियां᳚ च वि॒राज॑म॒न्तर्य॑न्त्यृ॒तुषु॒ मे\-ऽप्य॑सन्नक्ष॒त्रिया॑यां च वि॒राजीति॑॥९॥

%7.1.3.3
खलु॒ वै य॒ज्ञेन॒ यज॑मानो यजते॒ यस्य॑ त्रयस्त्रि॒ꣳ॒शम॑न्त॒र्यन्ति॑ दे॒वता॒स्तस्या॒न्तर्य॑न्ति दे॒वता॑सु॒ मे\-ऽप्य॑स॒दिति॒ खलु॒ वै य॒ज्ञेन॒ यज॑मानो यजते॒ यो वै स्तोमा॑नामव॒मं प॑र॒मतां॒ गच्छ॑न्तं॒ वेद॑ पर॒मता॑मे॒व ग॑च्छति त्रि॒वृद्वै स्तोमा॑नामव॒म\-स्त्रि॒वृत्प॑र॒मो य ए॒वं वेद॑ पर॒मता॑मे॒व ग॑च्छति॥१०॥

%7.1.4.0
{\anuvakamend[{स॒प्त॒द॒शम॑न्त॒र्यन्ति॑ वि॒राजीति॒ चतु॑श्चत्वारिꣳशच्च}]}%॥३॥

%7.1.4.1
अङ्गि॑रसो॒ वै स॒त्त्रमा॑सत॒ ते सु॑व॒र्गं लो॒कमा॑य॒न्तेषाꣳ॑ ह॒विष्माꣳ॑श्च हवि॒ष्कृच्चा॑हीयेता॒न्ताव॑कामयेताꣳ सुव॒र्गं लो॒कमि॑या॒वेति॒ तावे॒तं द्वि॑रा॒त्रम॑पश्यता॒न्तमाह॑रता॒न्तेना॑यजेता॒न्ततो॒ वै तौ सु॑व॒र्गं लो॒कमै॑तां॒ य ए॒वं वि॒द्वान्द्वि॑रा॒त्रेण॒ यज॑ते सुव॒र्गमे॒व लो॒कमे॑ति॒ तावैता॒म्पूर्वे॒णा\-ऽह्ना\-ऽग॑च्छता॒मुत्त॑रेण॥११॥

%7.1.4.2
अ॒भि॒प्ल॒वः पूर्व॒मह॑र्भवति॒ गति॒रुत्त॑रं॒ ज्योति॑ष्टोमो\-ऽग्निष्टो॒मः पूर्व॒मह॑र्भवति॒ तेज॒स्तेनाव॑ रुन्द्धे॒ सर्व॑स्तोमो\-ऽतिरा॒त्र उत्त॑र॒ꣳ॒ सर्व॒स्याप्त्यै॒ सर्व॒स्याव॑रुद्ध्यै गाय॒त्रम्पूर्वेह॒न्थ्साम॑ भवति॒ तेजो॒ वै गा॑य॒त्री गा॑य॒त्री ब्र॑ह्मवर्च॒सन्तेज॑ ए॒व ब्र॑ह्मवर्च॒स\-मा॒त्मन्ध॑त्ते॒ त्रैष्टु॑भ॒मुत्त॑र॒ ओजो॒ वै वी॒र्यं॑ त्रि॒ष्टुगोज॑ ए॒व वी॒र्य॑मा॒त्मन्ध॑त्ते रथन्त॒रम्पूर्वे᳚॥१२॥

%7.1.4.3
अह॒न्थ्साम॑ भवती॒यं वै र॑थन्त॒रम॒स्यामे॒व प्रति॑ तिष्ठति बृ॒हदुत्त॑रे॒\-ऽसौ वै बृ॒हद॒मुष्या॑मे॒व प्रति॑ तिष्ठति॒ तदा॑हुः॒ क्व॑ जग॑ती चानु॒ष्टुप्चेति॑ वैखान॒सम्पूर्वे\-ऽह॒न्थ्साम॑ भवति॒ तेन॒ जग॑त्यै॒ नैति॑ षोड॒श्युत्त॑रे॒ तेना॑नु॒ष्टुभो\-ऽथा॑हु॒र्यथ्स॑मा॒ने᳚\-ऽ र्धमा॒से स्याता॑मन्यत॒रस्याह्नो॑ वी॒र्य॑मनु॑ पद्ये॒तेत्य॑मावा॒स्या॑या॒म्पूर्व॒मह॑र्भव॒त्युत्त॑रस्मि॒न्नुत्त॑र॒न्नानै॒वार्ध॑मा॒सयो᳚र्भवतो॒ नाना॑वीर्ये भवतो ह॒विष्म॑न्निधन॒म्पूर्व॒मह॑र्भवति हवि॒ष्कृन्नि॑धन॒मुत्त॑रं॒ प्रति॑ष्ठित्यै॥१३॥

%7.1.5.0
{\anuvakamend[{उत्त॑रेण रथन्त॒रम्पूर्वे\-ऽन्वेक॑विꣳशतिश्च}]}%॥४॥

%7.1.5.1
आपो॒ वा इ॒दमग्रे॑ सलि॒लमा॑सी॒त्तस्मि॑न्प्र॒जाप॑तिर्वा॒युर्भू॒त्वाच॑र॒थ्स इ॒माम॑पश्य॒त्तां व॑रा॒हो भू॒त्वाह॑र॒त्तां वि॒श्वक॑र्मा भू॒त्वा व्य॑मा॒र्ट्थ्साप्र॑थत॒ सा पृ॑थि॒व्य॑भव॒त्तत्पृ॑थि॒व्यै पृ॑थिवि॒त्वन्तस्या॑मश्राम्यत्प्र॒जाप॑तिः॒ स दे॒वान॑सृजत॒ वसू᳚न्रु॒द्राना॑दि॒त्यान्ते दे॒वाः प्र॒जाप॑तिमब्रुव॒न्प्र जा॑यामहा॒ इति॒ सो᳚\-ऽब्रवीत्॥१४॥

%7.1.5.2
यथा॒हं यु॒ष्माꣴस्तप॒सासृ॑क्ष्ये॒वं तप॑सि प्र॒जन॑नमिच्छध्व॒मिति॒ तेभ्यो॒\-ऽग्निमा॒यत॑न॒म्प्राय॑च्छदे॒तेना॒यत॑नेन श्राम्य॒तेति॒ ते᳚\-ऽग्निना॒यत॑नेनाश्राम्य॒न्ते सं॑वथ्स॒र एकां॒ गाम॑सृजन्त॒ तां वसु॑भ्यो रु॒द्रेभ्य॑ आदि॒त्येभ्यः॒ प्राय॑च्छन्ने॒ताꣳ र॑क्षध्व॒मिति॒ तां वस॑वो रु॒द्रा आ॑दि॒त्या अ॑रक्षन्त॒ सा वसु॑भ्यो रु॒द्रेभ्य॑ आदि॒त्येभ्यः॒ प्राजा॑यत॒ त्रीणि॑ च॥१५॥

%7.1.5.3
श॒तानि॒ त्रय॑स्त्रिꣳशतं॒ चाथ॒ सैव स॑हस्रत॒म्य॑भव॒त्ते दे॒वाः प्र॒जाप॑तिमब्रुवन्थ्स॒हस्रे॑ण नो याज॒येति॒ सो᳚\-ऽग्निष्टो॒मेन॒ वसू॑नयाजय॒त्त इ॒मं लो॒कम॑जय॒न्तच्चा॑ददुः॒ स उ॒क्थ्ये॑न रु॒द्रान॑याजय॒त्ते᳚\-ऽन्तरि॑क्षमजय॒न्तच्चा॑ददुः॒ सो॑\-ऽतिरा॒त्रेणा॑\-दि॒त्यान॑याजय॒त्ते॑\-ऽमुं लो॒कम॑जय॒न्तच्चा॑ददु॒स्तद॒न्तरि॑क्षम्॥१६॥

%7.1.5.4
व्यवै᳚र्यत॒ तस्मा᳚द्रु॒द्रा घातु॑का अनायत॒ना हि तस्मा॑दाहुः शिथि॒लं वै म॑ध्य॒ममह॑स्त्रिरा॒त्रस्य॒ वि हि तद॒वैर्य॒तेति॒ त्रैष्टु॑भम्मध्य॒मस्याह्न॒ आज्य॑म्भवति सं॒याना॑नि सू॒क्तानि॑ शꣳसति षोड॒शिनꣳ॑ शꣳस॒त्यह्नो॒ धृत्या॒ अशि॑थिलम्भावाय॒ तस्मा᳚त्त्रिरा॒त्रस्या᳚ग्निष्टो॒म ए॒व प्र॑थ॒ममहः॑ स्या॒दथो॒क्थ्यो\-ऽथा॑तिरा॒त्र ए॒षां लो॒कानां॒ विधृ॑त्यै॒ त्रीणि॑त्रीणि श॒तान्य॑नूचीना॒हमव्य॑वच्छिन्नानि ददाति॥१७॥

%7.1.5.5
ए॒षां लो॒काना॒मनु॒ सन्त॑त्यै द॒शतं॒ न विच्छि॑न्द्याद्वि॒राजं॒ नेद्वि॑च्छि॒नदा॒नीत्यथ॒ या स॑हस्रत॒म्यासी॒त्तस्या॒मिन्द्र॑श्च॒ विष्णु॑श्च॒ व्याय॑च्छेता॒ꣳ॒ स इन्द्रो॑\-ऽमन्यता॒नया॒ वा इ॒दं विष्णुः॑ स॒हस्रं॑ वर्क्ष्यत॒ इति॒ तस्या॑मकल्पेतां॒ द्विभा॑ग॒ इन्द्र॒स्तृती॑ये॒ विष्णु॒स्तद्वा ए॒षाभ्यनू᳚च्यत उ॒भा जि॑ग्यथु॒रिति॒ तां वा ए॒ताम॑च्छावा॒कः॥१८॥

%7.1.5.6
ए॒व शꣳ॑स॒त्यथ॒ या स॑हस्रत॒मी सा होत्रे॒ देयेति॒ होता॑रं॒ वा अ॒भ्यति॑रिच्यते॒ यद॑ति॒रिच्य॑ते॒ होताना᳚प्तस्यापयि॒ता\-था॑हुरुन्ने॒त्रे देयेत्यति॑रिक्ता॒ वा ए॒षा स॒हस्र॒स्याति॑रिक्त उन्ने॒तर्त्विजा॒मथा॑हुः॒ सर्वे᳚भ्यः सद॒स्ये᳚भ्यो॒ देयेत्यथा॑हुरुदा॒कृत्या॒ सा वशं॑ चरे॒दित्यथा॑हुर्ब्र॒ह्मणे॑ चा॒ग्नीधे॑ च॒ देयेति॑॥१९॥

%7.1.5.7
द्विभा॑गम्ब्र॒ह्मणे॒ तृती॑यम॒ग्नीध॑ ऐ॒न्द्रो वै ब्र॒ह्मा वै᳚ष्ण॒वो᳚\-ऽग्नीद्यथै॒व तावक॑ल्पेता॒मित्यथा॑हु॒र्या क॑ल्या॒णी ब॑हुरू॒पा सा देयेत्यथा॑हु॒र्या द्वि॑रू॒पोभ॒यत॑एनी॒ सा देयेति॑ स॒हस्र॑स्य॒ परि॑गृहीत्यै॒ तद्वा ए॒तथ्स॒हस्र॒स्याय॑नꣳ स॒हस्रꣴ॑ स्तो॒त्रीयाः᳚ स॒हस्रं॒ दक्षि॑णाः स॒हस्र॑सम्मितः सुव॒र्गो लो॒कः सु॑व॒र्गस्य॑ लो॒कस्या॒भिजि॑त्यै॥२०॥

%7.1.6.0
{\anuvakamend[{अ॒ब्र॒वी॒च्च॒ तद॒न्तरि॑क्षन्ददात्यच्छावा॒कश्च॒ देयेति॑ स॒प्तच॑त्वारिꣳशच्च}]}%॥५॥

%7.1.6.1
सोमो॒ वै स॒हस्र॑मविन्द॒त्तमिन्द्रो\-ऽन्व॑विन्द॒त्तौ य॒मो न्याग॑च्छ॒त्ताव॑ब्रवी॒दस्तु॒ मे\-ऽत्रापीत्यस्तु॒ ही(३) इत्य॑ब्रूता॒ꣳ॒ स य॒म एक॑स्यां वी॒र्यं॑ पर्य॑पश्यदि॒यं वा अ॒स्य स॒हस्र॑स्य वी॒र्य॑म्बिभ॒र्तीति॒ ताव॑ब्रवीदि॒यम्ममास्त्वे॒तद्यु॒वयो॒रिति॒ ताव॑ब्रूता॒ꣳ॒ सर्वे॒ वा ए॒तदे॒तस्यां᳚ वी॒र्यम्᳚॥२१॥

%7.1.6.2
परि॑ पश्या॒मो\-ऽꣳश॒मा ह॑रामहा॒ इति॒ तस्या॒मꣳश॒माह॑रन्त॒ ताम॒फ्सु प्रावे॑शय॒न्थ्सोमा॑यो॒देहीति॒ सा रोहि॑णी पिङ्ग॒लैक॑हायनी रू॒पं कृ॒त्वा त्रय॑स्त्रिꣳशता च त्रि॒भिश्च॑ श॒तैः स॒होदैत्तस्मा॒द्रोहि॑ण्या पिङ्ग॒लयैक॑हायन्या॒ सोमं॑ क्रीणीया॒द्य ए॒वं वि॒द्वान्रोहि॑ण्या पिङ्ग॒लयैक॑हायन्या॒ सोमं॑ क्री॒णाति॒ त्रय॑स्त्रिꣳशता चै॒वास्य॑ त्रि॒भिश्च॑॥२२॥

%7.1.6.3
श॒तैः सोमः॑ क्री॒तो भ॑वति॒ सुक्री॑तेन यजते॒ ताम॒फ्सु प्रावे॑शय॒न्निन्द्रा॑यो॒देहीति॒ सा रोहि॑णी लक्ष्म॒णा प॑ष्ठौ॒ही वार्त्र॑घ्नी रू॒पं कृ॒त्वा त्रय॑स्त्रिꣳशता च त्रि॒भिश्च॑ श॒तैः स॒होदैत्तस्मा॒द्रोहि॑णीं लक्ष्म॒णाम्प॑ष्ठौ॒हीं वार्त्र॑घ्नीं दद्या॒द्य ए॒वं वि॒द्वान्रोहि॑णीं लक्ष्म॒णाम्प॑ष्ठौ॒हीं वार्त्र॑घ्नीं॒ ददा॑ति॒ त्रय॑स्त्रिꣳशच्चै॒वास्य॒ त्रीणि॑ च श॒तानि॒ सा द॒त्ता॥२३॥

%7.1.6.4
भ॒व॒ति॒ ताम॒फ्सु प्रावे॑शयन् य॒मायो॒देहीति॒ सा जर॑ती मू॒र्खा त॑ज्जघ॒न्या रू॒पं कृ॒त्वा त्रय॑स्त्रिꣳशता च त्रि॒भिश्च॑ श॒तैः स॒होदैत्तस्मा॒ज्जर॑तीम्मू॒र्खां त॑ज्जघ॒न्याम॑नु॒स्तर॑णीं कुर्वीत॒ य ए॒वं वि॒द्वाञ्जर॑तीम्मू॒र्खां त॑ज्जघ॒न्याम॑नु॒स्तर॑णीं कुरु॒ते त्रय॑स्त्रिꣳशच्चै॒वास्य॒ त्रीणि॑ च श॒तानि॒ सामुष्मि॑ल्लोँ॒के भ॑वति॒ वागे॒व स॑हस्रत॒मी तस्मा᳚त्॥२४॥

%7.1.6.5
वरो॒ देयः॒ सा हि वरः॑ स॒हस्र॑मस्य॒ सा द॒त्ता भ॑वति॒ तस्मा॒द्वरो॒ न प्र॑ति॒गृह्यः॒ सा हि वरः॑ स॒हस्र॑मस्य॒ प्रति॑गृहीत\-म्भवती॒यं वर॒ इति॑ ब्रूया॒दथा॒न्याम्ब्रू॑यादि॒यम्ममेति॒ तथा᳚स्य॒ तथ्स॒हस्र॒मप्र॑तिगृहीतम्भवत्युभयतए॒नी स्या॒त्तदा॑हुरन्यत\-ए॒नी स्या᳚थ्स॒हस्र॑म्प॒रस्ता॒देत॒मिति॒ यैव वरः॑॥२५॥

%7.1.6.6
क॒ल्या॒णी रू॒पस॑मृद्धा॒ सा स्या॒थ्सा हि वरः॒ समृ॑द्ध्यै॒ तामुत्त॑रे॒णाग्नी᳚ध्रं पर्या॒णीया॑हव॒नीय॒स्यान्ते᳚ द्रोणकल॒शमव॑ घ्रापये॒दा जि॑घ्र क॒लश॑म्मह्यु॒रुधा॑रा॒ पय॑स्व॒त्या त्वा॑ विश॒न्त्विन्द॑वः समु॒द्रमि॑व॒ सिन्ध॑वः॒ सा मा॑ स॒हस्र॒ आ भ॑ज प्र॒जया॑ प॒शुभिः॑ स॒ह पुन॒र्मा वि॑शताद्र॒यिरिति॑ प्र॒जयै॒वैन॑म्प॒शुभी॑ र॒य्या सम्॥२६॥

%7.1.6.7
अ॒र्ध॒य॒ति॒ प्र॒जावा᳚न्पशु॒मान्र॑यि॒मान्भ॑वति॒ य ए॒वं वेद॒ तया॑ स॒हाग्नी᳚ध्रम्प॒रेत्य॑ पु॒रस्ता᳚त्प्र॒तीच्यां॒ तिष्ठ॑न्त्यां जुहुयादु॒भा जि॑ग्यथु॒र्न परा॑ जयेथे॒ न परा॑ जिग्ये कत॒रश्च॒नैनोः᳚। इन्द्र॑श्च विष्णो॒ यदप॑स्पृधेथां त्रे॒धा स॒हस्रं॒ वि तदै॑रयेथा॒मिति॑ त्रेधाविभ॒क्तं वै त्रि॑रा॒त्रे स॒हस्रꣳ॑ साह॒स्रीमे॒वैनां᳚ करोति स॒हस्र॑स्यै॒वैना॒म्मात्रा᳚म्॥२७॥

%7.1.6.8
क॒रो॒ति॒ रू॒पाणि॑ जुहोति रू॒पैरे॒वैना॒ꣳ॒ सम॑र्धयति॒ तस्या॑ उपो॒त्थाय॒ कर्ण॒मा ज॑पे॒दिडे॒ रन्ते\-ऽदि॑ते॒ सर॑स्वति॒ प्रिये॒ प्रेय॑सि॒ महि॒ विश्रु॑त्ये॒तानि॑ ते अघ्निये॒ नामा॑नि सु॒कृतं॑ मा दे॒वेषु॑ ब्रूता॒दिति॑ दे॒वेभ्य॑ ए॒वैन॒मा वे॑दय॒त्यन्वे॑नं दे॒वा बु॑ध्यन्ते॥२८॥

%7.1.7.0
{\anuvakamend[{ए॒तदे॒तस्यां᳚ वी॒र्य॑मस्य त्रि॒भिश्च॑ द॒त्ता स॑हस्रत॒मी तस्मा॑दे॒व वरः॒ सम्मात्रा॒मेका॒न्नच॑त्वारि॒ꣳ॒शच्च॑}]}%॥६॥

%7.1.7.1
स॒ह॒स्र॒त॒म्या॑ वै यज॑मानः सुव॒र्गं लो॒कमे॑ति॒ सैनꣳ॑ सुव॒र्गं लो॒कं ग॑मयति॒ सा मा॑ सुव॒र्गं लो॒कं ग॑म॒येत्या॑ह सुव॒र्गमे॒वैनं॑ लो॒कं ग॑मयति॒ सा मा॒ ज्योति॑ष्मन्तं लो॒कं ग॑म॒येत्या॑ह॒ ज्योति॑ष्मन्तमे॒वैनं॑ लो॒कं ग॑मय॒ति सा मा॒ सर्वा॒न्पुण्या᳚\-ल्लोँ॒कान्ग॑म॒येत्या॑ह॒ सर्वा॑ने॒वैन॒म्पुण्या᳚ल्लोँ॒कान्ग॑मयति॒ सा॥२९॥

%7.1.7.2
मा॒ प्र॒ति॒ष्ठां ग॑मय प्र॒जया॑ प॒शुभिः॑ स॒ह पुन॒र्मा वि॑शताद्र॒यिरिति॑ प्र॒जयै॒वैन॑म्प॒शुभी॑ र॒य्यां प्रति॑ ष्ठापयति प्र॒जावा᳚न्पशु॒मान्र॑यि॒मान्भ॑वति॒ य ए॒वं वेद॒ ताम॒ग्नीधे॑ वा ब्र॒ह्मणे॑ वा॒ होत्रे॑ वोद्गा॒त्रे वा᳚ध्व॒र्यवे॑ वा दद्याथ्स॒हस्र॑मस्य॒ सा द॒त्ता भ॑वति स॒हस्र॑मस्य॒ प्रति॑गृहीतम्भवति॒ यस्तामवि॑द्वान्॥३०॥

%7.1.7.3
प्र॒ति॒गृ॒ह्णाति॒ तां प्रति॑ गृह्णीया॒देका॑सि॒ न स॒हस्र॒मेकां᳚ त्वा भू॒तां प्रति॑ गृह्णामि॒ न स॒हस्र॒मेका॑ मा भू॒ता वि॑श॒ मा स॒हस्र॒मित्येका॑मे॒वैनां᳚ भू॒तां प्रति॑ गृह्णाति॒ न स॒हस्रं॒ य ए॒वं वेद॑ स्यो॒नासि॑ सु॒षदा॑ सु॒शेवा᳚ स्यो॒ना मा वि॑श सु॒षदा॒ मा वि॑श सु॒शेवा॒ मा वि॑श॥३१॥

%7.1.7.4
इत्या॑ह स्यो॒नैवैनꣳ॑ सु॒षदा॑ सु॒शेवा॑ भू॒ता वि॑शति॒ नैनꣳ॑ हिनस्ति ब्रह्मवा॒दिनो॑ वदन्ति स॒हस्र॑ꣳ सहस्रत॒म्यन्वे॒ती(३) स॑हस्रत॒मीꣳ स॒हस्रा(३)मिति॒ यत्प्राची॑मुथ्सृ॒जेथ्स॒हस्रꣳ॑ सहस्रत॒म्यन्वि॑या॒त्तथ्स॒हस्र॑मप्रज्ञा॒त्रꣳ सु॑व॒र्गं लो॒कं न प्र जा॑नीयात्प्र॒तीची॒मुथ्सृ॑जति॒ ताꣳ स॒हस्र॒मनु॑ प॒र्याव॑र्तते॒ सा प्र॑जान॒ती सु॑व॒र्गं लो॒कमे॑ति॒ यज॑मानम॒भ्युथ्सृ॑जति क्षि॒प्रे स॒हस्र॒म्प्र जा॑यत उत्त॒मा नी॒यते᳚ प्रथ॒मा दे॒वान्ग॑च्छति॥३२॥

%7.1.8.0
{\anuvakamend[{लो॒कान्ग॑मयति॒ सावि॑द्वान्थ्सु॒शेवा॒ मावि॑श॒ यज॑मान॒न्द्वाद॑श च}]}%॥७॥

%7.1.8.1
अत्रि॑रददा॒दौर्वा॑य प्र॒जाम्पु॒त्रका॑माय॒ स रि॑रिचा॒नो॑\-ऽमन्यत॒ निर्वी᳚र्यः शिथि॒लो या॒तया॑मा॒ स ए॒तं च॑तूरा॒त्रम॑पश्य॒त् तमाह॑र॒त्तेना॑यजत॒ ततो॒ वै तस्य॑ च॒त्वारो॑ वी॒रा आजा॑यन्त॒ सुहो॑ता॒ सू᳚द्गाता॒ स्व॑ध्वर्युः॒ सुस॑भेयो॒ य ए॒वं वि॒द्वाꣴश्च॑तूरा॒त्रेण॒ यज॑त॒ आस्य॑ च॒त्वारो॑ वी॒रा जा॑यन्ते॒ सुहो॑ता॒ सू᳚द्गाता॒ स्व॑ध्वर्युः॒ सुस॑भेयो॒ ये च॑तुर्वि॒ꣳ॒शाः पव॑माना ब्रह्मवर्च॒सं तत्॥३३॥

%7.1.8.2
य उ॒द्यन्तः॒ स्तोमाः॒ श्रीः सात्रिꣴ॑ श्र॒द्धादे॑वं॒ यज॑मानं च॒त्वारि॑ वीर्याणि॒ नोपा॑नम॒न्तेज॑ इन्द्रि॒यम्ब्र॑ह्मवर्च॒सम॒न्नाद्य॒ꣳ॒ स ए॒ताꣴश्च॒तुर॒श्चतु॑ष्टोमा॒न्थ्सोमा॑नपश्य॒त्तानाह॑र॒त्तैर॑यजत॒ तेज॑ ए॒व प्र॑थ॒मेनावा॑रुन्द्धेन्द्रि॒यं द्वि॒तीये॑न ब्रह्मवर्च॒सं तृ॒तीये॑ना॒न्नाद्यं॑ चतु॒र्थेन॒ य ए॒वं वि॒द्वाꣴश्च॒तुर॒श्चतु॑ष्टोमा॒न्थ्सोमा॑ना॒हर॑ति॒ तैर्यज॑ते॒ तेज॑ ए॒व प्र॑थ॒मेनाव॑ रुन्द्ध इन्द्रि॒यं द्वि॒तीये॑न ब्रह्मवर्च॒सं तृ॒तीये॑ना॒न्नाद्यं॑ चतु॒र्थेन॒ यामे॒वात्रि॒र्\mbox{}ऋद्धि॒मार्ध्नो॒त्तामे॒व यज॑मान ऋध्नोति॥३४॥

%7.1.9.0
{\anuvakamend[{तत्तेज॑ ए॒वाष्टाद॑श च}]}%॥८॥

%7.1.9.1
ज॒मद॑ग्निः॒ पुष्टि॑कामश्चतूरा॒त्रेणा॑यजत॒ स ए॒तान्पोषाꣳ॑ अपुष्य॒त्तस्मा᳚त्पलि॒तौ जाम॑दग्नियौ॒ न सं जा॑नाते ए॒ताने॒व पोषा᳚न्पुष्यति॒ य ए॒वं वि॒द्वाꣴश्च॑तूरा॒त्रेण॒ यज॑ते पुरोडा॒शिन्य॑ उप॒सदो॑ भवन्ति प॒शवो॒ वै पु॑रो॒डाशः॑ प॒शूने॒वाव॑ रु॒न्द्धे\-ऽन्नं॒ वै पु॑रो॒डाशो\-ऽन्न॑मे॒वाव॑ रुन्द्धे\-ऽन्ना॒दः प॑शु॒मान्भ॑वति॒ य ए॒वं वि॒द्वाꣴश्च॑तूरा॒त्रेण॒ यज॑ते॥३५॥

%7.1.10.0
{\anuvakamend[{ज॒मद॑ग्निर॒ष्टाच॑त्वारिꣳशत्}]}%॥९॥

%7.1.10.1
सं॒व॒थ्स॒रो वा इ॒दमेक॑ आसी॒थ्सो॑\-ऽकामयत॒र्तून्थ्सृ॑जे॒येति॒ स ए॒तम्प॑ञ्चरा॒त्रम॑पश्य॒त्तमाह॑र॒त्तेना॑यजत॒ ततो॒ वै स ऋ॒तून॑सृजत॒ य ए॒वं वि॒द्वान्प॑ञ्चरा॒त्रेण॒ यज॑ते॒ प्रैव जा॑यते॒ त ऋ॒तवः॑ सृ॒ष्टा न व्याव॑र्तन्त॒ त ए॒तम्प॑ञ्चरा॒त्रम॑पश्य॒न् तमाह॑र॒न्तेना॑यजन्त॒ ततो॒ वै ते व्याव॑र्तन्त॥३६॥

%7.1.10.2
य ए॒वं वि॒द्वान्प॑ञ्चरा॒त्रेण॒ यज॑ते॒ वि पा॒प्मना॒ भ्रातृ॑व्ये॒णा व॑र्तते॒ सार्व॑सेनिः शौचे॒यो॑\-ऽकामयत पशु॒मान्थ्स्या॒मिति॒ स ए॒तम्प॑ञ्चरा॒त्रमाह॑र॒त्तेना॑यजत॒ ततो॒ वै स स॒हस्रं॑ प॒शून्प्राप्नो॒द्य ए॒वं वि॒द्वान्प॑ञ्चरा॒त्रेण॒ यज॑ते॒ प्र स॒हस्रं॑ प॒शूना᳚प्नोति बब॒रः प्रावा॑हणिरकामयत वा॒चः प्र॑वदि॒ता स्या॒मिति॒ स ए॒तम्प॑ञ्चरा॒त्रमा॥३७॥

%7.1.10.3
अ॒ह॒र॒त्तेना॑यजत॒ ततो॒ वै स वा॒चः प्र॑वदि॒ताभ॑व॒द्य ए॒वं वि॒द्वान्प॑ञ्चरा॒त्रेण॒ यज॑ते प्रवदि॒तैव वा॒चो भ॑व॒त्यथो॑ एनं वा॒चस्पति॒रित्या॑हु॒रना᳚प्तश्चतूरा॒त्रो\-ऽति॑रिक्तः षड्रा॒त्रो\-ऽथ॒ वा ए॒ष स॑म्प्र॒ति य॒ज्ञो यत्प॑ञ्चरा॒त्रो य ए॒वं वि॒द्वान्प॑ञ्चरा॒त्रेण॒ यज॑ते सम्प्र॒त्ये॑व य॒ज्ञेन॑ यजते पञ्चरा॒त्रो भ॑वति॒ पञ्च॒ वा ऋ॒तवः॑ संवथ्स॒रः॥३८॥

%7.1.10.4
ऋ॒तुष्वे॒व सं॑ वथ्स॒रे प्रति॑ तिष्ठ॒त्यथो॒ पञ्चा᳚क्षरा प॒ङ्क्तिः पाङ्क्तो॑ य॒ज्ञो य॒ज्ञमे॒वाव॑ रुन्द्धे त्रि॒वृद॑ग्निष्टो॒मो भ॑वति॒ तेज॑ ए॒वाव॑ रुन्द्धे पञ्चद॒शो भ॑वतीन्द्रि॒यमे॒वाव॑ रुन्द्धे सप्तद॒शो भ॑वत्य॒न्नाद्य॒स्याव॑रुद्ध्या॒ अथो॒ प्रैव तेन॑ जायते पञ्चवि॒ꣳ॒शो᳚\-ऽग्निष्टो॒मो भ॑वति प्र॒जाप॑ते॒राप्त्यै॑ महाव्र॒तवा॑न॒न्नाद्य॒स्याव॑रुद्ध्यै विश्व॒जिथ्सर्व॑पृष्ठो\-ऽतिरा॒त्रो भ॑वति॒ सर्व॑स्या॒भिजि॑त्यै॥३९॥

%7.1.11.0
{\anuvakamend[{ते व्याव॑र्तन्त प्रवदि॒ता स्या॒मिति॒ स ए॒तम्प़॑ञ्चरा॒त्रमा सं॑ वथ्स॒रो॑\-ऽभिजि॑त्यै}]}%॥10॥

%7.1.11.1
दे॒वस्य॑ त्वा सवि॒तुः प्र॑स॒वे᳚\-ऽश्विनो᳚र्बा॒हु\-भ्यां᳚ पू॒ष्णो हस्ता᳚भ्या॒मा द॑द इ॒माम॑गृभ्णन्रश॒नामृ॒तस्य॒ पूर्व॒ आयु॑षि वि॒दथे॑षु क॒व्या। तया॑ दे॒वाः सु॒तमा ब॑भूवुर्\mbox{}ऋ॒तस्य॒ साम᳚न्थ्स॒रमा॒रप॑न्ती। अ॒भि॒धा अ॑सि॒ भुव॑नमसि य॒न्तासि॑ ध॒र्तासि॒ सो᳚\-ऽग्निं वै᳚श्वान॒रꣳ सप्र॑थसं गच्छ॒ स्वाहा॑कृतः पृथि॒व्यां य॒न्ता राड्य॒न्तासि॒ यम॑नो ध॒र्तासि॑ ध॒रुणः॑ कृ॒ष्यै त्वा॒ क्षेमा॑य त्वा र॒य्यै त्वा॒ पोषा॑य त्वा पृथि॒व्यै त्वा॒न्तरि॑क्षाय त्वा दि॒वे त्वा॑ स॒ते त्वास॑ते त्वा॒द्भ्यस्त्वौष॑धीभ्यस्त्वा॒ विश्वे᳚भ्यस्त्वा भू॒तेभ्यः॑॥४०॥

%7.1.12.0
{\anuvakamend[{ध॒रुणः॒ प़ञ्च॑विꣳशतिश्च}]}%॥11॥

%7.1.12.1
वि॒भूर्मा॒त्रा प्र॒भूः पि॒त्राश्वो॑\-ऽसि॒ हयो॒\-ऽस्यत्यो॑\-ऽसि॒ नरो॒\-ऽस्यर्वा॑सि॒ सप्ति॑रसि वा॒ज्य॑सि॒ वृषा॑सि नृ॒मणा॑ असि॒ ययु॒र्नामा᳚स्यादि॒त्याना॒म्पत्वान्वि॑ह्य॒ग्नये॒ स्वाहा॒ स्वाहे᳚न्द्रा॒ग्निभ्या॒ꣴ॒ स्वाहा᳚ प्र॒जाप॑तये॒ स्वाहा॒ विश्वे᳚भ्यो दे॒वेभ्यः॒ स्वाहा॒ सर्वा᳚भ्यो दे॒वेता᳚भ्य इ॒ह धृतिः॒ स्वाहे॒ह विधृ॑तिः॒ स्वाहे॒ह रन्तिः॒ स्वाहे॒ह रम॑तिः॒ स्वाहा॒ भूर॑सि भु॒वे त्वा॒ भव्या॑य त्वा भविष्य॒ते त्वा॒ विश्वे᳚भ्यस्त्वा भू॒तेभ्यो॒ देवा॑ आशापाला ए॒तं दे॒वेभ्यो\-ऽश्व॒म्मेधा॑य॒ प्रोक्षि॑तं गोपायत॥४१॥

%7.1.13.0
{\anuvakamend[{रन्तिः॒ स्वाहा॒ द्वाविꣳ॑शतिश्च}]}%॥12॥

%7.1.13.1
आय॑नाय॒ स्वाहा॒ प्राय॑णाय॒ स्वाहो᳚द्द्रा॒वाय॒ स्वाहोद्द्रु॑ताय॒ स्वाहा॑ शूका॒राय॒ स्वाहा॒ शूकृ॑ताय॒ स्वाहा॒ पला॑यिताय॒ स्वाहा॒\-ऽ\-ऽपला॑यिताय॒ स्वाहा॒\-ऽ\-ऽवल्ग॑ते॒ स्वाहा॑ परा॒वल्ग॑ते॒ स्वाहा॑\-ऽ\-ऽय॒ते स्वाहा᳚ प्रय॒ते स्वाहा॒ सर्व॑स्मै॒ स्वाहा᳚॥४२॥

%7.1.14.0
{\anuvakamend[{आय॑ना॒योत्त॑रमा॒पला॑यिताय॒ षड्विꣳ॑शतिः}]}%॥13॥

%7.1.14.1
अ॒ग्नये॒ स्वाहा॒ सोमा॑य॒ स्वाहा॑ वा॒यवे॒ स्वाहा॒पाम्मोदा॑य॒ स्वाहा॑ सवि॒त्रे स्वाहा॒ सर॑स्वत्यै॒ स्वाहेन्द्रा॑य॒ स्वाहा॒ बृह॒स्पत॑ये॒ स्वाहा॑ मि॒त्राय॒ स्वाहा॒ वरु॑णाय॒ स्वाहा॒ सर्व॑स्मै॒ स्वाहा᳚॥४३॥

%7.1.15.0
{\anuvakamend[{}]}

%7.1.15.1
पृ॒थि॒व्यै स्वाहा॒\-ऽन्तरि॑क्षाय॒ स्वाहा॑ दि॒वे स्वाहा॒ सूर्या॑य॒ स्वाहा॑ च॒न्द्रम॑से॒ स्वाहा॒ नक्ष॑त्रेभ्यः॒ स्वाहा॒ प्राच्यै॑ दि॒शे स्वाहा॒ दक्षि॑णायै दि॒शे स्वाहा᳚ प्र॒तीच्यै॑ दि॒शे स्वाहोदी᳚च्यै दि॒शे स्वाहो॒र्ध्वायै॑ दि॒शे स्वाहा॑ दि॒ग्भ्यः स्वाहा॑\-ऽ\-वान्तरदि॒शाभ्यः॒ स्वाहा॒ समा᳚भ्यः॒ स्वाहा॑ श॒रद्भ्यः॒ स्वाहा॑\-ऽहोरा॒त्रेभ्यः॒ स्वाहा᳚\-ऽर्धमा॒सेभ्यः॒ स्वाहा॒ मासे᳚भ्यः॒ स्वाह॒र्तुभ्यः॒ स्वाहा॑ संवथ्स॒राय॒ स्वाहा॒ सर्व॑स्मै॒ स्वाहा᳚॥४४॥

%7.1.16.0
{\anuvakamend[{}]}

%7.1.16.1
अ॒ग्नये॒ स्वाहा॒ सोमा॑य॒ स्वाहा॑ सवि॒त्रे स्वाहा॒ सर॑स्वत्यै॒ स्वाहा॑ पू॒ष्णे स्वाहा॒ बृह॒स्पत॑ये॒ स्वाहा॒\-ऽपाम्मोदा॑य॒ स्वाहा॑ वा॒यवे॒ स्वाहा॑ मि॒त्राय॒ स्वाहा॒ वरु॑णाय॒ स्वाहा॒ सर्व॑स्मै॒ स्वाहा᳚॥४५॥

%7.1.17.0
{\anuvakamend[{}]}

%7.1.17.1
पृ॒थि॒व्यै स्वाहा॒\-ऽन्तरि॑क्षाय॒ स्वाहा॑ दि॒वे स्वाहा॒\-ऽग्नये॒ स्वाहा॒ सोमा॑य॒ स्वाहा॒ सूर्या॑य॒ स्वाहा॑ च॒न्द्रम॑से॒ स्वाहा\-ऽह्ने॒ स्वाहा॒ रात्रि॑यै॒ स्वाह॒र्जवे॒ स्वाहा॑ सा॒धवे॒ स्वाहा॑ सुक्षि॒त्यै स्वाहा᳚ क्षु॒धे स्वाहा॑\-ऽ\-ऽशिति॒म्ने स्वाहा॒ रोगा॑य॒ स्वाहा॑ हि॒माय॒ स्वाहा॑ शी॒ताय॒ स्वाहा॑\-ऽ\-ऽत॒पाय॒ स्वाहा\-ऽर॑ण्याय॒ स्वाहा॑ सुव॒र्गाय॒ स्वाहा॑ लो॒काय॒ स्वाहा॒ सर्व॑स्मै॒ स्वाहा᳚॥४६॥

%7.1.18.0
{\anuvakamend[{}]}

%7.1.18.1
भुवो॑ दे॒वानां॒ कर्म॑णा॒पस॒र्तस्य॑ प॒थ्या॑सि॒ वसु॑भिर्दे॒वेभि॑र्दे॒वत॑या गाय॒त्रेण॑ त्वा॒ छन्द॑सा युनज्मि वस॒न्तेन॑ त्व॒र्तुना॑ ह॒विषा॑ दीक्षयामि रु॒द्रेभि॑र्दे॒वेभि॑र्दे॒वत॑या॒ त्रैष्टु॑भेन त्वा॒ छन्द॑सा युनज्मि ग्री॒ष्मेण॑ त्व॒र्तुना॑ ह॒विषा॑ दीक्षयाम्यादि॒त्येभि॑\-र्दे॒वेभि॑र्दे॒वत॑या॒ जाग॑तेन त्वा॒ छन्द॑सा युनज्मि व॒र्\mbox{}षाभि॑स्त्व॒र्तुना॑ ह॒विषा॑ दीक्षयामि॒ विश्वे॑भिर्दे॒वेभि॑र्दे॒वत॒यानु॑ष्टुभेन त्वा॒ छन्द॑सा युनज्मि॥४७॥

%7.1.18.2
श॒रदा᳚ त्व॒र्तुना॑ ह॒विषा॑ दीक्षया॒म्यङ्गि॑रोभिर्दे॒वेभि॑र्दे॒वत॑या॒ पाङ्क्ते॑न त्वा॒ छन्द॑सा युनज्मि हेमन्तशिशि॒रा\-भ्यां᳚ त्व॒र्तुना॑ ह॒विषा॑ दीक्षया॒म्याहं दी॒क्षाम॑रुहमृ॒तस्य॒ पत्नीं᳚ गाय॒त्रेण॒ छन्द॑सा॒ ब्रह्म॑णा च॒र्तꣳ स॒त्ये॑\-ऽधाꣳ स॒त्यमृ॒ते॑\-ऽधाम्। म॒हीमू॒ षु सु॒त्रामा॑णमि॒ह धृतिः॒ स्वाहे॒ह विधृ॑तिः॒ स्वाहे॒ह रन्तिः॒ स्वाहे॒ह रम॑तिः॒ स्वाहा᳚॥४८॥

%7.1.19.0
{\anuvakamend[{}]}

%7.1.19.1
ई॒ङ्का॒राय॒ स्वाहें कृ॑ताय॒ स्वाहा॒ क्रन्द॑ते॒ स्वाहा॑\-ऽव॒क्रन्द॑ते॒ स्वाहा॒ प्रोथ॑ते॒ स्वाहा᳚ प्र॒प्रोथ॑ते॒ स्वाहा॑ ग॒न्धाय॒ स्वाहा᳚ घ्रा॒ताय॒ स्वाहा᳚ प्रा॒णाय॒ स्वाहा᳚ व्या॒नाय॒ स्वाहा॑\-ऽपा॒नाय॒ स्वाहा॑ सन्दी॒यमा॑नाय॒ स्वाहा॒ सन्दि॑ताय॒ स्वाहा॑ विचृ॒त्यमा॑नाय॒ स्वाहा॒ विचृ॑त्ताय॒ स्वाहा॑ पलायि॒ष्यमा॑णाय॒ स्वाहा॒ पला॑यिताय॒ स्वाहो॑परꣴस्य॒ते स्वाहोप॑रताय॒ स्वाहा॑ निवेक्ष्य॒ते स्वाहा॑ निवि॒शमा॑नाय॒ स्वाहा॒ निवि॑ष्टाय॒ स्वाहा॑ निषथ्स्य॒ते स्वाहा॑ नि॒षीद॑ते॒ स्वाहा॒ निष॑ण्णाय॒ स्वाहा॑॥४९॥



%7.1.19.2
आ॒सि॒ष्य॒ते स्वाहा\-ऽ\-ऽसी॑नाय॒ स्वाहा॑\-ऽ\-ऽसि॒ताय॒ स्वाहा॑ निपथ्स्य॒ते स्वाहा॑ नि॒पद्य॑मानाय॒ स्वाहा॒ निप॑न्नाय॒ स्वाहा॑ शयिष्य॒ते स्वाहा॒ शया॑नाय॒ स्वाहा॑ शयि॒ताय॒ स्वाहा॑ सम्मीलिष्य॒ते स्वाहा॑ स॒म्मील॑ते॒ स्वाहा॒ सम्मी॑लिताय॒ स्वाहा᳚ स्वफ्स्य॒ते स्वाहा᳚ स्वप॒ते स्वाहा॑ सु॒प्ताय॒ स्वाहा᳚ प्रभोथ्स्य॒ते स्वाहा᳚ प्र॒बुध्य॑मानाय॒ स्वाहा॒ प्रबु॑द्धाय॒ स्वाहा॑ जागरिष्य॒ते स्वाहा॒ जाग्र॑ते॒ स्वाहा॑ जागरि॒ताय॒ स्वाहा॒ शुश्रू॑षमाणाय॒ स्वाहा॑ शृण्व॒ते स्वाहा᳚ श्रु॒ताय॒ स्वाहा॑ वीक्षिष्य॒ते स्वाहा᳚॥५०॥

%7.1.19.3
वीक्ष॑माणाय॒ स्वाहा॒ वीक्षि॑ताय॒ स्वाहा॑ सꣳहास्य॒ते स्वाहा॑ स॒ञ्जिहा॑नाय॒ स्वाहो॒ज्जिहा॑नाय॒ स्वाहा॑ विवर्थ्स्य॒ते स्वाहा॑ वि॒वर्त॑मानाय॒ स्वाहा॒ विवृ॑त्ताय॒ स्वाहो᳚त्थास्य॒ते स्वाहो॒त्तिष्ठ॑ते॒ स्वाहोत्थि॑ताय॒ स्वाहा॑ विधविष्य॒ते स्वाहा॑ विधून्वा॒नाय॒ स्वाहा॒ विधू॑ताय॒ स्वाहो᳚त्क्रꣴस्य॒ते स्वाहो॒त्क्राम॑ते॒ स्वाहोत्क्रा᳚न्ताय॒ स्वाहा॑ चङ्क्रमिष्य॒ते स्वाहा॑ चङ्क्र॒म्यमा॑णाय॒ स्वाहा॑ चङ्क्रमि॒ताय॒ स्वाहा॑ कण्डूयिष्य॒ते स्वाहा॑ कण्डू॒यमा॑नाय॒ स्वाहा॑ कण्डूयि॒ताय॒ स्वाहा॑ निकषिष्य॒ते स्वाहा॑ नि॒कष॑माणाय॒ स्वाहा॒ निक॑षिताय॒ स्वाहा॒ यदत्ति॒ तस्मै॒ स्वाहा॒ यत्पिब॑ति॒ तस्मै॒ स्वाहा॒ यन्मेह॑ति॒ तस्मै॒ स्वाहा॒ यच्छकृ॑त्क॒रोति॒ तस्मै॒ स्वाहा॒ रेत॑से॒ स्वाहा᳚ प्र॒जाभ्यः॒ स्वाहा᳚ प्रजन॑नाय॒ स्वाहा॒ सर्व॑स्मै॒ स्वाहा᳚॥५१॥

%7.1.20.0
{\anuvakamend[{}]}

%7.1.20.1
अ॒ग्नये॒ स्वाहा॑ वा॒यवे॒ स्वाहा॒ सूर्या॑य॒ स्वाह॒र्तम॑स्यृ॒तस्य॒र्तम॑सि स॒त्यम॑सि स॒त्यस्य॑ स॒त्यम॑स्यृ॒तस्य॒ पन्था॑ असि दे॒वानां᳚ छा॒यामृ॑तस्य॒ नाम॒ तथ्स॒त्यं यत्त्वं प्र॒जाप॑ति॒रस्यधि॒ यद॑स्मिन्वा॒जिनी॑व॒ शुभः॒ स्पर्ध॑न्ते॒ दिवः॒ सूर्ये॑ण॒ विशो॒\-ऽपो वृ॑णा॒नः प॑वते क॒व्यन्प॒शुं न गो॒पा इर्यः॒ परि॑ज्मा॥५२॥

%7.2.0.0

%7.2.0.0
{\anuvakamend[{}]}

%%% END PRASHNA

\sect{द्वितीयः प्रश्नः}\setcounter{anuvakam}{0}
\dnsub{तैत्तिरीयसंहितायां सप्तमकाण्डे द्वितीयः प्रश्नः}
%7.2.1.0
%7.2.1.1
सा॒ध्या वै दे॒वाः सु॑व॒र्गका॑मा ए॒तꣳ ष॑ड्रा॒त्रम॑पश्य॒न्तमाह॑र॒न्तेना॑यजन्त॒ ततो॒ वै ते सु॑व॒र्गं लो॒कमा॑य॒न् य ए॒वं वि॒द्वाꣳसः॑ षड्रा॒त्रमास॑ते सुव॒र्गमे॒व लो॒कं य॑न्ति देवस॒त्त्रं वै ष॑ड्रा॒त्रः प्र॒त्यक्ष॒ꣴ॒ ह्ये॑तानि॑ पृ॒ष्ठानि॒ य ए॒वं वि॒द्वाꣳसः॑ षड्रा॒त्रमास॑ते सा॒क्षादे॒व दे॒वता॑ अ॒भ्यारो॑हन्ति षड्रा॒त्रो भ॑वति॒ षड्वा ऋ॒तवः॒ षट्पृ॒ष्ठानि॑॥१॥

%7.2.1.2
पृ॒ष्ठैरे॒वर्तून॒न्वारो॑हन्त्यृ॒तुभिः॑ संवथ्स॒रन्ते सं॑वथ्स॒र ए॒व प्रति॑ तिष्ठन्ति बृहद्रथन्त॒रा\-भ्यां᳚ यन्ती॒यं वाव र॑थन्त॒रम॒सौ बृ॒हदा॒भ्यामे॒व य॒न्त्यथो॑ अ॒नयो॑रे॒व प्रति॑ तिष्ठन्त्ये॒ते वै य॒ज्ञस्या᳚ञ्ज॒साय॑नी स्रु॒ती ताभ्या॑मे॒व सु॑व॒र्गं लो॒कं य॑न्ति त्रि॒वृद॑ग्निष्टो॒मो भ॑वति॒ तेज॑ ए॒वाव॑ रुन्धते पञ्चद॒शो भ॑वतीन्द्रि॒यमे॒वाव॑ रुन्धते सप्तद॒शः॥२॥

%7.2.1.3
भ॒व॒त्य॒न्नाद्य॒स्याव॑रुद्ध्या॒ अथो॒ प्रैव तेन॑ जायन्त एकवि॒ꣳ॒शो भ॑वति॒ प्रति॑ष्ठित्या॒ अथो॒ रुच॑मे॒वात्मन्द॑धते त्रिण॒वो भ॑वति॒ विजि॑त्यै त्रयस्त्रि॒ꣳ॒शो भ॑वति॒ प्रति॑ष्ठित्यै सदोहविर्धा॒निन॑ ए॒तेन॑ षड्रा॒त्रेण॑ यजेर॒न्नाश्व॑त्थी हवि॒र्धानं॒ चाग्नी᳚ध्रं च भवत॒स्तद्धि सु॑व॒र्ग्यं॑ च॒क्रीव॑ती भवतः सुव॒र्गस्य॑ लो॒कस्य॒ सम॑ष्ट्या उ॒लूख॑लबुध्नो॒ यूपो॑ भवति॒ प्रति॑ष्ठित्यै॒ प्राञ्चो॑ यान्ति॒ प्राङि॑व॒ हि सु॑व॒र्गः॥३॥

%7.2.1.4
लो॒कः सर॑स्वत्या यान्त्ये॒ष वै दे॑व॒यानः॒ पन्था॒स्तमे॒वान्वारो॑हन्त्या॒क्रोश॑न्तो या॒न्त्यव॑र्तिमे॒वान्यस्मि॑न्प्रति॒षज्य॑ प्रति॒ष्ठां ग॑च्छन्ति य॒दा दश॑ श॒तं कु॒र्वन्त्यथैक॑मु॒त्थानꣳ॑ श॒तायुः॒ पुरु॑षः श॒तेन्द्रि॑य॒ आयु॑ष्ये॒वेन्द्रि॒ये प्रति॑ तिष्ठन्ति य॒दा श॒तꣳ स॒हस्रं॑ कु॒र्वन्त्यथैक॑मु॒त्थानꣳ॑ स॒हस्र॑सम्मितो॒ वा अ॒सौ लो॒को॑\-ऽमुमे॒व लो॒कम॒भि ज॑यन्ति य॒दैषां᳚ प्र॒मीये॑त य॒दा वा॒ जीये॑र॒न्नथैक॑मु॒त्थान॒न्तद्धि ती॒र्थम्॥४॥

%7.2.2.0
{\anuvakamend[{पृ॒ष्ठानि॑ सप्तद॒शः सु॑व॒र्गो ज॑यन्ति य॒दैका॑दश च}]}%॥१॥

%7.2.2.1
कु॒सु॒रु॒बिन्द॒ औद्दा॑लकिरकामयत पशु॒मान्थ्स्या॒मिति॒ स ए॒तꣳ स॑प्तरा॒त्रमाह॑र॒त्तेना॑यजत॒ तेन॒ वै स याव॑न्तो ग्रा॒म्याः प॒शव॒स्तानवा॑रुन्द्ध॒ य ए॒वं वि॒द्वान्थ्स॑प्तरा॒त्रेण॒ यज॑ते॒ याव॑न्त ए॒व ग्रा॒म्याः प॒शव॒स्ताने॒वाव॑ रुन्द्धे सप्तरा॒त्रो भ॑वति स॒प्त ग्रा॒म्याः प॒शवः॑ स॒प्तार॒ण्याः स॒प्त छन्दाꣳ॑स्यु॒भय॒स्याव॑रुद्ध्यै त्रि॒वृद॑ग्निष्टो॒मो भ॑वति॒ तेजः॑॥५॥

%7.2.2.2
ए॒वाव॑ रुन्द्धे पञ्चद॒शो भ॑वतीन्द्रि॒यमे॒वाव॑ रुन्द्धे सप्तद॒शो भ॑वत्य॒न्नाद्य॒स्याव॑रुद्ध्या॒ अथो॒ प्रैव तेन॑ जायत एकवि॒ꣳ॒शो भ॑वति॒ प्रति॑ष्ठित्या॒ अथो॒ रुच॑मे॒वात्मन्ध॑त्ते त्रिण॒वो भ॑वति॒ विजि॑त्यै पञ्चवि॒ꣳ॒शो᳚\-ऽग्निष्टो॒मो भ॑वति प्र॒जाप॑ते॒राप्त्यै॑ महाव्र॒तवा॑न॒न्नाद्य॒स्याव॑रुद्ध्यै विश्व॒जिथ्सर्व॑पृष्ठो\-ऽतिरा॒त्रो भ॑वति॒ सर्व॑स्या॒भिजि॑त्यै॒ यत्प्र॒त्यक्ष॒म्पूर्वे॒ष्वहः॑सु पृ॒ष्ठान्यु॑पे॒युः प्र॒त्यक्षम्᳚॥६॥

%7.2.2.3
वि॒श्व॒जिति॒ यथा॑ दु॒ग्धामु॑प॒सीद॑त्ये॒वमु॑त्त॒ममहः॑ स्या॒न्नैक॑रा॒त्रश्च॒न स्या᳚द्बृहद्रथन्त॒रे पूर्वे॒ष्वहः॒सूप॑ यन्ती॒यं वाव र॑थन्त॒रम॒सौ बृ॒हदा॒भ्यामे॒व न य॒न्त्यथो॑ अ॒नयो॑रे॒व प्रति॑ तिष्ठन्ति॒ यत्प्र॒त्यक्षं॑ विश्व॒जिति॑ पृ॒ष्ठान्यु॑प॒यन्ति॒ यथा॒ प्रत्तां᳚ दु॒हे ता॒दृगे॒व तत्॥७॥

%7.2.3.0
{\anuvakamend[{तेज॑ उपे॒युः प्र॒त्यक्षं॒ द्विच॑त्वारिꣳशच्च}]}%॥२॥

%7.2.3.1
बृह॒स्पति॑रकामयत ब्रह्मवर्च॒सी स्या॒मिति॒ स ए॒तम॑ष्टरा॒त्रम॑पश्य॒त्तमाह॑र॒त्तेना॑यजत॒ ततो॒ वै स ब्र॑ह्मवर्च॒स्य॑भव॒द्य ए॒वं वि॒द्वान॑ष्टरा॒त्रेण॒ यज॑ते ब्रह्मवर्च॒स्ये॑व भ॑वत्यष्टरा॒त्रो भ॑वत्य॒ष्टाक्ष॑रा गाय॒त्री गा॑य॒त्री ब्र॑ह्मवर्च॒सम्गा॑यत्रि॒यैव ब्र॑ह्मवर्च॒समव॑ रुन्द्धे\-ऽष्टरा॒त्रो भ॑वति॒ चत॑स्रो॒ वै दिश॒श्चत॑स्रो\-ऽवान्तरदि॒शा दि॒ग्भ्य ए॒व ब्र॑ह्मवर्च॒समव॑ रुन्द्धे॥८॥

%7.2.3.2
त्रि॒वृद॑ग्निष्टो॒मो भ॑वति॒ तेज॑ ए॒वाव॑ रुन्द्धे पञ्चद॒शो भ॑वतीन्द्रि॒यमे॒वाव॑ रुन्द्धे सप्तद॒शो भ॑वत्य॒न्नाद्य॒स्याव॑रुद्ध्या॒ अथो॒ प्रैव तेन॑ जायत एकवि॒ꣳ॒शो भ॑वति॒ प्रति॑ष्ठित्या॒ अथो॒ रुच॑मे॒वात्मन्ध॑त्ते त्रिण॒वो भ॑वति॒ विजि॑त्यै त्रयस्त्रि॒ꣳ॒शो भ॑वति॒ प्रति॑ष्ठित्यै पञ्चवि॒ꣳ॒शो᳚\-ऽग्निष्टो॒मो भ॑वति प्र॒जाप॑ते॒राप्त्यै॑ महाव्र॒तवा॑न॒न्नाद्य॒स्याव॑रुद्ध्यै विश्व॒जिथ्सर्व॑पृष्ठो\-ऽतिरा॒त्रो भ॑वति॒ सर्व॑स्या॒भिजि॑त्यै॥९॥

%7.2.4.0
{\anuvakamend[{दि॒ग्भ्य ए॒व ब्र॑ह्मवर्च॒समव॑रुन्धे॒\-ऽभिजि॑त्यै}]}%॥३॥

%7.2.4.1
प्र॒जाप॑तिः प्र॒जा अ॑सृजत॒ ताः सृ॒ष्टाः क्षुधं॒ न्या॑य॒न्थ्स ए॒तं न॑वरा॒त्रम॑पश्य॒त्तमाह॑र॒त्तेना॑यजत॒ ततो॒ वै प्र॒जाभ्यो॑\-ऽ कल्पत॒ यर्\mbox{}हि॑ प्र॒जाः क्षुधं॑ नि॒गच्छे॑यु॒स्तर्\mbox{}हि॑ नवरा॒त्रेण॑ यजेते॒मे हि वा ए॒तासां᳚ लो॒का अकॢ॑प्ता॒ अथै॒ताः क्षुधं॒ नि ग॑च्छन्ती॒माने॒वाभ्यो॑ लो॒कान्क॑ल्पयति॒ तान्कल्प॑मानान्प्र॒जाभ्यो\-ऽनु॑ कल्पते॒ कल्प॑न्ते॥१०॥

%7.2.4.2
अ॒स्मा॒ इ॒मे लो॒का ऊर्जं॑ प्र॒जासु॑ दधाति त्रिरा॒त्रेणै॒वेमं लो॒कं क॑ल्पयति त्रिरा॒त्रेणा॒न्तरि॑क्षं त्रिरा॒त्रेणा॒मुं लो॒कं यथा॑ गु॒णे गु॒णम॒न्वस्य॑त्ये॒वमे॒व तल्लो॒के लो॒कमन्व॑स्यति॒ धृत्या॒ अशि॑थिलम्भावाय॒ ज्योति॒र्गौरायु॒रिति॑ ज्ञा॒ताः स्तोमा॑ भवन्ती॒यं वाव ज्योति॑र॒न्तरि॑क्षं॒ गौर॒सावायु॑रे॒ष्वे॑व लो॒केषु॒ प्रति॑ तिष्ठन्ति॒ ज्ञात्रं॑ प्र॒जाना᳚म्॥११॥

%7.2.4.3
ग॒च्छ॒ति॒ न॒व॒रा॒त्रो भ॑वत्यभिपू॒र्वमे॒वास्मि॒न्तेजो॑ दधाति॒ यो ज्योगा॑मयावी॒ स्याथ्स न॑वरा॒त्रेण॑ यजेत प्रा॒णा हि वा ए॒तस्याधृ॑ता॒ अथै॒तस्य॒ ज्योगा॑मयति प्रा॒णाने॒वास्मि॑न्दाधारो॒त यदी॒तासु॒र्भव॑ति॒ जीव॑त्ये॒व॥१२॥

%7.2.5.0
{\anuvakamend[{कल्प॑न्ते प्र॒जाना॒न्त्रय॑स्त्रिꣳशच्च}]}%॥४॥

%7.2.5.1
प्र॒जाप॑तिरकामयत॒ प्र जा॑ये॒येति॒ स ए॒तं दश॑होतारमपश्य॒त्तम॑जुहो॒त्तेन॑ दशरा॒त्रम॑सृजत॒ तेन॑ दशरा॒त्रेण॒ प्राजा॑यत दशरा॒त्राय॑ दीक्षि॒ष्यमा॑णो॒ दश॑होतारं जुहुया॒द्दश॑होत्रै॒व द॑शरा॒त्रꣳ सृ॑जते॒ तेन॑ दशरा॒त्रेण॒ प्र जा॑यते वैरा॒जो वा ए॒ष य॒ज्ञो यद्द॑शरा॒त्रो य ए॒वं वि॒द्वान्द॑शरा॒त्रेण॒ यज॑ते वि॒राज॑मे॒व ग॑च्छति प्राजाप॒त्यो वा ए॒ष य॒ज्ञो यद्द॑शरा॒त्रः॥१३॥

%7.2.5.2
य ए॒वं वि॒द्वान्द॑शरा॒त्रेण॒ यज॑ते॒ प्रैव जा॑यत॒ इन्द्रो॒ वै स॒दृङ्दे॒वता॑भिरासी॒थ्स न व्या॒वृत॑मगच्छ॒थ्स प्र॒जाप॑ति॒मुपा॑धाव॒त् तस्मा॑ ए॒तं द॑शरा॒त्रम्प्राय॑च्छ॒त्तमाह॑र॒त्तेना॑यजत॒ ततो॒ वै सो᳚\-ऽन्याभि॑र्दे॒वता॑भिर्व्या॒वृत॑मगच्छ॒द्य ए॒वं वि॒द्वान्द॑शरा॒त्रेण॒ यज॑ते व्या॒वृत॑मे॒व पा॒प्मना॒ भ्रातृ॑व्येण गच्छति त्रिक॒कुद्वै॥१४॥

%7.2.5.3
ए॒ष य॒ज्ञो यद्द॑शरा॒त्रः क॒कुत्प॑ञ्चद॒शः क॒कुदे॑कवि॒ꣳ॒शः क॒कुत्त्र॑यस्त्रि॒ꣳ॒शो य ए॒वं वि॒द्वान्द॑शरा॒त्रेण॒ यज॑ते त्रिक॒कुदे॒व स॑मा॒नानां᳚ भवति॒ यज॑मानः पञ्चद॒शो यज॑मान एकवि॒ꣳ॒शो यज॑मानस्त्रयस्त्रि॒ꣳ॒शः पुर॒ इत॑रा अभिच॒र्यमा॑णो दशरा॒त्रेण॑ यजेत देवपु॒रा ए॒व पर्यू॑हते॒ तस्य॒ न कुत॑श्च॒नोपा᳚व्या॒धो भ॑वति॒ नैन॑मभि॒चर᳚न्थ्स्तृणुते देवासु॒राः संय॑त्ता आस॒न्ते दे॒वा ए॒ताः॥१५॥

%7.2.5.4
दे॒व॒पु॒रा अ॑पश्य॒न् यद्द॑शरा॒त्रस्ताः पर्यौ॑हन्त॒ तेषां॒ न कुत॑श्च॒नोपा᳚व्या॒धो॑\-ऽभव॒त्ततो॑ दे॒वा अभ॑व॒न्परासु॑रा॒ यो भ्रातृ॑व्यवा॒न्थ्स्याथ्स द॑शरा॒त्रेण॑ यजेत देवपु॒रा ए॒व पर्यू॑हते॒ तस्य॒ न कुत॑श्च॒नोपा᳚व्या॒धो भ॑वति॒ भव॑त्या॒त्मना॒ परा᳚स्य॒ भ्रातृ॑व्यो भवति॒ स्तोमः॒ स्तोम॒स्योप॑स्तिर्भवति॒ भ्रातृ॑व्यमे॒वोप॑स्तिं कुरुते जा॒मि वै॥१६॥

%7.2.5.5
ए॒तत्कु॑र्वन्ति॒ यज्ज्यायाꣳ॑स॒ꣴ॒ स्तोम॑मु॒पेत्य॒ कनी॑याꣳसमुप॒यन्ति॒ यद॑ग्निष्टोमसा॒मान्य॒वस्ता᳚च्च प॒रस्ता᳚च्च॒ भव॒न्त्यजा॑मित्वाय त्रि॒वृद॑ग्निष्टो॒मो᳚\-ऽग्नि॒ष्टुदा᳚ग्ने॒यीषु॑ भवति॒ तेज॑ ए॒वाव॑ रुन्द्धे पञ्चद॒श उ॒क्थ्य॑ ऐ॒न्द्रीष्वि॑न्द्रि॒यमे॒वाव॑ रुन्द्धे त्रि॒वृद॑ग्निष्टो॒मो वै᳚श्वदे॒वीषु॒ पुष्टि॑मे॒वाव॑ रुन्द्धे सप्तद॒शो᳚\-ऽग्निष्टो॒मः प्रा॑जाप॒त्यासु॑ तीव्रसो॒मो᳚\-ऽन्नाद्य॒स्याव॑रुद्ध्या॒ अथो॒ प्रैव तेन॑ जायते॥१७॥

%7.2.5.6
ए॒क॒वि॒ꣳ॒श उ॒क्थ्यः॑ सौ॒रीषु॒ प्रति॑ष्ठित्या॒ अथो॒ रुच॑मे॒वात्मन्ध॑त्ते सप्तद॒शो᳚\-ऽग्निष्टो॒मः प्रा॑जाप॒त्यासू॑पह॒व्य॑ उपह॒वमे॒व ग॑च्छति त्रिण॒वाव॑ग्निष्टो॒माव॒भित॑ ऐ॒न्द्रीषु॒ विजि॑त्यै त्रयस्त्रि॒ꣳ॒श उ॒क्थ्यो॑ वैश्वदे॒वीषु॒ प्रति॑ष्ठित्यै विश्व॒जिथ्सर्व॑पृष्ठो\-ऽ तिरा॒त्रो भ॑वति॒ सर्व॑स्या॒भिजि॑त्यै॥१८॥

%7.2.6.0
{\anuvakamend[{प्रा॒जा॒प॒त्यो वा ए॒ष य॒ज्ञो यद्द॑शरा॒त्रस्त्रि॑क॒कुद्वा ए॒ता वै जा॑यत॒ एक॑त्रिꣳशच्च}]}%॥५॥

%7.2.6.1
ऋ॒तवो॒ वै प्र॒जाका॑माः प्र॒जां नावि॑न्दन्त॒ ते॑\-ऽकामयन्त प्र॒जाꣳ सृ॑जेमहि प्र॒जामव॑ रुन्धीमहि प्र॒जां वि॑न्देमहि प्र॒जाव॑न्तः स्या॒मेति॒ त ए॒तमे॑कादशरा॒त्रम॑पश्य॒न्तमाह॑र॒न्तेना॑यजन्त॒ ततो॒ वै ते प्र॒जाम॑सृजन्त प्र॒जामवा॑रुन्धत प्र॒जाम॑विन्दन्त प्र॒जाव॑न्तो\-ऽभव॒न्त ऋ॒तवो॑\-ऽभव॒न्तदा᳚र्त॒वाना॑मार्तव॒त्वमृ॑तू॒नां वा ए॒ते पु॒त्रास्तस्मा᳚त्॥१९॥

%7.2.6.2
आ॒र्त॒वा उ॑च्यन्ते॒ य ए॒वं वि॒द्वाꣳस॑ एकादशरा॒त्रमास॑ते प्र॒जामे॒व सृ॑जन्ते प्र॒जामव॑ रुन्धते प्र॒जां वि॑न्दन्ते प्र॒जाव॑न्तो भवन्ति॒ ज्योति॑रतिरा॒त्रो भ॑वति॒ ज्योति॑रे॒व पु॒रस्ता᳚द्दधते सुव॒र्गस्य॑ लो॒कस्यानु॑ख्यात्यै॒ पृष्ठ्यः॑ षड॒हो भ॑वति॒ षड्वा ऋ॒तवः॒ षट्पृ॒ष्ठानि॑ पृ॒ष्ठैरे॒वर्तून॒न्वारो॑हन्त्यृ॒तुभिः॑ संवथ्स॒रन्ते सं॑वथ्स॒र ए॒व प्रति॑ तिष्ठन्ति चतुर्वि॒ꣳ॒शो भ॑वति॒ चतु॑र्विꣳशत्यक्षरा गाय॒त्री॥२०॥

%7.2.6.3
गा॒य॒त्रम्ब्र॑ह्मवर्च॒सङ्गा॑यत्रि॒यामे॒व ब्र॑ह्मवर्च॒से प्रति॑ तिष्ठन्ति चतुश्चत्वारि॒ꣳ॒शो भ॑वति॒ चतु॑श्चत्वारिꣳशदक्षरा त्रि॒ष्टुगि॑न्द्रि॒यं त्रि॒ष्टुप्त्रि॒ष्टुभ्ये॒वेन्द्रि॒ये प्रति॑ तिष्ठन्त्यष्टाचत्वारि॒ꣳ॒शो भ॑वत्य॒ष्टाच॑त्वारिꣳशदक्षरा॒ जग॑ती॒ जाग॑ताः प॒शवो॒ जग॑त्यामे॒व प॒शुषु॒ प्रति॑ तिष्ठन्त्येकादशरा॒त्रो भ॑वति॒ पञ्च॒ वा ऋ॒तव॑ आर्त॒वाः पञ्च॒र्तुष्वे॒वार्त॒वेषु॑ संवथ्स॒रे प्र॑ति॒ष्ठाय॑ प्र॒जामव॑ रुन्धते\-ऽतिरा॒त्राव॒भितो॑ भवतः प्र॒जायै॒ परि॑गृहीत्यै॥२१॥

%7.2.7.0
{\anuvakamend[{तस्मा᳚द्गाय॒त्र्येका॒न्नप॑ञ्चा॒शच्च॑}]}%॥६॥

%7.2.7.1
ऐ॒न्द्र॒वा॒य॒वाग्रा᳚न्गृह्णीया॒द्यः का॒मये॑त यथापू॒र्वं प्र॒जाः क॑ल्पेर॒न्निति॑ य॒ज्ञस्य॒ वै कॢप्ति॒मनु॑ प्र॒जाः क॑ल्पन्ते य॒ज्ञस्याकॢ॑प्ति॒मनु॒ न क॑ल्पन्ते यथापू॒र्वमे॒व प्र॒जाः क॑ल्पयति॒ न ज्यायाꣳ॑सं॒ कनी॑या॒नति॑ क्रामत्यैन्द्रवाय॒वाग्रा᳚न्गृह्णीयादामया॒विनः॑ प्रा॒णेन॒ वा ए॒ष व्यृ॑ध्यते॒ यस्या॒मय॑ति प्रा॒ण ऐ᳚न्द्रवाय॒वः प्रा॒णेनै॒वैन॒ꣳ॒ सम॑र्धयति मैत्रावरु॒णाग्रा᳚न्गृह्णीर॒न् येषां᳚ दीक्षि॒तानां᳚ प्र॒मीये॑त॥२२॥

%7.2.7.2
प्रा॒णा॒पा॒नाभ्यां॒ वा ए॒ते व्यृ॑ध्यन्ते॒ येषां᳚ दीक्षि॒तानां᳚ प्र॒मीय॑ते प्राणापा॒नौ मि॒त्रावरु॑णौ प्राणापा॒नावे॒व मु॑ख॒तः परि॑ हरन्त आश्वि॒नाग्रा᳚न्गृह्णीतानुजाव॒रो᳚\-ऽश्विनौ॒ वै दे॒वाना॑मानुजाव॒रौ प॒श्चेवाग्रं॒ पर्यैताम॒श्विना॑वे॒तस्य॑ दे॒वता॒ य आ॑नुजाव॒रस्तावे॒वैन॒मग्रं॒ परि॑ णयतः शु॒क्राग्रा᳚न्गृह्णीत ग॒तश्रीः᳚ प्रति॒ष्ठाका॑मो॒\-ऽसौ वा आ॑दि॒त्यः शु॒क्र ए॒षो\-ऽन्तो\-ऽन्त॑म्मनु॒ष्यः॑॥२३॥

%7.2.7.3
श्रि॒यै ग॒त्वा नि व॑र्त॒ते\-ऽन्ता॑दे॒वान्त॒मा र॑भते॒ न ततः॒ पापी॑यान्भवति म॒न्थ्य॑ग्रान्गृह्णीताभि॒चर॑न्नार्तपा॒त्रं वा ए॒तद्यन्म॑न्थिपा॒त्रम्मृ॒त्युनै॒वैनं॑ ग्राहयति ता॒जगार्ति॒मार्च्छ॑त्याग्रय॒णाग्रा᳚न्गृह्णीत॒ यस्य॑ पि॒ता पि॑ताम॒हः पुण्यः॒ स्यादथ॒ तन्न प्रा᳚प्नु॒याद्वा॒चा वा ए॒ष इ॑न्द्रि॒येण॒ व्यृ॑ध्यते॒ यस्य॑ पि॒ता पि॑ताम॒हः पुण्यः॑॥२४॥

%7.2.7.4
भव॒त्यथ॒ तन्न प्रा॒प्नोत्युर॑ इवै॒तद्य॒ज्ञस्य॒ वागि॑व॒ यदा᳚ग्रय॒णो वा॒चैवैन॑मिन्द्रि॒येण॒ सम॑र्धयति॒ न ततः॒ पापी॑यान्भव\-त्यु॒क्थ्या᳚ग्रान्गृह्णीताभिच॒र्यमा॑णः॒ सर्वे॑षां॒ वा ए॒तत्पात्रा॑णामिन्द्रि॒यं यदु॑क्थ्यपा॒त्रꣳ सर्वे॑णै॒वैन॑मिन्द्रि॒येणाति॒ प्र यु॑ङ्क्ते॒ सर॑स्वत्य॒भि नो॑ नेषि॒ वस्य॒ इति॑ पुरो॒रुचं॑ कुर्या॒द्वाग्वै॥२५॥

%7.2.7.5
सर॑स्वती वा॒चैवैन॒मति॒ प्र यु॑ङ्क्ते॒ मा त्वत्क्षेत्रा॒ण्यर॑णानि ग॒न्मेत्या॑ह मृ॒त्योर्वै क्षेत्रा॒ण्यर॑णानि॒ तेनै॒व मृ॒त्योः क्षेत्रा॑णि॒ न ग॑च्छति पू॒र्णान्ग्रहा᳚न्गृह्णीयादामया॒विनः॑ प्रा॒णान् वा ए॒तस्य॒ शुगृ॑च्छति॒ यस्या॒मय॑ति प्रा॒णा ग्रहाः᳚ प्रा॒णाने॒वास्य॑ शु॒चो मु॑ञ्चत्यु॒त यदी॒तासु॒र्भव॑ति॒ जीव॑त्ये॒व पू॒र्णान्ग्रहा᳚न्गृह्णीया॒द्यर्\mbox{}हि॑ प॒र्जन्यो॒ न वर्\mbox{}षे᳚त्प्रा॒णान् वा ए॒तर्\mbox{}हि॑ प्र॒जाना॒ꣳ॒ शुगृ॑च्छति॒ यर्\mbox{}हि॑ प॒र्जन्यो॒ न॒ वर्\mbox{}ष॑ति प्रा॒णा ग्रहाः᳚ प्रा॒णाने॒व प्र॒जानाꣳ॑ शु॒चो मु॑ञ्चति ता॒जक्प्र व॑र्\mbox{}षति॥२६॥

%7.2.8.0
{\anuvakamend[{प्र॒मीये॑त मनु॒ष्य॑ ऋध्यते॒ यस्य॑ पि॒ता पि॑ताम॒हः पुण्यो॒ वाग्वा ए॒व पू॒र्णान्ग्रहा॒न्पञ्च॑विꣳशतिश्च}]}%॥७॥

%7.2.8.1
गा॒य॒त्रो वा ऐ᳚न्द्रवाय॒वो गा॑य॒त्रम्प्रा॑य॒णीय॒मह॒स्तस्मा᳚त्प्राय॒णीये\-ऽह॑न्नैन्द्रवाय॒वो गृ॑ह्यते॒ स्व ए॒वैन॑मा॒यत॑ने गृह्णाति॒ त्रैष्टु॑भो॒ वै शु॒क्रस्त्रैष्टु॑भं द्वि॒तीय॒मह॒स्तस्मा᳚द्द्वि॒तीये\-ऽह॑ञ्छु॒क्रो गृ॑ह्यते॒ स्व ए॒वैन॑मा॒यत॑ने गृह्णाति॒ जाग॑तो॒ वा आ᳚ग्रय॒णो जाग॑तं तृ॒तीय॒मह॒स्तस्मा᳚त्तृ॒तीये\-ऽह॑न्नाग्रय॒णो गृ॑ह्यते॒ स्व ए॒वैन॑मा॒यत॑ने गृह्णात्ये॒तद्वै॥२७॥

%7.2.8.2
य॒ज्ञमा॑प॒द्यच्छन्दाꣳ॑स्या॒प्नोति॒ यदा᳚ग्रय॒णः श्वो गृ॒ह्यते॒ यत्रै॒व य॒ज्ञमदृ॑श॒न्तत॑ ए॒वैन॒म्पुनः॒ प्र यु॑ङ्क्ते॒ जग॑न्मुखो॒ वै द्वि॒तीय॑स्त्रिरा॒त्रो जाग॑त आग्रय॒णो यच्च॑तु॒र्थे\-ऽह॑न्नाग्रय॒णो गृ॒ह्यते॒ स्व ए॒वैन॑मा॒यत॑ने गृह्णा॒त्यथो॒ स्वमे॒व छन्दो\-ऽनु॑ प॒र्याव॑र्तन्ते॒ राथं॑तरो॒ वा ऐ᳚न्द्रवाय॒वो राथं॑तरं पञ्च॒ममह॒स्तस्मा᳚त्पञ्च॒मे\-ऽहन्न्॑॥२८॥

%7.2.8.3
ऐ॒न्द्र॒वा॒य॒वो गृ॑ह्यते॒ स्व ए॒वैन॑मा॒यत॑ने गृह्णाति॒ बार्\mbox{}ह॑तो॒ वै शु॒क्रो बार्\mbox{}ह॑तꣳ ष॒ष्ठमह॒स्तस्मा᳚त्ष॒ष्ठे\-ऽह॑ञ्छु॒क्रो गृ॑ह्यते॒ स्व ए॒वैन॑मा॒यत॑ने गृह्णात्ये॒तद्वै द्वि॒तीयं॑ य॒ज्ञमा॑प॒द्यच्छन्दाꣳ॑स्या॒प्नोति॒ यच्छु॒क्रः श्वो गृ॒ह्यते॒ यत्रै॒व य॒ज्ञमदृ॑श॒न्तत॑ ए॒वैन॒म्पुनः॒ प्र यु॑ङ्क्ते त्रि॒ष्टुङ्मु॑खो॒ वै तृ॒तीय॑स्त्रिरा॒त्रस्त्रैष्टु॑भः॥२९॥

%7.2.8.4
शु॒क्रो यथ्स॑प्त॒मे\-ऽह॑ञ्छु॒क्रो गृ॒ह्यते॒ स्व ए॒वैन॑मा॒यत॑ने गृह्णा॒त्यथो॒ स्वमे॒व छन्दो\-ऽनु॑ प॒र्याव॑र्तन्ते॒ वाग्वा आ᳚ग्रय॒णो वाग॑ष्ट॒ममह॒स्तस्मा॑दष्ट॒मे\-ऽह॑न्नाग्रय॒णो गृ॑ह्यते॒ स्व ए॒वैन॑मा॒यत॑ने गृह्णाति प्रा॒णो वा ऐ᳚न्द्रवाय॒वः प्रा॒णो न॑व॒ममह॒स्तस्मा᳚न्नव॒मे\-ऽह॑न्नैन्द्रवाय॒वो गृ॑ह्यते॒ स्व ए॒वैन॑मा॒यत॑ने गृह्णात्ये॒तत्॥३०॥

%7.2.8.5
वै तृ॒तीयं॑ य॒ज्ञमा॑प॒द्यच्छन्दाꣳ॑स्या॒प्नोति॒ यदै᳚न्द्रवाय॒वः श्वो गृ॒ह्यते॒ यत्रै॒व य॒ज्ञमदृ॑श॒न्तत॑ ए॒वैन॒म्पुनः॒ प्र यु॒ङ्क्ते\-ऽथो॒ स्वमे॒व छन्दो\-ऽनु॑ प॒र्याव॑र्तन्ते प॒थो वा ए॒ते\-ऽध्यप॑थेन यन्ति॒ ये᳚\-ऽन्येनै᳚न्द्रवाय॒वात्प्र॑ति॒पद्य॒न्ते\-ऽन्तः॒ खलु॒ वा ए॒ष य॒ज्ञस्य॒ यद्द॑श॒ममह॑र्दश॒मे\-ऽह॑न्नैन्द्रवाय॒वो गृ॑ह्यते य॒ज्ञस्य॑॥३१॥

%7.2.8.6
ए॒वान्तं॑ ग॒त्वाप॑था॒त्पन्था॒मपि॑ य॒न्त्यथो॒ यथा॒ वही॑यसा प्रति॒सारं॒ वह॑न्ति ता॒दृगे॒व तच्छन्दाꣳ॑स्य॒न्यो᳚न्यस्य॑ लो॒कम॒भ्य॑ध्याय॒न्तान्ये॒तेनै॒व दे॒वा व्य॑वाहयन्नैन्द्रवाय॒वस्य॒ वा ए॒तदा॒यत॑नं॒ यच्च॑तु॒र्थमह॒स्तस्मि॑न्नाग्रय॒णो गृ॑ह्यते॒ तस्मा॑दाग्रय॒णस्या॒यत॑ने नव॒मे\-ऽह॑न्नैन्द्रवाय॒वो गृ॑ह्यते शु॒क्रस्य॒ वा ए॒तदा॒यत॑नं॒ यत्प॑ञ्च॒मम्॥३२॥

%7.2.8.7
अह॒स्तस्मि॑न्नैन्द्रवाय॒वो गृ॑ह्यते॒ तस्मा॑दैन्द्रवाय॒वस्या॒यत॑ने सप्त॒मे\-ऽह॑ञ्छु॒क्रो गृ॑ह्यत आग्रय॒णस्य॒ वा ए॒तदा॒यत॑नं॒ यत्ष॒ष्ठमह॒स्तस्मि॑ञ्छु॒क्रो गृ॑ह्यते॒ तस्मा᳚च्छु॒क्रस्या॒यत॑ने\-ऽष्ट॒मे\-ऽह॑न्नाग्रय॒णो गृ॑ह्यते॒ छन्दाꣳ॑स्ये॒व तद्वि वा॑हयति॒ प्र वस्य॑सो विवा॒हमा᳚प्नोति॒ य ए॒वं वेदाथो॑ दे॒वता᳚भ्य ए॒व य॒ज्ञे सं॒विदं॑ दधाति॒ तस्मा॑दि॒दमन्यो᳚न्यस्मै॑ ददाति॥३३॥

%7.2.9.0
{\anuvakamend[{ए॒तद्वै प॑ञ्च॒मे\-ऽह॒न्त्रैष्टु॑भ ए॒तद्गृ॑ह्यते य॒ज्ञस्य॑ प़ञ्च॒मम॒न्यस्मा॒ एक॑ञ्च}]}%॥८॥

%7.2.9.1
प्र॒जाप॑तिरकामयत॒ प्र जा॑ये॒येति॒ स ए॒तं द्वा॑दशरा॒त्रम॑पश्य॒त्तमाह॑र॒त्तेना॑यजत॒ ततो॒ वै स प्राजा॑यत॒ यः का॒मये॑त॒ प्र जा॑ये॒येति॒ स द्वा॑दशरा॒त्रेण॑ यजेत॒ प्रैव जा॑यते ब्रह्मवा॒दिनो॑ वदन्त्यग्निष्टो॒मप्रा॑यणा य॒ज्ञा अथ॒ कस्मा॑दतिरा॒त्रः पूर्वः॒ प्र यु॑ज्यत॒ इति॒ चक्षु॑षी॒ वा ए॒ते य॒ज्ञस्य॒ यद॑तिरा॒त्रौ क॒नीनि॑के अग्निष्टो॒मौ यत्॥३४॥

%7.2.9.2
अ॒ग्नि॒ष्टो॒मं पूर्व॑म्प्रयुञ्जी॒रन्ब॑हि॒र्धा क॒नीनि॑के दध्यु॒स्तस्मा॑दतिरा॒त्रः पूर्वः॒ प्र यु॑ज्यते॒ चक्षु॑षी ए॒व य॒ज्ञे धि॒त्वा म॑ध्य॒तः क॒नीनि॑के॒ प्रति॑ दधति॒ यो वै गा॑य॒त्रीं ज्योतिः॑पक्षां॒ वेद॒ ज्योति॑षा भा॒सा सु॑व॒र्गं लो॒कमे॑ति॒ याव॑ग्निष्टो॒मौ तौ प॒क्षौ ये\-ऽन्त॑रे॒\-ऽष्टावु॒क्थ्याः᳚ स आ॒त्मैषा वै गा॑य॒त्री ज्योतिः॑पक्षा॒ य ए॒वं वेद॒ ज्योति॑षा भा॒सा सु॑व॒र्गं लो॒कम्॥३५॥

%7.2.9.3
ए॒ति॒ प्र॒जाप॑तिर्वा ए॒ष द्वा॑दश॒धा विहि॑तो॒ यद्द्वा॑दशरा॒त्रो याव॑तिरा॒त्रौ तौ प॒क्षौ ये\-ऽन्त॑रे॒\-ऽष्टावु॒क्थ्याः᳚ स आ॒त्मा प्र॒जाप॑तिर्वावैष सन्थ्सद्ध॒ वै स॒त्त्रेण॑ स्पृणोति प्रा॒णा वै सत्प्रा॒णाने॒व स्पृ॑णोति॒ सर्वा॑सां॒ वा ए॒ते प्र॒जानां᳚ प्रा॒णैरा॑सते॒ ये स॒त्त्रमास॑ते॒ तस्मा᳚त्पृच्छन्ति॒ किमे॒ते स॒त्त्रिण॒ इति॑ प्रि॒यः प्र॒जाना॒मुत्थि॑तो भवति॒ य ए॒वं वेद॑॥३६॥

%7.2.10.0
{\anuvakamend[{अ॒ग्नि॒ष्टो॒मौ यथ्सु॑व॒र्गल्लों॒कं प्रि॒यः प्र॒जानां॒ पञ्च॑ च}]}%॥९॥

%7.2.10.1
न वा ए॒षो᳚\-ऽन्यतो॑वैश्वानरः सुव॒र्गाय॑ लो॒काय॒ प्राभ॑वदू॒र्ध्वो ह॒ वा ए॒ष आत॑त आसी॒त्ते दे॒वा ए॒तं वै᳚श्वान॒रं पर्यौ॑हन्थ्सुव॒र्गस्य॑ लो॒कस्य॒ प्रभू᳚त्या ऋ॒तवो॒ वा ए॒तेन॑ प्र॒जाप॑तिमयाजय॒न्तेष्वा᳚र्ध्नो॒दधि॒ तदृ॒ध्नोति॑ ह॒ वा ऋ॒त्विक्षु॒ य ए॒वं वि॒द्वान्द्वा॑दशा॒हेन॒ यज॑ते॒ ते᳚\-ऽस्मिन्नैच्छन्त॒ स रस॒मह॑ वस॒न्ताय॒ प्राय॑च्छत्॥३७॥

%7.2.10.2
यवं॑ ग्री॒ष्मायौष॑धीर्व॒र्\mbox{}षाभ्यो᳚ व्री॒हीञ्छ॒रदे॑ माषति॒लौ हे॑मन्तशिशि॒राभ्या॒न्तेनेन्द्रं॑ प्र॒जाप॑तिरयाजय॒त्ततो॒ वा इन्द्र॒ इन्द्रो॑\-ऽभव॒त्तस्मा॑दाहुरानुजाव॒रस्य॑ य॒ज्ञ इति॒ स ह्ये॑तेनाग्रे\-ऽय॑जतै॒ष ह॒ वै कु॒णप॑मत्ति॒ यः स॒त्त्रे प्र॑तिगृ॒ह्णाति॑ पुरुषकुण॒पम॑श्वकुण॒पङ्गौर्वा अन्नं॒ येन॒ पात्रे॒णान्न॒म्बिभ्र॑ति॒ यत्तन्न नि॒र्णेनि॑जति॒ ततो\-ऽधि॑॥३८॥

%7.2.10.3
मलं॑ जायत॒ एक॑ ए॒व य॑जे॒तैको॒ हि प्र॒जाप॑ति॒रार्ध्नो॒द्द्वाद॑श॒ रात्री᳚र्दीक्षि॒तः स्या॒द्द्वाद॑श॒ मासाः᳚ संवथ्स॒रः सं॑वथ्स॒रः प्र॒जाप॑तिः प्र॒जाप॑ति॒र्वावैष ए॒ष ह॒ त्वै जा॑यते॒ यस्तप॒सो\-ऽधि॒ जाय॑ते चतु॒र्धा वा ए॒तास्ति॒स्रस्ति॑स्रो॒ रात्र॑यो॒ यद्द्वाद॑शोप॒सदो॒ याः प्र॑थ॒मा य॒ज्ञं ताभिः॒ सम्भ॑रति॒ या द्वि॒तीया॑ य॒ज्ञं ताभि॒रा र॑भते॥३९॥

%7.2.10.4
यास्तृ॒तीयाः॒ पात्रा॑णि॒ ताभि॒र्निर्णे॑निक्ते॒ याश्च॑तु॒र्थीरपि॒ ताभि॑रा॒त्मान॑मन्तर॒तः शु॑न्धते॒ यो वा अ॑स्य प॒शुमत्ति॑ मा॒ꣳ॒सꣳ सो᳚\-ऽत्ति॒ यः पु॑रो॒डाश॑म्म॒स्तिष्क॒ꣳ॒ स यः प॑रिवा॒पं पुरी॑ष॒ꣳ॒ स य आज्य॑म्म॒ज्जान॒ꣳ॒ स यः सोमꣴ॒ स्वेद॒ꣳ॒ सो\-ऽपि॑ ह॒ वा अ॑स्य शीर्\mbox{}ष॒ण्या॑ नि॒ष्पदः॒ प्रति॑ गृह्णाति॒ यो द्वा॑दशा॒हे प्र॑तिगृ॒ह्णाति॒ तस्मा᳚द्द्वादशा॒हेन॒ न याज्य॑म्पा॒प्मनो॒ व्यावृ॑त्त्यै॥४०॥

%7.2.11.0
{\anuvakamend[{अय॑च्छ॒दधि॑ रभते द्वादशा॒हेन॑ च॒त्वारि॑ च}]}%॥10॥

%7.2.11.1
एक॑स्मै॒ स्वाहा॒ द्वाभ्या॒ꣴ॒ स्वाहा᳚ त्रि॒भ्यः स्वाहा॑ च॒तुर्भ्यः॒ स्वाहा॑ प॒ञ्चभ्यः॒ स्वाहा॑ ष॒ड्भ्यः स्वाहा॑ स॒प्तभ्यः॒ स्वाहा᳚\-ऽष्टा॒भ्यः स्वाहा॑ न॒वभ्यः॒ स्वाहा॑ द॒शभ्यः॒ स्वाहै॑काद॒शभ्यः॒ स्वाहा᳚ द्वाद॒शभ्यः॒ स्वाहा᳚ त्रयोद॒शभ्यः॒ स्वाहा॑ चतुर्द॒शभ्यः॒ स्वाहा॑ पञ्चद॒शभ्यः॒ स्वाहा॑ षोड॒शभ्यः॒ स्वाहा॑ सप्तद॒शभ्यः॒ स्वाहा᳚\-ऽष्टाद॒शभ्यः॒ स्वाहैका॒न्न विꣳ॑श॒त्यै स्वाहा॒ नव॑विꣳशत्यै॒ स्वाहैका॒न्न च॑त्वारि॒ꣳ॒शते॒ स्वाहा॒ नव॑चत्वारिꣳशते॒ स्वाहैका॒न्न ष॒ष्ट्यै स्वाहा॒ नव॑षष्ट्यै॒ स्वाहैका॒न्नाशी॒त्यै स्वाहा॒ नवा॑शीत्यै॒ स्वाहैका॒न्न श॒ताय॒ स्वाहा॑ श॒ताय॒ स्वाहा॒ द्वाभ्याꣳ॑ श॒ताभ्या॒ꣴ॒ स्वाहा॒ सर्व॑स्मै॒ स्वाहा᳚॥४१॥

%7.2.12.0
{\anuvakamend[{नव॑चत्वारिꣳशते॒ स्वाहैका॒न्नैक॑विꣳशतिश्च}]}%॥11॥

%7.2.12.1
एक॑स्मै॒ स्वाहा᳚ त्रि॒भ्यः स्वाहा॑ प॒ञ्चभ्यः॒ स्वाहा॑ स॒प्तभ्यः॒ स्वाहा॑ न॒वभ्यः॒ स्वाहै॑काद॒शभ्यः॒ स्वाहा᳚ त्रयोद॒शभ्यः॒ स्वाहा॑ पञ्चद॒शभ्यः॒ स्वाहा॑ सप्तद॒शभ्यः॒ स्वाहैका॒न्न विꣳ॑श॒त्यै स्वाहा॒ नव॑विꣳशत्यै॒ स्वाहैका॒न्न च॑त्वारि॒ꣳ॒शते॒ स्वाहा॒ नव॑चत्वारिꣳशते॒ स्वाहैका॒न्न ष॒ष्ट्यै स्वाहा॒ नव॑षष्ट्यै॒ स्वाहैका॒न्नाशी॒त्यै स्वाहा॒ नवा॑शीत्यै॒ स्वाहैका॒न्न श॒ताय॒ स्वाहा॑ श॒ताय॒ स्वाहा॒ सर्व॑स्मै॒ स्वाहा᳚॥४२॥

%7.2.13.0
{\anuvakamend[{एक॑स्मै त्रि॒भ्यः प॑ञ्चा॒शत्}]}%॥12॥

%7.2.13.1
द्वाभ्या॒ꣴ॒ स्वाहा॑ च॒तुर्भ्यः॒ स्वाहा॑ ष॒ड्भ्यः स्वाहा᳚\-ऽष्टा॒भ्यः स्वाहा॑ द॒शभ्यः॒ स्वाहा᳚ द्वाद॒शभ्यः॒ स्वाहा॑ चतुर्द॒शभ्यः॒ स्वाहा॑ षोड॒शभ्यः॒ स्वाहा᳚\-ऽष्टाद॒शभ्यः॒ स्वाहा॑ विꣳश॒त्यै स्वाहा॒\-ऽष्टान॑वत्यै॒ स्वाहा॑ श॒ताय॒ स्वाहा॒ सर्व॑स्मै॒ स्वाहा᳚॥४३॥

%7.2.14.0
{\anuvakamend[{द्वाभ्या॑म॒ष्टान॑वत्यै॒ षड्विꣳ॑शतिः}]}%॥13॥

%7.2.14.1
त्रि॒भ्यः स्वाहा॑ प॒ञ्चभ्यः॒ स्वाहा॑ स॒प्तभ्यः॒ स्वाहा॑ न॒वभ्यः॒ स्वाहै॑काद॒शभ्यः॒ स्वाहा᳚ त्रयोद॒शभ्यः॒ स्वाहा॑ पञ्चद॒शभ्यः॒ स्वाहा॑ सप्तद॒शभ्यः॒ स्वाहैका॒न्न विꣳ॑श॒त्यै स्वाहा॒ नव॑विꣳशत्यै॒ स्वाहैका॒न्न च॑त्वारि॒ꣳ॒शते॒ स्वाहा॒ नव॑चत्वारिꣳशते॒ स्वाहैका॒न्न ष॒ष्ट्यै स्वाहा॒ नव॑षष्ट्यै॒ स्वाहैका॒न्नाशी॒त्यै स्वाहा॒ नवा॑शीत्यै॒ स्वाहैका॒न्न श॒ताय॒ स्वाहा॑ श॒ताय॒ स्वाहा॒ सर्व॑स्मै॒ स्वाहा᳚॥४४॥

%7.2.15.0
{\anuvakamend[{त्रि॒भ्यो᳚\-ऽष्टाचत्वारि॒ꣳ॒शत्}]}%॥14॥

%7.2.15.1
च॒तुर्भ्यः॒ स्वाहा᳚\-ऽष्टा॒भ्यः स्वाहा᳚ द्वाद॒शभ्यः॒ स्वाहा॑ षोड॒शभ्यः॒ स्वाहा॑ विꣳश॒त्यै स्वाहा॒ षण्ण॑वत्यै॒ स्वाहा॑ श॒ताय॒ स्वाहा॒ सर्व॑स्मै॒ स्वाहा᳚॥४५॥

%7.2.16.0
{\anuvakamend[{च॒तुर्भ्यः॒ षण्ण॑वत्यै॒ षोड॑श}]}%॥15॥

%7.2.16.1
प॒ञ्चभ्यः॒ स्वाहा॑ द॒शभ्यः॒ स्वाहा॑ पञ्चद॒शभ्यः॒ स्वाहा॑ विꣳश॒त्यै स्वाहा॒ पञ्च॑नवत्यै॒ स्वाहा॑ श॒ताय॒ स्वाहा॒ सर्व॑स्मै॒ स्वाहा᳚॥४६॥

%7.2.17.0
{\anuvakamend[{प॒ञ्चभ्यः॒ प़ञ्च॑नवत्यै॒ चतु॑र्दश}]}%॥16॥

%7.2.17.1
द॒शभ्यः॒ स्वाहा॑ विꣳश॒त्यै स्वाहा᳚ त्रि॒ꣳ॒शते॒ स्वाहा॑ चत्वारि॒ꣳ॒शते॒ स्वाहा॑ पञ्चा॒शते॒ स्वाहा॑ ष॒ष्ट्यै स्वाहा॑ सप्त॒त्यै स्वाहा॑\-ऽशी॒त्यै स्वाहा॑ नव॒त्यै स्वाहा॑ श॒ताय॒ स्वाहा॒ सर्व॑स्मै॒ स्वाहा᳚॥४७॥

%7.2.18.0
{\anuvakamend[{द॒शभ्यो॒ द्वाविꣳ॑शतिः}]}%॥17॥

%7.2.18.1
वि॒ꣳ॒श॒त्यै स्वाहा॑ चत्वारि॒ꣳ॒शते॒ स्वाहा॑ ष॒ष्ट्यै स्वाहा॑\-ऽशी॒त्यै स्वाहा॑ श॒ताय॒ स्वाहा॒ सर्व॑स्मै॒ स्वाहा᳚॥४८॥

%7.2.19.0
{\anuvakamend[{वि॒ꣳ॒श॒त्यै द्वाद॑श}]}%॥18॥

%7.2.19.1
प॒ञ्चा॒शते॒ स्वाहा॑ श॒ताय॒ स्वाहा॒ द्वाभ्याꣳ॑ श॒ताभ्या॒ꣴ॒ स्वाहा᳚ त्रि॒भ्यः श॒तेभ्यः॒ स्वाहा॑ च॒तुर्भ्यः॑ श॒तेभ्यः॒ स्वाहा॑ प॒ञ्चभ्यः॑ श॒तेभ्यः॒ स्वाहा॑ ष॒ड्भ्यः श॒तेभ्यः॒ स्वाहा॑ स॒प्तभ्यः॑ श॒तेभ्यः॒ स्वाहा᳚\-ऽष्टा॒भ्यः श॒तेभ्यः॒ स्वाहा॑ न॒वभ्यः॑ श॒तेभ्यः॒ स्वाहा॑ स॒हस्रा॑य॒ स्वाहा॒ सर्व॑स्मै॒ स्वाहा᳚॥४९॥

%7.2.20.0
{\anuvakamend[{प॒ञ्चा॒शते॒ द्वात्रिꣳ॑शत्}]}%॥19॥

%7.2.20.1
श॒ताय॒ स्वाहा॑ स॒हस्रा॑य॒ स्वाहा॒\-ऽयुता॑य॒ स्वाहा॑ नि॒युता॑य॒ स्वाहा᳚ प्र॒युता॑य॒ स्वाहा\-ऽर्बु॑दाय॒ स्वाहा॒ न्य॑र्बुदाय॒ स्वाहा॑ समु॒द्राय॒ स्वाहा॒ मध्या॑य॒ स्वाहा\-ऽन्ता॑य॒ स्वाहा॑ परा॒र्धाय॒ स्वाहो॒षसे॒ स्वाहा॒ व्यु॑ष्ट्यै॒ स्वाहो॑देष्य॒ते स्वाहो᳚द्य॒ते स्वाहोदि॑ताय॒ स्वाहा॑ सुव॒र्गाय॒ स्वाहा॑ लो॒काय॒ स्वाहा॒ सर्व॑स्मै॒ स्वाहा᳚॥५०॥

%7.3.0.0
{\anuvakamend[{श॒ताया॒ष्टात्रिꣳ॑शत्}]}%॥20॥

%7.3.0.0

{\anuvakamend[{प्र॒जवं॑ ब्रह्मवा॒दिनः॒ किमे॒ष वा आ॒प्त आ॑दि॒त्या उ॒भयोः᳚ प्र॒जाप॑ति॒रन्वा॑य॒न्निन्द्रो॒ वै स॒दृङ्ङिन्द्रो॒ वै शि॑थि॒लः प्र॒जाप॑तिरकामयतान्ना॒दः सा वि॒राड॒सावा॑दि॒त्यो᳚\-ऽर्वाङ्भू॒तमा मे॒\-ऽग्निना॒ स्वाहा॒धिन्द॒द्भ्यो᳚\-ऽञ्ज्ये॒ताय॑ कृ॒ष्णायौष॑धीभ्यो॒ वन॒स्पति॑भ्यो विꣳश॒तिः}]%॥20॥
}
%%% END PRASHNA

\sect{तृतीयः प्रश्नः}\setcounter{anuvakam}{0}
\dnsub{तैत्तिरीयसंहितायां सप्तमकाण्डे तृतीयः प्रश्नः}
%7.3.1.0
%7.3.1.1
प्र॒जवं॒ वा ए॒तेन॑ यन्ति॒ यद्द॑श॒ममहः॑ पापाव॒हीयं॒ वा ए॒तेन॑ भवन्ति॒ यद्द॑श॒ममह॒र्यो वै प्र॒जवं॑ य॒तामप॑थेन प्रति॒पद्य॑ते॒ यः स्था॒णुꣳ हन्ति॒ यो भ्रेषं॒ न्येति॒ स ही॑यते॒ स यो वै द॑श॒मे\-ऽह॑न्नविवा॒क्य उ॑पह॒न्यते॒ स ही॑यते॒ तस्मै॒ य उप॑हताय॒ व्याह॒ तमे॒वान्वा॒रभ्य॒ सम॑श्ञु॒ते\-ऽथ॒ यो व्याह॒ सः॥१॥

%7.3.1.2
ही॒य॒ते॒ तस्मा᳚द्दश॒मे\-ऽह॑न्नविवा॒क्य उप॑हताय॒ न व्युच्य॒मथो॒ खल्वा॑हुर्य॒ज्ञस्य॒ वै समृ॑द्धेन दे॒वाः सु॑व॒र्गं लो॒कमा॑यन् य॒ज्ञस्य॒ व्यृ॑द्धे॒नासु॑रा॒न्परा॑भावय॒न्निति॒ यत्खलु॒ वै य॒ज्ञस्य॒ समृ॑द्धं॒ तद्यज॑मानस्य॒ यद्व्यृ॑द्धं॒ तद्भ्रातृ॑व्यस्य॒ स यो वै द॑श॒मे\-ऽह॑न्नविवा॒क्य उ॑पह॒न्यते॒ स ए॒वाति॑ रेचयति॒ ते ये बाह्या॑ दृशी॒कवः॑॥२॥

%7.3.1.3
स्युस्ते वि ब्रू॑यु॒र्यदि॒ तत्र॒ न वि॒न्देयु॑रन्तःसद॒साद्व्युच्यं॒ यदि॒ तत्र॒ न वि॒न्देयु॑र्गृ॒हप॑तिना॒ व्युच्य॒न्तद्व्युच्य॑मे॒वाथ॒ वा ए॒तथ्स॑र्परा॒ज्ञिया॑ ऋ॒ग्भिः स्तु॑वन्ती॒यं वै सर्प॑तो॒ राज्ञी॒ यद्वा अ॒स्यां किं चार्च॑न्ति॒ यदा॑नृ॒चुस्तेने॒यꣳ स॑र्परा॒ज्ञी ते यदे॒व किं च॑ वा॒चानृ॒चुर्यद॒तो\-ऽध्य॑र्चि॒तारः॑॥३॥

%7.3.1.4
तदु॒भय॑मा॒प्त्वाव॒रुध्योत्ति॑ष्ठा॒मेति॒ ताभि॒र्मन॑सा स्तुवते॒ न वा इ॒माम॑श्वर॒थो नाश्व॑तरीर॒थः स॒द्यः पर्या᳚प्तुमर्\mbox{}हति॒ मनो॒ वा इ॒माꣳ स॒द्यः पर्या᳚प्तुमर्\mbox{}हति॒ मनः॒ परि॑भवितु॒मथ॒ ब्रह्म॑ वदन्ति॒ परि॑मिता॒ वा ऋचः॒ परि॑मितानि॒ सामा॑नि॒ परि॑मितानि॒ यजू॒ꣳ॒ष्यथै॒तस्यै॒वान्तो॒ नास्ति॒ यद्ब्रह्म॒ तत्प्र॑तिगृण॒त आ च॑क्षीत॒ स प्र॑तिग॒रः॥४॥

%7.3.2.0
{\anuvakamend[{व्याह॒ स दृ॑शी॒कवो᳚\-ऽर्चि॒तारः॒ स एक॑ञ्च}]}%॥१॥

%7.3.2.1
ब्र॒ह्म॒वा॒दिनो॑ वदन्ति॒ किं द्वा॑दशा॒हस्य॑ प्रथ॒मेनाह्न॒र्त्विजां॒ यज॑मानो वृङ्क्त॒ इति॒ तेज॑ इन्द्रि॒यमिति॒ किं द्वि॒तीये॒नेति॑ प्रा॒णान॒न्नाद्य॒मिति॒ किं तृ॒तीये॒नेति॒ त्रीनि॒माल्लोण॒कानिति॒ किं च॑तु॒र्थेनेति॒ चतु॑ष्पदः प॒शूनिति॒ किम्प॑ञ्च॒मेनेति॒ पञ्चा᳚क्षराम्प॒ङ्क्तिमिति॒ किꣳ ष॒ष्ठेनेति॒ षडृ॒तूनिति॒ किꣳ स॑प्त॒मेनेति॑ स॒प्तप॑दा॒ꣳ॒ शक्व॑री॒मिति॑॥५॥

%7.3.2.2
किम॑ष्ट॒मेनेत्य॒ष्टाक्ष॑रां गाय॒त्रीमिति॒ किं न॑व॒मेनेति॑ त्रि॒वृत॒ꣴ॒ स्तोम॒मिति॒ किं द॑श॒मेनेति॒ दशा᳚क्षरां वि॒राज॒मिति॒ किमे॑काद॒शेनेत्येका॑दशाक्षरां त्रि॒ष्टुभ॒मिति॒ किं द्वा॑द॒शेनेति॒ द्वाद॑शाक्षरां॒ जग॑ती॒मित्ये॒ताव॒द्वा अ॑स्ति॒ याव॑दे॒तद्याव॑दे॒वास्ति॒ तदे॑षां वृङ्क्ते॥६॥

%7.3.3.0
{\anuvakamend[{शक्व॑री॒मित्येक॑चत्वारिꣳशच्च}]}%॥२॥

%7.3.3.1
ए॒ष वा आ॒प्तो द्वा॑दशा॒हो यत्त्र॑योदशरा॒त्रः स॑मा॒नꣴ ह्ये॑तदह॒र्यत्प्रा॑य॒णीय॑श्चोदय॒नीय॑श्च॒ त्र्य॑तिरात्रो भवति॒ त्रय॑ इ॒मे लो॒का ए॒षां लो॒काना॒माप्त्यै᳚ प्रा॒णो वै प्र॑थ॒मो॑\-ऽतिरा॒त्रो व्या॒नो द्वि॒तीयो॑\-ऽपा॒नस्तृ॒तीयः॑ प्राणापानोदा॒नेष्वे॒वान्नाद्ये॒ प्रति॑ तिष्ठन्ति॒ सर्व॒मायु॑र्यन्ति॒ य ए॒वं वि॒द्वाꣳ॑सस्त्रयोदशरा॒त्रमास॑ते॒ तदा॑हु॒र्वाग्वा ए॒षा वित॑ता॥७॥

%7.3.3.2
यद्द्वा॑दशा॒हस्तां विच्छि॑न्द्यु॒र्यन्मध्ये॑\-ऽतिरा॒त्रं कु॒र्युरु॑प॒दासु॑का गृ॒हप॑ते॒र्वाख्स्या॑दु॒परि॑ष्टाच्छन्दो॒माना᳚म्महाव्र॒तं कु॑र्वन्ति॒ सन्त॑तामे॒व वाच॒मव॑ रुन्द्ध॒ते\-ऽनु॑पदासुका गृ॒हप॑ते॒र्वाग्भ॑वति प॒शवो॒ वै छ॑न्दो॒मा अन्न॑म्महाव्र॒तं यदु॒परि॑ष्टाच्छन्दो॒माना᳚\-म्महाव्र॒तं कु॒र्वन्ति॑ प॒शुषु॑ चै॒वान्नाद्ये॑ च॒ प्रति॑ तिष्ठन्ति॥८॥

%7.3.4.0
{\anuvakamend[{वित॑ता॒ त्रिच॑त्वारिꣳशच्च}]}%॥३॥

%7.3.4.1
आ॒दि॒त्या अ॑कामयन्तो॒भयो᳚र्लो॒कयोर्॑ऋध्नुया॒मेति॒ त ए॒तं च॑तुर्दशरा॒त्रम॑पश्य॒न्तमाह॑र॒न्तेना॑यजन्त॒ ततो॒ वै त उ॒भयो᳚र्लो॒कयो॑रार्ध्नुवन्न॒स्मिꣴश्चा॒मुष्मिꣴ॑श्च॒ य ए॒वं वि॒द्वाꣳस॑श्चतुर्दशरा॒त्रमास॑त उ॒भयो॑रे॒व लो॒कयोर्\mbox{}॑ऋध्नुवन्त्य॒\-स्मिꣴश्चा॒मुष्मिꣴ॑श्च चतुर्दशरा॒त्रो भ॑वति स॒प्त ग्रा॒म्या ओष॑धयः स॒प्तार॒ण्या उ॒भयी॑षा॒मव॑रुद्ध्यै॒ यत्प॑रा॒चीना॑नि पृ॒ष्ठानि॑॥९॥

%7.3.4.2
भव॑न्त्य॒मुमे॒व तैर्लो॒कम॒भि ज॑यन्ति॒ यत्प्र॑ती॒चीना॑नि पृ॒ष्ठानि॒ भव॑न्ती॒ममे॒व तैर्लो॒कम॒भि ज॑यन्ति त्रयस्त्रि॒ꣳ॒शौ म॑ध्य॒तः स्तोमौ॑ भवतः॒ साम्रा᳚ज्यमे॒व ग॑च्छन्त्यधिरा॒जौ भ॑वतो\-ऽधिरा॒जा ए॒व स॑मा॒नानां᳚ भवन्त्यतिरा॒त्राव॒भितो॑ भवतः॒ परि॑गृहीत्यै॥१०॥

%7.3.5.0
{\anuvakamend[{पृ॒ष्ठानि॒ चतु॑स्त्रिꣳशच्च}]}%॥४॥

%7.3.5.1
प्र॒जाप॑तिः सुव॒र्गं लो॒कमै॒त्तं दे॒वा अन्वा॑य॒न्ताना॑दि॒त्याश्च॑ प॒शव॒श्चान्वा॑य॒न्ते दे॒वा अ॑ब्रुव॒न् यान्प॒शूनु॒पाजी॑विष्म॒ त इ॒मे᳚\-ऽन्वाग्म॒न्निति॒ तेभ्य॑ ए॒तं च॑तुर्दशरा॒त्रम्प्रत्यौ॑ह॒न्त आ॑दि॒त्याः पृ॒ष्ठैः सु॑व॒र्गं लो॒कमारो॑हन्त्र्य॒हाभ्या॑म॒स्मिल्लोँ॒के प॒शून्प्रत्यौ॑हन्पृ॒ष्ठैरा॑दि॒त्या अ॒मुष्मि॑ल्लोँ॒क आर्ध्नु॑वन्त्र्य॒हाभ्या॑म॒स्मिन्॥११॥

%7.3.5.2
लो॒के प॒शवो॒ य ए॒वं वि॒द्वाꣳस॑श्चतुर्दशरा॒त्रमास॑त उ॒भयो॑रे॒व लो॒कयोर्॑ऋध्नुवन्त्य॒स्मिꣴश्चा॒मुष्मिꣴ॑श्च पृ॒ष्ठैरे॒वामुष्मि॑ल्लोँ॒क ऋ॑ध्नु॒वन्ति॑ त्र्य॒हाभ्या॑म॒स्मिल्लोँ॒के ज्योति॒र्गौरायु॒रिति॑ त्र्य॒हो भ॑वती॒यं वाव ज्योति॑र॒न्तरि॑क्षं॒ गौर॒सावायु॑रि॒माने॒व लो॒कान॒भ्यारो॑हन्ति॒ यद॒न्यतः॑ पृ॒ष्ठानि॒ स्युर्विवि॑वधꣴ स्या॒न्मध्ये॑ पृ॒ष्ठानि॑ भवन्ति सविवध॒त्वाय॑॥१२॥

%7.3.5.3
ओजो॒ वै वी॒र्यं॑ पृ॒ष्ठान्योज॑ ए॒व वी॒र्य॑म्मध्य॒तो द॑धते बृहद्रथन्त॒रा\-भ्यां᳚ यन्ती॒यं वाव र॑थन्त॒रम॒सौ बृ॒हदा॒भ्यामे॒व य॒न्त्यथो॑ अ॒नयो॑रे॒व प्रति॑ तिष्ठन्त्ये॒ते वै य॒ज्ञस्या᳚ञ्ज॒साय॑नी स्रु॒ती ताभ्या॑मे॒व सु॑व॒र्गं लो॒कं य॑न्ति॒ परा᳚ञ्चो॒ वा ए॒ते सु॑व॒र्गं लो॒कम॒भ्यारो॑हन्ति॒ ये प॑रा॒चीना॑नि पृ॒ष्ठान्यु॑प॒यन्ति॑ प्र॒त्यङ्त्र्य॒हो भ॑वति प्र॒त्यव॑रूढ्या॒ अथो॒ प्रति॑ष्ठित्या उ॒भयो᳚र्लो॒कयोर्\mbox{}॑ऋ॒द्ध्वोत्ति॑ष्ठन्ति॒ चतु॑र्दशै॒तास्तासां॒ या दश॒ दशा᳚क्षरा वि॒राडन्नं॑ वि॒राड्वि॒राजै॒वान्नाद्य॒मव॑ रुन्धते॒ याश्चत॑स्र॒श्चत॑स्रो॒ दिशो॑ दि॒क्ष्वे॑व प्रति॑ तिष्ठन्त्यतिरा॒त्राव॒भितो॑ भवतः॒ परि॑गृहीत्यै॥१३॥

%7.3.6.0
{\anuvakamend[{आर्ध्नु॑वन्त्र्य॒हाभ्या॑म॒स्मिन्थ्स॑विवध॒त्वाय॒ प्रति॑ष्ठित्या॒ एक॑त्रिꣳशच्च}]}%॥५॥

%7.3.6.1
इन्द्रो॒ वै स॒दृङ्दे॒वता॑भिरासी॒थ्स न व्या॒वृत॑मगच्छ॒थ्स प्र॒जाप॑ति॒मुपा॑धाव॒त्तस्मा॑ ए॒तम्प॑ञ्चदशरा॒त्रम्प्राय॑च्छ॒त्तमाह॑र॒त् तेना॑यजत॒ ततो॒ वै सो᳚\-ऽन्याभि॑र्दे॒वता॑भिर्व्या॒वृत॑मगच्छ॒द्य ए॒वं वि॒द्वाꣳसः॑ पञ्चदशरा॒त्रमास॑ते व्या॒वृत॑मे॒व पा॒प्मना॒ भ्रातृ॑व्येण गच्छन्ति॒ ज्योति॒र्गौरायु॒रिति॑ त्र्य॒हो भ॑वती॒यं वाव ज्योति॑र॒न्तरि॑क्षम्॥१४॥

%7.3.6.2
गौर॒सावायु॑रे॒ष्वे॑व लो॒केषु॒ प्रति॑ तिष्ठ॒न्त्यस॑त्त्रं॒ वा ए॒तद्यद॑छन्दो॒मं यच्छ॑न्दो॒मा भव॑न्ति॒ तेन॑ स॒त्त्रं दे॒वता॑ ए॒व पृ॒ष्ठैरव॑ रुन्धते प॒शूञ्छ॑न्दो॒मैरोजो॒ वा वी॒र्यं॑ पृ॒ष्ठानि॑ प॒शव॑श्छन्दो॒मा ओज॑स्ये॒व वी॒र्ये॑ प॒शुषु॒ प्रति॑ तिष्ठन्ति पञ्चदशरा॒त्रो भ॑वति पञ्चद॒शो वज्रो॒ वज्र॑मे॒व भ्रातृ॑व्येभ्यः॒ प्र ह॑रन्त्यतिरा॒त्राव॒भितो॑ भवत इन्द्रि॒यस्य॒ परि॑गृहीत्यै॥१५॥

%7.3.7.0
{\anuvakamend[{अ॒न्तरि॑क्षमिन्द्रि॒यस्यैक॑ञ्च}]}%॥६॥

%7.3.7.1
इन्द्रो॒ वै शि॑थि॒ल इ॒वाप्र॑तिष्ठित आसी॒थ्सो\-ऽसु॑रेभ्यो\-ऽबिभे॒थ्स प्र॒जाप॑ति॒मुपा॑धाव॒त्तस्मा॑ ए॒तम्प॑ञ्चदशरा॒त्रं वज्र॒म्प्राय॑च्छ॒त् तेनासु॑रान्परा॒भाव्य॑ वि॒जित्य॒ श्रिय॑मगच्छदग्नि॒ष्टुता॑ पा॒प्मानं॒ निर॑दहत पञ्चदशरा॒त्रेणौजो॒ बल॑मिन्द्रि॒यं वी॒र्य॑मा॒त्मन्न॑धत्त॒ य ए॒वं वि॒द्वाꣳसः॑ पञ्चदशरा॒त्रमास॑ते॒ भ्रातृ॑व्याने॒व प॑रा॒भाव्य॑ वि॒जित्य॒ श्रियं॑ गच्छन्त्यग्नि॒ष्टुता॑ पा॒प्मानं॒ निः॥१६॥

%7.3.7.2
द॒ह॒न्ते॒ प॒ञ्च॒द॒श॒रा॒त्रेणौजो॒ बल॑मिन्द्रि॒यं वी॒र्य॑मा॒त्मन्द॑धत ए॒ता ए॒व प॑श॒व्याः᳚ पञ्च॑दश॒ वा अ॑र्धमा॒सस्य॒ रात्र॑यो\-ऽ\-र्धमास॒शः सं॑वथ्स॒र आ᳚प्यते संवथ्स॒रम्प॒शवो\-ऽनु॒ प्र जा॑यन्ते॒ तस्मा᳚त्पश॒व्या॑ ए॒ता ए॒व सु॑व॒र्ग्याः᳚ पञ्च॑दश॒ वा अ॑र्धमा॒सस्य॒ रात्र॑यो\-ऽर्धमास॒शः सं॑वथ्स॒र आ᳚प्यते संवथ्स॒रः सु॑व॒र्गो लो॒कस्तस्मा᳚थ्सुव॒र्ग्या᳚ ज्योति॒र्गौरायु॒रिति॑ त्र्य॒हो भ॑वती॒यं वाव ज्योति॑र॒न्तरि॑क्षम्॥१७॥

%7.3.7.3
गौर॒सावायु॑रि॒माने॒व लो॒कान॒भ्यारो॑हन्ति॒ यद॒न्यतः॑ पृ॒ष्ठानि॒ स्युर्विवि॑वधꣴ स्या॒न्मध्ये॑ पृ॒ष्ठानि॑ भवन्ति सविवध॒त्वायौजो॒ वै वी॒र्यं॑ पृ॒ष्ठान्योज॑ ए॒व वी॒र्य॑म्मध्य॒तो द॑धते बृहद्रथन्त॒रा\-भ्यां᳚ यन्ती॒यं वाव र॑थन्त॒रम॒सौ बृ॒हदा॒भ्यामे॒व य॒न्त्यथो॑ अ॒नयो॑रे॒व प्रति॑ तिष्ठन्त्ये॒ते वै य॒ज्ञस्या᳚ञ्ज॒साय॑नी स्रु॒ती ताभ्या॑मे॒व सु॑व॒र्गं लो॒कम्॥१८॥

%7.3.7.4
य॒न्ति॒ परा᳚ञ्चो॒ वा ए॒ते सु॑व॒र्गं लो॒कम॒भ्यारो॑हन्ति॒ ये प॑रा॒चीना॑नि पृ॒ष्ठान्यु॑प॒यन्ति॑ प्र॒त्यङ्त्र्य॒हो भ॑वति प्र॒त्यव॑रूढ्या॒ अथो॒ प्रति॑ष्ठित्या उ॒भयो᳚र्लो॒कयोर्॑ऋ॒द्ध्वोत्ति॑ष्ठन्ति॒ पञ्च॑दशै॒तास्तासां॒ या दश॒ दशा᳚क्षरा वि॒राडन्नं॑ वि॒राड्वि॒राजै॒वान्नाद्य॒\-मव॑ रुन्धते॒ याः पञ्च॒ पञ्च॒ दिशो॑ दि॒क्ष्वे॑व प्रति॑ तिष्ठन्त्यतिरा॒त्राव॒भितो॑ भवत इन्द्रि॒यस्य॑ वी॒र्य॑स्य प्र॒जायै॑ पशू॒नां परि॑गृहीत्यै॥१९॥

%7.3.8.0
{\anuvakamend[{ग॒च्छ॒न्त्य॒ग्नि॒ष्टुता॑ पा॒प्मान॒न्निर॒न्तरि॑क्षल्लोँ॒कं प्र॒जायै॒ द्वे च॑}]}%॥७॥

%7.3.8.1
प्र॒जाप॑तिरकामयतान्ना॒दः स्या॒मिति॒ स ए॒तꣳ स॑प्तदशरा॒त्रम॑पश्य॒त्तमाह॑र॒त्तेना॑यजत॒ ततो॒ वै सो᳚\-ऽन्ना॒दो॑\-ऽभव॒द्य ए॒वं वि॒द्वाꣳसः॑ सप्तदशरा॒त्रमास॑ते\-ऽन्ना॒दा ए॒व भ॑वन्ति पञ्चा॒हो भ॑वति॒ पञ्च॒ वा ऋ॒तवः॑ संवथ्स॒र ऋ॒तुष्वे॒व सं॑वथ्स॒रे प्रति॑ तिष्ठ॒न्त्यथो॒ पञ्चा᳚क्षरा प॒ङ्क्तिः पाङ्क्तो॑ य॒ज्ञो य॒ज्ञमे॒वाव॑ रुन्ध॒ते\-ऽस॑त्त्रं॒ वा ए॒तत्॥२०॥

%7.3.8.2
यद॑छन्दो॒मं यच्छ॑न्दो॒मा भव॑न्ति॒ तेन॑ स॒त्त्रं दे॒वता॑ ए॒व पृ॒ष्ठैरव॑ रुन्धते प॒शूञ्छ॑न्दो॒मैरोजो॒ वै वी॒र्यं॑ पृ॒ष्ठानि॑ प॒शव॑श्छन्दो॒मा ओज॑स्ये॒व वी॒र्ये॑ प॒शुषु॒ प्रति॑ तिष्ठन्ति सप्तदशरा॒त्रो भ॑वति सप्तद॒शः प्र॒जाप॑तिः प्र॒जाप॑ते॒राप्त्या॑ अतिरा॒त्राव॒भितो॑ भवतो॒\-ऽन्नाद्य॑स्य॒ परि॑गृहीत्यै॥२१॥

%7.3.9.0
{\anuvakamend[{ए॒तथ्स॒प्तत्रिꣳ॑श्चच्च}]}%॥८॥

%7.3.9.1
सा वि॒राड्वि॒क्रम्या॑तिष्ठ॒द्ब्रह्म॑णा दे॒वेष्वन्ने॒नासु॑रेषु॒ ते दे॒वा अ॑कामयन्तो॒भय॒ꣳ॒ सं वृ॑ञ्जीमहि॒ ब्रह्म॒ चान्नं॒ चेति॒ त ए॒ता विꣳ॑श॒तिꣳ रात्री॑रपश्य॒न्ततो॒ वै त उ॒भय॒ꣳ॒ सम॑वृञ्जत॒ ब्रह्म॒ चान्नं॑ च ब्रह्मवर्च॒सिनो᳚\-ऽन्ना॒दा अ॑भव॒न् य ए॒वं वि॒द्वाꣳस॑ ए॒ता आस॑त उ॒भय॑मे॒व सं वृ॑ञ्जते॒ ब्रह्म॒ चान्नं॑ च॥२२॥

%7.3.9.2
ब्र॒ह्म॒व॒र्च॒सिनो᳚\-ऽन्ना॒दा भ॑वन्ति॒ द्वे वा ए॒ते वि॒राजौ॒ तयो॑रे॒व नाना॒ प्रति॑ तिष्ठन्ति वि॒ꣳ॒शो वै पुरु॑षो॒ दश॒ हस्त्या॑ अ॒ङ्गुल॑यो॒ दश॒ पद्या॒ यावा॑ने॒व पुरु॑ष॒स्तमा॒प्त्वोत्ति॑ष्ठन्ति॒ ज्योति॒र्गौरायु॒रिति॑ त्र्य॒हा भ॑वन्ती॒यं वाव ज्योति॑र॒न्तरि॑क्षं॒ गौर॒सावायु॑रि॒माने॒व लो॒कान॒भ्यारो॑हन्त्यभिपू॒र्वं त्र्य॒हा भ॑वन्त्यभिपू॒र्वमे॒व सु॑व॒र्गम्॥२३॥

%7.3.9.3
लो॒कम॒भ्यारो॑हन्ति॒ यद॒न्यतः॑ पृ॒ष्ठानि॒ स्युर्विवि॑वधꣴ स्या॒न्मध्ये॑ पृ॒ष्ठानि॑ भवन्ति सविवध॒त्वायौजो॒ वै वी॒र्यं॑ पृ॒ष्ठान्योज॑ ए॒व वी॒र्य॑म्मध्य॒तो द॑धते बृहद्रथन्त॒रा\-भ्यां᳚ यन्ती॒यं वाव र॑थन्त॒रम॒सौ बृ॒हदाभ्यामे॒व य॒न्त्यथो॑ अ॒नयो॑रे॒व प्रति॑ तिष्ठन्त्ये॒ते वै य॒ज्ञस्या᳚ञ्ज॒साय॑नी स्रु॒ती ताभ्या॑मे॒व सु॑व॒र्गं लो॒कं य॑न्ति॒ परा᳚ञ्चो॒ वा ए॒ते सु॑व॒र्गं लो॒कम॒भ्यारो॑हन्ति॒ ये प॑रा॒चीना॑नि पृ॒ष्ठान्यु॑प॒यन्ति॑ प्र॒त्यङ्त्र्य॒हो भ॑वति प्र॒त्यव॑रूढ्या॒ अथो॒ प्रति॑ष्ठित्या उ॒भयो᳚र्लो॒कयोर्॑ ऋ॒द्ध्वोत्ति॑ष्ठन्त्यतिरा॒त्राव॒भितो॑ भवतो ब्रह्मवर्च॒सस्या॒न्नाद्य॑स्य॒ परि॑गृहीत्यै॥२४॥

%7.3.10.0
{\anuvakamend[{वृ॒ञ्ज॒ते॒ ब्रह्म॒ चान्न॑ञ्च सुव॒र्गमे॒ते सु॑व॒र्गन्त्रयो॑विꣳशतिश्च}]}%॥९॥

%7.3.10.1
अ॒सावा॑दि॒त्यो᳚\-ऽस्मिल्लोँ॒क आ॑सी॒त्तं दे॒वाः पृ॒ष्ठैः प॑रि॒गृह्य॑ सुव॒र्गं लो॒कम॑गमय॒न्परै॑र॒वस्ता॒त्पर्य॑गृह्णन्दिवाकी॒र्त्ये॑न सुव॒र्गे लो॒के प्रत्य॑स्थापय॒न्परैः᳚ प॒रस्ता॒त्पर्य॑गृह्णन्पृ॒ष्ठैरु॒पावा॑रोह॒न्थ्स वा अ॒सावा॑दि॒त्यो॑\-ऽमुष्मि॑ल्लोँ॒के परै॑रुभ॒यतः॒ परि॑गृहीतो॒ यत्पृ॒ष्ठानि॒ भव॑न्ति सुव॒र्गमे॒व तैर्लो॒कं यज॑माना यन्ति॒ परै॑र॒वस्ता॒त्परि॑ गृह्णन्ति दिवाकी॒र्त्ये॑न॥२५॥

%7.3.10.2
सु॒व॒र्गे लो॒के प्रति॑ तिष्ठन्ति॒ परैः᳚ प॒रस्ता॒त्परि॑ गृह्णन्ति पृ॒ष्ठैरु॒पाव॑रोहन्ति॒ यत्परे॑ प॒रस्ता॒न्न स्युः परा᳚ञ्चः सुव॒र्गाल्लो॒कान्निष्प॑द्येर॒न् यद॒वस्ता॒न्न स्युः प्र॒जा निर्द॑हेयुर॒भितो॑ दिवाकी॒र्त्यं॑ परः॑सामानो भवन्ति सुव॒र्ग ए॒वैना᳚ल्लोँ॒क उ॑भ॒यतः॒ परि॑ गृह्णन्ति॒ यज॑माना॒ वै दि॑वाकी॒र्त्यꣳ॑ संवथ्स॒रः परः॑सामानो॒\-ऽभितो॑ दिवाकी॒र्त्यं॑ परः॑ सामानो भवन्ति संवथ्स॒र ए॒वोभ॒यतः॑॥२६॥

%7.3.10.3
प्रति॑ तिष्ठन्ति पृ॒ष्ठं वै दि॑वाकी॒र्त्य॑म्पा॒र्श्वे परः॑सामानो॒\-ऽभितो॑ दिवाकी॒र्त्यं॑ परः॑सामानो भवन्ति॒ तस्मा॑द॒भितः॑ पृ॒ष्ठम्पा॒र्श्वे भूयि॑ष्ठा॒ ग्रहा॑ गृह्यन्ते॒ भूयि॑ष्ठꣳ शस्यते य॒ज्ञस्यै॒व तन्म॑ध्य॒तो ग्र॒न्थं ग्र॑थ्न॒न्त्यवि॑स्रꣳसाय स॒प्त गृ॑ह्यन्ते स॒प्त वै शी॑र्\mbox{}ष॒ण्याः᳚ प्रा॒णाः प्रा॒णाने॒व यज॑मानेषु दधति॒ यत्प॑रा॒चीना॑नि पृ॒ष्ठानि॒ भव॑न्त्य॒मुमे॒व तैर्लो॒कम॒भ्यारो॑हन्ति॒ यदि॒मं लो॒कं न॥२७॥

%7.3.10.4
प्र॒त्य॒व॒रोहे॑यु॒रुद्वा॒ माद्ये॑यु॒र्यज॑मानाः॒ प्र वा॑ मीयेर॒न् यत्प्र॑ती॒चीना॑नि पृ॒ष्ठानि॒ भव॑न्ती॒ममे॒व तैर्लो॒कम्प्र॒त्यव॑रोह॒न्त्यथो॑ अ॒स्मिन्ने॒व लो॒के प्रति॑ तिष्ठ॒न्त्यनु॑न्मादा॒येन्द्रो॒ वा अप्र॑तिष्ठित आसी॒थ्स प्र॒जाप॑ति॒मुपा॑धाव॒त्तस्मा॑ ए॒तमे॑कविꣳशतिरा॒त्रम्प्राय॑च्छ॒त्तमाह॑र॒त्तेना॑यजत॒ ततो॒ वै स प्रत्य॑तिष्ठ॒द्ये ब॑हुया॒जिनो\-ऽप्र॑तिष्ठिताः॥२८॥

%7.3.10.5
स्युस्त ए॑कविꣳशतिरा॒त्रमा॑सीर॒न्द्वाद॑श॒ मासाः॒ पञ्च॒र्तव॒स्त्रय॑ इ॒मे लो॒का अ॒सावा॑दि॒त्य ए॑कवि॒ꣳ॒श ए॒ताव॑न्तो॒ वै दे॑वलो॒कास्तेष्वे॒व य॑थापू॒र्वं प्रति॑ तिष्ठन्त्य॒सावा॑दि॒त्यो न व्य॑रोचत॒ स प्र॒जाप॑ति॒मुपा॑धाव॒त्तस्मा॑ ए॒तमे॑कविꣳशति\-रा॒त्रम्प्राय॑च्छ॒त्तमाह॑र॒त्तेना॑यजत॒ ततो॒ वै सो॑\-ऽरोचत॒ य ए॒वं वि॒द्वाꣳ॑स एकविꣳशतिरा॒त्रमास॑ते॒ रोच॑न्त ए॒वैक॑विꣳशतिरा॒त्रो भ॑वति॒ रुग्वा ए॑कवि॒ꣳ॒शो रुच॑मे॒व ग॑च्छ॒न्त्यथो᳚ प्रति॒ष्ठामे॒व प्र॑ति॒ष्ठा ह्ये॑कवि॒ꣳ॒शो॑-\-ऽ तिरा॒त्राव॒भितो॑ भवतो ब्रह्मवर्च॒सस्य॒ परि॑गृहीत्यै॥२९॥

%7.3.11.0
{\anuvakamend[{गृ॒ह्ण॒न्ति॒ दि॒वा॒की॒र्त्ये॑नै॒वोभ॒यतो॒ नाप्र॑तिष्ठिता॒ आस॑त॒ एक॑विꣳशतिश्च}]}%॥10॥

%7.3.11.1
अ॒र्वाङ्य॒ज्ञः सं क्रा॑मत्व॒मुष्मा॒दधि॒ माम॒भि। ऋषी॑णां॒ यः पु॒रोहि॑तः। निर्दे॑वं॒ निर्वी॑रं कृ॒त्वा विष्क॑न्धं॒ तस्मि॑न् हीयतां॒ यो᳚\-ऽस्मान्द्वेष्टि॑। शरी॑रं यज्ञशम॒लं कुसी॑दं॒ तस्मि᳚न्थ्सीदतु॒ यो᳚\-ऽस्मान्द्वेष्टि॑। यज्ञ॑ य॒ज्ञस्य॒ यत्तेज॒स्तेन॒ सं क्रा॑म॒ माम॒भि। ब्रा॒ह्म॒णानृ॒त्विजो॑ दे॒वान् य॒ज्ञस्य॒ तप॑सा॒ ते सवा॒हमा हु॑वे। इ॒ष्टेन॑ प॒क्वमुप॑॥३०॥

%7.3.11.2
ते॒ हु॒वे॒ स॒वा॒हम्। सन्ते॑ वृञ्जे सुकृ॒तꣳ सं प्र॒जां प॒शून्। प्रै॒षान्थ्सा॑मिधे॒नीरा॑घा॒रावाज्य॑भागा॒वाश्रु॑तम्प्र॒त्याश्रु॑त॒मा शृ॑णामि ते। प्र॒या॒जा॒नू॒या॒जान्थ्स्वि॑ष्ट॒कृत॒मिडा॑मा॒शिष॒ आ वृ॑ञ्जे॒ सुवः॑। अ॒ग्निनेन्द्रे॑ण॒ सोमे॑न॒ सर॑स्वत्या॒ विष्णु॑ना दे॒वता॑भिः। या॒ज्या॒नु॒वा॒क्या᳚भ्या॒मुप॑ ते हुवे स्वा॒हं य॒ज्ञमा द॑दे ते॒ वष॑ट्कृतम्। स्तु॒तꣳ श॒स्त्रम्प्र॑तिग॒रं ग्रह॒मिडा॑मा॒शिषः॑॥३१॥

%7.3.11.3
आ वृ॑ञ्जे॒ सुवः॑। प॒त्नी॒सं॒या॒जानुप॑ ते हुवे सवा॒हꣳ स॑मिष्टय॒जुरा द॑दे॒ तव॑। प॒शून्थ्सु॒तम्पु॑रो॒डाशा॒न्थ्सव॑ना॒न्योत य॒ज्ञम्। दे॒वान्थ्सेन्द्रा॒नुप॑ ते हुवे सवा॒हम॒ग्निमु॑खा॒न्थ्सोम॑वतो॒ ये च॒ विश्वे᳚॥३२॥

%7.3.12.0
{\anuvakamend[{उप॒ ग्रह॒मिडा॑मा॒शिषो॒ द्वात्रिꣳ॑शच्च}]}%॥11॥

%7.3.12.1
भू॒तम्भव्य॑म्भवि॒ष्यद्वष॒ट्थ्स्वाहा॒ नम॒ ऋख्साम॒ यजु॒र्वष॒ट्थ्स्वाहा॒ नमो॑ गाय॒त्री त्रि॒ष्टुब्जग॑ती॒ वष॒ट्थ्स्वाहा॒ नमः॑ पृथि॒व्य॑न्तरि॑क्षं॒ द्यौर्वष॒ट्थ्स्वाहा॒ नमो॒\-ऽग्निर्वा॒युः सूर्यो॒ वष॒ट्थ्स्वाहा॒ नमः॑ प्रा॒णो व्या॒नो॑\-ऽपा॒नो वष॒ट्थ्स्वाहा॒ नमो\-ऽन्नं॑ कृ॒षिर्वृष्टि॒र्वष॒ट्थ्स्वाहा॒ नमः॑ पि॒ता पु॒त्रः पौत्रो॒ वष॒ट्थ्स्वाहा॒ नमो॒ भूर्भुवः॒सुव॒र्वष॒ट्थ्स्वाहा॒ नमः॑॥३३॥

%7.3.13.0
{\anuvakamend[{भुव॑श्च॒त्वारि॑ च}]}%॥12॥

%7.3.13.1
आ मे॑ गृ॒हा भ॑वं᳚ त्वा प्र॒जा म॒ आ मा॑ य॒ज्ञो वि॑शतु वी॒र्या॑वान्। आपो॑ दे॒वीर्य॒ज्ञिया॒ मा वि॑शन्तु स॒हस्र॑स्य मा भू॒मा मा प्र हा॑सीत्। आ मे॒ ग्रहो॑ भव॒त्वा पु॑रो॒रुख्स्तु॑तश॒स्त्रे मा वि॑शताꣳ स॒मीची᳚। आ॒दि॒त्या रु॒द्रा वस॑वो मे सद॒स्याः᳚ स॒हस्र॑स्य मा भू॒मा मा प्र हा॑सीत्। आ मा᳚ग्निष्टो॒मो वि॑शतू॒क्थ्य॑श्चातिरा॒त्रो मा वि॑शत्वापिशर्व॒रः। ति॒रोअ॑ह्निया मा॒ सुहु॑ता॒ आ वि॑शन्तु स॒हस्र॑स्य मा भू॒मा मा॒ प्र हा॑सीत्॥३४॥

%7.3.14.0
{\anuvakamend[{अ॒ग्नि॒ष्टो॒मो वि॑शत्व॒ष्टाद॑श च}]}%॥13॥

%7.3.14.1
अ॒ग्निना॒ तपो\-ऽन्व॑भवद्वा॒चा ब्रह्म॑ म॒णिना॑ रू॒पाणीन्द्रे॑ण दे॒वान् वाते॑न प्रा॒णान्थ्सूर्ये॑ण॒ द्याञ्च॒न्द्रम॑सा॒ नक्ष॑त्राणि य॒मेन॑ पितॄन्राज्ञा॑ मनु॒ष्या᳚न्फ॒लेन॑ नादे॒यान॑जग॒रेण॑ स॒र्पान्व्या॒घ्रेणा॑र॒ण्यान्प॒शूञ्छ्ये॒नेन॑ पत॒त्रिणो॒ वृष्णाश्वा॑नृष॒भेण॒ गा ब॒स्तेना॒जा वृ॒ष्णिनावी᳚र्व्री॒हिणान्ना॑नि॒ यवे॒नौष॑धीर्न्य॒ग्रोधे॑न॒ वन॒स्पती॑नुदु॒म्बरे॒णोर्ज॑ङ्गायत्रि॒या छन्दाꣳ॑सि त्रि॒वृता॒ स्तोमा᳚न्ब्राह्म॒णेन॒ वाचम्᳚॥३५॥

%7.3.15.0
{\anuvakamend[{ब्रा॒ह्म॒णेनैक॑ञ्च}]}%॥14॥

%7.3.15.1
स्वाहा॒धिमाधी॑ताय॒ स्वाहा॒ स्वाहाधी॑त॒म्मन॑से॒ स्वाहा॒ स्वाहा॒ मनः॑ प्र॒जाप॑तये॒ स्वाहा॒ काय॒ स्वाहा॒ कस्मै॒ स्वाहा॑ कत॒मस्मै॒ स्वाहादि॑त्यै॒ स्वाहादि॑त्यै म॒ह्यै᳚ स्वाहादि॑त्यै सुमृडी॒कायै॒ स्वाहा॒ सर॑स्वत्यै॒ स्वाहा॒ सर॑स्वत्यै बृह॒त्यै᳚ स्वाहा॒ सर॑स्वत्यै पाव॒कायै॒ स्वाहा॑ पू॒ष्णे स्वाहा॑ पू॒ष्णे प्र॑प॒थ्या॑य॒ स्वाहा॑ पू॒ष्णे न॒रन्धि॑षाय॒ स्वाहा॒ त्वष्ट्रे॒ स्वाहा॒ त्वष्ट्रे॑ तु॒रीपा॑य॒ स्वाहा॒ त्वष्ट्रे॑ पुरु॒रूपा॑य॒ स्वाहा॒ विष्ण॑वे॒ स्वाहा॒ विष्ण॑वे निखुर्य॒पाय॒ स्वाहा॒ विष्ण॑वे निभूय॒पाय॒ स्वाहा॒ सर्वस्मै॒ स्वाहा᳚॥३६॥

%7.3.16.0
{\anuvakamend[{पु॒रु॒रूपा॑य॒ स्वाहा॒ दश॑ च}]}%॥15॥

%7.3.16.1
द॒द्भ्यः स्वाहा॒ हनू᳚भ्या॒ꣴ॒ स्वाहोष्ठा᳚भ्या॒ꣴ॒ स्वाहा॒ मुखा॑य॒ स्वाहा॒ नासि॑काभ्या॒ꣴ॒ स्वाहा॒क्षीभ्या॒ꣴ॒ स्वाहा॒ कर्णा᳚भ्या॒ꣴ॒ स्वाहा॑ पा॒र इ॒क्षवो॑\-ऽवा॒र्ये᳚भ्यः॒ पक्ष्म॑भ्यः॒ स्वाहा॑वा॒र इ॒क्षवः॑ पा॒र्ये᳚भ्यः॒ पक्ष्म॑भ्यः॒ स्वाहा॑ शी॒र्\mbox{}ष्णे स्वाहा᳚ भ्रू॒भ्याꣴ स्वाहा॑ ल॒लाटा॑य॒ स्वाहा॑ मू॒र्ध्ने स्वाहा॑ म॒स्तिष्का॑य॒ स्वाहा॒ केशे᳚भ्यः॒ स्वाहा॒ वहा॑य॒ स्वाहा᳚ ग्री॒वाभ्यः॒ स्वाहा᳚ स्क॒न्धेभ्यः॒ स्वाहा॒ कीक॑साभ्यः॒ स्वाहा॑ पृ॒ष्टीभ्यः॒ स्वाहा॑ पाज॒स्या॑य॒ स्वाहा॑ पा॒र्श्वाभ्या॒ꣴ॒ स्वाहा᳚॥३७॥

%7.3.16.2
अꣳसा᳚भ्या॒ꣴ॒ स्वाहा॑ दो॒षभ्या॒ꣴ॒ स्वाहा॑ बा॒हुभ्या॒ꣴ॒ स्वाहा॒ जङ्घा᳚भ्या॒ꣴ॒ स्वाहा॒ श्रोणी᳚भ्या॒ꣴ॒ स्वाहो॒रुभ्या॒ꣴ॒ स्वाहा᳚ष्ठी॒वद्भ्या॒ꣴ॒ स्वाहा॒ जङ्घा᳚भ्या॒ꣴ॒ स्वाहा॑ भ॒सदे॒ स्वाहा॑ शिख॒ण्डेभ्यः॒ स्वाहा॑ वाल॒धाना॑य॒ स्वाहा॒ण्डाभ्या॒ꣴ॒ स्वाहा॒ शेपा॑य॒ स्वाहा॒ रेत॑से॒ स्वाहा᳚ प्र॒जाभ्यः॒ स्वाहा᳚ प्र॒जन॑नाय॒ स्वाहा॑ प॒द्भ्यः स्वाहा॑ श॒फेभ्यः॒ स्वाहा॒ लोम॑भ्यः॒ स्वाहा᳚ त्व॒चे स्वाहा॒ लोहि॑ताय॒ स्वाहा॑ मा॒ꣳ॒साय॒ स्वाहा॒ स्नाव॑भ्यः॒ स्वाहा॒स्थभ्यः॒ स्वाहा॑ म॒ज्जभ्यः॒ स्वाहाङ्गे᳚भ्यः॒ स्वाहा॒त्मने॒ स्वाहा॒ सर्व॑स्मै॒ स्वाहा᳚॥३८॥

%7.3.17.0
{\anuvakamend[{पा॒र्श्वाभ्या॒ꣴ॒ स्वाहा॑ म॒ज्जभ्यः॒ स्वाहा॒ षट्च॑}]}%॥16॥

%7.3.17.1
अ॒ञ्ज्ये॒ताय॒ स्वाहा᳚ञ्जिस॒क्थाय॒ स्वाहा॑ शिति॒पदे॒ स्वाहा॒ शिति॑ककुदे॒ स्वाहा॑ शिति॒रन्ध्रा॑य॒ स्वाहा॑ शितिपृ॒ष्ठाय॒ स्वाहा॑ शि॒त्यꣳसा॑य॒ स्वाहा॑ पुष्प॒कर्णा॑य॒ स्वाहा॑ शि॒त्योष्ठा॑य॒ स्वाहा॑ शिति॒भ्रवे॒ स्वाहा॒ शिति॑भसदे॒ स्वाहा᳚ श्वे॒तानू॑काशाय॒ स्वाहा॒ञ्जये॒ स्वाहा॑ ल॒लामा॑य॒ स्वाहासि॑तज्ञवे॒ स्वाहा॑ कृष्णै॒ताय॒ स्वाहा॑ रोहितै॒ताय॒ स्वाहा॑रुणै॒ताय॒ स्वाहे॒दृशा॑य॒ स्वाहा॑ की॒दृशा॑य॒ स्वाहा॑ ता॒दृशा॑य॒ स्वाहा॑ स॒दृशा॑य॒ स्वाहा॒ विस॑दृशाय॒ स्वाहा॒ सुस॑दृ॒शाय॒ स्वाहा॑ रू॒पाय॒ स्वाहा॒ सर्व॑स्मै॒ स्वाहा᳚॥३९॥

%7.3.18.0
{\anuvakamend[{रू॒पाय॒ स्वाहा॒ द्वे च॑}]}%॥17॥

%7.3.18.1
कृ॒ष्णाय॒ स्वाहा᳚ श्वे॒ताय॒ स्वाहा॑ पि॒शङ्गा॑य॒ स्वाहा॑ सा॒रङ्गा॑य॒ स्वाहा॑रु॒णाय॒ स्वाहा॑ गौ॒राय॒ स्वाहा॑ ब॒भ्रवे॒ स्वाहा॑ नकु॒लाय॒ स्वाहा॒ रोहि॑ताय॒ स्वाहा॒ शोणा॑य॒ स्वाहा᳚ श्या॒वाय॒ स्वाहा᳚ श्या॒माय॒ स्वाहा॑ पाक॒लाय॒ स्वाहा॑ सुरू॒पाय॒ स्वाहानु॑रूपाय॒ स्वाहा॒ विरू॑पाय॒ स्वाहा॒ सरू॑पाय॒ स्वाहा॒ प्रति॑रूपाय॒ स्वाहा॑ श॒बला॑य॒ स्वाहा॑ कम॒लाय॒ स्वाहा॒ पृश्न॑ये॒ स्वाहा॑ पृश्ञिस॒क्थाय॒ स्वाहा॒ सर्व॑स्मै॒ स्वाहा᳚॥४०॥

%7.3.19.0
{\anuvakamend[{कृ॒ष्णाय॒ षट्च॑त्वारिꣳशत्}]}%॥18॥

%7.3.19.1
ओष॑धीभ्यः॒ स्वाहा॒ मूले᳚भ्यः॒ स्वाहा॒ तूले᳚भ्यः॒ स्वाहा॒ काण्डे᳚भ्यः॒ स्वाहा॒ वल्\mbox{}शे᳚भ्यः॒ स्वाहा॒ पुष्पे᳚भ्यः॒ स्वाहा॒ फले᳚भ्यः॒ स्वाहा॑ गृही॒तेभ्यः॒ स्वाहागृ॑हीतेभ्यः॒ स्वाहाव॑पन्नेभ्यः॒ स्वाहा॒ शया॑नेभ्यः॒ स्वाहा॒ सर्व॑स्मै॒ स्वाहा᳚॥४१॥

%7.3.20.0
{\anuvakamend[{ओष॑धीभ्य॒श्चतु॑र्विꣳशतिः}]}%॥19॥

%7.3.20.1
वन॒स्पति॑भ्यः॒ स्वाहा॒ मूले᳚भ्यः॒ स्वाहा॒ तूले᳚भ्यः॒ स्वाहा॒ स्कन्धो᳚भ्यः॒ स्वाहा॒ शाखा᳚भ्यः॒ स्वाहा॑ प॒र्णेभ्यः॒ स्वाहा॒ पुष्पे᳚भ्यः॒ स्वाहा॒ फले᳚भ्यः॒ स्वाहा॑ गृही॒तेभ्यः॒ स्वाहागृ॑हीतेभ्यः॒ स्वाहाव॑पन्नेभ्यः॒ स्वाहा॒ शया॑नेभ्यः॒ स्वाहा॑ शि॒ष्टाय॒ स्वाहाति॑शिष्टाय॒ स्वाहा॒ परि॑शिष्टाय॒ स्वाहा॒ सꣳशि॑ष्टाय॒ स्वाहोच्छि॑ष्टाय॒ स्वाहा॑ रि॒क्ताय॒ स्वाहारि॑क्ताय॒ स्वाहा॒ प्ररि॑क्ताय॒ स्वाहा॒ सꣳरि॑क्ताय॒ स्वाहोद्रि॑क्ताय॒ स्वाहा॒ सर्व॑स्मै॒ स्वाहा᳚॥४२॥

%7.4.0.0
{\anuvakamend[{वन॒स्पति॑भ्यः॒ स्कन्धो᳚भ्यः शि॒ष्टाय॑ रि॒क्ताय॒ षट्च॑त्वारिꣳशत्}]}%॥20॥

\prashnaend[{प्र॒जवं॑ प्र॒जाप॑ति॒र्यद॑छन्दो॒मन्ते॑ हुवे सवा॒हमोष॑धीभ्यो॒ द्विच॑त्वारिꣳशत्॥42॥ प्र॒जव॒ꣳ॒ सर्व॑स्मै॒ स्वाहा᳚॥}]
%%% END PRASHNA

\sect{चतुर्थः प्रश्नः}\setcounter{anuvakam}{0}
\dnsub{तैत्तिरीयसंहितायां सप्तमकाण्डे चतुर्थः प्रश्नः}
%7.4.1.0
%7.4.1.1
बृह॒स्पति॑रकामयत॒ श्रन्मे॑ दे॒वा दधी॑र॒न्गच्छे॑यं पुरो॒धामिति॒ स ए॒तं च॑तुर्विꣳशतिरा॒त्रम॑पश्य॒त्तमाह॑र॒त्तेना॑यजत॒ ततो॒ वै तस्मै॒ श्रद्दे॒वा अद॑ध॒ताग॑च्छत्पुरो॒धां य ए॒वं वि॒द्वाꣳस॑श्चतुर्विꣳशतिरा॒त्रमास॑ते॒ श्रदे᳚भ्यो मनु॒ष्या॑ दधते॒ गच्छ॑न्ति पुरो॒धां ज्योति॒र्गौरायु॒रिति॑ त्र्य॒हा भ॑वन्ती॒यं वाव ज्योति॑र॒न्तरि॑क्षं॒ गौर॒सावायुः॑॥१॥

%7.4.1.2
इ॒माने॒व लो॒कान॒भ्यारो॑हन्त्यभिपू॒र्वं त्र्य॒हा भ॑वन्त्यभिपू॒र्वमे॒व सु॑व॒र्गं लो॒कम॒भ्यारो॑ह॒न्त्यस॑त्त्रं॒ वा ए॒तद्यद॑छन्दो॒मं यच्छ॑न्दो॒मा भव॑न्ति॒ तेन॑ स॒त्त्रं दे॒वता॑ ए॒व पृ॒ष्ठैरव॑ रुन्धते प॒शूञ्छ॑न्दो॒मैरोजो॒ वै वी॒र्यं॑ पृ॒ष्ठानि॑ प॒शव॑श्छन्दो॒मा ओज॑स्ये॒व वी॒र्ये॑ प॒शुषु॒ प्रति॑ तिष्ठन्ति बृहद्रथन्त॒रा\-भ्यां᳚ यन्ती॒यं वाव र॑थन्त॒रम॒सौ बृ॒हदा॒भ्यामे॒व॥२॥

%7.4.1.3
य॒न्त्यथो॑ अ॒नयो॑रे॒व प्रति॑ तिष्ठन्त्ये॒ते वै य॒ज्ञस्या᳚ञ्ज॒साय॑नी स्रु॒ती ताभ्या॑मे॒व सु॑व॒र्गं लो॒कं य॑न्ति चतुर्विꣳशतिरा॒त्रो भ॑वति॒ चतु॑र्विꣳशतिरर्धमा॒साः सं॑वथ्स॒रः सं॑वथ्स॒रः सु॑व॒र्गो लो॒कः सं॑वथ्स॒र ए॒व सु॑व॒र्गे लो॒के प्रति॑ तिष्ठ॒न्त्यथो॒ चतु॑र्विꣳशत्यक्षरा गाय॒त्री गा॑य॒त्री ब्र॑ह्मवर्च॒सङ्गा॑यत्रि॒यैव ब्र॑ह्मवर्च॒समव॑ रुन्धते\-ऽतिरा॒त्राव॒भितो॑ भवतो ब्रह्मवर्च॒सस्य॒ परि॑गृहीत्यै॥३॥

%7.4.2.0
{\anuvakamend[{अ॒सावायु॑रा॒भ्यामे॒व पञ्च॑चत्वारिꣳशच्च}]}%॥१॥

%7.4.2.1
यथा॒ वै म॑नु॒ष्या॑ ए॒वं दे॒वा अग्र॑ आस॒न्ते॑\-ऽकामय॒न्ताव॑र्तिम्पा॒प्मान॑म्मृ॒त्युम॑प॒हत्य॒ दैवीꣳ॑ स॒ꣳ॒सदं॑ गच्छे॒मेति॒ त ए॒तं च॑तुर्विꣳशतिरा॒त्रम॑पश्य॒न्तमाह॑र॒न्तेना॑यजन्त॒ ततो॒ वै ते\-ऽव॑र्तिम्पा॒प्मान॑म्मृ॒त्युम॑प॒हत्य॒ दैवीꣳ॑ स॒ꣳ॒सद॑मगच्छ॒न् य ए॒वं वि॒द्वाꣳ॑सश्चतुर्विꣳशतिरा॒त्रमास॒ते\-ऽव॑र्तिमे॒व पा॒प्मान॑मप॒हत्य॒ श्रियं॑ गच्छन्ति॒ श्रीर्\mbox{}हि म॑नु॒ष्य॑स्य॥४॥

%7.4.2.2
दैवी॑ स॒ꣳ॒सज्ज्योति॑रतिरा॒त्रो भ॑वति सुव॒र्गस्य॑ लो॒कस्यानु॑ख्यात्यै॒ पृष्ठ्यः॑ षड॒हो भ॑वति॒ षड्वा ऋ॒तवः॑ संवथ्स॒रस्तम्मासा॑ अर्धमा॒सा ऋ॒तवः॑ प्र॒विश्य॒ दैवीꣳ॑ स॒ꣳ॒सद॑मगच्छ॒न् य ए॒वं वि॒द्वाꣳ॑सश्चतुर्विꣳशतिरा॒त्रमास॑ते संवथ्स॒रमे॒व प्र॒विश्य॒ वस्य॑सीꣳ स॒ꣳ॒सदं॑ गच्छन्ति॒ त्रय॑स्त्रयस्त्रि॒ꣳ॒शा अ॒वस्ता᳚द्भवन्ति॒ त्रय॑स्त्रयस्त्रि॒ꣳ॒शाः प॒रस्ता᳚त्त्रयस्त्रि॒ꣳ॒शैरे॒वोभ॒यतो\-ऽव॑र्तिम्पा॒प्मान॑मप॒हत्य॒ दैवीꣳ॑ स॒ꣳ॒सद॑म्मध्य॒तः॥५॥

%7.4.2.3
ग॒च्छ॒न्ति॒ पृ॒ष्ठानि॒ हि दैवी॑ स॒ꣳ॒सज्जा॒मि वा ए॒तत्कु॑र्वन्ति॒ यत्त्रय॑स्त्रयस्त्रि॒ꣳ॒शा अ॒न्वञ्चो॒ मध्ये\-ऽनि॑रुक्तो भवति॒ तेनाजा᳚म्यू॒र्ध्वानि॑ पृ॒ष्ठानि॑ भवन्त्यू॒र्ध्वाश्छ॑न्दो॒मा उ॒भाभ्याꣳ॑ रू॒पाभ्याꣳ॑ सुव॒र्गं लो॒कं य॒न्त्यस॑त्त्रं॒ वा ए॒तद्यद॑छन्दो॒मं यच्छ॑न्दो॒मा भव॑न्ति॒ तेन॑ स॒त्त्रं दे॒वता॑ ए॒व पृ॒त्ष्ठैरव॑ रुन्धते प॒शूञ्छ॑न्दो॒मैरोजो॒ वै वी॒र्यं॑ पृ॒ष्ठानि॑ प॒शवः॑॥६॥

%7.4.2.4
छ॒न्दो॒मा ओज॑स्ये॒व वी॒र्ये॑ प॒शुषु॒ प्रति॑ तिष्ठन्ति॒ त्रय॑स्त्रयस्त्रि॒ꣳ॒शा अ॒वस्ता᳚द्भवन्ति॒ त्रय॑स्त्रयस्त्रि॒ꣳ॒शाः प॒रस्ता॒न्मध्ये॑ पृ॒ष्ठान्युरो॒ वै त्र॑यस्त्रि॒ꣳ॒शा आ॒त्मा पृ॒ष्ठान्या॒त्मन॑ ए॒व तद्यज॑मानाः॒ शर्म॑ नह्य॒न्ते\-ऽना᳚र्त्यै बृहद्रथन्त॒रा\-भ्यां᳚ यन्ती॒यं वाव र॑थन्त॒रम॒सौ बृ॒हदा॒भ्यामे॒व य॒न्त्यथो॑ अ॒नयो॑रे॒व प्रति॑ तिष्ठन्त्ये॒ते वै य॒ज्ञस्या᳚ञ्ज॒साय॑नी स्रु॒ती ताभ्या॑मे॒व॥७॥

%7.4.2.5
सु॒व॒र्गं लो॒कं य॑न्ति॒ परा᳚ञ्चो॒ वा ए॒ते सु॑व॒र्गं लो॒कम॒भ्यारो॑हन्ति॒ ये प॑रा॒चीना॑नि पृ॒ष्ठान्यु॑प॒यन्ति॑ प्र॒त्यङ्क्ष॑ड॒हो भ॑वति प्र॒त्यव॑रूढ्या॒ अथो॒ प्रति॑ष्ठित्या उ॒भयो᳚र्लो॒कयोर्\mbox{}॑ऋ॒द्ध्वोत्ति॑ष्ठन्ति त्रि॒वृतो\-ऽधि॑ त्रि॒वृत॒मुप॑ यन्ति॒ स्तोमा॑ना॒ꣳ॒ सम्प॑त्त्यै प्रभ॒वाय॒ ज्योति॑रग्निष्टो॒मो भ॑वत्य॒यं वाव स क्षयो॒\-ऽस्मादे॒व तेन॒ क्षया॒न्न य॑न्ति चतुर्विꣳशतिरा॒त्रो भ॑वति॒ चतु॑र्विꣳशतिरर्धमा॒साः सं॑वथ्स॒रः सं॑वथ्स॒रः सु॑व॒र्गो लो॒कः सं॑वथ्स॒र ए॒व सु॑व॒र्गे लो॒के प्रति॑ तिष्ठ॒न्त्यथो॒ चतु॑र्विꣳशत्यक्षरा गाय॒त्री गा॑य॒त्री ब्र॑ह्मवर्च॒सङ्गा॑यत्रि॒यैव ब्र॑ह्मवर्च॒समव॑ रुन्धते\-ऽतिरा॒त्राव॒भितो॑ भवतो ब्रह्मवर्च॒सस्य॒ परि॑गृहीत्यै॥८॥

%7.4.3.0
{\anuvakamend[{म॒नु॒ष्य॑स्य मध्य॒तः प॒शव॒स्ताभ्या॑मे॒व सं॑ वथ्स॒रश्चतु॑र्विꣳशतिश्च}]}%॥२॥

%7.4.3.1
ऋ॒क्षा वा इ॒यम॑लो॒मका॑सी॒थ्साका॑मय॒तौष॑धीभि॒र्वन॒स्पति॑भिः॒ प्र जा॑ये॒येति॒ सैतास्त्रि॒ꣳ॒शत॒ꣳ॒ रात्री॑रपश्य॒त्ततो॒ वा इ॒यमोष॑धीभि॒र्वन॒स्पति॑भिः॒ प्राजा॑यत॒ ये प्र॒जाका॑माः प॒शुका॑माः॒ स्युस्त ए॒ता आ॑सीर॒न्प्रैव जा॑यन्ते प्र॒जया॑ प॒शुभि॑रि॒यं वा अ॑क्षुध्य॒थ्सैतां वि॒राज॑मपश्य॒त्तामा॒त्मन्धि॒त्वान्नाद्य॒मवा॑रु॒न्द्धौष॑धीः॥९॥

%7.4.3.2
वन॒स्पती᳚न्प्र॒जां प॒शून्तेना॑वर्धत॒ सा जे॒मान॑म्महि॒मान॑मगच्छ॒द्य ए॒वं वि॒द्वाꣳस॑ ए॒ता आस॑ते वि॒राज॑मे॒वात्मन्धि॒त्वा\-ऽन्नाद्य॒मव॑ रुन्धते॒ वर्ध॑न्ते प्र॒जया॑ प॒शुभि॑र्जे॒मान॑म्महि॒मानं॑ गच्छन्ति॒ ज्योति॑रतिरा॒त्रो भ॑वति सुव॒र्गस्य॑ लो॒कस्यानु॑\-ख्यात्यै॒ पृष्ठ्यः॑ षड॒हो भ॑वति॒ षड्वा ऋ॒तवः॒ षट्पृ॒ष्ठानि॑ पृ॒ष्ठैरे॒वर्तून॒न्वारो॑हन्त्यृ॒तुभिः॑ संवथ्स॒रन्ते सं॑वथ्स॒र ए॒व॥१०॥

%7.4.3.3
प्रति॑ तिष्ठन्ति त्रयस्त्रि॒ꣳ॒शात्त्र॑यस्त्रि॒ꣳ॒शमुप॑ यन्ति य॒ज्ञस्य॒ सन्त॑त्या॒ अथो᳚ प्र॒जाप॑ति॒र्वै त्र॑यस्त्रि॒ꣳ॒शः प्र॒जाप॑तिमे॒वा र॑भन्ते॒ प्रति॑ष्ठित्यै त्रिण॒वो भ॑वति॒ विजि॑त्या एकवि॒ꣳ॒शो भ॑वति॒ प्रति॑ष्ठित्या॒ अथो॒ रुच॑मे॒वात्मन्द॑धते त्रि॒वृद॑ग्नि॒ष्टुद्भ॑वति पा॒प्मान॑मे॒व तेन॒ निर्द॑ह॒न्ते\-ऽथो॒ तेजो॒ वै त्रि॒वृत्तेज॑ ए॒वात्मन्द॑धते पञ्चद॒श इ॑न्द्रस्तो॒मो भ॑वतीन्द्रि॒यमे॒वाव॑॥११॥

%7.4.3.4
रु॒न्ध॒ते॒ स॒प्त॒द॒शो भ॑वत्य॒न्नाद्य॒स्याव॑रुद्ध्या॒ अथो॒ प्रैव तेन॑ जायन्त एकवि॒ꣳ॒शो भ॑वति॒ प्रति॑ष्ठित्या॒ अथो॒ रुच॑मे॒वात्मन्द॑धते चतुर्वि॒ꣳ॒शो भ॑वति॒ चतु॑र्विꣳशतिरर्धमा॒साः सं॑वथ्स॒रः सं॑वथ्स॒रः सु॑व॒र्गो लो॒कः सं॑वथ्स॒र ए॒व सु॑व॒र्गे लो॒के प्रति॑ तिष्ठ॒न्त्यथो॑ ए॒ष वै वि॑षू॒वान् वि॑षू॒वन्तो॑ भवन्ति॒ य ए॒वं वि॒द्वाꣳस॑ ए॒ता आस॑ते चतुर्वि॒ꣳ॒शात्पृ॒ष्ठान्युप॑ यन्ति संवथ्स॒र ए॒व प्र॑ति॒ष्ठाय॑॥१२॥

%7.4.3.5
दे॒वता॑ अ॒भ्यारो॑हन्ति त्रयस्त्रि॒ꣳ॒शात्त्र॑यस्त्रि॒ꣳ॒शमुप॑ यन्ति॒ त्रय॑स्त्रिꣳश॒द्वै दे॒वता॑ दे॒वता᳚स्वे॒व प्रति॑ तिष्ठन्ति त्रिण॒वो भ॑वती॒मे वै लो॒कास्त्रि॑ण॒व ए॒ष्वे॑व लो॒केषु॒ प्रति॑ तिष्ठन्ति॒ द्वावे॑कवि॒ꣳ॒शौ भ॑वतः॒ प्रति॑ष्ठित्या॒ अथो॒ रुच॑मे॒वात्मन्द॑धते ब॒हवः॑ षोड॒शिनो॑ भवन्ति॒ तस्मा᳚द्ब॒हवः॑ प्र॒जासु॒ वृषा॑णो॒ यदे॒ते स्तोमा॒ व्यति॑षक्ता॒ भव॑न्ति॒ तस्मा॑दि॒यमोष॑धीभि॒र्वन॒स्पति॑भि॒र्व्यति॑षक्ता॥१३॥

%7.4.3.6
व्यति॑षज्यन्ते प्र॒जया॑ प॒शुभि॒र्य ए॒वं वि॒द्वाꣳस॑ ए॒ता आस॒ते\-ऽकॢ॑प्ता॒ वा ए॒ते सु॑व॒र्गं लो॒कं य॑न्त्युच्चाव॒चान् हि स्तोमा॑नुप॒यन्ति॒ यदे॒त ऊ॒र्ध्वाः कॢ॒प्ताः स्तोमा॒ भव॑न्ति कॢ॒प्ता ए॒व सु॑व॒र्गं लो॒कं य॑न्त्यु॒भयो॑रेभ्यो लो॒कयोः᳚ कल्पते त्रि॒ꣳ॒शदे॒तास्त्रि॒ꣳ॒शद॑क्षरा वि॒राडन्नं॑ वि॒राड्वि॒राजै॒वान्नाद्य॒मव॑ रुन्धते\-ऽतिरा॒त्राव॒भितो॑ भवतो॒\-ऽन्नाद्य॑स्य॒ परि॑गृहीत्यै॥१४॥

%7.4.4.0
{\anuvakamend[{ओष॑धीः सं वथ्स॒र ए॒वाव॑ प्रति॒ष्ठाय॒ व्यति॑ष॒क्तैका॒न्नप॑ञ्चा॒शच्च॑}]}%॥३॥

%7.4.4.1
प्र॒जाप॑तिः सुव॒र्गं लो॒कमै॒त्तं दे॒वा येन॑येन॒ छन्द॒सानु॒ प्रायु॑ञ्जत॒ तेन॒ नाप्नु॑व॒न्त ए॒ता द्वात्रिꣳ॑शत॒ꣳ॒ रात्री॑रपश्य॒न् द्वात्रिꣳ॑शदक्षरानु॒ष्टुगानु॑ष्टुभः प्र॒जाप॑तिः॒ स्वेनै॒व छन्द॑सा प्र॒जाप॑तिमा॒प्त्वाभ्या॒रुह्य॑ सुव॒र्गं लो॒कमा॑य॒न् य ए॒वं वि॒द्वाꣳस॑ ए॒ता आस॑ते॒ द्वात्रिꣳ॑शदे॒ता द्वात्रिꣳ॑शदक्षरानु॒ष्टुगानु॑ष्टुभः प्र॒जाप॑तिः॒ स्वेनै॒व छन्द॑सा प्र॒जाप॑तिमा॒प्त्वा श्रियं॑ गच्छन्ति॥१५॥

%7.4.4.2
श्रीर्\mbox{}हि म॑नु॒ष्य॑स्य सुव॒र्गो लो॒को द्वात्रिꣳ॑शदे॒ता द्वात्रिꣳ॑शदक्षरानु॒ष्टुग्वाग॑नु॒ष्टुफ्सर्वा॑मे॒व वाच॑माप्नुवन्ति॒ सर्वे॑ वा॒चो व॑दि॒तारो॑ भवन्ति॒ सर्वे॒ हि श्रियं॒ गच्छ॑न्ति॒ ज्योति॒र्गौरायु॒रिति॑ त्र्य॒हा भ॑वन्ती॒यं वाव ज्योति॑र॒न्तरि॑क्षं॒ गौर॒सावायु॑\-रि॒माने॒व लो॒कान॒भ्यारो॑हन्त्यभिपू॒र्वं त्र्य॒हा भ॑वन्त्यभिपू॒र्वमे॒व सु॑व॒र्गं लो॒कम॒भ्यारो॑हन्ति बृहद्रथन्त॒रा\-भ्यां᳚ यन्ति॥१६॥

%7.4.4.3
इ॒यं वाव र॑थन्त॒रम॒सौ बृ॒हदा॒भ्यामे॒व य॒न्त्यथो॑ अ॒नयो॑रे॒व प्रति॑ तिष्ठन्त्ये॒ते वै य॒ज्ञस्या᳚ञ्ज॒साय॑नी स्रु॒ती ताभ्या॑मे॒व सु॑व॒र्गं लो॒कं य॑न्ति॒ परा᳚ञ्चो॒ वा ए॒ते सु॑व॒र्गं लो॒कम॒भ्यारो॑हन्ति॒ ये परा॑चस्त्र्य॒हानु॑प॒यन्ति॑ प्र॒त्यङ्त्र्य॒हो भ॑वति॒ प्र॒त्यव॑रूढ्या॒ अथो॒ प्रति॑ष्ठित्या उ॒भयो᳚र्लो॒कयोर्\mbox{}॑ऋ॒द्ध्वोत्ति॑ष्ठन्ति॒ द्वात्रिꣳ॑शदे॒तास्तासां॒ यास्त्रि॒ꣳ॒शत्त्रि॒ꣳ॒शद॑क्षरा वि॒राडन्नं॑ वि॒राड्वि॒राजै॒वान्नाद्य॒मव॑ रुन्धते॒ ये द्वे अ॑होरा॒त्रे ए॒व ते उ॒भाभ्याꣳ॑ रू॒पाभ्याꣳ॑ सुव॒र्गं लो॒कं य॑न्त्यतिरा॒त्राव॒भितो॑ भवतः॒ परि॑गृहीत्यै॥१७॥

%7.4.5.0
{\anuvakamend[{ग॒च्छ॒न्ति॒ य॒न्ति॒ त्रि॒ꣳ॒शद॑क्षरा॒ द्वाविꣳ॑शतिश्च}]}%॥४॥

%7.4.5.1
द्वे वाव दे॑वस॒त्त्रे द्वा॑दशा॒हश्चै॒व त्र॑यस्त्रिꣳशद॒हश्च॒ य ए॒वं वि॒द्वाꣳस॑स्त्रयस्त्रिꣳशद॒हमास॑ते सा॒क्षादे॒व दे॒वता॑ अ॒भ्यारो॑हन्ति॒ यथा॒ खलु॒ वै श्रेया॑न॒भ्यारू॑ढः का॒मय॑ते॒ तथा॑ करोति॒ यद्य॑व॒विध्य॑ति॒ पापी॑यान्भवति॒ यदि॒ नाव॒विध्य॑ति स॒दृङ्य ए॒वं वि॒द्वाꣳस॑स्त्रयस्त्रिꣳशद॒हमास॑ते॒ वि पा॒प्मना॒ भ्रातृ॑व्ये॒णा व॑र्तन्ते\-ऽह॒र्भाजो॒ वा ए॒ता दे॒वा अग्र॒ आह॑रन्न्॥१८॥

%7.4.5.2
अह॒रेको\-ऽभ॑ज॒ताह॒रेक॒स्ताभि॒र्वै ते प्र॒बाहु॑गार्ध्नुव॒न् य ए॒वं वि॒द्वाꣳस॑स्त्रयस्त्रिꣳशद॒हमास॑ते॒ सर्व॑ ए॒व प्र॒बाहु॑गृध्नुवन्ति॒ सर्वे॒ ग्राम॑णीय॒म्प्राप्नु॑वन्ति पञ्चा॒हा भ॑वन्ति॒ पञ्च॒ वा ऋ॒तवः॑ संवथ्स॒र ऋ॒तुष्वे॒व सं॑वथ्स॒रे प्रति॑ तिष्ठ॒न्त्यथो॒ पञ्चा᳚क्षरा प॒ङ्क्तिः पाङ्क्तो॑ य॒ज्ञ य॒ज्ञमे॒वाव॑ रुन्धते॒ त्रीण्या᳚श्वि॒नानि॑ भवन्ति॒ त्रय॑ इ॒मे लो॒का ए॒षु॥१९॥

%7.4.5.3
ए॒व लो॒केषु॒ प्रति॑ तिष्ठ॒न्त्यथो॒ त्रीणि॒ वै य॒ज्ञस्ये᳚न्द्रि॒याणि॒ तान्ये॒वाव॑ रुन्धते विश्व॒जिद्भ॑वत्य॒न्नाद्य॒स्याव॑रुद्ध्यै॒ सर्व॑पृष्ठो भवति॒ सर्व॑स्या॒भिजि॑त्यै॒ वाग्वै द्वा॑दशा॒हो यत्पु॒रस्ता᳚द्द्वादशा॒हमु॑पे॒युरना᳚प्तां॒ वाच॒मुपे॑युरुप॒दासु॑कैषां॒ वाख्स्या॑दु॒परि॑ष्टाद्द्वादशा॒हमुप॑ यन्त्या॒प्तामे॒व वाच॒मुप॑ यन्ति॒ तस्मा॑दु॒परि॑ष्टाद्वा॒चा व॑दामो\-ऽवान्त॒रम्॥२०॥

%7.4.5.4
वै द॑शरा॒त्रेण॑ प्र॒जाप॑तिः प्र॒जा अ॑सृजत॒ यद्द॑शरा॒त्रो भव॑ति प्र॒जा ए॒व तद्यज॑मानाः सृजन्त ए॒ताꣳ ह॒ वा उ॑द॒ङ्कः शौ᳚ल्बाय॒नः स॒त्त्रस्यर्द्धि॑मुवाच॒ यद्द॑शरा॒त्रो यद्द॑शरा॒त्रो भव॑ति स॒त्त्रस्यर्द्ध्या॒ अथो॒ यदे॒व पूर्वे॒ष्वहः॑सु॒ विलो॑म क्रि॒यते॒ तस्यै॒वैषा शान्ति॑र्द्व्यनी॒का वा ए॒ता रात्र॑यो॒ यज॑माना विश्व॒जिथ्स॒हाति॑रा॒त्रेण॒ पूर्वाः॒ षोड॑श स॒हाति॑रा॒त्रेणोत्त॑राः॒ षोड॑श॒ य ए॒वं वि॒द्वाꣳस॑स्त्रयस्त्रिꣳशद॒हमास॑त॒ ऐ॑षां᳚ द्व्यनी॒का प्र॒जा जा॑यते\-ऽतिरा॒त्राव॒भितो॑ भवतः॒ परि॑गृहीत्यै॥२१॥

%7.4.6.0
{\anuvakamend[{अ॒ह॒र॒न्ने॒ष्व॑वान्त॒रꣳ षोड॑श स॒ह स॒प्तद॑श च}]}%॥५॥

%7.4.6.1
आ॒दि॒त्या अ॑कामयन्त सुव॒र्गं लो॒कमि॑या॒मेति॒ ते सु॑व॒र्गं लो॒कं न प्राजा॑न॒न्न सु॑व॒र्गं लो॒कमा॑य॒न्त ए॒तꣳ ष॑ट्त्रिꣳशद्रा॒त्रम॑पश्य॒न्तमाह॑र॒न्तेना॑यजन्त॒ ततो॒ वै ते सु॑व॒र्गं लो॒कम्प्राजा॑नन्थ्सुव॒र्गं लो॒कमा॑य॒न् य ए॒वं वि॒द्वाꣳसः॑ षट्त्रिꣳशद्रा॒त्रमास॑ते सुव॒र्गमे॒व लो॒कम्प्र जा॑नन्ति सुव॒र्गं लो॒कं य॑न्ति॒ ज्योति॑रतिरा॒त्रः॥२२॥

%7.4.6.2
भ॒व॒ति॒ ज्योति॑रे॒व पु॒रस्ता᳚द्दधते सुव॒र्गस्य॑ लो॒कस्यानु॑ख्यात्यै षड॒हा भ॑वन्ति॒ षड्वा ऋ॒तव॑ ऋ॒तुष्वे॒व प्रति॑ तिष्ठन्ति च॒त्वारो॑ भवन्ति॒ चत॑स्रो॒ दिशो॑ दि॒क्ष्वे॑व प्रति॑ तिष्ठ॒न्त्यस॑त्त्रं॒ वा ए॒तद्यद॑छन्दो॒मं यच्छ॑न्दो॒मा भव॑न्ति॒ तेन॑ स॒त्त्रं दे॒वता॑ ए॒व पृ॒ष्ठैरव॑ रुन्धते प॒शूञ्छ॑न्दो॒मैरोजो॒ वै वी॒र्यं॑ पृ॒ष्ठानि॑ प॒शव॑श्छन्दो॒मा ओज॑स्ये॒व॥२३॥

%7.4.6.3
वी॒र्ये॑ प॒शुषु॒ प्रति॑ तिष्ठन्ति षट्त्रिꣳशद्रा॒त्रो भ॑वति॒ षट्त्रिꣳ॑शदक्षरा बृह॒ती बार्\mbox{}ह॑ताः प॒शवो॑ बृह॒त्यैव प॒शूनव॑ रुन्धते बृह॒ती छन्द॑सा॒ꣴ॒ स्वारा᳚ज्यमाश्नुताश्नु॒वते॒ स्वारा᳚ज्यं॒ य ए॒वं वि॒द्वाꣳसः॑ षट्त्रिꣳशद्रा॒त्रमास॑ते सुव॒र्गमे॒व लो॒कं य॑न्त्यतिरा॒त्राव॒भितो॑ भवतः सुव॒र्गस्य॑ लो॒कस्य॒ परि॑गृहीत्यै॥२४॥

%7.4.7.0
{\anuvakamend[{अ॒ति॒रा॒त्र ओज॑स्ये॒व षट्त्रिꣳ॑शच्च}]}%॥६॥

%7.4.7.1
वसि॑ष्ठो ह॒तपु॑त्रो\-ऽकामयत वि॒न्देय॑ प्र॒जाम॒भि सौ॑दा॒सान्भ॑वेय॒मिति॒ स ए॒तमे॑कस्मान्नपञ्चा॒शम॑पश्य॒त्तमाह॑र॒त्तेना॑यजत॒ ततो॒ वै सो\-ऽवि॑न्दत प्र॒जाम॒भि सौ॑दा॒सान॑भव॒द्य ए॒वं वि॒द्वाꣳस॑ एकस्मान्नपञ्चा॒शमास॑ते वि॒न्दन्ते᳚ प्र॒जाम॒भि भ्रातृ॑व्यान्भवन्ति॒ त्रय॑स्त्रि॒वृतो᳚\-ऽग्निष्टो॒मा भ॑वन्ति॒ वज्र॑स्यै॒व मुख॒ꣳ॒ सꣴ श्य॑न्ति॒ दश॑ पञ्चद॒शा भ॑वन्ति पञ्चद॒शो वज्रः॑॥२५॥

%7.4.7.2
वज्र॑मे॒व भ्रातृ॑व्येभ्यः॒ प्र ह॑रन्ति षोडशि॒मद्द॑श॒ममह॑र्भवति॒ वज्र॑ ए॒व वी॒र्यं॑ दधति॒ द्वाद॑श सप्तद॒शा भ॑वन्त्य॒न्नाद्य॒स्याव॑रुद्ध्या॒ अथो॒ प्रैव तैर्जा॑यन्ते॒ पृष्ठ्यः॑ षड॒हो भ॑वति॒ षड्वा ऋ॒तवः॒ षट्पृ॒ष्ठानि॑ पृ॒ष्ठैरे॒वर्तून॒न्वारो॑हन्त्यृ॒तुभिः॑ संवथ्स॒रन्ते सं॑वथ्स॒र ए॒व प्रति॑ तिष्ठन्ति॒ द्वाद॑शैकवि॒ꣳ॒शा भ॑वन्ति॒ प्रति॑ष्ठित्या॒ अथो॒ रुच॑मे॒वात्मन्न्॥२६॥

%7.4.7.3
द॒ध॒ते॒ ब॒हवः॑ षोड॒शिनो॑ भवन्ति॒ विजि॑त्यै॒ षडा᳚श्वि॒नानि॑ भवन्ति॒ षड्वा ऋ॒तव॑ ऋ॒तुष्वे॒व प्रति॑ तिष्ठन्त्यूनातिरि॒क्ता वा ए॒ता रात्र॑य ऊ॒नास्तद्यदेक॑स्यै॒ न प॑ञ्चा॒शदति॑रिक्ता॒स्तद्यद्भूय॑सीर॒ष्टाच॑त्वारिꣳशत ऊ॒नाच्च॒ खलु॒ वा अति॑रिक्ताच्च प्र॒जाप॑तिः॒ प्राजा॑यत॒ ये प्र॒जाका॑माः प॒शुका॑माः॒ स्युस्त ए॒ता आ॑सीर॒न्प्रैव जा॑यन्ते प्र॒जया॑ प॒शुभि॑र्वैरा॒जो वा ए॒ष य॒ज्ञो यदे॑कस्मान्नपञ्चा॒शो य ए॒वं वि॒द्वाꣳस॑ एकस्मान्नपञ्चा॒शमास॑ते वि॒राज॑मे॒व ग॑च्छन्त्यन्ना॒दा भ॑वन्त्यतिरा॒त्राव॒भितो॑ भवतो॒\-ऽन्नाद्य॑स्य॒ परि॑गृहीत्यै॥२७॥

%7.4.8.0
{\anuvakamend[{वज्र॑ आ॒त्मन्प्र॒जया॒ द्वाविꣳ॑शतिश्च}]}%॥७॥

%7.4.8.1
सं॒व॒थ्स॒राय॑ दीक्षि॒ष्यमा॑णा एकाष्ट॒कायां᳚ दीक्षेरन्ने॒षा वै सं॑वथ्स॒रस्य॒ पत्नी॒ यदे॑काष्ट॒कैतस्यां॒ वा ए॒ष ए॒ताꣳ रात्रिं॑ वसति सा॒क्षादे॒व सं॑वथ्स॒रमा॒रभ्य॑ दीक्षन्त॒ आर्तं॒ वा ए॒ते सं॑वथ्स॒रस्या॒भि दी᳚क्षन्ते॒ य ए॑काष्ट॒कायां॒ दीक्ष॒न्ते\-ऽन्त॑नामानावृ॒तू भ॑वतो॒ व्य॑स्तं॒ वा ए॒ते सं॑वथ्स॒रस्या॒भि दी᳚क्षन्ते॒ य ए॑काष्ट॒कायां॒ दीक्ष॒न्ते\-ऽन्त॑नामानावृ॒तू भ॑वतः फल्गुनीपूर्णमा॒से दी᳚क्षेर॒न्मुखं॒ वा ए॒तत्॥२८॥

%7.4.8.2
सं॒व॒थ्स॒रस्य॒ यत्फ॑ल्गुनीपूर्णमा॒सो मु॑ख॒त ए॒व सं॑वथ्स॒रमा॒रभ्य॑ दीक्षन्ते॒ तस्यैकै॒व नि॒र्या यथ्साम्मे᳚घ्ये विषू॒वान्थ्स॒म्पद्य॑ते चित्रापूर्णमा॒से दी᳚क्षेर॒न्मुखं॒ वा ए॒तथ्सं॑वथ्स॒रस्य॒ यच्चि॑त्रापूर्णमा॒सो मु॑ख॒त ए॒व सं॑वथ्स॒रमा॒रभ्य॑ दीक्षन्ते॒ तस्य॒ न का च॒न नि॒र्या भ॑वति चतुर॒हे पु॒रस्ता᳚त्पौर्णमा॒स्यै दी᳚क्षेर॒न्तेषा॑मेकाष्ट॒कायां᳚ क्र॒यः सम्प॑द्यते॒ तेनै॑काष्ट॒कां न छ॒म्बट्कु॑र्वन्ति॒ तेषा᳚म्॥२९॥

%7.4.8.3
पू॒र्व॒प॒क्षे सु॒त्या सम्प॑द्यते पूर्वप॒क्षम्मासा॑ अ॒भि सम्प॑द्यन्ते॒ ते पू᳚र्वप॒क्ष उत्ति॑ष्ठन्ति॒ तानु॒त्तिष्ठ॑त॒ ओष॑धयो॒ वन॒स्पत॒यो\-ऽनूत्ति॑ष्ठन्ति॒ तान्क॑ल्या॒णी की॒र्तिरनूत्ति॑ष्ठ॒त्यरा᳚थ्सुरि॒मे यज॑माना॒ इति॒ तदनु॒ सर्वे॑ राध्नुवन्ति॥३०॥

%7.4.9.0
{\anuvakamend[{ए॒तच्छ॒म्बट्कु॑र्वन्ति॒ तेषा॒ञ्चतु॑स्त्रिꣳशच्च}]}%॥८॥

%7.4.9.1
सु॒व॒र्गं वा ए॒ते लो॒कं य॑न्ति॒ ये स॒त्त्रमु॑प॒यन्त्य॒भीन्ध॑त ए॒व दी॒क्षाभि॑रा॒त्मानꣴ॑ श्रपयन्त उप॒सद्भि॒र्द्वाभ्यां॒ लोमाव॑ द्यन्ति॒ द्वाभ्या॒न्त्वच॒न्द्वाभ्या॒मसृ॒द्द्वाभ्या᳚म्मा॒ꣳ॒सन्द्वाभ्या॒मस्थि॒ द्वाभ्या᳚म्म॒ज्जान॑मा॒त्मद॑क्षिणं॒ वै स॒त्त्रमा॒त्मान॑मे॒व दक्षि॑णां नी॒त्वा सु॑व॒र्गं लो॒कं य॑न्ति॒ शिखा॒मनु॒ प्र व॑पन्त॒ ऋद्ध्या॒ अथो॒ रघी॑याꣳसः सुव॒र्गं लो॒कम॑या॒मेति॑॥३१॥

%7.4.10.0
{\anuvakamend[{सु॒व॒र्गम्प॑ञ्चा॒शत्}]}%॥९॥

%7.4.10.1
ब्र॒ह्म॒वा॒दिनो॑ वदन्त्यतिरा॒त्रः प॑र॒मो य॑ज्ञक्रतू॒नां कस्मा॒त्तम्प्र॑थ॒ममुप॑ य॒न्तीत्ये॒तद्वा अ॑ग्निष्टो॒मम्प्र॑थ॒ममुप॑ य॒न्त्यथो॒क्थ्य॑मथ॑ षोड॒शिन॒मथा॑तिरा॒त्रम॑नुपू॒र्वमे॒वैतद्य॑ज्ञक्र॒तूनु॒पेत्य॒ ताना॒लभ्य॑ परि॒गृह्य॒ सोम॑मे॒वैतत्पिब॑न्त आसते॒ ज्योति॑ष्टोमम्प्रथ॒ममुप॑ यन्ति॒ ज्योति॑ष्टोमो॒ वै स्तोमा॑ना॒म्मुख॑म्मुख॒त ए॒व स्तोमा॒न्प्र यु॑ञ्जते॒ ते॥३२॥

%7.4.10.2
सꣴस्तु॑ता वि॒राज॑म॒भि सम्प॑द्यन्ते॒ द्वे चर्चा॒वति॑ रिच्येते॒ एक॑या॒ गौरति॑रिक्त॒ एक॒यायु॑रू॒नः सु॑व॒र्गो वै लो॒को ज्योति॒रूर्ग्वि॒राट्सु॑व॒र्गमे॒व तेन॑ लो॒कं य॑न्ति रथन्त॒रं दिवा॒ भव॑ति रथन्त॒रं नक्त॒मित्या॑हुर्ब्रह्मवा॒दिनः॒ केन॒ तदजा॒मीति॑ सौभ॒रं तृ॑तीयसव॒ने ब्र॑ह्मसा॒मम्बृ॒हत्तन्म॑ध्य॒तो द॑धति॒ विधृ॑त्यै॒ तेनाजा॑मि॥३३॥

%7.4.11.0
{\anuvakamend[{त एका॒न्नप॑ञ्चा॒शच्च॑}]}%॥10॥

%7.4.11.1
ज्योति॑ष्टोमम्प्रथ॒ममुप॑ यन्त्य॒स्मिन्ने॒व तेन॑ लो॒के प्रति॑ तिष्ठन्ति॒ गोष्टो॑मं द्वि॒तीय॒मुप॑ यन्त्य॒न्तरि॑क्ष ए॒व तेन॒ प्रति॑ तिष्ठ॒न्त्यायु॑ष्टोमं तृ॒तीय॒मुप॑ यन्त्य॒मुष्मि॑न्ने॒व तेन॑ लो॒के प्रति॑ तिष्ठन्ती॒यं वाव ज्योति॑र॒न्तरि॑क्षं॒ गौर॒सावायु॒र्यदे॒तान्थ्स्तोमा॑नुप॒यन्त्ये॒ष्वे॑व तल्लो॒केषु॑ स॒त्त्रिणः॑ प्रति॒तिष्ठ॑न्तो यन्ति॒ ते सꣴस्तु॑ता वि॒राजम्᳚॥३४॥

%7.4.11.2
अ॒भि सम्प॑द्यन्ते॒ द्वे चर्चा॒वति॑ रिच्येते॒ एक॑या॒ गौरति॑रिक्त॒ एक॒यायु॑रू॒नः सु॑व॒र्गो वै लो॒को ज्योति॒रूर्ग्वि॒राडूर्ज॑मे॒वाव॑ रुन्धते॒ ते न क्षु॒धार्ति॒मार्च्छ॒न्त्यक्षो॑धुका भवन्ति॒ क्षुथ्स॑म्बाधा इव॒ हि स॒त्त्रिणो᳚\-ऽग्निष्टो॒माव॒भितः॑ प्र॒धी तावु॒क्थ्या॑ मध्ये॒ नभ्यं॒ तत्तदे॒तत्प॑रि॒यद्दे॑वच॒क्रं यदे॒तेन॑॥३५॥

%7.4.11.3
ष॒ड॒हेन॒ यन्ति॑ देवच॒क्रमे॒व स॒मारो॑ह॒न्त्यरि॑ष्ट्यै॒ ते स्व॒स्ति सम॑श्नुवते षड॒हेन॑ यन्ति॒ षड्वा ऋ॒तव॑ ऋ॒तुष्वे॒व प्रति॑ तिष्ठन्त्युभ॒यतो᳚ज्योतिषा यन्त्युभ॒यत॑ ए॒व सु॑व॒र्गे लो॒के प्र॑ति॒तिष्ठ॑न्तो यन्ति॒ द्वौ ष॑ड॒हौ भ॑वत॒स्तानि॒ द्वाद॒शाहा॑नि॒ सम्प॑द्यन्ते द्वाद॒शो वै पुरु॑षो॒ द्वे स॒क्थ्यौ᳚ द्वौ बा॒हू आ॒त्मा च॒ शिर॑श्च च॒त्वार्यङ्गा॑नि॒ स्तनौ᳚ द्वाद॒शौ॥३६॥

%7.4.11.4
तत्पुरु॑ष॒मनु॑ प॒र्याव॑र्तन्ते॒ त्रयः॑ षड॒हा भ॑वन्ति॒ तान्य॒ष्टाद॒शाहा॑नि॒ सम्प॑द्यन्ते॒ नवा॒न्यानि॒ नवा॒न्यानि॒ नव॒ वै पुरु॑षे प्रा॒णास्तत्प्रा॒णाननु॑ प॒र्याव॑र्तन्ते च॒त्वारः॑ षड॒हा भ॑वन्ति॒ तानि॒ चतु॑र्विꣳशति॒रहा॑नि॒ सम्प॑द्यन्ते॒ चतु॑र्विꣳशतिरर्धमा॒साः सं॑वथ्स॒रस्तथ्सं॑वथ्स॒रमनु॑ प॒र्याव॑र्त॒न्ते\-ऽप्र॑तिष्ठितः संवथ्स॒र इति॒ खलु॒ वा आ॑हु॒र्वर्\mbox{}षी॑यान्प्रति॒ष्ठाया॒ इत्ये॒ताव॒द्वै सं॑वथ्स॒रस्य॒ ब्राह्म॑णं॒ याव॑न्मा॒सो मा॒सिमा᳚स्ये॒व प्र॑ति॒तिष्ठ॑न्तो यन्ति॥३७॥

%7.4.12.0
{\anuvakamend[{वि॒राज॑मे॒तेन॑ द्वाद॒शावे॒ताव॒द्वा अ॒ष्टौ च॑}]}%॥11॥

%7.4.12.1
मे॒षस्त्वा॑ पच॒तैर॑वतु॒ लोहि॑तग्रीव॒श्छागैः᳚ शल्म॒लिर्वृद्ध्या॑ प॒र्णो ब्रह्म॑णा प्ल॒क्षो मेधे॑न न्य॒ग्रोध॑श्चम॒सैरु॑दु॒म्बर॑ ऊ॒र्जा गा॑य॒त्री छन्दो॑भिस्त्रि॒वृथ्स्तोमै॒रव॑न्तीः॒ स्थाव॑न्तीस्त्वावन्तु प्रि॒यं त्वा᳚ प्रि॒याणां॒ वर्\mbox{}षि॑ष्ठ॒माप्या॑नां निधी॒नां त्वा॑ निधि॒पतिꣳ॑ हवामहे वसो मम॥३८॥

%7.4.13.0
{\anuvakamend[{मे॒षष्षट्त्रिꣳ॑शत्}]}%॥12॥

%7.4.13.1
कूप्या᳚भ्यः॒ स्वाहा॒ कूल्या᳚भ्यः॒ स्वाहा॑ विक॒र्या᳚भ्यः॒ स्वाहा॑\-ऽव॒ट्या᳚भ्यः॒ स्वाहा॒ खन्या᳚भ्यः॒ स्वाहा॒ ह्रद्या᳚भ्यः॒ स्वाहा॒ सूद्या᳚भ्यः॒ स्वाहा॑ सर॒स्या᳚भ्यः॒ स्वाहा॑ वैश॒न्तीभ्यः॒ स्वाहा॑ पल्व॒ल्या᳚भ्यः॒ स्वाहा॒ वर्ष्या᳚भ्यः॒ स्वाहा॑\-ऽव॒र्ष्याभ्यः॒ स्वाहा᳚ ह्रा॒दुनी᳚भ्यः॒ स्वाहा॒ पृष्वा᳚भ्यः॒ स्वाहा॒ स्यन्द॑मानाभ्यः॒ स्वाहा᳚ स्थाव॒राभ्यः॒ स्वाहा॑ नादे॒यीभ्यः॒ स्वाहा॑ सैन्ध॒वीभ्यः॒ स्वाहा॑ समु॒द्रिया᳚भ्यः॒ स्वाहा॒ सर्वा᳚भ्यः॒ स्वाहा᳚॥३९॥

%7.4.14.0
{\anuvakamend[{कूप्या᳚भ्यश्चत्वारि॒ꣳ॒शत्}]}%॥13॥

%7.4.14.1
अ॒द्भ्यः स्वाहा॒ वह॑न्तीभ्यः॒ स्वाहा॑ परि॒वह॑न्तीभ्यः॒ स्वाहा॑ सम॒न्तं वह॑न्तीभ्यः॒ स्वाहा॒ शीघ्रं॒ वह॑न्तीभ्यः॒ स्वाहा॒ शीभं॒ वह॑न्तीभ्यः॒ स्वाहो॒ग्रं वह॑न्तीभ्यः॒ स्वाहा॑ भी॒मं वह॑न्तीभ्यः॒ स्वाहा\-ऽम्भो᳚भ्यः॒ स्वाहा॒ नभो᳚भ्यः॒ स्वाहा॒ महो᳚भ्यः॒ स्वाहा॒ सर्व॑स्मै॒ स्वाहा᳚॥४०॥

%7.4.15.0
{\anuvakamend[{अ॒द्भ्य एका॒न्नत्रि॒ꣳ॒शत्}]}%॥14॥

%7.4.15.1
यो अर्व॑न्तं॒ जिघाꣳ॑सति॒ तम॒भ्य॑मीति॒ वरु॑णः। प॒रो मर्तः॑ प॒रः श्वा। अ॒हं च॒ त्वं च॑ वृत्रह॒न्थ्सम्ब॑भूव स॒निभ्य॒ आ। अ॒रा॒ती॒वा चि॑दद्रि॒वो\-ऽनु॑ नौ शूर मꣳसतै भ॒द्रा इन्द्र॑स्य रा॒तयः॑। अ॒भि क्रत्वे᳚न्द्र भू॒रध॒ ज्मन्न ते॑ विव्यङ्महि॒मान॒ꣳ॒ रजाꣳ॑सि। स्वेना॒ हि वृ॒त्रꣳ शव॑सा ज॒घन्थ॒ न शत्रु॒रन्तं॑ विविदद्यु॒धा ते᳚॥४१॥

%7.4.16.0
{\anuvakamend[{वि॒वि॒दद्द्वे च॑}]}%॥15॥

%7.4.16.1
नमो॒ राज्ञे॒ नमो॒ वरु॑णाय॒ नमो\-ऽश्वा॑य॒ नमः॑ प्र॒जाप॑तये॒ नमो\-ऽधि॑पत॒ये\-ऽधि॑पतिर॒स्यधि॑पतिं मा कु॒र्वधि॑पतिर॒हं प्र॒जानां᳚ भूयास॒म्मां धे॑हि॒ मयि॑ धेह्यु॒पाकृ॑ताय॒ स्वाहा\-ऽ\-ऽल॑ब्धाय॒ स्वाहा॑ हु॒ताय॒ स्वाहा᳚॥४२॥

%7.4.17.0
{\anuvakamend[{नम॒ एका॒न्नत्रि॒ꣳ॒शत्}]}%॥16॥

%7.4.17.1
म॒यो॒भूर्वातो॑ अ॒भि वा॑तू॒स्रा ऊर्ज॑स्वती॒रोष॑धी॒रा रि॑शन्ताम्। पीव॑स्वतीर्जी॒वध॑न्याः पिबन्त्वव॒साय॑ प॒द्वते॑ रुद्र मृड। याः सरू॑पा॒ विरू॑पा॒ एक॑रूपा॒ यासा॑म॒ग्निरिष्ट्या॒ नामा॑नि॒ वेद॑। या अङ्गि॑रस॒स्तप॑से॒ह च॒क्रुस्ताभ्यः॑ पर्जन्य॒ महि॒ शर्म॑ यच्छ। या दे॒वेषु॑ त॒नुव॒मैर॑यन्त॒ यासा॒ꣳ॒ सोमो॒ विश्वा॑ रू॒पाणि॒ वेद॑। ता अ॒स्मभ्य॒म्पय॑सा॒ पिन्व॑मानाः प्र॒जाव॑तीरिन्द्र॥४३॥

%7.4.17.2
गो॒ष्ठे रि॑रीहि। प्र॒जाप॑ति॒र्मह्य॑मे॒ता ररा॑णो॒ विश्वै᳚र्दे॒वैः पि॒तृभिः॑ संविदा॒नः। शि॒वाः स॒तीरुप॑ नो गो॒ष्ठमाक॒स्तासां᳚ व॒यं प्र॒जया॒ सꣳ स॑देम। इ॒ह धृतिः॒ स्वाहे॒ह विधृ॑तिः॒ स्वाहे॒ह रन्तिः॒ स्वाहे॒ह रम॑तिः॒ स्वाहा॑ म॒हीमू॒ षु सु॒त्रामा॑णम्॥४४॥

%7.4.18.0
{\anuvakamend[{इ॒न्द्रा॒ष्टात्रिꣳ॑शच्च}]}%॥17॥

%7.4.18.1
किꣴ स्वि॑दासीत्पू॒र्वचि॑त्तिः॒ किꣴ स्वि॑दासीद्बृ॒हद्वयः॑। किꣴ स्वि॑दासीत्पिशङ्गि॒ला किꣴ स्वि॑दासीत्पिलिप्पि॒ला। द्यौरा॑सीत्पू॒र्वचि॑त्ति॒रश्व॑ आसीद्बृ॒हद्वयः॑। रात्रि॑रासीत्पिशङ्गि॒लावि॑रासीत्पिलिप्पि॒ला। कः स्वि॑देका॒की च॑रति॒ क उ॑ स्विज्जायते॒ पुनः॑। किꣴ स्वि॑द्धि॒मस्य॑ भेष॒जं किꣴ स्वि॑दा॒वप॑नम्म॒हत्। सूर्य॑ एका॒की च॑रति॥४५॥

%7.4.18.2
च॒न्द्रमा॑ जायते॒ पुनः॑। अ॒ग्निर्\mbox{}हि॒मस्य॑ भेष॒जम्भूमि॑रा॒वप॑नम्म॒हत्। पृ॒च्छामि॑ त्वा॒ पर॒मन्तं॑ पृथि॒व्याः पृ॒च्छामि॑ त्वा॒ भुव॑नस्य॒ नाभिम्᳚। पृ॒च्छामि॑ त्वा॒ वृष्णो॒ अश्व॑स्य॒ रेतः॑ पृ॒च्छामि॑ वा॒चः प॑र॒मं व्यो॑म। वेदि॑माहुः॒ पर॒मन्तं॑ पृथि॒व्या य॒ज्ञमा॑हु॒र्भुव॑नस्य॒ नाभिम्᳚। सोम॑माहु॒र्वृष्णो॒ अश्व॑स्य॒ रेतो॒ ब्रह्मै॒व वा॒चः प॑र॒मं व्यो॑म॥४६॥

%7.4.19.0
{\anuvakamend[{सूर्य॑ एका॒की च॑रति॒ षट्च॑त्वारिꣳशच्च}]}%॥18॥

%7.4.19.1
अम्बे॒ अम्बा॒ल्यम्बि॑के॒ न मा॑ नयति॒ कश्च॒न। स॒सस्त्य॑श्व॒कः। सुभ॑गे॒ काम्पी॑लवासिनि सुव॒र्गे लो॒के सं प्रोर्ण्वा॑थाम्। आहम॑जानि गर्भ॒धमा त्वम॑जासि गर्भ॒धम्। तौ स॒ह च॒तुरः॑ प॒दः सम्प्र सा॑रयावहै। वृषा॑ वाꣳ रेतो॒धा रेतो॑ दधा॒तूथ्स॒क्थ्यो᳚र्गृ॒दं धे᳚ह्य॒ञ्जिमुद॑ञ्जि॒मन्व॑ज। यः स्त्री॒णां जी॑व॒भोज॑नो॒ य आ॑साम्॥४७॥

%7.4.19.2
बि॒ल॒धाव॑नः। प्रि॒यः स्त्री॒णाम॑पी॒च्यः॑। य आ॑सां कृ॒ष्णे लक्ष्म॑णि॒ सर्दि॑गृदिम्प॒राव॑धीत्। अम्बे॒ अम्बा॒ल्यम्बि॑के॒ न मा॑ यभति॒ कश्च॒न। स॒सस्त्य॑श्व॒कः। ऊ॒र्ध्वामे॑ना॒मुच्छ्र॑यताद्वेणुभा॒रं गि॒रावि॑व। अथा᳚स्या॒ मध्य॑मेधताꣳ शी॒ते वाते॑ पु॒नन्नि॑व। अम्बे॒ अम्बा॒ल्यम्बि॑के॒ न मा॑ यभति॒ कश्च॒न। स॒सस्त्य॑श्व॒कः। यद्ध॑रि॒णी यव॒मत्ति॒ न॥४८॥

%7.4.19.3
पु॒ष्टम्प॒शु म॑न्यते। शू॒द्रा यदर्य॑जारा॒ न पोषा॑य धनायति। अम्बे॒ अम्बा॒ल्यम्बि॑के॒ न मा॑ यभति॒ कश्च॒न। स॒सस्त्य॑श्व॒कः। इ॒यं य॒का श॑कुन्ति॒काहल॒मिति॒ सर्प॑ति। आह॑तं ग॒भे पसो॒ नि ज॑ल्गुलीति॒ धाणि॑का। अम्बे॒ अम्बा॒ल्यम्बि॑के॒ न मा॑ यभति॒ कश्च॒न। स॒सस्त्य॑श्व॒कः। मा॒ता च॑ ते पि॒ता च॒ ते\-ऽग्रं॑ वृ॒क्षस्य॑ रोहतः।॥४९॥

%7.4.19.4
प्र सु॑ला॒मीति॑ ते पि॒ता ग॒भे मु॒ष्टिम॑तꣳसयत्। द॒धि॒क्राव्ण्णो॑ अकारिषं जि॒ष्णोरश्व॑स्य वा॒जिनः॑। सु॒र॒भि नो॒ मुखा॑ कर॒त्प्र ण॒ आयूꣳ॑षि तारिषत्। आपो॒ हि ष्ठा म॑यो॒भुव॒स्ता न॑ ऊ॒र्जे द॑धातन। म॒हे रणा॑य॒ चक्ष॑से। यो वः॑ शि॒वत॑मो॒ रस॒स्तस्य॑ भाजयते॒ह नः॑। उ॒श॒तीरि॑व मा॒तरः॑। तस्मा॒ अरं॑ गमाम वो॒ यस्य॒ क्षया॑य॒ जिन्व॑थ। आपो॑ ज॒नय॑था च नः॥५०॥

%7.4.20.0
{\anuvakamend[{आ॒सा॒मत्ति॒ न रो॑हतो॒ जिन्व॑थ च॒त्वारि॑ च}]}%॥19॥

%7.4.20.1
भूर्भुवः॒ सुव॒र्वस॑वस्त्वाञ्जन्तु गाय॒त्रेण॒ छन्द॑सा रु॒द्रास्त्वा᳚ञ्जन्तु॒ त्रैष्टु॑भेन॒ छन्द॑सादि॒त्यास्त्वा᳚ञ्जन्तु॒ जाग॑तेन॒ छन्द॑सा॒ यद्वातो॑ अ॒पो अग॑म॒दिन्द्र॑स्य त॒नुव॑म्प्रि॒याम्। ए॒तꣴ स्तो॑तरे॒तेन॑ प॒था पुन॒रश्व॒मा व॑र्तयासि नः। लाजी (३) ञ्छाची (३) न् यशो॑ म॒मा (4)म्। य॒व्यायै॑ ग॒व्याया॑ ए॒तद्दे॑वा॒ अन्न॑मत्तै॒तदन्न॑मद्धि प्रजापते। यु॒ञ्जन्ति॑ ब्र॒ध्नम॑रु॒षं चर॑न्तं॒ परि॑ त॒स्थुषः॑। रोच॑न्ते रोच॒ना दि॒वि। यु॒ञ्जन्त्य॑स्य॒ काम्या॒ हरी॒ विप॑क्षसा॒ रथे᳚। शोणा॑ धृ॒ष्णू नृ॒वाह॑सा। के॒तुं कृ॒ण्वन्न॑के॒तवे॒ पेशो॑ मर्या अपे॒शसे᳚। समु॒षद्भि॑रजायथाः॥५१॥

%7.4.21.0
{\anuvakamend[{ब्र॒ध्नं पञ्च॑विꣳशतिश्च}]}%॥20॥

%7.4.21.1
प्रा॒णाय॒ स्वाहा᳚ व्या॒नाय॒ स्वाहा॑\-ऽपा॒नाय॒ स्वाहा॒ स्नाव॑भ्यः॒ स्वाहा॑ सन्ता॒नेभ्यः॒ स्वाहा॒ परि॑सन्तानेभ्यः॒ स्वाहा॒ पर्व॑भ्यः॒ स्वाहा॑ सं॒धाने᳚भ्यः॒ स्वाहा॒ शरी॑रेभ्यः॒ स्वाहा॑ य॒ज्ञाय॒ स्वाहा॒ दक्षि॑णाभ्यः॒ स्वाहा॑ सुव॒र्गाय॒ स्वाहा॑ लो॒काय॒ स्वाहा॒ सर्व॑स्मै॒ स्वाहा᳚॥५२॥

%7.4.22.0
{\anuvakamend[{प्रा॒णाया॒ष्टाविꣳ॑शतिः}]}%॥21॥

%7.4.22.1
सि॒ताय॒ स्वाहा\-ऽसि॑ताय॒ स्वाहा॒\-ऽभिहि॑ताय॒ स्वाहा\-ऽन॑भिहिताय॒ स्वाहा॑ यु॒क्ताय॒ स्वाहाH\-ऽयु॑क्ताय॒ स्वाहा॒ सुयु॑क्ताय॒ स्वाहोद्यु॑क्ताय॒ स्वाहा॒ विमु॑क्ताय॒ स्वाहा॒ प्रमु॑क्ताय॒ स्वाहा॒ वञ्च॑ते॒ स्वाहा॑ परि॒वञ्च॑ते॒ स्वाहा॑ सं॒वञ्च॑ते॒ स्वाहा॑\-ऽनु॒वञ्च॑ते॒ स्वाहो॒द्वञ्च॑ते॒ स्वाहा॑ य॒ते स्वाहा॒ धाव॑ते॒ स्वाहा॒ तिष्ठ॑ते॒ स्वाहा॒ सर्व॑स्मै॒ स्वाहा᳚॥५३॥

%7.5.0.0
{\anuvakamend[{सि॒ताया॒ष्टात्रिꣳ॑शत्}]}%॥22॥

{\anuvakamend[{बृह॒स्पतिः॒ श्रद्यथा॒ वा ऋ॒क्षा वै प्र॒जाप॑ति॒र्येन॑येन॒ द्वे वाव दे॑वस॒त्रे आ॑दि॒त्या अ॑कामयन्त सुव॒र्गं वसि॑ष्ठः सं वथ्स॒राय॑ सुव॒र्गं ये स॒त्रम्ब्र॑ह्मवा॒दिनो॑\-ऽतिरा॒त्रो ज्योति॑ष्टोमं मे॒षः कूप्या᳚भ्यो॒\-ऽद्भ्यो यो नमो॑ मयो॒भूः किꣴ स्वि॒दम्बे॒ भूः प्रा॒णाय॑ सि॒ताय॒ द्वाविꣳ॑शतिः}]%॥22॥

\prashnaend[{बृह॒स्पतिः॒ प्रति॑तिष्ठन्ति॒ वै द॑शरा॒त्रेण॑ सुव॒र्गं यो अर्व॑न्तं॒ भूस्त्रिप़॑ञ्चा॒शत्॥53॥ बृह॒स्पतिः॒ सर्व॑स्मै॒ स्वाहा᳚॥}]

%7.5.0.0

%%% END PRASHNA

\sect{पञ्चमः प्रश्नः}\setcounter{anuvakam}{0}
\dnsub{तैत्तिरीयसंहितायां सप्तमकाण्डे पञ्चमः प्रश्नः}
%7.5.1.0
%7.5.1.1
गावो॒ वा ए॒तथ्स॒त्त्रमा॑सताशृ॒ङ्गाः स॒तीः शृङ्गा॑णि नो जायन्ता॒ इति॒ कामे॑न॑ तासां॒ दश॒ मासा॒ निष॑ण्णा॒ आस॒न्नथ॒ शृङ्गा᳚ण्यजायन्त॒ ता उद॑तिष्ठ॒न्नरा॒थ्स्मेत्यथ॒ यासां॒ नाजा॑यन्त॒ ताः सं॑वथ्स॒रमा॒प्त्वोद॑तिष्ठ॒न्नरा॒थ्स्मेति॒ यासां॒ चाजा॑यन्त॒ यासां᳚ च॒ न ता उ॒भयी॒रुद॑तिष्ठ॒न्नरा॒थ्स्मेति॑ गोस॒त्त्रं वै॥१॥

%7.5.1.2
सं॒व॒थ्स॒रो य ए॒वं वि॒द्वाꣳसः॑ संवथ्स॒रमु॑प॒यन्त्यृ॑ध्नु॒वन्त्ये॒व तस्मा᳚त्तूप॒रा वार्\mbox{}षि॑कौ॒ मासौ॒ पर्त्वा॑ चरति स॒त्त्राभि॑जित॒ꣴ॒ ह्य॑स्यै॒ तस्मा᳚थ्संवथ्सर॒सदो॒ यत्किं च॑ गृ॒हे क्रि॒यते॒ तदा॒प्तमव॑रुद्धम॒भिजि॑तं क्रियते समु॒द्रं वा ए॒ते प्र प्ल॑वन्ते॒ ये सं॑वथ्स॒रमु॑प॒यन्ति॒ यो वै स॑मु॒द्रस्य॒ पारं॒ न पश्य॑ति॒ न वै स तत॒ उदे॑ति संवथ्स॒रः॥२॥

%7.5.1.3
वै स॑मु॒द्रस्तस्यै॒तत्पा॒रं यद॑तिरा॒त्रौ य ए॒वं वि॒द्वाꣳसः॑ संवथ्स॒रमु॑प॒यन्त्यना᳚र्ता ए॒वोदृचं॑ गच्छन्ती॒यं वै पूर्वो॑\-ऽतिरा॒त्रो॑\-ऽ\-सावुत्त॑रो॒ मनः॒ पूर्वो॒ वागुत्त॑रः प्रा॒णः पूर्वो॑\-ऽपा॒न उत्त॑रः प्र॒रोध॑न॒म्पूर्व॑ उ॒दय॑न॒मुत्त॑रो॒ ज्योति॑ष्टोमो वैश्वान॒रो॑\-ऽतिरा॒त्रो भ॑वति॒ ज्योति॑रे॒व पु॒रस्ता᳚द्दधते सुव॒र्गस्य॑ लो॒कस्यानु॑ख्यात्यै चतुर्वि॒ꣳ॒शः प्रा॑य॒णीयो॑ भवति चतु॑र्विꣳशतिरर्धमा॒साः॥३॥

%7.5.1.4
सं॒व॒थ्स॒रः प्र॒यन्त॑ ए॒व सं॑वथ्स॒रे प्रति॑ तिष्ठन्ति॒ तस्य॒ त्रीणि॑ च श॒तानि॑ ष॒ष्टिश्च॑ स्तो॒त्रीया॒स्ताव॑तीः संवथ्स॒रस्य॒ रात्र॑य उ॒भे ए॒व सं॑वथ्स॒रस्य॑ रू॒पे आ᳚प्नुवन्ति॒ ते सꣴस्थि॑त्या॒ अरि॑ष्ट्या॒ उत्त॑रै॒रहो॑भिश्चरन्ति षड॒हा भ॑वन्ति॒ षड्वा ऋ॒तवः॑ संवथ्स॒र ऋ॒तुष्वे॒व सं॑वथ्स॒रे प्रति॑ तिष्ठन्ति॒ गौश्चायु॑श्च मध्य॒तः स्तोमौ॑ भवतः संवथ्स॒रस्यै॒व तन्मि॑थु॒नम्म॑ध्य॒तः॥४॥

%7.5.1.5
द॒ध॒ति॒ प्र॒जन॑नाय॒ ज्योति॑र॒भितो॑ भवति वि॒मोच॑नमे॒व तच्छन्दाꣳ॑स्ये॒व तद्वि॒मोकं॑ य॒न्त्यथो॑ उभ॒यतो᳚ज्योतिषै॒व ष॑ड॒हेन॑ सुव॒र्गं लो॒कं य॑न्ति ब्रह्मवा॒दिनो॑ वद॒न्त्यास॑ते॒ केन॑ य॒न्तीति॑ देव॒याने॑न प॒थेति॑ ब्रूया॒च्छन्दाꣳ॑सि॒ वै दे॑व॒यानः॒ पन्था॑ गाय॒त्री त्रि॒ष्टुब्जग॑ती॒ ज्योति॒र्वै गा॑य॒त्री गौस्त्रि॒ष्टुगायु॒र्जग॑ती यदे॒ते स्तोमा॒ भव॑न्ति देव॒याने॑नै॒व॥५॥

%7.5.1.6
तत्प॒था य॑न्ति समा॒नꣳ साम॑ भवति देवलो॒को वै साम॑ देवलो॒कादे॒व न य॑न्त्य॒न्याअ॑न्या॒ ऋचो॑ भवन्ति मनुष्यलो॒को वा ऋचो॑ मनुष्यलो॒कादे॒वान्यम॑न्यं देवलो॒कम॑भ्या॒रोह॑न्तो यन्त्यभिव॒र्तो ब्र॑ह्मसा॒मम्भ॑वति सुव॒र्गस्य॑ लो॒कस्या॒भिवृ॑त्त्या अभि॒जिद्भ॑वति सुव॒र्गस्य॑ लो॒कस्या॒भिजि॑त्यै विश्व॒जिद्भ॑वति॒ विश्व॑स्य॒ जित्यै॑ मा॒सिमा॑सि पृ॒ष्ठान्युप॑ यन्ति मा॒सिमा᳚स्यतिग्रा॒ह्या॑ गृह्यन्ते मा॒सिमा᳚स्ये॒व वी॒र्यं॑ दधति मा॒सां प्रति॑ष्ठित्या उ॒परि॑ष्टान्मा॒सां पृ॒ष्ठान्युप॑ यन्ति॒ तस्मा॑दु॒परि॑ष्टा॒दोष॑धयः॒ फलं॑ गृह्णन्ति॥६॥

%7.5.2.0
{\anuvakamend[{गो॒स॒त्त्रं वा ए॑ति सं वथ्स॒रो᳚\-ऽर्धमा॒सा मि॑थु॒नम्म॑ध्य॒तो दे॑व॒याने॑नै॒व वी॒र्य॑न्त्रयो॑दश च}]}%॥१॥

%7.5.2.1
गावो॒ वा ए॒तथ्स॒त्त्रमा॑सताशृ॒ङ्गाः स॒तीः शृ॑ङ्गाणि॒ सिषा॑सन्ती॒स्तासां॒ दश॒ मासा॒ निष॑ण्णा॒ आस॒न्नथ॒ शृङ्गा᳚ण्यजायन्त॒ ता अ॑ब्रुव॒न्नरा॒थ्स्मोत्ति॑ष्ठा॒माव॒ तं काम॑मरुथ्स्महि॒ येन॒ कामे॑न॒ न्यष॑दा॒मेति॒ तासा॑मु॒ त्वा अ॑ब्रुवन्न॒र्धा वा॒ याव॑ती॒र्वासा॑महा ए॒वेमौ द्वा॑द॒शौ मासौ॑ संवथ्स॒रꣳ स॒म्पाद्योत्ति॑ष्ठा॒मेति॒ तासा᳚म्॥७॥

%7.5.2.2
द्वा॒द॒शे मा॒सि शृङ्गा॑णि॒ प्राव॑र्तन्त श्र॒द्धया॒ वाश्र॑द्धया वा॒ ता इ॒मा यास्तू॑प॒रा उ॒भय्यो॒ वाव ता आ᳚र्ध्नुव॒न् याश्च॒ शृङ्गा॒ण्यस॑न्व॒न् याश्चोर्ज॑म॒वारु॑न्धत॒र्ध्नोति॑ द॒शसु॑ मा॒सू᳚त्तिष्ठ॑न्नृ॒ध्नोति॑ द्वाद॒शसु॒ य ए॒वं वेद॑ प॒देन॒ खलु॒ वा ए॒ते य॑न्ति वि॒न्दति॒ खलु॒ वै प॒देन॒ यन्तद्वा ए॒तदृ॒द्धमय॑न॒न्तस्मा॑दे॒तद्गो॒सनि॑॥८॥

%7.5.3.0
{\anuvakamend[{ति॒ष्ठा॒मेति॒ तासा॒न्तस्मा॒द्द्वे च॑}]}%॥२॥

%7.5.3.1
प्र॒थ॒मे मा॒सि पृ॒ष्ठान्युप॑ यन्ति मध्य॒म उप॑ यन्त्युत्त॒म उप॑ यन्ति॒ तदा॑हु॒र्यां वै त्रिरेक॒स्याह्न॑ उप॒सीद॑न्ति द॒ह्रं वै साप॑राभ्यां॒ दोहा᳚भ्यां दु॒हे\-ऽथ॒ कुतः॒ सा धो᳚क्ष्यते॒ यां द्वाद॑श॒ कृत्व॑ उप॒सीद॒न्तीति॑ संवथ्स॒रꣳ स॒म्पाद्यो᳚त्त॒मे मा॒सि स॒कृत्पृ॒ष्ठान्युपे॑यु॒स्तद्यज॑माना य॒ज्ञं प॒शूनव॑ रुन्धते समु॒द्रं वै॥९॥

%7.5.3.2
ए॒ते॑\-ऽनवा॒रम॑पा॒रम्प्र प्ल॑वन्ते॒ ये सं॑वथ्स॒रमु॑प॒यन्ति॒ यद्बृ॑हद्रथन्त॒रे अ॒न्वर्जे॑यु॒र्यथा॒ मध्ये॑ समु॒द्रस्य॑ प्ल॒वम॒न्वर्जे॑युस्ता॒दृक्त\-दनु॑थ्सर्गम्बृहद्रथन्त॒राभ्या॑मि॒त्वा प्र॑ति॒ष्ठां ग॑च्छन्ति॒ सर्वे᳚भ्यो॒ वै कामे᳚भ्यः स॒न्धिर्दु॑हे॒ तद्यज॑मानाः॒ सर्वा॒न्कामा॒नव॑ रुन्धते॥१०॥

%7.5.4.0
{\anuvakamend[{स॒मु॒द्रं वै चतु॑स्त्रिꣳशच्च}]}%॥३॥

%7.5.4.1
स॒मा॒न्य॑ ऋचो॑ भवन्ति मनुष्यलो॒को वा ऋचो॑ मनुष्यलो॒कादे॒व न य॑न्त्य॒न्यद॑न्य॒थ्साम॑ भवति देवलो॒को वै साम॑ देवलो॒कादे॒वान्यम॑न्यम्मनुष्यलो॒कम्प्र॑त्यव॒रोह॑न्तो यन्ति॒ जग॑ती॒मग्र॒ उप॑ यन्ति॒ जग॑तीं॒ वै छन्दाꣳ॑सि प्र॒त्यव॑रोहन्त्याग्रय॒णं ग्रहा॑ बृ॒हत्पृ॒ष्ठानि॑ त्रयस्त्रि॒ꣳ॒शꣴ स्तोमा॒स्तस्मा॒ज्ज्यायाꣳ॑सं॒ कनी॑यान्प्र॒त्यव॑रोहति वैश्वकर्म॒णो गृ॑ह्यते॒ विश्वा᳚न्ये॒व तेन॒ कर्मा॑णि॒ यज॑माना॒ अव॑ रुन्धत आदि॒त्यः॥११॥

%7.5.4.2
गृ॒ह्य॒त॒ इ॒यं वा अदि॑तिर॒स्यामे॒व प्रति॑ तिष्ठन्त्य॒न्यो᳚न्यो गृह्येते मिथुन॒त्वाय॒ प्रजा᳚त्या अवान्त॒रं वै द॑शरा॒त्रेण॑ प्र॒जाप॑तिः प्र॒जा अ॑सृजत॒ यद्द॑शरा॒त्रो भव॑ति प्र॒जा ए॒व तद्यज॑मानाः सृजन्त ए॒ताꣳ ह॒ वा उ॑द॒ङ्कः शौ᳚ल्बाय॒नः स॒त्त्रस्यर्द्धि॑मुवाच॒ यद्द॑शरा॒त्रो यद्द॑शरा॒त्रो भव॑ति स॒त्त्रस्यर्द्ध्या॒ अथो॒ यदे॒व पूर्वे॒ष्वहः॑सु॒ विलो॑म क्रि॒यते॒ तस्यै॒वैषा शान्तिः॑॥१२॥

%7.5.5.0
{\anuvakamend[{आ॒दि॒त्यस्तस्यै॒व द्वे च॑}]}%॥४॥

%7.5.5.1
यदि॒ सोमौ॒ सꣳसु॑तौ॒ स्याता᳚म्मह॒ति रात्रि॑यै प्रातरनुवा॒कमु॒पाकु॑र्या॒त्पूर्वो॒ वाच॒म्पूर्वो॑ दे॒वताः॒ पूर्व॒श्छन्दाꣳ॑सि वृङ्क्ते॒ वृष॑ण्वतीं प्रति॒पदं॑ कुर्यात्प्रातःसव॒नादे॒वैषा॒मिन्द्रं॑ वृ॒ङ्क्ते\-ऽथो॒ खल्वा॑हुः सवनमु॒खेस॑वनमुखे का॒र्येति॑ सवनमु॒खाथ्स॑वनमुखादे॒वैषा॒मिन्द्रं॑ वृङ्क्ते संवे॒शायो॑पवे॒शाय॑ गायत्रि॒यास्त्रि॒ष्टुभो॒ जग॑त्या अनु॒ष्टुभः॑ प॒ङ्क्त्या अ॒भिभू᳚त्यै॒ स्वाहा॒ छन्दाꣳ॑सि॒ वै सं॑वे॒श उ॑पवे॒शश्छन्दो॑भिरे॒वैषा᳚म्॥१३॥

%7.5.5.2
छन्दाꣳ॑सि वृङ्क्ते सज॒नीय॒ꣳ॒ शस्यं॑ विह॒व्यꣳ॑ शस्य॑म॒गस्त्य॑स्य कयाशु॒भीय॒ꣳ॒ शस्य॑मे॒ताव॒द्वा अ॑स्ति॒ याव॑दे॒तद्याव॑दे॒वास्ति॒ तदे॑षां वृङ्क्ते॒ यदि॑ प्रातःसव॒ने क॒लशो॒ दीर्ये॑त वैष्ण॒वीषु॑ शिपिवि॒ष्टव॑तीषु स्तुवीर॒न् यद्वै य॒ज्ञस्या॑ति॒रिच्य॑ते॒ विष्णुं॒ तच्छि॑पिवि॒ष्टम॒भ्यति॑ रिच्यते॒ तद्विष्णुः॑ शिविपि॒ष्टो\-ऽति॑रिक्त ए॒वाति॑रिक्तं दधा॒त्यथो॒ अति॑रिक्तेनै॒वाति॑रिक्तमा॒प्त्वाव॑ रुन्धते॒ यदि॑ म॒ध्यन्दि॑ने॒ दीर्ये॑त वषट्का॒रनि॑धन॒ꣳ॒ साम॑ कुर्युर्वषट्का॒रो वै य॒ज्ञस्य॑ प्रति॒ष्ठा प्र॑ति॒ष्ठामे॒वैन॑द्गमयन्ति॒ यदि॑ तृतीयसव॒न ए॒तदे॒व॥१४॥


%7.5.6.0
{\anuvakamend[{छन्दो॑भिरे॒वैषा॒मवैका॒न्नविꣳ॑श॒तिश्च॑}]}%॥५॥

%7.5.6.1
ष॒ड॒हैर्मासा᳚न्थ्स॒म्पाद्याह॒रुथ्सृ॑जन्ति षड॒हैर्\mbox{}हि मासा᳚न्थ्स॒म्पश्य॑न्त्यर्धमा॒सैर्मासा᳚न्थ्स॒म्पाद्याह॒रुथ्सृ॑जन्त्यर्धमा॒सैर्\mbox{}हि मासा᳚न्थ्स॒म्पश्य॑न्त्यमावा॒स्य॑या॒ मासा᳚न्थ्स॒म्पाद्याह॒रुथ्सृ॑जन्त्यमावा॒स्य॑या॒ हि मासा᳚न्थ्स॒म्पश्य॑न्ति पौर्णमा॒स्या मासा᳚न्थ्स॒म्पाद्याह॒रुथ्सृ॑जन्ति पौर्णमा॒स्या हि मासा᳚न्थ्स॒म्पश्य॑न्ति॒ यो वै पू॒र्ण आ॑सि॒ञ्चति॒ परा॒ स सि॑ञ्चति॒ यः पू॒र्णादु॒दच॑ति॥१५॥

%7.5.6.2
प्रा॒णम॑स्मि॒न्थ्स द॑धाति॒ यत्पौ᳚र्णमा॒स्या मासा᳚न्थ्स॒म्पाद्याह॑रुथ्सृ॒जन्ति॑ संवथ्स॒रायै॒व तत्प्रा॒णं द॑धति॒ तदनु॑ स॒त्त्रिणः॒ प्राण॑न्ति॒ यदह॒र्नोथ्सृ॒जेयु॒र्यथा॒ दृति॒रुप॑नद्धो वि॒पत॑त्ये॒वꣳ सं॑वथ्स॒रो वि प॑ते॒दार्ति॒मार्च्छे॑यु॒र्यत्पौ᳚र्णमा॒स्या मासा᳚न्थ्स॒म्पाद्याह॑रुथ्सृ॒जन्ति॑ संवथ्स॒रायै॒व तदु॑दा॒नं द॑धति॒ तदनु॑ स॒त्त्रिण॒ उत्॥१६॥

%7.5.6.3
अ॒न॒न्ति॒ नार्ति॒मार्च्छ॑न्ति पू॒र्णमा॑से॒ वै दे॒वानाꣳ॑ सु॒तो यत्पौ᳚र्णमा॒स्या मासा᳚न्थ्स॒म्पाद्याह॑रुथ्सृ॒जन्ति॑ दे॒वाना॑मे॒व तद्य॒ज्ञेन॑ य॒ज्ञम्प्र॒त्यव॑रोहन्ति॒ वि वा ए॒तद्य॒ज्ञं छि॑न्दन्ति॒ यत्ष॑ड॒हसं॑तत॒ꣳ॒ सन्त॒मथाह॑रुथ्सृ॒जन्ति॑ प्राजाप॒त्यम्प॒शुमाल॑भन्ते प्र॒जाप॑तिः॒ सर्वा॑ दे॒वता॑ दे॒वता॑भिरे॒व य॒ज्ञꣳ सं त॑न्वन्ति॒ यन्ति॒ वा ए॒ते सव॑ना॒द्ये\-ऽहः॑॥१७॥

%7.5.6.4
उ॒थ्सृ॒जन्ति॑ तु॒रीयं॒ खलु॒ वा ए॒तथ्सव॑नं॒ यथ्सा᳚न्ना॒य्यं यथ्सा᳚न्ना॒य्यम्भव॑ति॒ तेनै॒व सव॑ना॒न्न य॑न्ति समुप॒हूय॑ भक्षयन्त्ये॒तथ्सो॑मपीथा॒ ह्ये॑तर्\mbox{}हि॑ यथायत॒नं वा ए॒तेषाꣳ॑ सवन॒भाजो॑ दे॒वता॑ गच्छन्ति॒ ये\-ऽह॑रुथ्सृ॒जन्त्य॑नुसव॒नं पु॑रो॒डाशा॒न्निर्व॑पन्ति यथायत॒नादे॒व स॑वन॒भाजो॑ दे॒वता॒ अव॑ रुन्धते॒\-ऽष्टाक॑पालान्प्रातःसव॒न एका॑दशकपाला॒\-न्माध्य॑न्दिने॒ सव॑ने॒ द्वाद॑शकपालाꣴस्तृतीयसव॒ने छन्दाꣳ॑स्ये॒वाप्त्वाव॑ रुन्धते वैश्वदे॒वं च॒रुं तृ॑तीयसव॒ने निर्व॑पन्ति वैश्वदे॒वं वै तृ॑तीयसव॒नन्तेनै॒व तृ॑तीयसव॒नान्न य॑न्ति॥१८॥

%7.5.7.0
{\anuvakamend[{उ॒दच॒त्युद्ये\-ऽह॑रा॒प्त्वा पञ्च॑दश च}]}%॥६॥

%7.5.7.1
उ॒थ्सृज्या (३) न्नोथ्सृज्या (३) मिति॑ मीमाꣳसन्ते ब्रह्मवा॒दिन॒स्तद्वा॑हुरु॒थ्सृज्य॑मे॒वेत्य॑मावा॒स्या॑यां च पौर्णमा॒स्यां चो॒थ्सृज्य॒मित्या॑हुरे॒ते हि य॒ज्ञं वह॑त॒ इति॒ ते त्वाव नोथ्सृज्ये॒ इत्या॑हु॒र्ये अ॑वान्त॒रं य॒ज्ञम्भे॒जाते॒ इति॒ या प्र॑थ॒मा व्य॑ष्टका॒ तस्या॑मु॒थ्सृज्य॒मित्या॑हुरे॒ष वै मा॒सो वि॑श॒र इति॒ नादि॑ष्टम्॥१९॥

%7.5.7.2
उथ्सृ॑जेयु॒र्यदादि॑ष्टमुथ्सृ॒जेयु॑र्या॒दृशे॒ पुनः॑ पर्याप्ला॒वे मध्ये॑ षड॒हस्य॑ स॒म्पद्ये॑त षड॒हैर्मासा᳚न्थ्स॒म्पाद्य॒ यथ्स॑प्त॒ममह॒\-स्तस्मि॒न्नुथ्सृ॑ज्येयु॒स्तद॒ग्नये॒ वसु॑मते पुरो॒डाश॑म॒ष्टाक॑पालं॒ निर्व॑पेयुरै॒न्द्रं दधीन्द्रा॑य म॒रुत्व॑ते पुरो॒डाश॒मेका॑दशकपालं वैश्वदे॒वं द्वाद॑शकपालम॒ग्नेर्वै वसु॑मतः प्रातःसव॒नं यद॒ग्नये॒ वसु॑मते पुरो॒डाश॑म॒ष्टाक॑पालं नि॒र्वप॑न्ति दे॒वता॑मे॒व तद्भा॒गिनीं᳚ कु॒र्वन्ति॑॥२०॥

%7.5.7.3
सव॑नमष्टा॒भिरुप॑ यन्ति॒ यदैन्द्रं दधि॒ भव॒तीन्द्र॑मे॒व तद्भा॑ग॒धेया॒न्न च्या॑वय॒न्तीन्द्र॑स्य॒ वै म॒रुत्व॑तो॒ माध्यं॑दिन॒ꣳ॒ सव॑नं॒ यदिन्द्रा॑य म॒रुत्व॑ते पुरो॒डाश॒मेका॑दशकपालं नि॒र्वप॑न्ति दे॒वता॑मे॒व तद्भा॒गिनीं᳚ कु॒र्वन्ति॒ सव॑नमेकाद॒शभि॒रुप॑ यन्ति॒ विश्वे॑षां॒ वै दे॒वाना॑मृभु॒मतां᳚ तृतीयसव॒नं यद्वै᳚श्वदे॒वं द्वाद॑शकपालं नि॒र्वप॑न्ति दे॒वता॑ ए॒व तद्भा॒गिनीः᳚ कु॒र्वन्ति॒ सव॑नं द्वाद॒शभिः॑॥२१॥

%7.5.7.4
उप॑ यन्ति प्राजाप॒त्यम्प॒शुमा ल॑भन्ते य॒ज्ञो वै प्र॒जाप॑तिर्य॒ज्ञस्यान॑नुसर्गायाभिव॒र्त इ॒तः षण्मा॒सो ब्र॑ह्मसा॒मम्भ॑वति॒ ब्रह्म॒ वा अ॑भिव॒र्तो ब्रह्म॑णै॒व तथ्सु॑व॒र्गं लो॒कम॑भिव॒र्तय॑न्तो यन्ति प्रतिकू॒लमि॑व॒ हीतः सु॑व॒र्गो लो॒क इन्द्र॒ क्रतुं॑ न॒ आ भ॑र पि॒ता पु॒त्रेभ्यो॒ यथा᳚। शिक्षा॑ नो अ॒स्मिन्पु॑रुहूत॒ याम॑नि जी॒वा ज्योति॑रशीम॒हीत्य॒मुत॑ आय॒ताꣳ षण्मा॒सो ब्र॑ह्मसा॒मम्भ॑वत्य॒यं वै लो॒को ज्योतिः॑ प्र॒जा ज्योति॑रि॒ममे॒व तल्लो॒कम्पश्य॑न्तो\-ऽभि॒वद॑न्त॒ आ य॑न्ति॥२२॥

%7.5.8.0
{\anuvakamend[{नादि॑ष्टङ्कु॒र्वन्ति॑ द्वाद॒शभि॒रिति॑ विꣳश॒तिश्च॑}]}%॥७॥

%7.5.8.1
दे॒वानां॒ वा अन्तं॑ ज॒ग्मुषा॑मिन्द्रि॒यं वी॒र्य॑मपा᳚क्राम॒त्तत्क्रो॒शेनावा॑ रुन्धत॒ तत्क्रो॒शस्य॑ क्रोश॒त्वं यत्क्रो॒शेन॒ चात्वा॑ल॒स्यान्ते᳚ स्तु॒वन्ति॑ य॒ज्ञस्यै॒वान्तं॑ ग॒त्वेन्द्रि॒यं वी॒र्य॑मव॑ रुन्धते स॒त्त्रस्यर्द्ध्या॑हव॒नीय॒स्यान्ते᳚ स्तुवन्त्य॒ग्निमे॒वोप॑द्र॒ष्टारं॑ कृ॒त्वर्द्धि॒मुप॑ यन्ति प्र॒जाप॑ते॒र्\mbox{}हृद॑येन हवि॒र्धाने॒\-ऽन्तः स्तु॑वन्ति प्रे॒माण॑मे॒वास्य॑ गच्छन्ति श्लो॒केन॑ पु॒रस्ता॒थ्सद॑सः॥२३॥

%7.5.8.2
स्तु॒व॒न्त्यनु॑श्लोकेन प॒श्चाद्य॒ज्ञस्यै॒वान्तं॑ ग॒त्वा श्लो॑क॒भाजो॑ भवन्ति न॒वभि॑रध्व॒र्युरुद्गा॑यति॒ नव॒ वै पुरु॑षे प्रा॒णाः प्रा॒णाने॒व यज॑मानेषु दधाति॒ सर्वा॑ ऐ॒न्द्रियो॑ भवन्ति प्रा॒णेष्वे॒वेन्द्रि॒यं द॑ध॒त्यप्र॑तिहृताभि॒रुद्गा॑यति॒ तस्मा॒त्पुरु॑षः॒ सर्वा᳚ण्य॒न्यानि॑ शी॒र्ष्णो\-ऽङ्गा॑नि॒ प्रत्य॑चति॒ शिर॑ ए॒व न पञ्च॑द॒शꣳ र॑थन्त॒रम्भ॑वतीन्द्रि॒यमे॒वाव॑ रुन्धते सप्तद॒शम्॥२४॥

%7.5.8.3
बृ॒हद॒न्नाद्य॒स्याव॑रुद्ध्या॒ अथो॒ प्रैव तेन॑ जायन्त एकवि॒ꣳ॒शम्भ॒द्रं द्वि॒पदा॑सु॒ प्रति॑ष्ठित्यै॒ पत्न॑य॒ उप॑ गायन्ति मिथुन॒त्वाय॒ प्रजा᳚त्यै प्र॒जाप॑तिः प्र॒जा अ॑सृजत॒ सो॑\-ऽकामयता॒साम॒हꣳ रा॒ज्यं परी॑या॒मिति॒ तासाꣳ॑ राज॒नेनै॒व रा॒ज्यं पर्यै॒त्तद्रा॑ज॒नस्य॑ राजन॒त्वं यद्रा॑ज॒नम्भव॑ति प्र॒जाना॑मे॒व तद्यज॑माना रा॒ज्यं परि॑ यन्ति पञ्चवि॒ꣳ॒शम्भ॑वति प्र॒जाप॑तेः॥२५॥

%7.5.8.4
आप्त्यै॑ प॒ञ्चभि॒स्तिष्ठ॑न्तः स्तुवन्ति देवलो॒कमे॒वाभि ज॑यन्ति प॒ञ्चभि॒रासी॑ना मनुष्यलो॒कमे॒वाभि ज॑यन्ति॒ दश॒ सम्प॑द्यन्ते॒ दशा᳚क्षरा वि॒राडन्नं॑ वि॒राजै॒वान्नाद्य॒मव॑ रुन्धते पञ्च॒धा वि॑नि॒षद्य॑ स्तुवन्ति॒ पञ्च॒ दिशो॑ दि॒क्ष्वे॑व प्रति॑ तिष्ठ॒न्त्येकै॑क॒यास्तु॑तया स॒माय॑न्ति दि॒ग्भ्य ए॒वान्नाद्य॒ꣳ॒ सम्भ॑रन्ति॒ ताभि॑रुद्गा॒तोद्गा॑यति दि॒ग्भ्य ए॒वान्नाद्यम्᳚॥२६॥

%7.5.8.5
स॒म्भृत्य॒ तेज॑ आ॒त्मन्द॑धते॒ तस्मा॒देकः॑ प्रा॒णः सर्वा॒ण्यङ्गा᳚न्यव॒त्यथो॒ यथा॑ सुप॒र्ण उ॑त्पति॒ष्यञ्छिर॑ उत्त॒मं कु॑रु॒त ए॒वमे॒व तद्यज॑मानाः प्र॒जाना॑मुत्त॒मा भ॑वन्त्यास॒न्दीमु॑द्गा॒ता रो॑हति॒ साम्रा᳚ज्यमे॒व ग॑च्छन्ति प्ले॒ङ्खꣳ होता॒ नाक॑स्यै॒व पृ॒ष्ठꣳ रो॑हन्ति कू॒र्चाव॑ध्व॒र्युर्ब्र॒ध्नस्यै॒व वि॒ष्टपं॑ गच्छन्त्ये॒ताव॑न्तो॒ वै दे॑वलो॒कास्तेष्वे॒व य॑थापू॒र्वं प्रति॑ तिष्ठ॒न्त्यथो॑ आ॒क्रम॑णमे॒व तथ्सेतुं॒ यज॑मानाः कुर्वते सुव॒र्गस्य॑ लो॒कस्य॒ सम॑ष्ट्यै॥२७॥

%7.5.9.0
{\anuvakamend[{सद॑सः सप्तद॒शं प्र॒जाप॑तेर्गायति दि॒ग्भ्य ए॒वान्नाद्य॒म्प्रत्येका॑दश च}]}%॥८॥

%7.5.9.1
अ॒र्क्ये॑ण॒ वै स॑हस्र॒शः प्र॒जाप॑तिः प्र॒जा अ॑सृजत॒ ताभ्य॒ इला᳚न्दे॒नेरां॒ लूता॒मवा॑रुन्द्ध॒ यद॒र्क्य॑म्भव॑ति प्र॒जा ए॒व तद्यज॑मानाः सृजन्त॒ इला᳚न्दम्भवति प्र॒जाभ्य॑ ए॒व सृ॒ष्टाभ्य॒ इरां॒ लूता॒मव॑ रुन्धते॒ तस्मा॒द्याꣳ समाꣳ॑ स॒त्त्रꣳ समृ॑द्धं॒ क्षोधु॑का॒स्ताꣳ समां᳚ प्र॒जा इष॒ꣴ॒ ह्या॑सा॒मूर्ज॑मा॒दद॑ते॒ याꣳ समां॒ व्यृ॑द्ध॒मक्षो॑धुका॒स्ताꣳ समां᳚ प्र॒जाः॥२८॥

%7.5.9.2
न ह्या॑सा॒मिष॒मूर्ज॑मा॒दद॑त उत्क्रो॒दं कु॑र्वते॒ यथा॑ ब॒न्धान्मु॑मुचा॒ना उ॑त्क्रो॒दं कु॒र्वत॑ ए॒वमे॒व तद्यज॑माना देवब॒न्धान्मु॑मुचा॒ना उ॑त्क्रो॒दं कु॑र्वत॒ इष॒मूर्ज॑मा॒त्मन्दधा॑ना वा॒णः श॒तत॑न्तुर्भवति श॒तायुः॒ पुरु॑षः श॒तेन्द्रि॑य॒ आयु॑ष्ये॒वेन्द्रि॒ये प्रति॑ तिष्ठन्त्या॒जिं धा॑व॒न्त्यन॑भिजितस्या॒भिजि॑त्यै दुन्दु॒भीन्थ्स॒माघ्न॑न्ति पर॒मा वा ए॒षा वाग्या दु॑न्दु॒भौ प॑र॒मामे॒व॥२९॥

%7.5.9.3
वाच॒मव॑ रुन्धते भूमिदुन्दु॒भिमा घ्न॑न्ति॒ यैवेमां वाक्प्रवि॑ष्टा॒ तामे॒वाव॑ रुन्ध॒ते\-ऽथो॑ इ॒मामे॒व ज॑यन्ति॒ सर्वा॒ वाचो॑ वदन्ति॒ सर्वा॑सां वा॒चामव॑रुद्ध्या आ॒र्द्रे चर्म॒न्व्याय॑च्छेते इन्द्रि॒यस्याव॑रुद्ध्या॒ आन्यः क्रोश॑ति॒ प्रान्यः शꣳ॑सति॒ य आ॒क्रोश॑ति पु॒नात्ये॒वैना॒न्थ्स स यः प्र॒शꣳस॑ति पू॒तेष्वे॒वान्नाद्यं॑ दधा॒त्यृषि॑कृतं च॥३०॥

%7.5.9.4
वा ए॒ते दे॒वकृ॑तं च॒ पूर्वै॒र्मासै॒रव॑ रुन्धते॒ यद्भू॑ते॒च्छदा॒ꣳ॒ सामा॑नि॒ भव॑न्त्यु॒भय॒स्याव॑रुद्ध्यै॒ यन्ति॒ वा ए॒ते मि॑थु॒नाद्ये सं॑वथ्स॒रमु॑प॒यन्त्य॑न्तर्वे॒दि मि॑थु॒नौ सम्भ॑वत॒स्तेनै॒व मि॑थु॒नान्न य॑न्ति॥३१॥

%7.5.10.0
{\anuvakamend[{व्यृ॑द्ध॒मक्षो॑धुका॒स्ताꣳ समां᳚ प्र॒जाः प॑र॒मामे॒व च॑ त्रि॒ꣳ॒शच्च॑}]}%॥९॥

%7.5.10.1
चर्माव॑ भिन्दन्ति पा॒प्मान॑मे॒वैषा॒मव॑ भिन्दन्ति॒ माप॑ राथ्सी॒र्माति॑ व्याथ्सी॒रित्या॑ह सम्प्र॒त्ये॑वैषां᳚ पा॒प्मान॒मव॑ भिन्दन्त्युदकु॒म्भान॑धिनि॒धाय॑ दा॒स्यो॑ मार्जा॒लीयं॒ परि॑ नृत्यन्ति प॒दो नि॑घ्न॒तीरि॒दम्म॑धुं॒ गाय॑न्त्यो॒ मधु॒ वै दे॒वानां᳚ पर॒मम॒न्नाद्यं॑ पर॒ममे॒वान्नाद्य॒मव॑ रुन्धते प॒दो नि घ्न॑न्ति मही॒यामे॒वैषु॑ दधति॥३२॥

%7.5.11.0
{\anuvakamend[{चर्मैका॒न्नप॑ञ्चा॒शत्}]}%॥10॥

%7.5.11.1
पृ॒थि॒व्यै स्वाहा॒न्तरि॑क्षाय॒ स्वाहा॑ दि॒वे स्वाहा॑ सम्प्लोष्य॒ते स्वाहा॑ स॒म्प्लव॑मानाय॒ स्वाहा॒ सम्प्लु॑ताय॒ स्वाहा॑ मेघायिष्य॒ते स्वाहा॑ मेघाय॒ते स्वाहा॑ मेघि॒ताय॒ स्वाहा॑ मे॒घाय॒ स्वाहा॑ नीहा॒राय॒ स्वाहा॑ नि॒हाका॑यै॒ स्वाहा᳚ प्रास॒चाय॒ स्वाहा᳚ प्रच॒लाका॑यै॒ स्वाहा॑ विद्योतिष्य॒ते स्वाहा॑ वि॒द्योत॑मानाय॒ स्वाहा॑ संवि॒द्योत॑मानाय॒ स्वाहा᳚ स्तनयिष्य॒ते स्वाहा᳚ स्त॒नय॑ते॒ स्वाहो॒ग्रꣴ स्त॒नय॑ते॒ स्वाहा॑ वर्\mbox{}षिष्य॒ते स्वाहा॒ वर्\mbox{}ष॑ते॒ स्वाहा॑भि॒वर्\mbox{}ष॑ते॒ स्वाहा॑ परि॒वर्\mbox{}ष॑ते॒ स्वाहा॑ सं॒वर्\mbox{}ष॑ते॥३३॥

%7.5.11.2
स्वाहा॑नु॒वर्\mbox{}ष॑ते॒ स्वाहा॑ शीकायिष्य॒ते स्वाहा॑ शीकाय॒ते स्वाहा॑ शीकि॒ताय॒ स्वाहा᳚ प्रोषिष्य॒ते स्वाहा᳚ प्रुष्ण॒ते स्वाहा॑ परिप्रुष्ण॒ते स्वाहो᳚द्ग्रहीष्य॒ते स्वाहो᳚द्गृह्ण॒ते स्वाहोद्गृ॑हीताय॒ स्वाहा॑ विप्लोष्य॒ते स्वाहा॑ वि॒प्लव॑मानाय॒ स्वाहा॒ विप्लु॑ताय॒ स्वाहा॑तफ्स्य॒ते स्वाहा॒तप॑ते॒ स्वाहो॒ग्रमा॒तप॑ते॒ स्वाह॒र्ग्भ्यः स्वाहा॒ यजु॑र्भ्यः॒ स्वाहा॒ साम॑भ्यः॒ स्वाहाङ्गि॑रोभ्यः॒ स्वाहा॒ वेदे᳚भ्यः॒ स्वाहा॒ गाथा᳚भ्यः॒ स्वाहा॑ नाराश॒ꣳ॒सीभ्यः॒ स्वाहा॒ रैभी᳚भ्यः॒ स्वाहा॒ सर्व॑स्मै॒ स्वाहा᳚॥३४॥

%7.5.12.0
{\anuvakamend[{सं॒ वर्\mbox{}ष॑ते॒ रैभी᳚भ्यः॒ स्वाहा॒ द्वे च॑}]}%॥11॥

%7.5.12.1
द॒त्वते॒ स्वाहा॑\-ऽद॒न्तका॑य॒ स्वाहा᳚ प्रा॒णिने॒ स्वाहा᳚\-ऽप्रा॒णाय॒ स्वाहा॒ मुख॑वते॒ स्वाहा॑\-ऽमु॒खाय॒ स्वाहा॒ नासि॑कवते॒ स्वाहा॑\-ऽनासि॒काय॒ स्वाहा᳚\-ऽक्ष॒ण्वते॒ स्वाहा॑\-ऽन॒क्षिका॑य॒ स्वाहा॑ क॒र्णिने॒ स्वाहा॑\-ऽक॒र्णका॑य॒ स्वाहा॑ शीर्\mbox{}ष॒ण्वते॒ स्वाहा॑\-ऽ\-शी॒र्\mbox{}षका॑य॒ स्वाहा॑ प॒द्वते॒ स्वाहा॑\-ऽपा॒दका॑य॒ स्वाहा᳚ प्राण॒ते स्वाहा\-ऽप्रा॑णते॒ स्वाहा॒ वद॑ते॒ स्वाहा\-ऽव॑दते॒ स्वाहा॒ पश्य॑ते॒ स्वाहा\-ऽप॑श्यते॒ स्वाहा॑ शृण्व॒ते स्वाहा\-ऽशृ॑ण्वते॒ स्वाहा॑ मन॒स्विने॒ स्वाहा᳚॥३५॥

%7.5.12.2
अ॒म॒नसे॒ स्वाहा॑ रेत॒स्विने॒ स्वाहा॑\-ऽरे॒तस्का॑य॒ स्वाहा᳚ प्र॒जाभ्यः॒ स्वाहा᳚ प्र॒जन॑नाय॒ स्वाहा॒ लोम॑वते॒ स्वाहा॑\-ऽलो॒मका॑य॒ स्वाहा᳚ त्व॒चे स्वाहा॒\-ऽत्वक्का॑य॒ स्वाहा॒ चर्म॑ण्वते॒ स्वाहा॑\-ऽच॒र्मका॑य॒ स्वाहा॒ लोहि॑तवते॒ स्वाहा॑\-ऽलोहि॒ताय॒ स्वाहा॑ माꣳस॒न्वते॒ स्वाहा॑\-ऽमा॒ꣳ॒सका॑य॒ स्वाहा॒ स्नाव॑भ्यः॒ स्वाहा᳚\-ऽस्ना॒वका॑य॒ स्वाहा᳚\-ऽस्थ॒न्वते॒ स्वाहा॑\-ऽन॒स्थिका॑य॒ स्वाहा॑ मज्ज॒न्वते॒ स्वाहा॑\-ऽम॒ज्जका॑य॒ स्वाहा॒\-ऽङ्गिने॒ स्वाहा॑\-ऽन॒ङ्गाय॒ स्वाहा॒\-ऽ\-ऽत्मने॒ स्वाहा\-ऽना᳚त्मने॒ स्वाहा॒ सर्व॑स्मै॒ स्वाहा᳚॥३६॥

%7.5.13.0
{\anuvakamend[{म॒न॒स्विने॒ स्वाहा\-ऽना᳚त्मने॒ स्वाहा॒ द्वे च॑}]}%॥12॥

%7.5.13.1
कस्त्वा॑ युनक्ति॒ स त्वा॑ युनक्तु॒ विष्णु॑स्त्वा युनक्त्व॒स्य य॒ज्ञस्यर्द्ध्यै॒ मह्य॒ꣳ॒ सन्न॑त्या अ॒मुष्मै॒ कामा॒यायु॑षे त्वा प्रा॒णाय॑ त्वा\-ऽपा॒नाय॑ त्वा व्या॒नाय॑ त्वा॒ व्यु॑ष्ट्यै त्वा र॒य्यै त्वा॒ राध॑से त्वा॒ घोषा॑य त्वा॒ पोषा॑य त्वाराद्घो॒षाय॑ त्वा॒ प्रच्यु॑त्यै त्वा॥३७॥

%7.5.14.0
{\anuvakamend[{कस्त्वा॒\-ऽष्टात्रिꣳ॑शत्}]}%॥13॥

%7.5.14.1
अ॒ग्नये॑ गाय॒त्राय॑ त्रि॒वृते॒ राथं॑तराय वास॒न्ताया॒ष्टाक॑पाल॒ इन्द्रा॑य॒ त्रैष्टु॑भाय पञ्चद॒शाय॒ बार्\mbox{}ह॑ताय॒ ग्रैष्मा॒यैका॑दशकपालो॒ विश्वे᳚भ्यो दे॒वेभ्यो॒ जाग॑तेभ्यः सप्तद॒शेभ्यो॑ वैरू॒पेभ्यो॒ वार्\mbox{}षि॑केभ्यो॒ द्वाद॑शकपालो मि॒त्रावरु॑णाभ्या॒मानु॑ष्टुभाभ्यामेक\-वि॒ꣳ॒शा\-भ्यां᳚ वैरा॒जाभ्याꣳ॑ शार॒दा\-भ्यां᳚ पय॒स्या॑ बृह॒स्पत॑ये॒ पाङ्क्ता॑य त्रिण॒वाय॑ शाक्व॒राय॒ हैम॑न्तिकाय च॒रुः स॑वि॒त्र आ॑तिच्छन्द॒साय॑ त्रयस्त्रि॒ꣳ॒शाय॑ रैव॒ताय॑ शैशि॒राय॒ द्वाद॑शकपा॒लो\-ऽदि॑त्यै॒ विष्णु॑पत्न्यै च॒रुर॒ग्नये॑ वैश्वान॒राय॒ द्वाद॑शकपा॒लो\-ऽनु॑मत्यै च॒रुः का॒य एक॑कपालः॥३८॥

%7.5.15.0
{\anuvakamend[{अ॒ग्नये\-ऽदि॑त्या॒ अनु॑मत्यै स॒प्तच॑त्वारिꣳशत्}]}%॥14॥

%7.5.15.1
यो वा अ॒ग्नाव॒ग्निः प्र॑ह्रि॒यते॒ यश्च॒ सोमो॒ राजा॒ तयो॑रे॒ष आ॑ति॒थ्यं यद॑ग्नीषो॒मीयो\-ऽथै॒ष रु॒द्रो यश्ची॒यते॒ यथ्सञ्चि॑ते॒\-ऽग्नावे॒तानि॑ ह॒वीꣳषि॒ न नि॒र्वपे॑दे॒ष ए॒व रु॒द्रो\-ऽशा᳚न्त उपो॒त्थाय॑ प्र॒जां प॒शून् यज॑मानस्या॒भि म॑न्येत॒ यथ्सञ्चि॑ते॒\-ऽग्नावे॒तानि॑ ह॒वीꣳषि॑ नि॒र्वप॑ति भाग॒धेये॑नै॒वैनꣳ॑ शमयति॒ नास्य॑ रु॒द्रो\-ऽशा᳚न्तः॥३९॥

%7.5.15.2
उ॒पो॒त्थाय॑ प्र॒जां प॒शून॒भि म॑न्यते॒ दश॑ ह॒वीꣳषि॑ भवन्ति॒ नव॒ वै पुरु॑षे प्रा॒णा नाभि॑र्दश॒मी प्रा॒णाने॒व यज॑माने दधा॒त्यथो॒ दशा᳚क्षरा वि॒राडन्नं॑ वि॒राज्ये॒वान्नाद्ये॒ प्रति॑ तिष्ठत्यृ॒तुभि॒र्वा ए॒ष छन्दो॑भिः॒ स्तोमैः᳚ पृ॒ष्ठैश्चे॑त॒व्य॑ इत्या॑हु॒र्यदे॒तानि॑ ह॒वीꣳषि॑ नि॒र्वप॑त्यृ॒तुभि॑रे॒वैनं॒ छन्दो॑भिः॒ स्तोमैः᳚ पृ॒ष्ठैश्चि॑नुते॒ दिशः॑ सुषुवा॒णेन॑॥४०॥

%7.5.15.3
अ॒भि॒जित्या॒ इत्या॑हु॒र्यदे॒तानि॑ ह॒वीꣳषि॑ नि॒र्वप॑ति दि॒शाम॒भिजि॑त्या ए॒तया॒ वा इन्द्रं॑ दे॒वा अ॑याजय॒न्तस्मा॑दिन्द्रस॒व ए॒तया॒ मनु॑म्मनु॒ष्या᳚स्तस्मा᳚न्मनुस॒वो यथेन्द्रो॑ दे॒वानां॒ यथा॒ मनु॑र्मनु॒ष्या॑णामे॒वम्भ॑वति॒ य ए॒वं वि॒द्वाने॒तयेष्ट्या॒ यज॑ते॒ दिग्व॑तीः पुरोनुवा॒क्या॑ भवन्ति॒ सर्वा॑सां दि॒शाम॒भिजि॑त्यै॥४१॥

%7.5.16.0
{\anuvakamend[{अशा᳚न्तः सुषुवा॒णेनैक॑चत्वारिꣳशच्च}]}%॥15॥

%7.5.16.1
यः प्रा॑ण॒तो नि॑मिष॒तो म॑हि॒त्वैक॒ इद्राजा॒ जग॑तो ब॒भूव॑। य ईशे॑ अ॒स्य द्वि॒पद॒श्चतु॑ष्पदः॒ कस्मै॑ दे॒वाय॑ ह॒विषा॑ विधेम। उ॒प॒या॒मगृ॑हीतो\-ऽसि प्र॒जाप॑तये त्वा॒ जुष्टं॑ गृह्णामि॒ तस्य॑ ते॒ द्यौर्म॑हि॒मा नक्ष॑त्राणि रू॒पमा॑दि॒त्यस्ते॒ तेज॒स्तस्मै᳚ त्वा महि॒म्ने प्र॒जाप॑तये॒ स्वाहा᳚॥४२॥

%7.5.17.0
{\anuvakamend[{यः प्रा॑ण॒तो द्यौरा॑दि॒त्यो᳚\-ऽष्टात्रिꣳ॑शत्}]}%॥16॥

%7.5.17.1
य आ᳚त्म॒दा ब॑ल॒दा यस्य॒ विश्व॑ उ॒पास॑ते प्र॒शिषं॒ यस्य॑ दे॒वाः। यस्य॑ छा॒यामृतं॒ यस्य॑ मृ॒त्युः कस्मै॑ दे॒वाय॑ ह॒विषा॑ विधेम। उ॒प॒या॒मगृ॑हीतो\-ऽसि प्र॒जाप॑तये त्वा॒ जुष्टं॑ गृह्णामि॒ तस्य॑ ते पृथि॒वी म॑हि॒मौष॑धयो॒ वन॒स्पत॑यो रू॒पम॒ग्निस्ते॒ तेज॒स्तस्मै᳚ त्वा महि॒म्ने प्र॒जाप॑तये॒ स्वाहा᳚॥४३॥

%7.5.18.0
{\anuvakamend[{य आ᳚त्म॒दाः पृ॑थि॒व्य॑ग्निरेका॒न्नच॑त्वारि॒ꣳ॒शत्}]}%॥17॥

%7.5.18.1
आ ब्रह्म॑न्ब्राह्म॒णो ब्र॑ह्मवर्च॒सी जा॑यता॒मा\-ऽस्मिन्रा॒ष्ट्रे रा॑ज॒न्य॑ इष॒व्यः॑ शूरो॑ महार॒थो जा॑यता॒न्दोग्ध्री॑ धे॒नुर्वोढा॑\-ऽ\-न॒ड्वाना॒शुः सप्तिः॒ पुरं॑धि॒र्योषा॑ जि॒ष्णू र॑थे॒ष्ठाः स॒भेयो॒ युवा\-ऽस्य यज॑मानस्य वी॒रो जा॑यतान्निका॒मेनि॑कामे नः प॒र्जन्यो॑ वर्\mbox{}षतु फ॒लिन्यो॑ न॒ ओष॑धयः पच्यन्तां योगक्षे॒मो नः॑ कल्पताम्॥४४॥

%7.5.19.0
{\anuvakamend[{आ ब्रह्म॒न्नेक॑चत्वारिꣳशत्}]}%॥18॥

%7.5.19.1
आक्रान्॑ वा॒जी पृ॑थि॒वीम॒ग्निं युज॑मकृत वा॒ज्यर्वाक्रान्॑ वा॒ज्य॑न्तरि॑क्षं वा॒युं युज॑मकृत वा॒ज्यर्वा॒ द्यां वा॒ज्या\-ऽक्रꣴ॑स्त॒ सूर्यं॒ युज॑मकृत वा॒ज्यर्वा॒ग्निस्ते॑ वाजि॒न् युङ्ङनु॒ त्वा र॑भे स्व॒स्ति मा॒ सम्पा॑रय वा॒युस्ते॑ वाजि॒न् युङ्ङनु॒ त्वा र॑भे स्व॒स्ति मा॒ सम्॥४५॥

%7.5.19.2
पा॒र॒यादि॒त्यस्ते॑ वाजि॒न् युङ्ङनु॒ त्वा र॑भे स्व॒स्ति मा॒ सम्पा॑रय प्राण॒धृग॑सि प्रा॒णं मे॑ दृꣳह व्यान॒धृग॑सि व्या॒नं मे॑ दृꣳहापान॒धृग॑स्यपा॒नं मे॑ दृꣳह॒ चक्षु॑रसि॒ चक्षु॒र्मयि॑ धेहि॒ श्रोत्र॑मसि॒ श्रोत्र॒म्मयि॑ धे॒ह्यायु॑र॒स्यायु॒र्मयि॑ धेहि॥४६॥

%7.5.20.0
{\anuvakamend[{वा॒युस्ते॑ वाजि॒न् युङ्ङनु॒ त्वा र॑भे स्व॒स्ति मा॒ सन्त्रिच॑त्वारिꣳशच्च}]}%॥19॥

%7.5.20.1
जज्ञि॒ बीजं॒ वर्\mbox{}ष्टा॑ प॒र्जन्यः॒ पक्ता॑ स॒स्यꣳ सु॑पिप्प॒ला ओष॑धयः स्वधिचर॒णेयꣳ सू॑पसद॒नो᳚\-ऽग्निः स्व॑ध्य॒क्षम॒न्तरि॑क्षꣳ सुपा॒वः पव॑मानः सूपस्था॒ना द्यौः शि॒वम॒सौ तप॑न् यथापू॒र्वम॑होरा॒त्रे प॑ञ्चद॒शिनो᳚\-ऽर्धमा॒सास्त्रि॒ꣳ॒शिनो॒ मासाः᳚ कॢ॒प्ता ऋ॒तवः॑ शा॒न्तः सं॑वथ्स॒रः॥४७॥

%7.5.21.0
{\anuvakamend[{जज्ञि॒ बीज॒मेक॑त्रिꣳशत्}]}%॥20॥

%7.5.21.1
आ॒ग्ने॒यो᳚\-ऽष्टाक॑पालः सौ॒म्यश्च॒रुः सा॑वि॒त्रो᳚\-ऽष्टाक॑पालः पौ॒ष्णश्च॒रू रौ॒द्रश्च॒रुर॒ग्नये॑ वैश्वान॒राय॒ द्वाद॑शकपालो मृगाख॒रे यदि॒ नागच्छे॑द॒ग्नये\-ऽꣳ॑हो॒मुचे॒\-ऽष्टाक॑पालः सौ॒र्यम्पयो॑ वाय॒व्य॑ आज्य॑भागः॥४८॥

%7.5.22.0
{\anuvakamend[{आ॒ग्ने॒यश्चतु॑र्विꣳशतिः}]}%॥21॥

%7.5.22.1
अ॒ग्नये\-ऽꣳ॑हो॒मुचे॒\-ऽष्टाक॑पाल॒ इन्द्रा॑याꣳहो॒मुच॒ एका॑दशकपालो मि॒त्रावरु॑णाभ्यामागो॒मुग्\-भ्यां᳚ पय॒स्या॑ वायोसावि॒त्र आ॑गो॒मुग्\-भ्यां᳚ च॒रुर॒श्विभ्या॑मागो॒मुग्\-भ्यां᳚ धा॒ना म॒रुद्भ्य॑ एनो॒मुग्भ्यः॑ स॒प्तक॑पालो॒ विश्वे᳚भ्यो दे॒वेभ्य॑ एनो॒मुग्भ्यो॒ द्वाद॑शकपा॒लो\-ऽनु॑मत्यै च॒रुर॒ग्नये॑ वैश्वान॒राय॒ द्वाद॑शकपालो॒ द्यावा॑पृथि॒वीभ्या॑मꣳहो॒मुग्\-भ्यां᳚ द्विकपा॒लः॥४९॥

%7.5.23.0
{\anuvakamend[{अ॒ग्नये\-ऽꣳ॑हो॒मुचे᳚ त्रि॒ꣳ॒शत्}]}%॥22॥

%7.5.23.1
अ॒ग्नये॒ सम॑नमत्पृथि॒व्यै सम॑नम॒द्यथा॒ग्निः पृ॑थि॒व्या स॒मन॑मदे॒वम्मह्य॑म्भ॒द्राः सन्न॑तयः॒ सं न॑मन्तु वा॒यवे॒ सम॑नमद॒न्तरि॑क्षाय॒ सम॑नम॒द्यथा॑ वा॒युर॒न्तरि॑क्षेण॒ सूर्या॑य॒ सम॑नमद्दि॒वे सम॑नम॒द्यथा॒ सूर्यो॑ दि॒वा च॒न्द्रम॑से॒ सम॑नम॒न्नक्ष॑त्रेभ्यः॒ सम॑नम॒द्यथा॑ च॒न्द्रमा॒ नक्ष॑त्रै॒र्वरु॑णाय॒ सम॑नमद॒द्भ्यः सम॑नम॒द्यथा᳚॥५०॥

%7.5.23.2
वरु॑णो॒\-ऽद्भिः साम्ने॒ सम॑नमदृ॒चे सम॑नम॒द्यथा॒ साम॒र्चा ब्रह्म॑णे॒ सम॑नमत्क्ष॒त्राय॒ सम॑नम॒द्यथा॒ ब्रह्म॑ क्ष॒त्रेण॒ राज्ञे॒ सम॑नमद्वि॒शे सम॑नम॒द्यथा॒ राजा॑ वि॒शा रथा॑य॒ सम॑नम॒दश्वे᳚भ्यः॒ सम॑नम॒द्यथा॒ रथो\-ऽश्वैः᳚ प्र॒जाप॑तये॒ सम॑नमद्भू॒तेभ्यः॒ सम॑नम॒द्यथा᳚ प्र॒जाप॑तिर्भू॒तैः स॒मन॑मदे॒वम्मह्य॑म्भ॒द्राः सन्न॑तयः॒ सं न॑मन्तु॥५१॥

%7.5.24.0
{\anuvakamend[{अ॒द्भ्यः सम॑नम॒द्यथा॒ मह्यं॑ च॒त्वारि॑ च}]}%॥23॥

%7.5.24.1
ये ते॒ पन्था॑नः सवितः पू॒र्व्यासो॑\-ऽरे॒णवो॒ वित॑ता अ॒न्तरि॑क्षे। तेभि॑र्नो अ॒द्य प॒थिभिः॑ सु॒गेभी॒ रक्षा॑ च नो॒ अधि॑ च देव ब्रूहि। नमो॒\-ऽग्नये॑ पृथिवि॒क्षिते॑ लोक॒स्पृते॑ लो॒कम॒स्मै यज॑मानाय देहि॒ नमो॑ वा॒यवे᳚\-ऽन्तरिक्ष॒क्षिते॑ लोक॒स्पृते॑ लो॒कम॒स्मै यज॑मानाय देहि॒ नमः॒ सूर्या॑य दिवि॒क्षिते॑ लोक॒स्पृते॑ लो॒कम॒स्मै यज॑मानाय देहि॥५२॥

%7.5.25.0
{\anuvakamend[{ये ते॒ चतु॑श्चत्वारिꣳशत्}]}%॥24॥

%7.5.25.1
यो वा अश्व॑स्य॒ मेध्य॑स्य॒ शिरो॒ वेद॑ शीर्\mbox{}ष॒ण्वान्मेध्यो॑ भवत्यु॒षा वा अश्व॑स्य॒ मेध्य॑स्य॒ शिरः॒ सूर्य॒श्चक्षु॒र्वातः॑ प्रा॒णश्च॒न्द्रमाः॒ श्रोत्र॒न्दिशः॒ पादा॑ अवान्तरदि॒शाः पर्\mbox{}श॑वो\-ऽहोरा॒त्रे नि॑मे॒षो᳚\-ऽर्धमा॒साः पर्वा॑णि॒ मासाः᳚ सं॒धाना᳚न्यृ॒तवो\-ऽङ्गा॑नि संवथ्स॒र आ॒त्मा र॒श्मयः॒ केशा॒ नक्ष॑त्राणि रू॒पन्तार॑का अ॒स्थानि॒ नभो॑ मा॒ꣳ॒सान्योष॑धयो॒ लोमा॑नि॒ वन॒स्पत॑यो॒ वाला॑ अ॒ग्निर्मुखं॑ वैश्वान॒रो व्यात्तम्᳚॥५३॥

%7.5.25.2
स॒मु॒द्र उ॒दर॑म॒न्तरि॑क्षम्पा॒युर्द्यावा॑पृथि॒वी आ॒ण्डौ ग्रावा॒ शेपः॒ सोमो॒ रेतो॒ यज्ज॑ञ्ज॒भ्यते॒ तद्वि द्यो॑तते॒ यद्वि॑धूनु॒ते तथ्स्त॑नयति॒ यन्मेह॑ति॒ तद्व॑र्\mbox{}षति॒ वागे॒वास्य॒ वागह॒र्वा अश्व॑स्य॒ जाय॑मानस्य महि॒मा पु॒रस्ता᳚ज्जायते॒ रात्रि॑रेनम्महि॒मा प॒श्चादनु॑ जायत ए॒तौ वै म॑हि॒माना॒वश्व॑म॒भितः॒ सम्ब॑भूवतु॒र्\mbox{}हयो॑ दे॒वान॑वह॒दर्वासु॑रान् वा॒जी ग॑न्ध॒र्वानश्वो॑ मनु॒ष्या᳚न्थ्समु॒द्रो वा अश्व॑स्य॒ योनिः॑ समु॒द्रो बन्धुः॑॥५४॥

{\anuvakamend[{व्यात्त॑मवह॒द्द्वाद॑श च}]}%॥25॥

{\anuvakamend[{गावो॒ गावः॒ सिषा॑सन्तीः प्रथ॒मे मा॒सि स॑मा॒न्यो॑ यदि॒ सोमौ॑ षड॒हैरु॒थ्सृज्या(३)ं दे॒वाना॑म॒र्क्ये॑ण॒ चर्माव॑ पृथि॒व्यै द॒त्वते॒ कस्त्वा॒ग्नये॒ यो वै यः प्रा॑ण॒तो य आ᳚त्म॒दा आ ब्रह्म॒न्नाक्रा॒ञ्जज्ञि॒ बीज॑माग्ने॒यो᳚\-ऽष्टाक॑पालो॒\-ऽग्नये\-ऽꣳ॑हो॒मुचे॒\-ऽ\-ष्टाक॑पालो॒\-ऽग्नये॒ सम॑नम॒द्ये ते॒ पन्था॑नो॒ यो वा अश्व॑स्य॒ मेध्य॑स्य॒ शिरः॒ प़ञ्च॑विꣳशतिः}]%॥25॥
}

\prashnaend[{गावः॑ समा॒न्यः॑ सव॑नमष्टा॒भिर्वा ए॒ते दे॒वकृ॑तञ्चाभि॒जित्या॒ इत्या॑हु॒र्वरु॑णो॒\-ऽद्भिः साम्ने॒ चतुः॑पञ्चा॒शत्॥54॥ गावो॒ योनिः॑ समु॒द्रो बन्धुः॑॥}]
%%% END PRASHNA
%%% END KANDAM
