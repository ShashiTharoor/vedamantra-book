\chapt{काण्डम् ४}
\sect{प्रथमः प्रश्नः}\setcounter{anuvakam}{0}
\dnsub{तैत्तिरीयसंहितायां चतुर्थकाण्डे प्रथमः प्रश्नः}
%4.1.1.0
%4.1.1.1
यु॒ञ्जा॒नः प्र॑थ॒मम्मन॑स्त॒त्वाय॑ सवि॒ता धियः॑। अ॒ग्निं ज्योति॑र्नि॒चाय्य॑ पृथि॒व्या अध्याभ॑रत्। यु॒क्त्वाय॒ मन॑सा दे॒वान्थ्सुव॑र्य॒तो धि॒या दिवम्᳚। बृ॒हज्ज्योतिः॑ करिष्य॒तः स॑वि॒ता प्र सु॑वति॒ तान्। यु॒क्तेन॒ मन॑सा व॒यं दे॒वस्य॑ सवि॒तुः स॒वे। सु॒व॒र्गेया॑य॒ शक्त्यै᳚। यु॒ञ्जते॒ मन॑ उ॒त यु॑ञ्जते॒ धियो॒ विप्रा॒ विप्र॑स्य बृह॒तो वि॑प॒श्चितः॑। वि होत्रा॑ दधे वयुना॒विदेक॒ इत्~(१)

%4.1.1.2
म॒ही दे॒वस्य॑ सवि॒तुः परि॑ष्टुतिः। यु॒जे वां॒ ब्रह्म॑ पू॒र्व्यं नमो॑भि॒र्वि श्लोका॑ यन्ति प॒थ्ये॑व॒ सूराः᳚। शृ॒ण्वन्ति॒ विश्वे॑ अ॒मृत॑स्य पु॒त्रा आ ये धामा॑नि दि॒व्यानि॑ त॒स्थुः। यस्य॑ प्र॒याण॒मन्व॒न्य इद्य॒युर्दे॒वा दे॒वस्य॑ महि॒मान॒मर्च॑तः। यः पार्थि॑वानि विम॒मे स एत॑शो॒ रजाꣳ॑सि दे॒वः स॑वि॒ता म॑हित्व॒ना। देव॑ सवितः॒ प्र सु॑व य॒ज्ञम्प्र सु॑व~(२)

%4.1.1.3
य॒ज्ञप॑ति॒म्भगा॑य दि॒व्यो ग॑न्ध॒र्वः। के॒त॒पूः केतं॑ नः पुनातु वा॒चस्पति॒र्वाच॑म॒द्य स्व॑दाति नः। इ॒मं नो॑ देव सवितर्य॒ज्ञं प्र सु॑व देवा॒युवꣳ॑ सखि॒विदꣳ॑ सत्रा॒जितं॑ धन॒जितꣳ॑ सुव॒र्जितम्᳚। ऋ॒चा स्तोम॒ꣳ॒ सम॑र्धय गाय॒त्रेण॑ रथन्त॒रम्। बृ॒हद्गा॑य॒त्रव॑र्तनि। दे॒वस्य॑ त्वा सवि॒तुः प्र॑स॒वे᳚\-ऽश्विनो᳚र्बा॒हु\-भ्यां᳚ पू॒ष्णो हस्ता᳚भ्याम्गाय॒त्रेण॒ छन्द॒सा\-ऽ\-ऽद॑दे\-ऽङ्गिर॒स्वदभ्रि॑रसि॒ नारिः॑~(३)

%4.1.1.4
अ॒सि॒ पृ॒थि॒व्याः स॒धस्था॑द॒ग्निम्पु॑री॒ष्य॑मङ्गिर॒स्वदा भ॑र॒ त्रैष्टु॑भेन त्वा॒ छन्द॒सा\-ऽ\-ऽद॑दे\-ऽङ्गिर॒स्वद्बभ्रि॑रसि॒ नारि॑रसि॒ त्वया॑ व॒यꣳ स॒धस्थ॒ आग्निꣳ श॑केम॒ खनि॑तुं पुरी॒ष्यं॑ जाग॑तेन त्वा॒ छन्द॒सा\-ऽ\-ऽद॑दे\-ऽङ्गिर॒स्वद्धस्त॑ आ॒धाय॑ सवि॒ता बिभ्र॒दभ्रिꣳ॑ हिर॒ण्ययी᳚म्। तया॒ ज्योति॒रज॑स्र॒मिद॒ग्निं खा॒त्वी न॒ आ भ॒रानु॑ष्टुभेन त्वा॒ छन्द॒सा\-ऽ\-ऽद॑दे\-ऽङ्गिर॒स्वत्॥

%4.1.2.0
{\anuvakamend[{इद्य॒ज्ञं प्र सु॑व॒ नारि॒रानु॑ष्टुभेन त्वा॒ छन्द॑सा॒ त्रीणि॑ च}]}%~(१)

%4.1.2.1
इ॒माम॑गृभ्णन्रश॒नामृ॒तस्य॒ पूर्व॒ आयु॑षि वि॒दथे॑षु क॒व्या। तया॑ दे॒वाः सु॒तमा ब॑भूवुर्\mbox{}ऋ॒तस्य॒ साम᳚न्थ्स॒रमा॒रप॑न्ती। प्रतू᳚र्तं वाजि॒न्ना द्र॑व॒ वरि॑ष्ठा॒मनु॑ सं॒वतम्᳚। दि॒वि ते॒ जन्म॑ पर॒मम॒न्तरि॑क्षे॒ नाभिः॑ पृथि॒व्यामधि॒ योनिः॑। यु॒ञ्जाथा॒ꣳ॒ रास॑भं यु॒वम॒स्मिन् यामे॑ वृषण्वसू। अ॒ग्निम्भर॑न्तमस्म॒युम्। योगे॑योगे त॒वस्त॑रं॒ वाजे॑वाजे हवामहे। सखा॑य॒ इन्द्र॑मू॒तये᳚। प्र॒तूर्वन्न्॑~(५)

%4.1.2.2
एह्य॑व॒क्राम॒न्नश॑स्ती रु॒द्रस्य॒ गाण॑पत्यान्मयो॒भूरेहि॑। उ॒र्व॑न्तरि॑क्ष॒मन्वि॑हि स्व॒स्तिग॑व्यूति॒रभ॑यानि कृ॒ण्वन्न्। पू॒ष्णा स॒युजा॑ स॒ह। पृ॒थि॒व्याः स॒धस्था॑द॒ग्निम्पु॑री॒ष्य॑मङ्गिर॒स्वदच्छे᳚ह्य॒ग्निम्पु॑री॒ष्य॑मङ्गिर॒स्वदच्छे॑मो॒\-ऽग्निम्पु॑री॒ष्य॑मङ्गिर॒\-स्वद्भ॑रिष्यामो॒\-ऽग्निम्पु॑री॒ष्य॑मङ्गिर॒स्वद्भ॑रामः। अन्व॒ग्निरु॒षसा॒मग्र॑मख्य॒दन्वहा॑नि प्रथ॒मो जा॒तवे॑दाः। अनु॒ सूर्य॑स्य~(६)

%4.1.2.3
पु॒रु॒त्रा च॑ र॒श्मीननु॒ द्यावा॑पृथि॒वी आ त॑तान। आ॒गत्य॑ वा॒ज्यध्व॑नः॒ सर्वा॒ मृधो॒ वि धू॑नुते। अ॒ग्निꣳ स॒धस्थे॑ मह॒ति चक्षु॑षा॒ नि चि॑कीषते। आ॒क्रम्य॑ वाजिन्पृथि॒वीम॒ग्निमि॑च्छ रु॒चा त्वम्। भूम्या॑ वृ॒त्वाय॑ नो ब्रूहि॒ यतः॒ खना॑म॒ तं व॒यम्। द्यौस्ते॑ पृ॒ष्ठं पृ॑थि॒वी स॒धस्थ॑मा॒त्मान्तरि॑क्षꣳ समु॒द्रस्ते॒ योनिः॑। वि॒ख्याय॒ चक्षु॑षा॒ त्वम॒भि ति॑ष्ठ~(७)

%4.1.2.4
पृ॒त॒न्य॒तः। उत्क्रा॑म मह॒ते सौभ॑गाया॒स्मादा॒स्थाना᳚द्द्रविणो॒दा वा॑जिन्न्। व॒यꣴ स्या॑म सुम॒तौ पृ॑थि॒व्या अ॒ग्निं ख॑नि॒ष्यन्त॑ उ॒पस्थे॑ अस्याः। उद॑क्रमीद्द्रविणो॒दा वा॒ज्यर्वाकः॒ स लो॒कꣳ सुकृ॑तं पृथि॒व्याः। ततः॑ खनेम सु॒प्रती॑कम॒ग्निꣳ सुवो॒ रुहा॑णा॒ अधि॒ नाक॑ उत्त॒मे। अ॒पो दे॒वीरुप॑ सृज॒ मधु॑मतीरय॒क्ष्माय॑ प्र॒जाभ्यः॑। तासा॒ꣴ॒ स्थाना॒दुज्जि॑हता॒मोष॑धयः सुपिप्प॒लाः। जिघ॑र्मि~(८)

%4.1.2.5
अ॒ग्निम्मन॑सा घृ॒तेन॑ प्रति॒क्ष्यन्त॒म्भुव॑नानि॒ विश्वा᳚। पृ॒थुं ति॑र॒श्चा वय॑सा बृ॒हन्तं॒ व्यचि॑ष्ठ॒मन्नꣳ॑ रभ॒सं विदा॑नम्। आ त्वा॑ जिघर्मि॒ वच॑सा घृ॒तेना॑र॒क्षसा॒ मन॑सा॒ तज्जु॑षस्व। मर्य॑श्रीः स्पृह॒यद्व॑र्णो अ॒ग्निर्नाभि॒मृशे॑ त॒नुवा॒ जर्\mbox{}हृ॑षाणः। परि॒ वाज॑पतिः क॒विर॒ग्निर्\mbox{}ह॒व्या न्य॑क्रमीत्। दध॒द्रत्ना॑नि दा॒शुषे᳚। परि॑ त्वाऽग्ने॒ पुरं॑ व॒यं विप्रꣳ॑ सहस्य धीमहि। धृ॒षद्व॑र्णं दि॒वेदि॑वे भे॒त्तार॑म्भङ्गु॒॒राव॑तः। त्वम॑ग्ने॒ द्युभि॒स्त्वमा॑शुशु॒क्षणि॒स्त्वम॒द्भ्यस्त्वमश्म॑न॒स्परि॑। त्वं वने᳚भ्य॒स्त्वमोष॑धीभ्य॒स्त्वं नृ॒णां नृ॑पते जायसे॒ शुचिः॑॥~(९)

%4.1.3.0
{\anuvakamend[{प्र॒तूर्व॒न्थ्सूर्य॑स्य तिष्ठ॒ जिघ॑र्मि भे॒त्तारं॑ विꣳश॒तिश्च॑}]}%~(२)

%4.1.3.1
दे॒वस्य॑ त्वा सवि॒तुः प्र॑स॒वे᳚\-ऽश्विनो᳚र्बा॒हु\-भ्यां᳚ पू॒ष्णो हस्ता᳚भ्यां पृथि॒व्याः स॒धस्थे॒\-ऽग्निम्पु॑री॒ष्य॑मङ्गिर॒स्व\-त्ख॑नामि। ज्योति॑ष्मन्तं त्वाग्ने सु॒प्रती॑क॒मज॑स्रेण भा॒नुना॒ दीद्या॑नम्। शि॒वं प्र॒जाभ्यो\-ऽहिꣳ॑सन्तं पृथि॒व्याः स॒धस्थे॒\-ऽग्निं पु॑री॒ष्य॑मङ्गिर॒स्वत्ख॑नामि। अ॒पां पृ॒ष्ठम॑सि स॒प्रथा॑ उ॒र्व॑ग्निम्भ॑रि॒ष्यदप॑रावपिष्ठम्। वर्ध॑मानम्म॒ह आ च॒ पुष्क॑रं दि॒वो मात्र॑या वरि॒णा प्र॑थस्व। शर्म॑ च स्थः~(१०)

%4.1.3.2
वर्म॑ च स्थो॒ अच्छि॑द्रे बहु॒ले उ॒भे। व्यच॑स्वती॒ सं व॑साथाम्भ॒र्तम॒ग्निम्पु॑री॒ष्यम्᳚। सं व॑साथाꣳ सुव॒र्विदा॑ स॒मीची॒ उर॑सा॒ त्मना᳚। अ॒ग्निम॒न्तर्भ॑रि॒ष्यन्ती॒ ज्योति॑ष्मन्त॒मज॑स्र॒मित्। पु॒री॒ष्यो॑\-ऽसि वि॒श्वभ॑राः। अथ॑र्वा त्वा प्रथ॒मो निर॑मन्थदग्ने। त्वाम॑ग्ने॒ पुष्क॑रा॒दध्यथ॑र्वा॒ निर॑मन्थत। मू॒र्ध्नो विश्व॑स्य वा॒घतः॑। तमु॑ त्वा द॒ध्यङ्ङृषिः॑ पु॒त्र ई॑धे~(११)

%4.1.3.3
अथ॑र्वणः। वृ॒त्र॒हणं॑ पुरन्द॒रम्। तमु॑ त्वा पा॒थ्यो वृषा॒ समी॑धे दस्यु॒हन्त॑मम्। ध॒नं॒ज॒यꣳ रणे॑रणे। सीद॑ होतः॒ स्व उ॑ लो॒के चि॑कि॒त्वान्थ्सा॒दया॑ य॒ज्ञꣳ सु॑कृ॒तस्य॒ योनौ᳚। दे॒वा॒वीर्दे॒वान् ह॒विषा॑ यजा॒स्यग्ने॑ बृ॒हद्यज॑माने॒ वयो॑ धाः। नि होता॑ होतृ॒षद॑ने॒ विदा॑नस्त्वे॒षो दी॑दि॒वाꣳ अ॑सदथ्सु॒दक्षः॑। अद॑ब्धव्रतप्रमति॒र्वसि॑ष्ठः सहस्रम्भ॒रः शुचि॑जिह्वो अ॒ग्निः। सꣳ सी॑दस्व म॒हाꣳ अ॑सि॒ शोच॑स्व~(१२)

%4.1.3.4
दे॒व॒वीत॑मः। वि धू॒मम॑ग्ने अरु॒षम्मि॑येध्य सृ॒ज प्र॑शस्त दर्\mbox{}श॒तम्। जनि॑ष्वा॒ हि जेन्यो॒ अग्रे॒ अह्नाꣳ॑ हि॒तो हि॒तेष्व॑रु॒षो वने॑षु। दमे॑दमे स॒प्त रत्ना॒ दधा॑नो॒\-ऽग्निर्\mbox{}होता॒ नि ष॑सादा॒ यजी॑यान्॥~(१३)

%4.1.4.0
{\anuvakamend[{स्थ॒ ई॒धे॒ शोच॑स्व स॒प्तविꣳ॑शतिश्च}]}%~(३)

%4.1.4.1
सं ते॑ वा॒युर्मा॑त॒रिश्वा॑ दधातूत्ता॒नायै॒ हृद॑यं॒ यद्विलि॑ष्टम्। दे॒वानां॒ यश्चर॑ति प्रा॒णथे॑न॒ तस्मै॑ च देवि॒ वष॑डस्तु॒ तुभ्यम्᳚। सुजा॑तो॒ ज्योति॑षा स॒ह शर्म॒ वरू॑थ॒मास॑दः॒ सुवः॑। वासो॑ अग्ने वि॒श्वरू॑प॒ꣳ॒ सं व्य॑यस्व विभावसो। उदु॑ तिष्ठ स्वध्व॒रावा॑ नो दे॒व्या कृ॒पा। दृ॒शे च॑ भा॒सा बृ॑ह॒ता सु॑शु॒क्वनि॒राग्ने॑ याहि सुश॒स्तिभिः॑।~(१४)

%4.1.4.2
ऊ॒र्ध्व ऊ॒ षु ण॑ ऊ॒तये॒ तिष्ठा॑ दे॒वो न स॑वि॒ता। ऊ॒र्ध्वो वाज॑स्य॒ सनि॑ता॒ यद॒ञ्जिभि॑र्वा॒घद्भि॑र्वि॒ह्वया॑महे। स जा॒तो गर्भो॑ असि॒ रोद॑स्यो॒रग्ने॒ चारु॒र्विभृ॑त॒ ओष॑धीषु। चि॒त्रः शिशुः॒ परि॒ तमाꣴ॑स्य॒क्तः प्र मा॒तृभ्यो॒ अधि॒ कनि॑क्रदद्गाः। स्थि॒रो भ॑व वी॒ड्व॑ङ्ग आ॒शुर्भ॑व वा॒ज्य॑र्वन्न्। पृ॒थुर्भ॑व सु॒षद॒स्त्वम॒ग्नेः पु॑रीष॒वाह॑नः। शि॒वो भ॑व~(१५)

%4.1.4.3
प्र॒जाभ्यो॒ मानु॑षीभ्य॒स्त्वम॑ङ्गिरः। मा द्यावा॑पृथि॒वी अ॒भि शू॑शुचो॒ मान्तरि॑क्षं॒ मा वन॒स्पतीन्॑। प्रैतु॑ वा॒जी कनि॑क्रद॒न्नान॑द॒द्रास॑भः॒ पत्वा᳚। भर॑न्न॒ग्निम्पु॑री॒ष्यं॑ मा पा॒द्यायु॑षः पु॒रा। रास॑भो वां॒ कनि॑क्रद॒थ्सुयु॑क्तो वृषणा॒ रथे᳚। स वा॑म॒ग्निम्पु॑री॒ष्य॑मा॒शुर्दू॒तो व॑हादि॒तः। वृषा॒ग्निं वृ॑षण॒म्भर॑न्न॒पां गर्भꣳ॑ समु॒द्रियम्᳚। अग्न॒ आ या॑हि~(१६)

%4.1.4.4
वी॒तय॑ ऋ॒तꣳ स॒त्यम्। ओष॑धयः॒ प्रति॑ गृह्णीता॒ग्निमे॒तꣳ शि॒वमा॒यन्त॑म॒भ्यत्र॑ यु॒ष्मान्। व्यस्य॒न्विश्वा॒ अम॑ती॒ररा॑तीर्नि॒षीद॑न्नो॒ अप॑ दुर्म॒तिꣳ ह॑नत्। ओष॑धयः॒ प्रति॑ मोदध्वमेन॒म्पुष्पा॑वतीः सुपिप्प॒लाः। अ॒यं वो॒ गर्भ॑ ऋ॒त्वियः॑ प्र॒त्नꣳ स॒धस्थ॒मास॑दत्॥~(१७)

%4.1.5.0
{\anuvakamend[{सु॒श॒स्तिभिः॑ शि॒वो भ॑व याहि॒ षट्त्रिꣳ॑शच्च}]}%~(४)

%4.1.5.1
वि पाज॑सा पृ॒थुना॒ शोशु॑चानो॒ बाध॑स्व द्वि॒षो र॒क्षसो॒ अमी॑वाः। सु॒शर्म॑णो बृह॒तः शर्म॑णि स्याम॒ग्नेर॒हꣳ सु॒हव॑स्य॒ प्रणी॑तौ। आपो॒ हि ष्ठा म॑यो॒भुव॒स्ता न॑ ऊ॒र्जे द॑धातन। म॒हे रणा॑य॒ चक्ष॑से। यो वः॑ शि॒वत॑मो॒ रस॒स्तस्य॑ भाजयते॒ह नः॑। उ॒श॒तीरि॑व मा॒तरः॑। तस्मा॒ अरं॑ गमाम वो॒ यस्य॒ क्षया॑य॒ जिन्व॑थ। आपो॑ ज॒नय॑था च नः। मि॒त्रः~(१८)

%4.1.5.2
स॒ꣳ॒सृज्य॑ पृथि॒वीम्भूमिं॑ च॒ ज्योति॑षा स॒ह। सुजा॑तं जा॒तवे॑दसम॒ग्निं वै᳚श्वान॒रं वि॒भुम्। अ॒य॒क्ष्माय॑ त्वा॒ सꣳ सृ॑जामि प्र॒जाभ्यः॑। विश्वे᳚ त्वा दे॒वा वै᳚श्वान॒राः सꣳ सृ॑ज॒न्त्वानु॑ष्टुभेन॒ छन्द॑साङ्गिर॒स्वत्। रु॒द्राः स॒म्भृत्य॑ पृथि॒वीम्बृ॒हज्ज्योतिः॒ समी॑धिरे। तेषां᳚ भा॒नुरज॑स्र॒ इच्छु॒क्रो दे॒वेषु॑ रोचते। सꣳसृ॑ष्टां॒ वसु॑भी रु॒द्रैर्धीरैः᳚ कर्म॒ण्या᳚म्मृदम्᳚। हस्ता᳚भ्याम्मृ॒द्वीं कृ॒त्वा सि॑नीवा॒ली क॑रोतु~(१९)

%4.1.5.3
ताम्। सि॒नी॒वा॒ली सु॑कप॒र्दा सु॑कुरी॒रा स्वौ॑प॒शा। सा तुभ्य॑मदिते मह॒ ओखां द॑धातु॒ हस्त॑योः। उ॒खां क॑रोतु॒ शक्त्या॑ बा॒हुभ्या॒मदि॑तिर्धि॒या। मा॒ता पु॒त्रं यथो॒पस्थे॒ साग्निम्बि॑भर्तु॒ गर्भ॒ आ। म॒खस्य॒ शिरो॑\-ऽसि य॒ज्ञस्य॑ प॒दे स्थः॑। वस॑वस्त्वा कृण्वन्तु गाय॒त्रेण॒ छन्द॑साङ्गिर॒स्वत्पृ॑थि॒व्य॑सि रु॒द्रास्त्वा॑ कृण्वन्तु॒ त्रैष्टु॑भेन॒ छन्द॑साङ्गिर॒स्वद॒न्तरि॑क्षमसि~(२०)

%4.1.5.4
आ॒दि॒त्यास्त्वा॑ कृण्वन्तु॒ जाग॑तेन॒ छन्द॑साङ्गिर॒स्वद्द्यौर॑सि॒ विश्वे᳚ त्वा दे॒वा वै᳚श्वान॒राः कृ॑ण्व॒न्त्वानु॑ष्टुभेन॒ छन्द॑साङ्गिर॒स्वद्दिशो॑\-ऽसि ध्रु॒वासि॑ धा॒रया॒ मयि॑ प्र॒जाꣳ रा॒यस्पोषं॑ गौप॒त्यꣳ सु॒वीर्यꣳ॑ सजा॒तान् यज॑माना॒यादि॑त्यै॒ रास्ना॒स्यदि॑तिस्ते॒ बिलं॑ गृह्णातु॒ पाङ्क्ते॑न॒ छन्द॑साङ्गिर॒स्वत्। कृ॒त्वाय॒ सा म॒हीमु॒खाम्मृ॒न्मयीं॒ योनि॑म॒ग्नये᳚। ताम्पु॒त्रेभ्यः॒ सम्प्राय॑च्छ॒ददि॑तिः श्र॒पया॒निति॑॥~(२१)

%4.1.6.0
{\anuvakamend[{मि॒त्रः क॑रोत्व॒न्तरि॑क्षमसि॒ प्र च॒त्वारि॑ च}]}%~(५)

%4.1.6.1
वस॑वस्त्वा धूपयन्तु गाय॒त्रेण॒ छन्द॑साङ्गिर॒स्वद्रु॒द्रास्त्वा॑ धूपयन्तु॒ त्रैष्टु॑भेन॒ छन्द॑साङ्गिर॒स्वदा॑दि॒त्यास्त्वा॑ धूपयन्तु॒ जाग॑तेन॒ छन्द॑साङ्गिर॒स्वद्विश्वे᳚ त्वा दे॒वा वै᳚श्वान॒रा धू॑पय॒न्त्वानु॑ष्टुभेन॒ छन्द॑साङ्गिर॒स्वदिन्द्र॑स्त्वा धूपयत्वङ्गिर॒स्वद्विष्णु॑स्त्वा धूपयत्वङ्गिर॒स्वद्वरु॑णस्त्वा धूपयत्वङ्गिर॒स्वददि॑तिस्त्वा दे॒वी वि॒श्वदे᳚व्यावती पृथि॒व्याः स॒धस्थे᳚\-ऽङ्गिर॒स्वत्ख॑नत्ववट दे॒वानां᳚ त्वा॒ पत्नीः᳚~(२२)

%4.1.6.2
दे॒वीर्वि॒श्वदे᳚व्यावतीः पृथि॒व्याः स॒धस्थे᳚\-ऽङ्गिर॒स्वद्द॑धतूखे धि॒षणा᳚स्त्वा दे॒वीर्वि॒श्वदे᳚व्यावतीः पृथि॒व्याः स॒धस्थे᳚\-ऽङ्गिर॒स्वद॒भीन्ध॑तामुखे॒ ग्नास्त्वा॑ दे॒वीर्वि॒श्वदे᳚व्यावतीः पृथि॒व्याः स॒धस्थे᳚\-ऽङ्गिर॒स्वच्छ्र॑पयन्तूखे॒ वरू᳚त्रयो॒ जन॑यस्त्वा दे॒वीर्वि॒श्वदे᳚व्यावतीः पृथि॒व्याः स॒धस्थे᳚\-ऽङ्गिर॒स्वत्प॑चन्तूखे। मित्रै॒तामु॒खाम्प॑चै॒षा मा भे॑दि। एातां ते॒ परि॑ ददा॒म्यभि॑त्त्यै। अ॒भीमाम्~(२३)

%4.1.6.3
म॒हि॒ना दिव॑म्मि॒त्रो ब॑भूव स॒प्रथाः᳚। उ॒त श्रव॑सा पृथि॒वीम्। मि॒त्रस्य॑ चर्\mbox{}षणी॒धृतः॒ श्रवो॑ दे॒वस्य॑ सान॒सिम्। द्यु॒म्नं चि॒त्रश्र॑वस्तमम्। दे॒वस्त्वा॑ सवि॒तोद्व॑पतु सुपा॒णिः स्व॑ङ्गु॒॒रिः। सु॒बा॒हुरु॒त शक्त्या᳚। अप॑द्यमाना पृथि॒व्याशा॒ दिश॒ आ पृ॑ण। उत्ति॑ष्ठ बृह॒ती भ॑वो॒र्ध्वा ति॑ष्ठ ध्रु॒वा त्वम्। वस॑व॒स्त्वाच्छृ॑न्दन्तु गाय॒त्रेण॒ छन्द॑साङ्गिर॒स्वद्रु॒द्रास्त्वा च्छृ॑न्दन्तु॒ त्रैष्टु॑भेन॒ छन्द॑साङ्गिर॒स्वदा॑दि॒त्यास्त्वाच्छृ॑न्दन्तु॒ जाग॑तेन॒ छन्द॑साङ्गिर॒स्वद्विश्वे᳚ त्वा दे॒वा वै᳚श्वान॒रा आच्छृ॑न्द॒न्त्वानु॑ष्टुभेन॒ छन्द॑साङ्गिर॒स्वत्॥~(२४)

%4.1.7.0
{\anuvakamend[{पत्नी॑रि॒माꣳ रु॒द्रास्त्वाच्छृ॑न्द॒न्त्वेका॒न्नविꣳ॑श॒तिश्च॑}]}%~(६)

%4.1.7.1
समा᳚स्त्वाग्न ऋ॒तवो॑ वर्धयन्तु संवथ्स॒रा ऋष॑यो॒ यानि॑ स॒त्या। सं दि॒व्येन॑ दीदिहि रोच॒नेन॒ विश्वा॒ आ भा॑हि प्र॒दिशः॑ पृथि॒व्याः। सं चे॒ध्यस्वा᳚ग्ने॒ प्र च॑ बोधयैन॒मुच्च॑ तिष्ठ मह॒ते सौभ॑गाय। मा च॑ रिषदुपस॒त्ता ते॑ अग्ने ब्र॒ह्माण॑स्ते य॒शसः॑ सन्तु॒ मान्ये। त्वाम॑ग्ने वृणते ब्राह्म॒णा इ॒मे शि॒वो अ॑ग्ने~(२५)

%4.1.7.2
सं॒वर॑णे भवा नः। स॒प॒त्न॒हा नो॑ अभिमाति॒जिच्च॒ स्वे गये॑ जागृ॒ह्यप्र॑युच्छन्न्। इ॒हैवाग्ने॒ अधि॑ धारया र॒यिं मा त्वा॒ नि क्र॑न्पूर्व॒चितो॑ निका॒रिणः॑। क्ष॒त्रम॑ग्ने सु॒यम॑मस्तु॒ तुभ्य॑मुपस॒त्ता व॑र्धतां ते॒ अनि॑ष्टृतः। क्ष॒त्रेणा᳚ग्ने॒ स्वायुः॒ सꣳ र॑भस्व मि॒त्रेणा᳚ग्ने मित्र॒धेये॑ यतस्व। स॒जा॒ताना᳚म्मध्यम॒स्था ए॑धि॒ राज्ञा॑मग्ने विह॒व्यो॑ दीदिही॒ह। अति॑~(२६)

%4.1.7.3
निहो॒ अति॒ स्रिधो\-ऽत्यचि॑त्ति॒मत्यरा॑तिमग्ने। विश्वा॒ ह्य॑ग्ने दुरि॒ता सह॒स्वाथा॒स्मभ्यꣳ॑ स॒हवी॑राꣳ र॒यिं दाः᳚। अ॒ना॒धृ॒ष्यो जा॒तवे॑दा॒ अनि॑ष्टृतो वि॒राड॑ग्ने क्षत्र॒भृद्दी॑दिही॒ह। विश्वा॒ आशाः᳚ प्रमु॒ञ्चन्मानु॑षीर्भि॒यः शि॒वाभि॑र॒द्य परि॑ पाहि नो वृ॒धे। बृह॑स्पते सवितर्बो॒धयै॑न॒ꣳ॒ सꣳशि॑तं चिथ्सन्त॒राꣳ सꣳ शि॑शाधि। व॒र्धयै॑नम्मह॒ते सौभ॑गाय~(२७)

%4.1.7.4
विश्व॑ एन॒मनु॑ मदन्तु दे॒वाः। अ॒मु॒त्र॒भूया॒दध॒ यद्य॒मस्य॒ बृह॑स्पते अ॒भिश॑स्ते॒रमु॑ञ्चः। प्रत्यौ॑हताम॒श्विना॑ मृ॒त्युम॑स्माद्दे॒वाना॑मग्ने भि॒षजा॒ शची॑भिः। उद्व॒यं तम॑स॒स्परि॒ पश्य॑न्तो॒ ज्योति॒रुत्त॑रम्। दे॒वं दे॑व॒त्रा सूर्य॒मग॑न्म॒ ज्योति॑रुत्त॒मम्॥~(२८)

%4.1.8.0
{\anuvakamend[{इ॒मे शि॒वो अ॒ग्ने\-ऽति॒ सौभ॑गाय॒ चतु॑स्त्रिꣳशच्च}]}%~(७)

%4.1.8.1
ऊ॒र्ध्वा अ॑स्य स॒मिधो॑ भवन्त्यू॒र्ध्वा शु॒क्रा शो॒चीꣳष्य॒ग्नेः। द्यु॒मत्त॑मा सु॒प्रती॑कस्य सू॒नोः। तनू॒नपा॒दसु॑रो वि॒श्ववे॑दा दे॒वो दे॒वेषु॑ दे॒वः। प॒थ आन॑क्ति॒ मध्वा॑ घृ॒तेन॑। मध्वा॑ य॒ज्ञं न॑क्षसे प्रीणा॒नो नरा॒शꣳसो॑ अग्ने। सु॒कृद्दे॒वः स॑वि॒ता वि॒श्ववा॑रः। अच्छा॒यमे॑ति॒ शव॑सा घृ॒तेने॑डा॒नो वह्नि॒र्नम॑सा। अ॒ग्निꣴ स्रुचो॑ अध्व॒रेषु॑ प्र॒यथ्सु॑। स य॑क्षदस्य महि॒मान॑म॒ग्नेः सः~(२९)

%4.1.8.2
ई॒ म॒न्द्रासु॑ प्र॒यसः॑। वसु॒श्चेति॑ष्ठो वसु॒धात॑मश्च। द्वारो॑ दे॒वीरन्व॑स्य॒ विश्वे᳚ व्र॒ता द॑दन्ते अ॒ग्नेः। उ॒रु॒व्यच॑सो॒ धाम्ना॒ पत्य॑मानाः। ते अ॑स्य॒ योष॑णे दि॒व्ये न योना॑वु॒षासा॒नक्ता᳚। इ॒मं य॒ज्ञम॑वतामध्व॒रं नः॑। दैव्या॑ होतारावू॒र्ध्वम॑ध्व॒रं नो॒\-ऽग्नेर्जि॒ह्वाम॒भि गृ॑णीतम्। कृ॒णु॒तं नः॒ स्वि॑ष्टिम्। ति॒स्रो दे॒वीर्ब॒र्हिरेदꣳ स॑द॒न्त्विडा॒ सर॑स्वती~(३०)

%4.1.8.3
भार॑ती। म॒ही गृ॑णा॒ना। तन्न॑स्तु॒रीप॒मद्भु॑तं पुरु॒क्षु त्वष्टा॑ सु॒वीरम्᳚। रा॒यस्पोषं॒ वि ष्य॑तु॒ नाभि॑म॒स्मे। वन॑स्प॒ते\-ऽव॑ सृजा॒ ररा॑ण॒स्त्मना॑ दे॒वेषु॑। अ॒ग्निर्\mbox{}ह॒व्यꣳ श॑मि॒ता सू॑दयाति। अग्ने॒ स्वाहा॑ कृणुहि जातवेद॒ इन्द्रा॑य ह॒व्यम्। विश्वे॑ दे॒वा ह॒विरि॒दं जु॑षन्ताम्। हि॒र॒ण्य॒ग॒र्भः सम॑वर्त॒ताग्रे॑ भू॒तस्य॑ जा॒तः पति॒रेक॑ आसीत्। स दा॑धार पृथि॒वीं द्याम्~(३१)

%4.1.8.4
उ॒तेमां कस्मै॑ दे॒वाय॑ ह॒विषा॑ विधेम। यः प्रा॑ण॒तो नि॑मिष॒तो म॑हि॒त्वैक॒ इद्राजा॒ जग॑तो ब॒भूव॑। य ईशे॑ अ॒स्य द्वि॒पद॒श्चतु॑ष्पदः॒ कस्मै॑ दे॒वाय॑ ह॒विषा॑ विधेम। य आ᳚त्म॒दा ब॑ल॒दा यस्य॒ विश्व॑ उ॒पास॑ते प्र॒शिषं॒ यस्य॑ दे॒वाः। यस्य॑ छा॒यामृतं॒ यस्य॑ मृ॒त्युः कस्मै॑ दे॒वाय॑ ह॒विषा॑ विधेम। यस्ये॒मे हि॒मव॑न्तो महि॒त्वा यस्य॑ समु॒द्रꣳ र॒सया॑ स॒ह~(३२)

%4.1.8.5
आ॒हुः। यस्ये॒माः प्र॒दिशो॒ यस्य॑ बा॒हू कस्मै॑ दे॒वाय॑ ह॒विषा॑ विधेम। यं क्रन्द॑सी॒ अव॑सा तस्तभा॒ने अ॒भ्यैक्षे॑ता॒म्मन॑सा॒ रेज॑माने। यत्राधि॒ सूर॒ उदि॑तौ॒ व्येति॒ कस्मै॑ दे॒वाय॑ ह॒विषा॑ विधेम। येन॒ द्यौरु॒ग्रा पृ॑थि॒वी च॑ दृ॒ढे येन॒ सुवः॑ स्तभि॒तं येन॒ नाकः॑। यो अ॒न्तरि॑क्षे॒ रज॑सो वि॒मानः॒ कस्मै॑ दे॒वाय॑ ह॒विषा॑ विधेम। आपो॑ ह॒ यन्म॑ह॒तीर्विश्वम्᳚~(३३)

%4.1.8.6
आय॒न्दक्षं॒ दधा॑ना ज॒नय॑न्तीर॒ग्निम्। ततो॑ दे॒वानां॒ निर॑वर्त॒तासु॒रेकः॒ कस्मै॑ दे॒वाय॑ ह॒विषा॑ विधेम। यश्चि॒दापो॑ महि॒ना प॒र्यप॑श्य॒द्दक्षं॒ दधा॑ना ज॒नय॑न्तीर॒ग्निम्। यो दे॒वेष्वधि॑ दे॒व एक॒ आसी॒त्कस्मै॑ दे॒वाय॑ ह॒विषा॑ विधेम॥~(३४)

%4.1.9.0
{\anuvakamend[{अ॒ग्नेः स सर॑स्वती॒ द्याꣳ स॒ह विश्व॒ञ्चतु॑स्त्रिHꣳशश्च}]}%~(८)

%4.1.9.1
आकू॑तिम॒ग्निम्प्र॒युज॒ꣴ॒ स्वाहा॒ मनो॑ मे॒धाम॒ग्निम्प्र॒युज॒ꣴ॒ स्वाहा॑ चि॒त्तं विज्ञा॑तम॒ग्निम्प्र॒युज॒ꣴ॒ स्वाहा॑ वा॒चो विधृ॑तिम॒ग्निम्प्र॒युज॒ꣴ॒ स्वाहा᳚ प्र॒जाप॑तये॒ मन॑वे॒ स्वाहा॒ग्नये॑ वैश्वान॒राय॒ स्वाहा॒ विश्वे॑ दे॒वस्य॑ ने॒तुर्मर्तो॑ वृणीत स॒ख्यं विश्वे॑ रा॒य इ॑षुध्यसि द्यु॒म्नं वृ॑णीत पु॒ष्यसे॒ स्वाहा॒ मा सु भि॑त्था॒ मा सु रि॑षो॒ दृꣳह॑स्व वी॒डय॑स्व॒ सु। अम्ब॑ धृष्णु वी॒रय॑स्व~(३५)

%4.1.9.2
अ॒ग्निश्चे॒दं क॑रिष्यथः। दृꣳह॑स्व देवि पृथिवि स्व॒स्तय॑ आसु॒री मा॒या स्व॒धया॑ कृ॒तासि॑। जुष्टं॑ दे॒वाना॑मि॒दम॑स्तु ह॒व्यमरि॑ष्टा॒ त्वमुदि॑हि य॒ज्ञे अ॒स्मिन्न्। मित्रै॒तामु॒खां त॑पै॒षा मा भे॑दि। ए॒तान्ते॒ परि॑ ददा॒म्यभि॑त्त्यै। द्र्व॑न्नः स॒र्पिरा॑सुतिः प्र॒त्नो होता॒ वरे᳚ण्यः। सह॑सस्पु॒त्रो अद्भु॑तः। पर॑स्या॒ अधि॑ सं॒वतो\-ऽव॑राꣳ अभ्या~(३६)

%4.1.9.3
त॒र॒। यत्रा॒हमस्मि॒ ताꣳ अ॑व। प॒र॒मस्याः᳚ परा॒वतो॑ रो॒हिद॑श्व इ॒हा ग॑हि। पु॒री॒ष्यः॑ पुरुप्रि॒यो\-ऽग्ने॒ त्वं त॑रा॒ मृधः॑। सीद॒ त्वं मा॒तुर॒स्या उ॒पस्थे॒ विश्वा᳚न्यग्ने व॒युना॑नि वि॒द्वान्। मैना॑म॒र्चिषा॒ मा तप॑सा॒भि शू॑शुचो॒\-ऽन्तर॑स्याꣳ शु॒क्रज्यो॑ति॒र्वि भा॑हि। अ॒न्तर॑ग्ने रु॒चा त्वमु॒खायै॒ सद॑ने॒ स्वे। तस्या॒स्त्वꣳ हर॑सा॒ तप॒ञ्जात॑वेदः शि॒वो भ॑व। शि॒वो भू॒त्वा मह्य॑म॒ग्ने\-ऽथो॑ सीद शि॒वस्त्वम्। शि॒वाः कृ॒त्वा दिशः॒ सर्वाः॒ स्वां योनि॑मि॒हास॑दः॥~(३७)

%4.1.10.0
{\anuvakamend[{वी॒रय॒स्वा तप॑न्विꣳश॒तिश्च॑}]}%~(९)

%4.1.10.1
यद॑ग्ने॒ यानि॒ कानि॒ चा ते॒ दारू॑णि द॒ध्मसि॑। तद॑स्तु॒ तुभ्य॒मिद्घृ॒तं तज्जु॑षस्व यविष्ठ्य। यदत्त्यु॑प॒जिह्वि॑का॒ यद्व॒म्रो अ॑ति॒सर्प॑ति। सर्वं॒ तद॑स्तु ते घृ॒तं तज्जु॑षस्व यविष्ठ्य। रात्रि॑ꣳरात्रि॒मप्र॑याव॒म्भर॒न्तो\-ऽश्वा॑येव॒ तिष्ठ॑ते घा॒समस्मै। रा॒यस्पोषे॑ण॒ समि॒षा मद॒न्तो\-ऽग्ने॒ मा ते॒ प्रति॑वेशा रिषाम। नाभा᳚~(३८)

%4.1.10.2
पृ॒थि॒व्याः स॑मिधा॒नम॒ग्निꣳ रा॒यस्पोषा॑य बृह॒ते ह॑वामहे। इ॒र॒म्म॒दम्बृ॒हदु॑क्थं॒ यज॑त्रं॒ जेता॑रम॒ग्निं पृ॑तनासु सास॒हिम्। याः सेना॑ अ॒भीत्व॑रीराव्या॒धिनी॒रुग॑णा उ॒त। ये स्ते॒ना ये च॒ तस्क॑रा॒स्ताꣴस्ते॑ अ॒ग्ने\-ऽपि॑ दधाम्या॒स्ये᳚। दꣴष्ट्रा᳚भ्याम्म॒लिम्लू॒ञ्जम्भ्यै॒स्तस्क॑राꣳ उ॒त। हनू᳚भ्याꣴस्ते॒नान्भ॑गव॒स्ताꣴस्त्वं खा॑द॒ सुखा॑दितान्। ये जने॑षु म॒लिम्ल॑वः स्ते॒नास॒स्तस्क॑रा॒ वने᳚। ये~(३९)

%4.1.10.3
कक्षे᳚ष्वघा॒यव॒स्ताꣴस्ते॑ दधामि॒ जम्भ॑योः। यो अ॒स्मभ्य॑मराती॒याद्यश्च॑ नो॒ द्वेष॑ते॒ जनः॑। निन्दा॒द्यो अ॒स्मान् दिफ्सा᳚च्च॒ सर्वं॒ तम्म॑स्म॒सा कु॑रु। सꣳशि॑तं मे॒ ब्रह्म॒ सꣳशि॑तं वीर्यं॑ बलम्᳚। सꣳशि॑तं क्ष॒त्रं जि॒ष्णु यस्या॒हमस्मि॑ पु॒रोहि॑तः। उदे॑षाम्बा॒हू अ॑तिर॒मुद्वर्च॒ उदू॒ बलम्᳚। क्षि॒णोमि॒ ब्रह्म॑णा॒मित्रा॒नुन्न॑यामि~(४०)

%4.1.10.4
स्वाꣳ अ॒हम्। दृ॒शा॒नो रु॒क्म उ॒र्व्या व्य॑द्यौद्दु॒र्मर्\mbox{}ष॒मायुः॑ श्रि॒ये रु॑चा॒नः। अ॒ग्निर॒मृतो॑ अभव॒द्वयो॑भि॒र्यदे॑नं॒ द्यौरज॑नयथ्सु॒रेताः᳚। विश्वा॑ रू॒पाणि॒ प्रति॑ मुञ्चते क॒विः प्रासा॑वीद्भ॒द्रं द्वि॒पदे॒ चतु॑ष्पदे। वि नाक॑मख्यथ्सवि॒ता वरे॒ण्यो\-ऽनु॑ प्र॒याण॑मु॒षसो॒ वि रा॑जति। नक्तो॒षासा॒ सम॑नसा॒ विरू॑पे धा॒पये॑ते॒ शिशु॒मेकꣳ॑ समी॒ची। द्यावा॒ क्षामा॑ रु॒क्मः~(४१)

%4.1.10.5
अ॒न्तर्वि भा॑ति दे॒वा अ॒ग्निं धा॑रयन्द्रविणो॒दाः। सु॒प॒र्णो\-ऽसि ग॒रुत्मा᳚न्त्रि॒वृत्ते॒ शिरो॑ गाय॒त्रं चक्षुः॒ स्तोम॑ आ॒त्मा साम॑ ते त॒नूर्वा॑मदे॒व्यम्बृ॑हद्रथन्त॒रे प॒क्षौ य॑ज्ञाय॒ज्ञिय॒म्पुच्छं॒ छन्दा॒ꣴ॒स्यङ्गा॑नि॒ धिष्णि॑याः श॒फा॒ यजूꣳ॑षि॒ नाम॑। सु॒प॒र्णो॑\-ऽसि ग॒रुत्मा॒न्दिवं॑ गच्छ॒ सुवः॑ पत॥~(४२)

%4.1.11.0
{\anuvakamend[{नाभा॒ वने॒ येन॑ यामि॒ क्षामा॑ रु॒क्मो᳚\-ऽष्टात्रिꣳ॑शच्च}]}%॥10॥

%4.1.11.1
अग्ने॒ यं य॒ज्ञम॑ध्व॒रं वि॒श्वतः॑ परि॒भूरसि॑। स इद्दे॒वेषु॑ गच्छति। सोम॒ यास्ते॑ मयो॒भुव॑ ऊ॒तयः॒ सन्ति॑ दा॒शुषे᳚। ताभि॑र्नो\-ऽवि॒ता भ॑व। अ॒ग्निर्मू॒र्धा भुवः॑। त्वं नः॑ सोम॒ या ते॒ धामा॑नि। तथ्स॑वि॒तुर्वरे᳚ण्य॒म्भर्गो॑ दे॒वस्य॑ धीमहि। धियो॒ यो नः॑ प्रचो॒दया᳚त्। अचि॑त्ती॒ यच्च॑कृ॒मा दैव्ये॒ जने॑ दी॒नैर्दक्षैः॒ प्रभू॑ती पूरुष॒त्वता᳚।~(४३)

%4.1.11.2
दे॒वेषु॑ च सवित॒र्मानु॑षेषु च॒ त्वं नो॒ अत्र॑ सुवता॒दना॑गसः। चो॒द॒यि॒त्री सू॒नृता॑नां॒ चेत॑न्ती सुमती॒नाम्। य॒ज्ञं द॑धे॒ सर॑स्वती। पावी॑रवी क॒न्या॑ चि॒त्रायुः॒ सर॑स्वती वी॒रप॑त्नी॒ धियं॑ धात्। ग्नाभि॒रच्छि॑द्रꣳ शर॒णꣳ स॒जोषा॑ दुरा॒धर्\mbox{}षं॑ गृण॒ते शर्म॑ यꣳसत्। पू॒षा गा अन्वे॑तु नः पू॒षा र॑क्ष॒त्वर्व॑तः। पू॒षा वाजꣳ॑ सनोतु नः। शु॒क्रं ते॑ अ॒न्यद्य॑ज॒तं ते॑ अ॒न्यत्~(४४)

%4.1.11.3
विषु॑रूपे॒ अह॑नी॒ द्यौरि॑वासि। विश्वा॒ हि मा॒या अव॑सि स्वधावो भ॒द्रा ते॑ पूषन्नि॒ह रा॒तिर॑स्तु। ते॑\-ऽवर्धन्त॒ स्वत॑वसो महित्व॒ना नाकं॑ त॒स्थुरु॒रु च॑क्रिरे॒ सदः॑। विष्णु॒र्यद्धाव॒द्वृष॑णम्मद॒च्युतं॒ वयो॒ न सी॑द॒न्नधि॑ ब॒र्\mbox{}हिषि॑ प्रि॒ये। प्र चि॒त्रम॒र्कं गृ॑ण॒ते तु॒राय॒ मारु॑ताय॒ स्वत॑वसे भरध्वम्। ये सहाꣳ॑सि॒ सह॑सा॒ सह॑न्ते~(४५)

%4.1.11.4
रेज॑ते अग्ने पृथि॒वी म॒खेभ्यः॑। विश्वे॑ दे॒वा विश्वे॑ देवाः। द्यावा॑ नः पृथि॒वी इ॒मꣳ सि॒ध्रम॒द्य दि॑वि॒स्पृशम्᳚। य॒ज्ञं दे॒वेषु॑ यच्छताम्। प्र पू᳚र्व॒जे पि॒तरा॒ नव्य॑सीभिर्गी॒र्भिः कृ॑णुध्व॒ꣳ॒ सद॑ने ऋ॒तस्य॑। आ नो᳚ द्यावापृथिवी॒ दैव्ये॑न॒ जने॑न यात॒म्महि॑ वां॒ वरू॑थम्। अ॒ग्निꣴ स्तोमे॑न बोधय समिधा॒नो अम॑र्त्यम्। ह॒व्या दे॒वेषु॑ नो दधत्। स ह॑व्य॒वाडम॑र्त्य उ॒शिग्दू॒तश्चनो॑हितः। अ॒ग्निर्धि॒या समृ॑ण्वति। शं नो॑ भवन्तु॒ वाजे॑वाजे॥~(४६)

%4.2.0.0
{\anuvakamend[{पू॒रु॒ष॒त्वता॑ यज॒तन्ते॑ अ॒न्यथ्सह॑न्ते॒ चनो॑हितो॒\-ऽष्टौ च॑}]}%॥11॥

%4.2.0.0

{\anuvakamend[{विष्णोः॒ क्रमो॑\-ऽसि दि॒वस्पर्यन्न॑प॒ते\-ऽपे॑त॒ समि॑तं॒ या जा॒ता मा नो॑ हिꣳसीद्ध्रु॒वा\-ऽस्या॑दि॒त्यङ्गर्भ॒मिन्द्रा᳚ग्नी रोच॒नैका॑\-दश}]}%॥11॥

\prashnaend{विष्णो॑रस्मिन् ह॒व्येति॑ त्वा॒\-ऽहं धी॒तिभि॒र्\mbox{}होत्रा॑ अ॒ष्टाच॑त्वारिꣳशत्॥48॥ विष्णोः॒ क्रमो॑\-ऽसि॒ स त्वन्नो॑ अग्ने॥}
%%% END PRASHNA

\sect{द्वितीयः प्रश्नः}\setcounter{anuvakam}{0}
\dnsub{तैत्तिरीयसंहितायां चतुर्थकाण्डे द्वितीयः प्रश्नः}
%4.2.1.0
%4.2.1.1
विष्णोः॒ क्रमो᳚\-ऽस्यभिमाति॒हा गा॑य॒त्रं छन्द॒ आ रो॑ह पृथि॒वीमनु॒ वि क्र॑मस्व॒ निर्भ॑क्तः॒ स यं द्वि॒ष्मो विष्णोः॒ क्रमो᳚\-ऽस्यभिशस्ति॒हा त्रैष्टु॑भं॒ छन्द॒ आ रो॑हा॒न्तरि॑क्ष॒मनु॒ वि क्र॑मस्व॒ निर्भ॑क्तः॒ स यं द्वि॒ष्मो विष्णोः॒ क्रमो᳚\-ऽ स्यरातीय॒तो ह॒न्ता जाग॑तं॒ छन्द॒ आ रो॑ह॒ दिव॒मनु॒ वि क्र॑मस्व॒ निर्भ॑क्तः॒ स यं द्वि॒ष्मो विष्णोः᳚~(१)

%4.2.1.2
क्रमो॑\-ऽसि शत्रूय॒तो ह॒न्तानु॑ष्टुभं॒ छन्द॒ आ रो॑ह॒ दिशो\-ऽनु॒ वि क्र॑मस्व॒ निर्भ॑क्तः॒ स यं द्वि॒ष्मः। अक्र॑न्दद॒ग्निः स्त॒नय॑न्निव॒ द्यौः क्षामा॒ रेरि॑हद्वी॒रुधः॑ सम॒ञ्जन्न्। स॒द्यो ज॑ज्ञा॒नो वि हीमि॒द्धो अख्य॒दा रोद॑सी भा॒नुना॑ भात्य॒न्तः। अग्ने᳚\-ऽभ्यावर्तिन्न॒भि न॒ आ व॑र्त॒स्वायु॑षा॒ वर्च॑सा स॒न्या मे॒धया᳚ प्र॒जया॒ धने॑न। अग्ने᳚~(२)

%4.2.1.3
अ॒ङ्गि॒रः॒ श॒तं ते॑ सन्त्वा॒वृतः॑ स॒हस्रं॑ त उपा॒वृतः॑। तासा॒म्पोष॑स्य॒ पोषे॑ण॒ पुन॑र्नो न॒ष्टमा कृ॑धि॒ पुन॑र्नो र॒यिमा कृ॑धि। पुन॑रू॒र्जा नि व॑र्तस्व॒ पुन॑रग्न इ॒षायु॑षा। पुन॑र्नः पाहि वि॒श्वतः॑। स॒ह र॒य्या नि व॑र्त॒स्वाग्ने॒ पिन्व॑स्व॒ धार॑या। वि॒श्वफ्स्नि॑या वि॒श्वत॒स्परि॑। उदु॑त्त॒मं व॑रुण॒ पाश॑म॒स्मदवा॑ध॒मम्~(३)

%4.2.1.4
वि म॑ध्य॒मꣴ श्र॑थाय। अथा॑ व॒यमा॑दित्य व्र॒ते तवाना॑गसो॒ अदि॑तये स्याम। आ त्वा॑हार्\mbox{}षम॒न्तर॑भूर्ध्रु॒वस्ति॒ष्ठा\-वि॑चाचलिः। विश॑स्त्वा॒ सर्वा॑ वाञ्छन्त्व॒स्मिन्रा॒ष्ट्रमधि॑ श्रय। अग्रे॑ बृ॒हन्नु॒षसा॑मू॒र्ध्वो अ॑स्थान्निर्जग्मि॒वान्तम॑सो॒ ज्योति॒षागा᳚त्। अ॒ग्निर्भा॒नुना॒ रुश॑ता॒ स्वङ्ग॒ आ जा॒तो विश्वा॒ सद्मा᳚न्यप्राः। सीद॒ त्वं मा॒तुर॒स्याः~(४)

%4.2.1.5
उ॒पस्थे॒ विश्वा᳚न्यग्ने व॒युना॑नि वि॒द्वान्। मैना॑म॒र्चिषा॒ मा तप॑सा॒भि शू॑शुचो॒\-ऽन्तर॑स्याꣳ शु॒क्रज्यो॑ति॒र्वि भा॑हि। अ॒न्तर॑ग्ने रु॒चा त्वमु॒खायै॒ सद॑ने॒ स्वे। तस्या॒स्त्वꣳ हर॑सा॒ तप॒ञ्जात॑वेदः शि॒वो भ॑व। शि॒वो भू॒त्वा मह्य॑म॒ग्ने\-ऽथो॑ सीद शि॒वस्त्वम्। शि॒वाः कृ॒त्वा दिशः॒ सर्वाः॒ स्वं योनि॑मि॒हास॑दः। ह॒ꣳ॒सः शु॑चि॒षद्वसु॑रन्तरिक्ष॒सद्धोता॑ वेदि॒षदति॑थिर्दुरोण॒सत्। नृ॒षद्व॑र॒सदृ॑त॒सद्व्यो॑म॒सद॒ब्जा गो॒जा ऋ॑त॒जा अ॑द्रि॒जा ऋ॒तम्बृ॒हत्॥~(५)

%4.2.2.0
{\anuvakamend[{दिव॒मनु॒ वि क्र॑मस्व॒ निर्भ॑क्तः॒ स यं द्वि॒ष्मो विष्णो॒र्धने॒नाग्ने॑\-ऽध॒मम॒स्याः शु॑चि॒षथ्षोड॑श च}]}%~(१)

%4.2.2.1
दि॒वस्परि॑ प्रथ॒मं ज॑ज्ञे अ॒ग्निर॒स्मद्द्वि॒तीयं॒ परि॑ जा॒तवे॑दाः। तृ॒तीय॑म॒फ्सु नृ॒मणा॒ अज॑स्र॒मिन्धा॑न एनं जरते स्वा॒धीः। वि॒द्मा ते॑ अग्ने त्रे॒धा त्र॒याणि॑ वि॒द्मा ते॒ सद्म॒ विभृ॑तं पुरु॒त्रा। वि॒द्मा ते॒ नाम॑ पर॒मं गुहा॒ यद्वि॒द्मा तमुथ्सं॒ यत॑ आज॒गन्थ॑। स॒मु॒द्रे त्वा॑ नृ॒मणा॑ अ॒फ्स्व॑न्तर्नृ॒चक्षा॑ ईधे दि॒वो अ॑ग्न॒ ऊधन्न्॑। तृ॒तीये᳚ त्वा~(६)

%4.2.2.2
रज॑सि तस्थि॒वाꣳस॑मृ॒तस्य॒ योनौ॑ महि॒षा अ॑हिन्वन्न्। अक्र॑न्दद॒ग्निः स्त॒नय॑न्निव॒ द्यौः क्षामा॒ रेरि॑हद्वी॒रुधः॑ सम॒ञ्जन्न्। स॒द्यो ज॑ज्ञा॒नो वि हीमि॒द्धो अख्य॒दा रोद॑सी भा॒नुना॑ भात्य॒न्तः। उ॒शिक्पा॑व॒को अ॑र॒तिः सु॑मे॒धा मर्ते᳚ष्व॒ग्निर॒मृतो॒ निधा॑यि। इय॑र्ति धू॒मम॑रु॒षम्भरि॑भ्र॒दुच्छु॒क्रेण॑ शो॒चिषा॒ द्यामिन॑क्षत्। विश्व॑स्य के॒तुर्भुव॑नस्य॒ गर्भ॒ आ~(७)

%4.2.2.3
रोद॑सी अपृणा॒ज्जाय॑मानः। वी॒डुं चि॒दद्रि॑मभिनत्परा॒यञ्जना॒ यद॒ग्निमय॑जन्त॒ पञ्च॑। श्री॒णामु॑दा॒रो ध॒रुणो॑ रयी॒णाम्म॑नी॒षाणा॒म्प्रार्प॑णः॒ सोम॑गोपाः। वसोः᳚ सू॒नुः सह॑सो अ॒फ्सु राजा॒ वि भा॒त्यग्र॑ उ॒षसा॑मिधा॒नः। यस्ते॑ अ॒द्य कृ॒णव॑द्भद्रशोचे\-ऽपू॒पं दे॑व घृ॒तव॑न्तमग्ने। प्र तं न॑य प्रत॒रां वस्यो॒ अच्छा॒भि द्यु॒म्नं दे॒वभ॑क्तं यविष्ठ। आ~(८)

%4.2.2.4
तम्भ॑ज सौश्रव॒सेष्व॑ग्न उ॒क्थउ॑क्थ॒ आ भ॑ज श॒स्यमा॑ने। प्रि॒यः सूर्ये᳚ प्रि॒यो अ॒ग्ना भ॑वा॒त्युज्जा॒तेन॑ भि॒नद॒दुज्जनि॑त्वैः। त्वाम॑ग्ने॒ यज॑माना॒ अनु॒ द्यून् विश्वा॒ वसू॑नि दधिरे॒ वार्या॑णि। त्वया॑ स॒ह द्रवि॑णमि॒च्छमा॑ना व्र॒जं गोम॑न्तमु॒शिजो॒ वि व॑व्रुः। दृ॒शा॒नो रु॒क्म उ॒र्व्या व्य॑द्यौद्दु॒र्मर्\mbox{}ष॒मायुः॑ श्रि॒ये रु॑चा॒नः। अ॒ग्निर॒मृतो॑ अभव॒द्वयो॑भि॒र्यदे॑नं॒ द्यौरज॑नयथ्सु॒रेताः᳚॥~(९)

%4.2.3.0
{\anuvakamend[{तृ॒तीये᳚ त्वा॒ गर्भ॒ आ य॑वि॒ष्ठा यच्च॒त्वारि॑ च}]}%~(२)

%4.2.3.1
अन्न॑प॒ते\-ऽन्न॑स्य नो देह्यनमी॒वस्य॑ शु॒ष्मिणः॑। प्रप्र॑दा॒तारं॑ तारिष॒ ऊर्जं॑ नो धेहि द्वि॒पदे॒ चतु॑ष्पदे। उदु॑ त्वा॒ विश्वे॑ दे॒वा अग्ने॒ भर॑न्तु॒ चित्ति॑भिः। स नो॑ भव शि॒वत॑मः सु॒प्रती॑को वि॒भाव॑सुः। प्रेद॑ग्ने॒ ज्योति॑ष्मान् याहि शि॒वेभि॑र॒र्चिभि॒स्त्वम्। बृ॒हद्भि॑र्भा॒नुभि॒र्भास॒न्मा हिꣳ॑सीस्त॒नुवा᳚ प्र॒जाः। स॒मिधा॒ग्निं दु॑वस्यत घृ॒तैर्बो॑धय॒ताति॑थिम्। आ~(१०)

%4.2.3.2
अ॒स्मि॒न् ह॒व्या जु॑होतन। प्रप्रा॒यम॒ग्निर्भ॑र॒तस्य॑ शृण्वे॒ वि यथ्सूर्यो॒ न रोच॑ते बृ॒हद्भाः। अ॒भि यः पू॒रुम् पृत॑नासु त॒स्थौ दी॒दाय॒ दैव्यो॒ अति॑थिः शि॒वो नः॑। आपो॑ देवीः॒ प्रति॑ गृह्णीत॒ भस्मै॒तथ्स्यो॒ने कृ॑णुध्वꣳ सुर॒भावु॑ लो॒के। तस्मै॑ नमन्तां॒ जन॑यः सु॒पत्नी᳚र्मा॒तेव॑ पु॒त्रम्बि॑भृ॒ता स्वे॑नम्। अ॒फ्स्व॑ग्ने॒ सधि॒ष्टव॑~(११)

%4.2.3.3
सौष॑धी॒रनु॑ रुध्यसे। गर्भे॒ सञ्जा॑यसे॒ पुनः॑। गर्भो॑ अ॒स्योष॑धीनां॒ गर्भो॒ वन॒स्पती॑नाम्। गर्भो॒ विश्व॑स्य भू॒तस्याग्ने॒ गर्भो॑ अ॒पाम॑सि। प्र॒सद्य॒ भस्म॑ना॒ योनि॑म॒पश्च॑ पृथि॒वीम॑ग्ने। स॒ꣳ॒सृज्य॑ मा॒तृभि॒स्त्वं ज्योति॑ष्मा॒न्पुन॒रास॑दः। पुन॑रा॒सद्य॒ सद॑नम॒पश्च॑ पृथि॒वीम॑ग्ने। शेषे॑ मा॒तुर्यथो॒पस्थे॒\-ऽन्तर॒स्याꣳ शि॒वत॑मः। पुन॑रू॒र्जा~(१२)

%4.2.3.4
नि व॑र्तस्व॒ पुन॑रग्न इ॒षायु॑षा। पुन॑र्नः पाहि वि॒श्वतः॑। स॒ह र॒य्या नि व॑र्त॒स्वाग्ने॒ पिन्व॑स्व॒ धार॑या। वि॒श्वफ्स्नि॑या वि॒श्वत॒स्परि॑। पुन॑स्त्वादि॒त्या रु॒द्रा वस॑वः॒ समि॑न्धता॒म्पुन॑र्ब्र॒ह्माणो॑ वसुनीथ य॒ज्ञैः। घृ॒तेन॒ त्वं त॒नुवो॑ वर्धयस्व स॒त्याः स॑न्तु॒ यज॑मानस्य॒ कामाः᳚। बोधा॑ नो अ॒स्य वच॑सो यविष्ठ॒ मꣳहि॑ष्ठस्य॒ प्रभृ॑तस्य स्वधावः। पीय॑ति त्वो॒ अनु॑ त्वो गृणाति व॒न्दारु॑स्ते त॒नुवं॑ वन्दे अग्ने। स बो॑धि सू॒रिर्म॒घवा॑ वसु॒दावा॒ वसु॑पतिः। यु॒यो॒ध्य॑स्मद्द्वेषाꣳ॑सि॥~(१३)

%4.2.4.0
{\anuvakamend[{आ तवो॒र्जा\-ऽनु॒ षोड॑श च}]}%~(३)

%4.2.4.1
अपे॑त॒ वीत॒ वि च॑ सर्प॒तातो॒ ये\-ऽत्र॒ स्थ पु॑रा॒णा ये च॒ नूत॑नाः। अदा॑दि॒दं य॒मो॑\-ऽव॒सानं॑ पृथि॒व्या अक्र॑न्नि॒मम् पि॒तरो॑ लो॒कम॑स्मै। अ॒ग्नेर्भस्मा᳚स्य॒ग्नेः पुरी॑षमसि सं॒ज्ञान॑मसि काम॒धर॑ण॒म्मयि॑ ते काम॒धर॑णम्भूयात्। सं या वः॑ प्रि॒यास्त॒नुवः॒ सम्प्रि॒या हृ॑दयानि वः। आ॒त्मा वो॑ अस्तु~(१४)

%4.2.4.2
सम्प्रि॑यः॒ सम्प्रि॑यास्त॒नुवो॒ मम॑। अ॒यꣳ सो अ॒ग्निर्यस्मि॒न्थ्सोम॒मिन्द्रः॑ सु॒तं द॒धे ज॒ठरे॑ वावशा॒नः। स॒ह॒स्रियं॒ वाज॒मत्यं॒ न सप्तिꣳ॑ सस॒वान्थ्सन्थ्स्तू॑यसे जातवेदः। अग्ने॑ दि॒वो अर्ण॒मच्छा॑ जिगा॒स्यच्छा॑ दे॒वाꣳ ऊ॑चिषे॒ धिष्णि॑या॒ ये। याः प॒रस्ता᳚द्रोच॒ने सूर्य॑स्य॒ याश्चा॒वस्ता॑दुप॒तिष्ठ॑न्त॒ आपः॑। अग्ने॒ यत्ते॑ दि॒वि वर्चः॑ पृथि॒व्यां यदोष॑धीषु~(१५)

%4.2.4.3
अ॒फ्सु वा॑ यजत्र। येना॒न्तरि॑क्षमु॒र्वा॑त॒तन्थ॑ त्वे॒षः स भा॒नुर॑र्ण॒वो नृ॒चक्षाः᳚। पु॒री॒ष्या॑सो अ॒ग्नयः॑ प्राव॒णेभिः॑ स॒जोष॑सः। जु॒षन्ताꣳ॑ ह॒व्यमाहु॑तमनमी॒वा इषो॑ म॒हीः। इडा॑मग्ने पुरु॒दꣳसꣳ॑ स॒निं गोः श॑श्वत्त॒मꣳ हव॑मानाय साध। स्यान्नः॑ सु॒नुस्तन॑यो वि॒जावाग्ने॒ सा ते॑ सुम॒तिर्भू᳚त्व॒स्मे। अ॒यं ते॒ योनि॑र्\mbox{}ऋ॒त्वियो॒ यतो॑ जा॒तो अरो॑चथाः। तं जा॒नन्न्~(१६)

%4.2.4.4
अ॒ग्न॒ आ रो॒हाथा॑ नो वर्धया र॒यिम्। चिद॑सि॒ तया॑ दे॒वत॑याङ्गिर॒स्वद्ध्रु॒वा सी॑द परि॒चिद॑सि॒ तया॑ दे॒वत॑या\-ऽ ङ्गिर॒स्वद्ध्रु॒वा सी॑द लो॒कं पृ॑ण छि॒द्रं पृ॒णाथो॑ सीद शि॒वा त्वम्। इ॒न्द्रा॒ग्नी त्वा॒ बृह॒स्पति॑र॒स्मिन् योना॑वसीषदन्न्। ता अ॑स्य॒ सूद॑दोहसः॒ सोमꣴ॑ श्रीणन्ति॒ पृश्न॑यः। जन्मं॑ दे॒वानां॒ विश॑स्त्रि॒ष्वा रो॑च॒ने दि॒वः॥~(१७)

%4.2.5.0
{\anuvakamend[{अ॒स्त्वोष॑धीषु जा॒नन्न॒ष्टाच॑त्वारिꣳशच्च}]}%~(४)

%4.2.5.1
समि॑त॒ꣳ॒ सं क॑ल्पेथा॒ꣳ॒ सम्प्रि॑यौ रोचि॒ष्णू सु॑मन॒स्यमा॑नौ। इष॒मूर्ज॑म॒भि सं॒वसा॑नौ॒ सं वा॒म्मनाꣳ॑सि॒ सं व्र॒ता समु॑ चि॒त्तान्याक॑रम्। अग्ने॑ पुरीष्याधि॒पा भ॑वा॒ त्वं नः॑। इष॒मूर्जं॒ यज॑मानाय धेहि। पु॒री॒ष्य॑स्त्वम॑ग्ने रयि॒मान्पु॑ष्टि॒माꣳ अ॑सि। शि॒वाः कृ॒त्वा दिशः॒ सर्वाः॒ स्वां योनि॑मि॒हास॑दः। भव॑तं नः॒ सम॑नसौ॒ समो॑कसौ~(१८)

%4.2.5.2
अ॒रे॒पसौ᳚। मा य॒ज्ञꣳ हिꣳ॑सिष्टं॒ मा य॒ज्ञप॑तिं जातवेदसौ शि॒वौ भ॑वतम॒द्य नः॑। मा॒तेव॑ पु॒त्रं पृ॑थि॒वी पु॑री॒ष्य॑म॒ग्निꣴ स्वे योना॑वभारु॒खा। तां विश्वै᳚र्दे॒वैर्\mbox{}ऋ॒तुभिः॑ संविदा॒नः प्र॒जाप॑तिर्वि॒श्वक॑र्मा॒ वि मु॑ञ्चतु। यद॒स्य पा॒रे रज॑सः शु॒क्रं ज्योति॒रजा॑यत। तन्नः॑ पर्\mbox{}ष॒दति॒ द्विषो\-ऽग्ने॑ वैश्वानर॒ स्वाहा᳚। नमः॒ सु ते॑ निर्\mbox{}ऋते विश्वरूपे~(१९)

%4.2.5.3
अ॒य॒स्मयं॒ वि चृ॑ता ब॒न्धमे॒तम्। य॒मेन॒ त्वं य॒म्या॑ संविदा॒नोत्त॒मं नाक॒मधि॑ रोहये॒मम्। यत्ते॑ दे॒वी निर्\mbox{}ऋ॑तिरा ब॒बन्ध॒ दाम॑ ग्री॒वास्व॑विच॒र्त्यम्। इ॒दं ते॒ तद्वि ष्या॒म्यायु॑षो॒ न मध्या॒दथा॑ जी॒वः पि॒तुम॑द्धि॒ प्रमु॑क्तः। यस्या᳚स्ते अ॒स्याः क्रू॒र आ॒सञ्जु॒होम्ये॒षाम्ब॒न्धाना॑मव॒सर्ज॑नाय। भूमि॒रिति॑ त्वा॒ जना॑ वि॒दुर्निर्\mbox{}ऋ॑तिः~(२०)

%4.2.5.4
इति॑ त्वा॒हं परि॑ वेद वि॒श्वतः॑। असु॑न्वन्त॒मय॑जमानमिच्छ स्ते॒नस्ये॒त्यां तस्क॑र॒स्यान्वे॑षि। अ॒न्यम॒स्मदि॑च्छ॒ सा त॑ इ॒त्या नमो॑ देवि निर्\mbox{}ऋते॒ तुभ्य॑मस्तु। दे॒वीम॒हं निर्\mbox{}ऋ॑तिं॒ वन्द॑मानः पि॒तेव॑ पु॒त्रं द॑सये॒ वचो॑भिः। विश्व॑स्य॒ या जाय॑मानस्य॒ वेद॒ शिरः॑शिरः॒ प्रति॑ सू॒री वि च॑ष्टे। नि॒वेश॑नः सं॒गम॑नो॒ वसू॑नां॒ विश्वा॑ रू॒पाभि च॑ष्टे~(२१)

%4.2.5.5
शची॑भिः। दे॒व इ॑व सवि॒ता स॒त्यध॒र्मेन्द्रो॒ न त॑स्थौ सम॒रे प॑थी॒नाम्। सं व॑र॒त्रा द॑धातन॒ निरा॑हा॒वान्कृ॑णोतन। सि॒ञ्चाम॑हा अव॒टमु॒द्रिणं॑ व॒यं विश्वाहाद॑स्त॒मक्षि॑तम्। निष्कृ॑ताहावमव॒टꣳ सु॑वर॒त्रꣳ सु॑षेच॒नम्। उ॒द्रिणꣳ॑ सिञ्चे॒ अक्षि॑तम्। सीरा॑ युञ्जन्ति क॒वयो॑ यु॒गा वि त॑न्वते॒ पृथ॑क्। धीरा॑ दे॒वेषु॑ सुम्न॒या। यु॒नक्त॒ सीरा॒ वि यु॒गा त॑नोत कृ॒ते योनौ॑ वपते॒ह~(२२)

%4.2.5.6
बीजम्᳚। गि॒रा च॑ श्रु॒ष्टिः सभ॑रा॒ अस॑न्नो॒ नेदी॑य॒ इथ्सृ॒ण्या॑ प॒क्वमाय॑त्। लाङ्ग॑ल॒म्पवी॑रवꣳ सु॒शेवꣳ॑ सुम॒तिथ्स॑रु। उदित्कृ॑षति॒ गामवि॑म्प्रफ॒र्व्यं॑ च॒ पीव॑रीम्। प्र॒स्थाव॑द्रथ॒वाह॑नम्। शु॒नं नः॒ फाला॒ वि तु॑दन्तु॒ भूमिꣳ॑ शु॒नं की॒नाशा॑ अ॒भि य॑न्तु वा॒हान्। शु॒नम्प॒र्जन्यो॒ मधु॑ना॒ पयो॑भिः॒ शुना॑सीरा शु॒नम॒स्मासु॑ धत्तम्। कामं॑ कामदुघे धुक्ष्व मि॒त्राय॒ वरु॑णाय च। इन्द्रा॑या॒ग्नये॑ पू॒ष्ण ओष॑धीभ्यः प्र॒जाभ्यः॑। घृ॒तेन॒ सीता॒ मधु॑ना॒ सम॑क्ता॒ विश्वै᳚र्दे॒वैरनु॑मता म॒रुद्भिः॑। ऊर्ज॑स्वती॒ पय॑सा॒ पिन्व॑माना॒स्मान्थ्सी॑ते॒ पय॑सा॒भ्याव॑वृथ्स्व॥~(२३)

%4.2.6.0
{\anuvakamend[{समो॑कसौ विश्वरूपे वि॒दुर्निर्\mbox{}ऋ॑तिर॒भि च॑ष्ट इ॒ह मि॒त्राय॒ द्वाविꣳ॑शतिश्च}]}%~(५)

%4.2.6.1
या जा॒ता ओष॑धयो दे॒वेभ्य॑स्त्रियु॒गम्पु॒रा। मन्दा॑मि ब॒भ्रूणा॑महꣳ श॒तं धामा॑नि स॒प्त च॑। श॒तं वो॑ अम्ब॒ धामा॑नि स॒हस्र॑मुत वो॒ रुहः॑। अथा॑ शतक्रत्वो यू॒यमि॒मं मे॑ अग॒दं कृ॑त। पुष्पा॑वतीः प्र॒सूव॑तीः फ॒लिनी॑रफ॒ला उ॒त। अश्वा॑ इव स॒जित्व॑रीर्वी॒रुधः॑ पारयि॒ष्णवः॑। ओष॑धी॒रिति॑ मातर॒स्तद्वो॑ देवी॒रुप॑ ब्रुवे। रपाꣳ॑सि विघ्न॒तीरि॑त॒ रपः॑~(२४)

%4.2.6.2
चा॒तय॑मानाः। अ॒श्व॒त्थे वो॑ नि॒षद॑नम्प॒र्णे वो॑ वस॒तिः कृ॒ता। गो॒भाज॒ इत्किला॑सथ॒ यथ्स॒नव॑थ॒ पूरु॑षम्। यद॒हं वा॒जय॑न्नि॒मा ओष॑धी॒र्\mbox{}हस्त॑ आद॒धे। आ॒त्मा यक्ष्म॑स्य नश्यति पु॒रा जी॑व॒गृभो॑ यथा। यदोष॑धयः सं॒गच्छ॑न्ते॒ राजा॑नः॒ समि॑ताविव। विप्रः॒ स उ॑च्यते भि॒षग्र॑क्षो॒हामी॑व॒चात॑नः। निष्कृ॑ति॒र्नाम॑ वो मा॒ताथा॑ यू॒यꣴ स्थ॒ सङ्कृ॑तीः। स॒राः प॑त॒त्रिणीः᳚~(२५)

%4.2.6.3
स्थ॒न॒ यदा॒मय॑ति॒ निष्कृ॑त। अ॒न्या वो॑ अ॒न्याम॑वत्व॒न्यान्यस्या॒ उपा॑वत। ताः सर्वा॒ ओष॑धयः संविदा॒ना इ॒दम्मे॒ प्राव॑ता॒ वचः॑। उच्छुष्मा॒ ओष॑धीनां॒ गावो॑ गो॒ष्ठादि॑वेरते। धनꣳ॑ सनि॒ष्यन्ती॑नामा॒त्मानं॒ तव॑ पूरुष। अति॒ विश्वाः᳚ परि॒ष्ठाः स्ते॒न इ॑व व्र॒जम॑क्रमुः। ओष॑धयः॒ प्राचु॑च्यवु॒र्यत् किं च॑ त॒नुवा॒ꣳ॒ रपः॑। याः~(२६)

%4.2.6.4
त॒ आ॒त॒स्थुरा॒त्मानं॒ या आ॑विवि॒शुः परुः॑परुः। तास्ते॒ यक्ष्मं॒ वि बा॑धन्तामु॒ग्रो म॑ध्यम॒शीरि॑व। सा॒कं य॑क्ष्म॒ प्र प॑त श्ये॒नेन॑ किकिदी॒विना᳚। सा॒कं वात॑स्य॒ ध्राज्या॑ सा॒कं न॑श्य नि॒हाक॑या। अ॒श्वा॒व॒तीꣳ सो॑मव॒तीमू॒र्जय॑न्ती॒\-मुदो॑जसम्। आ वि॑थ्सि॒ सर्वा॒ ओष॑धीर॒स्मा अ॑रि॒ष्टता॑तये। याः फ॒लिनी॒र्या अ॑फ॒ला अ॑पु॒ष्पा याश्च॑ पु॒ष्पिणीः᳚। बृह॒स्पति॑प्रसूता॒स्ता नो॑ मुञ्च॒न्त्वꣳह॑सः। याः~(२७)

%4.2.6.5
ओष॑धयः॒ सोम॑राज्ञीः॒ प्रवि॑ष्टाः पृथि॒वीमनु॑। तासां॒ त्वम॑स्युत्त॒मा प्र णो॑ जी॒वात॑वे सुव। अ॒व॒पत॑न्तीरवदन्दि॒व ओष॑धयः॒ परि॑। यं जी॒वम॒श्नवा॑महै॒ न स रि॑ष्याति॒ पूरु॑षः। याश्चे॒दमु॑पशृ॒ण्वन्ति॒ याश्च॑ दू॒रं परा॑गताः। इ॒ह सं॒गत्य॒ ताः सर्वा॑ अस्मै॒ सं द॑त्त भेष॒जम्। मा वो॑ रिषत्खनि॒ता यस्मै॑ चा॒हं खना॑मि वः। द्वि॒पच्चतु॑ष्पद॒स्माक॒ꣳ॒ सर्व॑म॒स्त्वना॑तुरम्। ओष॑धयः॒ सं व॑दन्ते॒ सोमे॑न स॒ह राज्ञा᳚। यस्मै॑ क॒रोति॑ ब्राह्म॒णस्तꣳ रा॑जन्पारयामसि॥~(२८)

%4.2.7.0
{\anuvakamend[{रपः॑ पत॒त्रिणी॒र्या अꣳह॑सो॒ याः खना॑मि वो॒\-ऽष्टाद॑श च}]}%~(६)

%4.2.7.1
मा नो॑ हिꣳसीज्जनि॒ता यः पृ॑थि॒व्या यो वा॒ दिवꣳ॑ स॒त्यध॑र्मा ज॒जान॑। यश्चा॒पश्च॒न्द्रा बृ॑ह॒तीर्ज॒जान॒ कस्मै॑ दे॒वाय॑ ह॒विषा॑ विधेम। अ॒भ्याव॑र्तस्व पृथिवि य॒ज्ञेन॒ पय॑सा स॒ह। व॒पां ते॑ अ॒ग्निरि॑षि॒तो\-ऽव॑ सर्पतु। अग्ने॒ यत्ते॑ शु॒क्रं यच्च॒न्द्रं यत्पू॒तं यद्य॒ज्ञियम्᳚। तद्दे॒वेभ्यो॑ भरामसि। इष॒मूर्ज॑म॒हमि॒त आ~(२९)

%4.2.7.2
द॒द॒ ऋ॒तस्य॒ धाम्नो॑ अ॒मृत॑स्य॒ योनेः᳚। आ नो॒ गोषु॑ विश॒त्वौष॑धीषु॒ जहा॑मि से॒दिमनि॑रा॒ममी॑वाम्। अग्ने॒ तव॒ श्रवो॒ वयो॒ महि॑ भ्राजन्त्य॒र्चयो॑ विभावसो। बृह॑द्भानो॒ शव॑सा॒ वाज॑मु॒क्थ्यं॑ दधा॑सि दा॒शुषे॑ कवे। इ॒र॒ज्यन्न॑ग्ने प्रथयस्व ज॒न्तुभि॑र॒स्मे रायो॑ अमर्त्य। स द॑र्\mbox{}श॒तस्य॒ वपु॑षो॒ वि रा॑जसि पृ॒णक्षि॑ सान॒सिꣳ र॒यिम्। ऊर्जो॑ नपा॒ज्जात॑वेदः सुश॒स्तिभि॒र्मन्द॑स्व~(३०)

%4.2.7.3
धी॒तिभि॑र्हि॒तः। त्वे इषः॒ सं द॑धु॒र्भूरि॑रेतसश्चि॒त्रोत॑यो वा॒मजा॑ताः। पा॒व॒कव॑र्चाः शु॒क्रव॑र्चा॒ अनू॑नवर्चा॒ उदि॑यर्\mbox{}षि भा॒नुना᳚। पु॒त्रः पि॒तरा॑ वि॒चर॒न्नुपा॑वस्यु॒भे पृ॑णक्षि॒ रोद॑सी। ऋ॒तावा॑नम्महि॒षं वि॒श्वच॑र्\mbox{}षणिम॒ग्निꣳ सु॒म्नाय॑ दधिरे पु॒रो जनाः᳚। श्रुत्क॑र्णꣳ स॒प्रथ॑स्तं᳚ त्वा गि॒रा दैव्य॒म्मानु॑षा यु॒गा। नि॒ष्क॒र्तार॑मध्व॒रस्य॒ प्रचे॑तसं॒ क्षय॑न्त॒ꣳ॒ राध॑से म॒हे। रा॒तिम्भृगू॑णामु॒शिजं॑ क॒विक्र॑तुं पृ॒णक्षि॑ सान॒सिम्~(३१)

%4.2.7.4
र॒यिम्। चितः॑ स्थ परि॒चित॑ ऊर्ध्व॒चितः॑ श्रयध्वं॒ तया॑ दे॒वत॑याङ्गिर॒स्वद् ध्रु॒वाः सी॑दत। आ प्या॑यस्व॒ समे॑तु ते वि॒श्वतः॑ सोम॒ वृष्णि॑यम्। भवा॒ वाज॑स्य सङ्ग॒थे। सं ते॒ पयाꣳ॑सि॒ समु॑ यन्तु॒ वाजाः॒ सं वृष्णि॑यान्यभिमाति॒षाहः॑। आ॒प्याय॑मानो अ॒मृता॑य सोम दि॒वि श्रवाꣴ॑स्युत्त॒मानि॑ धिष्व॥~(३२)

%4.2.8.0
{\anuvakamend[{आ मन्द॑स्व सान॒सिमेका॒न्नच॑त्वारि॒ꣳ॒शच्च॑}]}%~(७)

%4.2.8.1
अ॒भ्य॑स्था॒द्विश्वाः॒ पृत॑ना॒ अरा॑ती॒स्तद॒ग्निरा॑ह॒ तदु॒ सोम॑ आह। बृह॒स्पतिः॑ सवि॒ता तन्म॑ आह पु॒षा मा॑धाथ्सुकृ॒तस्य॑ लो॒के। यदक्र॑न्दः प्रथ॒मं जाय॑मान उ॒द्यन्थ्स॑मु॒द्रादुत वा॒ पुरी॑षात्। श्ये॒नस्य॑ प॒क्षा ह॑रि॒णस्य॑ बा॒हू उप॑स्तुतं॒ जनि॑म॒ तत्ते॑ अर्वन्न्। अ॒पां पृ॒ष्ठम॑सि॒ योनि॑र॒ग्नेः स॑मु॒द्रम॒भितः॒ पिन्व॑मानम्। वर्ध॑मानम्म॒हः~(३३)

%4.2.8.2
आ च॒ पुष्क॑रं दि॒वो मात्र॑या व॒रिणा प्र॑थस्व। ब्रह्म॑ जज्ञा॒नम्प्र॑थ॒मम्पु॒रस्ता॒द्वि सी॑म॒तः सु॒रुचो॑ वे॒न आ॑वः। स बु॒ध्निया॑ उप॒मा अ॑स्य वि॒ष्ठाः स॒तश्च॒ योनि॒मस॑तश्च॒ विवः॑। हि॒र॒ण्य॒ग॒र्भः सम॑वर्त॒ताग्रे॑ भू॒तस्य॑ जा॒तः पति॒रेक॑ आसीत्। स दा॑धार पृथि॒वीं द्यामु॒तेमां कस्मै॑ दे॒वाय॑ ह॒विषा॑ विधेम। द्र॒फ्सश्च॑स्कन्द पृथि॒वीमनु॑~(३४)

%4.2.8.3
द्यामि॒मं च॒ योनि॒मनु॒ यश्च॒ पूर्वः॑। तृ॒तीयं॒ योनि॒मनु॑ स॒ञ्चर॑न्तं द्र॒फ्सं जु॑हो॒म्यनु॑ स॒प्त होत्राः᳚। नमो॑ अस्तु स॒र्पेभ्यो॒ ये के च॑ पृथि॒वीमनु॑। ये अ॒न्तरि॑क्षे॒ ये दि॒वि तेभ्यः॑ स॒र्पेभ्यो॒ नमः॑। ये॑\-ऽदो रो॑च॒ने दि॒वो ये वा॒ सूर्य॑स्य र॒श्मिषु॑। येषा॑म॒फ्सु सदः॑ कृ॒तं तेभ्यः॑ स॒र्पेभ्यो॒ नमः॑। या इष॑वो यातु॒धाना॑नां॒ ये वा॒ वन॒स्पती॒ꣳ॒ रनु॑। ये वा॑व॒टेषु॒ शेर॑ते॒ तेभ्यः॑ स॒र्पेभ्यो॒ नमः॑॥~(३५)

%4.2.9.0
{\anuvakamend[{म॒हो\-ऽनु॑ यातु॒धाना॑ना॒मेका॑\-दश च}]}%~(८)

%4.2.9.1
ध्रु॒वासि॑ ध॒रुणास्तृ॑ता वि॒श्वक॑र्मणा॒ सुकृ॑ता। मा त्वा॑ समु॒द्र उद्व॑धी॒न्मा सु॑प॒र्णो\-ऽव्य॑थमाना पृथि॒वीं दृꣳ॑ह। प्र॒जाप॑तिस्त्वा सादयतु पृथि॒व्याः पृ॒ष्ठे व्यच॑स्वती॒म्प्रथ॑स्वती॒म्प्रथो॑\-ऽसि पृथि॒व्य॑सि॒ भूर॑सि॒ भूमि॑र॒स्यदि॑तिरसि वि॒श्वधा॑या॒ विश्व॑स्य॒ भुव॑नस्य ध॒र्त्री पृ॑थि॒वीं य॑च्छ पृथि॒वीं दृꣳ॑ह पृथि॒वीं मा हिꣳ॑सी॒र्विश्व॑स्मै प्रा॒णाया॑पा॒नाय॑ व्या॒नायो॑दा॒नाय॑ प्रति॒ष्ठायै᳚~(३६)

%4.2.9.2
च॒रित्रा॑या॒ग्निस्त्वा॒भि पा॑तु म॒ह्या स्व॒स्त्या छ॒र्दिषा॒ शन्त॑मेन॒ तया॑ दे॒वत॑याङ्गिर॒स्वद्ध्रु॒वा सी॑द। काण्डा᳚त्काण्डात् प्र॒रोह॑न्ती॒ परु॑षःपरुषः॒ परि॑। ए॒वा नो॑ दूर्वे॒ प्र त॑नु स॒हस्रे॑ण श॒तेन॑ च। या श॒तेन॑ प्रत॒नोषि॑ स॒हस्रे॑ण वि॒रोह॑सि। तस्या᳚स्ते देवीष्टके वि॒धेम॑ ह॒विषा॑ व॒यम्। अषा॑ढासि॒ सह॑माना॒ सह॒स्वारा॑तीः॒ सह॑स्वारातीय॒तः सह॑स्व॒ पृत॑नाः॒ सह॑स्व पृतन्य॒तः। स॒हस्र॑वीर्या~(३७)

%4.2.9.3
अ॒सि॒ सा मा॑ जिन्व। मधु॒ वाता॑ ऋताय॒ते मधु॑ क्षरन्ति॒ सिन्ध॑वः। माध्वी᳚र्नः स॒न्त्वोष॑धीः। मधु॒ नक्त॑मु॒तोषसि॒ मधु॑म॒त्पार्थि॑व॒ꣳ॒ रजः। मधु॒ द्यौर॑स्तु नः पि॒ता। मधु॑मान्नो॒ वन॒स्पति॒र्मधु॑माꣳ अस्तु॒ सूर्यः॑। माध्वी॒र्गावो॑ भवन्तु नः। म॒ही द्यौः पृ॑थि॒वी च॑ न इ॒मं य॒ज्ञम्मि॑मिक्षताम्। पि॒पृ॒तां नो॒ भरी॑मभिः। तद्विष्णोः᳚ पर॒मम्~(३८)

%4.2.9.4
प॒दꣳ सदा॑ पश्यन्ति सू॒रयः॑। दि॒वीव॒ चक्षु॒रात॑तम्। ध्रु॒वासि॑ पृथिवि॒ सह॑स्व पृतन्य॒तः। स्यू॒ता दे॒वेभि॑र॒मृते॒नागाः᳚। यास्ते॑ अग्ने॒ सूर्ये॒ रुच॑ उद्य॒तो दिव॑मात॒न्वन्ति॑ र॒श्मिभिः॑। ताभिः॒ सर्वा॑भी रु॒चे जना॑य नस्कृधि। या वो॑ देवाः॒ सूर्ये॒ रुचो॒ गोष्वश्वे॑षु॒ या रुचः॑। इन्द्रा᳚ग्नी॒ ताभिः॒ सर्वा॑भी॒ रुचं॑ नो धत्त बृहस्पते। वि॒राट्~(३९)

%4.2.9.5
ज्योति॑रधारयथ्स॒म्राड्ज्योति॑रधारयथ्स्व॒राड्ज्योति॑रधारयत्। अग्ने॑ यु॒क्ष्वा हि ये तवाश्वा॑सो देव सा॒धवः॑। अरं॒ वह॑न्त्या॒शवः॑। यु॒क्ष्वा हि दे॑व॒हूत॑मा॒ꣳ॒ अश्वाꣳ॑ अग्ने र॒थीरि॑व। नि होता॑ पू॒र्व्यः स॑दः। द्र॒फ्सश्च॑स्कन्द पृथि॒वीमनु॒ द्यामि॒मं च॒ योनि॒मनु॒ यश्च॒ पूर्वः॑। तृ॒तीयं॒ योनि॒मनु॑ स॒ञ्चर॑न्तं द्र॒फ्सं जु॑हो॒म्यनु॑ स॒प्त~(४०)

%4.2.9.6
होत्राः᳚। अभू॑दि॒दं विश्व॑स्य॒ भुव॑नस्य॒ वाजि॑नम॒ग्नेर्वै᳚श्वान॒रस्य॑ च। अ॒ग्निर्ज्योति॑षा॒ ज्योति॑ष्मान्रु॒क्मो वर्च॑सा॒ वर्च॑स्वान्। ऋ॒चे त्वा॑ रु॒चे त्वा॒ समिथ्स्र॑वन्ति स॒रितो॒ न धेनाः᳚। अ॒न्तर्\mbox{}हृ॒दा मन॑सा पू॒यमा॑नाः। घृ॒तस्य॒ धारा॑ अ॒भि चा॑कशीमि। हि॒र॒ण्ययो॑ वेत॒सो मध्य॑ आसाम्। तस्मि᳚न्थ्सुप॒र्णो म॑धु॒कृत्कु॑ला॒यी भज॑न्नास्ते॒ मधु॑ दे॒वता᳚भ्यः। तस्या॑सते॒ हर॑यः स॒प्त तीरे᳚ स्व॒धां दुहा॑ना अ॒मृत॑स्य॒ धारा᳚म्॥~(४१)

%4.2.10.0
{\anuvakamend[{प्र॒ति॒ष्ठायै॑ स॒हस्र॑वीर्या पर॒मं वि॒राट्थ्स॒प्त तीरे॑ च॒त्वारि॑ च}]}%~(९)

%4.2.10.1
आ॒दि॒त्यं गर्भ॒म्पय॑सा सम॒ञ्जन्थ्स॒हस्र॑स्य प्रति॒मां वि॒श्वरू॑पम्। परि॑ वृङ्ग्धि॒ हर॑सा॒ माभि मृ॑क्षः श॒तायु॑षं कृणुहि ची॒यमा॑नः। इ॒मं मा हिꣳ॑सीर्द्वि॒पाद॑म्पशू॒नाꣳ सह॑स्राक्ष॒ मेध॒ आ ची॒यमा॑नः। म॒युमा॑र॒ण्यमनु॑ ते दिशामि॒ तेन॑ चिन्वा॒नस्त॒नुवो॒ नि षी॑द। वात॑स्य॒ ध्राजिं॒ वरु॑णस्य॒ नाभि॒मश्वं॑ जज्ञा॒नꣳ स॑रि॒रस्य॒ मध्ये᳚। शिशुं॑ न॒दीना॒ꣳ॒ हरि॒मद्रि॑बुद्ध॒मग्ने॒ मा हिꣳ॑सीः~(४२)

%4.2.10.2
प॒र॒मे व्यो॑मन्न्। इ॒मं मा हिꣳ॑सी॒रेक॑शफम्पशू॒नां क॑निक्र॒दं वा॒जिनं॒ वाजि॑नेषु। गौ॒रमा॑र॒ण्यमनु॑ ते दिशामि॒ तेन॑ चिन्वा॒नस्त॒नुवो॒ नि षी॑द। अज॑स्र॒मिन्दु॑मरु॒षम्भु॑र॒ण्युम॒ग्निमी॑डे पू॒र्वचि॑त्तौ॒ नमो॑भिः। स पर्व॑भिर्\mbox{}ऋतु॒शः कल्प॑मानो॒ गां मा हिꣳ॑सी॒रदि॑तिं वि॒राजम्᳚। इ॒मꣳ स॑मु॒द्रꣳ श॒तधा॑र॒मुथ्सं॑ व्य॒च्यमा॑न॒म्भुव॑नस्य॒ मध्ये᳚। घृ॒तं दुहा॑ना॒मदि॑तिं॒ जना॒याग्ने॒ मा~(४३)

%4.2.10.3
हि॒ꣳ॒सीः॒ प॒र॒मे व्यो॑मन्न्। ग॒व॒यमा॑र॒ण्यमनु॑ ते दिशामि॒ तेन॑ चिन्वा॒नस्त॒नुवो॒ नि षी॑द। वरू᳚त्रिं॒ त्वष्टु॒र्वरु॑णस्य॒ नाभि॒मविं॑ जज्ञा॒नाꣳ रज॑सः॒ पर॑स्मात्। म॒हीꣳ सा॑ह॒स्रीमसु॑रस्य मा॒यामग्ने॒ मा हिꣳ॑सीः॒ पर॒मे व्यो॑मन्न्। इ॒मामू᳚र्णा॒युं वरु॑णस्य मा॒यां त्वच॑म्पशू॒नां द्वि॒पदां॒ चतु॑ष्पदाम्। त्वष्टुः॑ प्र॒जानां᳚ प्रथ॒मं ज॒नित्र॒मग्ने॒ मा हिꣳ॑सीः पर॒मे व्यो॑मन्न्। उष्ट्र॑मार॒ण्यमनु॑~(४४)

%4.2.10.4
ते॒ दि॒शा॒मि॒ तेन॑ चिन्वा॒नस्त॒नुवो॒ नि षी॑द। यो अ॒ग्निर॒ग्नेस्तप॒सो\-ऽधि॑ जा॒तः शोचा᳚त्पृथि॒व्या उ॒त वा॑ दि॒वस्परि॑। येन॑ प्र॒जा वि॒श्वक॑र्मा॒ व्यान॒ट्तम॑ग्ने॒ हेडः॒ परि॑ ते वृणक्तु। अ॒जा ह्य॑ग्नेरज॑निष्ट॒ गर्भा॒थ्सा वा अ॑पश्यज्जनि॒तार॒मग्रे᳚। तया॒ रोह॑माय॒न्नुप॒ मेध्या॑स॒स्तया॑ दे॒वा दे॒वता॒मग्र॑ आयन्न्। श॒र॒भमा॑र॒ण्यमनु॑ ते दिशामि॒ तेन॑ चिन्वा॒नस्त॒नुवो॒ नि षी॑द॥~(४५)

%4.2.11.0
{\anuvakamend[{अग्ने॒ मा हिꣳ॑सी॒रग्ने॒ मोष्ट्र॑मार॒ण्यमनु॑ शर॒भं नव॑ च}]}%॥10॥ आ॒दि॒त्यमि॒मं द्वि॒पाद॑म्म॒युं वात॒स्याश्व॑मि॒ममेक॑शफङ्गौ॒रमज॑स्रङ्गव॒यं वरू᳚त्रि॒मवि॑मि॒मामू᳚र्णा॒युमुष्ट्रं॒ यो अ॒ग्निर॒ग्नेः श॑र॒भम्॥

%4.2.11.1
इन्द्रा᳚ग्नी रोच॒ना दि॒वः परि॒ वाजे॑षु भूषथः। तद्वां᳚ चेति॒ प्र वी॒र्यम्᳚। श्नथ॑द्वृ॒त्रमु॒त स॑नोति॒ वाज॒मिन्द्रा॒ यो अ॒ग्नी सहु॑री सप॒र्यात्। इ॒र॒ज्यन्ता॑ वस॒व्य॑स्य॒ भूरेः॒ सह॑स्तमा॒ सह॑सा वाज॒यन्ता᳚। प्र च॑र्\mbox{}ष॒णिभ्यः॑ पृतना॒ हवे॑षु॒ प्र पृ॑थि॒व्या रि॑रिचाथे दि॒वश्च॑। प्र सिन्धु॑भ्यः॒ प्र गि॒रिभ्यो॑ महि॒त्वा प्रेन्द्रा᳚ग्नी॒ विश्वा॒ भुव॒नात्य॒न्या। मरु॑तो॒ यस्य॒ हि~(४६)

%4.2.11.2
क्षये॑ पा॒था दि॒वो वि॑महसः। स सु॑गो॒पात॑मो॒ जनः॑। य॒ज्ञैर्वा॑ यज्ञवाहसो॒ विप्र॑स्य वा मती॒नाम्। मरु॑तः शृणु॒ता हवम्᳚। श्रि॒यसे॒ कम्भा॒नुभिः॒ सम्मि॑मिक्षिरे॒ ते र॒श्मिभि॒स्त ऋक्व॑भिः सुखा॒दयः॑। ते वाशी॑मन्त इ॒ष्मिणो॒ अभी॑रवो वि॒द्रे प्रि॒यस्य॒ मारु॑तस्य॒ धाम्नः॑। अव॑ ते॒ हेड॒ उदु॑त्त॒मम्। कया॑ नश्चि॒त्र आ भु॑वदू॒ती स॒दावृ॑धः॒ सखा᳚। कया॒ शचि॑ष्ठया वृ॒ता।~(४७)

%4.2.11.3
को अ॒द्य यु॑ङ्क्ते धु॒रि गा ऋ॒तस्य॒ शिमी॑वतो भा॒मिनो॑ दुर्\mbox{}हृणा॒यून्। आ॒सन्नि॑षून् हृ॒थ्स्वसो॑ मयो॒भून् य ए॑षाम् भृ॒त्यामृ॒णध॒थ्स जी॑वात्। अग्ने॒ नया दे॒वाना॒ꣳ॒ शं नो॑ भवन्तु॒ वाजे॑वाजे। अ॒फ्स्व॑ग्ने॒ सधि॒ष्टव॒ सौष॑धी॒रनु॑ रुध्यसे। गर्भे॒ सञ्जा॑यसे॒ पुनः॑। वृषा॑ सोम द्यु॒माꣳ अ॑सि॒ वृषा॑ देव॒ वृष॑व्रतः। वृषा॒ धर्मा॑णि दधिषे। इ॒मं मे॑ वरुण तत्त्वा॑ यामि॒ त्वं नो॑ अग्ने॒ स त्वं नो॑ अग्ने॥~(४८)

%4.3.0.0
{\anuvakamend[{हि वृ॒ता म॒ एका॑\-दश च}]}%॥11॥

%4.3.0.0

{\anuvakamend[{अ॒पां त्वेम॑न्न॒यं पु॒रो भुवः॒ प्राची᳚ ध्रु॒वक्षि॑ति॒स्त्र्यवि॒रिन्द्रा᳚ग्नी॒ मा छन्द॑ आ॒शुस्त्रि॒वृद॒ग्नेर्भा॒गो᳚\-ऽस्येक॑ये॒यमे॒व सा याग्ने॑ जा॒तान॒ग्निर्वृ॒त्राणि॒ त्रयो॑दश}]}%॥13॥
\prashnaend{ अ॒पां त्वेन्द्रा᳚ग्नी इ॒यमे॒व दे॒वता॑ता॒ षट्त्रिꣳ॑शत्॥36॥ अ॒पां त्वेम॑न् ह॒विषा॒ वर्ध॑नेन॥}
%%% END PRASHNA

\sect{तृतीयः प्रश्नः}\setcounter{anuvakam}{0}
\dnsub{तैत्तिरीयसंहितायां चतुर्थकाण्डे तृतीयः प्रश्नः}
%4.3.1.0
%4.3.1.1
अ॒पां त्वेम᳚न्थ्सादयाम्य॒पां त्वोद्म᳚न्थ्सादयाम्य॒पां त्वा॒ भस्म᳚न्थ्सादयाम्य॒पां त्वा॒ ज्योति॑षि सादयाम्य॒पां त्वाय॑ने सादयाम्यर्ण॒वे सद॑ने सीद समु॒द्रे सद॑ने सीद सलि॒ले सद॑ने सीदा॒पां क्षये॑ सीदा॒पाꣳ सधि॑षि सीदा॒पां त्वा॒ सद॑ने सादयाम्य॒पां त्वा॑ स॒धस्थे॑ सादयाम्य॒पां त्वा॒ पुरी॑षे सादयाम्य॒पां त्वा॒ योनौ॑ सादयाम्य॒पां त्वा॒ पाथ॑सि सादयामि गाय॒त्री छन्द॑स्त्रि॒ष्टुप्छन्दो॒ जग॑ती॒ छन्दो॑\-ऽनु॒ष्टुप्छन्दः॑ प॒ङ्क्तिश्छन्दः॑॥~(१)

%4.3.2.0
{\anuvakamend[{योनौ॒ पञ्च॑दश च}]}%~(१)

%4.3.2.1
अ॒यम्पु॒रो भुव॒स्तस्य॑ प्रा॒णो भौ॑वाय॒नो व॑स॒न्तः प्रा॑णाय॒नो गा॑य॒त्री वा॑स॒न्ती गा॑यत्रि॒यै गा॑य॒त्रं गा॑य॒त्रादु॑पा॒ꣳ॒शु\-रु॑पा॒ꣳ॒शोस्त्रि॒वृत्त्रि॒वृतो॑ रथन्त॒रꣳ र॑थन्त॒राद्वसि॑ष्ठ॒ ऋषिः॑ प्र॒जाप॑तिगृहीतया॒ त्वया᳚ प्रा॒णं गृ॑ह्णामि प्र॒जाभ्यो॒\-ऽयं द॑क्षि॒णा वि॒श्वक॑र्मा॒ तस्य॒ मनो॑ वैश्वकर्म॒णं ग्री॒ष्मो मा॑न॒सस्त्रि॒ष्टुग्ग्रै॒ष्मी त्रि॒ष्टुभ॑ ऐ॒डमै॒डाद॑न्तर्या॒मो᳚\-ऽन्तर्या॒मात् प॑ञ्चद॒शः प॑ञ्चद॒शाद्बृ॒हद्बृ॑ह॒तो भ॒रद्वा॑ज॒ ऋषिः॑ प्र॒जाप॑तिगृहीतया॒ त्वया॒ मनः॑~(२)

%4.3.2.2
गृ॒ह्णा॒मि॒ प्र॒जाभ्यो॒\-ऽयम्प॒श्चाद्वि॒श्वव्य॑चा॒स्तस्य॒ चक्षु॑र्वैश्वव्यच॒सं व॒र्\mbox{}षाणि॑ चाक्षु॒षाणि॒ जग॑ती वा॒र्\mbox{}षी जग॑त्या॒ ऋक्ष॑म॒मृक्ष॑माच्छु॒क्रः शु॒क्राथ्स॑प्तद॒शः स॑प्तद॒शाद्वै॑रू॒पं वै॑रू॒पाद्वि॒श्वामि॑त्र॒ ऋषिः॑ प्र॒जाप॑तिगृहीतया॒ त्वया॒ चक्षु॑र्गृह्णामि प्र॒जाभ्य॑ इ॒दमु॑त्त॒राथ्सुव॒स्तस्य॒ श्रोत्रꣳ॑ सौ॒वꣳ श॒रच्छ्रौ॒त्र्य॑नु॒ष्टुप्छा॑र॒द्य॑नु॒ष्टुभः॑ स्वा॒रꣴ स्वा॒रान्म॒न्थी म॒न्थिन॑ एकवि॒ꣳ॒श ए॑कवि॒ꣳ॒शाद्वै॑रा॒जं वै॑रा॒जाज्ज॒मद॑ग्नि॒र्\mbox{}ऋषिः॑ प्र॒जाप॑तिगृहीतया~(३)

%4.3.2.3
त्वया॒ श्रोत्रं॑ गृह्णामि प्र॒जाभ्य॑ इ॒यमु॒परि॑ म॒तिस्तस्यै॒ वाङ्मा॒ती हे॑म॒न्तो वा᳚च्याय॒नः प॒ङ्क्तिर्\mbox{}है॑म॒न्ती प॒ङ्क्त्यै नि॒धन॑वन्नि॒धन॑वत आग्रय॒ण आ᳚ग्रय॒णात्त्रि॑णवत्रयस्त्रिꣳ॒शौ त्रि॑णवत्रयस्त्रिꣳ॒शाभ्याꣳ॑ शाक्वररैव॒ते शा᳚क्वररैव॒ता\-भ्यां᳚ वि॒श्वक॒र्मर्\mbox{}षिः॑ प्र॒जाप॑तिगृहीतया॒ त्वया॒ वाचं॑ गृह्णामि प्रजाभ्यः॥~(४)

%4.3.3.0
{\anuvakamend[{त्वया॒ मनो॑ ज॒मद॑ग्नि॒र्\mbox{}ऋषिः॑ प्र॒जाप॑तिगृहीतया त्रि॒ꣳ॒शच्च॑}]}%~(२)

%4.3.3.1
प्राची॑ दि॒शां व॑स॒न्त ऋ॑तू॒नाम॒ग्निर्दे॒वता॒ ब्रह्म॒ द्रवि॑णं त्रि॒वृथ्स्तोमः॒ स उ॑ पञ्चद॒शव॑र्तनि॒स्त्र्यवि॒र्वयः॑ कृ॒तमया॑नां पुरोवा॒तो वातः॒ सान॑ग॒ ऋषि॑र्दक्षि॒णा दि॒शां ग्री॒ष्म ऋ॑तू॒नामिन्द्रो॑ दे॒वता᳚ क्ष॒त्रं द्रवि॑णं पञ्चद॒शः स्तोमः॒ स उ॑ सप्तद॒शव॑र्तनिर्दित्य॒वाड्वय॒स्त्रेताया॑नां दक्षिणाद्वा॒तो वातः॑ सना॒तन॒ ऋषिः॑ प्र॒तीची॑ दि॒शां व॒र्\mbox{}षा ऋ॑तू॒नां विश्वे॑ दे॒वा दे॒वता॒ विट्~(५)

%4.3.3.2
द्रवि॑णꣳ सप्तद॒शः स्तोमः॒ स उ॑वेकवि॒ꣳ॒शव॑र्तनिस्त्रिव॒थ्सो वयो᳚ द्वाप॒रो\-ऽया॑नाम्पश्चाद्वा॒तो वातो॑\-ऽह॒भून॒ ऋषि॒रुदी॑ची दि॒शाꣳ श॒रदृ॑तू॒नाम्मि॒त्रावरु॑णौ दे॒वता॑ पु॒ष्टं द्रवि॑णमेकवि॒ꣳ॒शः स्तोमः॒ स उ॑ त्रिण॒वव॑र्तनिस्तुर्य॒वाड्वय॑ आस्क॒न्दो-\-ऽ या॑नामुत्तराद्वा॒तो वातः॑ प्र॒त्न ऋषि॑रू॒र्ध्वा दि॒शाꣳ हे॑मन्तशिशि॒रावृ॑तू॒नाम्बृह॒स्पति॑र्दे॒वता॒ वर्चो॒ द्रवि॑णं त्रिण॒वः स्तोमः॒ स उ॑ त्रयस्त्रि॒ꣳ॒शव॑र्तनिः पष्ठ॒वाद्वयो॑\-ऽभि॒भूरया॑नां विष्वग्वा॒तो वातः॑ सुप॒र्ण ऋषिः॑ पि॒तरः॑ पिताम॒हाः परे\-ऽव॑रे॒ ते नः॑ पान्तु॒ ते नो॑\-ऽवन्त्व॒स्मिन्ब्रह्म॑न्न॒स्मिन्क्ष॒त्रे᳚\-ऽस्यामा॒शिष्य॒स्याम्पु॑रो॒धाया॑म॒स्मिन्कर्म॑न्न॒स्यां दे॒वहू᳚त्याम्॥~(६)

%4.3.4.0
{\anuvakamend[{विट्प॑ष्ठ॒वाड्वयो॒\-ऽष्टाविꣳ॑शतिश्च}]}%~(३)

%4.3.4.1
ध्रु॒वक्षि॑तिर्ध्रु॒वयो॑निर्ध्रु॒वासि॑ ध्रु॒वं योनि॒मा सी॑द सा॒ध्या। उख्य॑स्य के॒तुम्प्र॑थ॒मम्पु॒रस्ता॑द॒श्विना᳚ध्व॒र्यू सा॑दयतामि॒ह त्वा᳚। स्वे दक्षे॒ दक्ष॑पिते॒ह सी॑द देव॒त्रा पृ॑थि॒वी बृ॑ह॒ती ररा॑णा। स्वा॒स॒स्था त॒नुवा॒ सं वि॑शस्व पि॒तेवै॑धि सू॒नव॒ आ सु॒शेवा॒श्विना᳚ध्व॒र्यू सा॑दयतामि॒ह त्वा᳚। कु॒ला॒यिनी॒ वसु॑मती वयो॒धा र॒यिं नो॑ वर्ध बहु॒लꣳ सु॒वीरम्᳚।~(७)

%4.3.4.2
अपा॑मतिं दुर्म॒तिम्बाध॑माना रा॒यस्पोषे॑ य॒ज्ञप॑तिमा॒भज॑न्ती॒ सुव॑र्धेहि॒ यज॑मानाय॒ पोष॑म॒श्विना᳚ध्व॒र्यू सा॑दयतामि॒ह त्वा᳚। अ॒ग्नेः पुरी॑षमसि देव॒यानी॒ तां त्वा॒ विश्वे॑ अ॒भि गृ॑णन्तु दे॒वाः। स्तोम॑पृष्ठा घृ॒तव॑ती॒ह सी॑द प्र॒जाव॑द॒स्मे द्रवि॒णा य॑जस्वा॒श्विना᳚ध्व॒र्यू सा॑दयतामि॒ह त्वा᳚। दि॒वो मू॒र्धासि॑ पृथि॒व्या नाभि॑र्वि॒ष्टम्भ॑नी दि॒शामधि॑पत्नी॒ भुव॑नानाम्।~(८)

%4.3.4.3
ऊ॒र्मिर्द्र॒फ्सो अ॒पाम॑सि वि॒श्वक॑र्मा त॒ ऋषि॑र॒श्विना᳚ध्व॒र्यू सा॑दयतामि॒ह त्वा᳚। स॒जूर्\mbox{}ऋ॒तुभिः॑ स॒जूर्वि॒धाभिः॑ स॒जूर्वसु॑भिः स॒जू रु॒द्रैः स॒जूरा॑दि॒त्यैः स॒जूर्विश्वै᳚र्दे॒वैः स॒जूर्दे॒वैः स॒जूर्दे॒वैर्व॑योना॒धैर॒ग्नये᳚ त्वा वैश्वान॒राया॒श्विना᳚ध्व॒र्यू सा॑दयतामि॒ह त्वा᳚। प्रा॒णं मे॑ पाह्यपा॒नं मे॑ पाहि व्या॒नं मे॑ पाहि॒ चक्षु॑र्म उ॒र्व्या वि भा॑हि॒ श्रोत्रं॑ मे श्लोकया॒पस्पि॒न्वौष॑धीर्जिन्व द्वि॒पात्पा॑हि॒ चतु॑ष्पादव दि॒वो वृष्टि॒मेर॑य॥~(९)

%4.3.5.0
{\anuvakamend[{सु॒वीरं॒ भुव॑नानामु॒र्व्या स॒प्तद॑श च}]}%~(४)

%4.3.5.1
त्र्यवि॒र्वय॑स्त्रि॒ष्टुप्छन्दो॑ दित्य॒वाड्वयो॑ वि॒राट्छन्दः॒ पञ्चा॑वि॒र्वयो॑ गाय॒त्री छन्द॑स्त्रिव॒थ्सो वय॑ उ॒ष्णिहा॒ छन्द॑स्तुर्य॒वाड्वयो॑\-ऽ\-नु॒ष्टुप्छन्दः॑ पष्ठ॒वाद्वयो॑ बृह॒ती छन्द॑ उ॒क्षा वयः॑ स॒तोबृ॑हती॒ छन्द॑ ऋष॒भो वयः॑ क॒कुच्छन्दो॑ धे॒नुर्वयो॒ जग॑ती॒ छन्दो॑\-ऽ\-न॒ड्वान् वयः॑ प॒ङ्क्तिश्छन्दो॑ ब॒स्तो वयो॑ विव॒लं छन्दो॑ वृ॒ष्णिर्वयो॑ विशा॒लं छन्दः॒ पुरु॑षो॒ वय॑स्त॒न्द्रं छन्दो᳚ व्या॒घ्रो वयो\-ऽ\-ना॑धृष्टं॒ छन्दः॑ सि॒ꣳ॒हो वय॑श्छ॒दिश्छन्दो॑ विष्ट॒म्भो वयो\-ऽधि॑पति॒श्छन्दः॑ क्ष॒त्रं वयो॒ मयं॑दं॒ छन्दो॑ वि॒श्वक॑र्मा॒ वयः॑ परमे॒ष्ठी छन्दो मू॒र्धा वयः॑ प्र॒जाप॑ति॒श्छन्दः॑॥~(१०)

%4.3.6.0
{\anuvakamend[{पुरु॑षो॒ वयः॒ षड्विꣳ॑शतिश्च}]}%~(५)

%4.3.6.1
इन्द्रा᳚ग्नी॒ अव्य॑थमाना॒मिष्ट॑कां दृꣳहतं यु॒वम्। पृ॒ष्ठेन॒ द्यावा॑पृथि॒वी अ॒न्तरि॑क्षं च॒ वि बा॑धताम्॥ वि॒श्वक॑र्मा त्वा सादयत्व॒न्तरि॑क्षस्य पृ॒ष्ठे व्यच॑स्वती॒म्प्रथ॑स्वती॒म्भास्व॑तीꣳ सूरि॒मती॒मा या द्याम्भास्या पृ॑थि॒वीमोर्व॑न्तरि॑क्षम॒न्तरि॑क्षं यच्छा॒न्तरि॑क्षं दृꣳहा॒न्तरि॑क्षं॒ मा हिꣳ॑सी॒र्विश्व॑स्मै प्रा॒णाया॑पा॒नाय॑ व्या॒नायो॑दा॒नाय॑ प्रति॒ष्ठायै॑ च॒रित्रा॑य वा॒युस्त्वा॒भि पा॑तु म॒ह्या स्व॒स्त्या छ॒र्दिषा᳚~(११)

%4.3.6.2
शन्त॑मेन॒ तया॑ दे॒वत॑याङ्गिर॒स्वद्ध्रु॒वा सी॑द। राज्ञ्य॑सि॒ प्राची॒ दिग्वि॒राड॑सि दक्षि॒णा दिख्स॒म्राड॑सि प्र॒तीची॒ दिख्स्व॒राड॒स्युदी॑ची॒ दिगधि॑पत्न्यसि बृह॒ती दिगायु॑र्मे पाहि प्रा॒णं मे॑ पाह्यपा॒नं मे॑ पाहि व्या॒नं मे॑ पाहि॒ चक्षु॑र्मे पाहि॒ श्रोत्रं॑ मे पाहि॒ मनो॑ मे जिन्व॒ वाचं॑ मे पिन्वा॒त्मानं॑ मे पाहि॒ ज्योति॑र्मे यच्छ॥~(१२)

%4.3.7.0
{\anuvakamend[{छ॒र्दिषा॑ पिन्व॒ षट्च॑}]}%~(६)

%4.3.7.1
मा छन्दः॑ प्र॒मा छन्दः॑ प्रति॒मा छन्दो᳚\-ऽस्री॒विश्छन्दः॑ प॒ङ्क्तिश्छन्द॑ उ॒ष्णिहा॒ छन्दो॑ बृह॒ती छन्दो॑\-ऽनु॒ष्टुप्छन्दो॑ वि॒राट्छन्दो॑ गाय॒त्री छन्द॑स्त्रि॒ष्टुप्छन्दो॒ जग॑ती॒ छन्दः॑ पृथि॒वी छन्दो॒\-ऽन्तरि॑क्षं॒ छन्दो॒ द्यौश्छन्दः॒ समा॒श्छन्दो॒ नक्ष॑त्राणि॒ छन्दो॒ मन॒श्छन्दो॒ वाक्छन्दः॑ कृ॒षिश्छन्दो॒ हिर॑ण्यं॒ छन्दो॒ गौश्छन्दो॒\-ऽजा छन्दो\-ऽश्व॒श्छन्दः॑। अ॒ग्निर्दे॒वता᳚~(१३)

%4.3.7.2
वातो॑ दे॒वता॒ सूर्यो॑ दे॒वता॑ च॒न्द्रमा॑ दे॒वता॒ वस॑वो दे॒वता॑ रु॒द्रा दे॒वता॑दि॒त्या दे॒वता॒ विश्वे॑ दे॒वा दे॒वता॑ म॒रुतो॑ दे॒वता॒ बृह॒स्पति॑र्दे॒वतेन्द्रो॑ दे॒वता॒ वरु॑णो दे॒वता॑ मू॒र्धासि॒ राड्ध्रु॒वासि॑ ध॒रुणा॑ य॒न्त्र्य॑सि॒ यमि॑त्री॒षे त्वो॒र्जे त्वा॑ कृ॒ष्यै त्वा॒ क्षेमा॑य त्वा॒ यन्त्री॒ राड्ध्रु॒वासि॒ धर॑णी ध॒र्त्र्य॑सि॒ धरि॒त्र्यायु॑षे त्वा॒ वर्च॑से॒ त्वौज॑से त्वा॒ बला॑य त्वा॥~(१४)

%4.3.8.0
{\anuvakamend[{दे॒वता\-ऽ\-ऽयु॑षे त्वा॒ षट्च॑}]}%~(७)

%4.3.8.1
आ॒शुस्त्रि॒वृद्भा॒न्तः प॑ञ्चद॒शो व्यो॑म सप्तद॒शः प्रतू᳚र्तिरष्टाद॒शस्तपो॑ नवद॒शो॑\-ऽभिव॒र्तः स॑वि॒ꣳ॒शो ध॒रुण॑ एकवि॒ꣳ॒शो वर्चो᳚ द्वावि॒ꣳ॒शः स॒म्भर॑णस्त्रयोवि॒ꣳ॒शो योनि॑श्चतुर्वि॒ꣳ॒शो गर्भाः᳚ पञ्चवि॒ꣳ॒श ओज॑स्त्रिण॒वः क्रतु॑रेकत्रि॒ꣳ॒शः प्र॑ति॒ष्ठा त्र॑यस्त्रि॒ꣳ॒शो ब्र॒ध्नस्य॑ वि॒ष्टपं॑ चतुस्त्रि॒ꣳ॒शो नाकः॑ षट्त्रि॒ꣳ॒शो वि॑व॒र्तो᳚\-ऽष्टाचत्वारि॒ꣳ॒शो ध॒र्त्रश्च॑तुष्टो॒मः॥~(१५)

%4.3.9.0
{\anuvakamend[{आ॒शुः स॒प्तत्रिꣳ॑शत्}]}%~(८)

%4.3.9.1
अ॒ग्नेर्भा॒गो॑\-ऽसि दी॒क्षाया॒ आधि॑पत्यं॒ ब्रह्म॑ स्पृ॒तं त्रि॒वृथ्स्तोम॒ इन्द्र॑स्य भा॒गो॑\-ऽसि॒ विष्णो॒राधि॑पत्यं क्ष॒त्रꣴ स्पृ॒तम्प॑ञ्चद॒शः स्तोमो॑ नृ॒चक्ष॑साम्भा॒गो॑\-ऽसि धा॒तुराधि॑पत्यं ज॒नित्रꣴ॑ स्पृ॒तꣳ स॑प्तद॒शः स्तोमो॑ मि॒त्रस्य॑ भा॒गो॑\-ऽसि॒ वरु॑ण॒स्याधि॑पत्यं दि॒वो वृ॒ष्टिर्वाताः᳚ स्पृ॒ता ए॑कवि॒ꣳ॒शः स्तोमो\-ऽदि॑त्यै भा॒गो॑\-ऽसि पू॒ष्ण आधि॑पत्य॒मोजः॑ स्पृ॒तं त्रि॑ण॒वः स्तोमो॒ वसू॑नाम्भा॒गो॑\-ऽसि~(१६)

%4.3.9.2
रु॒द्राणा॒माधि॑पत्यं॒ चतु॑ष्पाथ्स्पृ॒तं च॑तुर्वि॒ꣳ॒शः स्तोम॑ आदि॒त्यानां᳚ भा॒गो॑\-ऽसि म॒रुता॒माधि॑पत्यं॒ गर्भाः᳚ स्पृ॒ताः प॑ञ्चवि॒ꣳ॒शः स्तोमो॑ दे॒वस्य॑ सवि॒तुर्भा॒गो॑\-ऽसि॒ बृह॒स्पते॒राधि॑पत्यꣳ स॒मीची॒र्दिशः॑ स्पृ॒ताश्च॑तुष्टो॒मः स्तोमो॒ यावा॑नाम्भा॒गो᳚\-ऽस्यया॑वाना॒माधि॑पत्यं प्र॒जाः स्पृ॒ताश्च॑तुश्चत्वारि॒ꣳ॒शः स्तोम॑ ऋभू॒णाम्भा॒गो॑\-ऽसि॒ विश्वे॑षां दे॒वाना॒माधि॑पत्यम्भू॒तं निशा᳚न्तꣴ स्पृ॒तं त्र॑यस्त्रि॒ꣳ॒शः स्तोमः॑॥~(१७)

%4.3.10.0
{\anuvakamend[{वसू॑नां भा॒गो॑\-ऽसि॒ षट्च॑त्वारिꣳशच्च}]}%~(९)

%4.3.10.1
एक॑यास्तुवत प्र॒जा अ॑धीयन्त प्र॒जाप॑ति॒रधि॑पतिरासीत्ति॒सृभि॑रस्तुवत॒ ब्रह्मा॑सृज्यत॒ ब्रह्म॑ण॒स्पति॒रधि॑पतिरासीत् प॒ञ्चभि॑रस्तुवत भू॒तान्य॑सृज्यन्त भू॒ताना॒म्पति॒रधि॑पतिरासीथ्स॒प्तभि॑रस्तुवत सप्त॒र्\mbox{}षयो॑\-ऽसृज्यन्त धा॒ताधि॑पतिरा\-सीन्न॒वभि॑रस्तुवत पि॒तरो॑\-ऽसृज्य॒न्तादि॑ति॒रधि॑पत्न्यासीदेकाद॒शभि॑रस्तुवत॒र्तवो॑\-ऽसृज्यन्तार्त॒वो\-ऽधि॑पतिरासीत् त्रयोद॒शभि॑रस्तुवत॒ मासा॑ असृज्यन्त संवथ्स॒रो\-ऽधि॑पतिः~(१८)

%4.3.10.2
आ॒सी॒त्प॒ञ्च॒द॒शभि॑रस्तुवत क्ष॒त्रम॑सृज्य॒तेन्द्रो\-ऽधि॑पतिरासीथ्सप्तद॒शभि॑रस्तुवत प॒शवो॑\-ऽसृज्यन्त॒ बृह॒स्पति॒रधि॑पतिरासी\-न्नवद॒शभि॑रस्तुवत शूद्रा॒र्याव॑सृज्येतामहोरा॒त्रे अधि॑पत्नी आस्ता॒मेक॑विꣳशत्यास्तुव॒तैक॑शफाः प॒शवो॑\-ऽसृज्यन्त॒ वरु॒णो\-ऽधि॑पतिरासी॒त्त्रयो॑विꣳशत्यास्तुवत क्षु॒द्राः प॒शवो॑\-ऽसृज्यन्त पू॒षाधि॑पतिरासी॒त्पञ्च॑विꣳशत्यास्तुवतार॒ण्याः प॒शवो॑\-ऽसृज्यन्त वा॒युरधि॑पतिरासीथ्स॒प्तविꣳ॑शत्यास्तुवत॒ द्यावा॑पृथि॒वी वि~(१९)

%4.3.10.3
ऐ॒तां॒ वस॑वो रु॒द्रा आ॑दि॒त्या अनु॒ व्या॑य॒न्तेषा॒माधि॑पत्यमासी॒न्नव॑विꣳशत्यास्तुवत॒ वन॒स्पत॑यो\-ऽसृज्यन्त॒ सोमो\-ऽ\-धि॑पतिरासी॒देक॑त्रिꣳशतास्तुवत प्र॒जा अ॑सृज्यन्त॒ यावा॑नां॒ चाया॑वानां॒ चाधि॑पत्यमासी॒त्त्रय॑स्त्रिꣳशतास्तुवत भू॒तान्य॑शाम्यन्प्र॒जाप॑तिः परमे॒ष्ठ्यधि॑पतिरासीत्॥~(२०)

%4.3.11.0
{\anuvakamend[{सं॒ व॒थ्स॒रो\-ऽधि॑पति॒र्वि पञ्च॑त्रिꣳशच्च}]}%॥10॥

%4.3.11.1
इ॒यमे॒व सा या प्र॑थ॒मा व्यौच्छ॑द॒न्तर॒स्यां च॑रति॒ प्रवि॑ष्टा। व॒धूर्ज॑जान नव॒गज्जनि॑त्री॒ त्रय॑ एनाम्महि॒मानः॑ सचन्ते॥ छन्द॑स्वती उ॒षसा॒ पेपि॑शाने समा॒नं योनि॒मनु॑ स॒ञ्चर॑न्ती। सूर्य॑पत्नी॒ वि च॑रतः प्रजान॒ती के॒तुं कृ॑ण्वा॒ने अ॒जरे॒ भूरि॑रेतसा॥ ऋ॒तस्य॒ पन्था॒मनु॑ ति॒स्र आगु॒स्त्रयो॑ घ॒र्मासो॒ अनु॒ ज्योति॒षागुः॑। प्र॒जामेका॒ रक्ष॒त्यूर्ज॒मेका᳚~(२१)

%4.3.11.2
व्र॒तमेका॑ रक्षति देवयू॒नाम्॥ च॒तु॒ष्टो॒मो अ॑भव॒द्या तु॒रीया॑ य॒ज्ञस्य॑ प॒क्षावृ॑षयो॒ भव॑न्ती। गा॒य॒त्रीं त्रि॒ष्टुभं॒ जग॑तीमनु॒ष्टुभ॑म्बृ॒हद॒र्कं यु॑ञ्जा॒नाः सुव॒राभ॑रन्नि॒दम्॥ प॒ञ्चभि॑र्धा॒ता वि द॑धावि॒दं यत्तासा॒ꣴ॒ स्वसॄ॑रजनय॒त्पञ्च॑पञ्च। तासा॑मु यन्ति प्रय॒वेण॒ पञ्च॒ नाना॑ रू॒पाणि॒ क्रत॑वो॒ वसा॑नाः॥ त्रि॒ꣳ॒शथ्स्वसा॑र॒ उप॑ यन्ति निष्कृ॒तꣳ स॑मा॒नं के॒तुम्प्र॑तिमु॒ञ्चमा॑नाः।~(२२)

%4.3.11.3
ऋ॒तूꣴस्त॑न्वते क॒वयः॑ प्रजान॒तीर्मध्ये॑छन्दसः॒ परि॑ यन्ति॒ भास्व॑तीः। ज्योति॑ष्मती॒ प्रति॑ मुञ्चते॒ नभो॒ रात्री॑ दे॒वी सूर्य॑स्य व्र॒तानि॑। वि प॑श्यन्ति प॒शवो॒ जाय॑माना॒ नाना॑रूपा मा॒तुर॒स्या उ॒पस्थे᳚। ए॒का॒ष्ट॒का तप॑सा॒ तप्य॑माना ज॒जान॒ गर्भ॑म्महि॒मान॒मिन्द्रम्᳚। तेन॒ दस्यू॒न्व्य॑सहन्त दे॒वा ह॒न्तासु॑राणामभव॒च्छची॑भिः। अना॑नुजामनु॒जाम्माम॑कर्त स॒त्यं वद॒न्त्यन्वि॑च्छ ए॒तत्। भू॒यासम्᳚~(२३)

%4.3.11.4
अ॒स्य॒ सु॒म॒तौ यथा॑ यू॒यम॒न्या वो॑ अ॒न्यामति॒ मा प्र यु॑क्त। अभू॒न्मम॑ सुम॒तौ वि॒श्ववे॑दा॒ आष्ट॑ प्रति॒ष्ठामवि॑द॒द्धि गा॒धम्। भू॒यास॑मस्य सुम॒तौ यथा॑ यू॒यम॒न्या वो॑ अ॒न्यामति॒ मा प्र यु॑क्त। पञ्च॒ व्यु॑ष्टी॒रनु॒ पञ्च॒ दोहा॒ गां पञ्च॑नाम्नीमृ॒तवो\-ऽनु॒ पञ्च॑। पञ्च॒ दिशः॑ पञ्चद॒शेन॒ कॢ॒प्ताः स॑मा॒नमू᳚र्ध्नीर॒भि लो॒कमेकम्᳚~(२४)

%4.3.11.5
ऋ॒तस्य॒ गर्भः॑ प्रथ॒मा व्यू॒षुष्य॒पामेका॑ महि॒मान॑म्बिभर्ति। सूर्य॒स्यैका॒ चर॑ति निष्कृ॒तेषु॑ घ॒र्मस्यैका॑ सवि॒तैकां॒ नि य॑च्छति। या प्र॑थ॒मा व्यौच्छ॒थ्सा धे॒नुर॑भवद्य॒मे। सा नः॒ पय॑स्वती धु॒क्ष्वोत्त॑रामुत्तरा॒ꣳ॒ समा᳚म्। शु॒क्रर्\mbox{}ष॑भा॒ नभ॑सा॒ ज्योति॒षागा᳚द्वि॒श्वरू॑पा शब॒लीर॒ग्निके॑तुः। स॒मा॒नमर्थꣴ॑ स्वप॒स्यमा॑ना॒ बिभ्र॑ती ज॒राम॑जर उष॒ आगाः᳚। ऋ॒तू॒नाम्पत्नी᳚ प्रथ॒मेयमागा॒दह्नां᳚ ने॒त्री ज॑नि॒त्री प्र॒जाना᳚म्। एका॑ स॒ती ब॑हु॒धोषो॒ व्यु॑च्छ॒स्यजी᳚र्णा॒ त्वं ज॑रयसि॒ सर्व॑म॒न्यत्॥~(२५)

%4.3.12.0
{\anuvakamend[{ऊर्ज॒मेका᳚ प्रतिमु॒ञ्चमा॑ना भू॒यास॒मेकं॒ पत्न्येका॒न्नविꣳ॑श॒तिश्च॑}]}%॥11॥

%4.3.12.1
अग्ने॑ जा॒तान्प्र णु॑दा नः स॒पत्ना॒न्प्रत्यजा॑ताञ्जातवेदो नुदस्व। अ॒स्मे दी॑दिहि सु॒मना॒ अहे॑ड॒न्तव॑ स्या॒ꣳ॒ शर्म॑न्त्रि॒वरू॑थ उ॒द्भित्। सह॑सा जा॒तान्प्र णु॑दा नः स॒पत्ना॒न्प्रत्यजा॑ताञ्जातवेदो नुदस्व। अधि॑ नो ब्रूहि सुमन॒स्यमा॑नो व॒यꣴ स्या॑म॒ प्र णु॑दा नः स॒पत्नान्॑। च॒तु॒श्च॒त्वा॒रि॒ꣳ॒शः स्तोमो॒ वर्चो॒ द्रवि॑णꣳ षोड॒शः स्तोम॒ ओजो॒ द्रवि॑णं पृथि॒व्याः पुरी॑षमसि~(२६)

%4.3.12.2
अफ्सो॒ नाम॑। एव॒श्छन्दो॒ वरि॑व॒श्छन्दः॑ श॒म्भूश्छन्दः॑ परि॒भूश्छन्द॑ आ॒च्छच्छन्दो॒ मन॒श्छन्दो॒ व्यच॒श्छन्दः॒ सिन्धु॒श्छन्दः॑ समु॒द्रं छन्दः॑ सलि॒लं छन्दः॑ सं॒यच्छन्दो॑ वि॒यच्छन्दो॑ बृ॒हच्छन्दो॑ रथन्त॒रं छन्दो॑ निका॒यश्छन्दो॑ विव॒धश्छन्दो॒ गिर॒श्छन्दो॒ भ्रज॒श्छन्दः॑ स॒ष्टुप्छन्दो॑\-ऽनु॒ष्टुप्छन्दः॑ क॒कुच्छन्द॑स्त्रिक॒कुच्छन्दः॑ का॒व्यं छन्दो᳚\-ऽङ्कु॒पं छन्दः॑~(२७)

%4.3.12.3
प॒दप॑ङ्क्ति॒श्छन्दो॒\-ऽक्षर॑पङ्क्ति॒श्छन्दो॑ विष्टा॒रप॑ङ्क्ति॒श्छन्दः॑ क्षु॒रो भृज्वा॒ञ्छन्दः॑ प्र॒च्छच्छन्दः॑ प॒क्षश्छन्द॒ एव॒श्छन्दो॒ वरि॑व॒श्छन्दो॒ वय॒श्छन्दो॑ वय॒स्कृच्छन्दो॑ विशा॒लं छन्दो॒ विष्प॑र्धा॒श्छन्द॑श्छ॒दिश्छन्दो॑ दूरोह॒णं छन्द॑स्त॒न्द्रं छन्दो᳚\-ऽङ्का॒ङ्कं छन्दः॑॥~(२८)

%4.3.13.0
{\anuvakamend[{अ॒स्य॒ङ्कु॒पञ्छन्द॒स्त्रय॑स्त्रिꣳशच्च}]}%॥12॥

%4.3.13.1
अ॒ग्निर्वृ॒त्राणि॑ जङ्घनद्द्रविण॒स्युर्वि॑प॒न्यया᳚। समि॑द्धः शु॒क्र आहु॑तः॥ त्वꣳ सो॑मासि॒ सत्प॑ति॒स्त्वꣳ राजो॒त वृ॑त्र॒हा। त्वं भ॒द्रो अ॑सि॒ क्रतुः॑॥ भ॒द्रा ते॑ अग्ने स्वनीक स॒न्दृग्घो॒रस्य॑ स॒तो विषु॑णस्य॒ चारुः॑। न यत्ते॑ शो॒चिस्तम॑सा॒ वर॑न्त॒ न ध्व॒स्मान॑स्त॒नुवि॒ रेप॒ आ धुः॑॥ भ॒द्रं ते॑ अग्ने सहसि॒न्ननी॑कमुपा॒क आ रो॑चते॒ सूर्य॑स्य।~(२९)

%4.3.13.2
रुश॑द्दृ॒शे द॑दृशे नक्त॒या चि॒दरू᳚क्षितं दृ॒श आ रू॒पे अन्नम्᳚। सैनानी॑केन सुवि॒दत्रो॑ अ॒स्मे यष्टा॑ दे॒वाꣳ आय॑जिष्ठः स्व॒स्ति। अद॑ब्धो गो॒पा उ॒त नः॑ पर॒स्पा अग्ने᳚ द्यु॒मदु॒त रे॒वद्दि॑दीहि। स्व॒स्ति नो॑ दि॒वो अ॑ग्ने पृथि॒व्या वि॒श्वायु॑र्धेहि य॒जथा॑य देव। यथ्सी॒महि॑ दिविजात॒ प्रश॑स्तं॒ तद॒स्मासु॒ द्रवि॑णं धेहि चि॒त्रम्। यथा॑ होत॒र्मनु॑षः~(३०)

%4.3.13.3
दे॒वता॑ता य॒ज्ञेभिः॑ सूनो सहसो॒ यजा॑सि। ए॒वानो॑ अ॒द्य स॑म॒ना स॑मा॒नानु॒शन्न॑ग्न उश॒तो य॑क्षि दे॒वान्॥ अ॒ग्निमी॑डे पु॒रोहि॑तं य॒ज्ञस्य॑ दे॒वमृ॒त्विजम्᳚। होता॑रꣳ रत्न॒धात॑मम्॥ वृषा॑ सोम द्यु॒माꣳ अ॑सि॒ वृषा॑ देव॒ वृष॑व्रतः। वृषा॒ धर्मा॑णि दधिषे॥ सान्त॑पना इ॒दꣳ ह॒विर्मरु॑त॒स्तज्जु॑जुष्टन। यु॒ष्माको॒ती रि॑शादसः॥ यो नो॒ मर्तो॑ वसवो दुर्\mbox{}हृणा॒युस्ति॒रः स॒त्यानि॑ मरुतः~(३१)

%4.3.13.4
जिघाꣳ॑सात्। द्रु॒हः पाशं॒ प्रति॒ स मु॑चीष्ट॒ तपि॑ष्ठेन॒ तप॑सा हन्तना॒ तम्। सं॒व॒थ्स॒रीणा॑ म॒रुतः॑ स्व॒र्का उ॑रु॒क्षयाः॒ सग॑णा॒ मानु॑षेषु। ते᳚\-ऽस्मत्पाशा॒न्प्र मु॑ञ्च॒न्त्वꣳह॑सः सान्तप॒ना म॑दि॒रा मा॑दयि॒ष्णवः॑। पि॒प्री॒हि दे॒वाꣳ उ॑श॒तो य॑विष्ठ वि॒द्वाꣳ ऋ॒तूꣳर्\mbox{}ऋ॑तुपते यजे॒ह। ये दैव्या॑ ऋ॒त्विज॒स्तेभि॑रग्ने॒ त्वꣳ होतॄ॑णाम॒स्याय॑जिष्ठः। अग्ने॒ यद॒द्य वि॒शो अ॑ध्वरस्य होतः॒ पाव॑क~(३२)

%4.3.13.5
शो॒चे॒ वेष्ट्वꣳ हि यज्वा᳚। ऋ॒ता य॑जासि महि॒ना वि यद्भूर्\mbox{}ह॒व्या व॑ह यविष्ठ॒ या ते॑ अ॒द्य। अ॒ग्निना॑ र॒यिम॑श्नव॒त्पोष॑मे॒व दि॒वेदि॑वे। य॒शसं॑ वी॒रव॑त्तमम्॥ ग॒य॒स्फानो॑ अमीव॒हा व॑सु॒वित्पु॑ष्टि॒वर्ध॑नः। सु॒मि॒त्रः सो॑म नो भव। गृह॑मेधास॒ आ ग॑त॒ मरु॑तो॒ माप॑ भूतन। प्र॒मु॒ञ्चन्तो॑ नो॒ अꣳह॑सः। पू॒र्वीभि॒र्\mbox{}हि द॑दाशि॒म श॒रद्भि॑र्मरुतो व॒यम्। महो॑भिः~(३३)

%4.3.13.6
च॒र्\mbox{}ष॒णी॒नाम्। प्र बु॒ध्निया॑ ईरते वो॒ महाꣳ॑सि॒ प्र णामा॑नि प्रयज्यवस्तिरध्वम्। स॒ह॒स्रियं॒ दम्य॑म्भा॒गमे॒तं गृ॑हमे॒धीय॑म्मरुतो जुषध्वम्। उप॒ यमेति॑ युव॒तिः सु॒दक्षं॑ दो॒षा वस्तोर्\mbox{}॑ह॒विष्म॑ती घृ॒ताची᳚। उप॒ स्वैन॑म॒रम॑तिर्वसू॒युः। इ॒मो अ॑ग्ने वी॒तत॑मानि ह॒व्याज॑स्रो वक्षि दे॒वता॑ति॒मच्छ॑। प्रति॑ न ईꣳ सुर॒भीणि॑ वियन्तु। क्री॒डं वः॒ शर्धो॒ मारु॑तमन॒र्वाणꣳ॑ रथे॒शुभम्᳚।~(३४)

%4.3.13.7
कण्वा॑ अ॒भि प्र गा॑यत। अत्या॑सो॒ न ये म॒रुतः॒ स्वञ्चो॑ यक्ष॒दृशो॒ न शु॒भय॑न्त॒ मर्याः᳚। ते ह॑र्म्ये॒ष्ठाः शिश॑वो॒ न शु॒भ्रा व॒थ्सासो॒ न प्र॑क्री॒डिनः॑ पयो॒धाः। प्रैषा॒मज्मे॑षु विथु॒रेव॑ रेजते॒ भूमि॒र्यामे॑षु॒ यद्ध॑ यु॒ञ्जते॑ शु॒भे। ते क्री॒डयो॒ धुन॑यो॒ भ्राज॑दृष्टयः स्व॒यं म॑हि॒त्वं प॑नयन्त॒ धूत॑यः। उ॒प॒ह्व॒रेषु॒ यदचि॑ध्वं य॒यिं वय॑ इव मरुतः॒ केन॑~(३५)

%4.3.13.8
चि॒त्प॒था। श्चोत॑न्ति॒ कोशा॒ उप॑ वो॒ रथे॒ष्वा घृ॒तमु॑क्षता॒ मधु॑वर्ण॒मर्च॑ते। अ॒ग्निम॑ग्नि॒ꣳ॒ हवी॑मभिः॒ सदा॑ हवन्त वि॒श्पतिम्᳚। ह॒व्य॒वाहं॑ पुरुप्रि॒यम्। तꣳ हि शश्व॑न्त॒ ईड॑ते स्रु॒चा दे॒वं घृ॑त॒श्चुता᳚। अ॒ग्निꣳ ह॒व्याय॒ वोढ॑वे। इन्द्रा᳚ग्नी रोच॒ना दि॒वः श्नथ॑द्वृ॒त्रमिन्द्रं॑ वो वि॒श्वत॒स्परीन्द्रं॒ नरो॒ विश्व॑कर्मन् ह॒विषा॑ वावृधा॒नो विश्व॑कर्मन् ह॒विषा॒ वर्ध॑नेन॥~(३६)

%4.4.0.0
{\anuvakamend[{सूर्य॑स्य॒ मनु॑षो मरुतः॒ पाव॑क॒ महो॑भी रथे॒शुभं॒ केन॒ षट्च॑त्वारिꣳशच्च}]}%॥13॥

%4.4.0.0

{\anuvakamend[{र॒श्मिर॑सि॒ राज्ञ्य॑स्य॒यं पु॒रो हरि॑केशो॒\-ऽग्निर्मू॒र्धेन्द्रा॒ग्निभ्यां॒ बृह॒स्पति॑र्भूय॒स्कृद॑स्य॒ग्निना॑ विश्वा॒षाट्प्र॒जाप॑ति॒र्मन॑सा॒ कृत्ति॑का॒ मधु॑श्च स॒मिद्दि॒शां द्वाद॑श}]}%॥12॥
\prashnaend{ र॒श्मिर॑सि॒ प्रति॑ धे॒नुम॑सि स्तनयित्नु॒सनि॑रस्यादि॒त्यानाꣳ॑ स॒प्तत्रिꣳ॑शत्॥37॥ र॒श्मिर॑सि॒ को अ॒द्य यु॑ङ्क्ते॥}
%%% END PRASHNA

\sect{चतुर्थः प्रश्नः}\setcounter{anuvakam}{0}
\dnsub{तैत्तिरीयसंहितायां चतुर्थकाण्डे चतुर्थः प्रश्नः}
%4.4.1.0
%4.4.1.1
र॒श्मिर॑सि॒ क्षया॑य त्वा॒ क्षयं॑ जिन्व॒ प्रेति॑रसि॒ धर्मा॑य त्वा॒ धर्मं॑ जि॒न्वान्वि॑तिरसि दि॒वे त्वा॒ दिवं॑ जिन्व सं॒धिर॑स्य॒न्तरि॑क्षाय त्वा॒न्तरि॑क्षं जिन्व प्रति॒धिर॑सि पृथि॒व्यै त्वा॑ पृथि॒वीं जि॑न्व विष्ट॒म्भो॑\-ऽसि॒ वृष्ट्यै᳚ त्वा॒ वृष्टिं॑ जिन्व प्र॒वास्यह्ने॒ त्वाह॑र्जिन्वानु॒वासि॒ रात्रि॑यै त्वा॒ रात्रिं॑ जिन्वो॒शिग॑सि~(१)

%4.4.1.2
वसु॑भ्यस्त्वा॒ वसू᳚ञ्जिन्व प्रके॒तो॑\-ऽसि रु॒द्रेभ्य॑स्त्वा रु॒द्राञ्जि॑न्व सुदी॒तिर॑स्यादि॒त्येभ्य॑स्त्वा\-ऽ\-ऽदि॒त्याञ्जि॒न्वौजो॑\-ऽसि पि॒तृभ्य॑स्त्वा पि॒तॄञ्जि॑न्व॒ तन्तु॑रसि प्र॒जाभ्य॑स्त्वा प्र॒जा जि॑न्व पृतना॒षाड॑सि प॒शुभ्य॑स्त्वा प॒शूञ्जि॑न्व रे॒वद॒स्योष॑धीभ्य॒स्त्वौष॑धीर्जिन्वाभि॒जिद॑सि यु॒क्तग्रा॒वेन्द्रा॑य॒ त्वेन्द्रं॑ जि॒न्वाधि॑पतिरसि प्रा॒णाय॑~(२)

%4.4.1.3
त्वा॒ प्रा॒णं जि॑न्व य॒न्तास्य॑पा॒नाय॑ त्वापा॒नं जि॑न्व स॒ꣳ॒सर्पो॑\-ऽसि॒ चक्षु॑षे त्वा॒ चक्षु॑र्जिन्व वयो॒धा अ॑सि॒ श्रोत्रा॑य त्वा॒ श्रोत्रं॑ जिन्व त्रि॒वृद॑सि प्र॒वृद॑सि सं॒वृद॑सि वि॒वृद॑सि सꣳरो॒हो॑\-ऽसि नीरो॒हो॑\-ऽसि प्ररो॒हो᳚\-ऽस्यनुरो॒हो॑\-ऽसि वसु॒को॑\-ऽसि॒ वेष॑श्रिरसि॒ वस्य॑ष्टिरसि॥~(३)

%4.4.2.0
{\anuvakamend[{उ॒शिग॑सि प्रा॒णाय॒ त्रिच॑त्वारिꣳशच्च}]}%~(१)

%4.4.2.1
राज्ञ्य॑सि॒ प्राची॒ दिग्वस॑वस्ते दे॒वा अधि॑पतयो॒\-ऽग्निर्\mbox{}हे॑ती॒नाम्प्र॑तिध॒र्ता त्रि॒वृत्त्वा॒ स्तोमः॑ पृथि॒व्याꣴ श्र॑य॒त्वाज्य॑मु॒क्थ\-मव्य॑थयथ्स्तभ्नातु रथन्त॒रꣳ साम॒ प्रति॑ष्ठित्यै वि॒राड॑सि दक्षि॒णा दिग्रु॒द्रास्ते॑ दे॒वा अधि॑पतय॒ इन्द्रो॑ हेती॒नाम्प्र॑तिध॒र्ता प॑ञ्चद॒शस्त्वा॒ स्तोमः॑ पृथि॒व्याꣴ श्र॑यतु॒ प्रउ॑गमु॒क्थमव्य॑थयथ्स्तभ्नातु बृ॒हथ्साम॒ प्रति॑ष्ठित्यै स॒म्राड॑सि प्र॒तीची॒ दिक्~(४)

%4.4.2.2
आ॒दि॒त्यास्ते॑ दे॒वा अधि॑पतयः॒ सोमो॑ हेती॒नाम्प्र॑तिध॒र्ता स॑प्तद॒शस्त्वा॒ स्तोमः॑ पृथि॒व्याꣴ श्र॑यतु मरुत्व॒तीय॑मु॒क्थ\-मव्य॑थयथ्स्तभ्नातु वैरू॒पꣳ साम॒ प्रति॑ष्ठित्यै स्व॒राड॒स्युदी॑ची॒ दिग्विश्वे॑ ते दे॒वा अधि॑पतयो॒ वरु॑णो हेती॒नाम्प्र॑तिध॒र्तैक॑\-वि॒ꣳ॒शस्त्वा॒ स्तोमः॑ पृथि॒व्याꣴ श्र॑यतु॒ निष्के॑वल्यमु॒क्थमव्य॑थयथ्स्तभ्नातु वैरा॒जꣳ साम॒ प्रति॑ष्ठित्या॒ अधि॑पत्न्यसि बृह॒ती दिङ्म॒रुत॑स्ते दे॒वा अधि॑पतयः~(५)

%4.4.2.3
बृह॒स्पति॑र्\mbox{}हेती॒नाम्प्र॑तिध॒र्ता त्रि॑णवत्रयस्त्रि॒ꣳ॒शौ त्वा॒ स्तोमौ॑ पृथि॒व्याꣴ श्र॑यतां वैश्वदेवाग्निमारु॒ते उ॒क्थे अव्य॑थयन्ती स्तभ्नीताꣳ शाक्वररैव॒ते साम॑नी॒ प्रति॑ष्ठित्या अ॒न्तरि॑क्षा॒यर्\mbox{}ष॑यस्त्वा प्रथम॒जा दे॒वेषु॑ दि॒वो मात्र॑या वरि॒णा प्र॑थन्तु विध॒र्ता चा॒यमधि॑पतिश्च॒ ते त्वा॒ सर्वे॑ संविदा॒ना नाक॑स्य पृ॒ष्ठे सु॑व॒र्गे लो॒के यज॑मानं च सादयन्तु॥~(६)

%4.4.3.0
{\anuvakamend[{प्र॒तीची॒ दिङ्म॒रुत॑स्ते दे॒वा अधि॑पतयश्चत्वारि॒ꣳ॒शच्च॑}]}%~(२)

%4.4.3.1
अ॒यम्पु॒रो हरि॑केशः॒ सूर्य॑रश्मि॒स्तस्य॑ रथगृ॒थ्सश्च॒ रथौ॑जाश्च सेनानिग्राम॒ण्यौ॑ पुञ्जिकस्थ॒ला च॑ कृतस्थ॒ला चा᳚फ्स॒रसौ॑ यातु॒धाना॑ हे॒ती रक्षाꣳ॑सि॒ प्रहे॑तिर॒यं द॑क्षि॒णा वि॒श्वक॑र्मा॒ तस्य॑ रथस्व॒नश्च॒ रथे॑चित्रश्च सेनानिग्राम॒ण्यौ॑ मेन॒का च॑ सहज॒न्या चा᳚फ्स॒रसौ॑ द॒ङ्क्ष्णवः॑ प॒शवो॑ हे॒तिः पौरु॑षेयो व॒धः प्रहे॑तिर॒यम्प॒श्चाद्वि॒श्वव्य॑चा॒स्तस्य॒ रथ॑प्रोत॒श्चास॑मरथश्च सेनानिग्राम॒ण्यौ᳚ प्र॒म्लोच॑न्ती च~(७)

%4.4.3.2
अ॒नु॒म्लोच॑न्ती चाफ्स॒रसौ॑ स॒र्पा हे॒तिर्व्या॒घ्राः प्रहे॑तिर॒यमु॑त्त॒राथ्सं॒यद्व॑सु॒स्तस्य॑ सेन॒जिच्च॑ सु॒षेण॑श्च सेनानिग्राम॒ण्यौ॑ वि॒श्वाची॑ च घृ॒ताची॑ चाफ्स॒रसा॒वापो॑ हे॒तिर्वातः॒ प्रहे॑तिर॒यमु॒पर्य॒र्वाग्व॑सु॒स्तस्य॒ तार्क्ष्य॒श्चारि॑ष्टनेमिश्च सेनानिग्राम॒ण्या॑\-वु॒र्वशी॑ च पू॒र्वचि॑त्तिश्चाफ्स॒रसौ॑ वि॒द्युद्धे॒तिर॑व॒स्फूर्ज॒न्प्रहे॑ति॒स्तेभ्यो॒ नम॒स्ते नो॑ मृडयन्तु॒ ते यम्~(८)

%4.4.3.3
द्वि॒ष्मो यश्च॑ नो॒ द्वेष्टि॒ तं वो॒ जम्भे॑ दधाम्या॒योस्त्वा॒ सद॑ने सादया॒म्यव॑तश्छा॒यायां॒ नमः॑ समु॒द्राय॒ नमः॑ समु॒द्रस्य॒ चक्ष॑से परमे॒ष्ठी त्वा॑ सादयतु दि॒वः पृ॒ष्ठे व्यच॑स्वती॒म्प्रथ॑स्वतीं वि॒भूम॑तीम्प्र॒भूम॑तीं परि॒भूम॑तीं॒ दिवं॑ यच्छ॒ दिवं॑ दृꣳह॒ दिवं॒ मा हिꣳ॑सी॒र्विश्व॑स्मै प्रा॒णाया॑पा॒नाय॑ व्या॒नायो॑दा॒नाय॑ प्रति॒ष्ठायै॑ च॒रित्रा॑य॒ सूर्य॑स्त्वा॒भि पा॑तु म॒ह्या स्व॒स्त्या छ॒र्दिषा॒ शन्त॑मेन॒ तया॑ दे॒वत॑याङ्गिर॒स्वद्ध्रु॒वा सी॑द। प्रोथ॒दश्वो॒ न यव॑से अवि॒ष्यन् य॒दा म॒हः सं॒वर॑णा॒द्व्यस्था᳚त्। आद॑स्य॒ वातो॒ अनु॑ वाति शो॒चिरध॑ स्म ते॒ व्रज॑नं कृ॒ष्णम॑स्ति॥~(९)

%4.4.4.0
{\anuvakamend[{प्र॒म्लोच॑न्ती च॒ यꣴ स्व॒स्त्याष्टाविꣳ॑शतिश्च}]}%~(३)

%4.4.4.1
अ॒ग्निर्मू॒र्धा दि॒वः क॒कुत्पतिः॑ पृथि॒व्या अ॒यम्। अ॒पाꣳ रेताꣳ॑सि जिन्वति॥ त्वाम॑ग्ने॒ पुष्क॑रा॒दध्यथ॑र्वा॒ निर॑मन्थत। मू॒र्ध्नो विश्व॑स्य वा॒घतः॑॥ अ॒यम॒ग्निः स॑ह॒स्रिणो॒ वाज॑स्य श॒तिन॒स्पतिः॑। मू॒र्धा क॒वी र॑यी॒णाम्॥ भुवो॑ य॒ज्ञस्य॒ रज॑सश्च ने॒ता यत्रा॑ नि॒युद्भिः॒ सच॑से शि॒वाभिः॑। दि॒वि मू॒र्धानं॑ दधिषे सुव॒र्\mbox{}षां जि॒ह्वाम॑ग्ने चकृषे हव्य॒वाहम्᳚॥ अबो᳚ध्य॒ग्निः स॒मिधा॒ जना॑नाम्~(१०)

%4.4.4.2
प्रति॑ धे॒नुमि॑वाय॒तीमु॒षासम्᳚। य॒ह्वा इ॑व॒ प्र व॒यामु॒ज्जिहा॑नाः॒ प्र भा॒नवः॑ सिस्रते॒ नाक॒मच्छ॑। अवो॑चाम क॒वये॒ मेध्या॑य॒ वचो॑ व॒न्दारु॑ वृष॒भाय॒ वृष्णे᳚। गवि॑ष्ठिरो॒ नम॑सा॒ स्तोम॑म॒ग्नौ दि॒वीव॑ रु॒क्ममु॒र्व्यञ्च॑मश्रेत्। जन॑स्य गो॒पा अ॑जनिष्ट॒ जागृ॑विर॒ग्निः सु॒दक्षः॑ सुवि॒ताय॒ नव्य॑से। घृ॒तप्र॑तीको बृह॒ता दि॑वि॒स्पृशा᳚ द्यु॒मद्वि भा॑ति भर॒तेभ्यः॒ शुचिः॑। त्वाम॑ग्ने॒ अङ्गि॑रसः~(११)

%4.4.4.3
गुहा॑ हि॒तमन्व॑विन्दञ्छिश्रिया॒णं वने॑वने। स जा॑यसे म॒थ्यमा॑नः॒ सहो॑ म॒हत्त्वामा॑हुः॒ सह॑सस्पु॒त्रम॑ङ्गिरः। य॒ज्ञस्य॑ के॒तुम्प्र॑थ॒मम्पु॒रोहि॑तम॒ग्निं नर॑स्त्रिषध॒स्थे समि॑न्धते। इन्द्रे॑ण दे॒वैः स॒रथ॒ꣳ॒ स ब॒र्\mbox{}हिषि॒ सीद॒न्नि होता॑ य॒जथा॑य सु॒क्रतुः॑। त्वं चि॑त्रश्रवस्तम॒ हव॑न्ते वि॒क्षु ज॒न्तवः॑। शो॒चिष्के॑शं पुरुप्रि॒याग्ने॑ ह॒व्याय॒ वोढ॑वे। सखा॑यः॒ सं वः॑ स॒म्यञ्च॒मिषम्᳚~(१२)

%4.4.4.4
स्तोमं॑ चा॒ग्नये᳚। वर्\mbox{}षि॑ष्ठाय क्षिती॒नामू॒र्जो नप्त्रे॒ सह॑स्वते। सꣳस॒मिद्यु॑वसे वृष॒न्नग्ने॒ विश्वा᳚न्य॒र्य आ। इ॒डस्प॒दे समि॑ध्यसे॒ स नो॒ वसू॒न्या भ॑र। ए॒ना वो॑ अ॒ग्निं नम॑सो॒र्जो नपा॑त॒मा हु॑वे। प्रि॒यं चेति॑ष्ठमर॒तिꣴ स्व॑ध्व॒रं विश्व॑स्य दू॒तम॒मृतम्᳚। स यो॑जते अरु॒षो वि॒श्वभो॑जसा॒ स दु॑द्रव॒थ्स्वा॑हुतः। सु॒ब्रह्मा॑ य॒ज्ञः सु॒शमी᳚~(१३)

%4.4.4.5
वसू॑नां दे॒वꣳ राधो॒ जना॑नाम्। उद॑स्य शो॒चिर॑स्थादा॒जुह्वा॑नस्य मी॒ढुषः॑। उद्धू॒मासो॑ अरु॒षासो॑ दिवि॒स्पृशः॒ सम॒ग्निमि॑न्धते॒ नरः॑। अग्ने॒ वाज॑स्य॒ गोम॑त॒ ईशा॑नः सहसो यहो। अ॒स्मे धे॑हि जातवेदो॒ महि॒ श्रवः॑। स इ॑धा॒नो वसु॑ष्क॒विर॒ग्निरी॒डेन्यो॑ गि॒रा। रे॒वद॒स्मभ्यं॑ पुर्वणीक दीदिहि। क्ष॒पो रा॑जन्नु॒त त्मनाग्ने॒ वस्तो॑रु॒तोषसः॑। स ति॑ग्मजम्भ~(१४)

%4.4.4.6
र॒क्षसो॑ दह॒ प्रति॑। आ ते॑ अग्न इधीमहि द्यु॒मन्तं॑ देवा॒जरम्᳚। यद्ध॒ स्या ते॒ पनी॑यसी स॒मिद्दी॒दय॑ति॒ द्यवीषꣴ॑ स्तो॒तृभ्य॒ आ भ॑र। आ ते॑ अग्न ऋ॒चा ह॒विः शु॒क्रस्य॑ ज्योतिषस्पते। सुश्च॑न्द्र॒ दस्म॒ विश्प॑ते॒ हव्य॑वा॒ट्तुभ्यꣳ॑ हूयत॒ इषꣴ॑ स्तो॒तृभ्य॒ आ भ॑र। उ॒भे सु॑श्चन्द्र स॒र्पिषो॒ दर्वी᳚ श्रीणीष आ॒सनि॑। उ॒तो न॒ उत्पु॑पूर्याः~(१५)

%4.4.4.7
उ॒क्थेषु॑ शवसस्पत॒ इषꣴ॑ स्तो॒तृभ्य॒ आ भ॑र। अग्ने॒ तम॒द्याश्वं॒ न स्तोमैः॒ क्रतुं॒ न भ॒द्रꣳ हृ॑दि॒स्पृशम्᳚। ऋ॒ध्यामा॑ त॒ ओहैः᳚। अधा॒ ह्य॑ग्ने॒ क्रतो᳚र्भ॒द्रस्य॒ दक्ष॑स्य सा॒धोः। र॒थीर्\mbox{}ऋ॒तस्य॑ बृह॒तो ब॒भूथ॑। आ॒भिष्टे॑ अ॒द्य गी॒र्भिर्गृ॒णन्तो\-ऽग्ने॒ दाशे॑म। प्र ते॑ दि॒वो न स्त॑नयन्ति॒ शुष्माः᳚। ए॒भिर्नो॑ अ॒र्कैर्भवा॑ नो अ॒र्वाङ्~(१६)

%4.4.4.8
सुव॒र्न ज्योतिः॑। अग्ने॒ विश्वे॑भिः सु॒मना॒ अनी॑कैः। अ॒ग्निꣳ होता॑रम्मन्ये॒ दास्व॑न्तं॒ वसोः᳚ सू॒नुꣳ सह॑सो जा॒तवे॑दसम्। विप्रं॒ न जा॒तवे॑दसम्। य ऊ॒र्ध्वया᳚ स्वध्व॒रो दे॒वो दे॒वाच्या॑ कृ॒पा। घृ॒तस्य॒ विभ्रा᳚ष्टि॒मनु॑ शु॒क्रशो॑चिष आ॒जुह्वा॑नस्य स॒र्पिषः॑। अग्ने॒ त्वन्नो॒ अन्त॑मः। उ॒त त्रा॒ता शि॒वो भ॑व वरू॒थ्यः॑। तं त्वा॑ शोचिष्ठ दीदिवः। सु॒म्नाय॑ नू॒नमी॑महे॒ सखि॑भ्यः। वसु॑र॒ग्निर्वसु॑श्रवाः। अच्छा॑ नक्षि द्यु॒मत्त॑मो र॒यिं दाः᳚॥~(१७)

%4.4.5.0
{\anuvakamend[{जना॑ना॒मङ्गि॑रस॒ इषꣳ॑ सु॒शमी॑ तिग्मजम्भ पुपूर्या अ॒र्वाङ्वसु॑श्रवाः॒ पञ़्च॑ च}]}%~(४)

%4.4.5.1
इ॒न्द्रा॒ग्नि\-भ्यां᳚ त्वा स॒युजा॑ यु॒जा यु॑नज्म्याघा॒राभ्यां॒ तेज॑सा॒ वर्च॑सो॒क्थेभिः॒ स्तोमे॑भि॒श्छन्दो॑भी र॒य्यै पोषा॑य सजा॒ताना᳚म्मध्यम॒स्थेया॑य॒ मया᳚ त्वा स॒युजा॑ यु॒जा यु॑नज्म्य॒म्बा दु॒ला नि॑त॒त्निर॒भ्रय॑न्ती मे॒घय॑न्ती व॒र्\mbox{}षय॑न्ती चुपु॒णीका॒ नामा॑सि प्र॒जाप॑तिना त्वा॒ विश्वा॑भिर्धी॒भिरुप॑ दधामि पृथि॒व्यु॑दपु॒रमन्ने॑न वि॒ष्टा म॑नु॒ष्या᳚स्ते गो॒प्तारो॒\-ऽग्निर्विय॑त्तो\-ऽस्यां॒ ताम॒हम्प्र प॑द्ये॒ सा~(१८)

%4.4.5.2
मे॒ शर्म॑ च॒ वर्म॑ चा॒स्त्वधि॑द्यौर॒न्तरि॑क्षं॒ ब्रह्म॑णा वि॒ष्टा म॒रुत॑स्ते गो॒प्तारो॑ वा॒युर्विय॑त्तो\-ऽस्यां॒ ताम॒हम्प्र प॑द्ये॒ सा मे॒ शर्म॑ च॒ वर्म॑ चास्तु॒ द्यौरप॑राजिता॒मृते॑न वि॒ष्टादि॒त्यास्ते॑ गो॒प्तारः॒ सूर्यो॒ विय॑त्तो\-ऽस्यां॒ ताम॒हम्प्र प॑द्ये॒ सा मे॒ शर्म॑ च॒ वर्म॑ चास्तु॥~(१९)

%4.4.6.0
{\anuvakamend[{सा\-ऽष्टाच॑त्वारिꣳशच्च}]}%~(५)

%4.4.6.1
बृह॒स्पति॑स्त्वा सादयतु पृथि॒व्याः पृ॒ष्ठे ज्योति॑ष्मतीं॒ विश्व॑स्मै प्रा॒णाया॑पा॒नाय॒ विश्वं॒ ज्योति॑र्यच्छा॒ग्निस्ते\-ऽधि॑पतिर्\-वि॒श्वक॑र्मा त्वा सादयत्व॒न्तरि॑क्षस्य पृ॒ष्ठे ज्योति॑ष्मतीं॒ विश्व॑स्मै प्रा॒णाया॑पा॒नाय॒ विश्वं॒ ज्योति॑र्यच्छ वा॒युस्ते\-ऽधि॑पतिः प्र॒जाप॑तिस्त्वा सादयतु दि॒वः पृ॒ष्ठे ज्योति॑ष्मतीं॒ विश्व॑स्मै प्रा॒णाया॑पा॒नाय॒ विश्वं॒ ज्योति॑र्यच्छ परमे॒ष्ठी ते\-ऽधि॑पतिः पुरोवात॒सनि॑रस्यभ्र॒सनि॑रसि विद्यु॒थ्सनिः॑~(२०)

%4.4.6.2
अ॒सि॒ स्त॒न॒यि॒त्नु॒सनि॑रसि वृष्टि॒सनि॑रस्य॒ग्नेर्यान्य॑सि दे॒वाना॑मग्ने॒यान्य॑सि वा॒योर्यान्य॑सि दे॒वानां᳚ वायो॒यान्य॑स्य॒न्तरि॑क्षस्य॒ यान्य॑सि दे॒वाना॑मन्तरिक्ष॒यान्य॑स्य॒न्तरि॑क्षमस्य॒न्तरि॑क्षाय त्वा सलि॒लाय॑ त्वा॒ सर्णी॑काय त्वा॒ सती॑काय त्वा॒ केता॑य त्वा॒ प्रचे॑तसे त्वा॒ विव॑स्वते त्वा दि॒वस्त्वा॒ ज्योति॑ष आदि॒त्येभ्य॑स्त्व॒र्चे त्वा॑ रु॒चे त्वा᳚ द्यु॒ते त्वा॑ भा॒से त्वा॒ ज्योति॑षे त्वा यशो॒दां त्वा॒ यश॑सि तेजो॒दां त्वा॒ तेज॑सि पयो॒दां त्वा॒ पय॑सि वर्चो॒दां त्वा॒ वर्च॑सि द्रविणो॒दां त्वा॒ द्रवि॑णे सादयामि॒ तेनर्\mbox{}षि॑णा॒ तेन॒ ब्रह्म॑णा॒ तया॑ दे॒वत॑याङ्गिर॒स्वद्ध्रु॒वा सी॑द॥~(२१)

%4.4.7.0
{\anuvakamend[{वि॒द्यु॒थ्सनि॑र्द्यु॒त्वैका॒न्नत्रि॒ꣳ॒शच्च॑}]}%~(६)

%4.4.7.1
भू॒य॒स्कृद॑सि वरिव॒स्कृद॑सि॒ प्राच्य॑स्यू॒र्ध्वास्य॑न्तरिक्ष॒सद॑स्य॒न्तरि॑क्षे सीदाफ्सु॒षद॑सि श्येन॒सद॑सि गृध्र॒सद॑सि सुपर्ण॒सद॑सि नाक॒सद॑सि पृथि॒व्यास्त्वा॒ द्रवि॑णे सादयाम्य॒न्तरि॑क्षस्य त्वा॒ द्रवि॑णे सादयामि दि॒वस्त्वा॒ द्रवि॑णे सादयामि दि॒शां त्वा॒ द्रवि॑णे सादयामि द्रविणो॒दां त्वा॒ द्रवि॑णे सादयामि प्रा॒णं मे॑ पाह्यपा॒नं मे॑ पाहि व्या॒नम्मे᳚~(२२)

%4.4.7.2
पा॒ह्यायु॑र्मे पाहि वि॒श्वायु॑र्मे पाहि स॒र्वायु॑र्मे पा॒ह्यग्ने॒ यत्ते॒ पर॒ꣳ॒ हृन्नाम॒ तावेहि॒ सꣳ र॑भावहै॒ पाञ्च॑जन्ये॒ष्वप्ये᳚ध्यग्ने॒ यावा॒ अया॑वा॒ एवा॒ ऊमाः॒ सब्दः॒ सग॑रः सु॒मेकः॑॥~(२३)

%4.4.8.0
{\anuvakamend[{व्या॒नम्मे॒ द्वात्रिꣳ॑शच्च}]}%~(७)

%4.4.8.1
अ॒ग्निना॑ विश्वा॒षाट्थ्सूर्ये॑ण स्व॒राट्क्रत्वा॒ शची॒पति॑र्\mbox{}ऋष॒भेण॒ त्वष्टा॑ य॒ज्ञेन॑ म॒घवा॒न्दक्षि॑णया सुव॒र्गो म॒न्युना॑ वृत्र॒हा सौहा᳚र्द्येन तनू॒धा अन्ने॑न॒ गयः॑ पृथि॒व्यास॑नोदृ॒ग्भिर॑न्ना॒दो व॑षट्का॒रेण॒र्द्धः साम्ना॑ तनू॒पा वि॒राजा॒ ज्योति॑ष्मा॒न् ब्रह्म॑णा सोम॒पा गोभि॑र्य॒ज्ञं दा॑धार क्ष॒त्रेण॑ मनु॒ष्या॑नश्वे॑न च॒ रथे॑न च व॒ज्र्यृ॑तुभिः॑ प्र॒भुः सं॑वथ्स॒रेण॑ परि॒भूस्तप॒साना॑धृष्टः॒ सूर्यः॒ सन्त॒नूभिः॑॥~(२४)

%4.4.9.0
{\anuvakamend[{अ॒ग्निनैका॒न्नप॑ञ्चा॒शत्}]}%~(८)

%4.4.9.1
प्र॒जाप॑ति॒र्मन॒सान्धो\-ऽच्छे॑तो धा॒ता दी॒क्षायाꣳ॑ सवि॒ता भृ॒त्यां पू॒षा सो॑म॒क्रय॑ण्यां॒ वरु॑ण॒ उप॑न॒द्धो\-ऽसु॑रः क्री॒यमा॑णो मि॒त्रः क्री॒तः शि॑पिवि॒ष्ट आसा॑दितो न॒रन्धि॑षः प्रो॒ह्यमा॒णो\-ऽधि॑पति॒राग॑तः प्र॒जाप॑तिः प्रणी॒यमा॑नो॒\-ऽग्निराग्नी᳚ध्रे॒ बृह॒स्पति॒राग्नी᳚ध्रात्प्रणी॒यमा॑न॒ इन्द्रो॑ हवि॒र्धाने\-ऽदि॑ति॒रासा॑दितो॒ विष्णु॑रुपावह्रि॒यमा॒णो\-ऽथ॒र्वोपो᳚त्तो य॒मो॑\-ऽभिषु॑तो\-ऽपूत॒पा आ॑धू॒यमा॑नो वा॒युः पू॒यमा॑नो मि॒त्रः क्षी॑र॒श्रीर्म॒न्थी स॑क्तु॒श्रीर्वै᳚श्वदे॒व उन्नी॑तो रुद्र॒ आहु॑तो वा॒युरावृ॑त्तो नृ॒चक्षाः॒ प्रति॑ख्यातो भ॒क्ष आग॑तः पितृ॒णां ना॑राश॒ꣳ॒सो\-ऽसु॒रात्तः॒ सिन्धु॑रवभृ॒थम॑वप्र॒यन्थ्स॑मु॒द्रो\-ऽव॑गतः सलि॒लः प्रप्लु॑तः सुव॑रु॒दृचं॑ ग॒तः॥~(२५)

%4.4.10.0
{\anuvakamend[{रु॒द्र एक॑विꣳशतिश्च}]}%~(८)

%4.4.10.1
कृत्ति॑का॒ नक्ष॑त्रम॒ग्निर्दे॒वता॒ग्ने रुचः॑ स्थ प्र॒जाप॑तेर्धा॒तुः सोम॑स्य॒र्चे त्वा॑ रु॒चे त्वा᳚ द्यु॒ते त्वा॑ भा॒से त्वा॒ ज्योति॑षे त्वा रोहि॒णी नक्ष॑त्रं प्र॒जाप॑तिर्दे॒वता॑ मृगशी॒र्\mbox{}षं नक्ष॑त्र॒ꣳ॒ सोमो॑ दे॒वता॒र्द्रा नक्ष॑त्रꣳ रु॒द्रो दे॒वता॒ पुन॑र्वसू॒ नक्ष॑त्र॒मदि॑ति\-र्दे॒वता॑ ति॒ष्यो॑ नक्ष॑त्र॒म्बृह॒स्पति॑र्दे॒वता᳚श्रे॒षा नक्ष॑त्रꣳ स॒र्पा दे॒वता॑ म॒घा नक्ष॑त्रम्पि॒तरो॑ दे॒वता॒ फल्गु॑नी॒ नक्ष॑त्रम्~(२६)

%4.4.10.2
अ॒र्य॒मा दे॒वता॒ फल्गु॑नी॒ नक्ष॑त्र॒म्भगो॑ दे॒वता॒ हस्तो॒ नक्ष॑त्रꣳ सवि॒ता दे॒वता॑ चि॒त्रा नक्ष॑त्र॒मिन्द्रो॑ दे॒वता᳚ स्वा॒ती नक्ष॑त्रं वा॒युर्दे॒वता॒ विशा॑खे॒ नक्ष॑त्रमिन्द्रा॒ग्नी दे॒वता॑\-ऽनूरा॒धा नक्ष॑त्रम्मि॒त्रो दे॒वता॑ रोहि॒णी नक्ष॑त्र॒मिन्द्रो॑ दे॒वता॑ वि॒चृतौ॒ नक्ष॑त्रम्पि॒तरो॑ दे॒वता॑षा॒ढा नक्ष॑त्र॒मापो॑ दे॒वता॑षा॒ढा नक्ष॑त्रं॒ विश्वे॑ दे॒वा दे॒वता᳚ श्रो॒णा नक्ष॑त्त्रं॒ विष्णु॑र्दे॒वता॒ श्रवि॑ष्ठा॒ नक्ष॑त्रं॒ वस॑वः~(२७)

%4.4.10.3
दे॒वता॑ श॒तभि॑ष॒ङ्नक्ष॑त्र॒मिन्द्रो॑ दे॒वता᳚ प्रोष्ठप॒दा नक्ष॑त्रम॒ज एक॑पाद्दे॒वता᳚ प्रोष्ठप॒दा नक्ष॑त्र॒महि॑र्बु॒ध्नियो॑ दे॒वता॑ रे॒वती॒ नक्ष॑त्रं पू॒षा दे॒वता᳚श्व॒युजौ॒ नक्ष॑त्रम॒श्विनौ॑ दे॒वता॑प॒भर॑णी॒र्नक्ष॑त्रं य॒मो दे॒वता॑ पू॒र्णा प॒श्चाद्यत्ते॑ दे॒वा अद॑धुः॥~(२८)

%4.4.11.0
{\anuvakamend[{फल्गु॑नी॒ नक्ष॑त्रं॒ वस॑व॒स्त्रय॑स्त्रिꣳशच्च}]}%॥10॥

%4.4.11.1
मधु॑श्च॒ माध॑वश्च॒ वास॑न्तिकावृ॒तू शु॒क्रश्च॒ शुचि॑श्च॒ ग्रैष्मा॑वृ॒तू नभ॑श्च नभ॒स्य॑श्च॒ वार्\mbox{}षि॑कावृ॒तू इ॒षश्चो॒र्जश्च॑ शार॒दावृ॒तू सह॑श्च सह॒स्य॑श्च॒ हैम॑न्तिकावृ॒तू तप॑श्च तप॒स्य॑श्च शैशि॒रावृ॒तू अ॒ग्नेर॑न्तःश्ले॒षो॑\-ऽसि॒ कल्पे॑तां॒ द्यावा॑पृथि॒वी कल्प॑न्ता॒माप॒ ओष॑धीः॒ कल्प॑न्ताम॒ग्नयः॒ पृथ॒ङ्मम॒ ज्यैष्ठ्या॑य॒ सव्र॑ताः~(२९)

%4.4.11.2
ये᳚\-ऽग्नयः॒ सम॑नसो\-ऽन्त॒रा द्यावा॑पृथि॒वी शै॑शि॒रावृ॒तू अ॒भि कल्प॑माना॒ इन्द्र॑मिव दे॒वा अ॒भि सं वि॑शन्तु सं॒यच्च॒ प्रचे॑ताश्चा॒ग्नेः सोम॑स्य॒ सूर्य॑स्यो॒ग्रा च॑ भी॒मा च॑ पितृ॒णां य॒मस्येन्द्र॑स्य ध्रु॒वा च॑ पृथि॒वी च॑ दे॒वस्य॑ सवि॒तुर्म॒रुतां॒ वरु॑णस्य ध॒र्त्री च॒ धरि॑त्री च मि॒त्रावरु॑णयोर्मि॒त्रस्य॑ धा॒तुः प्राची॑ च प्र॒तीची॑ च॒ वसू॑नाꣳ रु॒द्राणा᳚म्~(३०)

%4.4.11.3
आ॒दि॒त्याना॒न्ते ते\-ऽधि॑पतय॒स्तेभ्यो॒ नम॒स्ते नो॑ मृडयन्तु॒ ते यं द्वि॒ष्मो यश्च॑ नो॒ द्वेष्टि॒ तं वो॒ जम्भे॑ दधामि स॒हस्र॑स्य प्र॒मा अ॑सि स॒हस्र॑स्य प्रति॒मा अ॑सि स॒हस्र॑स्य वि॒मा अ॑सि स॒हस्र॑स्यो॒न्मा अ॑सि साह॒स्रो॑\-ऽसि स॒हस्रा॑य त्वे॒मा मे॑ अग्न॒ इष्ट॑का धे॒नवः॑ स॒न्त्वेका॑ च श॒तं च॑ स॒हस्रं॑ चा॒युतं॑ च~(३१)

%4.4.11.4
नि॒युतं॑ च प्र॒युतं॒ चार्बु॑दं च॒ न्य॑र्बुदं च समु॒द्रश्च॒ मध्यं॒ चान्त॑श्च परा॒र्धश्चे॒मा मे॑ अग्न॒ इष्ट॑का धे॒नवः॑ सन्तु ष॒ष्टिः स॒हस्र॑म॒युत॒मक्षी॑यमाणा ऋत॒स्थाः स्थ॑र्ता॒वृधो॑ घृत॒श्चुतो॑ मधु॒श्चुत॒ ऊर्ज॑स्वतीः स्वधा॒विनी॒स्ता मे॑ अग्न॒ इष्ट॑का धे॒नवः॑ सन्तु वि॒राजो॒ नाम॑ काम॒दुघा॑ अ॒मुत्रा॒मुष्मि॑ल्लोँ॒के॥~(३२)

%4.4.12.0
{\anuvakamend[{सव्र॑ता रु॒द्राणा॑म॒युतं॑ च॒ पञ्च॑चत्वारिꣳशच्च}]}%॥11॥

%4.4.12.1
स॒मिद्दि॒शामा॒शया॑ नः सुव॒र्विन्मधो॒रतो॒ माध॑वः पात्व॒स्मान्। अ॒ग्निर्दे॒वो दु॒ष्टरी॑तु॒रदा᳚भ्य इ॒दं क्ष॒त्रꣳ र॑क्षतु॒ पात्व॒स्मान्। र॒थं॒त॒रꣳ साम॑भिः पात्व॒स्मान्गा॑य॒त्री छन्द॑सां वि॒श्वरू॑पा। त्रि॒वृन्नो॑ वि॒ष्ठया॒ स्तोमो॒ अह्नाꣳ॑ समु॒द्रो वात॑ इ॒दमोजः॑ पिपर्तु। उ॒ग्रा दि॒शाम॒भिभू॑तिर्वयो॒धाः शुचिः॑ शु॒क्रे अह॑न्योज॒सीना᳚। इन्द्राधि॑पतिः पिपृता॒दतो॑ नो॒ महि॑~(३३)

%4.4.12.2
क्ष॒त्रं वि॒श्वतो॑ धारये॒दम्। बृ॒हथ्साम॑ क्षत्र॒भृद्वृ॒द्धवृ॑ष्णियं त्रि॒ष्टुभौजः॑ शुभि॒तमु॒ग्रवी॑रम्। इन्द्र॒ स्तोमे॑न पञ्चद॒शेन॒ मध्य॑मि॒दं वाते॑न॒ सग॑रेण रक्ष। प्राची॑ दि॒शाꣳ स॒हय॑शा॒ यश॑स्वती॒ विश्वे॑ देवाः प्रा॒वृषाह्ना॒ꣳ॒ सुव॑र्वती। इ॒दं क्ष॒त्रं दु॒ष्टर॑म॒स्त्वोजो\-ऽना॑धृष्टꣳ सह॒स्रिय॒ꣳ॒ सह॑स्वत्। वै॒रू॒पे साम॑न्नि॒ह तच्छ॑केम॒ जग॑त्यैनं वि॒क्ष्वा वे॑शयामः। विश्वे॑ देवाः सप्तद॒शेन॑~(३४)

%4.4.12.3
वर्च॑ इ॒दं क्ष॒त्रꣳ स॑लि॒लवा॑तमु॒ग्रम्। ध॒र्त्री दि॒शां क्ष॒त्रमि॒दं दा॑धारोप॒स्थाशा॑नाम्मि॒त्रव॑द॒स्त्वोजः॑। मित्रा॑वरुणा श॒रदाह्नां᳚ चिकित्नू अ॒स्मै रा॒ष्ट्राय॒ महि॒ शर्म॑ यच्छतम्। वै॒रा॒जे साम॒न्नधि॑ मे मनी॒षानु॒ष्टुभा॒ सम्भृ॑तं वी॒र्यꣳ॑ सहः॑। इ॒दं क्ष॒त्रम्मि॒त्रव॑दा॒र्द्रदा॑नु॒ मित्रा॑वरुणा॒ रक्ष॑त॒माधि॑पत्यैः। स॒म्राड्दि॒शाꣳ स॒हसा᳚म्नी॒ सह॑स्वत्यृ॒तुर्\mbox{}हे॑म॒न्तो वि॒ष्ठया॑ नः पिपर्तु। अ॒व॒स्युवा॑ताः~(३५)

%4.4.12.4
बृ॒ह॒तीर्नु शक्व॑रीरि॒मं य॒ज्ञम॑वन्तु नो घृ॒ताचीः᳚। सुव॑र्वती सु॒दुघा॑ नः॒ पय॑स्वती दि॒शां दे॒व्य॑वतु नो घृ॒ताची᳚। त्वं गो॒पाः पु॑रए॒तोत प॒श्चाद्बृह॑स्पते॒ याम्यां᳚ युङ्ग्धि॒ वाचम्᳚। ऊ॒र्ध्वा दि॒शाꣳ रन्ति॒राशौष॑धीनाꣳ संवथ्स॒रेण॑ सवि॒ता नो॒ अह्ना᳚म्। रे॒वथ्सामाति॑च्छन्दा उ॒ छन्दोजा॑तशत्रुः स्यो॒ना नो॑ अस्तु। स्तोम॑त्रयस्त्रिꣳशे॒ भुव॑नस्य पत्नि॒ विव॑स्वद्वाते अ॒भि नः॑~(३६)

%4.4.12.5
गृ॒णा॒हि॒। घृ॒तव॑ती सवित॒राधि॑पत्यैः॒ पय॑स्वती॒ रन्ति॒राशा॑ नो अस्तु। ध्रु॒वा दि॒शां वि॑ष्णुप॒त्न्यघो॑रा॒स्येशा॑ना॒ सह॑सो॒ या म॒नोता᳚। बृह॒स्पति॑र्मात॒रिश्वो॒त वा॒युः सं॑धुवा॒ना वाता॑ अ॒भि नो॑ गृणन्तु। वि॒ष्ट॒म्भो दि॒वो ध॒रुणः॑ पृथि॒व्या अ॒स्येशा॑ना॒ जग॑तो॒ विष्णु॑पत्नी। वि॒श्वव्य॑चा इ॒षय॑न्ती॒ सुभू॑तिः शि॒वा नो॑ अ॒स्त्वदि॑तिरु॒पस्थे᳚। वै॒श्वा॒न॒रो न॑ ऊ॒त्या पृ॒ष्टो दि॒व्यनु॑ नो॒\-ऽद्यानु॑मति॒रन्विद॑नुमते॒ त्वङ्कया॑ नश्चि॒त्र आ भु॑व॒त्को अ॒द्य यु॑ङ्क्ते॥~(३७)

%4.5.0.0
{\anuvakamend[{महि॑ सप्तद॒शेना॑व॒स्युवा॑ता अ॒भि नो\-ऽनु॑ न॒श्चतु॑र्दश च}]}%॥12॥

%4.5.0.0

{\anuvakamend[{नम॑स्ते रुद्र॒ नमो॒ हिर॑ण्यबाहवे॒ नमः॒ सह॑मानाय॒ नम॑ आव्या॒धिनी᳚भ्यो॒ नमो॑ भ॒वाय॒ नमो᳚ ज्ये॒ष्ठाय॒ नमो॑ दुन्दु॒भ्या॑य॒ नमः॒ सोमा॑य॒ नम॑ इरि॒ण्या॑य॒ द्रापे॑ स॒हस्रा॒ण्येका॑\-दश}]}%॥11॥
\prashnaend{ नम॑स्ते रुद्र॒ नमो॑ भ॒वाय॒ द्रापे॑ स॒प्तविꣳ॑शतिः॥27॥ नम॑स्ते रुद्र॒ तं वो॒ जम्भे॑ दधामि॥}
%%% END PRASHNA

\sect{पञ्चमः प्रश्नः}\setcounter{anuvakam}{0}
\dnsub{तैत्तिरीयसंहितायां चतुर्थकाण्डे पञ्चमः प्रश्नः}
%4.5.1.0
%4.5.1.1
नम॑स्ते रुद्र म॒न्यव॑ उ॒तो त॒ इष॑वे॒ नमः॑। नम॑स्ते अस्तु॒ धन्व॑ने बा॒हुभ्या॑मु॒त ते॒ नमः॑। या त॒ इषुः॑ शि॒वत॑मा शि॒वम्ब॒भूव॑ ते॒ धनुः॑। शि॒वा श॑र॒व्या॑ या तव॒ तया॑ नो रुद्र मृडय। या ते॑ रुद्र शि॒वा त॒नूरघो॒रापा॑पकाशिनी। तया॑ नस्त॒नुवा॒ शन्त॑मया॒ गिरि॑शन्ता॒भि चा॑कशीहि। यामिषुं॑ गिरिशन्त॒ हस्ते᳚~(१)

%4.5.1.2
बिभ॒र्ष्यस्त॑वे। शि॒वां गि॑रित्र॒ तां कु॑रु॒ मा हिꣳ॑सीः॒ पुरु॑षं॒ जग॑त्। शि॒वेन॒ वच॑सा त्वा॒ गिरि॒शाच्छा॑ वदामसि। यथा॑ नः॒ सर्व॒मिज्जग॑दय॒क्ष्मꣳ सु॒मना॒ अस॑त्। अध्य॑वोचदधिव॒क्ता प्र॑थ॒मो दैव्यो॑ भि॒षक्। अहीꣴ॑श्च॒ सर्वा᳚ञ्ज॒म्भय॒न्थ्सर्वा᳚श्च यातुधा॒न्यः॑। अ॒सौ यस्ता॒म्रो अ॑रु॒ण उ॒त ब॒भ्रुः सु॑म॒ङ्गलः॑। ये चे॒माꣳ रु॒द्रा अ॒भितो॑ दि॒क्षु~(२)

%4.5.1.3
श्रि॒ताः स॑हस्र॒शो\-ऽवै॑षा॒ꣳ॒ हेड॑ ईमहे। अ॒सौ यो॑\-ऽव॒सर्प॑ति॒ नील॑ग्रीवो॒ विलो॑हितः। उ॒तैनं॑ गो॒पा अ॑दृश॒न्नदृ॑शन्नुदहा॒र्यः॑। उ॒तैनं॒ विश्वा॑ भू॒तानि॒ स दृ॒ष्टो मृ॑डयाति नः। नमो॑ अस्तु॒ नील॑ग्रीवाय सहस्रा॒क्षाय॑ मी॒ढुषे᳚। अथो॒ ये अ॑स्य॒ सत्वा॑नो॒\-ऽहं तेभ्यो॑\-ऽकरं॒ नमः॑। प्र मु॑ञ्च॒ धन्व॑न॒स्त्वमु॒भयो॒रार्त्नि॑यो॒र्ज्याम्। याश्च॑ ते॒ हस्त॒ इष॑वः~(३)

%4.5.1.4
परा॒ ता भ॑गवो वप। अ॒व॒तत्य॒ धनु॒स्त्वꣳ सह॑स्राक्ष॒ शते॑षुधे। नि॒शीर्य॑ श॒ल्याना॒म्मुखा॑ शि॒वो नः॑ सु॒मना॑ भव। विज्यं॒ धनुः॑ कप॒र्दिनो॒ विश॑ल्यो॒ बाण॑वाꣳ उ॒त। अने॑शन्न॒स्येष॑व आ॒भुर॑स्य निष॒ङ्गथिः॑। या ते॑ हे॒तिर्मी॑ढुष्टम॒ हस्ते॑ ब॒भूव॑ ते॒ धनुः॑। तया॒स्मान् वि॒श्वत॒स्त्वम॑यक्ष॒मया॒ परि॑ ब्भुज। नम॑स्ते अ॒स्त्वायु॑धा॒याना॑तताय धृ॒ष्णवे᳚। उ॒भाभ्या॑मु॒त ते॒ नमो॑ बा॒हुभ्यां॒ तव॒ धन्व॑ने। परि॑ ते॒ धन्व॑नो हे॒तिर॒स्मान्वृ॑णक्तु वि॒श्वतः॑। अथो॒ य इ॑षु॒धिस्तवा॒रे अ॒स्मन्नि धे॑हि॒ तम्॥~(४)

%4.5.2.0
{\anuvakamend[{हस्ते॑ दि॒क्ष्विष॑व उ॒भाभ्यां॒ द्वाविꣳ॑शतिश्च}]}%~(१)

%4.5.2.1
नमो॒ हिर॑ण्यबाहवे सेना॒न्ये॑ दि॒शां च॒ पत॑ये॒ नमो॒ नमो॑ वृ॒क्षेभ्यो॒ हरि॑केशेभ्यः पशू॒नाम्पत॑ये॒ नमो॒ नमः॑ स॒स्पिञ्ज॑राय॒ त्विषी॑मते पथी॒नाम्पत॑ये॒ नमो॒ नमो॑ बभ्लु॒शाय॑ विव्या॒धिने\-ऽन्ना॑ना॒म्पत॑ये॒ नमो॒ नमो॒ हरि॑केशायोपवी॒तिने॑ पु॒ष्टाना॒म्पत॑ये॒ नमो॒ नमो॑ भ॒वस्य॑ हे॒त्यै जग॑ता॒म्पत॑ये॒ नमो॒ नमो॑ रु॒द्राया॑तता॒विने॒ क्षेत्रा॑णा॒म्पत॑ये॒ नमो॒ नमः॑ सू॒तायाह॑न्त्याय॒ वना॑ना॒म्पत॑ये॒ नमो॒ नमः॑~(५)

%4.5.2.2
रोहि॑ताय स्थ॒पत॑ये वृ॒क्षाणा॒म्पत॑ये॒ नमो॒ नमो॑ म॒न्त्रिणे॑ वाणि॒जाय॒ कक्षा॑णा॒म्पत॑ये॒ नमो॒ नमो॑ भुव॒न्तये॑ वारिवस्कृ॒तायौष॑धीना॒म्पत॑ये॒ नमो॒ नम॑ उ॒च्चैर्घो॑षायाक्र॒न्दय॑ते पत्ती॒नाम्पत॑ये॒ नमो॒ नमः॑ कृथ्स्नवी॒ताय॒ धाव॑ते॒ सत्व॑ना॒म्पत॑ये॒ नमः॑॥~(६)

%4.5.3.0
{\anuvakamend[{वना॑ना॒म्पत॑ये॒ नमो॒ नम॒ एका॒न्नत्रि॒ꣳ॒शच्च॑}]}%~(२)

%4.5.3.1
नमः॒ सह॑मानाय निव्या॒धिन॑ आव्या॒धिनी॑ना॒म्पत॑ये॒ नमो॒ नमः॑ ककु॒भाय॑ निष॒ङ्गिणे᳚ स्ते॒नाना॒म्पत॑ये॒ नमो॒ नमो॑ निष॒ङ्गिण॑ इषुधि॒मते॒ तस्क॑राणा॒म्पत॑ये॒ नमो॒ नमो॒ वञ्च॑ते परि॒वञ्च॑ते स्तायू॒नाम्पत॑ये॒ नमो॒ नमो॑ निचे॒रवे॑ परिच॒रायार॑ण्याना॒म्पत॑ये॒ नमो॒ नमः॑ सृका॒विभ्यो॒ जिघाꣳ॑सद्भ्यो मुष्ण॒ताम्पत॑ये॒ नमो॒ नमो॑\-ऽसि॒मद्भ्यो॒ नक्तं॒ चर॑द्भ्यः प्रकृ॒न्ताना॒म्पत॑ये॒ नमो॒ नम॑ उष्णी॒षिणे॑ गिरिच॒राय॑ कुलु॒ञ्चाना॒म्पत॑ये॒ नमो॒ नमः॑~(७)

%4.5.3.2
इषु॑मद्भ्यो धन्वा॒विभ्य॑श्च वो॒ नमो॒ नम॑ आतन्वा॒नेभ्यः॑ प्रति॒दधा॑नेभ्यश्च वो॒ नमो॒ नम॑ आ॒यच्छ॑द्भ्यो विसृ॒जद्भ्य॑श्च वो॒ नमो॒ नमो\-ऽस्य॑द्भ्यो॒ विध्य॑द्भ्यश्च वो॒ नमो॒ नम॒ आसी॑नेभ्यः॒ शया॑नेभ्यश्च वो॒ नमो॒ नमः॑ स्व॒पद्भ्यो॒ जाग्र॑द्भ्यश्च वो॒ नमो॒ नम॒स्तिष्ठ॑द्भ्यो॒ धाव॑द्भ्यश्च वो॒ नमो॒ नमः॑ स॒भाभ्यः॑ स॒भाप॑तिभ्यश्च वो॒ नमो॒ नमो॒ अश्वे॒भ्यो\-ऽश्व॑पतिभ्यश्च वो॒ नमः॑॥~(८)

%4.5.4.0
{\anuvakamend[{कु॒लु॒ञ्चाना॒म्पत॑ये॒ नमो॒ नमो\-ऽश्व॑पतिभ्य॒स्त्रीणि॑ च}]}%~(३)

%4.5.4.1
नम॑ आव्या॒धिनी᳚भ्यो वि॒विध्य॑न्तीभ्यश्च वो॒ नमो॒ नम॒ उग॑णाभ्यस्तृꣳह॒तीभ्य॑श्च वो॒ नमो॒ नमो॑ गृ॒थ्सेभ्यो॑ गृ॒थ्सप॑तिभ्यश्च वो॒ नमो॒ नमो॒ व्राते᳚भ्यो॒ व्रात॑पतिभ्यश्च वो॒ नमो॒ नमो॑ ग॒णेभ्यो॑ ग॒णप॑तिभ्यश्च वो॒ नमो॒ नमो॒ विरू॑पेभ्यो वि॒श्वरू॑पेभ्यश्च वो॒ नमो॒ नमो॑ म॒हद्भ्यः॑ क्षुल्ल॒केभ्य॑श्च वो॒ नमो॒ नमो॑ र॒थिभ्यो॑\-ऽर॒थेभ्य॑श्च वो॒ नमो॒ नमो॒ रथे᳚भ्यः~(९)

%4.5.4.2
रथ॑पतिभ्यश्च वो॒ नमो॒ नमः॒ सेना᳚भ्यः सेना॒निभ्य॑श्च वो॒ नमो॒ नमः॑ क्ष॒त्तृभ्यः॑ सङ्ग्रही॒तृभ्य॑श्च वो॒ नमो॒ नम॒स्तक्ष॑भ्यो रथका॒रेभ्य॑श्च वो॒ नमो॒ नमः॒ कुला॑लेभ्यः क॒र्मारे᳚भ्यश्च वो॒ नमो॒ नमः॑ पु॒ञ्जिष्टे᳚भ्यो निषा॒देभ्य॑श्च वो॒ नमो॒ नम॑ इषु॒कृद्भ्यो॑ धन्व॒कृद्भ्य॑श्च वो॒ नमो॒ नमो॑ मृग॒युभ्यः॑ श्व॒निभ्य॑श्च वो॒ नमो॒ नमः॒ श्वभ्यः॒ श्वप॑तिभ्यश्च वो॒ नमः॑॥~(१०)

%4.5.5.0
{\anuvakamend[{रथे᳚भ्यः॒ श्वप॑तिभ्यश्च॒ द्वे च॑}]}%~(४)

%4.5.5.1
नमो॑ भ॒वाय॑ च रु॒द्राय॑ च॒ नमः॑ श॒र्वाय॑ च पशु॒पत॑ये च॒ नमो॒ नील॑ग्रीवाय च शिति॒कण्ठा॑य च॒ नमः॑ कप॒र्दिने॑ च॒ व्यु॑प्तकेशाय च॒ नमः॑ सहस्रा॒क्षाय॑ च श॒तध॑न्वने च॒ नमो॑ गिरि॒शाय॑ च शिपिवि॒ष्टाय॑ च॒ नमो॑ मी॒ढुष्ट॑माय॒ चेषु॑मते च॒ नमो᳚ ह्र॒स्वाय॑ च वाम॒नाय॑ च॒ नमो॑ बृह॒ते च॒ वर्\mbox{}षी॑यसे च॒ नमो॑ वृ॒द्धाय॑ च सं॒वृध्व॑ने च~(११)

%4.5.5.2
नमो॒ अग्रि॑याय च प्रथ॒माय॑ च॒ नम॑ आ॒शवे॑ चाजि॒राय॑ च॒ नमः॒ शीघ्रि॑याय च॒ शीभ्या॑य च॒ नम॑ ऊ॒र्म्या॑य चावस्व॒न्या॑य च॒ नमः॑ स्रोत॒स्या॑य च॒ द्वीप्या॑य च॥~(१२)

%4.5.6.0
{\anuvakamend[{सं॒ वृध्व॑ने च॒ पञ्च॑विꣳशतिश्च}]}%~(५)

%4.5.6.1
नमो᳚ ज्ये॒ष्ठाय॑ च कनि॒ष्ठाय॑ च॒ नमः॑ पूर्व॒जाय॑ चापर॒जाय॑ च॒ नमो॑ मध्य॒माय॑ चापग॒ल्भाय॑ च॒ नमो॑ जघ॒न्या॑य च॒ बुध्नि॑याय च॒ नमः॑ सो॒भ्या॑य च प्रतिस॒र्या॑य च॒ नमो॒ याम्या॑य च॒ क्षेम्या॑य च॒ नम॑ उर्व॒र्या॑य च॒ खल्या॑य च॒ नमः॒ श्लोक्या॑य चावसा॒न्या॑य च॒ नमो॒ वन्या॑य च॒ कक्ष्या॑य च॒ नमः॑ श्र॒वाय॑ च प्रतिश्र॒वाय॑ च~(१३)

%4.5.6.2
नम॑ आ॒शुषे॑णाय चा॒शुर॑थाय च॒ नमः॒ शूरा॑य चावभिन्द॒ते च॒ नमो॑ व॒र्मिणे॑ च वरू॒थिने॑ च॒ नमो॑ बि॒ल्मिने॑ च कव॒चिने॑ च॒ नमः॑ श्रु॒ताय॑ च श्रुतसे॒नाय॑ च॥~(१४)

%4.5.7.0
{\anuvakamend[{प्र॒ति॒श्र॒वाय॑ च॒ पञ्च॑विꣳशतिश्च}]}%~(६)

%4.5.7.1
नमो॑ दुन्दु॒भ्या॑य चाहन॒न्या॑य च॒ नमो॑ धृ॒ष्णवे॑ च प्रमृ॒शाय॑ च॒ नमो॑ दू॒ताय॑ च॒ प्रहि॑ताय च॒ नमो॑ निष॒ङ्गिणे॑ चेषुधि॒मते॑ च॒ नम॑स्ती॒क्ष्णेष॑वे चायु॒धिने॑ च॒ नमः॑ स्वायु॒धाय॑ च सु॒धन्व॑ने च॒ नमः॒ स्रुत्या॑य च॒ पथ्या॑य च॒ नमः॑ का॒ट्या॑य च नी॒प्या॑य च॒ नमः॒ सूद्या॑य च सर॒स्या॑य च॒ नमो॑ ना॒द्याय॑ च वैश॒न्ताय॑ च~(१५)

%4.5.7.2
नमः॒ कूप्या॑य चाव॒ट्या॑य च॒ नमो॒ वर्ष्या॑य चाव॒र्ष्याय॑ च॒ नमो॑ मे॒घ्या॑य च विद्यु॒त्या॑य च॒ नम॑ ई॒ध्रिया॑य चात॒प्या॑य च॒ नमो॒ वात्या॑य च॒ रेष्मि॑याय च॒ नमो॑ वास्त॒व्या॑य च वास्तु॒पाय॑ च॥~(१६)

%4.5.8.0
{\anuvakamend[{वै॒श॒न्ताय॑ च त्रि॒ꣳ॒शच्च॑}]}%~(७)

%4.5.8.1
नमः॒ सोमा॑य च रु॒द्राय॑ च॒ नम॑स्ता॒म्राय॑ चारु॒णाय॑ च॒ नमः॑ शं॒गाय॑ च पशु॒पत॑ये च॒ नम॑ उ॒ग्राय॑ च भी॒माय॑ च॒ नमो॑ अग्रेव॒धाय॑ च दूरेव॒धाय॑ च॒ नमो॑ ह॒न्त्रे च॒ हनी॑यसे च॒ नमो॑ वृ॒क्षेभ्यो॒ हरि॑केशेभ्यो॒ नम॑स्ता॒राय॒ नमः॑ श॒म्भवे॑ च मयो॒भवे॑ च॒ नमः॑ शङ्क॒राय॑ च मयस्क॒राय॑ च॒ नमः॑ शि॒वाय॑ च शि॒वत॑राय च~(१७)

%4.5.8.2
नम॒स्तीर्थ्या॑य च॒ कूल्या॑य च॒ नमः॑ पा॒र्या॑य चावा॒र्या॑य च॒ नमः॑ प्र॒तर॑णाय चो॒त्तर॑णाय च॒ नम॑ आता॒र्या॑य चाला॒द्या॑य च॒ नमः॒ शष्प्या॑य च॒ फेन्या॑य च॒ नमः॑ सिक॒त्या॑य च प्रवा॒ह्या॑य च॥~(१८)

%4.5.9.0
{\anuvakamend[{शि॒वत॑राय च त्रि॒ꣳ॒शच्च॑}]}%~(८)

%4.5.9.1
नम॑ इरि॒ण्या॑य च प्रप॒थ्या॑य च॒ नमः॑ किꣳशि॒लाय॑ च॒ क्षय॑णाय च॒ नमः॑ कप॒र्दिने॑ च पुल॒स्तये॑ च॒ नमो॒ गोष्ठ्या॑य च॒ गृह्या॑य च॒ नम॒स्तल्प्या॑य च॒ गेह्या॑य च॒ नमः॑ का॒ट्या॑य च गह्वरे॒ष्ठाय॑ च॒ नमो᳚ ह्रद॒य्या॑य च निवे॒ष्प्या॑य च॒ नमः॑ पाꣳस॒व्या॑य च रज॒स्या॑य च॒ नमः॒ शुष्क्या॑य च हरि॒त्या॑य च॒ नमो॒ लोप्या॑य चोल॒प्या॑य च~(१९)

%4.5.9.2
नम॑ ऊ॒र्व्या॑य च सू॒र्म्या॑य च॒ नमः॑ प॒र्ण्या॑य च पर्णश॒द्या॑य च॒ नमो॑\-ऽपगु॒रमा॑णाय चाभिघ्न॒ते च॒ नम॑ आक्खिद॒ते च॑ प्रक्खिद॒ते च॒ नमो॑ वः किरि॒केभ्यो॑ दे॒वाना॒ꣳ॒ हृद॑येभ्यो॒ नमो॑ विक्षीण॒केभ्यो॒ नमो॑ विचिन्व॒त्केभ्यो॒ नम॑ आनिर्\mbox{}ह॒तेभ्यो॒ नम॑ आमीव॒त्केभ्यः॑॥~(२०)

%4.5.10.0
{\anuvakamend[{उ॒ल॒प्या॑य च॒ त्रय॑स्त्रिꣳशच्च}]}%~(९)

%4.5.10.1
द्रापे॒ अन्ध॑सस्पते॒ दरि॑द्र॒न्नील॑लोहित। ए॒षां पुरु॑षाणामे॒षाम्प॑शू॒नां मा भेर्मारो॒ मो ए॑षां॒ किं च॒नाम॑मत्। या ते॑ रुद्र शि॒वा त॒नूः शि॒वा वि॒श्वाह॑भेषजी। शि॒वा रु॒द्रस्य॑ भेष॒जी तया॑ नो मृड जी॒वसे᳚। इ॒माꣳ रु॒द्राय॑ त॒वसे॑ कप॒र्दिने᳚ क्ष॒यद्वी॑राय॒ प्र भ॑रामहे म॒तिम्। यथा॑ नः॒ शमस॑द्द्वि॒पदे॒ चतु॑ष्पदे॒ विश्व॑म्पु॒ष्टम्ग्रा॒मे॑ अ॒स्मिन्न्~(२१)

%4.5.10.2
अना॑तुरम्। मृ॒डा नो॑ रु॒द्रोत नो॒ मय॑स्कृधि क्ष॒यद्वी॑राय॒ नम॑सा विधेम ते। यच्छं च॒ योश्च॒ मनु॑राय॒जे पि॒ता तद॑श्याम॒ तव॑ रुद्र॒ प्रणी॑तौ। मा नो॑ म॒हान्त॑मु॒त मा नो॑ अर्भ॒कं मा न॒ उक्ष॑न्तमु॒त मा न॑ उक्षि॒तम्। मा नो॑ वधीः पि॒तर॒म्मोत मा॒तर॑म्प्रि॒या मा न॑स्त॒नुवः॑~(२२)

%4.5.10.3
रु॒द्र॒ री॒रि॒षः॒। मा न॑स्तो॒के तन॑ये॒ मा न॒ आयु॑षि॒ मा नो॒ गोषु॒ मा नो॒ अश्वे॑षु रीरिषः। वी॒रान्मा नो॑ रुद्र भामि॒तो व॑धीर्\mbox{}ह॒विष्म॑न्तो॒ नम॑सा विधेम ते। आ॒रात्ते॑ गो॒घ्न उ॒त पू॑रुष॒घ्ने क्ष॒यद्वी॑राय सु॒म्नम॒स्मे ते॑ अस्तु। रक्षा॑ च नो॒ अधि॑ च देव ब्रू॒ह्यधा॑ च नः॒ शर्म॑ यच्छ द्वि॒बर्\mbox{}हाः᳚। स्तु॒हि~(२३)

%4.5.10.4
श्रु॒तं ग॑र्त॒सदं॒ युवा॑नम्मृ॒गं न भी॒ममु॑पह॒त्नुमु॒ग्रम्। मृ॒डा ज॑रि॒त्रे रु॑द्र॒ स्तवा॑नो अ॒न्यं ते॑ अ॒स्मन्नि व॑पन्तु॒ सेनाः᳚। परि॑ णो रु॒द्रस्य॑ हे॒तिर्वृ॑णक्तु॒ परि॑ त्वे॒षस्य॑ दुर्म॒तिर॑घा॒योः। अव॑ स्थि॒रा म॒घव॑द्भ्यस्तनुष्व॒ मीढ्व॑स्तो॒काय॒ तन॑याय मृडय। मीढु॑ष्टम॒ शिव॑तम शि॒वो नः॑ सु॒मना॑ भव। प॒र॒मे वृ॒क्ष आयु॑धं नि॒धाय॒ कृत्तिं॒ वसा॑न॒ आ च॑र॒ पिना॑कम्~(२४)

%4.5.10.5
बिभ्र॒दा ग॑हि। विकि॑रिद॒ विलो॑हित॒ नम॑स्ते अस्तु भगवः। यास्ते॑ स॒हस्रꣳ॑ हे॒तयो॒\-ऽन्यम॒स्मन्नि व॑पन्तु॒ ताः। स॒हस्रा॑णि सहस्र॒धा बा॑हु॒वोस्तव॑ हे॒तयः॑। तासा॒मीशा॑नो भगवः परा॒चीना॒ मुखा॑ कृधि॥~(२५)

%4.5.11.0
{\anuvakamend[{अ॒स्मिꣴ स्त॒नुवः॑ स्तु॒हि पिना॑क॒मेका॒न्नत्रि॒ꣳ॒शच्च॑}]}%॥10॥

%4.5.11.1
स॒हस्रा॑णि सहस्र॒शो ये रु॒द्रा अधि॒ भूम्या᳚म्। तेषाꣳ॑ सहस्रयोज॒ने\-ऽव॒ धन्वा॑नि तन्मसि। अ॒स्मिन्म॑ह॒त्य॑र्ण॒वे᳚\-ऽ\-न्तरि॑क्षे भ॒वा अधि॑। नील॑ग्रीवाः शिति॒कण्ठाः᳚ श॒र्वा अ॒धः क्ष॑माच॒राः। नील॑ग्रीवाः शिति॒कण्ठा॒ दिवꣳ॑ रु॒द्रा उप॑श्रिताः। ये वृ॒क्षेषु॑ स॒स्पिञ्ज॑रा॒ नील॑ग्रीवा॒ विलो॑हिताः। ये भू॒ताना॒मधि॑पतयो विशि॒खासः॑ कप॒र्दिनः॑। ये अन्ने॑षु वि॒विध्य॑न्ति॒ पात्रे॑षु॒ पिब॑तो॒ जनान्॑। ये प॒थाम्प॑थि॒रक्ष॑य ऐलबृ॒दा य॒व्युधः॑। ये ती॒र्थानि॑~(२६)

%4.5.11.2
प्र॒चर॑न्ति सृ॒काव॑न्तो निष॒ङ्गिणः॑। य ए॒ताव॑न्तश्च॒ भूयाꣳ॑सश्च॒ दिशो॑ रु॒द्रा वि॑तस्थि॒रे। तेषाꣳ॑ सहस्रयोज॒ने\-ऽव॒ धन्वा॑नि तन्मसि। नमो॑ रु॒द्रेभ्यो॒ ये पृ॑थि॒व्यां ये᳚\-ऽन्तरि॑क्षे॒ ये दि॒वि येषा॒मन्नं॒ वातो॑ व॒र्\mbox{}षमिष॑व॒स्तेभ्यो॒ दश॒ प्राची॒र्दश॑ दक्षि॒णा दश॑ प्र॒तीची॒र्दशोदी॑ची॒र्दशो॒र्ध्वास्तेभ्यो॒ नम॒स्ते नो॑ मृडयन्तु॒ ते यं द्वि॒ष्मो यश्च॑ नो॒ द्वेष्टि॒ तं वो॒ जम्भे॑ दधामि~(२७)

%4.6.0.0

%4.6.0.0
{\anuvakamend[{ती॒र्थानि॒ यश्च॒ षट्च॑}]}%॥11॥

%4.6.0.0
{\prashnaend[{अश्म॒न् य इ॒मोदे॑नमा॒शुः प्राचीं᳚ जी॒मूत॑स्य॒ यदक्र॑न्दो॒ मा नो॑ मि॒त्रो ये वा॒जिनं॒ नव॑~(९) अश्म॑न्मनो॒युजं॒ प्राची॒मनु॒ शर्म॑ यच्छतु॒ तेषा॑म॒भिगू᳚र्तिः॒ षट्च॑त्वारिꣳशत्। अश्म॑न् ह॒विष्मान्॑॥}]}

%%% END PRASHNA

\sect{षष्ठमः प्रश्नः}\setcounter{anuvakam}{0}
\dnsub{तैत्तिरीयसंहितायां चतुर्थकाण्डे षष्ठमः प्रश्नः}
%4.6.1.0
%4.6.1.1
अश्म॒न्नूर्जं॒ पर्व॑ते शिश्रिया॒णां वाते॑ प॒र्जन्ये॒ वरु॑णस्य॒ शुष्मे᳚। अ॒द्भ्य ओष॑धीभ्यो॒ वन॒स्पति॒भ्यो\-ऽधि॒ सम्भृ॑तां॒ तां न॒ इष॒मूर्जं॑ धत्त मरुतः सꣳररा॒णाः। अश्मꣴ॑स्ते॒ क्षुद॒मुं ते॒ शुगृ॑च्छतु॒ यं द्वि॒ष्मः। स॒मु॒द्रस्य॑ त्वा॒\-ऽवाक॒याग्ने॒ परि॑ व्ययामसि। पाव॒को अ॒स्मभ्यꣳ॑ शि॒वो भ॑व। हि॒मस्य॑ त्वा ज॒रायु॒णाग्ने॒ परि॑ व्ययामसि। पा॒व॒को अ॒स्मभ्यꣳ॑ शि॒वो भ॑व। उप॑~(१)

%4.6.1.2
ज्मन्नुप॑ वेत॒से\-ऽव॑त्तरं न॒दीष्वा। अग्ने॑ पि॒त्तम॒पाम॑सि। मण्डू॑कि॒ ताभि॒रा ग॑हि॒ सेमं नो॑ य॒ज्ञम्। पा॒व॒कव॑र्णꣳ शि॒वं कृ॑धि। पा॒व॒क आ चि॒तय॑न्त्या कृ॒पा। क्षाम॑न्रुरु॒च उ॒षसो॒ न भा॒नुना᳚। तूर्व॒न्न याम॒न्नेत॑शस्य॒ नू रण॒ आ यो घृ॒णे। न त॑तृषा॒णो अ॒जरः॑। अग्ने॑ पावक रो॒चिषा॑ म॒न्द्रया॑ देव जि॒ह्वया᳚। आ दे॒वान्~(२)

%4.6.1.3
व॒क्षि॒ यक्षि॑ च। स नः॑ पावक दीदि॒वो\-ऽग्ने॑ दे॒वाꣳ इ॒हा व॑ह। उप॑ य॒ज्ञꣳ ह॒विश्च॑ नः। अ॒पामि॒दं न्यय॑नꣳ समु॒द्रस्य॑ नि॒वेश॑नम्। अ॒न्यं ते॑ अ॒स्मत्त॑पन्तु हे॒तयः॑ पाव॒को अ॒स्मभ्यꣳ॑ शि॒वो भ॑व। नम॑स्ते॒ हर॑से शो॒चिषे॒ नम॑स्ते अस्त्व॒र्चिषे᳚। अ॒न्यं ते॑ अ॒स्मत्त॑पन्तु हे॒तयः॑ पाव॒को अ॒स्मभ्यꣳ॑ शि॒वो भ॑व। नृ॒षदे॒ वट्~(३)

%4.6.1.4
अ॒फ्सु॒षदे॒ वड्व॑न॒सदे॒ वड्ब॑र्\mbox{}हि॒षदे॒ वट्थ्सु॑व॒र्विदे॒ वट्। ये दे॒वा दे॒वानां᳚ य॒ज्ञिया॑ य॒ज्ञिया॑नाꣳ संवथ्स॒रीण॒मुप॑ भा॒गमास॑ते। अ॒हु॒तादो॑ ह॒विषो॑ य॒ज्ञे अ॒स्मिन्थ्स्व॒यं जु॑हुध्व॒म्मधु॑नो घृ॒तस्य॑। ये दे॒वा दे॒वेष्वधि॑ देव॒त्वमाय॒न् ये ब्रह्म॑णः पुरए॒तारो॑ अ॒स्य। येभ्यो॒ नर्ते पव॑ते॒ धाम॒ किं च॒न न ते दि॒वो न पृ॑थि॒व्या अधि॒ स्नुषु॑। प्रा॒ण॒दाः~(४)

%4.6.1.5
अ॒पा॒न॒दा व्या॑न॒दाश्च॑क्षु॒र्दा व॑र्चो॒दा व॑रिवो॒दाः। अ॒न्यं ते॑ अ॒स्मत्त॑पन्तु हे॒तयः॑ पाव॒को अ॒स्मभ्यꣳ॑ शि॒वो भ॑व। अ॒ग्निस्ति॒ग्मेन॑ शो॒चिषा॒ यꣳस॒द्विश्वं॒ न्य॑त्रिणम्᳚। अ॒ग्निर्नो॑ वꣳसते र॒यिम्। सैनानी॑केन सुवि॒दत्रो॑ अ॒स्मे यष्टा॑ दे॒वाꣳ आय॑जिष्ठः स्व॒स्ति। अद॑ब्धो गो॒पा उ॒त नः॑ पर॒स्पा अग्ने᳚ द्यु॒मदु॒त रे॒वद्दि॑दीहि॥~(५)

%4.6.2.0
{\anuvakamend[{उप॑ दे॒वान् वट्प्रा॑ण॒दाश्चतु॑श्चत्वारिꣳशच्च}]}%~(१)

%4.6.2.1
य इ॒मा विश्वा॒ भुव॑नानि॒ जुह्व॒दृषि॒र्\mbox{}होता॑ निष॒सादा॑ पि॒ता नः॑। स आ॒शिषा॒ द्रवि॑णमि॒च्छमा॑नः परम॒च्छदो॒ वर॒ आ वि॑वेश। वि॒श्वक॑र्मा॒ मन॑सा॒ यद्विहा॑या धा॒ता वि॑धा॒ता प॑र॒मोत स॒न्दृक्। तेषा॑मि॒ष्टानि॒ समि॒षा म॑दन्ति॒ यत्र॑ सप्त॒र्\mbox{}षीन्प॒र एक॑मा॒हुः। यो नः॑ पि॒ता ज॑नि॒ता यो वि॑धा॒ता यो नः॑ स॒तो अ॒भ्या सज्ज॒जान॑।~(६)

%4.6.2.2
यो दे॒वानां᳚ नाम॒धा एक॑ ए॒व तꣳ स॑म्प्र॒श्ञम्भुव॑ना यन्त्य॒न्या। त आय॑जन्त॒ द्रवि॑ण॒ꣳ॒ सम॑स्मा॒ ऋष॑यः॒ पूर्वे॑ जरि॒तारो॒ न भू॒ना। अ॒सूर्ता॒ सूर्ता॒ रज॑सो वि॒माने॒ ये भू॒तानि॑ स॒मकृ॑ण्वन्नि॒मानि॑। न तं वि॑दाथ॒ य इ॒दं ज॒जाना॒न्यद्यु॒ष्माक॒मन्त॑रम्भवाति। नी॒हा॒रेण॒ प्रावृ॑ता॒ जल्प्या॑ चासु॒तृप॑ उक्थ॒शास॑श्चरन्ति। प॒रो दि॒वा प॒र ए॒ना~(७)

%4.6.2.3
पृ॒थि॒व्या प॒रो दे॒वेभि॒रसु॑रै॒र्गुहा॒ यत्। कꣴ स्वि॒द्गर्भं॑ प्रथ॒मं द॑ध्र॒ आपो॒ यत्र॑ दे॒वाः स॒मग॑च्छन्त॒ विश्वे᳚। तमिद्गर्भ॑म्प्रथ॒मं द॑ध्र॒ आपो॒ यत्र॑ दे॒वाः स॒मग॑च्छन्त॒ विश्वे᳚। अ॒जस्य॒ नाभा॒वध्येक॒मर्पि॑तं॒ यस्मि॑न्नि॒दं विश्व॒म्भुव॑न॒\-मधि॑ श्रि॒तम्। वि॒श्वक॑र्मा॒ ह्यज॑निष्ट दे॒व आदिद्ग॑न्ध॒र्वो अ॑भवद्द्वि॒तीयः॑। तृ॒तीयः॑ पि॒ता ज॑नि॒तौष॑धीनाम्~(८)

%4.6.2.4
अ॒पां गर्भं॒ व्य॑दधात्पुरु॒त्रा। चक्षु॑षः पि॒ता मन॑सा॒ हि धीरो॑ घृ॒तमे॑ने अजन॒न्नन्न॑माने। य॒देदन्ता॒ अद॑दृꣳहन्त॒ पूर्व॒ आदिद्द्यावा॑पृथि॒वी अ॑प्रथेताम्। वि॒श्वत॑श्चक्षुरु॒त वि॒श्वतो॑मुखो वि॒श्वतो॑हस्त उ॒त वि॒श्वत॑स्पात्। सं बा॒हुभ्यां॒ नम॑ति॒ सम्पत॑त्रै॒र्द्यावा॑पृथि॒वी ज॒नयं॑ दे॒व एकः॑। किꣴ स्वि॑दासीदधि॒ष्ठान॑मा॒रम्भ॑णं कत॒मथ्स्वि॒त्किमा॑सीत्। यदी॒ भूमिं॑ ज॒नयन्न्॑~(९)

%4.6.2.5
वि॒श्वक॑र्मा॒ वि द्यामौर्णो᳚न्महि॒ना वि॒श्वच॑क्षाः। किꣴ स्वि॒द्वनं॒ क उ॒ स वृ॒क्ष आ॑सी॒द्यतो॒ द्यावा॑पृथि॒वी नि॑ष्टत॒क्षुः। मनी॑षिणो॒ मन॑सा पृ॒च्छतेदु॒ तद्यद॒ध्यति॑ष्ठ॒द्भुव॑नानि धा॒रयन्न्॑। या ते॒ धामा॑नि पर॒माणि॒ याव॒मा या म॑ध्य॒मा वि॑श्वकर्मन्नु॒तेमा। शिक्षा॒ सखि॑भ्यो ह॒विषि॑ स्वधावः स्व॒यं य॑जस्व त॒नुवं॑ जुषा॒णः। वा॒चस्पतिं॑ वि॒श्वक॑र्माणमू॒तये᳚~(१०)

%4.6.2.6
म॒नो॒युजं॒ वाजे॑ अ॒द्या हु॑वेम। स नो॒ नेदि॑ष्ठा॒ हव॑नानि जोषते वि॒श्वश॑म्भू॒रव॑से सा॒धुक॑र्मा। विश्व॑कर्मन् ह॒विषा॑ वावृधा॒नः स्व॒यं य॑जस्व त॒नुवं॑ जुषा॒णः। मुह्य॑न्त्व॒न्ये अ॒भितः॑ स॒पत्ना॑ इ॒हास्माक॑म्म॒घवा॑ सू॒रिर॑स्तु। विश्व॑कर्मन् ह॒विषा॒ वर्ध॑नेन त्रा॒तार॒मिन्द्र॑मकृणोरव॒ध्यम्। तस्मै॒ विशः॒ सम॑नमन्त पू॒र्वीर॒यमु॒ग्रो वि॑ह॒व्यो॑ यथास॑त्। स॒मु॒द्राय॑ व॒युना॑य॒ सिन्धू॑ना॒म्पत॑ये॒ नमः॑। न॒दीना॒ꣳ॒ सर्वा॑साम्पि॒त्रे जु॑हु॒ता वि॒श्वक॑र्मणे॒ विश्वाहाम॑र्त्यꣳ ह॒विः॥~(११)

%4.6.3.0
{\anuvakamend[{ज॒जानै॒नौष॑धीनां॒ भूमिं॑ ज॒नय॑न्नू॒तये॒ नमो॒ नव॑ च}]}%~(२)

%4.6.3.1
उदे॑नमुत्त॒रां न॒याग्ने॑ घृतेनाहुत। रा॒यस्पोषे॑ण॒ सꣳ सृ॑ज प्र॒जया॑ च॒ धने॑न च। इन्द्रे॒मम्प्र॑त॒रां कृ॑धि सजा॒ताना॑मसद्व॒शी। समे॑नं॒ वर्च॑सा सृज दे॒वेभ्यो॑ भाग॒धा अ॑सत्। यस्य॑ कु॒र्मो ह॒विर्गृ॒हे तम॑ग्ने वर्धया॒ त्वम्। तस्मै॑ दे॒वा अधि॑ ब्रवन्न॒यं च॒ ब्रह्म॑ण॒स्पतिः॑। उदु॑ त्वा॒ विश्वे॑ दे॒वाः~(१२)

%4.6.3.2
अग्ने॒ भर॑न्तु॒ चित्ति॑भिः। स नो॑ भव शि॒वत॑मः सु॒प्रती॑को वि॒भाव॑सुः। पञ्च॒ दिशो॒ दैवी᳚र्य॒ज्ञम॑वन्तु दे॒वीरपाम॑तिं दुर्म॒तिम्बाध॑मानाः। रा॒यस्पोषे॑ य॒ज्ञप॑तिमा॒भज॑न्तीः। रा॒यस्पोषे॒ अधि॑ य॒ज्ञो अ॑स्था॒थ्समि॑द्धे अ॒ग्नावधि॑ मामहा॒नः। उ॒क्थप॑त्त्र॒ ईड्यो॑ गृभी॒तस्त॒प्तं घ॒र्मं प॑रि॒गृह्या॑यजन्त। ऊ॒र्जा यद्य॒ज्ञमश॑मन्त दे॒वा दैव्या॑य ध॒र्त्रे जोष्ट्रे᳚। दे॒व॒श्रीः श्रीम॑णाः श॒तप॑याः~(१३)

%4.6.3.3
प॒रि॒गृह्य॑ दे॒वा य॒ज्ञमा॑यन्न्। सूर्य॑रश्मि॒र्\mbox{}हरि॑केशः पु॒रस्ता᳚थ्सवि॒ता ज्योति॒रुद॑या॒ꣳ॒ अज॑स्रम्। तस्य॑ पू॒षा प्र॑स॒वं या॑ति दे॒वः स॒म्पश्य॒न्विश्वा॒ भुव॑नानि गो॒पाः। दे॒वा दे॒वेभ्यो॑ अध्व॒र्यन्तो॑ अस्थुर्वी॒तꣳ श॑मि॒त्रे श॑मि॒ता य॒जध्यै᳚। तु॒रीयो॑ य॒ज्ञो यत्र॑ ह॒व्यमेति॒ ततः॑ पाव॒का आ॒शिषो॑ नो जुषन्ताम्। वि॒मान॑ ए॒ष दि॒वो मध्य॑ आस्त आपप्रि॒वान्रोद॑सी अ॒न्तरि॑क्षम्। स वि॒श्वाची॑र॒भि~(१४)

%4.6.3.4
च॒ष्टे॒ घृ॒ताची॑रन्त॒रा पूर्व॒मप॑रं च के॒तुम्। उ॒क्षा स॑मु॒द्रो अ॑रु॒णः सु॑प॒र्णः पूर्व॑स्य॒ योनि॑म्पि॒तुरा वि॑वेश। मध्ये॑ दि॒वो निहि॑तः॒ पृश्ञि॒रश्मा॒ वि च॑क्रमे॒ रज॑सः पा॒त्यन्तौ᳚। इन्द्रं॒ विश्वा॑ अवीवृधन्थ्समु॒द्रव्य॑चसं॒ गिरः॑। र॒थीत॑मꣳ रथी॒नां वाजा॑ना॒ꣳ॒ सत्प॑ति॒म्पतिम्᳚। सु॒म्न॒हूर्य॒ज्ञो दे॒वाꣳ आ च॑ वक्ष॒द्यक्ष॑द॒ग्निर्दे॒वो दे॒वाꣳ आ च॑ वक्षत्। वाज॑स्य मा प्रस॒वेनो᳚द्ग्रा॒भेणोद॑ग्रभीत्। अथा॑ स॒पत्ना॒ꣳ॒ इन्द्रो॑ मे निग्रा॒भेणाध॑राꣳ अकः। उ॒द्ग्रा॒भं च॑ निग्रा॒भं च॒ ब्रह्म॑ दे॒वा अ॑वीवृधन्न्। अथा॑ स॒पत्ना॑निन्द्रा॒ग्नी मे॑ विषू॒चीना॒न्व्य॑स्यताम्॥~(१५)

%4.6.4.0
{\anuvakamend[{दे॒वाः श॒तप॑या अ॒भि वाज॑स्य॒ षड्विꣳ॑शतिश्च}]}%~(३)

%4.6.4.1
आ॒शुः शिशा॑नो वृष॒भो न यु॒ध्मो घ॑नाघ॒नः क्षोभ॑णश्चर्\mbox{}षणी॒नाम्। स॒ङ्क्रन्द॑नो\-ऽनिमि॒ष ए॑कवी॒रः श॒तꣳ सेना॑ अजयथ्सा॒कमिन्द्रः॑। सं॒क्रन्द॑नेनानिमि॒षेण॑ जि॒ष्णुना॑ युत्का॒रेण॑ दुश्च्यव॒नेन॑ धृ॒ष्णुना᳚। तदिन्द्रे॑ण जयत॒ तथ्स॑हध्वं॒ युधो॑ नर॒ इषु॑हस्तेन॒ वृष्णा᳚। स इषु॑हस्तैः॒ स नि॑ष॒ङ्गिभि॑र्व॒शी सꣴस्र॑ष्टा॒ स युध॒ इन्द्रो॑ ग॒णेन। स॒ꣳ॒सृ॒ष्ट॒जिथ्सो॑म॒पा बा॑हुश॒र्ध्यू᳚र्ध्वध॑न्वा॒ प्रति॑हिताभि॒रस्ता᳚। बृह॑स्पते॒ परि॑ दीय~(१६)

%4.6.4.2
रथे॑न रक्षो॒हामित्राꣳ॑ अप॒बाध॑मानः। प्र॒भ॒ञ्जन्थ्सेनाः᳚ प्रमृ॒णो यु॒धा जय॑न्न॒स्माक॑मेध्यवि॒ता रथा॑नाम्। गो॒त्र॒भिदं॑ गो॒विदं॒ वज्र॑बाहुं॒ जय॑न्त॒मज्म॑ प्रमृ॒णन्त॒मोज॑सा। इ॒मꣳ स॑जाता॒ अनु॑ वीरयध्व॒मिन्द्रꣳ॑ सखा॒यो\-ऽनु॒ सꣳ र॑भध्वम्। ब॒ल॒वि॒ज्ञा॒यः स्थवि॑रः॒ प्रवी॑रः॒ सह॑स्वान् वा॒जी सह॑मान उ॒ग्रः। अ॒भिवी॑रो अ॒भिस॑त्वा सहो॒जा जैत्र॑मिन्द्र॒ रथ॒मा ति॑ष्ठ गो॒वित्। अ॒भि गो॒त्राणि॒ सह॑सा॒ गाह॑मानो\-ऽदा॒यः~(१७)

%4.6.4.3
वी॒रः श॒तम॑न्यु॒रिन्द्रः॑। दु॒श्च्य॒व॒नः पृ॑तना॒षाड॑यु॒ध्यो᳚\-ऽस्माक॒ꣳ॒ सेना॑ अवतु॒ प्र यु॒थ्सु। इन्द्र॑ आसां ने॒ता बृह॒स्पति॒र्दक्षि॑णा य॒ज्ञः पु॒र ए॑तु॒ सोमः॑। दे॒व॒से॒नाना॑मभिभञ्जती॒नां जय॑न्तीनाम्म॒रुतो॑ य॒न्त्वग्रे᳚। इन्द्र॑स्य॒ वृष्णो॒ वरु॑णस्य॒ राज्ञ॑ आदि॒त्याना᳚म्म॒रुता॒ꣳ॒ शर्ध॑ उ॒ग्रम्। म॒हाम॑नसाम्भुवनच्य॒वानां॒ घोषो॑ दे॒वानां॒ जय॑ता॒मुद॑स्थात्। अ॒स्माक॒मिन्द्रः॒ समृ॑तेषु ध्व॒जेष्व॒स्माकं॒ या इष॑व॒स्ता ज॑यन्तु।~(१८)

%4.6.4.4
अ॒स्माकं॑ वी॒रा उत्त॑रे भवन्त्व॒स्मानु॑ देवा अवता॒ हवे॑षु। उद्ध॑र्\mbox{}षय मघव॒न्नायु॑धा॒न्युथ्सत्व॑नाम्माम॒काना॒म्महाꣳ॑सि। उद्वृ॑त्रहन्वा॒जिनां॒ वाजि॑ना॒न्युद्रथा॑नां॒ जय॑तामेतु॒ घोषः॑। उप॒ प्रेत॒ जय॑ता नरः स्थि॒रा वः॑ सन्तु बा॒हवः॑। इन्द्रो॑ वः॒ शर्म॑ यच्छत्वनाधृ॒ष्या यथास॑थ। अव॑सृष्टा॒ परा॑ पत॒ शर॑व्ये॒ ब्रह्म॑सꣳशिता। गच्छा॒मित्रा॒न्प्र~(१९)

%4.6.4.5
वि॒श॒ मैषां॒ कं च॒नोच्छि॑षः। मर्मा॑णि ते॒ वर्म॑भिश्छादयामि॒ सोम॑स्त्वा॒ राजा॒मृते॑ना॒भि व॑स्ताम्। उ॒रोर्वरी॑यो॒ वरि॑वस्ते अस्तु॒ जय॑न्तं॒ त्वामनु॑ मदन्तु दे॒वाः। यत्र॑ बा॒णाः स॒म्पत॑न्ति कुमा॒रा वि॑शि॒खा इ॑व। इन्द्रो॑ न॒स्तत्र॑ वृत्र॒हा वि॑श्वा॒हा शर्म॑ यच्छतु॥~(२०)

%4.6.5.0
{\anuvakamend[{दी॒या॒ दा॒यो ज॑यन्त्व॒मित्रा॒न्प्र च॑त्वारि॒ꣳ॒शच्च॑}]}%~(४)

%4.6.5.1
प्राची॒मनु॑ प्र॒दिश॒म्प्रेहि॑ वि॒द्वान॒ग्नेर॑ग्ने पु॒रो अ॑ग्निर्भवे॒ह। विश्वा॒ आशा॒ दीद्या॑नो॒ वि भा॒ह्यूर्जं॑ नो धेहि द्वि॒पदे॒ चतु॑ष्पदे। क्रम॑ध्वम॒ग्निना॒ नाक॒मुख्य॒ꣳ॒ हस्ते॑षु॒ बिभ्र॑तः। दि॒वः पृ॒ष्ठꣳ सुव॑र्ग॒त्वा मि॒श्रा दे॒वेभि॑राद्ध्वम्। पृ॒थि॒व्या अ॒हमुद॒न्तरि॑क्ष॒मारु॑हम॒न्तरि॑क्षा॒द्दिव॒मारु॑हम्। दि॒वो नाक॑स्य पृ॒ष्ठाथ्सुव॒र्ज्योति॑रगाम्~(२१)

%4.6.5.2
अ॒हम्। सुव॒र्यन्तो॒ नापे᳚क्षन्त॒ आ द्याꣳ रो॑हन्ति॒ रोद॑सी। य॒ज्ञं ये वि॒श्वतो॑धार॒ꣳ॒ सुवि॑द्वाꣳसो वितेनि॒रे। अग्ने॒ प्रेहि॑ प्रथ॒मो दे॑वय॒तां चक्षु॑र्दे॒वाना॑मु॒त मर्त्या॑नाम्। इय॑क्षमाणा॒ भृगु॑भिः स॒जोषाः॒ सुव॑र्यन्तु॒ यज॑मानाः स्व॒स्ति। नक्तो॒षासा॒ सम॑नसा॒ विरू॑पे धा॒पये॑ते॒ शिशु॒मेकꣳ॑ समी॒ची। द्यावा॒ क्षामा॑ रु॒क्मो अ॒न्तर्विभा॑ति दे॒वा अ॒ग्निं धा॑रयन्द्रविणो॒दाः। अग्ने॑ सहस्राक्ष~(२२)

%4.6.5.3
श॒त॒मू॒र्ध॒ञ्छ॒तं ते᳚ प्रा॒णाः स॒हस्र॑मपा॒नाः। त्वꣳ सा॑ह॒स्रस्य॑ रा॒य ई॑शिषे॒ तस्मै॑ ते विधेम॒ वाजा॑य॒ स्वाहा᳚। सु॒प॒र्णो॑\-ऽसि ग॒रुत्मा᳚न्पृथि॒व्याꣳ सी॑द पृ॒ष्ठे पृ॑थि॒व्याः सी॑द भा॒सान्तरि॑क्ष॒मा पृ॑ण॒ ज्योति॑षा॒ दिव॒मुत्त॑भान॒ तेज॑सा॒ दिश॒ उद्दृꣳ॑ह। आ॒जुह्वा॑नः सु॒प्रती॑कः पु॒रस्ता॒दग्ने॒ स्वां योनि॒मा सी॑द सा॒ध्या। अ॒स्मिन्थ्स॒धस्थे॒ अध्युत्त॑रस्मि॒न्विश्वे॑ देवाः~(२३)

%4.6.5.4
यज॑मानश्च सीदत। प्रेद्धो॑ अग्ने दीदिहि पु॒रो नो\-ऽज॑स्रया सू॒र्म्या॑ यविष्ठ। त्वाꣳ शश्व॑न्त॒ उप॑ यन्ति॒ वाजाः᳚। वि॒धेम॑ ते पर॒मे जन्म॑न्नग्ने वि॒धेम॒ स्तोमै॒रव॑रे स॒धस्थे᳚। यस्मा॒द्योने॑रु॒दारि॑था॒ यजे॒ तम्प्र त्वे ह॒वीꣳषि॑ जुहुरे॒ समि॑द्धे। ताꣳ स॑वि॒तुर्वरे᳚ण्यस्य चि॒त्रामाहं वृ॑णे सुम॒तिं वि॒श्वज॑न्याम्। याम॑स्य॒ कण्वो॒ अदु॑ह॒त्प्रपी॑नाꣳ स॒हस्र॑धाराम्~(२४)

%4.6.5.5
पय॑सा म॒हीं गाम्। स॒प्त ते॑ अग्ने स॒मिधः॑ स॒प्त जि॒ह्वाः स॒प्तर्\mbox{}ष॑यः स॒प्त धाम॑ प्रि॒याणि॑। स॒प्त होत्राः᳚ सप्त॒धा त्वा॑ यजन्ति स॒प्त योनी॒रा पृ॑णस्वा घृ॒तेन॑। ई॒दृङ्चा᳚न्या॒दृङ्चै॑ता॒दृङ्च॑ प्रति॒दृङ्च॑ मि॒तश्च॒ सम्मि॑तश्च॒ सभ॑राः। शु॒क्रज्यो॑तिश्च चि॒त्रज्यो॑तिश्च स॒त्यज्यो॑तिश्च॒ ज्योति॑ष्माꣴश्च स॒त्यश्च॑र्त॒पाश्चात्यꣳ॑हाः।~(२५)

%4.6.5.6
ऋ॒त॒जिच्च॑ सत्य॒जिच्च॑ सेन॒जिच्च॑ सु॒षेण॒श्चान्त्य॑मित्रश्च दू॒रेअ॑मित्रश्च ग॒णः। ऋ॒तश्च॑ स॒त्यश्च॑ ध्रु॒वश्च॑ ध॒रुण॑श्च ध॒र्ता च॑ विध॒र्ता च॑ विधार॒यः। ई॒दृक्षा॑स एता॒दृक्षा॑स ऊ॒ षु णः॑ स॒दृक्षा॑सः॒ प्रति॑सदृक्षास॒ एत॑न। मि॒तास॑श्च॒ सम्मि॑तासश्च न ऊ॒तये॒ सभ॑रसो मरुतो य॒ज्ञे अ॒स्मिन्निन्द्रं॒ दैवी॒र्विशो॑ म॒रुतो\-ऽनु॑वर्त्मानो॒ यथेन्द्रं॒ दैवी॒र्विशो॑ म॒रुतो\-ऽनु॑वर्त्मान ए॒वमि॒मं यज॑मानं॒ दैवी᳚श्च॒ विशो॒ मानु॑षी॒श्चानु॑वर्त्मानो भवन्तु॥~(२६)

%4.6.6.0
{\anuvakamend[{अ॒गा॒ꣳ स॒ह॒स्रा॒क्ष॒ दे॒वाः॒ स॒हस्र॑धारा॒मत्यꣳ॑हा॒ अनु॑वर्त्मानः॒ षोड॑श च}]}%~(५)

%4.6.6.1
जी॒मूत॑स्येव भवति॒ प्रती॑कं॒ यद्व॒र्मी याति॑ स॒मदा॑मु॒पस्थे᳚। अना॑विद्धया त॒नुवा॑ जय॒ त्वꣳ स त्वा॒ वर्म॑णो महि॒मा पि॑पर्तु। धन्व॑ना॒ गा धन्व॑ना॒जिं ज॑येम॒ धन्व॑ना ती॒व्राः स॒मदो॑ जयेम। धनुः॒ शत्रो॑रपका॒मं कृ॑णोति॒ धन्व॑ना॒ सर्वाः᳚ प्र॒दिशो॑ जयेम। व॒क्ष्यन्ती॒वेदा ग॑नीगन्ति॒ कर्ण॑म्प्रि॒यꣳ सखा॑यं परिषस्वजा॒ना। योषे॑व शिङ्क्ते॒ वित॒ताधि॒ धन्वन्न्॑~(२७)

%4.6.6.2
ज्या इ॒यꣳ सम॑ने पा॒रय॑न्ती। ते आ॒चर॑न्ती॒ सम॑नेव॒ योषा॑ मा॒तेव॑ पु॒त्रम्बि॑भृतामु॒पस्थे᳚। अप॒ शत्र न्॑विध्यताꣳ संविदा॒ने आर्त्नी॑ इ॒मे वि॑ष्फु॒रन्ती॑ अ॒मित्रान्॑। ब॒ह्वी॒नाम्पि॒ता ब॒हुर॑स्य पु॒त्रश्चि॒श्चा कृ॑णोति॒ सम॑नाव॒गत्य॑। इ॒षु॒धिः सङ्काः॒ पृत॑नाश्च॒ सर्वाः᳚ पृ॒ष्ठे निन॑द्धो जयति॒ प्रसू॑तः। रथे॒ तिष्ठ॑न्नयति वा॒जिनः॑ पु॒रो यत्र॑यत्र का॒मय॑ते सुषार॒थिः। अ॒भीशू॑नाम्महि॒मानम्᳚~(२८)

%4.6.6.3
प॒ना॒य॒त॒ मनः॑ प॒श्चादनु॑ यच्छन्ति र॒श्मयः॑। ती॒व्रान्घोषा᳚न्कृण्वते॒ वृष॑पाण॒यो\-ऽश्वा॒ रथे॑भिः स॒ह वा॒जय॑न्तः। अ॒व॒क्राम॑न्तः॒ प्रप॑दैर॒मित्रा᳚न्क्षि॒णन्ति॒ शत्रू॒ꣳ॒रन॑पव्ययन्तः। र॒थ॒वाह॑नꣳ ह॒विर॑स्य॒ नाम॒ यत्रायु॑धं॒ निहि॑तमस्य॒ वर्म॑। तत्रा॒ रथ॒मुप॑ श॒ग्मꣳ स॑देम वि॒श्वाहा॑ व॒यꣳ सु॑मन॒स्यमा॑नाः। स्वा॒दु॒ष॒ꣳ॒सदः॑ पि॒तरो॑ वयो॒धाः कृ॑च्छ्रे॒श्रितः॒ शक्ती॑वन्तो गभी॒राः। चि॒त्रसे॑ना॒ इषु॑बला॒ अमृ॑ध्राः स॒तोवी॑रा उ॒रवो᳚ व्रातसा॒हाः। ब्राह्म॑णासः~(२९)

%4.6.6.4
पित॑रः॒ सोम्या॑सः शि॒वे नो॒ द्यावा॑पृथि॒वी अ॑ने॒हसा᳚। पू॒षा नः॑ पातु दुरि॒तादृ॑तावृधो॒ रक्षा॒ माकि॑र्नो अ॒घशꣳ॑स ईशत। सु॒प॒र्णं व॑स्ते मृ॒गो अ॑स्या॒ दन्तो॒ गोभिः॒ सन्न॑द्धा पतति॒ प्रसू॑ता। यत्रा॒ नरः॒ सं च॒ वि च॒ द्रव॑न्ति॒ तत्रा॒स्मभ्य॒मिष॑वः॒ शर्म॑ यꣳसन्न्। ऋजी॑ते॒ परि॑ वृङ्ग्धि॒ नो\-ऽश्मा॑ भवतु नस्त॒नूः। सोमो॒ अधि॑ ब्रवीतु॒ नो\-ऽदि॑तिः~(३०)

%4.6.6.5
शर्म॑ यच्छतु। आ ज॑ङ्घन्ति॒ सान्वे॑षां ज॒घना॒ꣳ॒ उप॑ जिघ्नते। अश्वा॑जनि॒ प्रचे॑त॒सो\-ऽश्वा᳚न्थ्स॒मथ्सु॑ चोदय। अहि॑रिव भो॒गैः पर्ये॑ति बा॒हुं ज्याया॑ हे॒तिं प॑रि॒बाध॑मानः। ह॒स्त॒घ्नो विश्वा॑ व॒युना॑नि वि॒द्वान्पुमा॒न्पुमाꣳ॑सं॒ परि॑ पातु वि॒श्वतः॑। वन॑स्पते वी॒ड्व॑ङ्गो॒ हि भू॒या अ॒स्मथ्स॑खा प्र॒तर॑णः सु॒वीरः॑। गोभिः॒ सन्न॑द्धो असि वी॒डय॑स्वास्था॒ता ते॑ जयतु॒ जेत्वा॑नि। दि॒वः पृ॑थि॒व्याः परि॑~(३१)

%4.6.6.6
ओज॒ उद्भृ॑तं॒ वन॒स्पति॑भ्यः॒ पर्याभृ॑त॒ꣳ॒ सहः॑। अ॒पामो॒ज्मानं॒ परि॒ गोभि॒रावृ॑त॒मिन्द्र॑स्य॒ वज्रꣳ॑ ह॒विषा॒ रथं॑ यज। इन्द्र॑स्य॒ वज्रो॑ म॒रुता॒मनी॑कम्मि॒त्रस्य॒ गर्भो॒ वरु॑णस्य॒ नाभिः॑। सेमां नो॑ ह॒व्यदा॑तिं जुषा॒णो देव॑ रथ॒ प्रति॑ ह॒व्या गृ॑भाय। उप॑ श्वासय पृथि॒वीमु॒त द्याम्पु॑रु॒त्रा ते॑ मनुतां॒ विष्ठि॑तं॒ जग॑त्। स दु॑न्दुभे स॒जूरिन्द्रे॑ण दे॒वैर्दू॒रात्~(३२)

%4.6.6.7
दवी॑यो॒ अप॑ सेध॒ शत्रून्॑। आ क्र॑न्दय॒ बल॒मोजो॑ न॒ आ धा॒ नि ष्ट॑निहि दुरि॒ता बाध॑मानः। अप॑ प्रोथ दुन्दुभे दु॒च्छुनाꣳ॑ इ॒त इन्द्र॑स्य मु॒ष्टिर॑सि वी॒डय॑स्व। आमूर॑ज प्र॒त्याव॑र्तये॒माः के॑तु॒मद्दु॑न्दु॒भिर्वा॑वदीति। समश्व॑पर्णा॒श्चर॑न्ति नो॒ नरो॒\-ऽस्माक॑मिन्द्र र॒थिनो॑ जयन्तु॥~(३३)

%4.6.7.0
{\anuvakamend[{धन्व॑न्महि॒मानं॒ ब्राह्म॑णा॒सो\-ऽदि॑तिः पृथि॒व्याः परि॑ दू॒रादेक॑चत्वारिꣳशच्च}]}%~(६)

%4.6.7.1
यदक्र॑न्दः प्रथ॒मं जाय॑मान उ॒द्यन्थ्स॑मु॒द्रादु॒त वा॒ पुरी॑षात्। श्ये॒नस्य॑ प॒क्षा ह॑रि॒णस्य॑ बा॒हू उ॑प॒स्तुत्य॒म्महि॑ जा॒तं ते॑ अर्वन्न्। य॒मेन॑ द॒त्तं त्रि॒त ए॑नमायुन॒गिन्द्र॑ एणम्प्रथ॒मो अध्य॑तिष्ठत्। ग॒न्ध॒र्वो अ॑स्य रश॒नाम॑गृभ्णा॒थ्सूरा॒दश्वं॑ वसवो॒ निर॑तष्ट। असि॑ य॒मो अस्या॑दि॒त्यो अ॑र्व॒न्नसि॑ त्रि॒तो गुह्ये॑न व्र॒तेन॑। असि॒ सोमे॑न स॒मया॒ विपृ॑क्तः~(३४)

%4.6.7.2
आ॒हुस्ते॒ त्रीणि॑ दि॒वि बन्ध॑नानि। त्रीणि॑ त आहुर्दि॒वि बन्ध॑नानि॒ त्रीण्य॒फ्सु त्रीण्य॒न्तः स॑मु॒द्रे। उ॒तेव॑ मे॒ वरु॑णश्छन्थ्स्यर्व॒न् यत्रा॑ त आ॒हुः प॑र॒मं ज॒नित्रम्᳚। इ॒मा ते॑ वाजिन्नव॒मार्ज॑नानी॒मा श॒फानाꣳ॑ सनि॒तुर्नि॒धाना᳚। अत्रा॑ ते भ॒द्रा र॑श॒ना अ॑पश्यमृ॒तस्य॒ या अ॑भि॒रक्ष॑न्ति गो॒पाः। आ॒त्मानं॑ ते॒ मन॑सा॒राद॑जानाम॒वो दि॒वा~(३५)

%4.6.7.3
प॒तय॑न्तम्पतं॒गम्। शिरो॑ अपश्यम्प॒थिभिः॑ सु॒गेभि॑ररे॒णुभि॒र्जेह॑मानम्पत॒त्रि। अत्रा॑ ते रू॒पमु॑त्त॒मम॑पश्यं॒ जिगी॑षमाणमि॒ष आ प॒दे गोः। य॒दा ते॒ मर्तो॒ अनु॒ भोग॒मान॒डादिद्ग्रसि॑ष्ठ॒ ओष॑धीरजीगः। अनु॑ त्वा॒ रथो॒ अनु॒ मर्यो॑ अर्व॒न्ननु॒ गावो\-ऽनु॒ भगः॑ क॒नीना᳚म्। अनु॒ व्राता॑स॒स्तव॑ स॒ख्यमी॑यु॒रनु॑ दे॒वा म॑मिरे वी॒र्यम्᳚~(३६)

%4.6.7.4
ते॒। हिर॑ण्यशृ॒ङ्गो\-ऽयो॑ अस्य॒ पादा॒ मनो॑जवा॒ अव॑र॒ इन्द्र॑ आसीत्। दे॒वा इद॑स्य हवि॒रद्य॑माय॒न् यो अर्व॑न्तम्प्रथ॒मो अ॒ध्यति॑ष्ठत्। ई॒र्मान्ता॑सः॒ सिलि॑कमध्यमासः॒ सꣳ शूर॑णासो दि॒व्यासो॒ अत्याः᳚। ह॒ꣳ॒सा इ॑व श्रेणि॒शो य॑तन्ते॒ यदाक्षि॑षुर्दि॒व्यमज्म॒मश्वाः᳚। तव॒ शरी॑रम्पतयि॒ष्ण्व॑र्व॒न्तव॑ चि॒त्तं वात॑ इव॒ ध्रजी॑मान्। तव॒ शृङ्गा॑णि॒ विष्ठि॑ता पुरु॒त्रार॑ण्येषु॒ जर्भु॑राणा चरन्ति। उप॑~(३७)

%4.6.7.5
प्रागा॒च्छस॑नं वा॒ज्यर्वा॑ देव॒द्रीचा॒ मन॑सा॒ दीध्या॑नः। अ॒जः पु॒रो नी॑यते॒ नाभि॑र॒स्यानु॑ प॒श्चात्क॒वयो॑ यन्ति रे॒भाः। उप॒ प्रागा᳚त्पर॒मं यथ्स॒धस्थ॒मर्वा॒ꣳ॒ अच्छा॑ पि॒तर॑म्मा॒तरं॑ च। अ॒द्या दे॒वां जुष्ट॑तमो॒ हि ग॒म्या अथा शा᳚स्ते दा॒शुषे॒ वार्या॑णि॥~(३८)

%4.6.8.0
{\anuvakamend[{विपृ॑क्तो दि॒वा वी॒र्य॑मुपैका॒न्नच॑त्वारि॒ꣳ॒शच्च॑}]}%~(७)

%4.6.8.1
मा नो॑ मि॒त्रो वरु॑णो अर्य॒मायुरिन्द्र॑ ऋभु॒क्षा म॒रुतः॒ परि॑ ख्यन्न्। यद्वा॒जिनो॑ दे॒वजा॑तस्य॒ सप्तेः᳚ प्रव॒क्ष्यामो॑ वि॒दथे॑ वी॒र्या॑णि। यन्नि॒र्णिजा॒ रेक्ण॑सा॒ प्रावृ॑तस्य रा॒तिं गृ॑भी॒ताम्मु॑ख॒तो नय॑न्ति। सुप्रा॑ङ॒जो मेम्य॑द्वि॒श्वरू॑प इन्द्रापू॒ष्णोः प्रि॒यमप्ये॑ति॒ पाथः॑। ए॒ष च्छागः॑ पु॒रो अश्वे॑न वा॒जिना॑ पू॒ष्णो भा॒गो नी॑यते वि॒श्वदे᳚व्यः। अ॒भि॒प्रियं॒ यत्पु॑रो॒डाश॒मर्व॑ता॒ त्वष्टेत्~(३९)

%4.6.8.2
ए॒न॒ꣳ॒ सौ॒श्र॒व॒साय॑ जिन्वति। यद्ध॒विष्य॑मृतु॒शो दे॑व॒यानं॒ त्रिर्मानु॑षाः॒ पर्यश्वं॒ नय॑न्ति। अत्रा॑ पू॒ष्णः प्र॑थ॒मो भा॒ग ए॑ति य॒ज्ञं दे॒वेभ्यः॑ प्रतिवे॒दय॑न्न॒जः। होता᳚ध्व॒र्युराव॑या अग्निमि॒न्धो ग्रा॑वग्रा॒भ उ॒त शꣴस्ता॒ सुवि॑प्रः। तेन॑ य॒ज्ञेन॑ स्व॑रं कृतेन॒ स्वि॑ष्टेन व॒क्षणा॒ आ पृ॑णध्वम्। यू॒प॒व्र॒स्का उ॒त ये यू॑पवा॒हाश्च॒षालं॒ ये अ॑श्वयू॒पाय॒ तक्ष॑ति। ये चार्व॑ते॒ पच॑नꣳ स॒म्भर॑न्त्यु॒तो~(४०)

%4.6.8.3
तेषा॑म॒भिगू᳚र्तिर्न इन्वतु। उप॒ प्रागा᳚थ्सु॒मन्मे॑\-ऽधायि॒ मन्म॑ दे॒वाना॒माशा॒ उप॑ वी॒तपृ॑ष्ठः। अन्वे॑नं॒ विप्रा॒ ऋष॑यो मदन्ति दे॒वानां᳚ पु॒ष्टे च॑कृमा सु॒बन्धुम्᳚। यद्वा॒जिनो॒ दाम॑ सं॒दान॒मर्व॑तो॒ या शी॑र्\mbox{}ष॒ण्या॑ रश॒ना रज्जु॑रस्य। यद्वा॑ घास्य॒ प्रभृ॑तमा॒स्ये॑ तृण॒ꣳ॒ सर्वा॒ ता ते॒ अपि॑ दे॒वेष्व॑स्तु। यदश्व॑स्य क्र॒विषः॑~(४१)

%4.6.8.4
मक्षि॒काश॒ यद्वा॒ स्वरौ॒ स्वधि॑तौ रि॒प्तमस्ति॑। यद्धस्त॑योः शमि॒तुर्यन्न॒खेषु॒ सर्वा॒ ता ते॒ अपि॑ दे॒वेष्व॑स्तु। यदूव॑ध्यमु॒दर॑स्याप॒वाति॒ य आ॒मस्य॑ क्र॒विषो॑ ग॒न्धो अस्ति॑। सु॒कृ॒ता तच्छ॑मि॒तारः॑ कृण्वन्तू॒त मेधꣳ॑ शृत॒पाकं॑ पचन्तु। यत्ते॒ गात्रा॑द॒ग्निना॑ प॒च्यमा॑नाद॒भि शूलं॒ निह॑तस्याव॒धाव॑ति। मा तद्भूम्या॒मा श्रि॑ष॒न्मा तृणे॑षु दे॒वेभ्य॒स्तदु॒शद्भ्यो॑ रा॒तम॑स्तु॥~(४२)

%4.6.9.0
{\anuvakamend[{इदु॒तो क्र॒विषः॑ श्रिषथ्स॒प्त च॑}]}%~(८)

%4.6.9.1
ये वा॒जिनं॑ परि॒पश्य॑न्ति प॒क्वं य ई॑मा॒हुः सु॑र॒भिर्निर्\mbox{}ह॒रेति॑। ये चार्व॑तो माꣳसभि॒क्षामु॒पास॑त उ॒तो तेषा॑म॒भिगू᳚र्तिर्न इन्वतु। यन्नीक्ष॑णम्मा॒ꣳ॒स्पच॑न्या उ॒खाया॒ या पात्रा॑णि यू॒ष्ण आ॒सेच॑नानि। ऊ॒ष्म॒ण्या॑पि॒धाना॑ चरू॒णाम॒ङ्काः सू॒नाः परि॑ भूष॒न्त्यश्वम्᳚। नि॒क्रम॑णं नि॒षद॑नं वि॒वर्त॑नं॒ यच्च॒ पड्बी॑श॒मर्व॑तः। यच्च॑ प॒पौ यच्च॑ घा॒सिम्~(४३)

%4.6.9.2
ज॒घास॒ सर्वा॒ ता ते॒ अपि॑ दे॒वेष्व॑स्तु। मा त्वा॒ग्निर्ध्व॑नयिद्धू॒मग॑न्धि॒र्मोखा भ्राज॑न्त्य॒भि वि॑क्त॒ जघ्रिः॑। इ॒ष्टं वी॒तम॒भिगू᳚र्तं॒ वष॑ट्कृतं॒ तं दे॒वासः॒ प्रति॑ गृभ्ण॒न्त्यश्वम्᳚। यदश्वा॑य॒ वास॑ उपस्तृ॒णन्त्य॑धीवा॒सं या हिर॑ण्यान्यस्मै। सं॒दान॒मर्व॑न्त॒म्पड्बी॑शम्प्रि॒या दे॒वेष्वा या॑मयन्ति। यत्ते॑ सा॒दे मह॑सा॒ शूकृ॑तस्य॒ पार्ष्णि॑या वा॒ कश॑या~(४४)

%4.6.9.3
वा॒ तु॒तोद॑। स्रु॒चेव॒ ता ह॒विषो॑ अध्व॒रेषु॒ सर्वा॒ ता ते॒ ब्रह्म॑णा सूदयामि। चतु॑स्त्रिꣳशद्वा॒जिनो॑ दे॒वब॑न्धो॒र्वङ्क्री॒रश्व॑स्य॒ स्वधि॑तिः॒ समे॑ति। अच्छि॑द्रा॒ गात्रा॑ व॒युना॑ कृणोत॒ परु॑ष्परुरनु॒घुष्या॒ वि श॑स्त। एक॒स्त्वष्टु॒रश्व॑स्या विश॒स्ता द्वा य॒न्तारा॑ भवत॒स्तथ॒र्तुः। या ते॒ गात्रा॑णामृतु॒था कृ॒णोमि॒ ताता॒ पिण्डा॑ना॒म्प्र जु॑होम्य॒ग्नौ। मा त्वा॑ तपत्~(४५)

%4.6.9.4
प्रि॒य आ॒त्मापि॒यन्तं॒ मा स्वधि॑तिस्त॒नुव॒ आ ति॑ष्ठिपत्ते। मा ते॑ गृ॒ध्नुर॑विश॒स्ताति॒हाय॑ छि॒द्रा गात्रा॑ण्य॒सिना॒ मिथू॑ कः। न वा उ॑ वे॒तन्म्रि॑यसे॒ न रि॑ष्यसि दे॒वाꣳ इदे॑षि प॒थिभिः॑ सु॒गेभिः॑। हरी॑ ते॒ युञ्जा॒ पृष॑ती अभूता॒मुपा᳚स्थाद्वा॒जी धु॒रि रास॑भस्य। सु॒गव्यं॑ नो वा॒जी स्वश्वि॑यम्पु॒ꣳ॒सः पु॒त्राꣳ उ॒त वि॑श्वा॒पुषꣳ॑ र॒यिम्। अ॒ना॒गा॒स्त्वं नो॒ अदि॑तिः कृणोतु क्ष॒त्रं नो॒ अश्वो॑ वनताꣳ ह॒विष्मान्॑~(४६)

%4.7.0.0

{\anuvakamend[{घा॒सिं कश॑या तपद्र॒यिं नव॑ च}]}%~(९)
%%% END PRASHNA

\sect{सप्तमः प्रश्नः}\setcounter{anuvakam}{0}
\dnsub{तैत्तिरीयसंहितायां चतुर्थकाण्डे सप्तमः प्रश्नः}
%4.7.1.0
%4.7.1.1
अग्ना॑विष्णू स॒जोष॑से॒मा व॑र्धन्तु वां॒ गिरः॑। द्यु॒म्नैर्वाजे॑भि॒राग॑तम्। वाज॑श्च मे प्रस॒वश्च॑ मे॒ प्रय॑तिश्च मे॒ प्रसि॑तिश्च मे धी॒तिश्च॑ मे॒ क्रतु॑श्च मे॒ स्वर॑श्च मे॒ श्लोक॑श्च मे श्रा॒वश्च॑ मे॒ श्रुति॑श्च मे॒ ज्योति॑श्च मे॒ सुव॑श्च मे प्रा॒णश्च॑ मे\-ऽपा॒नः~(१)

%4.7.1.2
च॒ मे॒ व्या॒नश्च॒ मे\-ऽसु॑श्च मे चि॒त्तं च॑ म॒ आधी॑तं च मे॒ वाक्च॑ मे॒ मन॑श्च मे॒ चक्षु॑श्च मे॒ श्रोत्रं॑ च मे॒ दक्ष॑श्च मे॒ बलं॑ च म॒ ओज॑श्च मे॒ सह॑श्च म॒ आयु॑श्च मे ज॒रा च॑ म आ॒त्मा च॑ मे त॒नूश्च॑ मे॒ शर्म॑ च मे॒ वर्म॑ च॒ मे\-ऽङ्गा॑नि च मे॒\-ऽस्थानि॑ च मे॒ परूꣳ॑षि च मे॒ शरी॑राणि च मे॥~(२)

%4.7.2.0
{\anuvakamend[{अ॒पा॒नस्त॒नूश्च॑ मे॒\-ऽष्टाद॑श च}]}%~(१)

%4.7.2.1
ज्यैष्ठ्यं॑ च म॒ आधि॑पत्यं च मे म॒न्युश्च॑ मे॒ भाम॑श्च॒ मे\-ऽम॑श्च॒ मे\-ऽम्भ॑श्च मे जे॒मा च॑ मे महि॒मा च॑ मे वरि॒मा च॑ मे प्रथि॒मा च॑ मे व॒र्ष्मा च॑ मे द्राघु॒या च॑ मे वृ॒द्धं च॑ मे॒ वृद्धि॑श्च मे स॒त्यं च॑ मे श्र॒द्धा च॑ मे॒ जग॑च्च~(३)

%4.7.2.2
मे॒ धनं॑ च मे॒ वश॑श्च मे॒ त्विषि॑श्च मे क्री॒डा च॑ मे॒ मोद॑श्च मे जा॒तं च॑ मे जनि॒ष्यमा॑णं च मे सू॒क्तं च॑ मे सुकृ॒तं च॑ मे वि॒त्तं च॑ मे॒ वेद्यं॑ च मे भू॒तं च॑ मे भवि॒ष्यच्च॑ मे सु॒गं च॑ मे सु॒पथं॑ च म ऋ॒द्धं च॑ म॒ ऋद्धि॑श्च मे कॢ॒प्तं च॑ मे॒ कॢप्ति॑श्च मे म॒तिश्च॑ मे सुम॒तिश्च॑ मे॥~(४)

%4.7.3.0
{\anuvakamend[{जग॒च्चर्द्धि॒श्चतु॑र्दश च}]}%~(२)

%4.7.3.1
शं च॑ मे॒ मय॑श्च मे प्रि॒यं च॑ मे\-ऽनुका॒मश्च॑ मे॒ काम॑श्च मे सौमन॒सश्च॑ मे भ॒द्रं च॑ मे॒ श्रेय॑श्च मे॒ वस्य॑श्च मे॒ यश॑श्च मे॒ भग॑श्च मे॒ द्रवि॑णं च मे य॒न्ता च॑ मे ध॒र्ता च॑ मे॒ क्षेम॑श्च मे॒ धृति॑श्च मे॒ विश्वं॑ च~(५)

%4.7.3.2
मे॒ मह॑श्च मे सं॒विच्च॑ मे॒ ज्ञात्रं॑ च मे॒ सूश्च॑ मे प्र॒सूश्च॑ मे॒ सीरं॑ च मे ल॒यश्च॑ म ऋ॒तं च॑ मे॒\-ऽमृतं॑ च मे\-ऽय॒क्ष्मं च॒ मे\-ऽना॑मयच्च मे जी॒वातु॑श्च मे दीर्घायु॒त्वं च॑ मे\-ऽनमि॒त्रं च॒ मे\-ऽभ॑यं च मे सु॒गं च॑ मे॒ शय॑नं च मे सू॒षा च॑ मे सु॒दिनं॑ च मे॥~(६)

%4.7.4.0
{\anuvakamend[{विश्वं॑ च॒ शय॑नम॒ष्टौ च॑}]}%~(३)

%4.7.4.1
ऊर्क्च॑ मे सू॒नृता॑ च मे॒ पय॑श्च मे॒ रस॑श्च मे घृ॒तं च॑ मे॒ मधु॑ च मे॒ सग्धि॑श्च मे॒ सपी॑तिश्च मे कृ॒षिश्च॑ मे॒ वृष्टि॑श्च मे॒ जैत्रं॑ च म॒ औद्भि॑द्यं च मे र॒यिश्च॑ मे॒ राय॑श्च मे पु॒ष्टं च॑ मे॒ पुष्टि॑श्च मे वि॒भु च॑~(७)

%4.7.4.2
मे॒ प्र॒भु च॑ मे ब॒हु च॑ मे॒ भूय॑श्च मे पू॒र्णं च॑ मे पू॒र्णत॑रं च॒ मे\-ऽक्षि॑तिश्च मे॒ कूय॑वाश्च॒ मे\-ऽन्नं॑ च॒ मे\-ऽक्षु॑च्च मे व्री॒हय॑श्च मे॒ यवा᳚श्च मे॒ माषा᳚श्च मे॒ तिला᳚श्च मे मु॒द्गाश्च॑ मे ख॒ल्वा᳚श्च मे गो॒धूमा᳚श्च मे म॒सुरा᳚श्च मे प्रि॒यङ्ग॑वश्च॒ मे\-ऽण॑वश्च मे श्या॒माका᳚श्च मे नी॒वारा᳚श्च मे॥~(८)

%4.7.5.0
{\anuvakamend[{वि॒भु च॑ म॒सुरा॒श्चतु॑र्दश च}]}%~(४)

%4.7.5.1
अश्मा॑ च मे॒ मृत्ति॑का च मे गि॒रय॑श्च मे॒ पर्व॑ताश्च मे॒ सिक॑ताश्च मे॒ वन॒स्पत॑यश्च मे॒ हिर॑ण्यं च॒ मे\-ऽय॑श्च मे॒ सीसं॑ च मे॒ त्रपु॑श्च मे श्या॒मं च॑ मे लो॒हं च॑ मे॒\-ऽग्निश्च॑ म॒ आप॑श्च मे वी॒रुध॑श्च म॒ ओष॑धयश्च मे कृष्टप॒च्यं च॑~(९)

%4.7.5.2
मे॒\-ऽकृ॒ष्ट॒प॒च्यं च॑ मे ग्रा॒म्याश्च॑ मे प॒शव॑ आर॒ण्याश्च॑ य॒ज्ञेन॑ कल्पन्तां वि॒त्तं च॑ मे॒ वित्ति॑श्च मे भू॒तं च॑ मे॒ भूति॑श्च मे॒ वसु॑ च मे वस॒तिश्च॑ मे॒ कर्म॑ च मे॒ शक्ति॑श्च॒ मे\-ऽर्थ॑श्च म॒ एम॑श्च म॒ इति॑श्च मे॒ गति॑श्च मे॥~(१०)

%4.7.6.0
{\anuvakamend[{कृ॒ष्ट॒प॒च्यञ्चा॒ष्टाच॑त्वारिꣳशच्च}]}%~(५)

%4.7.6.1
अ॒ग्निश्च॑ म॒ इन्द्र॑श्च मे॒ सोम॑श्च म॒ इन्द्र॑श्च मे सवि॒ता च॑ म॒ इन्द्र॑श्च मे॒ सर॑स्वती च म॒ इन्द्र॑श्च मे पू॒षा च॑ म॒ इन्द्र॑श्च मे॒ बृह॒स्पति॑श्च म॒ इन्द्र॑श्च मे मि॒त्रश्च॑ म॒ इन्द्र॑श्च मे॒ वरु॑णश्च म॒ इन्द्र॑श्च मे॒ त्वष्टा॑ च~(११)

%4.7.6.2
म॒ इन्द्र॑श्च मे धा॒ता च॑ म॒ इन्द्र॑श्च मे॒ विष्णु॑श्च म॒ इन्द्र॑श्च मे॒\-ऽश्विनौ॑ च म॒ इन्द्र॑श्च मे म॒रुत॑श्च म॒ इन्द्र॑श्च मे॒ विश्वे॑ च मे दे॒वा इन्द्र॑श्च मे पृथि॒वी च॑ म॒ इन्द्र॑श्च मे॒\-ऽन्तरि॑क्षञ्च म॒ इन्द्र॑श्च मे॒ द्यौश्च॑ म॒ इन्द्र॑श्च मे॒ दिश॑श्च म॒ इन्द्र॑श्च मे मू॒र्धा च॑ म॒ इन्द्र॑श्च मे प्र॒जाप॑तिश्च म॒ इन्द्र॑श्च मे॥~(१२)

%4.7.7.0
{\anuvakamend[{त्वष्टा॑ च॒ द्यौश्च॑ म॒ एक॑विꣳशतिश्च}]}%~(६)

%4.7.7.1
अ॒ꣳ॒शुश्च॑ मे र॒श्मिश्च॒ मे\-ऽदा᳚भ्यश्च॒ मे\-ऽधि॑पतिश्च म उपा॒ꣳ॒शुश्च॑ मे\-ऽन्तर्या॒मश्च॑ म ऐन्द्रवाय॒वश्च॑ मे मैत्रावरु॒णश्च॑ म आश्वि॒नश्च॑ मे प्रतिप्र॒स्थान॑श्च मे शु॒क्रश्च॑ मे म॒न्थी च॑ म आग्रय॒णश्च॑ मे वैश्वदे॒वश्च॑ मे ध्रु॒वश्च॑ मे वैश्वान॒रश्च॑ म ऋतुग्र॒हाश्च॑~(१३)

%4.7.7.2
मे॒\-ऽति॒ग्रा॒ह्या᳚श्च म ऐन्द्रा॒ग्नश्च॑ मे वैश्वदे॒वश्च॑ मे मरुत्व॒तीया᳚श्च मे माहे॒न्द्रश्च॑ म आदि॒त्यश्च॑ मे सावि॒त्रश्च॑ मे सारस्व॒तश्च॑ मे पौ॒ष्णश्च॑ मे पात्नीव॒तश्च॑ मे हारियोज॒नश्च॑ मे॥~(१४)

%4.7.8.0
{\anuvakamend[{ऋ॒तु॒ग्र॒हाश्च॒ चतु॑स्त्रिꣳशच्च}]}%~(७)

%4.7.8.1
इ॒ध्मश्च॑ मे ब॒र्\mbox{}हिश्च॑ मे॒ वेदि॑श्च मे॒ धिष्णि॑याश्च मे॒ स्रुच॑श्च मे चम॒साश्च॑ मे॒ ग्रावा॑णश्च मे॒ स्वर॑वश्च म उपर॒वाश्च॑ मे\-ऽधि॒षव॑णे च मे द्रोणकल॒शश्च॑ मे वाय॒व्या॑नि च मे पूत॒भृच्च॑ म आधव॒नीय॑श्च म॒ आग्नी᳚ध्रं च मे हवि॒र्धानं॑ च मे गृ॒हाश्च॑ मे॒ सद॑श्च मे पुरो॒डाशा᳚श्च मे पच॒ताश्च॑ मे\-ऽवभृ॒थश्च॑ मे स्वगाका॒रश्च॑ मे॥~(१५)

%4.7.9.0
{\anuvakamend[{गृ॒हाश्च॒ षोड॑श च}]}%~(८)

%4.7.9.1
अ॒ग्निश्च॑ मे घ॒र्मश्च॑ मे॒\-ऽर्कश्च॑ मे॒ सूर्य॑श्च मे प्रा॒णश्च॑ मे\-ऽश्वमे॒धश्च॑ मे पृथि॒वी च॒ मे\-ऽदि॑तिश्च मे॒ दिति॑श्च मे॒ द्यौश्च॑ मे॒ शक्व॑रीर॒ङ्गुल॑यो॒ दिश॑श्च मे य॒ज्ञेन॑ कल्पन्ता॒मृक्च॑ मे॒ साम॑ च मे॒ स्तोम॑श्च मे॒ यजु॑श्च मे दी॒क्षा च॑ मे॒ तप॑श्च म ऋ॒तुश्च॑ मे व्र॒तं च॑ मे\-ऽहोरा॒त्रयो᳚र्वृ॒ष्ट्या बृ॑हद्रथन्त॒रे च॑ मे य॒ज्ञेन॑ कल्पेताम्॥~(१६)

%4.7.10.0
{\anuvakamend[{दी॒क्षा\-ऽष्टाद॑श च}]}%~(९)

%4.7.10.1
गर्भा᳚श्च मे व॒थ्साश्च॑ मे॒ त्र्यवि॑श्च मे त्र्य॒वी च॑ मे दित्य॒वाट्च॑ मे दित्यौ॒ही च॑ मे॒ पञ्चा॑विश्च मे पञ्चा॒वी च॑ मे त्रिव॒थ्सश्च॑ मे त्रिव॒थ्सा च॑ मे तुर्य॒वाट्च॑ मे तुर्यौ॒ही च॑ मे पष्ठ॒वाच्च॑ मे पष्ठौ॒ही च॑ म उ॒क्षा च॑ मे व॒शा च॑ म ऋष॒भश्च॑~(१७)

%4.7.10.2
मे॒ वे॒हच्चमे\-ऽन॒ड्वां च॑ मे धे॒नुश्च॑ म॒ आयु॑र्य॒ज्ञेन॑ कल्पतां प्रा॒णो य॒ज्ञेन॑ कल्पतामपा॒नो य॒ज्ञेन॑ कल्पताव्व्याँ॒नो य॒ज्ञेन॑ कल्पतां॒ चक्षु॑र्य॒ज्ञेन॑ कल्पता॒ꣴ॒ श्रोत्रं॑ य॒ज्ञेन॑ कल्पता॒म्मनो॑ य॒ज्ञेन॑ कल्पतां॒ वाग्य॒ज्ञेन॑ कल्पतामा॒त्मा य॒ज्ञेन॑ कल्पतां य॒ज्ञो य॒ज्ञेन॑ कल्पताम्॥~(१८)

%4.7.11.0
{\anuvakamend[{ऋ॒ष॒भश्च॑ चत्वारि॒ꣳ॒शच्च॑}]}%॥10॥

%4.7.11.1
एका॑ च मे ति॒स्रश्च॑ मे॒ पञ्च॑ च मे स॒प्त च॑ मे॒ नव॑ च म॒ एका॑\-दश च मे॒ त्रयो॑दश च मे॒ पञ्च॑दश च मे स॒प्तद॑श च मे॒ नव॑दश च म॒ एक॑विꣳशतिश्च मे॒ त्रयो॑विꣳशतिश्च मे॒ पञ्च॑विꣳशतिश्च मे स॒प्तविꣳ॑शतिश्च मे॒ नव॑विꣳशतिश्च म॒ एक॑त्रिꣳशच्च मे॒ त्रय॑स्त्रिꣳशच्च~(१९)

%4.7.11.2
मे॒ चत॑स्रश्च मे॒\-ऽष्टौ च॑ मे॒ द्वाद॑श च मे॒ षोड॑श च मे विꣳश॒तिश्च॑ मे॒ चतु॑र्विꣳशतिश्च मे॒\-ऽष्टाविꣳ॑शतिश्च मे॒ द्वात्रिꣳ॑शच्च मे॒ षट्त्रिꣳ॑शच्च मे चत्वारि॒ꣳ॒शच्च॑ मे॒ चतु॑श्चत्वारिꣳशच्च मे॒\-ऽष्टाच॑त्वारिꣳशच्च मे॒ वाज॑श्च प्रस॒वश्चा॑पि॒जश्च॒ क्रतु॑श्च॒ सुव॑श्च मू॒र्धा च॒ व्यश्नि॑यश्चान्त्याय॒नश्चान्त्य॑श्च भौव॒नश्च॒ भुव॑न॒श्चाधि॑पतिश्च॥~(२०)

%4.7.12.0
{\anuvakamend[{त्रय॑स्त्रिꣳशच्च॒ व्यश्ञि॑य॒ एका॑\-दश च}]}%॥11॥

%4.7.12.1
वाजो॑ नः स॒प्त प्र॒दिश॒श्चत॑स्रो वा परा॒वतः॑। वाजो॑ नो॒ विश्वै᳚र्दे॒वैर्धन॑सातावि॒हाव॑तु। विश्वे॑ अ॒द्य म॒रुतो॒ विश्व॑ ऊ॒ती विश्वे॑ भवन्त्व॒ग्नयः॒ समि॑द्धाः। विश्वे॑ नो दे॒वा अव॒सा ग॑मन्तु॒ विश्व॑मस्तु॒ द्रवि॑णं॒ वाजो॑ अ॒स्मे। वाज॑स्य प्रस॒वं दे॑वा॒ रथै᳚र्याता हिर॒ण्ययैः᳚। अ॒ग्निरिन्द्रो॒ बृह॒स्पति॑र्म॒रुतः॒ सोम॑पीतये। वाजे॑वाजे\-ऽवत वाजिनो नो॒ धने॑षु~(२१)

%4.7.12.2
वि॒प्रा॒ अ॒मृ॒ता॒ ऋ॒त॒ज्ञाः॒। अ॒स्य मध्वः॑ पिबत मा॒दय॑ध्वं तृ॒प्ता या॑त प॒थिभि॑र्देव॒यानैः᳚। वाजः॑ पु॒रस्ता॑दु॒त म॑ध्य॒तो नो॒ वाजो॑ दे॒वाꣳ ऋ॒तुभिः॑ कल्पयाति। वाज॑स्य॒ हि प्र॑स॒वो नन्न॑मीति॒ विश्वा॒ आशा॒ वाज॑पतिर्भवेयम्। पयः॑ पृथि॒व्याम्पय॒ ओष॑धीषु॒ पयो॑ दिव्य॒न्तरि॑क्षे॒ पयो॑ धाम्। पय॑स्वतीः प्र॒दिशः॑ सन्तु॒ मह्यम्᳚। सम्मा॑ सृजामि॒ पय॑सा घृ॒तेन॒ सम्मा॑ सृजाम्य॒पः~(२२)

%4.7.12.3
ओष॑धीभिः। सो॑\-ऽहं वाजꣳ॑ सनेयमग्ने। नक्तो॒षासा॒ सम॑नसा॒ विरू॑पे धा॒पये॑ते॒ शिशु॒मेकꣳ॑ समी॒ची। द्यावा॒ क्षामा॑ रु॒क्मो अ॒न्तर्वि भा॑ति दे॒वा अ॒ग्निं धा॑रयन्द्रविणो॒दाः। स॒मु॒द्रो॑\-ऽसि॒ नभ॑स्वाना॒र्द्रदा॑नुः श॒म्भूर्म॑यो॒भूर॒भि मा॑ वाहि॒ स्वाहा॑ मारु॒तो॑\-ऽसि म॒रुतां᳚ ग॒णः श॒म्भूर्म॑यो॒भूर॒भि मा॑ वाहि॒ स्वाहा॑व॒स्युर॑सि॒ दुव॑स्वाञ्छ॒म्भूर्म॑यो॒भूर॒भि मा॑ वाहि॒ स्वाहा᳚॥~(२३)

%4.7.13.0
{\anuvakamend[{धने᳚ष्व॒पो दुव॑स्वाञ्छ॒म्भूर्म॑यो॒भूर॒भि मा॒ द्वे च॑}]}%॥12॥

%4.7.13.1
अ॒ग्निं यु॑नज्मि॒ शव॑सा घृ॒तेन॑ दि॒व्यꣳ सु॑प॒र्णं वय॑सा बृ॒हन्तम्᳚। तेन॑ व॒यं प॑तेम ब्र॒ध्नस्य॑ वि॒ष्टप॒ꣳ॒ सुवो॒ रुहा॑णा॒ अधि॒ नाक॑ उत्त॒मे। इ॒मौ ते॑ प॒क्षाव॒जरौ॑ पत॒त्रिणो॒ याभ्या॒ꣳ॒ रक्षाꣴ॑स्यप॒हꣴस्य॑ग्ने। ता\-भ्यां᳚ पतेम सु॒कृता॑मु लो॒कं यत्रर्\mbox{}ष॑यः प्रथम॒जा ये पु॑रा॒णाः। चिद॑सि समु॒द्रयो॑नि॒रिन्दु॒र्दक्षः॑ श्ये॒न ऋ॒तावा᳚। हिर॑ण्यपक्षः शकु॒नो भु॑र॒ण्युर्म॒हान्थ्स॒धस्थे᳚ ध्रु॒वः~(२४)

%4.7.13.2
आ निष॑त्तः। नम॑स्ते अस्तु॒ मा मा॑ हिꣳसी॒र्विश्व॑स्य मू॒र्धन्नधि॑ तिष्ठसि श्रि॒तः। स॒मु॒द्रे ते॒ हृद॑यम॒न्तरायु॒र्द्यावा॑पृथि॒वी भुव॑ने॒ष्वर्पि॑ते। उ॒द्नो द॑त्तोद॒धिम्भि॑न्त दि॒वः प॒र्जन्या॑द॒न्तरि॑क्षात्पृथि॒व्यास्ततो॑ नो॒ वृष्ट्या॑वत। दि॒वो मू॒र्धासि॑ पृथि॒व्या नाभि॒रूर्ग॒पामोष॑धीनाम्। वि॒श्वायुः॒ शर्म॑ स॒प्रथा॒ नम॑स्प॒थे। येनर्\mbox{}ष॑य॒स्तप॑सा स॒त्त्रम्~(२५)

%4.7.13.3
आस॒तेन्धा॑ना अ॒ग्निꣳ सुव॑रा॒भर॑न्तः। तस्मि॑न्न॒हं नि द॑धे॒ नाके॑ अ॒ग्निमे॒तं यमा॒हुर्मन॑वः स्ती॒र्णब॑र्\mbox{}हिषम्। तम्पत्नी॑भि॒रनु॑ गच्छेम देवाः पु॒त्रैर्भ्रातृ॑भिरु॒त वा॒ हिर॑ण्यैः। नाकं॑ गृह्णा॒नाः सु॑कृ॒तस्य॑ लो॒के तृ॒तीये॑ पृ॒ष्ठे अधि॑ रोच॒ने दि॒वः। आ वा॒चो मध्य॑मरुहद्भुर॒ण्युर॒यम॒ग्निः सत्प॑ति॒श्चेकि॑तानः। पृ॒ष्ठे पृ॑थि॒व्या निहि॑तो॒ दवि॑द्युतदधस्प॒दं कृ॑णुते~(२६)

%4.7.13.4
ये पृ॑त॒न्यवः॑। अ॒यम॒ग्निर्वी॒रत॑मो वयो॒धाः स॑ह॒स्रियो॑ दीप्यता॒मप्र॑युच्छन्न्। वि॒भ्राज॑मानः सरि॒रस्य॒ मध्य॒ उप॒ प्र या॑त दि॒व्यानि॒ धाम॑। सम्प्र च्य॑वध्व॒मनु॒ सम्प्र या॒ताग्ने॑ प॒थो दे॑व॒याना᳚न्कृणुध्वम्। अ॒स्मिन्थ्स॒धस्थे॒ अध्युत्त॑रस्मि॒न्विश्वे॑ देवा॒ यज॑मानश्च सीदत। येना॑ स॒हस्रं॒ वह॑सि॒ येना᳚ग्ने सर्ववेद॒सम्। तेने॒मं य॒ज्ञं नो॑ वह देव॒यानो॒ यः~(२७)

%4.7.13.5
उ॒त्त॒मः। उद्बु॑ध्यस्वाग्ने॒ प्रति॑ जागृह्येनमिष्टापू॒र्ते सꣳसृ॑जेथाम॒यं च॑। पुनः॑ कृ॒ण्वꣴस्त्वा॑ पि॒तरं॒ युवा॑नम॒न्वाताꣳ॑सी॒त् त्वयि॒ तन्तु॑मे॒तम्। अ॒यं ते॒ योनि॑र्\mbox{}ऋ॒त्वियो॒ यतो॑ जा॒तो अरो॑चथाः। तं जा॒नन्न॑ग्न॒ आ रो॒हाथा॑ नो वर्धया र॒यिम्॥~(२८)

%4.7.14.0
{\anuvakamend[{ध्रु॒वः स॒त्रं कृ॑णुते॒ यः स॒प्तत्रिꣳ॑शच्च}]}%॥13॥

%4.7.14.1
ममा᳚ग्ने॒ वर्चो॑ विह॒वेष्व॑स्तु व॒यं त्वेन्धा॑नास्त॒नुव॑म्पुषेम। मह्यं॑ नमन्ताम्प्र॒दिश॒श्चत॑स्र॒स्त्वयाध्य॑क्षेण॒ पृत॑ना जयेम। मम॑ दे॒वा वि॑ह॒वे स॑न्तु॒ सर्व॒ इन्द्रा॑वन्तो म॒रुतो॒ विष्णु॑र॒ग्निः। ममा॒न्तरि॑क्षमु॒रु गो॒पम॑स्तु॒ मह्यं॒ वातः॑ पवतां॒ कामे॑ अ॒स्मिन्न्। मयि॑ दे॒वा द्रवि॑ण॒मा य॑जन्ता॒म्मय्या॒शीर॑स्तु॒ मयि॑ दे॒वहू॑तिः। दैव्या॒ होता॑रा वनिषन्त~(२९)

%4.7.14.2
पूर्वे\-ऽरि॑ष्टाः स्याम त॒नुवा॑ सु॒वीराः᳚। मह्यं॑ यजन्तु॒ मम॒ यानि॑ ह॒व्याकू॑तिः स॒त्या मन॑सो मे अस्तु। एनो॒ मा नि गां᳚ कत॒मच्च॒नाहं विश्वे॑ देवासो॒ अधि॑ वोचता मे। देवीः᳚ षडुर्वीरु॒रु णः॑ कृणोत॒ विश्वे॑ देवास इ॒ह वी॑रयध्वम्। मा हा᳚स्महि प्र॒जया॒ मा त॒नूभि॒र्मा र॑धाम द्विष॒ते सो॑म राजन्न्। अ॒ग्निर्म॒न्युम्प्र॑तिनु॒दन्पु॒रस्ता᳚त्~(३०)

%4.7.14.3
अद॑ब्धो गो॒पाः परि॑ पाहि न॒स्त्वम्। प्र॒त्यञ्चो॑ यन्तु नि॒गुतः॒ पुन॒स्ते॑\-ऽमैषां᳚ चि॒त्तम्प्र॒बुधा॒ वि ने॑शत्। धा॒ता धा॑तृ॒णाम्भुव॑नस्य॒ यस्पति॑र्दे॒वꣳ स॑वि॒तार॑मभिमाति॒षाहम्᳚। इ॒मं य॒ज्ञम॒श्विनो॒भा बृह॒स्पति॑र्दे॒वाः पा᳚न्तु॒ यज॑मानं न्य॒र्थात्। उ॒रु॒व्यचा॑ नो महि॒षः शर्म॑ यꣳसद॒स्मिन् हवे॑ पुरुहू॒तः पु॑रु॒क्षु। स नः॑ प्र॒जायै॑ हर्यश्व मृड॒येन्द्र॒ मा~(३१)

%4.7.14.4
नो॒ री॒रि॒षो॒ मा परा॑ दाः। ये नः॑ स॒पत्ना॒ अप॒ ते भ॑वन्त्विन्द्रा॒ग्निभ्या॒मव॑ बाधामहे॒ तान्। वस॑वो रु॒द्रा आ॑दि॒त्या उ॑परि॒स्पृश॑म्मो॒ग्रं चेत्ता॑रमधिरा॒जम॑क्रन्न्। अ॒र्वाञ्च॒मिन्द्र॑म॒मुतो॑ हवामहे॒ यो गो॒जिद्ध॑न॒जिद॑श्व॒जिद्यः। इ॒मं नो॑ य॒ज्ञं वि॑ह॒वे जु॑षस्वा॒स्य कु॑र्मो हरिवो मे॒दिनं॑ त्वा॥~(३२)

%4.7.15.0
{\anuvakamend[{व॒नि॒ष॒न्त॒ पु॒रस्ता॒न्मा त्रिच॑त्वारिꣳशच्च}]}%॥14॥

%4.7.15.1
अ॒ग्नेर्म॑न्वे प्रथ॒मस्य॒ प्रचे॑तसो॒ यम्पाञ्च॑जन्यम्ब॒हवः॑ समि॒न्धते᳚। विश्व॑स्यां वि॒शि प्र॑विविशि॒वाꣳस॑मीमहे॒ स नो॑ मुञ्च॒त्वꣳह॑सः। यस्ये॒दं प्रा॒णन्नि॑मि॒षद्यदेज॑ति॒ यस्य॑ जा॒तं जन॑मानं च॒ केव॑लम्। स्तौम्य॒ग्निं ना॑थि॒तो जो॑हवीमि॒ स नो॑ मुञ्च॒त्वꣳह॑सः। इन्द्र॑स्य मन्ये प्रथ॒मस्य॒ प्रचे॑तसो वृत्र॒घ्नः स्तोमा॒ उप॒ मामु॒पागुः॑। यो दा॒शुषः॑ सु॒कृतो॒ हव॒मुप॒ गन्ता᳚~(३३)

%4.7.15.2
स नो॑ मुञ्च॒त्वꣳह॑सः। यः स॑ङ्ग्रमं नय॑ति॒ सं व॒शी यु॒धे यः पु॒ष्टानि॑ सꣳसृ॒जति॑ त्र॒याणि॑। स्तौमीन्द्रं॑ नाथि॒तो जो॑हवीमि॒ स नो॑ मुञ्च॒त्वꣳह॑सः। म॒न्वे वा᳚म्मित्रावरुणा॒ तस्य॑ वित्त॒ꣳ॒ सत्यौ॑जसा दृꣳहणा॒ यं नु॒देथे᳚। या राजा॑नꣳ स॒रथं॑ या॒थ उ॑ग्रा॒ ता नो॑ मुञ्चत॒माग॑सः। यो वा॒ꣳ॒ रथ॑ ऋ॒जुर॑श्मिः स॒त्यध॑र्मा॒ मिथु॒श्चर॑न्तमुप॒याति॑ दू॒षयन्न्॑। स्तौमि॑~(३४)

%4.7.15.3
मि॒त्रावरु॑णा नाथि॒तो जो॑हवीमि॒ तौ नो॑ मुञ्चत॒माग॑सः। वा॒योः स॑वि॒तुर्वि॒दथा॑नि मन्महे॒ यावा᳚त्म॒न्वद्बि॑भृ॒तो यौ च॒ रक्ष॑तः। यौ विश्व॑स्य परि॒भू ब॑भू॒वतु॒स्तौ नो॑ मुञ्चत॒माग॑सः। उप॒ श्रेष्ठा॑ न आ॒शिषो॑ दे॒वयो॒र्धर्मे॑ अस्थिरन्न्। स्तौमि॑ वा॒युꣳ स॑वि॒तारं॑ नाथि॒तो जो॑हवीमि॒ तौ नो॑ मुञ्चत॒माग॑सः। र॒थीत॑मौ रथी॒नाम॑ह्व ऊ॒तये॒ शुभं॒ गमि॑ष्ठौ सु॒यमे॑भि॒रश्वैः᳚। ययोः᳚~(३५)

%4.7.15.4
वां॒ दे॒वौ॒ दे॒वेष्वनि॑शित॒मोज॒स्तौ नो॑ मुञ्चत॒माग॑सः। यदया॑तं वह॒तुꣳ सू॒र्याया᳚स्त्रिच॒क्रेण॑ स॒ꣳ॒सद॑मि॒च्छमा॑नौ। स्तौमि॑ दे॒वाव॒श्विनौ॑ नाथि॒तो जो॑हवीमि॒ तौ नो॑ मुञ्चत॒माग॑सः। म॒रुता᳚म्मन्वे॒ अधि॑ नो ब्रुवन्तु॒ प्रेमां वाचं॒ विश्वा॑मवन्तु॒ विश्वे᳚। आ॒शून् हु॑वे सु॒यमा॑नू॒तये॒ ते नो॑ मुञ्च॒न्त्वेन॑सः। ति॒ग्ममायु॑धं वीडि॒तꣳ सह॑स्वद्दि॒व्यꣳ शर्धः॑~(३६)

%4.7.15.5
पृत॑नासु जि॒ष्णु। स्तौमि॑ दे॒वान्म॒रुतो॑ नाथि॒तो जो॑हवीमि॒ ते नो॑ मुञ्च॒न्त्वेन॑सः। दे॒वाना᳚म्मन्वे॒ अधि॑ नो ब्रुवन्तु॒ प्रेमां वाचं॒ विश्वा॑मवन्तु॒ विश्वे᳚। आ॒शून् हु॑वे सु॒यमा॑नू॒तये॒ ते नो॑ मुञ्च॒न्त्वेन॑सः। यदि॒दम्मा॑भि॒शोच॑ति॒ पौरु॑षेयेण॒ दैव्ये॑न। स्तौमि॒ विश्वां᳚ दे॒वान्ना॑थि॒तो जो॑हवीमि॒ ते नो॑ मुञ्च॒न्त्वेन॑सः। अनु॑ नो॒\-ऽद्यानु॑मति॒रनु॑~(३७)

%4.7.15.6
इद॑नुमते॒ त्वं वै᳚श्वान॒रो न॑ ऊ॒त्या पृ॒ष्टो दि॒वि। ये अप्र॑थेता॒ममि॑तेभि॒रोजो॑भि॒र्ये प्र॑ति॒ष्ठे अभ॑वतां॒ वसू॑नाम्। स्तौमि॒ द्यावा॑पृथि॒वी ना॑थि॒तो जो॑हवीमि॒ ते नो॑ मुञ्चत॒मꣳह॑सः। उर्वी॑ रोदसी॒ वरि॑वः कृणोतं॒ क्षेत्र॑स्य पत्नी॒ अधि॑ नो ब्रूयातम्। स्तौमि॒ द्यावा॑पृथि॒वी ना॑थि॒तो जो॑हवीमि॒ ते नो॑ मुञ्चत॒मꣳह॑सः। यत्ते॑ व॒यं पु॑रुष॒त्रा य॑वि॒ष्ठावि॑द्वाꣳसश्चकृ॒मा कच्च॒न~(३८)

%4.7.15.7
आगः॑। कृ॒धी स्व॑स्माꣳ अदि॑ते॒रना॑गा॒ व्येनाꣳ॑सि शिश्रथो॒ विष्व॑गग्ने। यथा॑ ह॒ तद्व॑सवो गौ॒र्यं॑ चित्प॒दि षि॒ताममु॑ञ्चता यजत्राः। ए॒वा त्वम॒स्मत्प्र मु॑ञ्चा॒ व्यꣳहः॒ प्राता᳚र्यग्ने प्रत॒रां न॒ आयुः॑~(३९)

%5.1.0.0

%5.1.0.0
{\anuvakamend[{गन्ता॑ दू॒षय॒न्थ्स्तौमि॒ ययोः॒ शर्धो\-ऽनु॑मति॒रनु॑ च॒न चतु॑स्त्रिꣳशच्च}]}%॥15॥

{\anuvakamend[{अ॒ग्निष्ट्वा॑ वा॒मश्वो॒ द्विच॑त्वारिꣳशच्च}]}%॥11॥
%%% END KANDAM
