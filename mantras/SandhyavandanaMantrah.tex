% !TeX program = XeLaTeX
% !TeX root = ../vedamantrabook.tex
\sect{प्रातः सन्ध्यावन्दनम् प्राशनम्}

सूर्यश्च मा मन्युश्च मन्युपतयश्च मन्यु॑कृते॒भ्यः। पापेभ्यो॑ रक्ष॒न्ताम्। यद्रात्रिया पाप॑मका॒र्‌षम्। मनसा वाचा॑ हस्ता॒भ्याम्। पद्भ्यामुदरे॑ण शि॒श्ञा। रात्रि॒स्तद॑वलु॒म्पतु। यत्किं च॑ दुरि॒तं मयि॑। इदमहं माममृ॑तयो॒नौ। सूर्ये ज्योतिषि जुहो॑मि स्वा॒हा॥

\sect{माध्याह्निकम् प्राशनम्}

आपः॑ पुनन्तु पृथि॒वीं पृ॑थि॒वी पू॒ता पु॑नातु॒ माम्। पु॒नन्तु॒ ब्रह्म॑ण॒स्पति॒र्ब्रह्म॑पू॒ता पु॑नातु॒ माम्। यदुच्छि॑ष्ट॒मभो᳚ज्यं॒ यद्वा॑ दु॒श्चरि॑तं॒ मम॑। सर्वं॑ पुनन्तु॒ मामापो॑ऽस॒तां च॑ प्रति॒ग्रह॒ꣴ॒ स्वाहा᳚॥

\sect{सायं सन्ध्यावन्दनम् प्राशनम्}

अग्निश्च मा मन्युश्च मन्युपतयश्च मन्यु॑कृते॒भ्यः। पापेभ्यो॑ रक्ष॒न्ताम्। यदह्ना पाप॑मका॒र्‌षम्। मनसा वाचा॑ हस्ता॒भ्याम्। पद्भ्यामुदरे॑ण शि॒श्ञा। अह॒स्तद॑वलु॒म्पतु। यत्किं च॑ दुरि॒तं मयि॑। इदमहं माममृ॑तयो॒नौ। सत्ये ज्योतिषि जुहो॑मि स्वा॒हा॥४८॥


\sect{पुनर्मार्जनम्}

द॒धि॒क्राव्ण्णो॑ अकारिषम्।\\
 जि॒ष्णोरश्व॑स्य वा॒जिनः॑।\\
सु॒र॒भि नो॒ मुखा॑कर॒त्।\\
प्रण॒ आयूꣳ॑षि तारिषत्॥\\



आपो॒ हि ष्ठा म॑यो॒ भुवः॑।\\
ता न॑ ऊ॒र्जे द॑धातन।\\
म॒हेरणा॑य॒ चक्ष॑से।\\
यो वः॑ शि॒वत॑मो॒ रसः॑।\\
तस्य॑ भाजयते॒ह नः॑।\\
उ॒श॒तीरि॑व मा॒तरः॑।\\
तस्मा॒ अरं॑ गमाम वः।\\
यस्य॒ क्षया॑य॒ जिन्व॑थ।\\
आपो॑ ज॒नय॑था च नः॥\\

\sect{प्रणवजपः—प्राणायामः}

ओं भूः। ओं भुव। ओꣳ सुव। ओं मह। ओं जन। ओं तप। ओꣳ स॒त्यम्॥
ओं तथ्स॑वि॒तुर्वरे᳚ण्यं॒ भर्गो॑ दे॒वस्य॑ धीमहि। धियो॒ यो नः॑ प्रचो॒दया᳚त्॥
ओमापो॒ ज्योती॒रसो॒ऽमृतं॒ ब्रह्म॒ भूर्भुवः॒ सुव॒रोम्॥

\sect{गायत्री-आवाहनम्}

आया॑तु॒ वर॑दा दे॒वी॒ अ॒क्षरं॑ ब्रह्म॒सम्मि॑तम्। गा॒य॒त्रीं᳚ छन्द॑सां मा॒तेदं ब्र॑ह्म जु॒षस्व॑ नः॥

ओजो॑ऽसि॒ सहो॑ऽसि॒ बल॑मसि॒ भ्राजो॑ऽसि दे॒वानां॒ धाम॒ नामा॑सि॒ विश्व॑मसि वि॒श्वायुः॒ सर्व॑मसि स॒र्वायुरभिभूरों गायत्रीमावा॑हया॒मि॒ सावित्रीमावा॑हया॒मि॒ सरस्वतीमावा॑ह\-या॒मि॒ सावित्र्या ऋषिर्विश्वामित्रः। निचृद्गायत्री छन्दः। सविता देवता।

\sect{गायत्री-जपः}

ओं।\\
भूर्भुवः॒ सुवः॑।\\
तथ्स॑वि॒तुर्वरे᳚ण्यम्।\\
भर्गो॑ दे॒वस्य॑ धीमहि।\\
धियो॒ यो नः॑ प्रचो॒दया᳚त्॥

\sect{गायत्री-उपस्थानम्}

उ॒त्तमे शिख॑रे दे॒वी॒ भू॒म्यां प॑र्वत॒मूर्ध॑नि।\\
ब्रा॒ह्मणे॑॑भ्यो ह्य॑नुज्ञा॒नं॒ ग॒च्छ दे॑वि य॒था सु॑खम्॥


\sect{प्रातः सन्ध्या सूर्योपस्थानम्}
मि॒त्रस्य॑ चर्\mbox{}षणी॒ धृतः॒ श्रवो॑ दे॒वस्य॑ सान॒सिम्। स॒त्यं चि॒त्रश्र॑वस्तमम्॥ मि॒त्रो जनान्॑ यातयति प्रजा॒नन्मि॒त्रो दा॑धार पृथि॒वीमु॒तद्याम्। मि॒त्रः कृ॒ष्टीरनि॑मिषा॒भिच॑ष्टे स॒त्याय॑ ह॒व्यं घृ॒तव॑द्विधेम॥ प्र समि॑त्र॒ मर्तो॑ अस्तु॒ प्रय॑स्वा॒न् यस्त॑ आदित्य॒ शिक्ष॑ति व्र॒तेन॑। न ह॑न्यते॒ न जी॑यते॒ त्वोतो॒ नैन॒मꣳहो॑ अश्नो॒त्यन्ति॑तो॒ न दू॒रात्॥


\sect{माध्याह्निक सूर्योपस्थानम्}
आ स॒त्येन॒ रज॑सा॒ वर्त॑मानो निवे॒शय॑न्न॒मृतं॒ मर्त्यं॑ च। हि॒र॒ण्यये॑न सवि॒ता रथे॒नादे॒वो या॑ति॒ भुव॑ना वि॒पश्यन्॑। उद्व॒यं तम॑स॒स्परि॒ पश्य॑न्तो॒ ज्योति॒रुत्त॑रम्। दे॒वं दे॑व॒त्रा सूर्य॒मग॑न्म॒ ज्योति॑रुत्त॒मम्। उदु॒त्यं जा॒तवे॑दसं दे॒वं व॑हन्ति के॒तवः॑। दृ॒शे विश्वा॑य॒ सूर्यम्। चि॒त्रं दे॒वाना॒मुद॑गा॒दनी॑कं॒ चक्षु॑र्मि॒त्रस्य॒ वरु॑णस्या॒ग्नेः। आ प्रा॒ द्यावा॑ पृथि॒वी अ॒न्तरि॑क्ष॒ꣳ॒ सूर्य॑ आ॒त्मा जग॑तस्त॒स्थुष॑श्च। तच्चक्षु॑र्दे॒वहि॑तं पु॒रस्ताच्छु॒क्रमु॒च्चर॑त्॥

पश्ये॑म श॒रदः॑ श॒तं जीवे॑म श॒रदः॑ श॒तं नन्दा॑म श॒रदः॑ श॒तं मोदा॑म श॒रदः॑ श॒तं भवा॑म श॒रदः॑ श॒तꣳ शृ॒णवा॑म श॒रदः॑ श॒तं प्रब्र॑वाम श॒रदः॑ श॒तमजी॑ताः स्याम श॒रदः॑ श॒तं ज्योक्च॒ सूर्यं॑ दृ॒शे। य उद॑गान्मह॒तोर्णवाद्वि॒भ्राज॑मानः सरि॒रस्य॒ मध्या॒थ्स मा॑ वृष॒भो लो॑हिता॒क्षः सूर्यो॑ विप॒श्चिन्मन॑सा पुनातु॥

\sect{सायं सन्ध्या सूर्योपस्थानम्}
इ॒मं मे॑ वरुण श्रुधी॒ हव॑म॒द्या च॑ मृडय। त्वाम॑व॒स्युराच॑के॥ तत्त्वा॑ यामि॒ ब्रह्म॑णा॒ वन्द॑मान॒स्तदाशास्ते॒ यज॑मानो ह॒विर्भिः॑। अहे॑डमानो वरुणे॒ह बो॒ध्युरु॑शꣳस॒ मा न॒ आयुः॒ प्रमो॑षीः॥
यच्चि॒द्धिते॒ विशो॑ यथा॒ प्रदे॑व वरुण व्र॒तम्। मि॒नी॒मसि॒ द्यवि॑द्यवि॥ यत्किं चे॒दं व॑रुण॒ दैव्ये॒ जने॑भिद्रो॒हं मनु॒ष्याश्चरा॑मसि। अचि॑त्ती॒यत्तव॒ धर्मा॑ युयोपि॒म मा न॒स्तस्मा॒देन॑सो देव रीरिषः॥ कि॒त॒वासो॒ यद्रि॑रि॒पुर्नदी॒वि यद्वा॑ घा स॒त्यमु॒त यं न वि॒द्म। सर्वा॒ताविष्य॑ शिथि॒रेव दे॒वाथा॑ ते स्याम वरुण प्रि॒यासः॑॥

\sect{हरिहर-वन्दनम्}

\twolineshloka*
{ऋ॒तꣳ स॒त्यं प॑रं ब्र॒ह्म॒ पु॒रुषं॑ कृष्ण॒पिङ्ग॑लम्}
{ऊ॒र्ध्वरे॑तं वि॑रूपा॒क्षं॒ वि॒श्वरू॑पाय॒ वै नमो॒ नमः॑}



\sect{रक्षा}

अ॒द्या नो॑ देव सवितः प्र॒जाव॑थ्सावीः॒ सौभ॑गम्।\\
परा॑ दुः॒ष्वप्नि॑यꣳ सुव।\\
विश्वा॑नि देव सवितर्दुरि॒तानि॒ परा॑ सुव।\\
यद्भ॒द्रं तन्म॒ आ सु॑व।


\sect{ब्रह्मयज्ञः}
विद्यु॑दसि॒ विद्य॑ मे पा॒पमान॑मृ॒तात् स॒त्यमुपै॑मि।


ॐ भूः । तत् स॑वि॒तुर्वरे॑॑ण्यम्।\
ॐ भुवः। भर्गो॑ दे॒वस्य॑ धीमहि।\\
ओꣳ सुवः। धियो॒ यो नः॑ प्रचो॒दया॑॑त्।\\
ॐ भूः तत् स॑वि॒तुर्वरे॑॑ण्यम्। भर्गो॑ दे॒वस्य॑ धीमहि।\\
ॐ भुवः। धियो॒ यो नः॑ प्रचो॒दया॑॑त्।\\
ओꣳसुवः। तत् स॑वि॒तुर्वरे॑॑ण्यम्। भर्गो॑ दे॒वस्य॑ धीमहि। धियो॒ यो नः॑ प्रचो॒दया॑॑त्।\\

\dnsub{वेदादयः}
हरिः ॐ। अ॒ग्निमी᳚ळे पु॒रोहि॑तं य॒ज्ञस्य॑ दे॒वमृ॒त्विजम्᳚। होता᳚रं रत्न॒-धात॑मम्॥ हरिः॑ ॐ॥\\

हरिः ॐ। इ॒षेत्वो॒र्जे त्वा॑ वा॒यवः॑ स्थो पा॒यवः॑ स्थ दे॒वो वः॑ सवि॒ता प्रार्प॑यतु॒ श्रेष्ठ॑तमाय॒ कर्म॑णे॥ हरिः॑ ॐ॥ \\

हरिः ॐ। अग्न॒ आया॑हि वी॒तये॑ गृणा॒नो ह॒व्यदा॑तये। नि होता॑ सथ्सि ब॒र्हिषि॑॥ हरिः॑ ॐ॥\\

हरिः ॐ। शन्नो॑ दे॒वीर॒भिष्ट॑य॒ आपो॑ भवुन्तु पी॒तये᳚। शं योर॒भिस्र॑वन्तु नः॥ हरिः॑ ॐ॥\\

ॐ भूर्भुवः॒ सुवः॑।

सत्यं तपः श्रद्धायां॑ जुहो॒मि॥

\dnsub{परिधानीया}

ॐ॥ नमो॒ ब्रह्म॑णे॒ नमो॑ अस्त्व॒ग्नये॒ नमः॑ पृथि॒व्यै नम॒ ओष॑धीभ्यः।
नमो॑ वा॒चे नमो॑ वा॒चस्पत॑ये॒ नमो॒ विष्ण॑वे बृह॒ते क॑रोमि॥

वृष्टि॑रसि॒ वृश्च॑मे पा॒प्मान॑मृ॒ताथ्स॒त्यमुपा॑गाम्॥
