% !TeX program = XeLaTeX
% !TeX root = ../vedamantrabook.tex


\chapt{गणपत्यथर्वशीर्षोपनिषत्}

ॐ भ॒द्रं कर्णे॑भिः शृणु॒याम॑ देवाः। भ॒द्रं प॑श्येमा॒क्षभि॒र्यज॑त्राः। 
स्थि॒रैरङ्गै᳚स्तुष्टु॒वाꣳ स॑स्त॒नूभिः॑। व्यशे॑म दे॒वहि॑तं॒ यदायुः॑। 
स्व॒स्ति न॒ इन्द्रो॑ वृ॒द्धश्र॑वाः। स्व॒स्ति नः॑ पू॒षा वि॒श्ववे॑दाः। 
स्व॒स्ति न॒स्तार्क्ष्यो॒ अरि॑ष्टनेमिः। स्व॒स्ति नो॒ बृह॒स्पति॑र्दधातु॥
ॐ शान्तिः॒ शान्तिः॒ शान्तिः॑॥

ॐ नम॑स्ते गणपतये। त्वमे॒व प्र॒त्यक्षं॒ तत्त्व॑मसि।
त्वमे॒व के॒वलं॒ कर्ता॑ऽसि। त्वमे॒व के॒वलं॒ धर्ता॑ऽसि।  त्वमे॒व के॒वलं॒ हर्ता॑ऽसि।
त्वमेव सर्वं खल्विदं॑ ब्रह्मा॒सि।  त्वं साक्षादात्मा॑ऽसि नि॒त्यम्॥१॥

ऋ॑तं व॒च्मि। स॑त्यं व॒च्मि॥२॥

अव॑ त्वं॒ माम्।  अव॑ व॒क्तारम्᳚। अव॑ श्रो॒तारम्᳚। अव॑ दा॒तारम्᳚।  अव॑ धा॒तारम्᳚।
अवानूचानम॑व शि॒ष्यम्। अव॑ प॒श्चात्ता᳚त्।  अव॑ पु॒रस्ता᳚त्। अवो॑त्त॒रात्ता᳚त्।
अव॑ दक्षि॒णात्ता᳚त्। अव॑ चो॒र्ध्वात्ता᳚त्।  अवाध॒रात्ता᳚त्।
सर्वतो मां पाहि पाहि॑ सम॒न्तात्॥३॥

त्वं वाङ्मयस्त्वं॑ चिन्म॒यः। त्वमानन्दमयस्त्वं॑ ब्रह्म॒मयः।  त्वं सच्चिदानन्दाद्वि॑तीयो॒ऽसि।
त्वं प्र॒त्यक्षं॒ ब्रह्मा॑सि।  त्वं ज्ञानमयो विज्ञान॑मयो॒ऽसि॥४॥

सर्वं जगदिदं त्वत्त्वो॑ जाय॒ते। सर्वं जगदिदं त्वत्त्व॑स्तिष्ठ॒ति।
सर्वं जगदिदं त्वयि लय॑मेष्य॒ति। सर्वं जगदिदं त्वयि॑ प्रत्ये॒ति।
त्वं भूमिरापोऽनलोऽनि॑लो न॒भः।  त्वं चत्वारि वाक्परिमिता॑ पदा॒नि॥५॥

त्वं गुणत्र॑याती॒तः।  त्वम् अवस्थात्र॑याती॒तः। त्वं देहत्र॑याती॒तः।
त्वं कालत्र॑याती॒तः। त्वं मूलाधारस्थितो॑ऽसि नि॒त्यम्। त्वं शक्तित्र॑यात्म॒कः।
त्वां योगिनो ध्याय॑न्ति नि॒त्यम्।  त्वं ब्रह्मा त्वं विष्णुस्त्वं रुद्रस्त्वमिन्द्रस्त्वमग्निस्त्वं
वायुस्त्वं सूर्यस्त्वं चन्द्रमास्त्वं ब्रह्म॒ भूर्भुवः॒ सुव॒रोम्॥६॥

ग॒णादिं पूर्व॑मुच्चा॒र्य॒ व॒र्णादिं त॑दनन्त॒रम्। अनुस्वारः प॑रत॒रः। अर्धे॑न्दुलसि॒तम्।
तारे॑ण ऋ॒द्धम्।  एतत्तव मनु॑स्वरू॒पम्। गकारः पू‍᳚र्वरू॒पम्।  अकारो मध्य॑मरू॒पम्।
अनुस्वारश्चा᳚न्त्य\-रू॒पम्।  बिन्दुरुत्त॑ररू॒पम्। नादः॑ सन्धा॒नम्।  सꣳहि॑ता स॒न्धिः। 
सैषा गणे॑शवि॒द्या। गण॑क ऋ॒षिः। निचृद्गाय॑त्रीच्छ॒न्दः।
श्रीमहागणपति॑र्देव॒ता। ॐ गं गणपतये॒ नमः॑॥७॥

{\centering
ए॒क॒द॒न्ताय॑ वि॒द्महे॑ वक्रतु॒ण्डाय॑ धीमहि।\\
  तन्नो॑ दन्ती प्रचो॒दया᳚त्॥८॥
\twolineshloka*{ए॒क॒द॒न्तं च॑तुर्ह॒स्तं॒ पा॒शम॑ङ्कुश॒धारि॑णम्}
{रदं॑ च॒ वर॑दं ह॒स्तै॒र्बि॒भ्रा॒णं मू॑षक॒ध्वजम्}
\twolineshloka*{रक्तं॑ ल॒म्बोद॑रं शू॒र्प॒क॒र्ण॒कं र॑क्तवा॒ससम्}
{रक्त॑ग॒न्धानु॑लिप्ता॒ङ्गं॒ र॒क्त॒पुष्पैः॑ सुपू॒जितम्}
\addtocounter{shlokacount}{8}
\threelineshloka{भक्ता॑नु॒कम्पि॑नं दे॒वं॒ ज॒गत्का॑रण॒मच्यु॑तम्}
{आ॒विर्भू॒तं च॑ सृष्ट्या॒दौ॒ प्र॒कृ॒तेः पु॑रुषा॒त्परम्}
{एवं॑ ध्या॒यति॑ यो नि॒त्यं॒ स॒ यो॒गी यो॑गिनां॒ वरः॑}
}

%\pagebreak[4]
नमो व्रातपतये नमो गणपतये नमः प्रमथपतये नमस्ते अस्तु लम्बोदरायैकदन्ताय विघ्नविनाशिने शिवसुताय श्रीवरदमू‍र्तये॑ नमो॒ नमः॥१०॥

एतदथर्वशीर्षं॑ योऽधी॒ते। स ब्रह्मभूयाय॑ कल्प॒ते।  स सर्वविघ्नैर्न॑ बाध्य॒ते।
स सर्वतः सुख॑मेध॒ते।  स पञ्चमहापापात् प्र॑मुच्य॒ते।
सायमधीयानो दिवसकृतं पापं॑ नाश॒यति।  प्रातरधीयानो रात्रिकृतं पापं॑ नाश॒यति।
सायं प्रातः प्रयुञ्जानो अपा॑पो भ॒वति।  सर्वत्राधीयानोऽपविघ्नो॑ भव॒ति।
धर्मार्थकाममोक्षं च॑ विन्द॒ति। इदमथर्वशीर्षमशिष्याय॑ न दे॒यम्।  यो यदि मोहाद्दास्यति स पापी॑यान् भ॒वति।
सहस्रावर्तनाद्यं यं काममधीते तं तमनेन॑ साध॒येत्॥११॥

अनेन गणपतिमभिषिञ्चति स वाग्मी॑ भव॒ति।  चतुर्थ्यामनश्नन् जपति स विद्या॑वान् भ॒वति।  इत्यथर्व॑णवा॒क्यम्। ब्रह्माद्यावरणं विद्यान्न बिभेति कदा॑चने॒ति॥१२॥

यो दूर्वाङ्कुरैर्यजति स वैश्रवणोप॑मो भ॒वति। यो लाजैर्यजति स यशो॑वान् भ॒वति स मेधा॑वान् भ॒वति। 
यो मोदकसहस्रेण यजति स वाञ्छितफलम॑वाप्नो॒ति। यः साज्यसमिद्भिर्यजति स सर्वं लभते स सर्वं॑ लभ॒ते॥१३॥

अष्टौ ब्राह्मणान् सम्यग्ग्रा॑हयि॒त्वा। सूर्यवर्चस्वी॑ भव॒ति।
सूर्यग्रहे महानद्यां प्रतिमासन्निधौ वा जप्त्वा सिद्धमन्त्रो॑ भव॒ति।
म॒हा॒वि॒घ्नात् प्र॑मुच्य॒ते।  म॒हा॒दो॒षात् प्र॑मुच्य॒ते। म॒हा॒पा॒पात् प्र॑मुच्य॒ते।
म॒हा॒प्र॒त्य॒वा॒यात् प्र॑मुच्य॒ते। स सर्वविद्भवति स सर्व॑विद्भ॒वति। य ए॑वं वे॒द।  इत्यु॑प॒निष॑त्॥१४॥

स॒ह ना॑ववतु। स॒ह नौ॑ भुनक्तु। स॒ह वी॒र्यं॑ करवावहै।
ते॒ज॒स्विना॒ऽवधी॑तमस्तु॒ मा वि॑द्विषा॒वहै᳚॥

\centerline{ॐ शान्तिः॒ शान्तिः॒ शान्तिः॑॥}