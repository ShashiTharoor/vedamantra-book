% !TeX program = XeLaTeX
% !TeX root = ../vedamantrabook.tex

\chapt{चमकप्रश्नः}

अग्ना॑विष्णू स॒जोष॑से॒मा व॑र्धन्तु वां॒ गिर॑। द्यु॒म्नैर्वाजे॑भि॒राग॑तम्॥ \\
वाज॑श्च मे प्रस॒वश्च॑ मे॒ प्रय॑तिश्च मे॒ प्रसि॑तिश्च मे धी॒तिश्च॑ मे॒ क्रतु॑श्च मे॒ स्वर॑श्च मे॒ श्लोक॑श्च मे श्रा॒वश्च॑ मे॒ श्रुति॑श्च मे॒ ज्योति॑श्च मे॒ सुव॑श्च मे प्रा॒णश्च॑ मेऽपा॒नश्च॑ मे व्या॒नश्च॒ मेऽसु॑श्च मे चि॒त्तं च॑ म॒ आधी॑तं च मे॒ वाक्च॑ मे॒ मन॑श्च मे॒ चक्षु॑श्च मे॒ श्रोत्रं॑ च मे॒ दक्ष॑श्च मे॒ बलं॑ च म॒ ओज॑श्च मे॒ सह॑श्च म॒ आयु॑श्च मे ज॒रा च॑ म आ॒त्मा च॑ मे त॒नूश्च॑ मे॒ शर्म॑ च मे॒ वर्म॑ च॒ मेऽङ्गा॑नि च मे॒ऽस्थानि॑ च मे॒ परूषि च मे॒ शरी॑राणि च मे॥१॥ 

ज्यैष्ठ्यं॑ च म॒ आधि॑पत्यं च मे म॒न्युश्च॑ मे॒ भाम॑श्च॒ मेऽम॑श्च॒ मेऽम्भ॑श्च मे जे॒मा च॑ मे महि॒मा च॑ मे वरि॒मा च॑ मे प्रथि॒मा च॑ मे व॒र्ष्मा च॑ मे द्राघु॒या च॑ मे वृ॒द्धं च॑ मे॒ वृद्धि॑श्च मे स॒त्यं च॑ मे श्र॒द्धा च॑ मे॒ जग॑च्च मे॒ धनं॑ च मे॒ वश॑श्च मे॒ त्विषि॑श्च मे क्री॒डा च॑ मे॒ मोद॑श्च मे जा॒तं च॑ मे जनि॒ष्यमा॑णं च मे सू॒क्तं च॑ मे सुकृ॒तं च॑ मे वि॒त्तं च॑ मे॒ वेद्यं॑ च मे भू॒तं च॑ मे भवि॒ष्यच्च॑ मे सु॒गं च॑ मे सु॒पथं॑ च म ऋ॒द्धं च॑ म॒ ऋद्धि॑श्च मे कॢ॒प्तं च॑ मे॒ कॢप्ति॑श्च मे म॒तिश्च॑ मे सुम॒तिश्च॑ मे॥२॥ 

शं च॑ मे॒ मय॑श्च मे प्रि॒यं च॑ मेऽनुका॒मश्च॑ मे॒ काम॑श्च मे सौमन॒सश्च॑ मे भ॒द्रं च॑ मे॒ श्रेय॑श्च मे॒ वस्य॑श्च मे॒ यश॑श्च मे॒ भग॑श्च मे॒ द्रवि॑णं च मे य॒न्ता च मे ध॒र्ता च मे॒ क्षेम॑श्च मे॒ धृति॑श्च मे॒ विश्वं॑ च मे॒ मह॑श्च मे सं॒विच्च॑ मे॒ ज्ञात्रं॑ च मे॒ सूश्च॑ मे प्र॒सूश्च॑ मे॒ सीरं॑ च मे ल॒यश्च॑ म ऋ॒तं च॑ मे॒ऽमृतं॑ च मेऽय॒क्ष्मं च॒ मेऽना॑मयच्च मे जी॒वातु॑श्च मे दीर्घायु॒त्वं च॑ मेऽनमि॒त्रं च॒ मेऽभ॑यं च मे सु॒गं च॑ मे॒ शय॑नं च मे सू॒षा च॑ मे सु॒दिनं॑ च मे॥३॥ 

ऊर्क्च॑ मे सू॒नृता॑ च मे॒ पय॑श्च मे॒ रस॑श्च मे घृ॒तं च॑ मे॒ मधु॑ च मे॒ सग्धि॑श्च मे॒ सपी॑तिश्च मे कृ॒षिश्च॑ मे॒ वृष्टि॑श्च मे॒ जैत्रं॑ च म॒ औद्भि॑द्यं च मे र॒यिश्च॑ मे॒ राय॑श्च मे पु॒ष्टं च॑ मे॒ पुष्टि॑श्च मे वि॒भु च॑ मे प्र॒भु च॑ मे ब॒हु च॑ मे॒ भूय॑श्च मे पू॒र्णं च॑ मे पू॒र्णत॑रं च॒ मेऽक्षि॑तिश्च मे॒ कूय॑वाश्च॒ मेऽन्नं॑ च॒ मेऽक्षु॑च्च मे व्री॒हय॑श्च मे॒ यवाश्च मे॒ माषाश्च मे॒ तिलाश्च मे मु॒द्गाश्च॑ मे ख॒ल्वाश्च मे गो॒धूमाश्च मे म॒सुराश्च मे प्रि॒यङ्ग॑वश्च॒ मेऽण॑वश्च मे श्या॒माकाश्च मे नी॒वाराश्च मे॥४॥ 

अश्मा॑ च मे॒ मृत्ति॑का च मे गि॒रय॑श्च मे॒ पर्व॑ताश्च मे॒ सिक॑ताश्च मे॒ वन॒स्पत॑यश्च मे॒ हिर॑ण्यं च॒ मेऽय॑श्च मे॒ सीसं॑ च मे॒ त्रपु॑श्च मे श्या॒मं च॑ मे लो॒हं च॑ मे॒ऽग्निश्च॑ म॒ आप॑श्च मे वी॒रुध॑श्च म॒ ओष॑धयश्च मे कृष्टप॒च्यं च॑ मेऽकृष्टप॒च्यं च॑ मे ग्रा॒म्याश्च॑ मे प॒शव॑ आर॒ण्याश्च॑ य॒ज्ञेन॑ कल्पन्तां वि॒त्तं च मे॒ वित्ति॑श्च मे भू॒तं च॑ मे॒ भूति॑श्च मे॒ वसु॑ च मे वस॒तिश्च॑ मे॒ कर्म॑ च मे॒ शक्ति॑श्च॒ मेऽर्थ॑श्च म॒ एम॑श्च म॒ इति॑श्च मे॒ गति॑श्च मे॥५॥ 

अ॒ग्निश्च॑ म॒ इन्द्र॑श्च मे॒ सोम॑श्च म॒ इन्द्र॑श्च मे सवि॒ता च॑ म॒ इन्द्र॑श्च मे॒ सर॑स्वती च म॒ इन्द्र॑श्च मे पू॒षा च॑ म॒ इन्द्र॑श्च मे॒ बृह॒स्पति॑श्च म॒ इन्द्र॑श्च मे मि॒त्रश्च॑ म॒ इन्द्र॑श्च मे॒ वरु॑णश्च म॒ इन्द्र॑श्च मे॒ त्वष्टा॑ च म॒ इन्द्र॑श्च मे धा॒ता च॑ म॒ इन्द्र॑श्च मे॒ विष्णु॑श्च म॒ इन्द्र॑श्च मे॒ऽश्विनौ॑ च म॒ इन्द्र॑श्च मे म॒रुत॑श्च म॒ इन्द्र॑श्च मे॒ विश्वे॑ च मे दे॒वा इन्द्र॑श्च मे पृथि॒वी च॑ म॒ इन्द्र॑श्च मे॒ऽन्तरि॑क्षं च म॒ इन्द्र॑श्च मे॒ द्यौश्च॑ म॒ इन्द्र॑श्च मे॒ दिश॑श्च म॒ इन्द्र॑श्च मे मू॒र्धा च॑ म॒ इन्द्र॑श्च मे प्र॒जाप॑तिश्च म॒ इन्द्र॑श्च मे॥६॥ 

अ॒शुश्च॑ मे र॒श्मिश्च॒ मेऽदाभ्यश्च॒ मेऽधि॑पतिश्च म उपा॒शुश्च॑ मेऽन्तर्या॒मश्च॑ म ऐन्द्रवाय॒वश्च॑ मे मैत्रावरु॒णश्च॑ म आश्वि॒नश्च॑ मे प्रतिप्र॒स्थान॑श्च मे शु॒क्रश्च॑ मे म॒न्थी च॑ म आग्रय॒णश्च॑ मे वैश्वदे॒वश्च॑ मे ध्रु॒वश्च॑ मे वैश्वान॒रश्च॑ म ऋतुग्र॒हाश्च॑ मेऽतिग्रा॒ह्याश्च म ऐन्द्रा॒ग्नश्च॑ मे वैश्वदे॒वश्च॑ मे मरुत्व॒तीयाश्च मे माहे॒न्द्रश्च॑ म आदि॒त्यश्च॑ मे सावि॒त्रश्च॑ मे सारस्व॒तश्च॑ मे पौ॒ष्णश्च॑ मे पात्नीव॒तश्च॑ मे हारियोज॒नश्च॑ मे॥७॥ 

इ॒ध्मश्च॑ मे ब॒र्हिश्च॑ मे॒ वेदि॑श्च मे॒ धिष्णि॑याश्च मे॒ स्रुच॑श्च मे चम॒साश्च॑ मे॒ ग्रावा॑णश्च मे॒ स्वर॑वश्च म उपर॒वाश्च॑ मेऽधि॒षव॑णे च मे द्रोणकल॒शश्च॑ मे वाय॒व्या॑नि च मे पूत॒भृच्च॑ म आधव॒नीय॑श्च म॒ आग्नीध्रं च मे हवि॒र्धानं॑ च मे गृ॒हाश्च॑ मे॒ सद॑श्च मे पुरो॒डाशाश्च मे पच॒ताश्च॑ मेऽवभृ॒थश्च॑ मे स्वगाका॒रश्च॑ मे॥८॥ 

अ॒ग्निश्च॑ मे घ॒र्मश्च॑ मे॒ऽर्कश्च॑ मे॒ सूर्य॑श्च मे प्रा॒णश्च॑ मेऽश्वमे॒धश्च॑ मे पृथि॒वी च॒ मेऽदि॑तिश्च मे॒ दिति॑श्च मे॒ द्यौश्च॑ मे॒ शक्व॑रीर॒ङ्गुल॑यो॒ दिश॑श्च मे य॒ज्ञेन॑ कल्पन्ता॒मृक्च॑ मे॒ साम॑ च मे॒ स्तोम॑श्च मे॒ यजु॑श्च मे दी॒क्षा च॑ मे॒ तप॑श्च म ऋ॒तुश्च॑ मे व्र॒तं च॑ मेऽहोरा॒त्रयोर्वृ॒ष्ट्या बृ॑हद्रथन्त॒रे च॑ मे य॒ज्ञेन॑ कल्पेताम्॥९॥ 

गर्भाश्च मे व॒त्साश्च॑ मे॒ त्र्यवि॑श्च मे त्र्य॒वी च॑ मे दित्य॒वाच्च॑ मे दित्यौ॒ही च॑ मे॒ पञ्चा॑विश्च मे पञ्चा॒वी च॑ मे त्रिव॒त्सश्च॑ मे त्रिव॒त्सा च॑ मे तुर्य॒वाच्च॑ मे तुर्यौ॒ही च॑ मे पष्ठ॒वाच्च॑ मे पष्ठौ॒ही च॑ म उ॒क्षा च॑ मे व॒शा च॑ म ऋष॒भश्च॑ मे वे॒हच्च॑ मेऽन॒ड्वां च॑ मे धे॒नुश्च॑ म॒ आयुर्य॒ज्ञेन॑ कल्पतां प्रा॒णो य॒ज्ञेन॑ कल्पतामपा॒नो य॒ज्ञेन॑ कल्पतां व्या॒नो य॒ज्ञेन॑ कल्पतां॒ चक्षु॑र्‌य॒ज्ञेन॑ कल्पता॒ श्रोत्रं॑ य॒ज्ञेन॑ कल्पतां॒ मनो॑ य॒ज्ञेन॑ कल्पतां॒ वाग्य॒ज्ञेन॑ कल्पतामा॒त्मा य॒ज्ञेन॑ कल्पतां य॒ज्ञो य॒ज्ञेन॑ कल्पताम्॥१०॥ 

एका॑ च मे ति॒स्रश्च॑ मे॒ पञ्च॑ च मे स॒प्त च॑ मे॒ नव॑ च म॒ एका॑दश च मे॒ त्रयो॑दश च मे॒ पञ्च॑दश च मे स॒प्तद॑श च मे॒ नव॑दश च म॒ एक॑विशतिश्च मे॒ त्रयो॑विशतिश्च मे॒ पञ्च॑विशतिश्च मे स॒प्तविशतिश्च मे॒ नव॑विशतिश्च म॒ एक॑त्रिशच्च मे॒ त्रय॑स्त्रिशच्च मे॒ चत॑स्रश्च मे॒ऽष्टौ च॑ मे॒ द्वाद॑श च मे॒ षोड॑श च मे विश॒तिश्च॑ मे॒ चतु॑र्विशतिश्च मे॒ऽष्टाविशतिश्च मे॒ द्वात्रिशच्च मे॒ षट्त्रिशच्च मे चत्वारि॒शच्च॑ मे॒ चतु॑श्चत्वारिशच्च मे॒ऽष्टाच॑त्वारिशच्च मे॒ वाज॑श्च प्रस॒वश्चा॑पि॒जश्च॒ क्रतु॑श्च॒ सुव॑श्च मू॒र्धा च॒ व्यश्नि॑यश्चान्त्याय॒नश्चान्त्य॑श्च भौव॒नश्च॒ भुव॑न॒श्चाधि॑पतिश्च॥११॥ 

इडा॑ देव॒हूर्मनु॑र्यज्ञ॒नीर्बृह॒स्पति॑रुक्थाम॒दानि॑ शसिष॒द्विश्वे॑दे॒वाः सूक्त॒वाच॒ पृथि॑वि मात॒र्मा मा॑ हिसी॒र्मधु॑ मनिष्ये॒ मधु॑ जनिष्ये॒ मधु॑ वक्ष्यामि॒ मधु॑ वदिष्यामि॒ मधु॑मतीं दे॒वेभ्यो॒ वाच॑मुद्यास शुश्रू॒षेण्यां मनु॒ष्येभ्य॒स्तं मा॑ दे॒वा अ॑वन्तु शो॒भायै॑ पि॒तरोऽनु॑मदन्तु॥ 

\centerline{॥ॐ शान्ति॒ शान्ति॒ शान्ति॑॥}


