% !TeX program = XeLaTeX
% !TeX root = ../vedamantrabook.tex
\chapt{कूष्माण्ड-होम-मन्त्राः}

\newcommand{\idam}{इदं न मम}


यद्दे॑वा देव॒हेळ॑नं॒ देवा॑सश्चकृ॒मा व॒यम्। 
आदि॑त्या॒स्तस्मा᳚न्मा मुञ्चत॒र्तस्य॒र्तेन॒ मामि॒त स्वाहा᳚॥१॥

(देवेभ्य आदित्येभ्य \idam)

देवा॑ जीवनका॒म्या यद्वा॒चाऽनृ॑त\-मूदि॒म। 
तस्मा᳚न्न इ॒ह मु॑ञ्चत॒ विश्वे॑ देवाः स॒जोष॑सः॒ स्वाहा᳚॥२॥

(विश्वेभ्यो देवेभ्य \idam)


ऋ॒तेन॑ द्यावापृथिवी ऋ॒तेन॒ त्वꣳ स॑रस्वति। 
कृ॒तान्नः॑ पा॒ह्येन॑सो॒ यत्किं चानृ॑त\-मूदि॒म स्वाहा᳚॥३॥

(द्यावापृथिवीभ्यां सरस्वत्या \idam)

इ॒न्द्रा॒ग्नी मि॒त्रावरु॑णौ॒ सोमो॑ धा॒ता बृह॒स्पतिः॑। 
ते नो॑ मुञ्च॒न्त्वेन॑सो॒ यद॒न्यकृ॑तमारि॒म स्वाहा᳚॥४॥

(इन्द्राग्निभ्यां मित्रावरुणाभ्यां सोमाय धात्रे बृहस्पतय \idam)


स॒जा॒त॒श॒ꣳ॒सादु॒त जा॑मिश॒ꣳ॒साज्ज्याय॑सः॒ शꣳसा॑दु॒त वा॒ कनी॑यसः। 
अना॑धृष्टं दे॒वकृ॑तं॒ यदेन॒स्तस्मा॒त् त्वम॒स्माञ्जा॑तवेदो मुमुग्धि॒ स्वाहा᳚॥५॥

(अग्नये जातवेदस \idam)

यद्वा॒चा यन्मन॑सा बा॒हुभ्या॑मू॒रुभ्या॑मष्ठी॒वद्भ्याꣳ॑ शि॒श्नैर्यदनृ॑तं चकृ॒मा व॒यम्। 
अ॒ग्निर्मा॒ तस्मा॒देन॑सो॒ गार्\mbox{}ह॑पत्यः॒ प्रमु॑ञ्चतु चकृ॒म यानि॑ दुष्कृ॒ता स्वाहा᳚॥६॥

(अग्नये गार्हपत्याय \idam)


येन॑ त्रि॒तो अ॑र्ण॒वान्नि॑र्ब॒भूव॒ येन॒ सूर्यं॒ तम॑सो निर्मु॒मोच॑। 
येनेन्द्रो॒ विश्वा॒ अज॑हा॒दरा॑ती॒स्तेना॒हं ज्योति॑षा॒ ज्योति॑रानशा॒न आ᳚क्षि॒ स्वाहा᳚॥७॥

(अग्नये ज्योतिष \idam)

यत्कुसी॑द॒मप्र॑तीत्तं॒ मये॒ह येन॑ य॒मस्य॑ नि॒धिना॒ चरा॑मि। 
ए॒तत्तद॑ग्ने अनृ॒णो भ॑वामि॒ जीव॑न्ने॒व प्रति॒ तत्ते॑ दधामि॒ स्वाहा᳚॥८॥

(अग्नय \idam)

%3.7.12

यन्मयि॑ मा॒ता गर्भे॑ स॒ति। एन॑श्च॒कार॒ यत्पि॒ता।
अ॒ग्निर्मा॒ तस्मा॒देन॑सः। गार्\mbox{}ह॑पत्यः॒ प्र मु॑ञ्चतु। 
दु॒रि॒ता यानि॑ चकृ॒म। क॒रोतु॒ माम॑ने॒नस॒ꣴ॒ स्वाहा᳚॥९॥

(अग्नये गार्हपत्याय \idam)

यदा॑ पि॒पेष॑ मा॒तरं॑ पि॒तरम्᳚।
पु॒त्रः प्रमु॑दितो॒ धयन्।
अहिꣳ॑सितौ पि॒तरौ॒ मया॒ तत्।
तद॑ग्ने अनृ॒णो भ॑वामि॒ स्वाहा᳚॥१०॥

(अग्नय \idam)


यद॒न्तरि॑क्षं पृथि॒वीमु॒त द्यां यन्मा॒तरं॑ पि॒तरं॑ वा जिहिꣳसि॒म॥
अ॒ग्निर्मा॒ तस्मा॒देन॑सः। गार्\mbox{}ह॑पत्यः॒ प्र मु॑ञ्चतु। 
दु॒रि॒ता यानि॑ चकृ॒म। क॒रोतु॒ माम॑ने॒नस॒ꣴ॒ स्वाहा᳚॥११॥

(अग्नये गार्हपत्याय \idam)



यदा॒शसा॑ नि॒शसा॒ यत्प॑रा॒शसा᳚॥ यदेन॑श्चकृ॒मा नूत॑नं॒ यत्पु॑रा॒णम्।
अ॒ग्निर्मा॒ तस्मा॒देन॑सः। गार्\mbox{}ह॑पत्यः॒ प्र मु॑ञ्चतु। 
दु॒रि॒ता यानि॑ चकृ॒म। क॒रोतु॒ माम॑ने॒नस॒ꣴ॒ स्वाहा᳚॥१२॥

(अग्नये गार्हपत्याय \idam)



अति॑ क्रामामि दुरि॒तं यदेनः॑। जहा॑मि रि॒प्रं प॑र॒मे स॒धस्थे᳚।
यत्र॒ यन्ति॑ सु॒कृतो॒ नापि॑ दु॒ष्कृतः॑।
तमा रो॑हामि सु॒कृतां॒ नु लो॒कꣴ स्वाहा᳚॥१३॥

(अग्नय \idam)

त्रि॒ते दे॒वा अ॑मृजतै॒तदेनः॑।
त्रि॒त ए॒तन्म॑नु॒ष्ये॑षु मामृजे।
ततो॑ मा॒ यदि॒ किञ्चि॑दान॒शे।
अ॒ग्निर्मा॒ तस्मा॒देन॑सः। गार्\mbox{}ह॑पत्यः॒ प्र मु॑ञ्चतु। 
दु॒रि॒ता यानि॑ चकृ॒म। क॒रोतु॒ माम॑ने॒नस॒ꣴ॒ स्वाहा᳚॥१४॥

(अग्नये गार्हपत्याय \idam)

दि॒वि जा॒ता अ॒फ्सु जा॒ताः।
या जा॒ता ओष॑धीभ्यः।
अथो॒ या अ॑ग्नि॒जा आपः॑।
ता नः॑ शुन्धन्तु॒ शुन्ध॑नीः॒ स्वाहा᳚॥१५॥

(अद्भ्यः शुन्धनीभ्य \idam)

यदापो॒ नक्तं॑ दुरि॒तं चरा॑म।
यद्वा॒ दिवा॒ नूत॑नं॒ यत्पु॑रा॒णम्।
हिर॑ण्यवर्णा॒स्तत॒ उत्पु॑नीत नः॒ स्वाहा᳚॥१६॥

(अद्भ्यो हिर्ण्यवर्णाभ्य \idam)


इ॒मं मे॑ वरुण श्रुधी॒ हव॑म॒द्या च॑ मृळय। त्वाम॑व॒स्युराच॑के॒ स्वाहा᳚॥१७॥

(वरुणाय \idam)

तत्त्वा॑ यामि॒ ब्रह्म॑णा॒ वन्द॑मान॒स्तदाऽऽशा᳚स्ते॒ यज॑मानो ह॒विर्भिः॑।
अहे॑ळमानो वरुणे॒ह बो॒द्ध्युरु॑शꣳस॒ मा न॒ आयुः॒ प्रमो॑षीः॒ स्वाहा᳚॥१८॥

(वरुणाय \idam)

%२।५।१२

त्वं नो॑ अग्ने॒ वरु॑णस्य वि॒द्वां दे॒वस्य॒ हेळो\-ऽव॑ यासिसीष्ठाः।
यजि॑ष्ठो॒ वह्नि॑तमः॒ शोशु॑चानो॒ विश्वा॒ द्वेषाꣳ॑सि॒ प्र मु॑मुग्ध्य॒स्मथ्स्वाहा᳚॥१९॥

(अग्नीवरुणाभ्याम् \idam)



स त्वं नो॑ अग्ने\-ऽव॒मो भ॑वो॒ती नेदि॑ष्ठो अ॒स्या उ॒षसो॒ व्यु॑ष्टौ।
अव॑ यक्ष्व नो॒ वरु॑ण॒ꣳ॒ ररा॑णो वी॒हि मृ॑ळी॒कꣳ सु॒हवो॑ न एधि॒ स्वाहा᳚॥२०॥

(अग्नीवरुणाभ्याम् \idam)

%2.4.1.9
प्र ण॒ आयूꣳ॑षि तारिषत्।
त्वम॑ग्ने अ॒यासि॑।
अ॒या सन्मन॑सा हि॒तः।
अ॒या सन् ह॒व्यमू॑हिषे।
अ॒या नो॑ धेहि भेष॒जꣴ स्वाहा᳚॥२१॥

(अग्नये अयस \idam)


यददी᳚व्यन्नृ॒णम॒हं ब॒भूवादि॑थ्सन्वा सञ्ज॒गर॒ जने᳚भ्यः। 
अ॒ग्निर्मा॒ तस्मा॒दिन्द्र॑श्च संविदा॒नौ प्रमु॑ञ्चता॒ꣴ॒ स्वाहा᳚॥२२॥

(इन्द्राग्निभ्याम् \idam)

यद्धस्ता᳚भ्यां च॒कर॒ किल्बि॑षाण्य॒क्षाणां᳚ व॒ग्नुमु॑प॒जिघ्न॑मानः। 
उ॒ग्रं॒ प॒श्या च॑ राष्ट्र॒भृच्च॒ तान्य॑फ्स॒रसा॒वनु॑दत्तामृ॒णानि॒ स्वाहा᳚॥२३॥

(उग्रं-पश्य-राष्ट्रभृद्भ्याम् अफ्सरोभ्याम् \idam)


उग्रं॑ पश्ये॒ राष्ट्र॑भृ॒त्किल्बि॑षाणि॒ यद॒क्षवृ॑त्त॒मनु॑दत्तमे॒तत्। 
नेन्न॑ ऋ॒णानृ॒णव॒ इथ्स॑मानो य॒मस्य॑ लो॒के अधि॑रज्जु॒राय॒ स्वाहा᳚॥२४॥

(उग्रं-पश्या राष्ट्रभृद्भ्याम् अफ्सरोभ्याम् \idam)


%1.5.11.3
अव॑ ते॒ हेडो॑ वरुण॒ नमो॑\-भि॒रव॑ य॒ज्ञेभि॑रीमहे ह॒विर्भिः॑।
क्षय॑न्न॒स्मभ्य॑मसुर प्रचेतो॒ राज॒न्नेनाꣳ॑सि शिश्रथः कृ॒तानि॑ स्वाहा᳚॥२५॥

(वरुणाय \idam)


उदु॑त्त॒मं व॑रुण॒ पाश॑\-म॒स्मद\-वा॑ध॒मं वि म॑ध्य॒मꣴ श्र॑थाय।
अथा॑ व॒यमा॑दित्य व्र॒ते तवाना॑गसो॒ अदि॑तये स्याम॒ स्वाहा᳚॥२६॥

(वरुणाय \idam)


इ॒मं मे॑ वरुण श्रुधी॒ हव॑म॒द्या च॑ मृळय। त्वाम॑व॒स्युराच॑के॒ स्वाहा᳚॥२७॥

(वरुणाय \idam)

तत्त्वा॑ यामि॒ ब्रह्म॑णा॒ वन्द॑मान॒स्तदाऽऽशा᳚स्ते॒ यज॑मानो ह॒विर्भिः॑।
अहे॑ळमानो वरुणे॒ह बो॒द्ध्युरु॑शꣳस॒ मा न॒ आयुः॒ प्रमो॑षीः॒ स्वाहा᳚॥२८॥

(वरुणाय \idam)


%२।५।१२

त्वं नो॑ अग्ने॒ वरु॑णस्य वि॒द्वां दे॒वस्य॒ हेळो\-ऽव॑ यासिसीष्ठाः।
यजि॑ष्ठो॒ वह्नि॑तमः॒ शोशु॑चानो॒ विश्वा॒ द्वेषाꣳ॑सि॒ प्र मु॑मुग्ध्य॒स्मथ्स्वाहा᳚॥२९॥

(अग्नीवरुणाभ्याम् \idam)



स त्वं नो॑ अग्ने\-ऽव॒मो भ॑वो॒ती नेदि॑ष्ठो अ॒स्या उ॒षसो॒ व्यु॑ष्टौ।
अव॑ यक्ष्व नो॒ वरु॑ण॒ꣳ॒ ररा॑णो वी॒हि मृ॑ळी॒कꣳ सु॒हवो॑ न एधि॒ स्वाहा᳚॥३०॥

(अग्नीवरुणाभ्याम् \idam)


सङ्कु॑सुको॒ विकु॑सुको निर्\mbox{}ऋ॒थो यश्च॑ निस्व॒नः। 
तेऽ(१\char"E009)स्मद्यक्ष्म॒मना॑\-गसो दू॒राद्दू॒रम॑चीचत॒ꣴ॒ स्वाहा᳚॥३१॥

(सङ्कुसुक विकु॑सुक निर्\mbox{}ऋ॒थ निस्व॒नेभ्य \idam)


निर्य॑क्ष्ममचीचते कृ॒त्यां निर्\mbox{}ऋ॑तिं च। 
तेन॒ योऽ(१\char"E009)स्मथ्समृ॑च्छातै॒ तम॑स्मै॒ प्रसु॑वामसि॒ स्वाहा᳚॥३२॥

(निर्\mbox{}ऋत्या \idam) [अप उपस्पृश्य]


दुः॒श॒ꣳ॒सा॒नु॒श॒ꣳ॒साभ्यां᳚ घ॒णेना॑नुघ॒णेन॑ च। 
तेना॒न्योऽ(१\char"E009)स्मथ्समृ॑च्छातै॒ तम॑स्मै॒ प्रसु॑वामसि॒ स्वाहा᳚॥३३॥

(दुः॒श॒ꣳ॒सा॒नु॒श॒ꣳ॒स-घ॒णेना॑नुघ॒णेभ्य \idam)



सं वर्च॑सा॒ पय॑सा॒ सन्त॒नूभि॒रग॑न्महि॒ मन॑सा॒ सꣳ शि॒वेन॑। 
त्वष्टा॑ नो॒ अत्र॒ विद॑धातु रा॒योऽनु॑मार्ष्टु त॒न्वो(१\char"E009) यद्विलि॑ष्ट॒ꣴ॒ स्वाहा᳚॥३४॥

(त्वष्ट्र \idam)


आयु॑ष्टे वि॒श्वतो॑ दधद॒यम॒ग्निर्वरे᳚ण्यः। 
पुन॑स्ते प्रा॒ण आया॑ति॒ परा॒यक्ष्मꣳ॑ सुवामि ते॒ स्वाहा᳚॥३५॥

(वरेण्यायाग्नय \idam)


आ॒यु॒र्दा अ॑ग्ने ह॒विषो॑ जुषा॒णो घृ॒तप्र॑तीको घृ॒तयो॑निरेधि। 
घृ॒तं पी॒त्वा मधु॒ चारु॒ गव्यं॑ पि॒तेव॑ पु॒त्रम॒भिर॑क्षतादि॒मꣴ स्वाहा᳚॥३६॥

(आयुर्दे अग्नय \idam)


इ॒मम॑ग्न॒ आयु॑षे॒ वर्च॑से कृधि ति॒ग्ममोजो॑ वरुण॒ सꣳशि॑शाधि। 
मा॒तेवा᳚स्मा अदिते॒ शर्म॑ यच्छ॒ विश्वे॑ देवा॒ जर॑दष्टि॒र्यथा\-ऽस॒थ्स्वाहा᳚॥३७॥

(अग्निवरुणादिति-विश्वेभ्यो देवेभ्य \idam)


अग्न॒ आयूꣳ॑षि पवस॒ आ सु॒वोर्ज॒मिषं॑ च नः। 
आ॒रे बा॑धस्व दु॒च्छुना॒ꣴ॒ स्वाहा᳚॥३८॥

(अग्नये पवमानाय \idam)

अग्ने॒ पव॑स्व॒ स्वपा॑ अ॒स्मे वर्चः॑ सु॒वीर्यम्᳚। 
दध॑द्र॒यिं मयि॒ पोष॒ꣴ॒ स्वाहा᳚॥३९॥

(अग्नये पवमानाय \idam)


अ॒ग्निर्\mbox{}ऋषिः॒ पव॑मानः॒ पाञ्च॑जन्यः पु॒रोहि॑तः। 
तमी॑महे महाग॒यꣴ स्वाहा᳚॥४०॥

(पाञ्चजन्याय अग्नय \idam)


अग्ने॑ जा॒तान्प्रणु॑दा नः स॒पत्ना॒न्प्रत्यजा॑ताञ्जातवेदो नुदस्व। 
अ॒स्मे दी॑दिहि सु॒मना॒ अहे॑ळ॒ञ्छर्म॑न्ते स्याम त्रि॒वरू॑थ उ॒द्भौ स्वाहा᳚॥४१॥

(अग्नये जातवेदस \idam)


सह॑सा जा॒तान्प्रणु॑दा नः स॒पत्ना॒न्प्रत्यजा॑ताञ्जातवेदो नुदस्व। 
अधि॑ नो ब्रूहि सुमन॒स्यमा॑नो व॒यꣴ स्या॑म॒ प्रणु॑दा नः स॒पत्ना॒न्थ्स्वाहा᳚॥४२॥

(अग्नये जातवेदस \idam)


अग्ने॒ यो नो॒ऽभितो॒ जनो॒ वृको॒ वारो॒ जिघाꣳ॑सति। 
ताꣴस्त्वं वृ॑त्रहं जहि॒ वस्व॒स्मभ्य॒माभ॑र॒ स्वाहा᳚॥४३॥

(अग्नय \idam)


अग्ने॒ यो नो॑ऽभि॒दास॑ति समा॒नो यश्च॒ निष्ट्यः॑। 
तं व॒यꣳ स॒मिधं॑ कृ॒त्वा तुभ्य॑म॒ग्नेऽपि॑ दध्मसि॒ स्वाहा᳚॥४४॥

(अग्नय \idam)


यो नः॒ शपा॒दश॑पतो॒ यश्च॑ नः॒ शप॑तः॒ शपा᳚त्। 
उ॒षाश्च॒ तस्मै॑ नि॒म्रुक्च॒ सर्वं॑ पा॒पꣳ समू॑हता॒ꣴ॒ स्वाहा᳚॥४५॥

(उषो निम्रुग्भ्याम् \idam)


यो नः॑ स॒पत्नो॒ यो रणो॒ मर्तो॑ऽभि॒दास॑ति देवाः। 
इ॒ध्मस्ये॑व प्र॒क्षाय॑तो॒ मा तस्योच्छे॑षि॒ किं च॒न स्वाहा᳚॥४६॥

(अग्नय \idam)


यो मां द्वेष्टि॑ जातवेदो॒ यं चा॒हं द्वेष्मि॒ यश्च॒ माम्। 
सर्वा॒ꣴ॒स्तान॑ग्ने॒ सन्द॑ह॒ याꣴश्चा॒हं द्वेष्मि॒ ये च॒ माꣴ स्वाहा᳚॥४७॥

(अग्नय \idam)


यो अ॒स्मभ्य॑मराती॒याद्यश्च॑ नो॒ द्वेष॑ते॒ जनः॑। 
निन्दा॒द्यो अ॒स्मान्दिफ्सा᳚च्च॒ सर्वा॒ꣴ॒स्तान्म॑ष्म॒षा कु॑रु॒ स्वाहा᳚॥४८॥

(अग्नय \idam)


सꣳशि॑तं मे॒ ब्रह्म॒ सꣳशि॑तं वी॒र्या(१\char"E009)म्बलम्᳚। 
सꣳशि॑तं क्ष॒त्रं मे॑ जि॒ष्णु यस्या॒हमस्मि॑ पु॒रोहि॑तः॒ स्वाहा᳚॥४९॥

(ब्रह्मण \idam)


उदे॑षां बा॒हू अ॑तिर॒मुद्वर्चो॒ अथो॒ बलम्᳚। 
क्षि॒णोमि॒ ब्रह्म॑णा॒ऽमित्रा॒नुन्न॑\-यामि॒ स्वा(१)म् अ॒हꣴ स्वाहा᳚॥५०॥

(अग्नय \idam)


पुन॒र्मनः॒ पुन॒रायु॑र्म॒ आगा॒त्पुन॒श्चक्षुः॒ पुनः॒ श्रोत्रं॑ म॒ आगा॒त्पुनः॑ प्रा॒णः पुन॒राकू॑तं म॒ आगा॒त्पुन॑श्चि॒त्तं पुन॒राधी॑तं म॒ आगा᳚त्। 
वै॒श्वा॒न॒रो मेऽद॑ब्धस्तनू॒पा अव॑बाधतां दुरि॒तानि॒ विश्वा॒ स्वाहा᳚॥५१॥

(अग्नये वैश्वानराय \idam)


\sect{बोधायन-मतानुसारेण अष्टौ होममन्त्राः}

%2.7.7.1
सि॒ꣳ॒हे व्या॒घ्र उ॒त या पृदा॑कौ।
त्विषि॑र॒ग्नौ ब्रा᳚ह्म॒णे सूर्ये॒ या।
इन्द्रं॒ या दे॒वी सु॒भगा॑ ज॒जान॑।
सा न॒ आग॒न्वर्च॑सा संविदा॒ना स्वाहा᳚॥१॥

(त्विष्यै देव्या \idam)


या रा॑ज॒न्ये॑ दुन्दु॒भावाय॑तायाम्।
अश्व॑स्य॒ क्रन्द्ये॒ पुरु॑षस्य मा॒यौ।
इन्द्रं॒ या दे॒वी सु॒भगा॑ ज॒जान॑।
सा न॒ आग॒न्वर्च॑सा संविदा॒ना स्वाहा᳚॥२॥

(त्विष्यै देव्या \idam)

या ह॒स्तिनि॑ द्वी॒पिनि॒ या हिर॑ण्ये।
त्विषि॒रश्वे॑षु॒ पुरु॑षेषु॒ गोषु॑॥१६॥
इन्द्रं॒ या दे॒वी सु॒भगा॑ ज॒जान॑।
सा न॒ आग॒न्वर्च॑सा संविदा॒ना स्वाहा᳚॥३॥

(त्विष्यै देव्या \idam)



रथे॑ अ॒क्षेषु॑ वृष॒भस्य॒ वाजे᳚।
वाते॑ प॒र्जन्ये॒ वरु॑णस्य॒ शुष्मे᳚।
इन्द्रं॒ या दे॒वी सु॒भगा॑ ज॒जान॑।
सा न॒ आग॒न्वर्च॑सा संविदा॒ना स्वाहा᳚॥४॥

(त्विष्यै देव्या \idam)


%4.2.1.2
अग्ने᳚\-ऽभ्यावर्तिन्न॒भि न॒ आ व॑र्त॒स्वायु॑षा॒ वर्च॑सा स॒न्या मे॒धया᳚ प्र॒जया॒ धने॑न॒ स्वाहा᳚॥५॥

(अग्नये अभ्यावर्तिन \idam)

अग्ने᳚ अङ्गिरः श॒तं ते॑ सन्त्वा॒वृतः॑ स॒हस्रं॑ त उपा॒वृतः॑।
तासा॒म्पोष॑स्य॒ पोषे॑ण॒ पुन॑र्नो न॒ष्टमा कृ॑धि॒ पुन॑र्नो र॒यिमा कृ॑धि॒ स्वाहा᳚॥६॥

(अग्नये अभ्यावर्तिन \idam)


पुन॑रू॒र्जा नि व॑र्तस्व॒ पुन॑रग्न इ॒षायु॑षा। पुन॑र्नः पाहि वि॒श्वतः॒ स्वाहा᳚॥७॥

(अग्नये अभ्यावर्तिन \idam)


स॒ह र॒य्या नि व॑र्त॒स्वाग्ने॒ पिन्व॑स्व॒ धार॑या। वि॒श्वफ्स्नि॑या वि॒श्वत॒स्परि॒ स्वाहा᳚॥८॥

(अग्नये अभ्यावर्तिन \idam)


\sect{उपस्थान-मन्त्राः}

\centerline{\scriptsize (तैत्तिरीयारण्यकम् २-६)}


वै॒श्वा॒न॒राय॒ प्रति॑वेदयामो॒ यदी॑नृ॒णꣳ स॑ङ्ग॒रो दे॒वता॑सु। 
स ए॒तान्पाशा᳚न् प्र॒मुच॒न् प्रवे॑द॒ स नो॑ मुञ्चातु दुरि॒तादव॒द्यात्। 
वै॒श्वा॒न॒रः पव॑यान्नः प॒वित्रै॒र्यथ्स॑ङ्ग॒रम॒भिधावा᳚म्या॒शाम्। 
अना॑जान॒न्मन॑सा॒ याच॑मानो॒ यदत्रैनो॒ अव॒ तथ्सु॑वामि। 
अ॒मी ये सु॒भगे॑ दि॒वि वि॒चृतौ॒ नाम॒ तार॑के। 
प्रेहामृत॑स्य यच्छतामे॒तद्ब॑द्धक॒मोच॑नम्। 
विजि॑हीर्ष्व लो॒कान्कृ॑धि ब॒न्धान्मु॑ञ्चासि॒ बद्ध॑कम्। 
योने॑रिव॒ प्रच्यु॑तो॒ गर्भः॒ सर्वा᳚न् प॒थो अ॑नुष्व। 
स प्र॑जा॒नन्प्रति॑गृभ्णीत वि॒द्वान्प्र॒जाप॑तिः प्रथम॒जा ऋ॒तस्य॑। 
अ॒स्माभि॑र्द॒त्तं ज॒रसः॑ प॒रस्ता॒दच्छि॑न्नं॒ तन्तु॑मनु॒सञ्च॑रेम॥९॥

त॒तं तन्तु॒मन्वेके॒ अनु॒ सञ्च॑रन्ति॒ येषां᳚ द॒त्तं पित्र्य॒माय॑नवत्। 
अ॒ब॒न्ध्वेके॒ दद॑तः प्र॒यच्छा॒द्दातुं॒ चेच्छ॒क्नवा॒ꣳ॒सः स्व॒र्ग ए॑षाम्। 
आर॑भेथा॒मनु॒ सꣳर॑भेथाꣳ समा॒नं पन्था॑मवथो घृ॒तेन॑। 
यद्वां᳚ पू॒र्तं परि॑विष्टं॒ यद॒ग्नौ तस्मै॒ गोत्रा॑ये॒ह जाया॑पती॒ सꣳर॑भेथाम्। 
यद॒न्तरि॑क्षं पृथि॒वीमु॒त द्यां यन्मा॒तरं॑ पि॒तरं॑ वा जिहिꣳसि॒म। 
अ॒ग्निर्मा॒ तस्मा॒देन॑सो॒ गार्\mbox{}ह॑पत्य॒ उन्नो॑ नेषद्दुरि॒ता यानि॑ चकृ॒म। 
भूमि॑र्मा॒ताऽदि॑तिर्नो ज॒नित्रं॒ भ्राता॒ऽन्तरि॑क्षम॒भिश॑स्त एनः। 
द्यौर्नः॑ पि॒ता पि॑तृ॒याच्छं भ॑वासि जा॒मि मि॒त्वा मा वि॑विथ्सि लो॒कात्। 
यत्र॑ सु॒हार्दः॑ सु॒कृतो॒ मद॑न्ते वि॒हाय॒ रोगं॑ त॒न्वा(१\char"E009)ꣴ॒ स्वाया᳚म्। 
अ॒श्लो॒णाङ्गै॒रह्रु॑ताः स्व॒र्गे तत्र॑ पश्येम पि॒तरं॑ च पु॒त्रम्। 
यदन्न॒मद्म्यनृ॑तेन देवा दा॒स्यन्नदा᳚स्यन्नु॒त वा॑ करि॒ष्यन्। 
यद्दे॒वानां॒ चक्षु॒ष्यागो॒ अस्ति॒ यदे॒व किं च॑ प्रतिजग्रा॒हम॒ग्निर्मा॒ तस्मा॑दनृ॒णं कृ॑णोतु। 
यदन्न॒मद्मि॑ बहु॒धा विरू॑पं॒ वासो॒ हिर॑ण्यमु॒त गाम॒जामविम्᳚। 
यद्दे॒वानां॒ चक्षु॒ष्यागो॒ अस्ति॒ यदे॒व किं च॑ प्रतिजग्रा॒हम॒ग्निर्मा॒ तस्मा॑दनृ॒णं कृ॑णोतु। 
य॒न्मया॑ मन॑सा वा॒चा॒ कृ॒त॒मेनः॑ कदा॒चन। 
सर्वस्मा᳚त्तस्मा᳚न्मेळि॑तो मो॒ग्धि॒ त्वꣳ हि वेत्थ॑ यथात॒थम्॥
