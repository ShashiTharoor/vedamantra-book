% !TeX program = XeLaTeX
% !TeX root = ../vedamantrabook.tex
\chapt{भूसूक्तम्}
\centerline{\normalsize (तैत्तिरीय संहिता काण्डम् – १/प्रपाठकः – ५/अनुवाकः – ३)}

भूमि॑र्भू॒म्ना द्यौर्व॑रि॒णाऽन्तरि॑क्षं महि॒त्वा। उ॒पस्थे॑ ते देव्यदिते॒ऽग्नि\-म॑न्ना॒दम॒न्नाद्या॒याऽऽद॑धे।
आऽयङ्गौः पृश्नि॑रक्रमी॒\-दस॑नन्मा॒तरं॒ पुन॑। पि॒तरं॑ च प्र॒यन्त्सुव॑।
त्रि॒शद्धाम॒ वि रा॑जति॒ वाक्प॑त॒ङ्गाय॑ शिश्रिये। प्रत्य॑स्य वह॒ द्युभि॑। अ॒स्य प्रा॒णाद॑पान॒त्य॑न्तश्च॑रति रोच॒ना।
व्य॑ख्यन् महि॒षः सुव॑॥

यत्त्वा क्रु॒द्धः प॑रो॒वप॑ म॒न्युना॒ यदव॑र्त्या। सु॒कल्प॑मग्ने॒ तत्तव॒ पुन॒स्त्वोद्दी॑पयामसि।
यत्ते॑ म॒न्युप॑रोप्तस्य पृथि॒वीमनु॑ दध्व॒से। आ॒दि॒त्या विश्वे॒ तद्दे॒वा वस॑वश्च स॒माभ॑रन्।
मनो॒ ज्योति॑र्जुषता॒माज्यं॒ विच्छि॑न्नं य॒ज्ञ समि॒मं द॑धातु। बृह॒स्पति॑स्तनुतामि॒मं नो॒ विश्वे॑ दे॒वा इ॒ह मा॑दयन्ताम्।
 स॒प्त ते॑ अग्ने स॒मिध॑ स॒प्त जि॒ह्वाः स॒प्तर्‌ष॑यः स॒प्त धाम॑ प्रि॒याणि॑। 

स॒प्त होत्रा सप्त॒धा त्वा॑ यजन्ति स॒प्त योनी॒रापृ॑णस्वा घृ॒तेन॑। पुन॑रू॒र्जा नि व॑र्तस्व॒ पुन॑रग्न इ॒षाऽऽयु॑षा। पुन॑र्नः पाहि वि॒श्वत॑। स॒ह र॒य्या नि व॑र्त॒स्वाऽग्ने॒ पिन्व॑स्व॒ धार॑या। वि॒श्वफ्स्नि॑या वि॒श्वत॒स्परि॑। लेक॒ सले॑कः सु॒लेक॒स्ते न॑ आदि॒त्या आज्यं॑ जुषा॒णा वि॑यन्तु॒ केत॒ सके॑तः सु॒केत॒स्ते न॑ आदि॒त्या आज्यं॑ जुषा॒णा वि॑यन्तु॒ विव॑स्वा॒ अदि॑ति॒र्देव॑जूति॒स्ते न॑ आदि॒त्या आज्यं॑ जुषा॒णा वि॑यन्तु।

%मे॒दिनी॑ दे॒वी व॒सुन्ध॑रा स्या॒द्वसु॑दा दे॒वी वा॒सवी।
%ब्र॒ह्म॒व॒र्च॒सः पि॑तृ॒णा श्रोत्रं॒ चक्षु॒र्मन॑॥
%दे॒वी हिर॑ण्यगर्भिणी दे॒वी प्र॒सूव॑री ( प्र॒सोद॑री )।
%रस॑ने ( सद॑ने ) स॒त्याय॑ने सीद॥
%स॒मु॒द्रव॑ती सावि॒त्री ह॒नो दे॒वी म॒ह्यङ्गी।
%म॒हीधर॑णी म॒होव्यथि॑ष्टा ( म॒होध्यति॑ष्ठा ) श्शृ॒ङ्गे श्रृ॑ङ्गे य॒ज्ञे य॑ज्ञे विभी॒षणी॥
%इन्द्र॑पत्नी व्या॒पिनी॑ सु॒रस॑रिदि॒ह (Accent unknown सरसिज इह )।
%वा॒यु॒मती॑ जल॒शय॑नी श्रि॒यन्धा॒ (Accent unknown स्वयंधा ) राजा॑ स॒त्यन्तो॒ ( धो॒ ) परि॑मेदिनी॥
%श्वो॒परि॑धत्त॒ परि॑गाय।(Accent Unknown सो परिधत्तंगाय )
%वि॒ष्णु॒प॒त्नीं म॑हीं दे॒वीं॒ मा॒ध॒वीं मा॑धव॒प्रियां।
%लक्ष्मी प्रि॒यस॑खीं दे॒वीं॒ न॒मा॒म्यच्॑युत व॒ल्लभां॥
%ॐ ध॒नुर्ध॒रायै॑ वि॒द्महे॑ सर्वसि॒द्ध्यै च॑ धीमहि।
%तन्नो॑ धरा प्रचो॒दयात्॥