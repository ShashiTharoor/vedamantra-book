% !TeX program = XeLaTeX
% !TeX root = ../vedamantrabook.tex
\sect{सन्ध्योपासन-मन्त्राः}
%ꣳꣳ॑ꣳ॒ꣴꣴ॒ꣴ॑
\dnsub{प्रातः सन्ध्या सूर्योपस्थानम्}
मि॒त्रस्य॑ चर्\mbox{}षणी॒ धृतः॒ श्रवो॑ दे॒वस्य॑ सान॒सिम्। स॒त्यं चि॒त्रश्र॑वस्तमम्॥ मि॒त्रो जनान्॑ यातयति प्रजा॒नन्मि॒त्रो दा॑धार पृथि॒वीमु॒तद्याम्। मि॒त्रः कृ॒ष्टीरनि॑मिषा॒भिच॑ष्टे स॒त्याय॑ ह॒व्यं घृ॒तव॑द्विधेम॥ प्र समि॑त्र॒ मर्तो॑ अस्तु॒ प्रय॑स्वा॒न्॒ यस्त॑ आदित्य॒ शिक्ष॑ति व्र॒तेन॑। न ह॑न्यते॒ न जी॑यते॒ त्वोतो॒ नैन॒मꣳहो॑ अश्नो॒त्यन्ति॑तो॒ न दू॒रात्॥

\dnsub{माध्याह्निक सूर्योपस्थानम्}
आ स॒त्येन॒ रज॑सा॒ वर्त॑मानो निवे॒शय॑न्न॒मृतं॒ मर्त्यं॑ च। हि॒र॒ण्यये॑न सवि॒ता रथे॒नादे॒वो या॑ति॒ भुव॑ना वि॒पश्यन्॑। उद्व॒यं तम॑स॒स्परि॒ पश्य॑न्तो॒ ज्योति॒रुत्त॑रम्। दे॒वं दे॑व॒त्रा सूर्य॒मग॑न्म॒ ज्योति॑रुत्त॒मम्। उदु॒त्यं जा॒तवे॑दसं दे॒वं व॑हन्ति के॒तवः॑। दृ॒शे विश्वा॑य॒ सूर्यम्᳚। चि॒त्रं दे॒वाना॒मुद॑गा॒दनी॑कं॒ चक्षु॑र्मि॒त्रस्य॒ वरु॑णस्या॒ग्नेः। आ प्रा॒ द्यावा॑ पृथि॒वी अ॒न्तरि॑क्ष॒ꣳ॒ सूर्य॑ आ॒त्मा जग॑तस्त॒स्थुष॑श्च। तच्चक्षु॑र्दे॒वहि॑तं पु॒रस्ता᳚च्छु॒क्रमु॒च्चर॑त्॥

पश्ये॑म श॒रदः॑ श॒तं जीवे॑म श॒रदः॑ श॒तं नन्दा॑म श॒रदः॑ श॒तं मोदा॑म श॒रदः॑ श॒तं भवा॑म श॒रदः॑ श॒तꣳ शृ॒णवा॑म श॒रदः॑ श॒तं प्रब्र॑वाम श॒रदः॑ श॒तमजी॑ताः स्याम श॒रदः॑ श॒तं ज्योक्च॒ सूर्यं॑ दृ॒शे। य उद॑गान्मह॒तोर्णवा᳚द्वि॒भ्राज॑मानः सरि॒रस्य॒ मध्या॒थ्स मा॑ वृष॒भो लो॑हिता॒क्षः सूर्यो॑ विप॒श्चिन्मन॑सा पुनातु॥

\dnsub{सायं सन्ध्या सूर्योपस्थानम्}
इ॒मं मे॑ वरुण श्रुधी॒ हव॑म॒द्या च॑ मृडय। त्वाम॑व॒स्युराच॑के॥ तत्त्वा॑ यामि॒ ब्रह्म॑णा॒ वन्द॑मान॒स्तदाशा᳚स्ते॒ यज॑मानो ह॒विर्भिः॑। अहे॑डमानो वरुणे॒ह बो॒ध्युरु॑शꣳस॒ मा न॒ आयुः॒ प्रमो॑षीः॥
यच्चि॒द्धिते॒ विशो॑ यथा॒ प्रदे॑व वरुण व्र॒तम्। मि॒नी॒मसि॒ द्यवि॑द्यवि॥ यत्किं चे॒दं व॑रुण॒ दैव्ये॒ जने॑भिद्रो॒हं मनु॒ष्या᳚श्चरा॑मसि। अचि॑त्ती॒यत्तव॒ धर्मा॑ युयोपि॒म मा न॒स्तस्मा॒देन॑सो देव रीरिषः॥ कि॒त॒वासो॒ यद्रि॑रि॒पुर्नदी॒वि यद्वा॑ घा स॒त्यमु॒त यं न वि॒द्म। सर्वा॒ताविष्य॑ शिथि॒रेव॑ दे॒वाथा॑ ते स्याम वरुण प्रि॒यासः॑॥