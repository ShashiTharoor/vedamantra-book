% !TeX program = XeLaTeX
% !TeX root = ../vedamantrabook.tex
%ꣳꣳ॑ꣳ॒ꣴꣴ॒ꣴ॑
\setmainfont[Script=Devanagari,Mapping=tex-text,Mapping=devanagarinumerals,AutoFakeBold=2.0]{Siddhanta}
\title{\Huge कृष्णयजुर्वेद-ब्रह्मयज्ञः}
\date{}
\maketitle
\clearemptydoublepage
\chapt{ब्रह्मयज्ञः}

\sect{सङ्कल्पः}

\textbf{आचमनम्।}

\twolineshloka*
{शुक्लाम्बरधरं विष्णुं शशिवर्णं चतुर्भुजम्}
{प्रसन्नवदनं ध्यायेत् सर्वविघ्नोपशान्तये}

\textbf{प्राणायामः।}

ममोपात्त-समस्त-दुरित-क्षय-द्वारा श्री-परमेश्वर-प्रीत्यर्थं ब्रह्मयज्ञं करिष्ये। ब्रह्मयज्ञेन यक्ष्ये।


\dnsub{ब्रह्मयज्ञ-प्रशंसा}
\vspace{-1ex}
\centerline{\scriptsize(तैत्तिरीयारण्यकम्/प्रपाठकः – २/अनुवाकः – १४-१५)}
{\normalsize
पञ्च॒ वा ए॒ते म॑हाय॒ज्ञाः स॑त॒ति प्रता॑यन्ते सत॒ति सन्ति॑ष्ठन्ते देवय॒ज्ञः पि॑तृय॒ज्ञो भू॑तय॒ज्ञो म॑नुष्यय॒ज्ञो ब्र॑ह्मय॒ज्ञ इति॒ यद॒ग्नौ जु॒होत्य॒पि स॒मिधं॒ तद्दे॑वय॒ज्ञः सन्ति॑ष्ठते॒ यत्पि॒तृभ्यः॑ स्व॒धा क॒रोत्यप्य॒पस्तत्पि॑तृय॒ज्ञः सन्ति॑ष्ठते॒ यद्भू॒तेभ्यो॑ ब॒लिꣳ हर॑ति॒ तद्भू॑तय॒ज्ञः सन्ति॑ष्ठते॒ यद्ब्रा᳚ह्म॒णेभ्योऽन्नं॒ ददा॑ति॒ तन्म॑नुष्यय॒ज्ञः सन्ति॑ष्ठते॒ यथ्स्वा᳚ध्या॒यमधी॑यी॒तैका॑मप्यृ॒चं यजुः॒ साम॑ वा॒ तद्ब्र॑ह्मय॒ज्ञः सन्ति॑ष्ठते॒ यदृ॒चोऽधी॑ते॒ पय॑सः॒ कूल्या॑ अस्य पि॒तॄन्थ्स्व॒धा अ॒भिव॑हन्ति॒ यद्यजूꣳ॑षि घृ॒तस्य॑ कूल्या॒ यथ्सामा॑नि॒ सोम॑ एभ्यः पवते॒ यदथ॑र्वाङ्गि॒रसो॒ मधोः᳚ कूल्या॒ यद्ब्रा᳚ह्म॒णानी॑तिहा॒सान् पु॑रा॒णानि॒ कल्पा॒न्गाथा॑ नाराश॒ꣳ॒सीर्मेद॑सः॒ कूल्या॑ अस्य पि॒तॄन्थ्स्व॒धा अ॒भिव॑हन्ति॒ यदृ॒चोऽधी॑ते॒ पय॑आहुतिभिरे॒व तद्दे॒वाꣴस्त॑र्पयति॒ यद्यजूꣳ॑षि घृ॒ताहु॑तिभि॒र्यथ्सामा॑नि॒ सोमा॑हुतिभि॒र्यदथ॑र्वाङ्गि॒रसो॒ मध्वा॑\-हुतिभि॒र्यद्ब्रा᳚ह्म॒णानी॑तिहा॒सान् पु॑रा॒णानि॒ कल्पा॒न्गाथा॑ नाराश॒ꣳ॒सीर्मे॑दाहु॒तिभि॑रे॒व तद्दे॒वाꣴस्त॑र्पयति॒ त ए॑नं तृ॒प्ता आयु॑षा॒ तेज॑सा॒ वर्च॑सा श्रि॒या यश॑सा ब्रह्मवर्च॒सेना॒न्नाद्ये॑न च तर्पयन्ति॥१४॥

ब्र॒ह्म॒य॒ज्ञेन॑ य॒क्ष्यमा॑णः॒ प्राच्यां᳚ दि॒शि ग्रामा॒दछ॑दिर्द॒र्\mbox{}श उदी᳚च्यां प्रागुदी॒च्यां वो॒दित॑ आदि॒त्ये द॑क्षिण॒त उ॑प॒वीयो॑प॒विश्य॒ हस्ता॑वव॒निज्य॒ त्रिराचा॑मे॒द्द्विः प॑रि॒मृज्य॑ स॒कृदु॑प॒स्पृश्य॒ शिर॒श्चक्षु॑षी॒ नासि॑के॒ श्रोत्रे॒ हृद॑यमा॒लभ्य॒ यत्त्रिरा॒चाम॑ति॒ तेन॒ ऋचः॑ प्रीणाति॒ यद्द्विः प॑रि॒मृज॑ति॒ तेन॒ यजूꣳ॑षि॒ यथ्स॒कृदु॑प॒स्पृश॑ति॒ तेन॒ सामा॑नि॒ यथ्स॒व्यं पा॒णिं पा॒दौ प्रो॒क्षति॒ यच्छिर॒श्चक्षु॑षी॒ नासि॑के॒ श्रोत्रे॒ हृद॑यमा॒लभ॑ते॒ तेनाथ॑र्वाङ्गि॒रसो᳚ ब्राह्म॒णानी॑तिहा॒सान् पु॑रा॒णानि॒ कल्पा॒न्गाथा॑ नाराश॒ꣳ॒सीः प्री॑णाति॒ दर्भा॑णां म॒हदु॑प॒स्तीर्यो॒पस्थं॑ कृ॒त्वा प्राङासी॑नः स्वाध्या॒यमधी॑यीता॒पां वा ए॒ष ओष॑धीना॒ꣳ॒ रसो॒ यद्द॒र्भाः सर॑समे॒व ब्रह्म॑ कुरुते दक्षिणोत्त॒रौ पा॒णी पा॒दौ कृ॒त्वा सप॒वित्रा॒वोमिति॒ प्रति॑पद्यत ए॒तद्वै यजु॑स्त्रयीं वि॒द्यां प्रत्ये॒षा वागे॒तत्प॑र॒मम॒क्षरं॒ तदे॒तदृ॒चाऽभ्यु॑क्तमृ॒चो अ॒क्षरे॑ पर॒मे व्यो॑म॒न् यस्मि॑न्दे॒वा अधि॒ विश्वे॑ निषे॒दुर्यस्तन्न वेद॒ किमृ॒चा क॑रिष्यति॒ य इत्तद्वि॒दुस्त इ॒मे समा॑सत॒ इति॒ त्रीने॒व प्रायु॑ङ्क्त॒ भूर्भुवः॒ स्व॑रित्या॑है॒तद्वै वा॒चः स॒त्यं यदे॒व वा॒चः स॒त्यं तत्प्रायु॒ङ्क्ताथ॑ सावि॒त्रीं गा॑य॒त्रीं त्रिरन्वा॑ह प॒च्छो᳚ऽर्धर्च॑शोऽनवा॒नꣳ स॑वि॒ता श्रियः॑ प्रसवि॒ता श्रिय॑मे॒वाऽऽप्नो॒त्यथो᳚ प्र॒ज्ञात॑यै॒व प्र॑ति॒पदा॒ छन्दाꣳ॑सि॒ प्रति॑पद्यते॥१५॥


}

\sect{यज्ञः}
विद्यु॑दसि॒ विद्य॑ मे पा॒पमान॑मृ॒तात् स॒त्यमुपै॑मि।

त्रिराचा॑मेत्।\\
द्विः प॑रि॒मृज्य॑।\\
स॒कृदु॑प॒स्पृश्य।\\
शि॒र॒श्चक्षु॑षी॒ नासि॑के॒ श्रोत्रे॒ हृद॑यमा॒लभ्य।

उपस्थं॑ कृ॒त्वा।{\scriptsize [Sit with right leg on top of the left leg]}

ॐ भूः । तत् स॑वि॒तुर्वरे॑॑ण्यम्।\
ॐ भुवः। भर्गो॑ दे॒वस्य॑ धीमहि।\\
ओꣳ सुवः। धियो॒ यो नः॑ प्रचो॒दया॑॑त्।\\
ॐ भूः तत् स॑वि॒तुर्वरे॑॑ण्यम्। भर्गो॑ दे॒वस्य॑ धीमहि।\\
ॐ भुवः। धियो॒ यो नः॑ प्रचो॒दया॑॑त्।\\
ओꣳसुवः। तत् स॑वि॒तुर्वरे॑॑ण्यम्। भर्गो॑ दे॒वस्य॑ धीमहि। धियो॒ यो नः॑ प्रचो॒दया॑॑त्।\\

\dnsub{वेदादयः}
हरिः ॐ। अ॒ग्निमी᳚ळे पु॒रोहि॑तं य॒ज्ञस्य॑ दे॒वमृ॒त्विजम्᳚। होता᳚रं रत्न॒-धात॑मम्॥ हरिः॑ ॐ॥\\

हरिः ॐ। इ॒षेत्वो॒र्जे त्वा॑ वा॒यवः॑ स्थो पा॒यवः॑ स्थ दे॒वो वः॑ सवि॒ता प्रार्प॑यतु॒ श्रेष्ठ॑तमाय॒ कर्म॑णे॥ हरिः॑ ॐ॥ \\

हरिः ॐ। अग्न॒ आया॑हि वी॒तये॑ गृणा॒नो ह॒व्यदा॑तये। नि होता॑ सथ्सि ब॒र्हिषि॑॥ हरिः॑ ॐ॥\\

हरिः ॐ। शन्नो॑ दे॒वीर॒भिष्ट॑य॒ आपो॑ भवन्तु पी॒तये᳚। शं योर॒भिस्र॑वन्तु नः॥ हरिः॑ ॐ॥\\

{\centering
\twolineshloka*
{ब्रह्मयज्ञे जपन् सूक्तं पौरुषं चिन्तयन् हरिम्}
{स सर्वान्स्तु जपेन्वेदान् साङ्गोपाङ्गान् द्विजोत्तमः}
}
{\hfill [वैद्यनाथ-दीक्षितीयम्]}

{\scriptsize [take a little water on the palm and rotate the palm around the head]}\\
ॐ भूर्भुवः॒ सुवः॑।

सत्यं तपः श्रद्धायां॑ जुहो॒मि॥

\dnsub{परिधानीया}

{\scriptsize [chant three times]}\\

ॐ॥ नमो॒ ब्रह्म॑णे॒ नमो॑ अस्त्व॒ग्नये॒ नमः॑ पृथि॒व्यै नम॒ ओष॑धीभ्यः।
नमो॑ वा॒चे नमो॑ वा॒चस्पत॑ये॒ नमो॒ विष्ण॑वे बृह॒ते क॑रोमि॥ (त्रिः)

{\scriptsize [wash the front and back of both the hands while chanting the following mantra]}\\
वृष्टि॑रसि॒ वृश्च॑ मे पा॒प्मान॑मृ॒ताथ्स॒त्यमुपा॑गाम्॥

\sect{देवर्षिपितृ-तर्पणम्}
\vspace{-1ex}
{\scriptsize [उपवीती। सकृत् देवतीर्थेन।]}\\
ब्रह्मादयो ये देवाः तान् देवाꣳस्तर्पयामि।\\
सर्वान् देवाꣳस्तर्पयामि।\\
सर्वदेवगणाꣳस्तर्पयामि।\\
सर्वदेवपत्नीस्तर्पयामि।\\
सर्वदेवगणपत्नीस्तर्पयामि॥

{\scriptsize [निवीती। द्विः। ऋषितीर्थेन।]}\\
कृष्णद्वैपायनादयो ये ऋषयस्तान् ऋषीꣳस्तर्पयामि।\\
सर्वान् ऋषीꣳस्तर्पयामि।\\
सर्वर्षिगणाꣳस्तर्पयामि।\\
सर्वर्षिपत्नीस्तर्पयामि।\\
सर्वर्षिगणपत्नीस्तर्पयामि।\\
प्रजापतिं काण्डऋषिं तर्पयामि।\\
सोमं काण्डऋषिं तर्पयामि।\\
अग्निं काण्डऋषिं तर्पयामि।\\
विश्वान् देवान् काण्डऋषीꣳस्तर्पयामि।
\pagebreak[4]

{\scriptsize [सकृत् देवतीर्थेन।]}\\
साꣳहितीर्देवताः उपनिषदस्तर्पयामि।\\
याज्ञिकीर्देवताः उपनिषदस्तर्पयामि।\\
वारुणीर्देवताः उपनिषदस्तर्पयामि।\\
हव्यवाहं तर्पयामि।\\
विश्वान् देवान् काण्डऋषीꣳस्तर्पयामि।

{\scriptsize [द्विः। ब्रह्मतीर्थेन।]}\\
ब्रह्माणं स्वयम्भुवं तर्पयामि।

{\scriptsize [पुनः ऋषितीर्थेन। द्विः।]}\\
विश्वान् देवान् काण्डऋषीꣳस्तर्पयामि।\\
अरुणान् काण्डऋषीꣳस्तर्पयामि।

{\scriptsize [सकृत् देवतीर्थेन।]}\\
सदसस्पतिं तर्पयामि।\\
ऋग्वेदं तर्पयामि।\\
यजुर्वेदं तर्पयामि।\\
सामवेदं तर्पयामि।\\
अथर्ववेदं तर्पयामि।\\
इतिहासपुराणं तर्पयामि। कल्पं तर्पयामि।

{\scriptsize [प्राचीनावीती। त्रिः। पितृतीर्थेन।]}\\
सोमः पितृमान् यमोऽङ्गिरस्वान् अग्निः कव्यवाहनः इत्यादयो ये पितरस्तान् पितॄꣳस्तर्पयामि।\\
सर्वान् पितॄꣳस्तर्पयामि।\\
सर्वपितृगणाꣳस्तर्पयामि।\\
सर्वपितृ पत्नीस्तर्पयामि।\\
सर्वपितृ गणपत्नीस्तर्पयामि।\\
ऊर्जं वहन्ती-रमृतं घृतं पयः कीलालं परिस्रुतं स्वधास्थ तर्पयत मे पितॄन् तृप्यत तृप्यत तृप्यत॥

{\scriptsize [उपवीती।]}\\
\dnsub{समर्पणम्}
\vspace{-3ex}
\fourlineindentedshloka*
{कायेन वाचा मनसेन्द्रियैर्वा}
{बुद्‌ध्याऽऽत्मना वा प्रकृतेः स्वभावात्}
{करोमि यद्यत् सकलं परस्मै}
{नारायणायेति समर्पयामि}

\textbf{आचमनम्।}

