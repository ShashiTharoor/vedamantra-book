% !TeX program = XeLaTeX
% !TeX root = ../vedamantrabook.tex
\newcommand{\akhkhi}{\char"0916\char"094D\char"0916\char"093F\char"E009\hspace{-2ex} \char"E009\hspace{1.2ex}}

\chapt{रुद्रपदपाठः}
ॐ। ग॒णाना᳚म्। त्वा॒। ग॒णप॑ति॒मिति॑ ग॒ण-प॒ति॒म्॒। ह॒वा॒म॒हे॒। क॒विम्। क॒वी॒नाम्। उ॒प॒मश्र॑वस्तम॒मित्यु॑प॒मश्र॑वः-त॒म॒म्॒॥ 
ज्ये॒ष्ठ॒राज॒मिति॑ ज्येष्ठ-राजम्᳚। ब्रह्म॑णाम्। ब्र॒ह्म॒णः॒। प॒ते॒। एति॑। नः॒। शृ॒ण्वन्। ऊ॒तिभि॒रित्यू॒ति-भिः॒। सी॒द॒। साद॑नम्॥ 


नमः॑। ते॒। रु॒द्र॒। म॒न्यवे᳚। उ॒तो इति॑। ते॒। इष॑वे। नमः॑॥ 
नमः॑। ते॒। अ॒स्तु॒। धन्व॑ने। बा॒हुभ्या॒मिति॑ बा॒हु-भ्या॒म्॒। उ॒त। ते॒। नमः॑॥ 
या। ते॒। इषुः॑। शि॒वत॒मेति॑ शि॒व-त॒मा॒। शि॒वम्। ब॒भूव॑। ते॒। धनुः॑॥ 
शि॒वा। श॒र॒व्या᳚। या। तव॑। तया᳚। नः॒। रु॒द्र॒। मृ॒ड॒य॒॥ 
या। ते॒। रु॒द्र॒। शि॒वा। त॒नूः। अघो॑रा। अपा॑पकाशि॒नीत्यपा॑प-का॒शि॒नी॒॥ 
तया᳚। नः॒। त॒नुवा᳚। शन्त॑म॒येति॒ शम्-त॒म॒या॒। गिरि॑श॒न्तेति॒ गिरि॑-श॒न्त॒। अभीति॑। चा॒क॒शी॒हि॒॥ 
याम्। इषुम्᳚। गि॒रि॒श॒न्तेति॑ गिरि-श॒न्त॒। हस्ते᳚।~(१)


बिभ॑र्‌षि। अस्त॑वे॥ 
शि॒वाम्। गि॒रि॒त्रेति॑ गिरि-त्र॒। ताम्। कु॒रु॒। मा। हि॒ꣳ॒सीः॒। पुरु॑षम्। जग॑त्॥ 
शि॒वेन॑। वच॑सा। त्वा॒। गिरि॑श। अच्छा॑। व॒दा॒म॒सि॒॥ 
यथा᳚। नः॒। सर्वम्᳚। इत्। जग॑त्। अ॒य॒क्ष्मम्। सु॒मना॒ इति॑ सु-मनाः᳚। अस॑त्॥ 
अधीति॑। अ॒वो॒च॒त्॒। अ॒धि॒व॒क्तेत्य॑धि-व॒क्ता। प्र॒थ॒मः। दैव्यः॑। भि॒षक्॥ 
अहीन्। च॒। सर्वान्। ज॒म्भयन्। सर्वाः᳚। च॒। या॒तु॒धा॒न्य॑ इति॑ यातु-धा॒न्यः॑॥ 
अ॒सौ। यः। ता॒म्रः। अ॒रु॒णः। उ॒त। ब॒भ्रुः। सु॒म॒ङ्गल॒ इति॑ सु-म॒ङ्गलः॑॥ 
ये। च॒। इ॒माम्। रु॒द्राः। अ॒भितः॑। दि॒क्षु।~(२)


श्रि॒ताः। स॒ह॒स्र॒श इति॑ सहस्र-शः। अवेति॑। ए॒षा॒म्। हेडः॑। ई॒म॒हे॒॥ 
अ॒सौ। यः। अ॒व॒सर्प॒तीत्य॑व-सर्प॑ति। नील॑ग्रीव॒ इति॒ नील॑-ग्री॒वः॒। विलो॑हित॒ इति॒ वि-लो॒हि॒तः॒॥ 
उ॒त। ए॒न॒म्॒। गो॒पा इति॑ गो-पाः। अ॒दृ॒श॒न्। अदृ॑शन्। उ॒द॒हा॒र्य॑ इत्यु॑द-हा॒र्यः॑॥ 
उ॒त। ए॒न॒म्। विश्वा᳚। भू॒तानि॑। सः। दृ॒ष्टः। मृ॒ड॒या॒ति॒। नः॒॥ 
नमः॑। अ॒स्तु॒। नील॑ग्रीवा॒येति॒ नील॑-ग्री॒वा॒य॒। स॒ह॒स्रा॒क्षायेति॑ सहस्र-अ॒क्षाय॑। मी॒ढुषे᳚॥
अथो॒ इति॑। ये। अ॒स्य॒। सत्वा॑नः। अ॒हम्। तेभ्यः॑। अ॒क॒र॒म्। नमः॑॥ 
प्रेति॑। मु॒ञ्च॒। धन्व॑नः। त्वम्। उ॒भयोः᳚। आर्त्नि॑योः। ज्याम्॥ 
याः। च॒। ते॒। हस्ते᳚। इष॑वः।~(३)


परेति॑। ताः। भ॒ग॒व॒ इति॑ भग-वः॒। व॒प॒॥ 
अ॒व॒तत्येत्य॑व-तत्य॑। धनुः॑। त्वम्। सह॑स्रा॒क्षेति॒ सह॑स्र-अ॒क्ष॒। शते॑षुध॒ इति॒ शत॑-इ॒षु॒धे॒॥ 
नि॒शीर्येति॑ नि-शीर्य॑। श॒ल्याना᳚म्। मुखा᳚। शि॒वः। नः॒। सु॒मना॒ इति॑ सु-मनाः᳚। भ॒व॒॥ 
विज्य॒मिति॒ वि-ज्य॒म्। धनुः॑। क॒प॒र्दिनः॑। विश॑ल्य॒ इति॒ वि-श॒ल्यः॒। बाण॑वा॒निति॒ बाण॑-वा॒न्॒। उ॒त॥ 
अने॑शन्। अ॒स्य॒। इष॑वः। आ॒भुः। अ॒स्य॒। नि॒ष॒ङ्गथिः॑॥ 
या। ते॒। हे॒तिः। मी॒ढु॒ष्ट॒मेति॑ मीढुः-त॒म॒। हस्ते᳚। ब॒भूव॑। ते॒। धनुः॑॥ 
तया᳚। अ॒स्मान्। वि॒श्वतः॑। त्वम्। अ॒य॒क्ष्मया᳚। परीति॑। भु॒ज॒॥ 
नमः॑। ते॒। अ॒स्तु॒। आयु॑धाय। अना॑तता॒येत्यना᳚-त॒ता॒य॒। धृ॒ष्णवे᳚॥ 
उ॒भाभ्या᳚म्। उ॒त। ते॒। नमः॑। बा॒हुभ्या॒मिति॑ बा॒हु-भ्या॒म्॒। तव॑। धन्व॑ने॥ 
परीति॑। ते॒। धन्व॑नः। हे॒तिः। अ॒स्मान्। वृ॒ण॒क्तु॒। वि॒श्वतः॑॥
 अथो॒ इति॑। यः। इ॒षु॒धिरिती॑षु-धिः। तव॑। आ॒रे। अ॒स्मत्। नीति॑। धे॒हि॒। तम्॥~(४)


नमः॑। हिर॑ण्यबाहव॒ इति॒ हिर॑ण्य-बा॒ह॒वे॒। से॒ना॒न्य॑ इति॑ सेना-न्ये᳚। दि॒शाम्। च॒। पत॑ये। नमः॑। 
नमः॑। वृ॒क्षेभ्यः॑। हरि॑केशेभ्य॒ इति॒ हरि॑-के॒शे॒भ्यः॒। प॒शू॒नाम्। पत॑ये। नमः॑। 
नमः॑। स॒स्पिञ्ज॑राय। त्विषी॑मत॒ इति॒ त्विषी॑-म॒ते॒। प॒थी॒नाम्। पत॑ये। नमः॑। 
नमः॑। ब॒भ्लु॒शाय॑। वि॒व्या॒धिन॒ इति॑ वि-व्या॒धिने᳚। अन्ना॑नाम्। पत॑ये। नमः॑। 
नमः॑। हरि॑केशा॒येति॒ हरि॑-के॒शा॒य॒। उ॒प॒वी॒तिन॒ इत्यु॑प-वी॒तिने᳚। पु॒ष्टाना᳚म्। पत॑ये। नमः॑। 
नमः॑। भ॒वस्य॑। हे॒त्यै। जग॑ताम्। पत॑ये। नमः॑। 
नमः॑। रु॒द्राय॑। आ॒त॒ता॒विन॒ इत्या᳚-त॒ता॒विने᳚। क्षेत्रा॑णाम्। पत॑ये। नमः॑। 
नमः॑। सू॒ताय॑। अह॑न्त्याय। वना॑नाम्। पत॑ये। नमः॑। 
नमः॑।~(५)


रोहि॑ताय। स्थ॒पत॑ये। वृ॒क्षाणा᳚म्। पत॑ये। नमः॑। 
नमः॑। म॒न्त्रिणे᳚। वा॒णि॒जाय॑। कक्षा॑णाम्। पत॑ये। नमः॑। 
नमः॑। भु॒व॒न्तये᳚। वा॒रि॒व॒स्कृ॒तायेति॑ वारिवः-कृ॒ताय॑। ओष॑धीनाम्। पत॑ये। नमः॑। 
नमः॑। उ॒च्चैर्घो॑षा॒येत्यु॒च्चैः-घो॒षा॒य॒। आ॒क्र॒न्दय॑त॒ इत्या᳚-क्र॒न्दय॑ते। प॒त्ती॒नाम्। पत॑ये। नमः॑। 
नमः॑। कृ॒थ्स्न॒वी॒तायेति॑ कृथ्स्न-वी॒ताय॑। धाव॑ते। सत्व॑नाम्। पत॑ये। नमः॑॥~(६)


नमः॑। सह॑मानाय। नि॒व्या॒धिन॒ इति॑ नि-व्या॒धिने᳚। आ॒व्या॒धिनी॑ना॒\-मित्या᳚-व्या॒धिनी॑नाम्। पत॑ये। नमः॑। 
नमः॑। क॒कु॒भाय॑। नि॒ष॒ङ्गिण॒ इति॑ नि-स॒ङ्गिने᳚। स्ते॒नाना᳚म्। पत॑ये। नमः॑। 
नमः॑। नि॒ष॒ङ्गिण॒ इति॑ नि-स॒ङ्गिने᳚। इ॒षु॒धि॒मत॒ इती॑षुधि-मते᳚। तस्क॑राणाम्। पत॑ये। नमः॑। 
नमः॑। वञ्च॑ते। प॒रि॒वञ्च॑त॒ इति॑ परि-वञ्च॑ते। स्ता॒यू॒नाम्। पत॑ये। नमः॑। 
नमः॑। नि॒चे॒रव॒ इति॑ नि-चे॒रवे᳚। प॒रि॒च॒रायेति॑ परि-च॒राय॑। अर॑ण्यानाम्। पत॑ये। नमः॑। 
नमः॑। सृ॒का॒विभ्य॒ इति॑ सृका॒वि-भ्यः॒। जिघाꣳ॑सद्भ्य॒ इति॒ जिघाꣳ॑सत्-भ्यः॒। मु॒ष्ण॒ताम्। पत॑ये। नमः॑। 
नमः॑। अ॒सि॒मद्भ्य॒ इत्य॑सि॒मत्-भ्यः॒। नक्तम्᳚। चर॑द्भ्य॒ इति॒ चर॑त्-भ्यः॒। प्र॒कृ॒न्ताना॒मिति॑ प्र-कृ॒न्ताना᳚म्। पत॑ये। नमः॑। 
नमः॑। उ॒ष्णी॒षिणे᳚। गि॒रि॒च॒रायेति॑ गिरि-च॒राय॑। कु॒लु॒ञ्चाना᳚म्। पत॑ये। नमः॑। नमः॑।~(७)


इषु॑मद्भ्य॒ इतीषु॑मत्-भ्यः॒। ध॒न्वा॒विभ्य॒ इति॑ धन्वा॒वि-भ्यः॒। च॒। वः॒। नमः॑। 
नमः॑। आ॒त॒न्वा॒नेभ्य॒ इत्या᳚-त॒न्वा॒नेभ्यः॑। प्र॒ति॒दधा॑नेभ्य॒ इति॑ प्रति-दधा॑नेभ्यः। च॒। वः॒। नमः॑। 
नमः॑। आ॒यच्छ॑द्भ्य॒ इत्या॒यच्छ॑त्-भ्यः॒। वि॒सृ॒जद्भ्य॒ इति॑ विसृ॒जत्-भ्यः॒। च॒। वः॒। नमः॑। 
नमः॑। अस्य॑द्भ्य॒ इत्यस्य॑त्-भ्यः॒। विध्य॑द्भ्य॒ इति॒ विध्य॑त्-भ्यः॒। च॒। वः॒। नमः॑। 
नमः॑। आसी॑नेभ्यः। शया॑नेभ्यः। च॒। वः॒। नमः॑। 
नमः॑। स्व॒पद्भ्य॒ इति॑ स्व॒पत्-भ्यः॒। जाग्र॑द्भ्य॒ इति॒ जाग्र॑त्-भ्यः॒। च॒। वः॒। नमः॑। 
नमः॑। तिष्ठ॑द्भ्य॒ इति॒ तिष्ठ॑त्-भ्यः॒। धाव॑द्भ्य॒ इति॒ धाव॑त्-भ्यः॒। च॒। वः॒। नमः॑। 
नमः॑। स॒भाभ्यः॑। स॒भाप॑तिभ्य॒ इति॑ स॒भाप॑ति-भ्यः॒। च॒। वः॒। नमः॑। 
नमः॑। अश्वे᳚भ्यः। अश्व॑पतिभ्य॒ इत्यश्व॑पति-भ्यः॒। च॒। वः॒। नमः॑॥~(८)


नमः॑। आ॒व्या॒धिनी᳚भ्य॒ इत्या᳚-व्या॒धिनी᳚भ्यः। वि॒विध्य॑न्तीभ्य॒ इति॑ वि-विध्य॑न्तीभ्यः। च॒। वः॒। नमः॑। नमः॑। उग॑णाभ्यः। तृ॒ꣳ॒ह॒तीभ्यः॑। च॒। वः॒। नमः॑। 
नमः॑। गृ॒थ्सेभ्यः॑। गृ॒थ्सप॑तिभ्य॒ इति॑ गृ॒थ्सप॑ति-भ्यः॒। च॒। वः॒। नमः॑। 
नमः॑। व्राते᳚भ्यः। व्रात॑पतिभ्य॒ इति॒ व्रात॑पति-भ्यः॒। च॒। वः॒। नमः॑। 
नमः॑। ग॒णेभ्यः॑। ग॒णप॑तिभ्य॒ इति॑ ग॒णप॑ति-भ्यः॒। च॒। वः॒। नमः॑। 
नमः॑। विरू॑पेभ्य॒ इति॑ वि-रू॒पे॒भ्यः॒। वि॒श्वरू॑पेभ्य॒ इति॑ वि॒श्व-रू॒पे॒भ्यः॒। च॒। वः॒। नमः॑। 
नमः॑। म॒हद्भ्य॒ इति॑ म॒हत्-भ्यः॒। क्षु॒ल्ल॒केभ्यः॑। च॒। वः॒। नमः॑। 
नमः॑। र॒थिभ्य॒ इति॑ र॒थि-भ्यः॒। अ॒र॒थेभ्यः॑। च॒। वः॒। नमः॑। 
नमः॑। रथे᳚भ्यः।~(९)


रथ॑पतिभ्य॒ इति॒ रथ॑पति-भ्यः॒। च॒। वः॒। नमः॑। 
नमः॑। सेना᳚भ्यः। से॒ना॒निभ्य॒ इति॑ सेना॒नि-भ्यः॒। च॒। वः॒। नमः॑। 
नमः॑। क्ष॒त्तृभ्य॒ इति॑ क्ष॒त्तृ-भ्यः॒। स॒ङ्ग्र॒ही॒तृभ्य॒ इति॑ सङ्ग्रही॒तृ-भ्यः॒। च॒। वः॒। नमः॑। 
नमः॑। तक्ष॑भ्य॒ इति॑ तक्ष॑-भ्यः॒। र॒थ॒का॒रेभ्य॒ इति॑ रथ-का॒रेभ्यः॑। च॒। वः॒। नमः॑। 
नमः॑। कुला॑लेभ्यः। क॒र्मारे᳚भ्यः। च॒। वः॒। नमः॑। 
नमः॑। पु॒ञ्जिष्टे᳚भ्यः। नि॒षा॒देभ्यः॑। च॒। वः॒। नमः॑। 
नमः॑। इ॒षु॒कृद्भ्य॒ इती॑षु॒कृत्-भ्यः॒। ध॒न्व॒कृद्भ्य॒ इति॑ धन्व॒कृत्-भ्यः॒। च॒। वः॒। नमः॑। 
नमः॑। मृ॒ग॒युभ्य॒ इति॑ मृग॒यु-भ्यः॒। श्व॒निभ्य॒ इति॑ श्व॒नि-भ्यः॒। च॒। वः॒। नमः॑। 
नमः॑। श्वभ्य॒ इति॒ श्व-भ्यः॒। श्वप॑तिभ्य॒ इति॒ श्वप॑ति-भ्यः॒। च॒। वः॒। नमः॑॥~(१०)


नमः॑। भ॒वाय॑। च॒। रु॒द्राय॑। च॒। 
नमः॑। श॒र्वाय॑। च॒। प॒शु॒पत॑य॒ इति॑ पशु-पत॑ये। च॒। 
नमः॑। नील॑ग्रीवा॒येति॒ नील॑-ग्री॒वा॒य॒। च॒। शि॒ति॒कण्ठा॒येति॑ शिति-कण्ठा॑य। च॒। 
नमः॑। क॒प॒र्दिने᳚। च॒। व्यु॑प्तकेशा॒येति॒ व्यु॑प्त-के॒शा॒य॒। च॒। 
नमः॑। स॒ह॒स्रा॒क्षायेति॑ सहस्र-अ॒क्षाय॑। च॒। श॒त॒ध॑न्वन॒ इति॑ श॒त-ध॒न्व॒ने॒। च॒। 
नमः॑। गि॒रि॒शाय॑। च॒। शि॒पि॒वि॒ष्टायेति॑ शिपि-वि॒ष्टाय॑। च॒। 
नमः॑। मी॒ढुष्ट॑मा॒येति॑ मी॒ढुः-त॒मा॒य॒। च॒। इषु॑मत॒ इतीषु॑-म॒ते॒। च॒। 
नमः॑। ह्र॒स्वाय॑। च॒। वा॒म॒नाय॑। च॒। 
नमः॑। बृ॒ह॒ते। च॒। वर्‌षी॑यसे। च॒। 
नमः॑। वृ॒द्धाय॑। च॒। सं॒वृध्व॑न॒ इति॑ सम्-वृध्व॑ने। च॒।~(११)


नमः॑। अग्रि॑याय। च॒। प्र॒थ॒माय॑। च॒। 
नमः॑। आ॒शवे᳚। च॒। अ॒जि॒राय॑। च॒। 
नमः॑। शीघ्रि॑याय। च॒। शीभ्या॑य। च॒। 
नमः॑। ऊ॒र्म्या॑य। च॒। अ॒व॒स्व॒न्या॑येत्य॑व-स्व॒न्या॑य। च॒। 
नमः॑। स्रो॒त॒स्या॑य। च॒। द्वीप्या॑य। च॒॥~(१२)


नमः॑। ज्ये॒ष्ठाय॑। च॒। क॒नि॒ष्ठाय॑। च॒। 
नमः॑। पू॒र्व॒जायेति॑ पूर्व-जाय॑। च॒। अ॒प॒र॒जायेत्य॑पर-जाय॑। च॒। 
नमः॑। म॒ध्य॒माय॑। च॒। अ॒प॒ग॒ल्भायेत्य॑प-ग॒ल्भाय॑। च॒। 
नमः॑। ज॒घ॒न्या॑य। च॒। बुध्नि॑याय। च॒। 
नमः॑। सो॒भ्या॑य। च॒। प्र॒ति॒स॒र्या॑येति॑ प्रति-स॒र्या॑य। च॒। 
नमः॑। याम्या॑य। च॒। क्षेम्या॑य। च॒। 
नमः॑। उ॒र्व॒र्या॑य। च॒। खल्या॑य। च॒। 
नमः॑। श्लोक्या॑य। च॒। अ॒व॒सा॒न्या॑येत्य॑व-सा॒न्या॑य। च॒। 
नमः॑। वन्या॑य। च॒। कक्ष्या॑य। च॒। 
नमः॑। श्र॒वाय॑। च॒। प्र॒ति॒श्र॒वायेति॑ प्रति-श्र॒वाय॑। च॒।~(१३)


नमः॑। आ॒शुषे॑णा॒येत्या॒शु-से॒ना॒य॒। च॒। आ॒शु॒र॑था॒येत्या॒शु-र॒था॒य॒। च॒। 
नमः॑। शूरा॑य। च॒। अ॒व॒भि॒न्द॒त इत्य॑व-भि॒न्द॒ते। च॒। 
नमः॑। व॒र्मिणे᳚। च॒। व॒रू॒थिने᳚। च॒। 
नमः॑। बि॒ल्मिने᳚। च॒। क॒व॒चिने᳚। च॒। 
नमः॑। श्रु॒ताय॑। च॒। श्रु॒त॒से॒नायेति॑ श्रुत-से॒ना॒य॑। च॒॥~(१४)


नमः॑। दु॒न्दु॒भ्या॑य। च॒। आ॒ह॒न॒न्या॑येत्या᳚-ह॒न॒न्या॑य। च॒। 
नमः॑। धृ॒ष्णवे᳚। च॒। प्र॒मृ॒शायेति॑ प्र-मृ॒शाय॑। च॒। 
नमः॑। दू॒ताय॑। च॒। प्रहि॑ता॒येति॒ प्र-हि॒ता॒य॒। च॒। 
नमः॑। नि॒ष॒ङ्गिण॒ इति॑ नि-स॒ङ्गिने᳚। च॒। इ॒षु॒धि॒मत॒ इती॑षुधि-मते᳚। च॒। 
नमः॑। ती॒क्ष्णेष॑व॒ इति॑ ती॒क्ष्ण-इ॒ष॒वे॒। च॒। आ॒यु॒धिने᳚। च॒। 
नमः॑। स्वा॒यु॒धायेति॑ सु-आ॒यु॒धाय॑। च॒। सु॒धन्व॑न॒ इति॑ सु-धन्व॑ने। च॒। 
नमः॑। स्रुत्या॑य। च॒। पथ्या॑य। च॒। 
नमः॑। का॒ट्या॑य। च॒। नी॒प्या॑य। च॒। 
नमः॑। सूद्या॑य। च॒। स॒र॒स्या॑य। च॒। 
नमः॑। ना॒द्याय॑। च॒। वै॒श॒न्ताय॑। च॒।~(१५)


नमः॑। कूप्या॑य। च॒। अ॒व॒ट्या॑य। च॒। 
नमः॑। वर्ष्या॑य। च॒। अ॒व॒र्ष्याय॑। च॒। 
नमः॑। मे॒घ्या॑य। च॒। वि॒द्यु॒त्या॑येति॑ वि-द्यु॒त्या॑य। च॒। 
नमः॑। ई॒ध्रिया॑य। च॒। आ॒त॒प्या॑येत्या᳚-त॒प्या॑य। च॒। 
नमः॑। वात्या॑य। च॒। रेष्मि॑याय। च॒। 
नमः॑। वा॒स्त॒व्या॑य। च॒। वा॒स्तु॒पायेति॑ वास्तु-पाय॑। च॒॥~(१६)


नमः॑। सोमा॑य। च॒। रु॒द्राय॑। च॒। 
नमः॑। ता॒म्राय॑। च॒। अ॒रु॒णाय॑। च॒। 
नमः॑। श॒ङ्गाय॑। च॒। प॒शु॒पत॑य॒ इति॑ पशु-पत॑ये। च॒। 
नमः॑। उ॒ग्राय॑। च॒। भी॒माय॑। च॒। 
नमः॑। अ॒ग्रे॒व॒धायेत्य॑ग्रे-व॒धाय॑। च॒। दू॒रे॒व॒धायेति॑ दूरे-व॒धाय॑। च॒। 
नमः॑। ह॒न्त्रे। च॒। हनी॑यसे। च॒। 
नमः॑। वृ॒क्षेभ्यः॑। हरि॑केशेभ्य॒ इति॒ हरि॑-के॒शे॒भ्यः॒। 
नमः॑। ता॒राय॑। नमः॑। श॒म्भव॒ इति॑ शम्-भवे᳚। च॒। म॒यो॒भव॒ इति॑ मयः-भवे᳚। च॒। 
नमः॑। श॒ङ्क॒रायेति॑ शम्-क॒राय॑। च॒। म॒य॒स्क॒रायेति॑ मयः-क॒राय॑। च॒। 
नमः॑। शि॒वाय॑। च॒। शि॒वत॑रा॒येति॑ शि॒व-त॒रा॒य॒। च॒।~(१७)


नमः॑। तीर्थ्या॑य। च॒। कूल्या॑य। च॒। 
नमः॑। पा॒र्या॑य। च॒। अ॒वा॒र्या॑य। च॒। 
नमः॑। प्र॒तर॑णा॒येति॑ प्र-तर॑णाय। च॒। उ॒त्त॒र॑णा॒येत्यु॑त्-तर॑णाय। च॒। 
नमः॑। आ॒ता॒र्या॑येत्या᳚-ता॒र्या॑य। च॒। आ॒ला॒द्या॑येत्या᳚-ला॒द्या॑य। च॒। 
नमः॑। शष्प्या॑य। च॒। फेन्या॑य। च॒। 
नमः॑। सि॒क॒त्या॑य। च॒। प्र॒वा॒ह्या॑येति॑ प्र-वा॒ह्या॑य। च॒॥~(१८)


नमः॑। इ॒रि॒ण्या॑य। च॒। प्र॒प॒थ्या॑येति॑ प्र-प॒थ्या॑य। च॒। 
नमः॑। कि॒ꣳ॒शि॒लाय॑। च॒। क्षय॑णाय। च॒। 
नमः॑। क॒प॒र्दिने᳚। च॒। पु॒ल॒स्तये᳚। च॒। 
नमः॑। गोष्ठ्या॒येति॒ गो-स्थ्या॒य॒। च॒। गृह्या॑य। च॒। 
नमः॑। तल्प्या॑य। च॒। गेह्या॑य। च॒। 
नमः॑। का॒ट्या॑य। च॒। ग॒ह्व॒रे॒ष्ठायेति॑ गह्वरे-स्थाय॑। च॒। 
नमः॑। ह्र॒द॒य्या॑य। च॒। नि॒वे॒ष्प्या॑येति॑ नि-वे॒ष्प्या॑य। च॒। 
नमः॑। पा॒ꣳ॒स॒व्या॑य। च॒। र॒ज॒स्या॑य। च॒। 
नमः॑। शुष्क्या॑य। च॒। ह॒रि॒त्या॑य। च॒। 
नमः॑। लोप्या॑य। च॒। उ॒ल॒प्या॑य। च॒।~(१९)


नमः॑। ऊ॒र्व्या॑य। च॒। सू॒र्म्या॑य। च॒। 
नमः॑। प॒र्ण्या॑य। च॒। प॒र्ण॒श॒द्या॑येति॑ पर्ण-श॒द्या॑य। च॒। 
नमः॑। अ॒प॒गु॒रमा॑णा॒येत्य॑प-गु॒रमा॑णाय। च॒। अ॒भि॒घ्न॒त इत्यभि-घ्न॒ते। च॒। 
नमः॑। \mbox{आ॒\akhkhi{}द॒त} इत्या᳚-खि॒द॒ते। च॒। \mbox{प्र॒\akhkhi{}द॒त} इति॑ प्र-खि॒द॒ते। च॒। 
नमः॑। वः॒। कि॒रि॒केभ्यः॑। दे॒वाना᳚म्। हृद॑येभ्यः। नमः॑। 
वि॒क्षी॒ण॒केभ्य॒ इति॑ वि-क्षी॒ण॒केभ्यः॑। नमः॑। वि॒चि॒न्व॒त्केभ्य॒ इति॑ वि-चि॒न्व॒त्केभ्यः॑। नमः॑। 
आ॒नि॒र्॒‌ह॒तेभ्य॒ इत्या॑निः-ह॒तेभ्यः॑। नमः॑। आ॒मी॒व॒त्केभ्य॒ इत्या᳚-मी॒व॒त्केभ्यः॑॥~(२०)


द्रापे᳚। अन्ध॑सः। प॒ते॒। दरि॑द्रत्। नील॑लोहि॒तेति॒ नील॑-लो॒हि॒त॒॥ 
ए॒षाम्। पुरु॑षाणाम्। ए॒षाम्। प॒शू॒नाम्। मा। भेः। मा। अ॒रः॒। मो इति॑। ए॒षा॒म्। किम्। च॒न। आ॒म॒म॒त्॒॥ 
या। ते॒। रु॒द्र॒। शि॒वा। त॒नूः। शि॒वा। वि॒श्वाह॑भेष॒जीति॑ वि॒श्वाह॑-भे॒ष॒जी॒॥ 
शि॒वा। रु॒द्रस्य॑। भे॒ष॒जी। तया᳚। नः॒। मृ॒ड॒। जी॒वसे᳚॥ 
इ॒माम्। रु॒द्राय॑। त॒वसे᳚। क॒प॒र्दिने᳚। क्ष॒यद्वी॑रा॒येति॑ क्ष॒यत्-वी॒रा॒य॒। प्रेति॑। भ॒रा॒म॒हे॒। म॒तिम्॥ 
यथा᳚। नः॒। शम्। अस॑त्। द्वि॒पद॒ इति॑ द्वि-पदे᳚। चतु॑ष्पद॒ इति॒ चतुः॑-प॒दे॒। विश्वम्᳚। पु॒ष्टम्। ग्रामे᳚। अ॒स्मिन्।~(२१)


अना॑तुर॒मित्यना᳚-तु॒र॒म्॒॥ 
मृ॒डा। नः॒। रु॒द्र॒। उ॒त। नः॒। मयः॑। कृ॒धि॒। क्ष॒यद्वी॑रा॒येति॑ क्ष॒यत्-वी॒रा॒य॒। नम॑सा। वि॒धे॒म॒। ते॒॥ 
यत्। शम्। च॒। योः। च॒। मनुः॑। आ॒य॒ज इत्या᳚-य॒जे। पि॒ता। तत्। अ॒श्या॒म॒। तव॑। रु॒द्र॒। प्रणी॑ता॒विति॒ प्र-नी॒तौ॒॥ 
मा। नः॒। म॒हान्तम्᳚। उ॒त। मा। नः॒। अ॒र्भ॒कम्। मा। नः॒। उक्ष॑न्तम्। उ॒त। मा। नः॒। उ॒क्षि॒तम्॥ 
मा। नः॒। व॒धीः॒। पि॒तरम्᳚। मा। उ॒त। मा॒तरम्᳚। प्रि॒याः। मा। नः॒। त॒नुवः॑।~(२२)


रु॒द्र॒। री॒रि॒षः॒॥ 
मा। नः॒। तो॒के। तन॑ये। मा। नः॒। आयु॑षि। मा। नः॒। गोषु॑। मा। नः॒। अश्वे॑षु। री॒रि॒षः॒॥ 
वी॒रान्। मा। नः॒। रु॒द्र॒। भा॒मि॒तः। व॒धीः॒। ह॒विष्म॑न्तः। नम॑सा। वि॒धे॒म॒। ते॒॥ 
आ॒रात्। ते॒। गो॒घ्न इति॑ गो-घ्ने। उ॒त। पू॒रु॒ष॒घ्न इति॑ पूरुष-घ्ने। क्ष॒यद्वी॑रा॒येति॑ क्ष॒यत्-वी॒रा॒य॒। सु॒म्नम्। अ॒स्मे इति॑। ते॒। अ॒स्तु॒॥ 
रक्षा॑। च॒। नः॒। अधीति॑। च॒। दे॒व॒। ब्रू॒हि॒। अधा॑। च॒। नः॒। शर्म॑। य॒च्छ॒। द्वि॒बर्हा॒ इति॑ द्वि-बर्हाः᳚॥ 
स्तु॒हि।~(२३)


श्रु॒तम्। ग॒र्त॒सद॒मिति॑ गर्त-सदम्᳚। युवा॑नम्। मृ॒गम्। न। भी॒मम्। उ॒प॒ह॒त्नुम्। उ॒ग्रम्॥ 
मृ॒डा। ज॒रि॒त्रे। रु॒द्र॒। स्तवा॑नः। अ॒न्यम्। ते॒। अ॒स्मत्। नीति॑। व॒प॒न्तु॒। सेनाः᳚॥ 
परीति॑। नः॒। रु॒द्रस्य॑। हे॒तिः। वृ॒ण॒क्तु॒। परीति॑। त्वे॒षस्य॑। दु॒र्म॒तिरिति॑ दुः-म॒तिः। अ॒घा॒योरित्य॑घा-योः॥ 
अवेति॑। स्थि॒रा। म॒घव॑द्भ्य॒ इति॑ म॒घव॑त्-भ्यः॒। त॒नु॒ष्व॒। मीढ्वः॑। तो॒काय॑। तन॑याय। मृ॒ड॒य॒॥ 
मीढु॑ष्ट॒मेति॒ मीढुः॑-त॒म॒। शिव॑त॒मेति॒ शिव॑-त॒म॒। शि॒वः। नः॒। सु॒मना॒ इति॑ सु-मनाः᳚। भ॒व॒॥ 
प॒र॒मे। वृ॒क्षे। आयु॑धम्। नि॒धायेति॑ नि-धाय॑। कृत्तिम्᳚। वसा॑नः। एति॑। च॒र॒। पिना॑कम्।~(२४)


बिभ्र॑त्। एति॑। ग॒हि॒॥ 
विकि॑रि॒देति॒ वि-कि॒रि॒द॒। विलो॑हि॒तेति॒ वि-लो॒हि॒त॒। नमः॑। ते॒। अ॒स्तु॒। भ॒ग॒व॒ इति॑ भग-वः॒॥ 
याः। ते॒। स॒हस्रम्᳚। हे॒तयः॑। अ॒न्यम्। अ॒स्मत्। नीति॑। व॒प॒न्तु॒। ताः॥ 
स॒हस्रा॑णि। स॒ह॒स्र॒धेति॑ सहस्र-धा। बा॒हु॒वोः। तव॑। हे॒तयः॑॥ 
तासा᳚म्। ईशा॑नः। भ॒ग॒व॒ इति॑ भग-वः॒। प॒रा॒चीना᳚। मुखा᳚। कृ॒धि॒॥~(२५)


स॒हस्रा॑णि। स॒ह॒स्र॒श इति॑ सहस्र-शः। ये। रु॒द्राः। अधीति॑। भूम्या᳚म्॥ 
तेषा᳚म्। स॒ह॒स्र॒यो॒ज॒न इति॑ सहस्र-यो॒ज॒ने। अवेति॑। धन्वा॑नि। त॒न्म॒सि॒॥ 
अ॒स्मिन्। म॒ह॒ति। अ॒र्ण॒वे। अ॒न्तरि॑क्षे। भ॒वाः। अधि॑॥ 
नील॑ग्रीवा॒ इति॒ नील॑-ग्री॒वाः॒। शि॒ति॒कण्ठा॒ इति॑ शिति-कण्ठाः᳚। श॒र्वाः। अ॒धः। क्ष॒मा॒च॒राः॥ 
नील॑ग्रीवा॒ इति॒ नील॑-ग्री॒वाः॒। शि॒ति॒कण्ठा॒ इति॑ शिति-कण्ठाः᳚। दिवम्᳚। रु॒द्राः। उप॑श्रिता॒ इत्युप॑-श्रि॒ताः॒॥ 
ये। वृ॒क्षेषु॑। स॒स्पिञ्ज॑राः। नील॑ग्रीवा॒ इति॒ नील॑-ग्री॒वाः॒। विलो॑हिता॒ इति॒ वि-लो॒हि॒ताः॒॥ 
ये। भू॒ताना᳚म्। अधि॑पतय॒ इत्यधि॑-प॒त॒यः॒। वि॒शि॒खास॒ इति॑ वि-शि॒खासः॑। क॒प॒र्दिनः॑॥ 
ये। अन्ने॑षु। वि॒विध्य॒न्तीति॑ वि-विध्य॑न्ति। पात्रे॑षु। पिब॑तः। जनान्॥ 
ये। प॒थाम्। प॒थि॒रक्ष॑य॒ इति॑ पथि-रक्ष॑यः। ऐ॒ल॒बृ॒दाः। य॒व्युधः॑॥ 
ये। ती॒र्थानि॑।~(२६)


प्र॒चर॒न्तीति॑ प्र-चर॑न्ति। सृ॒काव॑न्त॒ इति॑ सृ॒का-व॒न्तः॒। नि॒ष॒ङ्गिण॒ इति॑ नि-स॒ङ्गिनः॑॥ 
ये। ए॒ताव॑न्तः। च॒। भूयाꣳ॑सः। च॒। दिशः॑। रु॒द्राः। वि॒त॒स्थि॒र इति॑ वि-त॒स्थि॒रे॥ 
तेषा᳚म्। स॒ह॒स्र॒यो॒ज॒न इति॑ सहस्र-यो॒ज॒ने। अवेति॑। धन्वा॑नि। त॒न्म॒सि॒॥ 
नमः॑। रु॒द्रेभ्यः॑। ये। पृ॒थि॒व्याम्। ये। अ॒न्तरि॑क्षे। ये। दि॒वि। येषा᳚म्। अन्नम्᳚। वातः॑। व॒र्॒‌षम्।
 इष॑वः। तेभ्यः॑। दश॑। प्राची᳚। दश॑। द॒क्षि॒णा। दश॑। प्र॒तीची᳚। दश॑। उदी॑चीः। दश॑। ऊ॒र्ध्वाः।
  तेभ्यः॑। नमः॑। ते। नः॒। मृ॒ड॒य॒न्तु॒। ते। यम्। द्वि॒ष्मः। यः। च॒। नः॒। द्वेष्टि॑। तम्। वः॒। जम्भे᳚। द॒धा॒मि॒॥~(२७)

त्र्य॑म्बक॒मिति॒ त्रि-अ॒म्ब॒क॒म्॒। य॒जा॒म॒हे॒। सु॒ग॒न्धिमिति॑ सु-ग॒न्धिम्। पु॒ष्टि॒वर्ध॑न॒मिति॑ पुष्टि-वर्ध॑नम्॥ 
उ॒र्वा॒रु॒कम्। इ॒व॒। बन्ध॑नात्। मृ॒त्योः। मु॒क्षी॒य॒। मा। अ॒मृता᳚त्॥ 
यः। रु॒द्रः। अ॒ग्नौ। यः। अ॒फ्स्वित्य॑प्-सु। यः। ओष॑धीषु। यः। रु॒द्रः। विश्वा᳚। भुव॑ना। आ॒वि॒वेशेत्या᳚-वि॒वेश॑। तस्मै᳚। रु॒द्राय॑। नमः॑। अ॒स्तु॒॥

\centerline{॥ॐ शान्तिः॒ शान्तिः॒ शान्तिः॑॥}


