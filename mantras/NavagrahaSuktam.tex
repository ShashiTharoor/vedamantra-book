% !TeX program = XeLaTeX
% !TeX root = ../vedamantrabook.tex

\chapt{नवग्रहसूक्तम्}

आ स॒त्येन॒ रज॑सा॒ वर्त॑मानो निवे॒शय॑न्न॒मृतं॒ मर्त्यं॑ च। हि॒र॒ण्यये॑न सवि॒ता रथे॒नाऽदे॒वो या॑ति॒ भुव॑ना वि॒पश्यन्। अ॒ग्निं दू॒तं वृ॑णीमहे॒ होता॑रं वि॒श्ववे॑दसम्। अ॒स्य य॒ज्ञस्य॑ सु॒क्रतुम्᳚॥ येषा॒मीशे॑ पशु॒पतिः॑ पशू॒नां चतु॑ष्पदामु॒त च॑ द्वि॒पदा᳚म्। निष्क्री॑तो॒ऽयं य॒ज्ञियं॑ भा॒गमे॑तु रा॒यस्पोषा॒ यज॑मानस्य सन्तु॥ \\
अधिदेवता प्रत्यधिदेवता सहिताय आदित्याय॒ नम॥१॥ 

अ॒ग्निर्मू॒र्धा दि॒वः क॒कुत्पतिः॑ पृथि॒व्या अ॒यम्। अ॒पाꣳ रेताꣳ॑सि जिन्वति। स्यो॒ना पृ॑थिवि॒ भवा॑ऽनृक्ष॒रा नि॒वेश॑नी। यच्छा॑नः॒ शर्म॑ स॒प्रथाः᳚। क्षेत्र॑स्य॒ पति॑ना व॒यꣳ हि॒ते ने॑व जयामसि। गामश्वं॑ पोषयि॒त्न्वा स नो॑ मृडाती॒दृशे᳚॥ \\
अधिदेवता प्रत्यधिदेवता सहिताय अङ्गारकाय॒ नम॥२॥ 

प्र वः॑ शु॒क्राय॑ भा॒नवे॑ भरध्वꣳ ह॒व्यं म॒तिं चा॒ग्नये॒ सुपू॑तम्॥ यो दैव्या॑नि॒ मानु॑षा ज॒नूꣴष्य॒न्तर्विश्वा॑नि वि॒द्मना॒ जिगा॑ति॥ इ॒न्द्रा॒णीमा॒सु नारि॑षु सु॒पत्नी॑म॒हम॑श्रवम्। न ह्य॑स्या अप॒रञ्च॒न ज॒रसा॒ मर॑ते॒ पतिः॑॥ इन्द्रं॑ वो वि॒श्वत॒स्परि॒ हवा॑महे॒ जने᳚भ्यः। अ॒स्माक॑मस्तु॒ केव॑लः॥ \\
अधिदेवता प्रत्यधिदेवता सहिताय शुक्राय॒ नम॥३॥ 

आप्या॑यस्व॒ समे॑तु ते वि॒श्वतः॑ सोम॒ वृष्णि॑यम्। भवा॒ वाज॑स्य सङ्ग॒थे॥ अ॒फ्सु मे॒ सोमो॑ अब्रवीद॒न्तर्विश्वा॑नि भेष॒जा। अ॒ग्निं च॑ वि॒श्वश॑म्भुव॒माप॑श्च वि॒श्वभे॑षजीः। गौ॒री मि॑माय सलि॒लानि॒ तक्ष॒ती। एक॑पदी द्वि॒पदी॒ सा चतु॑ष्पदी। अ॒ष्टाप॑दी॒ नव॑पदी बभू॒वुषी᳚। स॒हस्रा᳚क्षरा पर॒मे व्यो॑मन्। \\
अधिदेवता प्रत्यधिदेवता सहिताय सोमाय॒ नम॥४॥ 

उद्बु॑ध्यस्वाग्ने॒ प्रति॑जागृह्येनमिष्टापू॒र्ते सꣳसृ॑जेथाम॒यं च॑। पुनः॑ कृ॒ण्वꣴस्त्वा॑ पि॒तरं॒ युवा॑नम॒न्वाताꣳ॑सी॒त्वयि॒ तन्तु॑मे॒तम्॥ इ॒दं विष्णु॒र्विच॑क्रमे त्रे॒धा निद॑धे प॒दम्। समू॑ढमस्यपाꣳ सु॒रे॥ विष्णो॑ र॒राट॑मसि॒ विष्णोः᳚ पृ॒ष्ठम॑सि॒ विष्णोः॒ श्नप्त्रे᳚स्थो॒ विष्णोः॒ स्यूर॑सि॒ विष्णो᳚र्ध्रु॒वम॑सि वैष्ण॒वम॑सि॒ विष्ण॑वे त्वा। \\
अधिदेवता प्रत्यधिदेवता सहिताय बुधाय॒ नम॥५॥ 

बृह॑स्पते॒ अति॒यद॒र्यो अर्हा᳚द्वि॒मद्वि॒भाति॒ क्रतु॑म॒ज्जने॑षु। यद्दी॒दय॒च्छव॑सर्त\-प्रजात॒ तद॒स्मासु॒ द्रवि॑णं धेहि चि॒त्रम्॥ इन्द्र॑मरुत्व इ॒ह पा॑हि॒ सोमं॒ यथा॑ शार्या॒ते अपि॑बः सु॒तस्य॑। तव॒ प्रणी॑ती॒ तव॑ शूर॒शर्म॒न्नावि॑वासन्ति क॒वयः॑ सुय॒ज्ञाः॥ ब्रह्म॑जज्ञा॒नं प्र॑थ॒मं पु॒रस्ता॒द्विसी॑म॒तः सु॒रुचो॑ वे॒न आ॑वः। सबु॒ध्निया॑ उप॒मा अ॑स्य वि॒ष्ठाः स॒तश्च॒ योनि॒मस॑तश्च॒ विवः॑॥\\
अधिदेवता प्रत्यधिदेवता सहिताय बृहस्पतये॒ नम॥६॥ 

शं नो॑ दे॒वीर॒भिष्ट॑य॒ आपो॑ भवन्तु पी॒तये᳚। शंयोर॒भिस्र॑वन्तु नः॥ प्रजा॑पते॒ न त्वदे॒तान्य॒न्यो विश्वा॑ जा॒तानि॒ परि॒ता ब॑भूव। यत्का॑मास्ते जुहु॒मस्तन्नो॑ अस्तु व॒यꣴ स्या॑म॒ पत॑यो रयी॒णाम्। इ॒मं य॑मप्रस्त॒रमाहि सीदाऽङ्गि॑रोभिः पि॒तृभिः॑ संविदा॒नः। आत्वा॒ मन्त्राः᳚ कविश॒स्ता व॑हन्त्वे॒ना रा॑जन् ह॒विषा॑ मादयस्व॥ \\
अधिदेवता प्रत्यधिदेवता सहिताय शनैश्चराय॒ नम॥७॥ 

कया॑ नश्चि॒त्र आभु॑वदू॒ती स॒दावृ॑धः॒ सखा᳚। कया॒ शचि॑ष्ठया वृ॒ता। आऽयङ्गौः पृश्नि॑रक्रमी॒दस॑नन्मा॒तरं॒ पुनः॑। पि॒तरं॑ च प्र॒यन्थ्सुवः॑। यत्ते॑ दे॒वी निर्ऋ॑तिराब॒बन्ध॒ दाम॑ ग्री॒वास्व॑विच॒र्त्यम्। इ॒दं  ते॒ तद्विष्या॒म्यायु॑षो॒ न मध्या॒दथा॑जी॒वः पि॒तुम॑द्धि॒ प्रमु॑क्तः॥ \\
अधिदेवता प्रत्यधिदेवता सहिताय राहवे॒ नम॥८॥ 

के॒तुं कृ॒ण्वन्न॑के॒तवे॒ पेशो॑ मर्या अपे॒शसे᳚। समु॒षद्भि॑रजायथाः॥ ब्र॒ह्मा दे॒वानां᳚ पद॒वीः क॑वी॒नामृषि॒र्विप्रा॑णां महि॒षो मृ॒गाणा᳚म्। श्ये॒नो गृध्रा॑णा॒ꣴ॒ स्वधि॑ति॒र्वना॑ना॒ꣳ॒ सोमः॑ प॒वित्र॒मत्ये॑ति॒ रेभन्। (ऋक्) सचि॑त्र चि॒त्रं चि॒तयन्᳚ तम॒स्मे चित्र॑क्षत्र चि॒त्रत॑मं वयो॒धाम्। च॒न्द्रं र॒यिं पु॑रु॒वीरं᳚ बृ॒हन्तं॒ चन्द्र॑च॒न्द्राभि॑र्गृण॒ते यु॑वस्व॥ \\
अधिदेवता प्रत्यधिदेवता सहिताय केतवे॒ नम॥९॥ 

\centerline{॥ॐ आदित्यादि नवग्रहदेव॑ताभ्यो॒ नमो॒ नमः॑॥ }

\centerline{॥ॐ शान्तिः॒ शान्तिः॒ शान्तिः॑॥}