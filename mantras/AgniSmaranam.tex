% !TeX program = XeLaTeX
% !TeX root = ../vedamantrabook.tex
\chapt{अग्नि-स्मरणम्}

च॒त्वारि॒ शृङ्गा॒ त्रयो॑ अस्य॒ पादा॒ द्वे शी॒र्\mbox{}षे स॒प्त हस्ता॑सो अ॒स्य। त्रिधा॑ ब॒द्धो वृ॑ष॒भो रो॑रवीति म॒हो दे॒वो मर्त्या॒ आवि॑वेश॥

ए॒ष हि दे॒वः प्र॒दिशोऽनु॒ सर्वा॒ पूर्वो॑ हि जा॒तः स उ॒ गर्भे॑ अ॒न्तः। स वि॒जाय॑मानः स जनि॒ष्यमा॑णः प्र॒त्यङ्मुखास्तिष्ठति वि॒श्वतो॑मुखः॥ 

प्राङ्मुखो देव हे अग्ने मम अभिमुखो भव॥

इन्द्राय नमः। अग्नये नमः। यमाय नमः। निर्‌ऋतये नमः। 
वरुणाय नमः। वायवे नमः। सोमाय नमः। ईशानाय नमः। 

अग्नये नमः। 

आत्मने नमः। 

सर्वेभ्यो ब्राह्मणेभ्यो नमः॥

अनाज्ञातं॒ यदाज्ञा॑तं य॒ज्ञस्य॑ क्रि॒यते॒ मिथु॑।
अग्ने॒ तद॑स्य कल्पय॒ त्व हि वेत्थ॑ यथात॒थम्।

पुरु॑षसम्मितो य॒ज्ञो य॒ज्ञः पुरु॑षसम्मितः।
अग्ने॒ तद॑स्य कल्पय॒ त्व हि वेत्थ॑ यथात॒थम्।

यत्पा॑क॒त्रा मन॑सा दी॒नद॑क्षा॒ न य॒ज्ञस्य॑ म॒न्वते॒ मर्ता॑सः।
अ॒ग्निष्टद्धोता॑ क्रतु॒विद्वि॑जा॒नन् यजि॑ष्ठो दे॒वा ऋ॑तु॒शो य॑जाति॥

स॒प्त ते॑ अग्ने स॒मिध॑ स॒प्त जि॒ह्वाः स॒प्तर्‌ष॑यः स॒प्त धाम॑ प्रि॒याणि॑। स॒प्त होत्रा सप्त॒धा त्वा॑ यजन्ति स॒प्त योनी॒रापृ॑णस्वा घृ॒तेन॑॥


अग्ने॒ नय॑ सु॒पथा॑ रा॒ये अ॒स्मान् विश्वा॑नि देव व॒युना॑नि वि॒द्वान्। यु॒यो॒ध्य॑स्मज्जु॑हुरा॒णमेनो॒ भूयि॑ष्ठां ते॒ नम॑ उक्तिं विधेम॥ 

प्रव॑ शु॒क्राय॑ भा॒नवे॑ भरध्व ह॒व्यं म॒तिं चा॒ग्नये॒ सुपू॑तम्॥ यो दैव्या॑नि॒ मानु॑षा ज॒नूष्य॒न्तर्विश्वा॑नि वि॒द्म ना॒ जिगा॑ति॥

अग्ने॒ त्वं पा॑रया॒ नव्यो॑ अ॒स्मान्थ्स्व॒स्तिभि॒रति॑ दु॒र्गाणि॒ विश्वा।
पूश्च॑ पृ॒थ्वी ब॑हु॒ला न॑ उ॒र्वी भवा॑ तो॒काय॒ तन॑याय॒ शं योः॥

\twolineshloka*
{नमस्ते गार्हपत्याय नमस्ते दक्षिणाग्नये}
{नम आहवनीयाय महावेद्यै नमो नमः}

\twolineshloka*
{काण्डद्वयोपपाद्याय कर्मब्रह्मस्वरूपिणे}
{स्वर्गापवर्गरूपाय यज्ञेशाय नमो नमः}

\twolineshloka*
{यज्ञेशाच्युत गोविन्द माधवानन्त केशव}
{कृष्ण विष्णो हृषीकेश वासुदेव नमोऽस्तु ते}

स॒कृत्ते॑ अग्ने॒ नम॑। द्विस्ते॒ नम॑। त्रिस्ते॒ नम॑। च॒तुस्ते॒ नम॑। प॒ञ्च॒कृत्व॑स्ते॒ नम॑। द॒श॒कृत्व॑स्ते॒ नम॑। श॒त॒कृत्व॑स्ते॒ नम॑। आ॒स॒ह॒स्र॒कृत्व॑स्ते॒ नम॑। अ॒प॒रि॒मि॒त॒कृत्व॑स्ते॒ नम॑। नम॑स्ते अस्तु॒ मा मा॑ हिसीः॥

अग्नये नमः।

मन्त्रहीनं क्रियाहीनं भक्तिहीनं हुताशन।
यद्धुतं तु मया देव परिपूर्णं तदस्तु ते॥

प्रायश्चित्तान्यशेषाणि तपःकर्मात्मकानि वै।
यानि तेषामशेषाणां कृष्णानुस्मरणं परम्॥
कृष्ण कृष्ण कृष्ण॥

बृ॒हत्साम॑ क्षत्र॒भृद्वृ॒द्ध वृ॑ष्णियं त्रि॒ष्टुभौजः॑ शुभि॒तमु॒ग्रवी॑रम्।
इन्द्र॒स्तोमे॑न पञ्चद॒शेन॒ मध्य॑मि॒दं वाते॑न॒ सग॑रेण रक्ष॥
