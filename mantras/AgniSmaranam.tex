% !TeX program = XeLaTeX
% !TeX root = ../vedamantrabook.tex
\chapt{अग्नि-स्मरणम्}

च॒त्वारि॒ शृङ्गा॒ त्रयो॑ अस्य॒ पादा॒ द्वे शी॒र्\mbox{}षे स॒प्त हस्ता॑सो अ॒स्य। त्रिधा॑ ब॒द्धो वृ॑ष॒भो रो॑रवीति म॒हो दे॒वो मर्त्या॒ आवि॑वेश॥

ए॒ष हि दे॒वः प्र॒दिशोऽनु॒ सर्वा॒ पूर्वो॑ हि जा॒तः स उ॒ गर्भे॑ अ॒न्तः। स वि॒जाय॑मानः स जनि॒ष्यमा॑णः प्र॒त्यङ्मुखास्तिष्ठति वि॒श्वतो॑मुखः॥ 

प्राङ्मुखो देव हे अग्ने मम अभिमुखो भव॥

इन्द्राय नमः। अग्नये नमः। यमाय नमः। निर्‌ऋतये नमः। 
वरुणाय नमः। वायवे नमः। सोमाय नमः। ईशानाय नमः। 

अग्नये नमः। 

% अग्निश्च जात॑वेदा॒श्च। सहोजा अ॑जिरा॒प्रभुः। वैश्वानरो न॑र्यापा॒श्च। 
% प॒ङ्क्तिरा॑धाश्च॒ सप्त॑मः। विसर्पेवाऽष्ट॑मोऽग्नी॒नाम्‌। 

आत्मने नमः। 

सर्वेभ्यो ब्राह्मणेभ्यो नमः॥


होष्यामि। [समिधं हुत्वा]

जुहुधि। [पत्न्या वक्तव्यम्]

\dnsub{होमः}

[प्रातः] सूर्या॑य॒ स्वाहा। सूर्या॑य॒ इदं॑ न मम।

[सायम्] अ॒ग्नये॒ स्वाहा। अ॒ग्न॑य इदं॑ न मम।

अग्नये᳚ स्विष्ट॒कृते॒ स्वाहा᳚। अग्नये᳚ स्विष्ट॒कृते॒ इदं॑ न मम।


यत्र॒ वेत्थ॑ वनस्पते दे॒वानां॒ गुह्या॒ नामा॑नि । तत्र॑ ह॒व्यानि॑ गामय॥ [समिधं हुत्वा]
प्रजा॑पतय इदं न मम।


अना᳚ज्ञातं॒ यदाज्ञा॑तम्। य॒ज्ञस्य॑ क्रि॒यते॒ मिथु॑।
अग्ने॒ तद॑स्य कल्पय। त्व हि वेत्थ॑ यथात॒थम्।

पुरु॑षसम्मितो य॒ज्ञः। य॒ज्ञः पुरु॑षसम्मितः।
अग्ने॒ तद॑स्य कल्पय। त्व हि वेत्थ॑ यथात॒थम्।

यत्पा॑क॒त्रा मन॑सा दी॒नद॑क्षा॒ न। य॒ज्ञस्य॑ म॒न्वते॒ मर्ता॑सः।
अ॒ग्निष्टद्धोता᳚ क्रतु॒विद्वि॑जा॒नन्। यजि॑ष्ठो दे॒वा ऋ॑तु॒शो य॑जाति॥

कृष्ण कृष्ण कृष्ण॥ इ॒दं विष्णु॒र्विच॑क्रमे त्रे॒धा निद॑धे प॒दम्। समू॑ढमस्यपा सु॒रे॥



\dnsub{उपस्थानम्}

अग्ने॒ नय॑ सु॒पथा॑ रा॒ये अ॒स्मान् विश्वा॑नि देव व॒युना॑नि वि॒द्वान्। यु॒यो॒ध्य॑स्मज्जु॑हुरा॒णमेनो॒ भूयि॑ष्ठां ते॒ नम॑ उक्तिं विधेम॥ 

प्रव॑ शु॒क्राय॑ भा॒नवे॑ भरध्व ह॒व्यं म॒तिं चा॒ग्नये॒ सुपू॑तम्॥ यो दैव्या॑नि॒ मानु॑षा ज॒नूष्य॒न्तर्विश्वा॑नि वि॒द्म ना॒ जिगा॑ति॥

अच्छा॒ गिरो॑ म॒तयो॑ देव॒यन्तीः᳚। अ॒ग्निं य॑न्ति॒ द्रवि॑णं॒ भिक्ष॑माणाः।
सु॒स॒न्दृशꣳ॑ सु॒प्रती॑क॒ स्वञ्चम्᳚। ह॒व्य॒वाह॑मर॒तिं मानु॑षाणाम्॥

अग्ने॒ त्वम॒स्मद्यु॑यो॒ध्यमी॑वाः। अन॑ग्नित्रा अ॒भ्यम॑न्त कृ॒ष्टीः।
पुन॑र॒स्मभ्य सुवि॒ताय॑ देव। क्षां विश्वे॑भिर॒जरे॑भिर्यजत्र॥ 

अग्ने॒ त्वं पा॑रया॒ नव्यो॑ अ॒स्मान्थ्स्व॒स्तिभि॒रति॑ दु॒र्गाणि॒ विश्वा।
पूश्च॑ पृ॒थ्वी ब॑हु॒ला न॑ उ॒र्वी भवा॑ तो॒काय॒ तन॑याय॒ शं योः॥

प्र का॑रवो मन॒ना व॒च्यमा॑नाः। दे॒व॒द्रीचीं᳚ नयत देव॒यन्तः॑।
द॒क्षि॒णा॒वाड्वा॒जिनी॒ प्राच्ये॑ति। ह॒विर्भर॑न्त्य॒ग्नये॑ घृ॒ताची᳚॥

स॒कृत्ते॑ अग्ने॒ नम॑। द्विस्ते॒ नम॑। त्रिस्ते॒ नम॑। च॒तुस्ते॒ नम॑। प॒ञ्च॒कृत्व॑स्ते॒ नम॑। द॒श॒कृत्व॑स्ते॒ नम॑। श॒त॒कृत्व॑स्ते॒ नम॑। आ॒स॒ह॒स्र॒कृत्व॑स्ते॒ नम॑। अ॒प॒रि॒मि॒त॒कृत्व॑स्ते॒ नम॑। नम॑स्ते अस्तु॒ मा मा॑ हिसीः॥

\twolineshloka*
{नमस्ते गार्हपत्याय नमस्ते दक्षिणाग्नये}
{नम आहवनीयाय महावेद्यै नमो नमः}

\twolineshloka*
{काण्डद्वयोपपाद्याय कर्मब्रह्मस्वरूपिणे}
{स्वर्गापवर्गरूपाय यज्ञेशाय नमो नमः}

\twolineshloka*
{यज्ञेशाच्युत गोविन्द माधवानन्त केशव}
{कृष्ण विष्णो हृषीकेश वासुदेव नमोऽस्तु ते}

वै॒श्वा॒न॒राय॑ वि॒द्महे॑ लाली॒लाय॑ धीमहि। 
तन्नो॑ अग्निः प्रचो॒दयात्। 

अग्नये नमः। मन्त्रहीनं क्रियाहीनं भक्तिहीनं हुताशन।
यद्धुतं तु मया देव परिपूर्णं तदस्तु ते॥

प्रायश्चित्तान्यशेषाणि तपःकर्मात्मकानि वै।
यानि तेषामशेषाणां कृष्णानुस्मरणं परम्॥ कृष्ण (१२)॥

अभिवादये+नमस्कारः॥

\dnsub{रक्षा}
बृ॒हत्साम॑ क्षत्र॒भृद्वृ॒द्ध वृ॑ष्णियं त्रि॒ष्टुभौजः॑ शुभि॒तमु॒ग्रवी॑रम्।
इन्द्र॒स्तोमे॑न पञ्चद॒शेन॒ मध्य॑मि॒दं वाते॑न॒ सग॑रेण रक्ष॥

दशास्यां पु॒त्राना धे॑हि॒ पति॑मेकाद॒शं कृ॑धि॥
