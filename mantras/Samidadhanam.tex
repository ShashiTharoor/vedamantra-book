% !TeX program = XeLaTeX
% !TeX root = ../vedamantrabook.tex
%ꣳꣳ॑ꣳ॒ꣴꣴ॒ꣴ॑
\setmainfont[Script=Devanagari,Mapping=tex-text,Mapping=devanagarinumerals,AutoFakeBold=2.0]{Siddhanta}
\title{\Huge यजुर्वेद-समिदाधानम्}
\date{}
\maketitle
% \mbox{}\\\thispagestyle{empty}
% \clearpage
\tableofcontents 

\clearpage
\chapt{समिदाधानम्}

\sect{सङ्कल्पः}
\textbf{आचमनम्।}

\twolineshloka*
{शुक्लाम्बरधरं विष्णुं शशिवर्णं चतुर्भुजम्}
{प्रसन्नवदनं ध्यायेत् सर्वविघ्नोपशान्तये}
 
\textbf{प्राणायामः।} 

ममोपात्त समस्त दुरितक्षयद्वारा श्री परमेश्वर प्रीत्यर्थं प्रातः (सायं) समिदाधानं करिष्ये।

(लौकिकाग्निं प्रतिष्ठाप्य। अग्निमिध्वा। प्रज्वाल्य।)

\sect{परिषेचनम्}
परि॑ त्वाऽग्ने॒ परि॑मृजा॒म्यायु॑षा च॒ बले॑न च सु॒प्र॒जाः प्र॒जया॑ भूयासꣳ सु॒वीरो॑ वी॒रैः सु॒वर्चा॒ वर्च॑सा सु॒पोषः॒ पोषैः᳚ सु॒गृहो॑ गृ॒हैः सु॒पतिः॒ पत्या॑ सुमे॒धा मे॒धया॑ सु॒ब्रह्मा ब्र॑ह्मचा॒रिभिः॑ ।

(तूष्णीं परिषिच्य) [देव॑ सवितः॒ प्रसु॑व।]

\sect{होमः}
अ॒ग्नये॑ स॒मिध॒माहा॑र्\mbox{}षं बृह॒ते जा॒तवे॑दसे।\\
यथा॒ त्वम॑ग्ने स॒मिधा॑ समि॒द्ध्यस॑ ए॒वं मामायु॑षा॒ वर्च॑सा स॒न्या मे॒धया प्र॒जया॑ प॒शुभि॑र्ब्रह्मवर्च॒सेना॒न्नाद्ये॑न॒ समे॑धय॒ स्वाहा॥१॥

एधोऽस्येधिषी॒महि॒ स्वाहा॥२॥

स॒मिद॑सि समेधिषी॒महि॒ स्वाहा॥३॥

तेजो॑सि॒ तेजो॒ मयि॑ धेहि॒ स्वाहा॥४॥

अपो॑ अ॒द्यान्व॑चारिष॒ꣳ॒ रसे॑न॒ सम॑सृक्ष्महि। पय॑स्वाꣳ अग्न॒ आग॑मं॒ तं मा॒ सꣳसृ॑ज॒ वर्च॑सा॒ स्वाहा॥५॥

सं माऽग्ने॒ वर्च॑सा सृज प्र॒जया॑ च॒ धने॑न च॒ स्वाहा॥६॥

वि॒द्युन्मे॑ अस्य दे॒वा इन्द्रो॑ वि॒द्याथ्स॒हर्\mbox{}षि॑भिः॒ स्वाहा॥७॥

अ॒ग्नये॑ बृह॒ते नाका॑य॒ स्वाहा॥८॥

द्यावा॑पृथि॒वीभ्या॒ꣴ॒ स्वाहा॥९॥

ए॒षा ते॑ अग्ने स॒मित्तया॒ वर्ध॑स्व॒ चाप्या॑यस्व च॒ तया॒ऽहं वर्ध॑मानो भूयासमा॒प्याय॑मानश्च॒ स्वाहा॥१०॥

यो माऽग्ने भा॒गिनꣳ॑ स॒न्तमथा॑भा॒गञ्चिकी॑र्\mbox{}षति। अभा॒गम॑ग्ने॒ तं कु॑रु॒ माम॑ग्ने भा॒गिनं॑ कुरु॒ स्वाहा॥११॥

स॒मिध॑मा॒धायाग्ने॒ सर्व॑व्रतो भूयास॒ꣴ॒ स्वाहा॥१२॥

(परिषिच्य) [देव॑ सवितः॒ प्रासा॑वीः।]

स्वाहा॥१३॥


\sect{उपस्थानम्}
(अग्नेः उपस्थानं करिष्ये।)

यत्ते॑ अग्ने॒ तेज॒स्तेना॒हं ते॑ज॒स्वी भू॑यासम्।\\
यत्ते॑ अग्ने॒ वर्च॒स्तेना॒हं व॑र्च॒स्वी भू॑यासम्।\\
यत्ते॑ अग्ने॒ हर॒स्तेना॒हं ह॑र॒स्वी भूयासम्।

मयि॑ मे॒धां मयि॑ प्र॒जां मय्य॒ग्निस्तेजो॑ दधातु॒।\\
मयि॑ मे॒धां मयि॑ प्र॒जां मयीन्द्र॑ इन्द्रि॒यं द॑धातु।\\
मयि॑ मे॒धां मयि॑ प्र॒जां मयि॒ सूर्यो॒ भ्राजो॑ दधातु॥
 
अग्नये नमः। 

\twolineshloka*
{मन्त्रहीनं क्रियाहीनं भक्तिहीनं हुताशन}
{यद्धुतं तु मया देव परिपूर्णं तदस्तु ते}

\twolineshloka*
{प्रायश्चित्तान्यशेषाणि तपः कर्मात्मकानि वै}
{यानि तेषामशेषाणां कृष्णानुस्मरणं परम्}

कृष्ण कृष्ण कृष्ण। कृष्ण कृष्ण कृष्ण। कृष्ण कृष्ण कृष्ण। कृष्ण कृष्ण कृष्ण॥

अभिवादये ( ) (त्रयार्षेय) प्रवरान्वित ( ) गोत्रः\\
(आपस्तम्ब) सूत्रः यजुःशाखा अध्यायी\\
() शर्मा नामाहम् अस्मि भोः॥

\textbf{नमस्कारः।}


\sect{भस्मधारणम्}

(होमभस्म सङ्गृह्य। वामकरतले निधाय। अद्भिः सेचयित्वा। अनामिकया पेषयित्वा।)

 मा न॑स्तो॒के तन॑ये॒ मा न॒ आयु॑षि॒ मा नो॒ गोषु॒ मा नो॒ अश्वे॑षु रीरिषः। वी॒रान्मा नो॑ रुद्र भामि॒तो व॑धीर्‌ह॒विष्म॑न्तो॒ नम॑सा विधेम ते॥
 
 मे॒धा॒वी भू॑यासम्। (forehead)
 
 ते॒ज॒स्वी भू॑यासम्। (heart)
 
 व॒र्च॒स्वी भू॑यासम्। (right arm)
 
 ब्र॒ह्म॒वर्चसी भू॑यासम्। (left arm)
 
 आ॒यु॒ष्मान् भू॑यासम्। (neck)
 
 अ॒न्ना॒दो भू॑यासम्। (back of neck)
 
 स्व॒स्ति भू॑यासम्। (top of head)
 
 
 \twolineshloka*
{श्रद्धां मेधां यशः प्रज्ञां विद्यां बुद्धिं श्रियं बलम्}
{आयुष्यं तेज आरोग्यं देहि मे हव्यवाहन}
 
श्रियं देहि मे हव्यवाहन ॐ नम इति।

\sect{समर्पणम्}
\fourlineindentedshloka*
{कायेन वाचा मनसेन्द्रियैर्वा}
{बुद्‌ध्याऽऽत्मना वा प्रकृतेः स्वभावात्}
{करोमि यद्यत् सकलं परस्मै}
{नारायणायेति समर्पयामि}

\textbf{आचमनम्।}


 