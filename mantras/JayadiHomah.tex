% !TeX program = XeLaTeX
% !TeX root = ../vedamantrabook.tex
\chapt{जयादि होमः}

एतत्कर्मसमृद्ध्यर्थं जयादिहोमं करिष्ये॥

१. चि॒त्तं च॒ स्वाहा। चित्तायेदम्।\\
२. चित्ति॑श्च॒ स्वाहा। चित्त्या इदम्।\\
३. आकू॑तं च॒ स्वाहा। आकूतायेदम्।\\
४. आकू॑तिश्च॒ स्वाहा। आकूत्या इदम्।\\
५. विज्ञा॑तं च॒ स्वाहा। विज्ञातायेदम्।\\
६. वि॒ज्ञानं॑ च॒ स्वाहा। विज्ञानायेदं न मम।\\
७. मन॑श्च॒ स्वाहा। मनस इदम्।\\
८. शक्व॑रीश्च॒ स्वाहा। शक्वरीभ्य इदम्।\\
९. दर्श॑श्च॒ स्वाहा। दर्शायेदम्।\\
१०. पू॒र्णमा॑सश्च॒ स्वाहा। पूर्णमासायेदम्।\\
११. बृ॒हच्च॒ स्वाहा। बृहत इदम्।\\
१२. र॒थ॒न्त॒रं च॒ स्वाहा। रथन्तरायेदम्।\\
१३. प्र॒जाप॑ति॒र्जया॒न्द्रिया॑य॒वृष्णे॒ प्राय॑च्छदु॒ग्रः पृ॑त॒नाज्ये॑षु॒ तस्मै॒ विशः॒ सम॑नमन्त॒ सर्वाः॒ सः उ॒ग्रः स हि॒ हव्यो॑ ब॒भूब॒ स्वाहा। प्रजापतय इदम्।\\

\dnsub{अभ्यातानाः}
१. अ॒ग्निर्भू॒ताना॒मधि॑पतिः॒ स मा॑ऽवत्व॒स्मिन् ब्रह्म॑न्न॒स्मिन् क्ष॒त्रेऽस्यामा॒शिष्य॒स्यां पु॑रो॒धाया॑म॒स्मिन् कर्म॑न्न॒स्यां दे॒वहूत्या॒ꣴ स्वाहा।\\
अग्नय इदम्।

२. इन्द्रो ज्ये॒ष्ठाना॒मधि॑पतिः॒ स मा॑ऽवत्व॒स्मिन् ब्रह्म॑न्न॒स्मिन् क्ष॒त्रेऽस्यामा॒शिष्य॒स्यां पु॑रो॒धाया॑म॒स्मिन् कर्म॑न्न॒स्यां दे॒वहूत्या॒ꣴ स्वाहा।\\
इन्द्रायेदम्।

३. य॒मः पृ॑थि॒व्या अधि॑पतिः॒ स मा॑ऽवत्व॒स्मिन् ब्रह्म॑न्न॒स्मिन् क्ष॒त्रेऽस्यामा॒शिष्य॒स्यां पु॑रो॒धाया॑म॒स्मिन् कर्म॑न्न॒स्यां दे॒वहूत्या॒ꣴ स्वाहा।\\
यमायेदम्।

४. वा॒युर॒न्तरि॑क्ष॒स्याधि॑पतिः॒ स मा॑ऽवत्व॒स्मिन् ब्रह्म॑न्न॒स्मिन् क्ष॒त्रेऽस्यामा॒शिष्य॒स्यां पु॑रो॒धाया॑म॒स्मिन् कर्म॑न्न॒स्यां दे॒वहूत्या॒ꣴ स्वाहा।\\
वायव इदम्।

५. सूर्यो॑ दि॒वोऽधि॑पतिः॒ स मा॑ऽवत्व॒स्मिन् ब्रह्म॑न्न॒स्मिन् क्ष॒त्रेऽस्यामा॒शिष्य॒स्यां पु॑रो॒धाया॑म॒स्मिन् कर्म॑न्न॒स्यां दे॒वहूत्या॒ꣴ स्वाहा।\\
सूर्यायेदम्।

६. च॒न्द्रमा॒ नक्ष॑त्राणा॒मधि॑पतिः॒ स मा॑ऽवत्व॒स्मिन् ब्रह्म॑न्न॒स्मिन् क्ष॒त्रेऽस्यामा॒शिष्य॒स्यां पु॑रो॒धाया॑म॒स्मिन् कर्म॑न्न॒स्यां दे॒वहूत्या॒ꣴ स्वाहा।\\
चन्द्रमस इदम्।

७. बृह॒पति॒र्ब्रह्म॒णोऽधि॑पतिः स मा॑ऽवत्व॒स्मिन् ब्रह्म॑न्न॒स्मिन् क्ष॒त्रेऽस्यामा॒शिष्य॒स्यां पु॑रो॒धाया॑म॒स्मिन् कर्म॑न्न॒स्यां दे॒वहूत्या॒ꣴ स्वाहा।\\
बृहस्पतय इदम्।

८. मि॒त्रः स॒त्याना॒मधि॑पतिः॒ स मा॑ऽवत्व॒स्मिन् ब्रह्म॑न्न॒स्मिन् क्ष॒त्रेऽस्यामा॒शिष्य॒स्यां पु॑रो॒धाया॑म॒स्मिन् कर्म॑न्न॒स्यां दे॒वहूत्या॒ꣴ स्वाहा।\\
मित्रायेदम्।

९. वरु॑णो॒ऽपामधि॑पतिः॒ स मा॑ऽवत्व॒स्मिन् ब्रह्म॑न्न॒स्मिन् क्ष॒त्रेऽस्यामा॒शिष्य॒स्यां पु॑रो॒धाया॑म॒स्मिन् कर्म॑न्न॒स्यां दे॒वहूत्या॒ꣴ स्वाहा।\\
वरुणायेदम्।

१०. स॒मु॒द्रः स्रो॒त्याना॒मधि॑पतिः॒ स मा॑ऽवत्व॒स्मिन् ब्रह्म॑न्न॒स्मिन् क्ष॒त्रेऽस्यामा॒शिष्य॒स्यां पु॑रो॒धाया॑म॒स्मिन् कर्म॑न्न॒स्यां दे॒वहूत्या॒ꣴ स्वाहा।\\
समुद्रायेदम्।

११. अन्न॒ꣳ साम्रा॑॑ज्याना॒मधि॑पति॒ तन्मा॑वत्व॒स्मिन् ब्रह्म॑न्न॒स्मिन् क्ष॒त्रेऽस्यामा॒शिष्य॒स्यां पु॑रो॒धाया॑म॒स्मिन् कर्म॑न्न॒स्यां दे॒वहूत्या॒ꣴ स्वाहा।\\
अन्नायेदम्।

१२. सोम॒ ओष॑धीना॒मधि॑पतिः॒ स मा॑ऽवत्व॒स्मिन् ब्रह्म॑न्न॒स्मिन् क्ष॒त्रेऽस्यामा॒शिष्य॒स्यां पु॑रो॒धाया॑म॒स्मिन् कर्म॑न्न॒स्यां दे॒वहूत्या॒ꣴ स्वाहा।\\
सोमायेदम्।

१३. स॒वि॒ता प्र॑स॒वाना॒मधि॑पतिः॒ स मा॑ऽवत्व॒स्मिन् ब्रह्म॑न्न॒स्मिन् क्ष॒त्रेऽस्यामा॒शिष्य॒स्यां पु॑रो॒धाया॑म॒स्मिन् कर्म॑न्न॒स्यां दे॒वहूत्या॒ꣴ स्वाहा।\\
सवित्र इदम्।

१४. रु॒द्रः प॑शू॒नामधि॑पतिः॒ स मा॑ऽवत्व॒स्मिन् ब्रह्म॑न्न॒स्मिन् क्ष॒त्रेऽस्यामा॒शिष्य॒स्यां पु॑रो॒धाया॑म॒स्मिन् कर्म॑न्न॒स्यां दे॒वहूत्या॒ꣴ स्वाहा।\\
रुद्रायेदं न मम। (अप उपस्पृश्य)

१५. त्वष्टा॑ रू॒पाणा॒मधि॑पतिः॒ स मा॑ऽवत्व॒स्मिन् ब्रह्म॑न्न॒स्मिन् क्ष॒त्रेऽस्यामा॒शिष्य॒स्यां पु॑रो॒धाया॑म॒स्मिन् कर्म॑न्न॒स्यां दे॒वहूत्या॒ꣴ स्वाहा।\\
त्वष्ट्र इदं न मम।

१६. विष्णुः॒ पर्व॑ताना॒मधि॑पतिः॒ स मा॑ऽवत्व॒स्मिन् ब्रह्म॑न्न॒स्मिन् क्ष॒त्रेऽस्यामा॒शिष्य॒स्यां पु॑रो॒धाया॑म॒स्मिन् कर्म॑न्न॒स्यां दे॒वहूत्या॒ꣴ स्वाहा।\\
विष्णव इदम्।

१७. म॒रुतो॑ ग॒णाना॒मधि॑पतय॒स्ते मा॑ऽवन्त्व॒स्मिन् ब्रह्म॑न्न॒स्मिन् क्ष॒त्रेऽस्यामा॒शिष्य॒स्यां पु॑रो॒धाया॑म॒स्मिन् कर्म॑न्न॒स्यां दे॒वहूत्या॒ꣴ स्वाहा।\\
मरुद्भ्य इदम्।

१८. पित॑रः पितामहाः परेऽवरे॒ तता॑॑स्ततामहा इ॒ह मा॑ऽवत। अ॒स्मिन् ब्रह्म॑न्न॒स्मिन् क्ष॒त्रेऽस्यामा॒शिष्य॒स्यां पु॑रो॒धाया॑म॒स्मिन् कर्म॑न्न॒स्यां दे॒वहूत्या॒ꣴ स्वाहा।\\
पितृभ्य इदम्॥ (अप उपस्पृश्य)


\dnsub{राष्ट्रभृतः}
१. ऋ॒ता॒षाडृ॒तधा॑मा॒ग्निर्ग॑न्ध॒र्वः तस्यौष॑धयोऽफ्स॒रस॒ ऊर्जो॒ नाम॒
स इ॒दं ब्रह्म॑ क्ष॒त्रं पा॑तु॒ ता इ॒दं ब्रह्म॑ क्ष॒त्रं पान्तु॒ तस्मै॒ स्वाहा। 
अग्नये गन्धर्वायेदम्। ताभ्यः॒ स्वाहा। ओषधीभ्य अपसरोभ्य इदम्॥

२. स॒ꣳहि॒तो वि॒श्वसा॑मा॒ सूर्यो॑ गन्ध॒र्वस्तस्य॒ मरी॑चयोऽफ्स॒रस॑ आ॒युवो॒ नाम॒ + स्वाहा।
सूर्याय गन्धर्वायेदम्॥ ताभ्यः॒ स्वाहा। मरीचिभ्योऽफ्ससोभ्य इदम्॥

३. सु॒षु॒म्नः सूर्य॑रश्मिश्च॒न्द्रमा॑ गन्ध॒र्वस्तस्य॒ नक्ष॑त्राण्यपस॒रसो॑ बे॒कुर॑यो॒ नाम॒ + स्वाहा।
चन्द्रमसे गन्धर्वायेदम्। ताभ्यः॒ स्वाहा। नक्षत्रेभ्योऽफ्सरोभ्य इदम्॥

४. भु॒ज्युः सु॑प॒र्णो य॒ज्ञो ग॑न्ध॒र्वस्तस्य॒ दक्षि॑णा अफ्स॒रसः॑ स्त॒वा नाम॒ + स्वाहा।
यज्ञाय गन्धर्वायेदम्। ताभ्यः॒ स्वाहा। दक्षिणाभ्योऽफ्सरोभ्य इदम्॥

५. प्र॒जाप॑तिर्वि॒श्वक॑र्मा॒ मनो॑ गन्ध॒र्वस्तस्य॑र्ख्सा॒मान्य॑फ्स॒रसो॒ वह्न॑यो॒ नाम॒ + स्वाहा।
मनसे गन्धर्वायेदम्। ताभ्यः॒ स्वाहा। ऋख्सामेभ्योऽफ्सरोभ्य इदम्॥

६. इ॒षि॒रो वि॒श्वव्य॑चा॒ वातो॑ गन्ध॒र्वस्तस्याऽऽपो॑॑ऽफ्स॒रसो मु॒दा नाम॒ + स्वाहा।
वाताय गन्धर्वायेदम्। ताभ्यः॒ स्वाहा। अद्भ्योऽफ्सरोभ्यः इदम्॥

७. भुव॑नस्य पते॒ यस्य॑ ते उ॒परि॑ गृ॒हा इ॒ह च॑।
स नो॑ रास्वाऽज्या॑निंꣳ रा॒यस्पोष॑ꣳ सु॒वीर्य॑ꣳ संवथ्स॒रिणा॑ꣳ स्व॒स्ति स्वाहा॥ भुवनस्य पत्य इदं॥

८. प॒र॒मे॒ष्ठ्यधि॑पतिर्मृ॒त्युर्ग॑न्ध॒र्वस्तस्य॒ विश्व॑मफ्स॒रसो॒ भुवो॒ नाम॒ + स्वाहा।
मृत्यवे गन्धर्वायेदम्। ताभ्यः॒ स्वाहा। विश्वस्मा अफ्सरोभ्य इदम्॥

९. सु॒क्षि॒तिः सुभू॑तिर्भद्र॒कृत् सुव॑र्वान् प॒र्जन्यो॑ गन्ध॒र्वस्तस्य॑ वि॒द्युतो॑॑ऽफ्स॒रसो॒ रुचो॒ नाम॒ + स्वाहा। पर्जन्याय गन्धर्वाय इदम्। ताभ्यः॒ स्वाहा। विद्युद्भ्योऽफ्सरोभ्य इदम्॥

१०. दू॒रे हे॑तिरमृड॒यो मृ॒त्युर्ग॑न्ध॒र्वस्तस्य॑ प्र॒जा अ॑फ्स॒रसो भी॒रुवो॒ नाम॒ + स्वाहा।
मृत्यवे गन्धर्वायेदम्। ताभ्यः॒ स्वाहा। प्रजाभ्यऽफ्सरोभ्य इदम्॥

११. चारुः॑ कृपणका॒शी कामो॑ गन्ध॒र्वस्तस्या॒धयो॑॑ऽफ्स॒रसः॑ शो॒चय॑न्ती॒र्नाम॒ + स्वाहा।
कामाय गन्धर्वायेदम्। ताभ्यः॒ स्वाहा। आधिभ्यऽफ्सरोभ्य इदम्॥

१२. स नो॑ भुवनस्य पते॒ यस्य॑ त उ॒परि॑ गृ॒हा इ॒ह च॑।
उ॒रु ब्रह्म॑णे॒ऽस्मै क्ष॒त्राय॒ महि॒ शर्म॑ यच्छ॒ स्वाहा॥ भुवनस्य पत्यै ब्रह्मण इदम्॥

\dnsub{प्रजापत्या}
प्रजा॑पते॒ न त्वदे॒तान्य॒न्यो विश्वा॑ जा॒तानि॒ परि॒ ता ब॑भू॒व।
यत्का॑मास्ते जुहु॒मस्तन्नो॑ अस्तु व॒यꣴ स्या॑म॒ पत॑यो रयी॒णाꣴ स्वाहा। प्रजापतय इदं न मम॥

\dnsub{व्याहृतयः}
भूः स्वाहा। अग्नय इदं न मम॥
भुवः॒ स्वाहा। वायव इदं न मम॥
सुवः॒ स्वाहा। सूर्याय इदं न मम॥   

\dnsub{सौविष्टकृती}
यद॑स्य॒ कर्म॒णोऽत्यरी॑रिचं॒ यद्वा॒ न्यू॑मि॒हाक॑रम्। अग्नि॒ष्टत् स्वि॑ष्ट॒कृद् वि॒द्वान् सर्व॒ꣴस्वि॑ष्टं॒ꣳल् सुहु॑तं करोतु॒ स्वाहा। अग्नये स्विष्टकृत इदम्॥