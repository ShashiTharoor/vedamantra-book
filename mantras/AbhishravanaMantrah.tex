% !TeX program = XeLaTeX
% !TeX root = ../vedamantrabook.tex
\chapt{अभिश्रवण-मन्त्राः}

\sect{पुरुषसूक्तम्}
\centerline{\scriptsize(तैत्तिरीयारण्यकम्/प्रपाठकः – ३/अनुवाकः – १२--१३)}

स॒हस्र॑शीर्‌षा॒ पुरु॑षः। 
स॒ह॒स्रा॒क्षः स॒हस्र॑पात्। 
स भूमिं॑ वि॒श्वतो॑ वृ॒त्वा। 
अत्य॑तिष्ठद्दशाङ्गु॒लम्॥ 
% 
पुरु॑ष ए॒वेदꣳ सर्वम्᳚। 
यद्भू॒तं यच्च॒ भव्यम्᳚। 
उ॒तामृ॑त॒त्वस्येशा॑नः। 
यदन्ने॑नाति॒रोह॑ति॥ 
% 
ए॒तावा॑नस्य महि॒मा। 
अतो॒ ज्यायाꣴ॑श्च॒ पूरु॑षः। 
पादो᳚ऽस्य॒ विश्वा॑ भू॒तानि॑। 
त्रि॒पाद॑स्या॒मृतं॑ दि॒वि॥ 
% 
त्रि॒पादू॒र्ध्व उदै॒त्पुरु॑षः। 
पादो᳚ऽस्ये॒हाऽऽभ॑वा॒त्पुनः॑। 
ततो॒ विश्व॒ङ्व्य॑क्रामत्। 
सा॒श॒ना॒न॒श॒ने अ॒भि॥ 
% 
तस्मा᳚द्वि॒राड॑जायत। 
वि॒राजो॒ अधि॒ पूरु॑षः। 
स जा॒तो अत्य॑रिच्यत। 
प॒श्चाद्भूमि॒मथो॑ पु॒रः॥ 
% 
 यत्पुरु॑षेण ह॒विषा᳚। 
दे॒वा य॒ज्ञमत॑न्वत। 
व॒स॒न्तो अ॑स्याऽऽसी॒दाज्यम्᳚। 
ग्री॒ष्म इ॒ध्मः श॒रद्ध॒विः॥ 
% 
 स॒प्तास्या॑ऽऽसन्  परि॒धयः॑। 
त्रिः स॒प्त स॒मिधः॑ कृ॒ताः। 
दे॒वा यद्य॒ज्ञं त॑न्वा॒नाः। 
अब॑ध्न॒न् पुरु॑षं प॒शुम्॥ 
% 
 तं य॒ज्ञं ब॒\ar हिषि॒ प्रौक्ष\sn। 
पुरु॑षं जा॒तम॑ग्र॒तः। 
तेन॑ दे॒वा अय॑जन्त। 
सा॒ध्या ऋष॑यश्च॒ ये॥ 
% 
तस्मा᳚द्य॒ज्ञाथ्स॑र्व॒हुतः॑। 
सम्भृ॑तं पृषदा॒ज्यम्। 
प॒शूꣴस्ताꣴश्च॑क्रे वाय॒व्या\sn{}। 
आ॒र॒ण्यान्ग्रा॒म्याश्च॒ ये॥ 
% 
 तस्मा᳚द्य॒ज्ञाथ्स॑र्व॒हुतः॑। 
ऋचः॒ सामा॑नि जज्ञिरे। 
छन्दाꣳ॑सि जज्ञिरे॒ तस्मा᳚त्। 
यजु॒स्तस्मा॑दजायत॥ 
% 
तस्मा॒दश्वा॑ अजायन्त। 
ये के चो॑भ॒याद॑तः। 
गावो॑ ह जज्ञिरे॒ तस्मा᳚त्। 
तस्मा᳚ज्जा॒ता अ॑जा॒वयः॑॥ 
% 
यत्पुरु॑षं॒ व्य॑दधुः। 
क॒ति॒धा व्य॑कल्पयन्। 
मुखं॒ किम॑स्य॒ कौ बा॒हू। 
कावू॒रू पादा॑वुच्येते॥ 
% 
ब्रा॒ह्म॒णो᳚ऽस्य॒ मुख॑मासीत्। 
बा॒हू रा॑ज॒न्यः॑ कृ॒तः। 
ऊ॒रू तद॑स्य॒ यद्वैश्यः॑। 
प॒द्भ्याꣳ शू॒द्रो अ॑जायत॥ 
% 
च॒न्द्रमा॒ मन॑सो जा॒तः। 
चक्षोः॒ सूर्यो॑ अजायत। 
मुखा॒दिन्द्र॑श्चा॒ग्निश्च॑। 
प्रा॒णाद्वा॒युर॑जायत॥ 
% 
नाभ्या॑ आसीद॒न्तरि॑क्षम्। 
शी॒र्ष्णो द्यौः सम॑वर्तत। 
प॒द्भ्यां भूमि॒र्दिशः॒ श्रोत्रा᳚त्। 
तथा॑ लो॒काꣳ अ॑कल्पयन्॥ 
% 
वेदा॒हमे॒तं पुरु॑षं म॒हान्तम्᳚। 
आ॒दि॒त्यव॑र्णं॒ तम॑स॒स्तु पा॒रे॥ 
% 
सर्वा॑णि रू॒पाणि॑ वि॒चित्य॒ धीरः॑। 
नामा॑नि कृ॒त्वाऽभि॒वद॒\an{} यदास्ते᳚॥ 
% 
धा॒ता पु॒रस्ता॒द्यमु॑दाज॒हार॑। 
श॒क्रः प्रवि॒द्वान्  प्र॒दिश॒श्चत॑स्रः। 
तमे॒वं वि॒द्वान॒मृत॑ इ॒ह भ॑वति। 
नान्यः पन्था॒ अय॑नाय विद्यते॥ 
% 
य॒ज्ञेन॑ य॒ज्ञम॑यजन्त दे॒वाः। 
तानि॒ धर्मा॑णि प्रथ॒मान्या॑सन्। 
ते ह॒ नाकं॑ महि॒मानः॑ सचन्ते। 
यत्र॒ पूर्वे॑ सा॒ध्याः सन्ति॑ दे॒वाः॥ 
% 
अ॒द्भ्यः सम्भू॑तः पृथि॒व्यै रसा᳚च्च। 
वि॒श्वक॑र्मणः॒ सम॑वर्त॒ताधि॑। 
तस्य॒ त्वष्टा॑ वि॒दध॑द्रू॒पमे॑ति। 
तत्पुरु॑षस्य॒ विश्व॒माजा॑न॒मग्रे᳚॥ 
% 
वेदा॒हमे॒तं पुरु॑षं म॒हान्तम्᳚। 
आ॒दि॒त्यव॑र्णं॒ तम॑सः॒ पर॑स्तात्। 
तमे॒वं वि॒द्वान॒मृत॑ इ॒ह भ॑वति। 
नान्यः पन्था॑ विद्य॒तेऽय॑नाय॥ 
% 
प्र॒जाप॑तिश्चरति॒ गर्भे॑ अ॒न्तः। 
अ॒जाय॑मानो बहु॒धा विजा॑यते। 
तस्य॒ धीराः॒ परि॑जानन्ति॒ योनिम्᳚। 
मरी॑चीनां प॒दमि॑च्छन्ति वे॒धसः॑॥ 
% 
यो दे॒वेभ्य॒ आत॑पति। 
यो दे॒वानां᳚ पु॒रोहि॑तः। 
पूर्वो॒ यो दे॒वेभ्यो॑ जा॒तः। 
नमो॑ रु॒चाय॒ ब्राह्म॑ये॥ 
% 
रुचं॑ ब्रा॒ह्मं ज॒नय॑न्तः। 
दे॒वा अग्रे॒ तद॑ब्रुवन्। 
यस्त्वै॒वं ब्रा᳚ह्म॒णो वि॒द्यात्। 
तस्य॑ दे॒वा अस॒न् वशे᳚॥ 
% 
ह्रीश्च॑ ते ल॒क्ष्मीश्च॒ पत्न्यौ᳚। 
अ॒हो॒रा॒त्रे पा॒र्श्वे। 
नक्ष॑त्राणि रू॒पम्। 
अ॒श्विनौ॒ व्यात्तम्᳚। 
इ॒ष्टं म॑निषाण। 
अ॒मुं म॑निषाण। 
सर्वं॑ मनिषाण॥ 
% 
\centerline{॥ॐ शान्तिः॒ शान्तिः॒ शान्तिः॑॥}

\chapt{नारायणसूक्तम्}
\centerline{\scriptsize(तैत्तिरीयारण्यकम्/प्रपाठकः – १०/अनुवाकः – १३)}

स॒ह॒स्र॒शी\sr%
षं दे॒वं॒ वि॒श्वाक्षं॑ वि॒श्वश॑म्भुवम्। विश्वं॑ ना॒राय॑णं दे॒व॒म॒क्षरं॑ पर॒मं प॒दम्। 
वि॒श्वतः॒ पर॑मान्नि॒त्यं॒ वि॒श्वं ना॑राय॒णꣳ ह॑रिम्। विश्व॑मे॒वेदं पुरु॑ष॒स्तद्विश्व॒मुप॑जीवति। 
पतिं॒   विश्व॑स्या॒ऽ॒ऽ॒त्मेश्व॑र॒ꣳ॒ शाश्व॑तꣳ शि॒वम॑च्युतम्। ना॒राय॒णं म॑हाज्ञे॒यं॒ वि॒श्वात्मा॑नं प॒राय॑णम्। ना॒राय॒णप॑रो ज्यो॒ति॒रा॒त्मा ना॑राय॒णः प॑रः। ना॒राय॒ण प॑रं ब्र॒ह्म॒ त॒त्त्वं ना॑राय॒णः प॑रः। ना॒राय॒णप॑रो ध्या॒ता॒ ध्या॒नं ना॑राय॒णः प॑रः। यच्च॑ कि॒ञ्चिज्ज॑गथ्स॒र्वं॒ दृ॒श्यते᳚ श्रूय॒तेऽपि॑ वा॥ 

अन्त॑र्ब॒हिश्च॑ तथ्स॒र्वं॒ व्या॒प्य ना॑राय॒णः स्थि॑तः। अन॑न्त॒मव्य॑यं क॒विꣳ स॑मु॒द्रेऽन्तं॑  वि॒श्वश॑म्भुवम्। प॒द्म॒को॒श प्र॑तीका॒श॒ꣳ॒ हृ॒दयं॑ चाप्य॒धोमु॑खम्। अधो॑ नि॒ष्ट्या वि॑तस्त्या॒न्ते॒ ना॒भ्यामु॑परि॒ तिष्ठ॑ति। ज्वा॒ल॒मा॒लाकु॑लं भा॒ती॒ वि॒श्वस्या॑ऽऽयत॒नं म॑हत्। सन्त॑तꣳ शि॒लाभि॑स्तु॒\-लम्ब॑त्याकोश॒सन्नि॑भम्। तस्यान्ते॑ सुषि॒रꣳ सू॒क्ष्मं तस्मि᳚न्थ्स॒र्वं प्रति॑ष्ठितम्। तस्य॒ मध्ये॑ म॒हान॑\-ग्निर्वि॒श्वार्चि॑र्वि॒श्वतो॑मुखः। सोऽग्र॑भु॒ग्विभ॑जन्ति॒ष्ठ॒न्नाहा॑रमज॒रः क॒विः। ति॒र्य॒गू॒र्ध्वम॑धः शा॒यी॒ र॒श्मय॑स्तस्य॒ सन्त॑ता। स॒न्ता॒पय॑ति स्वं दे॒हमापा॑दतल॒मस्त॑कः। तस्य॒ मध्ये॒ वह्नि॑शिखा अ॒णीयो᳚र्ध्वा व्य॒वस्थि॑तः। नी॒लतो॑यद॑\-मध्य॒स्था॒द्वि॒द्युल्ले॑खेव॒  भास्व॑रा। नी॒वार॒शूक॑वत्त॒न्वी॒ पी॒ता भा᳚स्वत्य॒णूप॑मा। तस्याः᳚ शिखा॒या म॑ध्ये प॒रमा᳚त्मा व्य॒वस्थि॑तः। स ब्रह्म॒ स शिवः॒ स हरिः॒ सेन्द्रः॒ सोऽक्ष॑रः पर॒मः स्व॒राट्॥ 
ऋ॒तꣳ स॒त्यं प॑रं ब्र॒ह्म॒ पु॒रुषं॑ कृष्ण॒पिङ्ग॑लम्। ऊ॒र्ध्वरे॑तं वि॑रूपा॒क्षं॒ वि॒श्वरू॑पाय॒ वै नमो॒ नमः॑। 

ना॒रा॒य॒णाय॑ वि॒द्महे॑ वासुदे॒वाय॑ धीमहि। तन्नो॑ विष्णुः प्रचो॒दया᳚त्। 

विष्णो॒र्नु कं॑ वी॒र्या॑णि॒ प्रवो॑चं॒ यः पार्थि॑वानि विम॒मे रजाꣳ॑सि॒ यो अस्क॑भाय॒दुत्त॑रꣳ स॒धस्थं॑ विचक्रमा॒णस्त्रे॒धोरु॑गा॒यो विष्णो॑र॒राट॑मसि॒ विष्णोः᳚ पृ॒ष्ठम॑सि॒ विष्णोः॒ श्नप्त्रे᳚स्थो॒ विष्णोः॒ स्यूर॑सि॒ विष्णो᳚र्ध्रु॒वम॑सि वैष्ण॒वम॑सि॒ विष्ण॑वे त्वा॥ 

\centerline{॥ॐ शान्तिः॒ शान्तिः॒ शान्तिः॑॥}

कृ॒णु॒ष्व पाजः॒ प्रसि॑तिं॒ न पृ॒थ्वीं या॒हि राजे॒वाम॑वा॒ꣳ॒ इभे॑न। तृ॒ष्वीमनु॒ प्रसि॑तिं द्रूणा॒नोऽस्ता॑सि॒ विध्य॑ र॒क्षस॒स्तपि॑ष्ठैः। तव॑ भ्र॒मास॑ आशु॒या प॑त॒न्त्यनु॑ स्पृश धृष॒ता शोशु॑चानः। तपूꣴ॑ष्यग्ने जु॒ह्वा॑ पत॒ङ्गानस॑न्दितो॒ वि सृ॑ज॒ विष्व॑गु॒ल्काः। प्रति॒ स्पशो॒ वि सृ॑ज॒ तूर्णि॑तमो॒ भवा॑ पा॒युर्वि॒शो अ॒स्या अद॑ब्धः। यो नो॑ दू॒रे अ॒घशꣳ॑सो॒  यो अन्त्यग्ने॒ माकि॑ष्टे॒ व्यथि॒रा द॑धर्षीत्॥१॥

उद॑ग्ने तिष्ठ॒ प्रत्याऽऽत॑नुष्व॒ न्य॑मित्राꣳ॑ ओषतात्तिग्महेते। यो नो॒ अरा॑तिꣳ समिधान च॒क्रे नी॒चा तं ध॑क्ष्यत॒सं न शुष्कम्᳚। ऊ॒र्ध्वो भ॑व॒ प्रति॑ वि॒ध्याध्य॒स्मदा॒विष्कृ॑णुष्व॒ दैव्या᳚न्यग्ने। अव॑ स्थि॒रा त॑नुहि यातु॒जूनां᳚ जा॒मिमजा॑मिं॒ प्र मृ॑णीहि॒ शत्रून्॑। स ते॑ जानाति सुम॒तिं य॑विष्ठ॒ य ईव॑ते॒ ब्रह्म॑णे गा॒तुमैर॑त्॥२॥

 विश्वा᳚न्यस्मै सु॒दिना॑नि रा॒यो द्यु॒म्नान्य॒र्यो वि दुरो॑ अ॒भि द्यौ᳚त्। सेद॑ग्ने अस्तु सु॒भगः॑ सु॒दानु॒र्यस्त्वा॒ नित्ये॑न ह॒विषा॒ य उ॒क्थैः। पिप्री॑षति॒ स्व आयु॑षि दुरो॒णे विश्वेद॑स्मै सु॒दिना॒ साऽस॑दि॒ष्टिः। अर्चा॑मि ते सुम॒तिं घोष्य॒र्वाख्सं ते॑ वा॒वाता॑ जरतामि॒यङ्गीः॥३॥
 
स्वश्वा᳚स्त्वा सु॒रथा॑ मर्जयेमा॒स्मे क्ष॒त्राणि॑ धारये॒रनु॒ द्यून्। इ॒ह त्वा॒ भूर्या च॑रे॒दुप॒ त्मन्दोषा॑\-वस्तर्दीदि॒वाꣳ\-स॒मनु॒ द्यून्। कीड॑न्तस्त्वा सु॒मन॑सः सपेमा॒भि द्यु॒म्ना त॑स्थि॒वाꣳसो॒ जना॑नाम्। यस्त्वा॒ स्वश्वः॑ सुहिर॒ण्यो अ॑ग्न उप॒याति॒ वसु॑मता॒ रथे॑न। तस्य॑ त्रा॒ता भ॑वसि॒ तस्य॒ सखा॒ यस्त॑ आति॒थ्यमा॑नु॒षग्जुजो॑षत्। म॒हो रु॑जामि ब॒न्धुता॒ वचो॑भि॒स्तन्मा॑ पि॒तुर्गोत॑मा॒दन्वि॑याय॥४॥

 त्वं नो॑ अ॒स्य वच॑सश्चिकिद्धि॒ होत॑र्यविष्ठ सुक्रतो॒ दमू॑नाः। अस्व॑प्नजस्त॒रण॑यः सु॒शेवा॒ अत॑न्द्रासोऽवृ॒का अश्र॑मिष्ठाः। ते पा॒यवः॑ स॒ध्रिय॑ञ्चो नि॒षद्याऽग्ने॒ तव॑ नः पान्त्वमूर। ये पा॒यवो॑ मामते॒यं ते॑ अग्ने॒ पश्य॑न्तो अ॒न्धं दु॑रि॒तादर॑क्षन्। र॒रक्ष॒ तान्थ्सु॒कृतो॑ वि॒श्ववे॑दा॒ दिफ्स॑न्त॒ इद्रि॒पवो॒ ना ह॑ देभुः॥५॥
 
त्वया॑ व॒यꣳ स॑ध॒न्य॑स्त्वोता॒स्तव॒ प्रणी᳚त्यश्याम॒ वाजा\sn{}। उ॒भा शꣳसा॑ सूदय सत्यतातेऽनुष्ठु॒या कृ॑णुह्यह्रयाण। अ॒या ते॑ अग्ने स॒मिधा॑ विधेम॒ प्रति॒ स्तोमꣳ॑ श॒स्यमा॑नं गृभाय। दहा॒ऽ॒शसो॑ र॒क्षसः॑ पा॒ह्य॑स्मान्द्रु॒हो नि॒दो मि॑त्रमहो अव॒द्यात्। र॒क्षो॒हणं॑ वा॒जिन॒माऽऽजि॑घर्मि मि॒त्रं प्रथि॑ष्ठ॒मुप॑ यामि॒ शर्म॑। शिशा॑नो अ॒ग्निः क्रतु॑भिः॒ समि॑द्धः॒ स नो॒ दिवा॒ स रि॒षः पा॑तु॒ नक्तम्᳚॥६॥

 वि ज्योति॑षा बृह॒ता भा᳚त्य॒ग्निरा॒विर्विश्वा॑नि कृणुते महि॒त्वा। प्रादे॑वीर्मा॒याः स॑हते दु॒रेवाः॒ शिशी॑ते॒ शृङ्गे॒ रक्ष॑से वि॒निक्षे᳚। उ॒त स्वा॒नासो॑ दि॒विष॑न्त्व॒ग्नेस्ति॒ग्मायु॑धा॒ रक्ष॑से॒ हन्त॒वा उ॑। मदे॑ चिदस्य॒ प्ररु॑जन्ति॒ भामा॒ न व॑रन्ते परि॒बाधो॒ अदे॑वीः॥७॥[१.२.१४]

\centerline{\scriptsize(तैत्तिरीय-ब्राह्मणम्/अष्टकम्–१/प्रश्नः—२/अनुवाकः–३)}

%1.2.3.1
सन्त॑ति॒र्वा ए॒ते ग्रहाः᳚।
यत्परः॑ सामानः।
वि॒षू॒वान्दि॑वा\-की॒र्त्यम्᳚।
यथा॒ शाला॑यै॒ पक्ष॑सी।
ए॒वꣳ सं॑वथ्स॒रस्य॒ पक्ष॑सी।
यदे॒तेन गृ॒ह्येर\sn{}।
विषू॑ची संवथ्स॒रस्य॒ पक्ष॑सी॒ व्यव॑स्रꣳसेयाताम्।
आर्ति॒मार्च्छे॑युः।
यदे॒ते गृ॒ह्यन्ते᳚।
यथा॒ शाला॑यै॒ पक्ष॑सी मध्य॒मं व॒ꣳ॒शम॒भि स॑मा॒यच्छ॑ति॥३३॥

%1.2.3.2
ए॒वꣳ सं॑वथ्स॒रस्य॒ पक्ष॑सी दिवाकी॒र्त्य॑म॒भि सं त॑न्वन्ति।
नार्ति॒मार्च्छ॑न्ति।
ए॒क॒वि॒ꣳ॒शमह॑र्भवति।
शु॒क्राग्रा॒ ग्रहा॑ गृह्यन्ते।
प्रत्युत्त॑ब्ध्यै सय॒त्वाय॑।
सौ॒र्य॑ ए॒तदहः॑ प॒शुराल॑भ्यते।
सौ॒र्यो॑\-ऽतिग्रा॒ह्यो॑ गृह्यते।
अह॑रे॒व रू॒पेण॒ सम॑र्धयन्ति।
अथो॒ अह्न॑ ए॒वैष ब॒लिर्\mbox{}ह्रि॑यते।
स॒प्तैतदह॑रतिग्रा॒ह्या॑ गृह्यन्ते॥३४॥

%1.2.3.3
स॒प्त वै शी॑र्\mbox{}ष॒ण्या᳚ प्रा॒णाः।
अ॒सावा॑दि॒त्यः शिरः॑ प्र॒जाना᳚म्।
शी॒र्॒षन्ने॒व प्र॒जानां᳚ प्रा॒णान्द॑धाति।
तस्मा᳚थ्स॒प्त शी॒र्॒षन्प्रा॒णाः।
इन्द्रो॑ वृ॒त्रꣳ ह॒त्वा।
असु॑रान्परा॒भाव्य॑।
स इ॒माँल्लो॒कान॒भ्य॑जयत्।
तस्या॒सौ लो॒को\-ऽन॑भिजित आसीत्।
तं वि॒श्वक॑र्मा भू॒त्वा\-ऽभ्य॑जयत्।
यद्वै᳚श्वकर्म॒णो गृ॒ह्यते᳚॥३५॥

%1.2.3.4
सु॒व॒र्गस्य॑ लो॒कस्या॒भिजि॑त्यै।
प्र वा ए॒ते᳚\-ऽस्माल्लो॒काच्च्य॑वन्ते।
ये वै᳚श्वकर्म॒णं गृ॒ह्णते᳚।
आ॒दि॒त्यः श्वो गृ॑ह्यते।
इ॒यं वा अदि॑तिः।
अ॒स्यामे॒व प्रति॑ तिष्ठन्ति।
अ॒न्यो᳚न्यो गृह्येते।
विश्वा᳚न्ये॒वान्येन॒ कर्मा॑णि कुर्वा॒णा य॑न्ति।
अ॒स्याम॒न्येन॒ प्रति॑ तिष्ठन्ति।
तावाऽप॑रा॒र्धाथ्सं॑वथ्स॒रस्या॒न्यो᳚न्यो गृह्येते।
तावु॒भौ स॒ह म॑हाव्र॒ते गृ॑ह्येते।
य॒ज्ञस्यै॒वान्तं॑ ग॒त्वा।
उ॒भयो᳚र्लो॒कयोः॒ प्रति॑ तिष्ठन्ति।
अ॒र्क्य॑मु॒क्थं भ॑वति।
अ॒न्नाद्य॒स्याव॑रुद्ध्यै॥३६॥



\centerline{\scriptsize(तैत्तिरीयकाठकम्/प्रश्नः–३/अनुवाकः–९)}

   ऋ॒चां प्राची॑ मह॒ती दिगु॑च्यते।
   दक्षि॑णामाहु॒र्यजु॑षामपा॒राम्।
   अथ॑र्वणा॒मङ्गि॑रसां प्र॒तीची᳚।
   साम्ना॒मुदी॑ची मह॒ती दिगु॑च्यते।
   ऋ॒ग्भिः पू᳚र्वा॒ह्णे दि॒वि दे॒व ई॑यते।
   य॒जु॒र्वे॒दे ति॑ष्ठति॒ मध्ये॒ अह्नः॑।
   सा॒म॒वे॒देना᳚ऽस्तम॒ये मही॑यते।
   वेदै॒रशू᳚न्यस्त्रि॒भिरे॑ति॒ सूर्यः॑।
   ऋ॒ग्भ्यो जा॒ताꣳ स॑र्व॒शो मूर्ति॑माहुः।
   सर्वा॒ गति॑र्याजु॒षी है॒व शश्व॑त्॥४९॥

   सर्वं॒ तेजः॑ सामरू॒प्यꣳ ह॑ शश्वत्।
   सर्वꣳ॑ हे॒दं ब्रह्म॑णा है॒व सृ॒ष्टम्।
   ऋ॒ग्भ्यो जा॒तं वैश्यं॒ वर्ण॑माहुः।
   य॒जु॒र्वे॒दं क्ष॑त्रि॒यस्या॑ऽऽहु॒र्योनिम्᳚।
   सा॒म॒वे॒दो ब्रा᳚ह्म॒णानां॒ प्रसू॑तिः।
   पूर्वे॒ पूर्वे᳚भ्यो॒ वच॑ ए॒तदू॑चुः।
   आ॒द॒र्\mbox{}शम॒ग्निं चि॑न्वा॒नाः।
   पूर्वे॑ विश्व॒सृजो॒ऽमृताः᳚।
   श॒तं व॑र्‌षसह॒स्राणि॑।
   दी॒क्षि॒ताः स॒त्रमा॑सत॥५०॥

   तप॑ आसीद्गृ॒हप॑तिः।
   ब्रह्म॑ ब्र॒ह्माऽभ॑वथ्स्व॒यम्।
   स॒त्यꣳ ह॒ होतै॑षा॒मासी᳚त्।
   यद्वि॑श्व॒सृज॒ आस॑त।
   अ॒मृत॑मेभ्य॒ उद॑गायत्।
   स॒हस्रं॑ परिवथ्स॒रान्।
   भू॒तꣳ ह॑ प्रस्तो॒तैषा॒मासी᳚त्।
   भ॒वि॒ष्यत्प्रति॑ चाहरत्।
   प्रा॒णो अ॑ध्व॒र्युर॑भवत्।
   इ॒दꣳ सर्व॒ꣳ॒ सिषा॑सताम्॥५१॥

   अ॒पा॒नो वि॒द्वाना॒वृतः॑।
   प्रति॒प्राति॑ष्ठदध्व॒रे।
   आ॒र्त॒वा उ॑पगा॒तारः॑।
   स॒द॒स्या॑ ऋ॒तवो॑ऽभवन्।
   अ॒र्ध॒मा॒साश्च॒ मासा᳚श्च।
   च॒म॒सा॒ध्व॒र्य॒वोऽभ॑वन्।
   अ॒शꣳ॑स॒द्ब्रह्म॑ण॒स्तेजः॑।
   अ॒च्छा॒वा॒कोऽभ॑व॒द्यशः॑।
   ऋ॒तमे॑षां प्रशा॒स्ताऽऽसी᳚त्।
   यद्वि॑श्व॒सृज॒ आस॑त॥५२॥

   ऊर्ग्राजा॑न॒मुद॑वहत्।
   ध्रु॒व॒गो॒पः सहो॑ऽभवत्।
   ओजो॒ऽभ्य॑ष्टौ॒\-द्ग्राव्ण्णः॑।
   यद्वि॑श्व॒सृज॒ आस॑त।
   अप॑चितिः पो॒त्रीया॑मयजत्।
   ने॒ष्ट्रीया॑म\-यज॒त्त्विषिः॑।
   आग्नी᳚द्ध्राद्वि॒दुषी॑ स॒त्यम्।
   श्र॒द्धा है॒वाय॑जथ्स्व॒यम्।
   इरा॒ पत्नी॑ विश्व॒सृजा᳚म्।
   आकू॑तिरपिन\-ड्ढ॒विः॥५३॥

   इ॒ध्मꣳ ह॒ क्षुच्चै᳚भ्य उ॒ग्रे।
   तृ॒ष्णा चाऽऽव॑हतामु॒भे।
   वागे॑षाꣳ सुब्रह्म॒ण्याऽऽसी᳚त्।
   छ॒न्दो॒यो॒गान् वि॑जान॒ती।
   क॒ल्प॒त॒न्त्राणि॑ तन्वा॒नाऽहः॑।
   स॒ꣴ॒स्थाश्च॑ सर्व॒शः।
   अ॒हो॒रा॒त्रे प॑शुपा॒ल्यौ।
   मु॒हू॒र्ताः प्रे॒ष्या॑ अभवन्।
   मृ॒त्युस्तद॑भवद्धा॒ता।
   श॒मि॒तोग्रो वि॒शां पतिः॑॥५४॥

   वि॒श्व॒सृजः॑ प्रथ॒माः स॒त्रमा॑सत।
   स॒हस्र॑समं॒ प्रसु॑तेन॒ यन्तः॑।
   ततो॑ ह जज्ञे॒ भुव॑नस्य गो॒पाः।
   हि॒र॒ण्मयः॑ श॒कुनि॒र्ब्रह्म॒ नाम॑।
   येन॒ सूर्य॒स्तप॑ति॒ तेज॑से॒द्धः।
   पि॒ता पु॒त्रेण॑ पितृ॒मान् योनि॑योनौ।
   नावे॑दविन्मनुते॒ तं बृ॒हन्तम्᳚।
   स॒र्वा॒नु॒भुमा॒त्मानꣳ॑ सम्परा॒ये।
   ए॒ष नि॒त्यो म॑हि॒मा ब्रा᳚ह्म॒णस्य॑।
   न कर्म॑णा वर्धते॒ नो कनी॑यान्॥५५॥

   तस्यै॒वाऽऽत्मा प॑द॒वित्तं वि॑दित्वा।
   न कर्म॑णा लिप्यते॒ पाप॑केन।
   पञ्च॑पञ्चा॒शत॑स्त्रि॒वृतः॑ संवथ्स॒राः।
   पञ्च॑पञ्चा॒शतः॑ पञ्चद॒शाः।
   पञ्च॑पञ्चा॒शतः॑ सप्तद॒शाः।
   पञ्च॑पञ्चा॒शत॑ एकवि॒ꣳ॒शाः।
   वि॒श्व॒सृजाꣳ॑ स॒हस्र॑संवथ्सरम्।
   ए॒तेन॒ वै वि॑श्व॒सृज॑ इ॒दं विश्व॑मसृजन्त।
   यद्विश्व॒मसृ॑जन्त।
   तस्मा᳚द्विश्व॒सृजः॑।
   विश्व॑मेना॒ननु॒ प्रजा॑यते।
   ब्रह्म॑णः॒ सायु॑ज्यꣳ सलो॒कतां᳚ यन्ति।
   ए॒तासा॑मे॒व दे॒वता॑ना॒ꣳ॒ सायु॑ज्यम्।
   सा॒र्ष्टिताꣳ॑ समानलो॒कतां᳚ यन्ति।
   य ए॒तदु॑प॒यन्ति॑।
   ये चै॑न॒त्प्राहुः॑।
   येभ्य॑श्चैन॒त्प्राहुः॑॥५६॥
 ॐ॥


\closesection