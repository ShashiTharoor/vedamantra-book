% !TeX program = XeLaTeX
% !TeX root = ../vedamantrabook.tex

\chapt{यजुर्वेद उपाकर्म}

\sect{कामोकार्षीन्मन्त्र जपम्}

\sect{महा-सङ्कल्पम्}
आचम्य, दर्भेषु आसीनः, दर्भान् धारयमाणः । शुक्लाम्बरधरं + शान्तये । प्राणानायम्य । 
    ममोपात्त समस्तदुरितक्षयद्वारा श्रीपरमेश्वरप्रीत्यर्थं   तदेव लग्नं सुदिनं तदेव ताराबलं चन्द्रबलं तदेव विद्याबलं दैवबलं तदेव लक्ष्मीपते ते अङ्घ्रियुगं स्मरामि।
    \twolineshloka*
    {अपवित्रः पवित्रो वा सर्वावस्थागतोऽपि वा}
    {यः स्मरेत्पुण्डरीकाक्षं स बाह्याभ्यन्तरः शुचिः}

     \twolineshloka*
  {मानसं वाचिकं पापं कर्मणा समुपार्जितम्}
{श्रीरामस्मरणेनैव व्यपोहति न संशयः }

श्रीराम राम राम। 

    \twolineshloka*
{तिथिर्विष्णुस्तथा वारो नक्षत्रं विष्णुरेव च}
{योगश्च करणं चैव सर्वं विष्णुमयं जगत्}

श्रीगोविन्द गोविन्द गोविन्द। 

अद्य श्रीभगवतः आदिविष्णोः आदिनारायणस्य अचिन्त्यया अपरिमितया शक्त्या भ्रियमाणस्य महाजलौघस्य मध्ये परिभ्रमताम् अनेककोटिब्रह्माण्डानाम् एकतमे अव्यक्त महदहङ्कार पृथिव्यप्-तेजो-वाय्वाकाशाद्यैः आवरणैः आवृते अस्मिन् महति ब्रह्माण्डकरण्डमण्डले आधारशक्ति आदिकूर्मादि अनन्तादि अष्टदिग्गजोपरि प्रतिष्ठितस्य अतल वितल सुतल रसातल तलातल महातल पाताळाख्य लोकसप्तकस्य उपरितले पुण्यकृताम् निवासभूते भुवः सुवः महर्ज्जन तपः सत्याख्य लोकषट्कस्य अधोभागे महानाळायमानफणिराजशेषस्य सहस्र फणामणि मण्डल मण्डिते दिग्दन्ति शुण्डादण्ड उत्तम्भिते पञ्चाशत्कोटियोजन विस्तीर्ण्णे लोकालोक अचलेन वलयिते लवणेक्षु सुरासर्पि दधि क्षीर शुद्धोदकार्ण्णवैः परिवृते जम्बू-प्लक्ष-शाक-शाल्मलि-कुश-क्रौञ्च-पुष्कराख्य-सप्तद्वीपविराजिते जम्बूद्वीपे भारत-किम्पुरुष-हरि-इला वृत-रम्यक-हिरण्मय- कुरु-भद्राश्व-केतुमाल-नववर्षमध्ये भारतवर्षे इन्द्र-चेरु-ताम्र-गभस्ति-नाग- सौम्य-गन्धर्व-चारण-भरत-नवखण्डमध्ये भरतखण्डे सुमेरु-निषध-हेमकूट -हिमाचल-माल्यवत्-पारियात्रक -गन्धमादन-कैलास-विन्ध्याचलादि-महाशैलमध्ये दण्डकारण्य-चम्प कारण्य-विन्ध्यारण्य-वीक्षारण्य- श्वेतारण्य-वेदारण्यादि-अनेकपुण्यारण्यानां मध्ये कर्मभूमौ रामसेतुकेदारयोः मध्ये भागीरथी-गौतमी -कृष्णवेणी-यमुना-नर्मदा- तुङ्गभद्रा-त्रिवेणी-महापहारिणी-कावेर्यादि-अनेकपुण्यनदी-विराजिते इन्द्रप्रस्थ-यमप्रस्थ-अवन्तिकापुरी-हस्तिनापुरी-अयोध्यापुरी-मथुरापुरी-मायापुरी-काशीपुरी-काञ्ची पुरी-द्वारकादि- अनेकपुण्यपुरी-विराजिते     सकलजगत्स्रष्टुः परार्धद्वयजीविनः ब्रह्मणः द्वितीयपरार्धे पञ्चाशद्-अब्दादौ प्रथमे वर्षे प्रथमे मासे प्रथमे पक्षे प्रथमे दिवसे अह्नि द्वितीये यामे तृतीये मुहूर्ते स्वायम्भुव-स्वारोचिष-उत्तम-तामस-रैवत-चाक्षुषाख्येषु षट्सु मनुषु अतीतेषु सप्तमे वैवस्वतमन्वन्तरे अष्टाविंशतितमे कलियुगे प्रथमे पादे अस्मिन् वर्तमाने व्यावहारिकाणां प्रभवादीनां षष्ट्याः संवत्सराणां मध्ये ...नाम संवत्सरे दक्षिणायने ग्रीष्मऋतौ सिंह मासे शुक्लपक्षे पौर्णमास्यां शुभतिथौ ...वासरयुक्तायां ...नक्षत्रयुक्तायां ...योग...करण युक्तायाम् एवंगुणविशेषणविशिष्टायाम् अस्यां पौर्णमास्यां शुभतिथौ 
    अनादि-अविद्या-वासनया प्रवर्तमाने अस्मिन् महति संसारचक्रे विचित्राभिः कर्मगतिभिः विचित्रासु योनिषु पुनःपुनः अनेकधा जनित्वा केनापि पुण्यकर्मविशेषेण इदानीन्तन-मानुष-द्विजजन्मविशेषं प्राप्तवतः मम जन्माभ्यासात् जन्मप्रभृति एतत्क्षणपर्यन्तं बाल्ये कौमारे यौवने मध्यमे वयसि वार्धके च जागृत्-स्वप्न-सुषुप्ति-अवस्थासु मनो-वाक्-कायाख्यत्रिकरणचेष्टया कर्मेन्द्रिय-ज्ञानेन्द्रिय-व्यापारैः सम्भावितानाम् इह जन्मनि जन्मान्तरे च ज्ञानाज्ञानकृतानां महापातकानां महापातक-अनुमन्तृत्वादीनां समपातकानाम् उपपातकानां मलिनीकरणानां निन्दितधन-आदान-उपजीवनादीनाम् अपात्रीकरणानां जातिभ्रंशकराणां विहितकर्मत्यागादीनां ज्ञानतः सकृत् कृतानाम् अज्ञानतः असकृत् कृतानां सर्वेषां पापानां सद्यः अपनोदनार्थं  
    विनायकादिसमस्तहरिहरदेवतासन्निधौ ... पौर्णमास्याम् अध्यायोपक्रमकर्म करिष्ये । 
    तदङ्गं शरीरशुद्ध्यर्थं शुद्धोदकस्नानम् अहं करिष्ये ।
    
    \twolineshloka*
    {अतिक्रूर महाकाय कल्पान्त दहनोपम}
     {भैरवाय नमस्तुभ्यम् अनुज्ञां दातुम् अर्हसि}

    
\sect{यज्ञोपवीत-धारणम्}

\sect{काण्डऋषि तर्पणम्}

\sect{वेदव्यास (काण्डऋषि पूजा)}

\sect{होमम्}

\sect{वेदारम्भम्}