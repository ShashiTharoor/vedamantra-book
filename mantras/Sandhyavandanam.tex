% !TeX program = XeLaTeX
% !TeX root = ../vedamantrabook.tex
%ꣳꣳ॑ꣳ॒ꣴꣴ॒ꣴ॑
\setmainfont[Script=Devanagari,Mapping=tex-text,Mapping=devanagarinumerals,AutoFakeBold=2.0]{Siddhanta}
\title{\Huge यजुर्वेद-सन्ध्यावन्दनम्}
\date{}
\maketitle
% \mbox{}\\\thispagestyle{empty}
% \clearpage
\tableofcontents 

\clearpage
\chapt{प्रातः सन्ध्यावन्दनम्}
\renewcommand{\sectionmark}[1]{%
\markboth{\large #1 (प्रातः सन्ध्या)}{}}
\sect{आचमनम्}

(कुक्कुटासने)

अच्युताय नमः। अनन्ताय नमः। गोविन्दाय नमः। 

द्विः परिमृज्य।

\dnsub{अङ्गवन्दनम्}

\begin{enumerate}
    \item \makebox[3em][l]{केशव} {\scriptsize (touch right cheek with right thumb)}
    \item \makebox[3em][l]{नारायण} {\scriptsize (touch left cheek with right thumb)}
    \item \makebox[3em][l]{माधव} {\scriptsize (touch right eye with right ring finger)}
    \item \makebox[3em][l]{गोविन्द} {\scriptsize (touch left eye with right ring finger)}
    \item \makebox[3em][l]{विष्णो} {\scriptsize (touch right nostril with right index finger)}
    \item \makebox[3em][l]{मधुसूदन} {\scriptsize (touch left nostril with right index finger)}
    \item \makebox[3em][l]{त्रिविक्रम} {\scriptsize (touch right ear with right little finger)}
    \item \makebox[3em][l]{वामन} {\scriptsize (touch left ear with right little finger)}
    \item \makebox[3em][l]{श्रीधर} {\scriptsize (touch right shoulder with right middle finger)}
    \item \makebox[3em][l]{हृषीकेश} {\scriptsize (touch left shoulder with right middle finger)}
    \item \makebox[3em][l]{पद्मनाभ} {\scriptsize (touch navel with right four fingers)}
    \item \makebox[3em][l]{दामोदर} {\scriptsize (touch the centre of the head with all five fingers)}
\end{enumerate}


\sect{विघ्नेश्वर-ध्यानम्}

भृगुः—अङ्गुलीपृष्ठभागाभ्यां कुट्टणं पञ्चवारकम्।

{\scriptsize (strike gently on the temples five times with the back side of the fingers)}

\twolineshloka*
{शुक्लाम्बरधरं विष्णुं शशिवर्णं चतुर्भुजम्}
{प्रसन्नवदनं ध्यायेत् सर्वविघ्नोपशान्तये}

\sect{प्राणायामः}

ओं भूः। ओं भुव। ओꣳ सुव। ओं मह। ओं जन। ओं तप। ओꣳ स॒त्यम्॥
ओं तथ्स॑वि॒तुर्वरे᳚ण्यं॒ भर्गो॑ दे॒वस्य॑ धीमहि। धियो॒ यो नः॑ प्रचो॒दया᳚त्॥
ओमापो॒ ज्योती॒रसो॒ऽमृतं॒ ब्रह्म॒ भूर्भुवः॒ सुव॒रोम्॥

\sect{सङ्कल्पः}

ममोपात्त समस्त दुरितक्षयद्वारा श्री परमेश्वर प्रीत्यर्थं प्रातः सन्ध्यामुपासिष्ये।

\sect{मार्जनम्}

ॐ श्री केशवाय नमः।

आपो॒ हि ष्ठा म॑यो॒ भुवः॑।\\
ता न॑ ऊ॒र्जे द॑धातन।\\
म॒हेरणा॑य॒ चक्ष॑से।\\
यो वः॑ शि॒वत॑मो॒ रसः॑।\\
तस्य॑ भाजयते॒ ह नः॑।\\
उ॒श॒तीरि॑व मा॒तरः॑।\\
तस्मा॒ अरं॑ गमाम वः।\\
यस्य॒ क्षया॑य॒ जिन्व॑थ।\\
आपो॑ ज॒नय॑था च नः॥\\

ओं भूर्भुवः॒ सुवः॑॥ (आत्म-परिषेचनम्)

\sect{प्राशनम्}


सूर्यश्च मा मन्युश्च मन्युपतयश्च मन्यु॑कृते॒भ्यः। पापेभ्यो॑ रक्ष॒न्ताम्। यद्रात्रिया पाप॑मका॒रिषम्। मनसा वाचा॑ हस्ता॒भ्याम्। पद्भ्यामुदरे॑ण शि॒श्ञा। रात्रि॒स्तद॑वलु॒म्पतु। यत्किं च॑ दुरि॒तं मयि॑। इदमहं माममृ॑तयो॒नौ। सूर्ये ज्योतिषि जुहो॑मि स्वा॒हा॥

\sect{पुनर्मार्जनम्}

\textbf{आचमनम्।}

द॒धि॒क्राव्ण्णो॑ अकारिषम्।\\
 जि॒ष्णोरश्व॑स्य वा॒जिनः॑।\\
सु॒र॒भि नो॒ मुखा॑कर॒त्।\\
प्रण॒ आयूꣳ॑षि तारिषत्॥\\



आपो॒ हि ष्ठा म॑यो॒ भुवः॑।\\
ता न॑ ऊ॒र्जे द॑धातन।\\
म॒हेरणा॑य॒ चक्ष॑से।\\
यो वः॑ शि॒वत॑मो॒ रसः॑।\\
तस्य॑ भाजयते॒ ह नः॑।\\
उ॒श॒तीरि॑व मा॒तरः॑।\\
तस्मा॒ अरं॑ गमाम वः।\\
यस्य॒ क्षया॑य॒ जिन्व॑थ।\\
आपो॑ ज॒नय॑था च नः॥\\

ओं भूर्भुवः॒ सुवः॑॥ (आत्म-परिषेचनम्)


\sect{अर्घ्यप्रदानम्}

ओं भूर्भुवः॒ सुवः॑। तथ्स॑वि॒तुर्वरे᳚ण्यं॒ भर्गो॑ दे॒वस्य॑ धीमहि। धियो॒ यो नः॑ प्रचो॒दया᳚त्॥

\hfill{(एवं त्रिः)}

\sect{प्रायश्चित्तार्घ्यम्}

\textbf{प्राणायामः॥}

(ममोपात्त समस्त दुरितक्षयद्वारा श्री परमेश्वर प्रीत्यर्थं प्रातः सन्ध्या)
कालातीतप्रायश्चित्तार्थम् अर्घ्यप्रदानम् करिष्ये॥

ओं भूर्भुवः॒ सुवः॑। तथ्स॑वि॒तुर्वरे᳚ण्यं॒ भर्गो॑ दे॒वस्य॑ धीमहि। धियो॒ यो नः॑ प्रचो॒दया᳚त्॥

(आत्मप्रदक्षिणं परिषेचनं च)

\sect{ऐक्यानुसन्धानम्}

असावादित्यो ब्रह्म। ब्रह्मैवाहमस्मि॥

ध्यानम्॥

\textbf{आचमनम्।}


\sect{देवतर्पणम्}

\dnsub{नवग्रहदेवता-तर्पणम्}
\begin{enumerate}
 \item आदित्यं तर्पयामि।
 \item सोमं तर्पयामि।
 \item अङ्गारकं तर्पयामि।
 \item बुधं तर्पयामि।
 \item बृहस्पतिं तर्पयामि।
 \item शुक्रं तर्पयामि।
 \item शनैश्चरं तर्पयामि।
 \item राहुं तर्पयामि।
 \item केतुं तर्पयामि।
\end{enumerate}

\dnsub{केशवादि-तर्पणम्}

\begin{enumerate}
\item केशवं तर्पयामि।
\item नारायणं तर्पयामि।
\item माधवं तर्पयामि।
\item गोविन्दं तर्पयामि।
\item विष्णुं तर्पयामि।
\item मधुसूदनं तर्पयामि।
\item त्रिविक्रमं तर्पयामि।
\item वामनं तर्पयामि।
\item श्रीधरं तर्पयामि।
\item हृषीकेशं तर्पयामि।
\item पद्मनाभं तर्पयामि।
\item दामोदरं तर्पयामि।
\end{enumerate}

\textbf{आचमनम्।}


\centerline{॥इति प्रातः सन्ध्यावन्दन-पूर्वभागः॥}

\dnsub{सन्ध्यावन्दन-उत्तरभागः}

\sect{जप-सङ्कल्पः}

\twolineshloka*
{शुक्लाम्बरधरं विष्णुं शशिवर्णं चतुर्भुजम्}
{प्रसन्नवदनं ध्यायेत् सर्वविघ्नोपशान्तये}

\textbf{प्राणायामः।}

ममोपात्त समस्त दुरितक्षयद्वारा श्री परमेश्वर प्रीत्यर्थं प्रातः सन्ध्या-गायत्री-महामन्त्र-जपं करिष्ये।


\sect{प्रणवजपः—प्राणायामः}
प्रणवस्य ऋषिर्ब्रह्मा।
देवी गायत्री छन्दः।
परमात्मा देवता।

भूरादिसप्त व्याहृतीनाम् अत्रि-भृगु-कुत्स-वसिष्ठ-गौतम-काश्यप-आङ्गिरस ऋषयः।

गायत्री-उष्णिक्-अनुष्टुप्-बृहती-पङ्क्ती-त्रिष्टुप्-जगत्यः छन्दांसि।

अग्नि-वायु-अर्क-वागीश-वरुण-इन्द्र-विश्वेदेवा देवताः।
      
प्राणायामे विनियोगः॥


ओं भूः। ओं भुव। ओꣳ सुव। ओं मह। ओं जन। ओं तप। ओꣳ स॒त्यम्॥
ओं तथ्स॑वि॒तुर्वरे᳚ण्यं॒ भर्गो॑ दे॒वस्य॑ धीमहि। धियो॒ यो नः॑ प्रचो॒दया᳚त्॥
ओमापो॒ ज्योती॒रसो॒ऽमृतं॒ ब्रह्म॒ भूर्भुवः॒ सुव॒रोम्॥



\sect{गायत्री-आवाहनम्}

आयात्वित्यनुवाकस्य वामदेव ऋषिः।
अनुष्टुप् छन्दः।
गायत्री देवता।

आया॑तु॒ वर॑दा दे॒वी॒ अ॒क्षरं॑ ब्रह्म॒सम्मि॑तम्। गा॒य॒त्रीं᳚ छन्द॑सां मा॒तेदं ब्र॑ह्म जु॒षस्व॑ नः॥

ओजो॑ऽसि॒ सहो॑ऽसि॒ बल॑मसि॒ भ्राजो॑ऽसि दे॒वानां॒ धाम॒ नामा॑सि॒ विश्व॑मसि वि॒श्वायुः॒ सर्व॑मसि स॒र्वायुरभिभूरों गायत्रीमावा॑हया॒मि॒ सावित्रीमावा॑हया॒मि॒ सरस्वतीमावा॑ह\-या॒मि॒ सावित्र्या ऋषिर्विश्वामित्रः। निचृद्गायत्री छन्दः। सविता देवता।

गायत्री-जपे विनियोगः॥

\sect{गायत्री-जपः}

\dnsub{ध्यानम्}

\fourlineindentedshloka*
{मुक्ता-विद्रुम-हेम-नील-धवळच्छायैर्मुखैस्त्र्यक्षणैः}
{युक्तामिन्दु-निबद्ध-रत्न-मकुटां तत्त्वार्थ-वर्णात्मिकाम्}
{गायत्रीं वरदाभयाङ्कुशकशाः शुभ्रं कपालं गदाम्}
{शङ्खं चक्रमथारविन्दयुगलं हस्तैर्वहन्तीं भजे}

\twolineshloka*
{यो देवः सविताऽस्माकं धियो धर्माधि-गोचरः}
{प्रेरयेत् तस्य यद्भर्गस्तद्वरेण्यमुपास्महे}


ओं।\\
भूर्भुवः॒ सुवः॑।\\
तथ्स॑वि॒तुर्वरे᳚ण्यम्।\\
भर्गो॑ दे॒वस्य॑ धीमहि।\\
धियो॒ यो नः॑ प्रचो॒दया᳚त्॥

\textbf{प्राणायामः।}

\sect{गायत्री-उपस्थानम्}

प्रातः सन्ध्या गायत्री उपस्थानम करिष्ये।

उ॒त्तमे शिख॑रे दे॒वी॒ भू॒म्यां प॑र्वत॒मूर्ध॑नि।\\
ब्रा॒ह्मणे॑॑भ्यो ह्य॑नुज्ञा॒नं॒ ग॒च्छ दे॑वि य॒था सु॑खम्॥


\sect{प्रातः सन्ध्या सूर्योपस्थानम्}
मि॒त्र॒स्य॑ चर्\mbox{}षणी॒ धृतः॒ श्रवो॑ दे॒वस्य॑ सान॒सिम्। स॒त्यं चि॒त्रश्र॑वस्तमम्॥ मि॒त्रो जनान्॑ यातयति प्रजा॒नन् मि॒त्रो दा॑धार पृथि॒वीमु॒तद्याम्। मि॒त्रः कृ॒ष्टीरनि॑मिषा॒भिच॑ष्टे स॒त्याय॑ ह॒व्यं घृ॒तव॑द्विधेम॥ प्र समि॑त्र॒ मर्तो॑ अस्तु॒ प्रय॑स्वा॒न् यस्त॑ आदित्य॒ शिक्ष॑ति व्र॒तेन॑। न ह॑न्यते॒ न जी॑यते॒ त्वोतो॒ नैन॒मꣳहो॑ अश्नो॒त्यन्ति॑तो॒ न दू॒रात्॥


\sect{समष्ट्यभिवादनम्}

सन्ध्यायै नमः।  {\scriptsize (East)}\\
सावित्र्यै नमः। {\scriptsize (South)}\\
गायत्र्यै नमः।  {\scriptsize (West)}\\
सरस्वत्यै नमः।  {\scriptsize (North)}

सर्वाभ्यो देवताभ्यो नमो नमः। {\scriptsize (East)}

कामोऽकार्\mbox{}षी᳚न्मन्युरकार्\mbox{}षी᳚न्नमो॒ नमः।

अभिवादये ( ) (त्रयार्षेय) प्रवरान्वित ( ) गोत्रः\\
(आपस्तम्ब) सूत्रः यजुःशाखा अध्यायी\\
() शर्मा नामाहम् अस्मि भोः॥

\textbf{नमस्कारः।}

\sect{दिग्देवता-वन्दनम्}

प्राच्यै दिशे नमः।   {\scriptsize (East)}\\
दक्षिणायै दिशे नमः।  {\scriptsize (South)}\\
प्रतीच्यै दिशे नमः।   {\scriptsize (West)}\\
उदीच्यै दिशे नमः।   {\scriptsize (North)}\\
ऊर्ध्वाय नमः।   {\scriptsize (up)}\\
अधराय नमः।   {\scriptsize (down)}\\
अन्तरिक्षाय नमः। {\scriptsize (up)}\\
भूम्यै नमः। {\scriptsize (down)}\\
ब्रह्मणे नमः। {\scriptsize (up)}\\
विष्णवे नमः।  {\scriptsize (down)}\\
मृत्यवे नमः।

\sect{यम-वन्दनम्}
यमाय नमः   {\scriptsize (South)}

\twolineshloka*
{यमाय   धर्मराजाय   मृत्यवे   चान्तकाय   च}
{वैवस्वताय   कालाय   सर्वभूतक्षयाय   च}

\twolineshloka*
{औदुम्बराय   दध्नाय   नीलाय   परमेष्ठिने}
{वृकोदराय   चित्राय   चित्रगुप्ताय   वै  नमः}

चित्रगुप्ताय   वै  नम ओं नम इति॥

\sect{सर्प-रक्षा}
 {\scriptsize (North)}
\twolineshloka*
{नर्मदायै नमः प्रातर्नर्मदायै नमो निशि}
{नमोऽस्तु नर्मदे तुभ्यं त्राहि मां विषसर्पतः}

\twolineshloka*
{सर्पापसर्प भद्रं ते गच्छ सर्प महाविष}
{जनमेजयस्य यज्ञान्ते आस्तीकवचनं स्मरन्}


\twolineshloka*
{जरत्कारोर्जरत्कार्वां समुत्पन्नो महायशाः}
{अस्तीकः सत्यसन्धो मां पन्नगेभ्योऽभिरक्षतु}

पन्नगेभ्योऽभिरक्षत्वोन्नम इति॥

\sect{हरिहर-वन्दनम्}
 {\scriptsize (North)}

ऋ॒तꣳ स॒त्यं प॑रं ब्र॒ह्म॒ पु॒रुषं॑ कृष्ण॒पिङ्ग॑लम्।\\
ऊ॒र्ध्वरे॑तं वि॑रूपा॒क्षं॒ वि॒श्वरू॑पाय॒ वै नमो॒ नमः॑॥


वि॒श्वरू॑पाय॒ वै नम ओं नम इति॥
\sect{सूर्यनारायण-वन्दनम्}
{\scriptsize (East)}

\fourlineindentedshloka*
{नमः सवित्रे जगदेकचक्षुषे}
{जगत्प्रसूति-स्थिति-नाश-हेतवे}
{त्रयीमयाय त्रिगुणात्मधारिणे}
{विरिञ्चि-नारायण-शङ्करात्मने}

\fourlineindentedshloka*
{ध्येयः सदा सवितृमण्डल-मध्यवर्ती}
{नारायणः सरसि-जासन-सन्निविष्टः}
{केयूरवान् मकरकुण्डलवान् किरीटी}
{हारी हिरण्मयवपुर्धृतशङ्खचक्रः}

\twolineshloka*
{शङ्ख-चक्र-गदापाणे द्वारकानिलयाच्युत}
{गोविन्द पुण्डरीकाक्ष रक्ष मां शरणागतम्}

\twolineshloka*
{आकाशात् पतितं तोयं यथा गच्छति सागरम्}
{सर्वदेवनमस्कारः केशवं प्रतिगच्छति}

श्री केशवं प्रतिगच्छत्यों नम इति॥

अभिवादये ( ) (त्रयार्षेय) प्रवरान्वित ( ) गोत्रः\\
(आपस्तम्ब) सूत्रः यजुःशाखा अध्यायी\\
() शर्मा नामाहम् अस्मि भोः॥

\textbf{नमस्कारः।}

\sect{समर्पणम्}
\fourlineindentedshloka*
{कायेन वाचा मनसेन्द्रियैर्वा}
{बुद्‌ध्याऽऽत्मना वा प्रकृतेः स्वभावात्}
{करोमि यद्यत् सकलं परस्मै}
{नारायणायेति समर्पयामि}

\textbf{आचमनम्।}

\sect{रक्षा}

अ॒द्या नो॑ देव सवितः प्र॒जाव॑थ्सावीः॒ सौभ॑गम्।\\
परा॑ दु॒ष्वप्नि॑यꣳ सुव।\\
विश्वा॑नि देव सवितर्दुरि॒तानि॒ परा॑ सुव।\\
यद्भ॒द्रं तन्म॒ आ सु॑व।

\centerline{॥इति प्रातः सन्ध्यावन्दन-उत्तरभागः॥}


\chapt{माध्याह्निकम्}
\renewcommand{\sectionmark}[1]{%
\markboth{\large #1 (माध्याह्निकम्)}{}}

\sect{आचमनम्}

(कुक्कुटासने)

अच्युताय नमः। अनन्ताय नमः। गोविन्दाय नमः। 

द्विः परिमृज्य।

\dnsub{अङ्गवन्दनम्}

\begin{enumerate}
    \item \makebox[3em][l]{केशव} {\scriptsize (touch right cheek with right thumb)}
    \item \makebox[3em][l]{नारायण} {\scriptsize (touch left cheek with right thumb)}
    \item \makebox[3em][l]{माधव} {\scriptsize (touch right eye with right ring finger)}
    \item \makebox[3em][l]{गोविन्द} {\scriptsize (touch left eye with right ring finger)}
    \item \makebox[3em][l]{विष्णो} {\scriptsize (touch right nostril with right index finger)}
    \item \makebox[3em][l]{मधुसूदन} {\scriptsize (touch left nostril with right index finger)}
    \item \makebox[3em][l]{त्रिविक्रम} {\scriptsize (touch right ear with right little finger)}
    \item \makebox[3em][l]{वामन} {\scriptsize (touch left ear with right little finger)}
    \item \makebox[3em][l]{श्रीधर} {\scriptsize (touch right shoulder with right middle finger)}
    \item \makebox[3em][l]{हृषीकेश} {\scriptsize (touch left shoulder with right middle finger)}
    \item \makebox[3em][l]{पद्मनाभ} {\scriptsize (touch navel with right four fingers)}
    \item \makebox[3em][l]{दामोदर} {\scriptsize (touch the centre of the head with all five fingers)}
\end{enumerate}


\sect{विघ्नेश्वर-ध्यानम्}

भृगुः—अङ्गुलीपृष्ठभागाभ्यां कुट्टणं पञ्चवारकम्।

{\scriptsize (strike gently on the temples five times with the back side of the fingers)}

\twolineshloka*
{शुक्लाम्बरधरं विष्णुं शशिवर्णं चतुर्भुजम्}
{प्रसन्नवदनं ध्यायेत् सर्वविघ्नोपशान्तये}

\sect{प्राणायामः}

ओं भूः। ओं भुव। ओꣳ सुव। ओं मह। ओं जन। ओं तप। ओꣳ स॒त्यम्॥
ओं तथ्स॑वि॒तुर्वरे᳚ण्यं॒ भर्गो॑ दे॒वस्य॑ धीमहि। धियो॒ यो नः॑ प्रचो॒दया᳚त्॥
ओमापो॒ ज्योती॒रसो॒ऽमृतं॒ ब्रह्म॒ भूर्भुवः॒ सुव॒रोम्॥

\sect{सङ्कल्पः}

ममोपात्त समस्त दुरितक्षयद्वारा श्री परमेश्वर प्रीत्यर्थं माध्याह्निकं करिष्ये।

\sect{मार्जनम्}

ॐ श्री केशवाय नमः।

आपो॒ हि ष्ठा म॑यो॒ भुवः॑।\\
ता न॑ ऊ॒र्जे द॑धातन।\\
म॒हेरणा॑य॒ चक्ष॑से।\\
यो वः॑ शि॒वत॑मो॒ रसः॑।\\
तस्य॑ भाजयते॒ ह नः॑।\\
उ॒श॒तीरि॑व मा॒तरः॑।\\
तस्मा॒ अरं॑ गमाम वः।\\
यस्य॒ क्षया॑य॒ जिन्व॑थ।\\
आपो॑ ज॒नय॑था च नः॥\\

ओं भूर्भुवः॒ सुवः॑॥ (आत्म-परिषेचनम्)

\sect{प्राशनम्}

आपः॑ पुनन्तु पृथि॒वीं पृ॑थि॒वी पू॒ता पु॑नातु॒ माम्। पु॒नन्तु॒ ब्रह्म॑ण॒स्पति॒र्ब्रह्म॑पू॒ता पु॑नातु॒ माम्। यदुच्छि॑ष्ट॒मभो᳚ज्यं॒ यद्वा॑ दु॒श्चरि॑तं॒ मम॑। सर्वं॑ पुनन्तु॒ मामापो॑ऽस॒तां च॑ प्रति॒ग्रह॒ꣴ॒ स्वाहा᳚॥

\sect{पुनर्मार्जनम्}

\textbf{आचमनम्।}

द॒धि॒क्राव्ण्णो॑ अकारिषम्।\\
 जि॒ष्णोरश्व॑स्य वा॒जिनः॑।\\
सु॒र॒भि नो॒ मुखा॑कर॒त्।\\
प्रण॒ आयूꣳ॑षि तारिषत्॥\\



आपो॒ हि ष्ठा म॑यो॒ भुवः॑।\\
ता न॑ ऊ॒र्जे द॑धातन।\\
म॒हेरणा॑य॒ चक्ष॑से।\\
यो वः॑ शि॒वत॑मो॒ रसः॑।\\
तस्य॑ भाजयते॒ ह नः॑।\\
उ॒श॒तीरि॑व मा॒तरः॑।\\
तस्मा॒ अरं॑ गमाम वः।\\
यस्य॒ क्षया॑य॒ जिन्व॑थ।\\
आपो॑ ज॒नय॑था च नः॥\\

ओं भूर्भुवः॒ सुवः॑॥ (आत्म-परिषेचनम्)


\sect{अर्घ्यप्रदानम्}

ओं भूर्भुवः॒ सुवः॑। तथ्स॑वि॒तुर्वरे᳚ण्यं॒ भर्गो॑ दे॒वस्य॑ धीमहि। धियो॒ यो नः॑ प्रचो॒दया᳚त्॥

\hfill{(एवं द्विः)}

\sect{प्रायश्चित्तार्घ्यम्}

\textbf{प्राणायामः॥}

(ममोपात्त समस्त दुरितक्षयद्वारा श्री परमेश्वर प्रीत्यर्थं माध्याह्निक)
कालातीतप्रायश्चित्तार्थम् अर्घ्यप्रदानम् करिष्ये॥

ओं भूर्भुवः॒ सुवः॑। तथ्स॑वि॒तुर्वरे᳚ण्यं॒ भर्गो॑ दे॒वस्य॑ धीमहि। धियो॒ यो नः॑ प्रचो॒दया᳚त्॥

(आत्मप्रदक्षिणं परिषेचनं च)

\sect{ऐक्यानुसन्धानम्}

असावादित्यो ब्रह्म। ब्रह्मैवाहमस्मि॥

ध्यानम्॥

\textbf{आचमनम्।}


\sect{देवतर्पणम्}

\dnsub{नवग्रहदेवता-तर्पणम्}
\begin{enumerate}
 \item आदित्यं तर्पयामि।
 \item सोमं तर्पयामि।
 \item अङ्गारकं तर्पयामि।
 \item बुधं तर्पयामि।
 \item बृहस्पतिं तर्पयामि।
 \item शुक्रं तर्पयामि।
 \item शनैश्चरं तर्पयामि।
 \item राहुं तर्पयामि।
 \item केतुं तर्पयामि।
\end{enumerate}

\dnsub{केशवादि-तर्पणम्}

\begin{enumerate}
\item केशवं तर्पयामि।
\item नारायणं तर्पयामि।
\item माधवं तर्पयामि।
\item गोविन्दं तर्पयामि।
\item विष्णुं तर्पयामि।
\item मधुसूदनं तर्पयामि।
\item त्रिविक्रमं तर्पयामि।
\item वामनं तर्पयामि।
\item श्रीधरं तर्पयामि।
\item हृषीकेशं तर्पयामि।
\item पद्मनाभं तर्पयामि।
\item दामोदरं तर्पयामि।
\end{enumerate}

\textbf{आचमनम्।}


\centerline{॥इति माध्याह्निक-पूर्वभागः॥}

\dnsub{सन्ध्यावन्दन-उत्तरभागः}

\sect{जप-सङ्कल्पः}

\twolineshloka*
{शुक्लाम्बरधरं विष्णुं शशिवर्णं चतुर्भुजम्}
{प्रसन्नवदनं ध्यायेत् सर्वविघ्नोपशान्तये}

\textbf{प्राणायामः।}

ममोपात्त समस्त दुरितक्षयद्वारा श्री परमेश्वर प्रीत्यर्थं माध्याह्निक-गायत्री-महामन्त्र-जपं करिष्ये।


\sect{प्रणवजपः—प्राणायामः}
प्रणवस्य ऋषिर्ब्रह्मा।
देवी गायत्री छन्दः।
परमात्मा देवता।

भूरादिसप्त व्याहृतीनाम् अत्रि-भृगु-कुत्स-वसिष्ठ-गौतम-काश्यप-आङ्गिरस ऋषयः।

गायत्री-उष्णिक्-अनुष्टुप्-बृहती-पङ्क्ती-त्रिष्टुप्-जगत्यः छन्दांसि।

अग्नि-वायु-अर्क-वागीश-वरुण-इन्द्र-विश्वेदेवा देवताः।
      
प्राणायामे विनियोगः॥


ओं भूः। ओं भुव। ओꣳ सुव। ओं मह। ओं जन। ओं तप। ओꣳ स॒त्यम्॥
ओं तथ्स॑वि॒तुर्वरे᳚ण्यं॒ भर्गो॑ दे॒वस्य॑ धीमहि। धियो॒ यो नः॑ प्रचो॒दया᳚त्॥
ओमापो॒ ज्योती॒रसो॒ऽमृतं॒ ब्रह्म॒ भूर्भुवः॒ सुव॒रोम्॥



\sect{गायत्री-आवाहनम्}

आयात्वित्यनुवाकस्य वामदेव ऋषिः।
अनुष्टुप् छन्दः।
गायत्री देवता।

आया॑तु॒ वर॑दा दे॒वी॒ अ॒क्षरं॑ ब्रह्म॒सम्मि॑तम्। गा॒य॒त्रीं᳚ छन्द॑सां मा॒तेदं ब्र॑ह्म जु॒षस्व॑ नः॥

ओजो॑ऽसि॒ सहो॑ऽसि॒ बल॑मसि॒ भ्राजो॑ऽसि दे॒वानां॒ धाम॒ नामा॑सि॒ विश्व॑मसि वि॒श्वायुः॒ सर्व॑मसि स॒र्वायुरभिभूरों गायत्रीमावा॑हया॒मि॒ सावित्रीमावा॑हया॒मि॒ सरस्वतीमावा॑ह\-या॒मि॒ सावित्र्या ऋषिर्विश्वामित्रः। निचृद्गायत्री छन्दः। सविता देवता।

गायत्री-जपे विनियोगः॥

\sect{गायत्री-जपः}

\dnsub{ध्यानम्}

\fourlineindentedshloka*
{मुक्ता-विद्रुम-हेम-नील-धवळच्छायैर्मुखैस्त्र्यक्षणैः}
{युक्तामिन्दु-निबद्ध-रत्न-मकुटां तत्त्वार्थ-वर्णात्मिकाम्}
{गायत्रीं वरदाभयाङ्कुशकशाः शुभ्रं कपालं गदाम्}
{शङ्खं चक्रमथारविन्दयुगलं हस्तैर्वहन्तीं भजे}

\twolineshloka*
{यो देवः सविताऽस्माकं धियो धर्माधि-गोचरः}
{प्रेरयेत् तस्य यद्भर्गस्तद्वरेण्यमुपास्महे}


ओं।\\
भूर्भुवः॒ सुवः॑।\\
तथ्स॑वि॒तुर्वरे᳚ण्यम्।\\
भर्गो॑ दे॒वस्य॑ धीमहि।\\
धियो॒ यो नः॑ प्रचो॒दया᳚त्॥

\textbf{प्राणायामः।}

\sect{गायत्री-उपस्थानम्}

आदित्य उपस्थानं करिष्ये।

उ॒त्तमे शिख॑रे दे॒वी॒ भू॒म्यां प॑र्वत॒मूर्ध॑नि।\\
ब्रा॒ह्मणे॑॑भ्यो ह्य॑नुज्ञा॒नं॒ ग॒च्छ दे॑वि य॒था सु॑खम्॥



\sect{माध्याह्निक सूर्योपस्थानम्}
आ स॒त्येन॒ रज॑सा॒ वर्त॑मानो निवे॒शय॑न्न॒मृतं॒ मर्त्यं॑ च। हि॒र॒ण्यये॑न सवि॒ता रथे॒नादे॒वो या॑ति॒ भुव॑ना वि॒पश्यन्॑। उद्व॒यं तम॑स॒स्परि॒ पश्य॑न्तो॒ ज्योति॒रुत्त॑रम्। दे॒वं दे॑व॒त्रा सूर्य॒मग॑न्म॒ ज्योति॑रुत्त॒मम्। उदु॒त्यं जा॒तवे॑दसं दे॒वं व॑हन्ति के॒तवः॑। दृ॒शे विश्वा॑य॒ सूर्यम्। चि॒त्रं दे॒वाना॒मुद॑गा॒दनी॑कं॒ चक्षु॑र्मि॒त्रस्य॒ वरु॑णस्या॒ग्नेः। आ प्रा॒ द्यावा॑ पृथि॒वी अ॒न्तरि॑क्ष॒ꣳ॒ सूर्य॑ आ॒त्मा जग॑तस्त॒स्थुष॑श्च। तच्चक्षु॑र्दे॒वहि॑तं पु॒रस्ताच्छु॒क्रमु॒च्चर॑त्॥

पश्ये॑म श॒रदः॑ श॒तं जीवे॑म श॒रदः॑ श॒तं नन्दा॑म श॒रदः॑ श॒तं मोदा॑म श॒रदः॑ श॒तं भवा॑म श॒रदः॑ श॒तꣳ शृ॒णवा॑म श॒रदः॑ श॒तं प्रब्र॑वाम श॒रदः॑ श॒तमजी॑ताः स्याम श॒रदः॑ श॒तं ज्योक्च॒ सूर्यं॑ दृ॒शे। य उद॑गान्मह॒तोर्णवाद्वि॒भ्राज॑मानः सरि॒रस्य॒ मध्या॒थ्स मा॑ वृष॒भो लो॑हिता॒क्षः सूर्यो॑ विप॒श्चिन्मन॑सा पुनातु॥


\sect{समष्ट्यभिवादनम्}

सन्ध्यायै नमः।  {\scriptsize (East)}\\
सावित्र्यै नमः। {\scriptsize (South)}\\
गायत्र्यै नमः।  {\scriptsize (West)}\\
सरस्वत्यै नमः।  {\scriptsize (North)}

सर्वाभ्यो देवताभ्यो नमो नमः। {\scriptsize (East)}

कामोऽकार्\mbox{}षी᳚न्मन्युरकार्\mbox{}षी᳚न्नमो॒ नमः।

अभिवादये ( ) (त्रयार्षेय) प्रवरान्वित ( ) गोत्रः\\
(आपस्तम्ब) सूत्रः यजुःशाखा अध्यायी\\
() शर्मा नामाहम् अस्मि भोः॥

\textbf{नमस्कारः।}

\sect{दिग्देवता-वन्दनम्}

प्राच्यै दिशे नमः।   {\scriptsize (East)}\\
दक्षिणायै दिशे नमः।  {\scriptsize (South)}\\
प्रतीच्यै दिशे नमः।   {\scriptsize (West)}\\
उदीच्यै दिशे नमः।   {\scriptsize (North)}\\
ऊर्ध्वाय नमः।   {\scriptsize (up)}\\
अधराय नमः।   {\scriptsize (down)}\\
अन्तरिक्षाय नमः। {\scriptsize (up)}\\
भूम्यै नमः। {\scriptsize (down)}\\
ब्रह्मणे नमः। {\scriptsize (up)}\\
विष्णवे नमः।  {\scriptsize (down)}\\
मृत्यवे नमः।

\sect{यम-वन्दनम्}
यमाय नमः   {\scriptsize (South)}

\twolineshloka*
{यमाय   धर्मराजाय   मृत्यवे   चान्तकाय   च}
{वैवस्वताय   कालाय   सर्वभूतक्षयाय   च}

\twolineshloka*
{औदुम्बराय   दध्नाय   नीलाय   परमेष्ठिने}
{वृकोदराय   चित्राय   चित्रगुप्ताय   वै  नमः}

चित्रगुप्ताय   वै  नम ओं नम इति॥

\sect{सर्प-रक्षा}
 {\scriptsize (North)}
\twolineshloka*
{नर्मदायै नमः प्रातर्नर्मदायै नमो निशि}
{नमोऽस्तु नर्मदे तुभ्यं त्राहि मां विषसर्पतः}

\twolineshloka*
{सर्पापसर्प भद्रं ते गच्छ सर्प महाविष}
{जनमेजयस्य यज्ञान्ते आस्तीकवचनं स्मरन्}


\twolineshloka*
{जरत्कारोर्जरत्कार्वां समुत्पन्नो महायशाः}
{अस्तीकः सत्यसन्धो मां पन्नगेभ्योऽभिरक्षतु}

पन्नगेभ्योऽभिरक्षत्वोन्नम इति॥

\sect{हरिहर-वन्दनम्}
 {\scriptsize (North)}

ऋ॒तꣳ स॒त्यं प॑रं ब्र॒ह्म॒ पु॒रुषं॑ कृष्ण॒पिङ्ग॑लम्।\\
ऊ॒र्ध्वरे॑तं वि॑रूपा॒क्षं॒ वि॒श्वरू॑पाय॒ वै नमो॒ नमः॑॥


वि॒श्वरू॑पाय॒ वै नम ओं नम इति॥
\sect{सूर्यनारायण-वन्दनम्}
{\scriptsize (East)}

\fourlineindentedshloka*
{नमः सवित्रे जगदेकचक्षुषे}
{जगत्प्रसूति-स्थिति-नाश-हेतवे}
{त्रयीमयाय त्रिगुणात्मधारिणे}
{विरिञ्चि-नारायण-शङ्करात्मने}

\fourlineindentedshloka*
{ध्येयः सदा सवितृमण्डल-मध्यवर्ती}
{नारायणः सरसि-जासन-सन्निविष्टः}
{केयूरवान् मकरकुण्डलवान् किरीटी}
{हारी हिरण्मयवपुर्धृतशङ्खचक्रः}

\twolineshloka*
{शङ्ख-चक्र-गदापाणे द्वारकानिलयाच्युत}
{गोविन्द पुण्डरीकाक्ष रक्ष मां शरणागतम्}

\twolineshloka*
{आकाशात् पतितं तोयं यथा गच्छति सागरम्}
{सर्वदेवनमस्कारः केशवं प्रतिगच्छति}

श्री केशवं प्रतिगच्छत्यों नम इति॥

अभिवादये ( ) (त्रयार्षेय) प्रवरान्वित ( ) गोत्रः\\
(आपस्तम्ब) सूत्रः यजुःशाखा अध्यायी\\
() शर्मा नामाहम् अस्मि भोः॥

\textbf{नमस्कारः।}

\sect{समर्पणम्}
\fourlineindentedshloka*
{कायेन वाचा मनसेन्द्रियैर्वा}
{बुद्‌ध्याऽऽत्मना वा प्रकृतेः स्वभावात्}
{करोमि यद्यत् सकलं परस्मै}
{नारायणायेति समर्पयामि}

\textbf{आचमनम्।}

\sect{रक्षा}

अ॒द्या नो॑ देव सवितः प्र॒जाव॑थ्सावीः॒ सौभ॑गम्।\\
परा॑ दु॒ष्वप्नि॑यꣳ सुव।\\
विश्वा॑नि देव सवितर्दुरि॒तानि॒ परा॑ सुव।\\
यद्भ॒द्रं तन्म॒ आ सु॑व।

\centerline{॥इति माध्याह्निकस्-उत्तरभागः॥}


\chapt{सायं सन्ध्यावन्दनम्}
\renewcommand{\sectionmark}[1]{%
\markboth{\large #1 (सायं सन्ध्या)}{}}

{\scriptsize (Begin facing North)}

\sect{आचमनम्}

(कुक्कुटासने)

अच्युताय नमः। अनन्ताय नमः। गोविन्दाय नमः। 

द्विः परिमृज्य।

\dnsub{अङ्गवन्दनम्}

\begin{enumerate}
    \item \makebox[3em][l]{केशव} {\scriptsize (touch right cheek with right thumb)}
    \item \makebox[3em][l]{नारायण} {\scriptsize (touch left cheek with right thumb)}
    \item \makebox[3em][l]{माधव} {\scriptsize (touch right eye with right ring finger)}
    \item \makebox[3em][l]{गोविन्द} {\scriptsize (touch left eye with right ring finger)}
    \item \makebox[3em][l]{विष्णो} {\scriptsize (touch right nostril with right index finger)}
    \item \makebox[3em][l]{मधुसूदन} {\scriptsize (touch left nostril with right index finger)}
    \item \makebox[3em][l]{त्रिविक्रम} {\scriptsize (touch right ear with right little finger)}
    \item \makebox[3em][l]{वामन} {\scriptsize (touch left ear with right little finger)}
    \item \makebox[3em][l]{श्रीधर} {\scriptsize (touch right shoulder with right middle finger)}
    \item \makebox[3em][l]{हृषीकेश} {\scriptsize (touch left shoulder with right middle finger)}
    \item \makebox[3em][l]{पद्मनाभ} {\scriptsize (touch navel with right four fingers)}
    \item \makebox[3em][l]{दामोदर} {\scriptsize (touch the centre of the head with all five fingers)}
\end{enumerate}


\sect{विघ्नेश्वर-ध्यानम्}

भृगुः—अङ्गुलीपृष्ठभागाभ्यां कुट्टणं पञ्चवारकम्।

{\scriptsize (strike gently on the temples five times with the back side of the fingers)}

\twolineshloka*
{शुक्लाम्बरधरं विष्णुं शशिवर्णं चतुर्भुजम्}
{प्रसन्नवदनं ध्यायेत् सर्वविघ्नोपशान्तये}

\sect{प्राणायामः}

ओं भूः। ओं भुव। ओꣳ सुव। ओं मह। ओं जन। ओं तप। ओꣳ स॒त्यम्॥
ओं तथ्स॑वि॒तुर्वरे᳚ण्यं॒ भर्गो॑ दे॒वस्य॑ धीमहि। धियो॒ यो नः॑ प्रचो॒दया᳚त्॥
ओमापो॒ ज्योती॒रसो॒ऽमृतं॒ ब्रह्म॒ भूर्भुवः॒ सुव॒रोम्॥

\sect{सङ्कल्पः}

ममोपात्त समस्त दुरितक्षयद्वारा श्री परमेश्वर प्रीत्यर्थं सायं सन्ध्यामुपासिष्ये।

\sect{मार्जनम्}

ॐ श्री केशवाय नमः।

आपो॒ हि ष्ठा म॑यो॒ भुवः॑।\\
ता न॑ ऊ॒र्जे द॑धातन।\\
म॒हेरणा॑य॒ चक्ष॑से।\\
यो वः॑ शि॒वत॑मो॒ रसः॑।\\
तस्य॑ भाजयते॒ ह नः॑।\\
उ॒श॒तीरि॑व मा॒तरः॑।\\
तस्मा॒ अरं॑ गमाम वः।\\
यस्य॒ क्षया॑य॒ जिन्व॑थ।\\
आपो॑ ज॒नय॑था च नः॥\\

ओं भूर्भुवः॒ सुवः॑॥ (आत्म-परिषेचनम्)

\sect{प्राशनम्}

अग्निश्च मा मन्युश्च मन्युपतयश्च मन्यु॑कृते॒भ्यः। पापेभ्यो॑ रक्ष॒न्ताम्। यदह्ना पाप॑मका॒रिषम्। मनसा वाचा॑ हस्ता॒भ्याम्। पद्भ्यामुदरे॑ण शि॒श्ञा। अह॒स्तद॑वलु॒म्पतु। यत्किं च॑ दुरि॒तं मयि॑। इदमहं माममृ॑तयो॒नौ। सत्ये ज्योतिषि जुहो॑मि स्वा॒हा॥४८॥

\sect{पुनर्मार्जनम्}

\textbf{आचमनम्।}

द॒धि॒क्राव्ण्णो॑ अकारिषम्।\\
 जि॒ष्णोरश्व॑स्य वा॒जिनः॑।\\
सु॒र॒भि नो॒ मुखा॑कर॒त्।\\
प्रण॒ आयूꣳ॑षि तारिषत्॥\\



आपो॒ हि ष्ठा म॑यो॒ भुवः॑।\\
ता न॑ ऊ॒र्जे द॑धातन।\\
म॒हेरणा॑य॒ चक्ष॑से।\\
यो वः॑ शि॒वत॑मो॒ रसः॑।\\
तस्य॑ भाजयते॒ ह नः॑।\\
उ॒श॒तीरि॑व मा॒तरः॑।\\
तस्मा॒ अरं॑ गमाम वः।\\
यस्य॒ क्षया॑य॒ जिन्व॑थ।\\
आपो॑ ज॒नय॑था च नः॥\\

ओं भूर्भुवः॒ सुवः॑॥ (आत्म-परिषेचनम्)


\sect{अर्घ्यप्रदानम्}
{\scriptsize (Turn West)}

ओं भूर्भुवः॒ सुवः॑। तथ्स॑वि॒तुर्वरे᳚ण्यं॒ भर्गो॑ दे॒वस्य॑ धीमहि। धियो॒ यो नः॑ प्रचो॒दया᳚त्॥

\hfill{(एवं त्रिः)}

\sect{प्रायश्चित्तार्घ्यम्}

\textbf{प्राणायामः॥}

(ममोपात्त समस्त दुरितक्षयद्वारा श्री परमेश्वर प्रीत्यर्थं सायं सन्ध्या)
कालातीतप्रायश्चित्तार्थम् अर्घ्यप्रदानम् करिष्ये॥

ओं भूर्भुवः॒ सुवः॑। तथ्स॑वि॒तुर्वरे᳚ण्यं॒ भर्गो॑ दे॒वस्य॑ धीमहि। धियो॒ यो नः॑ प्रचो॒दया᳚त्॥

(आत्मप्रदक्षिणं परिषेचनं च)

\sect{ऐक्यानुसन्धानम्}

असावादित्यो ब्रह्म। ब्रह्मैवाहमस्मि॥

ध्यानम्॥

{\scriptsize (Face North Again)}

\textbf{आचमनम्।}


\sect{देवतर्पणम्}

\dnsub{नवग्रहदेवता-तर्पणम्}
\begin{enumerate}
 \item आदित्यं तर्पयामि।
 \item सोमं तर्पयामि।
 \item अङ्गारकं तर्पयामि।
 \item बुधं तर्पयामि।
 \item बृहस्पतिं तर्पयामि।
 \item शुक्रं तर्पयामि।
 \item शनैश्चरं तर्पयामि।
 \item राहुं तर्पयामि।
 \item केतुं तर्पयामि।
\end{enumerate}

\dnsub{केशवादि-तर्पणम्}

\begin{enumerate}
\item केशवं तर्पयामि।
\item नारायणं तर्पयामि।
\item माधवं तर्पयामि।
\item गोविन्दं तर्पयामि।
\item विष्णुं तर्पयामि।
\item मधुसूदनं तर्पयामि।
\item त्रिविक्रमं तर्पयामि।
\item वामनं तर्पयामि।
\item श्रीधरं तर्पयामि।
\item हृषीकेशं तर्पयामि।
\item पद्मनाभं तर्पयामि।
\item दामोदरं तर्पयामि।
\end{enumerate}

\textbf{आचमनम्।}


\centerline{॥इति सायं सन्ध्यावन्दन-पूर्वभागः॥}

\dnsub{सन्ध्यावन्दन-उत्तरभागः}


\sect{जप-सङ्कल्पः}
{\scriptsize (Face West)}

\twolineshloka*
{शुक्लाम्बरधरं विष्णुं शशिवर्णं चतुर्भुजम्}
{प्रसन्नवदनं ध्यायेत् सर्वविघ्नोपशान्तये}

\textbf{प्राणायामः।}

ममोपात्त समस्त दुरितक्षयद्वारा श्री परमेश्वर प्रीत्यर्थं सायं सन्ध्या-गायत्री-महामन्त्र-जपं करिष्ये।


\sect{प्रणवजपः—प्राणायामः}
प्रणवस्य ऋषिर्ब्रह्मा।
देवी गायत्री छन्दः।
परमात्मा देवता।

भूरादिसप्त व्याहृतीनाम् अत्रि-भृगु-कुत्स-वसिष्ठ-गौतम-काश्यप-आङ्गिरस ऋषयः।

गायत्री-उष्णिक्-अनुष्टुप्-बृहती-पङ्क्ती-त्रिष्टुप्-जगत्यः छन्दांसि।

अग्नि-वायु-अर्क-वागीश-वरुण-इन्द्र-विश्वेदेवा देवताः।
      
प्राणायामे विनियोगः॥


ओं भूः। ओं भुव। ओꣳ सुव। ओं मह। ओं जन। ओं तप। ओꣳ स॒त्यम्॥
ओं तथ्स॑वि॒तुर्वरे᳚ण्यं॒ भर्गो॑ दे॒वस्य॑ धीमहि। धियो॒ यो नः॑ प्रचो॒दया᳚त्॥
ओमापो॒ ज्योती॒रसो॒ऽमृतं॒ ब्रह्म॒ भूर्भुवः॒ सुव॒रोम्॥



\sect{गायत्री-आवाहनम्}

आयात्वित्यनुवाकस्य वामदेव ऋषिः।
अनुष्टुप् छन्दः।
गायत्री देवता।

आया॑तु॒ वर॑दा दे॒वी॒ अ॒क्षरं॑ ब्रह्म॒सम्मि॑तम्। गा॒य॒त्रीं᳚ छन्द॑सां मा॒तेदं ब्र॑ह्म जु॒षस्व॑ नः॥

ओजो॑ऽसि॒ सहो॑ऽसि॒ बल॑मसि॒ भ्राजो॑ऽसि दे॒वानां॒ धाम॒ नामा॑सि॒ विश्व॑मसि वि॒श्वायुः॒ सर्व॑मसि स॒र्वायुरभिभूरों गायत्रीमावा॑हया॒मि॒ सावित्रीमावा॑हया॒मि॒ सरस्वतीमावा॑ह\-या॒मि॒ सावित्र्या ऋषिर्विश्वामित्रः। निचृद्गायत्री छन्दः। सविता देवता।

गायत्री-जपे विनियोगः॥

\sect{गायत्री-जपः}

\dnsub{ध्यानम्}

\fourlineindentedshloka*
{मुक्ता-विद्रुम-हेम-नील-धवळच्छायैर्मुखैस्त्र्यक्षणैः}
{युक्तामिन्दु-निबद्ध-रत्न-मकुटां तत्त्वार्थ-वर्णात्मिकाम्}
{गायत्रीं वरदाभयाङ्कुशकशाः शुभ्रं कपालं गदाम्}
{शङ्खं चक्रमथारविन्दयुगलं हस्तैर्वहन्तीं भजे}

\twolineshloka*
{यो देवः सविताऽस्माकं धियो धर्माधि-गोचरः}
{प्रेरयेत् तस्य यद्भर्गस्तद्वरेण्यमुपास्महे}


ओं।\\
भूर्भुवः॒ सुवः॑।\\
तथ्स॑वि॒तुर्वरे᳚ण्यम्।\\
भर्गो॑ दे॒वस्य॑ धीमहि।\\
धियो॒ यो नः॑ प्रचो॒दया᳚त्॥

\textbf{प्राणायामः।}

\sect{गायत्री-उपस्थानम्}

सायं सन्ध्या गायत्री उपस्थानं करिष्ये।

उ॒त्तमे शिख॑रे दे॒वी॒ भू॒म्यां प॑र्वत॒मूर्ध॑नि।\\
ब्रा॒ह्मणे॑॑भ्यो ह्य॑नुज्ञा॒नं॒ ग॒च्छ दे॑वि य॒था सु॑खम्॥


\sect{सायं सन्ध्या सूर्योपस्थानम्}
इ॒मं मे॑ वरुण श्रुधी॒ हव॑म॒द्या च॑ मृडय। त्वाम॑व॒स्युराच॑के॥ तत्त्वा॑ यामि॒ ब्रह्म॑णा॒ वन्द॑मान॒स्तदाशास्ते॒ यज॑मानो ह॒विर्भिः॑। अहे॑डमानो वरुणे॒ह बो॒ध्युरु॑शꣳस॒ मा न॒ आयुः॒ प्रमो॑षीः॥
यच्चि॒द्धिते॒ विशो॑ यथा॒ प्रदे॑व वरुण व्र॒तम्। मि॒नी॒मसि॒ द्यवि॑द्यवि॥ यत्किं चे॒दं व॑रुण॒ दैव्ये॒ जने॑भिद्रो॒हं मनु॒ष्याश्चरा॑मसि। अचि॑त्ती॒यत्तव॒ धर्मा॑ युयोपि॒म मा न॒स्तस्मा॒देन॑सो देव रीरिषः॥ कि॒त॒वासो॒ यद्रि॑रि॒पुर्नदी॒वि यद्वा॑ घा स॒त्यमु॒त यं न वि॒द्म। सर्वा॒ताविष्य॑ शिथि॒रेव दे॒वाथा॑ ते स्याम वरुण प्रि॒यासः॑॥

\sect{समष्ट्यभिवादनम्}

सन्ध्यायै नमः।  {\scriptsize (West)}\\
सावित्र्यै नमः। {\scriptsize (North)}\\
गायत्र्यै नमः।  {\scriptsize (East)}\\
सरस्वत्यै नमः।  {\scriptsize (South)}

सर्वाभ्यो देवताभ्यो नमो नमः। {\scriptsize (West)}

कामोऽकार्\mbox{}षी᳚न्मन्युरकार्\mbox{}षी᳚न्नमो॒ नमः।

अभिवादये ( ) (त्रयार्षेय) प्रवरान्वित ( ) गोत्रः\\
(आपस्तम्ब) सूत्रः यजुःशाखा अध्यायी\\
() शर्मा नामाहम् अस्मि भोः॥

\textbf{नमस्कारः।}

\sect{दिग्देवता-वन्दनम्}

प्रतीच्यै दिशे नमः।   {\scriptsize (West)}\\
उदीच्यै दिशे नमः।   {\scriptsize (North)}\\
प्राच्यै दिशे नमः।   {\scriptsize (East)}\\
दक्षिणायै दिशे नमः।  {\scriptsize (South)}\\
ऊर्ध्वाय नमः।   {\scriptsize (up)}\\
अधराय नमः।   {\scriptsize (down)}\\
अन्तरिक्षाय नमः। {\scriptsize (up)}\\
भूम्यै नमः। {\scriptsize (down)}\\
ब्रह्मणे नमः। {\scriptsize (up)}\\
विष्णवे नमः।  {\scriptsize (down)}\\
मृत्यवे नमः।

\sect{यम-वन्दनम्}
यमाय नमः   {\scriptsize (South)}

\twolineshloka*
{यमाय   धर्मराजाय   मृत्यवे   चान्तकाय   च}
{वैवस्वताय   कालाय   सर्वभूतक्षयाय   च}

\twolineshloka*
{औदुम्बराय   दध्नाय   नीलाय   परमेष्ठिने}
{वृकोदराय   चित्राय   चित्रगुप्ताय   वै  नमः}

चित्रगुप्ताय   वै  नम ओं नम इति॥

\sect{सर्प-रक्षा}
 {\scriptsize (North)}
\twolineshloka*
{नर्मदायै नमः प्रातर्नर्मदायै नमो निशि}
{नमोऽस्तु नर्मदे तुभ्यं त्राहि मां विषसर्पतः}

\twolineshloka*
{सर्पापसर्प भद्रं ते गच्छ सर्प महाविष}
{जनमेजयस्य यज्ञान्ते आस्तीकवचनं स्मरन्}


\twolineshloka*
{जरत्कारोर्जरत्कार्वां समुत्पन्नो महायशाः}
{अस्तीकः सत्यसन्धो मां पन्नगेभ्योऽभिरक्षतु}

पन्नगेभ्योऽभिरक्षत्वोन्नम इति॥

\sect{हरिहर-वन्दनम्}
 {\scriptsize (North)}

ऋ॒तꣳ स॒त्यं प॑रं ब्र॒ह्म॒ पु॒रुषं॑ कृष्ण॒पिङ्ग॑लम्।\\
ऊ॒र्ध्वरे॑तं वि॑रूपा॒क्षं॒ वि॒श्वरू॑पाय॒ वै नमो॒ नमः॑॥


वि॒श्वरू॑पाय॒ वै नम ओं नम इति॥
\sect{सूर्यनारायण-वन्दनम्}
{\scriptsize (West)}

\fourlineindentedshloka*
{नमः सवित्रे जगदेकचक्षुषे}
{जगत्प्रसूति-स्थिति-नाश-हेतवे}
{त्रयीमयाय त्रिगुणात्मधारिणे}
{विरिञ्चि-नारायण-शङ्करात्मने}

\fourlineindentedshloka*
{ध्येयः सदा सवितृमण्डल-मध्यवर्ती}
{नारायणः सरसि-जासन-सन्निविष्टः}
{केयूरवान् मकरकुण्डलवान् किरीटी}
{हारी हिरण्मयवपुर्धृतशङ्खचक्रः}

\twolineshloka*
{शङ्ख-चक्र-गदापाणे द्वारकानिलयाच्युत}
{गोविन्द पुण्डरीकाक्ष रक्ष मां शरणागतम्}

\twolineshloka*
{आकाशात् पतितं तोयं यथा गच्छति सागरम्}
{सर्वदेवनमस्कारः केशवं प्रतिगच्छति}

श्री केशवं प्रतिगच्छत्यों नम इति॥

अभिवादये ( ) (त्रयार्षेय) प्रवरान्वित ( ) गोत्रः\\
(आपस्तम्ब) सूत्रः यजुःशाखा अध्यायी\\
() शर्मा नामाहम् अस्मि भोः॥

\textbf{नमस्कारः।}


\sect{समर्पणम्}
{\scriptsize (Sit down facing North)}
\fourlineindentedshloka*
{कायेन वाचा मनसेन्द्रियैर्वा}
{बुद्‌ध्याऽऽत्मना वा प्रकृतेः स्वभावात्}
{करोमि यद्यत् सकलं परस्मै}
{नारायणायेति समर्पयामि}

\textbf{आचमनम्।}

\sect{रक्षा}

अ॒द्या नो॑ देव सवितः प्र॒जाव॑थ्सावीः॒ सौभ॑गम्।\\
परा॑ दु॒ष्वप्नि॑यꣳ सुव।\\
विश्वा॑नि देव सवितर्दुरि॒तानि॒ परा॑ सुव।\\
यद्भ॒द्रं तन्म॒ आ सु॑व।

\centerline{॥इति सायं सन्ध्यावन्दन-उत्तरभागः॥}
 

% \clearpage
% \chapt{समिधादानम्}


% \textbf{आचमनम्।}

% \twolineshloka*
% {शुक्लाम्बरधरं विष्णुं शशिवर्णं चतुर्भुजम्}
% {प्रसन्नवदनं ध्यायेत् सर्वविघ्नोपशान्तये}

% \textbf{प्राणायामः।}


% ममोपात्त समस्त दुरितक्षयद्वारा श्री परमेश्वर प्रीत्यर्थं प्रातः (सायं) समिधादानं करिष्ये।

% (लौकिकाग्निं प्रतिष्ठाप्य। अग्निमिध्वा। प्रज्वाल्य।)

% {\scriptsize(Pray to Agni)}

% परि॑त्वाग्ने\\ 
% परि॑मृजामि\\
% आयु॑षा च\\
% धने॑न च



% \clearpage
% \chapt{ब्रह्मयज्ञः}



