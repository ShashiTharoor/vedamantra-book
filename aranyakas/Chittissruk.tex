% !TeX program = XeLaTeX
% !TeX root = ../AraNyakabook-kindle.tex

%  
%  
%॒य्यँ --> ं॒ य
%य्यँ॒ -> ं य॒
% व्वँ -> ं व
% ॒व्वँ -> ं॒ व
% र्् -> र्\mbox{}
% ॒स्स्-> ः॒ स्
% ॑स्स्-> ॑ स्
% ॒स्सु-> ः॒ सु
% ॑स्स-> ॑ स
% ॒स्स-> ः॒ स
% स्स -> ः स
% ->  //
%द्ध्व -> ध्[carefully]
% ंब -> म्ब
% ं॒ब -> ॒म्ब
% ंभ -> म्भ
% ं॒भ -> ॒म्भ
% ॒श्श् -> ः॒ श्
% ॑श्श् -> ॑ श्
% ॒श्श -> ः॒ श
%॑श्श -> ॑ श
%व्वँ॑ -> ंव॑[may need to split word]
%व्वँ॒ -> ंव॒
% न्॒ -> \an{}
% ः ->    [complete]
% ॑ -> ॑  [complete]
%  श्श->  ः  श
% व्वेँ -> ं वे
%॒ञ्च -> ं॒ च [check carefully]
%॑ञ्च -> ं॑ च [check carefully]
%॑ञ्ज -> ं॑ ज [check carefully]
%॒ञ्ज -> ं॒ ज [check carefully]
%फ् -> प्
%ङ्क -> ं क [check carefully]
%॒ङ्क -> ं॒ क [check carefully]
% ॑न्त -> ं॑ त
% ॒न्त
% ङ्ग॑च्छ-> ं ग॑च्छ
% ङ्गच्छ -> ं गच्छ
%न्द्वि॒ -> ं द्वि॒
%॒न्न -> ं॒ न
%॑न्न -> None!
% न्न -> ं न
\sect{तृतीयः प्रश्नः}\setcounter{anuvakam}{0}
ॐ तच्छं॒ योरावृ॑णीमहे। गा॒तुं य॒ज्ञाय॑। गा॒तुं य॒ज्ञप॑तये। 
दैवी स्व॒स्तिर॑स्तु नः। स्व॒स्तिर्मानु॑षेभ्यः। ऊ॒र्ध्वं जि॑गातु भेष॒जम्। 
शं नो॑ अस्तु द्वि॒पदे। शं चतु॑ष्पदे। ॐ शान्ति॒ शान्ति॒ शान्ति॑॥

चित्ति॒ स्रुक्। चि॒त्तमाज्यम्। वाग्वेदि॑। आधी॑तं ब॒\ar{}हिः। केतो॑ अ॒ग्निः। विज्ञा॑तम॒ग्निः। वाक्प॑ति॒र्\mbox{}होता। मन॑ उपव॒क्ता। प्रा॒णो ह॒विः। सामाध्व॒र्युः। वाच॑स्पते विधे नामन्। वि॒धेम॑ ते॒ नाम॑। वि॒धेस्त्वम॒स्माकं॒ नाम॑। वा॒चस्पतिः॒ सोमं॑ पिबतु। आऽस्मासु॑ नृ॒म्णन्धा॒त्स्वाहा॥१॥
\anuvakamend[अ॒ध्व॒र्युः पञ्च॑ च]
पृ॒थि॒वी होता। द्यौर॑ध्व॒र्युः। रु॒द्रोऽग्नीत्। बृह॒स्पति॑रुपव॒क्ता। वाच॑स्पते वा॒चो वी॒र्ये॑ण। सम्भृ॑ततमे॒नाय॑क्ष्यसे। यज॑मानाय॒ वार्यम्। आसुव॒स्कर॑स्मै। वा॒चस्पतिः॒ सोमं॑ पिबतु। ज॒जन॒दिन्द्र॑\-मिन्द्रि॒याय॒ स्वाहा॥२॥%
\anuvakamend[पृ॒थि॒वी होता॒ दश॑]
अ॒ग्निर्\mbox{}होता। अ॒श्विनाऽध्व॒र्यू। त्वष्टा॒ऽग्नीत्। मि॒त्र उ॑पव॒क्ता। सोमः॒ सोम॑स्य पुरो॒गाः। शु॒क्रः  शु॒क्रस्य॑ पुरो॒गाः। श्रा॒तास्त॑ इन्द्र॒ सोमा। वाता॑पेर्\mbox{}हवन॒श्रुतः॒ स्वाहा॥३॥%
\anuvakamend[अ॒ग्निर्होता॒ऽष्टौ]
सूर्यं॑ ते॒ चक्षु॑। वातं॑ प्रा॒णः। द्यां पृ॒ष्ठम्। अ॒न्तरि॑क्षमा॒त्मा। अङ्गैर्\mbox{}य॒ज्ञम्। पृ॒थि॒वी शरी॑रैः। वाच॑स्प॒तेऽच्छि॑द्रया वा॒चा। अच्छि॑द्रया जु॒ह्वा। दि॒वि दे॑वा॒वृध॒ होत्रा॒ मेर॑यस्व॒ स्वाहा॥४॥
\anuvakamend[सूर्यं॑ ते॒ नव॑]
म॒हाह॑वि॒र्\mbox{}होता। स॒त्यह॑विरध्व॒र्युः। अच्यु॑तपाजा अ॒ग्नीत्। अच्यु॑तमना उपव॒क्ता। अ॒ना॒धृ॒ष्यश्चाप्रतिधृ॒ष्यश्च॑ य॒ज्ञस्या॑भिग॒रौ। अ॒यास्य॑ उद्गा॒ता। वाच॑स्पते हृद्विधे नामन्। वि॒धेम॑ ते॒ नाम॑। वि॒धेस्त्वम॒स्माकं॒ नाम॑। वा॒चस्पतिः॒ सोम॑मपात्। मा दैव्य॒स्तन्तु॒श्छेदि॒ मा म॑नु॒ष्य॑। नमो॑ दि॒वे। नम॑ पृथि॒व्यै स्वाहा॥५॥%
\anuvakamend[अ॒पा॒त्त्रीणि॑ च]
वाग्घोता। दी॒क्षा पत्नी। वातोऽध्व॒र्युः। आपो॑ऽभिग॒रः। मनो॑ ह॒विः। तप॑सि जुहोमि। भूर्भुवः॒ सुव॑। ब्रह्म॑ स्वय॒म्भु। ब्रह्म॑णे स्वय॒म्भुवे॒ स्वाहा॥६॥
\anuvakamend[वाग्घोता॒ नव॑]
%ब्रा॒ह्म॒णो य॒ज्ञोऽग्निर्भ॒र्ता पृ॑थि॒वी प्र॑ति॒ष्ठाऽन्तरि॑क्षं  वि॒ष्ठा वा॒युः प्रा॒णश्च॒न्द्रमा॑ ऋ॒तूनन्नं॑ प्रा॒णस्य॑ प्रा॒णो द्यौर॑नाधृ॒ष्य आ॑दि॒त्यः स ते॑ज॒स्वी प्र॒जाप॑तिरि॒द सर्व॒ सर्वं॑ च मे भूयात्॥
ब्रा॒ह्म॒ण एक॑होता। स य॒ज्ञः। स मे॑ ददातु प्र॒जां प॒शून्पुष्टिं॒ यश॑। य॒ज्ञश्च॑ मे भूयात्। अ॒ग्निर्द्विहो॑ता। स भ॒र्ता। स मे॑ ददातु प्र॒जां प॒शून्पुष्टिं॒ यश॑। भ॒र्ता च॑ मे भूयात्। पृ॒थि॒वी त्रिहो॑ता। स प्र॑ति॒ष्ठा॥७॥%
स मे॑ ददातु प्र॒जां प॒शून्पुष्टिं॒ यश॑। प्र॒ति॒ष्ठा च॑ मे भूयात्। अ॒न्तरि॑क्षं॒ चतु॑र्\mbox{}होता। स वि॒ष्ठाः। स मे॑ ददातु प्र॒जां प॒शून्पुष्टिं॒ यश॑। वि॒ष्ठाश्च॑ मे भूयात्। वा॒युः पञ्च॑होता। स प्रा॒णः। स मे॑ ददातु प्र॒जां प॒शून्पुष्टिं॒ यश॑। प्रा॒णश्च॑ मे भूयात्॥८॥
च॒न्द्रमाः॒ षड्ढो॑ता। स ऋ॒तून्क॑ल्पयाति। स मे॑ ददातु प्र॒जां प॒शून्पुष्टिं॒ यश॑। ऋ॒तव॑श्च मे कल्पन्ताम्। अन्न स॒प्तहो॑ता। स प्रा॒णस्य॑ प्रा॒णः। स मे॑ ददातु प्र॒जां प॒शून्पुष्टिं॒ यश॑। प्रा॒णस्य॑ च मे प्रा॒णो भू॑यात्। द्यौर॒ष्टहो॑ता। सो॑ऽनाधृ॒ष्यः॥९॥%
स मे॑ ददातु प्र॒जां प॒शून्पुष्टिं॒ यश॑। अ॒ना॒धृ॒ष्यश्च॑ भूयासम्। आ॒दि॒त्यो नव॑होता। स ते॑ज॒स्वी। स मे॑ ददातु प्र॒जां प॒शून्पुष्टिं॒ यश॑। ते॒ज॒स्वी च॑ भूयासम्। प्र॒जाप॑ति॒र्दश॑होता। स इ॒द सर्वम्। स मे॑ ददातु प्र॒जां प॒शून्पुष्टिं॒ यश॑। सर्वं॑ च मे भूयात्॥१०॥%
\anuvakamend[प्र॒ति॒ष्ठा प्रा॒णश्च॑ मे भूयादनाधृ॒ष्यः सर्वं॑ च मे भूयात्]
%८.१
अ॒ग्निर्यजु॑र्भिः। स॒वि॒ता स्तोमै। इन्द्र॑ उक्थाम॒दैः। मि॒त्रावरु॑णा\-वा॒शिषा। अङ्गि॑रसो॒ धिष्णि॑यैर॒ग्निभि॑। म॒रुत॑ सदोहविर्धा॒नाभ्याम्। आपः॒ प्रोक्ष॑णीभिः। ओष॑धयो ब॒र्\mbox{}हि॒षा। अदि॑ति॒र्वेद्या। सोमो॑ दी॒क्षया॥११॥
%८.२
त्वष्टे॒ध्मेन॑। विष्णु॑र्\mbox{}य॒ज्ञेन॑। वस॑व॒ आज्ये॑न। आ॒दि॒त्या दक्षि॑णाभिः। विश्वे॑ दे॒वा ऊ॒र्जा। पू॒षा स्व॑गाका॒रेण॑। बृह॒स्पति॑ पुरो॒धया। प्र॒जाप॑तिरुद्गी॒थेन॑। अ॒न्तरि॑क्षं प॒वित्रे॑ण। वा॒युः पात्रै। अ॒ह श्र॒द्धया॥१२॥
९.०
अ॒नु॒ष्टुग्दिशः॒ षट्च॑। ९।
\anuvakamend[दी॒क्षया॒ पात्रै॒रेकं॑ च]
९.१
सेनेन्द्र॑स्य। धेना॒ बृह॒स्पते। प॒त्थ्या॑ पू॒ष्णः। वाग्वा॒योः। दी॒क्षा सोम॑स्य। पृ॒थि॒व्य॑ग्नेः। वसू॑नां  गाय॒त्री। रु॒द्राणान्त्रि॒ष्टुक्। आ॒दि॒त्यानां॒ जग॑ती। विष्णो॑रनु॒ष्टुक्॥१३॥%
९.२
वरु॑णस्य वि॒राट्। य॒ज्ञस्य॑ प॒ङ्क्तिः। प्र॒जाप॑ते॒रनु॑मतिः। मि॒त्रस्य॑ श्र॒द्धा। स॒वि॒तुः प्रसू॑तिः। सूर्य॑स्य॒ मरी॑चिः। च॒न्द्रम॑सो रोहि॒णी। ऋषी॑णामरुन्ध॒ती। प॒र्जन्य॑स्य वि॒द्युत्। चत॑स्रो॒ दिश॑। चत॑स्रोऽवान्तरदि॒शाः। अह॑श्च॒ रात्रि॑श्च। कृ॒षिश्च॒ वृष्टि॑श्च। त्विषि॒श्चाप॑चितिश्च। आप॒श्चौष॑धयश्च। ऊर्क्च॑ सू॒नृता॑ च दे॒वानां॒ पत्न॑यः॥१४॥%
१०.०
दा॒ता पुरु॑ष॒मप॑ प्रतिग्रही॒त्रे नव॑ च॥ १०।
\anuvakamend
१०.१
दे॒वस्य॑ त्वा सवि॒तुः प्र॑स॒वे। अ॒श्विनोर्बा॒हुभ्याम्। पू॒ष्णो हस्ताभ्यां॒ प्रति॑गृह्णामि। राजा त्वा॒ वरु॑णो नयतु देवि दक्षिणे॒ऽग्नये॒ हिर॑ण्यम्। तेना॑मृत॒त्वम॑श्याम्। वयो॑ दा॒त्रे। मयो॒ मह्य॑मस्तु प्रतिग्रही॒त्रे। क इ॒दं कस्मा॑ अदात्। कामः॒ कामा॑य। कामो॑ दा॒ता॥१५॥
१०.२
काम॑ प्रतिग्रही॒ता। काम समु॒द्रमावि॑श। कामे॑न त्वा॒ प्रति॑गृह्णामि। कामै॒तत्ते। ए॒षा ते॑ काम॒ दक्षि॑णा। उ॒त्ता॒नस्त्वाङ्गीर॒सः प्रति॑गृह्णातु। सोमा॑य॒ वास॑। रु॒द्राय॒ गाम्। वरु॑णा॒याश्वम्। प्र॒जाप॑तये॒ पुरु॑षम्॥१६॥%
१०.३
मन॑वे॒ तल्पम्। त्वष्ट्रे॒ऽजाम्। पू॒ष्णेऽविम्। निर्\mbox{}ऋ॑त्या अश्वतरगर्द॒भौ। हि॒मव॑तो ह॒स्तिनम्। ग॒न्ध॒र्वा॒प्स॒राभ्य॑ स्रगलं कर॒णे। विश्वेभ्यो दे॒वेभ्यो॑ धा॒न्यम्। वा॒चेऽन्नम्। ब्रह्म॑ण ओद॒नम्। स॒मु॒द्रायाप॑॥१७॥
१०.४
उ॒त्ता॒नायाङ्गीर॒सायान॑। वै॒श्वा॒न॒राय॒ रथम्। वै॒श्वा॒न॒रः प्र॒त्नथा॒ नाक॒मारु॑हत्। दि॒वः पृ॒ष्ठं भन्द॑मानः सु॒मन्म॑भिः। स पूर्व॒वज्ज॒नय॑ज्ज॒न्तवे॒ धनम्। स॒मा॒नम॑ज्मा॒ परि॑याति॒ जागृ॑विः। राजा त्वा॒ वरु॑णो नयतु देविदक्षिणे वैश्वान॒राय॒ रथम्। तेना॑मृत॒त्वम॑श्याम्। वयो॑ दा॒त्रे। मयो॒ मह्य॑मस्तु प्रतिग्रही॒त्रे॥१८॥
१०.५
क इ॒दं कस्मा॑ अदात्। कामः॒ कामा॑य। कामो॑ दा॒ता। काम॑ प्रतिग्रही॒ता। काम समु॒द्रमा वि॑श। कामे॑न त्वा॒ प्रति॑गृह्णामि। कामै॒तत्ते। ए॒षा ते॑ काम॒ दक्षि॑णा। उ॒त्ता॒नस्त्वाङ्गीर॒सः प्रति॑गृह्णातु॥१९॥
११.०
आ॒त्मा जना॑नां  विकु॒र्वन्तं॑  विप॒श्चिं प्र॒जानां वसु॒धानीं  वि॒राजं॒ चर॑न्तं॒  गोम॑तीं मे॒ निय॑च्छ॒त्वेक॑चक्र॒व्व्योँ॑मन्मा॒यया॑ दे॒व एक॑रूपा अ॒ष्टौ च॑। ११।
\anuvakamend
११.१
सु॒वर्णं॑ घ॒र्मं परि॑वेद वे॒नम्। इन्द्र॑स्या॒त्मानं॑ दश॒धा चर॑न्तम्। अ॒न्तः स॑मु॒द्रे मन॑सा॒ चर॑न्तम्। ब्रह्मान्व॑विन्द॒द्दश॑होतार॒मर्णे। अ॒न्तः प्रवि॑ष्टः  शा॒स्ता जना॑नाम्। एकः॒ सन्ब॑हु॒धा वि॑चारः। श॒त शु॒क्राणि॒ यत्रैकं॒ भव॑न्ति। सर्वे॒ वेदा॒ यत्रैकं॒ भव॑न्ति। सर्वे॒ होता॑रो॒ यत्रैकं॒ भव॑न्ति। स॒मान॑सीन आ॒त्मा जना॑नाम्॥२०॥%
११.२
अ॒न्तः प्रवि॑ष्टः  शा॒स्ता जना॑ना॒ सर्वात्मा। सर्वा प्र॒जा यत्रैकं॒ भव॑न्ति। चतु॑र्\mbox{}होतारो॒ यत्र॑ सं॒पदं॒ गच्छ॑न्ति दे॒वैः। स॒मान॑सीन आ॒त्मा जना॑नाम्। ब्रह्मेन्द्र॑म॒ग्निं जग॑तः प्रति॒ष्ठाम्। दि॒व आ॒त्मान सवि॒तारं॒ बृह॒स्पतिम्। चतु॑र्\mbox{}होतारं प्र॒दिशोऽनु॑ कॢ॒प्तम्। वा॒चो वी॒र्यं॑ तप॒साऽन्व॑विन्दत्। अ॒न्तः प्रवि॑ष्टं क॒र्तार॑मे॒तम्। त्वष्टा॑र रू॒पाणि॑ विकु॒र्वन्तं॑ विप॒श्चिम्॥२१॥
११.३
अ॒मृत॑स्य प्रा॒णं य॒ज्ञमे॒तम्। चतु॑र्\mbox{}होतृणामा॒त्मानं॑ क॒वयो॒ निचि॑क्युः। अ॒न्तः प्रवि॑ष्टं क॒र्तार॑मे॒तम्। दे॒वानां॒ बन्धु॒ निहि॑त॒ङ्गुहा॑सु। अ॒मृते॑न कॢ॒प्तं य॒ज्ञमे॒तम्। चतु॑र्\mbox{}होतृणामा॒त्मान॑ङ्क॒वयो॒ निचि॑क्युः। श॒तन्नि॒युतः॒ परि॑वेद॒ विश्वा॑ वि॒श्ववा॑रः। विश्व॑मि॒दव्वृं॑णाति। इन्द्र॑स्या॒त्मा निहि॑तः॒ पञ्च॑होता। अ॒मृतं॑ दे॒वाना॒मायु॑ प्र॒जानाम्॥२२॥%
११.४
इन्द्र॒ राजा॑न सवि॒तार॑मे॒तम्। वा॒योरा॒त्मान॑ङ्क॒वयो॒ निचि॑क्युः। र॒श्मि र॑श्मी॒नां मध्ये॒ तप॑न्तम्। ऋ॒तस्य॑ प॒दे क॒वयो॒ निपान्ति। य आण्डको॒शे भुव॑नं बि॒भर्ति॑। अनि॑र्भिण्णः॒ सन्नथ॑ लो॒कान् वि॒चष्टे। यस्याण्डको॒श शुष्म॑मा॒हुः प्रा॒णमुल्बम्। तेन॑ कॢ॒प्तो॑ऽमृते॑ना॒हम॑स्मि। सु॒वर्णं॒ कोश॒ रज॑सा॒ परी॑वृतम्। दे॒वानां वसु॒धानीं  वि॒राजम्॥२३॥%
११.५
अ॒मृत॑स्य पू॒र्णान्तामु॑ क॒लां  विच॑क्षते। पाद॒ षड्ढो॑तु॒र्न किला॑विवित्से। येन॒र्तव॑ पञ्च॒धोत कॢ॒प्ताः। उ॒त वा॑ ष़॒ड्धा मन॒सोत कॢ॒प्ताः। त षड्ढो॑तारमृ॒तुभिः॒ कल्प॑मानम्। ऋ॒तस्य॑ प॒दे क॒वयो॒ निपान्ति। अ॒न्तः प्रवि॑ष्टं क॒र्तार॑मे॒तम्। अ॒न्तश्च॒न्द्रम॑सि॒ मन॑सा॒ चर॑न्तम्। स॒हैव सन्तं॒ न विजा॑नन्ति दे॒वाः। इन्द्र॑स्या॒त्मान शत॒धा चर॑न्तम्॥२४॥%
११.६
इन्द्रो॒ राजा॒ जग॑तो॒ य ईशे। स॒प्तहो॑ता सप्त॒धा विकॢ॑प्तः। परे॑ण॒ तन्तुं॑ परिषि॒च्यमा॑नम्। अ॒न्तरा॑दि॒त्ये मन॑सा॒ चर॑न्तम्। दे॒वाना॒ हृद॑यं॒ ब्रह्मान्व॑विन्दत्। ब्रह्मै॒तद्ब्रह्म॑ण॒ उज्ज॑भार। अ॒र्क श्चोत॑न्त सरि॒रस्य॒ मध्ये। आ यस्मिन्त्स॒प्त पेर॑वः। मेह॑न्ति बहु॒ला श्रियम्। ब॒ह्व॒श्वामि॑न्द्र॒ गोम॑तीम्॥२५॥%
११.७
अच्यु॑तां बहु॒ला श्रियम्। स हरि॑र्वसु॒वित्त॑मः। पे॒रुरिन्द्रा॑य पिन्वते। ब॒ह्व॒श्वामि॑न्द्र॒ गोम॑तीम्। अच्यु॑तां बहु॒ला श्रियम्। मह्य॒मिन्द्रो॒ निय॑च्छतु। श॒त श॒ता अ॑स्य यु॒क्ता हरी॑णाम्। अ॒र्वाङा या॑तु॒ वसु॑भी र॒श्मिरिन्द्र॑। प्रमह॑माणो बहु॒ला श्रियम्। र॒श्मिरिन्द्र॑ सवि॒ता मे॒ निय॑च्छतु॥२६॥%
११.८
घृ॒तन्तेजो॒ मधु॑मदिन्द्रि॒यम्। मय्य॒यम॒ग्निर्द॑धातु। हरि॑ पत॒ङ्गः प॑ट॒री सु॑प॒र्णः। दि॒वि॒क्षयो॒ नभ॑सा॒ य एति॑। स न॒ इन्द्र॑ कामव॒रन्द॑दातु। पञ्चा॑रं च॒क्रं परि॑वर्तते पृ॒थु। हिर॑ण्यज्योतिः सरि॒रस्य॒ मध्ये। अज॑स्रं॒ ज्योति॒र्नभ॑सा॒ सर्प॑देति। स न॒ इन्द्र॑ कामव॒रन्द॑दातु। स॒प्त यु॑ञ्जन्ति॒ रथ॒मेक॑चक्रम्॥२७॥%
११.९
एको॒ अश्वो॑ वहति सप्तना॒मा। त्रि॒णाभि॑ च॒क्रम॒जर॒मन॑र्वम्। येने॒मा विश्वा॒ भुव॑नानि तस्थुः। भ॒द्रं पश्य॑न्त॒ उप॑सेदु॒रग्रे। तपो॑ दी॒क्षामृष॑यः सुव॒र्विद॑। तत॑ क्ष॒त्रं बल॒मोज॑श्च जा॒तम्। तद॒स्मै दे॒वा अ॒भि सन्न॑मन्तु। श्वे॒त र॒श्मिं बो॑भु॒ज्यमा॑नम्। अ॒पां ने॒तारं॒ भुव॑नस्य गो॒पाम्। इन्द्रं॒ निचि॑क्युः पर॒मे व्यो॑मन्॥२८॥
११.१०
रोहि॑णीः पिङ्ग॒ला एक॑रूपाः। क्षर॑न्तीः पिङ्ग॒ला एक॑रूपाः। श॒त स॒हस्रा॑णि प्र॒युता॑नि॒ नाव्या॑नाम्। अ॒यं यः श्वे॒तो र॒श्मिः। परि॒ सर्व॑मि॒दं जग॑त्। प्र॒जां प॒शून्धना॑नि। अ॒स्माकं॑ ददातु। श्वे॒तो र॒श्मिः परि॒ सर्वं॑ बभूव। सुव॒न्मह्यं॑ प॒शून् वि॒श्वरू॑पान्। प॒त॒ङ्गम॒क्तमसु॑रस्य मा॒यया॥२९॥%
११.११
हृ॒दा प॑श्यन्ति॒ मन॑सा मनी॒षिण॑। स॒मु॒द्रे अ॒न्तः क॒वयो॒ विच॑क्षते। मरी॑चीनां प॒दमि॑च्छन्ति वे॒धस॑। प॒त॒ङ्गो वाचं॒ मन॑सा बिभर्ति। ताङ्ग॑न्ध॒र्वो॑ऽवद॒द्गर्भे॑ अ॒न्तः। तां द्योत॑माना स्व॒र्यं॑ मनी॒षाम्। ऋ॒तस्य॑ प॒दे क॒वयो॒ निपान्ति। ये ग्रा॒म्याः प॒शवो॑ वि॒श्वरू॑पाः। विरू॑पाः॒ सन्तो॑ बहु॒धैक॑रूपाः। अ॒ग्निस्ता अग्रे॒ प्रमु॑मोक्तु दे॒वः॥३०॥
११.१२
प्र॒जाप॑तिः प्र॒जया॑ संविदा॒नः। वी॒त स्तु॑केस्तुके। यु॒वम॒स्मासु॒ निय॑च्छतम्। प्रप्र॑ य॒ज्ञप॑तिन्तिर। ये ग्रा॒म्याः प॒शवो॑ वि॒श्वरू॑पाः। विरू॑पाः॒ सन्तो॑ बहु॒धैक॑रूपाः। तेषा सप्ता॒नामि॒ह रन्ति॑रस्तु। रा॒यस्पोषा॑य सुप्रजा॒स्त्वाय॑ सु॒वीर्या॑य। य आ॑र॒ण्याः प॒शवो॑ वि॒श्वरू॑पाः। विरू॑पाः॒ सन्तो॑ बहु॒धैक॑रूपाः। वा॒युस्ता अग्रे॒ प्रमु॑मोक्तु दे॒वः। प्र॒जाप॑तिः प्र॒जया॑ संविदा॒नः। इडा॑यै सृ॒प्तङ्घृ॒तव॑च्चराच॒रम्। दे॒वा अन्व॑विन्द॒न्गुहा॑ हि॒तम्। य आ॑र॒ण्याः प॒शवो॑ वि॒श्वरू॑पाः। विरू॑पाः॒ सन्तो॑ बहु॒धैक॑रूपाः। तेषा सप्ता॒नामि॒ह रन्ति॑रस्तु। रा॒यस्पोषा॑य सुप्रजा॒स्त्वाय॑ सु॒वीर्या॑य॥३१॥
१२.०
पूरु॑षः पु॒रोऽग्र॒तो॑ऽजायत कृ॒तो॑ऽकल्पयन्नासं॒ द्वे च॑॥ १२॥ज्याया॒नधि॒ पूरु॑षः। अ॒न्यत्र॒ पुरु॑षः॥ ॥
\anuvakamend
१२.१
स॒हस्र॑शीर्\mbox{}षा॒ पुरु॑षः। स॒ह॒स्रा॒क्षः स॒हस्र॑पात्। स भूमिं॑ वि॒श्वतो॑ वृ॒त्वा। अत्य॑तिष्ठद्दशाङ्गु॒लम्। पुरु॑ष ए॒वेद सर्वम्। यद्भू॒तं यच्च॒ भव्यम्। उ॒तामृ॑त॒त्वस्येशा॑नः। यदन्ने॑नाति॒रोह॑ति। ए॒तावा॑नस्य महि॒मा। अतो॒ ज्यायाश्च॒ पूरु॑षः। [३२]
१२.२
पादोऽस्य॒ विश्वा॑ भू॒तानि॑। त्रि॒पाद॑स्या॒मृतं॑ दि॒वि। त्रि॒पादू॒र्ध्व उदै॒त्पुरु॑षः। पादोऽस्ये॒हाभ॑वा॒त्पुन॑। ततो॒ विष्व॒ङ्व्य॑क्रामत्। सा॒श॒ना॒न॒श॒ने अ॒भि। तस्माद्वि॒राड॑जायत। वि॒राजो॒ अधि॒ पूरु॑षः। स जा॒तो अत्य॑रिच्यत। प॒श्चाद्भूमि॒मथो॑ पु॒रः। [३३]
१२.३
यत्पुरु॑षेण ह॒विषा। दे॒वा य॒ज्ञमत॑न्वत। व॒स॒न्तो अ॑स्यासी॒दाज्यम्। ग्री॒ष्म इ॒ध्मः  श॒रद्ध॒विः। स॒प्तास्या॑सन्परि॒धय॑। त्रिः स॒प्त स॒मिध॑ कृ॒ताः। दे॒वा यद्य॒ज्ञन्त॑न्वा॒नाः। अब॑ध्न॒न्पुरु॑षं प॒शुम्। तं य॒ज्ञं ब॒\ar{}हिषि॒ प्रौक्ष\sn{}। पुरु॑षं जा॒तम॑ग्र॒तः। [३४]
१२.४
तेन॑ दे॒वा अय॑जन्त। सा॒ध्या ऋष॑यश्च॒ ये। तस्माद्य॒ज्ञात्स॑र्व॒हुत॑। सम्भृ॑तं पृषदा॒ज्यम्। प॒शूस्ताश्च॑क्रे वाय॒व्यान्॑। आ॒र॒ण्यान्ग्रा॒म्याश्च॒ ये। तस्माद्य॒ज्ञात्स॑र्व॒हुत॑। ऋचः॒ सामा॑नि जज्ञिरे। छन्दासि जज्ञिरे॒ तस्मात्। यजु॒स्तस्मा॑दजायत। [३५]
१२.५
तस्मा॒दश्वा॑ अजायन्त। ये के चो॑भ॒याद॑तः। गावो॑ ह जज्ञिरे॒ तस्मात्। तस्माज्जा॒ता अ॑जा॒वय॑। यत्पुरु॑षं॒ व्य॑दधुः। क॒ति॒धा व्य॑कल्पयन्। मुखं॒ किम॑स्य॒ कौ बा॒हू। कावू॒रू पादा॑वुच्येते। ब्रा॒ह्म॒णोऽस्य॒ मुख॑मासीत्। बा॒हू रा॑ज॒न्य॑ कृ॒तः। [३६]
१२.६
ऊ॒रू तद॑स्य॒ यद्वैश्य॑। प॒द्भ्या शू॒द्रो अ॑जायत। च॒न्द्रमा॒ मन॑सो जा॒तः। चक्षोः॒ सूर्यो॑ अजायत। मुखा॒दिन्द्र॑श्चा॒ग्निश्च॑। प्रा॒णाद्वा॒युर॑जायत। नाभ्या॑ आसीद॒न्तरि॑क्षम्। शी॒र्ष्णो द्यौः सम॑वर्तत। प॒द्भ्यां भूमि॒र्दिशः॒ श्रोत्रात्। तथा॑ लो॒का अ॑कल्पयन्। [३७]
१२.७
वेदा॒हमे॒तं पुरु॑षं म॒हान्तम्। आ॒दि॒त्यव॑र्णं॒ तम॑स॒स्तु पा॒रे। सर्वा॑णि रू॒पाणि॑ वि॒चित्य॒ धीर॑। नामा॑नि कृ॒त्वाऽभि॒वद॒\an{} यदास्ते। धा॒ता पु॒रस्ता॒द्यमु॑दाज॒हार॑। श॒क्रः प्रवि॒द्वान्प्र॒दिश॒श्चत॑स्रः। तमे॒वं वि॒द्वान॒मृत॑ इ॒ह भ॑वति। नान्यः पन्था॒ अय॑नाय विद्यते। य॒ज्ञेन॑ य॒ज्ञम॑यजन्त दे॒वाः। तानि॒ धर्मा॑णि प्रथ॒मान्या॑सन्। ते ह॒ नाकं॑ महि॒मान॑ सचन्ते। यत्र॒ पूर्वे॑ सा॒ध्याः सन्ति॑ दे॒वाः॥ [३८]
१३.०
जा॒य॒ते॒ वशे॑ स॒प्त च॑॥ १३॥
\anuvakamend
१३.१
अ॒द्भ्यः सम्भू॑तः पृथि॒व्यै रसाच्च। वि॒श्वक॑र्मणः॒ सम॑वर्त॒ताधि॑। तस्य॒ त्वष्टा॑ वि॒दध॑द्रू॒पमे॑ति। तत्पुरु॑षस्य॒ विश्व॒माजा॑न॒मग्रे। वेदा॒हमे॒तं पुरु॑षं म॒हान्तम्। आ॒दि॒त्यव॑र्णं॒ तम॑सः॒ पर॑स्तात्। तमे॒वं वि॒द्वान॒मृत॑ इ॒ह भ॑वति। नान्यः पन्था॑ विद्य॒तेऽय॑नाय। प्र॒जाप॑तिश्चरति॒ गर्भे॑ अ॒न्तः। अ॒जाय॑मानो बहु॒धा विजा॑यते। [३९]
१३.२
तस्य॒ धीराः॒ परि॑जानन्ति॒ योनिम्। मरी॑चीनां प॒दमि॑च्छन्ति वे॒धस॑। यो दे॒वेभ्य॒ आत॑पति। यो दे॒वानां पु॒रोहि॑तः। पूर्वो॒ यो दे॒वेभ्यो॑ जा॒तः। नमो॑ रु॒चाय॒ ब्राह्म॑ये। रुचं॑ ब्रा॒ह्मं ज॒नय॑न्तः। दे॒वा अग्रे॒ तद॑ब्रुवन्। यस्त्वै॒वं ब्राह्म॒णो वि॒द्यात्। तस्य॑ दे॒वा अस॒न्वशे। ह्रीश्च॑ ते ल॒क्ष्मीश्च॒ पत्न्यौ। अ॒हो॒रा॒त्रे पा॒र्श्वे। नक्ष॑त्राणि रू॒पम्। अ॒श्विनौ॒ व्यात्तम्। इ॒ष्टं म॑निषाण। अ॒मुं म॑निषाण। सर्वं॑ मनिषाण॥ [४०]
१४.०
एकं॑ प्र॒जानाङ्गसाथां॒ नव॑॥ १४।
\anuvakamend
१४.१
भ॒र्ता सन्भ्रि॒यमा॑णो बिभर्ति। एको॑ दे॒वो ब॑हु॒धा निवि॑ष्टः। य॒दा भा॒रन्त॒न्द्रय॑ते॒ स भर्तुम्। नि॒धाय॑ भा॒रं पुन॒रस्त॑मेति। तमे॒व मृ॒त्युम॒मृतं॒ तमा॑हुः। तं भ॒र्तारं॒ तमु॑ गो॒प्तार॑माहुः। स भृ॒तो भ्रि॒यमा॑णो बिभर्ति। य ए॑नं॒ वेद॑ स॒त्येन॒ भर्तुम्। स॒द्यो जा॒तमु॒त ज॑हात्ये॒षः। उ॒तो जर॑न्तं॒ न ज॑हा॒त्येकम्॥४१॥
१४.२
उ॒तो ब॒हूनेक॒मह॑र्जहार। अत॑न्द्रो दे॒वः सद॑मे॒व प्रार्थ॑। यस्तद्वेद॒ यत॑ आब॒भूव॑। स॒न्धां च॒ या स॑न्द॒धे ब्रह्म॑णै॒षः। रम॑ते॒ तस्मि॑न्नु॒त जी॒र्णे शया॑ने। नैनं॑ जहा॒त्यह॑ सु पू॒र्व्येषु॑। त्वामापो॒ अनु॒ सर्वाश्चरन्ति जान॒तीः। व॒त्सं पय॑सा पुना॒नाः। त्वम॒ग्नि ह॑व्य॒वाह॒ समिन्त्से। त्वं भ॒र्ता मा॑त॒रिश्वा प्र॒जानाम्॥४२॥
१४.३
त्वं य॒ज्ञस्त्वमु॑वे॒वासि॒ सोम॑। तव॑ दे॒वा हव॒माय॑न्ति॒ सर्वे। त्वमेको॑ऽसि ब॒हूननु॒प्रवि॑ष्टः। नम॑स्ते अस्तु सु॒हवो॑ म एधि। नमो॑ वामस्तु शृणु॒त हवं॑ मे। प्राणा॑पानावजि॒र स॒ञ्चर॑न्तौ। ह्वया॑मि वां॒ ब्रह्म॑णा तू॒र्तमेतम्। यो मां द्वेेष्टि॒ तं ज॑हितं युवाना। प्राणा॑पानौ संविदा॒नौ ज॑हितम्। अ॒मुष्यासु॑ना॒मा सङ्ग॑साथाम्॥४३॥
१४.४
तं मे॑ देवा॒ ब्रह्म॑णा संविदा॒नौ। व॒धाय॑ दत्तं॒ तम॒ह ह॑नामि। अस॑ज्जजान स॒त आब॑भूव। यं यं॑ ज॒जान॒ स उ॑ गो॒पो अ॑स्य। य॒दा भा॒रन्त॒न्द्रय॑ते॒ स भर्तुम्। प॒रास्य॑ भा॒रं पुन॒रस्त॑मेति। तद्वै त्वं प्रा॒णो अ॑भवः। म॒हान्भोग॑ प्र॒जाप॑तेः। भुज॑ करि॒ष्यमा॑णः। यद्दे॒वान्प्राण॑यो॒ नव॑॥४४॥
१५.०
मृ॒त्यवे॑ वी॒राश्च॒त्वारि॑ च॥ १५।
\anuvakamend
१५.१
हरि॒ हर॑न्त॒मनु॑यन्ति दे॒वाः। विश्व॒स्येशा॑नं वृष॒भं म॑ती॒नाम्। ब्रह्म॒ सरू॑प॒मनु॑मे॒दमागात्। अय॑नं॒ मा विव॑धी॒र्विक्र॑मस्व। मा छि॑दो मृत्यो॒ मा व॑धीः। मा मे॒ बलं॒  विवृ॑हो॒ मा प्रमो॑षीः। प्र॒जां मा मे॑ रीरिष॒ आयु॑रुग्र। नृ॒चक्ष॑सन्त्वा ह॒विषा॑ विधेम। स॒द्यश्च॑कमा॒नाय॑। प्र॒वे॒पा॒नाय॑ मृ॒त्यवे॥४५॥
१५.२
प्रास्मा॒ आशा॑ अशृण्वन्। कामे॑नाजनय॒न्पुन॑। कामे॑न मे॒ काम॒ आगात्। हृद॑या॒द्धृद॑यं मृ॒त्योः। यद॒मीषा॑म॒दः प्रि॒यम्। तदैतूप॒माम॒भि। परं॑ मृत्यो॒ अनु॒ परे॑हि॒ पन्थाम्। यस्ते॒ स्व इत॑रो देव॒यानात्। चक्षु॑ष्मते शृण्व॒ते ते ब्रवीमि। मा न॑ प्र॒जा री॑रिषो॒ मोत वी॒रान्। प्र पू॒र्व्यं मन॑सा॒ वन्द॑मानः। नाध॑मानो वृष॒भं च॑र्\mbox{}षणी॒नाम्। यः प्र॒जाना॑मेक॒राण्मानु॑षीणाम्। मृ॒त्युं॒ य॑जे प्रथम॒जामृ॒तस्य॑॥४६॥
\anuvakamend
१६.१
त॒रणि॑र्वि॒श्वद॑र्\mbox{}शतो ज्योति॒ष्कृद॑सि सूर्य। विश्व॒मा भा॑सि रोच॒नम्। उ॒प॒या॒मगृ॑हीतोऽसि॒ सूर्या॑य त्वा॒ भ्राज॑स्वत ए॒ष ते॒ योनिः॒ सूर्या॑य त्वा॒ भ्राज॑स्वते॥४७॥ १६॥
\anuvakamend
१७.१
आ प्या॑यस्व मदिन्तम॒ सोम॒ विश्वा॑भिरू॒तिभि॑। भवा॑ नः स॒प्रथ॑स्तमः॥४८॥ १७॥
\anuvakamend
१८.१
ई॒युष्टे ये पूर्व॑तरा॒मप॑श्यन् व्यु॒च्छन्ती॑मु॒षसं॒ मर्त्या॑सः। अ॒स्माभि॑रू॒ नु प्र॑ति॒चक्ष्या॑ऽभू॒दो ते य॑न्ति॒ ये अ॑प॒रीषु॒ पश्यान्॑॥ ४९। १८॥
\anuvakamend
१९.१
ज्योति॑ष्मतीं त्वा सादयामि ज्योति॒ष्कृतं॑ त्वा सादयामि ज्योति॒र्विदं॑ त्वा सादयामि॒ भास्व॑तीं त्वा सादयामि॒ ज्वल॑न्तीं त्वा सादयामि मल्मला॒भव॑न्तीं त्वा सादयामि॒ दीप्य॑मानां त्वा सादयामि॒ रोच॑मानां त्वा सादया॒म्यज॑स्रां त्वा सादयामि बृ॒हज्ज्यो॑तिषं त्वा सादयामि बो॒धय॑न्तीं त्वा सादयामि॒ जाग्र॑तीं त्वा सादयामि॥ ५०॥ १९॥
\anuvakamend
२०.१
प्र॒या॒साय॒ स्वाहा॑ऽऽया॒साय॒ स्वाहा॑ विया॒साय॒ स्वाहा॑ संया॒साय॒ स्वाहोद्या॒साय॒ स्वाहा॑ऽवया॒साय॒ स्वाहा॑ शु॒चे स्वाहा॒ शेका॑य॒ स्वाहा॑ तप्य॒त्वै स्वाहा॒ तप॑ते॒ स्वाहा ब्रह्मह॒त्यायै॒ स्वाहा॒ सर्व॑स्मै॒ स्वाहा॥५१॥% २०।
\anuvakamend
२१.१
चि॒त्त स॑न्ता॒नेन॑ भ॒वं य॒क्ना रु॒द्रन्तनि॑म्ना पशु॒पति स्थूलहृद॒येना॒ग्नि हृद॑येन रु॒द्रं लोहि॑तेन श॒र्वम्मत॑स्नाभ्यां महादे॒वम॒न्तः पार्श्वेनौषिष्ठ॒हन शिङ्गीनिको॒श्याभ्याम्॥५२॥% २१।
\anuvakamend

ॐ तच्छं॒ योरावृ॑णीमहे। गा॒तुं य॒ज्ञाय॑। गा॒तुं य॒ज्ञप॑तये। 
दैवी स्व॒स्तिर॑स्तु नः। स्व॒स्तिर्मानु॑षेभ्यः। ऊ॒र्ध्वं जि॑गातु भेष॒जम्। 
शं नो॑ अस्तु द्वि॒पदे। शं चतु॑ष्पदे। ॐ शान्ति॒ शान्ति॒ शान्ति॑॥

\closesection

\sect{चतुर्थः प्रश्नः}\setcounter{anuvakam}{0}
न॒मोऽनु॑मदन्तु। ॐ शान्तिः॒ शान्तिः॒ शान्ति॑॥ 

नमो॑ वा॒चे या चो॑दि॒ता या चानु॑दिता॒ तस्यै॑ वा॒चे नमो॒ नमो॑ वा॒चे नमो॑ वा॒चस्पत॑ये॒ नम॒ ऋषि॑भ्यो मन्त्र॒कृद्भ्यो॒ मन्त्र॑पतिभ्यो॒ मा मामृष॑यो मन्त्र॒कृतो॑ मन्त्र॒पत॑यः॒ परा॑दु॒र्माहमृषीन्मन्त्र॒कृतो॑ मन्त्र॒पती॒न्परा॑दां वैश्वदे॒वीं वाच॑मुद्यास शि॒वामद॑स्तां॒ जुष्टान्दे॒वेभ्यः॒ शर्म॑ मे॒ द्यौः  शर्म॑ पृथि॒वी शर्म॒ विश्व॑मि॒दं जग॑त्। शर्म॑ च॒न्द्रश्च॒ सूर्य॑श्च॒ शर्म॑ ब्रह्मप्रजाप॒ती। भू॒तं व॑दिष्ये॒ भुव॑नं वदिष्ये॒ तेजो॑ वदिष्ये॒ यशो॑ वदिष्ये॒ तपो॑ वदिष्ये॒ ब्रह्म॑ वदिष्ये स॒त्यं व॑दिष्ये॒ तस्मा॑ अ॒हमि॒दमु॑प॒स्तर॑ण॒मुप॑स्तृण उप॒स्तर॑णं मे प्र॒जायै॑ पशू॒नां भू॑यादुप॒स्तर॑णम॒हं प्र॒जायै॑ पशू॒नां भू॑यासं॒ प्राणा॑पानौ मृ॒त्योर्मा॑ पातं॒ प्राणा॑पानौ॒ मा मा॑ हासिष्टं॒ मधु॑ मनिष्ये॒ मधु॑ जनिष्ये॒ मधु॑ वक्ष्यामि॒ मधु॑ वदिष्यामि॒ मधु॑मतीं दे॒वेभ्यो॒ वाच॑मुद्यास शुश्रू॒षेण्यां मनु॒ष्येभ्य॒स्तं मा॑ दे॒वा अ॑वन्तु शो॒भायै॑ पि॒तरोऽनु॑मदन्तु॥१॥

७.२.०
प॒ते॒ शिर॑ ऋतावरीर्\mbox{}ऋ॒द्ध्यास॑म॒द्य म॒खस्य॒ शिरः॒ शिरः॒ शिरो॑ऽसि॒ नव॑ च।२॥इय॑ति॒ देवी॒रिन्द्र॒स्यौजोऽस्यग्नि॒जा अ॒स्यायु॑र्द्धेहि प्रा॒णं पञ्च॑॥ ॥
\anuvakamend
७.२.१
यु॒ञ्जते॒ मन॑ उ॒त यु॑ञ्जते॒ धिय॑। विप्रा॒ विप्र॑स्य बृह॒तो वि॑प॒श्चित॑। वि होत्रा॑ दधे वयुना॒विदेक॒ इत्। म॒ही दे॒वस्य॑ सवि॒तुः परि॑ष्टुतिः। दे॒वस्य॑ त्वा सवि॒तुः प्र॑स॒वे। अ॒श्विनोर्बा॒हुभ्याम्। पू॒ष्णो हस्ताभ्या॒मा द॑दे। अभ्रि॑रसि॒ नारि॑रसि। अ॒ध्व॒र॒कृद्दे॒वेभ्य॑। उत्ति॑ष्ठ ब्रह्मणस्पते॥२॥
७.२.२
दे॒व॒यन्त॑स्त्वेमहे। उप॒ प्रय॑न्तु म॒रुत॑ सु॒दान॑वः। इन्द्र॑ प्रा॒शूर्भ॑वा॒ सचा। प्रैतु॒ ब्रह्म॑ण॒स्पति॑। प्र दे॒व्ये॑तु सू॒नृता। अच्छा॑ वी॒रं नर्यं॑ प॒ङ्क्तिरा॑धसम्। दे॒वा य॒ज्ञं न॑यन्तु नः। देवी द्यावापृथिवी॒ अनु॑ मे मसाथाम्। ऋ॒द्ध्यास॑म॒द्य। म॒खस्य॒ शिर॑॥३॥
७.२.३
म॒खाय॑ त्वा। म॒खस्य॑ त्वा शी॒र्ष्णे। इय॒त्यग्र॑ आसीः। ऋ॒द्ध्यास॑म॒द्य। म॒खस्य॒ शिर॑। म॒खाय॑ त्वा। म॒खस्य॑ त्वा शी॒र्ष्णे। देवीर्वम्रीर॒स्य भू॒तस्य॑ प्रथमजा ऋतावरीः। ऋ॒द्ध्यास॑म॒द्य। म॒खस्य॒ शिर॑॥४॥
७.२.४
म॒खाय॑ त्वा। म॒खस्य॑ त्वा शी॒र्ष्णे। इन्द्र॒स्यौजो॑ऽसि। ऋ॒द्ध्यास॑म॒द्य। म॒खस्य॒ शिर॑। म॒खाय॑ त्वा। म॒खस्य॑ त्वा शी॒र्ष्णे। अ॒ग्नि॒जा अ॑सि प्र॒जाप॑ते॒ रेत॑। ऋ॒द्ध्यास॑म॒द्य। म॒खस्य॒ शिर॑॥५॥
७.२.५
म॒खाय॑ त्वा। म॒खस्य॑ त्वा शी॒र्ष्णे। आयु॑र्धेहि प्रा॒णन्धे॑हि। अ॒पा॒नन्धे॑हि व्या॒नन्धे॑हि। चक्षु॑र्धेहि॒ श्रोत्र॑न्धेहि। मनो॑ धेहि॒ वाच॑न्धेहि। आ॒त्मान॑न्धेहि प्रति॒ष्ठान्धे॑हि। मान्धे॑हि॒ मयि॑ धेहि। मधु॑ त्वा मधु॒ला क॑रोतु। म॒खस्य॒ शिरो॑ऽसि॥६॥
७.२.६
य॒ज्ञस्य॑ प॒दे स्थ॑। गा॒य॒त्रेण॑ त्वा॒ छन्द॑सा करोमि। त्रैष्टु॑भेन त्वा॒ छन्द॑सा करोमि। जाग॑तेन त्वा॒ छन्द॑सा करोमि। म॒खस्य॒ रास्ना॑ऽसि। अदि॑तिस्ते॒ बिल॑ङ्गृह्णातु। पाङ्क्ते॑न॒ छन्द॑सा। सूर्य॑स्य॒ हर॑सा श्राय। म॒खो॑ऽसि॥७॥
७.३.०
पृ॒थि॒वीं भ॑व॒ वाख्षट्च॑। ३।
\anuvakamend
७.३.१
वृष्णो॒ अश्व॑स्य नि॒ष्पद॑सि। वरु॑णस्त्वा धृ॒तव्र॑त॒ आधू॑पयतु। मि॒त्रावरु॑णयोर्ध्रु॒वेण॒ धर्म॑णा। अ॒र्चिषे त्वा। शो॒चिषे त्वा। ज्योति॑षे त्वा। तप॑से त्वा। अ॒भीमं म॑हि॒ना दिवम्। मि॒त्रो ब॑भूव स॒प्रथा। उ॒त श्रव॑सा पृथि॒वीम्॥८॥
७.३.२
मि॒त्रस्य॑ चर्\mbox{}षणी॒धृत॑। श्रवो॑ दे॒वस्य॑ सान॒सिम्। द्यु॒म्नं चि॒त्रश्र॑वस्तमम्। सिध्यै त्वा। दे॒वस्त्वा॑ सवि॒तोद्व॑पतु। सु॒पा॒णिः स्व॑ङ्गु॒रिः। सु॒बा॒हुरु॒त शक्त्या। अप॑द्यमानः पृथि॒व्याम्। आशा॒ दिश॒ आ पृ॑ण। उत्ति॑ष्ठ बृ॒हन्भ॑व॥९॥
७.३.३
ऊ॒र्ध्वस्ति॑ष्ठ ध्रु॒वस्त्वम्। सूर्य॑स्य त्वा॒ चक्षु॒षाऽन्वीक्षे। ऋ॒जवे त्वा। सा॒धवे त्वा। सु॒क्षि॒त्यै त्वा॒ भूत्यै त्वा। इ॒दम॒हम॒मुमा॑मुष्याय॒णं  वि॒शा प॒शुभि॑र्ब्रह्मवर्च॒सेन॒ पर्यू॑हामि। गा॒य॒त्रेण॑ त्वा॒ छन्द॒सा च्छृ॑णद्मि। त्रैष्टु॑भेन त्वा॒ छन्द॒सा च्छृ॑णद्मि। जाग॑तेन त्वा॒ छन्द॒सा च्छृ॑णद्मि। छृ॒णत्तु॑ त्वा॒ वाक्। छृ॒णत्तु॒ त्वोर्क्। छृ॒णत्तु॑ त्वा ह॒विः। छृ॒न्धि वाचम्। छृ॒न्ध्यूर्जम्। छृ॒न्धि ह॒विः। देव॑ पुरश्चर स॒ग्घ्यास॑न्त्वा॥१०॥
७.४.०
अहृ॑णीयमानो॒ द्वे च॑॥ ४।
\anuvakamend
७.४.१
ब्रह्म॑न् प्रव॒र्ग्ये॑ण॒ प्रच॑रिष्यामः। होत॑र्घ॒र्मम॒भिष्टु॑हि। अग्नी॒द्रौहि॑णौ पुरो॒डाशा॒वधि॑श्रय। प्रति॑प्रस्थात॒र्विह॑र। प्रस्तो॑तः॒ सामा॑नि गाय। यजु॑र्\mbox{}युक्त॒ साम॑भि॒राक्त॑खन्त्वा। विश्वैर्दे॒वैरनु॑मतं म॒रुद्भि॑। दक्षि॑णाभिः॒ प्रत॑तं पारयि॒ष्णुम्। स्तुभो॑ वहन्तु सुमन॒स्यमा॑नम्। स नो॒ रुच॑न्धे॒ह्यहृ॑णीयमानः। भूर्भुवः॒ सुव॑। ओमिन्द्र॑वन्तः॒ प्रच॑रत॥११॥
७.५.०
अ॒न॒क्त्व॒सा॒दी॒दु॒त्त॒र॒तः पा॑हि प्रति॒मा अ॑सि यज॒तन्ते॑ अ॒न्यज्जाग॑तम॒स्येकं॑ च॥ ५॥
\anuvakamend
७.५.१
ब्रह्म॒न्प्रच॑रिष्यामः। होत॑र्घ॒र्मम॒भिष्टु॑हि। य॒माय॑ त्वा म॒खाय॑ त्वा। सूर्य॑स्य॒ हर॑से त्वा। प्रा॒णाय॒ स्वाहा व्या॒नाय॒ स्वाहा॑ऽपा॒नाय॒ स्वाहा। चक्षु॑षे॒ स्वाहा॒ श्रोत्रा॑य॒ स्वाहा। मन॑से॒ स्वाहा॑ वा॒चे सर॑स्वत्यै॒ स्वाहा। दक्षा॑य॒ स्वाहा॒ क्रत॑वे॒ स्वाहा। ओज॑से॒ स्वाहा॒ बला॑य॒ स्वाहा। दे॒वस्त्वा॑ सवि॒ता मध्वा॑ऽनक्तु॥१२॥
७.५.२
पृ॒थि॒वीन्तप॑सस्त्रायस्व। अ॒र्चिर॑सि शो॒चिर॑सि॒ ज्योति॑रसि॒ तपो॑ऽसि। ससी॑दस्व म॒हा अ॑सि। शोच॑स्व देव॒वीत॑मः। विधू॒मम॑ग्ने अरु॒षं मि॑येध्य। सृ॒ज प्र॑शस्तदर्\mbox{}श॒तम्। अ॒ञ्जन्ति॒ यं प्र॒थय॑न्तो॒ न विप्रा। व॒पाव॑न्तं॒ नाग्निना॒ तप॑न्तः। पि॒तुर्न पु॒त्र उप॑सि॒ प्रेष्ठ॑। आ घ॒र्मो अ॒ग्निमृ॒तय॑न्नसादीत्॥१३॥
७.५.३
अ॒ना॒धृ॒ष्या पु॒रस्तात्। अ॒ग्नेराधि॑पत्ये। आयु॑र्मे दाः। पु॒त्रव॑ती दक्षिण॒तः। इन्द्र॒स्याधि॑पत्ये। प्र॒जां मे॑ दाः। सु॒षदा॑ प॒श्चात्। दे॒वस्य॑ सवि॒तुराधि॑पत्ये। प्रा॒णं मे॑ दाः। आश्रु॑तिरुत्तर॒तः॥१४॥
७.५.४
मि॒त्रावरु॑णयो॒राधि॑पत्ये। श्रोत्रं॑ मे दाः। विधृ॑तिरु॒परि॑ष्टात्। बृह॒स्पते॒राधि॑पत्ये। ब्रह्म॑ मे दाः क्ष॒त्रं मे॑ दाः। तेजो॑ मे धा॒ वर्चो॑ मे धाः। यशो॑ मे धा॒स्तपो॑ मे धाः। मनो॑ मे धाः। मनो॒रश्वा॑ऽसि॒ भूरि॑पुत्रा। विश्वाभ्यो मा ना॒ष्ट्राभ्य॑ पाहि॥१५॥
७.५.५
सू॒प॒सदा॑ मे भूया॒ मा मा॑ हिसीः। तपो॒ष्व॑ग्ने॒ अन्त॑रा अ॒मित्रान्॑। तपा॒शस॑मर॒रुषः॒ पर॑स्य। तपा॑वसो चिकिता॒नो अ॒चित्तान्॑। वि ते॑ तिष्ठन्ताम॒जरा॑ अ॒यास॑। चित॑ स्थ परि॒चित॑। स्वाहा॑ म॒रुद्भिः॒ परि॑श्रयस्व। मा अ॑सि। प्र॒मा अ॑सि। प्र॒ति॒मा अ॑सि॥१६॥
७.५.६
स॒म्मा अ॑सि। वि॒मा अ॑सि। उ॒न्मा अ॑सि। अ॒न्तरि॑क्षस्यान्त॒र्द्धिर॑सि। दिवं॒ तप॑सस्त्रायस्व। आ॒भिर्गी॒र्भिर्यदतो॑ न ऊ॒नम्। आप्या॑यय हरिवो॒ वर्द्ध॑मानः। य॒दा स्तो॒तृभ्यो॒ महि॑ गो॒त्रा रु॒जासि॑। भू॒यि॒ष्ठ॒भाजो॒ अध॑ ते स्याम। शु॒क्रन्ते॑ अ॒न्यद्य॑ज॒तन्ते॑ अ॒न्यत्॥१७॥
७.५.७
विषु॑रूपे॒ अह॑नी॒ द्यौरि॑वासि। विश्वा॒ हि मा॒या अव॑सि स्वधावः। भ॒द्रा ते॑ पूषन्नि॒ह रा॒तिर॑स्तु। अर्\mbox{}ह॑न्बिभर्\mbox{}षि॒ साय॑कानि॒ धन्व॑। अर्\mbox{}हं नि॒ष्कं य॑ज॒तं  वि॒श्वरू॑पम्। अर्\mbox{}हं॑ नि॒दन्द॑यसे॒ विश्व॒मब्भु॑वम्। न वा ओजी॑यो रुद्र॒ त्वद॑स्ति। गा॒य॒त्रम॑सि। त्रैष्टु॑भमसि। जाग॑तमसि। मधु॒ मधु॒ मधु॑॥१८॥
७.६.०
रो॒च॒य॒ धे॒हि॒ नव॑ च। ६।
\anuvakamend
७.६.१
दश॒ प्राची॒र्दश॑ भासि दक्षि॒णा। दश॑ प्र॒तीची॒र्दश॑ भा॒स्युदी॑चीः। दशो॒र्ध्वा भा॑सि सुमन॒स्यमा॑नः। स नो॒ रुच॑न्धे॒ह्यहृ॑णीयमानः। अ॒ग्निष्ट्वा॒ वसु॑भिः पु॒रस्ताद्रोचयतु गाय॒त्रेण॒ छन्द॑सा। स मा॑ रुचि॒तो रो॑चय। इन्द्र॑स्त्वा रु॒द्रैर्द॑क्षिण॒तो रो॑चयतु॒ त्रैष्टु॑भेन॒ छन्द॑सा। स मा॑ रुचि॒तो रो॑चय। वरु॑णस्त्वादि॒त्यैः प॒श्चाद्रो॑चयतु॒ जाग॑तेन॒ छन्द॑सा। स मा॑ रुचि॒तो रो॑चय॥१९॥
७.६.२
द्यु॒ता॒नस्त्वा॑ मारु॒तो म॒रुद्भि॑रुत्तर॒तो रो॑चय॒त्वानु॑ष्टुभेन॒ छन्द॑सा। स मा॑ रुचि॒तो रो॑चय। बृह॒स्पति॑स्त्वा॒ विश्वैर्दे॒वैरु॒परि॑ष्टाद्रोचयतु॒ पाङ्क्ते॑न॒ छन्द॑सा। स मा॑ रुचि॒तो रो॑चय। रो॒चि॒तस्त्वन्दे॑व घर्म दे॒वेष्वसि॑। रो॒चि॒षी॒याहं म॑नु॒ष्ये॑षु। सम्राड्घर्म रुचि॒तस्त्वन्दे॒वेष्वायु॑ष्मास्तेज॒स्वी ब्र॑ह्मवर्च॒स्य॑सि। रु॒चि॒तो॑ऽहं म॑नु॒ष्येष्वायु॑ष्मास्तेज॒स्वी ब्र॑ह्मवर्च॒सी भू॑यासम्। रुग॑सि। रुचं॒ मयि॑ धेहि॥२०॥
७.६.३
मयि॒ रुक्। दश॑ पु॒रस्ताद्रोचसे। दश॑ दक्षि॒णा। दश॑ प्र॒त्यङ्ङ्। दशोदङ्ङ्॑। दशो॒र्ध्वो भा॑सि सुमन॒स्यमा॑नः। स न॑ सम्रा॒डिष॒मूर्ज॑न्धेहि। वा॒जी वा॒जिने॑ पवस्व। रो॒चि॒तो घ॒र्मो रु॑ची॒य॥२१॥
७.७.०
रो॒च॒ते॒ सूर्या॑य त्वा देवा॒युवं॑ द्रविणो॒दा दधा॑ना॒ द्वे च॑॥ ७॥
\anuvakamend
७.७.१
अप॑श्यङ्गो॒पामनि॑पद्यमानम्। आ च॒ परा॑ च प॒थिभि॒श्चर॑न्तम्। स स॒ध्रीचीः॒ स विषू॑ची॒र्वसा॑नः। आ व॑रीवर्ति॒ भुव॑नेष्व॒न्तः। अत्र॑ प्रा॒वीः। मधु॒ माध्वीभ्यां॒ मधु॒ माधू॑चीभ्याम्। अनु॑ वान्दे॒ववी॑तये। सम॒ग्निर॒ग्निना॑ गत। सन्दे॒वेन॑ सवि॒त्रा। स सूर्ये॑ण रोचते॥२२॥
७.७.२
स्वाहा॒ सम॒ग्निस्तप॑सा गत। सन्दे॒वेन॑ सवि॒त्रा। स सूर्ये॑णारोचिष्ट। ध॒र्ता दि॒वो विभा॑सि॒ रज॑सः। पृ॒थि॒व्या ध॒र्ता। उ॒रोर॒न्तरि॑क्षस्य ध॒र्ता। ध॒र्ता दे॒वो दे॒वानाम्। अम॑र्त्यस्तपो॒जाः। हृ॒दे त्वा॒ मन॑से त्वा। दि॒वे त्वा॒ सूर्या॑य त्वा॥२३॥
७.७.३
ऊ॒र्ध्वमि॒म॑ध्व॒रं कृ॑धि। दि॒वि दे॒वेषु॒ होत्रा॑ यच्छ। विश्वा॑सां भुवां पते। विश्व॑स्य भुवनस्पते। विश्व॑स्य मनसस्पते। विश्व॑स्य वचसस्पते। विश्व॑स्य तपसस्पते। विश्व॑स्य ब्रह्मणस्पते। दे॒व॒श्रूस्त्वन्दे॑व घर्म दे॒वान्पा॑हि। त॒पो॒जां वाच॑म॒स्मे निय॑च्छ देवा॒युवम्॥२४॥
७.७.४
गर्भो॑ दे॒वानाम्। पि॒ता म॑ती॒नाम्। पति॑ प्र॒जानाम्। मति॑ कवी॒नाम्। सन्दे॒वो दे॒वेन॑ सवि॒त्रा य॑तिष्ट। स सूर्ये॑णारुक्त। आ॒यु॒र्दास्त्वम॒स्मभ्य॑ङ्घर्म वर्चो॒दा अ॑सि। पि॒ता नो॑ऽसि पि॒ता नो॑ बोध। आ॒यु॒र्द्धास्त॑नू॒धाः प॑यो॒धाः। व॒र्चो॒दा व॑रिवो॒दा द्र॑विणो॒दाः॥२५॥
७.७.५
अ॒न्त॒रि॒क्ष॒प्र॒ उ॒रोर्वरी॑यान्। अ॒शी॒महि॑ त्वा॒ मा मा॑ हिसीः। त्वम॑ग्ने गृ॒हप॑तिर्वि॒शाम॑सि। विश्वा॑सां॒ मानु॑षीणाम्। श॒तं पू॒र्भिर्य॑विष्ठ पा॒ह्यह॑सः। स॒मे॒द्धार श॒त हिमा। त॒न्द्रा॒विण हार्दिवा॒नम्। इ॒हैव रा॒तय॑ सन्तु। त्वष्टी॑मती ते सपेय। सु॒रेता॒ रेतो॒ दधा॑ना। वी॒रं  वि॑देय॒ तव॑ स॒न्दृशि॑। माऽह रा॒यस्पोषे॑ण॒ वि यो॑षम्॥२६॥
७.८.०
एहि॑ पाहि पिन्वस्व गृह्णामि॒ नव॑ च॥ ८॥
\anuvakamend
७.८.१
दे॒वस्य॑ त्वा सवि॒तुः प्र॑स॒वे। अ॒श्विनोर्बा॒हुभ्याम्। पू॒ष्णो हस्ताभ्या॒मा द॑दे। अदि॑त्यै॒ रास्ना॑सि। इड॒ एहि॑। अदि॑त॒ एहि॑। सर॑स्व॒त्येहि॑। असा॒वेहि॑। असा॒वेहि॑। असा॒वेहि॑॥२७॥
७.८.२
अदि॑त्या उ॒ष्णीष॑मसि। वा॒युर॑स्यै॒डः। पू॒षा त्वो॒पाव॑सृजतु। अ॒श्विभ्यां॒ प्र दा॑पय। यस्ते॒ स्तन॑ शश॒यो यो म॑यो॒भूः। येन॒ विश्वा॒ पुष्य॑सि॒ वार्या॑णि। यो र॑त्न॒धा व॑सु॒विद्यः सु॒दत्र॑। सर॑स्वति॒ तमि॒ह धात॑वेकः। उस्र॑ घ॒र्म शिष। उस्र॑ घ॒र्मं पा॑हि॥२८॥
७.८.३
घ॒र्माय॑ शिष। बृह॒स्पति॒स्त्वोप॑सीदतु। दान॑वः स्थ॒ पेर॑वः। वि॒ष्व॒ग्वृतो॒ लोहि॑तेन। अ॒श्विभ्यां पिन्वस्व। सर॑स्वत्यै पिन्वस्व। पू॒ष्णे पि॑न्वस्व। बृह॒स्पत॑ये पिन्वस्व। इन्द्रा॑य पिन्वस्व। इन्द्रा॑य पिन्वस्व॥२९॥
७.८.४
गा॒य॒त्रो॑ऽसि। त्रैष्टु॑भोऽसि। जाग॑तमसि। स॒होर्जो भा॒गेनोप॒मेहि॑। इन्द्राश्विना॒ मधु॑नः सार॒घस्य॑। घ॒र्मं पा॑त वसवो॒ यज॑ता॒ वट्। स्वाहा त्वा॒ सूर्य॑स्य र॒श्मये॑ वृष्टि॒वन॑ये जुहोमि। मधु॑ ह॒विर॑सि। सूर्य॑स्य॒ तप॑स्तप। द्यावा॑पृथि॒वीभ्यान्त्वा॒ परि॑गृह्णामि॥३०॥
७.८.५
अ॒न्तरि॑क्षेण॒ त्वोप॑यच्छामि। दे॒वानान्त्वा पितृ॒णामनु॑मतो॒ भर्तु शकेयम्। तेजो॑ऽसि। तेजोऽनु॒ प्रेहि॑। दि॒वि॒स्पृङ्मा मा॑ हिसीः। अ॒न्त॒रि॒क्ष॒स्पृङ्मा मा॑ हिसीः। पृ॒थि॒वि॒स्पृङ्मा मा॑ हिसीः। सुव॑रसि॒ सुव॑र्मे यच्छ। दिवं॑ यच्छ दि॒वो मा॑ पाहि॥३१॥
७.९.०
आ॒दि॒त्यव॑ते॒ स्वाहा॑ हार्दिवा॒नं पृ॑थि॒व्या अ॒ष्टौ च॑॥ ९॥
\anuvakamend
७.९.१
स॒मु॒द्राय॑ त्वा॒ वाता॑य॒ स्वाहा। स॒लि॒लाय॑ त्वा॒ वाता॑य॒ स्वाहा। अ॒ना॒धृ॒ष्याय॑ त्वा॒ वाता॑य॒ स्वाहा। अ॒प्र॒ति॒धृ॒ष्याय॑ त्वा॒ वाता॑य॒ स्वाहा। अ॒व॒स्यवे त्वा॒ वाता॑य॒ स्वाहा। दुव॑स्वते त्वा॒ वाता॑य॒ स्वाहा। शिमि॑द्वते त्वा॒ वाता॑य॒ स्वाहा। अ॒ग्नये त्वा॒ वसु॑मते॒ स्वाहा। सोमा॑य त्वा रु॒द्रव॑ते॒ स्वाहा। वरु॑णाय त्वाऽऽदि॒त्यव॑ते॒ स्वाहा॥३२॥
७.९.२
बृह॒स्पत॑ये त्वा वि॒श्वदेव्यावते॒ स्वाहा। स॒वि॒त्रे त्व॑र्भु॒मते॑ विभु॒मते प्रभु॒मते॒ वाज॑वते॒ स्वाहा। य॒माय॒ त्वाऽङ्गि॑रस्वते पितृ॒मते॒ स्वाहा। विश्वा॒ आशा॑ दक्षिण॒सत्। विश्वान्दे॒वान॑याडि॒ह। स्वाहा॑कृतस्य घ॒र्मस्य॑। मधो पिबतमश्विना। स्वाहा॒ऽग्नये॑ य॒ज्ञिया॑य। शं यजु॑र्भिः। अश्वि॑ना घ॒र्मं पा॑त हार्दिवा॒नम्॥३३॥
७.९.३
अह॑र्दि॒वाभि॑रू॒तिभि॑। अनु॑वा॒न्द्यावा॑पृथि॒वी मसाताम्। स्वाहेन्द्रा॑य। स्वाहेन्द्रा॒वट्। घ॒र्मम॑पातमश्विना हार्दिवा॒नम्। अह॑र्दि॒वाभि॑रू॒तिभि॑। अनु॑वा॒न्द्यावा॑पृथि॒वी अ॑मसाताम्। तं प्रा॒व्यं॑ यथा॒ वट्। नमो॑ दि॒वे। नम॑ पृथि॒व्यै॥३४॥
७.९.४
दि॒विधा॑ इ॒मं य॒ज्ञम्। य॒ज्ञमि॒मन्दि॒वि धा। दिवं॑ गच्छ। अ॒न्तरि॑क्षं गच्छ। पृ॒थि॒वीं ग॑च्छ। पञ्च॑ प्र॒दिशो॑ गच्छ। दे॒वान्घ॑र्म॒पान्ग॑च्छ। पि॒तॄन्घ॑र्म॒पान्ग॑च्छ॥३५॥
७.१०.०
ब्र॒ह्म॒व॒र्च॒साय॑ पीपिहि स्क॒न्दयाद्रु॒द्राय॑ रु॒द्रहोत्रे॒ स्वाहाऽह्नो॑ मा पाह्य॒ग्नौ स॒प्त च॑॥ १०॥
\anuvakamend
७.१०.१
इ॒षे पी॑पिहि। ऊ॒र्जे पी॑पिहि। ब्रह्म॑णे पीपिहि। क्ष॒त्राय॑ पीपिहि। अ॒द्भ्यः पी॑पिहि। ओष॑धीभ्यः पीपिहि। वन॒स्पति॑भ्यः पीपिहि। द्यावा॑पृ॒थिवीभ्यां पीपिहि। सु॒भू॒ताय॑ पीपिहि। ब्र॒ह्म॒व॒र्च॒साय॑ पीपिहि॥३६॥
७.१०.२
यज॑मानाय पीपिहि। मह्यं॒ ज्यैष्ठ्या॑य पीपिहि। त्विष्यै त्वा। द्यु॒म्नाय॑ त्वा। इ॒न्द्रि॒याय॑ त्वा॒ भूत्यै त्वा। धर्मा॑सि सु॒धर्मा मेन्य॒स्मे। ब्रह्मा॑णि धारय। क्ष॒त्राणि॑ धारय। विश॑न्धारय। नेत्त्वा॒ वात॑ स्क॒न्दयात्॥३७॥
७.१०.३
अ॒मुष्य॑ त्वा प्रा॒णे सा॑दयामि। अ॒मुना॑ स॒ह नि॑र॒र्थं ग॑च्छ। योऽस्मान्द्वेष्टि॑। यं च॑ व॒यं द्वि॒ष्मः। पू॒ष्णे शर॑से॒ स्वाहा। ग्राव॑भ्यः॒ स्वाहा। प्र॒ति॒रेभ्यः॒ स्वाहा। द्यावा॑पृथि॒वीभ्या॒ स्वाहा। पि॒तृभ्यो॑ घर्म॒पेभ्यः॒ स्वाहा। रु॒द्राय॑ रु॒द्रहोत्रे॒ स्वाहा॥३८॥
७.१०.४
अह॒र्ज्योति॑ के॒तुना॑ जुषताम्। सु॒ज्यो॒तिर्ज्योति॑षा॒ स्वाहा। रात्रि॒र्ज्योति॑ के॒तुना॑ जुषताम्। सु॒ज्यो॒तिर्ज्योति॑षा॒ स्वाहा। अपी॑परो॒ माऽह्नो॒ रात्रि॑यै मा पाहि। ए॒षा ते॑ अग्ने स॒मित्। तया॒ समि॑ध्यस्व। आयु॑र्मे दाः। वर्च॑सा माञ्जीः। अपी॑परो मा॒ रात्रि॑या॒ अह्नो॑ मा पाहि॥३९॥
७.१०.५
ए॒षा ते॑ अग्ने स॒मित्। तया॒ समि॑ध्यस्व। आयु॑र्मे दाः। वर्च॑सा माञ्जीः। अ॒ग्निर्ज्योति॒र्ज्योति॑र॒ग्निः स्वाहा। सूर्यो॒ ज्योति॒र्ज्योतिः॒ सूर्यः॒ स्वाहा। भूः स्वाहा। हु॒त ह॒विः। मधु॑ ह॒विः। इन्द्र॑तमे॒ऽग्नौ॥४०॥
७.१०.६
पि॒ता नो॑ऽसि॒ मा मा॑ हिसीः। अ॒श्याम॑ ते देवघर्म। मधु॑मतो॒ वाज॑वतः पितु॒मत॑। अङ्गि॑रस्वतः स्वधा॒विन॑। अ॒शी॒महि॑ त्वा॒ मा मा॑ हिसीः। स्वाहा त्वा॒ सूर्य॑स्य र॒श्मिभ्य॑। स्वाहा त्वा॒ नक्ष॑त्रेभ्यः॥४१॥
७.११.०
याऽऽग्नीध्रे॒ तान्त॑ ए॒तेनाव॑ यजे॒ स्वाहा॒ धर्म॑णा शं॒ युधा॑याः प्यासिषी॒महि॒ पोषे॑ण॒ निष॑त्तो वि॒द्म स॑न्त्व॒ष्टौ
\anuvakamend
७.११.१
घर्म॒ या ते॑ दि॒वि शुक्। या गा॑य॒त्रे छन्द॑सि। या ब्राह्म॒णे। या ह॑वि॒र्द्धाने। तान्त॑ ए॒तेनाव॑ यजे॒ स्वाहा। घर्म॒ या ते॒ऽन्तरि॑क्षे॒ शुक्। या त्रैष्टु॑भे॒ छन्द॑सि। या रा॑ज॒न्ये। याऽऽग्नीध्रे। तान्त॑ ए॒तेनाव॑ यजे॒ स्वाहा॥४२॥
७.११.२
घर्म॒ या ते॑ पृथि॒व्या शुक्। या जाग॑ते॒ छन्द॑सि। या वैश्ये। या सद॑सि। तान्त॑ ए॒तेनाव॑ यजे॒ स्वाहा। अनु॑नो॒ऽद्यानु॑मतिः। अन्विद॑नुमते॒ त्वम्। दि॒वस्त्वा॑ पर॒स्पाया। अ॒न्तरि॑क्षस्य त॒नुव॑ पाहि। पृ॒थि॒व्यास्त्वा॒ धर्म॑णा॥४३॥
७.११.३
व॒यमनु॑क्रामाम सुवि॒ताय॒ नव्य॑से। ब्रह्म॑णस्त्वा पर॒स्पाया। क्ष॒त्रस्य॑ त॒नुव॑ पाहि। वि॒शस्त्वा॒ धर्म॑णा। व॒यमनु॑क्रामाम सुवि॒ताय॒ नव्य॑से। प्रा॒णस्य॑ त्वा पर॒स्पायै। चक्षु॑षस्त॒नुव॑ पाहि। श्रोत्र॑स्य त्वा॒ धर्म॑णा। व॒यमनु॑क्रामाम सुवि॒ताय॒ नव्य॑से। व॒ल्गुर॑सि शं॒ युधा॑याः॥४४॥
७.११.४
शिशु॒र्जन॑धायाः। शं च॒ वक्षि॒ परि॑ च॒ वक्षि॑। चतु॑ स्रक्ति॒र्नाभि॑र्\mbox{}ऋ॒तस्य॑। सदो॑ वि॒श्वायुः॒ शर्म॑ स॒प्रथा। अप॒ द्वेषो॒ अप॒ह्वर॑। अ॒न्यद्व्र॑तस्य सश्चिम। घर्मै॒तत्तेऽन्न॑मे॒तत्पुरी॑षम्। तेन॒ वर्द्ध॑स्व॒ चा च॑ प्यायस्व। व॒र्द्धि॒षी॒महि॑ च व॒यम्। आ च॑ प्यासिषी॒महि॑॥४५॥
७.११.५
रन्ति॒र्नामा॑सि दि॒व्यो ग॑न्ध॒र्वः। तस्य॑ ते प॒द्वद्ध॑वि॒र्द्धानम्। अ॒ग्निरध्य॑क्षाः। रु॒द्रोऽधि॑पतिः। सम॒हमायु॑षा। सं प्रा॒णेन॑। सं वर्च॑सा। सं पय॑सा। सङ्गौ॑प॒त्येन॑। स रा॒यस्पोषे॑ण॥४६॥
७.११.६
व्य॑सौ। योऽस्मान्द्वेष्टि॑। यं च॑ व॒यं द्वि॒ष्मः। अचि॑क्रद॒द्वृषा॒ हरि॑। म॒हान्मि॒त्रो न द॑र्\mbox{}श॒तः। स सूर्ये॑ण रोचते। चिद॑सि समु॒द्रयो॑निः। इन्दु॒र्दक्ष॑ श्ये॒न ऋ॒तावा। हिर॑ण्यपक्षः  शकु॒नो भु॑र॒ण्युः। म॒हान्त्स॒धस्थे ध्रु॒व आनिष॑त्तः॥४७॥
७.११.७
नम॑स्ते अस्तु॒ मा मा॑ हिसीः। वि॒श्वाव॑सु सोम गन्ध॒र्वम्। आपो॑ ददृ॒शुषी। तदृ॒तेना॒व्या॑यन्। तद॒न्ववैत्। इन्द्रो॑ रारहा॒ण आ॑साम्। परि॒ सूर्य॑स्य परि॒धी र॑पश्यत्। वि॒श्वाव॑सुर॒भि तन्नो॑ गृणातु। दि॒व्यो ग॑न्ध॒र्वो रज॑सो वि॒मान॑। यद्वा॑ घा स॒त्यमु॒त यन्न वि॒द्म॥४७॥
७.११.८
धियो॑ हिन्वा॒नो धिय॒ इन्नो॑ अव्यात्। सस्नि॑मविन्द॒च्चर॑णे न॒दीनाम्। अपा॑वृणो॒द्दुरो॒ अश्म॑व्रजानाम्। प्रासान्गन्ध॒र्वो अ॒मृता॑नि वोचत्। इन्द्रो॒ दक्षं॒ परि॑जानाद॒हीनम्। ए॒तत्त्वन्दे॑व घर्म दे॒वो दे॒वानुपा॑गाः। इ॒दम॒हं म॑नु॒ष्यो॑ मनु॒ष्यान्॑। सोम॑पी॒थानु॒मेहि॑। स॒ह प्र॒जया॑ स॒ह रा॒यस्पोषे॑ण। सु॒मि॒त्रा न॒ आप॒ ओष॑धयः सन्तु॥४९॥
७.११.९
दु॒र्मि॒त्रास्तस्मै॑ भूयासुः। योऽस्मान्द्वेष्टि॑। यं च॑ व॒यं द्वि॒ष्मः। उद्व॒यन्तम॑स॒स्परि॑। उदु॒त्यं चि॒त्रम्। इ॒ममू॒षुत्यम॒स्मभ्य स॒निम्। गा॒य॒त्रन्नवी॑यासम्। अग्ने॑ दे॒वेषु॒ प्रवो॑चः॥५०॥
७.१२.०
\anuvakamend
७.१२.१
म॒ही॒नां पयो॑ऽसि॒ विहि॑तन्देव॒त्रा। ज्यो॒ति॒र्भा अ॑सि॒ वन॒स्पती॑ना॒मोष॑धीना॒ रस॑। वा॒जिनं॑ त्वा वा॒जिनोऽव॑ नयामः। ऊ॒र्ध्वं मन॑ सुव॒र्गम्॥५१॥
७.१३.०
\anuvakamend
७.१३.१
अस्का॒न्द्यौः पृ॑थि॒वीम्। अस्का॑नृष॒भो युवा॒गाः। स्क॒न्नेमा विश्वा॒ भुव॑ना। स्क॒न्नो य॒ज्ञः प्रज॑नयतु। अस्का॒नज॑नि॒ प्राज॑नि। आ स्क॒न्नाज्जा॑यते॒ वृषा। स्क॒न्नात् प्रज॑निषीमहि॥५२॥
७.१४.०
-। १४।
\anuvakamend
७.१४.१
या पु॒रस्ताद्वि॒द्युदाप॑तत्। तान्त॑ ए॒तेनाव॑ यजे॒ स्वाहा। या द॑क्षिण॒तः। या प॒श्चात्। योत्त॑र॒तः। योपरि॑ष्टाद्वि॒द्युदाप॑तत्। तान्त॑ ए॒तेनाव॑ यजे॒ स्वाहा॥५३॥
७.१५.०
-। १५।
\anuvakamend

७.१५.१
प्रा॒णाय॒ स्वाहा व्या॒नाय॒ स्वाहा॑ऽपा॒नाय॒ स्वाहा। चक्षु॑षे॒ स्वाहा॒ श्रोत्रा॑य॒ स्वाहा। मन॑से॒ स्वाहा॑ वा॒चे सर॑स्वत्यै॒ स्वाहा॥५४॥
७.१६.०
-। १६।
\anuvakamend


७.१६.१
पू॒ष्णे स्वाहा॑ पू॒ष्णे शर॑से॒ स्वाहा। पू॒ष्णे प्र॑प॒त्थ्या॑य॒ स्वाहा॑ पू॒ष्णे न॒रन्धि॑षाय॒ स्वाहा। पू॒ष्णेऽङ्घृ॑णये॒ स्वाहा॑ पू॒ष्णे न॒रुणा॑य॒ स्वाहा। पू॒ष्णे सा॑के॒ताय॒ स्वाहा॥५५॥
७.१७.०
-। १७।
\anuvakamend

७.१७.१
उद॑स्य॒ शुष्माद्भा॒नुर्नार्त॒ बिभ॑र्ति। भा॒रं पृ॑थि॒वी न भूम॑। प्र शु॒क्रैतु॑ दे॒वी म॑नी॒षा। अ॒स्मत्सुत॑ष्टो॒ रथो॒ न वा॒जी। अर्च॑न्त॒ एके॒ महि॒ साम॑मन्वत। तेन॒ सूर्य॑मधारयन्। तेन॒ सूर्य॑मरोचयन्। घ॒र्मः  शिर॒स्तद॒यम॒ग्निः। पुरी॑षमसि॒ संप्रि॑यं प्र॒जया॑ प॒शुभि॑र्भुवत्। प्र॒जापति॑स्त्वा सादयतु। तया॑ दे॒वत॑याऽङ्गिर॒स्वद्ध्रु॒वा सी॑द॥५६॥
७.१८.०
-। १८।
\anuvakamend
७.१८.१
यास्ते॑ अग्न आ॒र्द्रा योन॑यो॒ याः कु॑ला॒यिनी। ये ते॑ अग्न॒ इन्द॑वो॒ या उ॒ नाभ॑यः। यास्ते॑ अग्ने त॒नुव॒ ऊर्जो॒ नाम॑। ताभि॒स्त्वमु॒भयी॑भिः संविदा॒नः। प्र॒जाभि॑रग्ने॒ द्रवि॑णे॒ह सी॑द। प्र॒जाप॑तिस्त्वा सादयतु। तया॑ दे॒वत॑याऽङ्गिर॒स्वद्ध्रु॒वा सी॑द॥५७॥
७.१९.०
चित॑यो॒ नव॑ च॥ १९।
\anuvakamend
७.१९.१
अ॒ग्निर॑सि वैश्वान॒रो॑ऽसि। सं॒व॒त्स॒रो॑ऽसि परिवत्स॒रो॑ऽसि। इ॒दा॒व॒त्स॒रो॑ऽसीदुवत्स॒रो॑ऽसि। इ॒द्व॒त्स॒रो॑ऽसि वत्स॒रो॑ऽसि। तस्य॑ ते वस॒न्तः  शिर॑। ग्री॒ष्मो दक्षि॑णः प॒क्षः। व॒र्\mbox{}षाः पुच्छम्। श॒रदुत्त॑रः प॒क्षः। हे॒म॒न्तो मध्यम्। पू॒र्व॒प॒क्षाश्चित॑यः। अ॒प॒र॒प॒क्षाः पुरी॑षम्। अ॒हो॒रा॒त्राणीष्ट॑काः। तस्य॑ ते॒ मासाश्चार्द्धमा॒साश्च॑ कल्पन्ताम्। ऋ॒तव॑स्ते कल्पन्ताम्। सं॒व॒त्स॒रस्ते॑ कल्पताम्। अ॒हो॒रा॒त्राणि॑ ते कल्पन्ताम्। एति॒ प्रेति॒ वीति॒ समित्युदिति॑। प्र॒जाप॑तिस्त्वा सादयतु। तया॑ दे॒वत॑याऽङ्गिर॒स्वद्ध्रु॒वा सी॑द॥५८॥
७.२०.०
पु॒रो॒वसु॑र्\mbox{}हीडिषाता सुप॒र्णाः॥ २०॥
\anuvakamend

७.२०.१
भूर्भुवः॒ सुव॑। ऊ॒र्ध्व ऊ॒षुण॑ ऊ॒तये। ऊ॒र्ध्वो न॑ पा॒ह्यह॑सः। वि॒धुन्द्र॑द्रा॒ण सम॑ने बहू॒नाम्। युवा॑न॒ सन्तं॑ पलि॒तो ज॑गार। दे॒वस्य॑ पश्य॒ काव्यं॑ महि॒त्वाद्या म॒मार॑। सह्यः॒ समा॑न। यदृ॒ते चि॑दभि॒श्रिष॑। पु॒रा ज॒र्तृभ्य॑ आ॒तृद॑। सन्धा॑ता स॒न्धिं म॒घवा॑ पुरो॒वसु॑॥५९॥
७.२०.२
निष्क॑र्ता॒ विह्रु॑तं॒ पुन॑। पुन॑रू॒र्जा स॒ह र॒य्या। मा नो॑ घर्म व्यथि॒तो वि॑व्यथो नः। मा नः॒ पर॒मध॑र॒म्मा रजो॑नैः। मोष्व॑स्मा स्तम॑स्यन्त॒रा धा। मा रु॒द्रिया॑सो अ॒भिगु॑र्वृ॒धान॑। मा नः॒ क्रतु॑भिर्\mbox{}हीडि॒तेभि॑र॒स्मान्। द्विषा॑सुनीते॒ मा परा॑ दाः। मा नो॑ रु॒द्रो निर्\mbox{}ऋ॑ति॒र्मा नो॒ अस्ता। मा द्यावा॑पृथि॒वी ही॑डिषाताम्॥६०॥
७.२०.३
उप॑ नो मित्रावरुणावि॒हाव॑तम्। अ॒न्वादीध्याथामि॒ह न॑ सखाया। आ॒दि॒त्यानां॒ प्रसि॑तिर्\mbox{}हे॒तिः। उ॒ग्रा श॒तापाष्ठा घ॒विषा॒ परि॑ णो वृणक्तु। इ॒मं मे॑ वरुण॒ तत्त्वा॑ यामि। त्वं नो॑ अग्ने॒ स त्वं नो॑ अग्ने। त्वम॑ग्ने अ॒यासि॑। उद्व॒यन्तम॑स॒स्परि॑। उदु॒त्यं चि॒त्रम्। वय॑ सुपर्णाः॥६१॥
७.२१.०
यश॑सा स॒ह षट्च॑॥ २१॥
\anuvakamend

७.२१.१
भूर्भुवः॒ सुव॑। मयि॒ त्यदि॑न्द्रि॒यं म॒हत्। मयि॒ दक्षो॒ मयि॒ क्रतु॑। मयि॑ धायि सु॒वीर्यम्। त्रिशु॑ग्घ॒र्मो विभा॑तु मे। आकूत्या॒ मन॑सा स॒ह। वि॒राजा॒ ज्योति॑षा स॒ह। य॒ज्ञेन॒ पय॑सा स॒ह। ब्रह्म॑णा॒ तेज॑सा स॒ह। क्ष॒त्रेण॒ यश॑सा स॒ह। स॒त्येन॒ तप॑सा स॒ह। तस्य॒ दोह॑मशीमहि। तस्य॑ सु॒म्नम॑शीमहि। तस्य॑ भ॒क्षम॑शीमहि। तस्य॑ त॒ इन्द्रे॑ण पी॒तस्य॒ मधु॑मतः। उप॑हूत॒स्योप॑हूतो भक्षयामि॥६२॥
७.२२.०
-। २२।
\anuvakamend


७.२२.१
यास्ते॑ अग्ने घो॒रास्त॒नुव॑। क्षुच्च॒ तृष्णा च॑। अस्नु॒क्चाना॑हुतिश्च। अ॒श॒न॒या च॑ पिपा॒सा च॑। से॒दिश्चाम॑तिश्च। ए॒तास्ते॑ अग्ने घो॒रास्त॒नुव॑। ताभि॑र॒मुं ग॑च्छ। योऽस्मान्द्वेष्टि॑। यं च॑ व॒यं द्वि॒ष्मः॥६३॥
७.२३.०
-। २३।
\anuvakamend

७.२३.१
स्निक्च॒ स्नीहि॑तिश्च॒ स्निहि॑तिश्च। उ॒ष्णा च॑ शी॒ता च॑। उ॒ग्रा च॑ भी॒मा च॑। स॒दाम्नी॑ से॒दिरनि॑रा। ए॒तास्ते॑ अग्ने घो॒रास्त॒नुव॑। ताभि॑र॒मुं ग॑च्छ। योऽस्मान्द्वेष्टि॑। यं च॑ व॒यं द्वि॒ष्मः॥६४॥
७.२४.०
-। २४।
\anuvakamend

७.२४.१
धुनि॑श्च ध्वा॒न्तश्च॑ ध्व॒नश्च॑ ध्व॒नयश्च। नि॒लिं॒पश्च॑ विलिं॒पश्च॑ विक्षि॒पः॥६५॥
७.२५.०
-। २५।
\anuvakamend

७.२५.१
उ॒ग्रश्च॒ धुनि॑श्च ध्वा॒न्तश्च॑ ध्व॒नश्च॑ ध्व॒नयश्च। स॒ह॒स॒ह्वाश्च॒ सह॑मानश्च॒ सह॑स्वाश्च॒ सही॑याश्च। एत्य॒ प्रेत्य॑ विक्षि॒पः॥६६॥
७.२६.०
-। २६।
\anuvakamend

७.२६.१
अ॒हो॒रा॒त्रे त्वोदी॑रयताम्। अ॒र्द्ध॒मा॒सास्त्वोदीं जयन्तु। मासास्त्वा श्रपयन्तु। ऋ॒तव॑स्त्वा पचन्तु। सं॒व॒त्स॒रस्त्वा॑ हन्त्वसौ॥६७॥
७.२७.०
-। २७।
\anuvakamend

७.२७.१
खट् फट् ज॒हि। छि॒न्धी भि॒न्धी ह॒न्धी कट्। इति॒ वाच॑ क्रूरा॒णि॥६८॥
७.२८.०
त्रिस्ते॒ नम॑ स॒प्त च॑। २८।
७.२८.१
विगा इ॑न्द्र वि॒चरन्त्स्पाशयस्व। स्व॒पन्त॑मिन्द्र पशु॒मन्त॑मिच्छ। वज्रे॑णा॒मुं बो॑धय दुर्वि॒दत्रम्। स्व॒प॒तोऽस्य॒ प्रह॑र॒ भोज॑नेभ्यः। अग्ने॑ अ॒ग्निना॒ संव॑दस्व। मृत्यो॑ मृ॒त्युना॒ संव॑दस्व। नम॑स्ते अस्तु भगवः। स॒कृत्ते॑ अग्ने॒ नम॑। द्विस्ते॒ नम॑। त्रिस्ते॒ नम॑। च॒तुस्ते॒ नम॑। प॒ञ्च॒कृत्व॑स्ते॒ नम॑। द॒श॒कृत्व॑स्ते॒ नम॑। श॒त॒कृत्व॑स्ते॒ नम॑। आ॒स॒ह॒स्र॒कृत्व॑स्ते॒ नम॑। अ॒प॒रि॒मि॒त॒कृत्व॑स्ते॒ नम॑। नम॑स्ते अस्तु॒ मा मा॑ हिसीः॥६९॥
७.२९.०
-। २९।
\anuvakamend

७.२९.१
असृ॑न्मुखो रुधि॒रेणा॒व्य॑क्तः। य॒मस्य॑ दू॒तः  श्वपा॒द्विधा॑वसि। गृध्र॑ सुप॒र्णः कु॒णपं॒ निषे॑वसे। य॒मस्य॑ दू॒तः प्रहि॑तो भ॒वस्य॑ चो॒भयो॥७०॥
७.३०.०
-। ३०।
\anuvakamend

७.३०.१
यदे॒तद्वृ॑क॒सो भू॒त्वा। वाग्देव्यभि॒राय॑सि। द्वि॒षन्तं॑ मे॒ऽभिरा॑य। तं मृ॑त्यो मृ॒त्यवे॑ नय। स आर्त्यार्ति॒मार्च्छ॑तु॥७१॥
७.३१.०
-। ३१।
\anuvakamend

७.३१.१
यदी॑षि॒तो यदि॑ वा स्वका॒मी। भ॒येड॑को वद॑ति॒ वाच॑मे॒ताम्। तामि॑न्द्रा॒ग्नी ब्रह्म॑णा संविदा॒नौ। शि॒वाम॒स्मभ्य॑ङ्कृणुतङ्गृ॒हेषु॑॥७२॥
७.३२.०
-। ३२।
\anuvakamend

७.३२.१
दीर्घ॑मुखि॒ दुर्\mbox{}ह॑णु। मा स्म॑ दक्षिण॒तो व॑दः। यदि॑ दक्षिण॒तो वदाद्द्वि॒षन्तं॒ मेऽव॑ बाधासै॥७३॥
७.३३.०
-। ३३।
\anuvakamend

७.३३.१
इ॒त्थादुलू॑क॒ आप॑प्तत्। हि॒र॒ण्या॒क्षो अयो॑मुखः। रक्ष॑सान्दू॒त आग॑तः। तमि॒तो ना॑शयाग्ने॥७४॥
७.३४.०
-। ३४।
\anuvakamend

७.३४.१
यदे॒तद्भू॒तान्य॑न्वा॒विश्य॑। दैवीं॒ वाचं॑ व॒दसि॑। द्वि॒षतो॑ नः॒ परा॑वद। तान्मृ॑त्यो मृ॒त्यवे॑ नय। त आर्त्याऽऽर्ति॒मार्च्छ॑न्तु। अ॒ग्निना॒ऽग्निः संव॑दताम्॥७५॥
७.३५.०
-। ३५।
\anuvakamend

७.३५.१
प्र॒सार्य॑ स॒क्थ्यौ॑ पत॑सि। स॒व्यमक्षि॑ नि॒पेपि॑ च। मेहक॑स्य च॒नाम॑मत्॥७६॥
७.३६.०
-। ३६।
\anuvakamend

७.३६.१
अत्रि॑णा त्वा क्रिमे हन्मि। कण्वे॑न ज॒मद॑ग्निना। वि॒श्वाव॑सो॒र्ब्रह्म॑णा ह॒तः। क्रिमी॑णा॒ राजा। अप्ये॑षा स्थ॒पति॑र्ह॒तः। अथो॑ मा॒ताऽथो॑ पि॒ता। अथो स्थू॒रा अथो क्षु॒द्राः। अथो॑ कृ॒ष्णा अथो श्वे॒ताः। अथो॑ आ॒शाति॑का ह॒ताः। श्वे॒ताभि॑ स॒ह सर्वे॑ ह॒ताः॥७७॥
७.३७.०
-। ३७।
\anuvakamend

७.३७.१
आह॒राव॑द्य। शृ॒तस्य॑ ह॒विषो॒ यथा। तत्स॒त्यम्। यद॒मुं य॒मस्य॒ जम्भ॑योः। आद॑धामि॒ तथा॒ हि तत्। खण्फण्म्रसि॑॥७८॥
७.३८.०
-। ३८।
\anuvakamend

७.३८.१
ब्रह्म॑णा त्वा शपामि। ब्रह्म॑णस्त्वा श॒पथे॑न शपामि। घो॒रेण॑ त्वा॒ भृगू॑णां॒ चक्षु॑षा॒ प्रेक्षे। रौ॒द्रेण॒ त्वाङ्गि॑रसां॒ मन॑सा ध्यायामि। अ॒घस्य॑ त्वा॒ धार॑या विद्ध्यामि। अध॑रो॒ मत्प॑द्यस्वाऽसौ॥७९॥
७.३९.०
-। ३९।
\anuvakamend

७.३९.१
उत्तु॑द शिमिजावरि। तल्पे॑जे॒ तल्प॒ उत्तु॑द। गि॒री रनु॒ प्रवे॑शय। मरी॑ची॒रुप॒ सन्नु॑द। याव॑दि॒तः पु॒रस्ता॑दु॒दया॑ति॒ सूर्य॑। ताव॑दि॒तो॑ऽमुन्ना॑शय। योऽस्मान्द्वेष्टि॑। यं च॑ व॒यं द्वि॒ष्मः॥८०॥
७.४०.०
-। ४०।
\anuvakamend

७.४०.१
भूर्भुवः॒ सुवो॒ भूर्भुवः॒ सुवो॒ भूर्भुवः॒ सुव॑। भुवोऽद्धायि॒ भुवोऽद्धायि॒ भुवोऽद्धायि। नृ॒म्णायि नृ॒म्णं नृ॒म्णायि नृ॒म्णं नृ॒म्णायि नृ॒म्णम्। नि॒धाय्यो॑ वायि नि॒धाय्यो॑ वायि नि॒धाय्यो॑ वायि। ए अ॒स्मे अ॒स्मे। सुव॒र्न ज्योती॥८१॥
७.४१.०
स॒मित्समि॑न्धे व्र॒तं च॑रिष्या॒म्यायु॑षा॒ तेज॑सा॒ वर्च॑सा श्रि॒या यश॑सा ब्रह्मवर्च॒सेना॒ष्टौ च॑। ४१।
\anuvakamend

७.४१.१
पृ॒थि॒वी स॒मित्। ताम॒ग्निः समि॑न्धे। साऽग्नि समि॑न्धे। ताम॒ह समि॑न्धे। सा मा॒ समि॑द्धा। आयु॑षा॒ तेज॑सा। वर्च॑सा श्रि॒या। यश॑सा ब्रह्मवर्च॒सेन॑। अ॒न्नाद्ये॑न॒ समि॑न्ता॒ स्वाहा। अ॒न्तरि॑क्ष॒ स॒मित्॥८२॥
७.४१.२
तां॒ वा॒युः समि॑न्धे। सा वा॒यु समि॑न्धे। ताम॒ह समि॑न्धे। सा मा॒ समि॑द्धा। आयु॑षा॒ तेज॑सा। वर्च॑सा श्रि॒या। यश॑सा ब्रह्मवर्च॒सेन॑। अ॒न्नाद्ये॑न॒ समि॑न्ता॒ स्वाहा। द्यौः स॒मित्। तामा॑दि॒त्यः समि॑न्धे॥८३॥
७.४१.३
साऽऽदि॒त्य समि॑न्धे। ताम॒ह समि॑न्धे। सा मा॒ समि॑द्धा। आयु॑षा॒ तेज॑सा। वर्च॑सा श्रि॒या। यश॑सा ब्रह्मवर्च॒सेन॑। अ॒न्नाद्ये॑न॒ समिन्ता॒ स्वाहा। प्रा॒जा॒प॒त्या मे॑ स॒मिद॑सि सपत्न॒क्षय॑णी। भ्रा॒तृ॒व्य॒हा मे॑ऽसि॒ स्वाहा। अग्ने व्रतपते व्र॒तं च॑रिष्यामि॥८४॥
७.४१.४
तच्छ॑केयं॒ तन्मे॑ राध्यताम्। वायो व्रतपत॒ आदि॑त्य व्रतपते। व्र॒तानां व्रतपते व्र॒तं च॑रिष्यामि। तच्छ॑केयं॒ तन्मे॑ राध्यताम्। द्यौः स॒मित्। तामा॑दि॒त्यः समि॑न्धे। साऽऽदि॒त्य समि॑न्धे। ताम॒ह समि॑न्धे। सा मा॒ समि॑द्धा। आयु॑षा॒ तेज॑सा॥८५॥
७.४१.५
वर्च॑सा श्रि॒या। यश॑सा ब्रह्मवर्च॒सेन॑। अ॒न्नाद्ये॑न॒ समि॑न्ता॒ स्वाहा। अ॒न्तरि॑क्ष स॒मित्। तां॒ वा॒युः समि॑न्धे। सा वा॒यु समि॑न्धे। ताम॒ह समि॑न्धे। सा मा॒ समि॑द्धा। आयु॑षा॒ तेज॑सा। वर्च॑सा श्रि॒या॥८६॥
७.४१.६
यश॑सा ब्रह्मवर्च॒सेन॑। अ॒न्नाद्ये॑न॒ समि॑न्ता॒ स्वाहा। पृ॒थि॒वी स॒मित्। ताम॒ग्निः समि॑न्धे। साऽग्नि समि॑न्धे। ताम॒ह समि॑न्धे। सा मा॒ समि॑द्धा। आयु॑षा॒ तेज॑सा। वर्च॑सा श्रि॒या। यश॑सा ब्रह्मवर्च॒सेन॑॥८७॥
७.४१.७
अ॒न्नाद्ये॑न॒ समि॑न्ता॒ स्वाहा। प्रा॒जा॒प॒त्या मे॑ स॒मिद॑सि सपत्न॒क्षय॑णी। भ्रा॒तृ॒व्य॒हा मे॑ऽसि॒ स्वाहा। आदि॑त्य व्रतपते व्र॒तम॑चारिषम्। तद॑शकं॒ तन्मे॑ऽराधि। वायो व्रतप॒तेऽग्ने व्रतपते। व्र॒तानां व्रतपते व्र॒तम॑चारिषम्। तद॑शकं॒ तन्मे॑ऽराधि॥८८॥
७.४२.०
प॒रा॒वतो॑ दधातु ब॒द्धां जिन्व॑थ दृ॒शे स॒प्त च॑॥ ४२॥
\anuvakamend

७.४२.१
शं नो॒ वात॑ पवतां मात॒रिश्वा॒ शं न॑स्तपतु॒ सूर्य॑। अहा॑नि॒शं भ॑वन्तु नः॒ श रात्रिः॒ प्रति॑धीयताम्। शमु॒षा नो॒ व्यु॑च्छतु॒ शमा॑दि॒त्य उदे॑तु नः। शि॒वा नः॒ शन्त॑मा भव सुमृडी॒का सर॑स्वति। मा ते॒ व्यो॑म स॒न्दृशि॑। इडा॑यै॒ वास्त्व॑सि वास्तु॒मद्वास्तु॒मन्तो॑ भूयास्म॒ मा वास्तोश्छित्स्मह्यवा॒स्तुः स भू॑या॒द्योऽस्मान्द्वेष्टि॒ यं च॑ व॒यं द्वि॒ष्मः। प्र॒ति॒ष्ठासि॑ प्रति॒ष्ठाव॑न्तो भूयास्म॒ मा प्र॑ति॒ष्ठायाश्छित्स्मह्यप्रति॒ष्ठः स भू॑या॒द्योऽस्मान्द्वेष्टि॒ यं च॑ व॒यं द्वि॒ष्मः। आ वा॑त वाहि भेष॒जं वि वा॑त वाहि॒ यद्रप॑। त्व हि वि॒श्वभे॑षजो दे॒वानान्दू॒त ईय॑से। द्वावि॒मौ वातौ॑ वात॒ आ सिन्धो॑रा प॑रा॒वत॑॥८९॥
७.४२.२
दक्षं॑ मे अ॒न्य आ॒वातु॒ परा॒न्यो वा॑तु॒ यद्रप॑। यद॒दो वा॑तते गृ॒हे॑ऽमृत॑स्य नि॒धिर्\mbox{}हि॒तः। ततो॑ नो देहि जी॒वसे॒ ततो॑ नो धेहि भेष॒जम्। ततो॑ नो॒ मह॒ आव॑ह॒ वात॒ आवा॑तु भेष॒जम्। श॒म्भूर्म॑यो॒भूर्नो॑ हृ॒दे प्र ण॒ आयूषि तारिषत्। इन्द्र॑स्य गृ॒हो॑ऽसि॒ तं त्वा॒ प्रप॑द्ये॒ सगुः॒ साश्व॑। स॒ह यन्मे॒ अस्ति॒ तेन॑। भूः प्रप॑द्ये॒ भुवः॒ प्रप॑द्ये॒ सुवः॒ प्रप॑द्ये॒ भूर्भुवः॒ सुवः॒ प्रप॑द्ये वा॒युं प्रप॒द्येऽनार्तां दे॒वतां॒ प्रप॒द्येऽश्मा॑नमाख॒णं प्रप॑द्ये प्र॒जाप॑तेर्ब्रह्मको॒शं ब्रह्म॒ प्रप॑द्य॒ ओं प्रप॑द्ये। अ॒न्तरि॑क्षं म उ॒र्व॑न्तरं॑ बृ॒हद॒ग्नयः॒ पर्व॑ताश्च॒ यया॒ वात॑ स्व॒स्त्या स्व॑स्ति॒मान्तया स्व॒स्त्या स्व॑स्ति॒मान॑सानि। प्राणा॑पानौ मृ॒त्योर्मा॑ पातं॒ प्राणा॑पानौ॒ मा मा॑ हासिष्टं॒ मयि॑ मे॒धां मयि॑ प्र॒जां मय्य॒ग्निस्तेजो॑ दधातु॒ मयि॑ मे॒धां मयि॑ प्र॒जां मयीन्द्र॑ इन्द्रि॒यं द॑धातु॒ मयि॑ मे॒धां मयि॑ प्र॒जां मयि॒ सूर्यो॒ भ्राजो॑ दधातु॥९०॥
७.४२.३
द्यु॒भिर॒क्तुभिः॒ परि॑पातम॒स्मानरि॑ष्टेभिरश्विना॒ सौभ॑गेभिः। तन्नो॑ मि॒त्रो वरु॑णो मामहन्ता॒मदि॑तिः॒ सिन्धु॑ पृथि॒वी उ॒त द्यौः। कया॑ नश्चि॒त्र आ भु॑वदू॒ती स॒दावृ॑धः॒ सखा। कया॒ शचि॑ष्ठया वृ॒ता। कस्त्वा॑ स॒त्यो मदा॑नां॒ महि॑ष्ठो मत्स॒दन्ध॑सः। दृ॒ढाचि॑दा॒रुजे॒ वसु॑। अ॒भी षु णः॒ सखी॑नामवि॒ता ज॑रितॄ॒णाम्। श॒तं भ॑वास्यू॒तिभि॑। वय॑ सुप॒र्णा उप॑सेदु॒रिन्द्रं॑ प्रि॒यमे॑धा॒ ऋष॑यो॒ नाध॑मानाः। अप॑ ध्वा॒न्तमूर्णु॒हि पू॒र्धि चक्षु॑र्मुमु॒ग्ध्य॑स्मान्नि॒धये॑व ब॒द्धान्॥९१॥
७.४२.४
शं नो॑ दे॒वीर॒भिष्ट॑य॒ आपो॑ भवन्तु पी॒तये। शं॒योर॒भिस्र॑वन्तु नः। ईशा॑ना॒ वार्या॑णां॒ क्षय॑न्तीश्चर्\mbox{}षणी॒नाम्। अ॒पो या॑चामि भेष॒जम्। सु॒मि॒त्रा न॒ आप॒ ओष॑धयः सन्तु दुर्मि॒त्रास्तस्मै॑ भूयासु॒र्योऽस्मान्द्वेष्टि॒ यं च॑ व॒यं द्वि॒ष्मः। आपो॒ हि ष्ठा म॑यो॒भुव॒स्ता न॑ ऊ॒र्जे द॑धातन। म॒हे रणा॑य॒ चक्ष॑से। यो व॑ शि॒वत॑मो॒ रस॒स्तस्य॑ भाजयते॒ह न॑। उ॒श॒तीरि॑व मा॒तर॑। तस्मा॒ अरं॑ गमाम वो॒ यस्य॒ क्षया॑य॒ जिन्व॑थ॥९२॥
७.४२.५
आपो॑ ज॒नय॑था च नः। पृ॒थि॒वी शा॒न्ता साग्निना॑ शा॒न्ता सा मे॑ शा॒न्ता शुच शमयतु। अ॒न्तरि॑क्ष शा॒न्तं तद्वा॒युना॑ शा॒न्तं तन्मे॑ शा॒न्त शुच शमयतु। द्यौः  शा॒न्ता सादि॒त्येन॑ शा॒न्ता सा मे॑ शा॒न्ता शुच शमयतु। पृ॒थि॒वी शान्ति॑र॒न्तरि॑क्ष॒ शान्ति॒र्द्यौः  शान्ति॒र्दिशः॒ शान्ति॑रवान्तरदि॒शाः  शान्ति॑र॒ग्निः  शान्ति॑र्वा॒युः  शान्ति॑रादि॒त्यः  शान्ति॑श्च॒न्द्रमाः॒ शान्ति॒र्नक्ष॑त्राणि॒ शान्ति॒रापः॒ शान्ति॒रोष॑धयः॒ शान्ति॒र्वन॒स्पत॑यः॒ शान्ति॒र्गौः  शान्ति॑र॒जा शान्ति॒रश्वः॒ शान्तिः॒ पुरु॑षः॒ शान्ति॒र्ब्रह्म॒ शान्ति॑र्ब्राह्म॒णः  शान्तिः॒ शान्ति॑रे॒व शान्तिः॒ शान्ति॑र्मे अस्तु॒ शान्ति॑। तया॒ह शा॒न्त्या स॑र्वशा॒न्त्या मह्यं॑ द्वि॒पदे॒ चतु॑ष्पदे च॒ शान्तिं॑ करोमि॒ शान्ति॑र्मे अस्तु॒ शान्ति॑। एह॒ श्रीश्च॒ ह्रीश्च॒ धृति॑श्च॒ तपो॑ मे॒धा प्र॑ति॒ष्ठा श्र॒द्धा स॒त्यन्धर्म॑श्चै॒तानि॒ मोत्ति॑ष्ठन्त॒मनूत्ति॑ष्ठन्तु॒ मा मा॒ श्रीश्च॒ ह्रीश्च॒ धृति॑श्च॒ तपो॑ मे॒धा प्र॑ति॒ष्ठा श्र॒द्धा स॒त्यन्धर्म॑श्चै॒तानि॑ मा॒ मा हा॑सिषुः। उदायु॑षा स्वा॒युषोदोष॑धीना॒ रसे॒नोत्प॒र्जन्य॑स्य॒ शुष्मे॒णोद॑स्थाम॒मृता॒ अनु॑। तच्चक्षु॑र्दे॒वहि॑तं पु॒रस्ताच्छु॒क्रमु॒च्चर॑त्। पश्ये॑म श॒रद॑ श॒तं जीवे॑म श॒रद॑ श॒तं नन्दा॑म श॒रद॑ श॒तं मोदा॑म श॒रद॑ श॒तं भवा॑म श॒रद॑ श॒त शृ॒णवा॑म श॒रद॑ श॒तं प्रब्र॑वाम श॒रद॑ श॒तमजी॑ताः स्याम श॒रद॑ श॒तं ज्योक्च॒ सूर्यं॑ दृ॒शे। य उद॑गान्मह॒तोऽर्णवाद्वि॒भ्राज॑मानः सरि॒रस्य॒ मध्या॒त्स मा वृष॒भो लो॑हिता॒क्षः सूर्यो॑ विप॒श्चिन्मन॑सा पुनातु। ब्रह्म॑ण॒श्चोत॑न्यसि॒ ब्रह्म॑ण आ॒णी स्थो॒ ब्रह्म॑ण आ॒वप॑नमसि धारि॒तेयं पृ॑थि॒वी ब्रह्म॑णा म॒ही धा॑रि॒तमे॑नेन म॒हद॒न्तरि॑क्षं॒ दिवं॑ दाधार पृथि॒वी सदे॑वां॒ यद॒हं वेद॒ तद॒हन्धा॑रयाणि॒ मा मद्वेदोऽधि॒विस्र॑सत्। मे॒धा॒म॒नी॒षे मावि॑शता स॒मीची॑ भू॒तस्य॒ भव्य॒स्याव॑रुध्यै॒ सर्व॒मायु॑रयाणि॒ सर्व॒मायु॑रयाणि। आ॒भिर्गी॒र्भिर्यदतो॑ न ऊ॒नमाप्या॑यय हरिवो॒ वर्ध॑मानः। य॒दा स्तो॒तृभ्यो॒ महि॑ गो॒त्रा रु॒जासि॑ भूयिष्ठ॒भाजो॒ अध॑ ते स्याम। ब्रह्म॒ प्रावा॑दिष्म॒ तन्नो॒ मा हा॑सीत्। ॐ शान्तिः॒ शान्तिः॒ शान्ति॑॥९३॥

न॒मोऽनु॑मदन्तु। ॐ शान्तिः॒ शान्तिः॒ शान्ति॑॥ 

\closesection
\clearpage

\sect{पञ्चमः प्रश्नः}\setcounter{anuvakam}{0}

ॐ शं न॒स्तन्नो॒ मा हा॑सीत्॥ ॐ शान्तिः॒ शान्तिः॒शान्ति॑॥

दे॒वा वै स॒त्रमा॑सत। ऋद्धि॑परिमितं॒ यश॑स्कामाः। तेऽब्रुवन्। यन्न॑ प्रथ॒मं यश॑ ऋ॒च्छात्। सर्वे॑षां न॒स्तत्स॒हास॒दिति॑। तेषां कुरुक्षे॒त्रं वेदि॑रासीत्। तस्यै॑ खाण़्ड॒वो द॑क्षिणा॒र्द्ध आ॑सीत्। तूर्घ्न॑मुत्तरा॒र्द्धः। प॒री॒णज्ज॑घना॒र्द्धः। म॒रव॑ उत्क॒रः॥१॥
८.१.२
तेषाम्म॒खं॒ वैष्ण॒वं यश॑ आर्च्छत्। तन्न्य॑कामयत। तेनापाक्रामत्। तन्दे॒वा अन्वा॑यन्। यशो॑ऽव॒रुरु॑त्समानाः। तस्या॒न्वाग॑तस्य। स॒व्याद्धनु॒रजा॑यत। दक्षि॑णा॒दिष॑वः। तस्मा॑दिषुध॒न्वं पुण्य॑जन्म। य॒ज्ञज॑न्मा॒ हि॥२॥
८.१.३
तमेक॒ सन्तम्। ब॒हवो॒ नाभ्य॑धृष्णुवन्। तस्मा॒देक॑मिषुध॒न्विनम्। ब॒हवो॑ऽनिषुध॒न्वा नाभिधृ॑ष्णुवन्ति। सोऽस्मयत। एकं॑ मा॒ सन्तं॑ ब॒हवो॒ नाभ्य॑धर्\mbox{}षिषु॒रिति॑। तस्य॑ सिष्मिया॒णस्य॒ तेजोऽपाक्रामत्। तद्दे॒वा ओष॑धीषु॒ न्य॑मृजुः। ते श्या॒माका॑ अभवन्। स्म॒याका॒ वै नामै॒ते॥३॥
८.१.४
तत्स्म॒याका॑ना स्मयाक॒त्वम्। तस्माद्दीक्षि॒तेना॑पि॒गृह्य॑ स्मेत॒व्यम्। तेज॑सो॒ धृत्यै। स धनु॑ प्रति॒ष्कभ्या॑तिष्ठत्। ता उ॑प॒दीका॑ अब्रुव॒न्वरं॑ वृणामहै। अथ॑ व इ॒म र॑न्धयाम। यत्र॒ क्व॑ च॒ खना॑म। तद॒पो॑ऽभितृ॑णदा॒मेति॑। तस्मा॑दुप॒दीका॒ यत्र॒ क्व॑ च॒ खन॑न्ति। तद॒पो॑ऽभितृ॑न्दन्ति॥४॥
८.१.५
वारे॑वृत॒ ह्या॑साम्। तस्य॒ ज्यामप्या॑दन्। तस्य॒ धनु॑र्वि॒प्रव॑माण॒ शिर॒ उद॑वर्तयत्। तद्द्यावा॑पृथि॒वी अनु॒प्राव॑र्तत। यत् प्राव॑र्तत। तत्प्र॑व॒र्ग्य॑स्य प्रवर्ग्य॒त्वम्। यद्घ्राँ(४)इत्यप॑तत्। तद्ध॒र्मस्य॑ घर्म॒त्वम्। म॒ह॒तो वी॒र्य॑मपप्त॒दिति॑। तन्म॑हावी॒रस्य॑ महावीर॒त्वम्॥५॥
८.१.६
यद॒स्याः स॒मभ॑रन्। तत्स॒म्राज्ञ॑ सम्रा॒ट्त्वम्। त स्तृ॒तन्दे॒वतास्त्रे॒धा व्य॑गृह्णत। अ॒ग्निः प्रा॑तः सव॒नम्। इन्द्रो॒ माध्यं॑ दिन॒ सव॑नम्। विश्वे॑दे॒वास्तृ॑तीयसव॒नम्। तेनाप॑शीर्ष्णा य॒ज्ञेन॒ यज॑मानाः। नाशिषो॒ऽवारु॑न्धत। न सु॑व॒र्गं लो॒कम॒भ्य॑जयन्। ते दे॒वा अ॒श्विना॑वब्रुवन्॥६॥
८.१.७
भि॒षजौ॒ वै स्थ॑। इ॒दं य॒ज्ञस्य॒ शिरः॒ प्रति॑धत्त॒मिति॑। ताव॑ब्रूतां॒ वरं॑ वृणावहै। ग्रह॑ ए॒व ना॒वत्रापि॑ गृह्यता॒मिति॑। ताभ्या॑मे॒तमाश्वि॒नम॑गृह्णन्। तावे॒तद्य॒ज्ञस्य॒ शिरः॒ प्रत्य॑धत्ताम्। यत्प्र॑व॒र्ग्य॑। तेन॒ सशीर्ष्णा य॒ज्ञेन॒ यज॑मानाः। अवा॒शिषोऽरु॑न्धत। अ॒भि सु॑व॒र्गं लो॒कम॑जयन्। यत्प्र॑व॒र्ग्यं॑ प्रवृ॒णक्ति॑। य॒ज्ञस्यै॒व तच्छिरः॒ प्रति॑दधाति। तेन॒ सशीर्ष्णा य॒ज्ञेन॒ यज॑मानः। अवा॒शिषो॑ रु॒न्धे। अ॒भि सु॑व॒र्गं लो॒कं ज॑यति। तस्मा॑दे॒ष आश्वि॒नप्र॑वया इव। यत्प्र॑व॒र्ग्य॑॥७॥
\anuvakamend


८.२.०
या॒ज्या॑यै॒ न जु॑हु॒यादवि॑श॒द्वेणुः॒ शान्त्यै॑ प॒ङ्क्तिरा॑धस॒मित्या॑ह हरति दिहन्ति प॒राक्र॑म॒तावि॑शत् प्र॒जाय॑मानाना सृजति श॒न्त्वाया॒ष्टौ च॑॥ २॥
८.२.१
सा॒वि॒त्रं जु॑होति॒ प्रति॑ष्ठित्यै। च॒तु॒र्गृ॒ही॒तेन॑ जुहोति। चतु॑ष्पादः प॒शव॑। प॒शूने॒वाव॑रुन्धे। चत॑स्रो॒ दिश॑। दि॒क्ष्वे॑व प्रति॑तिष्ठति। छन्दासि दे॒वेभ्योऽपाक्रामन्। न वो॑ऽभा॒गानि॑ ह॒व्यं व॑क्ष्याम॒ इति॑। तेभ्य॑ ए॒तच्च॑तुर्गृही॒तम॑धारयन्। पु॒रो॒नु॒वा॒क्या॑यै या॒ज्या॑यै॥८॥
८.२.२
दे॒वता॑यै वषट्का॒राय॑। यच्च॑तुर्गृही॒तं जु॒होति॑। छन्दास्ये॒व तत् प्री॑णाति। तान्य॑स्य प्री॒तानि॑ दे॒वेभ्यो॑ ह॒व्यं व॑हन्ति। ब्र॒ह्म॒वा॒दिनो॑ वदन्ति। हो॒त॒व्यं॑ दीक्षि॒तस्य॑ गृ॒हा(३)इ न हो॑त॒व्या(३)मिति॑। ह॒विर्\mbox{}वै दीक्षि॒तः। यज्जु॑हु॒यात्। ह॒विष्कृ॑तं॒ यज॑मानम॒ग्नौ प्रद॑ध्यात्। यन्न जु॑हु॒यात्॥९॥
८.२.३
य॒ज्ञ॒प॒रुर॒न्तरि॑यात्। यजु॑रे॒व व॑देत्। न ह॒विष्कृ॑तं॒ यज॑मानम॒ग्नौ प्र॒दधा॑ति। न य॑ज्ञप॒रुर॒न्तरे॑ति। गा॒य॒त्री छन्दा॒स्यत्य॑मन्यत। तस्यै॑ वषट्का॒रोऽभ्यय्य॒ शिरोऽच्छिनत्। तस्यै द्वे॒धा रसः॒ परा॑पतत्। पृ॒थि॒वीम॒र्द्धः प्रावि॑शत्। प॒शून॒र्द्धः। यः पृ॑थि॒वीं प्रावि॑शत्॥१०॥
८.२.४
स ख॑दि॒रो॑ऽभवत्। यः प॒शून्। सो॑ऽजाम्। यत्खा॑दि॒र्यभ्रि॒र्भव॑ति। छन्द॑सामे॒व रसे॑न य॒ज्ञस्य॒ शिरः॒ सम्भ॑रति। यदौदुं॑बरी। ऊर्ग्वा उ॑दु॒म्बर॑। ऊ॒र्जैव य॒ज्ञस्य॒ शिरः॒ सम्भ॑रति। यद्वै॑ण॒वी। तेजो॒ वै वेणु॑॥११॥
८.२.५
तेज॑सै॒व य॒ज्ञस्य॒ शिरः॒ सम्भ॑रति। यद्वैक॑ङ्कती। भा ए॒वाव॑रुन्धे। दे॒वस्य॑ त्वा सवि॒तुः प्र॑स॒व इत्यभ्रि॒माद॑त्ते॒ प्रसूत्यै। अ॒श्विनोर्बा॒हुभ्या॒मित्या॑ह। अ॒श्विनौ॒ हि दे॒वाना॑मध्व॒र्यू आस्ताम्। पू॒ष्णो हस्ताभ्या॒मित्या॑ह॒ यत्यै। वज्र॑ इव॒ वा ए॒षा। यदभ्रि॑। अभ्रि॑रसि॒ नारि॑र॒सीत्या॑ह॒ शान्त्यै॥१२॥
८.२.६
अ॒ध्व॒र॒कृद्दे॒वेभ्य॒ इत्या॑ह। य॒ज्ञो वा अ॑ध्व॒रः। य॒ज्ञ॒कृद्दे॒वेभ्य॒ इति॒ वावैतदा॑ह। उत्ति॑ष्ठ ब्रह्मणस्पत॒ इत्या॑ह। ब्रह्म॑णै॒व य॒ज्ञस्य॒ शिरोऽच्छै॑ति। प्रैतु॒ ब्रह्म॑ण॒स्पति॒रित्या॑ह। प्रेत्यै॒व य॒ज्ञस्य॒ शिरोऽच्छै॑ति। प्र दे॒व्ये॑तु सू॒नृतेत्या॑ह। य॒ज्ञो वै सू॒नृता। अच्छा॑ वी॒रं नर्यं॑ प॒ङ्क्तिरा॑धस॒मित्या॑ह॥१३॥
८.२.७
पाङ्क्तो॒ हि य॒ज्ञः। दे॒वा य॒ज्ञं न॑यन्तु न॒ इत्या॑ह। दे॒वाने॒व य॑ज्ञ॒निय॑ कुरुते। देवी द्यावापृथिवी॒ अनु॑ मे मसाथा॒मित्या॑ह। आ॒भ्यामे॒वानु॑मतो य॒ज्ञस्य॒ शिरः॒ सम्भ॑रति। ऋ॒द्ध्यास॑म॒द्य म॒खस्य॒ शिर॒ इत्या॑ह। य॒ज्ञो वै म॒खः। ऋ॒द्ध्यास॑म॒द्य य॒ज्ञस्य॒ शिर॒ इति वावैतदा॑ह। म॒खाय॑ त्वा म॒खस्य॑ त्वा शी॒र्ष्ण इत्या॑ह। नि॒र्दिश्यै॒वैन॑द्धरति॥१४॥
८.२.८
त्रिर्\mbox{}ह॑रति। त्रय॑ इ॒मे लो॒काः। ए॒भ्य ए॒व लो॒केभ्यो॑ य॒ज्ञस्य॒ शिरः॒ सम्भ॑रति। तू॒ष्णीं च॑तु॒र्थ ह॑रति। अप॑रिमितादे॒व य॒ज्ञस्य॒ शिरः॒ सम्भ॑रति। मृ॒त्ख॒नादग्रे॑ हरति। तस्मान्मृत्ख॒नः क॑रु॒ण्य॑तरः। इय॒त्यग्र॑ आसी॒रित्या॑ह। अ॒स्यामे॒वाछं॑बट्कारं य॒ज्ञस्य॒ शिरः॒ सम्भ॑रति। ऊर्जं॒ वा ए॒त रसं॑ पृथि॒व्या उ॑प॒दीका॒ उद्दि॑हन्ति॥१५॥
८.२.९
यद्व॒ल्मीकम्। यद्व॑ल्मीकव॒पा सं॑भा॒रो भव॑ति। ऊर्ज॑मे॒व रसं॑ पृथि॒व्या अव॑रुन्धे। अथो॒ श्रोत्र॑मे॒व। श्रोत्र॒ ह्ये॑तत्पृ॑थि॒व्याः। यद्व॒ल्मीक॑। अब॑धिरो भवति। य ए॒वं वेद॑। इन्द्रो॑ वृ॒त्राय॒ वज्र॒मुद॑यच्छत्। स यत्र॑यत्र प॒राक्र॑मत॥१६॥
८.२.१०
तन्नाद्ध्रि॑यत। स पू॑तीकस्त॒म्बे पराक्रमत। सोऽद्ध्रियत। सोऽब्रवीत्। ऊ॒तिं वै मे॑ धा॒ इति॑। तदू॒तीका॑नामूतीक॒त्वम्। यदू॒तीका॒ भव॑न्ति। य॒ज्ञायै॒वोतिन्द॑धति। अ॒ग्नि॒जा अ॑सि प्र॒जाप॑ते॒ रेत॒ इत्या॑ह। य ए॒व रस॑ प॒शून्प्रावि॑शत्॥१७॥
८.२.११
तमे॒वाव॑रुन्धे। पञ्चै॒ते सं॑भा॒रा भ॑वन्ति। पाङ्क्तो॑ य॒ज्ञः। यावा॑ने॒व य॒ज्ञः। तस्य॒ शिरः॒ सम्भ॑रति। यद्ग्रा॒म्याणां पशू॒नां चर्म॑णा स॒म्भरेत्। ग्रा॒म्यान्प॒शूञ्छु॒चाऽर्प॑येत्। कृ॒ष्णा॒जि॒नेन॒ सम्भ॑रति। आ॒र॒ण्याने॒व प॒शूञ्छु॒चार्प॑यति। तस्मात्स॒माव॑त्पशू॒नां प्र॒जाय॑मानानाम्॥१८॥
८.२.१२
आ॒र॒ण्याः प॒शवः॒ कनी॑यासः। शु॒चा ह्यृ॑ताः। लो॒म॒तः सम्भ॑रति। अतो॒ ह्य॑स्य॒ मेध्यम्। प॒रि॒गृह्या य॑न्ति। रक्ष॑सा॒मप॑हत्यै। ब॒हवो॑ हरन्ति। अप॑चितिमे॒वास्मि॑न्दधति। उद्ध॑ते॒ सिक॑तोपोप्ते॒ परि॑श्रिते॒ निद॑धति॒ शान्त्यै। मद॑न्तीभि॒रुप॑ सृजति॥१९॥
८.२.१३
तेज॑ ए॒वास्मि॑न्दधाति। मधु॑ त्वा मधु॒ला क॑रो॒त्वित्या॑ह। ब्रह्म॑णै॒वास्मि॒न्तेजो॑ दधाति। यद्ग्रा॒म्याणां॒ पात्रा॑णां क॒पालै ससृ॒जेत्। ग्रा॒म्याणि॒ पात्रा॑णि शु॒चाऽर्प॑येत्। अ॒र्म॒क॒पा॒लैः ससृ॑जति। ए॒तानि॒ वा अ॑नुपजीवनी॒यानि॑। तान्ये॒व शु॒चार्प॑यति। शर्क॑राभिः॒ ससृ॑जति॒ धृत्यै। अथो॑ श॒न्त्वाय॑। अ॒ज॒लो॒मैः ससृ॑जति। ए॒षा वा अ॒ग्नेः प्रि॒या त॒नूः। यद॒जा। प्रि॒ययै॒वैनं॑ त॒नुवा॒ ससृ॑जति। अथो॒ तेज॑सा। कृ॒ष्णा॒जि॒नस्य॒ लोम॑भिः॒ ससृ॑जति। य॒ज्ञो वै कृ॑ष्णाजि॒नम्। य॒ज्ञेनै॒व यज्ञ ससृ॑जति॥२०॥
\anuvakamend

८.३.०
स्या॒द्यत् प्र॑व॒र्ग्य॑श्छन्दो॑भिः करोति वी॒र्य॑सम्मितं॒ छन्दासि नि॒ष्पत्पृ॒णेत्या॑ह सुक्षि॒तिरनाच्छृण्ण॒ञ्छन्दा॒स्याच्छृ॑णत्त्य॒ष्टौ च॑॥ ३॥
८.३.१
परि॑श्रिते करोति। ब्र॒ह्म॒व॒र्च॒सस्य॒ परि॑गृहीत्यै। न कु॒र्वन्न॒भि प्राण्यात्। यत्कु॒र्वन्न॑भि प्रा॒ण्यात्। प्रा॒णाञ्छु॒चार्प॑येत्। अ॒प॒हाय॒ प्राणि॑ति। प्रा॒णानाङ्गोपी॒थाय॑। न प्र॑व॒र्ग्यं॑ चादि॒त्यं चा॒न्तरे॑यात्। यद॑न्तरे॒यात्। दु॒श्चर्मा स्यात्॥२१॥
८.३.२
तस्मा॒न्नान्त॒राय्यम्। आ॒त्मनो॑ गोपी॒थाय॑। वेणु॑ना करोति। तेजो॒ वै वेणु॑। तेज॑ प्रव॒र्ग्य॑। तेज॑सै॒व तेजः॒ सम॑र्द्धयति। म॒खस्य॒ शिरो॒ऽसीत्या॑ह। य॒ज्ञो वै म॒खः। तस्यै॒तच्छिर॑। यत्प्र॑व॒र्ग्य॑॥२२॥
८.३.३
तस्मा॑दे॒वमा॑ह। य॒ज्ञस्य॑ प॒दे स्थ॒ इत्या॑ह। य॒ज्ञस्य॒ ह्ये॑ते प॒दे। अथो॒ प्रति॑ष्ठित्यै। गा॒य॒त्रेण॑ त्वा॒ छन्द॑सा करो॒मीत्या॑ह। छन्दो॑भिरे॒वैन॑ङ्करोति। त्र्यु॑द्धिं करोति। त्रय॑ इ॒मे लो॒काः। ए॒षाल्लोँ॒काना॒माप्त्यै। छन्दो॑भिः करोति॥२३॥
८.३.४
वी॒र्यं॑ वै छन्दासि। वी॒र्ये॑णै॒वैन॑ङ्करोति। यजु॑षा॒ बिल॑ङ्करोति॒ व्यावृ॑त्यै। इयं॑ तं करोति। प्र॒जाप॑तिना यज्ञमु॒खेन॒ सम्मि॑तम्। इयं॑ तं करोति। य॒ज्ञ॒प॒रुषा॒ संमि॑तम्। इयं॑ तं करोति। ए॒ताव॒द्वै पुरु॑षे वी॒र्यम्। वी॒र्य॑सम्मितम्॥२४॥
८.३.५
अप॑रिमितं करोति। अप॑रिमित॒स्याव॑रुद्ध्यै। प॒रि॒ग्री॒वं क॑रोति॒ धृत्यै। सूर्य॑स्य॒ हर॑सा श्रा॒येत्या॑ह। य॒था॒य॒जुरे॒वैतत्। अ॒श्व॒श॒केन॑ धूपयति। प्रा॒जा॒प॒त्यो वा अश्व॑ सयोनि॒त्वाय॑। वृष्णो॒ अश्व॑स्य नि॒ष्पद॒सीत्या॑ह। अ॒सौ वा आ॑दि॒त्यो वृषाऽश्व॑। तस्य॒ छन्दासि नि॒ष्पत्॥२५॥
८.३.६
छन्दो॑भिरे॒वैन॑न्धूपयति। अ॒र्चिषे त्वा शो॒चिषे॒ त्वेत्या॑ह। तेज॑ ए॒वास्मि॑न्दधाति। वा॒रु॒णो॑ऽभीद्ध॑। मै॒त्रियोपै॑ति॒ शान्त्यै। सिद्ध्यै॒ त्वेत्या॑ह। य॒था॒य॒जुरे॒वैतत्। दे॒वस्त्वा॑ सवि॒तोद्व॑प॒त्वित्या॑ह। स॒वि॒तृप्र॑सूत ए॒वैनं॒ ब्रह्म॑णा दे॒वता॑भि॒रुद्व॑पति। अप॑द्यमानः पृथि॒व्यामाशा॒ दिश॒ आपृ॒णेत्या॑ह॥२६॥
८.३.७
तस्मा॑द॒ग्निः सर्वा॒ दिशोऽनु॒ विभा॑ति। उत्ति॑ष्ठ बृ॒हन्भ॑वो॒र्ध्वस्ति॑ष्ठ ध्रु॒वस्त्वमित्या॑ह॒ प्रति॑ष्ठित्यै। ई॒श्व॒रो वा ए॒षोऽन्धो भवि॑तोः। यः प्र॑व॒र्ग्य॑म॒न्वीक्ष॑ते। सूर्य॑स्य त्वा॒ चक्षु॒षाऽन्वीक्ष॒ इत्या॑ह। चक्षु॑षो गोपी॒थाय॑। ऋ॒जवे त्वा सा॒धवे त्वा सुक्षि॒त्यै त्वा॒ भूत्यै॒ त्वेत्या॑ह। इ॒यं वा ऋ॒जुः। अ॒न्तरि॑क्ष सा॒धु। अ॒सौ सु॑क्षि॒तिः॥२७॥
८.३.८
दिशो॒ भूति॑। इ॒माने॒वास्मै॑ लो॒कान्क॑ल्पयति। अथो॒ प्रति॑ष्ठित्यै। इ॒दम॒हम॒मुमा॑मुष्याय॒णं  वि॒शा प॒शुभि॑र्ब्रह्मवर्च॒सेन॒ पर्यू॑हा॒मीत्या॑ह। वि॒शैवैनं॑ प॒शुभि॑र्ब्रह्मवर्च॒सेन॒ पर्यू॑हति। वि॒शेति॑ राज॒न्य॑स्य ब्रूयात्। वि॒शैवैनं॒ पर्यू॑हति। प॒शुभि॒रिति॒ वैश्य॑स्य। प॒शुभि॑रे॒वैनं॒ पर्यू॑हति। अ॒सु॒र्यं॑ पात्र॒मनाच्छृण्णम्॥२८॥
८.३.९
आच्छृ॑णत्ति। दे॒व॒त्राक॑। अ॒ज॒क्षी॒रेणाच्छृ॑णत्ति। प॒र॒मं वा ए॒तत्पय॑। यद॑जक्षी॒रम्। प॒र॒मेणै॒वैनं॒ पय॒साच्छृ॑णत्ति। यजु॑षा॒ व्यावृ॑त्त्यै। छन्दो॑भि॒राच्छृ॑णत्ति। छन्दो॑भि॒र्वा ए॒ष क्रि॑यते। छन्दो॑भिरे॒व छन्दा॒स्याच्छृ॑णत्ति। छृ॒न्धि वाच॒मित्या॑ह। वाच॑मे॒वाव॑रुन्धे। छृ॒न्ध्यूर्ज॒मित्या॑ह। ऊर्ज॑मे॒वाव॑रुन्धे। छृ॒न्धि ह॒विरित्या॑ह। ह॒विरे॒वाक॑। देव॑ पुरश्चर स॒घ्यास॒न्त्वेत्या॑ह। य॒था॒य॒जुरे॒वैतत्॥२९॥
\anuvakamend

८.४.०
द॒धा॒ती॒वान्वा॑ह य॒ज्ञस्या॑है॒ष उ॒परि॑ष्टादाशीर॒न्यो व्यास्था॒पय॑न्ति र॒श्मयो॑ भवन्ति॒ धन्वेत्या॑ह य॒ज्ञश्च॑क्राम॒ सम॑ष्ट्यै॒ द्वे च॑॥ ४॥
८.४.१
ब्रह्म॒न्प्रच॑रिष्यामो॒ होत॑र्घ॒र्मम॒भिष्टु॒हीत्या॑ह। ए॒ष वा ए॒तर्\mbox{}हि॒ बृह॒स्पति॑। यद्ब्र॒ह्मा। तस्मा॑ ए॒व प्र॑ति॒प्रोच्य॒ प्रच॑रति। आ॒त्मनोऽनार्त्यै। य॒माय॑ त्वा म॒खाय॒ त्वेत्या॑ह। ए॒ता वा ए॒तस्य॑ दे॒वता। ताभि॑रे॒वैन॒ सम॑र्द्धयति। मद॑न्तीभिः॒ प्रोक्ष॑ति। तेज॑ ए॒वास्मि॑न्दधाति॥३०॥
८.४.२
अ॒भि॒पू॒र्वं प्रोक्ष॑ति। अ॒भि॒पू॒र्वमे॒वास्मि॒न्तेजो॑ दधाति। त्रिः प्रोक्ष॑ति। त्र्या॑वृ॒द्धि य॒ज्ञः। अथो॑ मेध्य॒त्वाय॑। होताऽन्वा॑ह। रक्ष॑सा॒मप॑हत्यै। अन॑वानम्। प्रा॒णाना॒ सन्त॑त्यै। त्रि॒ष्टुभ॑ स॒तीर्गा॑य॒त्रीरि॒वान्वा॑ह॥३१॥
८.४.३
गा॒य॒त्रो हि प्रा॒णः। प्रा॒णमे॒व यज॑माने दधाति। सन्त॑त॒मन्वा॑ह। प्रा॒णाना॑म॒न्नाद्य॑स्य॒ सन्त॑त्यै। अथो॒ रक्ष॑सा॒मप॑हत्यै। यत्परि॑मिता अनुब्रू॒यात्। परि॑मित॒मव॑रुन्धीत। अप॑रिमिता॒ अन्वा॑ह। अप॑रिमित॒स्याव॑रुद्ध्यै। शिरो॒ वा ए॒तद्य॒ज्ञस्य॑॥३२॥
८.४.४
यत्प्र॑व॒र्ग्य॑। ऊर्ङ्मुञ्जा। यन्मौ॒ञ्जो वे॒दो भव॑ति। ऊ॒र्जैव य॒ज्ञस्य॒ शिरः॒ सम॑र्द्धयति। प्रा॒णा॒हु॒तीर्जु॑होति। प्रा॒णाने॒व यज॑माने दधाति। स॒प्त जु॑होति। स॒प्त वै शी॑र्\mbox{}ष॒ण्या प्रा॒णाः। प्रा॒णाने॒वास्मि॑न्दधाति। दे॒वस्त्वा॑ सवि॒ता मध्वा॑ऽन॒क्त्वित्या॑ह॥३३॥
८.४.५
तेज॑सै॒वैन॑मनक्ति। पृ॒थि॒वीं तप॑सस्त्राय॒स्वेति॒ हिर॑ण्य॒मुपास्यति। अ॒स्या अन॑तिदाहाय। शिरो॒ वा ए॒तद्य॒ज्ञस्य॑। यत्प्र॑व॒र्ग्य॑। अ॒ग्निः सर्वा॑ दे॒वता। प्र॒ल॒वाना॒दीप्योपास्यति। दे॒वतास्वे॒व य॒ज्ञस्य॒ शिरः॒ प्रति॑दधाति। अप्र॑तिशीर्णाग्रं भवति। ए॒तद्ब॑र्\mbox{}हि॒र्\mbox{}ह्ये॑षः॥३४॥
८.४.६
अ॒र्चिर॑सि शो॒चिर॒सीत्या॑ह। तेज॑ ए॒वास्मि॑न्ब्रह्मवर्च॒सन्द॑धाति। ससी॑दस्व म॒हा अ॒सीत्या॑ह। म॒हान् ह्ये॑षः। ब्र॒ह्म॒वा॒दिनो॑ वदन्ति। ए॒ते वाव त ऋ॒त्विज॑। ये द॑र्\mbox{}शपूर्णमा॒सयो। अथ॑ क॒था होता॒ यज॑मानाया॒शिषो॒ नाशास्त॒ इति॑। पु॒रस्ता॑दाशीः॒ खलु॒ वा अ॒न्यो य॒ज्ञः। उ॒परि॑ष्टादाशीर॒न्यः॥३५॥
८.४.७
अ॒ना॒धृ॒ष्या पु॒रस्ता॒दिति॒ यदे॒तानि॒ यजू॒ष्याह॑। शी॒र्\mbox{}ष॒त ए॒व य॒ज्ञस्य॒ यज॑मान आ॒शिषोऽव॑रुन्धे। आयु॑ पु॒रस्ता॑दाह। प्र॒जान्द॑क्षिण॒तः। प्रा॒णं प॒श्चात्। श्रोत्र॑मुत्तर॒तः। विधृ॑तिमु॒परि॑ष्टात्। प्रा॒णाने॒वास्मै॑ स॒मीचो॑ दधाति। ई॒श्व॒रो वा ए॒ष दिशोऽनून्म॑दितोः। यन्दिशोऽनु॑ व्यास्था॒पय॑न्ति॥३६॥
८.४.८
मनो॒रश्वा॑सि॒ भूरि॑पु॒त्रेती॒माम॒भिमृ॑शति। इ॒यं वै मनो॒रश्वा॒ भूरि॑पुत्रा। अ॒स्यामे॒व प्रति॑तिष्ठ॒त्यनु॑न्मादाय। सू॒प॒सदा॑ मे भूया॒ मा मा॑ हिसी॒रित्या॒हाहिसायै। चित॑ स्थ परि॒चित॒ इत्या॑ह। अप॑चितिमे॒वास्मि॑न्दधाति। शिरो॒ वा ए॒तद्य॒ज्ञस्य॑। यत्प्र॑व॒र्ग्य॑। अ॒सौ खलु॒ वा आ॑दि॒त्यः प्र॑व॒र्ग्य॑। तस्य॑ म॒रुतो॑ र॒श्मय॑॥३७॥
८.४.९
स्वाहा॑ म॒रुद्भिः॒ परि॑श्रय॒स्वेत्या॑ह। अ॒मुमे॒वादि॒त्य र॒श्मिभिः॒ पर्यू॑हति। तस्मा॑द॒सावा॑दि॒त्यो॑ऽमुष्मि॑ल्लोँ॒के र॒श्मिभिः॒ पर्यू॑ढः। तस्मा॒द्राजा॑ वि॒शा पर्यू॑ढः। तस्माद्ग्राम॒णीः स॑जा॒तैः पर्यू॑ढः। अ॒ग्नेः सृ॒ष्टस्य॑ य॒तः। विक॑ङ्कतं॒ भा आर्च्छत्। यद्वैक॑ङ्कताः परि॒धयो॒ भव॑न्ति। भा ए॒वाव॑रुन्धे। द्वाद॑श भवन्ति॥३८॥
८.४.१०
द्वाद॑श॒ मासा संवत्स॒रः। सं॒व॒त्स॒रमे॒वाव॑रुन्धे। अस्ति॑ त्रयोद॒शो मास॒ इत्या॑हुः। यत्त्र॑योद॒शः प॑रि॒धिर्भव॑ति। तेनै॒व त्र॑योद॒शं मास॒मव॑रुन्धे। अ॒न्तरि॑क्षस्यान्त॒र्द्धिर॒सीत्या॑ह॒ व्यावृ॑त्त्यै। दिवं॒ तप॑सस्त्राय॒स्वेत्यु॒परि॑ष्टा॒द्धिर॑ण्य॒मधि॒ निद॑धाति। अ॒मुष्या॒ अन॑तिदाहाय। अथो॑ आ॒भ्यामे॒वैन॑मुभ॒यतः॒ परि॑गृह्णाति। अर्\mbox{}ह॑न् बिभर्\mbox{}षि॒ साय॑कानि॒ धन्वेत्या॑ह॥३९॥
८.४.११
स्तौत्ये॒वैन॑मे॒तत्। गा॒य॒त्रम॑सि॒ त्रैष्टु॑भमसि॒ जाग॑तम॒सीति॑ ध॒वित्रा॒ण्याद॑त्ते। छन्दो॑भिरे॒वैना॒न्याद॑त्ते। मधु॒ मध्विति॑ धूनोति। प्रा॒णो वै मधु॑। प्रा॒णमे॒व यज॑माने दधाति। त्रिः परि॑यन्ति। त्रि॒वृद्धि प्रा॒णः। त्रिः परि॑यन्ति। त्र्या॑वृ॒द्धि य॒ज्ञः॥४०॥
८.४.१२
अथो॒ रक्ष॑सा॒मप॑हत्यै। त्रिः पुनः॒ परि॑यन्ति। षट्थ्सम्प॑द्यन्ते। षड्वा ऋ॒तव॑। ऋ॒तुष्वे॒व प्रति॑तिष्ठन्ति। यो वै घ॒र्मस्य॑ प्रि॒यां त॒नुव॑मा॒क्राम॑ति। दु॒श्चर्मा॒ वै स भ॑वति। ए॒ष ह॒ वा अ॑स्य प्रि॒यां त॒नुव॒माक्रा॑मति। यत् त्रिः प॒रीत्य॑ चतु॒र्थं पर्ये॑ति। ए॒ता ह॒ वा अ॑स्यो॒ग्रदे॑वो॒ राज॑नि॒राच॑क्राम॥४१॥
८.४.१३
ततो॒ वै स दु॒श्चर्मा॑ऽभवत्। तस्मा॒त्त्रिः प॒रीत्य॒ न च॑तु॒र्थं परी॑यात्। आ॒त्मनो॑ गोपी॒थाय॑। प्रा॒णा वै ध॒वित्रा॑णि। अव्य॑तिषङ्गन्धून्वन्ति। प्रा॒णाना॒मव्य॑तिषङ्गाय॒ क्लृप्त्यै। वि॒नि॒षद्य॑ धून्वन्ति। दि॒क्ष्वे॑व प्रति॑तिष्ठन्ति। ऊ॒र्ध्वं धून्वन्ति। सु॒व॒र्गस्य॑ लो॒कस्य॒ सम॑ष्ट्यै। स॒र्वतो॑ धून्वन्ति। तस्मा॑द॒य स॒र्वत॑ पवते॥४२॥
\anuvakamend

८.५.०
प॒श्चाद्रो॑चयति॒ जाग॑तेन॒ छन्द॑सा॒ पाङ्क्ते॑न॒ छन्द॑सा॒ समा॑रुचि॒तो रो॑च॒येत्या॑हा॒शिष॑मे॒वैतामाशास्ते शास्ते॒ऽष्टौ च॑॥ ५॥
८.५.१
अ॒ग्निष्ट्वा॒ वसु॑भिः पु॒रस्ताद्रोचयतु गाय॒त्रेण॒ छन्द॒सेत्या॑ह। अ॒ग्निरे॒वैनं॒ वसु॑भिः पु॒रस्ताद्रोचयति गाय॒त्रेण॒ छन्द॑सा। समा॑रुचि॒तो रो॑च॒येत्या॑ह। आ॒शिष॑मे॒वैतामाशास्ते। इन्द्र॑स्त्वा रु॒द्रैर्द॑क्षिण॒तो रो॑चयतु॒ त्रैष्टु॑भेन॒ छन्द॒सेत्या॑ह। इन्द्र॑ ए॒वैन रु॒द्रैर्द॑क्षिण॒तो रो॑चयति॒ त्रैष्टु॑भेन॒ छन्द॑सा। समा॑रुचि॒तो रो॑च॒येत्या॑ह। आ॒शिष॑मे॒वैतामा शास्ते। वरु॑णस्त्वाऽऽदि॒त्यैः प॒श्चाद्रो॑चयतु॒ जाग॑तेन॒ छन्द॒सेत्या॑ह। वरु॑ण ए॒वैन॑मादि॒त्यैः प॒श्चाद्रो॑चयति॒ जाग॑तेन॒ छन्द॑सा॥४३॥
८.५.२
समा॑रुचि॒तो रो॑च॒येत्या॑ह। आ॒शिष॑मे॒वैतामाशास्ते। द्यु॒ता॒नस्त्वा॑ मारु॒तो म॒रुद्भि॑रुत्तर॒तो रो॑चय॒त्वानु॑ष्टुभेन॒ छन्द॒सेत्या॑ह। द्यु॒ता॒न ए॒वैनं॑ मारु॒तो म॒रुद्भि॑रुत्तर॒तो रो॑चय॒त्यानु॑ष्टुभेन॒ छन्द॑सा। समा॑रुचि॒तो रो॑च॒येत्या॑ह। आ॒शिष॑मे॒वैतामा शास्ते। बृह॒स्पति॑स्त्वा॒ विश्वैर्दे॒वैरु॒परि॑ष्टाद्रोचयतु॒ पाङ्क्ते॑न॒ छन्द॒सेत्या॑ह। बृह॒स्पति॑रे॒वैनं॒  विश्वैर्दे॒वैरु॒परि॑ष्टाद्रोचयति॒ पाङ्क्ते॑न॒ छन्द॑सा। समा॑रुचि॒तो रो॑च॒येत्या॑ह। आ॒शिष॑मे॒वैतामाशास्ते॥४४॥
८.५.३
रो॒चि॒तस्त्वन्दे॑व घर्म दे॒वेष्वसीत्या॑ह। रो॒चि॒तो ह्ये॑ष दे॒वेषु॑। रो॒चि॒षी॒याहं म॑नु॒ष्येष्वित्या॑ह। रोच॑त ए॒वैष म॑नु॒ष्ये॑षु। सम्राड्घर्म रुचि॒तस्त्वन्दे॒वेष्वायु॑ष्मा स्तेज॒स्वी ब्र॑ह्मवर्च॒स्य॑सीत्या॑ह। रु॒चि॒तो ह्ये॑ष दे॒वेष्वायु॑ष्मास्तेज॒स्वी ब्र॑ह्मवर्च॒सी। रु॒चि॒तो॑ऽहं म॑नु॒ष्येष्वायु॑ष्मास्तेज॒स्वी ब्र॑ह्मवर्च॒सी भू॑यास॒मित्या॑ह। रु॒चि॒त ए॒वैष म॑नु॒ष्येष्वायु॑ष्मास्तेज॒स्वी ब्र॑ह्मवर्च॒सी भ॑वति। रुग॑सि॒ रुचं॒ मयि॑ धेहि॒ मयि॒ रुगित्या॑ह। आ॒शिष॑मे॒वैतामाशास्ते। तं यदे॒तैर्यजु॑र्भि॒ररो॑चयि॒त्वा। रु॒चि॒तो घ॒र्म इति॑ प्रब्रू॒यात्। अरो॑चुकोऽध्व॒र्युः स्यात्। अरो॑चुको॒ यज॑मानः। अथ॒ यदे॑नमे॒तैर्यजु॑र्भी रोचयि॒त्वा। रु॒चि॒तो घर्म॒ इति॒ प्राह॑। रोचु॑कोऽध्व॒र्युर्भव॑ति। रोचु॑को॒ यज॑मानः॥४५॥
\anuvakamend

८.६.०
ऋ॒तवो॒ हि शिरः॒ सर्व॑पृष्ठे॒ प्रवृ॑ण॒क्त्यनि॑पद्यमान॒मित्या॑ह ग॒तेत्या॑ह शार॒दावे॒वास्मा॑ ऋ॒तू क॑ल्पयति रुन्धे कवी॒नामित्या॑ह प्रा॒णाः प्रति॑दधाति भवन्ति वाचयति च॒त्वारि॑ च॥ ६॥
८.६.१
शिरो॒ वा ए॒तद्य॒ज्ञस्य॑। यत् प्र॑व॒र्ग्य॑। ग्री॒वा उ॑प॒सद॑। पु॒रस्ता॑दुप॒सदां प्रव॒र्ग्यं॑ प्रवृ॑णक्ति। ग्री॒वास्वे॒व य॒ज्ञस्य॒ शिरः॒ प्रति॑दधाति। त्रिः प्रवृ॑णक्ति। त्रय॑ इ॒मे लो॒काः। ए॒भ्य ए॒व लो॒केभ्यो॑ य॒ज्ञस्य॒ शिरोऽव॑रुन्धे। षट्थ्सम्प॑द्यन्ते। षड्वा ऋ॒तव॑॥४६॥
८.६.२
ऋ॒तुभ्य॑ ए॒व य॒ज्ञस्य॒ शिरोऽव॑रुन्धे। द्वाद॑श॒कृत्वः॒ प्रवृ॑णक्ति। द्वाद॑श॒ मासा संवत्स॒रः। सं॒व॒त्स॒रादे॒व य॒ज्ञस्य॒ शिरोऽव॑रुन्धे। चतु॑र्विशतिः॒ सम्प॑द्यन्ते। चतु॑र्विशतिरर्द्धमा॒साः। अ॒र्द्ध॒मा॒सेभ्य॑ ए॒व य॒ज्ञस्य॒ शिरोऽव॑रुन्धे। अथो॒ खलु॑। स॒कृदे॒व प्र॒वृज्य॑। एक॒ हि शिर॑॥४७॥
८.६.३
अ॒ग्नि॒ष्टो॒मे प्रवृ॑णक्ति। ए॒तावा॒\an{} वै य॒ज्ञः। यावा॑नग्निष्टो॒मः। यावा॑ने॒व य॒ज्ञः। तस्य॒ शिरः॒ प्रति॑दधाति। नोक्थ्ये प्रवृ॑ञ्ज्यात्। प्र॒जा वै प॒शव॑ उ॒क्थानि॑। यदु॒क्थ्ये प्रवृ॒ञ्ज्यात्। प्र॒जां प॒शून॑स्य॒ निर्द॑हेत्। वि॒श्व॒जिति॒ सर्व॑पृष्ठे॒ प्रवृ॑णक्ति॥४८॥
८.६.४
पृ॒ष्ठानि॒ वा अच्यु॑तं च्यावयन्ति। पृ॒ष्ठैरे॒वास्मा॒ अच्यु॑तं च्यावयि॒त्वाऽव॑रुन्धे। अप॑श्यङ्गो॒पामित्या॑ह। प्रा॒णो वै गो॒पाः। प्रा॒णमे॒व प्र॒जासु॒ विया॑तयति। अप॑श्यङ्गो॒पामित्या॑ह। अ॒सौ वा आ॑दि॒त्यो गो॒पाः। स हीमाः प्र॒जा गो॑पा॒यति॑। तमे॒व प्र॒जानाङ्गो॒प्तार॑ङ्कुरुते। अनि॑पद्यमान॒मित्या॑ह॥४९॥
८.६.५
न ह्ये॑ष नि॒पद्य॑ते। आ च॒ परा॑ च प॒थिभि॒श्चर॑न्त॒मित्या॑ह। आ च॒ ह्ये॑ष परा॑ च प॒थिभि॒श्चर॑ति। स स॒ध्रीचीः॒ स विषू॑ची॒र्वसा॑न॒ इत्या॑ह। स॒ध्रीचीश्च॒ ह्ये॑ष विषू॑चीश्च॒ वसा॑नः प्र॒जा अ॒भि वि॒पश्य॑ति। आव॑रीवर्ति॒ भुव॑नेष्व॒न्तरित्या॑ह। आ ह्ये॑ष व॑री॒वर्ति॒ भुव॑नेष्व॒न्तः। अत्र॑ प्रा॒वीर्मधु॒ माध्वीभ्यां॒ मधु॒ माधू॑चीभ्या॒मित्या॑ह। वास॑न्तिकावे॒वास्मा॑ ऋ॒तू क॑ल्पयति। सम॒ग्निर॒ग्निना॑ ग॒तेत्या॑ह॥५०॥
८.६.६
ग्रैष्मा॑वे॒वास्मा॑ ऋ॒तू क॑ल्पयति। सम॒ग्निर॒ग्निना॑ ग॒तेत्या॑ह। अ॒ग्निर्ह्ये॑वैषोऽग्निना॑ स॒ङ्गच्छ॑ते। स्वाहा॒ सम॒ग्निस्तप॑सा ग॒तेत्या॑ह। पूर्व॑मे॒वादि॒तम्। उत्त॑रेणा॒भिगृ॑णाति। ध॒र्ता दि॒वो विभा॑सि॒ रज॑सः पृथि॒व्या इत्या॑ह। शा॒र॒दावे॒वास्मा॑ ऋ॒तू क॑ल्पयति॥५१॥
८.६.७
दि॒वि दे॒वेषु॒ होत्रा॑ य॒च्छेत्या॑ह। होत्रा॑भिरे॒वेमाल्लोँ॒कान्त्सन्द॑धाति। विश्वा॑सां भुवां पत॒ इत्या॑ह। हैम॑न्तिकावे॒वास्मा॑ ऋ॒तू क॑ल्पयति। दे॒व॒श्रूस्त्वन्दे॑व घर्म दे॒वान्पा॒हीत्या॑ह। शै॒शि॒रावे॒वास्मा॑ ऋ॒तू क॑ल्पयति। त॒पो॒जां वाच॑म॒स्मे निय॑च्छ देवा॒युव॒मित्या॑ह। या वै मेध्या॒ वाक्। सा त॑पो॒जाः। तामे॒वाव॑रुन्धे॥५२॥
८.६.८
गर्भो॑ दे॒वाना॒मित्या॑ह। गर्भो॒ ह्ये॑ष दे॒वानाम्। पि॒ता म॑ती॒नामित्या॑ह। प्र॒जा वै म॒तय॑। तासा॑मे॒ष ए॒व पि॒ता। यत् प्र॑व॒र्ग्य॑। तस्मा॑दे॒वमा॑ह। पति॑ प्र॒जाना॒मित्या॑ह। पति॒र्ह्ये॑ष प्र॒जानाम्। मति॑ कवी॒नामित्या॑ह॥५३॥
८.६.९
मति॒र्ह्ये॑ष क॑वी॒नाम्। सन्दे॒वो दे॒वेन॑ सवि॒त्रा य॑तिष्ट॒ स सूर्ये॑णारु॒क्तेत्या॑ह। अ॒मुं चै॒वादि॒त्यं प्र॑व॒र्ग्यं॑ च॒ सशास्ति। आ॒यु॒र्दास्त्वम॒स्मभ्य॑ङ्घर्म वर्चो॒दा अ॒सीत्या॑ह। आ॒शिष॑मे॒वैतामाशास्ते। पि॒ता नो॑ऽसि पि॒ता नो॑ बो॒धेत्या॑ह। बो॒धय॑त्ये॒वैनम्। न वै॒ ते॑ऽवका॒शा भ॑वन्ति। पत्नि॑यै दश॒मः। नव॒ वै पुरु॑षे प्रा॒णाः॥५४॥
८.६.१०
नाभि॑र्दश॒मी। प्रा॒णाने॒व यज॑माने दधाति। अथो॒ दशाक्षरा वि॒राट्। अन्नं॑  वि॒राट्। वि॒राजै॒वान्नाद्य॒मव॑रुन्धे। य॒ज्ञस्य॒ शिरोऽच्छिद्यत। तद्दे॒वा होत्रा॑भिः॒ प्रत्य॑दधुः। ऋ॒त्विजोऽवेक्षन्ते। ए॒ता वै होत्रा। होत्रा॑भिरे॒व य॒ज्ञस्य॒ शिरः॒ प्रति॑दधाति॥५५॥
८.६.११
रु॒चि॒तमवेक्षन्ते। रु॒चि॒ताद्वै प्र॒जाप॑तिः प्र॒जा अ॑सृजत। प्र॒जाना॒ सृष्ट्यै। रु॒चि॒तमवेक्षन्ते। रु॒चि॒ताद्वै प॒र्जन्यो॑ वर्\mbox{}षति। वर्\mbox{}षु॑कः प॒र्जन्यो॑ भवति। सं प्र॒जा ए॑धन्ते। रु॒चि॒तमवेक्षन्ते। रु॒चि॒तं वै ब्र॑ह्मवर्च॒सम्। ब्र॒ह्म॒व॒र्च॒सिनो॑ भवन्ति॥५६॥
८.६.१२
अ॒धी॒यन्तोऽवेक्षन्ते। सर्व॒मायु॑र्\mbox{}यन्ति। न पत्न्यवेक्षेत। यत्पन्त्य॒वेक्षे॑त। प्रजा॑येत। प्र॒जां त्व॑स्यै॒ निर्द॑हेत्। यन्नावेक्षे॑त। न प्रजा॑येत। नास्यै प्र॒जां निर्द॑हेत्। ति॒र॒स्कृत्य॒ यजु॑र्वाचयति। प्रजा॑यते। नास्यै प्र॒जां निर्द॑हति। त्वष्टी॑मती ते सपे॒येत्या॑ह। सपा॒द्धि प्र॒जाः प्र॒जाय॑न्ते॥५७॥
\anuvakamend

८.७.०
म॒नु॒ष्य॒ना॒मानि॑ प॒शव॑ सीद॒त्वित्या॒हेन्द्रा॒येत्या॑हार्द्धयति घ्नन्ति गृह्णा॒त्यहिसायै॒ पञ्चा॑ऽहादि॒त्यव॑ते॒ स्वाहेत्या॑ह पितृ॒माने॑ति च॒त्वारि॑ च। ७।
८.७.१
दे॒वस्य॑ त्वा सवि॒तुः प्र॑स॒व इति॑ रश॒नामाद॑त्ते॒ प्रसूत्यै। अ॒श्विनोर्बा॒हुभ्या॒मित्या॑ह। अ॒श्विनौ॒ हि दे॒वाना॑मध्व॒र्यू आस्ताम्। पू॒ष्णो हस्ताभ्या॒मित्या॑ह॒ यत्यै। आद॒देऽदि॑त्यै॒ रास्ना॒ऽसीत्या॑ह॒ यजु॑ष्कृत्यै। इड॒ एह्यदि॑त॒ एहि॒ सर॑स्व॒त्येहीत्या॑ह। ए॒तानि॒ वा अ॑स्यै देवना॒मानि॑। दे॒व॒ना॒मैरे॒वैना॒मा ह्व॑यति। असा॒वेह्यसा॒वेह्यसा॒वेहीत्या॑ह। ए॒तानि॒ वा अ॑स्यै मनुष्यना॒मानि॑॥५८॥
८.७.२
म॒नु॒ष्य॒ना॒मैरे॒वैना॒मा ह्व॑यति। षट्थ्सम्प॑द्यन्ते। षड्वा ऋ॒तव॑। ऋ॒तुभि॑रे॒वैना॒मा ह्व॑यति। अदि॑त्या उ॒ष्णीष॑म॒सीत्या॑ह। य॒था॒य॒जुरे॒वैतत्। वा॒युर॑स्यै॒ड इत्या॑ह। वा॒यु॒दे॒व॒त्यो॑ वै व॒त्सः। पू॒षा त्वो॒पाव॑सृज॒त्वित्या॑ह। पौ॒ष्णा वै दे॒वत॑या प॒शव॑॥५९॥
८.७.३
स्वयै॒वैनं॑ दे॒वत॑यो॒पाव॑सृजति। अ॒श्विभ्यां॒ प्रदा॑प॒येत्या॑ह। अ॒श्विनौ॒ वै दे॒वानां भि॒षजौ। ताभ्या॑मे॒वास्मै॑ भेष॒जं क॑रोति। यस्ते॒ स्तन॑ शश॒य इत्या॑ह। स्तौत्ये॒वैनाम्। उस्र॑ घ॒र्म शि॒षोस्र॑ घ॒र्मं पा॑हि घ॒र्माय॑ शि॒षेत्या॑ह। यथा ब्रू॒याद॒मुष्मै॑ दे॒हीति॑। ता॒दृगे॒व तत्। बृह॒स्पति॒स्त्वोप॑ सीद॒त्वित्या॑ह॥६०॥
८.७.४
ब्रह्म॒ वै दे॒वानां॒ बृह॒स्पति॑। ब्रह्म॑णै॒वैना॒मुप॑सीदति। दान॑वः स्थ॒ पेर॑व॒ इत्या॑ह। मेध्या॑ने॒वैनान्करोति। वि॒ष्व॒ग्वृतो॒ लोहि॑ते॒नेत्या॑ह॒ व्यावृ॑त्त्यै। अ॒श्विभ्यां पिन्वस्व॒ सर॑स्वत्यै पिन्वस्व पू॒ष्णे पि॑न्वस्व॒ बृह॒स्पत॑ये पिन्व॒स्वेत्या॑ह। ए॒ताभ्यो॒ ह्ये॑षा दे॒वताभ्यः॒ पिन्व॑ते। इन्द्रा॑य पिन्व॒स्वेन्द्रा॑य पिन्व॒स्वेत्या॑ह। इन्द्र॑मे॒व भा॑ग॒धेये॑न॒ सम॑र्द्धयति। द्विरिन्द्रा॒येत्या॑ह॥६१॥
८.७.५
तस्मा॒दिन्द्रो॑ दे॒वता॑नां भूयिष्ठ॒भाक्त॑मः। गा॒य॒त्रो॑ऽसि॒ त्रैष्टु॑भोऽसि॒ जाग॑तम॒सीति॑ शफोपय॒मानाद॑त्ते। छन्दो॑भिरे॒वैना॒नाद॑त्ते। स॒होर्जो भा॒गेनोप॒मेहीत्या॑ह। ऊ॒र्ज ए॒वैनं॑ भा॒गम॑कः। अ॒श्विनौ॒ वा ए॒तद्य॒ज्ञस्य॒ शिर॑ प्रति॒दध॑तावब्रूताम्। आ॒वाभ्या॑मे॒व पूर्वाभ्यां॒ वष॑ट्क्रियाता॒ इति॑। इन्द्राश्विना॒ मधु॑नः सार॒घस्येत्या॑ह। अ॒श्विभ्या॑मे॒व पूर्वाभ्यां॒ वष॑ट्करोति। अथो॑ अ॒श्विना॑वे॒व भा॑ग॒धेये॑न॒ सम॑र्द्धयति॥६२॥
८.७.६
घ॒र्मं पा॑त वसवो॒ यज॑ता॒ वडित्या॑ह। वसू॑ने॒व भा॑ग॒धेये॑न॒ सम॑र्द्धयति। यद्व॑षट्कु॒र्यात्। या॒तया॑माऽस्य वषट्का॒रः स्यात्। यन्न व॑षट्कु॒र्यात्। रक्षासि य॒ज्ञ ह॑न्युः। वडित्या॑ह। प॒रोक्ष॑मे॒व वष॑ट्करोति। नास्य॑ या॒तया॑मा वषट्का॒रो भव॑ति। न य॒ज्ञ रक्षासि घ्नन्ति॥६३॥
८.७.७
स्वाहा त्वा॒ सूर्य॑स्य र॒श्मये॑ वृष्टि॒वन॑ये जुहो॒मीत्या॑ह। यो वा अ॑स्य॒ पुण्यो॑ र॒श्मिः। स वृ॑ष्टि॒वनि॑। तस्मा॑ ए॒वैनं॑ जुहोति। मधु॑ ह॒विर॒सीत्या॑ह। स्व॒दय॑त्ये॒वैनम्। सूर्य॑स्य॒ तप॑स्त॒पेत्या॑ह। य॒था॒य॒जुरे॒वैतत्। द्यावा॑पृथि॒वीभ्यां त्वा॒ परि॑गृह्णा॒मीत्या॑ह। द्यावा॑पृथि॒वीभ्या॑मे॒वैनं॒ परि॑गृह्णाति॥६४॥
८.७.८
अ॒न्तरि॑क्षेण॒ त्वोप॑यच्छा॒मीत्या॑ह। अ॒न्तरि॑क्षेणै॒वैन॒मुप॑यच्छति। न वा ए॒तं म॑नु॒ष्यो॑ भर्तु॑मर्\mbox{}हति। दे॒वानां त्वा पितृ॒णामनु॑मतो॒ भर्तु शकेय॒मित्या॑ह। दे॒वैरे॒वैनं॑ पि॒तृभि॒रनु॑मत॒ आद॑त्ते। वि वा ए॑नमे॒तद॑र्द्धयन्ति। यत्प॒श्चाप्र॒वृज्य॑ पु॒रो जुह्व॑ति। तेजो॑ऽसि॒ तेजोऽनु॒ प्रेहीत्या॑ह। तेज॑ ए॒वास्मि॑न्दधाति। दि॒वि॒स्पृङ्मा मा॑ हिसीरन्तरिक्ष॒स्पृङ्मा मा॑ हिसीः पृथिवि॒स्पृङ्मा मा॑ हिसी॒रित्या॒हाहिसायै॥६५॥
८.७.९
सुव॑रसि॒ सुव॑र्मे यच्छ॒ दिवं॑ यच्छ दि॒वो मा॑ पा॒हीत्या॑ह। आ॒शिष॑मे॒वैतामाशास्ते। शिरो॒ वा ए॒तद्य॒ज्ञस्य॑। यत्प्र॑व॒र्ग्य॑। आ॒त्मा वा॒युः। उ॒द्यत्य॑ वातना॒मान्या॑ह। आ॒त्मन्ने॒व य॒ज्ञस्य॒ शिरः॒ प्रति॑दधाति। अन॑वानम्। प्रा॒णाना॒ सन्त॑त्यै। पञ्चा॑ह॥६६॥
८.७.१०
पाङ्क्तो॑ य॒ज्ञः। यावा॑ने॒व य॒ज्ञः। तस्य॒ शिरः॒ प्रति॑दधाति। अ॒ग्नये त्वा॒ वसु॑मते॒ स्वाहेत्या॑ह। अ॒सौ वा आ॑दि॒त्योऽग्निर्वसु॑मान्। तस्मा॑ ए॒वैनं॑ जुहोति। सोमा॑य त्वा रु॒द्रव॑ते॒ स्वाहेत्या॑ह। च॒न्द्रमा॒ वै सोमो॑ रु॒द्रवान्॑। तस्मा॑ ए॒वैनं॑ जुहोति। वरु॑णाय त्वाऽऽदि॒त्यव॑ते॒ स्वाहेत्या॑ह॥६७॥
८.७.११
अ॒प्सु वै वरु॑ण आदि॒त्यवान्॑। तस्मा॑ ए॒वैनं॑ जुहोति। बृह॒स्पत॑ये त्वा वि॒श्वदेव्यावते॒ स्वाहेत्या॑ह। ब्रह्म॒ वै दे॒वानां॒ बृह॒स्पति॑। ब्रह्म॑णै॒वैनं॑ जुहोति। स॒वि॒त्रे त्व॑र्भु॒मते॑ विभु॒मते प्रभु॒मते॒ वाज॑वते॒ स्वाहेत्या॑ह। सं॒व॒त्स॒रो वै स॑वि॒तर्भु॒मान् वि॑भु॒मान्प्र॑भु॒मान् वाज॑वान्। तस्मा॑ ए॒वैनं॑ जुहोति। य॒माय॒ त्वाऽङ्गि॑रस्वते पितृ॒मते॒ स्वाहेत्या॑ह। प्रा॒णो वै य॒मोऽङ्गि॑रस्वान्पितृ॒मान्॥६८॥
८.७.१२
तस्मा॑ ए॒वैनं॑ जुहोति। ए॒ताभ्य॑ ए॒वैनं॑ दे॒वताभ्यो जुहोति। दश॒ सम्प॑द्यन्ते। दशाक्षरा वि॒राट्। अन्नं॑  वि॒राट्। वि॒राजै॒वान्नाद्य॒मव॑रुन्धे। रौ॒हि॒णाभ्यां॒ वै दे॒वाः सु॑व॒र्गं॑ लो॒कमा॑यन्। तद्रौ॑हि॒णयो॑ रौहिण॒त्वम्। यद्रौ॑हि॒णौ भव॑तः। रौ॒हि॒णाभ्या॑मे॒व तद्यज॑मानः सुव॒र्गं लो॒कमे॑ति। अह॒र्ज्योति॑ के॒तुना॑ जुषता सुज्यो॒तिर्ज्योति॑षा॒ स्वाहा॒ रात्रि॒र्ज्योति॑ के॒तुना॑ जुषता सुज्यो॒तिर्ज्योति॑षा॒ स्वाहेत्या॑ह। आ॒दि॒त्यमे॒व तद॒मुष्मि॑ल्लोँ॒केऽह्ना॑ प॒रस्ताद्दाधार। रात्रि॑या अ॒वस्तात्। तस्मा॑द॒सावा॑दि॒त्यो॑ऽमुष्मि॑ल्लोँ॒के॑ऽहोरा॒त्राभ्यान्धृ॒तः॥६९॥
\anuvakamend

८.८.०
अ॒क॒र॒श्वि॒नेत्या॑ह प्र॒दिशो॑ ग॒च्छेत्या॑ह पितृ॒णाम॑न्तःपरि॒धि पि॑न्वयति धार॒येत्या॑ह॒ वाचो॑ घर्म॒पास्तेभ्य॑ ए॒वैनं॑ जुहोत्य॒न्वीक्षे॑त होत॒व्या॥३॥ मित्य॒ग्नावित्या॑ह दधतेऽगोपायत्स॒प्त च॑॥ ८॥
८.८.१
विश्वा॒ आशा॑ दक्षिण॒सदित्या॑ह। विश्वा॑ने॒व दे॒वान्प्री॑णाति। अथो॒ दुरि॑ष्ट्या ए॒वैनं॑ पाति। विश्वान्दे॒वान॑याडि॒हेत्या॑ह। विश्वा॑ने॒व दे॒वान्भा॑ग॒धेये॑न॒ सम॑र्द्धयति। स्वाहा॑कृतस्य घ॒र्मस्य॒ मधो पिबतमश्वि॒नेत्या॑ह। अ॒श्विना॑वे॒व भा॑ग॒धेये॑न॒ सम॑र्द्धयति। स्वाहा॒ऽग्नये॑ य॒ज्ञिया॑य॒ शं यजु॑र्भि॒रित्या॑ह। अ॒भ्ये॑वैन॑ङ्घारयति। अथो॑ ह॒विरे॒वाक॑॥७०॥
८.८.२
अश्वि॑ना घ॒र्मं पा॑त हार्दिवा॒नमह॑र्दि॒वाभि॑रू॒तिभि॒रित्या॑ह। अ॒श्विना॑वे॒व भा॑ग॒धेये॑न॒ सम॑र्द्धयति। अनु॑ वा॒न्द्यावा॑पृथि॒वी मसाता॒मित्या॒हानु॑मत्यै। स्वाहेन्द्रा॑य॒ स्वाहेन्द्रा॒वडित्या॑ह। इन्द्रा॑य॒ हि पु॒रो हू॒यते। आ॒श्राव्या॑ह घ॒र्मस्य॑ य॒जेति॑। वष॑ट्कृते जुहोति। रक्ष॑सा॒मप॑हत्यै। अनु॑यजति स्व॒गाकृ॑त्यै। घ॒र्मम॑पातमश्वि॒नेत्या॑ह॥७१॥
८.८.३
पूर्व॑मे॒वोदि॒तम्। उत्त॑रेणा॒भिगृ॑णाति। अनु॑वा॒न्द्यावा॑पृथि॒वी अ॑मसाता॒मित्या॒हानु॑मत्यै। तं प्रा॒व्यं॑ यथा॒वण्णमो॑ दि॒वे नम॑ पृथि॒व्या इत्या॑ह। य॒था॒य॒जुरे॒वैतत्। दि॒विधा॑ इ॒मं य॒ज्ञं य॒ज्ञमि॒मन्दि॒विधा॒ इत्या॑ह। सु॒व॒र्गमे॒वैनं॑ लो॒कं ग॑मयति। दिवं॑ गच्छा॒न्तरि॑क्षं गच्छ पृथि॒वीङ्ग॒च्छेत्या॑ह। ए॒ष्वे॑वैनं॑ लो॒केषु॒ प्रति॑ष्ठापयति। पञ्च॑ प्र॒दिशो॑ ग॒च्छेत्या॑ह॥७२॥
८.८.४
दि॒क्ष्वे॑वैनं॒ प्रति॑ष्ठापयति। दे॒वान्घ॑र्म॒पान्ग॑च्छ पि॒तॄन्घ॑र्म॒पान्ग॒च्छेत्या॑ह। उ॒भयेष्वे॒वैनं॒ प्रति॑ष्ठापयति। यत्पिन्व॑ते। वर्\mbox{}षु॑कः प॒र्जन्यो॑ भवति। तस्मा॒त्पिन्व॑मानः॒ पुण्य॑। यत्प्राङ्पिन्व॑ते। तद्दे॒वानाम्। यद्द॑क्षि॒णा। तत्पि॑तृ॒णाम्॥७३॥
८.८.५
यत्प्र॒त्यक्। तन्म॑नु॒ष्या॑णाम्। यदुदङ्ङ्॑। तद्रु॒द्राणाम्। प्राञ्च॒मुद॑ञ्चं पिन्वयति। दे॒व॒त्राक॑। अथो॒ खलु॑। सर्वा॒ अनु॒ दिश॑ पिन्वयति। सर्वा॒ दिशः॒ समे॑धन्ते। अ॒न्तः॒प॒रि॒धि पि॑न्वयति॥७४॥
८.८.६
तेज॒सोऽस्क॑न्दाय। इ॒षे पी॑पिह्यू॒र्जे पी॑पि॒हीत्या॑ह। इष॑मे॒वोर्जं॒ यज॑माने दधाति। यज॑मानाय पीपि॒हीत्या॑ह। यज॑मानायै॒वैतामा॒शिष॒मा शास्ते। मह्यं॒ ज्यैष्ठ्या॑य पीपि॒हीत्या॑ह। आ॒त्मन॑ ए॒वैतामा॒शिष॒माशास्ते। त्विष्यै त्वा द्यु॒म्नाय॑ त्वेन्द्रि॒याय॑ त्वा॒ भूत्यै॒ त्वेत्या॑ह। य॒था॒य॒जुरे॒वैतत्। धर्मा॑सि सु॒धर्मा मेन्य॒स्मे ब्रह्मा॑णि धार॒येत्या॑ह॥७५॥
८.८.७
ब्रह्म॑न्ने॒वैनं॒ प्रति॑ष्ठापयति। नेत्त्वा॒ वात॑ स्क॒न्दया॒दिति॒ यद्य॑भि॒चरेत्। अ॒मुष्य॑ त्वा प्रा॒णे सा॑दयाम्य॒मुना॑ स॒ह नि॑र॒र्थङ्ग॒च्छेति॑ ब्रूया॒द्यं द्वि॒ष्यात्। यमे॒व द्वेष्टि॑। तेनै॑न स॒ह नि॑र॒र्थङ्ग॑मयति। पू॒ष्णे शर॑से॒ स्वाहेत्या॑ह। या ए॒व दे॒वता॑ हु॒तभा॑गाः। ताभ्य॑ ए॒वैनं॑ जुहोति। ग्राव॑भ्यः॒ स्वाहेत्या॑ह। या ए॒वान्तरि॑क्षे॒ वाच॑॥७६॥
८.८.८
ताभ्य॑ ए॒वैनं॑ जुहोति। प्र॒ति॒रेभ्यः॒ स्वाहेत्या॑ह। प्रा॒णा वै दे॒वाः प्र॑ति॒राः। तेभ्य॑ ए॒वैनं॑ जुहोति। द्यावा॑पृथि॒वीभ्या॒ स्वाहेत्या॑ह। द्यावा॑पृथि॒वीभ्या॑मे॒वैनं॑ जुहोति। पि॒तृभ्यो॑ घर्म॒पेभ्यः॒ स्वाहेत्या॑ह। ये वै यज्वा॑नः। ते पि॒तरो॑ घर्म॒पाः। तेभ्य॑ ए॒वैनं॑ जुहोति॥७७॥
८.८.९
रु॒द्राय॑ रु॒द्रहोत्रे॒ स्वाहेत्या॑ह। रु॒द्रमे॒व भा॑ग॒धेये॑न॒ सम॑र्द्धयति। स॒र्वतः॒ सम॑नक्ति। स॒र्वत॑ ए॒व रु॒द्रं नि॒रव॑दयते। उद॑ञ्चं॒ निर॑स्यति। ए॒षा वै रु॒द्रस्य॒ दिक्। स्वाया॑मे॒व दि॒शि रु॒द्रं नि॒रव॑दयते। अ॒प उप॑स्पृशति मेध्य॒त्वाय॑। नान्वीक्षेत। यद॒न्वीक्षे॑त॥७८॥
८.८.१०
चक्षु॑रस्य प्र॒मायु॑क स्यात्। तस्मा॒न्नान्वीक्ष्य॑। अपी॑परो॒ माऽह्नो॒ रात्रि॑यै मा पाह्ये॒षा ते॑ अग्ने स॒मित्तया॒ समि॑ध्य॒स्वायु॑र्मे दा॒ वर्च॑सा माञ्जी॒रित्या॑ह। आयु॑रे॒वास्मि॒न्वर्चो॑ दधाति। अपी॑परो मा॒ रात्रि॑या॒ अह्नो॑ मा पाह्ये॒षा ते॑ अग्ने स॒मित्तया॒ समि॑ध्य॒स्वायु॑र्मे दा॒ वर्च॑सा माञ्जी॒रित्या॑ह। आयु॑रे॒वास्मि॒न्वर्चो॑ दधाति। अ॒ग्निर्ज्योति॒र्ज्योति॑र॒ग्निः स्वाहा॒ सूर्यो॒ ज्योति॒र्ज्योतिः॒ सूर्यः॒ स्वाहेत्या॑ह। य॒था॒य॒जुरे॒वैतत्। ब्र॒ह्म॒वा॒दिनो॑ वदन्ति। हो॒त॒व्य॑मग्निहो॒त्रा(३)न्न हो॑त॒व्या(३)मिति॑॥७९॥
८.८.११
यद्यजु॑षा जुहु॒यात्। अय॑थापूर्व॒माहु॑ती जुहुयात्। यन्न जु॑हु॒यात्। अ॒ग्निः परा॑भवेत्। भूः स्वाहेत्ये॒व हो॑त॒व्यम्। य॒था॒पू॒र्वमाहु॑ती जु॒होति॑। नाग्निः परा॑भवति। हु॒त ह॒विर्मधु॑ ह॒विरित्या॑ह। स्व॒दय॑त्ये॒वैनम्। इन्द्र॑तमे॒ऽग्नावित्या॑ह॥८०॥
८.८.१२
प्रा॒णो वा इन्द्र॑तमो॒ऽग्निः। प्रा॒ण ए॒वैन॒मिन्द्र॑तमे॒ऽग्नौ जु॑होति। पि॒ता नो॑ऽसि॒ मा मा॑ हिसी॒रित्या॒हाहिसायै। अ॒श्याम॑ ते देव घर्म॒ मधु॑मतो॒ वाज॑वतः पितु॒मत॒ इत्या॑ह। आ॒शिष॑मे॒वैतामाशास्ते। स्व॒धा॒विनो॑ऽशी॒महि॑ त्वा॒ मा मा॑ हिसी॒रित्या॒हाहिसायै। तेज॑सा॒ वा ए॒ते व्यृ॑ध्यन्ते। ये प्र॑व॒र्ग्ये॑ण॒ चर॑न्ति। प्राश्ञ॑न्ति। तेज॑ ए॒वात्मन्द॑धते॥८१॥
८.८.१३
सं॒व॒त्स॒रं न मा॒सम॑श्ञीयात्। न रा॒मामुपे॑यात्। न मृ॒न्मये॑न पिबेत्। नास्य॑ रा॒म उच्छि॑ष्टं पिबेत्। तेज ए॒व तत्सश्य॑ति। दे॒वा॒सु॒राः संय॑त्ता आसन्। ते दे॒वा वि॑ज॒यमु॑प॒यन्त॑। वि॒भ्राजि॑ सौ॒र्ये ब्रह्म॒सन्न्य॑दधत। यत्किं च॑ दिवाकी॒र्त्यम्। तदे॒तेनै॒व व्र॒तेना॑गोपायत्। तस्मा॑दे॒तद्व्र॒तं चा॒र्यम्। तेज॑सो गोपी॒थाय॑। तस्मा॑दे॒तानि॒ यजूषि वि॒भ्राज॑ सौ॒र्यस्येत्या॑हुः। स्वाहा त्वा॒ सूर्य॑स्य र॒श्मिभ्य॒ इति॑ प्रा॒तः ससा॑दयति। स्वाहा त्वा॒ नक्ष॑त्रेभ्य॒ इति॑ सा॒यम्। ए॒ता वा ए॒तस्य॑ दे॒वता। ताभि॑रे॒वैन॒ सम॑र्द्धयति॥८२॥
\anuvakamend

८.९.०
ब्रह्म॑णस्त्वा पर॒स्पाया॒ इत्या॑ह दधात्य॒न्वित्य॑ रक्ष॒स्वी रक्ष॑सा॒मप॑हत्यै॒ वै हिर॑ण्यमाहार्द्धयति॒ ह्ये॑ष गृ॑णा॒त्वित्या॑ह मनु॒ष्या॑नित्या॑हास्यै॒षोऽष्टौ च॑॥ ९॥
८.९.१
घर्म॒ या ते॑ दि॒वि शुगिति॑ ति॒स्र आहु॑तीर्जुहोति। छन्दो॑भिरे॒वास्यै॒भ्यो लो॒केभ्यः॒ शुच॒मव॑ यजते। इय॒त्यग्रे॑ जुहोति। अथेय॒त्यथेय॑ति। त्रय॑ इ॒मे लो॒काः। अनु॑ नो॒ऽद्यानु॑मति॒रित्या॒हानु॑मत्यै। दि॒वस्त्वा॑ पर॒स्पाया॒ इत्या॑ह। दि॒व ए॒वेमाल्लोँ॒कान्दा॑धार। ब्रह्म॑णस्त्वा पर॒स्पाया॒ इत्या॑ह॥८३॥
८.९.२
ए॒ष्वे॑व लो॒केषु॑ प्र॒जा दा॑धार। प्रा॒णस्य॑ त्वा पर॒स्पाया॒ इत्या॑ह। प्र॒जास्वे॒व प्रा॒णान्दा॑धार। शिरो॒ वा ए॒तद्य॒ज्ञस्य॑। यत्प्र॑व॒र्ग्य॑। अ॒सौ खलु॒ वा आ॑दि॒त्यः प्र॑व॒र्ग्य॑। तं यद्द॑क्षि॒णा प्र॒त्यञ्च॒मुद॑ञ्चमुद्वा॒सयेत्। जि॒ह्मं य॒ज्ञस्य॒ शिरो॑ हरेत्। प्राञ्च॒मुद्वा॑सयति। पु॒रस्ता॑दे॒व य॒ज्ञस्य॒ शिरः॒ प्रति॑दधाति॥८४॥
८.९.३
प्राञ्च॒मुद्वा॑सयति। तस्मा॑द॒सावा॑दि॒त्यः पु॒रस्ता॒दुदे॑ति। श॒फो॒प॒य॒मान्ध॒वित्रा॑णि॒ धृष्टी॒ इत्य॒न्वव॑हरन्ति। सात्मा॑नमे॒वैन॒ सत॑नुं करोति। सात्मा॒ऽमुष्मि॑ल्लोँ॒के भ॑वति। य ए॒वं वेद॑। औदुं॑बराणि भवन्ति। ऊर्ग्वा उ॑दु॒म्बर॑। ऊर्ज॑मे॒वाव॑रुन्धे। वर्त्म॑ना॒ वा अ॒न्वित्य॑॥८५॥
८.९.४
य॒ज्ञ रक्षासि जिघासन्ति। साम्ना प्रस्तो॒ताऽन्ववै॑ति। साम॒ वै र॑क्षो॒हा। रक्ष॑सा॒मप॑हत्यै। त्रिर्नि॒धन॒मुपै॑ति। त्रय॑ इ॒मे लो॒काः। ए॒भ्य ए॒व लो॒केभ्यो॒ रक्षा॒स्यप॑हन्ति। पुरु॑षःपुरुषो नि॒धन॒मुपै॑ति। पुरु॑षःपुरुषो॒ हि र॑क्ष॒स्वी। रक्ष॑सा॒मप॑हत्यै॥८६॥
८.९.५
यत्पृ॑थि॒व्यामु॑द्वा॒सयेत्। पृ॒थि॒वी शु॒चाऽर्प॑येत्। यद॒प्सु। अ॒पः  शु॒चार्प॑येत्। यदोष॑धीषु। ओष॑धीः  शु॒चाऽर्प॑येत्। यद्वन॒स्पति॑षु। वन॒स्पतीञ्छु॒चार्प॑येत्। हिर॑ण्यं नि॒धायोद्वा॑सयति। अ॒मृतं॒ वै हिर॑ण्यम्॥८७॥
८.९.६
अ॒मृत॑ ए॒वैनं॒ प्रति॑ष्ठापयति। व॒ल्गुर॑सि शं॒ युधा॑या॒ इति॒ त्रिः प॑रिषि॒ञ्चन्पर्ये॑ति। त्रि॒वृद्वा अ॒ग्निः। यावा॑ने॒वाग्निः। तस्य॒ शुच शमयति। त्रिः पुनः॒ पर्ये॑ति। षट्थ्सम्प॑द्यन्ते। षड्वा ऋ॒तव॑। ऋ॒तुभि॑रे॒वास्य॒ शुच शमयति। चतु॑ स्रक्ति॒र्नाभि॑र्\mbox{}ऋ॒तस्येत्या॑ह॥८८॥
८.९.७
इ॒यं वा ऋ॒तम्। तस्या॑ ए॒ष ए॒व नाभि॑। यत् प्र॑व॒र्ग्य॑। तस्मा॑दे॒वमा॑ह। सदो॑ वि॒श्वायु॒रित्या॑ह। सदो॒ हीयम्। अप॒ द्वेषो॒ अप॒ ह्वर॒ इत्या॑ह॒ भ्रातृ॑व्यापनुत्त्यै। घर्मै॒तत्तेऽन्न॑मे॒तत्पुरी॑ष॒मिति॑ द॒ध्ना म॑धुमि॒श्रेण॑ पूरयति। ऊर्ग्वा अ॒न्नाद्यं॒ दधि॑। ऊ॒र्जैवैन॑म॒न्नाद्ये॑न॒ सम॑र्द्धयति॥८९॥
८.९.८
अन॑शनायुको भवति। य ए॒वं वेद॑। रन्ति॒र्नामा॑सि दि॒व्यो ग॑न्ध॒र्व इत्या॑ह। रू॒पमे॒वास्यै॒तन्म॑हि॒मान॒ रन्तिं॑ ब॒न्धुतां॒ व्याच॑ष्टे। सम॒हमायु॑षा॒ सं प्रा॒णेनेत्या॑ह। आ॒शिष॑मे॒वैतामा शास्ते। व्य॑सौ योऽस्मान्द्वेष्टि॒ यं च॑ व॒यं द्वि॒ष्म इत्या॑ह। अ॒भि॒चा॒र ए॒वास्यै॒षः। अचि॑क्रद॒द्वृषा॒ हरि॒रित्या॑ह। वृषा॒ ह्ये॑षः॥९०॥
८.९.९
वृषा॒ हरि॑। म॒हान्मि॒त्रो न द॑र्\mbox{}श॒त इत्या॑ह। स्तौत्ये॒वैन॑मे॒तत्। चिद॑सि समु॒द्रयो॑नि॒रित्या॑ह। स्वामे॒वैनं॒ योनिं॑ गमयति। नम॑स्ते अस्तु॒ मा मा॑ हिसी॒रित्या॒हाहिसायै। वि॒श्वाव॑सु सोम गन्ध॒र्वमित्या॑ह। यदे॒वास्य॑ क्रि॒यमा॑णस्यान्त॒र्यन्ति॑। तदे॒वास्यै॒तेना प्या॑ययति। वि॒श्वाव॑सुर॒भि तन्नो॑ गृणा॒त्वित्या॑ह॥९१॥
८.९.१०
पूर्व॑मे॒वोदि॒तम्। उत्त॑रेणा॒भि गृ॑णाति। धियो॑ हिन्वा॒नो धिय॒ इन्नो॑ अव्या॒दित्या॑ह। ऋ॒तूने॒वास्मै॑ कल्पयति। प्रासाङ्गन्ध॒र्वो अ॒मृता॑नि वोच॒दित्या॑ह। प्रा॒णा वा अ॒मृता। प्रा॒णाने॒वास्मै॑ कल्पयति। ए॒तत्त्वन्दे॑व घर्म दे॒वो दे॒वानुपा॑गा॒ इत्या॑ह। दे॒वो ह्ये॑ष सन्दे॒वानु॒पैति॑। इ॒दम॒हं म॑नु॒ष्यो॑ मनु॒ष्या॑नित्या॑ह॥९२॥
८.९.११
म॒नु॒ष्यो॑ हि। ए॒ष सन्म॑नु॒ष्या॑नु॒पैति॑। ई॒श्व॒रो वै प्र॑व॒र्ग्य॑मुद्वा॒सय\sn{}। प्र॒जां प॒शून्त्सो॑मपी॒थम॑नू॒द्वासः॒ सोम॑ पी॒थानु॒मेहि॑। स॒ह प्र॒जया॑ स॒ह रा॒यस्पोषे॒णेत्या॑ह। प्र॒जामे॒व प॒शून्त्सो॑मपी॒थमा॒त्मन्ध॑त्ते। सु॒मि॒त्रा न॒ आप॒ ओष॑धयः स॒न्त्वित्या॑ह। आ॒शिष॑मे॒वैतामा शास्ते। दु॒र्मि॒त्रास्तस्मै॑ भूयासु॒र्योऽस्मान्द्वेष्टि॒ यं च॑ व॒यं द्वि॒ष्म इत्या॑ह। अ॒भि॒चा॒र ए॒वास्यै॒षः। प्र वा ए॒षोऽस्माल्लो॒काच्च्य॑वते। यः प्र॑व॒र्ग्य॑मुद्वा॒सयति॑। उदु॒त्यं चि॒त्रमिति॑ सौ॒रीभ्या॑मृ॒ग्भ्यां पुन॒रेत्य॒ गार्\mbox{}ह॑पत्ये जुहोति। अ॒यं वै लो॒को गार्\mbox{}ह॑पत्यः। अ॒स्मिन्ने॒व लो॒के प्रति॑तिष्ठति। अ॒सौ खलु॒ वा आ॑दि॒त्यः सु॑व॒र्गो लो॒कः। यत्सौ॒री भव॑तः। तेनै॒व सु॑व॒र्गाल्लो॒कान्नैति॑॥९३॥
\anuvakamend

८.१०.०
गोः पय॑ उत्तरवे॒दिरा॑सते स्थापयति घ॒र्मो य॑न्ति॥ १०॥
८.१०.१
प्र॒जाप॑तिं॒ वै दे॒वाः  शु॒क्रं पयो॑ऽदुह्रन्। तदेभ्यो॒ न व्य॑भवत्। तद॒ग्निर्व्य॑करोत्। तानि॒ शुक्रि॑याणि॒ सामान्यभवन्। तेषां॒ यो रसो॒ऽत्यक्ष॑रत्। तानि॑ शुक्रय॒जूष्य॑भवन्। शुक्रि॑याणां॒ वा ए॒तानि॒ शुक्रि॑याणि। सा॒म॒प॒य॒सं वा ए॒तयो॑र॒न्यत्। दे॒वाना॑म॒न्यत्पय॑। यद्गोः पय॑॥९४॥
८.१०.२
तत्साम्नः॒ पय॑। यद॒जायै॒ पय॑। तद्दे॒वानां॒ पय॑। तस्मा॒द्यत्रै॒तैर्यजु॑र्भि॒श्चर॑न्ति। तत्पय॑सा चरन्ति। प्र॒जाप॑तिमे॒व तत्पय॑सा॒ऽन्नाद्ये॑न॒ सम॑र्द्धयन्ति। ए॒ष ह त्वै सा॒क्षात्प्र॑व॒र्ग्यं॑ भक्षयति। यस्यै॒वं  वि॒दुष॑ प्रव॒र्ग्य॑ प्रवृ॒ज्यते। उ॒त्त॒र॒वे॒द्यामुद्वा॑सये॒त्तेज॑स्कामस्य। तेजो॒ वा उ॑त्तरवे॒दिः॥९५॥
८.१०.३
तेज॑ प्रव॒र्ग्य॑। तेज॑सै॒व तेजः॒ सम॑र्द्धयति। उ॒त्त॒र॒वे॒द्यामुद्वा॑सये॒दन्न॑कामस्य। शिरो॒ वा ए॒तद्य॒ज्ञस्य॑। यत्प्र॑व॒र्ग्य॑। मुख॑मुत्तरवे॒दिः। शी॒र्ष्णैव मुख॒ सन्द॑धात्य॒न्नाद्या॑य। अ॒न्ना॒द ए॒व भ॑वति। यत्र॒ खलु॒ वा ए॒तमुद्वा॑सितं॒ वयासि प॒र्यास॑ते। परि॒ वै ता समां प्र॒जा वयास्यासते॥९६॥
८.१०.४
तस्मा॑दुत्तरवे॒द्यामे॒वोद्वा॑सयेत्। प्र॒जानाङ्गोपी॒थाय॑। पु॒रो वा॑ प॒श्चाद्वोद्वा॑सयेत्। पु॒रस्ता॒द्वा ए॒तज्ज्योति॒रुदे॑ति। तत्प॒श्चान्निम्रो॑चति। स्वामे॒वैनं॒ योनि॒मनूद्वा॑सयति। अ॒पां मध्य॒ उद्वा॑सयेत्। अ॒पां वा ए॒तन्मध्या॒ज्ज्योति॑रजायत। ज्योति॑ प्रव॒र्ग्य॑। स्वयै॒वैनं॒ योनौ॒ प्रति॑ष्ठापयति॥९७॥
८.१०.५
यं द्वि॒ष्यात्। यत्र॒ स स्यात्। तस्यान्दि॒श्युद्वा॑सयेत्। ए॒ष वा अ॒ग्निर्वैश्वान॒रः। यत्प्र॑व॒र्ग्य॑। अ॒ग्निनै॒वैनं॑ वैश्वान॒रेणा॒भि प्रव॑र्तयति। औदु॑म्बर्या॒ शाखा॑या॒मुद्वा॑सयेत्। ऊर्ग्वा उ॑दु॒म्बर॑। अन्नं॑ प्रा॒णः। शुग्घ॒र्मः॥९८॥
८.१०.६
इ॒दम॒हम॒मुष्या॑मुष्याय॒णस्य॑ शु॒चा प्रा॒णमपि॑ दहा॒मीत्या॑ह। शु॒चैवास्य॑ प्रा॒णमपि॑ दहति। ता॒जगार्ति॒मार्च्छ॑ति। यत्र॑ द॒र्भा उ॑प॒दीक॑सन्तताः॒ स्युः। तदुद्वा॑सये॒द्वृष्टि॑कामस्य। ए॒ता वा अ॒पाम॑नू॒ज्झाव॑र्यो॒ नाम॑। यद्द॒र्भाः। अ॒सौ खलु॒ वा आ॑दि॒त्य इ॒तो वृष्टि॒मुदी॑रयति। अ॒सावे॒वास्मा॑ आदि॒त्यो वृष्टिं॒ निय॑च्छति। ता आपो॒ निय॑ता॒ धन्व॑ना यन्ति॥९९॥
\anuvakamend

८.११.०
व॒द॒न्ति॒ त॒नुवा॒ सस॑न्नो हू॒यमा॑नो ब्रूया॒दन्नं॑ प्र॒जाप॑ति॒रेकं॑ च॥ ११॥
८.११.१
प्र॒जाप॑तिः सम्भ्रि॒यमा॑णः। स॒म्राट्थ्सम्भृ॑तः। घ॒र्मः प्रवृ॑क्तः। म॒हा॒वी॒र उद्वा॑सितः। अ॒सौ खलु॒ वावैष आ॑दि॒त्यः। यत्प्र॑व॒र्ग्य॑। स ए॒तानि॒ नामान्यकुरुत। य ए॒वं वेद॑। वि॒दुरे॑नं॒ नाम्ना। ब्र॒ह्म॒वा॒दिनो॑ वदन्ति॥१००॥
८.११.२
यो वै वसी॑यासं यथाना॒ममु॑प॒चर॑ति। पुण्यार्तिं॒ वै स तस्मै॑ कामयते। पुण्यार्तिमस्मै कामयन्ते। य ए॒वं वेद॑। तस्मा॑दे॒वं  वि॒द्वान्। घ॒र्म इति॒ दिवाऽऽच॑क्षीत। स॒म्राडिति॒ नक्तम्। ए॒ते वा ए॒तस्य॑ प्रि॒ये त॒नुवौ। ए॒ते अ॑स्य प्रि॒ये नाम॑नी। प्रि॒ययै॒वैनं॑ त॒नुवा॥१०१॥
८.११.३
प्रि॒येण॒ नाम्ना॒ सम॑र्द्धयति। की॒र्तिर॑स्य॒ पूर्वाग॑च्छति ज॒नता॑माय॒तः। गा॒य॒त्री दे॒वेभ्योऽपाक्रामत्। तान्दे॒वाः प्र॑व॒र्ग्ये॑णै॒वानु॒ व्य॑भवन्। प्र॒व॒र्ग्ये॑णाप्नुवन्। यच्च॑तुर्विशति॒कृत्व॑ प्रव॒र्ग्यं॑ प्रवृ॒णक्ति॑। गा॒य॒त्रीमे॒व तदनु॒ विभ॑वति। गा॒य॒त्रीमाप्नोति। पूर्वाऽस्य॒ जनं॑ य॒तः की॒र्तिर्ग॑च्छति। वै॒श्व॒दे॒वः सस॑न्नः॥१०२॥
८.११.४
वस॑वः॒ प्रवृ॑क्तः। सोमो॑ऽभिकी॒र्यमा॑णः। आ॒श्वि॒नः पय॑स्यानी॒यमा॑ने। मा॒रु॒तः क्वथ\sn{}। पौ॒ष्ण उद॑न्तः। सा॒र॒स्व॒तो वि॒ष्यन्द॑मानः। मै॒त्रः  शरो॑ गृही॒तः। तेज॒ उद्य॑तः। वा॒युर्ह्रि॒यमा॑णः। प्र॒जाप॑तिर्\mbox{}हू॒यमा॑नः॥१०३॥
८.११.५
वाग्घु॒तः। अ॒सौ खलु॒ वावैष आ॑दि॒त्यः। यत्प्र॑व॒र्ग्य॑। स ए॒तानि॒ नामान्यकुरुत। य ए॒वं वेद॑। वि॒दुरे॑नं॒ नाम्ना। ब्र॒ह्म॒वा॒दिनो॑ वदन्ति। यन्मृ॒न्मय॒माहु॑तिं॒ नाश्ञु॒तेऽथ॑। कस्मा॑दे॒षोऽश्ञुत॒ इति॑। वागे॒ष इति॑ ब्रूयात्॥१०४॥
८.११.६
वा॒च्ये॑व वाचं॑ दधाति। तस्मा॑दश्ञुते। प्र॒जाप॑ति॒र्वा ए॒ष द्वा॑दश॒धा विहि॑तः। यत्प्र॑व॒र्ग्य॑। यत्प्राग॑वका॒शेभ्य॑। तेन॑ प्र॒जा अ॑सृजत। अ॒व॒का॒शैर्दे॑वासु॒रान॑सृजत। यदू॒र्ध्वम॑वका॒शेभ्य॑। तेनान्न॑मसृजत। अन्नं॑ प्र॒जाप॑तिः। प्र॒जाप॑ति॒र्वावैषः॥१०५॥
\anuvakamend

८.१२.०
नक्ष॑त्राण्येति वि॒राज॑मेति तपति॥ १२॥ दे॒वा वै स॒त्र सा॑वि॒त्रं परि॑श्रिते॒ ब्रह्म॒न् प्रच॑रिष्यामो॒ऽग्निष्ट्वा॒ शिरो ग्री॒वा दे॒वस्य॑ रश॒नां  विश्वा॒ आशा॒ घर्म॒ या ते प्र॒जाप॑ति शुक्रं प्र॒जाप॑तिः सम्भ्रि॒यमा॑णः सवि॒ता भू॒त्वा द्वाद॑श॥ १२॥ दे॒वा वै स॒त्र स ख॑दि॒रः परि॑श्रितेऽभिपू॒र्वमथो॒ रक्ष॑सा॒ङ्ग्रैष्मा॑वे॒वास्मै॒ ब्रह्म॒ वै दे॒वाना॒मश्वि॑ना घ॒र्मं पा॑तं प्रा॒णो वै वृषा॒ हरि॒र्यो वै वसी॑यासं यथाना॒मम॒ष्टोत्त॑रशतम्॥ १०८॥ दे॒वा वै स॒त्रमैव त॑पति॥ ॐ शान्तिः  शान्तिः  शान्तिः॥ हरि॑ ओम्। श्रीकृष्णार्पणमस्तु॥
८.१२.१
स॒वि॒ता भू॒त्वा प्र॑थ॒मेऽह॒न्प्रवृ॑ज्यते। तेन॒ कामा एति। यद्द्वि॒तीयेऽह॑न्प्रवृ॒ज्यते। अ॒ग्निर्भू॒त्वा दे॒वाने॑ति। यत्तृ॒तीयेऽह॑न्प्रवृ॒ज्यते। वा॒युर्भू॒त्वा प्रा॒णाने॑ति। यच्च॑तु॒र्थेऽह॑न्प्रवृ॒ज्यते। आ॒दि॒त्यो भू॒त्वा र॒श्मीने॑ति। यत्प॑ञ्च॒मेऽह॑न्प्रवृ॒ज्यते। च॒न्द्रमा॑ भू॒त्वा नक्ष॑त्राण्येति॥१०६॥
८.१२.२
यथ्ष॒ष्ठेऽह॑न्प्रवृ॒ज्यते। ऋ॒तुर्भू॒त्वा सं॑वत्स॒रमे॑ति। यत्स॑प्त॒मेऽह॑न्प्रवृ॒ज्यते। धा॒ता भू॒त्वा शक्व॑रीमेति। यद॑ष्ट॒मेऽह॑न्प्रवृ॒ज्यते। बृह॒स्पति॑र्भू॒त्वा गा॑य॒त्रीमे॑ति। यन्न॑व॒मेऽह॑न्प्रवृ॒ज्यते। मि॒त्रो भू॒त्वा त्रि॒वृत॑ इ॒माल्लोँ॒काने॑ति। यद्द॑श॒मेऽह॑न्प्रवृ॒ज्यते। वरु॑णो भू॒त्वा वि॒राज॑मेति॥१०७॥
८.१२.३
यदे॑काद॒शेऽह॑न्प्रवृ॒ज्यते। इन्द्रो॑ भू॒त्वा त्रि॒ष्टुभ॑मेति। यद्द्वा॑द॒शेऽह॑न्प्रवृ॒ज्यते। सोमो॑ भू॒त्वा सु॒त्यामे॑ति। यत्पु॒रस्ता॑दुप॒सदां प्रवृ॒ज्यते। तस्मा॑दि॒तः परा॑ङ॒मूल्लोँ॒कास्तप॑न्नेति। यदु॒परि॑ष्टादुप॒सदां प्रवृ॒ज्यते। तस्मा॑द॒मुतो॒ऽर्वाङि॒माल्लोँ॒कास्तप॑न्नेति। य ए॒वं वेद॑। ऐव त॑पति॥१०८॥

ॐ शं नस्तन्नो॒ मा हा॑सीत्॥ ॐ शान्तिः॒ शान्तिः॒शान्ति॑॥

\closesection
\clearpage

\sect{षष्ठः प्रश्नः}\setcounter{anuvakam}{0}
ॐ सन्त्वा॑ सिञ्चामि॒ यजुषा॑ प्र॒जामायु॒र्धनं॑ च॥ ॐ शान्तिः  शान्तिः  शन्तिः॥

४.०.०
प॒रे॒यु॒वासं॒ प्रवि॒द्वान्भुव॑नस्या॒भ्याव॑वृत्स्वा॒जो भा॒गो॑ऽयं वै चतु॑श्चत्वारिशत्। य ए॒तस्य॒ त्वत्पञ्च॑। प्रके॒तुने॒दन्ते॒ नाके॑ सुप॒र्णमपी॑हि॒ यौ ते॒ ये युध्य॑न्ते॒ तप॒साऽश्म॑न्वती रेवतीः॒ सर॑भध्व स॒हस्र॑धारम॒ष्टाविशतिः। यन्ते॒ यत्त॒ उत्ति॒ष्ठात॑ इ॒दन्त॒ उत्ति॑ष्ठ॒ प्रेह्यश्म॒\an{} यद्वा उद्व॒यम॒यं पञ्च॑विशतिः। आया॑तु त्रि॒शत्। वै॒श्वा॒न॒रे तस्मि॑न्द्र॒प्स इ॒ममपे॒ताहो॑भिर्युज्यन्तामघ्नि॒या अ॑दिते पा॒रं व॒ आप्या॑यस्व स॒प्तविशतिः। उत्ते॑ गृ॒हेऽक्षि॑ति॒स्तेभ्य॑ पृथिवि॒ षड्ढो॑ता॒ परं॑ मे श॒ग्माः पृ॑थि॒व्या अ॒न्तरि॑क्षस्य॒ द्वात्रिशत्। अ॒पू॒पवा॑नसौ॒ दश॑ श॒त द॑श। ए॒तास्ते॑ ते॒ दिशः॒ सर्वा॑ इ॒दमश्म॑न्विश॒तिः। आरो॑हत त॒नुवै क्रू॒रं च॒कार॒ पुन॑र्मृ॒त्यवे॒ मा नोनु॑गाद्दद्मह इ॒मा नारीः॒ परि॒ त्रयो॑विशतिः। अप॑नः सुक्षेत्रि॒या प्रयद्भन्दि॑ष्ठः॒ प्रयद॒ग्नेः प्रयत्ते॑ अग्ने॒ त्व हि द्विषः॒ सनः॒ सिन्धु॒माप॑ प्रव॒णादु॑द्व॒नादा॑न॒न्दाय॒ न वै तत्र॒ यत्रे॒दं चतु॑र्विशतिः। अप॑श्या॒मावृ॑णीहि॒ द्वाद॑श द्वादश। १२ प॒रे॒यु॒वास॒मायात्वे॒तास्ते॑ स॒प्तविशतिः। २७ प॒रे॒यु॒वास॒मोमुत्सृ॒जत।

४.१.०
वि॒द्वान॒भ्याव॑वृत्स्वा॒भिमा॑तीर्जयेम॒ शरी॑रैश्च॒त्वारि॑ च॥ १॥
४.१.१
प॒रे॒यु॒वासं॑ प्र॒वतो॑ म॒हीरनु॑ ब॒हुभ्यः॒ पन्था॑मनपस्पशा॒नम्। वै॒व॒स्व॒त स॒ङ्गम॑नं॒ जना॑नां य॒म राजा॑न ह॒विषा॑ दुवस्यत। इ॒दं त्वा॒ वस्त्रं॑ प्रथ॒मन्वाग॒न्नपै॒तदू॑ह॒ यदि॒हाबि॑भः पु॒रा। इ॒ष्टा॒पू॒र्तमनु॒ संप॑श्य॒ दक्षि॑णां॒ यथा॑ ते द॒त्तं ब॑हु॒धा विब॑न्धुषु। इ॒मौ यु॑नज्मि ते व॒ह्नी असु॑नीथाय वो॒ढवे। याभ्यां य॒मस्य॒ साद॑न सु॒कृतां॒ चापि॑ गच्छतात्। पू॒षा त्वे॒तश्च्या॑वयतु॒ प्रवि॒द्वानन॑ष्टपशु॒र्भुव॑नस्य गो॒पाः। स त्वै॒तेभ्यः॒ परि॑ददात्पि॒तृभ्यो॒ऽग्निर्दे॒वेभ्य॑ सुवि॒दत्रेभ्यः। पू॒षेमा आशा॒ अनु॑वेद॒ सर्वाः॒ सो अ॒स्मा अभ॑यतमेन नेषत्। स्व॒स्ति॒दा अघृ॑णिः॒ सर्व॑वी॒रोऽप्र॑युच्छन्पु॒र ए॑तु॒ प्रवि॒द्वान्॥१॥
४.१.२
आयु॑र्वि॒श्वायुः॒ परि॑पासति त्वा पू॒षा त्वा॑ पातु॒ प्रप॑थे पु॒रस्तात्। यत्रास॑ते सु॒कृतो॒ यत्र॒ ते य॒युस्तत्र॑ त्वा दे॒वः स॑वि॒ता द॑धातु। भुव॑नस्य पत इ॒द ह॒विः। अ॒ग्नये॑ रयि॒मते॒ स्वाहा। पुरु॑षस्य सयाव॒र्यपेद॒घानि॑ मृज्महे। यथा॑ नो॒ अत्र॒ नाप॑रः पु॒रा ज॒रस॒ आय॑ति। पुरु॑षस्य सयावरि वि ते प्रा॒णम॑सि स्रसम्। शरी॑रेण म॒हीमिहि॑ स्व॒धयेहि॑ पि॒तॄनुप॑ प्र॒जया॒ऽस्मानि॒हाव॑ह। मैव॑म्मा॒स्ता प्रि॑ये॒ऽहन्दे॒वी स॒ती पि॑तृलो॒कं यदैषि॑। वि॒श्ववा॑रा॒ नभ॑सा॒ संव्य॑यन्त्यु॒भौ नो॑ लो॒कौ पय॑सा॒ऽभ्याव॑वृत्स्व॥२॥
४.१.३
इ॒यं नारी॑ पतिलो॒कं वृणा॒ना निप॑द्यत॒ उप॑ त्वा मर्त्य॒ प्रेतम्। विश्वं॑ पुरा॒णमनु॑ पा॒लय॑न्ती॒ तस्यै प्र॒जान्द्रवि॑णं चे॒ह धे॑हि। उदीर्ष्व नार्य॒भि जी॑वलो॒कमि॒तासु॑मे॒तमुप॑शेष॒ एहि॑। ह॒स्त॒ग्रा॒भस्य॑ दिधि॒षोस्त्वमे॒तत्पत्यु॑र्जनि॒त्वम॒भि सम्ब॑भूव। सु॒वर्ण॒ हस्ता॑दा॒ददा॑ना मृ॒तस्य॑ श्रि॒यै ब्रह्म॑णे॒ तेज॑से॒ बला॑य। अत्रै॒व त्वमि॒ह व॒य सु॒शेवा॒ विश्वाः॒ स्पृधो॑ अ॒भिमा॑तीर्जयेम। धनु॒र्\mbox{}हस्ता॑दा॒ददा॑ना मृ॒तस्य॑ श्रि॒यै क्ष॒त्रायौज॑से॒ बला॑य। अत्रै॒व त्वमि॒ह व॒य सु॒शेवा॒ विश्वाः॒ स्पृधो॑ अ॒भिमा॑तीर्जयेम। मणि॒ हस्ता॑दा॒ददा॑ना मृ॒तस्य॒ श्रि॒यै वि॒शे पुष्ट्यै॒ बला॑य। अत्रै॒व त्वमि॒ह व॒य सु॒शेवा॒ विश्वाः॒ स्पृधो॑ अ॒भिमा॑तीर्जयेम॥३॥
४.१.४
इ॒मम॑ग्ने चम॒सं मा विजीह्वरः प्रि॒यो दे॒वाना॑मु॒त सो॒म्यानाम्। ए॒ष यश्च॑म॒सो दे॑व॒पान॒स्तस्मि॑न्दे॒वा अ॒मृता॑ मादयन्ताम्। अ॒ग्नेर्वर्म॒ परि॒ गोभि॑र्व्ययस्व॒ सं प्रोर्णु॑ष्व॒ मेद॑सा॒ पीव॑सा च। नेत्त्वा॑ धृ॒ष्णुऱ्\mbox{}हर॑सा॒ जर्\mbox{}हृ॑षाणो॒ दध॑द्विध॒क्ष्यन्पर्य॒ङ्खया॑तै। मैन॑मग्ने॒ विद॑हो॒ माऽभिशो॑चो॒ माऽस्य॒ त्वचं॑ चिक्षिपो॒ मा शरी॑रम्। य॒दा शृ॒तं क॒रवो॑ जातवे॒दोऽथे॑मेनं॒ प्रहि॑णुतात्पि॒तृभ्य॑। शृ॒तं य॒दा क॒रसि॑ जातवे॒दोऽथे॑मेनं॒ परि॑दत्तात्पि॒तृभ्य॑। य॒दा गच्छा॒त्यसु॑नीतिमे॒तामथा॑ दे॒वानां वश॒नीर्भ॑वाति। सूर्यं॑ ते॒ चक्षु॑र्गच्छतु॒ वात॑मा॒त्मा द्यां च॒ गच्छ॑ पृथि॒वीं च॒ धर्म॑णा। अ॒पो वा॑ गच्छ॒ यदि॒ तत्र॑ ते हि॒तमोष॑धीषु॒ प्रति॑तिष्ठा॒ शरी॑रैः। अ॒जो भा॒गस्तप॑सा॒ तं त॑पस्व॒ तं ते॑ शो॒चिस्त॑पतु॒ तं ते॑ अ॒र्चिः। यास्ते॑ शि॒वास्त॒नुवो॑ जातवेद॒स्ताभि॑र्वहे॒म सु॒कृतां॒ यत्र॑ लो॒काः। अ॒यव्वैं त्वम॒स्मादधि॒ त्वमे॒तद॒यव्वैं तद॑स्य॒ योनि॑रसि। वै॒श्वा॒न॒रः पु॒त्रः पि॒त्रे लो॑क॒कृज्जा॑तवेदो॒ वहे॑म सु॒कृतां॒ यत्र॑ लो॒काः॥४॥
\anuvakamend

४.२.०
य ए॒तस्य॒ त्वत्पञ्च॑॥ २॥
४.२.१
य ए॒तस्य॑ प॒थो गो॒प्तार॒स्तेभ्यः॒ स्वाहा॒ य ए॒तस्य॑ प॒थो र॑क्षि॒तार॒स्तेभ्यः॒ स्वाहा॒ य ए॒तस्य॑ प॒थो॑भिऽर॑क्षि॒तार॒स्तेभ्यः॒ स्वाहाऽऽख्या॒त्रे स्वाहा॑ऽपाख्या॒त्रे स्वाहा॑ऽभि॒लाल॑पते॒ स्वाहा॑ऽप॒लाल॑पते॒ स्वाहा॒ऽग्नये॑ कर्म॒कृते॒ स्वाहा॒ यमत्र॒ नाधी॒मस्तस्मै॒ स्वाहा। यस्त॑ इ॒ध्मं ज॒भर॑त्सिष्विदा॒नो मू॒र्धानं॑ वात॒ तप॑ते त्वा॒या। दिवो॒ विश्व॑स्मात्सीमघाय॒त उ॑रुष्यः। अ॒स्मात्त्वमधि॑ जा॒तो॑ऽसि॒ त्वद॒यं जा॑यतां॒ पुन॑। अ॒ग्नये॑ वैश्वान॒राय॑ सुव॒र्गाय॑ लो॒काय॒ स्वाहा॥५॥
\anuvakamend

४.३.०
धे॒ह्युत्त॑रेमा॒ष्टौ च॑॥ ३।
४.३.१
प्र के॒तुना॑ बृह॒ता भात्य॒ग्निरा॒विर्विश्वा॑नि वृष॒भो रो॑रवीति। दि॒वश्चि॒दन्ता॒दुप॒ मामु॒दान॑ड॒पामु॒पस्थे॑ महि॒षो व॑वर्ध। इ॒दं त॒ एकं॑ प॒र ऊ॑त॒ एकं॑ तृ॒तीये॑न॒ ज्योति॑षा॒ संवि॑शस्व। सं॒वेश॑नस्त॒नुवै॒ चारु॑रेधि प्रि॒यो दे॒वानां पर॒मे स॒धस्थे। नाके॑ सुप॒र्णमुप॒ यत्पत॑न्त हृ॒दा वेन॑न्तो अ॒भ्यच॑क्षत त्वा। हिर॑ण्यपक्षं॒ वरु॑णस्य दू॒तं य॒मस्य॒ योनौ॑ शकु॒नं भु॑र॒ण्युम्। अति॑द्रव सारमे॒यौ श्वानौ॑ चतुर॒क्षौ श॒बलौ॑ सा॒धुना॑ प॒था। अथा॑ पि॒तॄन्त्सु॑वि॒दत्रा॒ अपी॑हि य॒मेन॒ ये स॑ध॒मादं॒ मद॑न्ति। यौ ते॒ श्वानौ॑ यमरक्षि॒तारौ॑ चतुर॒क्षौ प॑थि॒रक्षी॑ नृ॒चक्ष॑सा। ताभ्या राज॒न्परि॑ देह्येन स्व॒स्ति चास्मा अनमी॒वं च॑ धेहि॥६॥
४.३.२
उ॒रु॒ण॒साव॑सु॒तृपा॑वुलुम्ब॒लौ य॒मस्य॑ दू॒तौ च॑रतो॒ वशा॒ अनु॑। ताव॒स्मभ्यं॑ दृ॒शये॒ सूर्या॑य॒ पुन॑र्दत्ता॒ वसु॑म॒द्येह भ॒द्रम्। सोम॒ एकेभ्यः पवते घृ॒तमेक॒ उपा॑सते। येभ्यो॒ मधु॑ प्र॒धाव॑ति॒ ताश्चि॑दे॒वापि॑ गच्छतात्। ये युध्य॑न्ते प्र॒धने॑षु॒ शूरा॑सो॒ ये त॑नु॒त्यज॑। ये वा॑ स॒हस्र॑दक्षिणा॒स्ता श्चि॑दे॒वापि॑ गच्छतात्। तप॑सा॒ ये अ॑नाधृ॒ष्यास्तप॑सा॒ ये सुव॑र्ग॒ताः। तपो॒ ये च॑क्रि॒रे म॒हत्ताश्चि॑दे॒वापि॑ गच्छतात्। अश्म॑न्वती रेवतीः॒ स र॑भध्व॒ मुत्ति॑ष्ठत प्रत॑रता सखायः। अत्रा॑ जहाम॒ ये अस॒न्नशे॑वाः  शि॒वान् व॒यम॒भि वाजा॒नुत्त॑रेम॥७॥
४.३.३
यद्वै दे॒वस्य॑ सवि॒तुः प॒वित्र स॒हस्र॑धारं॒  वित॑तम॒न्तरि॑क्षे। येनापु॑ना॒दिन्द्र॒मना॑र्त॒मार्त्यै॒ तेना॒हं मा स॒र्वत॑नुं पुनामि। या रा॒ष्ट्रात्प॒न्नादप॒ यन्ति॒ शाखा॑ अ॒भिमृ॑ता नृ॒पति॑मि॒च्छमा॑नाः। धा॒तुस्ताः सर्वाः॒ पव॑नेन पू॒ताः प्र॒जया॒स्मान्र॒य्या वर्च॑सा॒ ससृ॑जाथ। उद्व॒यं तम॑स॒स्परि॒ पश्य॑न्तो॒ ज्योति॒रुत्त॑रम्। दे॒वं दे॑व॒त्रा सूर्य॒मग॑न्म॒ ज्योति॑रुत्त॒मम्। धा॒ता पु॑नातु सवि॒ता पु॑नातु। अ॒ग्नेस्तेज॑सा॒ सूर्य॑स्य॒ वर्च॑सा॥८॥
\anuvakamend

४.४.०
अव॑शीयता स॒धस्थे॒ पञ्च॑ च॥ ४।
४.४.१
यन्ते॑ अ॒ग्निमम॑न्थाम वृष॒भाये॑व॒ पक्त॑वे। इ॒मन्त श॑मयामसि क्षी॒रेण॑ चोद॒केन॑ च। यन्त्वम॑ग्ने स॒मद॑ह॒स्त्वमु॒ निर्वा॑पया॒ पुन॑। क्या॒म्बूरत्र॑ जायतां पाकदू॒र्वा व्य॑ल्कशा। शीति॑के॒ शीति॑कावति॒ ह्लादु॑के॒ ह्लादु॑कावति। म॒ण्डू॒क्या॑ सुसङ्ग॒मये॒म स्व॑ग्नि श॒मय॑। शं ते॑ धन्व॒न्या आपः॒ शमु॑ ते सन्त्वनू॒क्या। शं ते॑ समु॒द्रिया॒ आपः॒ शमु॑ ते सन्तु॒ वर्ष्या। शं ते॒ स्रव॑न्तीस्त॒नुवे॒ शमु॑ ते सन्तु॒ कूप्या। शन्ते॑ नीहा॒रो व॑र्\mbox{}षतु॒ शमु॒ पृष्वाऽव॑शीयताम्॥९॥
४.४.२
अव॑ सृज॒ पुन॑रग्ने पि॒तृभ्यो॒ यस्त॒ आहु॑त॒श्चर॑ति स्व॒धाभि॑। आयु॒र्वसा॑न॒ उप॑ यातु॒ शेष॒ सङ्ग॑च्छतां त॒नुवा॑ जातवेदः। सङ्ग॑च्छस्व पि॒तृभि॒ स स्व॒धाभिः॒ समि॑ष्टापू॒र्तेन॑ पर॒मे व्यो॑मन्। यत्र॒ भूम्यै॑ वृ॒णसे॒ तत्र॑ गच्छ॒ तत्र॑ त्वा दे॒वः स॑वि॒ता द॑धातु। यत्ते॑ कृ॒ष्णः  श॑कु॒न आ॑तु॒तोद॑ पिपी॒लः स॒र्प उ॒त वा॒ श्वाप॑दः। अ॒ग्निष्टद्विश्वा॑दनृ॒णं कृ॑णोतु॒ सोम॑श्च॒ यो ब्राह्म॒णमा॑वि॒वेश॑। उत्ति॒ष्ठात॑स्त॒नुव॒ सम्भ॑रस्व॒ मेह गात्र॒मव॑हा॒ मा शरी॑रम्। यत्र॒ भूम्यै॑ वृ॒णसे॒ तत्र॑ गच्छ॒ तत्र॑ त्वा दे॒वः स॑वि॒ता द॑धातु। इ॒दं त॒ एकं॑ प॒र ऊ॑त॒ एकं॑ तृ॒तीये॑न॒ ज्योति॑षा॒ संवि॑शस्व। सं॒वेश॑नस्त॒नुवै॒ चारु॑रेधि प्रि॒यो दे॒वानां पर॒मे स॒धस्थे। उत्ति॑ष्ठ॒ प्रेहि॒ प्रद्र॒वौक॑ कृणुष्व पर॒मे व्यो॑मन्। य॒मेन॒ त्वं य॒म्या॑ संविदा॒नोत्त॒मन्नाक॒मधि॑ रोहे॒मम्। अश्म॑न्वती रेवती॒र्यद्वै दे॒वस्य॑ सवि॒तुः प॒वित्रं॒ या रा॒ष्ट्रात्प॒न्नादुद्व॒यं तम॑स॒स्परि॑ धा॒ता पु॑नातु। अ॒स्मात्त्वमधि॑ जा॒तो॑ऽसि॒ त्वद॒यं जा॑यतां॒ पुन॑। अ॒ग्नये॑ वैश्वान॒राय॑ सुव॒र्गाय॑ लो॒काय॒ स्वाहा॥१०॥
\anuvakamend

४.५.०
प॒थि॒कृद्भ्यो॑ विजान॒तेऽनु॑ वेनति॥ ५॥
४.५.१
आया॑तु दे॒वः सु॒मना॑भिरू॒तिभि॑र्य॒मो ह॑वे॒ह प्रय॑ताभिर॒क्ता। आसी॑दता सुप्र॒यते॑ह ब॒र्\mbox{}हिष्यूर्जा॑यजात्यै मम॑ शत्रु॒हत्यै। य॒मे इ॑व॒ यत॑माने॒ यदैतं॒ प्रवाम्भर॒न्मानु॑षा देव॒यन्त॑। आसी॑दत॒ स्वमु॑ लो॒कं  विदा॑ने स्वास॒स्थे भ॑वत॒मिन्द॑वे नः। य॒माय॒ सोम सुनुत य॒माय॑ जुहुता ह॒विः। य॒म ह॑ य॒ज्ञो ग॑च्छत्य॒ग्निदू॑तो॒ अर॑ङ्कृतः। य॒माय॑ घृ॒तव॑द्ध॒विर्जु॒होत॒ प्र च॑ तिष्ठत। स नो॑ दे॒वेष्वाय॑मद्दी॒र्घमायुः॒ प्र जी॒वसे। य॒माय॒ मधु॑मत्तम॒ राज्ञे॑ ह॒व्यं जु॑होतन। इ॒दं नम॒ ऋषि॑भ्यः पूर्व॒जेभ्यः॒ पूर्वेभ्यः पथि॒कृद्भ्य॑॥११॥
४.५.२
योऽस्य॒ कौष्ठ्य॒ जग॑तः॒ पार्थि॑व॒स्यैक॑ इद्व॒शी। य॒मं भ॑ङ्ग्यश्र॒वो गा॑य॒ यो राजा॑नप॒रोध्य॑। य॒मङ्गाय॑ भङ्ग्य॒श्रवो॒ यो राजा॑नप॒रोध्य॑। येना॒पो न॒द्यो॑ धन्वा॑नि॒ येन॒ द्यौः पृ॑थि॒वी दृ॒ढा। हि॒र॒ण्य॒क॒क्ष्यान् सु॒धुरान्॑ हिरण्या॒क्षान॑यः  श॒फान्। अश्वा॑न॒नश्य॑तो दा॒नं॒ य॒मो रा॑जाभि॒ तिष्ठ॑ति। य॒मो दा॑धार पृथि॒वीं य॒मो विश्व॑मि॒दं जग॑त्। य॒माय॒ सर्व॒मित्र॑स्थे॒ यत् प्रा॒णद्वा॒युर॑क्षि॒तम्। यथा॒ पञ्च॒ यथा॒ षड्य॒था पञ्च॑ द॒शर्\mbox{}ष॑यः। य॒मं यो वि॑द्या॒त्स ब्रू॑याद्य॒थैक ऋषि॑र्विजान॒ते॥१२॥
४.५.३
त्रिक॑द्रुकेभिः॒ पत॑ति॒ षडु॒र्वीरेक॒मिद्बृ॒हत्। गा॒य॒त्री त्रि॑ष्टुप्छन्दासि॒ सर्वा॒ ता य॒म आहि॑ता। अह॑रह॒र्नय॑मानो॒ गामश्वं॒ पुरु॑षं॒ जग॑त्। वैव॑स्वतो॒ न तृ॑प्यति॒ पञ्च॑भि॒र्मान॑वैर्य॒मः। वैव॑स्वते॒ विवि॑च्यन्ते॒ यमे॒ राज॑नि ते ज॒नाः। ये चे॒ह स॒त्येनेच्छ॑न्ते॒ य उ॒ चानृ॑तवादि॒नः। ते रा॑जन्नि॒ह विवि॑च्यन्ते॒ऽथा य॑न्ति त्वा॒मुप॑। दे॒वाश्च॒ ये न॑म॒स्यन्ति॒ ब्राह्म॑णाश्चाप॒चित्य॑ति। यस्मि॑न्वृ॒क्षे सु॑पला॒शे दे॒वैः सं॒पिब॑ते य॒मः। अत्रा॑ नो वि॒श्पति॑ पि॒ता पु॑रा॒णा अनु॑वेनति॥१३॥
\anuvakamend

४.६.०
अ॒घ्नि॒या अ॑गन्म स॒प्त च॑॥ ६॥
४.६.१
वै॒श्वा॒न॒रे ह॒विरि॒दं जु॑होमि साह॒स्रमुत्स श॒तधा॑रमे॒तम्। तस्मि॑न्ने॒ष पि॒तरं॑ पिताम॒हं प्रपि॑तामहं बिभर॒त्पिन्व॑माने। द्र॒प्सश्च॑स्कन्द पृथि॒वीमनु॒ द्यामि॒मं च॒ योनि॒मनु॒ यश्च॒ पूर्व॑। तृ॒तीय॒य्योंनि॒मनु॑ स॒ञ्चर॑न्तं द्र॒प्सं जु॑हो॒म्यनु॑ स॒प्त होत्रा। इ॒म स॑मु॒द्र श॒तधा॑र॒मुत्स॑व्व्यँ॒च्यमा॑नं॒ भुव॑नस्य॒ मध्ये। घृ॒तन्दुहा॑ना॒मदि॑तिं॒ जना॒याग्ने॒ मा हिसीः पर॒मे व्यो॑मन्। अपे॑त॒ वीत॒ वि च॑ सर्प॒तातो॒ येऽत्र॒ स्थ पु॑रा॒णा ये च॒ नूत॑नाः। अहो॑भिर॒द्भिर॒क्तुभि॒ र्व्य॑क्तं य॒मो द॑दात्वव॒सान॑मस्मै। स॒वि॒तैतानि॒ शरी॑राणि पृथि॒व्यै मा॒तुरु॒पस्थ॒ आद॑धे। तेभि॑र्युज्यन्तामघ्नि॒याः॥१४॥
४.६.२
शु॒नं॑ वा॒हाः  शु॒नं ना॒राः  शु॒नं कृ॑षतु॒ लाङ्ग॑लम्। शु॒नं व॑र॒त्रा ब॑ध्यन्ता शु॒नमष्ट्रा॒मुदि॑ङ्गय॒ शुना॑सीरा शु॒नम॒स्मासु॑ धत्तम्। शुना॑सीरा वि॒मां वाचं॒ यद्दि॒वि च॑क्र॒थुः पय॑। तेने॒मामुप॑ सिञ्चतम्। सीते॒ वन्दा॑महे त्वा॒ऽर्वाची॑ सुभगे भव। यथा॑ नः सु॒भगा स॑सि॒ यथा॑ नः सु॒फला स॑सि। स॒वि॒तैतानि॒ शरी॑राणि पृथि॒व्यै मा॒तुरु॒पस्थ॒ आद॑धे। तेभि॑रदिते॒ शं भव। विमु॑च्यध्वमघ्नि॒या दे॑व॒याना॒ अता॑रिष्म॒ तम॑सस्पा॒रम॒स्य। ज्योति॑रापाम॒ सुव॑रगन्म॥१५॥
४.६.३
प्र वाता॒ वान्ति॑ प॒तय॑न्ति वि॒द्युत॒ उदोष॑धीर्जिहते॒ पिन्व॑ते॒ सुव॑। इरा॒ विश्व॑स्मै॒ भुव॑नाय जायते॒ यत्प॒र्जन्य॑ पृथि॒वी रेत॒साऽव॑ति। यथा॑ य॒माय॑ हा॒र्म्यमव॑प॒न्पञ्च॑ मान॒वाः। ए॒वं व॑पामि हा॒र्म्यं यथासा॑म जीवलो॒के भूर॑यः। चित॑ स्थ परि॒चित॑ ऊर्ध्व॒चितः  श्रयध्वं पि॒तरो॑ दे॒वता। प्र॒जाप॑तिर्वः सादयतु॒ तया॑ दे॒वत॑या। आप्या॑यस्व॒ सन्ते॥१६॥
\anuvakamend

४.७.०
अन॑पस्फुरन्ती॒रुत्त॑र दे॒वत॑या॒ द्वे च॑॥ ७।
४.७.१
उत्ते॑ तभ्नोमि पृथि॒वीं त्वत्परी॒मँल्लो॒कं नि॒दध॒न्मो अ॒ह रि॑षम्। ए॒ता स्थूणां पि॒तरो॑ धारयन्तु॒ तेऽत्रा॑ य॒मः साद॑नात्ते मिनोतु। उप॑सर्प मा॒तरं॒ भूमि॑मे॒तामु॑रु॒व्यच॑सं पृथि॒वी सु॒शेवाम्। ऊर्ण॑म्रदा युव॒तिर्दक्षि॑णावत्ये॒षा त्वा॑ पातु॒ निर्\mbox{}ऋ॑त्या उ॒पस्थे। उछ्म॑ञ्चस्व पृथिवि॒ मा विबा॑धिथाः सूपाय॒नास्मै॑ भव सूपवञ्च॒ना। मा॒ता पु॒त्रं यथा॑सि॒चाभ्ये॑नं भूमि वृणु। उ॒छ्मञ्च॑माना पृथि॒वी हि तिष्ठ॑सि स॒हस्रं॒ मित॒ उप॒ हि श्रय॑न्ताम्। ते गृ॒हासो॑ मधु॒श्चुतो॒ विश्वाहास्मै शर॒णाः स॒न्त्वत्र॑। एणीर्धा॒ना हरि॑णी॒रर्जु॑नीः सन्तु धे॒नव॑। तिल॑वत्सा॒ ऊर्ज॑मस्मै॒ दुहा॑ना॒ विश्वाहा॑ स॒न्त्वनप॑स्फुरन्तीः॥१७॥
४.७.२
ए॒षा ते॑ यम॒साद॑ने स्व॒धा निधी॑यते गृ॒हे। अक्षि॑ति॒र्नाम॑ ते असौ। इ॒दं पि॒तृभ्यः॒ प्रभ॑रेम ब॒र्\mbox{}हिर्दे॒वेभ्यो॒ जीव॑न्त॒ उत्त॑रं भरेम। तत्त्व॑मारो॒हासो॒ मेघ्यो॒ भवं॑ य॒मेन॒ त्वं य॒म्या॑ सव्विंदा॒नः। मा त्वा॑ वृ॒क्षौ सम्बा॑धिष्टा॒म्मा मा॒ता पृ॑थिवि॒ त्वम्। पि॒तॄन् ह्यत्र॒ गच्छा॒स्येधा॑सं यम॒राज्ये। मा त्वा॑ वृ॒क्षौ सम्बा॑धेथां॒ मा मा॒ता पृ॑थि॒वी म॒ही। वै॒व॒स्व॒त हि गच्छा॑सि यम॒राज्ये॒ विरा॑जसि। न॒ळं प्ल॒वमारो॑है॒तं न॒ळेन॑ प॒थोऽन्वि॑हि। स त्वं॑ न॒ळप्ल॑वो भू॒त्वा॒ सन्त॑र॒ प्रत॒रोत्त॑र॥१८॥
४.७.३
स॒वि॒तैतानि॒ शरी॑राणि पृथि॒व्यै मा॒तुरु॒पस्थ॒ आद॑धे। तेभ्य॑ पृथिवि॒ शं भ॑व। षड्ढो॑ता॒ सूर्यं॑ ते॒ चक्षु॑र्गच्छतु॒ वात॑मा॒त्मा द्यां च॒ गच्छ॑ पृथि॒वीं च॒ धर्म॑णा। अ॒पो वा॑ गच्छ॒ यदि॒ तत्र॑ ते हि॒तमोष॑धीषु॒ प्रति॑तिष्ठा॒ शरी॑रैः। परं॑ मृत्यो॒ अनु॒परे॑हि॒ पन्थां॒ यस्ते॒ स्व इत॑रो देव॒यानात्। चक्षु॑ष्मते शृण्व॒ते ते ब्रवीमि॒ मा न॑ प्र॒जा री॑रिषो॒ मोत वी॒रान्। शं वातः॒ श हि ते॒ घृणिः॒ शमु॑ ते स॒न्त्वोष॑धीः। कल्प॑न्ताम्मे॒ दिश॑ श॒ग्माः। पृ॒थि॒व्यास्त्वा॑ लो॒के सा॑दयाम्य॒मुष्य॒ शर्मा॑सि पि॒तरो॑ दे॒वता। प्र॒जाप॑तिस्त्वा सादयतु॒ तया॑ दे॒वत॑या। अ॒न्तरि॑क्षस्य त्वा दि॒वस्त्वा॑ दि॒शां त्वा॒ नाक॑स्य त्वा पृ॒ष्ठे ब्र॒ध्नस्य॑ त्वा वि॒ष्टपे॑ सादयाम्य॒मुष्य॒ शर्मा॑सि पि॒तरो॑ दे॒वता। प्र॒जाप॑तिस्त्वा सादयतु॒ तया॑ दे॒वत॑या॥१९॥
४.८.१
अ॒पू॒पवान्घृ॒तवाश्च॒रुरेह सी॑दतूत्तभ्नु॒वन्पृ॑थि॒वीन्द्यामु॒तोपरि॑। यो॒नि॒कृत॑ पथि॒कृत॑ सपर्यत॒ ये दे॒वानाङ्घृ॒तभा॑गा इ॒ह स्थ। ए॒षा ते यम॒साद॑ने स्व॒धा निधी॑यते गृ॒हे॑ऽसौ। दशाक्षरा॒ ता र॑क्षस्व॒ ताङ्गो॑पायस्व॒ तां ते॒ परि॑ददामि॒ तस्यां त्वा॒ मा द॑भन्पि॒तरो॑ दे॒वता। प्र॒जाप॑तिस्त्वा सादयतु॒ तया॑ दे॒वत॑या। अ॒पू॒पवाञ्छृ॒तवान्क्षी॒रवा॒न्दधि॑वा॒न्मधु॑माश्च॒रुरेह सी॑दतूत्तभ्नु॒वन्पृ॑थि॒वीन्द्यामु॒तोपरि॑। यो॒नि॒कृत॑ पथि॒कृत॑ सपर्यत॒ ये दे॒वाना शृ॒तभा॑गाः क्षी॒रभा॑गा॒ दधि॑भागा॒ मधु॑भागा इ॒ह स्थ। ए॒षा ते॑ यम॒साद॑ने स्व॒धा निधी॑यते गृ॒हे॑ऽसौ। श॒ताक्ष॑रा स॒हस्राक्षरा॒युताक्ष॒राऽच्यु॑ताक्षरा॒ ता र॑क्षस्व॒ ताङ्गो॑पायस्व॒ तां ते॒ परि॑ददामि॒ तस्यां त्वा॒ मा द॑भन्पि॒तरो॑ दे॒वता। प्र॒जाप॑तिस्त्वा सादयतु॒ तया॑ दे॒वत॑या॥२०॥ ८॥
\anuvakamend

४.९.०
फलं॑ पुनातु॥ ९।
४.९.१
ए॒तास्ते स्व॒धा अ॒मृता करोमि॒ यास्ते॑ धा॒नाः प॑रि॒किरा॒म्यत्र॑। तास्ते॑ य॒मः पि॒तृभि॑ संविदा॒नोऽत्र॑ धे॒नूः का॑म॒दुघा करोतु। त्वामर्जु॒नौष॑धीनां॒ पयो ब्र॒ह्माण॒ इद्वि॑दुः। तासां त्वा॒ मध्या॒दाद॑दे च॒रुभ्यो॒ अपि॑धातवे। दू॒र्वाणा स्त॒म्बमाह॑रै॒तां प्रि॒यत॑मां॒ मम॑। इ॒मान्दिशं॑ मनु॒ष्या॑णां॒ भूयि॒ष्ठानु॒ वि रो॑हतु। काशा॑ना स्त॒म्बमाह॑र॒ रक्ष॑सा॒मप॑हत्यै। य ए॒तस्यै॑ दि॒शः प॒राभ॑वन्नघा॒यवो॒ यथा॒ तेनाभ॑वा॒न्पुन॑। द॒र्भाणा स्त॒म्बमाह॑र पितृ॒णा मोष॑धीं प्रि॒याम्। अन्वस्यै॒ मूलं॑ जीवा॒दनु॒ काण्ड॒मथो॒ फलम्॥२१॥
४.९.२
लो॒कं पृ॑ण॒ ता अ॑स्य॒ सूद॑दोहसः। शं वातः॒ शं हि ते॒ घृणिः॒ शमु॑ ते स॒न्त्वोष॑धीः। कल्पन्तां ते॒ दिशः॒ सर्वा। इ॒दमे॒व मेतोप॑रा॒ मार्ति॑माराम॒ काञ्च॒न। तथा॒ तद॒श्विभ्यां कृ॒तं मि॒त्रेण॒ वरु॑णेन च। व॒र॒णो वा॑रयादि॒दन्दे॒वो वन॒स्पति॑। आर्त्यै॒ निर्\mbox{}ऋ॑त्यै॒ द्वेषाच्च॒ वन॒स्पति॑। विधृ॑तिरसि॒ विधा॑रया॒स्मद॒घा द्वेषासि श॒मि श॒मया॒स्मद॒घा द्वेषासि य॒व य॒वया॒स्मद॒घा द्वेषासि। पृ॒थि॒वीं ग॑च्छा॒न्तरि॑क्षं गच्छ॒ दिवं॑ गच्छ॒ दिशो॑ गच्छ॒ सुव॑र्गच्छ॒ सुव॑र्गच्छ॒ दिशो॑ गच्छ॒ दिवं॑ गच्छा॒न्तरि॑क्षं गच्छ पृथि॒वीं ग॑च्छा॒पो वा॑ गच्छ॒ यदि॒ तत्र॑ ते हि॒तमोष॑धीषु॒ प्रति॑तिष्ठा॒ शरी॑रैः। अश्म॑न्वती रेवती॒र्यद्वै दे॒वस्य॑ सवि॒तुः प॒वित्रं॒ या रा॒ष्ट्रात्प॒न्नादुद्व॒यं तम॑स॒स्परि॑ धा॒ता पु॑नातु॥२२॥
\anuvakamend

४.१०.०
भ॒व॒ ज॒म्भ॒या॒म॒सि॒ त्रीणि॑ च। १०।
४.१०.१
आ रो॑ह॒तायु॑र्ज॒रस॑ङ्गृणा॒ना अ॑नुपू॒र्वं यत॑माना॒ यति॒ष्ट। इ॒ह त्वष्टा॑ सु॒जनि॑मा सु॒रत्नो॑ दी॒र्घमायु॑ करतु जी॒वसे॑ वः। यथाहान्यनुपूर्वं भव॑न्ति॒ यथ॒र्तव॑ ऋ॒तुभि॒र्यन्ति॑ कॢ॒प्ताः। यथा॒ न पूर्व॒मप॑रो॒ जहात्ये॒वा धा॑त॒रायूषि कल्पयैषाम्। न हि॑ ते अग्ने त॒नुवै क्रू॒रं च॒कार॒ मर्त्य॑। क॒पिर्ब॑भस्ति॒ तेज॑नं॒ पुन॑र्ज॒रायु॒ गौरि॑व। अप॑ नः॒ शोशु॑चद॒घमग्ने॑ शुशु॒ध्या र॒यिम्। अप॑ नः॒ शोशु॑चद॒घं मृ॒त्यवे॒ स्वाहा। अ॒न॒ड्वाह॑म॒न्वार॑भामहे स्व॒स्तये। स न॒ इन्द्र॑ इव दे॒वेभ्यो॒ वह्नि॑ स॒म्पार॑णो भव॥२३॥
४.१०.२
इ॒मे जी॒वा वि॑मृ॒तैराव॑वर्ति॒न्नभूद्भ॒द्रा दे॒वहू॑तिं नो अ॒द्य। प्राञ्जो॑गामानृ॒तये॒ हसा॑य॒ द्राघी॑य॒ आयु॑ प्रत॒रान्दधा॑नाः। मृ॒त्योः प॒दं यो॒पय॑न्तो॒ यदैम॒ द्राघी॑य॒ आयु॑ प्रत॒रान्दधा॑नाः। आ॒प्याय॑मानाः प्र॒जया॒ धने॑न शु॒द्धाः पू॒ता भ॑वथ यज्ञियासः। इ॒मं जी॒वेभ्य॑ परि॒धिन्द॑धामि॒ मा नोऽनु॑ गा॒दप॑रो॒ अर्ध॑मे॒तम्। श॒तञ्जी॑वन्तु श॒रद॑ पुरू॒चीस्ति॒रो मृ॒त्युन्द॑द्महे॒ पर्व॑तेन। इ॒मा नारी॑रविध॒वाः सु॒पत्नी॒राञ्ज॑नेन स॒र्पिषा॒ संमृ॑शन्ताम्। अ॒न॒श्रवो॑ अनमी॒वाः सु॒शेवा॒ आरो॑हन्तु॒ जन॑यो॒ योनि॒मग्रे। यदाञ्ज॑नन्त्रैककु॒दञ्जा॒त हि॒मव॑त॒स्परि॑। तेना॒मृत॑स्य॒ मूले॒नारा॑तीर्जम्भयामसि। यथा॒ त्वमु॑द्भि॒नत्स्यो॑षधे पृथि॒व्या अधि॑। ए॒वमि॒म उद्भि॑न्दन्तु की॒र्त्या यश॑सा ब्रह्मवर्च॒सेन॑। अ॒जोऽस्यजा॒स्मद॒घा द्वेषासि य॒वो॑ऽसि य॒वया॒स्मद॒घा द्वेषासि॥२४॥
\anuvakamend

४.११.०
अ॒घम॒घं च॒त्वारि॑ च॥ ११।
४.११.१
अप॑ नः॒ शोशु॑चद॒घमग्ने॑ शुशु॒ध्या र॒यिम्। अप॑ नः॒ शोशु॑चद॒घम्। सु॒क्षे॒त्रि॒या सु॑गातु॒या व॑सू॒या च॑ यजामहे। अप॑ नः॒ शोशु॑चद॒घम्। प्रयद्भन्दि॑ष्ठ एषां॒ प्रास्माका॑सश्च सू॒रय॑। अप॑ नः॒ शोशु॑चद॒घम्। प्रयद॒ग्नेः सह॑स्वतो वि॒श्वतो॒ यन्ति॑ सू॒रय॑। अप॒ नः॒ शोशु॑चद॒घम्। प्रयत्ते॑ अग्ने सू॒रयो॒ जाये॑महि॒ प्र ते॑ व॒यम्। अप॑ नः॒ शोशु॑चद॒घम्॥२५॥
४.११.२
त्व हि वि॑श्वतोमुख वि॒श्वत॑ परि॒भूरसि॑। अप॑ नः॒ शोशु॑चद॒घम्। द्विषो॑ नो विश्वतोमु॒खाऽति॑ ना॒वेव॑ पारय। अप॑ नः॒ शोशु॑चद॒घम्। स नः॒ सिन्धु॑मिव ना॒वयाति॑ पर्\mbox{}षा स्व॒स्तये। अप॑ नः॒ शोशु॑चद॒घम्। आप॑ प्रव॒णादि॑व य॒तीरपा॒स्मत्स्य॑न्दताम॒घम्। अप॑ नः॒ शोशु॑चद॒घम्। उ॒द्व॒नादु॑द॒कानी॒वापा॒स्मत्स्य॑न्दताम॒घम्। अप॑ नः॒ शोशु॑चद॒घम्। आ॒न॒न्दाय॑ प्रमो॒दाय॒ पुन॒रागा॒ स्वान्गृ॒हान्। अप॑ नः॒ शोशु॑चद॒घम्। न वै तत्र॒ प्रमी॑यते॒ गौरश्वः॒ पुरु॑षः प॒शुः। यत्रे॒दं ब्रह्म॑ क्रि॒यते॑ परि॒धिर्जीव॑नाय॒कमप॑ नः॒ शोशु॑चद॒घम्॥२६॥
\anuvakamend

४.१२.०
व॒धि॒ष्ट॒ द्वे च॑॥ १२।
४.१२.१
अप॑श्याम युव॒तिमा॒चर॑न्तीं मृ॒ताय॑ जी॒वां प॑रिणी॒यमा॑नाम्। अ॒न्धेन॒ या तम॑सा॒ प्रावृ॑ताऽसि॒ प्राची॒मवा॑ची॒मव॒यन्नरि॑ष्ट्यै। मयै॒तां मा॒स्तां भ्रि॒यमा॑णा दे॒वी स॒ती पि॑तृलो॒कं यदैषि॑। वि॒श्ववा॑रा॒ नभ॑सा॒ संव्य॑यन्त्यु॒भौ नो॑ लो॒कौ पय॒साऽऽवृ॑णीहि। रयि॑ष्ठाम॒ग्निं मधु॑मन्तमू॒र्मिण॒मूर्ज॑ सन्तं त्वा॒ पय॒सोप॒ सस॑देम। स र॒य्या समु॒ वर्च॑सा॒ सच॑स्वा नः स्व॒स्तये। ये जी॒वा ये च॑ मृ॒ता ये जा॒ता ये च॒ जन्त्या। तेभ्यो॑ घ़ृ॒तस्य॑ धारयितुं॒ मधु॑धारा व्युन्द॒ती। मा॒ता रु॒द्राणान्दुहि॒ता वसू॑ना॒ स्वसा॑दि॒त्याना॑म॒मृत॑स्य॒ नाभि॑। प्रणु॒वोचं॑ चिकि॒तुषे॒ जना॑य॒ मागामना॑गा॒मदि॑तिं वधिष्ट। पिब॑तूद॒कं तृणान्यत्तु। ओमुत्सृ॒जत॥२७॥

सु॒म॒ङ्ग॒लीरि॒यं व॒धूरि॒मा स॑मेत॒ पश्य॑त । सौभाग्यम॒स्यै द॒त्त्वायाथास्तं॒ वि परे॑तन।इ॒मां त्वमि॑न्द्र मीढ्वः सुपु॒त्रा सु॒भगां कुरु।दशास्यां पु॒त्राना धे॑हि॒ पति॑मेकाद॒शं कृ॑धि॥ आ॒वह॑न्ती वितन्वा॒ना कु॒र्वा॒णा चीर॑मा॒त्मन॑। वासासि॒ मम॒ गाव॑श्च। अ॒न्न॒पा॒ने च॑ सर्व॒दा। ततो॑ मे॒ श्रिय॒माव॑ह।

सन्त्वा॑ सिञ्चामि॒ यजुषा॑ प्र॒जामायु॒र्धनं॑ च॥ ॐ शान्तिः  शान्तिः  शन्तिः॥

\closesection
\clearpage

\sect{सप्तमः प्रश्नः --- शीक्षावल्ली}\setcounter{anuvakam}{0}
५.०.०
॥ तैत्तिरीयीरण्यके पञ्चमप्रश्नः प्रारम्भः॥ तैत्तिरीयोपनिषत्॥
५.०.०
शं नः॒ शीक्षा स॒ह नौ॒ यश्छन्द॑सां॒ भूः स यः पृथिव्योमित्यृतं चा॒हं वेदमनूच्य॒ शं न॑ स॒ह ना॑ववतु॒ ब्रह्म॒विद्भृगुः॒ पञ्च॑दश॥ १५॥ शं नो॒ मह॒ इति॒ ये तत्र॑ भी॒षाऽस्मा॒दन्नं॑ ब॒हु कु॑र्वीत॒ द्विच॑त्वारिशत्॥ ४२॥ शं न॒ इत्यु॑प॒निष॑त्॥ ॐ शान्तिः॒ शान्तिः॒ शान्ति॑॥ हरि॑ ओम्। श्रीकृष्णार्पणमस्तु॥ तैत्तिरीयारण्यके पञ्चमप्रश्नः समाप्तः।
५.१.०
स॒त्यं व॑दिष्यामि॒ पञ्च॑ च॥ १॥
५.१.१
शं नो॑ मि॒त्रः  शं वरु॑णः। शं नो॑ भवत्वर्य॒मा। शं न॒ इन्द्रो॒ बृह॒स्पति॑। शं नो॒ विष्णु॑रुरुक्र॒मः। नमो॒ ब्रह्म॑णे। नम॑स्ते वायो। त्वमे॒व प्र॒त्यक्षं॒ ब्रह्मा॑सि। त्वामे॒व प्र॒त्यक्षं॒ ब्रह्म॑ वदिष्यामि। ऋ॒तं व॑दिष्यामि। स॒त्यं व॑दिष्यामि। तन्माम॑वतु। तद्व॒क्तार॑मवतु। अव॑तु॒ माम्। अव॑तु व॒क्तारम्। ओम् शान्तिः॒ शान्तिः॒ शान्ति॑॥१॥
\anuvakamend

५.२.०
शीक्षां पञ्च॑॥ २॥
५.२.१
शीक्षाव्व्याँख्यास्या॒मः। वर्णः॒ स्वरः। मात्रा॒ बलम्। साम॑ सन्ता॒नः। इत्युक्तः  शीक्षाध्या॒यः॥२॥
\anuvakamend

५.३.०
स॒न्धिराचार्यः पूर्वरू॒पमित्यधि॒प्रजल्लोँ॑के॒न॥ ३॥
५.३.१
स॒ह नौ॒ यशः। स॒ह नौ ब्र॑ह्मव॒र्चसम्। अथातः सहिताया उपनिषदव्व्याँख्यास्या॒मः। पञ्चस्वधिक॑रणे॒षु। अधिलोकमधिज्यौतिषमधिविद्यमधिप्रज॑मध्या॒त्मम्। ता महासहिता इ॑त्याच॒क्षते। अथा॑धिलो॒कम्। पृथिवी पूर्वरू॒पम्। द्यौरुत्त॑ररू॒पम्। आका॑शः स॒न्धिः॥३॥
५.३.२
वायु॑ सन्धा॒नम्। इत्य॑धिलो॒कम्। अथा॑धिज्यौ॒तिषम्। अग्निः पूर्वरू॒पम्। आदित्य उत्त॑ररू॒पम्। आ॑पः स॒न्धिः। वैद्युत॑ सन्धा॒नम्। इत्य॑धिज्यौ॒तिषम्। अथा॑धिवि॒द्यम्। आचार्यः पूर्वरू॒पम्॥४॥
५.३.३
अन्तेवास्युत्त॑ररू॒पम्। वि॑द्या स॒न्धिः। प्रवचन सन्धा॒नम्। इत्य॑धिवि॒द्यम्। अथाधि॒प्रजम्। माता पूर्वरू॒पम्। पितोत्त॑ररू॒पम्। प्र॑जा स॒न्धिः। प्रजनन सन्धा॒नम्। इत्यधि॒प्रजम्॥५॥
५.३.४
अथाध्या॒त्मम्। अधराहनुः पूर्वरू॒पम्। उत्तराहनुरुत्त॑ररू॒पम्। वाख्स॒न्धिः। जिह्वा॑ सन्धा॒नम्। इत्यध्या॒त्मम्। इतीमा म॑हास॒हिताः। य एवमेता महासहिता व्याख्या॑ता वे॒द। सन्धीयते प्रज॑या प॒शुभिः। ब्रह्मवर्चसेनान्नाद्येन सुवर्ग्येण॑ लोके॒न॥६॥
\anuvakamend

५.४.०
वि॒त॒न्वा॒ना वि॑श॒ स्वाहा॑ स॒प्त च॑॥ ४॥
५.४.१
यश्छन्द॑सामृष॒भो वि॒श्वरू॑पः। छन्दो॒भ्योऽध्य॒मृतात्सम्ब॒भूव॑। स मेन्द्रो॑ मे॒धया स्पृणोतु। अ॒मृत॑स्य देव॒ धार॑णो भूयासम्। शरी॑रं मे॒ विच॑र्\mbox{}षणम्। जि॒ह्वा मे॒ मधु॑मत्तमा। कर्णाभ्यां॒ भूरि॒ विश्रु॑वम्। ब्रह्म॑णः को॒शो॑ऽसि मे॒धयापि॑हितः। श्रु॒तं मे॑ गोपाय। आ॒वह॑न्ती वितन्वा॒ना॥७॥
५.४.२
कु॒र्वा॒णा चीर॑मा॒त्मन॑। वासासि॒ मम॒ गाव॑श्च। अ॒न्न॒पा॒ने च॑ सर्व॒दा। ततो॑ मे॒ श्रिय॒माव॑ह। लो॒म॒शां प॒शुभि॑ स॒ह स्वाहा। आ मा॑ यन्तु ब्रह्मचा॒रिणः॒ स्वाहा। यशो॒ जने॑ऽसानि॒ स्वाहा। श्रेया॒\an{} वस्य॑सोऽसानि॒ स्वाहा। तं त्वा॑ भग॒ प्रवि॑शानि॒ स्वाहा। स मा॑ भग॒ प्रवि॑श॒ स्वाहा। तस्मिन्त्स॒हस्र॑शाखे। निभ॑गा॒हं त्वयि॑ मृजे॒ स्वाहा। यथापः॒ प्रव॑ता॒ यन्ति॑। यथा॒ मासा॑ अहर्ज॒रम्। ए॒वं मां ब्र॑ह्मचा॒रिण॑। धात॒राय॑न्तु स॒र्वतः॒ स्वाहा। प्र॒ति॒वे॒शो॑सि॒ प्र मा॑ भाहि॒ प्र मा॑ पद्यस्व॥८॥
\anuvakamend

५.५.०
अ॒सौ लो॒को यजूषि॒ वेद॒ द्वे च॑॥ ५॥
५.५.१
भूर्भुवः॒ सुव॒रिति॒ वा ए॒तास्ति॒स्रो व्याहृ॑तयः। तासा॑मुहस्मै॒ तां च॑तु॒र्थीम्। माहा॑चमस्यः॒ प्रवे॑दयते। मह॒ इति॑। तद्ब्रह्म॑। स आ॒त्मा। अङ्गान्य॒न्या दे॒वता। भूरिति॒ वा अ॒यं लो॒कः। भुव॒ इत्य॒न्तरि॑क्षम्। सुव॒रित्य॒सौ लो॒कः॥९॥
५.५.२
मह॒ इत्या॑दि॒त्यः। आ॒दि॒त्येन॒ वाव सर्वे॑ लो॒का मही॑यन्ते। भूरिति॒ वा अ॒ग्निः। भुव॒ इति॑ वा॒युः। सुव॒रित्या॑दि॒त्यः। मह॒ इति॑ च॒न्द्रमा। च॒न्द्रम॑सा॒ वाव सर्वा॑णि॒ ज्योतीषि॒ मही॑यन्ते। भूरिति॒ वा ऋच॑। भुव॒ इति॒ सामा॑नि। सुव॒रिति॒ यजूषि॥१०॥
५.५.३
मह॒ इति॒ ब्रह्म॑। ब्रह्म॑णा॒ वाव सर्वे॑ वे॒दा मही॑यन्ते। भूरिति॒ वै प्रा॒णः। भुव॒ इत्य॑पा॒नः। सुव॒रिति॑ व्या॒नः। मह॒ इत्यन्नम्। अन्ने॑न॒ वाव सर्वे प्रा॒णा मही॑यन्ते। ता वा ए॒ताश्च॑तस्रश्चतु॒र्धा। चत॑स्रश्चतस्रो॒ व्याहृ॑तयः। ता यो वेद॑। स वे॑द॒ ब्रह्म॑। सर्वेऽस्मै दे॒वा ब॒लिमाव॑हन्ति॥११॥
\anuvakamend

५.६.०
वा॒याव॒मृत॒मेकं॑ च॥ ६॥
५.६.१
स य ए॒षोऽन्तर्\mbox{}हृ॑दय आका॒शः। तस्मि॑न्न॒यं पुरु॑षो मनो॒मय॑। अमृ॑तो हिर॒ण्मय॑। अन्त॑रेण॒ तालु॑के। य ए॒ष स्तन॑ इवाव॒लम्ब॑ते। सेन्द्रयो॒निः। यत्रा॒सौ के॑शा॒न्तो वि॒वर्त॑ते। व्य॒पोह्य॑ शीर्\mbox{}षकपा॒ले। भूरित्य॒ग्नौ प्रति॑तिष्ठति। भुव॒ इति॑ वा॒यौ॥१२॥
५.६.२
सुव॒रित्या॑दि॒त्ये। मह॒ इति॒ ब्रह्म॑णि। आ॒प्नोति॒ स्वाराज्यम्। आ॒प्नोति॒ मन॑स॒स्पतिम्। वाक्प॑ति॒श्चक्षु॑ष्पतिः। श्रोत्र॑पतिर्वि॒ज्ञान॑पतिः। ए॒तत्ततो॑ भवति। आ॒का॒शश॑रीरं॒ ब्रह्म॑। स॒त्यात्म॑प्रा॒णारा॑मं॒ मन॑ आनन्दम्। शान्ति॑समृद्धम॒मृतम्। इति॑ प्राचीनयो॒ग्योपास्व॥१३॥
\anuvakamend

५.७.०
सर्व॒मेकं॑ च॥ ७॥
५.७.१
पृ॒थि॒व्य॑न्तरि॑क्ष॒न्द्यौर्दिशो॑ऽवान्तरदि॒शाः। अ॒ग्निर्वा॒युरा॑दि॒त्यश्च॒न्द्रमा॒ नक्ष॑त्राणि। आप॒ ओष॑धयो॒ वन॒स्पत॑य आका॒श आ॒त्मा। इत्य॑धिभू॒तम्। अथाध्या॒त्मम्। प्रा॒णो व्या॒नो॑ऽपा॒न उ॑दा॒नः स॑मा॒नः। चक्षुः॒ श्रोत्रं॒ मनो॒ वाक्त्वक्। चर्म॑ मा॒स स्नावास्थि॑ म॒ज्जा। ए॒तद॑धि वि॒धाय॒र्\mbox{}षि॒रवो॑चत्। पाङ्क्तं॒ वा इ॒द सर्वम्। पाङ्क्ते॑नै॒व पाङ्क्त स्पृणो॒तीति॑॥१४॥
\anuvakamend

५.८.०
ओन्दश॑॥ ८॥
५.८.१
ओमिति॒ ब्रह्म॑। ओमिती॒द सर्वम्। ओमित्ये॒तद॑नुकृति ह स्म॒ वा अ॒प्योश्रा॑व॒येत्याश्रा॑वयन्ति। ओमिति॒ सामा॑नि गायन्ति। ओशोमिति॑ श॒स्त्राणि॑ शसन्ति। ओमित्य॑ध्व॒र्युः प्र॑तिग॒रं प्रति॑गृणाति। ओमिति॒ ब्रह्मा॒ प्रसौ॑ति। ओमित्य॑ग्निहो॒त्रमनु॑जानाति। ओमिति॑ ब्राह्म॒णः प्र॑व॒क्ष्यन्ना॑ह॒ ब्रह्मोपाप्नवा॒नीति॑। ब्रह्मै॒वोपाप्नोति॥१५॥
\anuvakamend

५.९.०
प्रजा च स्वाध्यायप्रव॑चने॒ च षट्च॑॥ ९॥
५.९.१
ऋतं च स्वाध्यायप्रव॑चने॒ च। सत्यं च स्वाध्यायप्रव॑चने॒ च। तपश्च स्वाध्यायप्रव॑चने॒ च। दमश्च स्वाध्यायप्रव॑चने॒ च। शमश्च स्वाध्यायप्रव॑चने॒ च। अग्नयश्च स्वाध्यायप्रव॑चने॒ च। अग्निहोत्रं च स्वाध्यायप्रव॑चने॒ च। अतिथयश्च स्वाध्यायप्रव॑चने॒ च। मानुषं च स्वाध्यायप्रव॑चने॒ च। प्रजा च स्वाध्यायप्रव॑चने॒ च। प्रजनश्च स्वाध्यायप्रव॑चने॒ च। प्रजातिश्च स्वाध्यायप्रव॑चने॒ च। सत्यमिति सत्यवचा॑ राथी॒तरः। तप इति तपोनित्यः पौ॑रुशि॒ष्टिः। स्वाध्यायप्रवचने एवेति नाको॑ मौद्ग॒ल्यः। तद्धि तप॑स्तद्धि॒ तपः॥१६॥
\anuvakamend

५.१०.०
अ॒ह षट्॥ १०॥
५.१०.१
अ॒हं॒ वृ॒क्षस्य॒ रेरि॑वा। की॒र्तिः पृ॒ष्ठङ्गि॒रेरि॑व। ऊ॒र्ध्वप॑वित्रो वा॒जिनी॑व स्व॒मृत॑मस्मि। द्रवि॑ण॒ सव॑र्चसम्। सुमेधा अ॑मृतो॒क्षितः। इति त्रिशङ्कोर्वेदा॑नुव॒चनम्॥१७॥
\anuvakamend

५.११.०
स्वाध्यायप्रवचनाभ्यान्न प्रम॑दित॒व्यं तानि त्वयो॑पास्या॒नि स्यात्तेषु॑ वर्ते॒रन्त्स॒प्त च॑। ११।
५.११.१
वेदमनूच्याचार्योऽन्तेवासिनम॑नुशा॒स्ति। सत्यं॒ वद। धर्मं॒ चर। स्वाध्यायान्मा प्र॒मदः। आचार्याय प्रियन्धनमाहृत्य प्रजातन्तुं मा व्य॑वच्छे॒त्सीः। सत्यान्न प्रम॑दित॒व्यम्। धर्मान्न प्रम॑दित॒व्यम्। कुशलान्न प्रम॑दित॒व्यम्। भूत्यै न प्रम॑दित॒व्यम्। स्वाध्यायप्रवचनाभ्यान्न प्रम॑दित॒व्यम्॥१८॥
५.११.२
देवपितृकार्याभ्यान्न प्रम॑दित॒व्यम्। मातृ॑देवो॒ भव। पितृ॑देवो॒ भव। आचार्य॑देवो॒ भव। अतिथि॑देवो॒ भव। यान्यनवद्यानि॑ कर्मा॒णि। तानि सेवि॑तव्या॒नि। नो इ॑तरा॒णि। यान्यस्माक सुच॑रिता॒नि। तानि त्वयो॑पास्या॒नि॥१९॥
५.११.३
नो इ॑तरा॒णि। ये के चास्मच्छ्रेयासो ब्रा॒ह्मणाः। तेषां त्वयाऽऽसनेन प्रश्व॑सित॒व्यम्। श्रद्ध॑या दे॒यम्। अश्रद्ध॑याऽदे॒यम्। श्रि॑या दे॒यम्। ह्रि॑या दे॒यम्। भि॑या दे॒यम्। संवि॑दा दे॒यम्। अथ यदि ते कर्मविचिकित्सा वा वृत्तविचिकि॑त्सा वा॒ स्यात्॥२०॥
५.११.४
ये तत्र ब्राह्मणा सम्म॒र्\mbox{}शिनः। युक्ता॑ आयु॒क्ताः। अलूक्षा॑ धर्म॑कामाः॒ स्युः। यथा ते॑ तत्र॑ वर्ते॒रन्। तथा तत्र॑ वर्ते॒थाः। अथाभ्याख्याते॒षु। ये तत्र ब्राह्मणा सम्म॒र्\mbox{}शिनः। युक्ता॑ आयु॒क्ताः। अलूक्षा॑ धर्म॑कामाः॒ स्युः। यथा ते॑ तेषु॑ वर्ते॒रन्। तथा तेषु॑ वर्ते॒थाः। एष॑ आदे॒शः। एष उ॑पदे॒शः। एषा वे॑दोप॒निषत्। एतद॑नुशा॒सनम्। एवमुपा॑सित॒व्यम्। एवमु चैत॑दुपा॒स्यम्॥२१॥
\anuvakamend

५.१२.०
स॒त्यम॑वादिषं॒ पञ्च॑ च॥ १ २॥
५.१२.१
शं नो॑ मि॒त्रः  शं वरु॑णः। शं नो॑ भवत्वर्य॒मा। शं न॒ इन्द्रो॒ बृह॒स्पति॑। शं नो॒ विष्णु॑रुरुक्र॒मः। नमो॒ ब्रह्म॑णे। नम॑स्ते वायो। त्वमे॒व प्र॒त्यक्षं॒ ब्रह्मा॑सि। त्वामे॒व प्र॒त्यक्षं॒ ब्रह्मावा॑दिषम्। ऋ॒तम॑वादिषम्। स॒त्यम॑वादिषम्। तन्मामा॑वीत्। तद्व॒क्तार॑मावीत्। आवी॒न्माम्। आवीद्व॒क्तारम्। ॐ शान्तिः॒ शान्तिः॒ शान्ति॑॥२२॥

\closesection
\clearpage

\sect{अष्टमः प्रश्नः --- ब्रह्मानन्दवल्ली}\setcounter{anuvakam}{0}
५.१३.०
स॒ह ना॑ववतु॒ पञ्च॑॥ १३॥
५.१३.१
स॒ह ना॑ववतु। स॒ह नौ॑ भुनक्तु। स॒ह वी॒र्यं॑ करवावहै। ते॒ज॒स्वि ना॒वधी॑तमस्तु॒ मा वि॑द्विषा॒वहै। ॐ शान्तिः॒ शान्तिः॒ शान्ति॑॥२३॥
\anuvakamend

५.१४.०
(ब्र॒ह्म॒विदिदमयमिदमेक॑विशतिः। अन्ना॒दन्न॑रस॒मयात् प्राणो॒ व्यानोऽपान आका॑शः॒ पृथिवी पुच्छ॒ षड्विशतिः। प्रा॒णं प्रा॑ण॒मयान्मनो॒ यजु॒\ar{} ऋख्सामादे॒शोऽथर्वाङ्गिरसः पुच्छ॒न्द्वाविशतिः। यत॑ श्र॒द्धर्त्त सत्यं यो॑गो॒ महोऽष्टाद॑श। वि॒ज्ञानं॒ प्रियं॒ मोदः प्रमोद आन॑न्दो॒ ब्रह्म पुच्छ॒न्द्वाविशतिः। अस॑न्ने॒वाथाष्टाविशतिः। अस॒थ्षोड॑श। भी॒षाऽस्मा॒न्मानुषो॒ मनुष्यगन्धर्वाणा॒न्देवगन्धर्वाणां॒ पितृणां चिरलोकलोकाना॒माजानजानां कर्मदेवानां ये कर्मणा देवाना॒मिन्द्र॑स्य॒ बृहस्पतेः॒ प्रजापते॒र्ब्रह्म॑णः॒ स यश्च॑ सङ्क्रा॒मत्येक॑पञ्चा॒शत्। यतः॒ कुत॑श्च॒नैतमेका॑दश॒ नव॑॥ ९॥ ॥
५.१४.१
ब्र॒ह्म॒विदाप्नोति॒ परम्। तदे॒षाभ्यु॑क्ता। स॒त्यं ज्ञा॒नम॑न॒न्तं ब्रह्म॑। यो वेद॒ निहि॑त॒ङ्गुहा॑यां पर॒मे व्यो॑मन्। सोऽश्नुते॒ सर्वा॒न्कामान्त्स॒ह। ब्रह्म॑णा विप॒श्चितेति॑। तस्मा॒द्वा ए॒तस्मा॑दा॒त्मन॑ आका॒शः सम्भू॑तः। आ॒का॒शाद्वा॒युः। वा॒योर॒ग्निः। अ॒ग्नेराप॑। अ॒द्भ्यः पृ॑थि॒वी। पृ॒थि॒व्या ओष॑धयः। ओष॑धी॒भ्योऽन्नम्। अन्ना॒त्पुरु॑षः। स वा एष पुरुषोऽन्न॑रस॒मयः। तस्येद॑मेव॒ शिरः। अयन्दक्षि॑णः प॒क्षः। अयमुत्त॑रः प॒क्षः। अयमात्मा। इदं पुच्छं॑ प्रति॒ष्ठा। तदप्येष श्लो॑को भ॒वति॥२४॥
५.१४.२
अन्ना॒द्वै प्र॒जाः प्र॒जाय॑न्ते। याः काश्च॑ पृथि॒वी श्रि॒ताः। अथो॒ अन्ने॑नै॒व जी॑वन्ति। अथै॑न॒दपि॑ यन्त्यन्त॒तः। अन्न॒ हि भू॒तानां॒ ज्येष्ठम्। तस्मात्सर्वौष॒धमु॑च्यते। सर्वं॒ वै तेऽन्न॑माप्नुवन्ति। येऽन्नं॒ ब्रह्मो॒पास॑ते। अन्न॒ हि भू॒तानां॒ ज्येष्ठम्। तस्मात्सर्वौष॒धमु॑च्यते। अन्नाद्भू॒तानि॒ जाय॑न्ते। जाता॒न्यन्ने॑न वर्धन्ते। अद्यतेऽत्ति च॑ भूता॒नि। तस्मादन्नं तदुच्य॑त इ॒ति। तस्माद्वा एतस्मादन्न॑रस॒मयात्। अन्योऽन्तर आत्मा प्राण॒मयः। तेनै॑ष पू॒र्णः। स वा एष पुरुषवि॑ध ए॒व। तस्य पुरु॑षवि॒धताम्। अन्वयं॑ पुरुष॒विधः। तस्य प्राण॑ एव॒ शिरः। व्यानो दक्षि॑णः प॒क्षः। अपान उत्त॑रः प॒क्षः। आका॑श आ॒त्मा। पृथिवी पुच्छं॑ प्रति॒ष्ठा। तदप्येष श्लो॑को भ॒वति॥२५॥
५.१४.३
प्रा॒णन्दे॒वा अनु॒ प्राण॑न्ति। म॒नु॒ष्या प॒शव॑श्च॒ ये। प्रा॒णो हि भू॒ताना॒मायु॑। तस्मात्सर्वायु॒षमु॑च्यते। सर्व॑मे॒व त॒ आयु॑र्\mbox{}यन्ति। ये प्रा॒णं ब्रह्मो॒पास॑ते। प्राणो हि भूता॑नामा॒युः। तस्मात्सर्वायुषमुच्य॑त इ॒ति। तस्यैष एव शारी॑र आ॒त्मा। य॑ पूर्व॒स्य। तस्माद्वा एतस्मात् प्राण॒मयात्। अन्योऽन्तर आत्मा॑ मनो॒मयः। तेनै॑ष पू॒र्णः। स वा एष पुरुषवि॑ध ए॒व। तस्य पुरु॑षवि॒धताम्। अन्वयं॑ पुरुष॒विधः। तस्य यजु॑रेव॒ शिरः। ऋग्दक्षि॑णः प॒क्षः। सामोत्त॑रः प॒क्षः। आदे॑श आ॒त्मा। अथर्वाङ्गिरसः पुच्छं॑ प्रति॒ष्ठा। तदप्येष श्लो॑को भ॒वति॥२६॥
५.१४.४
यतो॒ वाचो॒ निव॑र्तन्ते। अप्राप्य॒ मन॑सा स॒ह। आनन्दं ब्रह्म॑णो वि॒द्वान्। न बिभेति कदा॑चने॒ति। तस्यैष एव शारी॑र आ॒त्मा। य॑ पूर्व॒स्य। तस्माद्वा एतस्मान्मनो॒मयात्। अन्योऽन्तर आत्मा वि॑ज्ञान॒मयः। तेनै॑ष पू॒र्णः। स वा एष पुरुषवि॑ध ए॒व। तस्य पुरु॑षवि॒धताम्। अन्वयं॑ पुरुष॒विधः। तस्य श्र॑द्धैव॒ शिरः। ऋतन्दक्षि॑णः प॒क्षः। सत्यमुत्त॑रः प॒क्षः। यो॑ग आ॒त्मा। महः पुच्छं॑ प्रति॒ष्ठा। तदप्येष श्लो॑को भ॒वति॥२७॥
५.१४.५
वि॒ज्ञानं॑ य॒ज्ञं त॑नुते। कर्मा॑णि तनु॒तेऽपि॑ च। वि॒ज्ञानं॑ दे॒वाः सर्वे। ब्रह्म॒ ज्येष्ठ॒मुपा॑सते। वि॒ज्ञानं॒ ब्रह्म॒ चेद्वेद॑। तस्मा॒च्चेन्न प्र॒माद्य॑ति। शरीरे॑ पाप्म॑नो हि॒त्वा। सर्वान्कामान्त्समश्नु॑त इ॒ति। तस्यैष एव शारी॑र आ॒त्मा। य॑ पूर्व॒स्य। तस्माद्वा एतस्माद्वि॑ज्ञान॒मयात्। अन्योऽन्तर आत्मा॑ऽऽनन्द॒मयः। तेनै॑ष पू॒र्णः। स वा एष पुरुषवि॑ध ए॒व। तस्य पुरु॑षवि॒धताम्। अन्वयं॑ पुरुष॒विधः। तस्य प्रिय॑मेव॒ शिरः। मोदो दक्षि॑णः प॒क्षः। प्रमोद उत्त॑रः प॒क्षः। आन॑न्द आ॒त्मा। ब्रह्म पुच्छं॑ प्रति॒ष्ठा। तदप्येष श्लो॑को भ॒वति॥२८॥
५.१४.६
अस॑न्ने॒व स॑ भवति। अस॒द्ब्रह्मेति॒ वेद॒ चेत्। अस्ति ब्रह्मेति॑ चेद्वे॒द। सन्तमेनं ततो वि॑दुरि॒ति। तस्यैष एव शारी॑र आ॒त्मा। य॑ पूर्व॒स्य। अथातो॑ऽनुप्र॒श्ञाः। उ॒ता वि॒द्वान॒मुं लो॒कं प्रेत्य॑। कश्च॒न ग॑च्छ॒ती(३)॥ आहो॑ वि॒द्वान॒मुल्लोँ॒कं प्रेत्य॑। कश्चि॒त्सम॑श्ञु॒ता(३) उ॒। सो॑ऽकामयत। ब॒हु स्यां॒ प्रजा॑ये॒येति॑। स तपो॑ऽतप्यत। स तप॑स्त॒प्त्वा। इ॒द सर्व॑मसृजत। यदि॒दं किं च॑। तत्सृ॒ष्ट्वा। तदे॒वानु॒ प्रावि॑शत्। तद॑नुप्र॒विश्य॑। सच्च॒ त्यच्चा॑भवत्। नि॒रुक्तं॒ चानि॑रुक्तं च। नि॒लय॑नं॒ चानि॑लयनं च। वि॒ज्ञानं॒ चावि॑ज्ञानं च। सत्यं चानृतं च स॑त्यम॒भवत्। यदि॑दं किं॒ च। तत्सत्यमि॑त्याच॒क्षते। तदप्येष श्लो॑को भ॒वति॥२९॥
५.१४.७
अस॒द्वा इ॒दमग्र॑ आसीत्। ततो॒ वै सद॑जायत। तदात्मान स्वय॑मकु॒रुत। तस्मात्तत्सुकृतमुच्य॑त इ॒ति। यद्वै॑ तत्सु॒कृतम्। र॑सो वै॒ सः। रस ह्येवायल्लँब्ध्वाऽऽन॑न्दी भ॒वति। को ह्येवान्यात्कः प्रा॒ण्यात्। यदेष आकाश आन॑न्दो न॒ स्यात्। एष ह्येवान॑न्दया॒ति। य॒दा ह्ये॑वैष॒ एतस्मिन्नदृश्येऽनात्म्येऽनिरुक्तेऽनिलयनेऽभयं प्रति॑ष्ठां  वि॒न्दते। अथ सोऽभयङ्ग॑तो भ॒वति। य॒दा ह्ये॑वैष॒ एतस्मिन्नु दरमन्त॑रं कु॒रुते। अथ तस्य भ॑यं भ॒वति। तत्त्वेव भयं  विदुषोऽम॑न्वान॒स्य। तदप्येष श्लो॑को भ॒वति॥३०॥
५.१४.८
भी॒षाऽस्मा॒द्वात॑ पवते। भी॒षोदे॑ति॒ सूर्य॑। भीषाऽस्मादग्नि॑श्चेन्द्र॒श्च। मृत्युर्धावति पञ्च॑म इ॒ति। सैषाऽऽनन्दस्य मीमासा भ॒वति। युवा स्यात्साधु यु॑वाऽध्या॒यकः। आशिष्ठो दृढिष्ठो॑ बलि॒ष्ठः। तस्येयं पृथिवी सर्वा वित्तस्य॑ पूर्णा॒ स्यात्। स एको मानुष॑ आन॒न्दः। ते ये शतं मानुषा॑ आन॒न्दाः। स एको मनुष्यगन्धर्वाणा॑मान॒न्दः। श्रोत्रियस्य चाकाम॑हत॒स्य। ते ये शतं मनुष्यगन्धर्वाणा॑मान॒न्दाः। स एको देवगन्धर्वाणा॑मान॒न्दः। श्रोत्रियस्य चाकाम॑हत॒स्य। ते ये शतन्देवगन्धर्वाणा॑मान॒न्दाः। स एकः पितृणां चिरलोकलोकाना॑मान॒न्दः। श्रोत्रियस्य चाकाम॑हत॒स्य। ते ये शतं पितृणां चिरलोकलोकाना॑मान॒न्दाः। स एक आजानजानान्देवाना॑मान॒न्दः। श्रोत्रियस्य चाकाम॑हत॒स्य। ते ये शतमाजानजानान्देवाना॑मान॒न्दाः। स एकः कर्मदेवानान्देवाना॑मान॒न्दः। ये कर्मणा देवान॑पिय॒न्ति। श्रोत्रियस्य चाकाम॑हत॒स्य। ते ये शतं कर्मदेवानान्देवाना॑मान॒न्दाः। स एको देवाना॑मान॒न्दः। श्रोत्रियस्य चाकाम॑हत॒स्य। ते ये शतन्देवाना॑मान॒न्दाः। स एक इन्द्र॑स्यान॒न्दः। श्रोत्रियस्य चाकाम॑हत॒स्य। ते ये शतमिन्द्र॑स्यान॒न्दाः। स एको बृहस्पते॑रान॒न्दः। श्रोत्रियस्य चाकाम॑हत॒स्य। ते ये शतं बृहस्पते॑रान॒न्दाः। स एकः प्रजापते॑रान॒न्दः। श्रोत्रियस्य चाकाम॑हत॒स्य। ते ये शतं प्रजापते॑रान॒न्दाः। स एको ब्रह्मण॑ आन॒न्दः। श्रोत्रियस्य चाकाम॑हत॒स्य। स यश्चा॑यं पु॒रुषे। यश्चासा॑वादि॒त्ये। स एक॑। स य॑ एवं॒वित्। अस्माल्लो॑कात्प्रे॒त्य। एतमन्नमयमात्मानमुप॑सङ्क्रा॒मति। एतं प्राणमयमात्मानमुप॑सङ्क्रा॒मति। एतम्मनोमयमात्मानमुप॑सङ्क्रा॒मति। एतं विज्ञानमयमात्मानमुप॑सङ्क्रा॒मति। एतमानन्दमयमात्मानमुप॑सङ्क्रा॒मति। तदप्येष श्लो॑को भ॒वति॥३१॥
५.१४.९
यतो॒ वाचो॒ निव॑र्तन्ते। अप्राप्य॒ मन॑सा स॒ह। आनन्दं ब्रह्म॑णो वि॒द्वान्। न बिभेति कुत॑श्चने॒ति। एत ह वाव॑ न त॒पति। किमह साधु॑ नाक॒रवम्। किमहं पापमकर॑वमि॒ति। स य एवं  विद्वानेते आत्मा॑न स्पृ॒णुते। उ॒भे ह्ये॑वैष॒ एते आत्मा॑न स्पृ॒णुते। य ए॒वं वेद॑। इत्यु॑प॒निष॑त्॥३२॥

ॐ स॒ह ना॑ववतु। स॒ह नौ॑ भुनक्तु। स॒ह वी॒र्यं॑ करवावहै। ते॒ज॒स्वि ना॒वधी॑तमस्तु॒ मा वि॑द्विषा॒वहै। ॐ शान्तिः॒ शान्तिः॒ शान्ति॑॥

\closesection
\clearpage

\sect{नवमः प्रश्नः --- भृगुवल्ली}\setcounter{anuvakam}{0}
ॐ स॒ह ना॑ववतु। स॒ह नौ॑ भुनक्तु। स॒ह वी॒र्यं॑ करवावहै। ते॒ज॒स्वि ना॒वधी॑तमस्तु॒ मा वि॑द्विषा॒वहै। ॐ शान्तिः॒ शान्तिः॒ शान्ति॑॥

५.१५.१
भृगु॒र्वै वा॑रु॒णिः। वरु॑णं॒ पित॑र॒मुप॑ससार। अधी॑हि भगवो॒ ब्रह्मेति॑। तस्मा॑ ए॒तत्प्रो॑वाच। अन्नं॑ प्रा॒णं चक्षुः॒ श्रोत्रं॒ मनो॒ वाच॒मिति॑। त हो॑वाच। यतो॒ वा इ॒मानि॒ भूता॑नि॒ जाय॑न्ते। येन॒ जाता॑नि॒ जीव॑न्ति। यत्प्रय॑न्त्य॒भि संवि॑शन्ति। तद्विजि॑ज्ञासस्व। तद्ब्रह्मेति॑। स तपो॑ऽतप्यत। स तप॑स्त॒प्त्वा॥३३॥
५.१५.२
अन्नं॒ ब्रह्मेति॒ व्य॑जानात्। अ॒न्नाद्ध्ये॑व खल्वि॒मानि॒ भूता॑नि॒ जाय॑न्ते। अन्ने॑न॒ जाता॑नि॒ जीव॑न्ति। अ॒न्नं प्रय॑न्त्य॒भि  संवि॑श॒न्तीति॑। तद्वि॒ज्ञाय॑। पुन॑रे॒व वरु॑णं॒ पित॑र॒मुप॑ससार। अधी॑हि भगवो॒ ब्रह्मेति॑। त हो॑वाच। तप॑सा॒ ब्रह्म॒ विजि॑ज्ञासस्व। तपो॒ ब्रह्मेति॑। स तपो॑ऽतप्यत। स तप॑स्त॒प्त्वा॥३४॥
५.१५.३
प्रा॒णो ब्र॒ह्मेति॒ व्य॑जानात्। प्रा॒णाद्ध्ये॑व खल्वि॒मानि॒ भूता॑नि॒ जाय॑न्ते। प्रा॒णेन॒ जाता॑नि॒ जीव॑न्ति। प्रा॒णं प्रय॑न्त्य॒भि  संवि॑श॒न्तीति॑। तद्वि॒ज्ञाय॑। पुन॑रे॒व वरु॑णं॒ पित॑र॒मुप॑ससार। अधी॑हि भगवो॒ ब्रह्मेति॑। त हो॑वाच। तप॑सा॒ ब्रह्म॒ विजि॑ज्ञासस्व। तपो॒ ब्रह्मेति॑। स तपो॑ऽतप्यत। स तप॑स्त॒प्त्वा॥३५॥
५.१५.४
मनो॒ ब्रह्मेति॒ व्य॑जानात्। मन॑सो॒ ह्ये॑व खल्वि॒मानि॒ भूता॑नि॒ जाय॑न्ते। मन॑सा॒ जाता॑नि॒ जीव॑न्ति। मनः॒ प्रय॑न्त्य॒भि  संवि॑श॒न्तीति॑। तद्वि॒ज्ञाय॑। पुन॑रे॒व वरु॑णं॒ पित॑र॒मुप॑ससार। अधी॑हि भगवो॒ ब्रह्मेति॑। त हो॑वाच। तप॑सा॒ ब्रह्म॒ विजि॑ज्ञासस्व। तपो॒ ब्रह्मेति॑। स तपो॑ऽतप्यत। स तप॑स्त॒प्त्वा॥३६॥
५.१५.५
वि॒ज्ञानं॒ ब्रह्मेति॒ व्य॑जानात्। वि॒ज्ञाना॒द्ध्ये॑व खल्वि॒मानि॒ भूता॑नि॒ जाय॑न्ते। वि॒ज्ञाने॑न॒ जाता॑नि॒ जीव॑न्ति। वि॒ज्ञानं॒ प्रय॑न्त्य॒भि संवि॑श॒न्तीति॑। तद्वि॒ज्ञाय॑। पुन॑रे॒व वरु॑णं॒ पित॑र॒मुप॑ससार। अधी॑हि भगवो॒ ब्रह्मेति॑। त हो॑वाच। तप॑सा॒ ब्रह्म॒ विजि॑ज्ञासस्व। तपो॒ ब्रह्मेति॑। स तपो॑ऽतप्यत। स तप॑स्त॒प्त्वा॥३७॥
५.१५.६
आ॒न॒न्दो ब्र॒ह्मेति॒ व्य॑जानात्। आ॒नन्दा॒द्ध्ये॑व खल्वि॒मानि॒ भूता॑नि॒ जाय॑न्ते। आ॒न॒न्देन॒ जाता॑नि॒ जीव॑न्ति। आ॒न॒न्दं प्रय॑न्त्य॒भि संवि॑श॒न्तीति॑। सैषा भार्ग॒वी वा॑रु॒णी वि॒द्या। प॒र॒मे व्यो॑म॒न् प्रति॑ष्ठिता। य ए॒वं वेद॒ प्रति॑तिष्ठति। अन्न॑वानन्ना॒दो भ॑वति। म॒हान्भ॑वति प्र॒जया॑ प॒शुभि॑र्ब्रह्मवर्च॒सेन॑। म॒हान्की॒र्त्या॥३८॥
५.१५.७
अन्नं॒ न नि॑न्द्यात्। तद्व्र॒तम्। प्रा॒णो वा अन्नम्। शरी॑रमन्ना॒दम्। प्रा॒णे शरी॑रं॒ प्रति॑ष्ठितम्। शरी॑रे प्रा॒णः प्रति॑ष्ठितः। तदे॒तदन्न॒मन्ने॒ प्रति॑ष्ठितम्। स य ए॒तदन्न॒मन्ने॒ प्रति॑ष्ठितं॒ वेद॒ प्रति॑तिष्ठति। अन्न॑वानन्ना॒दो भ॑वति। म॒हान्भ॑वति प्र॒जया॑ प॒शुभि॑र्ब्रह्मवर्च॒सेन॑। म॒हान्की॒र्त्या॥३९॥
५.१५.८
अन्नं॒ न परि॑चक्षीत। तद्व्र॒तम्। आपो॒ वा अन्नम्। ज्योति॑रन्ना॒दम्। अ॒प्सु ज्योतिः॒ प्रति॑ष्ठितम्। ज्योति॒ष्यापः॒ प्रति॑ष्ठिताः। तदे॒तदन्न॒मन्ने॒ प्रति॑ष्ठितम्। स य ए॒तदन्न॒मन्ने॒ प्रति॑ष्ठितं॒ वेद॒ प्रति॑तिष्ठति। अन्न॑वानन्ना॒दो भ॑वति। म॒हान्भ॑वति प्र॒जया॑ प॒शुभि॑र्ब्रह्मवर्च॒सेन॑। म॒हान्की॒र्त्या॥४०॥
५.१५.९
अन्नं॑ ब॒हु कु॑र्वीत। तद्व्र॒तम्। पृ॒थि॒वी वा अन्नम्। आ॒का॒शोऽन्ना॒दः। पृ॒थि॒व्यामा॑का॒शः प्रति॑ष्ठितः। आ॒का॒शे पृ॑थि॒वी प्रति॑ष्ठिता। तदे॒तदन्न॒मन्ने॒ प्रति॑ष्ठितम्। स य ए॒तदन्न॒मन्ने॒ प्रति॑ष्ठितं॒ वेद॒ प्रति॑तिष्ठति। अन्न॑वानन्ना॒दो भ॑वति। म॒हान्भ॑वति प्र॒जया॑ प॒शुभि॑र्ब्रह्मवर्च॒सेन॑। म॒हान्की॒र्त्या॥४१॥
५.१५.१०
न कञ्चन वसतौ प्रत्या॑चक्षी॒त। तद्व्र॒तम्। तस्माद्यया कया च विधया बह्व॑न्नं प्रा॒प्नुयात्। अराध्यस्मा अन्नमि॑त्याच॒क्षते। एतद्वै मुखतोऽन्न रा॒द्धम्। मुखतोऽस्मा अ॑न्न रा॒ध्यते। एतद्वै मध्यतोऽन्न रा॒द्धम्। मध्यतोऽस्मा अ॑न्न रा॒ध्यते। एतद्वा अन्ततोऽन्न रा॒द्धम्। अन्ततोऽस्मा अ॑न्न रा॒ध्यते। य ए॑वं वे॒द। क्षेम इ॑ति वा॒चि। योगक्षेम इति प्रा॑णापा॒नयोः। कर्मे॑ति ह॒स्तयोः। गतिरि॑ति पा॒दयोः। विमुक्तिरि॑ति पा॒यौ। इति मानुषी समा॒ज्ञाः। अथ दै॒वीः। तृप्तिरि॑ति वृ॒ष्टौ। बलमि॑ति वि॒द्युति। यश इ॑ति प॒शुषु। ज्योतिरिति न॑क्षत्रे॒षु। प्रजातिरमृतमानन्द इ॑त्युप॒स्थे। सर्वमि॑त्याका॒शे। तत्प्रतिष्ठेत्यु॑पासी॒त। प्रतिष्ठा॑वान्भ॒वति। तन्मह इत्यु॑पासी॒त। म॑हान्भ॒वति। तन्मन इत्यु॑पासी॒त। मान॑वान्भ॒वति। तन्नम इत्यु॑पासी॒त। नम्यन्तेऽस्मै का॒माः। तद्ब्रह्मेत्यु॑पासी॒त। ब्रह्म॑वान्भ॒वति। तद्ब्रह्मणः परिमर इत्यु॑पासी॒त। पर्येणं म्रियन्ते द्विषन्त॑ सप॒त्नाः। परि येऽप्रिया भ्रातृ॒व्याः। स यश्चा॑यं पु॒रुषे। यश्चासा॑वादि॒त्ये। स एक॑। स य॑ एवं॒वित्। अस्माल्लो॑कात्प्रे॒त्य। एतमन्नमयमात्मानमुप॑सङ्क्र॒म्य। एतं प्राणमयमात्मानमुप॑सङ्क्र॒म्य। एतं मनोमयमात्मानमुप॑सङ्क्र॒म्य। एतं विज्ञानमयमात्मानमुप॑\-सङ्क्र॒म्य। एतमानन्दमयमात्मानमुप॑सङ्क्र॒म्य। इमाल्लोँकान्कामान्नी कामरूप्य॑नुस॒ञ्चरन्। एतत्साम गा॑यन्ना॒स्ते। हा(३) वु॒ हा(३) वु॒ हा(३) वु॑। अ॒हमन्नम॒हमन्नम॒हमन्नम्। अ॒हमन्ना॒दो(२)\-\aav{}हमन्ना॒दो(२)\aav{}हमन्ना॒दः। अ॒ह श्लोक॒कृद॒ह श्लोक॒कृद॒ह श्लोक॒कृत्। अहमस्मि प्रथमजा ऋता(३) स्य॒। पूर्वन्देवेभ्यो अमृतस्य ना(३) भा॒इ॒। यो मा ददाति स इदेव मा(३) वाः॒। अ॒हमन्न॒मन्न॑म॒दन्त॒मा(३) द्मि॒। अ॒हं  विश्वं॒ भुव॑न॒मभ्य॑भ॒वाम्। सुव॒र्न ज्योती। य ए॒वं वेद॑। इत्यु॑प॒निष॑त्॥४२॥

ॐ स॒ह ना॑ववतु। स॒ह नौ॑ भुनक्तु। स॒ह वी॒र्यं॑ करवावहै। ते॒ज॒स्वि ना॒वधी॑तमस्तु॒ मा वि॑द्विषा॒वहै। ॐ शान्तिः॒ शान्तिः॒ शान्ति॑॥

\closesection
\clearpage

\sect{दशमः प्रश्नः --- महानारायणोपनिषत्}\setcounter{anuvakam}{0}

%६.१.०

%६.१.१
अम्भ॑स्यपा॒रे भुव॑नस्य॒ मध्ये॒ नाक॑स्य पृ॒ष्ठे म॑ह॒तो मही॑यान्। शु॒क्रेण॒ ज्योतीषि समनु॒प्रवि॑ष्टः प्र॒जाप॑तिश्चरति॒ गर्भे॑ अ॒न्तः॥ यस्मि॑न्नि॒द सं च॒ विचैति॒ सर्वं॒ यस्मि॑न्दे॒वा अधि॒ विश्वे॑ निषे॒दुः। तदे॒व भू॒तं तदु॒ भव्य॑मा इ॒दं तद॒क्षरे॑ पर॒मे व्यो॑मन्॥ येना॑वृ॒तं खं च॒ दिवं॑ म॒हीं च॒ येना॑दि॒त्यस्तप॑ति॒ तेज॑सा॒ भ्राज॑सा च। यम॒न्तः स॑मु॒द्रे क॒वयो॒ वय॑न्ति॒ यद॒क्षरे॑ पर॒मे प्र॒जाः॥ यत॑ प्रसू॒ता ज॒गत॑ प्रसूती॒ तोये॑न जी॒वान्व्यस॑सर्ज॒ भूम्याम्। यदोष॑धीभिः पु॒रुषान्प॒शूश्च॒ विवे॑श भू॒तानि॑ चराच॒राणि॑॥ अत॑ परं॒ नान्य॒दणी॑यस हि॒ परात्परं॒ यन्मह॑तो म॒हान्तम्। यदे॑कम॒व्यक्त॒मन॑न्तरूपं॒  विश्वं॑ पुरा॒णं तम॑सः॒ पर॑स्तात्॥१॥

%६.१.२
तदे॒वर्तं तदु॑ स॒त्यमा॑हु॒स्तदे॒व ब्रह्म॑ पर॒मं क॑वी॒नाम्। इ॒ष्टा॒पू॒र्तं ब॑हु॒धा जा॒तं जाय॑मानं  वि॒श्वं बि॑भर्ति॒ भुव॑नस्य॒ नाभि॑॥ तदे॒वाग्निस्तद्वा॒युस्तत्सूर्य॒स्तदु॑ च॒न्द्रमा। तदे॒व शु॒क्रम॒मृतं॒ तद्ब्रह्म॒ तदापः॒ स प्र॒जाप॑तिः॥ सर्वे॑ निमे॒षा ज॒ज्ञिरे॑ वि॒द्युतः॒ पुरु॑षा॒दधि॑। क॒ला मु॑हू॒र्ताः काष्ठाश्चाहोरा॒त्राश्च॑ सर्व॒शः॥ अ॒र्द्ध॒मा॒सा मासा॑ ऋ॒तव॑ संवत्स॒रश्च॑ कल्पताम्। स आप॑ प्रदु॒घे उ॒भे इ॒मे अ॒न्तरि॑क्ष॒मथो॒ सुव॑॥ नैन॑मू॒र्ध्वन्न ति॒र्यञ्चं॒ न मध्ये॒ परि॑जग्रभत्। न तस्ये॑शे॒ कश्च॒न तस्य॑ नाम म॒हद्यश॑॥२॥

%६.१.३
न स॒न्दृशे॑ तिष्ठति॒ रूप॑मस्य॒ न चक्षु॑षा पश्यति॒ कश्च॒नैनम्। हृ॒दा म॑नी॒षा मन॑सा॒ऽभिक्लृ॑प्तो॒ य ए॑नं  वि॒दुरमृ॑ता॒स्ते भ॑वन्ति॥ अ॒द्भ्यः सम्भू॑तो हिरण्यग॒र्भ इत्य॒ष्टौ॥ ए॒ष हि दे॒वः प्र॒दिशोनु॒ सर्वाः॒ पूर्वो॑ हि जा॒तः स उ॒ गर्भे॑ अ॒न्तः। स वि॒जाय॑मानः स जनि॒ष्यमा॑णः प्र॒त्यङ्मुखास्तिष्ठति वि॒श्वतो॑मुखः॥ वि॒श्वत॑श्चक्षुरु॒त वि॒श्वतो॑मुखो वि॒श्वतो॑हस्त उ॒त वि॒श्वत॑स्पात्। सं बा॒हुभ्यां॒ नम॑ति॒ सं पत॑त्रै॒र्द्यावा॑पृथि॒वी ज॒नय॑न्दे॒व एक॑॥ वे॒नस्तत्पश्य॒न्विश्वा॒ भुव॑नानि वि॒द्वान् यत्र॒ विश्वं॒ भव॒त्येक॑नीळम्। यस्मि॑न्नि॒द सं च॒ विचैक॒ स ओतः॒ प्रोत॑श्च वि॒भुः प्र॒जासु॑। प्र तद्वो॑चे अ॒मृतं॒ नु वि॒द्वान्ग॑न्ध॒र्वो नाम॒ निहि॑त॒ङ्गुहा॑सु॥३॥

%६.१.४
त्रीणि॑ प॒दा निहि॑ता॒ गुहा॑सु॒ यस्तद्वेद॑ सवि॒तुः पि॒ताऽस॑त्। स नो॒ बन्धु॑र्जनि॒ता स वि॑धा॒ता धामा॑नि॒ वेद॒ भुव॑नानि॒ विश्वा। यत्र॑ दे॒वा अ॒मृत॑मानशा॒नास्तृ॒तीये॒ धामान्य॒भ्यैर॑यन्त। परि॒ द्यावा॑पृथि॒वी य॑न्ति स॒द्यः परि॑ लो॒कान् परि॒ दिशः॒ परि॒ सुव॑। ऋ॒तस्य॒ तन्तुं॑  विततं  वि॒चृत्य॒ तद॑पश्य॒त्तद॑भवत् प्र॒जासु॑। प॒रीत्य॑ लो॒कान्प॒रीत्य॑ भू॒तानि॑ प॒रीत्य॒ सर्वा प्र॒दिशो॒ दिश॑श्च। प्र॒जाप॑तिः प्रथम॒जा ऋ॒तस्य॒\aav{}\aav{}त्मन॒\aav{}\aav{}त्मान॑म॒भिसम्ब॑भूव। सद॑स॒स्पति॒मद्भु॑तं प्रि॒यमिन्द्र॑स्य॒ काम्यम्। सनिं॑ मे॒धाम॑यासिषम्। उद्दीप्यस्व जातवेदोऽप॒घ्नन्निर्\mbox{}ऋ॑तिं॒ मम॑॥४॥

%६.१.५
प॒शूश्च॒ मह्य॒माव॑ह॒ जीव॑नं च॒ दिशो॑ दिश। मा नो॑ हिसीज्जातवेदो॒ गामश्वं॒ पुरु॑षं॒ जग॑त्। अबि॑भ्र॒दग्न॒ आग॑हि श्रि॒या मा॒ परि॑पातय। पुरु॑षस्य विद्म सहस्रा॒क्षस्य॑ महादे॒वस्य॑ धीमहि। तन्नो॑ रुद्रः प्रचो॒दयात्। तत्पुरु॑षाय वि॒द्महे॑ महादे॒वाय॑ धीमहि। तन्नो॑ रुद्रः प्रचो॒दयात्। तत्पुरु॑षाय वि॒द्महे॑ वक्रतु॒ण्डाय॑ धीमहि। तन्नो॑ दन्तिः प्रचो॒दयात्। तत्पुरु॑षाय वि॒द्महे॑ सुवर्णप॒क्षाय॑ धीमहि॥५॥

%६.१.६
तन्नो॑ गरुडः प्रचो॒दयात्। का॒त्या॒य॒नाय॑ वि॒द्महे॑ कन्यकु॒मारि॑ धीमहि। तन्नो॑ दुर्गिः प्रचो॒दयात्। ना॒रा॒य॒णाय॑ वि॒द्महे॑ वासुदे॒वाय॑ धीमहि। तन्नो॑ विष्णुः प्रचो॒दयात्। स॒ह॒स्र॒पर॑मा दे॒वी॒ श॒तमू॑ला श॒ताङ्कु॑रा। सर्व हरतु॑ मे पा॒प॒न्दू॒र्वा दु॑स्वप्न॒नाशि॑नी। अश्व॑क्रा॒न्ते र॑थक्रा॒न्ते॒ वि॒ष्णुक्रान्ते व॒सुन्ध॑रा। शिरसा॑ धारि॑ता दे॒वी॒ र॒क्ष॒स्व मां पदे॒ पदे। उ॒द्धृता॑सि व॑राहे॒ण॒ कृ॒ष्णे॒न श॑तबा॒हुना॥६॥

%६.१.७
भूमिर्द्धेनुर्धरणी लो॑कधा॒रिणी। मृ॒त्तिके॑ हन॑ मे पा॒पं॒ य॒न्म॒या दु॑ष्कृतं॒ कृतम्। त्वया॑ ह॒तेन॑ पापे॒न॒ जी॒वा॒मि श॑रदः॒ शतम्। मृ॒त्तिके॑ देहि॑ मे पु॒ष्टिं॒ त्व॒यि स॑र्वं प्र॒तिष्ठि॑तम्। ग॒न्ध॒द्वा॒रान्दु॑राध॒र्\mbox{}षान्नि॒त्यपु॑ष्टां करी॒षिणीम्। ई॒श्वरी सर्व॑भूता॒नां॒ त्वामि॒होप॑ह्वये॒ श्रियम्। हिर॑ण्यशृङ्गं॒ वरु॑णं॒ प्रप॑द्ये ती॒र्थं मे देहि॒ याचि॑तः। य॒न्मया॑ भु॒क्तम॒साधू॑नां पा॒पेभ्य॑श्च प्र॒तिग्र॑हः। यन्मे॒ मन॑सा वा॒चा॒ क॒र्म॒णा वा दु॑ष्कृतं॒ कृतम्। तन्न॒ इन्द्रो॒ वरु॑णो॒ बृह॒स्पति॑ सवि॒ता च॑ पुनन्तु॒ पुन॑ पुनः॥७॥

%६.१.८
सु॒मि॒त्रा न॒ आप॒ ओष॑धयः सन्तु दुर्मि॒त्रास्तस्मै॑ भूयासु॒र्योऽस्मान्द्वेष्टि॒ यं च॑ व॒यं द्वि॒ष्मः। नमो॒ऽग्नयेऽप्सु॒मते॒ नम॒ इन्द्रा॑य॒ नमो॒ वरु॑णाय॒ नमो वारुण्यै॑ नमो॒ऽद्भ्यः। यद॒पां क्रू॒रं यद॑मे॒ध्यं यद॑शा॒न्तं तदप॑गच्छतात्। अ॒त्या॒श॒नाद॑तीपा॒ना॒द्य॒च्च उ॒ग्रात् प्र॑ति॒ग्रहात्। तन्मे॒ वरु॑णो रा॒जा॒ पा॒णिना ह्यव॒मर्\mbox{}श॑तु। सो॑ऽहम॑पा॒पो वि॒रजो॒ निर्मु॒क्तो मु॑क्तकि॒ल्बिषः। नाक॑स्य पृ॒ष्ठमारु॑ह्य॒ गच्छे॒द्ब्रह्म॑सलो॒कताम्। इ॒मं मे॑ गङ्गे यमुने सरस्वति॒ शुतु॑द्रि॒ स्तोम सचता॒ परु॒ष्णिया। अ॒सि॒क्नि॒या म॑रुद्वृधे वि॒तस्त॒याऽऽर्जी॑कीये श्रुणु॒ह्या सु॒षोम॑या। ऋ॒तं च॑ स॒त्यं चा॒भीद्धा॒त्तप॒सोऽध्य॑जायत॥८॥

%६.१.९
ततो॒ रात्रि॑रजायत॒ तत॑ समु॒द्रो अ॑र्ण॒वः। स॒मु॒द्राद॑र्ण॒वादधि॑ संवत्स॒रो अ॑जायत। अ॒हो॒रा॒त्राणि॑ वि॒दध॒द्विश्व॑स्य मिष॒तो व॒शी। सू॒र्या॒च॒न्द्र॒मसौ॑ धा॒ता य॑थापू॒र्वम॑कल्पयत्। दिवं॑ च पृथि॒वीं चा॒न्तरि॑क्ष॒मथो॒ सुव॑। यत्पृ॑थि॒व्या रज॑स्व॒ मान्तरि॑क्षे वि॒रोद॑सी। इ॒मास्तदा॒पो व॑रुणः पु॒नात्व॑घमर्\mbox{}ष॒णः। ए॒ष भू॒तस्य॑ भ॒व्ये भुव॑नस्य गो॒प्ता। ए॒ष पु॒ण्यकृ॑तां लो॒का॒ने॒ष मृ॒त्योर्\mbox{}हि॑र॒ण्मयम्। द्यावा॑पृथि॒व्योर्\mbox{}हि॑र॒ण्मय॒ सश्रि॑त॒ सुव॑॥९॥

%६.१.१०
स नः॒ सुवः॒ सशि॑शाधि। आर्द्रं॒ ज्वल॑ति॒ ज्योति॑र॒हम॑स्मि। ज्योति॒र्ज्वल॑ति॒ ब्रह्मा॒हम॑स्मि। यो॑ऽहम॑स्मि॒ ब्रह्मा॒हम॑स्मि। अ॒हमे॒वाहं मां जु॑होमि॒ स्वाहा। अ॒का॒र्य॒का॒र्य॑वकी॒र्णी स्ते॒नो भ्रू॑ण॒हा गु॑रुत॒ल्पगः। वरु॑णो॒ऽपाम॑घमर्\mbox{}ष॒णस्तस्मात्पा॒पात् प्रमु॑च्यते। र॒जो भूमि॑स्त्व॒मा रोद॑यस्व॒ प्रव॑दन्ति॒ धीरा। पु॒नन्तु॒ ऋष॑यः पु॒नन्तु॒ वस॑वः पु॒नातु॒ वरु॑णः पु॒नात्व॑घमर्\mbox{}ष॒णः। आक्रान्त्समु॒द्रः प्र॑थ॒मे विध॑र्मं ज॒नय॑न्प्र॒जा भुव॑नस्य॒ राजा॥१०॥

%६.१.११
वृषा॑ प॒वित्रे॒ अधि॒ सानो॒ अव्ये॑ बृ॒हत्सोमो॑ वावृधे सुवा॒न इन्दु॑। जा॒तवे॑दसे सुनवाम॒ सोम॑मरातीय॒तो निज॑हाति॒ वेद॑। स न॑ पर्\mbox{}ष॒दति॑ दु॒र्गाणि॒ विश्वा॑ ना॒वेव॒ सिन्धुं॑ दुरि॒ताऽत्य॒ग्निः। ताम॒ग्निव॑र्णां॒ तप॑सा ज्वल॒न्तीं॒ वै॑रोच॒नीं क॑र्मफ॒लेषु॒ जुष्टाम्। दु॒र्गान्दे॒वी शर॑णम॒हं प्रप॑द्ये सु॒तर॑सि तरसे॒ नम॑। अग्ने॒ त्वं पा॑रया॒ नव्यो॑ अ॒स्मान्त्स्व॒स्तिभि॒रति॑ दु॒र्गाणि॒ विश्वा। पूश्च॑ पृ॒थ्वी ब॑हु॒ला न॑ उ॒र्वी भवा॑ तो॒काय॒ तन॑याय॒ शँय्योः। विश्वा॑नि नो दु॒र्गहा॑ जातवेदः॒ सिन्धुं॒ न ना॒वा दु॑रि॒ताति॑ पर्\mbox{}षि। अग्ने॑ अत्रि॒वन्मन॑सा गृणा॒नोऽस्माकं॑ बोध्यवि॒ता त॒नूनाम्। पृ॒त॒ना॒जित॒ सह॑मानम॒ग्निमु॒ग्र हु॑वेम पर॒मात्स॒धस्थात्। स न॑ पर्\mbox{}ष॒दति॑ दु॒र्गाणि॒ विश्वा॒ क्षाम॑द्दे॒वो अति॑ दुरि॒ताऽत्य॒ग्निः। प्र॒त्नोषि॑ क॒मीड्यो॑ अध्व॒रेषु॑ स॒नाच्च॒ होता॒ नव्य॑श्च॒ सत्सि॑। स्वाञ्चाग्ने त॒नुवं॑ पि॒प्रय॑स्वा॒स्मभ्यं॑ च॒ सौभ॑ग॒माय॑जस्व॥११॥
\anuvakamend[पर॑स्ता॒द्यशो॒ गुहा॑सु॒ मम॑ सुवर्णप॒क्षाय॑ धीमहि शतबा॒हुना पुन॑ पुनरजायत॒ सुवो॒ राजा॑ स॒धस्था॒त्त्रीणि॑ च। १।]



%६.२.१
भूर॒ग्नये॑ पृथि॒व्यै स्वाहा॒ भुवो॑ वा॒यवे॒ऽन्तरि॑क्षाय॒ स्वाहा॒ सुव॑रादि॒त्याय॑ दि॒वे स्वाहा॒ भूर्भुवः॒ सुव॑श्च॒न्द्रम॑से दि॒ग्भ्यः स्वाहा॒ नमो॑ दे॒वेभ्य॑ स्व॒धा पि॒तृभ्यो॒ भूर्भुवः॒ सुव॒रोम्॥१२॥
\anuvakamend


%६.३.१
भूरन्न॑म॒ग्नये॑ पृथि॒व्यै स्वाहा॒ भुवोऽन्नं॑ वा॒यवे॒ऽन्तरि॑क्षाय॒ स्वाहा॒ सुव॒रन्न॑मादि॒त्याय॑ दि॒वे स्वाहा॒ भूर्भुवः॒ सुव॒रन्नं॑ च॒न्द्रम॑से दि॒ग्भ्यः स्वाहा॒ नमो॑ दे॒वेभ्य॑ स्व॒धा पि॒तृभ्यो॒ भूर्भुवः॒ सुव॒रन्न॒मोम्॥१३॥
\anuvakamend


%६.४.१
भूर॒ग्नये॑ च पृथि॒व्यै च॑ मह॒ते च॒ स्वाहा॒ भुवो॑ वा॒यवे॑ चा॒न्तरि॑क्षाय च मह॒ते च॒ स्वाहा॒ सुव॑रादि॒त्याय॑ च दि॒वे च॑ मह॒ते च॒ स्वाहा॒ भूर्भुवः॒ सुव॑श्च॒न्द्रम॑से च॒ नक्ष॑त्रेभ्यश्च दि॒ग्भ्यश्च॑ मह॒ते च॒ स्वाहा॒ नमो॑ दे॒वेभ्य॑ स्व॒धा पि॒तृभ्यो॒ भूर्भुवः॒ सुव॒र्मह॒रोम्॥१४॥ %६.५.०
\anuvakamend


%६.५.१
पाहि नो अग्न एन॑से स्वा॒हा। पाहि नो विश्ववेद॑से स्वा॒हा। यज्ञं पाहि विभाव॑सो स्वा॒हा। सर्वं पाहि शतक्र॑तो स्वा॒हा॥१५॥
%६.६.०
\anuvakamend


%६.६.१
यश्छन्द॑सामृष॒भो वि॒श्वरू॑प॒श्छन्दोभ्य॒श्छन्दास्यावि॒वेश॑। सता शिक्यः पुरोवाचो॑पनि॒षदिन्द्रो ज्ये॒ष्ठ इ॑न्द्रि॒याय॒ ऋषि॑भ्यो॒ नमो॑ दे॒वेभ्य॑ स्व॒धा पि॒तृभ्यो॒ भूर्भुवः॒ सुव॒रोम्॥१६॥
%६.७.०
\anuvakamend


%६.७.१
नमो॒ ब्रह्म॑णे धा॒रणं॑ मे अ॒स्त्वनि॑राकरणन्धा॒रयि॑ता भूयासं॒ कर्ण॑योः  श्रु॒तं मा च्योढ्वं॒ ममा॒मुष्य॒ ओम्॥१७॥
%६.८.०
\anuvakamend


%६.८.१
ऋ॒तं तप॑ स॒त्यं तप॑ श्रु॒तं तप॑ शा॒न्तं तपो॒ दानं॒ तपो॒ यज्ञ॒स्तपो॒ भूर्भुवः॒ सुव॒र्ब्रह्मै॒तदुपास्यै॒तत्तप॑॥१८॥
\anuvakamend


%६.९.०

%६.९.१
यथा॑ वृ॒क्षस्य॑ सं॒पुष्पि॑तस्य दू॒राद्ग॒न्धो वात्ये॒वं पुण्य॑स्य क॒र्मणो॑ दू॒राद्ग॒न्धो वा॑ति॒ यथा॑ऽसिधा॒रां क॒र्तेऽव॑हितामव॒क्रामे॒द्यद्युवे॒ युवे॒ ह वा॑ वि॒ह्वदि॑ष्यामि क॒र्तं प॑तिष्या॒मीत्ये॒वम॒नृता॑दा॒त्मानं॑ जु॒गुप्सेत्॥१९॥

%६.१०.०
अजोन्यः सुव॒ नाभिः॒ सर्व॑म॒ष्टौ च॑॥ १०।
\anuvakamend


%६.१०.१
अ॒णोरणी॑यान्मह॒तो मही॑याना॒त्मा गुहा॑यां॒ निहि॑तोऽस्य ज॒न्तोः। तम॑क्रतुं पश्यति वीतशो॒को धा॒तुः प्र॒सादान्महि॒मान॑मीशम्। स॒प्त प्रा॒णाः प्र॒भव॑न्ति॒ तस्मात्स॒प्तार्चिष॑ स॒मिध॑ स॒प्त जि॒ह्वाः। स॒प्त इ॒मे लो॒का येषु॒ चर॑न्ति प्रा॒णा गु॒हाश॑यां॒ निहि॑ताः स॒प्तस॑प्त। अत॑ समु॒द्रा गि॒रय॑श्च॒ सर्वे॒ऽस्मात्स्यन्द॑न्ते॒ सिन्ध॑वः॒ सर्व॑रूपाः। अत॑श्च॒ विश्वा॒ ओष॑धयो॒ रसाश्च॒ येनै॑ष भू॒तस्ति॑ष्ठत्यन्तरा॒त्मा। ब्र॒ह्मा दे॒वानां पद॒वीः क॑वी॒नामृषि॒र्विप्रा॑णां महि॒षो मृ॒गाणाम्। श्ये॒नो गृध्रा॑णा॒ स्वधि॑ति॒र्वना॑ना॒ सोम॑ प॒वित्र॒मत्ये॑ति॒ रेभ\sn{}। अ॒जामेकाँ॒ल्लोहि॑तशुक्लकृ॒ष्णां ब॒ह्वीं प्र॒जां ज॒नय॑न्ती॒ सरू॑पाम्। अ॒जो ह्येको॑ जु॒षमा॑णोऽनु॒शेते॒ जहात्येनां भु॒क्तभो॑गा॒मजोऽन्यः॥२०॥

%६.१०.२
ह॒सः  शु॑चि॒षद्वसु॑रन्तरिक्ष॒सद्धोता॑ वेदि॒षदति॑थिर्दुरोण॒सत्। नृ॒षद्व॑र॒सदृ॑त॒सद्व्यो॑म॒सद॒ब्जा गो॒जा ऋ॑त॒जा अ॑द्रि॒जा ऋ॒तं बृ॒हत्। यस्माज्जा॒ता न प॒रा नैव॒ किं च॒नास॒ य आ॑वि॒वेश॒ भुव॑नानि॒ विश्वा। प्र॒जाप॑तिः प्र॒जया॑ संविदा॒नस्त्रीणि॒ ज्योतीषि सचते॒ स षो॑ड॒शी। वि॒ध॒र्तार हवामहे॒ वसो कु॒विद्व॒नाति॑ नः। स॒वि॒तार॑न्नृ॒चक्ष॑सम्। अ॒द्या नो॑ देव सवितः प्र॒जाव॑त्सावीः॒ सौभ॑गम्। परा॑ दु॒ष्वप्नि॑य सुव। विश्वा॑नि देव सवितर्दुरि॒तानि॒ परा॑ सुव। यद्भ॒द्रं तन्म॒ आ सु॑व॥२१॥

%६.१०.३
मधु॒ वाता॑ ऋताय॒ते मधु॑ क्षरन्ति॒ सिन्ध॑वः। माध्वीर्नः स॒न्त्वोष॑धीः। मधु॒ नक्त॑मु॒तोषसि॒ मधु॑म॒त्पार्थि॑व॒ रज॑। मधु॒ द्यौर॑स्तु नः पि॒ता। मधु॑मान्नो॒ वन॒स्पति॒र्मधु॑मा अस्तु॒ सूर्य॑। माध्वी॒र्गावो॑ भवन्तु नः। घृ॒तं मि॑मिक्षे घृ॒तम॑स्य॒ योनि॑र्घृ॒ते श्रि॒तो घृ॒तमु॑वस्य॒ धाम॑। अ॒नु॒ष्व॒धमाव॑ह मा॒दय॑स्व॒ स्वाहा॑कृतं॑ वृषभ वक्षि ह॒व्यम्। स॒मु॒द्रादू॒र्मिमधु॑मा॒ उदा॑रदुपा॒शुना॒ सम॑मृत॒त्वमा॑नट्। घृ॒तस्य॒ नाम॒ गुह्यं॒ यदस्ति॑ जि॒ह्वा दे॒वाना॑म॒मृत॑स्य॒ नाभि॑॥२२॥

%६.१०.४
व॒यं नाम॒ प्रब्र॑वामा घृ॒तेना॒स्मिन् य॒ज्ञे धा॑रयामा॒ नमो॑भिः। उप॑ ब्र॒ह्माशृ॑णवच्छ॒स्यमा॑नं॒ चतु॑ शृङ्गोवमीद्गौ॒र ए॒तत्। च॒त्वारि॒ शृङ्गा॒ त्रयो॑ अस्य॒ पादा॒ द्वे शी॒र्\mbox{}षे स॒प्त हस्ता॑सो अ॒स्य। त्रिधा॑ ब॒द्धो वृ॑ष॒भो रो॑रवीति म॒हो दे॒वो मर्त्या॒ आवि॑वेश। त्रिधा॑ हि॒तं प॒णिभि॑र्गु॒ह्यमा॑न॒ङ्गवि॑ दे॒वासो॑ घृ॒तमन्व॑विन्दन्। इन्द्र॒ एक॒ सूर्य॒ एकं॑ जजान वे॒नादेक स्व॒धया॒ निष्ट॑तक्षुः। यो दे॒वानां प्रथ॒मं पु॒रस्ता॒द्विश्वा॒धिको॑ रु॒द्रो म॒हर्\mbox{}षि॑। हि॒र॒ण्य॒ग॒र्भं प॑श्यत॒ जाय॑मान॒ स नो॑ दे॒वः  शु॒भया॒ स्मृत्या॒ संयु॑नक्तु। यस्मा॒त्परं॒ नाप॑र॒मस्ति॒ किञ्चि॒द्यस्मां॒ नाणी॑यो॒ न ज्यायोऽस्ति॒ कश्चि॑त्। वृ॒क्ष इ॑व स्तब्धो दि॒वि ति॑ष्ठ॒त्येक॒स्तेने॒दं पू॒र्णं पुरु॑षेण॒ सर्वम्॥२३॥

%६.१०.५
न कर्म॑णा न प्र॒जया॒ धने॑न॒ त्यागे॑नैके अमृत॒त्वमा॑न॒शुः। परे॑ण॒ नाकं॒ निहि॑त॒ङ्गुहा॑यां  वि॒भ्राज॑ते॒ यद्यत॑यो वि॒शन्ति॑। वे॒दा॒न्त॒वि॒ज्ञान॒सुनि॑श्चिता॒र्थाः सन्न्या॑सयो॒गाद्यत॑यः  शुद्ध॒सत्वा। ते ब्र॑ह्मलो॒के तु॒ परान्तकाले॒ परा॑मृता॒त्परि॑मुच्यन्ति॒ सर्वे। द॒ह्रं॒  वि॒पा॒पं प॒रमेश्मभूतं॒ यत्पु॑ण्डरी॒कं पु॒रम॑ध्यस॒स्थम्। त॒त्रा॒पि द॒ह्रङ्ग॒गनं॑  विशोक॒स्तस्मि॑न् यद॒न्तस्तदुपा॑सित॒व्यम्। यद्वेदादौ स्व॑रः प्रो॒क्तो॒ वे॒दान्ते॑ च प्र॒तिष्ठि॑तः। तस्य॑ प्र॒कृति॑लीन॒स्य॒ यः॒ पर॑ स म॒हेश्व॑रः॥२४॥

%६.११.०
ना॒रा॒य॒णः स्थि॑तो व्य॒वस्थि॑तश्च॒त्वारि॑ च॥ ११॥
\anuvakamend


%६.११.१
स॒ह॒स्र॒शीर्॑षं  दे॒वं॒  वि॒श्वाक्षं॑  वि॒श्वश॑म्भुवम्। विश्वं॑ ना॒राय॑णं दे॒व॒म॒क्षरं॑ पर॒मं प॒दम्। वि॒श्वतः॒ पर॑मान्नि॒त्यं॒  वि॒श्वं ना॑राय॒ण ह॑रिम्। विश्व॑मे॒वेदं पुरु॑ष॒स्तद्विश्व॒मुप॑जीवति। पतिं॒  विश्व॑स्या॒त्मेश्व॑र॒ शाश्व॑त शि॒वम॑च्युतम्। ना॒राय॒णं म॑हाज्ञे॒यं॒  वि॒श्वात्मा॑नं प॒राय॑णम्। ना॒राय॒णप॑रं ब्र॒ह्म॒ त॒त्वं ना॑राय॒णः प॑रः। ना॒राय॒णप॑रो ज्यो॒ति॒रा॒त्मा ना॑राय॒णः प॑रः। यच्च॑ किं॒ चिज्ज॑गत्य॒स्मि॒न्दृ॒श्यते श्रूय॒तेऽपि॑ वा। अन्त॑र्ब॒हिश्च॑ तत्स॒र्व॒व्व्याँ॒प्य ना॑राय॒णः स्थि॑तः॥२५॥

%६.११.२
अन॑न्त॒मव्य॑यं क॒वि स॑मु॒द्रेन्तं॑  वि॒श्वश॑म्भुवम्। प॒द्म॒को॒शप्र॑तीका॒श॒ हृ॒दयं॑ चाप्य॒धोमु॑खम्। अधो॑ नि॒ष्ट्या वि॑तस्त्या॒न्ते॒ ना॒भ्यामु॑परि॒ तिष्ठ॑ति। हृ॒दयं॑ तद्वि॑जानी॒या॒द्वि॒श्वस्या॑यत॒नं म॑हत्। सन्त॑त सि॒राभि॑स्तु॒ लम्ब॑त्याकोश॒सन्नि॑भम्। तस्यान्ते॑ सुषि॒र सू॒क्ष्मं तस्मिन्त्स॒र्वं प्रति॑ष्ठितम्। तस्य॒ मध्ये॑ म॒हान॑ग्निर्वि॒श्वार्चि॑र्वि॒श्वतो॑मुखः। सोऽग्र॑भु॒ग्विभ॑जन्ति॒ष्ठं॒ नाहा॑रमज॒रः क॒विः। स॒न्ता॒पय॑ति स्वन्दे॒हमापा॑दतल॒मस्त॑कम्। तस्य॒ मध्ये॒ वह्नि॑शिखा अ॒णीयोर्ध्वा व्य॒वस्थि॑तः। नी॒लतो॑यद॑मध्य॒स्था॒ वि॒द्युल्ले॑खेव॒ भास्व॑रा। नी॒वार॒शूक॑वत्त॒न्वी॒ पी॒ताभास्यात्त॒नूप॑मा। तस्या  शिखा॒या म॑ध्ये प॒रमात्मा व्य॒वस्थि॑तः। स ब्रह्मा॒ स शिवः॒ सेन्द्रः॒ सोऽक्ष॑रः पर॒मः स्व॒राट्॥२६॥
\anuvakamend


%६.१२.१
ऋ॒त स॒त्यं प॑रं ब्र॒ह्म॒ पु॒रुष॑ङ्कृष्ण॒पिङ्ग॑लम्। ऊ॒र्ध्वरे॑तं  वि॑रूपा॒क्षं॒  वि॒श्वरू॑पाय॒ वै नम॑॥२७॥%१२।
%६.१३.०
\anuvakamend


%६.१३.१
आ॒दि॒त्यो वा ए॒ष ए॒तन्म॒ण्डलं॒ तप॑ति॒ तत्र॒ ता ऋच॒स्तदृ॒चां म॒ण्डल॒ स ऋ॒चां लो॒कोऽथ॒ य ए॒ष ए॒तस्मि॑न्म॒ण्डले॒ऽर्चिर्दी॒प्यते॒ तानि॒ सामा॑नि॒ स सा॒म्नाम्म॒ण्डल॒ स सा॒म्नां लो॒कोऽथ॒ य ए॒ष ए॒तस्मि॑न्म॒ण्डले॒ऽर्चिषि॒ पुरु॑ष॒स्तानि॒ यजूषि॒ स यजु॑षां म॒ण्डल॒ स यजु॑षां लो॒कः सैषा त्र॒य्येव॑ वि॒द्या त॑पति॒ य ए॒षोऽन्तरा॑दि॒त्ये हि॑र॒ण्मयः॒ पुरु॑षः॥२८॥
%६.१४.०
\anuvakamend


%६.१४.१
आ॒दि॒त्यो वै तेज॒ ओजो॒ बलं॒ यश॒श्चक्षुः॒ श्रोत्र॑मा॒त्मा मनो॑ म॒न्युर्मनु॑र्मृ॒त्युः स॒त्यो मि॒त्रो वा॒युरा॑का॒शः प्रा॒णो लो॑कपा॒लः कः किं कं तत्स॒त्यमन्न॒मायु॑र॒मृतो॑ जी॒वो विश्व॑ कत॒मः स्व॑य॒म्भुः प्र॒जाप॑ति संवत्स॒र इति॑ संवत्स॒रो॑ऽसावा॑दि॒त्यो य ए॒ष पुरु॑ष ए॒ष भू॒ताना॒मधि॑पति॒र्ब्रह्म॑णः॒ सायु॑ज्य सलो॒कता॑माप्नोत्ये॒तासा॑मे॒व दे॒वता॑ना॒ सायु॑ज्य सा॒र्ष्टिता समानलो॒कता॑माप्नोति॒ य ए॒वं वेदेत्युप॒निषत्॥२९॥
%६.१५.०
\anuvakamend


%६.१५.१
घृणिः॒ सूर्य॑ आदि॒त्योम॑र्चयन्ति॒ तप॑ स॒त्यं मधु॑ क्षरन्ति॒ तद्ब्रह्म॒ तदाप॒ आपो॒ ज्योती॒रसो॒ऽमृतं॒ ब्रह्म॒ भूर्भु॑वः॒ सुव॒रोम्॥३०॥
%६.१६.०
%६.१६.०
\anuvakamend


%६.१६.१
सर्वो॒ वै रु॒द्रस्तस्मै॑ रु॒द्राय॒ नमो॑ अस्तु। पुरु॑षो॒ वै रु॒द्रः सन्म॒हो नमो॒ नम॑। विश्वं॑ भू॒तं भुव॑नं चि॒त्रं ब॑हु॒धा जा॒तं जाय॑मानं च॒ यत्। सर्वो॒ ह्ये॑ष रु॒द्रस्तस्मै॑ रु॒द्राय॒ नमो॑ अस्तु॥३१॥
\anuvakamend


%६.१७.१
कद्रु॒द्राय॒ प्रचे॑तसे मी॒ढुष्ट॑माय॒ तव्य॑से। वो॒चेम॒ शन्त॑म हृ॒दे। सर्वो॒ ह्ये॑ष रु॒द्रस्तस्मै॑ रु॒द्राय॒ नमो॑ अस्तु॥३२॥
%६.१८.०
\anuvakamend


%६.१८.१
नमो हिरण्यबाहवे हिरण्यपतयेऽम्बिकापतय उमापतये॑ नमो॒ नमः॥३३॥
%६.१९.०
\anuvakamend


%६.१९.१
यस्य॒ वैक॑ङ्कत्यग्निहोत्र॒हव॑णी भवति॒ प्रति॑ष्ठिताः॒ प्रत्ये॒वास्याहु॑तयस्तिष्ठ॒न्त्यथो॒ प्रति॑ष्ठित्यै॥३४॥

%६.२०.०
। २०।
\anuvakamend


%६.२०.१
कृ॒णु॒ष्व पाज॒ इति॒ पञ्च॑॥३५॥
%६.२१.०
\anuvakamend


%६.२१.१
अदि॑तिर्दे॒वा ग॑न्ध॒र्वा म॑नु॒ष्या पि॒तरोऽसु॑रा॒स्तेषा सर्वभू॒तानां मा॒ता मे॒दिनी॑ मह॒ती म॒ही सा॑वि॒त्री गा॑य॒त्री जग॑त्यु॒र्वी पृ॒थ्वी ब॑हु॒ला विश्वा॑ भू॒ता क॑त॒मा का या सा स॒त्येत्य॒मृतेति॑ वसि॒ष्ठः॥३६॥
%६.२२.०
\anuvakamend


%६.२२.१
आपो॒ वा इ॒द सर्वं॒  विश्वा॑ भू॒तान्याप॑ प्रा॒णा वा आप॑ प॒शव॒ आपो॒ऽमृत॒मापोऽन्न॒माप॑ स॒म्राडापो॑ वि॒राडाप॑ स्व॒राडाप॒श्छन्दा॒स्यापो॒ ज्योती॒ष्याप॑ स॒त्यमापः॒ सर्वा॑ दे॒वता॒ आपो॒ भूर्भुवः॒ सुव॒राप॒ ओम्॥३७॥
%६.२३.०
\anuvakamend


%६.२३.१
आप॑ पुनन्तु पृथि॒वीं पृ॑थि॒वी पू॒ता पु॑नातु॒ माम्। पु॒नन्तु॒ ब्रह्म॑ण॒स्पति॒र्ब्रह्म॑पू॒ता पु॑नातु॒ माम्। यदुच्छि॑ष्ट॒मभोज्यं॒ यद्वा॑ दु॒श्चरि॑तं॒ मम॑। सर्वं॑ पुनन्तु॒ मामापो॑ऽस॒तां च॑ प्रति॒ग्रह॒ स्वाहा॥३८€॥
%६.२४.०
\anuvakamend


%६.२४.१
अग्निश्च मा मन्युश्च मन्युपतयश्च मन्यु॑कृते॒भ्यः। पापेभ्यो॑ रक्ष॒न्ताम्। यदह्ना पाप॑मका॒र्\mbox{}षम्। मनसा वाचा॑ हस्ता॒भ्याम्। पद्भ्यामुदरे॑ण शि॒श्ञा। अह॒स्तद॑वलु॒म्पतु। यत्किं च॑ दुरि॒तं मयि॑। इदमहं माममृ॑तयो॒नौ। सत्ये ज्योतिषि जुहो॑मि स्वा॒हा॥३९॥
%६.२५.०
\anuvakamend


%६.२५.१
सूर्यश्च मा मन्युश्च मन्युपतयश्च मन्यु॑कृते॒भ्यः। पापेभ्यो॑ रक्ष॒न्ताम्। यद्रात्रिया पाप॑मका॒र्\mbox{}षम्। मनसा वाचा॑ हस्ता॒भ्याम्। पद्भ्यामुदरे॑ण शि॒श्ञा। रात्रि॒स्तद॑वलु॒म्पतु। यत्किं च॑ दुरि॒तं मयि॑। इदमहं माममृ॑तयो॒नौ। सूर्ये ज्योतिषि जुहो॑मि स्वा॒हा॥४०॥
%६.२६.०
\anuvakamend


%६.२६.१
आया॑तु॒ वर॑दा दे॒वी॒ अ॒क्षरं॑ ब्रह्म॒संमि॑तम्। गा॒य॒त्रीञ्छन्द॑सां मा॒तेदं ब्र॑ह्म जु॒षस्व॑ नः। ओजो॑ऽसि॒ सहो॑ऽसि॒ बल॑मसि॒ भ्राजो॑ऽसि दे॒वाना॒न्धाम॒ नामा॑सि॒ विश्व॑मसि वि॒श्वायुः॒ सर्व॑मसि स॒र्वायुरभिभूरोङ्गायत्रीमावा॑हया॒मि॥४१॥
%६.२७.०
\anuvakamend


%६.२७.१
ओं भूः। ओं भुव॑। ओ सुव॑। ओं मह॑। ओं जन॑। ओं तप॑। ओ स॒त्यम्। ओं तत्स॑वि॒तुर्वरेण्यं॒ भर्गो॑ दे॒वस्य॑ धीमहि। धियो॒ यो न॑ प्रचो॒दयात्। ओमापो॒ ज्योती॒रसो॒ऽमृतं॒ ब्रह्म॒ भूर्भुवः॒ सुव॒रोम्॥४२॥
%६.२८.०
। २८।
\anuvakamend


%६.२८.१
ओं भूर्भुवः॒ सुव॒र्मह॒र्जन॒स्तप॑ स॒त्यं तद्ब्रह्म॒ तदाप॒ आपो॒ ज्योती॒रसो॒ऽमृतं॒ ब्रह्म॒ भूर्भुवः॒ सुव॒रोम्॥४३॥
%६.२९.०
\anuvakamend


%६.२९.१
ओं तद्ब्र॒ह्म। ओं तद्वा॒युः। ओं तदा॒त्मा। ओं तत्सर्वम्। ओं तत्पुरो॒र्नम॑॥४४॥
%६.३०.०

%६.३०.१
उ॒त्तमे॑ शिख॑रे दे॒वी॒ भू॒म्यां प॑र्वत॒मूर्ध॑नि। ब्रा॒ह्म॒णेभ्यो ह्य॑नुज्ञा॒नं॒ गच्छ दे॑वि य॒थासु॑खम्॥४५॥
\anuvakamend


%६.३१.०
। ३१।

%६.३१.१
ओमन्तश्चरति॑ भूते॒षु॒ गुहायां वि॑श्वमू॒र्तिषु। त्वं यज्ञस्त्वं  विष्णुस्त्वं व॑षट्का॒र॒स्त्व रुद्रस्त्वं ब्रह्मा त्वं॑ प्रजा॒पति॑॥४६॥
%६.३२.०
\anuvakamend


%६.३२.१
अ॒मृ॒तो॒प॒स्तर॑णमसि॥४७॥
%६.३३.०
\anuvakamend


%६.३३.१
प्रा॒णे निवि॑ष्टो॒ऽमृतं॑ जुहोमि। प्रा॒णाय॒ स्वाहा। अ॒पा॒ने निवि॑ष्टो॒ऽमृतं॑ जुहोमि। अ॒पा॒नाय॒ स्वाहा। व्या॒ने निवि॑ष्टो॒ऽमृतं॑ जुहोमि। व्यानाय॒ स्वाहा। उ॒दा॒ने निवि॑ष्टो॒ऽमृतं॑ जुहोमि। उ॒दा॒नाय॒ स्वाहा। स॒मा॒ने निवि॑ष्टो॒ऽमृतं॑ जुहोमि। स॒मा॒नाय॒ स्वाहा। ब्रह्म॑णि म आ॒त्माऽमृ॑त॒त्वाय॑॥४८॥
\anuvakamend


%६.३४.१
प्रा॒णे निवि॑ष्टो॒ऽमृतं॑ जुहोमि शि॒वो मा॑ वि॒शा प्र॑दाहाय प्रा॒णाय॒ स्वाहा। अ॒पा॒ने निवि॑ष्टो॒ऽमृतं॑ जुहोमि शि॒वो मा॑ वि॒शा प्र॑दाहायापा॒नाय॒ स्वाहा। व्या॒ने निवि॑ष्टो॒ऽमृतं॑ जुहोमि शि॒वो मा॑ वि॒शा प्र॑दाहाय व्या॒नाय॒ स्वाहा। उ॒दा॒ने निवि॑ष्टो॒ऽमृतं॑ जुहोमि शि॒वो मा॑ वि॒शा प्र॑दाहायोदा॒नाय॒ स्वाहा। स॒मा॒ने निवि॑ष्टो॒ऽमृतं॑ जुहोमि शि॒वो मा॑ वि॒शा प्र॑दाहाय समा॒नाय॒ स्वाहा। ब्रह्म॑णि म आ॒त्माऽमृ॑त॒त्वाय॑॥४९॥
%६.३५.०
\anuvakamend


%६.३५.१
अ॒मृ॒ता॒पि॒धा॒नम॑सि॥५०॥
%६.३६.०
\anuvakamend


%६.३६.१
श्र॒द्धायां प्रा॒णे निवि॑श्या॒मृत हु॒तम्प्रा॒णमन्ने॑नाप्यायस्व। अ॒पा॒ने निवि॑श्या॒मृत हु॒तम॑पा॒नमन्ने॑नाप्यायस्व। व्या॒ने निवि॑श्या॒मृत हु॒तव्व्याँ॒नमन्ने॑नाप्यायस्व। उ॒दा॒ने निवि॑श्या॒मृत हु॒तमु॑दा॒नमन्ने॑नाप्यायस्व। स॒मा॒ने निवि॑श्या॒मृत हु॒त स॑मा॒नमन्ने॑नाप्यायस्व। ब्रह्म॑णि म आ॒त्माऽमृ॑त॒त्वाय॑॥५१॥
%६.३७.०
\anuvakamend


%६.३७.१
प्राणानाङ्ग्रन्थिरसि रुद्रो मा॑ऽऽविशा॒न्तकस्तेनान्नेनाप्याय॒स्व॥५२॥
%६.३८.०
\anuvakamend


%६.३८.१
अङ्गुष्ठमात्रः पुरुषोऽङ्गुष्ठं च॑ समा॒श्रितः। ईशः सर्वस्य जगतः प्रभुः प्रीणाति॑ विश्व॒भुक्॥५३॥
%६.३९.०
\anuvakamend


%६.३९.१
मे॒धा दे॒वी जु॒षमा॑णा न॒ आगाद्वि॒श्वाची॑ भ॒द्रा सु॑मन॒स्यमा॑ना। त्वया॒ जुष्टा॑ जु॒षमा॑णा दु॒रुक्तान्बृ॒हद्व॑देम वि॒दथे॑ सु॒वीरा॥ त्वया॒ जुष्ट॑ ऋ॒षिर्भ॑वति देवि॒ त्वया॒ ब्रह्मा॑ऽऽग॒तश्री॑रु॒त त्वया। त्वया॒ जुष्ट॑श्चि॒त्रं  वि॑न्दते वसु॒ सा नो॑ जुषस्व॒ द्रवि॑णो न मेधे॥५४॥
%६.४०.०
\anuvakamend


%६.४०.१
मे॒धां म॒ इन्द्रो॑ ददातु मे॒धान्दे॒वी सर॑स्वती। मे॒धां मे॑ अ॒श्विनौ॑ दे॒वावाध॑त्तां॒ पुष्क॑रस्रजा॥५५॥
%६.४१.०
\anuvakamend


%६.४१.१
अ॒प्स॒रासु॑ च॒ या मे॒धा ग॑न्ध॒र्वेषु॑ च॒ यन्मन॑। दैवी॑ मे॒धा म॑नुष्य॒जा सा मां मे॒धा सु॒रभि॑र्जुषताम्॥५६॥
%६.४२.०
\anuvakamend


%६.४२.१
आ मां मे॒धा सु॒रभि॑र्वि॒श्वरू॑पा॒ हिर॑ण्यवर्णा॒ जग॑ती जग॒म्या। ऊर्ज॑स्वती॒ पय॑सा॒ पिन्व॑माना॒ सा मां मे॒धा सु॒प्रती॑का जुषताम्॥५७॥
%६.४३.०
\anuvakamend


%६.४३.१
स॒द्योजा॒तं प्र॑पद्या॒मि॒ स॒द्योजा॒ताय॒ वै नम॑। भ॒वेभ॑वे॒ नाति॑भवे भजस्व॒ मां भ॒वोद्भ॑वाय॒ नम॑॥५८॥
%६.४४.०
\anuvakamend


%६.४४.१
वा॒म॒दे॒वाय॒ नमो ज्ये॒ष्ठाय॒ नमो॑ रु॒द्राय॒ नमः॒ काला॑य॒ नमः॒ कल॑विकरणाय॒ नमो॒ बल॑विकरणाय॒ नमो॒ बल॑प्रमथनाय॒ नमः॒ सर्व॑भूतदमनाय॒ नमो॑ म॒नोन्म॑नाय॒ नम॑॥५९॥
%६.४५.०
\anuvakamend


%६.४५.१
अ॒घोरेभ्योऽथ॒ घोरेभ्यो॒ घोर॒घोर॑तरेभ्यः स॒र्वत॑ शर्व॒ सर्वेभ्यो॒ नम॑स्ते अस्तु रु॒द्ररू॑पेभ्यः॥६०॥
%६.४६.०
\anuvakamend


%६.४६.१
तत्पुरु॑षाय वि॒द्महे॑ महादे॒वाय॑ धीमहि। तन्नो॑ रुद्रः प्रचो॒दयात्॥६१॥

%६.४७.०
। ४७।
\anuvakamend


%६.४७.१
ईशानः सर्व॑विद्या॒ना॒मीश्वरः सर्व॑भूता॒नां॒ ब्रह्माधि॑पति॒र्ब्रह्म॒णोऽधि॑पति॒र्ब्रह्मा॑ शि॒वो मे॑ अस्तु सदाशि॒वोम्॥६२॥

%६.४८.०

%६.४८.१
ब्रह्म॑मेतु॒ माम्। मधु॑मेतु॒ माम्। ब्रह्म॑मे॒व मधु॑मेतु॒ माम्। यास्ते॑ सोम प्र॒जाव॒त्सोभि॒ सो अ॒हम्। दुस्व॑प्न॒हन्दु॑रुष्व॒हा। यास्ते॑ सोम प्रा॒णास्तां जु॑होमि। त्रिसु॑पर्ण॒मया॑चितं ब्राह्म॒णाय॑ दद्यात्। ब्र॒ह्म॒ह॒त्यां वा ए॒ते घ्न॑न्ति। ये ब्राह्म॒णास्त्रिसु॑पर्णं॒ पठ॑न्ति। ते सोमं॒ प्राप्नु॑वन्ति। आ॒स॒ह॒स्रात्प॒ङ्क्तिं पुन॑न्ति। ओम्॥६३॥

%६.४९.०
। ४९।
\anuvakamend


%६.४९.१
ब्रह्म॑ मे॒धया। मधु॑ मे॒धया। ब्रह्म॑मे॒व मधु॑ मे॒धया। अ॒द्या नो॑ देव सवितः प्र॒जाव॑त्सावीः॒ सौभ॑गम्। परा॑ दु॒ष्वप्नि॑य सुव। विश्वा॑नि देव सवितर्दुरि॒तानि॒ परा॑ सुव। यद्भ॒द्रं तन्म॒ आ सु॑व। मधु॒ वाता॑ ऋताय॒ते मधु॑ क्षरन्ति॒ सिन्ध॑वः। माध्वीर्नः स॒न्त्वोष॑धीः। मधु॒ नक्त॑मु॒तोषसि॒ मधु॑म॒त्पार्थि॑व॒ रज॑। मधु॒ द्यौर॑स्तु नः पि॒ता। मधु॑मान्नो॒ वन॒स्पति॒र्मधु॑मा अस्तु॒ सूर्य॑। माध्वी॒र्गावो॑ भवन्तु नः। य इ॒मं त्रिसु॑पर्ण॒मया॑चितं ब्राह्म॒णाय॑ दद्यात्। भ्रू॒ण॒ह॒त्यां वा ए॒ते घ्न॑न्ति। ये ब्राह्म॒णास्त्रिसु॑पर्णं॒ पठ॑न्ति। ते सोमं॒ प्राप्नु॑वन्ति। आ॒स॒ह॒स्रात्प॒ङ्क्तिं पुन॑न्ति। ओम्॥६४॥
%६.५०.०
\anuvakamend


%६.५०.१
ब्रह्म॑ मे॒धवा। मधु॑ मे॒धवा। ब्रह्म॑मे॒व मधु॑ मे॒धवा। ब्र॒ह्मा दे॒वानां पद॒वीः क॑वी॒नामृषि॒र्विप्रा॑णां महि॒षो मृ॒गाणाम्। श्ये॒नो गृध्रा॑णा॒ स्वधि॑ति॒र्वना॑ना॒ सोम॑ प॒वित्र॒मत्ये॑ति॒ रेभ\sn{}। ह॒सः  शु॑चि॒षद्वसु॑रन्तरिक्ष॒सद्धोता॑ वेदि॒षदति॑थिर्दुरोण॒सत्। नृ॒षद्व॑र॒सदृ॑त॒सद्व्यो॑म॒सद॒ब्जा गो॒जा ऋ॑त॒जा अ॑द्रि॒जा ऋ॒तं बृ॒हत्। य इ॒मं त्रिसु॑पर्ण॒मया॑चितं ब्राह्म॒णाय॑ दद्यात्। वी॒र॒ह॒त्यां वा ए॒ते घ्न॑न्ति। ये ब्राह्म॒णास्त्रिसु॑पर्णं॒ पठ॑न्ति। ते सोमं॒ प्राप्नु॑वन्ति। आ॒स॒ह॒स्रात्प॒ङ्क्तिं पुन॑न्ति। ओम्॥६५॥

%६.५१.०
। ५१।
\anuvakamend


%६.५१.१
प्राणापानव्यानोदानसमाना मे॑ शुद्ध्य॒न्तां॒ ज्योति॑र॒हं  वि॒रजा॑ विपा॒प्मा भू॑यास॒ स्वाहा॥६६॥
%६.५२.०
\anuvakamend


%६.५२.१
वाङ्मनश्चक्षुःश्रोत्रजिह्वाघ्राणरेतोबुध्याकूतिसङ्कल्पा मे॑ शुद्ध्य॒न्तां॒ ज्योति॑र॒हं  वि॒रजा॑ विपा॒प्मा भू॑यास॒ स्वाहा॥६७॥
%६.५३.०
\anuvakamend


%६.५३.१
शिरःपाणिपादपार्श्वपृष्ठोदरजङ्घशिश्ञोपस्थपायवो मे॑ शुद्ध्य॒न्तां॒ ज्योति॑र॒हं  वि॒रजा॑ विपा॒प्मा भू॑यास॒ स्वाहा॥६८॥
%६.५४.०
\anuvakamend


%६.५४.१
त्वक्चर्ममासरुधिरमेदोऽस्थिमज्जा मे॑ शुद्ध्य॒न्तां॒ ज्योति॑र॒हं  वि॒रजा॑ विपा॒प्मा भू॑यास॒ स्वाहा॥६९॥
%६.५५.०
\anuvakamend


%६.५५.१
शब्दस्पर्शरूपरसगन्धा मे॑ शुद्ध्य॒न्तां॒ ज्योति॑र॒हं  वि॒रजा॑ विपा॒प्मा भू॑यास॒ स्वाहा॥७०॥
%६.५६.०
\anuvakamend


%६.५६.१
पृथिव्यप्तेजोवाय्वाकाशा मे॑ शुद्ध्य॒न्तां॒ ज्योति॑र॒हं  वि॒रजा॑ विपा॒प्मा भू॑यास॒ स्वाहा॥७१॥
%६.५७.०
\anuvakamend


%६.५७.१
अन्नमयप्राणमयमनोमयविज्ञानमयानन्दमया मे॑ शुद्ध्य॒न्तां॒ ज्योति॑र॒हं  वि॒रजा॑ विपा॒प्मा भू॑यास॒ स्वाहा॥७२॥
%६.५८.०
\anuvakamend


%६.५८.१
विवि॑ट्टि॒ स्वाहा॥७३॥
%६.५९.०
\anuvakamend


%६.५९.१
घ॒षोत्काय॒ स्वाहा॥७४॥

%६.६०.०
। ६०।
\anuvakamend


%६.६०.१
उत्तिष्ठ पुरुषा हरी लोहितपिङ्गलाक्षि देहि देहि ददापयिता मे॑ शुद्ध्य॒न्तां॒ ज्योति॑र॒हं  वि॒रजा॑ विपा॒प्मा भू॑यास॒ स्वाहा॥७५॥
%६.६१.०
\anuvakamend


%६.६१.१
ओ स्वाहा॥७६॥
%६.६२.०
\anuvakamend


%६.६२.१
स॒त्यं परं॒ पर स॒त्य स॒त्येन॒ न सु॑व॒र्गाल्लो॒काच्च्य॑वन्ते क॒दाच॒न स॒ता हि स॒त्यं तस्मात्स॒त्ये र॑मन्ते॒ तप॒ इति॒ तपो॒ नानश॑ना॒त्परं॒ यद्धि परं॒ तप॒स्तद्दुर्द्ध॑र्\mbox{}षं॒ तद्दुरा॑धर्\mbox{}षं॒ तस्मा॒त्तप॑सि रमन्ते॒ दम॒ इति॒ निय॑तं ब्रह्मचा॒रिण॒स्तस्मा॒द्दमे॑ रमन्ते॒ शम॒ इत्यर॑ण्ये मु॒नय॒स्तस्मा॒च्छमे॑ रमन्ते दा॒नमिति॒ सर्वा॑णि भू॒तानि॑ प्र॒शस॑न्ति दा॒नान्नाति॑ दु॒ष्करं॒ तस्माद्दा॒ने र॑मन्ते ध॒र्म इति॒ धर्मे॑ण॒ सर्व॑मि॒दं परि॑गृहीतं ध॒र्मान्नाति॑ दु॒श्चरं॒ तस्माद्ध॒र्मे र॑मन्ते प्र॒जन॒ इति॒ भूयास॒स्तस्मा॒द्भूयि॑ष्ठाः॒ प्रजा॑यन्ते॒ तस्मा॒द्भूयि॑ष्ठाः प्र॒जन॑ने रमन्ते॒ऽग्नय॒ इत्या॑ह॒ तस्मा॑द॒ग्नय॒ आधा॑तव्या अग्निहो॒त्रमित्या॑ह॒ तस्मा॑दग्निहो॒त्रे र॑मन्ते य॒ज्ञ इति॑ य॒ज्ञो हि दे॒वानां य॒ज्ञेन॒ हि दे॒वा दिव॑ङ्ग॒तास्तस्माद्य॒ज्ञे र॑मन्ते मान॒समिति॑ वि॒द्वास॒स्तस्मा॑द्वि॒द्वास॑ ए॒व मा॑न॒से र॑मन्ते न्या॒स इति॑ ब्र॒ह्मा ब्र॒ह्मा हि परः॒ परो॑ हि ब्र॒ह्मा तानि॒ वा ए॒तान्यव॑राणि॒ तपासि न्या॒स ए॒वात्य॑रेचय॒द्य ए॒वं वेदेत्युप॒निषत्॥७७॥
%६.६३.०
\anuvakamend


%६.६३.१
प्रा॒जा॒प॒त्यो हारु॑णिः सुप॒र्णेय॑ प्र॒जाप॑तिं पि॒तर॒मुप॑ससार॒ किं भ॑गव॒न्तः प॑र॒मं व॑द॒न्तीति॒ तस्मै॒ प्रो॑वाच स॒त्येन॑ वा॒युरावा॑ति स॒त्येना॑दि॒त्यो रो॑चते दि॒वि स॒त्यं वा॒चः प्र॑ति॒ष्ठा स॒त्ये स॒र्वं प्रति॑ष्ठितं॒ तस्मात्स॒त्यं प॑र॒मं वद॑न्ति॒ तप॑सा दे॒वा दे॒वता॒मग्र॑ आय॒न्तप॒सर्\mbox{}ष॑यः॒ सुव॒रन्व॑विन्दं॒ तप॑सा स॒पत्ना॒प्रणु॑दा॒मारा॑ती॒स्तप॑सि स॒र्वं प्रति॑ष्ठितं॒ तस्मा॒त्तप॑ पर॒मं वद॑न्ति॒ दमे॑न दा॒न्ताः कि॒ल्बिष॑मवधू॒न्वन्ति॒ दमे॑न ब्रह्मचा॒रिणः॒ सुव॑रगच्छ॒न्दमो॑ भू॒तानान्दुरा॒धर्\mbox{}ष॒न्दमे॑ स॒र्वं प्रति॑ष्ठितं॒ तस्मा॒द्दम॑ पर॒मं वद॑न्ति॒ शमे॑न शा॒न्ताः  शि॒वमा॒चर॑न्ति॒ शमे॑न ना॒कं मु॒नयो॒ऽन्ववि॑न्द॒ञ्छमो॑ भू॒तानान्दुरा॒धर्\mbox{}ष॒ञ्छमे॑ स॒र्वं प्रति॑ष्ठितं॒ तस्मा॒च्छम॑ पर॒मं वद॑न्ति दा॒नं य॒ज्ञानां॒ वरू॑थ॒न्दक्षि॑णा लो॒के दा॒तार सर्वभू॒तान्यु॑पजी॒वन्ति॑ दा॒नेनारा॑ती॒रपा॑नुदन्त दा॒नेन॑ द्विष॒न्तो मि॒त्रा भ॑वन्ति दा॒ने स॒र्वं प्रति॑ष्ठितं॒ तस्माद्दा॒नं प॑र॒मं वद॑न्ति ध॒र्मो विश्व॑स्य॒ जग॑तः प्रति॒ष्ठा लो॒के ध॒र्मिष्ठं॑ प्र॒जा उ॑पस॒र्पन्ति॑ ध॒र्मेण॑ पा॒पम॑प॒नुद॑ति ध॒र्मे स॒र्वं प्रति॑ष्ठितं॒ तस्माद्ध॒र्मं प॑र॒मं वद॑न्ति प्र॒जन॑नं॒ वै प्र॑ति॒ष्ठा लो॒के सा॒धु प्र॒जायास्त॒न्तुं त॑न्वा॒नः पि॑तृ॒णाम॑नृ॒णो भव॑ति॒ तदे॑व त॒स्यानृ॑णं॒ तस्मात् प्र॒जन॑नं पर॒मं वद॑न्त्य॒ग्नयो॒ वै त्रयी॑ वि॒द्या दे॑व॒यानः॒ पन्था॑ गार्\mbox{}हप॒त्य ऋक्पृ॑थि॒वी र॑थन्त॒रम॑न्वाहार्य॒पच॑नो॒ यजु॑र॒न्तरि॑क्षं वामदे॒व्यमा॑हव॒नीयः॒ साम॑ सुव॒र्गो लो॒को बृ॒हत्तस्मा॑द॒ग्नीन्प॑र॒मं वद॑न्त्यग्निहो॒त्र सा॑यं प्रा॒तर्गृ॒हाणां॒ निष्कृ॑तिः॒ स्वि॑ष्ट सुहु॒तं य॑ज्ञक्रतू॒नां प्राय॑ण सुव॒र्गस्य॑ लो॒कस्य॒ ज्योति॒स्तस्मा॑दग्निहो॒त्रं प॑र॒मं वद॑न्ति य॒ज्ञ इति॑ य॒ज्ञो हि दे॒वानां य॒ज्ञेन॒ हि दे॒वा दिव॑ङ्ग॒ता य॒ज्ञेनासु॑रा॒नपा॑नुदन्त य॒ज्ञेन॑ द्विष॒न्तो मि॒त्रा भ॑वन्ति य॒ज्ञे स॒र्वं प्रति॑ष्ठितं॒ तस्माद्य॒ज्ञं प॑र॒मं वद॑न्ति मान॒सं वै प्रा॑जाप॒त्यं प॒वित्रं॑ मान॒सेन॒ मन॑सा सा॒धु प॑श्यति मान॒सा ऋष॑यः प्र॒जा अ॑सृजन्त मान॒से स॒र्वं प्रति॑ष्ठितं॒ तस्मान्मान॒सं प॑र॒मं वद॑न्ति न्या॒स इ॒त्याहु॑र्मनी॒षिणो ब्र॒ह्माणं॑ ब्र॒ह्मा विश्व॑ कत॒मः स्व॑य॒म्भुः प्र॒जाप॑तिः संवत्स॒र इति॑ संवत्स॒रो॑ऽसावा॑दि॒त्यो य ए॒ष आ॑दि॒त्ये पुरु॑षः॒ स प॑रमे॒ष्ठी ब्रह्मा॒त्मा याभि॑रादि॒त्यस्तप॑ति र॒श्मिभि॒स्ताभि॑ प॒र्जन्यो॑ वर्\mbox{}षति प॒र्जन्ये॑नौषधिवनस्प॒तयः॒ प्रजा॑यन्त ओषधिवनस्प॒तिभि॒रन्नं॑ भव॒त्यन्ने॑न प्रा॒णाः प्रा॒णैर्बलं॒ बले॑न॒ तप॒स्तप॑सा श्र॒द्धा श्र॒द्धया॑ मे॒धा मे॒धया॑ मनी॒षा म॑नी॒षया॒ मनो॒ मन॑सा॒ शान्तिः॒ शान्त्या॑ चि॒त्तं चि॒त्तेन॒ स्मृति॒ स्मृत्या॒ स्मार॒ स्मारे॑ण वि॒ज्ञानं॑  वि॒ज्ञाने॑ना॒त्मानं॑ वेदयति॒ तस्मा॑द॒न्नं दद॒न्त्सर्वाण्ये॒तानि॑ ददा॒त्यन्नात् प्रा॒णा भ॑वन्ति भू॒तानां प्रा॒णैर्मनो॒ मन॑सश्च वि॒ज्ञानं॑  वि॒ज्ञाना॑दान॒न्दो ब्र॑ह्मयो॒निः स वा ए॒ष पुरु॑षः पञ्च॒धा प॑ञ्चा॒त्मा येन॒ सर्व॑मि॒दं प्रोतं॑ पृथि॒वी चा॒न्तरि॑क्षं च॒ द्यौश्च॒ दिश॑श्चावान्तरदि॒शाश्च॒ स वै सर्व॑मि॒दं जग॒त्स च॒ भूत स भ॒व्यं जि॑ज्ञासकॢ॒प्त ऋ॑त॒जा रयि॑ष्ठाः  श्र॒द्धा स॒त्यो मह॑स्वान्त॒मसो॒परि॑ष्टा॒द्ज्ञात्वा॑ तमे॒वं मन॑सा हृ॒दा च॒ भूयो॑ न मृ॒त्युमुप॑याहि वि॒द्वान्तस्मान्न्या॒समे॒षां तप॑सामतिरिक्त॒माहु॑र्वसुर॒ण्यो॑ वि॒भूर॑सि प्रा॒णे त्वमसि॑ सन्धा॒ता ब्रह्मं॑ त्वम॑सि विश्व॒सृक्ते॑जो॒दास्त्वम॑स्य॒ग्नेर्व॑र्चो॒दास्त्वम॑सि॒ सूर्य॑स्य द्युम्नो॒दास्त्वम॑सि च॒न्द्रम॑स उपया॒मगृ॑हीतोऽसि ब्र॒ह्मणे त्वा॒ महस॒ ओमित्या॒त्मानं॑ युञ्जीतै॒तद्वै म॑होप॒निष॑दन्दे॒वाना॒ङ्गुह्यं॒ य ए॒वं वेद॑ ब्र॒ह्मणो॑ महि॒मान॑माप्नोति॒ तस्माद्ब्र॒ह्मणो॑ महि॒मान॑मित्युप॒निष॑त्॥७८॥
\anuvakamend


%६.६४.१
तस्यै॒वं  वि॒दुषो॑ य॒ज्ञस्या॒त्मा यज॑मानः श्र॒द्धा पत्नी॒ शरी॑रमि॒ध्ममुरो॒ वेदि॒र्लोमा॑नि ब॒\ar{}हिर्वे॒दः  शिखा॒ हृद॑यं॒ यूपः॒ काम॒ आज्यं॑ म॒न्युः प॒शुस्तपो॒ऽग्निः  श॑मयि॒ता दक्षि॑णा॒ वाग्घोता प्रा॒ण उ॑द्गा॒ता चक्षु॑रध्व॒र्युर्मनो॒ ब्रह्मा॒ श्रोत्र॑म॒ग्नीद्याव॒द्ध्रिय॑ते॒ सा दी॒क्षा यदश्ञा॑ति॒ यत्पिब॑ति॒ तद॑स्य सोमपा॒नं यद्रम॑ते॒ तदु॑प॒सदो॒ यत्स॒ञ्चर॑त्युप॒विश॑त्यु॒त्तिष्ठ॑ते च॒ स प्र॑व॒र्ग्यो॑ यन्मुखं॒ तदा॑हव॒नीयो॒ यद॑स्य वि॒ज्ञानं॒ तज्जु॒होति॒ यत्सा॒यं प्रा॒तर॑त्ति॒ तत्स॒मिधो॒ यत्सा॒यं प्रा॒तर्म॒ध्यन्दि॑नं च॒ तानि॒ सव॑नानि॒ ये अ॑होरा॒त्रे ते द॑र्\mbox{}शपूर्णमा॒सौ येऽर्द्धमा॒साश्च॒ मासाश्च॒ ते चा॑तुर्मा॒स्यानि॒ य ऋ॒तव॒स्ते प॑शुब॒न्धा ये सं॑वत्स॒राश्च॑ परिवत्स॒राश्च॒ तेऽह॑र्ग॒णाः स॑र्ववेद॒सं वा ए॒तत्स॒त्रं यन्मर॑णं॒ तद॑व॒भृथ॑ ए॒तद्वै ज॑रामर्यमग्निहो॒त्र स॒त्रं य ए॒वं  वि॒द्वानु॑द॒गय॑ने प्र॒मीय॑ते दे॒वाना॑मे॒व म॑हि॒मान॑ङ्ग॒त्वाऽऽदि॒त्यस्य॒ सायु॑ज्यं गच्छ॒त्यथ॒ यो द॑क्षि॒णे प्र॒मीय॑ते पितृ॒णामे॒व म॑हि॒मान॑ङ्ग॒त्वा च॒न्द्रम॑सः॒ सायु॑ज्यं गच्छत्ये॒तौ वै सूर्याचन्द्र॒मसोर्महि॒मानौ ब्राह्म॒णो वि॒द्वान॒भिज॑यति॒ तस्माद्ब्र॒ह्मणो॑ महि॒मान॑माप्नोति॒ तस्माद्ब्र॒ह्मणो॑ महि॒मान॑मित्युप॒निष॑त्॥७९॥
\anuvakamend

%अम्भ॑सि॒ भूर॒ग्नये॒ भूरन्नं॒ भूर॒ग्नये॑ च॒ पाहि नो यश्छन्द॑सां॒ नमो॒ ब्रह्म॑ण ऋ॒तं तपो॒ यथा॑ वृ॒क्षस्या॒णोरणी॑यान्त्सहस्र॒शीर्\mbox{}ष॑मृ॒तमा॑दि॒त्यो वा ए॒ष आ॑दि॒त्यो वै तेज॒ ओजो॒ घृणिः॒ सर्वो॒ वै कद्रु॒द्राय॒ नमो हिरण्यबाहवे यस्य॒ वैक॑ङ्कती कृणु॒ष्व पाजोऽदि॑ति॒रापो॒ वा इद सर्व॒माप॑ पुन॒न्त्वग्निश्च सूर्यश्चाया॒त्वों भूरों भूर्भुवः॒ सुव॒रोन्तदु॒त्तम॒ ओमन्तश्चरत्य॑मृतोप॒स्तर॑णमसि प्रा॒णे निवि॑ष्टः प्रा॒णे नि॑विष्टः  शि॒वो॑ऽमृतापिधा॒नम॑सि श्र॒द्धायां प्रा॒णे निवि॑श्य॒ प्राणानामङ्गुष्ठमात्रो मे॒धा दे॒वी मे॒धां म॒ इन्द्रो॑ ददात्वप्स॒रास्वामां मे॒धा स॒द्यो वा॑मदे॒वाया॒घोरेभ्य॒स्तत्पुरु॑षा॒येशानो ब्रह्म॑ मेतु॒ ब्रह्म॑ मे॒धया॒ ब्रह्म॑ मे॒धवा॒ प्राणापानवाङ्मनः  शिरःपाणित्वक्चर्मशब्दस्पर्शपृथिव्यन्नमय विवि॑ट्टि घ॒षोक्ता॒योत्तिष्ठो स॒त्यं परं॑ प्राजाप॒त्यस्तस्यै॒वञ्चतु॑षष्टिः॥

%अम्भ॑सि॒ वृषा॑ ह॒सः सर्वो॒ वा आया॑तु श्र॒द्धायां॒ तत्पुरु॑षाय॒ पृथिव्यप्तेजो नव॑सप्ततिः॥ ७९। अम्भ॒सीत्यु॑प॒निष॑त्॥

\centerline{॥ॐ शान्ति॒ शान्ति॒ शान्ति॑॥  हरि॑ ओम्॥}

\closesection
\clearpage
