% !TeX program = XeLaTeX
% !TeX root = ../AraNyakabook-kindle.tex

%  
%  
\sect{द्वितीयः प्रश्नः}\setcounter{anuvakam}{0}
ॐ नमो॒ ब्रह्म॑णे॒ नमो॑ अस्त्व॒ग्नये॒ नम॑ पृथि॒व्यै नम॒ ओष॑धीभ्यः। 
नमो॑ वा॒चे नमो॑ वा॒चस्पत॑ये॒ नमो॒ विष्ण॑वे बृह॒ते क॑रोमि॥
ॐ शान्ति॒ शान्ति॒ शान्ति॑॥

सह॒ वै दे॒वानां॒ चासु॑राणां च य॒ज्ञौ प्रत॑तावास्तां व॒य स्व॒र्गं लो॒कमेष्यामो व॒यमेष्याम॒ इति॒ तेऽसु॑राः स॒न्नह्य॒ सह॑सै॒वाच॑रन् ब्रह्म॒चर्ये॑ण॒ तप॑सैव दे॒वास्तेऽसु॑रा अमुह्य॒स्ते न प्राजा॑न॒स्ते परा॑ऽभव॒न्ते न स्व॒र्गं लो॒कमा॑य॒न् प्रसृ॑तेन॒ वै य॒ज्ञेन॑ दे॒वाः स्व॒र्गं लो॒कमा॑य॒न्न प्रसृ॑ते॒नासु॑रा॒न् परा॑भावय॒न् प्रसृ॑तो ह॒ वै य॑ज्ञोपवी॒तिनो॑ य॒ज्ञोऽप्र॑सृ॒तोऽनु॑पवी॒तिनो॒ यत्किं च॑ ब्राह्म॒णो य॑ज्ञोपवी॒त्यधी॑ते॒ यज॑त ए॒व तत्तस्माद्यज्ञोपवी॒त्ये॑वाधी॑यीत या॒जये॒द्यजे॑त वा य॒ज्ञस्य॒ प्रसृ॑त्या॒ अजि॑नं॒ वासो॑ वा दक्षिण॒त उ॑प॒वीय॒ दक्षि॑णं बा॒हुमुद्ध॑र॒तेऽव॑ धत्ते स॒व्यमिति॑ यज्ञोपवी॒तमे॒तदे॒व विप॑रीतं प्राचीनावी॒त सं॒वीतं॑ मानु॒षम्॥१॥\anuvakamend

रक्षासि॒ ह वा॑ पुरोऽनुवा॒के तपोग्र॑मतिष्ठन्त॒ तान् प्र॒जाप॑तिर्व॒रेणो॒पा\-म॑न्त्रयत॒ तानि॒ वर॑मवृणीताऽऽदि॒त्यो नो॒ योद्धा॒ इति॒ तान् प्र॒जाप॑तिरब्रवी॒द्योध॑य॒ध्वमिति॒ तस्मा॒दुत्ति॑ष्ठन्त॒ ह वा॒ तानि॒ रक्षास्यादि॒त्यं योध॑यन्ति॒ याव॑दस्त॒मन्व॑गा॒त्तानि॑ ह॒ वा ए॒तानि॒ रक्षासि गायत्रि॒या\-ऽभि॑मन्त्रिते॒नाम्भ॑सा शाम्यन्ति॒ तदु॑ ह॒ वा ए॒ते ब्र॑ह्मवा॒दिन॑ पू॒र्वाभि॑मु॒खाः स॒न्ध्यायां गायत्रि॒या\-ऽभि॑मन्त्रिता॒ आप॑ ऊ॒र्ध्वं विक्षि॑पन्ति॒ ता ए॒ता आपो॑ व॒ज्रीभू॒त्वा तानि॒ रक्षासि म॒न्देहारु॑णे द्वी॒पे प्रक्षि॑पन्ति॒ यत्प्र॑दक्षि॒णं प्रक्र॑मन्ति॒ तेन॑ पा॒प्मान॒म् अव॑धून्वन्त्यु॒द्यन्त॑मस्तं॒ यन्त॑म् आदि॒त्यम॑भिध्या॒यन् कु॒र्वन् ब्राह्म॒णो वि॒द्वान्त्स॒कलं॑ भ॒द्रम॑श्नुते॒ऽसावा॑दि॒त्यो ब्र॒ह्मेति॒ ब्रह्मै॒व सन् ब्रह्मा॒प्येति॒ य ए॒वं वेद॑॥२॥\anuvakamend

यद्दे॑वा देव॒हेळ॑नं॒ देवा॑सश्चकृ॒मा व॒यम्। 
आदि॑त्या॒स्तस्मान्मा मुञ्चत॒र्तस्य॒र्तेन॒ मामि॒त। 
देवा॑ जीवनका॒म्या यद्वा॒चाऽनृ॑त\-मूदि॒म। 
तस्मान्न इ॒ह मु॑ञ्चत॒ विश्वे॑ देवाः स॒जोष॑सः। 
ऋ॒तेन॑ द्यावापृथिवी ऋ॒तेन॒ त्व स॑रस्वति। 
कृ॒तान्न॑ पा॒ह्येन॑सो॒ यत्किं चानृ॑त\-मूदि॒म। 
इ॒न्द्रा॒ग्नी मि॒त्रावरु॑णौ॒ सोमो॑ धा॒ता बृह॒स्पति॑। 
ते नो॑ मुञ्च॒न्त्वेन॑सो॒ यद॒न्यकृ॑तमारि॒म। 
स॒जा॒त॒श॒सादु॒त जा॑मिश॒साज्ज्याय॑स॒ शसा॑दु॒त वा॒ कनी॑यसः। 
अना॑धृष्टं दे॒वकृ॑तं॒ यदेन॒स्तस्मा॒त् त्वम॒स्माञ्जा॑तवेदो मुमुग्धि॥३॥

यद्वा॒चा यन्मन॑सा बा॒हुभ्या॑मू॒रुभ्या॑मष्ठी॒वद्भ्या शि॒श्नैर्यदनृ॑तं चकृ॒मा व॒यम्। 
अ॒ग्निर्मा॒ तस्मा॒देन॑सो॒ गार्\mbox{}ह॑पत्य॒ प्रमु॑ञ्चतु चकृ॒म यानि॑ दुष्कृ॒ता। 
येन॑ त्रि॒तो अ॑र्ण॒वान्नि॑र्ब॒भूव॒ येन॒ सूर्यं॒ तम॑सो निर्मु॒मोच॑। 
येनेन्द्रो॒ विश्वा॒ अज॑हा॒दरा॑ती॒स्तेना॒हं ज्योति॑षा॒ ज्योति॑रानशा॒न आक्षि। 
यत्कुसी॑द॒मप्र॑तीत्तं॒ मये॒ह येन॑ य॒मस्य॑ नि॒धिना॒ चरा॑मि। 
ए॒तत्तद॑ग्ने अनृ॒णो भ॑वामि॒ जीव॑न्ने॒व प्रति॒ तत्ते॑ दधामि। 
यन्मयि॑ मा॒ता यदा॑ पि॒पेष॒ यद॒न्तरि॑क्षं॒ यदा॒शसाति॑क्रामामि त्रि॒ते दे॒वा दि॒वि जा॒ता यदाप॑ इ॒मं मे॑ वरुण॒ तत्त्वा॑ यामि॒ त्वं नो॑ अग्ने॒ स त्वं नो॑ अग्ने॒ त्वम॑ग्ने अ॒यासि॑॥४॥\anuvakamend

यददीव्यन्नृ॒णम॒हं ब॒भूवादि॑त्सन्वा सञ्ज॒गर॒ जनेभ्यः। 
अ॒ग्निर्मा॒ तस्मा॒दिन्द्र॑श्च संविदा॒नौ प्रमु॑ञ्चताम्। 
यद्धस्ताभ्यां च॒कर॒ किल्बि॑षाण्य॒क्षाणां व॒ग्नुमु॑प॒जिघ्न॑मानः। 
उ॒ग्रं॒ प॒श्या च॑ राष्ट्र॒भृच्च॒ तान्य॑प्स॒रसा॒वनु॑दत्तामृ॒णानि॑। 
उग्रं॑ पश्ये॒ राष्ट्र॑भृ॒त्किल्बि॑षाणि॒ यद॒क्षवृ॑त्त॒मनु॑दत्तमे॒तत्। 
नेन्न॑ ऋ॒णानृ॒णव॒ इत्स॑मानो य॒मस्य॑ लो॒के अधि॑रज्जु॒राय॑। 
अव॑ ते॒ हेळ॒ उदु॑त्त॒ममि॒मं मे॑ वरुण॒ तत्त्वा॑ यामि॒ त्वं नो॑ अग्ने॒ स त्वं नो अग्ने। 
सङ्कु॑सुको॒ विकु॑सुको निर्\mbox{}ऋ॒थो यश्च॑ निस्व॒नः। 
तेऽ(१\char"E009)स्मद्यक्ष्म॒मना॑गसो दू॒राद्दू॒रम॑चीचतम्। 
निर्य॑क्ष्ममचीचते कृ॒त्यां निर्\mbox{}ऋ॑तिं च। 
तेन॒ योऽ(१\char"E009)स्मत्समृ॑च्छातै॒ तम॑स्मै॒ प्रसु॑वामसि। 
दु॒श॒सा॒नु॒श॒साभ्यां घ॒णेना॑नुघ॒णेन॑ च। 
तेना॒न्योऽ(१\char"E009)स्मत्समृ॑च्छातै॒ तम॑स्मै॒ प्रसु॑वामसि। 
सं वर्च॑सा॒ पय॑सा॒ सन्त॒नूभि॒रग॑न्महि॒ मन॑सा॒ स शि॒वेन॑। 
त्वष्टा॑ नो॒ अत्र॒ विद॑धातु रा॒योऽनु॑मार्ष्टु त॒न्वो(१\char"E009) यद्विलि॑ष्टम्॥५॥\anuvakamend


आयु॑ष्टे वि॒श्वतो॑ दधद॒यम॒ग्निर्वरेण्यः। 
पुन॑स्ते प्रा॒ण आया॑ति॒ परा॒यक्ष्म सुवामि ते। 
आ॒यु॒र्दा अ॑ग्ने ह॒विषो॑ जुषा॒णो घृ॒तप्र॑तीको घृ॒तयो॑निरेधि। 
घृ॒तं पी॒त्वा मधु॒ चारु॒ गव्यं॑ पि॒तेव॑ पु॒त्रम॒भिर॑क्षतादि॒मम्। 
इ॒मम॑ग्न॒ आयु॑षे॒ वर्च॑से कृधि ति॒ग्ममोजो॑ वरुण॒ सशि॑शाधि। 
मा॒तेवास्मा अदिते॒ शर्म॑ यच्छ॒ विश्वे॑ देवा॒ जर॑दष्टि॒र्यथाऽस॑त्। 
अग्न॒ आयूषि पवस॒ आ सु॒वोर्ज॒मिषं॑ च नः। 
आ॒रे बा॑धस्व दु॒च्छुनाम्। 
अग्ने॒ पव॑स्व॒ स्वपा॑ अ॒स्मे वर्च॑ सु॒वीर्यम्। 
दध॑द्र॒यिं मयि॒ पोषम्॥६॥

अ॒ग्निर्\mbox{}ऋषि॒ पव॑मान॒ पाञ्च॑जन्यः पु॒रोहि॑तः। 
तमी॑महे महाग॒यम्। 
अग्ने॑ जा॒तान्प्रणु॑दा नः स॒पत्ना॒न्प्रत्यजा॑ताञ्जातवेदो नुदस्व। 
अ॒स्मे दी॑दिहि सु॒मना॒ अहे॑ळ॒ञ्छर्म॑न्ते स्याम त्रि॒वरू॑थ उ॒द्भौ। 
सह॑सा जा॒तान्प्रणु॑दा नः स॒पत्ना॒न्प्रत्यजा॑ताञ्जातवेदो नुदस्व। 
अधि॑ नो ब्रूहि सुमन॒स्यमा॑नो व॒य स्या॑म॒ प्रणु॑दा नः स॒पत्ना\sn{}। 
अग्ने॒ यो नो॒ऽभितो॒ जनो॒ वृको॒ वारो॒ जिघासति। 
तास्त्वं वृ॑त्रहं जहि॒ वस्व॒स्मभ्य॒माभ॑र। 
अग्ने॒ यो नो॑ऽभि॒दास॑ति समा॒नो यश्च॒ निष्ट्य॑। 
तं व॒य स॒मिधं॑ कृ॒त्वा तुभ्य॑म॒ग्नेऽपि॑ दध्मसि॥७॥

यो न॒ शपा॒दश॑पतो॒ यश्च॑ न॒ शप॑त॒ शपात्। 
उ॒षाश्च॒ तस्मै॑ नि॒म्रुक्च॒ सर्वं॑ पा॒प समू॑हताम्। 
यो न॑ स॒पत्नो॒ यो रणो॒ मर्तो॑ऽभि॒दास॑ति देवाः। 
इ॒ध्मस्ये॑व प्र॒क्षाय॑तो॒ मा तस्योच्छे॑षि॒ किं च॒न। 
यो मां द्वेष्टि॑ जातवेदो॒ यं चा॒हं द्वेष्मि॒ यश्च॒ माम्। 
सर्वा॒स्तान॑ग्ने॒ सन्द॑ह॒ याश्चा॒हं द्वेष्मि॒ ये च॒ माम्। 
यो अ॒स्मभ्य॑मराती॒याद्यश्च॑ नो॒ द्वेष॑ते॒ जन॑। 
निन्दा॒द्यो अ॒स्मान्दिप्साच्च॒ सर्वा॒स्तान्म॑ष्म॒षा कु॑रु। 
सशि॑तं मे॒ ब्रह्म॒ सशि॑तं वी॒र्या(१\char"E009)म्बलम्। 
सशि॑तं क्ष॒त्रं मे॑ जि॒ष्णु यस्या॒हमस्मि॑ पु॒रोहि॑तः। 
उदे॑षां बा॒हू अ॑तिर॒मुद्वर्चो॒ अथो॒ बलम्। 
क्षि॒णोमि॒ ब्रह्म॑णा॒ऽमित्रा॒नुन्न॑यामि॒ स्वा(१)म् अ॒हम्। 
पुन॒र्मन॒ पुन॒रायु॑र्म॒ आगा॒त्पुन॒श्चक्षु॒ पुन॒ श्रोत्रं॑ म॒ आगा॒त्पुन॑ प्रा॒णः पुन॒राकू॑तं म॒ आगा॒त्पुन॑श्चि॒त्तं पुन॒राधी॑तं म॒ आगात्। 
वै॒श्वा॒न॒रो मेऽद॑ब्धस्तनू॒पा अव॑बाधतां दुरि॒तानि॒ विश्वा॥८॥\anuvakamend

वै॒श्वा॒न॒राय॒ प्रति॑वेदयामो॒ यदी॑नृ॒ण स॑ङ्ग॒रो दे॒वता॑सु। 
स ए॒तान्पाशान् प्र॒मुच॒न् प्रवे॑द॒ स नो॑ मुञ्चातु दुरि॒तादव॒द्यात्। 
वै॒श्वा॒न॒रः पव॑यान्नः प॒वित्रै॒र्यत्स॑ङ्ग॒रम॒भिधावाम्या॒शाम्। 
अना॑जान॒न्मन॑सा॒ याच॑मानो॒ यदत्रैनो॒ अव॒ तत्सु॑वामि। 
अ॒मी ये सु॒भगे॑ दि॒वि वि॒चृतौ॒ नाम॒ तार॑के। 
प्रेहामृत॑स्य यच्छतामे॒तद्ब॑द्धक॒मोच॑नम्। 
विजि॑हीर्ष्व लो॒कान्कृ॑धि ब॒न्धान्मु॑ञ्चासि॒ बद्ध॑कम्। 
योने॑रिव॒ प्रच्यु॑तो॒ गर्भ॒ सर्वान् प॒थो अ॑नुष्व। 
स प्र॑जा॒नन्प्रति॑गृभ्णीत वि॒द्वान्प्र॒जाप॑तिः प्रथम॒जा ऋ॒तस्य॑। 
अ॒स्माभि॑र्द॒त्तं ज॒रस॑ प॒रस्ता॒दच्छि॑न्नं॒ तन्तु॑मनु॒सञ्च॑रेम॥९॥

त॒तं तन्तु॒मन्वेके॒ अनु॒ सञ्च॑रन्ति॒ येषां द॒त्तं पित्र्य॒माय॑नवत्। 
अ॒ब॒न्ध्वेके॒ दद॑तः प्र॒यच्छा॒द्दातुं॒ चेच्छ॒क्नवा॒सः स्व॒र्ग ए॑षाम्। 
आर॑भेथा॒मनु॒ सर॑भेथा समा॒नं पन्था॑मवथो घृ॒तेन॑। 
यद्वां पू॒र्तं परि॑विष्टं॒ यद॒ग्नौ तस्मै॒ गोत्रा॑ये॒ह जाया॑पती॒ सर॑भेथाम्। 
यद॒न्तरि॑क्षं पृथि॒वीमु॒त द्यां यन्मा॒तरं॑ पि॒तरं॑ वा जिहिसि॒म। 
अ॒ग्निर्मा॒ तस्मा॒देन॑सो॒ गार्\mbox{}ह॑पत्य॒ उन्नो॑ नेषद्दुरि॒ता यानि॑ चकृ॒म। 
भूमि॑र्मा॒ताऽदि॑तिर्नो ज॒नित्रं॒ भ्राता॒ऽन्तरि॑क्षम॒भिश॑स्त एनः। 
द्यौर्न॑ पि॒ता पि॑तृ॒याच्छं भ॑वासि जा॒मि मि॒त्वा मा वि॑वित्सि लो॒कात्। 
यत्र॑ सु॒हार्द॑ सु॒कृतो॒ मद॑न्ते वि॒हाय॒ रोगं॑ त॒न्वा(१\char"E009) स्वायाम्। 
अ॒श्लो॒णाङ्गै॒रह्रु॑ताः स्व॒र्गे तत्र॑ पश्येम पि॒तरं॑ च पु॒त्रम्। 
यदन्न॒मद्म्यनृ॑तेन देवा दा॒स्यन्नदास्यन्नु॒त वा॑ करि॒ष्यन्। 
यद्दे॒वानां॒ चक्षु॒ष्यागो॒ अस्ति॒ यदे॒व किं च॑ प्रतिजग्रा॒हम॒ग्निर्मा॒ तस्मा॑दनृ॒णं कृ॑णोतु। 
यदन्न॒मद्मि॑ बहु॒धा विरू॑पं॒ वासो॒ हिर॑ण्यमु॒त गाम॒जामविम्। 
यद्दे॒वानां॒ चक्षु॒ष्यागो॒ अस्ति॒ यदे॒व किं च॑ प्रतिजग्रा॒हम॒ग्निर्मा॒ तस्मा॑दनृ॒णं कृ॑णोतु। 
य॒न्मया॑ मन॑सा वा॒चा॒ कृ॒त॒मेन॑ कदा॒चन। 
सर्वस्मात्तस्मान्मेळि॑तो मो॒ग्धि॒ त्व हि वेत्थ॑ यथात॒थम्॥१०॥\anuvakamend

वात॑रशना ह॒ वा ऋष॑यः श्रम॒णा ऊ॒र्ध्वम॑न्थि॒नो ब॑भूवु॒स्तानृष॑यो॒\-ऽर्थमा॑य॒स्ते नि॒लाय॑मचर॒स्तेऽनु॑प्रविशुः कूश्मा॒ण्डानि॒ तास्तेष्वन्व॑विन्दञ्छ्र॒द्धया॑ च॒ तप॑सा च॒ तानृष॑योऽब्रुवन्क॒था नि॒लायं॑ चर॒थेति॒ त ऋषी॑नब्रुव॒न्नमो॑ वोऽस्तु भगवन्तो॒ऽस्मिन्धाम्नि॒ केन॑ वः सपर्या॒मेति॒ तानृष॑योऽब्रुवन्प॒वित्रं॑ नो ब्रूत॒ येना॑रे॒पस॑ स्या॒मेति॒ त ए॒तानि॑ सू॒क्तान्य॑पश्य॒न् यद्दे॑वा देव॒हेळ॑नं॒ यददीव्यन्नृ॒णम॒हं ब॒भूवाऽऽयु॑ष्टे वि॒श्वतो॑ दध॒दित्ये॒तैराज्यं॑ जुहुत वैश्वान॒राय॒ प्रति॑वेदयाम॒ इत्युप॑तिष्ठत॒ यद॑र्वा॒चीन॒मेनो भ्रूणह॒त्याया॒\-स्तस्मान्मोक्ष्यध्व॒ इति॒ त ए॒तैर॑जुहवु॒स्तेऽरे॒पसो॑\-ऽभवन्कर्मा॒दिष्वे॒तैर्जु॑हुयात्पू॒तो दे॑वलो॒कान्त्सम॑श्नुते॥११॥\anuvakamend


कू॒श्मा॒ण्डैर्जु॑हुया॒द्योऽपू॑त इव॒ मन्ये॑त॒ यथा स्ते॒नो यथा भ्रूण॒हैवमे॒ष भ॑वति॒ योऽयोनौ॒ रेत॑ सि॒ञ्चति॒ यद॑र्वा॒चीन॒मेनो भ्रूणह॒त्याया॒स्तस्मान्मुच्यते॒ याव॒देनो॑ दी॒क्षामुपै॑ति दीक्षि॒त ए॒तैः स॑त॒ति जु॑होति संवत्स॒रं दीक्षि॒तो भ॑वति संवत्स॒रादे॒वाऽऽत्मानं॑ पुनीते॒ मासं॑ दीक्षि॒तो भ॑वति॒ यो मास॒ स सं॑वत्स॒रः सं॑वत्स॒रादे॒वाऽऽत्मानं॑ पुनीते॒ चतु॑र्विशति॒ रात्रीर्दीक्षि॒तो भ॑वति॒ चतु॑र्विशतिरर्धमा॒साः सं॑वत्स॒रः सं॑वत्स॒रादे॒वाऽऽत्मानं॑ पुनीते॒ द्वाद॑श॒ रात्रीर्दीक्षि॒तो भ॑वति॒ द्वाद॑श॒ मासा संवत्स॒रः सं॑वत्स॒रादे॒वाऽऽत्मानं॑ पुनीते॒ षड्रात्रीर्दीक्षि॒तो भ॑वति॒ षड्वा ऋ॒तव॑ संवत्स॒रः सं॑वत्स॒रादे॒वाऽऽत्मानं॑ पुनीते ति॒स्रो रात्रीर्दीक्षि॒तो भ॑वति त्रि॒पदा॑ गाय॒त्री गा॑यत्रि॒या ए॒वाऽऽत्मानं॑ पुनीते॒ न मा॒सम॑श्नीया॒न्न स्त्रिय॒मुपे॑या॒न्नोपर्या॑सीत॒ जुगु॑प्से॒तानृ॑ता॒त्पयो ब्राह्म॒णस्य॑ व्र॒तं य॑वा॒गू रा॑ज॒न्य॑स्या॒मिक्षा॒ वैश्य॒स्याथो॑ सौ॒म्येप्य॑ध्व॒र ए॒तद्व्र॒तं ब्रू॑या॒द्यदि॒ मन्ये॑तोप॒दस्या॒मीत्यो॑द॒नं धा॒नाः सक्तूं घृ॒तमित्यनु॑व्रतयेदा॒त्मनोऽनु॑पदासाय॥१२॥ 
\anuvakamend

अ॒जान् ह॒ वै पृश्नीस्तप॒स्यमा॑ना॒न् ब्रह्म॑ स्वय॒म्भ्व॑भ्यान॑र्\mbox{}ष॒त्त ऋष॑योऽभव॒न्तदृषी॑णामृषि॒त्वं तां दे॒वता॒मुपा॑तिष्ठन्त य॒ज्ञका॑मा॒स्त ए॒तं ब्र॑ह्मय॒ज्ञम॑पश्य॒न्तमाह॑र॒न्तेना॑यजन्त॒ यदृ॒चोऽध्यगी॑षत॒ ताः पय॑आहुतयो दे॒वाना॑मभव॒न् यद्यजूषि घृ॒ताहु॑तयो॒ यत्सामा॑नि॒ सोमा॑हुतयो॒ यदथ॑र्वाङ्गि॒रसो॒ मध्वा॑हुतयो॒ यद्ब्राह्म॒णानी॑तिहा॒सान् पु॑रा॒णानि॒ कल्पा॒न्गाथा॑ नाराश॒सीर्मे॑दाहु॒तयो॑ दे॒वाना॑मभव॒न्ताभि॒ क्षुधं॑ पा॒प्मान॒म\-पाघ्न॒न्नप॑हतपाप्मानो दे॒वाः स्व॒र्गं लो॒कमा॑य॒न् ब्रह्म॑ण॒ सायु॑ज्य॒मृष॑योऽगच्छन्॥१३॥\anuvakamend

पञ्च॒ वा ए॒ते म॑हाय॒ज्ञाः स॑त॒ति प्रता॑यन्ते सत॒ति सन्ति॑ष्ठन्ते देवय॒ज्ञः पि॑तृय॒ज्ञो भू॑तय॒ज्ञो म॑नुष्यय॒ज्ञो ब्र॑ह्मय॒ज्ञ इति॒ यद॒ग्नौ जु॒होत्य॒पि स॒मिधं॒ तद्दे॑वय॒ज्ञः सन्ति॑ष्ठते॒ यत्पि॒तृभ्य॑ स्व॒धा क॒रोत्यप्य॒पस्तत्पि॑तृय॒ज्ञः सन्ति॑ष्ठते॒ यद्भू॒तेभ्यो॑ ब॒लि हर॑ति॒ तद्भू॑तय॒ज्ञः सन्ति॑ष्ठते॒ यद्ब्राह्म॒णेभ्योऽन्नं॒ ददा॑ति॒ तन्म॑नुष्यय॒ज्ञः सन्ति॑ष्ठते॒ यत्स्वाध्या॒यमधी॑यी॒तैका॑मप्यृ॒चं यजु॒ साम॑ वा॒ तद्ब्र॑ह्मय॒ज्ञः सन्ति॑ष्ठते॒ यदृ॒चोऽधी॑ते॒ पय॑स॒ कूल्या॑ अस्य पि॒तॄन्त्स्व॒धा अ॒भिव॑हन्ति॒ यद्यजूषि घृ॒तस्य॑ कूल्या॒ यत्सामा॑नि॒ सोम॑ एभ्यः पवते॒ यदथ॑र्वाङ्गि॒रसो॒ मधो कूल्या॒ यद्ब्राह्म॒णानी॑तिहा॒सान् पु॑रा॒णानि॒ कल्पा॒न्गाथा॑ नाराश॒सीर्मेद॑स॒ कूल्या॑ अस्य पि॒तॄन्त्स्व॒धा अ॒भिव॑हन्ति॒ यदृ॒चोऽधी॑ते॒ पय॑आहुतिभिरे॒व तद्दे॒वास्त॑र्पयति॒ यद्यजूषि घृ॒ताहु॑तिभि॒र्यत्सामा॑नि॒ सोमा॑हुतिभि॒र्यदथ॑र्वाङ्गि॒रसो॒ मध्वा॑\-हुतिभि॒र्यद्ब्राह्म॒णानी॑तिहा॒सान् पु॑रा॒णानि॒ कल्पा॒न्गाथा॑ नाराश॒सीर्मे॑दाहु॒तिभि॑रे॒व तद्दे॒वास्त॑र्पयति॒ त ए॑नं तृ॒प्ता आयु॑षा॒ तेज॑सा॒ वर्च॑सा श्रि॒या यश॑सा ब्रह्मवर्च॒सेना॒न्नाद्ये॑न च तर्पयन्ति॥१४॥
\anuvakamend


ब्र॒ह्म॒य॒ज्ञेन॑ य॒क्ष्यमा॑ण॒ प्राच्यां दि॒शि ग्रामा॒दछ॑दिर्द॒र्\mbox{}श उदीच्यां प्रागुदी॒च्यां वो॒दित॑ आदि॒त्ये द॑क्षिण॒त उ॑प॒वीयो॑प॒विश्य॒ हस्ता॑वव॒निज्य॒ त्रिराचा॑मे॒द्द्विः प॑रि॒मृज्य॑ स॒कृदु॑प॒स्पृश्य॒ शिर॒श्चक्षु॑षी॒ नासि॑के॒ श्रोत्रे॒ हृद॑यमा॒लभ्य॒ यत्त्रिरा॒चाम॑ति॒ तेन॒ ऋच॑ प्रीणाति॒ यद्द्विः प॑रि॒मृज॑ति॒ तेन॒ यजूषि॒ यत्स॒कृदु॑प॒स्पृश॑ति॒ तेन॒ सामा॑नि॒ यत्स॒व्यं पा॒णिं पा॒दौ प्रो॒क्षति॒ यच्छिर॒श्चक्षु॑षी॒ नासि॑के॒ श्रोत्रे॒ हृद॑यमा॒लभ॑ते॒ तेनाथ॑र्वाङ्गि॒रसो ब्राह्म॒णानी॑तिहा॒सान् पु॑रा॒णानि॒ कल्पा॒न्गाथा॑ नाराश॒सीः प्री॑णाति॒ दर्भा॑णां म॒हदु॑प॒स्तीर्यो॒पस्थं॑ कृ॒त्वा प्राङासी॑नः स्वाध्या॒यमधी॑यीता॒पां वा ए॒ष ओष॑धीना॒ रसो॒ यद्द॒र्भाः सर॑समे॒व ब्रह्म॑ कुरुते दक्षिणोत्त॒रौ पा॒णी पा॒दौ कृ॒त्वा सप॒वित्रा॒वोमिति॒ प्रति॑पद्यत ए॒तद्वै यजु॑स्त्रयीं वि॒द्यां प्रत्ये॒षा वागे॒तत्प॑र॒मम॒क्षरं॒ तदे॒तदृ॒चाऽभ्यु॑क्तमृ॒चो अ॒क्षरे॑ पर॒मे व्यो॑म॒न् यस्मि॑न्दे॒वा अधि॒ विश्वे॑ निषे॒दुर्यस्तन्न वेद॒ किमृ॒चा क॑रिष्यति॒ य इत्तद्वि॒दुस्त इ॒मे समा॑सत॒ इति॒ त्रीने॒व प्रायु॑ङ्क्त॒ भूर्भुव॒ स्व॑रित्या॑है॒तद्वै वा॒चः स॒त्यं यदे॒व वा॒चः स॒त्यं तत्प्रायु॒ङ्क्ताथ॑ सावि॒त्रीं गा॑य॒त्रीं त्रिरन्वा॑ह प॒च्छोऽर्धर्च॑शोऽनवा॒न स॑वि॒ता श्रिय॑ प्रसवि॒ता श्रिय॑मे॒वाऽऽप्नो॒त्यथो प्र॒ज्ञात॑यै॒व प्र॑ति॒पदा॒ छन्दासि॒ प्रति॑पद्यते॥१५॥\anuvakamend


ग्रामे॒ मन॑सा स्वाध्या॒यमधी॑यीत॒ दिवा॒ नक्तं॑ वे॒ति ह॑ स्मा॒ऽऽह शौ॒च आह्ने॒य उ॒तार॑ण्ये॒ऽबल॑ उ॒त वा॒चोत तिष्ठ॑न्नु॒त व्रज॑न्नु॒ताऽऽसी॑न उ॒त शया॑नो॒ऽधीयी॑तै॒व स्वाध्या॒यं तप॑स्वी॒ पुण्यो॑ भवति॒ य ए॒वं विद्वान्त्स्वाध्या॒यमधी॑ते॒ नमो॒ ब्रह्म॑णे॒ नमो॑ अस्त्व॒ग्नये॒ नम॑ पृथि॒व्यै नम॒ ओष॑धीभ्यः। 
नमो॑ वा॒चे नमो॑ वा॒चस्पत॑ये॒ नमो॒ विष्ण॑वे बृह॒ते क॑रोमि॥१६॥
\anuvakamend


म॒ध्यन्दि॑ने प्र॒बल॒मधी॑यीता॒सौ खलु॒ वावैष आ॑दि॒त्यो यद्ब्राह्म॒ण\-स्तस्मा॒त्तर्\mbox{}हि॒ तेऽक्ष्णि॑ष्ठं तपति॒ तदे॒षाऽभ्यु॑क्ता। 
चि॒त्रं दे॒वाना॒मुद॑गा॒\-दनी॑कं॒ चक्षु॑र्मि॒त्रस्य॒ वरु॑णस्या॒ग्नेः। 
आऽप्रा॒ द्यावा॑पृथि॒वी अ॒न्तरि॑क्ष॒ सूर्य॑ आ॒त्मा जग॑तस्त॒स्थुष॒श्चेति॒ स वा ए॒ष य॒ज्ञः स॒द्यः प्रता॑यते स॒द्यः सन्ति॑ष्ठते॒ तस्य॒ प्राक् सा॒यम॑वभृ॒थो नमो॒ ब्रह्म॑ण॒ इति॑ परिधा॒नीयां॒ त्रिरन्वा॑हा॒प उ॑प॒स्पृश्य॑ गृ॒हाने॑ति॒ ततो॒ यत्किं च॒ ददा॑ति॒ सा दक्षि॑णा॥१७॥\anuvakamend


तस्य॒ वा ए॒तस्य॑ य॒ज्ञस्य॒ मेघो॑ हवि॒र्धानं॑ वि॒द्युद॒ग्निर्\mbox{}व॒र्\mbox{}ष ह॒विः स्त॑नयि॒त्नुर्व॑षट्का॒रो यद॑व॒स्फूर्ज॑ति॒ सोऽनु॑वषट्का॒रो वा॒युरा॒त्माऽमा॑वा॒स्या स्विष्ट॒कृद्य ए॒वं वि॒द्वान्मे॒घे व॒र्\mbox{}षति॑ वि॒द्योत॑माने स्त॒नय॑त्यव॒स्फूर्ज॑ति॒ पव॑माने वा॒याव॑मावा॒स्या॑या स्वाध्या॒यमधी॑ते॒ तप॑ ए॒व तत्त॑प्यते॒ तपो॑ हि स्वाध्या॒य इत्यु॑त्त॒मं नाक रोहत्युत्त॒मः स॑मा॒नानां भवति॒ याव॑न्त ह॒ वा इ॒मां वि॒त्तस्य॑ पू॒र्णां दद॑त्स्व॒र्गं लो॒कं ज॑यति॒ ताव॑न्तं लो॒कं ज॑यति॒ भूयासं चाक्ष॒य्यं चाप॑ पुनर्मृ॒त्युं ज॑यति॒ ब्रह्म॑ण॒ सायु॑ज्यं गच्छति॥१८॥\anuvakamend


तस्य॒ वा ए॒तस्य॑ य॒ज्ञस्य॒ द्वाव॑नध्या॒यौ यदा॒त्माऽशुचि॒र्यद्दे॒शः समृ॑द्धिर्दैव॒तानि॒ य ए॒वं वि॒द्वान्म॑हारा॒त्र उ॒षस्युदि॑ते॒ व्रज॒स्तिष्ठ॒न्नासी॑न॒ शया॑नो॒ऽरण्ये ग्रामे॒ वा याव॑त्त॒रस स्वाध्या॒यमधी॑ते॒ सर्वाँल्लो॒काञ्ज॑यति॒ सर्वाँल्लो॒कान॑नृ॒णोऽनु॒\-सञ्च॑रति॒ तदे॒षाभ्यु॑क्ता। 
अ॒नृ॒णा अ॒स्मिन्न॑नृ॒णाः पर॑स्मि\-स्तृ॒तीये॑ लो॒के अ॑नृ॒णाः स्या॑म। 
ये दे॑व॒याना॑ उ॒त पि॑तृ॒याणा॒ सर्वान्प॒थो अ॑नृ॒णा आक्षी॑ये॒मेत्य॒ग्निं वै जा॒तं पा॒प्मा ज॑ग्राह॒ तं दे॒वा आहु॑तीभिः पा॒प्मान॒मपाघ्न॒न्नाहु॑तीनां य॒ज्ञेन॑ य॒ज्ञस्य॒ दक्षि॑णाभि॒र्दक्षि॑णानां ब्राह्म॒णेन॑ ब्राह्म॒णस्य॒ छन्दो॑भि॒श्छन्द॑सा स्वाध्या॒येनाप॑हतपाप्मा स्वाध्या॒यो॑ दे॒वप॑वित्रं॒ वा ए॒तत्तं योऽनूत्सृ॒जत्यभा॑गो वा॒चि भ॑व॒त्यभा॑गो ना॒के तदे॒षाऽभ्यु॑क्ता। 
यस्ति॒त्याज॑ सखि॒विद॒ सखा॑यं॒ न तस्य॑ वा॒च्यपि॑ भा॒गो अ॑स्ति। 
यदी शृ॒णोत्य॒लक शृणोति॒ न हि प्र॒वेद॑ सुकृ॒तस्य॒ पन्था॒मिति॒ तस्मात्स्वाध्या॒योऽध्ये॑त॒व्यो॑ यं यं॑ क्र॒तुमधी॑ते॒ तेन॑ तेनास्ये॒ष्टं भ॑वत्य॒ग्नेर्वा॒योरा॑दि॒त्यस्य॒ सायु॑ज्यं गच्छति॒ तदे॒षाऽभ्यु॑क्ता। 
ये अ॒र्वाङु॒त वा॑ पुरा॒णे वे॒दं वि॒द्वास॑म॒भितो॑ वदन्त्यादि॒त्यमे॒व ते परि॑वदन्ति॒ सर्वे॑ अ॒ग्निं द्वि॒तीयं॑ तृ॒तीयं॑ च ह॒समिति॒ याव॑ती॒र्वै दे॒वता॒स्ताः सर्वा॑ वेद॒विदि॑ ब्राह्म॒णे व॑सन्ति॒ तस्माद्ब्राह्म॒णेभ्यो॑ वेद॒विद्भ्यो॑ दि॒वे दि॑वे॒ नम॑स्कुर्या॒न्नाश्ली॒लं कीर्तयेदे॒ता ए॒व दे॒वता प्रीणाति॥१९॥
\anuvakamend

रिच्य॑त इव॒ वा ए॒ष प्रेव रि॑च्यते॒ यो या॒जय॑ति॒ प्रति॑ वा गृ॒ह्णाति॑ या॒जयि॑त्वा प्रतिगृ॒ह्य वाऽन॑श्न॒न्त्रिः स्वाध्या॒यं वे॒दमधी॑यीत त्रिरा॒त्रं वा॑ सावि॒त्रीं गा॑य॒त्रीम॒न्वाति॑रेचयति॒ वरो॒ दक्षि॑णा॒ वरे॑णै॒व वर स्पृणोत्या॒त्मा हि वर॑॥२०॥
\anuvakamend

दु॒हे ह॒ वा ए॒ष छन्दासि॒ यो या॒जय॑ति॒ स येन॑ यज्ञक्र॒तुना॑ या॒जये॒त्सोऽर॑ण्यं प॒रेत्य॑ शुचौ दे॒शे स्वाध्या॒यमे॒वैन॒मधी॑यन्नासी॒त तस्या॒नश॑नं दी॒क्षा स्था॒नमु॑प॒सद॒ आस॑न सु॒त्या वाग्जु॒हूर्मन॑ उप॒भृद्धृ॒तिर्ध्रु॒वा प्रा॒णो ह॒विः सामाध्व॒र्युः स वा ए॒ष य॒ज्ञः प्रा॒णद॑क्षि॒णोऽन॑न्त\-दक्षिण॒ समृ॑द्धतरः॥२१॥
\anuvakamend


क॒ति॒धाव॑कीर्णी प्रवि॒शति॑ चतु॒र्धेत्या॑हुर्ब्रह्मवा॒दिनो॑ म॒रुत॑ प्रा॒णैरिन्द्रं॒ बले॑न॒ बृह॒स्पतिं॑ ब्रह्मवर्च॒सेना॒ग्निमे॒वेत॑रेण॒ सर्वे॑ण॒ तस्यै॒तां प्राय॑श्चित्तिं वि॒दां च॑कार सुदे॒वः काश्य॒पो यो ब्र॑ह्मचा॒र्य॑व॒किरे॑दमावा॒स्या॑या॒ रात्र्या॑म॒ग्निं प्र॒णीयो॑पसमा॒धाय॒ द्विराज्य॑स्योप॒घातं॑ जुहोति॒ कामाव॑कीर्णो॒ऽस्म्यव॑कीर्णोऽस्मि॒ काम॒ कामा॑य॒ स्वाहा॒ कामाभि॑द्रुग्धो॒ऽस्म्यभि॑द्रुग्धोऽस्मि॒ काम॒ कामा॑य॒ स्वाहेत्य॒मृतं॒ वा आज्य॑म॒मृत॑मे॒वाऽऽत्मन्ध॑त्ते हु॒त्वा प्रय॑ताञ्ज॒लिः कवा॑तिर्यङ्ङ॒ग्निम॒भि\-म॑न्त्रयेत॒ सं मा॑ऽऽसिञ्चन्तु म॒रुत॒ समिन्द्र॒ सं बृह॒स्पति॑। 
सं मा॒ऽयम॒ग्निः सि॑ञ्च॒त्वायु॑षा च॒ बले॑न॒ चाऽऽयु॑ष्मन्तं करोत॒ मेति॒ प्रति॑ हास्मै म॒रुत॑ प्रा॒णान्द॑धति॒ प्रतीन्द्रो॒ बलं॒ प्रति॒ बृह॒स्पति॑र्ब्रह्मवर्च॒सं प्रत्य॒ग्निरि॒तर॒त्सर्व॒ सर्व॑तनुर्भू॒त्वा सर्व॒मायु॑रेति॒ त्रिर॒भिम॑न्त्रयेत॒ त्रिष॑त्या॒ हि दे॒वा योऽपू॑त इव॒ मन्ये॑त॒ स इ॒त्थं जु॑हुयादि॒त्थम॒भिम॑न्त्रयेत॒ पुनी॑त ए॒वाऽऽत्मान॒मायु॑रे॒वाऽऽत्मन्ध॑त्ते॒ वरो॒ दक्षि॑णा॒ वरे॑णै॒व वर स्पृणोत्या॒त्मा हि वर॑॥२२॥
\anuvakamend

भूः प्रप॑द्ये॒ भुव॒ प्रप॑द्ये॒ स्व॑ प्रप॑द्ये॒ भूर्भुव॒ स्व॑ प्रप॑द्ये॒ ब्रह्म॒ प्रप॑द्ये ब्रह्मको॒शं प्रप॑द्ये॒ऽमृतं॒ प्रप॑द्येऽमृतको॒शं प्रप॑द्ये चतुर्जा॒लं ब्र॑ह्मको॒शं यं मृ॒त्युर्नाव॒पश्य॑ति॒ तं प्रप॑द्ये दे॒वान् प्रप॑द्ये देवपु॒रं प्रप॑द्ये॒ परी॑वृतो॒ वरी॑वृतो॒ ब्रह्म॑णा॒ वर्म॑णा॒ऽहं तेज॑सा॒ कश्य॑पस्य॒ यस्मै॒ नम॒स्तच्छिरो॒ धर्मो॑ मू॒र्धानं॑ ब्र॒ह्मोत्त॑रा॒ हनु॑र्य॒ज्ञोऽध॑रा॒ विष्णु॒\ar{}हृद॑य संवत्स॒रः प्र॒जन॑नम॒श्विनौ॑ पूर्व॒पादा॑व॒त्रिर्मध्यं॑ मि॒त्रावरु॑णावपर॒पादा॑व॒ग्निः पुच्छ॑स्य प्रथ॒मं काण्डं॒ तत॒ इन्द्र॒स्तत॑ प्र॒जाप॑ति॒रभ॑यं चतु॒र्थ स वा ए॒ष दि॒व्यः शाक्व॒रः शिशु॑मार॒स्त ह॒ य ए॒वं वेदाप॑ पुनर्मृ॒त्युं ज॑यति॒ जय॑ति स्व॒र्गं लो॒कं नाध्वनि॒ प्रमी॑यते॒ नाप्सु प्रमी॑यते॒ नाग्नौ प्रमी॑यते॒ नान॒पत्य॑ प्रमी॒यते॑ ल॒घ्वान्नो॑ भवति ध्रु॒वस्त्वम॑सि ध्रु॒वस्य क्षि॑तमसि॒ त्वं भू॒ताना॒मधि॑पतिरसि॒ त्वं भू॒ताना॒ श्रेष्ठो॑ऽसि॒ त्वां भू॒तान्युप॑ प॒र्याव॑र्तन्ते॒ नम॑स्ते॒ नम॒ सर्वं॑ ते॒ नमो॒ नम॑ शिशुकुमाराय॒ नम॑॥२३॥
\anuvakamend

नम॒ प्राच्यै॑ दि॒शे याश्च॑ दे॒वता॑ ए॒तस्यां॒ प्रति॑वसन्त्ये॒ताभ्य॑श्च॒  नमो॒ नमो दक्षि॑णायै दि॒शे याश्च॑ दे॒वता॑ ए॒तस्यां॒ प्रति॑वसन्त्ये॒ताभ्य॑श्च॒  नमो॒ नम॒ प्रतीच्यै दि॒शे याश्च॑ दे॒वता॑ ए॒तस्यां॒ प्रति॑वसन्त्ये॒ताभ्य॑श्च॒  नमो॒ नम॒ उदीच्यै दि॒शे याश्च॑ दे॒वता॑ ए॒तस्यां॒ प्रति॑वसन्त्ये॒ताभ्य॑श्च॒  नमो॒ नम॑ ऊ॒र्ध्वायै॑ दि॒शे याश्च॑ दे॒वता॑ ए॒तस्यां॒ प्रति॑वसन्त्ये॒ताभ्य॑श्च॒  नमो॒ नमोऽध॑रायै दि॒शे याश्च॑ दे॒वता॑ ए॒तस्यां॒ प्रति॑वसन्त्ये॒ताभ्य॑श्च॒  नमो॒ नमो॑ऽवान्त॒रायै॑ दि॒शे याश्च॑ दे॒वता॑ ए॒तस्यां॒ प्रति॑वसन्त्ये॒ताभ्य॑श्च॒  नमो॒ नमो गङ्गायमुनयोर्मध्ये ये॑ वस॒न्ति॒ ते मे प्रसन्नात्मानश्चिरं जीवितं व॑र्धय॒न्ति॒ नमो गङ्गायमुनयोर्मुनि॑भ्यश्च॒ नमो॒ नमो गङ्गायमुनयोर्मुनि॑भ्यश्च॒ नमः॥२४॥
\anuvakamend

ॐ नमो॒ ब्रह्म॑णे॒ नमो॑ अस्त्व॒ग्नये॒ नम॑ पृथि॒व्यै नम॒ ओष॑धीभ्यः। 
नमो॑ वा॒चे नमो॑ वा॒चस्पत॑ये॒ नमो॒ विष्ण॑वे बृह॒ते क॑रोमि॥\\
\centerline{॥ॐ शान्ति॒ शान्ति॒ शान्ति॑॥}
