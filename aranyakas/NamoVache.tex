% !TeX program = XeLaTeX
% !TeX root = ../AraNyakabook-kindle.tex
\sect{चतुर्थः प्रश्नः}\setcounter{anuvakam}{0}
नमो॑ वा॒चे या चो॑दि॒ता या चानु॑दिता॒ तस्यै॑ वा॒चे नमो॒ नमो॑ वा॒चे नमो॑ वा॒चस्पत॑ये॒ नम॒ ऋषि॑भ्यो मन्त्र॒कृद्भ्यो॒ मन्त्र॑पतिभ्यो॒ मा मामृष॑यो मन्त्र॒कृतो॑ मन्त्र॒पत॑य॒ परा॑दु॒र्माहमृषीन्मन्त्र॒कृतो॑ मन्त्र॒पती॒न्परा॑दां वैश्वदे॒वीं वाच॑मुद्यास शि॒वामद॑स्तां॒ जुष्टां दे॒वेभ्य॒ शर्म॑ मे॒ द्यौः  शर्म॑ पृथि॒वी शर्म॒ विश्व॑मि॒दं जग॑त्। शर्म॑ च॒न्द्रश्च॒ सूर्य॑श्च॒ शर्म॑ ब्रह्मप्रजाप॒ती। भू॒तं व॑दिष्ये॒ भुव॑नं वदिष्ये॒ तेजो॑ वदिष्ये॒ यशो॑ वदिष्ये॒ तपो॑ वदिष्ये॒ ब्रह्म॑ वदिष्ये स॒त्यं व॑दिष्ये॒ तस्मा॑ अ॒हमि॒दमु॑प॒स्तर॑ण॒मुप॑स्तृण उप॒स्तर॑णं मे प्र॒जायै॑ पशू॒नां भू॑यादुप॒स्तर॑णम॒हं प्र॒जायै॑ पशू॒नां भू॑यासं॒ प्राणा॑पानौ मृ॒त्योर्मा॑ पातं॒ प्राणा॑पानौ॒ मा मा॑ हासिष्टं॒ मधु॑ मनिष्ये॒ मधु॑ जनिष्ये॒ मधु॑ वक्ष्यामि॒ मधु॑ वदिष्यामि॒ मधु॑मतीं दे॒वेभ्यो॒ वाच॑मुद्यास शुश्रू॒षेण्यां मनु॒ष्येभ्य॒स्तं मा॑ दे॒वा अ॑वन्तु शो॒भायै॑ पि॒तरोऽनु॑मदन्तु। ॐ शान्ति॒ शान्ति॒ शान्ति॑॥ 

नमो॑ वा॒चे या चो॑दि॒ता या चानु॑दिता॒ तस्यै॑ वा॒चे नमो॒ नमो॑ वा॒चे नमो॑ वा॒चस्पत॑ये॒ नम॒ ऋषि॑भ्यो मन्त्र॒कृद्भ्यो॒ मन्त्र॑पतिभ्यो॒ मा मामृष॑यो मन्त्र॒कृतो॑ मन्त्र॒पत॑य॒ परा॑दु॒र्माहमृषीन्मन्त्र॒कृतो॑ मन्त्र॒पती॒न्परा॑दां वैश्वदे॒वीं वाच॑मुद्यास शि॒वामद॑स्तां॒ जुष्टां दे॒वेभ्य॒ शर्म॑ मे॒ द्यौः  शर्म॑ पृथि॒वी शर्म॒ विश्व॑मि॒दं जग॑त्। शर्म॑ च॒न्द्रश्च॒ सूर्य॑श्च॒ शर्म॑ ब्रह्मप्रजाप॒ती। भू॒तं व॑दिष्ये॒ भुव॑नं वदिष्ये॒ तेजो॑ वदिष्ये॒ यशो॑ वदिष्ये॒ तपो॑ वदिष्ये॒ ब्रह्म॑ वदिष्ये स॒त्यं व॑दिष्ये॒ तस्मा॑ अ॒हमि॒दमु॑प॒स्तर॑ण॒मुप॑स्तृण उप॒स्तर॑णं मे प्र॒जायै॑ पशू॒नां भू॑यादुप॒स्तर॑णम॒हं प्र॒जायै॑ पशू॒नां भू॑यासं॒ प्राणा॑पानौ मृ॒त्योर्मा॑ पातं॒ प्राणा॑पानौ॒ मा मा॑ हासिष्टं॒ मधु॑ मनिष्ये॒ मधु॑ जनिष्ये॒ मधु॑ वक्ष्यामि॒ मधु॑ वदिष्यामि॒ मधु॑मतीं दे॒वेभ्यो॒ वाच॑मुद्यास शुश्रू॒षेण्यां मनु॒ष्येभ्य॒स्तं मा॑ दे॒वा अ॑वन्तु शो॒भायै॑ पि॒तरोऽनु॑मदन्तु॥१॥\anuvakamend

%७.२.१
यु॒ञ्जते॒ मन॑ उ॒त यु॑ञ्जते॒ धिय॑। विप्रा॒ विप्र॑स्य बृह॒तो वि॑प॒श्चित॑। वि होत्रा॑ दधे वयुना॒विदेक॒ इत्। म॒ही दे॒वस्य॑ सवि॒तुः परि॑ष्टुतिः। दे॒वस्य॑ त्वा सवि॒तुः प्र॑स॒वे। अ॒श्विनोर्बा॒हुभ्याम्। पू॒ष्णो हस्ताभ्या॒माद॑दे। अभ्रि॑रसि॒ नारि॑रसि। अ॒ध्व॒र॒कृद्दे॒वेभ्य॑। उत्ति॑ष्ठ ब्रह्मणस्पते॥२॥

%७.२.२
दे॒व॒यन्त॑स्त्वेमहे। उप॒ प्रय॑न्तु म॒रुत॑ सु॒दान॑वः। इन्द्र॑ प्रा॒शूर्भ॑वा॒ सचा। प्रैतु॒ ब्रह्म॑ण॒स्पति॑। प्र दे॒व्ये॑तु सू॒नृता। अच्छा॑ वी॒रं नर्यं॑ प॒ङ्क्तिरा॑धसम्। दे॒वा य॒ज्ञं न॑यन्तु नः। देवी द्यावापृथिवी॒ अनु॑ मे मसाथाम्। ऋ॒द्ध्यास॑म॒द्य। म॒खस्य॒ शिर॑॥३॥

%७.२.३
म॒खाय॑ त्वा। म॒खस्य॑ त्वा शी॒र्ष्णे। इय॒त्यग्र॑ आसीः। ऋ॒द्ध्यास॑म॒द्य। म॒खस्य॒ शिर॑। म॒खाय॑ त्वा। म॒खस्य॑ त्वा शी॒र्ष्णे। देवीर्वम्रीर॒स्य भू॒तस्य॑ प्रथमजा ऋतावरीः। ऋ॒द्ध्यास॑म॒द्य। म॒खस्य॒ शिर॑॥४॥

%७.२.४
म॒खाय॑ त्वा। म॒खस्य॑ त्वा शी॒र्ष्णे। इन्द्र॒स्यौजो॑ऽसि। ऋ॒द्ध्यास॑म॒द्य। म॒खस्य॒ शिर॑। म॒खाय॑ त्वा। म॒खस्य॑ त्वा शी॒र्ष्णे। अ॒ग्नि॒जा अ॑सि प्र॒जाप॑ते॒ रेत॑। ऋ॒द्ध्यास॑म॒द्य। म॒खस्य॒ शिर॑॥५॥

%७.२.५
म॒खाय॑ त्वा। म॒खस्य॑ त्वा शी॒र्ष्णे। आयु॑र्धेहि प्रा॒णं धे॑हि। अ॒पा॒नं धे॑हि व्या॒नं धे॑हि। चक्षु॑र्धेहि॒ श्रोत्रं॑ धेहि। मनो॑ धेहि॒ वाचं॑ धेहि। आ॒त्मानं॑ धेहि प्रति॒ष्ठां धे॑हि। मां धे॑हि॒ मयि॑ धेहि। मधु॑ त्वा मधु॒ला क॑रोतु। म॒खस्य॒ शिरो॑ऽसि॥६॥

%७.२.६
य॒ज्ञस्य॑ प॒दे स्थ॑। गा॒य॒त्रेण॑ त्वा॒ छन्द॑सा करोमि। त्रैष्टु॑भेन त्वा॒ छन्द॑सा करोमि। जाग॑तेन त्वा॒ छन्द॑सा करोमि। म॒खस्य॒ रास्ना॑ऽसि। अदि॑तिस्ते॒ बिलं॑ गृह्णातु। पाङ्क्ते॑न॒ छन्द॑सा। सूर्य॑स्य॒ हर॑सा श्राय। म॒खो॑ऽसि॥७॥
\anuvakamend[प॒ते॒ शिर॑ ऋतावरीर्\mbox{}ऋ॒द्ध्यास॑म॒द्य म॒खस्य॒ शिर॒ शिर॒ शिरो॑ऽसि॒ नव॑ च%।२॥इय॑ति॒ देवी॒रिन्द्र॒स्यौजोऽस्यग्नि॒जा अ॒स्यायु॑र्द्धेहि प्रा॒णं पञ्च॑॥ ॥
]

%७.३.१
वृष्णो॒ अश्व॑स्य नि॒ष्पद॑सि। वरु॑णस्त्वा धृ॒तव्र॑त॒ आधू॑पयतु। मि॒त्रावरु॑णयोर्ध्रु॒वेण॒ धर्म॑णा। अ॒र्चिषे त्वा। शो॒चिषे त्वा। ज्योति॑षे त्वा। तप॑से त्वा। अ॒भीमं म॑हि॒ना दिवम्। मि॒त्रो ब॑भूव स॒प्रथा। उ॒त श्रव॑सा पृथि॒वीम्॥८॥

%७.३.२
मि॒त्रस्य॑ चर्\mbox{}षणी॒धृत॑। श्रवो॑ दे॒वस्य॑ सान॒सिम्। द्यु॒म्नं चि॒त्रश्र॑वस्तमम्। सिध्यै त्वा। दे॒वस्त्वा॑ सवि॒तोद्व॑पतु। सु॒पा॒णिः स्व॑ङ्गु॒रिः। सु॒बा॒हुरु॒त शक्त्या। अप॑द्यमानः पृथि॒व्याम्। आशा॒ दिश॒ आ पृ॑ण। उत्ति॑ष्ठ बृ॒हन्भ॑व॥९॥

%७.३.३
ऊ॒र्ध्वस्ति॑ष्ठद्ध्रु॒वस्त्वम्। सूर्य॑स्य त्वा॒ चक्षु॒षाऽन्वीक्षे। ऋ॒जवे त्वा। सा॒धवे त्वा। सु॒क्षि॒त्यै त्वा॒ भूत्यै त्वा। इ॒दम॒हम॒मुमा॑मुष्याय॒णं  वि॒शा प॒शुभि॑र्ब्रह्मवर्च॒सेन॒ पर्यू॑हामि। गा॒य॒त्रेण॑ त्वा॒ छन्द॒साऽऽच्छृ॑णद्मि। त्रैष्टु॑भेन त्वा॒ छन्द॒साऽऽच्छृ॑णद्मि। जाग॑तेन त्वा॒ छन्द॒साऽऽच्छृ॑णद्मि। छृ॒णत्तु॑ त्वा॒ वाक्। छृ॒णत्तु॒ त्वोर्क्। छृ॒णत्तु॑ त्वा ह॒विः। छृ॒न्धि वाचम्। छृ॒न्ध्यूर्जम्। छृ॒न्धि ह॒विः। देव॑ पुरश्चर स॒ग्घ्यासं॑ त्वा॥१०॥
\anuvakamend[पृ॒थि॒वीं भ॑व॒ वाख्षट्च॑]

%७.४.१
ब्रह्म॑न् प्रव॒र्ग्ये॑ण॒ प्रच॑रिष्यामः। होत॑र्घ॒र्मम॒भिष्टु॑हि। अग्नी॒द्रौहि॑णौ पुरो॒डाशा॒वधि॑श्रय। प्रति॑प्रस्थात॒र्विह॑र। प्रस्तो॑त॒ सामा॑नि गाय। यजु॑र्\mbox{}युक्त॒ साम॑भि॒राक्त॑खन्त्वा। विश्वैर्दे॒वैरनु॑मतं म॒रुद्भि॑। दक्षि॑णाभि॒ प्रत॑तं पारयि॒ष्णुम्। स्तुभो॑ वहन्तु सुमन॒स्यमा॑नम्। स नो॒ रुचं॑ धे॒ह्यहृ॑णीयमानः। भूर्भुव॒ सुव॑। ओमिन्द्र॑वन्त॒ प्रच॑रत॥११॥
\anuvakamend[अहृ॑णीयमानो॒ द्वे च॑]

%७.५.१
ब्रह्म॒न्प्रच॑रिष्यामः। होत॑र्घ॒र्मम॒भिष्टु॑हि। य॒माय॑ त्वा म॒खाय॑ त्वा। सूर्य॑स्य॒ हर॑से त्वा। प्रा॒णाय॒ स्वाहा व्या॒नाय॒ स्वाहा॑ऽपा॒नाय॒ स्वाहा। चक्षु॑षे॒ स्वाहा॒ श्रोत्रा॑य॒ स्वाहा। मन॑से॒ स्वाहा॑ वा॒चे सर॑स्वत्यै॒ स्वाहा। दक्षा॑य॒ स्वाहा॒ क्रत॑वे॒ स्वाहा। ओज॑से॒ स्वाहा॒ बला॑य॒ स्वाहा। दे॒वस्त्वा॑ सवि॒ता मध्वा॑ऽनक्तु॥१२॥

%७.५.२
पृ॒थि॒वीं तप॑सस्त्रायस्व। अ॒र्चिर॑सि शो॒चिर॑सि॒ ज्योति॑रसि॒ तपो॑ऽसि। ससी॑दस्व म॒हा अ॑सि। शोच॑स्व देव॒वीत॑मः। विधू॒मम॑ग्ने अरु॒षं मि॑येध्य। सृ॒ज प्र॑शस्तदर्\mbox{}श॒तम्। अ॒ञ्जन्ति॒ यं प्र॒थय॑न्तो॒ न विप्रा। व॒पाव॑न्तं॒ नाग्निना॒ तप॑न्तः। पि॒तुर्न पु॒त्र उप॑सि॒ प्रेष्ठ॑। आ घ॒र्मो अ॒ग्निमृ॒तय॑न्नसादीत्॥१३॥

%७.५.३
अ॒ना॒धृ॒ष्या पु॒रस्तात्। अ॒ग्नेराधि॑पत्ये। आयु॑र्मे दाः। पु॒त्रव॑ती दक्षिण॒तः। इन्द्र॒स्याधि॑पत्ये। प्र॒जां मे॑ दाः। सु॒षदा॑ प॒श्चात्। दे॒वस्य॑ सवि॒तुराधि॑पत्ये। प्रा॒णं मे॑ दाः। आश्रु॑तिरुत्तर॒तः॥१४॥

%७.५.४
मि॒त्रावरु॑णयो॒राधि॑पत्ये। श्रोत्रं॑ मे दाः। विधृ॑तिरु॒परि॑ष्टात्। बृह॒स्पते॒राधि॑पत्ये। ब्रह्म॑ मे दाः क्ष॒त्रं मे॑ दाः। तेजो॑ मे धा॒ वर्चो॑ मे धाः। यशो॑ मे धा॒स्तपो॑ मे धाः। मनो॑ मे धाः। मनो॒रश्वा॑ऽसि॒ भूरि॑पुत्रा। विश्वाभ्यो मा ना॒ष्ट्राभ्य॑ पाहि॥१५॥

%७.५.५
सू॒प॒सदा॑ मे भूया॒ मा मा॑ हिसीः। तपो॒ष्व॑ग्ने॒ अन्त॑रा अ॒मित्रान्॑। तपा॒शस॑मर॒रुष॒ पर॑स्य। तपा॑वसो चिकिता॒नो अ॒चित्तान्॑। वि ते॑ तिष्ठन्ताम॒जरा॑ अ॒यास॑। चित॑ स्थ परि॒चित॑। स्वाहा॑ म॒रुद्भि॒ परि॑श्रयस्व। मा अ॑सि। प्र॒मा अ॑सि। प्र॒ति॒मा अ॑सि॥१६॥

%७.५.६
स॒म्मा अ॑सि। वि॒मा अ॑सि। उ॒न्मा अ॑सि। अ॒न्तरि॑क्षस्यान्त॒र्द्धि\-र॑सि। दिवं॒ तप॑सस्त्रायस्व। आ॒भिर्गी॒र्भिर्यदतो॑ न ऊ॒नम्। आप्या॑यय हरिवो॒ वर्ध॑मानः। य॒दा स्तो॒तृभ्यो॒ महि॑ गो॒त्रा रु॒जासि॑। भू॒यि॒ष्ठ॒भाजो॒ अध॑ ते स्याम। शु॒क्रं ते॑ अ॒न्यद्य॑ज॒तं ते॑ अ॒न्यत्॥१७॥

%७.५.७
विषु॑रूपे॒ अह॑नी॒ द्यौरि॑वासि। विश्वा॒ हि मा॒या अव॑सि स्वधावः। भ॒द्रा ते॑ पूषन्नि॒ह रा॒तिर॑स्तु। अर्\mbox{}ह॑न्बिभर्\mbox{}षि॒ साय॑कानि॒ धन्व॑। अर्\mbox{}हं॑ नि॒ष्कं य॑ज॒तं  वि॒श्वरू॑पम्। अर्\mbox{}हं॑ नि॒दन्द॑यसे॒ विश्व॒मब्भु॑वम्। न वा ओजी॑यो रुद्र॒ त्वद॑स्ति। गा॒य॒त्रम॑सि। त्रैष्टु॑भमसि। जाग॑तमसि। मधु॒ मधु॒ मधु॑॥१८॥
\anuvakamend[अ॒न॒क्त्व॒सा॒दी॒दु॒त्त॒र॒तः पा॑हि प्रति॒मा अ॑सि यज॒तन्ते॑ अ॒न्यज्जाग॑तम॒स्येकं॑ च]

%७.६.१
दश॒ प्राची॒र्दश॑ भासि दक्षि॒णा। दश॑ प्र॒तीची॒र्दश॑ भा॒स्युदी॑चीः। दशो॒र्ध्वा भा॑सि सुमन॒स्यमा॑नः। स नो॒ रुचं॑ धे॒ह्यहृ॑णीयमानः। अ॒ग्निष्ट्वा॒ वसु॑भिः पु॒रस्ताद्रोचयतु गाय॒त्रेण॒ छन्द॑सा। स मा॑ रुचि॒तो रो॑चय। इन्द्र॑स्त्वा रु॒द्रैर्द॑क्षिण॒तो रो॑चयतु॒ त्रैष्टु॑भेन॒ छन्द॑सा। स मा॑ रुचि॒तो रो॑चय। वरु॑णस्त्वादि॒त्यैः प॒श्चाद्रो॑चयतु॒ जाग॑तेन॒ छन्द॑सा। स मा॑ रुचि॒तो रो॑चय॥१९॥

%७.६.२
द्यु॒ता॒नस्त्वा॑ मारु॒तो म॒रुद्भि॑रुत्तर॒तो रो॑चय॒त्वानु॑ष्टुभेन॒ छन्द॑सा। स मा॑ रुचि॒तो रो॑चय। बृह॒स्पति॑स्त्वा॒ विश्वैर्दे॒वैरु॒परि॑ष्टाद्रोचयतु॒ पाङ्क्ते॑न॒ छन्द॑सा। स मा॑ रुचि॒तो रो॑चय। रो॒चि॒तस्त्वं दे॑व घर्म दे॒वेष्वसि॑। रो॒चि॒षी॒याहं म॑नु॒ष्ये॑षु। सम्राड्घर्म रुचि॒तस्त्वं दे॒वेष्वायु॑ष्मास्तेज॒स्वी ब्र॑ह्मवर्च॒स्य॑सि। रु॒चि॒तो॑ऽहं म॑नु॒ष्येष्वायु॑ष्मास्तेज॒स्वी ब्र॑ह्मवर्च॒सी भू॑यासम्। रुग॑सि। रुचं॒ मयि॑ धेहि॥२०॥

%७.६.३
मयि॒ रुक्। दश॑ पु॒रस्ताद्रोचसे। दश॑ दक्षि॒णा। दश॑ प्र॒त्यङ्ङ्। दशोदङ्ङ्॑। दशो॒र्ध्वो भा॑सि सुमन॒स्यमा॑नः। स न॑ सम्रा॒डिष॒मूर्जं॑ धेहि। वा॒जी वा॒जिने॑ पवस्व। रो॒चि॒तो घ॒र्मो रु॑ची॒य॥२१॥
\anuvakamend[रो॒च॒य॒ धे॒हि॒ नव॑ च]

%७.७.१
अप॑श्यं गो॒पामनि॑पद्यमानम्। आ च॒ परा॑ च प॒थिभि॒श्चर॑न्तम्। स स॒ध्रीची॒ स विषू॑ची॒र्वसा॑नः। आ व॑रीवर्ति॒ भुव॑नेष्व॒न्तः। अत्र॑ प्रा॒वीः। मधु॒ माध्वीभ्यां॒ मधु॒ माधू॑चीभ्याम्। अनु॑ वां दे॒ववी॑तये। सम॒ग्निर॒ग्निना॑ गत। सं दे॒वेन॑ सवि॒त्रा। स सूर्ये॑ण रोचते॥२२॥

%७.७.२
स्वाहा॒ सम॒ग्निस्तप॑सा गत। सं दे॒वेन॑ सवि॒त्रा। स सूर्ये॑णारोचिष्ट। ध॒र्ता दि॒वो विभा॑सि॒ रज॑सः। पृ॒थि॒व्या ध॒र्ता। उ॒रोर॒न्तरि॑क्षस्य ध॒र्ता। ध॒र्ता दे॒वो दे॒वानाम्। अम॑र्त्यस्तपो॒जाः। हृ॒दे त्वा॒ मन॑से त्वा। दि॒वे त्वा॒ सूर्या॑य त्वा॥२३॥

%७.७.३
ऊ॒र्ध्वमि॒मम॑ध्व॒रं कृ॑धि। दि॒वि दे॒वेषु॒ होत्रा॑ यच्छ। विश्वा॑सां भुवां पते। विश्व॑स्य भुवनस्पते। विश्व॑स्य मनसस्पते। विश्व॑स्य वचसस्पते। विश्व॑स्य तपसस्पते। विश्व॑स्य ब्रह्मणस्पते। दे॒व॒श्रूस्त्वं दे॑व घर्म दे॒वान्पा॑हि। त॒पो॒जां वाच॑म॒स्मे निय॑च्छ देवा॒युवम्॥२४॥

%७.७.४
गर्भो॑ दे॒वानाम्। पि॒ता म॑ती॒नाम्। पति॑ प्र॒जानाम्। मति॑ कवी॒नाम्। सं दे॒वो दे॒वेन॑ सवि॒त्रा य॑तिष्ट। स सूर्ये॑णारुक्त। आ॒यु॒र्दास्त्वम॒स्मभ्यं॑ घर्म वर्चो॒दा अ॑सि। पि॒ता नो॑ऽसि पि॒ता नो॑ बोध। आ॒यु॒र्द्धास्त॑नू॒धाः प॑यो॒धाः। व॒र्चो॒दा व॑रिवो॒दा द्र॑विणो॒दाः॥२५॥

%७.७.५
अ॒न्त॒रि॒क्ष॒प्र॒ उ॒रोर्वरी॑यान्। अ॒शी॒महि॑ त्वा॒ मा मा॑ हिसीः। त्वम॑ग्ने गृ॒हप॑तिर्वि॒शाम॑सि। विश्वा॑सां॒ मानु॑षीणाम्। श॒तं पू॒र्भिर्य॑विष्ठ पा॒ह्यह॑सः। स॒मे॒द्धार श॒त हिमा। त॒न्द्रा॒विण हार्दिवा॒नम्। इ॒हैव रा॒तय॑ सन्तु। त्वष्टी॑मती ते सपेय। सु॒रेता॒ रेतो॒ दधा॑ना। वी॒रं  वि॑देय॒ तव॑ स॒न्दृशि॑। माऽह रा॒यस्पोषे॑ण॒ वि यो॑षम्॥२६॥
\anuvakamend[रो॒च॒ते॒ सूर्या॑य त्वा देवा॒युवं॑ द्रविणो॒दा दधा॑ना॒ द्वे च॑]


%७.८.१
दे॒वस्य॑ त्वा सवि॒तुः प्र॑स॒वे। अ॒श्विनोर्बा॒हुभ्याम्। पू॒ष्णो हस्ताभ्या॒माद॑दे। अदि॑त्यै॒ रास्ना॑सि। इड॒ एहि॑। अदि॑त॒ एहि॑। सर॑स्व॒त्येहि॑। असा॒वेहि॑। असा॒वेहि॑। असा॒वेहि॑॥२७॥

%७.८.२
अदि॑त्या उ॒ष्णीष॑मसि। वा॒युर॑स्यै॒डः। पू॒षा त्वो॒पाव॑सृजतु। अ॒श्विभ्यां॒ प्रदा॑पय। यस्ते॒ स्तन॑ शश॒यो यो म॑यो॒भूः। येन॒ विश्वा॒ पुष्य॑सि॒ वार्या॑णि। यो र॑त्न॒धा व॑सु॒विद्यः सु॒दत्र॑। सर॑स्वति॒ तमि॒ह धात॑वेकः। उस्र॑ घ॒र्म शिष। उस्र॑ घ॒र्मं पा॑हि॥२८॥

%७.८.३
घ॒र्माय॑ शिष। बृह॒स्पति॒स्त्वोप॑सीदतु। दान॑वः स्थ॒ पेर॑वः। वि॒ष्व॒ग्वृतो॒ लोहि॑तेन। अ॒श्विभ्यां पिन्वस्व। सर॑स्वत्यै पिन्वस्व। पू॒ष्णे पि॑न्वस्व। बृह॒स्पत॑ये पिन्वस्व। इन्द्रा॑य पिन्वस्व। इन्द्रा॑य पिन्वस्व॥२९॥

%७.८.४
गा॒य॒त्रो॑ऽसि। त्रैष्टु॑भोऽसि। जाग॑तमसि। स॒होर्जो भा॒गेनोप॒मेहि॑। इन्द्राश्विना॒ मधु॑नः सार॒घस्य॑। घ॒र्मं पा॑त वसवो॒ यज॑ता॒ वट्। स्वाहा त्वा॒ सूर्य॑स्य र॒श्मये॑ वृष्टि॒वन॑ये जुहोमि। मधु॑ ह॒विर॑सि। सूर्य॑स्य॒ तप॑स्तप। द्यावा॑पृथि॒वीभ्यां त्वा॒ परि॑गृह्णामि॥३०॥

%७.८.५
अ॒न्तरि॑क्षेण॒ त्वोप॑यच्छामि। दे॒वानां त्वा पितृ॒णामनु॑मतो॒ भर्तु शकेयम्। तेजो॑ऽसि। तेजोऽनु॒ प्रेहि॑। दि॒वि॒स्पृङ्मा मा॑ हिसीः। अ॒न्त॒रि॒क्ष॒स्पृङ्मा मा॑ हिसीः। पृ॒थि॒वि॒स्पृङ्मा मा॑ हिसीः। सुव॑रसि॒ सुव॑र्मे यच्छ। दिवं॑ यच्छ दि॒वो मा॑ पाहि॥३१॥
\anuvakamend[एहि॑ पाहि पिन्वस्व गृह्णामि॒ नव॑ च]

%७.९.१
स॒मु॒द्राय॑ त्वा॒ वाता॑य॒ स्वाहा। स॒लि॒लाय॑ त्वा॒ वाता॑य॒ स्वाहा। अ॒ना॒धृ॒ष्याय॑ त्वा॒ वाता॑य॒ स्वाहा। अ॒प्र॒ति॒धृ॒ष्याय॑ त्वा॒ वाता॑य॒ स्वाहा। अ॒व॒स्यवे त्वा॒ वाता॑य॒ स्वाहा। दुव॑स्वते त्वा॒ वाता॑य॒ स्वाहा। शिमि॑द्वते त्वा॒ वाता॑य॒ स्वाहा। अ॒ग्नये त्वा॒ वसु॑मते॒ स्वाहा। सोमा॑य त्वा रु॒द्रव॑ते॒ स्वाहा। वरु॑णाय त्वाऽऽदि॒त्यव॑ते॒ स्वाहा॥३२॥

%७.९.२
बृह॒स्पत॑ये त्वा वि॒श्वदेव्यावते॒ स्वाहा। स॒वि॒त्रे त्व॑र्भु॒मते॑ विभु॒मते प्रभु॒मते॒ वाज॑वते॒ स्वाहा। य॒माय॒ त्वाऽङ्गि॑रस्वते पितृ॒मते॒ स्वाहा। विश्वा॒ आशा॑ दक्षिण॒सत्। विश्वां दे॒वान॑याडि॒ह। स्वाहा॑कृतस्य घ॒र्मस्य॑। मधो पिबतमश्विना। स्वाहा॒ऽग्नये॑ य॒ज्ञिया॑य। शं यजु॑र्भिः। अश्वि॑ना घ॒र्मं पा॑त हार्दिवा॒नम्॥३३॥

%७.९.३
अह॑र्दि॒वाभि॑रू॒तिभि॑। अनु॑ वां॒ द्यावा॑पृथि॒वी मसाताम्। स्वाहेन्द्रा॑य। स्वाहेन्द्रा॒वट्। घ॒र्मम॑पातमश्विना हार्दिवा॒नम्। अह॑र्दि॒वाभि॑रू॒तिभि॑। अनु॑ वां॒ द्यावा॑पृथि॒वी अ॑मसाताम्। तं प्रा॒व्यं॑ यथा॒ वट्। नमो॑ दि॒वे। नम॑ पृथि॒व्यै॥३४॥

%७.९.४
दि॒वि धा॑ इ॒मं य॒ज्ञम्। य॒ज्ञमि॒मं दि॒वि धा। दिवं॑ गच्छ। अ॒न्तरि॑क्षं गच्छ। पृ॒थि॒वीं ग॑च्छ। पञ्च॑ प्र॒दिशो॑ गच्छ। दे॒वान्घ॑र्म॒पान्ग॑च्छ। पि॒तॄन्घ॑र्म॒पान्ग॑च्छ॥३५॥
\anuvakamend[आ॒दि॒त्यव॑ते॒ स्वाहा॑ हार्दिवा॒नं पृ॑थि॒व्या अ॒ष्टौ च॑]

%७.१०.१
इ॒षे पी॑पिहि। ऊ॒र्जे पी॑पिहि। ब्रह्म॑णे पीपिहि। क्ष॒त्राय॑ पीपिहि। अ॒द्भ्यः पी॑पिहि। ओष॑धीभ्यः पीपिहि। वन॒स्पति॑भ्यः पीपिहि। द्यावा॑पृ॒थिवीभ्यां पीपिहि। सु॒भू॒ताय॑ पीपिहि। ब्र॒ह्म॒व॒र्च॒साय॑ पीपिहि॥३६॥

%७.१०.२
यज॑मानाय पीपिहि। मह्यं॒ ज्यैष्ठ्या॑य पीपिहि। त्विष्यै त्वा। द्यु॒म्नाय॑ त्वा। इ॒न्द्रि॒याय॑ त्वा॒ भूत्यै त्वा। धर्मा॑ऽसि सु॒धर्मा मे न्य॒स्मे। ब्रह्मा॑णि धारय। क्ष॒त्राणि॑ धारय। विशं॑ धारय। नेत्त्वा॒ वात॑ स्क॒न्दयात्॥३७॥

%७.१०.३
अ॒मुष्य॑ त्वा प्रा॒णे सा॑दयामि। अ॒मुना॑ स॒ह नि॑र॒र्थं ग॑च्छ। योऽस्मान्द्वेष्टि॑। यं च॑ व॒यं द्वि॒ष्मः। पू॒ष्णे शर॑से॒ स्वाहा। ग्राव॑भ्य॒ स्वाहा। प्र॒ति॒रेभ्य॒ स्वाहा। द्यावा॑पृथि॒वीभ्या॒ स्वाहा। पि॒तृभ्यो॑ घर्म॒पेभ्य॒ स्वाहा। रु॒द्राय॑ रु॒द्रहोत्रे॒ स्वाहा॥३८॥

%७.१०.४
अह॒र्ज्योति॑ के॒तुना॑ जुषताम्। सु॒ज्यो॒तिर्ज्योति॑षा॒ स्वाहा। रात्रि॒र्ज्योति॑ के॒तुना॑ जुषताम्। सु॒ज्यो॒तिर्ज्योति॑षा॒ स्वाहा। अपी॑परो॒ माऽह्नो॒ रात्रि॑यै मा पाहि। ए॒षा ते॑ अग्ने स॒मित्। तया॒ समि॑ध्यस्व। आयु॑र्मे दाः। वर्च॑सा माञ्जीः। अपी॑परो मा॒ रात्रि॑या॒ अह्नो॑ मा पाहि॥३९॥

%७.१०.५
ए॒षा ते॑ अग्ने स॒मित्। तया॒ समि॑ध्यस्व। आयु॑र्मे दाः। वर्च॑सा माञ्जीः। अ॒ग्निर्ज्योति॒र्ज्योति॑र॒ग्निः स्वाहा। सूर्यो॒ ज्योति॒र्ज्योति॒ सूर्य॒ स्वाहा। भूः स्वाहा। हु॒त ह॒विः। मधु॑ ह॒विः। इन्द्र॑तमे॒ऽग्नौ॥४०॥

%७.१०.६
पि॒ता नो॑ऽसि॒ मा मा॑ हिसीः। अ॒श्याम॑ ते देवघर्म। मधु॑मतो॒ वाज॑वतः पितु॒मत॑। अङ्गि॑रस्वतः स्वधा॒विन॑। अ॒शी॒महि॑ त्वा॒ मा मा॑ हिसीः। स्वाहा त्वा॒ सूर्य॑स्य र॒श्मिभ्य॑। स्वाहा त्वा॒ नक्ष॑त्रेभ्यः॥४१॥
\anuvakamend[ब्र॒ह्म॒व॒र्च॒साय॑ पीपिहि स्क॒न्दयाद्रु॒द्राय॑ रु॒द्रहोत्रे॒ स्वाहाऽह्नो॑ मा पाह्य॒ग्नौ स॒प्त च॑]

%७.११.१
घर्म॒ या ते॑ दि॒वि शुक्। या गा॑य॒त्रे छन्द॑सि। या ब्राह्म॒णे। या ह॑वि॒र्द्धाने। तान्त॑ ए॒तेनाव॑ यजे॒ स्वाहा। घर्म॒ या ते॒ऽन्तरि॑क्षे॒ शुक्। या त्रैष्टु॑भे॒ छन्द॑सि। या रा॑ज॒न्ये। याऽऽग्नीध्रे। तान्त॑ ए॒तेनाव॑ यजे॒ स्वाहा॥४२॥

%७.११.२
घर्म॒ या ते॑ पृथि॒व्या शुक्। या जाग॑ते॒ छन्द॑सि। या वैश्ये। या सद॑सि। तान्त॑ ए॒तेनाव॑ यजे॒ स्वाहा। अनु॑नो॒ऽद्यानु॑मतिः। अन्विद॑नुमते॒ त्वम्। दि॒वस्त्वा॑ पर॒स्पाया। अ॒न्तरि॑क्षस्य त॒नुव॑ पाहि। पृ॒थि॒व्यास्त्वा॒ धर्म॑णा॥४३॥

%७.११.३
व॒यमनु॑क्रामाम सुवि॒ताय॒ नव्य॑से। ब्रह्म॑णस्त्वा पर॒स्पाया। क्ष॒त्रस्य॑ त॒नुव॑ पाहि। वि॒शस्त्वा॒ धर्म॑णा। व॒यमनु॑क्रामाम सुवि॒ताय॒ नव्य॑से। प्रा॒णस्य॑ त्वा पर॒स्पायै। चक्षु॑षस्त॒नुव॑ पाहि। श्रोत्र॑स्य त्वा॒ धर्म॑णा। व॒यमनु॑क्रामाम सुवि॒ताय॒ नव्य॑से। व॒ल्गुर॑सि शं॒ युधा॑याः॥४४॥

%७.११.४
शिशु॒र्जन॑धायाः। शं च॒ वक्षि॒ परि॑ च॒ वक्षि॑। चतु॑ स्रक्ति॒र्नाभि॑र्\mbox{}ऋ॒तस्य॑। सदो॑ वि॒श्वायु॒ शर्म॑ स॒प्रथा। अप॒ द्वेषो॒ अप॒ह्वर॑। अ॒न्यद्व्र॑तस्य सश्चिम। घर्मै॒तत्तेऽन्न॑मे॒तत्पुरी॑षम्। तेन॒ वर्ध॑स्व॒ चाऽऽ च॑ प्यायस्व। व॒र्धि॒षी॒महि॑ च व॒यम्। आ च॑ प्यासिषी॒महि॑॥४५॥

%७.११.५
रन्ति॒र्नामा॑सि दि॒व्यो ग॑न्ध॒र्वः। तस्य॑ ते प॒द्वद्ध॑वि॒र्द्धानम्। अ॒ग्निरध्य॑क्षाः। रु॒द्रोऽधि॑पतिः। सम॒हमायु॑षा। सं प्रा॒णेन॑। सं वर्च॑सा। सं पय॑सा। सं गौ॑प॒त्येन॑। स रा॒यस्पोषे॑ण॥४६॥

%७.११.६
व्य॑सौ। योऽस्मान्द्वेष्टि॑। यं च॑ व॒यं द्वि॒ष्मः। अचि॑क्रद॒द्वृषा॒ हरि॑। म॒हान्मि॒त्रो न द॑र्\mbox{}श॒तः। स सूर्ये॑ण रोचते। चिद॑सि समु॒द्रयो॑निः। इन्दु॒र्दक्ष॑ श्ये॒न ऋ॒तावा। हिर॑ण्यपक्षः  शकु॒नो भु॑र॒ण्युः। म॒हान्त्स॒धस्थे ध्रु॒व आनिष॑त्तः॥४७॥

%७.११.७
नम॑स्ते अस्तु॒ मा मा॑ हिसीः। वि॒श्वाव॑सु सोम गन्ध॒र्वम्। आपो॑ ददृ॒शुषी। तदृ॒तेना॒व्या॑यन्। तद॒न्ववैत्। इन्द्रो॑ रारहा॒ण आ॑साम्। परि॒ सूर्य॑स्य परि॒धी र॑पश्यत्। वि॒श्वाव॑सुर॒भि तन्नो॑ गृणातु। दि॒व्यो ग॑न्ध॒र्वो रज॑सो वि॒मान॑। यद्वा॑ घा स॒त्यमु॒त यन्न वि॒द्म॥४७॥

%७.११.८
धियो॑ हिन्वा॒नो धिय॒ इन्नो॑ अव्यात्। सस्नि॑मविन्द॒च्चर॑णे न॒दीनाम्। अपा॑वृणो॒द्दुरो॒ अश्म॑व्रजानाम्। प्रासान्गन्ध॒र्वो अ॒मृता॑नि वोचत्। इन्द्रो॒ दक्षं॒ परि॑जानाद॒हीनम्। ए॒तत्त्वं दे॑व घर्म दे॒वो दे॒वानुपा॑गाः। इ॒दम॒हं म॑नु॒ष्यो॑ मनु॒ष्यान्॑। सोम॑पी॒थानु॒मेहि॑। स॒ह प्र॒जया॑ स॒ह रा॒यस्पोषे॑ण। सु॒मि॒त्रा न॒ आप॒ ओष॑धयः सन्तु॥४९॥

%७.११.९
दु॒र्मि॒त्रास्तस्मै॑ भूयासुः। योऽस्मान्द्वेष्टि॑। यं च॑ व॒यं द्वि॒ष्मः। उद्व॒यं तम॑स॒स्परि॑। उदु॒त्यं चि॒त्रम्। इ॒ममू॒षुत्यम॒स्मभ्य स॒निम्। गा॒य॒त्रं नवी॑यासम्। अग्ने॑ दे॒वेषु॒ प्रवो॑चः॥५०॥
\anuvakamend[याऽऽग्नीध्रे॒ तान्त॑ ए॒तेनाव॑ यजे॒ स्वाहा॒ धर्म॑णा शं॒ युधा॑याः प्यासिषी॒महि॒ पोषे॑ण॒ निष॑त्तो वि॒द्म स॑न्त्व॒ष्टौ]

%७.१२.१
म॒ही॒नां पयो॑ऽसि॒ विहि॑तं देव॒त्रा। ज्यो॒ति॒र्भा अ॑सि॒ वन॒स्पती॑ना॒\-मोष॑धीना॒ रस॑। वा॒जिनं॑ त्वा वा॒जिनोऽव॑ नयामः। ऊ॒र्ध्वं मन॑ सुव॒र्गम्॥५१॥
\anuvakamend

%७.१३.१
अस्का॒न्द्यौः पृ॑थि॒वीम्। अस्का॑नृष॒भो युवा॒गाः। स्क॒न्नेमा विश्वा॒ भुव॑ना। स्क॒न्नो य॒ज्ञः प्रज॑नयतु। अस्का॒नज॑नि॒ प्राज॑नि। आ स्क॒न्नाज्जा॑यते॒ वृषा। स्क॒न्नात् प्रज॑निषीमहि॥५२॥
\anuvakamend

%७.१४.१
या पु॒रस्ताद्वि॒द्युदाप॑तत्। तान्त॑ ए॒तेनाव॑ यजे॒ स्वाहा। या द॑क्षिण॒तः। या प॒श्चात्। योत्त॑र॒तः। योपरि॑ष्टाद्वि॒द्युदाप॑तत्। तान्त॑ ए॒तेनाव॑ यजे॒ स्वाहा॥५३॥
\anuvakamend

%७.१५.१
प्रा॒णाय॒ स्वाहा व्या॒नाय॒ स्वाहा॑ऽपा॒नाय॒ स्वाहा। चक्षु॑षे॒ स्वाहा॒ श्रोत्रा॑य॒ स्वाहा। मन॑से॒ स्वाहा॑ वा॒चे सर॑स्वत्यै॒ स्वाहा॥५४॥%
\anuvakamend

%७.१६.१
पू॒ष्णे स्वाहा॑ पू॒ष्णे शर॑से॒ स्वाहा। पू॒ष्णे प्र॑प॒त्थ्या॑य॒ स्वाहा॑ पू॒ष्णे न॒रन्धि॑षाय॒ स्वाहा। पू॒ष्णेऽङ्घृ॑णये॒ स्वाहा॑ पू॒ष्णे न॒रुणा॑य॒ स्वाहा। पू॒ष्णे सा॑के॒ताय॒ स्वाहा॥५५॥
\anuvakamend


%७.१७.१
उद॑स्य॒ शुष्माद्भा॒नुर्नार्त॒ बिभ॑र्ति। भा॒रं पृ॑थि॒वी न भूम॑। प्र शु॒क्रैतु॑ दे॒वी म॑नी॒षा। अ॒स्मत्सुत॑ष्टो॒ रथो॒ न वा॒जी। अर्च॑न्त॒ एके॒ महि॒ साम॑मन्वत। तेन॒ सूर्य॑मधारयन्। तेन॒ सूर्य॑मरोचयन्। घ॒र्मः  शिर॒स्तद॒यम॒ग्निः। पुरी॑षमसि॒ सं प्रि॑यं प्र॒जया॑ प॒शुभि॑र्भुवत्। प्र॒जापति॑स्त्वा सादयतु। तया॑ दे॒वत॑याऽङ्गिर॒स्वद्ध्रु॒वा सी॑द॥५६॥
\anuvakamend

%७.१८.१
यास्ते॑ अग्न आ॒र्द्रा योन॑यो॒ याः कु॑ला॒यिनी। ये ते॑ अग्न॒ इन्द॑वो॒ या उ॒ नाभ॑यः। यास्ते॑ अग्ने त॒नुव॒ ऊर्जो॒ नाम॑। ताभि॒स्त्वमु॒भयी॑भिः संविदा॒नः। प्र॒जाभि॑रग्ने॒ द्रवि॑णे॒ह सी॑द। प्र॒जाप॑तिस्त्वा सादयतु। तया॑ दे॒वत॑याऽङ्गिर॒स्वद्ध्रु॒वा सी॑द॥५७॥
\anuvakamend

%७.१९.१
अ॒ग्निर॑सि वैश्वान॒रो॑ऽसि। सं॒व॒त्स॒रो॑ऽसि परिवत्स॒रो॑ऽसि। इ॒दा॒व॒त्स॒रो॑ऽसीदुवत्स॒रो॑ऽसि। इ॒द्व॒त्स॒रो॑ऽसि वत्स॒रो॑ऽसि। तस्य॑ ते वस॒न्तः  शिर॑। ग्री॒ष्मो दक्षि॑णः प॒क्षः। व॒र्\mbox{}षाः पुच्छम्। श॒रदुत्त॑रः प॒क्षः। हे॒म॒न्तो मध्यम्। पू॒र्व॒प॒क्षाश्चित॑यः। अ॒प॒र॒प॒क्षाः पुरी॑षम्। अ॒हो॒रा॒त्राणीष्ट॑काः। तस्य॑ ते॒ मासाश्चार्द्धमा॒साश्च॑ कल्पन्ताम्। ऋ॒तव॑स्ते कल्पन्ताम्। सं॒व॒त्स॒रस्ते॑ कल्पताम्। अ॒हो॒रा॒त्राणि॑ ते कल्पन्ताम्। एति॒ प्रेति॒ वीति॒ समित्युदिति॑। प्र॒जाप॑तिस्त्वा सादयतु। तया॑ दे॒वत॑याऽङ्गिर॒स्वद्ध्रु॒वः सी॑द॥५८॥
\anuvakamend[चित॑यो॒ नव॑ च]


%७.२०.१
भूर्भुव॒ सुव॑। ऊ॒र्ध्व ऊ॒षुण॑ ऊ॒तये। ऊ॒र्ध्वो न॑ पा॒ह्यह॑सः। वि॒धुन्द॑द्रा॒ण सम॑ने बहू॒नाम्। युवा॑न॒ सन्तं॑ पलि॒तो ज॑गार। दे॒वस्य॑ पश्य॒ काव्यं॑ महि॒त्वाद्या म॒मार॑। सह्य॒ समा॑न। यदृ॒ते चि॑दभि॒श्रिष॑। पु॒रा ज॒र्तृभ्य॑ आ॒तृद॑। सन्धा॑ता स॒न्धिं म॒घवा॑ पुरो॒वसु॑॥५९॥

%७.२०.२
निष्क॑र्ता॒ विह्रु॑तं॒ पुन॑। पुन॑रू॒र्जा स॒ह र॒य्या। मा नो॑ घर्म व्यथि॒तो वि॑व्यथो नः। मा न॒ पर॒मध॑रं॒ मा रजो॑ऽनैः। मोष्व॑स्मा स्तम॑स्यन्त॒रा धा। मा रु॒द्रिया॑सो अ॒भिगु॑र्वृ॒धान॑। मा न॒ क्रतु॑भिर्\mbox{}हीडि॒तेभि॑र॒स्मान्। द्विषा॑सुनीते॒ मा परा॑ दाः। मा नो॑ रु॒द्रो निर्\mbox{}ऋ॑ति॒र्मा नो॒ अस्ता। मा द्यावा॑पृथि॒वी ही॑डिषाताम्॥६०॥

%७.२०.३
उप॑ नो मित्रावरुणावि॒हाव॑तम्। अ॒न्वादीध्याथामि॒ह न॑ सखाया। आ॒दि॒त्यानां॒ प्रसि॑तिर्\mbox{}हे॒तिः। उ॒ग्रा श॒तापाष्ठा घ॒विषा॒ परि॑ णो वृणक्तु। इ॒मं मे॑ वरुण॒ तत्त्वा॑ यामि। त्वं नो॑ अग्ने॒ स त्वं नो॑ अग्ने। त्वम॑ग्ने अ॒यासि॑। उद्व॒यं तम॑स॒स्परि॑। उदु॒त्यं चि॒त्रम्। वय॑ सुप॒र्णाः॥६१॥
\anuvakamend[पु॒रो॒वसु॑र्\mbox{}हीडिषाता सुप॒र्णाः]


%७.२१.१
भूर्भुव॒ सुव॑। मयि॒ त्यदि॑न्द्रि॒यं म॒हत्। मयि॒ दक्षो॒ मयि॒ क्रतु॑। मयि॑ धायि सु॒वीर्यम्। त्रिशु॑ग्घ॒र्मो विभा॑तु मे। आकूत्या॒ मन॑सा स॒ह। वि॒राजा॒ ज्योति॑षा स॒ह। य॒ज्ञेन॒ पय॑सा स॒ह। ब्रह्म॑णा॒ तेज॑सा स॒ह। क्ष॒त्रेण॒ यश॑सा स॒ह। स॒त्येन॒ तप॑सा स॒ह। तस्य॒ दोह॑मशीमहि। तस्य॑ सु॒म्नम॑शीमहि। तस्य॑ भ॒क्षम॑शीमहि। तस्य॑ त॒ इन्द्रे॑ण पी॒तस्य॒ मधु॑मतः। उप॑हूत॒स्योप॑हूतो भक्षयामि॥६२॥
\anuvakamend[यश॑सा स॒ह षट्च॑]

%७.२२.१
यास्ते॑ अग्ने घो॒रास्त॒नुव॑। क्षुच्च॒ तृष्णा च॑। अस्नु॒क्चाना॑हुतिश्च। अ॒श॒न॒या च॑ पिपा॒सा च॑। से॒दिश्चाम॑तिश्च। ए॒तास्ते॑ अग्ने घो॒रास्त॒नुव॑। ताभि॑र॒मुं ग॑च्छ। योऽस्मान्द्वेष्टि॑। यं च॑ व॒यं द्वि॒ष्मः॥६३॥
\anuvakamend

%७.२३.१
स्निक्च॒ स्नीहि॑तिश्च॒ स्निहि॑तिश्च। उ॒ष्णा च॑ शी॒ता च॑। उ॒ग्रा च॑ भी॒मा च॑। स॒दाम्नी॑ से॒दिरनि॑रा। ए॒तास्ते॑ अग्ने घो॒रास्त॒नुव॑। ताभि॑र॒मुं ग॑च्छ। योऽस्मान्द्वेष्टि॑। यं च॑ व॒यं द्वि॒ष्मः॥६४॥
\anuvakamend

%७.२४.१
धुनि॑श्च ध्वा॒न्तश्च॑ ध्व॒नश्च॑ ध्व॒नयश्च। नि॒लि॒म्पश्च॑ विलि॒म्पश्च॑ विक्षि॒पः॥६५॥
\anuvakamend

%७.२५.१
उ॒ग्रश्च॒ धुनि॑श्च ध्वा॒न्तश्च॑ ध्व॒नश्च॑ ध्व॒नयश्च। स॒ह॒स॒ह्वाश्च॒ सह॑मानश्च॒ सह॑स्वाश्च॒ सही॑याश्च। एत्य॒ प्रेत्य॑ विक्षि॒पः॥६६॥
\anuvakamend


%७.२६.१
अ॒हो॒रा॒त्रे त्वोदी॑रयताम्। अ॒र्ध॒मा॒सास्त्वोदीं जयन्तु। मासास्त्वा श्रपयन्तु। ऋ॒तव॑स्त्वा पचन्तु। सं॒व॒त्स॒रस्त्वा॑ हन्त्वसौ॥६७॥
\anuvakamend

%७.२७.१
खट् फट् ज॒हि। छि॒न्धी भि॒न्धी ह॒न्धी कट्। इति॒ वाच॑ क्रूरा॒णि॥६८॥
\anuvakamend

%७.२८.१
विगा इ॑न्द्र वि॒चरन्त्स्पाशयस्व। स्व॒पन्त॑मिन्द्र पशु॒मन्त॑मिच्छ। वज्रे॑णा॒मुं बो॑धय दुर्वि॒दत्रम्। स्व॒प॒तोऽस्य॒ प्रह॑र॒ भोज॑नेभ्यः। अग्ने॑ अ॒ग्निना॒ संव॑दस्व। मृत्यो॑ मृ॒त्युना॒ संव॑दस्व। नम॑स्ते अस्तु भगवः। स॒कृत्ते॑ अग्ने॒ नम॑। द्विस्ते॒ नम॑। त्रिस्ते॒ नम॑। च॒तुस्ते॒ नम॑। प॒ञ्च॒कृत्व॑स्ते॒ नम॑। द॒श॒कृत्व॑स्ते॒ नम॑। श॒त॒कृत्व॑स्ते॒ नम॑। आ॒स॒ह॒स्र॒कृत्व॑स्ते॒ नम॑। अ॒प॒रि॒मि॒त॒कृत्व॑स्ते॒ नम॑। नम॑स्ते अस्तु॒ मा मा॑ हिसीः॥६९॥
\anuvakamend[त्रिस्ते॒ नम॑ स॒प्त च॑]

%७.२९.१
असृ॑न्मुखो रुधि॒रेणा॒व्य॑क्तः। य॒मस्य॑ दू॒तः  श्वपा॒द्विधा॑वसि। गृध्र॑ सुप॒र्णः कु॒णपं॒ निषे॑वसे। य॒मस्य॑ दू॒तः प्रहि॑तो भ॒वस्य॑ चो॒भयो॥७०॥
\anuvakamend

%७.३०.१
यदे॒तद्वृ॑क॒सो भू॒त्वा। वाग्देव्यभि॒राय॑सि। द्वि॒षन्तं॑ मे॒ऽभिरा॑य। तं मृ॑त्यो मृ॒त्यवे॑ नय। स आर्त्यार्ति॒मार्च्छ॑तु॥७१॥
\anuvakamend

%७.३१.१
यदी॑षि॒तो यदि॑ वा स्वका॒मी। भ॒येड॑को॒ वद॑ति॒ वाच॑मे॒ताम्। तामि॑न्द्रा॒ग्नी ब्रह्म॑णा संविदा॒नौ। शि॒वाम॒स्मभ्यं॑ कृणुतं गृ॒हेषु॑॥७२॥
\anuvakamend

%७.३२.१
दीर्घ॑मुखि॒ दुर्\mbox{}ह॑णु। मा स्म॑ दक्षिण॒तो व॑दः। यदि॑ दक्षिण॒तो वदाद्द्वि॒षन्तं॒ मेऽव॑ बाधासै॥७३॥
\anuvakamend

%७.३३.१
इ॒त्थादुलू॑क॒ आप॑प्तत्। हि॒र॒ण्या॒क्षो अयो॑मुखः। रक्ष॑सां दू॒त आग॑तः। तमि॒तो ना॑शयाग्ने॥७४॥
\anuvakamend


%७.३४.१
यदे॒तद्भू॒तान्य॑न्वा॒विश्य॑। दैवीं॒ वाचं॑ व॒दसि॑। द्वि॒षतो॑ न॒ परा॑वद। तान्मृ॑त्यो मृ॒त्यवे॑ नय। त आर्त्याऽऽर्ति॒मार्च्छ॑न्तु। अ॒ग्निना॒ऽग्निः संव॑दताम्॥७५॥
\anuvakamend


%७.३५.१
प्र॒सार्य॑ स॒क्थ्यौ॑ पत॑सि। स॒व्यमक्षि॑ नि॒पेपि॑ च। मेहक॑स्य च॒नाम॑मत्॥७६॥
\anuvakamend


%७.३६.१
अत्रि॑णा त्वा क्रिमे हन्मि। कण्वे॑न ज॒मद॑ग्निना। वि॒श्वाव॑सो॒र्ब्रह्म॑णा ह॒तः। क्रिमी॑णा॒ राजा। अप्ये॑षा स्थ॒पति॑र्\mbox{}ह॒तः। अथो॑ मा॒ताऽथो॑ पि॒ता। अथो स्थू॒रा अथो क्षु॒द्राः। अथो॑ कृ॒ष्णा अथो श्वे॒ताः। अथो॑ आ॒शाति॑का ह॒ताः। श्वे॒ताभि॑ स॒ह सर्वे॑ ह॒ताः॥७७॥
\anuvakamend


%७.३७.१
आह॒राव॑द्य। शृ॒तस्य॑ ह॒विषो॒ यथा। तत्स॒त्यम्। यद॒मुं य॒मस्य॒ जम्भ॑योः। आद॑धामि॒ तथा॒ हि तत्। खण्फण्म्रसि॑॥७८॥
\anuvakamend


%७.३८.१
ब्रह्म॑णा त्वा शपामि। ब्रह्म॑णस्त्वा श॒पथे॑न शपामि। घो॒रेण॑ त्वा॒ भृगू॑णां॒ चक्षु॑षा॒ प्रेक्षे। रौ॒द्रेण॒ त्वाङ्गि॑रसां॒ मन॑सा ध्यायामि। अ॒घस्य॑ त्वा॒ धार॑या विद्ध्यामि। अध॑रो॒ मत्प॑द्यस्वाऽसौ॥७९॥%
\anuvakamend


%७.३९.१
उत्तु॑द शिमिजावरि। तल्पे॑जे॒ तल्प॒ उत्तु॑द। गि॒री रनु॒ प्रवे॑शय। मरी॑ची॒रुप॒ सन्नु॑द। याव॑दि॒तः पु॒रस्ता॑दु॒दया॑ति॒ सूर्य॑। ताव॑दि॒तो॑ऽमुं ना॑शय। योऽस्मान्द्वेष्टि॑। यं च॑ व॒यं द्वि॒ष्मः॥८०॥
\anuvakamend


%७.४०.१
भूर्भुव॒ सुवो॒ भूर्भुव॒ सुवो॒ भूर्भुव॒ सुव॑। भुवोऽद्धायि॒ भुवोऽद्धायि॒ भुवोऽद्धायि। नृ॒म्णायि नृ॒म्णं नृ॒म्णायि नृ॒म्णं नृ॒म्णायि नृ॒म्णम्। नि॒धाय्यो॑ वायि नि॒धाय्यो॑ वायि नि॒धाय्यो॑ वायि। ए अ॒स्मे अ॒स्मे। सुव॒र्न ज्योती॥८१॥
\anuvakamend


%७.४१.१
पृ॒थि॒वी स॒मित्। ताम॒ग्निः समि॑न्धे। साऽग्नि समि॑न्धे। ताम॒ह समि॑न्धे। सा मा॒ समि॑द्धा। आयु॑षा॒ तेज॑सा। वर्च॑सा श्रि॒या। यश॑सा ब्रह्मवर्च॒सेन॑। अ॒न्नाद्ये॑न॒ समि॑न्ता॒ स्वाहा। अ॒न्तरि॑क्ष स॒मित्॥८२॥

%७.४१.२
तां वा॒युः समि॑न्धे। सा वा॒यु समि॑न्धे। ताम॒ह समि॑न्धे। सा मा॒ समि॑द्धा। आयु॑षा॒ तेज॑सा। वर्च॑सा श्रि॒या। यश॑सा ब्रह्मवर्च॒सेन॑। अ॒न्नाद्ये॑न॒ समि॑न्ता॒ स्वाहा। द्यौः स॒मित्। तामा॑दि॒त्यः समि॑न्धे॥८३॥

%७.४१.३
साऽऽदि॒त्य समि॑न्धे। ताम॒ह समि॑न्धे। सा मा॒ समि॑द्धा। आयु॑षा॒ तेज॑सा। वर्च॑सा श्रि॒या। यश॑सा ब्रह्मवर्च॒सेन॑। अ॒न्नाद्ये॑न॒ समि॑न्ता॒ स्वाहा। प्रा॒जा॒प॒त्या मे॑ स॒मिद॑सि सपत्न॒क्षय॑णी। भ्रा॒तृ॒व्य॒हा मे॑ऽसि॒ स्वाहा। अग्ने व्रतपते व्र॒तं च॑रिष्यामि॥८४॥

%७.४१.४
तच्छ॑केयं॒ तन्मे॑ राध्यताम्। वायो व्रतपत॒ आदि॑त्य व्रतपते। व्र॒तानां व्रतपते व्र॒तं च॑रिष्यामि। तच्छ॑केयं॒ तन्मे॑ राध्यताम्। द्यौः स॒मित्। तामा॑दि॒त्यः समि॑न्धे। साऽऽदि॒त्य समि॑न्धे। ताम॒ह समि॑न्धे। सा मा॒ समि॑द्धा। आयु॑षा॒ तेज॑सा॥८५॥

%७.४१.५
वर्च॑सा श्रि॒या। यश॑सा ब्रह्मवर्च॒सेन॑। अ॒न्नाद्ये॑न॒ समि॑न्ता॒ स्वाहा। अ॒न्तरि॑क्ष स॒मित्। तां वा॒युः समि॑न्धे। सा वा॒यु समि॑न्धे। ताम॒ह समि॑न्धे। सा मा॒ समि॑द्धा। आयु॑षा॒ तेज॑सा। वर्च॑सा श्रि॒या॥८६॥

%७.४१.६
यश॑सा ब्रह्मवर्च॒सेन॑। अ॒न्नाद्ये॑न॒ समि॑न्ता॒ स्वाहा। पृ॒थि॒वी स॒मित्। ताम॒ग्निः समि॑न्धे। साऽग्नि समि॑न्धे। ताम॒ह समि॑न्धे। सा मा॒ समि॑द्धा। आयु॑षा॒ तेज॑सा। वर्च॑सा श्रि॒या। यश॑सा ब्रह्मवर्च॒सेन॑॥८७॥

%७.४१.७
अ॒न्नाद्ये॑न॒ समि॑न्ता॒ स्वाहा। प्रा॒जा॒प॒त्या मे॑ स॒मिद॑सि सपत्न॒क्षय॑णी। भ्रा॒तृ॒व्य॒हा मे॑ऽसि॒ स्वाहा। आदि॑त्य व्रतपते व्र॒तम॑चारिषम्। तद॑शकं॒ तन्मे॑ऽराधि। वायो व्रतप॒तेऽग्ने व्रतपते। व्र॒तानां व्रतपते व्र॒तम॑चारिषम्। तद॑शकं॒ तन्मे॑ऽराधि॥८८॥
\anuvakamend[स॒मित्समि॑न्धे व्र॒तं च॑रिष्या॒म्यायु॑षा॒ तेज॑सा॒ वर्च॑सा श्रि॒या यश॑सा ब्रह्मवर्च॒सेना॒ष्टौ च॑]


%७.४२.१
शं नो॒ वात॑ पवतां मात॒रिश्वा॒ शं न॑स्तपतु॒ सूर्य॑। अहा॑नि॒शं भ॑वन्तु न॒ श रात्रि॒ प्रति॑धीयताम्। शमु॒षा नो॒ व्यु॑च्छतु॒ शमा॑दि॒त्य उदे॑तु नः। शि॒वा न॒ शन्त॑मा भव सुमृडी॒का सर॑स्वति। मा ते॒ व्यो॑म स॒न्दृशि॑। इडा॑यै॒ वास्त्व॑सि वास्तु॒मद्वास्तु॒मन्तो॑ भूयास्म॒ मा वास्तोश्छित्स्मह्यवा॒स्तुः स भू॑या॒द्योऽस्मान्द्वेष्टि॒ यं च॑ व॒यं द्वि॒ष्मः। प्र॒ति॒ष्ठासि॑ प्रति॒ष्ठाव॑न्तो भूयास्म॒ मा प्र॑ति॒ष्ठायाश्छित्स्मह्यप्रति॒ष्ठः स भू॑या॒द्योऽस्मान्द्वेष्टि॒ यं च॑ व॒यं द्वि॒ष्मः। आ वा॑त वाहि भेष॒जं वि वा॑त वाहि॒ यद्रप॑। त्व हि वि॒श्वभे॑षजो दे॒वानां दू॒त ईय॑से। द्वावि॒मौ वातौ॑ वात॒ आ सिन्धो॒रा प॑रा॒वत॑॥८९॥

%७.४२.२
दक्षं॑ मे अ॒न्य आ॒वातु॒ परा॒न्यो वा॑तु॒ यद्रप॑। यद॒दो वा॑तते गृ॒हे॑ऽमृत॑स्य नि॒धिर्\mbox{}हि॒तः। ततो॑ नो देहि जी॒वसे॒ ततो॑ नो धेहि भेष॒जम्। ततो॑ नो॒ मह॒ आव॑ह॒ वात॒ आवा॑तु भेष॒जम्। श॒म्भूर्म॑यो॒भूर्नो॑ हृ॒दे प्र ण॒ आयूषि तारिषत्। इन्द्र॑स्य गृ॒हो॑ऽसि॒ तं त्वा॒ प्रप॑द्ये॒ सगु॒ साश्व॑। स॒ह यन्मे॒ अस्ति॒ तेन॑। भूः प्रप॑द्ये॒ भुव॒ प्रप॑द्ये॒ सुव॒ प्रप॑द्ये॒ भूर्भुव॒ सुव॒ प्रप॑द्ये वा॒युं प्रप॒द्येऽनार्तां दे॒वतां॒ प्रप॒द्येऽश्मा॑नमाख॒णं प्रप॑द्ये प्र॒जाप॑तेर्ब्रह्मको॒शं ब्रह्म॒ प्रप॑द्य॒ ओं प्रप॑द्ये। अ॒न्तरि॑क्षं म उ॒र्व॑न्तरं॑ बृ॒हद॒ग्नय॒ पर्व॑ताश्च॒ यया॒ वात॑ स्व॒स्त्या स्व॑स्ति॒मान्तया स्व॒स्त्या स्व॑स्ति॒मान॑सानि। प्राणा॑पानौ मृ॒त्योर्मा॑ पातं॒ प्राणा॑पानौ॒ मा मा॑ हासिष्टं॒ मयि॑ मे॒धां मयि॑ प्र॒जां मय्य॒ग्निस्तेजो॑ दधातु॒ मयि॑ मे॒धां मयि॑ प्र॒जां मयीन्द्र॑ इन्द्रि॒यं द॑धातु॒ मयि॑ मे॒धां मयि॑ प्र॒जां मयि॒ सूर्यो॒ भ्राजो॑ दधातु॥९०॥

%७.४२.३
द्यु॒भिर॒क्तुभि॒ परि॑पातम॒स्मानरि॑ष्टेभिरश्विना॒ सौभ॑गेभिः। तन्नो॑ मि॒त्रो वरु॑णो मामहन्ता॒मदि॑ति॒ सिन्धु॑ पृथि॒वी उ॒त द्यौः। कया॑ नश्चि॒त्र आ भु॑वदू॒ती स॒दावृ॑ध॒ सखा। कया॒ शचि॑ष्ठया वृ॒ता। कस्त्वा॑ स॒त्यो मदा॑नां॒ महि॑ष्ठो मत्स॒दन्ध॑सः। दृ॒ढाचि॑दा॒रुजे॒ वसु॑। अ॒भी षु ण॒ सखी॑नामवि॒ता ज॑रितॄ॒णाम्। श॒तं भ॑वास्यू॒तिभि॑। वय॑ सुप॒र्णा उप॑सेदु॒रिन्द्रं॑ प्रि॒यमे॑धा॒ ऋष॑यो॒ नाध॑मानाः। अप॑ ध्वा॒न्तमूर्णु॒हि पू॒र्धि चक्षु॑र्मुमु॒ग्ध्य॑स्मान्नि॒धये॑व ब॒द्धान्॥९१॥

%७.४२.४
शं नो॑ दे॒वीर॒भिष्ट॑य॒ आपो॑ भवन्तु पी॒तये। शं॒ योर॒भिस्र॑वन्तु नः। ईशा॑ना॒ वार्या॑णां॒ क्षय॑न्तीश्चर्\mbox{}षणी॒नाम्। अ॒पो या॑चामि भेष॒जम्। सु॒मि॒त्रा न॒ आप॒ ओष॑धयः सन्तु दुर्मि॒त्रास्तस्मै॑ भूयासु॒र्योऽस्मान्द्वेष्टि॒ यं च॑ व॒यं द्वि॒ष्मः। आपो॒ हि ष्ठा म॑यो॒भुव॒स्ता न॑ ऊ॒र्जे द॑धातन। म॒हे रणा॑य॒ चक्ष॑से। यो व॑ शि॒वत॑मो॒ रस॒स्तस्य॑ भाजयते॒ह न॑। उ॒श॒तीरि॑व मा॒तर॑। तस्मा॒ अरं॑ गमाम वो॒ यस्य॒ क्षया॑य॒ जिन्व॑थ॥९२॥

%७.४२.५
आपो॑ ज॒नय॑था च नः। पृ॒थि॒वी शा॒न्ता साऽग्निना॑ शा॒न्ता सा मे॑ शा॒न्ता शुच शमयतु। अ॒न्तरि॑क्ष शा॒न्तं तद्वा॒युना॑ शा॒न्तं तन्मे॑ शा॒न्त शुच शमयतु। द्यौः  शा॒न्ता साऽऽदि॒त्येन॑ शा॒न्ता सा मे॑ शा॒न्ता शुच शमयतु। पृ॒थि॒वी शान्ति॑र॒न्तरि॑क्ष॒ शान्ति॒र्द्यौः  शान्ति॒र्दिश॒ शान्ति॑रवान्तरदि॒शाः  शान्ति॑र॒ग्निः  शान्ति॑र्वा॒युः  शान्ति॑रादि॒त्यः  शान्ति॑श्च॒न्द्रमा॒ शान्ति॒र्नक्ष॑त्राणि॒ शान्ति॒राप॒ शान्ति॒रोष॑धय॒ शान्ति॒र्वन॒स्पत॑य॒ शान्ति॒र्गौः  शान्ति॑र॒जा शान्ति॒रश्व॒ शान्ति॒ पुरु॑ष॒ शान्ति॒र्ब्रह्म॒ शान्ति॑र्ब्राह्म॒णः  शान्ति॒ शान्ति॑रे॒व शान्ति॒ शान्ति॑र्मे अस्तु॒ शान्ति॑। तया॒ह शा॒न्त्या स॑र्वशा॒न्त्या मह्यं॑ द्वि॒पदे॒ चतु॑ष्पदे च॒ शान्तिं॑ करोमि॒ शान्ति॑र्मे अस्तु॒ शान्ति॑। एह॒ श्रीश्च॒ ह्रीश्च॒ धृति॑श्च॒ तपो॑ मे॒धा प्र॑ति॒ष्ठा श्र॒द्धा स॒त्यं धर्म॑श्चै॒तानि॒ मोत्ति॑ष्ठन्त॒मनूत्ति॑ष्ठन्तु॒ मा मा॒ श्रीश्च॒ ह्रीश्च॒ धृति॑श्च॒ तपो॑ मे॒धा प्र॑ति॒ष्ठा श्र॒द्धा स॒त्यं धर्म॑श्चै॒तानि॑ मा॒ मा हा॑सिषुः। उदायु॑षा स्वा॒युषोदोष॑धीना॒ रसे॒नोत्प॒र्जन्य॑स्य॒ शुष्मे॒णोद॑स्थाम॒मृता॒ अनु॑। तच्चक्षु॑र्दे॒वहि॑तं पु॒रस्ताच्छु॒क्रमु॒च्चर॑त्। पश्ये॑म श॒रद॑ श॒तं जीवे॑म श॒रद॑ श॒तं नन्दा॑म श॒रद॑ श॒तं मोदा॑म श॒रद॑ श॒तं भवा॑म श॒रद॑ श॒त शृ॒णवा॑म श॒रद॑ श॒तं प्रब्र॑वाम श॒रद॑ श॒तमजी॑ताः स्याम श॒रद॑ श॒तं ज्योक्च॒ सूर्यं॑ दृ॒शे। य उद॑गान्मह॒तोऽर्णवाद्वि॒भ्राज॑मानः सरि॒रस्य॒ मध्या॒त्स मा॑ वृष॒भो लो॑हिता॒क्षः सूर्यो॑ विप॒श्चिन्मन॑सा पुनातु। ब्रह्म॑ण॒श्चोत॑न्यसि॒ ब्रह्म॑ण आ॒णी स्थो॒ ब्रह्म॑ण आ॒वप॑नमसि धारि॒तेयं पृ॑थि॒वी ब्रह्म॑णा म॒ही धा॑रि॒तमे॑नेन म॒हद॒न्तरि॑क्षं॒ दिवं॑ दाधार पृथि॒वी सदे॑वां॒ यद॒हं वेद॒ तद॒हं धा॑रयाणि॒ मा मद्वेदोऽधि॒विस्र॑सत्। मे॒धा॒म॒नी॒षे मावि॑शता स॒मीची॑ भू॒तस्य॒ भव्य॒स्याव॑रुध्यै॒ सर्व॒मायु॑रयाणि॒ सर्व॒मायु॑रयाणि। आ॒भिर्गी॒र्भिर्यदतो॑ न ऊ॒नमाप्या॑यय हरिवो॒ वर्ध॑मानः। य॒दा स्तो॒तृभ्यो॒ महि॑ गो॒त्रा रु॒जासि॑ भूयिष्ठ॒भाजो॒ अध॑ ते स्याम। ब्रह्म॒ प्रावा॑दिष्म॒ तन्नो॒ मा हा॑सीत्। ॐ शान्ति॒ शान्ति॒ शान्ति॑॥९३॥\anuvakamend[प॒रा॒वतो॑ दधातु ब॒द्धां जिन्व॑थ दृ॒शे स॒प्त च॑]

नमो॑ वा॒चे या चो॑दि॒ता या चानु॑दिता॒ तस्यै॑ वा॒चे नमो॒ नमो॑ वा॒चे नमो॑ वा॒चस्पत॑ये॒ नम॒ ऋषि॑भ्यो मन्त्र॒कृद्भ्यो॒ मन्त्र॑पतिभ्यो॒ मा मामृष॑यो मन्त्र॒कृतो॑ मन्त्र॒पत॑य॒ परा॑दु॒र्माहमृषीन्मन्त्र॒कृतो॑ मन्त्र॒पती॒न्परा॑दां वैश्वदे॒वीं वाच॑मुद्यास शि॒वामद॑स्तां॒ जुष्टां दे॒वेभ्य॒ शर्म॑ मे॒ द्यौः  शर्म॑ पृथि॒वी शर्म॒ विश्व॑मि॒दं जग॑त्। शर्म॑ च॒न्द्रश्च॒ सूर्य॑श्च॒ शर्म॑ ब्रह्मप्रजाप॒ती। भू॒तं व॑दिष्ये॒ भुव॑नं वदिष्ये॒ तेजो॑ वदिष्ये॒ यशो॑ वदिष्ये॒ तपो॑ वदिष्ये॒ ब्रह्म॑ वदिष्ये स॒त्यं व॑दिष्ये॒ तस्मा॑ अ॒हमि॒दमु॑प॒स्तर॑ण॒मुप॑स्तृण उप॒स्तर॑णं मे प्र॒जायै॑ पशू॒नां भू॑यादुप॒स्तर॑णम॒हं प्र॒जायै॑ पशू॒नां भू॑यासं॒ प्राणा॑पानौ मृ॒त्योर्मा॑ पातं॒ प्राणा॑पानौ॒ मा मा॑ हासिष्टं॒ मधु॑ मनिष्ये॒ मधु॑ जनिष्ये॒ मधु॑ वक्ष्यामि॒ मधु॑ वदिष्यामि॒ मधु॑मतीं दे॒वेभ्यो॒ वाच॑मुद्यास शुश्रू॒षेण्यां मनु॒ष्येभ्य॒स्तं मा॑ दे॒वा अ॑वन्तु शो॒भायै॑ पि॒तरोऽनु॑मदन्तु। ॐ शान्ति॒ शान्ति॒ शान्ति॑॥ 

\closesection