% !TeX program = XeLaTeX
% !TeX root = ../AraNyakabook.tex
\chapt{कृष्णयजुर्वेदीयतैत्तिरीय-काठकम्}
\setcounter{anuvakam}{0}
\sect{प्रथमः प्रश्नः}

   सं॒ज्ञानं॑ वि॒ज्ञानं॑ प्र॒ज्ञानं॑ जा॒नद॑भिजा॒नत्।
   स॒ङ्कल्प॑मानं प्र॒कल्प॑मानमुप॒कल्प॑मान॒मुप॑कॢप्तं कॢ॒प्तम्।
   श्रेयो॒ वसी॑य आ॒यत्सम्भू॑तं भू॒तम्।
   चि॒त्रः के॒तुः प्र॒भाना॒भान्त्सम्भा॒न्।
   ज्योति॑ष्मास्तेज॑स्वाना॒तप॒स्तप॑न्नभि॒तप\sn{}।
   रो॒च॒नो रोच॑मानः शोभ॒नः शोभ॑मानः क॒ल्याण॑।
   दर्शा॑ दृ॒ष्टा द॑र्श॒ता वि॒श्वरू॑पा सुदर्श॒ना।
   आ॒प्याय॑माना॒ प्याय॑माना॒ प्याया॑ सू॒नृतेरा।
   आ॒पूर्य॑माणा॒ पूर्य॑माणा पू॒रय॑न्ती पू॒र्णा पौर्णमा॒सी।
   दा॒ता प्र॑दा॒ताऽऽन॒न्दो मो॒द॑ प्रमो॒दः॥१॥

   आ॒वे॒शय॑न्निवे॒शयन्त्सं॒वेश॑न॒ सशान्तः शा॒न्तः।
   आ॒भव॑न्प्र॒भवन्त्\-स॒म्भव॒न्त्सम्भू॑तो भू॒तः।
   प्रस्तु॑तं॒ विष्टु॑त॒ सस्तु॑तं क॒ल्याणं॑ वि॒श्वरू॑पम्।
   शु॒क्रम॒मृतं॑ तेज॒स्वि तेज॒ समि॑द्धम्।
   अ॒रु॒णं भा॑नु॒मन्मरी॑चिमदभि॒तप॒त्तप॑स्वत्।
   स॒वि॒ता प्र॑सवि॒ता दी॒प्तो दी॒पय॒न्दीप्य॑मानः।
   ज्वल॑ञ्ज्वलि॒ता तप॑न्वि॒तपन्त्स॒न्तप\sn{}।
   रो॒च॒नो रोच॑मानः शु॒म्भूः शुम्भ॑मानो वा॒मः।
   सु॒ता सु॑न्व॒ती प्रसु॑ता सू॒यमा॑नाऽभिषू॒यमा॑णा।
   पीती प्र॒पा स॒म्पा तृप्ति॑स्त॒र्पय॑न्ती॥२॥

   का॒न्ता का॒म्या का॒मजा॒ताऽऽयु॑ष्मती काम॒दुघा।
   अ॒भि॒शा॒स्ताऽ\-नु॑म॒न्ताऽऽन॒न्दो मोद॑ प्रमो॒दः।
   आ॒सा॒दय॑न्निषा॒दयन्त्स॒साद॑न॒ सस॑न्नः स॒न्नः।
   आ॒भूर्वि॒भूः प्र॒भूः श॒म्भूर्भुव॑।
   प॒वित्रं॑ पवियि॒ष्यन्पू॒तो मेध्य॑।
   यशो॒ यश॑स्वाना॒युर॒मृत॑।
   जी॒वो जी॑वि॒ष्यन्त्स्व॒र्गो लो॒कः।
   सह॑स्वा॒न्त्सही॑या॒नोज॑स्वा॒न्त्सह॑मानः।
   जय॑न्नभि॒जयन्त्सु॒द्रवि॑णो द्रविणो॒दाः।
   आ॒र्द्रप॑वित्रो॒ हरि॑केशो॒ मोद॑ प्रमो॒दः॥३॥

   अ॒रु॒णो॑ऽरु॒णर॑जाः पु॒ण्डरी॑को विश्व॒जिद॑भि॒जित्।
   आ॒र्द्रः पिन्व॑मा॒नोऽन्न॑वा॒न्रस॑वा॒निरा॑वान्।
   स॒र्वौ॒ष॒धः स॑म्भ॒रो मह॑स्वान्।
   ए॒ज॒त्का जो॑व॒त्काः।
   क्षु॒ल्ल॒काः शि॑पिविष्ट॒काः।
   स॒रि॒स्र॒राः सु॒शेर॑वः।
   अ॒जिरासो॑ गमि॒ष्णव॑।
   इ॒दानीं त॒दानी॑मे॒तर्\-\mbox{हि॑} क्षि॒प्रम॑जि॒रम्।
   आ॒शुर्नि॑मे॒षः फ॒णो द्रव॑न्नति॒द्रव\sn{‌}।
   त्वर॒स्त्वर॑माण आ॒शुराशी॑याञ्ज॒वः।
   अ॒ग्नि॒ष्टो॒म उ॒क्थ्यो॑ऽतिरा॒त्रो द्वि॑रा॒त्रस्त्रि॑रा॒त्रश्च॑तूरा॒त्रः।
   अ॒ग्निर्{‌}ऋ॒तुः सूर्य॑ ऋ॒तुश्च॒न्द्रमा॑ ऋ॒तुः।
   प्र॒जाप॑तिः संवत्स॒रो म॒हान्क॥४॥
   \anuvakamend
   
   भूर॒ग्निं च॑ पृथि॒वीं च॒ मां च॑।
   त्रीश्च॑ लो॒कान्त्सं॑वत्स॒रं च॑।
   प्र॒जाप॑तिस्त्वा सादयतु।
   तया॑ दे॒वत॑याऽङ्गिर॒स्वद्ध्रु॒वा सी॑द।
   भुवो॑ वा॒युं चा॒न्तरि॑क्षं च॒ मां च॑।
   त्रीश्च॑ लो॒कान्त्सं॑वत्स॒रं च॑।
   प्र॒जाप॑तिस्त्वा सादयतु।
   तया॑ दे॒वत॑याऽङ्गिर॒स्वद्ध्रु॒वा सी॑द।
   स्व॑रादि॒त्यं च॒ दिवं॑ च॒ मां च॑।
   त्रीश्च॑ लो॒कान्त्सं॑वत्स॒रं च॑।
   प्र॒जाप॑तिस्त्वा सादयतु।
   तया॑ दे॒वत॑याऽङ्गिर॒स्वद्ध्रु॒वा सी॑द।
   भूर्भुव॒ स्व॑श्च॒न्द्रम॑सं च॒ दिश॑श्च॒ मां च॑।
   त्रीश्च॑ लो॒कान्त्सं॑वत्स॒रं च॑।
   प्र॒जाप॑तिस्त्वा सादयतु।
   तया॑ दे॒वत॑याऽङ्गिर॒स्वद्ध्रु॒वा सी॑द॥५॥
   \anuvakamend
   
   त्वमे॒व त्वां वेत्थ॒ यो॑ऽसि॒ सोऽसि॑।
   त्वमे॒व त्वाम॑चैषीः।
   चि॒तश्चासि॒ सञ्चि॑तश्चास्यग्ने।
   ए॒तावा॒श्चासि॒ भूयाश्चास्यग्ने।
   यत्ते॑ अग्ने॒ न्यू॑नं॒ यदु॒ तेऽति॑रिक्तम्।
   आ॒दि॒त्यास्तदङ्गि॑रसश्चिन्वन्तु।
   विश्वे॑ ते दे॒वाश्चिति॒मापू॑रयन्तु।
   चि॒तश्चासि॒ सञ्चि॑तश्चास्यग्ने।
   ए॒तावा॒श्चासि॒ भूयाश्चास्यग्ने।
   मा ते॑ अग्ने च॒ येन॒ माऽति॑ च॒ येनाऽऽयु॒रावृ॑क्षि।
   सर्वे॑षां॒ ज्योति॑षां॒ ज्योति॒र्यद॒दावु॒देति॑ ।
   तप॑सो जा॒तमनि॑भृष्ट॒मोज॑।
   तत्ते॒ ज्योति॑रिष्टके।
   तेन॑ मे तप।
   तेन॑ मे ज्वल।
   तेन॑ मे दीदिहि।
   याव॑द्दे॒वाः।
   याव॒दसा॑ति॒ सूर्य॑।
   याव॑दु॒तापि॒ ब्रह्म॑॥६॥
\anuvakamend

   सं॒व॒त्स॒रो॑ऽसि परिवत्स॒रो॑ऽसि।
   इ॒दा॒व॒त्स॒रो॑ऽसीदुवत्स॒रो॑ऽसि।
   इ॒द्व॒त्स॒रो॑ऽसि वत्स॒रो॑ऽसि।
   तस्य॑ ते वस॒न्तः शिर॑।
   ग्री॒ष्मो दक्षि॑णः प॒क्षः।
   व॒\ar{‌}षाः पुच्छम्।
   श॒रदुत्त॑रः प॒क्षः।
   हे॒म॒न्तो मध्यम्।
   पू॒र्व॒प॒क्षाश्चित॑यः।
   अ॒प॒र॒प॒क्षाः पुरी॑षम्॥७॥

   अ॒हो॒रा॒त्राणीष्ट॑काः।
   ऋ॒ष॒भो॑ऽसि स्व॒र्गो लो॒कः।
   यस्यां दि॒शि म॒हीय॑से।
   ततो॑ नो॒ मह॒ आव॑ह।
   वा॒युर्भू॒त्वा सर्वा॒ दिश॒ आवा॑हि।
   सर्वा॒ दिशोऽनु॒विवा॑हि।
   सर्वा॒ दिशोऽनु॒संवा॑हि।
   चित्त्या॒ चिति॒मापृ॑ण।
   अचि॑त्त्या॒  चिति॒मापृ॑ण।
   चिद॑सि समु॒द्रयो॑निः॥८॥

   इन्दु॒र्दक्ष॑ श्ये॒न ऋ॒तावा।
   हिर॑ण्यपक्षः शकु॒नो भु॑र॒ण्युः।
   म॒हान्त्स॒धस्थे ध्रु॒व आनिष॑त्तः।
   नम॑स्ते अस्तु॒ मा मा॑ हिसीः।
   एति॒ प्रेति॒ वीति॒ समित्युदिति॑।
   दिवं॑ मे यच्छ।
   अ॒न्तरि॑क्षं मे यच्छ।
   पृ॒थि॒वीं मे॑ यच्छ।
   पृ॒थि॒वीं मे॑ यच्छ।
   अ॒न्तरि॑क्षं मे यच्छ।
   दिवं॑ मे यच्छ।
   अह्ना॒ प्रसा॑रय।
   रात्र्या॒ सम॑च।
   रात्र्या॒ प्रसा॑रय।
   अह्ना॒ सम॑च।
   कामं॒ प्रसा॑रय।
   काम॒ सम॑च॥९॥
   \anuvakamend

   भूर्भुव॒ स्व॑।
   ओजो॒ बलम्।
   ब्रह्म॑ क्ष॒त्रम्।
   यशो॑ म॒हत्।
   स॒त्यं तपो॒ नाम॑।
   रू॒पम॒मृतम्।
   चक्षु॒ स्रोत्रम्।
   मन॒ आयु॑।
   विश्वं॒ यशो॑ म॒हः।
   स॒मं तपो॒ हरो॒ भाः।
   जा॒तवे॑दा॒ यदि॑ वा पाव॒कोऽसि॑।
   वै॒श्वा॒न॒रो यदि॑ वा वैद्यु॒तोऽसि॑।
   शं प्र॒जाभ्यो॒ यज॑मानाय लो॒कम्।
   ऊर्जं॒ पुष्टिं॒ दद॑द॒भ्याव॑वृत्स्व॥१०॥
   \anuvakamend

   राज्ञी॑ वि॒राज्ञी।
   स॒म्राज्ञी स्व॒राज्ञी।
   अ॒र्चिः शो॒चिः।
   तपो॒ हरो॒ भाः।
   अ॒ग्निरिन्द्रो॒ बृह॒स्पति॑।
   विश्वे॑ दे॒वा भुव॑नस्य गो॒पाः।
   ते मा॒ सर्वे॒ यश॑सा॒ ससृ॑जन्तु॥११॥
\anuvakamend

   अस॑वे॒ स्वाहा॒ वस॑वे॒ स्वाहा।
   विभु॑वे॒ स्वाहा॒ विव॑स्वते॒ स्वाहा।
   अ॒भि॒भुवे॒ स्वाहाऽधि॑पतये॒ स्वाहा।
   दि॒वां पत॑ये॒ स्वाहाऽहस्प॒त्याय॒ स्वाहा।
   चा॒क्षु॒ष्म॒त्याय॒ स्वाहा ज्योतिष्म॒त्याय॒ स्वाहा।
   राज्ञे॒ स्वाहा॑ वि॒राज्ञे॒ स्वाहा।
   स॒म्राज्ञे॒ स्वाहा स्व॒राज्ञे॒ स्वाहा।
   शूषा॑य॒ स्वाहा॒ सूर्या॑य॒ स्वाहा।
   च॒न्द्रम॑से॒ स्वाहा॒ ज्योति॑षे॒ स्वाहा।
   स॒सर्पा॑य॒ स्वाहा॑ क॒ल्याणा॑य॒ स्वाहा।
   अर्जु॑नाय॒ स्वाहा॥१२॥
\anuvakamend

   वि॒प॒श्चिते॒ पव॑मानाय गायत।
   म॒ही न धाराऽत्यन्धो॑ अर्{‌}षति।
   अहि॑र्{‌}ह जी॒र्णामति॑सर्पति॒ त्वचम्।
   अत्यो॒ न क्रीड॑न्नसर॒द्वृषा॒ हरि॑।
   उ॒प॒या॒मगृ॑हीतोऽसि मृ॒त्यवे त्वा॒ जुष्टं॑ गृह्णामि।
   ए॒ष ते॒ योनि॑र्मृ॒त्यवे त्वा।
   अप॑मृ॒त्युमप॒क्षुधम्।
   अपे॒तः श॒पथं॑ जहि।
   अधा॑ नो अग्न॒ आव॑ह।
   रा॒यस्पोष सह॒स्रिणम्॥१३॥

   ये ते॑ स॒हस्र॑म॒युतं॒ पाशा।
   मृत्यो॒ मर्त्या॑य॒ हन्त॑वे।
   तान् य॒ज्ञस्य॑ मा॒यया।
   सर्वा॒नव॑यजामहे।
   भ॒क्षोऽस्यमृतभ॒क्षः।
   तस्य॑ ते मृ॒त्युपी॑तस्या॒मृत॑वतः।
   स्व॒गाकृ॑तस्य॒ मधु॑मतः।
   उप॑हूत॒स्योप॑हूतो भक्षयामि।
   म॒न्द्राऽभिभू॑तिः के॒तुर्{‌}य॒ज्ञानां॒ वाक्।
   असा॒वेहि॑॥१४॥

   अ॒न्धो जागृ॑विः प्राण।
   असा॒वेहि॑।
   ब॒धि॒र आक्रन्दयितरपान।
   असा॒वेहि॑।
   अ॒ह॒स्तोस्त्वा॒ चक्षु॑।
   असा॒वेहि॑।
   अ॒पा॒दाशो॒ मन॑।
   असा॒वेहि॑।
   कवे॒ विप्र॑चित्ते॒ श्रोत्र॑।
   असा॒वेहि॑॥१५॥

   सु॒ह॒स्तः सु॑वा॒साः।
   शू॒षो नामास्य॒मृतो॒ मर्त्ये॑षु।
   तं त्वा॒ऽहं तथा॒ वेद॑।
   असा॒वेहि॑।
   अ॒ग्निर्मे॑ वा॒चि श्रि॒तः।
   वाग्घृद॑ये।
   हृद॑यं॒ मयि॑।
   अ॒हम॒मृते।
   अ॒मृतं॒ ब्रह्म॑णि।
   वा॒युर्मे प्रा॒णे श्रि॒तः॥१६॥

   प्रा॒णो हृद॑ये।
   हृद॑यं॒ मयि॑।
   अ॒हम॒मृते।
   अ॒मृतं॒ ब्रह्म॑णि।
   सूर्यो॑ मे॒ चक्षु॑षि श्रि॒तः।
   चक्षु॒र्{‌}हृद॑ये।
   हृद॑यं॒ मयि॑।
   अ॒हम॒मृते।
   अ॒मृतं॒ ब्रह्म॑णि।
   च॒न्द्रमा॑ मे॒ मन॑सि श्रि॒तः॥१७॥

   मनो॒ हृद॑ये।
   हृद॑यं॒ मयि॑।
   अ॒हम॒मृते।
   अ॒मृतं॒ ब्रह्म॑णि।
   दिशो॑ मे॒ श्रोत्रे श्रि॒ताः।
   श्रोत्र॒ हृद॑ये।
   हृद॑यं॒ मयि॑।
   अ॒हम॒मृते।
   अ॒मृतं॒ ब्रह्म॑णि।
   आपो॑ मे॒ रेत॑सि श्रि॒ताः॥१८॥

   रेतो॒ हृद॑ये।
   हृद॑यं॒ मयि॑।
   अ॒हम॒मृते।
   अ॒मृतं॒ ब्रह्म॑णि।
   पृ॒थि॒वी मे॒ शरी॑रे श्रि॒ता।
   शरी॑र॒ हृद॑ये।
   हृद॑यं॒ मयि॑।
   अ॒हम॒मृते।
   अ॒मृतं॒ ब्रह्म॑णि।
   ओ॒ष॒धि॒व॒न॒स्प॒तयो॑ मे॒ लोम॑सु श्रि॒ताः॥१९॥

   लोमा॑नि॒ हृद॑ये।
   हृद॑यं॒ मयि॑।
   अ॒हम॒मृते।
   अ॒मृतं॒ ब्रह्म॑णि।
   इन्द्रो॑ मे॒ बले श्रि॒तः।
   बल॒ हृद॑ये।
   हृद॑यं॒ मयि॑।
   अ॒हम॒मृते।
   अ॒मृतं॒ ब्रह्म॑णि।
   प॒र्जन्यो॑ मे मू॒र्ध्नि श्रि॒तः॥२०॥

   मू॒र्धा हृद॑ये।
   हृद॑यं॒ मयि॑।
   अ॒हम॒मृते।
   अ॒मृतं॒ ब्रह्म॑णि।
   ईशा॑नो मे म॒न्यौ श्रि॒तः।
   म॒न्युर्{‌}हृद॑ये।
   हृद॑यं॒ मयि॑।
   अ॒हम॒मृते।
   अ॒मृतं॒ ब्रह्म॑णि।
   आ॒त्मा म॑ आ॒त्मनि॑ श्रि॒तः॥२१॥
   आ॒त्मा हृद॑ये।
   हृद॑यं॒ मयि॑।
   अ॒हम॒मृते।
   अ॒मृतं॒ ब्रह्म॑णि।
   पुन॑र्म आ॒त्मा पुन॒रायु॒रागात्।
   पुन॑ प्रा॒णः पु॒नराकू॑त॒मागात्।
   वै॒श्वा॒न॒रो र॒श्मिभि॑र्वावृधा॒नः।
   अ॒न्तस्ति॑ष्ठत्व॒मृत॑स्य गो॒पाः॥२२॥
\anuvakamend

   प्र॒जाप॑तिर्दे॒वान॑सृजत।
   ते पा॒प्मना॒ सन्दि॑ता अजायन्त।
   तान्व्य॑द्यत्।
   यद्व्य॒द्यत्।
   तस्माद्वि॒द्युत्।
   तम॑वृश्चत्।
   यदवृ॑श्चत्।
   तस्मा॒द्वृष्टि॑।
   तस्मा॒द्यत्रै॒ते दे॒वते॑ अभि॒प्राप्नु॑तः ।
   वि च॑ है॒वास्य॒ तत्र॑ पा॒प्मानं॒ द्यत॑॥२३॥

   वृ॒श्चत॑श्च।
   सैषा मी॑मा॒साऽग्नि॑हो॒त्र ए॒व सं॑पन्ना।
   अथो॑ आहुः।
   सर्वे॑षु यज्ञक्र॒तुष्विति॑।
   होष्य॑न्न॒प उप॑स्पृशेत्।
   विद्यु॑दसि॒ विद्य॑ मे पा॒प्मान॒मिति॑।
   अथ॑ हु॒त्वोप॑स्पृशेत्।
   वृष्टि॑रसि॒ वृश्च॑ मे पा॒प्मान॒मिति॑।
   य॒क्ष्यमा॑णो वे॒ष्ट्वा वा।
   वि च॑ है॒वास्यै॒ते दे॒वते॑ पा॒प्मानं॒ द्यत॑॥२४॥

   वृ॒श्चत॑श्च।
   अ॒त्य॒हो हाऽऽरु॑णिः।
   ब्र॒ह्म॒चा॒रिणे प्र॒श्नान्प्रोच्य॒ प्रजि॑घाय।
   परे॑हि।
   प्ल॒क्षं दय्यांपातिं पृच्छ।
   वेत्थ॑ सावि॒त्रा(३)न्न वे॒त्था(३) इति॑।
   तमा॒गत्य॑ पप्रच्छ।
   आ॒चार्यो॑ मा॒ प्राहै॑षीत्।
   वेत्थ॑ सावि॒त्रा(३)न्न वे॒त्था(३) इति॑।
   स हो॑वाच॒ वेदेति॑॥२५॥

   स कस्मि॒न्प्रति॑ष्ठित॒ इति॑।
   प॒रोर॑ज॒सीति॑।
   कस्तद्यत्प॒रोर॑जा॒ इति॑।
   ए॒ष वाव स प॒रोर॑जा॒ इति॑ होवाच।
   य ए॒ष तप॑ति।
   ए॒षोऽर्वाग्र॑जा॒ इति॑।
   स कस्मि॑न्त्वे॒ष इति॑।
   स॒त्य इति॑।
   किं तत्स॒त्यमिति॑।
   तप॒ इति॑॥२६॥

   कस्मि॒न्नु तप॒ इति॑।
   बल॒ इति॑।
   किं तद्बल॒मिति॑।
   प्रा॒ण इति॑।
   मा स्म॑ प्रा॒णमति॑पृच्छ॒ इति॑ माऽऽचा॒र्योऽब्रवी॒दिति॑ होवाच ब्रह्मचा॒री।
   स हो॑वाच प्ल॒क्षो दय्यांपातिः।
   यद्वै ब्र॑ह्मचारिन्प्रा॒णमत्य॑प्रक्ष्यः।
   मू॒र्धा ते॒ व्यप॑तिष्यत्।
   अ॒हमु॑त आचा॒र्याच्छ्रेयान्भविष्यामि।
   यो मा॑ सावि॒त्रे स॒मवा॑दि॒ष्टेति॑॥२७॥

   तस्मात्सावि॒त्रे न संव॑देत।
   स यो ह॒ वै सा॑वि॒त्रं वि॒दुषा॑ सावि॒त्रे सं॒वद॑ते।
   सहास्मि॒ञ्छ्रियं॑ दधाति।
   अनु॑ ह॒ वा अ॑स्मा अ॒सौ तप॒ञ्छ्रियं॑ मन्यते।
   अन्व॑स्मै॒ श्रीस्तपो॑ मन्यते।
   अन्व॑स्मै॒ तपो॒ बलं॑ मन्यते।
   अन्व॑स्मै॒ बलं॑ प्रा॒णं म॑न्यते।
   स यदाह॑।
   सं॒ज्ञानं॑ वि॒ज्ञानं॒ दर्शा॑ दृ॒ष्टेति॑।
   ए॒ष ए॒व तत्॥२८॥

   अथ॒ यदाह॑।
   प्रस्तु॑तं॒ विष्टु॑त सु॒ता सु॑न्व॒तीति॑।
   ए॒ष ए॒व तत्।
   ए॒ष ह्ये॑व तान्यहा॑नि।
   ए॒ष रात्र॑यः।
   अथ॒ यदाह॑।
   चि॒त्रः के॒तुर्दा॒ता प्र॑दा॒ता स॑वि॒ता प्र॑सवि॒ताऽभि॑शा॒स्ताऽनु॑म॒न्तेति॑।
   ए॒ष ए॒व तत्।
   ए॒ष ह्ये॑व तेऽह्नो॑ मुहू॒र्ताः।
   ए॒ष रात्रे॥२९॥

   अथ॒ यदाह॑।
   प॒वित्रं॑ पवयि॒ष्यन्त्सह॑स्वा॒न्त्सही॑यानरु॒णो॑ऽरु॒णर॑जा॒ इति॑।
   ए॒ष ए॒व तत्।
   ए॒ष ह्ये॑व तेऽर्धमा॒साः।
   ए॒ष मासा।
   अथ॒ यदाह॑।
   अ॒ग्नि॒ष्टो॒म उ॒क्थ्योऽग्निर्{‌}ऋ॒तुः प्र॒जाप॑तिः संवत्स॒र इति॑।
   ए॒ष ए॒व तत्।
   ए॒ष ह्ये॑व ते य॑ज्ञक्र॒तव॑।
   ए॒ष ऋ॒तव॑॥३०॥

   ए॒ष सं॑वत्स॒रः।
   अथ॒ यदाह॑।
   इ॒दानीं त॒दानी॒मिति॑।
   ए॒ष ए॒व तत्।
   ए॒ष ह्ये॑व ते मु॑हू॒र्तानां मुहू॒र्ताः।
    ज॒न॒को ह॒ वैदे॑हः।
   अ॒हो॒रा॒त्रैः स॒माज॑गाम।
   त हो॑चुः।
   यो वा अ॒स्मान् वेद॑।
   वि॒जह॑त्पा॒प्मान॑मेति॥३१॥

   सर्व॒मायु॑रेति।
   अ॒भि स्व॒र्गं लो॒कं ज॑यति।
   नास्या॒मुष्मिँ॑ल्लो॒केऽन्नं॑ क्षीयत॒ इति॑।
   वि॒जह॑द्ध॒ वै पा॒प्मान॑मेति।
   सर्व॒मायु॑रेति।
   अ॒भि स्व॒र्गं लो॒कं ज॑यति।
   नास्या॒मुष्मिँ॑ल्लो॒केऽन्नं॑ क्षीयते।
   य ए॒वं वेद॑।
   अही॑ना॒ हाऽऽश्व॑थ्यः।
   सा॒वि॒त्रं वि॒दाञ्च॑कार॥३२॥

   स ह॑ ह॒सो हिर॒ण्मयो॑ भू॒त्वा।
   स्व॒र्गं लो॒कमि॑याय।
   आ॒दि॒त्यस्य॒ सायु॑ज्यम्।
   ह॒सो ह॒ वै हि॑र॒ण्मयो॑ भू॒त्वा।
   स्व॒र्गं लो॒कमे॑ति।
   आ॒दि॒त्यस्य॒ सायु॑ज्यम्।
   य ए॒वं वेद॑।
   दे॒व॒भा॒गो ह॑ श्रौत॒र्{‌}षः।
   सा॒वि॒त्रं वि॒दाञ्च॑कार।
   त ह॒ वागदृ॑श्यमा॒नाऽभ्यु॑वाच॥३३॥

   सर्वं॑ बत गौत॒मो वेद॑।
   यः सा॑वि॒त्रं वेदेति॑।
   स हो॑वाच।
   कैषा वाग॒सीति॑।
   अ॒यम॒ह सा॑वि॒त्रः।
   दे॒वाना॑मुत्त॒मो लो॒कः।
   गुह्यं॒ महो॒ बिभ्र॒दिति॑।
   ए॒ताव॑ति ह गौत॒मः।
   य॒ज्ञो॒प॒वी॒तं कृ॒त्वाऽधो निप॑पात।
   नमो॒ नम॒ इति॑॥३४॥

   स हो॑वाच।
   मा भै॑षीर्गौतम।
   जि॒तो वै ते॑ लो॒क इति॑।
   तस्मा॒द्ये के च॑ सावि॒त्रं वि॒दुः।
   सर्वे॒ ते जि॒तलो॑काः।
   स यो ह॒ वै सा॑वि॒त्रस्या॒ष्टाक्ष॑रं प॒द श्रि॒याऽभिषिक्तं॒ वेद॑।
   श्रि॒या है॒वाभिषि॑च्यते।
   घृ॒णिरि॒ति द्वे अ॒क्षरे।
   सूर्य॒ इति॒ त्रीणि॑।
   आ॒दि॒त्य इति॒ त्रीणि॑॥३५॥

   ए॒तद्वै सा॑वि॒त्रस्या॒ष्टाक्ष॑रं प॒द श्रि॒याऽभिषि॑क्तम्।
   य ए॒वं वेद॑।
   श्रि॒या है॒वाभिषि॑च्यते।
   तदे॒तदृ॒चाऽभ्यु॑क्तम्।
   ऋ॒चो अ॒क्षरे॑ पर॒मे व्यो॑मन्।
   यस्मि॑न्दे॒वा अधि॒ विश्वे॑ निषे॒दुः।
   यस्तं न वेद॒ किमृ॒चा क॑रिष्यति।
   य इत्तद्वि॒दुस्त इ॒मे समा॑सत॒ इति॑।
   न ह॒ वा ए॒तस्य॒र्चा न यजु॑षा॒ न साम्नाऽर्थोऽस्ति।
   यः सा॑वि॒त्रं वेद॑॥३६॥

   तदे॒तत्प॑रि॒ यद्दे॑वच॒क्रम्।
   आ॒र्द्रं पिन्व॑मान स्व॒र्गे लो॒क ए॑ति।
   वि॒जह॒द्विश्वा॑ भू॒तानि॑ स॒म्पश्य॑त्।
   आ॒र्द्रो ह॒ वै पिन्व॑मानः।
   स्व॒र्गे लो॒क ए॑ति।
   वि॒जह॒न्विश्वा॑ भू॒तानि॑ स॒म्पश्य॑न्।
   य ए॒वं वेद॑।
   शू॒षो ह॒ वै वार्ष्णे॒यः।
   आ॒दि॒त्येन॑ स॒माज॑गाम।
   त हो॑वाच।
   एहि॑ सावि॒त्रं वि॑द्धि।
   अ॒यं वै स्व॒र्ग्योऽग्निः पा॑रयि॒ष्णुर॒मृता॒त्सम्भू॑त॒ इति॑।
   ए॒ष वाव स सा॑वि॒त्रः।
   य ए॒ष तप॑ति।
   एहि॒ मां वि॑द्धि।
   इति॑ है॒वैनं॒ तदु॑वाच॥३७॥
   \anuvakamend

   इ॒यं वाव स॒रघा।
   तस्या॑ अ॒ग्निरे॒व सा॑र॒घं मधु॑।
   या ए॒ताः पूर्वपक्षापरप॒क्षयो॒ रात्र॑यः।
   ता म॑धु॒कृत॑।
   यान्यहा॑नि।
   ते म॑धुवृ॒षाः।
   स यो ह॒ वा ए॒ता म॑धु॒कृत॑श्च मधुवृ॒षाश्च॒ वेद॑।
   कु॒र्वन्ति॑ हास्यै॒ता अ॒ग्नौ मधु॑।
   नास्येष्टापू॒र्तं ध॑यन्ति।
   अथ॒ यो न वेद॑॥३८॥

   न हास्यै॒ता अ॒ग्नौ मधु॑ कुर्वन्ति।
   धय॑न्त्यस्येष्टापू॒र्तम्।
   यो ह॒ वा अ॑होरा॒त्राणां नाम॒धेया॑नि॒ वेद॑।
   नाहो॑रा॒त्रेष्वार्ति॒मार्च्छ॑ति।
   सं॒ज्ञानं॑ वि॒ज्ञानं॒ दर्शा॑ दृ॒ष्टेति॑।
   ए॒ताव॑नुवा॒कौ पूर्वप॒क्षस्या॑होरा॒त्राणां नाम॒धेया॑नि।
   प्रस्तु॑तं॒ विष्टु॑त सु॒ता सु॑न्व॒तीति॑।
   ए॒ताव॑नुवा॒काव॑पर\-प॒क्षस्या॑होरा॒त्राणां नाम॒धेया॑नि।
   नाहो॑रा॒त्रेष्वार्ति॒मार्च्छ॑ति।
   य ए॒वं वेद॑॥३९॥

   यो ह॒ वै मु॑हू॒र्तानां नाम॒धेया॑नि॒ वेद॑।
   न मु॑हू॒र्तेष्वार्ति॒मार्च्छ॑ति।
   चि॒त्रः के॒तुर्दा॒ता प्र॑दा॒ता स॑वि॒ता प्रसवि॒ताऽभि॑शा॒स्ताऽनु॑म॒न्तेति॑।
   ए॒ते॑ऽनुवा॒का मुहू॒र्तानां नाम॒धेया॑नि।
   न मुहू॒र्तेष्वार्ति॒मार्च्छ॑ति।
   य ए॒वं वेद॑।
   यो ह॒ वा अ॑र्धमा॒सानां च॒ मासा॑नां च नाम॒धेया॑नि॒ वेद॑।
   नार्ध॑मा॒सेषु॒ न मासे॒ष्वार्ति॒मार्च्छ॑ति।
   प॒वित्रं॑ पवियि॒ष्यन्त्सह॑\-स्वा॒न्त्सही॑यानरु॒णो॑ऽरु॒णर॑जा॒ इति॑।
   ए॒ते॑ऽनुवा॒का अ॑र्धमा॒सानां च॒ मासा॑नां च नाम॒धेया॑नि॥४०॥

   नार्ध॑मा॒सेषु॒ न मासे॒ष्वार्ति॒मार्च्छ॑ति।
   य ए॒वं वेद॑।
   यो ह॒ वै य॑ज्ञक्रतू॒नां च॑र्तू॒नां च॑ संवत्स॒रस्य॑ च नाम॒धेया॑नि॒ वेद॑।
   न य॑ज्ञक्र॒तुषु॒ नर्तुषु॒ न सं॑वत्स॒र आर्ति॒मार्च्छ॑ति।
   अ॒ग्नि॒ष्टो॒म उ॒क्थ्योऽग्निर्{‌}ऋ॒तुः प्र॒जाप॑तिः संवत्स॒र इति॑।
   ए॒ते॑ऽनुवा॒का य॑ज्ञक्रतू॒नां च॑र्तू॒नां च॑ संवत्स॒रस्य॑ च नाम॒धेया॑नि॥४१॥
   न य॑ज्ञक्र॒तुषु॒ नर्तुषु॒ न सं॑वत्स॒र आर्ति॒मार्च्छ॑ति।
   य ए॒वं वेद॑।
   यो ह॒ वै मु॑हू॒र्तानां मुहू॒र्तान् वेद॑।
   न मु॑हू॒र्तानां मुहू॒र्तेष्वार्ति॒मार्च्छ॑ति।
   इ॒दानीं त॒दानी॒मिति॑।
   ए॒ते वै मु॑हू॒र्तानां मुहू॒र्ताः।
   न मु॑हू॒र्तानां मुहू॒र्तेष्वार्ति॒मार्च्छ॑ति।
   य ए॒वं वेद॑।
   अथो॒ यथा क्षेत्र॒ज्ञो भू॒त्वाऽनु॑प्र॒विश्यान्न॒मत्ति॑।
   ए॒वमे॒वैतान्क्षेत्र॒ज्ञो भू॒त्वाऽनु॑प्र॒विश्यान्न॑मत्ति।
   स ए॒तेषा॑मे॒व स॑लो॒कता॒ सायु॑ज्यमश्नुते।
   अप॑ पुनर्मृ॒त्युं ज॑यति।
   य ए॒वं वेद॑॥४२॥
\anuvakamend
				
   कश्चि॑द्ध॒ वा अ॒स्माल्लो॒कात्प्रेत्य॑।
   आ॒त्मानं॑ वेद।
   अ॒यम॒हम॒स्मीति॑।
   कश्चि॒त्स्वं लो॒कं न प्रति॒प्रजा॑नाति।
   अ॒ग्निमु॑ग्धो है॒व धू॒मतान्तः।
   स्वं लो॒कं न प्रति॒प्रजा॑नाति।
   अथ॒ यो है॒वैतम॒ग्नि सा॑वि॒त्रं वेद॑।
   स ए॒वास्माल्लो॒कात्प्रेत्य॑।
   आ॒त्मानं॑ वेद।
   अ॒यम॒हम॒स्मीति॑॥४३॥

   स स्वं लो॒कं प्रति॒प्रजा॑नाति।
   ए॒ष उ॑ चैवैनं॒ तत्सा॑वि॒त्रः।
   स्व॒र्गं लो॒कम॒भिव॑हति।
   अ॒हो॒रा॒त्रैर्वा इ॒द स॒युग्भि॑ क्रियते।
   इ॒ति॒रा॒त्राया॑दीक्षिषत।
   इ॒ति॒रा॒त्राय॑ व्र॒तमुपा॑गु॒रिति॑।
   तानि॒हाने॑वं वि॒दुष॑।
   अ॒मुष्मिँ॑ल्लो॒के शे॑व॒धिं ध॑यन्ति।
   धी॒त है॒व स शे॑व॒धिमनु॒ परै॑ति।
   अथ॒ यो है॒वैत॒मग्नि सा॑वि॒त्रं वेद॑॥४४॥

   तस्य॑ है॒वाहो॑रा॒त्राणि॑।
   अ॒मुष्मिँ॑ल्लो॒के शे॑व॒धिं न ध॑यन्ति।
   अधी॑त है॒व स शे॑व॒धिमनु॒ परै॑ति।
   भ॒रद्वा॑जो ह त्रि॒भिरायु॑र्भिर्ब्रह्म॒चर्य॑मुवास।
   त ह॒ जीर्णि॒ स्थवि॑र॒ शया॑नम्।
   इन्द्र॑ उप॒व्रज्यो॑वाच।
   भर॑द्वाज।
   यत्ते॑ चतु॒र्थमायु॑र्द॒द्याम्।
   किमे॑नेन कुर्या॒ इति॑।
   ब्र॒ह्म॒चर्य॑मे॒वैने॑न चरेय॒मिति॑ होवाच॥४५॥

   त ह॒ त्रीन्गि॒रिरू॑पा॒नवि॑ज्ञातानिव दर्श॒यां च॑कार।
   तेषा॒ हैकै॑कस्मान्मु॒ष्टिनाद॑दे।
   स हो॑वाच।
   भर॑द्वा॒जेत्या॒मन्त्र्य॑।
   वेदा॒ वा ए॒ते।
   अ॒न॒न्ता वै वेदा।
   ए॒तद्वा ए॒तैस्त्रि॒भिरायु॑र्भि॒रन्व॑वोचथाः।
   अथ॑ त॒ इत॑र॒दन॑नूक्तमे॒व।
   एही॒मं वि॑द्धि।
   अ॒यं वै स॑र्ववि॒द्येति॑॥४६॥

   तस्मै॑ है॒तम॒ग्नि सा॑वि॒त्रमु॑वाच।
   त स वि॑दि॒त्वा।
   अ॒मृतो॑ भू॒त्वा।
   स्व॒र्गं लो॒कमि॑याय।
   आ॒दि॒त्यस्य॒ सायु॑ज्यम्।
   अ॒मृतो॑ है॒व भू॒त्वा।
   स्व॒र्गं लो॒कमे॑ति।
   आ॒दि॒त्यस्य॒ सायु॑ज्यम्।
   य ए॒वं वेद॑।
   ए॒षो ए॒व त्रयी॑ वि॒द्या॥४७॥

   याव॑न्त ह॒ वै त्र॒य्या वि॒द्यया॑ लो॒कं ज॑यति।
   ताव॑न्तं लो॒कं ज॑यति।
   य ए॒वं वेद॑।
   अ॒ग्नेर्वा ए॒तानि॑ नाम॒धेया॑नि।
   अ॒ग्नेरे॒व सायु॑ज्य सलो॒कता॑माप्नोति।
   य ए॒वं वेद॑।
   वा॒योर्वा ए॒तानि॑ नाम॒धेया॑नि।
   वा॒योरे॒व सायु॑ज्य सलो॒कता॑माप्नोति।
   य ए॒वं वेद॑।
   इन्द्र॑स्य॒ वा ए॒तानि॑ नाम॒धेया॑नि॥४८॥

   इन्द्र॑स्यै॒व सायु॑ज्य सलो॒कता॑माप्नोति।
   य ए॒वं वेद॑।
   बृह॒स्पते॒र्वा ए॒तानि॑ नाम॒धेया॑नि।
   बृह॒स्पते॑रे॒व सायु॑ज्य सलो॒कता॑माप्नोति।
   य ए॒वं वेद॑।
   प्र॒जाप॑ते॒र्वा एता॒नि॑ नाम॒धेया॑नि।
   प्र॒जाप॑तेरे॒व सायु॑ज्य सलो॒कता॑माप्नोति।
   य ए॒वं वेद॑।
   ब्रह्म॑णो॒ वा ए॒तानि॑ नाम॒धेया॑नि।
   ब्रह्म॑ण ए॒व सायु॑ज्य सलो॒कता॑माप्नोति।
   य ए॒वं वेद॑।
   स वा ए॒षोऽग्निर॑पक्षपु॒च्छो वा॒युरे॒व।
   तस्या॒ग्निर्मुखम्।
   अ॒सावा॑दि॒त्यः शिर॑।
   स यदे॒ते दे॒वते॒ अन्त॑रेण।
   त॒त्सर्व सीव्यति।
   तस्मात्सावि॒त्रः॥४९॥
   \anuvakamend
   
॥इति कृष्णयजुर्वेदीयतैत्तिरीयकाठके प्रथमः प्रश्नः समाप्तः॥१॥

\sect{द्वितीयः प्रश्नः}
\setcounter{anuvakam}{0}

   लो॒को॑ऽसि स्व॒र्गो॑ऽसि।
   अ॒न॒न्तोऽस्यपा॒रो॑ऽसि।
   अक्षि॑तोऽस्यक्ष॒य्यो॑\-ऽसि।
   तप॑सः प्रति॒ष्ठा।
   त्वयी॒दम॒न्तः।
   विश्वं॑ य॒क्षं विश्वं॑ भू॒तं विश्व सुभू॒तम्।
   विश्व॑स्य भ॒र्ता विश्व॑स्य जनयि॒ता।
   तं त्वोप॑दधे काम॒दुघ॒मक्षि॑तम्।
   प्र॒जाप॑तिस्त्वा सादयतु।
   तया॑ दे॒वतया॑ऽङ्गिर॒स्वद्ध्रु॒वा सी॑द॥१॥

   तपो॑ऽसि लो॒के श्रि॒तम्।
   तेज॑सः प्रति॒ष्ठा।
   त्वयी॒दम॒न्तः।
   विश्वं॑ य॒क्षं विश्वं॑ भू॒तं विश्व सुभू॒तम्।
   विश्व॑स्य भ॒र्तृ विश्व॑स्य जनयि॒तृ।
   तत्त्वोप॑दधे काम॒दुघ॒मक्षि॑तम्।
   प्र॒जाप॑तिस्त्वा सादयतु।
   तया॑ दे॒वतया॑ऽङ्गिर॒स्वद्ध्रु॒वा सी॑द॥२॥

   तेजो॑ऽसि॒ तप॑सि श्रि॒तम्।
   स॒मु॒द्रस्य॑ प्रति॒ष्ठा।
   त्वयी॒दम॒न्तः।
   विश्वं॑ य॒क्षं विश्वं॑ भू॒तं विश्व सुभू॒तम्।
   विश्व॑स्य भ॒र्तृ विश्व॑स्य जनयि॒तृ।
   तत्त्वोप॑दधे काम॒दुघ॒मक्षि॑तम्।
   प्र॒जाप॑तिस्त्वा सादयतु।
   तया॑ दे॒वत॑याऽङ्गिर॒स्वद्ध्रु॒वा सी॑द॥३॥

   स॒मु॒द्रो॑ऽसि॒ तेज॑सि श्रि॒तः।
   अ॒पां प्र॑ति॒ष्ठा।
   त्वयी॒दम॒न्तः।
   विश्वं॑ य॒क्षं विश्वं॑ भू॒तं विश्व सुभू॒तम्।
   विश्व॑स्य भ॒र्ता विश्व॑स्य जनयि॒ता।
   तं त्वोप॑दधे काम॒दुघ॒मक्षि॑तम्।
   प्र॒जाप॑तिस्त्वा सादयतु।
   तया॑ दे॒वत॑याऽङ्गिर॒स्वद्ध्रु॒वा सी॑द॥४॥

   आप॑ स्थ समु॒द्रे श्रि॒ताः।
   पृ॒थि॒व्याः प्र॑ति॒ष्ठा यु॒ष्मासु॑।
   इ॒दम॒न्तः।
   विश्वं॑ य॒क्षं विश्वं॑ भू॒तं विश्व सुभू॒तम्।
   विश्व॑स्य भ॒र्त्र्यो॑ विश्व॑स्य जनयि॒त्र्य॑।
   ता व॒ उप॑दधे काम॒दुघा॒ अक्षि॑ताः।
   प्र॒जाप॑तिस्त्वा सादयतु।
   तया॑ दे॒वतया॑ऽङ्गिर॒स्वद्ध्रु॒वा सी॑द॥५॥

   पृ॒थि॒व्य॑स्य॒प्सु श्रि॒ता।
   अ॒ग्नेः प्र॑ति॒ष्ठा।
   त्वयी॒दम॒न्तः।
   विश्वं॑ य॒क्षं विश्वं॑ भू॒तं विश्व सुभू॒तम्।
   विश्व॑स्य भ॒र्त्री विश्व॑स्य जनयि॒त्री।
   तां त्वोप॑दधे काम॒दुघा॒मक्षि॑ताम्।
   प्र॒जाप॑तिस्त्वा सादयतु।
   तया॑ दे॒वत॑याऽङ्गिर॒स्वद्ध्रु॒वा सी॑द॥६॥

   अ॒ग्निर॑सि पृथि॒व्या श्रि॒तः।
   अ॒न्तरि॑क्षस्य प्रति॒ष्ठा।
   त्वयी॒दम॒न्तः।
   विश्वं॑ य॒क्षं विश्वं॑ भू॒तं विश्व सुभू॒तम्।
   विश्व॑स्य भ॒र्ता विश्व॑स्य जनयि॒ता।
   तं त्वोप॑दधे काम॒दुघ॒मक्षि॑तम्।
   प्र॒जाप॑तिस्त्वा सादयतु।
   तया॑ दे॒वत॑याऽङ्गिर॒स्वद्ध्रु॒वा सी॑द॥७॥

   अ॒न्तरि॑क्षमस्य॒ग्नौ श्रि॒तम्।
   वा॒योः प्र॑ति॒ष्ठा।
   त्वयी॒दम॒न्तः।
   विश्वं॑ य॒क्षं विश्वं॑ भू॒तं विश्व सुभू॒तम्।
   विश्व॑स्य भ॒र्तृ विश्व॑स्य जनयि॒तृ।
   तत्त्वोप॑दधे काम॒दुघ॒मक्षि॑तम्।
   प्र॒जाप॑तिस्त्वा सादयतु।
   तया॑ दे॒वत॑याऽङ्गिर॒स्वद्ध्रु॒वा सी॑द॥८॥

   वा॒युर॑स्य॒न्तरि॑क्षे श्रि॒तः।
   दि॒वः प्र॑ति॒ष्ठा।
   त्वयी॒दम॒न्तः।
   विश्वं॑ य॒क्षं विश्वं॑ भू॒तं विश्व सुभू॒तम्।
   विश्व॑स्य भ॒र्ता विश्व॑स्य जनयि॒ता।
   तं त्वोप॑दधे काम॒दुघ॒मक्षि॑तम्।
   प्र॒जाप॑तिस्त्वा सादयतु।
   तया॑ दे॒वत॑याऽङ्गिर॒स्वद्ध्रु॒वा सी॑द॥९॥

   द्यौर॑सि वा॒यौ श्रि॒ता।
   आ॒दि॒त्यस्य॑ प्रति॒ष्ठा।
   त्वयी॒दम॒न्तः।
   विश्वं॑ य॒क्षं विश्वं॑ भू॒तं विश्व सुभू॒तम्।
   विश्व॑स्य भ॒र्त्री विश्व॑स्य जनयि॒त्री।
   तं त्वोप॑दधे काम॒दुघ॒मक्षि॑तम्।
   प्र॒जाप॑तिस्त्वा सादयतु।
   तया॑ दे॒वत॑याऽङ्गिर॒स्वद्ध्रु॒वा सी॑द॥१०॥

   आ॒दि॒त्यो॑ऽसि दि॒वि श्रि॒तः।
   च॒न्द्रम॑सः प्रति॒ष्ठा।
   त्वयी॒दम॒न्तः।
   विश्वं॑ य॒क्षं विश्वं॑ भू॒तं विश्व सुभू॒तम्।
   विश्व॑स्य भ॒र्ता विश्व॑स्य जनयि॒ता।
   तं त्वोप॑दधे काम॒दुघ॒मक्षि॑तम्।
   प्र॒जाप॑तिस्त्वा सादयतु।
   तया॑ दे॒वत॑याऽङ्गिर॒स्वद्ध्रु॒वा सी॑द॥११॥

   च॒न्द्रमा॑ अस्यादि॒त्ये श्रि॒तः।
   नक्ष॑त्राणां प्रति॒ष्ठा।
   त्वयी॒दम॒न्तः।
   विश्वं॑ य॒क्षं विश्वं॑ भू॒तं विश्व सुभू॒तम्।
   विश्व॑स्य भ॒र्ता विश्व॑स्य जनयि॒ता।
   तं त्वोप॑दधे काम॒दुघ॒मक्षि॑तम्।
   प्र॒जाप॑तिस्त्वा सादयतु।
   तया॑ दे॒वत॑याऽङ्गिर॒स्वद्ध्रु॒वा सी॑द॥१२॥

   नक्ष॑त्राणि स्थ च॒न्द्रम॑सि श्रि॒तानि॑।
   सं॒व॒त्स॒रस्य॑ प्रति॒ष्ठा यु॒ष्मासु॑। 
   इ॒दम॒न्तः।
   विश्वं॑ य॒क्षं विश्वं॑ भू॒तं विश्व सुभू॒तम्।
   विश्व॑स्य भ॒र्तॄणि॒ विश्व॑स्य जनयि॒तॄणि॑।
   तानि॑ व॒ उप॑दधे काम॒दुघा॒न्यक्षि॑तानि।
   प्र॒जाप॑तिस्त्वा सादयतु।
   तया॑ दे॒वत॑याऽङ्गिर॒स्वद्ध्रु॒वा सी॑द॥१३॥

   सं॒व॒त्स॒रो॑ऽसि॒ नक्ष॑त्रेषु श्रि॒तः।
   ऋ॒तू॒नां प्र॑ति॒ष्ठा। 
   त्वयी॒दम॒न्तः।
   विश्वं॑ य॒क्षं विश्वं॑ भू॒तं विश्व सुभू॒तम्।
   विश्व॑स्य भ॒र्ता विश्व॑स्य जनयि॒ता।
   तं त्वोप॑दधे काम॒दुघ॒मक्षि॑तम्।
   प्र॒जाप॑तिस्त्वा सादयतु।
   तया॑ दे॒वत॑याऽङ्गिर॒स्वद्ध्रु॒वा सी॑द॥१४॥

   ऋ॒तव॑ स्थ संवत्स॒रे श्रि॒ताः।
   मासा॑नां प्रति॒ष्ठा यु॒ष्मासु॑। 
   इ॒दम॒न्तः।
   विश्वं॑ य॒क्षं विश्वं॑ भू॒तं विश्व सुभू॒तम्।
   विश्व॑स्य भ॒र्तारो॒ विश्व॑स्य जनयि॒तार॑।
   तान् व॒ उप॑दधे काम॒दुघा॒नक्षि॑तान्।
   प्र॒जाप॑तिस्त्वा सादयतु।
   तया॑ दे॒वत॑याऽङ्गिर॒स्वद्ध्रु॒वा सी॑द॥१५॥

   मासा स्थ॒र्तषु॑ श्रि॒ताः।
   अ॒र्ध॒मा॒सानां प्रति॒ष्ठा यु॒ष्मासु॑। 
   इ॒दम॒न्तः।
   विश्वं॑ य॒क्षं विश्वं॑ भू॒तं विश्व सुभू॒तम्।
   विश्व॑स्य भ॒र्तारो॒ विश्व॑स्य जनयि॒तार॑।
   तान्व॒ उप॑दधे काम॒दुघा॒नक्षि॑तान्।
   प्र॒जाप॑तिस्त्वा सादयतु।
   तया॑ दे॒वत॑याऽङ्गिर॒स्वद्ध्रु॒वा सी॑द॥१६॥

   अ॒र्ध॒मा॒साः स्थ॑ मा॒सु श्रि॒ताः।
   अ॒हो॒रा॒त्रयो प्रति॒ष्ठा यु॒ष्मासु॑। 
   इ॒दम॒न्तः।
   विश्वं॑ य॒क्षं विश्वं॑ भू॒तं विश्व सुभू॒तम्।
   विश्व॑स्य भ॒र्तारो॒ विश्व॑स्य जनयि॒तार॑।
   तान्व॒ उप॑दधे काम॒दुघा॒नक्षि॑तान्।
   प्र॒जाप॑तिस्त्वा सादयतु।
   तया॑ दे॒वत॑याऽङ्गिर॒स्वद्ध्रु॒वा सी॑द॥१७॥

   अ॒हो॒रा॒त्रे स्थोऽर्धमा॒सेषु॑ श्रि॒ते।
   भू॒तस्य॑ प्रति॒ष्ठे भव्य॑स्य प्रति॒ष्ठे।
   यु॒वयो॑रि॒दम॒न्तः।
   विश्वं॑ य॒क्षं विश्वं॑ भू॒तं विश्व सुभू॒तम्।
   विश्व॑स्य भ॒र्त्र्यौ॑ विश्व॑स्य जनयि॒त्र्यौ।
   ते वा॒मुप॑दधे काम॒दुघे॒ अक्षि॑ते।
   प्र॒जाप॑तिस्त्वा सादयतु।
   तया॑ दे॒वत॑याऽङ्गिर॒स्वद्ध्रु॒वा सी॑द॥१८॥

   पौ॒र्ण॒मा॒स्यष्ट॑काऽमावा॒स्या।
   अ॒न्ना॒दाः स्थान्न॒दुघो॑ यु॒ष्मासु॑। 
   इ॒दम॒न्तः।
   विश्वं॑ य॒क्षं विश्वं॑ भू॒तं विश्व सुभू॒तम्।
   विश्व॑स्य भ॒र्त्र्यो॑ विश्व॑स्य जनयि॒त्र्य॑।
   ता व॒ उप॑दधे काम॒दुघा॒ अक्षि॑ताः।
   प्र॒जाप॑तिस्त्वा सादयतु।
   तया॑ दे॒वत॑याऽङ्गिर॒स्वद्ध्रु॒वा सी॑द॥१९॥

   राड॑सि बृह॒ती श्रीर॒सीन्द्र॑पत्नी॒ धर्म॑पत्नी।
   विश्वं॑ भू॒तमनु॒प्रभू॑ता।
   त्वयी॒दम॒न्तः।
   विश्वं॑ य॒क्षं विश्वं॑ भू॒तं विश्व सुभू॒तम्।
   विश्व॑स्य भ॒र्त्री विश्व॑स्य जनयि॒त्री।
   तां त्वोप॑दधे काम॒दुघा॒मक्षि॑ताम्।
   प्र॒जाप॑तिस्त्वा सादयतु।
   तया॑ दे॒वत॑याऽङ्गिर॒स्वद्ध्रु॒वा सी॑द॥२०॥

   ओजो॑ऽसि॒ सहो॑ऽसि।
   बल॑मसि॒ भ्राजो॑ऽसि।
   दे॒वानां॒ धामा॒मृतम्।
   अम॑र्त्यस्तपो॒जाः।
   त्वयी॒दम॒न्तः।
   विश्वं॑ य॒क्षं विश्वं॑ भू॒तं विश्व सुभू॒तम्।
   विश्व॑स्य भ॒र्ता विश्व॑स्य जनयि॒ता।
   तं त्वोप॑दधे काम॒दुघ॒मक्षि॑तम्।
   प्र॒जाप॑तिस्त्वा सादयतु।
   तया॑ दे॒वत॑याऽङ्गिर॒स्वद्ध्रु॒वा सी॑द॥२१॥
\anuvakamend

   त्वम॑ग्ने रु॒द्रो असु॑रो म॒हो दि॒वः।
   त्व शर्धो॒ मारु॑तं पृ॒क्ष ई॑शिषे।
   त्वं वातै॑ररु॒णैर्या॑सि शङ्ग॒यः।
   त्वं पू॒षा वि॑ध॒तः पा॑सि॒ नु त्मना।
   देवा॑ दे॒वेषु॑ श्रयध्वम्।
   प्रथ॑मा द्वि॒तीये॑षु श्रयध्वम्।
   द्विती॑यास्तृ॒तीये॑षु श्रयध्वम्।
   तृती॑याश्चतु॒र्थेषु॑ श्रयध्वम्।
   च॒तु॒र्थाः प॑ञ्च॒मेषु॑ श्रयध्वम्।
   प॒ञ्च॒माः ष॒ष्ठेषु॑ श्रयध्वम्॥२२॥

   ष॒ष्ठाः स॑प्त॒मेषु॑ श्रयध्वम्।
   स॒प्त॒मा अ॑ष्ट॒मेषु॑ श्रयध्वम्।
   अ॒ष्ट॒मा न॑व॒मेषु॑ श्रयध्वम्।
   न॒व॒मा द॑श॒मेषु॑ श्रयध्वम्।
   द॒श॒मा ए॑काद॒शेषु॑ श्रयध्वम्।
   ए॒का॒द॒शा द्वा॑द॒शेषु॑ श्रयध्वम्।
   द्वा॒द॒शास्त्र॑योद॒शेषु॑ श्रयध्वम्।
   त्र॒यो॒द॒शाश्च॑तुर्द॒शेषु॑ श्रयध्वम्।
   च॒तु॒र्द॒शाः प॑ञ्चद॒शेषु॑ श्रयध्वम्।
   प॒ञ्च॒द॒शाः षो॑ड॒शेषु॑ श्रयध्वम्॥२३॥

   षो॒ड॒शाः स॑प्तद॒शेषु॑ श्रयध्वम्।
   स॒प्त॒द॒शा अ॑ष्टाद॒शेषु॑ श्रयध्वम्।
   अ॒ष्टा॒द॒शा ए॑कान्नवि॒शेषु॑ श्रयध्वम्।
   ए॒का॒न्न॒वि॒शा वि॒शेषु॑ श्रयध्वम्।
   वि॒शा ए॑कवि॒शेषु॑ श्रयध्वम्।
   ए॒क॒वि॒शा द्वा॑वि॒शेषु॑ श्रयध्वम्।
   द्वा॒वि॒शास्त्र॑योवि॒शे॑षु श्रयध्वम्।
   त्र॒यो॒वि॒शाश्च॑तुर्वि॒शेषु॑ श्रयध्वम्।
   च॒तु॒र्वि॒शाः प॑ञ्चवि॒शेषु॑ श्रयध्वम्।
   प॒ञ्च॒वि॒शाः ष॑ड्वि॒शेषु॑ श्रयध्वम्॥२४॥

   ष॒ड्वि॒शाः स॑प्तवि॒शेषु॑ श्रयध्वम्।
   स॒प्त॒वि॒शा अ॑ष्टावि॒शेषु॑ श्रयध्वम्।
   अ॒ष्टा॒वि॒शा ए॑कान्नत्रि॒शेषु॑ श्रयध्वम्।
   ए॒का॒न्न॒त्रि॒शास्त्रि॒शेषु॑ श्रयध्वम्।
   त्रि॒शा ए॑कत्रि॒शेषु॑ श्रयध्वम्।
   ए॒क॒त्रि॒शा द्वात्रि॒शेषु॑ श्रयध्वम्।
   द्वा॒त्रि॒शास्त्र॑यस्त्रि॒शेषु॑ श्रयध्वम्।
   देवास्त्रिरेकादशा॒स्त्रिस्त्र॑यस्त्रिशाः।
   उत्त॑रे भवत।
   उत्त॑रवर्त्मान॒ उत्त॑रसत्वानः।
   यत्का॑म इ॒दं जु॒होमि॑।
   तन्मे॒ समृ॑ध्यताम्।
   व॒य स्या॑म॒ पत॑यो रयी॒णाम्।
   भूर्भुव॒ स्व॑ स्वाहा॥२५॥
   \anuvakamend
   
   अग्ना॑विष्णू स॒जोष॑सा।
   इ॒मा व॑र्धन्तु वां॒ गिर॑।
   द्यु॒म्नैर्वाजे॑भि॒राग॑तम्।
   राज्ञी॑ वि॒राज्ञी।
   स॒म्राज्ञी स्व॒राज्ञी।
   अ॒र्चिः शो॒चिः।
   तपो॒ हरो॒ भाः।
   अ॒ग्निः सोमो॒ बृह॒स्पति॑।
   विश्वे॑ दे॒वा भुव॑नस्य गो॒पाः।
   ते सर्वे॑ स॒ङ्गत्य॑।
   इ॒दं मे॒ प्राव॑ता॒ वच॑।
   व॒य स्या॑म॒ पत॑यो रयी॒णाम्।
   भूर्भुव॒ स्व॑ स्वाहा॥२६॥
   \anuvakamend
   
   अन्न॑प॒तेऽन्न॑स्य नो देहि।
   अ॒न॒मी॒वस्य॑ शु॒ष्मिण॑।
   प्र प्र॑दा॒तारं॑ तारिषः।
   ऊर्जं॑ नो धेहि द्वि॒पदे॒ चतु॑ष्पदे।
   अग्ने॑ पृथिवीपते।
   सोम॑ वीरुधां पते।
   त्वष्ट॑ समिधां पते।
   विष्ण॑वाशानां पते।
   मित्र॑ सत्यानां पते।
   वरु॑ण धर्मणां पते॥२७॥

   म॒रुतो॑  गणानां पतयः।
   रुद्र॑ पशूनां पते।
   इन्द्रौ॑जसां पते।
   बृह॑स्पते ब्रह्मणस्पते।
   आ रु॒चा रो॑चे॒ऽह स्व॒यम्।
   रु॒चा रु॑रुचे॒ रोच॑मानः।
   अ॒तीत्या॒दः स्व॑राभ॑रे॒ह।
   तस्मि॒न् योनौ प्रज॒नौ प्रजा॑येय।
   व॒य स्या॑म॒ पत॑यो रयी॒णाम्।
   भूर्भुव॒ स्व॑ स्वाहा॥२८॥
  \anuvakamend
  
    स॒प्त ते॑ अग्ने स॒मिध॑ स॒प्त जि॒ह्वाः।
   स॒प्तर्{‌}ष॑यः स॒प्त धाम॑ प्रि॒याणि॑।
   स॒प्त होत्रा॑ अनुवि॒द्वान्।
   स॒प्त योनी॒रापृ॑णस्वा घृ॒तेन॑।
   प्राची॒ दिक्।
   अ॒ग्निर्दे॒वता।
   अ॒ग्नि स दि॒शां दे॒वं दे॒वता॑नामृच्छतु।
   यो मै॒तस्यै॑ दि॒शो॑ऽभि॒दास॑ति।
   द॒क्षि॒णा दिक्।
   इन्द्रो॑ दे॒वता॥२९॥

   इन्द्र॒ स दि॒शां दे॒वं दे॒वता॑नामृच्छतु।
   यो मै॒तस्यै॑ दि॒शो॑ऽभि॒दास॑ति।
   प्र॒तीची॒ दिक्।
   सोमो॑ दे॒वता।
   सोम॒ स दि॒शां दे॒वं दे॒वता॑नामृच्छतु।
   यो मै॒तस्यै॑ दि॒शो॑ऽभि॒दास॑ति।
   उदी॑ची॒ दिक्।
   मि॒त्रावरु॑णौ दे॒वता।
   मि॒त्रावरु॑णौ॒ स दि॒शां दे॒वौ दे॒वता॑नामृच्छतु।
   यो मै॒तस्यै दि॒शोऽभि॒दास॑ति॥३०॥

   ऊ॒र्ध्वा दिक्।
   बृह॒स्पति॑र्दे॒वता।
   बृह॒स्पति॒ स दि॒शां दे॒वं दे॒वता॑नामृच्छतु।
   यो मै॒तस्यै॑ दि॒शो॑ऽभि॒दास॑ति।
   इ॒यं दिक्।
   अदि॑तिर्दे॒वता।
   अदि॑ति॒ स दि॒शां दे॒वीं दे॒वता॑नामृच्छतु।
   यो मै॒तस्यै॑ दि॒शो॑ऽभि॒दास॑ति।
   पुरु॑षो॒ दिक्।
   पुरु॑षो मे॒ कामा॒न्त्सम॑र्धयतु॥३१॥
   
   अ॒न्धो जागृ॑विः प्राण।
   असा॒वेहि॑।
   ब॒धि॒र आक्रन्दयितरपान।
   असा॒वेहि॑।
   उ॒षस॑मुषसमशीय।
   अ॒हमसो॒ ज्योति॑रशीय।
   अ॒हमसो॒ऽपो॑ऽशीय।
   व॒य स्या॑म॒ पत॑यो रयी॒णाम्।
   भूर्भुव॒ स्व॑ स्वाहा॥३२॥
\anuvakamend
  
   यत्तेऽचि॑तं॒ यदु॑ चि॒तं ते॑ अग्ने।
   यत्त॑ ऊ॒नं यदु॒ तेऽति॑रिक्तम्।
   आ॒दि॒त्यास्त॒दङ्गि॑रसश्चिन्वन्तु।
   विश्वे॑ ते दे॒वाश्चिति॒मापू॑रयन्तु।
   चि॒तश्चासि॒ सञ्चि॑तश्चास्यग्ने।
   ए॒तावा॒श्चासि॒ भूया॑श्चास्यग्ने।
   लो॒कं पृ॑ण च्छि॒द्रं पृ॑ण।
   अथो॑ सीद शि॒वा त्वम्।
   इ॒न्द्रा॒ग्नी त्वा॒ बृह॒स्पति॑।
   अ॒स्मिन् योना॑वसीषदन्॥३३॥

   तया॑ दे॒वत॑याऽङ्गिर॒स्वद्ध्रु॒वा सी॑द।
   ता अ॑स्य॒ सूद॑दोहसः।
   सोम श्रीणन्ति॒ पृश्न॑यः।
   जन्म॑न्दे॒वानां॒ विश॑।
   त्रि॒ष्वा रो॑च॒ने दि॒वः।
   तया॑ दे॒वत॑याऽङ्गिर॒स्वद्ध्रु॒वा सी॑द।
   अग्ने॑ दे॒वा इ॒हाऽऽव॑ह।
   ज॒ज्ञा॒नो वृ॒क्तब॑र्{‌}हिषे।
   असि॒ होता॑ न॒ ईड्य॑।
   अग॑न्म म॒हा मन॑सा॒ यवि॑ष्ठम्॥३४॥

   यो दी॒दाय॒ समि॑द्ध॒ स्वे दु॑रो॒णे।
   चि॒त्रभा॑नू॒ रोद॑सी अ॒न्तरु॒र्वी।
   स्वा॑हुतं वि॒श्वत॑ प्र॒त्यञ्चम्।
   मे॒धा॒का॒रं वि॒दथ॑स्य प्र॒साध॑नम्।
   अ॒ग्नि होता॑रं परि॒भूत॑मं म॒तिम्।
   त्वामर्भ॑स्य ह॒विष॑ समा॒नमित्।
   त्वां म॒हो वृ॑णते॒ नरो॒ नान्यं त्वत्।
   म॒नु॒ष्वत्त्वा॒ निधी॑महि।
   म॒नु॒ष्वत्समि॑धीमहि।
   अग्ने॑ मनु॒ष्वद॑ङ्गिरः॥३५॥

   दे॒वान्दे॑वाय॒ते य॑ज।
   अ॒ग्निर्{‌}हि वा॒जिनं॑ वि॒शे।
   ददा॑ति वि॒श्वच॑र्{‌}षणिः।
   अ॒ग्नी रा॒ये स्वा॒भुवम्।
   स प्री॒तो या॑ति॒ वार्यम्।
   इष स्तो॒तृभ्य॒ आभ॑र।
   पृ॒ष्टो दि॒वि पृ॒ष्टो अ॒ग्निः पृ॑थि॒व्याम्।
   पृ॒ष्टो विश्वा॒ ओष॑धी॒रावि॑वेश।
   वै॒श्वा॒न॒रः सह॑सा पृ॒ष्टो अ॒ग्निः।
   स नो॒ दिवा॒ स रि॒षः पा॑तु॒ नक्तम्॥३६॥
\anuvakamend
  
   अ॒यं वाव यः पव॑ते।
   सोऽग्निर्ना॑चिके॒तः।
   स यत्प्राङ् पव॑ते।
   तद॑स्य॒ शिर॑।
   अथ॒ यद्द॑क्षि॒णा।
   स दक्षि॑णः प॒क्षः।
   अथ॒ यत्प्र॒त्यक्।
   तत्पुच्छम्।
   यदुदङ्{‌}\sG{}।
   स उत्त॑रः प॒क्षः॥३७॥

   अथ॒ यत्सं॒वाति॑।
   तद॑स्य स॒मञ्च॑नं च प्र॒सार॑णं च।
   अथो॑ स॒म्पदे॒वास्य॒ सा।
   स ह॒ वा अ॑स्मै॒ स काम॑ पद्यते।
   यत्का॑मो॒ यज॑ते।
   योऽग्निं ना॑चिके॒तं चि॑नु॒ते।
   य उ॑ चैनमे॒वं वेद॑।
   यो ह॒ वा अ॒ग्नेर्ना॑चिके॒तस्या॒ऽऽयत॑नं प्रति॒ष्ठां वेद॑।
   आ॒यत॑नवान्भवति।
   गच्छ॑ति प्रति॒ष्ठाम्॥३८॥

   हिर॑ण्यं॒ वा अ॒ग्नेर्ना॑चिके॒तस्या॒ऽऽयत॑नं प्रति॒ष्ठा।
   य ए॒वं वेद॑।
   आ॒यत॑नवान्भवति।
   गच्छ॑ति प्रति॒ष्ठाम्।
   यो ह॒ वा अ॒ग्नेर्ना॑चिके॒तस्य॒ शरी॑रं॒ वेद॑।
   सश॑रीर ए॒व स्व॒र्गं लो॒कमे॑ति।
   हिर॑ण्यं॒ वा अ॒ग्नेर्ना॑चिके॒तस्य॒ शरी॑रम्।
   य ए॒वं वेद॑।
   सश॑रीर ए॒व स्व॒र्गं लो॒कमे॑ति।
   अथो॒ यथा॑ रु॒क्म उत्त॑प्तो भा॒य्यात्॥३९॥

   ए॒वमे॒व स तेज॑सा॒ यश॑सा।
   अ॒स्मिश्च॑ लो॒के॑ऽमुष्मि॑श्च भाति।
   उ॒रवो॑ ह॒ वै नामै॒ते लो॒काः।
   येऽव॑रेणाऽऽदि॒त्यम्।
   अथ॑ है॒ते वरी॑यासो लो॒काः।
   ये परे॑णाऽऽदि॒त्यम्।
   अन्त॑वन्त ह॒ वा ए॒ष क्ष॒य्यं लो॒कं ज॑यति।
   योऽव॑रेणाऽऽदि॒त्यम्।
   अथ॑ है॒षो॑ऽन॒न्तम॑पा॒रम॑क्ष॒य्यं लो॒कं ज॑यति।
   यः परे॑णाऽऽदि॒त्यम्॥४०॥
   अ॒न॒न्त ह॒ वा अ॑पा॒रम॑क्ष॒य्यं लो॒कं ज॑यति।
   योऽग्निं ना॑चिके॒तं चि॑नु॒ते।
   य उ॑ चैनमे॒वं वेद॑।
   अथो॒ यथा॒ रथे॒ तिष्ठ॒न्पक्ष॑सी पर्या॒वर्त॑माने प्र॒त्यपेक्षते।
   ए॒वम॑होरा॒त्रे प्र॒त्यपेक्षते।
   नास्या॑होरा॒त्रे लो॒कमाप्नुतः।
   योऽग्निं ना॑चिके॒तं चि॑नु॒ते।
   य उ॑ चैनमे॒वं वेद॑॥४१॥
\anuvakamend
  
   उ॒शन् ह॒ वै वा॑जश्रव॒सः स॑र्ववेद॒सं द॑दौ।
   तस्य॑ ह॒ नचि॑केता॒ नाम॑ पु॒त्र आ॑स।
   त ह॑ कुमा॒र सन्तम्।
   दक्षि॑णासु नी॒यमा॑नासु श्र॒द्धाऽऽवि॑वेश।
   स हो॑वाच।
   तत॒ कस्मै॒ मां दास्य॒सीति॑।
   द्वि॒तीयं॑ तृ॒तीयम्।
   त ह॒ परी॑त उवाच।
   मृ॒त्यवे त्वा ददा॒मीति॑।
   त ह॒ स्मोत्थि॑तं॒ वाग॒भिव॑दति॥४२॥

   गौत॑म कुमा॒रमिति॑।
   स हो॑वाच।
   परे॑हि मृ॒त्योर्गृ॒हान्।
   मृ॒त्यवे॒ वै त्वा॑ऽदा॒मिति॑।
   तं वै प्र॒वस॑न्तं ग॒न्तासीति॑ होवाच।
   तस्य॑ स्म ति॒स्रो रात्री॒रनाश्वान्गृ॒हे व॑सतात्।
   स यदि॑ त्वा पृ॒च्छेत्।
   कुमा॑र॒ कति॒ रात्री॑रवात्सी॒रिति॑।
   ति॒स्र इति॒ प्रति॑ब्रूतात्।
   किं प्र॑थ॒मा रात्रि॑माश्ना॒ इति॑॥४३॥

   प्र॒जां त॒ इति॑।
   किं द्वि॒तीया॒मिति॑।
   प॒शूस्त॒ इति॑।
   किं तृ॒तीया॒मिति॑।
   सा॒धु॒कृ॒त्यां त॒ इति॑।
   तं वै प्र॒वस॑न्तं जगाम।
   तस्य॑ ह ति॒स्रो रात्री॒रनाश्वान्गृ॒ह उ॑वास।
   तमा॒गत्य॑ पप्रच्छ।
   कुमा॑र॒ कति॒ रात्री॑रवात्सी॒रिति॑।
   ति॒स्र इति॒ प्रत्यु॑वाच॥४४॥

   किं प्र॑थ॒मा रात्रि॑माश्ना॒ इति॑।
   प्र॒जां त॒ इति॑।
   किं द्वि॒तीया॒मिति॑।
   प॒शूस्त॒ इति॑।
   किं तृ॒तीया॒मिति॑।
   सा॒धु॒कृ॒त्यां त॒ इति॑।
   नम॑स्ते अस्तु भगव॒ इति॑ होवाच।
   वरं॑ वृणी॒ष्वेति॑।
   पि॒तर॑मे॒व जीव॑न्नया॒नीति॑।
   द्वि॒तीयं॑ वृणी॒ष्वेति॑॥४५॥

   इ॒ष्टा॒पू॒र्तयो॒र्मेऽक्षि॑तिं ब्रू॒हीति॑ होवाच।
   तस्मै॑ है॒तम॒ग्निं ना॑चिके॒तमु॑वाच।
   ततो॒ वै तस्येष्टापू॒र्ते ना क्षी॑येते।
   नास्येष्टापू॒र्ते क्षी॑येते।
   योऽग्निं ना॑चिके॒तं चि॑नु॒ते।
   य उ॑ चैनमे॒वं वेद॑।
   तृ॒तीयं॑ वृणी॒ष्वेति॑।
   पु॒न॒र्मृ॒त्योर्मेऽप॑चितिं ब्रू॒हीति॑ होवाच।
   तस्मै॑ है॒तम॒ग्निं ना॑चिके॒तमु॑वाच।
   ततो॒ वै सोऽप॑ पुनर्मृ॒त्युम॑जयत्॥४६॥

   अप॑ पुनर्मृ॒त्युं ज॑यति।
   योऽग्निं ना॑चिके॒तं चि॑नु॒ते।
   य उ॑ चैनमे॒वं वेद॑।
   प्र॒जाप॑ति॒र्वै प्र॒जाका॑म॒स्तपो॑ऽतप्यत।
   स हि॑रण्य॒मुदास्यत्।
   तद॒ग्नौ प्रास्य॑त्।
   तद॑स्मै॒ नाच्छ॑दयत्।
   तद्द्वि॒तीयं॒ प्रास्य॑त्।
   तद॑स्मै॒ नैवाच्छ॑दयत्।
   तत्तृ॒तीयं॒ प्रास्य॑त्॥४७॥

   तद॑स्मै॒ नैवाच्छ॑दयत्।
   तदा॒त्मन्ने॒व हृ॑द॒य्येऽग्नौ वैश्वान॒रे प्रास्य॑त्।
   तद॑स्मा अच्छदयत्।
   तस्मा॒द्धिर॑ण्यं॒ कनि॑ष्ठं॒ धना॑नाम्।
   भु॒ञ्जत्प्रि॒यत॑मम्।
   हृ॒द॒य॒ज हि।
   स वै तमे॒व नावि॑न्दत्।
   यस्मै॒ तां दक्षि॑णा॒मनेष्यत्।
   ता स्वायै॒व हस्ता॑य॒ दक्षि॑णायानयत्।
   तां प्रत्य॑गृह्णात्॥४८॥

   दक्षा॑य त्वा॒ दक्षि॑णां॒ प्रति॑गृह्णा॒मीति॑।
   सो॑ऽदक्षत॒ दक्षि॑णां प्रति॒गृह्य॑।
   दक्ष॑ते ह॒ वै दक्षि॑णां प्रति॒गृह्य॑।
   य ए॒वं वेद॑।
   ए॒तद्ध॑ स्म॒ वै तद्वि॒द्वासो॑ वाजश्रव॒सा गोत॑माः।
   अप्य॑नूदे॒श्यां दक्षि॑णां॒ प्रति॑गृह्णन्ति।
   उ॒भये॑न व॒यं द॑क्षिष्यामह ए॒व दक्षि॑णां प्रति॒गृह्येति॑।
   ते॑ऽदक्षन्त॒ दक्षि॑णां प्रति॒गृह्य॑।
   दक्ष॑ते ह॒ वै दक्षि॑णां प्रति॒गृह्य॑।
   य ए॒वं वेद॑।
   प्र  हा॒न्यं व्ली॑नाति॥४९॥
   \anuvakamend
  
   त है॒तमेके॑ पशुब॒न्ध ए॒वोत्त॑रवे॒द्यां चि॑न्वते।
   उ॒त्त॒र॒वे॒दिस॑म्मित ए॒षोऽग्निरिति॒ वद॑न्तः।
   तन्न तथा॑ कु॒र्यात्।
   ए॒तम॒ग्निं कामे॑न॒ व्य॑र्धयेत्।
   स ए॑नं॒ कामे॑न॒ व्यृ॑द्धः।
   कामे॑न॒ व्य॑र्धयेत्।
   सौ॒म्ये वावैन॑मध्व॒रे चि॑न्वी॒त।
   यत्र॑ वा॒ भूयि॑ष्ठा॒ आहु॑तयो हू॒येर\sn{}।
   ए॒तम॒ग्निं कामे॑न॒ सम॑र्धयति।
   स ए॑नं॒ कामे॑न॒ समृ॑द्धः॥५०॥

   कामे॑न॒ सम॑र्धयति।
   अथ॑ हैनं पु॒रर्{‌}ष॑यः।
   उ॒त्त॒र॒वे॒द्यामे॒व स॒त्त्रिय॑मचिन्वत।
   ततो॒ वै तेऽवि॑न्दन्त प्र॒जाम्।
   अ॒भि स्व॒र्गं लो॒कम॑जयन्।
   वि॒न्दत॑ ए॒व प्र॒जाम्।
   अ॒भि स्व॒र्गं लो॒कं ज॑यति।
   योऽग्निं ना॑चिके॒तं चि॑नु॒ते।
   य उ॑ चैनमे॒वं वेद॑।
   अथ॑ हैनं वा॒युर्{‌}ऋद्धि॑कामः॥५१॥

   य॒था॒न्यु॒प्तमे॒वोप॑दधे।
   ततो॒ वै स ए॒तामृद्धि॑मार्ध्नोत्।
   यामि॒दं वा॒युर्{‌}ऋ॒द्धः।
   ए॒तामृद्धि॑मृध्नोति।
   यामि॒दं वा॒युर्{‌}ऋ॒द्धः।
   योऽग्निं ना॑चिके॒तं चि॑नु॒ते।
   य उ॑ चैनमे॒वं वेद॑।
   अथ॑ हैनं गोब॒लो वार्ष्ण॑ प॒शुका॑मः।
   पाङ्क्त॑मे॒व चि॑क्ये।
   पञ्च॑ पु॒रस्तात्॥५२॥

   पञ्च॑ दक्षिण॒तः।
   पञ्च॑ प॒श्चात्।
   पञ्चोत्तर॒तः।
   एकां॒ मध्ये।
   ततो॒ वै स स॒हस्रं॑ प॒शून्प्राप्नोत्।
   प्र स॒हस्रं॑ प॒शूनाप्नोति।
   योऽग्निं ना॑चिके॒तं चि॑नु॒ते।
   य उ॑ चैनमे॒वं वेद॑।
   अथ॑ हैनं प्र॒जाप॑ति॒र्ज्यैष्ठ्य॑कामो॒ यश॑स्कामः प्र॒जन॑नकामः।
   त्रि॒वृत॑मे॒व चि॑क्ये॥५३॥

   स॒प्त पु॒रस्तात्।
   ति॒स्रो द॑क्षिण॒तः।
   स॒प्त प॒श्चात्।
   ति॒स्र उ॑त्तर॒तः।
   एकां॒ मध्ये।
   ततो॒ वै स प्र यशो॒ ज्यैष्ठ्य॑माप्नोत्।
   ए॒तां प्रजा॑तिं॒ प्राजा॑यत।
   यामि॒दं प्र॒जाः प्र॒जाय॑न्ते।
   त्रि॒वृद्वै ज्यैष्ठ्यम्।
   मा॒ता पि॒ता पु॒त्रः॥५४॥

   त्रि॒वृत्प्र॒जन॑नम्।
   उ॒पस्थो॒ योनि॑र्मध्य॒मा।
   प्र यशो॒ ज्यैष्ठ्य॑माप्नोति।
   ए॒तां प्रजा॑तिं॒ प्रजा॑यते।
   यामि॒दं प्र॒जाः प्र॒जाय॑न्ते।
   योऽग्निं ना॑चिके॒तं चि॑नु॒ते।
   य उ॑ चैनमे॒वं वेद॑।
   अथ॑ हैन॒मिन्द्रो॒ ज्यैष्ठ्य॑कामः।
   ऊ॒र्ध्वा ए॒वोप॑दधे।
   ततो॒ वै स ज्यैष्ठ्य॑मगच्छत्॥५५॥

   ज्यैष्ठ्यं॑ गच्छति।
   योऽग्निं ना॑चिके॒तं चि॑नु॒ते।
   य उ॑ चैनमे॒वं वेद॑।
   अथ॑ हैनम॒सावा॑दि॒त्यः स्व॒र्गका॑मः।
   प्राची॑रे॒वोप॑दधे।
   ततो॒ वै सो॑ऽभि स्व॒र्गं लो॒कम॑जयत्।
   अ॒भि स्व॒र्गं लो॒कं ज॑यति।
   योऽग्निं ना॑चिके॒तं चि॑नु॒ते।
   य उ॑ चैनमे॒वं वेद॑।
   स यदी॒च्छेत्॥५६॥

   ते॒ज॒स्वी य॑श॒स्वी ब्र॑ह्मवर्च॒सी स्या॒मिति॑।
   प्राङाहोतु॒र्धिष्ण्या॒दुत्स॑र्पेत्।
   येयं प्रागा॒द्यश॑स्वती।
   सा मा॒ प्रोर्णो॑तु।
   तेज॑सा॒ यश॑सा ब्रह्मवर्च॒सेनेति॑।
   ते॒ज॒स्व्ये॑व य॑श॒स्वी ब्र॑ह्मवर्च॒सी भ॑वति।
   अथ॒ यदी॒च्छेत्।
   भूयि॑ष्ठं मे॒ श्रद्द॑धीरन्।
   भूयि॑ष्ठा॒ दक्षि॑णा नयेयु॒रिति॑।
   दक्षि॑णासु नी॒यमा॑नासु॒ प्राच्येहि॒ प्राच्ये॒हीति॒ प्राची॑ जुषा॒णा वेत्वाज्य॑स्य॒ स्वाहेति॑ स्रु॒वेणो॑प॒हत्या॑ऽऽहव॒नीये॑ जुहुयात्॥५७॥

   भूयि॑ष्ठमे॒वास्मै॒ श्रद्द॑धते।
   भूयि॑ष्ठा॒ दक्षि॑णा नयन्ति।
   पुरी॑षमुप॒धाय॑।
   चि॒ति॒कॢ॒प्तिभि॑रभि॒मृश्य॑।
   अ॒ग्निं प्र॒णीयो॑पसमा॒धाय॑।
   चत॑स्र ए॒ता आहु॑तीर्जुहोति।
   त्वम॑ग्ने रु॒द्र इति॑ शतरु॒द्रीय॑स्य रू॒पम्।
   अग्ना॑विष्णू॒ इति॑ वसो॒र्धारा॑याः।
   अन्न॑पत॒ इत्य॑न्नहो॒मः।
   स॒प्त ते॑ अग्ने स॒मिध॑ स॒प्त जि॒ह्वा इति॑ विश्व॒प्रीः॥५८॥
\anuvakamend
  
   यां प्र॑थ॒मामिष्ट॑कामुप॒दधा॑ति।
   इ॒मं तया॑ लो॒कम॒भिज॑यति।
   अथो॒ या अ॒स्मिँल्लो॒के दे॒वता।
   तासा॒ सायु॑ज्य सलो॒कता॑माप्नोति।
   यां द्वि॒तीया॑मुप॒दधा॑ति।
   अ॒न्त॒रि॒क्ष॒लो॒कं तया॒ऽभिज॑यति।
   अथो॒ या अ॑न्तरिक्षलो॒के दे॒वता।
   तासा॒ सायु॑ज्य सलो॒कता॑माप्नोति।
   यां तृ॒तीया॑मुप॒दधा॑ति।
   अ॒मुं तया॑ लो॒कम॒भिज॑यति॥५९॥

   अथो॒ या अ॒मुष्मिँल्लो॒के दे॒वता।
   तासा॒ सायु॑ज्य सलो॒कता॑माप्नोति।
   अथो॒ या अ॒मूरित॑रा अ॒ष्टाद॑श।
   य ए॒वामी उ॒रव॑श्च॒ वरी॑यासश्च लो॒काः।
   ताने॒व ताभि॑र॒भिज॑यति॥
   का॒म॒चारो॑ ह॒ वा अ॑स्यो॒रुषु॑ च॒ वरी॑यःसु च लो॒केषु॑ भवति।
   योऽग्निं ना॑चिके॒तं चि॑नु॒ते।
   य उ॑ चैनमे॒वं वेद॑।
   सं॒व॒त्स॒रो वा अ॒ग्निर्ना॑चिके॒तः।
   तस्य॑ वस॒न्तः शिर॑॥६०॥

   ग्री॒ष्मो दक्षि॑णः प॒क्षः।
   व॒र्{‌}षा उत्त॑रः।
   श॒रत्पुच्छम्।
   मासा कर्मका॒राः।
   अ॒हो॒रा॒त्रे श॑तरु॒द्रीयम्।
   प॒र्जन्यो॒ वसो॒र्धारा।
   यथा॒ वै प॒र्जन्य॒ सुवृ॑ष्टं वृ॒ष्ट्वा।
   प्र॒जाभ्य॒ सर्वा॒न्कामान्त्सम्पू॒रय॑ति।
   ए॒वमे॒व स तस्य॒ सर्वा॒न्कामा॒न्त्सम्पू॑रयति।
   योऽग्निं ना॑चिके॒तं चि॑नु॒ते॥६१॥

   य उ॑ चैनमे॒वं वेद॑।
   सं॒व॒त्स॒रो वा अ॒ग्निर्ना॑चिके॒तः।
   तस्य॑ वस॒न्तः शिर॑।
   ग्री॒ष्मो दक्षि॑णः प॒क्षः।
   व॒र्{‌}षाः पुच्छम्।
   श॒रदुत्त॑रः प॒क्षः।
   हे॒म॒न्तो मध्यम्।
   पू॒र्व॒प॒क्षाश्चित॑यः।
   अ॒प॒र॒प॒क्षाः पुरी॑षम्।
   अ॒हो॒रा॒त्राणीष्ट॑काः।
   ए॒ष वाव सोऽग्निर॑ग्नि॒मय॑ पुनर्ण॒वः।
   अ॒ग्नि॒मयो॑ ह॒ वै पु॑नर्ण॒वो भू॒त्वा।
   स्व॒र्गं लो॒कमे॑ति।
   आ॒दि॒त्यस्य॒ सायु॑ज्यम्।
   योऽग्निं ना॑चिके॒तं चि॑नु॒ते।
   य उ॑ चैनमे॒वं वेद॑॥६२॥
\anuvakamend

  ॥इति कृष्णयजुर्वेदीयतैत्तितरीय काठके द्वितीयः प्रश्नः समाप्तः॥२॥


\sect{तृतीयः प्रश्नः}
\setcounter{anuvakam}{0}

   तुभ्यं॒ ता अ॑ङ्गिरस्तमा॒ऽश्याम॒ तं काम॑मग्ने।
   आशा॑नां त्वा॒ विश्वा॒ आशा।
   अनु॑ नो॒ऽद्यानु॑मति॒रन्विद॑नुमते॒ त्वम्।
   कामो॑ भू॒तस्य॒ काम॒स्तदग्रे।
   ब्रह्म॑ जज्ञा॒नं पि॒ता वि॒राजाम्।
   य॒ज्ञो रा॒यो॑ऽयं य॒ज्ञः।
   आपो॑ भ॒द्रा आदित्प॑श्यामि।
   तुभ्यं॑ भरन्ति॒ यो दे॒ह्यः।
   पूर्वं॑ देवा॒ अप॑रेण प्राणापा॒नौ।
   ह॒व्य॒वाह॒ स्वि॑ष्टम्॥१॥
   \anuvakamend
  

   दे॒वेभ्यो॒ वै स्व॒र्गो लो॒कस्ति॒रो॑ऽभवत्।
   ते प्र॒जाप॑तिमब्रुवन्।
   प्रजा॑पते स्व॒र्गो वै नो॑ लो॒कस्ति॒रो॑ऽभूत्।
   तमन्वि॒च्छेति॑।
   तं य॑ज्ञक्र॒तुभि॒रन्वैच्छत्।
   तं य॑ज्ञक्र॒तुभि॒र्नान्व॑विन्दत्।
   तमिष्टि॑भि॒रन्वैच्छत्।
   तमिष्टि॑भि॒रन्व॑विन्दत्।
   तदिष्टी॑नामिष्टि॒त्वम्।
   एष्ट॑यो ह॒ वै नाम॑।
   ता इष्ट॑य॒ इत्याच॑क्षते प॒रोक्षे॑ण।
   प॒रोक्ष॑प्रिया इव॒ हि दे॒वाः॥२॥

   तमाशाऽब्रवीत्।
   प्रजा॑पत आ॒शया॒ वै श्राम्यसि।
   अ॒हमु॒ वा आशाऽस्मि।
   मां नु य॑जस्व।
   अथ॑ ते स॒त्याऽऽशा॑ भविष्यति।
   अनु॑ स्व॒र्गं लो॒कं वे॒त्स्यसीति॑।
   स ए॒तम॒ग्नये॒ कामा॑य पुरो॒डाश॑म॒ष्टाक॑पालं॒ निर॑वपत्।
   आ॒शायै॑ च॒रुम्।
   अनु॑मत्यै च॒रुम्।
   ततो॒ वै तस्य॑ स॒त्याऽऽशा॑ऽभवत्।
   अनु॑ स्व॒र्गं लो॒कम॑विन्दत्।
   स॒त्या ह॒ वा अ॒स्याऽऽशा॑ भवति।
   अनु॑ स्व॒र्गं लो॒कं वि॑न्दति।
   य ए॒तेन॑ ह॒विषा॒ यज॑ते।
   य उ॑ चैनदे॒वं वेद॑।
   सोऽत्र॑ जुहोति।
   अ॒ग्नये॒ कामा॑य॒ स्वाहा॒ऽऽशायै॒ स्वाहा।
   अनु॑मत्यै॒ स्वाहा प्र॒जाप॑तये॒ स्वाहा।
   स्व॒र्गाय॑ लो॒काय॒ स्वाहा॒ऽग्नये स्विष्ट॒कृते॒ स्वाहेति॑॥३॥

   तं कामोऽब्रवीत्।
   प्रजा॑पते॒ कामे॑न॒ वै श्राम्यसि।
   अ॒हमु॒ वै कामोऽस्मि।
   मां नु य॑जस्व।
   अथ॑ ते स॒त्यः कामो॑ भविष्यति।
   अनु॑ स्व॒र्गं लो॒कं वे॒त्स्यसीति॑।
   स ए॒तम॒ग्नये॒ कामा॑य पुरो॒डाश॑म॒ष्टाक॑पालं॒ निर॑वपत्।
   कामा॑य च॒रुम्।
   अनु॑मत्यै च॒रुम्।
   ततो॒ वै तस्य॑ स॒त्यः कामो॑ऽभवत्।
   अनु॑ स्व॒र्गं लो॒कम॑विन्दत्।
   स॒त्यो ह॒ वा अ॑स्य॒ कामो॑ भवति।
   अनु॑ स्व॒र्गं लो॒कं वि॑न्दति।
   य ए॒तेन॑ ह॒विषा॒ यज॑ते।
   य उ॑ चैनदे॒वं वेद॑।
   सोऽत्र॑ जुहोति।
   अ॒ग्नये॒ कामा॑य॒ स्वाहा॒ कामा॑य॒ स्वाहा।
   अनु॑मत्यै॒ स्वाहा प्र॒जाप॑तये॒ स्वाहा।
   स्व॒र्गाय॑ लो॒काय॒ स्वाहा॒ऽग्नये स्विष्ट॒कृते॒ स्वाहेति॑॥४॥

   तं ब्रह्माब्रवीत्।
   प्रजा॑पते॒ ब्रह्म॑णा॒ वै श्राम्यसि।
   अ॒हमु॒ वै ब्रह्मास्मि।
   मां नु यज॑स्व।
   अथ॑ ते ब्रह्म॒ण्वान् य॒ज्ञो भ॑विष्यति।
   अनु॑ स्व॒र्गं लो॒कं वे॒त्स्यसीति॑।
   स ए॒तम॒ग्नये॒ कामा॑य पुरो॒डाश॑म॒ष्टाक॑पालं॒ निर॑वपत्।
   ब्रह्म॑णे च॒रुम्।
   अनु॑मत्यै च॒रुम्।
   ततो॒ वै तस्य॑ ब्रह्म॒ण्वान् य॒ज्ञो॑ऽभवत्।
   अनु॑ स्व॒र्गं लो॒कम॑विन्दत्।
   ब्र॒ह्म॒ण्वान् ह॒ वा अ॑स्य य॒ज्ञो भ॑वति।
   अनु॑ स्व॒र्गं लो॒कं वि॑न्दति।
   य ए॒तेन॑ ह॒विषा॒ यज॑ते।
   य उ॑ चैनदे॒वं वेद॑।
   सोऽत्र॑ जुहोति।
   अ॒ग्नये॒ कामा॑य॒ स्वाहा॒ ब्रह्म॑णे॒ स्वाहा।
   अनु॑मत्यै॒ स्वाहा प्र॒जाप॑तये॒ स्वाहा।
   स्व॒र्गाय॑ लो॒काय॒ स्वाहा॒ऽग्नये स्विष्ट॒कृते॒ स्वाहेति॑॥५॥

   तं य॒ज्ञोऽब्रवीत्।
   प्रजा॑पते य॒ज्ञेन॒ वै श्राम्यसि।
   अ॒हमु॒ वै य॒ज्ञोऽस्मि।
   मां नु य॑जस्व।
   अथ॑ ते स॒त्यो य॒ज्ञो भ॑वष्यति।
   अनु॑ स्व॒र्गं लो॒कं वे॒त्स्यसीति॑।
   स ए॒तम॒ग्नये॒ कामा॑य पुरो॒डाश॑म॒ष्टाक॑पालं॒ निर॑वपत्।
   य॒ज्ञाय॑ च॒रुम्।
   अनु॑मत्यै च॒रुम्।
   ततो॒ वै तस्य॑ स॒त्यो य॒ज्ञो॑ऽभवत्।
   अनु॑ स्व॒र्गं लो॒कम॑विन्दत्।
   स॒त्यो ह॒ वा अ॑स्य य॒ज्ञो भ॑वति।
   अनु॑ स्व॒र्गं लो॒कं वि॑न्दति।
   य ए॒तेन॑ ह॒विषा॒ यज॑ते।
   य उ॑ चैनदे॒वं वेद॑।
   सोऽत्र॑ जुहोति।
   अ॒ग्नये॒ कामा॑य॒ स्वाहा॑ य॒ज्ञाय॒ स्वाहा।
   अनु॑मत्यै॒ स्वाहा प्र॒जाप॑तये॒ स्वाहा।
   स्व॒र्गाय॑ लो॒काय॒ स्वाहा॒ऽग्नये स्विष्ट॒कृते॒ स्वाहेति॑॥६॥

   तमापोऽब्रुवन्।
   प्रजा॑पते॒ऽप्सु वै सर्वे॒ कामा श्रि॒ताः।
   व॒यमु॒ वा आप॑ स्मः।
   अ॒स्मान्नु य॑जस्व।
   अथ॒ त्वयि॒ सर्वे॒ कामा श्रयिष्यन्ते।
   अनु॑ स्व॒र्गं लो॒कं वे॒त्स्यसीति॑।
   स ए॒तम॒ग्नये॒ कामा॑य पुरो॒डाश॑म॒ष्टाक॑पालं॒ निर॑वपत्।
   अ॒द्भ्यश्च॒रुम्।
   अनु॑मत्यै च॒रुम्।
   ततो॒ वै तस्मि॒न्त्सर्वे॒ कामा॑ अश्रयन्त।
   अनु॑ स्व॒र्गं लो॒कम॑विन्दत्।
   सर्वे॑ ह॒ वा अ॑स्मि॒न्कामा श्रयन्ते।
   अनु॑ स्व॒र्गं लो॒कं वि॑न्दति।
   य ए॒तेन॑ ह॒विषा॒ यज॑ते।
   य उ॑ चैनदे॒वं वेद॑।
   सोऽत्र॑ जुहोति।
   अ॒ग्नये॒ कामा॑य॒ स्वाहा॒ऽद्भ्यः स्वाहा।
   अनु॑मत्यै॒ स्वाहा प्र॒जाप॑तये॒ स्वाहा।
   स्व॒र्गाय॑ लो॒काय॒ स्वाहा॒ऽग्नये स्विष्ट॒कृते॒ स्वाहेति॑॥७॥

   तम॒ग्निर्ब॑लि॒मान॑ब्रवीत्।
   प्रजा॑पते॒ऽग्नये॒ वै ब॑लि॒मते॒ सर्वा॑णि भू॒तानि॑ ब॒लि ह॑रन्ति।
   अ॒हमु॒ वा अ॒ग्निर्ब॑लि॒मान॑स्मि।
   मां नु य॑जस्व।
   अथ॑ ते॒ सर्वा॑णि भू॒तानि॑ ब॒लि ह॑रिष्यन्ति।
   अनु॑ स्व॒र्गं लो॒कं वे॒त्स्यसीति॑।
   स ए॒तम॒ग्नये॒ कामा॑य पुरो॒डाश॑म॒ष्टाक॑पालं॒ निर॑वपत्।
   अ॒ग्नये॑ बलि॒मते॑ च॒रुम्।
   अनु॑मत्यै च॒रुम्।
   ततो॒ वै तस्मै॒ सर्वा॑णि भू॒तानि॑ ब॒लिम॑हरन्।
   अनु॑ स्व॒र्गं लो॒कम॑विन्दत्।
   सर्वा॑णि ह॒ वा अ॑स्मै भू॒तानि॑ ब॒लि ह॑रन्ति।
   अनु॑ स्व॒र्गं लो॒कं वि॑न्दति।
   य ए॒तेन॑ ह॒विषा॒ यज॑ते।
   य उ॑ चैनदे॒वं वेद॑।
   सोऽत्र॑ जुहोति।
   अ॒ग्नये॒ कामा॑य॒ स्वाहा॒ऽग्नये॑ बलि॒मते॒ स्वाहा।
   अनु॑मत्यै॒ स्वाहा प्र॒जाप॑तये॒ स्वाहा।
   स्व॒र्गाय॑ लो॒काय॒ स्वाहा॒ऽग्नये स्विष्ट॒कृते॒ स्वाहेति॑॥८॥

   तमनु॑वित्तिरब्रवीत्।
   प्रजा॑पते स्व॒र्गं वै लो॒कमनु॑विवित्ससि।
   अ॒हमु॒ वा अनु॑वित्तिरस्मि।
   मां नु य॑जस्व।
   अथ॑ ते स॒त्याऽनु॑वित्तिर्भविष्यति।
   अनु॑ स्व॒र्गं लो॒कं वे॒त्स्यसीति॑।
   स ए॒तम॒ग्नये॒ कामा॑य पुरो॒डाश॑म॒ष्टाक॑पालं॒ निर॑वपत्।
   अनु॑वित्त्यै च॒रुम्।
   अनु॑मत्यै च॒रुम्।
   ततो॒ वै तस्य॑ स॒त्याऽनु॑वित्तिरभवत्।
   अनु॑ स्व॒र्गं लो॒कम॑विन्दत्।
   स॒त्या ह॒ वा अ॒स्यानु॑वित्तिर्भवति।
   अनु॑ स्व॒र्गं लो॒कं वि॑न्दति।
   य ए॒तेन॑ ह॒विषा॒ यज॑ते।
   य उ॑ चैनदे॒वं वेद॑।
   सोऽत्र॑ जुहोति।
   अ॒ग्नये॒ कामा॑य॒ स्वाहाऽनु॑वित्त्यै॒ स्वाहा।
   अनु॑मत्यै॒ स्वाहा प्र॒जाप॑तये॒ स्वाहा।
   स्व॒र्गाय॑ लो॒काय॒ स्वाहा॒ऽग्नये स्विष्ट॒कृते॒ स्वाहेति॑॥९॥

   ता वा ए॒ताः स॒प्त स्व॒र्गस्य॑ लो॒कस्य॒ द्वार॑।
   दि॒वःश्ये॑न॒योऽनु॑वित्तयो॒ नाम॑।
   आशा प्रथ॒मा र॑क्षति।
   कामो द्वि॒तीयाम्।
   ब्रह्म॑ तृ॒तीयाम्।
   य॒ज्ञश्च॑तु॒र्थीम्।
   आप॑ पञ्च॒मीम्।
   अ॒ग्निर्ब॑लि॒मान्त्ष॒ष्ठीम्।
   अनु॑वित्तिः सप्त॒मीम्।
   अनु॑ ह॒ वै स्व॒र्गं लो॒कं वि॑न्दति।
   का॒म॒चारोऽस्य स्व॒र्गे लो॒के भ॑वति।
   य ए॒ताभि॒रिष्टि॑भि॒र्यज॑ते।
   य उ॑ चैना ए॒वं वेद॑।
   तास्व॑न्वि॒ष्टि।
   प॒ष्ठौ॒ही॒व॒रां द॑द्यात्क॒सं च॑।
   स्त्रियै॑ चाऽऽभा॒र समृ॑द्ध्यै॥१०॥
\anuvakamend
  
   तप॑सा दे॒वा दे॒वता॒मग्र॑ आयन्।
   तप॒सर्‌ष॑य॒ स्व॑रन्व॑विन्दन्।
   तप॑सा स॒पत्ना॒न्प्रणु॑दा॒मारा॑तीः।
   येने॒दं विश्वं॒ परि॑भूतं॒ यदस्ति॑।
   प्र॒थ॒म॒जं दे॒व ह॒विषा॑ विधेम।
   स्व॒य॒म्भु ब्रह्म पर॒मं तपो॒ यत्।
   स ए॒व पु॒त्रः स पि॒ता स मा॒ता।
   तपो॑ ह य॒क्षं प्र॑थ॒म सम्ब॑भूव।
   श्र॒द्धया दे॑वो देव॒त्वम॑श्नुते।
   श्र॒द्धा प्र॑ति॒ष्ठा लो॒कस्य॑ दे॒वी॥११॥

   सा नो॑ जुषा॒णोप॑ य॒ज्ञमागात्।
   काम॑वत्सा॒ऽमृतं॒ दुहा॑ना।
   श्र॒द्धा दे॒वी प्र॑थम॒जा ऋ॒तस्य॑।
   विश्व॑स्य भ॒र्त्री जग॑तः प्रति॒ष्ठा।
   ता श्र॒द्धा ह॒विषा॑ यजामहे।
   सा नो॑ लो॒कम॒मृतं॑ दधातु।
   ईशा॑ना दे॒वी भुव॑न॒स्याधि॑पत्नी।
   आगात्स॒त्य ह॒विरि॒दं जु॑षा॒णम्।
   यस्माद्दे॒वा ज॑ज्ञिरे॒ भुव॑नं च॒ विश्वे।
   तस्मै॑ विधेम ह॒विषा॑ घृ॒तेन॑॥१२॥

   यथा॑ दे॒वैः स॑ध॒मादं॑ मदेम।
   यस्य॑ प्रति॒ष्ठोर्व॑न्तरि॑क्षम्।
   यस्माद्दे॒वा ज॑ज्ञिरे॒ भुव॑नं च॒ सर्वे।
   तत्स॒त्यमर्च॒दुप॑ य॒ज्ञं न॒ आगात्।
   ब्रह्माऽऽहु॑ती॒रुप॒मोद॑मानम्।
   मन॑सो॒ वशे॒ सर्व॑मि॒दं ब॑भूव।
   नान्यस्य॒ मनो॒ वश॒मन्वि॑याय।
   भी॒ष्मो हि दे॒वः सह॑स॒ सही॑यान्।
   स नो॑ जुषा॒ण उप॑ य॒ज्ञमागात्।
   आकू॑तीना॒मधि॑पतिं॒ चेत॑सां च॥१३॥

   स॒ङ्क॒ल्पजू॑तिं दे॒वं वि॑प॒श्चिम्।
   मनो॒ राजा॑नमि॒ह व॒र्धय॑न्तः।
   उ॒प॒ह॒वेऽस्य सुम॒तौ स्या॑म।
   चर॑णं प॒वित्रं॒ वित॑तं पुरा॒णम्।
   येन॑ पू॒तस्तर॑ति दुष्कृ॒तानि॑।
   तेन॑ प॒वित्रे॑ण शु॒द्धेन॑ पू॒ताः।
   अति॑ पा॒प्मान॒मरा॑तिं तरेम।
   लो॒कस्य॒ द्वार॑मर्चि॒मत्प॒वित्रम्।
   ज्योति॑ष्म॒द्भ्राज॑मानं॒ मह॑स्वत्।
   अ॒मृत॑स्य॒ धारा॑ बहु॒धा दोह॑मानम्।
   चर॑णं नो लो॒के सुधि॑तां दधातु।
   अ॒ग्निर्मू॒र्धा भुव॑।
   अनु॑ नो॒ऽद्यानु॑मति॒रन्विद॑नुमते॒ त्वम्।
   ह॒व्य॒वाह॒ स्वि॑ष्टम्॥१४॥
\anuvakamend
  
   दे॒वेभ्यो॒ वै स्व॒र्गो लो॒कस्ति॒रो॑ऽभवत्।
   ते प्र॒जाप॑तिमब्रुवन्।
   प्रजा॑पते स्व॒र्गो वै नो॑ लो॒कस्ति॒रो॑ऽभूत्।
   तमन्वि॒च्छेति॑।
   तं य॑ज्ञक्र॒तुभि॒रन्वैच्छत्।
   तं य॑ज्ञक्र॒तुभि॒र्नान्व॑विन्दत्।
   तमिष्टिभि॒\-रन्वै॑च्छत्।
   तमिष्टि॑भि॒रन्व॑विन्दत्।
   तदिष्टी॑नामिष्टि॒त्वम्।
   एष्ट॑यो ह॒ वै नाम॑।
   ता इष्ट॑य॒ इत्याच॑क्षते प॒रोक्षे॑ण।
   प॒रोक्ष॑प्रिया इव॒ हि दे॒वाः॥१५॥

   तं तपोऽब्रवीत्।
   प्रजा॑पते॒ तप॑सा॒ वै श्राम्यसि।
   अ॒हमु॒ वै तपोऽस्मि।
   मां नु य॑जस्व।
   अथ॑ ते स॒त्यं तपो॑ भविष्यति।
   अनु॑ स्व॒र्गं लो॒कं वे॒त्स्यसीति॑।
   स ए॒तमाग्ने॒यम॒ष्टाक॑पालं॒ निर॑वपत्।
   तप॑से च॒रुम्।
   अनु॑मत्यै च॒रुम्।
   ततो॒ वै तस्य॑ स॒त्यं तपो॑ऽभवत्।
   अनु॑ स्व॒र्गं लो॒कम॑विन्दत्।
   स॒त्य ह॒ वा अ॑स्य॒ तपो॑ भवति।
   अनु॑ स्व॒र्गं लो॒कं वि॑न्दति।
   य ए॒तेन॑ ह॒विषा॒ यज॑ते।
   य उ॑ चैनदे॒वं वेद॑।
   सोऽत्र॑ जुहोति।
   अ॒ग्नये॒ स्वाहा॒ तप॑से॒ स्वाहा।
   अनु॑मत्यै॒ स्वाहा प्र॒जाप॑तये॒ स्वाहा।
   स्व॒र्गाय॑ लो॒काय॒ स्वाहा॒ऽग्नये स्विष्ट॒कृते॒ स्वाहेति॑॥१६॥

   त श्र॒द्धाऽब्र॑वीत्।
   प्रजा॑पते श्र॒द्धया॒ वै श्राम्यसि।
   अ॒हमु॒ वै श्र॒द्धाऽस्मि॑।
   मां नु य॑जस्व।
   अथ॑ ते स॒त्या श्र॒द्धा भ॑विष्यति।
   अनु॑ स्व॒र्गं लो॒कं वे॒त्स्यसीति॑।
   स ए॒तमाग्ने॒यम॒ष्टाक॑पालं॒ निर॑वपत्।
   श्र॒द्धायै॑ च॒रुम्।
   अनु॑मत्यै च॒रुम्।
   ततो॒ वै तस्य॑ स॒त्या श्र॒द्धाऽभ॑वत्।
   अनु॑ स्व॒र्गं लो॒कम॑विन्दत्।
   स॒त्या ह॒ वा अ॑स्य श्र॒द्धा भ॑वति।
   अनु॑ स्व॒र्गं लो॒कं वि॑न्दति।
   य ए॒तेन॑ ह॒विषा॒ यज॑ते।
   य उ॑ चैनदे॒वं वेद॑।
   सोऽत्र॑ जुहोति।
   अ॒ग्नये॒ स्वाहा श्र॒द्धायै॒ स्वाहा।
   अनु॑मत्यै॒ स्वाहा प्र॒जाप॑तये॒ स्वाहा।
   स्व॒र्गाय॑ लो॒काय॒ स्वाहा॒ऽग्नये स्विष्ट॒कृते॒ स्वाहेति॑॥१७॥

   त स॒त्यम॑ब्रवीत्।
   प्रजा॑पते स॒त्येन॒ वै श्राम्यसि।
   अ॒हमु॒ वै स॒त्यम॑स्मि।
   मां नु य॑जस्व ।
   अथ॑ ते स॒त्य स॒त्यं भ॑विष्यति।
   अनु॑ स्व॒र्गं लो॒कं वे॒त्स्यसीति॑।
   स ए॒तमाग्ने॒यम॒ष्टाक॑पालं॒ निर॑वपत्।
   स॒त्याय॑ च॒रुम्।
   अनु॑मत्यै च॒रुम्।
   ततो॒ वै तस्य॑ स॒त्य स॒त्यम॑भवत्।
   अनु॑ स्व॒र्गं लो॒कम॑विन्दत्।
   सत्य ह॒ वा अ॑स्य स॒त्यं भ॑वति।
   अनु॑ स्व॒र्गं लो॒कं वि॑न्दति।
   य ए॒तेन॑ ह॒विषा॒ यज॑ते।
   य उ॑ चैनदे॒वं वेद॑।
   सोऽत्र॑ जुहोति।
   अ॒ग्नये॒ स्वाहा॑ स॒त्याय॒ स्वाहा।
   अनु॑मत्यै॒ स्वाहा प्र॒जाप॑तये॒ स्वाहा।
   स्व॒र्गाय॑ लो॒काय॒ स्वाहा॒ऽग्नये स्विष्ट॒कृते॒ स्वाहेति॑॥१८॥

   तं मनोऽब्रवीत्।
   प्रजा॑पते॒ मन॑सा॒ वै श्राम्यसि।
   अ॒हमु॒ वै मनोऽस्मि।
   मां नु य॑जस्व।
   अथ॑ ते स॒त्यं मनो॑ भविष्यति।
   अनु॑ स्व॒र्गं लो॒कं वे॒त्स्यसीति॑।
   स ए॒तमाग्ने॒यम॒ष्टाक॑पालं॒ निर॑वपत्।
   मन॑से च॒रुम्।
   अनु॑मत्यै च॒रुम्।
   ततो॒ वै तस्य॑ स॒त्यं मनो॑ऽभवत्।
   अनु॑ स्व॒र्गं लो॒कम॑विन्दत्।
   स॒त्य ह॒ वा अ॑स्य॒ मनो॑ भवति।
   अनु॑ स्व॒र्गं लो॒कं वि॑न्दति।
   य ए॒तेन॑ ह॒विषा॒ यज॑ते।
   य उ॑ चैनदे॒वं वेद॑।
   सोऽत्र॑ जुहोति।
   अ॒ग्नये॒ स्वाहा॒ मन॑से॒ स्वाहा।
   अनु॑मत्यै॒ स्वाहा प्र॒जाप॑तये॒ स्वाहा।
   स्व॒र्गाय॑ लो॒काय॒ स्वाहा॒ऽग्नये स्विष्ट॒कृते॒ स्वाहाति॑॥१९॥

   तं चर॑णमब्रवीत्।
   प्रजा॑पते॒ चर॑णेन॒ वै श्राम्यसि।
   अ॒हमु॒ वै चर॑णमस्मि।
   मां नु य॑जस्व।
   अथ॑ ते स॒त्यं चर॑णं भविष्यति।
   अनु॑ स्व॒र्गं लो॒कं वे॒त्स्यसीति॑।
   स ए॒तमाग्ने॒यम॒ष्टाक॑पालं॒ निर॑वपत्।
   चर॑णाय च॒रुम्।
   अनु॑मत्यै च॒रुम्।
   ततो॒ वै तस्य॑ स॒त्यं चर॑णमभवत्।
   अनु॑ स्व॒र्गं लो॒कम॑विन्दत्।
   स॒त्य ह॒ वा अ॑स्य॒ चर॑णं भवति।
   अनु॑ स्व॒र्गं लो॒कं वि॑न्दति।
   य ए॒तेन॑ ह॒विषा॒ यज॑ते।
   य उ॑ चैनदे॒वं वेद॑।
   सोऽत्र॑ जुहोति।
   अ॒ग्नये॒ स्वाहा॒ चर॑णाय॒ स्वाहा।
   अनु॑मत्यै॒ स्वाहा प्र॒जाप॑त॒ये स्वाहा।
   स्व॒र्गाय॑ लो॒काय॒ स्वाहा॒ऽग्नये स्विष्ट॒कृते॒ स्वाहेति॑॥२०॥

   ता वा ए॒ताः पञ्च॑ स्व॒र्गस्य॑ लो॒कस्य॒ द्वार॑।
   अपा॑घा॒ अनु॑वित्तयो॒ नाम॑।
   तप॑ प्रथ॒मा र॑क्षति।
   श्र॒द्धा द्वि॒तीयाम्।
   स॒त्यं तृ॒तीयाम्।
   मन॑श्चतु॒र्थीम्।
   चर॑णं पञ्च॒मीम्।
   अनु॑ ह॒ वै स्व॒र्गं लो॒कं वि॑न्दति।
   का॒म॒चारोऽस्य स्व॒र्गे लो॒के भ॑वति।
   य ए॒ताभि॒रिष्टि॑भि॒र्यज॑ते।
   य उ॑ चैना ए॒वं वेद॑।
   तास्व॑न्वि॒ष्टि।
   प॒ष्ठौ॒ही॒व॒रां द॑द्यात्क॒सं च॑।
   स्त्रियै॑ चाऽऽभा॒र समृ॑द्ध्यै॥२१॥
\anuvakamend
  
   ब्रह्म॒ वै चतु॑र्‌होतारः।
   चतु॑र्‌होतृ॒भ्योऽधि॑य॒ज्ञो निर्मि॑तः।
   नैन श॒प्तम्।
   नाभिच॑रित॒माग॑च्छति।
   य ए॒वं वेद॑।
   यो ह॒ वै चतु॑र्‌होतृणां चतुर्‌होतृ॒त्वं वेद॑।
   अथो॒ पञ्च॑होतृत्वम्।
   सर्वा॑ हास्मै॒ दिश॑ कल्पन्ते।
   वा॒चस्पति॒र्‌होता॒ दश॑होतॄणाम्।
   पृ॒थि॒वी होता॒ चतु॑र्‌होतॄणाम्॥२२॥

   अ॒ग्निर्‌होता॒ पञ्च॑होतॄणाम्।
   वाग्घोता॒ षड्ढो॑तॄणाम्।
   म॒हाह॑वि॒र्‌होता॑ स॒प्तहो॑तॄणाम्।
   ए॒तद्वै चतु॑र्‌होतृणां चतुर्‌होतृ॒त्वम्।
   अथो॒ पञ्च॑होतृत्वम्।
   सर्वा॑ हास्मै॒ दिश॑ कल्पन्ते।
   य ए॒वं वेद॑।
   ए॒षा वै स॑र्ववि॒द्या।
   ए॒तद्भे॑ष॒जम्।
   ए॒षा प॒ङ्क्तिः स्व॒र्गस्य॑ लो॒कस्याञ्ज॒साऽय॑निः स्रु॒तिः॥२३॥

   ए॒तान् योऽध्यैत्यछ॑दिर्द॒र्‌शे याव॑त्त॒रसम्।
   स्व॑रेति।
   अ॒न॒प॒ब्र॒वः सर्व॒मायु॑रेति।
   वि॒न्दते प्र॒जाम्।
   रा॒यस्पोषं॑ गौप॒त्यम्।
   ब्र॒ह्म॒व॒र्च॒सी भ॑वति।
   ए॒तान् योऽध्यैति॑।
   स्पृ॒णोत्या॒त्मानम्।
   प्र॒जां पि॒तॄन्।
   ए॒तान् वा अ॑रु॒ण औ॑पवे॒शिर्वि॒दाञ्च॑कार॥२४॥

   ए॒तैर॑धिवा॒दमपा॑जयत्।
   अथो॒ विश्वं॑ पा॒प्मानम्।
   स्व॑र्ययौ।
   ए॒तान्योऽध्यैति॑।
   अ॒धि॒वा॒दं ज॑यति।
   अथो॒ विश्वं॑ पा॒प्मानम्।
   स्व॑रेति।
   ए॒तैर॒ग्निं चि॑न्वीत स्व॒र्गका॑मः।
   ए॒तैरायु॑ष्कामः।
   प्र॒जाप॒शुका॑मो वा॥२५॥

   पु॒रस्ता॒द्दश॑होतार॒मुद॑ञ्च॒मुप॑दधाति यावत्प॒दम्।
   हृद॑यं॒ यजु॑षी॒ पत्न्यौ॑ च।
   द॒क्षि॒ण॒तः प्राञ्चं॒ चतु॑र्‌होतारम्।
   प॒श्चादुद॑ञ्चं॒ पञ्च॑होतारम्।
   उ॒त्त॒र॒तः प्राञ्च॒ षड्ढो॑तारम्।
   उ॒परिष्टा॒त्प्राञ्च स॒प्तहो॑तारम्।
   हृद॑यं॒ यजूषि॒ पत्न्य॑श्च ।
   य॒था॒व॒का॒शं ग्रहा\sn{}।
   य॒था॒व॒का॒शं प्र॑तिग्र॒हाँल्लो॑कम्पृ॒णाश्च॑।
   सर्वा॑ हास्यै॒ता दे॒वता प्री॒ता अ॒भीष्टा॑ भवन्ति॥२६॥

   सदे॑वम॒ग्निं चि॑नुते।
   र॒थस॑म्मितश्चेत॒व्य॑।
   वज्रो॒ वै रथ॑।
   वज्रे॑णै॒व पा॒प्मानं॒ भ्रातृ॑व्य स्तृणुते।
   प॒क्षः स॑म्मितश्चेत॒व्य॑।
   ए॒तावा॒न् वै रथ॑।
   याव॑त्प॒क्षः।
   र॒थस॑म्मितमे॒व चि॑नुते।
   इ॒ममे॒व लो॒कं प॑शुब॒न्धेना॒भिज॑यति।
   अथो॑ अग्निष्टो॒मेन॑॥२७॥

   अ॒न्तरि॑क्षमु॒क्थ्ये॑न।
   स्व॑रतिरा॒त्रेण॑।
   सर्वाँल्लो॒कान॑ही॒नेन॑।
   अथो॑ स॒त्रेण॑।
   वरो॒ दक्षि॑णा।
   वरे॑णै॒व वर स्पृणोति।
   आ॒त्मा हि वर॑।
   एक॑विशति॒र्दक्षि॑णा ददाति।
   ए॒क॒वि॒शो वा इ॒तः स्व॒र्गो लो॒कः।
   प्र स्व॒र्गं लो॒कमाप्नोति॥२८॥

   अ॒सावा॑दि॒त्य ए॑कवि॒शः।
   अ॒मुमे॒वाऽऽदि॒त्यमाप्नोति।
   श॒तं ददा॑ति।
   श॒तायु॒ पुरु॑षः श॒तेन्द्रि॑यः।
   आयु॑ष्ये॒वेन्द्रि॒ये प्रति॑तिष्ठति।
   स॒हस्रं॑ ददाति।
   स॒हस्र॑सम्मितः स्व॒र्गो लो॒कः।
   स्व॒र्गस्य॑ लो॒कस्या॒भिजि॑त्यै।
   अ॒न्वि॒ष्ट॒कं दक्षि॑णा ददाति।
   सर्वा॑णि॒ वयासि॥२९॥

   सर्व॒स्याऽऽप्त्यै।
   सर्व॒स्याव॑रुद्ध्यै।
   यदि॒ न वि॒न्देत॑।
   म॒न्थाने॑ताव॒तो द॑द्यादोद॒नान् वा।
   अ॒श्नु॒ते तं कामम्।
   यस्मै॒ कामा॑य॒\aav{}ग्निश्ची॒यते।
   प॒ष्ठौ॒हीं त्व॒न्तर्व॑तीं दद्यात्।
   सा हि सर्वा॑णि॒ वयासि।
   सर्व॒स्याऽऽप्त्यै।
   सर्व॒स्याव॑रुद्ध्यै॥३०॥

   हिर॑ण्यं ददाति।
   हिर॑ण्यज्योतिरे॒व स्व॒र्गं लो॒कमे॑ति।
   वासो॑ ददाति।
   तेनऽऽयु॒ प्रति॑रते।
   वे॒दि॒तृ॒ती॒ये य॑जेत।
   त्रिष॑त्या॒ हि दे॒वाः।
   स स॑त्यम॒ग्निं चि॑नुते।
   तदे॒तत्प॑शुब॒न्धे ब्राह्म॑णं ब्रूयात्।
   नेत॑रेषु य॒ज्ञेषु॑।
   यो ह॒ वै चतु॑र्‌होतॄननुसव॒नं त॑र्पयित॒व्या\sn{} वेद॑॥३१॥

   तृप्य॑ति प्र॒जया॑ प॒शुभि॑।
   उपै॑न सोमपी॒थो न॑मति।
   ए॒ते वै चतु॑र्‌होतारोऽनुसव॒नं त॑र्पयित॒व्या।
   ये ब्राह्म॒णा ब॑हु॒विद॑।
   तेभ्यो॒ यद्दक्षि॑णा॒ न नयेत्।
   दुरि॑ष्ट स्यात्।
   अ॒ग्निम॑स्य वृञ्जीरन्।
   तेभ्यो॑ यथाश्र॒द्धं द॑द्यात्।
   स्वि॑ष्टमे॒वैतत्क्रि॑यते।
   नास्या॒ग्निं वृ॑ञ्जते॥३२॥

   हि॒र॒ण्ये॒ष्टको भ॑वति।
   याव॑दुत्त॒मम॑ङ्गुलिका॒ण्डं य॑ज्ञप॒रुषा॒ सम्मि॑तम्।
   तेजो॒ हिर॑ण्यम्।
   यदि॒ हिर॑ण्यं॒ न वि॒न्देत्।
   शर्क॑रा अ॒क्ता उप॑दध्यात्।
   तेजो॑ घृ॒तम्।
   सते॑जसमे॒वाग्निं चि॑नुते।
   अ॒ग्निं चि॒त्वा सौत्राम॒ण्या य॑जेत मैत्रावरु॒ण्या वा।
   वी॒र्ये॑ण॒ वा ए॒ष व्यृ॑ध्यते।
   योऽग्निं चि॑नु॒ते॥३३॥
   याव॑दे॒व वी॒र्यम्।
   तद॑स्मिन्दधाति।
   ब्रह्म॑ण॒ सायु॑ज्य सलो॒कता॑माप्नोति।
   ए॒तासा॑मे॒व दे॒वता॑ना॒ सायु॑ज्यम्।
   सा॒र्ष्टिता समानलो॒कता॑माप्नोति।
   य ए॒तम॒ग्निं चि॑नु॒ते।
   य उ॑ चैनमे॒वं वेद॑।
   ए॒तदे॒व सा॑वि॒त्रे ब्राह्म॑णम्।
   अथो॑ नाचिके॒ते॥३४॥
   \anuvakamend
  
   यच्चा॒मृतं॒ यच्च॒ मर्त्यम्।
   यच्च॒ प्राणि॑ति॒ यच्च॒ न।
   सर्वा॒स्ता इष्ट॑काः कृ॒त्वा।
   उप॑ काम॒दुघा॑ दधे।
   तेनर्‌षि॑णा॒ तेन॒ ब्रह्म॑णा।
   तया॑ दे॒वत॑याऽङ्गिर॒स्वद्ध्रु॒वा सी॑द।
   सर्वा॒ स्त्रिय॒ सर्वान्पु॒सः।
   सर्वं॒ न स्त्री॑पुमं च॒ यत्।
   सर्वा॒स्ताः।
   याव॑न्तः पा॒सवो॒ भूमे॥३५॥

   सङ्ख्या॑ता देवमा॒यया।
   सर्वा॒स्ताः।
   याव॑न्त॒ ऊषा पशू॒नाम्।
   पृ॒थि॒व्यां पुष्टि॑र्‌हि॒ताः।
   सर्वा॒स्ताः।
   याव॑ती॒ सिक॑ता॒ सर्वा।
   अ॒प्स्व॑न्तश्च॒ याः श्रि॒ताः।
   सर्वा॒स्ताः।
   याव॑ती॒ शर्क॑रा॒ धृत्यै।
   अ॒स्यां पृ॑थि॒व्यामधि॑॥३६॥	

   सर्वा॒स्ताः।
   याव॒न्तोऽश्मा॑नो॒ऽस्यां पृ॑थि॒व्याम्।
   प्र॒ति॒ष्ठासु॒ प्रति॑ष्ठिताः।
   सर्वा॒स्ताः।
   याव॑तीर्वी॒रुध॒ सर्वा।
   विष्ठि॑ताः पृथि॒वीमनु॑।
   सर्वा॒स्ताः।
   याव॑ती॒रोष॑धी॒ सर्वा।
   विष्ठि॑ताः पृथि॒वीमनु॑।
   सर्वा॒स्ताः॥३७॥

   याव॑न्तो॒ वन॒स्पत॑यः।
   अ॒स्यां पृ॑थि॒व्यामधि॑।
   सर्वा॒स्ताः।
   याव॑न्तो ग्रा॒म्याः प॒शव॒ सर्वे।
   आ॒र॒ण्याश्च॒ ये।
   सर्वा॒स्ताः।
   ये द्वि॒पाद॒श्चतु॑ष्पादः।
   अ॒पाद॑ उदरस॒र्पिण॑।
   सर्वा॒स्ताः।
   याव॒दाञ्ज॑नमु॒च्यते॥३८॥

   दे॒व॒त्रा यच्च॑ मानु॒षम्।
   सर्वा॒स्ताः॥
   याव॑त्कृ॒ष्णाय॑स॒ सर्वम्।
   दे॒व॒त्रा यच्च॑ मानु॒षम्।
   सर्वा॒स्ताः।
   याव॑ल्लो॒हाय॑स॒ सर्वम्।
   दे॒व॒त्रा यच्च॑ मानु॒षम्।
   सर्वा॒स्ताः।
   सर्व॒ सीस॒ सर्वं॒ त्रपु॑।
   दे॒व॒त्रा यच्च॑ मानु॒षम्॥३९॥

   सर्वा॒स्ताः।
   सर्व॒ हिर॑ण्य रज॒तम्।
   दे॒व॒त्रा यच्च॑ मानु॒षम्।
   सर्वा॒स्ताः।
   सर्व॒ सुव॑र्ण॒ हरि॑तम्।
   दे॒व॒त्रा यच्च॑ मानु॒षम्।
   सर्वा॒स्ता इष्ट॑काः कृ॒त्वा।
   उप॑ काम॒दुघा॑ दधे।
   तेनर्‌षि॑णा॒ तेन॒ ब्रह्म॑णा।
   तया॑ दे॒वत॑याऽङ्गिर॒स्वद्ध्रु॒वा सी॑द॥४०॥
   \anuvakamend
  
   सर्वा॒ दिशो॑ दि॒क्षु।
   यच्चा॒न्तर्भू॒तं प्रति॑ष्ठितम्।
   सर्वा॒स्ता इष्ट॑काः कृ॒त्वा।
   उप॑ काम॒दुघा॑ दधे।
   तेनर्‌षि॑णा॒ तेन॒ ब्रह्म॑णा।
   तया॑ दे॒वत॑याऽङ्गिर॒स्वद्ध्रु॒वा सी॑द।
   अ॒न्तरि॑क्षं च॒ केव॑लम्।
   यच्चा॒स्मिन्न॑न्त॒राहि॑तम्।
   सर्वा॒स्ताः।
   आ॒न्त॒रि॒क्ष्य॑श्च॒ याः प्र॒जाः॥४१॥

   ग॒न्ध॒र्वा॒प्स॒रस॑श्च॒ ये।
   सर्वा॒स्ताः।
   सर्वा॑नुदा॒रान्त्स॒लिला॑न्।
   अ॒न्तरि॑क्षे॒ प्रति॑ष्ठितान्।
   सर्वा॒स्ताः।
   सर्वा॑नुदा॒रान्त्स॑लि॒लान्।
   स्था॒व॒राः प्रो॒ष्याश्च॒ ये।
   सर्वा॒स्ताः।
   सर्वां॒ धुनि॒ सर्वान्ध्व॒सान्।
   हि॒मो यच्च॑ शी॒यते॥४२॥

   सर्वा॒स्ताः।
   सर्वा॒न्मरी॑ची॒\an{} वित॑तान्।
   नी॒हा॒रो यच्च॑ शी॒यते।
   सर्वा॒स्ताः।
   सर्वा॑ वि॒द्युत॒ सर्वान्त्स्तनयि॒त्नून्।
   ह्रा॒दुनी॒र्यच्च॑ शी॒यते।
   सर्वा॒स्ताः।
   सर्वा॒ स्रव॑न्तीः स॒रित॑।
   सर्व॑मप्सुच॒रं च॒ यत्।
   सर्वा॒स्ताः॥४३॥

   याश्च॒ कूप्या॒ याश्च॑ ना॒द्या समु॒द्रिया।
   याश्च॑ वैश॒न्तीरु॒त प्रा॑स॒चीर्याः।
   सर्वा॒स्ताः।
   ये चो॒त्तिष्ठ॑न्ति जी॒मूता।
   याश्च॒ वर्‌ष॑न्ति वृ॒ष्टय॑।
   सर्वा॒स्ताः।
   तप॒स्तेज॑ आका॒शं।
   यच्चा॑ऽऽका॒शे प्रति॑ष्ठितम्।
   सर्वा॒स्ताः।
   वा॒युं वयासि॒ सर्वा॑णि॥४४॥

   अ॒न्त॒रि॒क्ष॒च॒रं च॒ यत्।
   सर्वा॒स्ताः।
   अ॒ग्नि सूर्यं॑ च॒न्द्रम्।
   मि॒त्रं वरु॑णं॒ भगम्।
   सर्वा॒स्ताः।
   स॒त्य श्र॒द्धां तपो॒ दमम्।
   नाम॑ रू॒पं च॑ भू॒तानाम्।
   सर्वा॒स्ता इष्ट॑काः कृ॒त्वा।
   उप॑ काम॒दुघा॑ दधे।
   तेनर्‌षि॑णा॒ तेन॒ ब्रह्म॑णा।
   तया॑ दे॒वत॑याऽङ्गिर॒स्वद्ध्रु॒वा सी॑द॥४५॥
\anuvakamend
  
   सर्वा॒न्दिव॒ सर्वान्दे॒वान्दि॒वि।
   यच्चा॒न्तर्भू॒तं प्रति॑ष्ठितम्।
   सर्वा॒स्ता इष्ट॑काः कृ॒त्वा।
   उप॑ काम॒दुघा॑ दधे।
   तेनर्‌षि॑णा तेन॒ ब्रह्म॑णा।
   तया॑ दे॒वत॑याऽङ्गिर॒स्वद्ध्रु॒वा सी॑द।
   याव॑ती॒स्तार॑का॒ सर्वा।
   वित॑ता रोच॒ने दि॒वि।
   सर्वा॒स्ताः।
   ऋचो॒ यजूषि॒ सामा॑नि॥४६॥

   अ॒थ॒र्वा॒ङ्गि॒रस॑श्च॒ ये।
   सर्वा॒स्ताः।
   इ॒ति॒हा॒स॒पु॒रा॒णं च॑।
   स॒र्प॒दे॒व॒ज॒नाश्च॒ ये।
   सर्वा॒स्ताः।
   ये च लो॒का ये चा॑लो॒काः।
   अ॒न्तर्भू॒तं प्रति॑ष्ठितम्।
   सर्वा॒स्ताः।
   यच्च॒ ब्रह्म॒ यच्चाब्र॒ह्म।
   अ॒न्तर्ब्र॒ह्मन्प्रति॑ष्ठितम्॥४७॥

   सर्वा॒स्ताः।
   अ॒हो॒रा॒त्राणि॒ सर्वा॑णि।
   अ॒र्ध॒मा॒साश्च॒ केव॑लान्।
   सर्वा॒स्ताः।
   सर्वा॑नृ॒तून्त्सर्वान्मा॒सान्।
   सं॒व॒त्स॒रं च॒ केव॑लम्।
   सर्वा॒स्ताः।
   सर्वं॒ भूत॒ सर्वं॒ भव्यम्।
   यच्चा॒तोऽधि॑भवि॒ष्यति॑।
   सर्वा॒स्ता इष्ट॑काः कृ॒त्वा।
   उप॑ काम॒दुघा॑ दधे।
   तेनर्‌षि॑णा॒ तेन॒ ब्रह्म॑णा।
   तया॑ दे॒वत॑याऽङ्गिर॒स्वद्ध्रु॒वा सी॑द॥४८॥
\anuvakamend
  
   ऋ॒चां प्राची॑ मह॒ती दिगु॑च्यते।
   दक्षि॑णामाहु॒र्यजु॑षामपा॒राम्।
   अथ॑र्वणा॒मङ्गि॑रसां प्र॒तीची।
   साम्ना॒मुदी॑ची मह॒ती दिगु॑च्यते।
   ऋ॒ग्भिः पूर्वा॒ह्णे दि॒वि दे॒व ई॑यते।
   य॒जु॒र्वे॒दे तिष्ठ॒ति मध्ये॒ अह्न॑।
   सा॒म॒वे॒देनाऽस्तम॒ये मही॑यते ।
   वेदै॒रशून्यस्त्रि॒भिरे॑ति॒ सूर्य॑।
   ऋ॒ग्भ्यो जा॒ता स॑र्व॒शो मूर्ति॑माहुः।
   सर्वा॒ गति॑र्याजु॒षी है॒व शश्व॑त्॥४९॥

   सर्वं॒ तेज॑ सामरू॒प्य ह॑ शश्वत्।
   सर्व हे॒दं ब्रह्म॑णा है॒व सृ॒ष्टम्।
   ऋ॒ग्भ्यो जा॒तं वैश्यं॒ वर्ण॑माहुः।
   य॒जु॒र्वे॒दं क्ष॑त्रि॒यस्या॑ऽऽहु॒र्योनिम्।
   सा॒म॒वे॒दो ब्राह्म॒णानां॒ प्रसू॑तिः।
   पूर्वे॒ पूर्वेभ्यो॒ वच॑ ए॒तदू॑चुः।
   आ॒द॒र्शम॒ग्निं चि॑न्वा॒नाः।
   पूर्वे॑ विश्व॒सृजो॒ऽमृता।
   श॒तं व॑र्‌षसह॒स्राणि॑।
   दी॒क्षि॒ताः स॒त्त्रमा॑सत॥५०॥

   तप॑ आसीद्गृ॒हप॑तिः।
   ब्रह्म॑ ब्र॒ह्माऽभ॑वत्स्व॒यम्।
   स॒त्य ह॒ होतै॑षा॒मासीत्।
   यद्वि॑श्व॒सृज॒ आस॑त।
   अ॒मृत॑मेभ्य॒ उद॑गायत्।
   स॒हस्रं॑ परिवत्स॒रान्।
   भू॒त ह॑ प्रस्तो॒तैषा॒मासीत्।
   भ॒वि॒ष्यत्प्रति॑ चाहरत्।
   प्रा॒णो अ॑ध्व॒र्युर॑भवत्।
   इ॒द सर्व॒ सिषा॑सताम्॥५१॥

   अ॒पा॒नो वि॒द्वाना॒वृत॑।
   प्रति॒प्राति॑ष्ठदध्व॒रे।
   आ॒र्त॒वा उ॑पगा॒तार॑।
   स॒द॒स्या॑ ऋ॒तवो॑ऽभवन्।
   अ॒र्ध॒मा॒साश्च॒ मासाश्च।
   च॒म॒सा॒ध्व॒र्य॒वोऽभ॑वन्।
   अ॒शस॒द्ब्रह्म॑ण॒स्तेज॑।
   अ॒च्छा॒वा॒कोऽभ॑व॒द्यश॑।
   ऋ॒तमे॑षां प्रशा॒स्ताऽऽसीत्।
   यद्वि॑श्व॒सृज॒ आस॑त॥५२॥

   ऊर्ग्राजा॑न॒मुद॑वहत्।
   ध्रु॒व॒गो॒पः सहो॑ऽभवत्।
   ओजो॒ऽभ्य॑ष्टौ॒द्ग्राव्ण्ण॑।
   यद्वि॑श्व॒सृज॒ आस॑त।
   अप॑चितिः पो॒त्रीया॑मयजत्।
   ने॒ष्ट्रीया॑मयज॒त्त्विषि॑।
   आग्नी॑द्ध्राद्वि॒दुषी॑ स॒त्यम्।
   श्र॒द्धा है॒वाय॑जत्स्व॒यम्।
   इरा॒ पत्नी॑ विश्व॒सृजाम्।
   आकू॑तिरपिनड्ढ॒विः॥५३॥

   इ॒ध्म ह॒ क्षुच्चैभ्य उ॒ग्रे।
   तृ॒ष्णा चाऽऽव॑हतामु॒भे।
   वागे॑षा सुब्रह्म॒ण्याऽऽसीत्।
   छ॒न्दो॒यो॒गान् वि॑जान॒ती।
   क॒ल्प॒त॒न्त्राणि॑ तन्वा॒नाऽह॑।
   स॒स्थाश्च॑ सर्व॒शः ।
   अ॒हो॒रा॒त्रे प॑शुपा॒ल्यौ।
   मु॒हू॒र्ताः प्रेष्या॑ अभवन्।
   मृ॒त्युस्तद॑भवद्धा॒ता।
   श॒मि॒तोग्रो वि॒शां पति॑॥५४॥

   वि॒श्व॒सृज॑ प्रथ॒माः स॒त्रमा॑सत।
   स॒हस्र॑समं॒ प्रसु॑तेन॒ यन्त॑।
   ततो॑ ह जज्ञे॒ भुव॑नस्य गो॒पाः।
   हि॒र॒ण्मय॑ श॒कुनि॒र्ब्रह्म॒ नाम॑।
   येन॒ सूर्य॒स्तप॑ति॒ तेज॑से॒द्धः।
   पि॒ता पु॒त्रेण॑ पितृ॒मान् योनि॑योनौ।
   नावे॑दविन्मनुते॒ तं बृ॒हन्तम्।
   स॒र्वा॒नु॒भुमा॒त्मान सम्परा॒ये।
   ए॒ष नि॒त्यो म॑हि॒मा ब्राह्म॒णस्य॑।
   न कर्म॑णा वर्धते॒ नो कनी॑यान्॥५५॥

   तस्यै॒वाऽऽत्मा प॑द॒वित्तं वि॑दित्वा।
   न कर्म॑णा लिप्यते॒ पाप॑केन।
   पञ्च॑पञ्चा॒शत॑स्त्रि॒वृत॑ संवत्स॒राः।
   पञ्च॑पञ्चा॒शत॑ पञ्चद॒शाः।
   पञ्च॑पञ्चा॒शत॑ सप्तद॒शाः।
   पञ्च॑पञ्चा॒शत॑ एकवि॒शाः।
   वि॒श्व॒सृजा स॒हस्र॑संवत्सरम्।
   ए॒तेन॒ वै वि॑श्व॒सृज॑ इ॒दं विश्व॑मसृजन्त।
   यद्विश्व॒मसृ॑जन्त।
   तस्माद्विश्व॒सृज॑।
   विश्व॑मेना॒ननु॒ प्रजा॑यते।
   ब्रह्म॑ण॒ सायु॑ज्य सलो॒कतां यन्ति।
   ए॒तासा॑मे॒व दे॒वता॑ना॒ सायु॑ज्यम्।
   सा॒र्ष्टिता समानलो॒कतां यन्ति।
   य ए॒तदु॑प॒यन्ति॑।
   ये चै॑न॒त्प्राहु॑।
   येभ्य॑श्चैन॒त्प्राहु॑॥५६॥
   \anuvakamend

      \begin{center}
  ॥ॐ॥\\
॥इति कृष्णयजुर्वेदीयतैत्तिरीयकाठके तृतीयः प्रश्नः समाप्तः॥३॥
॥इति कृष्णयजुर्वेदीयतैत्तिरीयकाठकं समाप्तम्॥\\
   हरि॑  ॐ॥
\end{center}
