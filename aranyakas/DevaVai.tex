% !TeX program = XeLaTeX
% !TeX root = ../AraNyakabook-kindle.tex
\sect{पञ्चमः प्रश्नः}\setcounter{anuvakam}{0}

ॐ शं न॒स्तन्नो॒ मा हा॑सीत्॥ ॐ शान्तिः॒ शान्तिः॒ शान्तिः॑॥

दे॒वा वै स॒त्रमा॑सत। 
ऋद्धि॑परिमितं॒ यश॑स्कामाः। 
ते᳚ऽब्रुवन्। 
यन्नः॑ प्रथ॒मं यश॑ ऋ॒च्छात्। 
सर्वे॑षां न॒स्तत्स॒हास॒दिति॑। 
तेषां᳚ कुरुक्षे॒त्रं वेदि॑रासीत्। 
तस्यै॑ खाण्ड॒वो द॑क्षिणा॒र्द्ध आ॑सीत्। 
तूर्घ्न॑मुत्तरा॒र्द्धः। 
प॒री॒णज्ज॑घना॒र्द्धः। 
म॒रव॑ उत्क॒रः॥१॥

%८.१.२
तेषां᳚ म॒खं वै᳚ष्ण॒वं यश॑ आर्च्छत्। 
तन्न्य॑कामयत। 
तेनापा᳚क्रामत्। 
तं दे॒वा अन्वा॑यन्। 
यशो॑ऽव॒रुरु॑त्समानाः। 
तस्या॒न्वाग॑तस्य। 
स॒व्याद्धनु॒रजा॑यत। 
दक्षि॑णा॒दिष॑वः। 
तस्मा॑दिषुध॒न्वं पुण्य॑जन्म। 
य॒ज्ञज॑न्मा॒ हि॥२॥

%८.१.३
तमेक॒ꣳ॒ सन्तम्᳚। 
ब॒हवो॒ नाभ्य॑धृष्णुवन्। 
तस्मा॒देक॑मिषुध॒न्वि\-नम्᳚। 
ब॒हवो॑ऽनिषुध॒न्वा नाभिधृ॑ष्णुवन्ति। 
सो᳚ऽस्मयत। 
एकं॑ मा॒ सन्तं॑ ब॒हवो॒ नाभ्य॑धर्\mbox{}षिषु॒रिति॑। 
तस्य॑ सिष्मिया॒णस्य॒ तेजोऽपा᳚क्रामत्। 
तद्दे॒वा ओष॑धीषु॒ न्य॑मृजुः। 
ते श्या॒माका॑ अभवन्। 
स्म॒याका॒ वै नामै॒ते॥३॥

%८.१.४
तत्स्म॒याका॑नाꣴ स्मयाक॒त्वम्। 
तस्मा᳚द्दीक्षि॒तेना॑पि॒गृह्य॑ स्मेत॒व्यम्᳚। 
तेज॑सो॒ धृत्यै᳚। 
स धनुः॑ प्रति॒ष्कभ्या॑तिष्ठत्। 
ता उ॑प॒दीका॑ अब्रुव॒न्वरं॑ वृणामहै। 
अथ॑ व इ॒मꣳ र॑न्धयाम। 
यत्र॒ क्व॑ च॒ खना॑म। 
तद॒पो॑ऽभितृ॑णदा॒मेति॑। 
तस्मा॑दुप॒दीका॒ यत्र॒ क्व॑ च॒ खन॑न्ति। 
तद॒पो॑ऽभितृ॑न्दन्ति॥४॥

%८.१.५
वारे॑वृत॒ꣴ॒ ह्या॑साम्। 
तस्य॒ ज्यामप्या॑दन्। 
तस्य॒ धनु॑र्वि॒प्रव॑माण॒ꣳ॒ शिर॒ उद॑वर्तयत्। 
तद्द्यावा॑पृथि॒वी अनु॒प्राव॑र्तत। 
यत् प्राव॑र्तत। 
तत्प्र॑व॒र्ग्य॑स्य प्रवर्ग्य॒त्वम्। 
यद्घ्राँ(४)इत्यप॑तत्। 
तद्\mbox{}घ॒र्मस्य॑ घर्म॒त्वम्। 
म॒ह॒तो वी॒र्य॑मपप्त॒दिति॑। 
तन्म॑हावी॒रस्य॑ महावीर॒त्वम्॥५॥

%८.१.६
यद॒स्याः स॒मभ॑रन्। 
तत्स॒म्राज्ञः॑ सम्रा॒ट्त्वम्। 
तꣴ स्तृ॒तं दे॒वता᳚स्त्रे॒धा व्य॑गृह्णत। 
अ॒ग्निः प्रा॑तः सव॒नम्। 
इन्द्रो॒ माध्यं॑ दिन॒ꣳ॒ सव॑नम्। 
विश्वे॑दे॒वास्तृ॑तीयसव॒नम्। 
तेनाप॑शीर्ष्णा य॒ज्ञेन॒ यज॑मानाः। 
नाशिषो॒ऽवारु॑न्धत। 
न सु॑व॒र्गं लो॒कम॒भ्य॑जयन्। 
ते दे॒वा अ॒श्विना॑वब्रुवन्॥६॥

%८.१.७
भि॒षजौ॒ वै स्थः॑। 
इ॒दं य॒ज्ञस्य॒ शिरः॒ प्रति॑धत्त॒मिति॑। 
ताव॑ब्रूतां॒ वरं॑ वृणावहै। 
ग्रह॑ ए॒व ना॒वत्रापि॑ गृह्यता॒मिति॑। 
ताभ्या॑मे॒तमा᳚श्वि॒नम॑गृह्णन्। 
तावे॒तद्य॒ज्ञस्य॒ शिरः॒ प्रत्य॑धत्ताम्। 
यत्प्र॑व॒र्ग्यः॑। 
तेन॒ सशी᳚र्ष्णा य॒ज्ञेन॒ यज॑मानाः। 
अवा॒शिषो\-ऽरु॑न्धत। 
अ॒भि सु॑व॒र्गं लो॒कम॑जयन्। 
यत्प्र॑व॒र्ग्यं॑ प्रवृ॒णक्ति॑। 
य॒ज्ञस्यै॒व तच्छिरः॒ प्रति॑दधाति। 
तेन॒ सशी᳚र्ष्णा य॒ज्ञेन॒ यज॑मानः। 
अवा॒शिषो॑ रु॒न्धे। 
अ॒भि सु॑व॒र्गं लो॒कं ज॑यति। 
तस्मा॑दे॒ष आ᳚श्वि॒नप्र॑वया इव। 
यत्प्र॑व॒र्ग्यः॑॥७॥
\anuvakamend[उ॒त्क॒रो ह्ये॑ते तृ॑न्दन्ति महावीर॒त्वम॑ब्रुवन्नजयन्त्स॒प्त च॑]

%८.२.१
सा॒वि॒त्रं जु॑होति॒ प्रसू᳚त्यै। 
च॒तु॒र्गृ॒ही॒तेन॑ जुहोति। 
चतु॑ष्पादः प॒शवः॑। 
प॒शूने॒वाव॑रुन्धे। 
चत॑स्रो॒ दिशः॑। 
दि॒क्ष्वे॑व प्रति॑तिष्ठति। 
छन्दाꣳ॑सि दे॒वेभ्योऽपा᳚क्रामन्। 
न वो॑ऽभा॒गानि॑ ह॒व्यं व॑क्ष्याम॒ इति॑। 
तेभ्य॑ ए॒तच्च॑तुर्गृही॒तम॑धारयन्। 
पु॒रो॒नु॒वा॒क्या॑यै या॒ज्या॑यै॥८॥

%८.२.२
दे॒वता॑यै वषट्का॒राय॑। 
यच्च॑तुर्गृही॒तं जु॒होति॑। 
छन्दाꣴ॑स्ये॒व तत् प्री॑णाति। 
तान्य॑स्य प्री॒तानि॑ दे॒वेभ्यो॑ ह॒व्यं व॑हन्ति। 
ब्र॒ह्म॒वा॒दिनो॑ वदन्ति। 
हो॒त॒व्यं॑ दीक्षि॒तस्य॑ गृ॒हा(३)इ न हो॑त॒व्या(३)मिति॑। 
ह॒विर्\mbox{}वै दी᳚क्षि॒तः। 
यज्जु॑हु॒यात्। 
ह॒विष्कृ॑तं॒ यज॑मानम॒ग्नौ प्रद॑ध्यात्। 
यन्न जु॑हु॒यात्॥९॥

%८.२.३
य॒ज्ञ॒प॒रुर॒न्तरि॑यात्। 
यजु॑रे॒व व॑देत्। 
न ह॒विष्कृ॑तं॒ यज॑मानम॒ग्नौ प्र॒दधा॑ति। 
न य॑ज्ञप॒रुर॒न्तरे॑ति। 
गा॒य॒त्री छन्दा॒ꣴ॒स्यत्य॑मन्यत। 
तस्यै॑ वषट्का॒रो᳚ऽभ्यय्य॒ शिरो᳚ऽच्छिनत्। 
तस्यै᳚ द्वे॒धा रसः॒ परा॑पतत्। 
पृ॒थि॒वीम॒र्द्धः प्रावि॑शत्। 
प॒शून॒र्द्धः। 
यः पृ॑थि॒वीं प्रावि॑शत्॥१०॥

%८.२.४
स ख॑दि॒रो॑ऽभवत्। 
यः प॒शून्। 
सो॑ऽजाम्। 
यत्खा॑दि॒र्यभ्रि॒र्भ\-व॑ति। 
छन्द॑सामे॒व रसे॑न य॒ज्ञस्य॒ शिरः॒ सम्भ॑रति। 
यदौदु॑म्बरी। 
ऊर्ग्वा उ॑दु॒म्बरः॑। 
ऊ॒र्जैव य॒ज्ञस्य॒ शिरः॒ सम्भ॑रति। 
यद्वै॑ण॒वी। 
तेजो॒ वै वेणुः॑॥११॥

%८.२.५
तेज॑सै॒व य॒ज्ञस्य॒ शिरः॒ सम्भ॑रति। 
यद्वैक॑ङ्कती। 
भा ए॒वाव॑रुन्धे। 
दे॒वस्य॑ त्वा सवि॒तुः प्र॑स॒व इत्यभ्रि॒माद॑त्ते॒ प्रसू᳚त्यै। 
अ॒श्विनो᳚र्बा॒हुभ्या॒\-मित्या॑ह। 
अ॒श्विनौ॒ हि दे॒वाना॑मध्व॒र्यू आस्ता᳚म्। 
पू॒ष्णो हस्ता᳚भ्या॒मित्या॑ह॒ यत्यै᳚। 
वज्र॑ इव॒ वा ए॒षा। 
यदभ्रिः॑। 
अभ्रि॑रसि॒ नारि॑र॒सीत्या॑ह॒ शान्त्यै᳚॥१२॥

%८.२.६
अ॒ध्व॒र॒कृद्दे॒वेभ्य॒ इत्या॑ह। 
य॒ज्ञो वा अ॑ध्व॒रः। 
य॒ज्ञ॒कृद्दे॒वेभ्य॒ इति॒ वावैतदा॑ह। 
उत्ति॑ष्ठ ब्रह्मणस्पत॒ इत्या॑ह। 
ब्रह्म॑णै॒व य॒ज्ञस्य॒ शिरोऽच्छै॑ति। 
प्रैतु॒ ब्रह्म॑ण॒स्पति॒रित्या॑ह। 
प्रेत्यै॒व य॒ज्ञस्य॒ शिरोऽच्छै॑ति। 
प्र दे॒व्ये॑तु सू॒नृतेत्या॑ह। 
य॒ज्ञो वै सू॒नृता᳚। 
अच्छा॑ वी॒रं नर्यं॑ प॒ङ्क्तिरा॑धस॒मित्या॑ह॥१३॥

%८.२.७
पाङ्क्तो॒ हि य॒ज्ञः। 
दे॒वा य॒ज्ञं न॑यन्तु न॒ इत्या॑ह। 
दे॒वाने॒व य॑ज्ञ॒नियः॑ कुरुते। 
देवी᳚ द्यावापृथिवी॒ अनु॑ मे मꣳसाथा॒मित्या॑ह। 
आ॒भ्यामे॒वानु॑मतो य॒ज्ञस्य॒ शिरः॒ सम्भ॑रति। 
ऋ॒द्ध्यास॑म॒द्य म॒खस्य॒ शिर॒ इत्या॑ह। 
य॒ज्ञो वै म॒खः। 
ऋ॒द्ध्यास॑म॒द्य य॒ज्ञस्य॒ शिर॒ इति वावैतदा॑ह। 
म॒खाय॑ त्वा म॒खस्य॑ त्वा शी॒र्ष्ण इत्या॑ह। 
नि॒र्दिश्यै॒वैन॑द्धरति॥१४॥

%८.२.८
त्रिर्\mbox{}ह॑रति। 
त्रय॑ इ॒मे लो॒काः। 
ए॒भ्य ए॒व लो॒केभ्यो॑ य॒ज्ञस्य॒ शिरः॒ सम्भ॑रति। 
तू॒ष्णीं च॑तु॒र्थꣳ ह॑रति। 
अप॑रिमितादे॒व य॒ज्ञस्य॒ शिरः॒ सम्भ॑रति। 
मृ॒त्ख॒नादग्रे॑ हरति। 
तस्मा᳚न्मृत्ख॒नः क॑रु॒ण्य॑तरः। 
इय॒त्यग्र॑ आसी॒रित्या॑ह। 
अ॒स्यामे॒वाछ॑म्बट्कारं य॒ज्ञस्य॒ शिरः॒ सम्भ॑रति। 
ऊर्जं॒ वा ए॒तꣳ रसं॑ पृथि॒व्या उ॑प॒दीका॒ उद्दि॑हन्ति॥१५॥

%८.२.९
यद्व॒ल्मीकम्᳚। 
यद्व॑ल्मीकव॒पा स॑म्भा॒रो भव॑ति। 
ऊर्ज॑मे॒व रसं॑ पृथि॒व्या अव॑रुन्धे। 
अथो॒ श्रोत्र॑मे॒व। 
श्रोत्र॒ꣴ॒ ह्ये॑तत्पृ॑थि॒व्याः। 
यद्व॒ल्मीकः॑। 
अब॑धिरो भवति। 
य ए॒वं वेद॑। 
इन्द्रो॑ वृ॒त्राय॒ वज्र॒मुद॑यच्छत्। 
स यत्र॑ यत्र प॒राक्र॑मत॥१६॥

%८.२.१०
तन्नाद्ध्रि॑यत। 
स पू॑तीकस्त॒म्बे परा᳚क्रमत। 
सो᳚ऽद्ध्रियत। 
सो᳚ऽब्रवीत्। 
ऊ॒तिं वै मे॑ धा॒ इति॑। 
तदू॒तीका॑नामूतीक॒त्वम्। 
यदू॒तीका॒ भव॑न्ति। 
य॒ज्ञायै॒वोतिं द॑धति। 
अ॒ग्नि॒जा अ॑सि प्र॒जाप॑ते॒ रेत॒ इत्या॑ह। 
य ए॒व रसः॑ प॒शून्प्रावि॑शत्॥१७॥

%८.२.११
तमे॒वाव॑रुन्धे। 
पञ्चै॒ते स॑म्भा॒रा भ॑वन्ति। 
पाङ्क्तो॑ य॒ज्ञः। 
यावा॑ने॒व य॒ज्ञः। 
तस्य॒ शिरः॒ सम्भ॑रति। 
यद्ग्रा॒म्याणां᳚ पशू॒नां चर्म॑णा स॒म्भरे᳚त्। 
ग्रा॒म्यान्प॒शूञ्छु॒चाऽर्प॑येत्। 
कृ॒ष्णा॒जि॒नेन॒ सम्भ॑रति। 
आ॒र॒ण्याने॒व प॒शूञ्छु॒चार्प॑यति। 
तस्मा᳚त्स॒माव॑त्पशू॒नां प्र॒जाय॑मानानाम्॥१८॥

%८.२.१२
आ॒र॒ण्याः प॒शवः॒ कनी॑याꣳसः। 
शु॒चा ह्यृ॑ताः। 
लो॒म॒तः सम्भ॑रति। 
अतो॒ ह्य॑स्य॒ मेध्यम्᳚। 
प॒रि॒गृह्या य॑न्ति। 
रक्ष॑सा॒मप॑हत्यै। 
ब॒हवो॑ हरन्ति। 
अप॑चितिमे॒वास्मि॑न्दधति। 
उद्ध॑ते॒ सिक॑तोपोप्ते॒ परि॑श्रिते॒ निद॑धति॒ शान्त्यै᳚। 
मद॑न्तीभि॒रुप॑ सृजति॥१९॥

%८.२.१३
तेज॑ ए॒वास्मि॑न्दधाति। 
मधु॑ त्वा मधु॒ला क॑रो॒त्वित्या॑ह। 
ब्रह्म॑णै॒वास्मि॒न्तेजो॑ दधाति। 
यद्ग्रा॒म्याणां॒ पात्रा॑णां क॒पालैः᳚ सꣳसृ॒जेत्। 
ग्रा॒म्याणि॒ पात्रा॑णि शु॒चाऽर्प॑येत्। 
अ॒र्म॒क॒पा॒लैः सꣳसृ॑जति। 
ए॒तानि॒ वा अ॑नुपजीवनी॒यानि॑। 
तान्ये॒व शु॒चार्प॑यति। 
शर्क॑राभिः॒ सꣳसृ॑जति॒ धृत्यै᳚। 
अथो॑ श॒न्त्वाय॑। 
अ॒ज॒लो॒मैः सꣳसृ॑जति। 
ए॒षा वा अ॒ग्नेः प्रि॒या त॒नूः। 
यद॒जा। 
प्रि॒ययै॒वैनं॑ त॒नुवा॒ सꣳसृ॑जति। 
अथो॒ तेज॑सा। 
कृ॒ष्णा॒जि॒नस्य॒ लोम॑भिः॒ सꣳसृ॑जति। 
य॒ज्ञो वै कृ॑ष्णाजि॒नम्। 
य॒ज्ञेनै॒व य॒ज्ञꣳ सꣳसृ॑जति॥२०॥
\anuvakamend[या॒ज्या॑यै॒ न जु॑हु॒यादवि॑श॒द्वेणुः॒ शान्त्यै॑ प॒ङ्क्तिरा॑धस॒मित्या॑ह हरति दिहन्ति प॒राक्र॑म॒तावि॑शत् प्र॒जाय॑मानानाꣳ सृजति श॒न्त्वाया॒ष्टौ च॑]

%८.३.१
परि॑श्रिते करोति। 
ब्र॒ह्म॒व॒र्च॒सस्य॒ परि॑गृहीत्यै। 
न कु॒र्वन्न॒भि प्रा᳚ण्यात्। 
यत्कु॒र्वन्न॑भि प्रा॒ण्यात्। 
प्रा॒णाञ्छु॒चार्प॑येत्। 
अ॒प॒हाय॒ प्राणि॑ति। 
प्रा॒णानां᳚ गोपी॒थाय॑। 
न प्र॑व॒र्ग्यं॑ चादि॒त्यं चा॒न्तरे॑यात्। 
यद॑न्तरे॒यात्। 
दु॒श्चर्मा᳚ स्यात्॥२१॥

%८.३.२
तस्मा॒न्नान्त॒राय्यम्᳚। 
आ॒त्मनो॑ गोपी॒थाय॑। 
वेणु॑ना करोति। 
तेजो॒ वै वेणुः॑। 
तेजः॑ प्रव॒र्ग्यः॑। 
तेज॑सै॒व तेजः॒ सम॑र्द्धयति। 
म॒खस्य॒ शिरो॒ऽसीत्या॑ह। 
य॒ज्ञो वै म॒खः। 
तस्यै॒तच्छिरः॑। 
यत्प्र॑व॒र्ग्यः॑॥२२॥

%८.३.३
तस्मा॑दे॒वमा॑ह। 
य॒ज्ञस्य॑ प॒दे स्थ॒ इत्या॑ह। 
य॒ज्ञस्य॒ ह्ये॑ते प॒दे। 
अथो॒ प्रति॑ष्ठित्यै। 
गा॒य॒त्रेण॑ त्वा॒ छन्द॑सा करो॒मीत्या॑ह। 
छन्दो॑भिरे॒वैनं॑ करोति। 
त्र्यु॑द्धिं करोति। 
त्रय॑ इ॒मे लो॒काः। 
ए॒षां लो॒काना॒माप्त्यै᳚। 
छन्दो॑भिः करोति॥२३॥

%८.३.४
वी॒र्यं॑ वै छन्दाꣳ॑सि। 
वी॒र्ये॑णै॒वैनं॑ करोति। 
यजु॑षा॒ बिलं॑ करोति॒ व्यावृ॑त्यै। 
इयं॑ तं करोति। 
प्र॒जाप॑तिना यज्ञमु॒खेन॒ सम्मि॑तम्। 
इयं॑ तं करोति। 
य॒ज्ञ॒प॒रुषा॒ सम्मि॑तम्। 
इयं॑ तं करोति। 
ए॒ताव॒द्वै पुरु॑षे वी॒र्यम्᳚। 
वी॒र्य॑सम्मितम्॥२४॥

%८.३.५
अप॑रिमितं करोति। 
अप॑रिमित॒स्याव॑रुद्ध्यै। 
प॒रि॒ग्री॒वं क॑रोति॒ धृत्यै᳚। 
सूर्य॑स्य॒ हर॑सा श्रा॒येत्या॑ह। 
य॒था॒य॒जुरे॒वैतत्। 
अ॒श्व॒श॒केन॑ धूपयति। 
प्रा॒जा॒प॒त्यो वा अश्वः॑ सयोनि॒त्वाय॑। 
वृष्णो॒ अश्व॑स्य नि॒ष्पद॒सीत्या॑ह। 
अ॒सौ वा आ॑दि॒त्यो वृषाऽश्वः॑। 
तस्य॒ छन्दाꣳ॑सि नि॒ष्पत्॥२५॥

%८.३.६
छन्दो॑भिरे॒वैनं॑ धूपयति। 
अ॒र्चिषे᳚ त्वा शो॒चिषे॒ त्वेत्या॑ह। 
तेज॑ ए॒वास्मि॑न्दधाति। 
वा॒रु॒णो॑ऽभीद्धः॑। 
मै॒त्रियोपै॑ति॒ शान्त्यै᳚। 
सिद्ध्यै॒ त्वेत्या॑ह। 
य॒था॒य॒जुरे॒वैतत्। 
दे॒वस्त्वा॑ सवि॒तोद्व॑प॒त्वित्या॑ह। 
स॒वि॒तृप्र॑सूत ए॒वैनं॒ ब्रह्म॑णा दे॒वता॑भि॒रुद्व॑पति। 
अप॑द्यमानः पृथि॒व्यामाशा॒ दिश॒ आपृ॒णेत्या॑ह॥२६॥

%८.३.७
तस्मा॑द॒ग्निः सर्वा॒ दिशोऽनु॒ विभा॑ति। 
उत्ति॑ष्ठ बृ॒हन्भ॑वो॒र्ध्वस्ति॑ष्ठ ध्रु॒वस्त्वमित्या॑ह॒ प्रति॑ष्ठित्यै। 
ई॒श्व॒रो वा ए॒षो᳚ऽन्धो भवि॑तोः। 
यः प्र॑व॒र्ग्य॑म॒न्वीक्ष॑ते। 
सूर्य॑स्य त्वा॒ चक्षु॒षाऽन्वी᳚क्ष॒ इत्या॑ह। 
चक्षु॑षो गोपी॒थाय॑। 
ऋ॒जवे᳚ त्वा सा॒धवे᳚ त्वा सुक्षि॒त्यै त्वा॒ भूत्यै॒ त्वेत्या॑ह। 
इ॒यं वा ऋ॒जुः। 
अ॒न्तरि॑क्षꣳ सा॒धु। 
अ॒सौ सु॑क्षि॒तिः॥२७॥

%८.३.८
दिशो॒ भूतिः॑। 
इ॒माने॒वास्मै॑ लो॒कान्क॑ल्पयति। 
अथो॒ प्रति॑ष्ठित्यै। 
इ॒दम॒हम॒मुमा॑मुष्याय॒णं  वि॒शा प॒शुभि॑र्ब्रह्मवर्च॒सेन॒ पर्यू॑हा॒मीत्या॑ह। 
वि॒शैवैनं॑ प॒शुभि॑र्ब्रह्मवर्च॒सेन॒ पर्यू॑हति। 
वि॒शेति॑ राज॒न्य॑स्य ब्रूयात्। 
वि॒शैवैनं॒ पर्यू॑हति। 
प॒शुभि॒रिति॒ वैश्य॑स्य। 
प॒शुभि॑रे॒वैनं॒ पर्यू॑हति। 
अ॒सु॒र्यं॑ पात्र॒मना᳚च्छृण्णम्॥२८॥

%८.३.९
आच्छृ॑णत्ति। 
दे॒व॒त्राकः॑। 
अ॒ज॒क्षी॒रेणाऽऽच्छृ॑णत्ति। 
प॒र॒मं वा ए॒तत्पयः॑। 
यद॑जक्षी॒रम्। 
प॒र॒मेणै॒वैनं॒ पय॒साऽऽच्छृ॑णत्ति। 
यजु॑षा॒ व्यावृ॑त्त्यै। 
छन्दो॑भि॒राच्छृ॑णत्ति। 
छन्दो॑भि॒र्वा ए॒ष क्रि॑यते। 
छन्दो॑भिरे॒व छन्दा॒ꣴ॒स्याच्छृ॑णत्ति। 
छृ॒न्धि वाच॒मित्या॑ह। 
वाच॑मे॒वाव॑रुन्धे। 
छृ॒न्ध्यूर्ज॒मित्या॑ह। 
ऊर्ज॑मे॒वाव॑रुन्धे। 
छृ॒न्धि ह॒विरित्या॑ह। 
ह॒विरे॒वाकः॑। 
देव॑ पुरश्चर स॒घ्यास॒न्त्वेत्या॑ह। 
य॒था॒य॒जुरे॒वैतत्॥२९॥
\anuvakamend[स्या॒द्यत् प्र॑व॒र्ग्य॑श्छन्दो॑भिः करोति वी॒र्य॑सम्मितं॒ छन्दाꣳ॑सि नि॒ष्पत्पृ॒णेत्या॑ह सुक्षि॒तिरना᳚च्छृण्ण॒ञ्छन्दा॒ꣴ॒स्या\-च्छृ॑णत्त्य॒ष्टौ च॑]

%८.४.१
ब्रह्म॒न्प्रच॑रिष्यामो॒ होत॑र्घ॒र्मम॒भिष्टु॒हीत्या॑ह। 
ए॒ष वा ए॒तर्\mbox{}हि॒ बृह॒स्पतिः॑। 
यद्ब्र॒ह्मा। 
तस्मा॑ ए॒व प्र॑ति॒प्रोच्य॒ प्रच॑रति। 
आ॒त्मनोऽना᳚र्त्यै। 
य॒माय॑ त्वा म॒खाय॒ त्वेत्या॑ह। 
ए॒ता वा ए॒तस्य॑ दे॒वताः᳚। 
ताभि॑रे॒वैन॒ꣳ॒ सम॑र्द्धयति। 
मद॑न्तीभिः॒ प्रोक्ष॑ति। 
तेज॑ ए॒वास्मि॑न्दधाति॥३०॥

%८.४.२
अ॒भि॒पू॒र्वं प्रोक्ष॑ति। 
अ॒भि॒पू॒र्वमे॒वास्मि॒न्तेजो॑ दधाति। 
त्रिः प्रोक्ष॑ति। 
त्र्या॑वृ॒द्धि य॒ज्ञः। 
अथो॑ मेध्य॒त्वाय॑। 
होताऽन्वा॑ह। 
रक्ष॑सा॒मप॑हत्यै। 
अन॑वानम्। 
प्रा॒णाना॒ꣳ॒ सन्त॑त्यै। 
त्रि॒ष्टुभः॑ स॒तीर्गा॑य॒त्रीरि॒वान्वा॑ह॥३१॥

%८.४.३
गा॒य॒त्रो हि प्रा॒णः। 
प्रा॒णमे॒व यज॑माने दधाति। 
सन्त॑त॒मन्वा॑ह। 
प्रा॒णाना॑म॒न्नाद्य॑स्य॒ सन्त॑त्यै। 
अथो॒ रक्ष॑सा॒मप॑हत्यै। 
यत्परि॑मिता अनुब्रू॒यात्। 
परि॑मित॒मव॑रुन्धीत। 
अप॑रिमिता॒ अन्वा॑ह। 
अप॑रिमित॒स्याव॑रुद्ध्यै। 
शिरो॒ वा ए॒तद्य॒ज्ञस्य॑॥३२॥

%८.४.४
यत्प्र॑व॒र्ग्यः॑। 
ऊर्ङ्मुञ्जाः᳚। 
यन्मौ॒ञ्जो वे॒दो भव॑ति। 
ऊ॒र्जैव य॒ज्ञस्य॒ शिरः॒ सम॑र्द्धयति। 
प्रा॒णा॒हु॒तीर्जु॑होति। 
प्रा॒णाने॒व यज॑माने दधाति। 
स॒प्त जु॑होति। 
स॒प्त वै शी॑र्\mbox{}ष॒ण्याः᳚ प्रा॒णाः। 
प्रा॒णाने॒वास्मि॑न्दधाति। 
दे॒वस्त्वा॑ सवि॒ता मध्वा॑ऽन॒क्त्वित्या॑ह॥३३॥

%८.४.५
तेज॑सै॒वैन॑मनक्ति। 
पृ॒थि॒वीं तप॑सस्त्राय॒स्वेति॒ हिर॑ण्य॒मुपा᳚स्यति। 
अ॒स्या अन॑तिदाहाय। 
शिरो॒ वा ए॒तद्य॒ज्ञस्य॑। 
यत्प्र॑व॒र्ग्यः॑। 
अ॒ग्निः सर्वा॑ दे॒वताः᳚। 
प्र॒ल॒वाना॒दीप्योपा᳚स्यति। 
दे॒वता᳚स्वे॒व य॒ज्ञस्य॒ शिरः॒ प्रति॑दधाति। 
अप्र॑तिशीर्णाग्रं भवति। 
ए॒तद्ब॑र्\mbox{}हि॒र्\mbox{}ह्ये॑षः॥३४॥

%८.४.६
अ॒र्चिर॑सि शो॒चिर॒सीत्या॑ह। 
तेज॑ ए॒वास्मि॑न्ब्रह्मवर्च॒सं द॑धाति। 
सꣳसी॑दस्व म॒हाꣳ अ॒सीत्या॑ह। 
म॒हान् ह्ये॑षः। 
ब्र॒ह्म॒वा॒दिनो॑ वदन्ति। 
ए॒ते वाव त ऋ॒त्विजः॑। 
ये द॑र्\mbox{}शपूर्णमा॒सयोः᳚। 
अथ॑ क॒था होता॒ यज॑मानाया॒ऽऽशिषो॒ नाशा᳚स्त॒ इति॑। 
पु॒रस्ता॑दाशीः॒ खलु॒ वा अ॒न्यो य॒ज्ञः। 
उ॒परि॑ष्टादाशीर॒न्यः॥३५॥

%८.४.७
अ॒ना॒धृ॒ष्या पु॒रस्ता॒दिति॒ यदे॒तानि॒ यजू॒ꣴ॒ष्याह॑। 
शी॒\ar{}\mbox{}ष॒त ए॒व य॒ज्ञस्य॒ यज॑मान आ॒शिषोऽव॑रुन्धे। 
आयुः॑ पु॒रस्ता॑दाह। 
प्र॒जां द॑क्षिण॒तः। 
प्रा॒णं प॒श्चात्। 
श्रोत्र॑मुत्तर॒तः। 
विधृ॑तिमु॒परि॑ष्टात्। 
प्रा॒णाने॒वास्मै॑ स॒मीचो॑ दधाति। 
ई॒श्व॒रो वा ए॒ष दिशोऽनून्म॑दितोः। 
यं दिशोऽनु॑ व्यास्था॒पय॑न्ति॥३६॥

%८.४.८
मनो॒रश्वा॑सि॒ भूरि॑पु॒त्रेती॒माम॒भिमृ॑शति। 
इ॒यं वै मनो॒रश्वा॒ भूरि॑पुत्रा। 
अ॒स्यामे॒व प्रति॑तिष्ठ॒त्यनु॑न्मादाय। 
सू॒प॒सदा॑ मे भूया॒ मा मा॑ हिꣳसी॒रित्या॒हाहिꣳ॑सायै। 
चितः॑ स्थ परि॒चित॒ इत्या॑ह। 
अप॑चितिमे॒वास्मि॑न्दधाति। 
शिरो॒ वा ए॒तद्य॒ज्ञस्य॑। 
यत्प्र॑व॒र्ग्यः॑। 
अ॒सौ खलु॒ वा आ॑दि॒त्यः प्र॑व॒र्ग्यः॑। 
तस्य॑ म॒रुतो॑ र॒श्मयः॑॥३७॥

%८.४.९
स्वाहा॑ म॒रुद्भिः॒ परि॑श्रय॒स्वेत्या॑ह। 
अ॒मुमे॒वाऽऽदि॒त्यꣳ र॒श्मिभिः॒ पर्यू॑हति। 
तस्मा॑द॒सावा॑दि॒त्यो॑ऽमुष्मिँ॑ल्लो॒के र॒श्मिभिः॒ पर्यू॑ढः। 
तस्मा॒द्राजा॑ वि॒शा पर्यू॑ढः। 
तस्मा᳚द्ग्राम॒णीः स॑जा॒तैः पर्यू॑ढः। 
अ॒ग्नेः सृ॒ष्टस्य॑ य॒तः। 
विक॑ङ्कतं॒ भा आ᳚र्च्छत्। 
यद्वैक॑ङ्कताः परि॒धयो॒ भव॑न्ति। 
भा ए॒वाव॑रुन्धे। 
द्वाद॑श भवन्ति॥३८॥

%८.४.१०
द्वाद॑श॒ मासाः᳚ संवत्स॒रः। 
सं॒व॒त्स॒रमे॒वाव॑रुन्धे। 
अस्ति॑ त्रयोद॒शो मास॒ इत्या॑हुः। 
यत्त्र॑योद॒शः प॑रि॒धिर्भव॑ति। 
तेनै॒व त्र॑योद॒शं मास॒मव॑रुन्धे। 
अ॒न्तरि॑क्षस्यान्त॒र्द्धिर॒सीत्या॑ह॒ व्यावृ॑त्त्यै। 
दिवं॒ तप॑सस्त्राय॒स्वेत्यु॒परि॑ष्टा॒द्धिर॑ण्य॒मधि॒ निद॑धाति। 
अ॒मुष्या॒ अन॑तिदाहाय। 
अथो॑ आ॒भ्यामे॒वैन॑मुभ॒यतः॒ परि॑गृह्णाति। 
अर्\mbox{}ह॑न् बिभर्\mbox{}षि॒ साय॑कानि॒ धन्वेत्या॑ह॥३९॥

%८.४.११
स्तौत्ये॒वैन॑मे॒तत्। 
गा॒य॒त्रम॑सि॒ त्रैष्टु॑भमसि॒ जाग॑तम॒सीति॑ ध॒वित्रा॒ण्याद॑त्ते। 
छन्दो॑भिरे॒वैना॒न्याद॑त्ते। 
मधु॒ मध्विति॑ धूनोति। 
प्रा॒णो वै मधु॑। 
प्रा॒णमे॒व यज॑माने दधाति। 
त्रिः परि॑यन्ति। 
त्रि॒वृद्धि प्रा॒णः। 
त्रिः परि॑यन्ति। 
त्र्या॑वृ॒द्धि य॒ज्ञः॥४०॥

%८.४.१२
अथो॒ रक्ष॑सा॒मप॑हत्यै। 
त्रिः पुनः॒ परि॑यन्ति। 
षट्थ्सम्प॑द्यन्ते। 
षड्वा ऋ॒तवः॑। 
ऋ॒तुष्वे॒व प्रति॑तिष्ठन्ति। 
यो वै घ॒र्मस्य॑ प्रि॒यां त॒नुव॑मा॒क्राम॑ति। 
दु॒श्चर्मा॒ वै स भ॑वति। 
ए॒ष ह॒ वा अ॑स्य प्रि॒यां त॒नुव॒माक्रा॑मति। 
यत् त्रिः प॒रीत्य॑ चतु॒र्थं पर्ये॑ति। 
ए॒ताꣳ ह॒ वा अ॑स्यो॒ग्रदे॑वो॒ राज॑नि॒राच॑क्राम॥४१॥

%८.४.१३
ततो॒ वै स दु॒श्चर्मा॑ऽभवत्। 
तस्मा॒त्त्रिः प॒रीत्य॒ न च॑तु॒र्थं परी॑यात्। 
आ॒त्मनो॑ गोपी॒थाय॑। 
प्रा॒णा वै ध॒वित्रा॑णि। 
अव्य॑तिषङ्गं धून्वन्ति। 
प्रा॒णाना॒मव्य॑तिषङ्गाय॒ क्लृप्त्यै᳚। 
वि॒नि॒षद्य॑ धून्वन्ति। 
दि॒क्ष्वे॑व प्रति॑तिष्ठन्ति। 
ऊ॒र्ध्वं धू᳚न्वन्ति। 
सु॒व॒र्गस्य॑ लो॒कस्य॒ सम॑ष्ट्यै। 
स॒र्वतो॑ धून्वन्ति। 
तस्मा॑द॒यꣳ स॒र्वतः॑ पवते॥४२॥
\anuvakamend[द॒धा॒ती॒वान्वा॑ह य॒ज्ञस्या॑है॒ष उ॒परि॑ष्टादाशीर॒न्यो व्या᳚स्था॒पय॑न्ति र॒श्मयो॑ भवन्ति॒ धन्वेत्या॑ह य॒ज्ञश्च॑क्राम॒ सम॑ष्ट्यै॒ द्वे च॑]

%८.५.१
अ॒ग्निष्ट्वा॒ वसु॑भिः पु॒रस्ता᳚द्रोचयतु गाय॒त्रेण॒ छन्द॒सेत्या॑ह। 
अ॒ग्निरे॒वैनं॒ वसु॑भिः पु॒रस्ता᳚द्रोचयति गाय॒त्रेण॒ छन्द॑सा। 
समा॑रुचि॒तो रो॑च॒येत्या॑ह। 
आ॒शिष॑मे॒वैतामाशा᳚स्ते। 
इन्द्र॑स्त्वा रु॒द्रैर्द॑क्षिण॒तो रो॑चयतु॒ त्रैष्टु॑भेन॒ छन्द॒सेत्या॑ह। 
इन्द्र॑ ए॒वैनꣳ॑ रु॒द्रैर्द॑क्षिण॒तो रो॑चयति॒ त्रैष्टु॑भेन॒ छन्द॑सा। 
समा॑रुचि॒तो रो॑च॒येत्या॑ह। 
आ॒शिष॑मे॒वैतामाशा᳚स्ते। 
वरु॑णस्त्वाऽऽदि॒त्यैः प॒श्चाद्रो॑चयतु॒ जाग॑तेन॒ छन्द॒सेत्या॑ह। 
वरु॑ण ए॒वैन॑मादि॒त्यैः प॒श्चाद्रो॑चयति॒ जाग॑तेन॒ छन्द॑सा॥४३॥

%८.५.२
समा॑रुचि॒तो रो॑च॒येत्या॑ह। 
आ॒शिष॑मे॒वैतामाशा᳚स्ते। 
द्यु॒ता॒नस्त्वा॑ मारु॒तो म॒रुद्भि॑रुत्तर॒तो रो॑चय॒त्वानु॑ष्टुभेन॒ छन्द॒सेत्या॑ह। 
द्यु॒ता॒न ए॒वैनं॑ मारु॒तो म॒रुद्भि॑रुत्तर॒तो रो॑चय॒त्यानु॑ष्टुभेन॒ छन्द॑सा। 
समा॑रुचि॒तो रो॑च॒येत्या॑ह। 
आ॒शिष॑\-मे॒वैतामाशा᳚स्ते। 
बृह॒स्पति॑स्त्वा॒ विश्वै᳚र्दे॒वैरु॒परि॑ष्टा\-द्रोचयतु॒ पाङ्क्ते॑न॒ छन्द॒सेत्या॑ह। 
बृह॒स्पति॑रे॒वैनं॒  विश्वै᳚र्दे॒वै\-रु॒परि॑ष्टाद्रोचयति॒ पाङ्क्ते॑न॒ छन्द॑सा। 
समा॑रुचि॒तो रो॑च॒येत्या॑ह। 
आ॒शिष॑मे॒वैतामाशा᳚स्ते॥४४॥

%८.५.३
रो॒चि॒तस्त्वं दे॑व घर्म दे॒वेष्वसीत्या॑ह। 
रो॒चि॒तो ह्ये॑ष दे॒वेषु॑। 
रो॒चि॒षी॒याहं म॑नु॒ष्ये᳚ष्वित्या॑ह। 
रोच॑त ए॒वैष म॑नु॒ष्ये॑षु। 
सम्रा᳚ड्घर्म रुचि॒तस्त्वं दे॒वेष्वायु॑ष्माꣴस्तेज॒स्वी ब्र॑ह्मवर्च॒स्य॑सीत्या॑ह। 
रु॒चि॒तो ह्ये॑ष दे॒वेष्वायु॑ष्माꣴस्तेज॒स्वी ब्र॑ह्मवर्च॒सी। 
रु॒चि॒तो॑ऽहं म॑नु॒ष्ये᳚ष्वायु॑ष्माꣴस्तेज॒स्वी ब्र॑ह्मवर्च॒सी भू॑यास॒मित्या॑ह। 
रु॒चि॒त ए॒वैष म॑नु॒ष्ये᳚ष्वायु॑ष्माꣴस्तेज॒स्वी ब्र॑ह्मवर्च॒सी भ॑वति। 
रुग॑सि॒ रुचं॒ मयि॑ धेहि॒ मयि॒ रुगित्या॑ह। 
आ॒शिष॑मे॒वैतामाशा᳚स्ते। 
तं यदे॒तैर्यजु॑र्भि॒ररो॑चयित्वा। 
रु॒चि॒तो घ॒र्म इति॑ प्रब्रू॒यात्। 
अरो॑चुकोऽध्व॒र्युः स्यात्। 
अरो॑चुको॒ यज॑मानः। 
अथ॒ यदे॑नमे॒तैर्यजु॑र्भी रोचयि॒त्वा। 
रु॒चि॒तो घर्म॒ इति॒ प्राह॑। 
रोचु॑कोऽध्व॒र्युर्भव॑ति। 
रोचु॑को॒ यज॑मानः॥४५॥
\anuvakamend[प॒श्चाद्रो॑चयति॒ जाग॑तेन॒ छन्द॑सा॒ पाङ्क्ते॑न॒ छन्द॑सा॒ समा॑रुचि॒तो रो॑च॒येत्या॑हा॒शिष॑मे॒वैतामाशा᳚स्ते शास्ते॒ऽष्टौ च॑]

%८.६.१
शिरो॒ वा ए॒तद्य॒ज्ञस्य॑। 
यत् प्र॑व॒र्ग्यः॑। 
ग्री॒वा उ॑प॒सदः॑। 
पु॒रस्ता॑दुप॒सदां᳚ प्रव॒र्ग्यं॑ प्रवृ॑णक्ति। 
ग्री॒वास्वे॒व य॒ज्ञस्य॒ शिरः॒ प्रति॑दधाति। 
त्रिः प्रवृ॑णक्ति। 
त्रय॑ इ॒मे लो॒काः। 
ए॒भ्य ए॒व लो॒केभ्यो॑ य॒ज्ञस्य॒ शिरोऽव॑रुन्धे। 
षट्थ्सम्प॑द्यन्ते। 
षड्वा ऋ॒तवः॑॥४६॥

%८.६.२
ऋ॒तुभ्य॑ ए॒व य॒ज्ञस्य॒ शिरोऽव॑रुन्धे। 
द्वाद॑श॒कृत्वः॒ प्रवृ॑णक्ति। 
द्वाद॑श॒ मासाः᳚ संवत्स॒रः। 
सं॒व॒त्स॒रादे॒व य॒ज्ञस्य॒ शिरोऽव॑रुन्धे। 
चतु॑र्विꣳशतिः॒ सम्प॑द्यन्ते। 
चतु॑र्विꣳशतिरर्द्धमा॒साः। 
अ॒र्द्ध॒मा॒सेभ्य॑ ए॒व य॒ज्ञस्य॒ शिरोऽव॑रुन्धे। 
अथो॒ खलु॑। 
स॒कृदे॒व प्र॒वृज्यः॑। 
एक॒ꣳ॒ हि शिरः॑॥४७॥

%८.६.३
अ॒ग्नि॒ष्टो॒मे प्रवृ॑णक्ति। 
ए॒तावा॒\an{} वै य॒ज्ञः। 
यावा॑नग्निष्टो॒मः। 
यावा॑ने॒व य॒ज्ञः। 
तस्य॒ शिरः॒ प्रति॑दधाति। 
नोक्थ्ये᳚ प्रवृ॑ञ्ज्यात्। 
प्र॒जा वै प॒शव॑ उ॒क्थानि॑। 
यदु॒क्थ्ये᳚ प्रवृ॒ञ्ज्यात्। 
प्र॒जां प॒शून॑स्य॒ निर्द॑हेत्। 
वि॒श्व॒जिति॒ सर्व॑पृष्ठे॒ प्रवृ॑णक्ति॥४८॥

%८.६.४
पृ॒ष्ठानि॒ वा अच्यु॑तं च्यावयन्ति। 
पृ॒ष्ठैरे॒वास्मा॒ अच्यु॑तं च्यावयि॒त्वाऽव॑रुन्धे। 
अप॑श्यं गो॒पामित्या॑ह। 
प्रा॒णो वै गो॒पाः। 
प्रा॒णमे॒व प्र॒जासु॒ विया॑तयति। 
अप॑श्यं गो॒पामित्या॑ह। 
अ॒सौ वा आ॑दि॒त्यो गो॒पाः। 
स हीमाः प्र॒जा गो॑पा॒यति॑। 
तमे॒व प्र॒जानां᳚ गो॒प्तारं॑ कुरुते। 
अनि॑पद्यमान॒मित्या॑ह॥४९॥

%८.६.५
न ह्ये॑ष नि॒पद्य॑ते। 
आ च॒ परा॑ च प॒थिभि॒श्चर॑न्त॒मित्या॑ह। 
आ च॒ ह्ये॑ष परा॑ च प॒थिभि॒श्चर॑ति। 
स स॒ध्रीचीः॒ स विषू॑ची॒र्वसा॑न॒ इत्या॑ह। 
स॒ध्रीची᳚श्च॒ ह्ये॑ष विषू॑चीश्च॒ वसा॑नः प्र॒जा अ॒भि वि॒पश्य॑ति। 
आव॑रीवर्ति॒ भुव॑नेष्व॒न्तरित्या॑ह। 
आ ह्ये॑ष व॑री॒वर्ति॒ भुव॑नेष्व॒न्तः। 
अत्र॑ प्रा॒वीर्मधु॒ माध्वी᳚भ्यां॒ मधु॒ माधू॑चीभ्या॒मित्या॑ह। 
वास॑न्तिकावे॒वास्मा॑ ऋ॒तू क॑ल्पयति। 
सम॒ग्निर॒ग्निना॑ ग॒तेत्या॑ह॥५०॥

%८.६.६
ग्रैष्मा॑वे॒वास्मा॑ ऋ॒तू क॑ल्पयति। 
सम॒ग्निर॒ग्निना॑ ग॒तेत्या॑ह। 
अ॒ग्निर्ह्ये॑वैषो᳚ऽग्निना॑ स॒ङ्गच्छ॑ते। 
स्वाहा॒ सम॒ग्निस्तप॑सा ग॒तेत्या॑ह। 
पूर्व॑मे॒वोदि॒तम्। 
उत्त॑रेणा॒भिगृ॑णाति। 
ध॒र्ता दि॒वो विभा॑सि॒ रज॑सः पृथि॒व्या इत्या॑ह। 
शा॒र॒दावे॒वास्मा॑ ऋ॒तू क॑ल्पयति॥५१॥

%८.६.७
दि॒वि दे॒वेषु॒ होत्रा॑ य॒च्छेत्या॑ह। 
होत्रा॑भिरे॒वेमाँल्लो॒कान्त्सन्द॑\-धाति। 
विश्वा॑सां भुवां पत॒ इत्या॑ह। 
हैम॑न्तिकावे॒वास्मा॑ ऋ॒तू क॑ल्पयति। 
दे॒व॒श्रूस्त्वं दे॑व घर्म दे॒वान्पा॒हीत्या॑ह। 
शै॒शि॒रावे॒वास्मा॑ ऋ॒तू क॑ल्पयति। 
त॒पो॒जां वाच॑म॒स्मे निय॑च्छ देवा॒युव॒मित्या॑ह। 
या वै मेध्या॒ वाक्। 
सा त॑पो॒जाः। 
तामे॒वाव॑रुन्धे॥५२॥

%८.६.८
गर्भो॑ दे॒वाना॒मित्या॑ह। 
गर्भो॒ ह्ये॑ष दे॒वाना᳚म्। 
पि॒ता म॑ती॒नामित्या॑ह। 
प्र॒जा वै म॒तयः॑। 
तासा॑मे॒ष ए॒व पि॒ता। 
यत् प्र॑व॒र्ग्यः॑। 
तस्मा॑दे॒वमा॑ह। 
पतिः॑ प्र॒जाना॒मित्या॑ह। 
पति॒र्ह्ये॑ष प्र॒जाना᳚म्। 
मतिः॑ कवी॒नामित्या॑ह॥५३॥

%८.६.९
मति॒र्ह्ये॑ष क॑वी॒नाम्। 
सं दे॒वो दे॒वेन॑ सवि॒त्रा य॑तिष्ट॒ सꣳ सूर्ये॑णारु॒क्तेत्या॑ह। 
अ॒मुं चै॒वाऽऽदि॒त्यं प्र॑व॒र्ग्यं॑ च॒ सꣳशा᳚स्ति। 
आ॒यु॒र्दास्त्वम॒स्मभ्यं॑ घर्म वर्चो॒दा अ॒सीत्या॑ह। 
आ॒शिष॑मे॒वैतामाशा᳚स्ते। 
पि॒ता नो॑ऽसि पि॒ता नो॑ बो॒धेत्या॑ह। 
बो॒धय॑त्ये॒वैनम्᳚। 
न वै॒ ते॑ऽवका॒शा भ॑वन्ति। 
पत्नि॑यै दश॒मः। 
नव॒ वै पुरु॑षे प्रा॒णाः॥५४॥

%८.६.१०
नाभि॑र्दश॒मी। 
प्रा॒णाने॒व यज॑माने दधाति। 
अथो॒ दशा᳚क्षरा वि॒राट्। 
अन्नं॑  वि॒राट्। 
वि॒राजै॒वान्नाद्य॒मव॑रुन्धे। 
य॒ज्ञस्य॒ शिरो᳚ऽच्छिद्यत। 
तद्दे॒वा होत्रा॑भिः॒ प्रत्य॑दधुः। 
ऋ॒त्विजोऽवे᳚क्षन्ते। 
ए॒ता वै होत्राः᳚। 
होत्रा॑भिरे॒व य॒ज्ञस्य॒ शिरः॒ प्रति॑दधाति॥५५॥

%८.६.११
रु॒चि॒तमवे᳚क्षन्ते। 
रु॒चि॒ताद्वै प्र॒जाप॑तिः प्र॒जा अ॑सृजत। 
प्र॒जाना॒ꣳ॒ सृष्ट्यै᳚। 
रु॒चि॒तमवे᳚क्षन्ते। 
रु॒चि॒ताद्वै प॒र्जन्यो॑ वर्\mbox{}षति। 
वर्\mbox{}षु॑कः प॒र्जन्यो॑ भवति। 
सं प्र॒जा ए॑धन्ते। 
रु॒चि॒तमवे᳚क्षन्ते। 
रु॒चि॒तं वै ब्र॑ह्मवर्च॒सम्। 
ब्र॒ह्म॒व॒र्च॒सिनो॑ भवन्ति॥५६॥

%८.६.१२
अ॒धी॒यन्तोऽवे᳚क्षन्ते। 
सर्व॒मायु॑र्\mbox{}यन्ति। 
न पत्न्यवे᳚क्षेत। 
यत्पत्न्य॒वेक्षे॑त। 
प्रजा॑येत। 
प्र॒जां त्व॑स्यै॒ निर्द॑हेत्। 
यन्नावेक्षे॑त। 
न प्रजा॑येत। 
नास्यै᳚ प्र॒जां निर्द॑हेत्। 
ति॒र॒स्कृत्य॒ यजु॑र्वाचयति। 
प्रजा॑यते। 
नास्यै᳚ प्र॒जां निर्द॑हति। 
त्वष्टी॑मती ते सपे॒येत्या॑ह। 
सपा॒द्धि प्र॒जाः प्र॒जाय॑न्ते॥५७॥
\anuvakamend[ऋ॒तवो॒ हि शिरः॒ सर्व॑पृष्ठे॒ प्रवृ॑ण॒क्त्यनि॑पद्यमान॒मित्या॑ह ग॒तेत्या॑ह शार॒दावे॒वास्मा॑ ऋ॒तू क॑ल्पयति रुन्धे कवी॒नामित्या॑ह प्रा॒णाः प्रति॑दधाति भवन्ति वाचयति च॒त्वारि॑ च]

%८.७.१
दे॒वस्य॑ त्वा सवि॒तुः प्र॑स॒व इति॑ रश॒नामाद॑त्ते॒ प्रसू᳚त्यै। 
अ॒श्विनो᳚र्बा॒हुभ्या॒मित्या॑ह। 
अ॒श्विनौ॒ हि दे॒वाना॑मध्व॒र्यू आस्ता᳚म्। 
पू॒ष्णो हस्ता᳚भ्या॒मित्या॑ह॒ यत्यै᳚। 
आद॒देऽदि॑त्यै॒ रास्ना॒ऽसीत्या॑ह॒ यजु॑ष्कृत्यै। 
इड॒ एह्यदि॑त॒ एहि॒ सर॑स्व॒त्येहीत्या॑ह। 
ए॒तानि॒ वा अ॑स्यै देवना॒मानि॑। 
दे॒व॒ना॒मैरे॒वैना॒माह्व॑यति। 
असा॒वेह्यसा॒वेह्यसा॒वेहीत्या॑ह। 
ए॒तानि॒ वा अ॑स्यै मनुष्यना॒मानि॑॥५८॥

%८.७.२
म॒नु॒ष्य॒ना॒मैरे॒वैना॒माह्व॑यति। 
षट्थ्सम्प॑द्यन्ते। 
षड्वा ऋ॒तवः॑। 
ऋ॒तुभि॑रे॒वैना॒माह्व॑यति। 
अदि॑त्या उ॒ष्णीष॑म॒सीत्या॑ह। 
य॒था॒य॒जुरे॒वैतत्। 
वा॒युर॑स्यै॒ड इत्या॑ह। 
वा॒यु॒दे॒व॒त्यो॑ वै व॒त्सः। 
पू॒षा त्वो॒पाव॑सृज॒त्वित्या॑ह। 
पौ॒ष्णा वै दे॒वत॑या प॒शवः॑॥५९॥

%८.७.३
स्वयै॒वैनं॑ दे॒वत॑यो॒पाव॑सृजति। 
अ॒श्विभ्यां॒ प्रदा॑प॒येत्या॑ह। 
अ॒श्विनौ॒ वै दे॒वानां᳚ भि॒षजौ᳚। 
ताभ्या॑मे॒वास्मै॑ भेष॒जं क॑रोति। 
यस्ते॒ स्तनः॑ शश॒य इत्या॑ह। 
स्तौत्ये॒वैना᳚म्। 
उस्र॑ घ॒र्मꣳ शि॒ꣳ॒षोस्र॑ घ॒र्मं पा॑हि घ॒र्माय॑ शि॒ꣳ॒षेत्या॑ह। 
यथा᳚ ब्रू॒याद॒मुष्मै॑ दे॒हीति॑। 
ता॒दृगे॒व तत्। 
बृह॒स्पति॒स्त्वोप॑ सीद॒त्वित्या॑ह॥६०॥

%८.७.४
ब्रह्म॒ वै दे॒वानां॒ बृह॒स्पतिः॑। 
ब्रह्म॑णै॒वैना॒मुप॑सीदति। 
दान॑वः स्थ॒ पेर॑व॒ इत्या॑ह। 
मेध्या॑ने॒वैना᳚न्करोति। 
वि॒ष्व॒ग्वृतो॒ लोहि॑ते॒नेत्या॑ह॒ व्यावृ॑त्त्यै। 
अ॒श्विभ्यां᳚ पिन्वस्व॒ सर॑स्वत्यै पिन्वस्व पू॒ष्णे पि॑न्वस्व॒ बृह॒स्पत॑ये पिन्व॒स्वेत्या॑ह। 
ए॒ताभ्यो॒ ह्ये॑षा दे॒वता᳚भ्यः॒ पिन्व॑ते। 
इन्द्रा॑य पिन्व॒स्वेन्द्रा॑य पिन्व॒स्वेत्या॑ह। 
इन्द्र॑मे॒व भा॑ग॒धेये॑न॒ सम॑र्द्धयति। 
द्विरिन्द्रा॒येत्या॑ह॥६१॥

%८.७.५
तस्मा॒दिन्द्रो॑ दे॒वता॑नां भूयिष्ठ॒भाक्त॑मः। 
गा॒य॒त्रो॑ऽसि॒ त्रैष्टु॑भोऽसि॒ जाग॑तम॒सीति॑ शफोपय॒मानाद॑त्ते। 
छन्दो॑भि\-रे॒वैना॒नाद॑त्ते। 
स॒होर्जो भा॒गेनोप॒मेहीत्या॑ह। 
ऊ॒र्ज ए॒वैनं॑ भा॒गम॑कः। 
अ॒श्विनौ॒ वा ए॒तद्य॒ज्ञस्य॒ शिरः॑ प्रति॒दध॑तावब्रूताम्। 
आ॒वाभ्या॑मे॒व पूर्वा᳚भ्यां॒ वष॑ट्क्रियाता॒ इति॑। 
इन्द्रा᳚श्विना॒ मधु॑नः सार॒घस्येत्या॑ह। 
अ॒श्विभ्या॑मे॒व पूर्वा᳚भ्यां॒ वष॑ट्करोति। 
अथो॑ अ॒श्विना॑वे॒व भा॑ग॒धेये॑न॒ सम॑र्द्धयति॥६२॥

%८.७.६
घ॒र्मं पा॑त वसवो॒ यज॑ता॒ वडित्या॑ह। 
वसू॑ने॒व भा॑ग॒धेये॑न॒ सम॑र्द्धयति। 
यद्व॑षट्कु॒र्यात्। 
या॒तया॑माऽस्य वषट्का॒रः स्या᳚त्। 
यन्न व॑षट्कु॒र्यात्। 
रक्षाꣳ॑सि य॒ज्ञꣳ ह॑न्युः। 
वडित्या॑ह। 
प॒रोक्ष॑मे॒व वष॑ट्करोति। 
नास्य॑ या॒तया॑मा वषट्का॒रो भव॑ति। 
न य॒ज्ञꣳ रक्षाꣳ॑सि घ्नन्ति॥६३॥

%८.७.७
स्वाहा᳚ त्वा॒ सूर्य॑स्य र॒श्मये॑ वृष्टि॒वन॑ये जुहो॒मीत्या॑ह। 
यो वा अ॑स्य॒ पुण्यो॑ र॒श्मिः। 
स वृ॑ष्टि॒वनिः॑। 
तस्मा॑ ए॒वैनं॑ जुहोति। 
मधु॑ ह॒विर॒सीत्या॑ह। 
स्व॒दय॑त्ये॒वैनम्᳚। 
सूर्य॑स्य॒ तप॑स्त॒पेत्या॑ह। 
य॒था॒य॒जुरे॒वैतत्। 
द्यावा॑पृथि॒वीभ्यां᳚ त्वा॒ परि॑गृह्णा॒मीत्या॑ह। 
द्यावा॑पृथि॒वीभ्या॑मे॒वैनं॒ परि॑गृह्णाति॥६४॥

%८.७.८
अ॒न्तरि॑क्षेण॒ त्वोप॑यच्छा॒मीत्या॑ह। 
अ॒न्तरि॑क्षेणै॒वैन॒मुप॑यच्छति। 
न वा ए॒तं म॑नु॒ष्यो॑ भर्तु॑मर्\mbox{}हति। 
दे॒वानां᳚ त्वा पितृ॒णामनु॑मतो॒ भर्तुꣳ॑ शकेय॒मित्या॑ह। 
दे॒वैरे॒वैनं॑ पि॒तृभि॒रनु॑मत॒ आद॑त्ते। 
वि वा ए॑नमे॒तद॑र्द्धयन्ति। 
यत्प॒श्चात्प्र॒वृज्य॑ पु॒रो जुह्व॑ति। 
तेजो॑ऽसि॒ तेजोऽनु॒ प्रेहीत्या॑ह। 
तेज॑ ए॒वास्मि॑न्दधाति। 
दि॒वि॒स्पृङ्मा मा॑ हिꣳसीरन्तरिक्ष॒स्पृङ्मा मा॑ हिꣳसीः पृथिवि॒स्पृङ्मा मा॑ हिꣳसी॒रित्या॒हाहिꣳ॑सायै॥६५॥

%८.७.९
सुव॑रसि॒ सुव॑र्मे यच्छ॒ दिवं॑ यच्छ दि॒वो मा॑ पा॒हीत्या॑ह। 
आ॒शिष॑मे॒वैतामाशा᳚स्ते। 
शिरो॒ वा ए॒तद्य॒ज्ञस्य॑। 
यत्प्र॑व॒र्ग्यः॑। 
आ॒त्मा वा॒युः। 
उ॒द्यत्य॑ वातना॒मान्या॑ह। 
आ॒त्मन्ने॒व य॒ज्ञस्य॒ शिरः॒ प्रति॑दधाति। 
अन॑वानम्। 
प्रा॒णाना॒ꣳ॒ सन्त॑त्यै। 
पञ्चा॑ह॥६६॥

%८.७.१०
पाङ्क्तो॑ य॒ज्ञः। 
यावा॑ने॒व य॒ज्ञः। 
तस्य॒ शिरः॒ प्रति॑दधाति। 
अ॒ग्नये᳚ त्वा॒ वसु॑मते॒ स्वाहेत्या॑ह। 
अ॒सौ वा आ॑दि॒त्यो᳚ऽग्निर्वसु॑मान्। 
तस्मा॑ ए॒वैनं॑ जुहोति। 
सोमा॑य त्वा रु॒द्रव॑ते॒ स्वाहेत्या॑ह। 
च॒न्द्रमा॒ वै सोमो॑ रु॒द्रवान्॑। 
तस्मा॑ ए॒वैनं॑ जुहोति। 
वरु॑णाय त्वाऽऽदि॒त्यव॑ते॒ स्वाहेत्या॑ह॥६७॥

%८.७.११
अ॒प्सु वै वरु॑ण आदि॒त्यवान्॑। 
तस्मा॑ ए॒वैनं॑ जुहोति। 
बृह॒स्पत॑ये त्वा वि॒श्वदे᳚व्यावते॒ स्वाहेत्या॑ह। 
ब्रह्म॒ वै दे॒वानां॒ बृह॒स्पतिः॑। 
ब्रह्म॑णै॒वैनं॑ जुहोति। 
स॒वि॒त्रे त्व॑र्भु॒मते॑ विभु॒मते᳚ प्रभु॒मते॒ वाज॑वते॒ स्वाहेत्या॑ह। 
सं॒व॒त्स॒रो वै स॑वि॒तर्भु॒मान् वि॑भु॒मान्प्र॑भु॒मान् वाज॑वान्। 
तस्मा॑ ए॒वैनं॑ जुहोति। 
य॒माय॒ त्वाऽङ्गि॑रस्वते पितृ॒मते॒ स्वाहेत्या॑ह। 
प्रा॒णो वै य॒मोऽङ्गि॑रस्वान्पितृ॒मान्॥६८॥

%८.७.१२
तस्मा॑ ए॒वैनं॑ जुहोति। 
ए॒ताभ्य॑ ए॒वैनं॑ दे॒वता᳚भ्यो जुहोति। 
दश॒ सम्प॑द्यन्ते। 
दशा᳚क्षरा वि॒राट्। 
अन्नं॑  वि॒राट्। 
वि॒राजै॒वान्नाद्य॒मव॑रुन्धे। 
रौ॒हि॒णाभ्यां॒ वै दे॒वाः सु॑व॒र्गं॑ लो॒कमा॑यन्। 
तद्रौ॑हि॒णयो॑ रौहिण॒त्वम्। 
यद्रौ॑हि॒णौ भव॑तः। 
रौ॒हि॒णाभ्या॑मे॒व तद्यज॑मानः सुव॒र्गं लो॒कमे॑ति। 
अह॒र्ज्योतिः॑ के॒तुना॑ जुषताꣳ सुज्यो॒तिर्ज्योति॑षा॒ꣴ॒ स्वाहा॒ रात्रि॒र्ज्योतिः॑ के॒तुना॑ जुषताꣳ सुज्यो॒तिर्ज्योति॑षा॒ꣴ॒ स्वाहेत्या॑ह। 
आ॒दि॒त्यमे॒व तद॒मुष्मिँ॑ल्लो॒केऽह्ना॑ प॒रस्ता᳚द्दाधार। 
रात्रि॑या अ॒वस्ता᳚त्। 
तस्मा॑द॒सावा॑दि॒त्यो॑ऽमुष्मिँ॑ल्लो॒के॑ऽहोरा॒त्राभ्यां᳚ धृ॒तः॥६९॥
\anuvakamend[म॒नु॒ष्य॒ना॒मानि॑ प॒शवः॑ सीद॒त्वित्या॒हेन्द्रा॒येत्या॑हार्द्धयति घ्नन्ति गृह्णा॒त्यहिꣳ॑सायै॒ पञ्चा॑ऽहादि॒त्यव॑ते॒ स्वाहेत्या॑ह पितृ॒माने॑ति च॒त्वारि॑ च]

%८.८.१
विश्वा॒ आशा॑ दक्षिण॒सदित्या॑ह। 
विश्वा॑ने॒व दे॒वान्प्री॑णाति। 
अथो॒ दुरि॑ष्ट्या ए॒वैनं॑ पाति। 
विश्वां᳚ दे॒वान॑याडि॒हेत्या॑ह। 
विश्वा॑ने॒व दे॒वान्भा॑ग॒धेये॑न॒ सम॑र्द्धयति। 
स्वाहा॑कृतस्य घ॒र्मस्य॒ मधोः᳚ पिबतमश्वि॒नेत्या॑ह। 
अ॒श्विना॑वे॒व भा॑ग॒धेये॑न॒ सम॑र्द्धयति। 
स्वाहा॒ऽग्नये॑ य॒ज्ञिया॑य॒ शं यजु॑र्भि॒रित्या॑ह। 
अ॒भ्ये॑वैनं॑ घारयति। 
अथो॑ ह॒विरे॒वाकः॑॥७०॥

%८.८.२
अश्वि॑ना घ॒र्मं पा॑तꣳ हार्दिवा॒नमह॑र्दि॒वाभि॑रू॒तिभि॒रित्या॑ह। 
अ॒श्विना॑वे॒व भा॑ग॒धेये॑न॒ सम॑र्द्धयति। 
अनु॑ वां॒ द्यावा॑पृथि॒वी मꣳ॑साता॒मित्या॒हानु॑मत्यै। 
स्वाहेन्द्रा॑य॒ स्वाहेन्द्रा॒वडित्या॑ह। 
इन्द्रा॑य॒ हि पु॒रो हू॒यते᳚। 
आ॒श्राव्या॑ह घ॒र्मस्य॑ य॒जेति॑। 
वष॑ट्कृते जुहोति। 
रक्ष॑सा॒मप॑हत्यै। 
अनु॑यजति स्व॒गाकृ॑त्यै। 
घ॒र्मम॑पातमश्वि॒नेत्या॑ह॥७१॥

%८.८.३
पूर्व॑मे॒वोदि॒तम्। 
उत्त॑रेणा॒भिगृ॑णाति। 
अनु॑ वां॒ द्यावा॑पृथि॒वी अ॑मꣳसाता॒मित्या॒हानु॑मत्यै। 
तं प्रा॒व्यं॑ यथा॒वण्णमो॑ दि॒वे नमः॑ पृथि॒व्या इत्या॑ह। 
य॒था॒य॒जुरे॒वैतत्। 
दि॒विधा॑ इ॒मं य॒ज्ञं य॒ज्ञमि॒मं दि॒विधा॒ इत्या॑ह। 
सु॒व॒र्गमे॒वैनं॑ लो॒कं ग॑मयति। 
दिवं॑ गच्छा॒न्तरि॑क्षं गच्छ पृथि॒वीं ग॒च्छेत्या॑ह। 
ए॒ष्वे॑वैनं॑ लो॒केषु॒ प्रति॑ष्ठापयति। 
पञ्च॑ प्र॒दिशो॑ ग॒च्छेत्या॑ह॥७२॥

%८.८.४
दि॒क्ष्वे॑वैनं॒ प्रति॑ष्ठापयति। 
दे॒वान्घ॑र्म॒पान्ग॑च्छ पि॒तॄन्घ॑र्म॒पान्ग॒च्छे\-त्या॑ह। 
उ॒भये᳚ष्वे॒वैनं॒ प्रति॑ष्ठापयति। 
यत्पिन्व॑ते। 
वर्\mbox{}षु॑कः प॒र्जन्यो॑ भवति। 
तस्मा॒त्पिन्व॑मानः॒ पुण्यः॑। 
यत्प्राङ्पिन्व॑ते। 
तद्दे॒वाना᳚म्। 
यद्द॑क्षि॒णा। 
तत्पि॑तृ॒णाम्॥७३॥

%८.८.५
यत्प्र॒त्यक्। 
तन्म॑नु॒ष्या॑णाम्। 
यदुदङ्ङ्॑। 
तद्रु॒द्राणा᳚म्। 
प्राञ्च॒मुद॑ञ्चं पिन्वयति। 
दे॒व॒त्राकः॑। 
अथो॒ खलु॑। 
सर्वा॒ अनु॒ दिशः॑ पिन्वयति। 
सर्वा॒ दिशः॒ समे॑धन्ते। 
अ॒न्तः॒प॒रि॒धि पि॑न्वयति॥७४॥

%८.८.६
तेज॒सोऽस्क॑न्दाय। 
इ॒षे पी॑पिह्यू॒र्जे पी॑पि॒हीत्या॑ह। 
इष॑मे॒वोर्जं॒ यज॑माने दधाति। 
यज॑मानाय पीपि॒हीत्या॑ह। 
यज॑मानायै॒वैतामा॒शिष॒माशा᳚स्ते। 
मह्यं॒ ज्यैष्ठ्या॑य पीपि॒हीत्या॑ह। 
आ॒त्मन॑ ए॒वैतामा॒शिष॒माशा᳚स्ते। 
त्विष्यै᳚ त्वा द्यु॒म्नाय॑ त्वेन्द्रि॒याय॑ त्वा॒ भूत्यै॒ त्वेत्या॑ह। 
य॒था॒य॒जुरे॒वैतत्। 
धर्मा॑सि सु॒धर्मा मे᳚ न्य॒स्मे ब्रह्मा॑णि धार॒येत्या॑ह॥७५॥

%८.८.७
ब्रह्म॑न्ने॒वैनं॒ प्रति॑ष्ठापयति। 
नेत्त्वा॒ वातः॑ स्क॒न्दया॒दिति॒ यद्य॑भि॒चरे᳚त्। 
अ॒मुष्य॑ त्वा प्रा॒णे सा॑दयाम्य॒मुना॑ स॒ह नि॑र॒र्थं ग॒च्छेति॑ ब्रूया॒द्यं द्वि॒ष्यात्। 
यमे॒व द्वेष्टि॑। 
तेनै॑नꣳ स॒ह नि॑र॒र्थं ग॑मयति। 
पू॒ष्णे शर॑से॒ स्वाहेत्या॑ह। 
या ए॒व दे॒वता॑ हु॒तभा॑गाः। 
ताभ्य॑ ए॒वैनं॑ जुहोति। 
ग्राव॑भ्यः॒ स्वाहेत्या॑ह। 
या ए॒वान्तरि॑क्षे॒ वाचः॑॥७६॥

%८.८.८
ताभ्य॑ ए॒वैनं॑ जुहोति। 
प्र॒ति॒रेभ्यः॒ स्वाहेत्या॑ह। 
प्रा॒णा वै दे॒वाः प्र॑ति॒राः। 
तेभ्य॑ ए॒वैनं॑ जुहोति। 
द्यावा॑पृथि॒वीभ्या॒ꣴ॒ स्वाहेत्या॑ह। 
द्यावा॑पृथि॒वीभ्या॑मे॒वैनं॑ जुहोति। 
पि॒तृभ्यो॑ घर्म॒पेभ्यः॒ स्वाहेत्या॑ह। 
ये वै यज्वा॑नः। 
ते पि॒तरो॑ घर्म॒पाः। 
तेभ्य॑ ए॒वैनं॑ जुहोति॥७७॥

%८.८.९
रु॒द्राय॑ रु॒द्रहो᳚त्रे॒ स्वाहेत्या॑ह। 
रु॒द्रमे॒व भा॑ग॒धेये॑न॒ सम॑र्द्धयति। 
स॒र्वतः॒ सम॑नक्ति। 
स॒र्वत॑ ए॒व रु॒द्रं नि॒रव॑दयते। 
उद॑ञ्चं॒ निर॑स्यति। 
ए॒षा वै रु॒द्रस्य॒ दिक्। 
स्वाया॑मे॒व दि॒शि रु॒द्रं नि॒रव॑दयते। 
अ॒प उप॑स्पृशति मेध्य॒त्वाय॑। 
नान्वी᳚क्षेत। 
यद॒न्वीक्षे॑त॥७८॥

%८.८.१०
चक्षु॑रस्य प्र॒मायु॑कꣴ स्यात्। 
तस्मा॒न्नान्वीक्ष्यः॑। 
अपी॑परो॒ माऽह्नो॒ रात्रि॑यै मा पाह्ये॒षा ते॑ अग्ने स॒मित्तया॒ समि॑ध्य॒स्वायु॑र्मे दा॒ वर्च॑सा माऽऽञ्जी॒रित्या॑ह। 
आयु॑रे॒वास्मि॒न्वर्चो॑ दधाति। 
अपी॑परो मा॒ रात्रि॑या॒ अह्नो॑ मा पाह्ये॒षा ते॑ अग्ने स॒मित्तया॒ समि॑ध्य॒स्वाऽऽयु॑र्मे दा॒ वर्च॑सा माऽऽञ्जी॒रित्या॑ह। 
आयु॑रे॒वास्मि॒न्वर्चो॑ दधाति। 
अ॒ग्निर्ज्योति॒र्ज्योति॑र॒ग्निः स्वाहा॒ सूर्यो॒ ज्योति॒र्ज्योतिः॒ सूर्यः॒ स्वाहेत्या॑ह। 
य॒था॒य॒जुरे॒वैतत्। 
ब्र॒ह्म॒वा॒दिनो॑ वदन्ति। 
हो॒त॒व्य॑मग्निहो॒त्रा(३)न्न हो॑त॒व्या(३)मिति॑॥७९॥

%८.८.११
यद्यजु॑षा जुहु॒यात्। 
अय॑थापूर्व॒माहु॑ती जुहुयात्। 
यन्न जु॑हु॒यात्। 
अ॒ग्निः परा॑भवेत्। 
भूः स्वाहेत्ये॒व हो॑त॒व्यम्᳚। 
य॒था॒पू॒र्वमाहु॑ती जु॒होति॑। 
नाग्निः परा॑भवति। 
हु॒तꣳ ह॒विर्मधु॑ ह॒विरित्या॑ह। 
स्व॒दय॑त्ये॒वैनम्᳚। 
इन्द्र॑तमे॒ऽग्नावित्या॑ह॥८०॥

%८.८.१२
प्रा॒णो वा इन्द्र॑तमो॒ऽग्निः। 
प्रा॒ण ए॒वैन॒मिन्द्र॑तमे॒ऽग्नौ जु॑होति। 
पि॒ता नो॑ऽसि॒ मा मा॑ हिꣳसी॒रित्या॒हाहिꣳ॑सायै। 
अ॒श्याम॑ ते देव घर्म॒ मधु॑मतो॒ वाज॑वतः पितु॒मत॒ इत्या॑ह। 
आ॒शिष॑मे॒वैतामाशा᳚स्ते। 
स्व॒धा॒विनो॑ऽशी॒महि॑ त्वा॒ मा मा॑ हिꣳसी॒रित्या॒हाहिꣳ॑सायै। 
तेज॑सा॒ वा ए॒ते व्यृ॑ध्यन्ते। 
ये प्र॑व॒र्ग्ये॑ण॒ चर॑न्ति। 
प्राश्ञ॑न्ति। 
तेज॑ ए॒वात्मन्द॑धते॥८१॥

%८.८.१३
सं॒व॒त्स॒रं न मा॒ꣳ॒सम॑श्ञीयात्। 
न रा॒मामुपे॑यात्। 
न मृ॒न्मये॑न पिबेत्। 
नास्य॑ रा॒म उच्छि॑ष्टं पिबेत्। 
तेज ए॒व तत्सꣴश्य॑ति। 
दे॒वा॒सु॒राः संय॑त्ता आसन्। 
ते दे॒वा वि॑ज॒यमु॑प॒यन्तः॑। 
वि॒भ्राजि॑ सौ॒र्ये ब्रह्म॒सन्न्य॑दधत। 
यत्किं च॑ दिवाकी॒र्त्यम्᳚। 
तदे॒तेनै॒व व्र॒तेना॑गोपायत्। 
तस्मा॑दे॒तद्व्र॒तं चा॒र्यम्᳚। 
तेज॑सो गोपी॒थाय॑। 
तस्मा॑दे॒तानि॒ यजूꣳ॑षि वि॒भ्राजः॑ सौ॒र्यस्येत्या॑हुः। 
स्वाहा᳚ त्वा॒ सूर्य॑स्य र॒श्मिभ्य॒ इति॑ प्रा॒तः सꣳसा॑दयति। 
स्वाहा᳚ त्वा॒ नक्ष॑त्रेभ्य॒ इति॑ सा॒यम्। 
ए॒ता वा ए॒तस्य॑ दे॒वताः᳚। 
ताभि॑रे॒वैन॒ꣳ॒ सम॑र्द्धयति॥८२॥
\anuvakamend[अ॒क॒र॒श्वि॒नेत्या॑ह प्र॒दिशो॑ ग॒च्छेत्या॑ह पितृ॒णाम॑न्तःपरि॒धि पि॑न्वयति धार॒येत्या॑ह॒ वाचो॑ घर्म॒पास्तेभ्य॑ ए॒वैनं॑ जुहोत्य॒न्वीक्षे॑त होत॒व्या(३)मित्य॒ग्नावित्या॑ह दधतेऽगोपायत्स॒प्त च॑]

%८.९.१
घर्म॒ या ते॑ दि॒वि शुगिति॑ ति॒स्र आहु॑तीर्जुहोति। 
छन्दो॑भिरे॒वास्यै॒भ्यो लो॒केभ्यः॒ शुच॒मव॑ यजते। 
इय॒त्यग्रे॑ जुहोति। 
अथेय॒त्यथेय॑ति। 
त्रय॑ इ॒मे लो॒काः। 
अनु॑ नो॒ऽद्यानु॑\-मति॒रित्या॒हानु॑मत्यै। 
दि॒वस्त्वा॑ पर॒स्पाया॒ इत्या॑ह। 
दि॒व ए॒वेमाँल्लो॒कान्दा॑धार। 
ब्रह्म॑णस्त्वा पर॒स्पाया॒ इत्या॑ह॥८३॥

%८.९.२
ए॒ष्वे॑व लो॒केषु॑ प्र॒जा दा॑धार। 
प्रा॒णस्य॑ त्वा पर॒स्पाया॒ इत्या॑ह। 
प्र॒जास्वे॒व प्रा॒णान्दा॑धार। 
शिरो॒ वा ए॒तद्य॒ज्ञस्य॑। 
यत्प्र॑व॒र्ग्यः॑। 
अ॒सौ खलु॒ वा आ॑दि॒त्यः प्र॑व॒र्ग्यः॑। 
तं यद्द॑क्षि॒णा प्र॒त्यञ्च॒मुद॑ञ्चमुद्वा॒सये᳚त्। 
जि॒ह्मं य॒ज्ञस्य॒ शिरो॑ हरेत्। 
प्राञ्च॒मुद्वा॑सयति। 
पु॒रस्ता॑दे॒व य॒ज्ञस्य॒ शिरः॒ प्रति॑दधाति॥८४॥

%८.९.३
प्राञ्च॒मुद्वा॑सयति। 
तस्मा॑द॒सावा॑दि॒त्यः पु॒रस्ता॒दुदे॑ति। 
श॒फो॒प॒य॒\-मान्ध॒वित्रा॑णि॒ धृष्टी॒ इत्य॒न्वव॑हरन्ति। 
सात्मा॑नमे॒वैन॒ꣳ॒ सत॑नुं करोति। 
सात्मा॒ऽमुष्मिँ॑ल्लो॒के भ॑वति। 
य ए॒वं वेद॑। 
औदु॑म्बराणि भवन्ति। 
ऊर्ग्वा उ॑दु॒म्बरः॑। 
ऊर्ज॑मे॒वाव॑रुन्धे। 
वर्त्म॑ना॒ वा अ॒न्वित्य॑॥८५॥

%८.९.४
य॒ज्ञꣳ रक्षाꣳ॑सि जिघाꣳसन्ति। 
साम्ना᳚ प्रस्तो॒ताऽन्ववै॑ति। 
साम॒ वै र॑क्षो॒हा। 
रक्ष॑सा॒मप॑हत्यै। 
त्रिर्नि॒धन॒मुपै॑ति। 
त्रय॑ इ॒मे लो॒काः। 
ए॒भ्य ए॒व लो॒केभ्यो॒ रक्षा॒ꣴ॒स्यप॑हन्ति। 
पुरु॑षः पुरुषो नि॒धन॒मुपै॑ति। 
पुरु॑षः पुरुषो॒ हि र॑क्ष॒स्वी। 
रक्ष॑सा॒मप॑हत्यै॥८६॥

%८.९.५
यत्पृ॑थि॒व्यामु॑द्वा॒सये᳚त्। 
पृ॒थि॒वीꣳ शु॒चाऽर्प॑येत्। 
यद॒प्सु। 
अ॒पः  शु॒चार्प॑येत्। 
यदोष॑धीषु। 
ओष॑धीः  शु॒चाऽर्प॑येत्। 
यद्वन॒स्पति॑षु। 
वन॒स्पती᳚ञ्छु॒चार्प॑येत्। 
हिर॑ण्यं नि॒धायोद्वा॑सयति। 
अ॒मृतं॒ वै हिर॑ण्यम्॥८७॥

%८.९.६
अ॒मृत॑ ए॒वैनं॒ प्रति॑ष्ठापयति। 
व॒ल्गुर॑सि शं॒ युधा॑या॒ इति॒ त्रिः प॑रिषि॒ञ्चन्पर्ये॑ति। 
त्रि॒वृद्वा अ॒ग्निः। 
यावा॑ने॒वाग्निः। 
तस्य॒ शुचꣳ॑ शमयति। 
त्रिः पुनः॒ पर्ये॑ति। 
षट्थ्सम्प॑द्यन्ते। 
षड्वा ऋ॒तवः॑। 
ऋ॒तुभि॑रे॒वास्य॒ शुचꣳ॑ शमयति। 
चतुः॑ स्रक्ति॒र्नाभि॑र्\mbox{}ऋ॒तस्येत्या॑ह॥८८॥

%८.९.७
इ॒यं वा ऋ॒तम्। 
तस्या॑ ए॒ष ए॒व नाभिः॑। 
यत् प्र॑व॒र्ग्यः॑। 
तस्मा॑दे॒वमा॑ह। 
सदो॑ वि॒श्वायु॒रित्या॑ह। 
सदो॒ हीयम्। 
अप॒ द्वेषो॒ अप॒ ह्वर॒ इत्या॑ह॒ भ्रातृ॑व्यापनुत्त्यै। 
घर्मै॒तत्तेऽन्न॑मे॒तत्पुरी॑ष॒मिति॑ द॒ध्ना म॑धुमि॒श्रेण॑ पूरयति। 
ऊर्ग्वा अ॒न्नाद्यं॒ दधि॑। 
ऊ॒र्जैवैन॑म॒न्नाद्ये॑न॒ सम॑र्द्धयति॥८९॥

%८.९.८
अन॑शनायुको भवति। 
य ए॒वं वेद॑। 
रन्ति॒र्नामा॑सि दि॒व्यो ग॑न्ध॒र्व इत्या॑ह। 
रू॒पमे॒वास्यै॒तन्म॑हि॒मान॒ꣳ॒ रन्तिं॑ ब॒न्धुतां॒ व्याच॑ष्टे। 
सम॒हमायु॑षा॒ सं प्रा॒णेनेत्या॑ह। 
आ॒शिष॑मे॒वैतामाशा᳚स्ते। 
व्य॑सौ यो᳚ऽस्मान्द्वेष्टि॒ यं च॑ व॒यं द्वि॒ष्म इत्या॑ह। 
अ॒भि॒चा॒र ए॒वास्यै॒षः। 
अचि॑क्रद॒द्वृषा॒ हरि॒रित्या॑ह। 
वृषा॒ ह्ये॑षः॥९०॥

%८.९.९
वृषा॒ हरिः॑। 
म॒हान्मि॒त्रो न द॑र्\mbox{}श॒त इत्या॑ह। 
स्तौत्ये॒वैन॑मे॒तत्। 
चिद॑सि समु॒द्रयो॑नि॒रित्या॑ह। 
स्वामे॒वैनं॒ योनिं॑ गमयति। 
नम॑स्ते अस्तु॒ मा मा॑ हिꣳसी॒रित्या॒हाहिꣳ॑सायै। 
वि॒श्वाव॑सुꣳ सोम गन्ध॒र्वमित्या॑ह। 
यदे॒वास्य॑ क्रि॒यमा॑णस्यान्त॒र्यन्ति॑। 
तदे॒वास्यै॒तेना प्या॑ययति। 
वि॒श्वाव॑सुर॒भि तन्नो॑ गृणा॒त्वि\-त्या॑ह॥९१॥

%८.९.१०
पूर्व॑मे॒वोदि॒तम्। 
उत्त॑रेणा॒भि गृ॑णाति। 
धियो॑ हिन्वा॒नो धिय॒ इन्नो॑ अव्या॒दित्या॑ह। 
ऋ॒तूने॒वास्मै॑ कल्पयति। 
प्राऽऽसां᳚ गन्ध॒र्वो अ॒मृता॑नि वोच॒दित्या॑ह। 
प्रा॒णा वा अ॒मृताः᳚। 
प्रा॒णाने॒वास्मै॑ कल्पयति। 
ए॒तत्त्वं दे॑व घर्म दे॒वो दे॒वानुपा॑गा॒ इत्या॑ह। 
दे॒वो ह्ये॑ष सं दे॒वानु॒पैति॑। 
इ॒दम॒हं म॑नु॒ष्यो॑ मनु॒ष्या॑नित्या॑ह॥९२॥

%८.९.११
म॒नु॒ष्यो॑ हि। 
ए॒ष सन्म॑नु॒ष्या॑नु॒पैति॑। 
ई॒श्व॒रो वै प्र॑व॒र्ग्य॑मुद्वा॒सय\sn{}। 
प्र॒जां प॒शून्त्सो॑मपी॒थम॑नू॒द्वासः॒ सोम॑ पी॒थानु॒मेहि॑। 
स॒ह प्र॒जया॑ स॒ह रा॒यस्पोषे॒णेत्या॑ह। 
प्र॒जामे॒व प॒शून्त्सो॑मपी॒थमा॒त्मन्ध॑त्ते। 
सु॒मि॒त्रा न॒ आप॒ ओष॑धयः स॒न्त्वित्या॑ह। 
आ॒शिष॑मे॒वैतामाशा᳚स्ते। 
दु॒र्मि॒त्रास्तस्मै॑ भूयासु॒र्यो᳚ऽस्मान्द्वेष्टि॒ यं च॑ व॒यं द्वि॒ष्म इत्या॑ह। 
अ॒भि॒चा॒र ए॒वास्यै॒षः। 
प्र वा ए॒षो᳚ऽस्माल्लो॒काच्च्य॑वते। 
यः प्र॑व॒र्ग्य॑मुद्वा॒सयति॑। 
उदु॒त्यं चि॒त्रमिति॑ सौ॒रीभ्या॑मृ॒ग्भ्यां पुन॒रेत्य॒ गार्\mbox{}ह॑पत्ये जुहोति। 
अ॒यं वै लो॒को गार्\mbox{}ह॑पत्यः। 
अ॒स्मिन्ने॒व लो॒के प्रति॑तिष्ठति। 
अ॒सौ खलु॒ वा आ॑दि॒त्यः सु॑व॒र्गो लो॒कः। 
यत्सौ॒री भव॑तः। 
तेनै॒व सु॑व॒र्गाल्लो॒कान्नैति॑॥९३॥
\anuvakamend[ब्रह्म॑णस्त्वा पर॒स्पाया॒ इत्या॑ह दधात्य॒न्वित्य॑ रक्ष॒स्वी रक्ष॑सा॒मप॑हत्यै॒ वै हिर॑ण्यमाहार्द्धयति॒ ह्ये॑ष गृ॑णा॒त्वित्या॑ह मनु॒ष्या॑नित्या॑हास्यै॒षो᳚ऽष्टौ च॑]

%८.१०.१
प्र॒जाप॑तिं॒ वै दे॒वाः  शु॒क्रं पयो॑ऽदुह्रन्। 
तदे᳚भ्यो॒ न व्य॑भवत्। 
तद॒ग्निर्व्य॑करोत्। 
तानि॒ शुक्रि॑याणि॒ सामा᳚न्यभवन्। 
तेषां॒ यो रसो॒ऽत्यक्ष॑रत्। 
तानि॑ शुक्रय॒जूꣴष्य॑भवन्। 
शुक्रि॑याणां॒ वा ए॒तानि॒ शुक्रि॑याणि। 
सा॒म॒प॒य॒सं वा ए॒तयो॑र॒न्यत्। 
दे॒वाना॑म॒न्यत्पयः॑। 
यद्गोः पयः॑॥९४॥

%८.१०.२
तत्साम्नः॒ पयः॑। 
यद॒जायै॒ पयः॑। 
तद्दे॒वानां॒ पयः॑। 
तस्मा॒द्यत्रै॒तैर्यजु॑र्भि॒\-श्चर॑न्ति। 
तत्पय॑सा चरन्ति। 
प्र॒जाप॑तिमे॒व तत्पय॑सा॒ऽन्नाद्ये॑न॒ सम॑र्द्धयन्ति। 
ए॒ष ह त्वै सा॒क्षात्प्र॑व॒र्ग्यं॑ भक्षयति। 
यस्यै॒वं  वि॒दुषः॑ प्रव॒र्ग्यः॑ प्रवृ॒ज्यते᳚। 
उ॒त्त॒र॒वे॒द्यामुद्वा॑स\-ये॒त्तेज॑स्कामस्य। 
तेजो॒ वा उ॑त्तरवे॒दिः॥९५॥

%८.१०.३
तेजः॑ प्रव॒र्ग्यः॑। 
तेज॑सै॒व तेजः॒ सम॑र्द्धयति। 
उ॒त्त॒र॒वे॒द्यामुद्वा॑सये॒\-दन्न॑\-कामस्य। 
शिरो॒ वा ए॒तद्य॒ज्ञस्य॑। 
यत्प्र॑व॒र्ग्यः॑। 
मुख॑मुत्तरवे॒दिः। 
शी॒र्ष्णैव मुख॒ꣳ॒ सन्द॑धात्य॒न्नाद्या॑य। 
अ॒न्ना॒द ए॒व भ॑वति। 
यत्र॒ खलु॒ वा ए॒तमुद्वा॑सितं॒ वयाꣳ॑सि प॒र्यास॑ते। 
परि॒ वै ताꣳ समां᳚ प्र॒जा वयाꣴ॑स्यासते॥९६॥

%८.१०.४
तस्मा॑दुत्तरवे॒द्यामे॒वोद्वा॑सयेत्। 
प्र॒जानां᳚ गोपी॒थाय॑। 
पु॒रो वा॑ प॒श्चाद्वोद्वा॑सयेत्। 
पु॒रस्ता॒द्वा ए॒तज्ज्योति॒रुदे॑ति। 
तत्प॒श्चान्निम्रो॑चति। 
स्वामे॒वैनं॒ योनि॒मनूद्वा॑सयति। 
अ॒पां मध्य॒ उद्वा॑सयेत्। 
अ॒पां वा ए॒तन्मध्या॒ज्ज्योति॑रजायत। 
ज्योतिः॑ प्रव॒र्ग्यः॑। 
स्वयै॒वैनं॒ योनौ॒ प्रति॑ष्ठापयति॥९७॥

%८.१०.५
यं द्वि॒ष्यात्। 
यत्र॒ स स्यात्। 
तस्यां᳚ दि॒श्युद्वा॑सयेत्। 
ए॒ष वा अ॒ग्निर्वै᳚श्वान॒रः। 
यत्प्र॑व॒र्ग्यः॑। 
अ॒ग्निनै॒वैनं॑ वैश्वान॒रेणा॒भि प्रव॑र्तयति। 
औदु॑म्बर्या॒ꣳ॒ शाखा॑या॒मुद्वा॑सयेत्। 
ऊर्ग्वा उ॑दु॒म्बरः॑। 
अन्नं॑ प्रा॒णः। 
शुग्घ॒र्मः॥९८॥

%८.१०.६
इ॒दम॒हम॒मुष्या॑मुष्याय॒णस्य॑ शु॒चा प्रा॒णमपि॑ दहा॒मीत्या॑ह। 
शु॒चैवास्य॑ प्रा॒णमपि॑ दहति। 
ता॒जगार्ति॒मार्च्छ॑ति। 
यत्र॑ द॒र्भा उ॑प॒दीक॑सन्तताः॒ स्युः। 
तदुद्वा॑सये॒द्वृष्टि॑कामस्य। 
ए॒ता वा अ॒पाम॑नू॒ज्झाव॑र्यो॒ नाम॑। 
यद्द॒र्भाः। 
अ॒सौ खलु॒ वा आ॑दि॒त्य इ॒तो वृष्टि॒मुदी॑रयति। 
अ॒सावे॒वास्मा॑ आदि॒त्यो वृष्टिं॒ निय॑च्छति। 
ता आपो॒ निय॑ता॒ धन्व॑ना यन्ति॥९९॥
\anuvakamend[गोः पय॑ उत्तरवे॒दिरा॑सते स्थापयति घ॒र्मो य॑न्ति]

%८.११.१
प्र॒जाप॑तिः सम्भ्रि॒यमा॑णः। 
स॒म्राट्थ्सम्भृ॑तः। 
घ॒र्मः प्रवृ॑क्तः। 
म॒हा॒वी॒र उद्वा॑सितः। 
अ॒सौ खलु॒ वावैष आ॑दि॒त्यः। 
यत्प्र॑व॒र्ग्यः॑। 
स ए॒तानि॒ नामा᳚न्यकुरुत। 
य ए॒वं वेद॑। 
वि॒दुरे॑नं॒ नाम्ना᳚। 
ब्र॒ह्म॒वा॒दिनो॑ वदन्ति॥१००॥

%८.११.२
यो वै वसी॑याꣳसं यथाना॒ममु॑प॒चर॑ति। 
पुण्या᳚र्तिं॒ वै स तस्मै॑ कामयते। 
पुण्या᳚र्तिमस्मै कामयन्ते। 
य ए॒वं वेद॑। 
तस्मा॑दे॒वं  वि॒द्वान्। 
घ॒र्म इति॒ दिवाऽऽच॑क्षीत। 
स॒म्राडिति॒ नक्तम्᳚। 
ए॒ते वा ए॒तस्य॑ प्रि॒ये त॒नुवौ᳚। 
ए॒ते अ॑स्य प्रि॒ये नाम॑नी। 
प्रि॒ययै॒वैनं॑ त॒नुवा᳚॥१०१॥

%८.११.३
प्रि॒येण॒ नाम्ना॒ सम॑र्द्धयति। 
की॒र्तिर॑स्य॒ पूर्वाग॑च्छति ज॒नता॑माय॒तः। 
गा॒य॒त्री दे॒वेभ्योऽपा᳚क्रामत्। 
तां दे॒वाः प्र॑व॒र्ग्ये॑णै॒वानु॒ व्य॑भवन्। 
प्र॒व॒र्ग्ये॑णाप्नुवन्। 
यच्च॑तुर्विꣳशति॒कृत्वः॑ प्रव॒र्ग्यं॑ प्रवृ॒णक्ति॑। 
गा॒य॒त्रीमे॒व तदनु॒ विभ॑वति। 
गा॒य॒त्रीमा᳚प्नोति। 
पूर्वा᳚ऽस्य॒ जनं॑ य॒तः की॒र्तिर्ग॑च्छति। 
वै॒श्व॒दे॒वः सꣳस॑न्नः॥१०२॥

%८.११.४
वस॑वः॒ प्रवृ॑क्तः। 
सोमो॑ऽभिकी॒र्यमा॑णः। 
आ॒श्वि॒नः पय॑स्यानी॒यमा॑ने। 
मा॒रु॒तः क्वथ\sn{}। 
पौ॒ष्ण उद॑न्तः। 
सा॒र॒स्व॒तो वि॒ष्यन्द॑मानः। 
मै॒त्रः  शरो॑ गृही॒तः। 
तेज॒ उद्य॑तः। 
वा॒युर्ह्रि॒यमा॑णः। 
प्र॒जाप॑तिर्\mbox{}हू॒यमा॑नो॒ वाग्घु॒तः॥१०३॥

%८.११.५
अ॒सौ खलु॒ वावैष आ॑दि॒त्यः। 
यत्प्र॑व॒र्ग्यः॑। 
स ए॒तानि॒ नामा᳚न्यकुरुत। 
य ए॒वं वेद॑। 
वि॒दुरे॑नं॒ नाम्ना᳚। 
ब्र॒ह्म॒वा॒दिनो॑ वदन्ति। 
यन्मृ॒न्मय॒माहु॑तिं॒ नाश्ञु॒तेऽथ॑। 
कस्मा॑दे॒षो᳚ऽश्ञुत॒ इति॑। 
वागे॒ष इति॑ ब्रूयात्। 
वा॒च्ये॑व वाचं॑ दधाति॥१०४॥

%८.११.६
तस्मा॑दश्ञुते। 
प्र॒जाप॑ति॒र्वा ए॒ष द्वा॑दश॒धा विहि॑तः। 
यत्प्र॑व॒र्ग्यः॑। 
यत्प्राग॑वका॒शेभ्यः॑। 
तेन॑ प्र॒जा अ॑सृजत। 
अ॒व॒का॒शैर्दे॑वासु॒रान॑सृजत। 
यदू॒र्ध्वम॑वका॒शेभ्यः॑। 
तेनान्न॑म\-सृजत। 
अन्नं॑ प्र॒जाप॑तिः। 
प्र॒जाप॑ति॒र्वावैषः॥१०५॥
\anuvakamend[व॒द॒न्ति॒ त॒नुवा॒ सꣳस॑न्नो हू॒यमा॑नो॒ वाग्घु॒तो द॑धात्ये॒षः]

%८.१२.१
स॒वि॒ता भू॒त्वा प्र॑थ॒मेऽह॒न्प्रवृ॑ज्यते। 
तेन॒ कामाꣳ॑ एति। 
यद्द्वि॒तीयेऽह॑न्प्रवृ॒ज्यते᳚। 
अ॒ग्निर्भू॒त्वा दे॒वाने॑ति। 
यत्तृ॒तीयेऽह॑न्प्र\-वृ॒ज्यते᳚। 
वा॒युर्भू॒त्वा प्रा॒णाने॑ति। 
यच्च॑तु॒र्थेऽह॑न्प्रवृ॒ज्यते᳚। 
आ॒दि॒त्यो भू॒त्वा र॒श्मीने॑ति। 
यत्प॑ञ्च॒मेऽह॑न्प्रवृ॒ज्यते᳚। 
च॒न्द्रमा॑ भू॒त्वा नक्ष॑त्राण्येति॥१०६॥

%८.१२.२
यत्ष॒ष्ठेऽह॑न्प्रवृ॒ज्यते᳚। 
ऋ॒तुर्भू॒त्वा सं॑वत्स॒रमे॑ति। 
यत्स॑प्त॒मेऽह॑न्प्र\-वृ॒ज्यते᳚। 
धा॒ता भू॒त्वा शक्व॑रीमेति। 
यद॑ष्ट॒मेऽह॑न्प्रवृ॒ज्यते᳚। 
बृह॒स्पति॑र्भू॒त्वा गा॑य॒त्रीमे॑ति। 
यन्न॑व॒मेऽह॑न्प्रवृ॒ज्यते᳚। 
मि॒त्रो भू॒त्वा त्रि॒वृत॑ इ॒माँल्लो॒काने॑ति। 
यद्द॑श॒मेऽह॑न्प्रवृ॒ज्यते᳚। 
वरु॑णो भू॒त्वा वि॒राज॑मेति॥१०७॥

%८.१२.३
यदे॑काद॒शेऽह॑न्प्रवृ॒ज्यते᳚। 
इन्द्रो॑ भू॒त्वा त्रि॒ष्टुभ॑मेति। 
यद्द्वा॑द॒शेऽह॑न्प्र\-वृ॒ज्यते᳚। 
सोमो॑ भू॒त्वा सु॒त्यामे॑ति। 
यत्पु॒रस्ता॑दुप॒सदां᳚ प्रवृ॒ज्यते᳚। 
तस्मा॑दि॒तः परा॑ङ॒मूँल्लो॒काꣴ\-स्तप॑न्नेति। 
यदु॒परि॑ष्टादुप॒सदां᳚ प्रवृ॒ज्यते᳚। 
तस्मा॑द॒मुतो॒ऽर्वा\-ङि॒माँल्लो॒काꣴ\-स्तप॑न्नेति। 
य ए॒वं वेद॑। 
ऐव त॑पति॥१०८॥
\anuvakamend[नक्ष॑त्राण्येति वि॒राज॑मेति तपति]

ॐ शं न॒स्तन्नो॒ मा हा॑सीत्॥ ॐ शान्तिः॒ शान्तिः॒ शान्तिः॑॥
%दे॒वा वै स॒त्रꣳ सा॑वि॒त्रं परि॑श्रिते॒ ब्रह्म॒न् प्रच॑रिष्यामो॒ऽग्निष्ट्वा॒ शिरो᳚ ग्री॒वा दे॒वस्य॑ रश॒नां  विश्वा॒ आशा॒ घर्म॒ या ते᳚ प्र॒जाप॑तिꣳ शुक्रं प्र॒जाप॑तिः सम्भ्रि॒यमा॑णः सवि॒ता भू॒त्वा द्वाद॑श॥ १२॥ दे॒वा वै स॒त्रꣳ स ख॑दि॒रः परि॑श्रितेऽभिपू॒र्वमथो॒ रक्ष॑सा॒ङ्ग्रैष्मा॑वे॒वास्मै॒ ब्रह्म॒ वै दे॒वाना॒मश्वि॑ना घ॒र्मं पा॑तं प्रा॒णो वै वृषा॒ हरि॒र्यो वै वसी॑याꣳसं यथाना॒मम॒ष्टोत्त॑रशतम्॥ १०८॥ दे॒वा वै स॒त्रमैव त॑पति॥ ॐ शान्तिः  शान्तिः  शान्तिः॥ हरिः॑ ओम्।  श्रीकृष्णार्पणमस्तु॥

\closesection