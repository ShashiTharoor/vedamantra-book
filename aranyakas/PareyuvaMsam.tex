% !TeX program = XeLaTeX
% !TeX root = ../AraNyakabook-kindle.tex
\sect{षष्ठः प्रश्नः}\setcounter{anuvakam}{0}
ॐ सन्त्वा॑ सिञ्चामि॒ यजुषा॑ प्र॒जामायु॒र्धनं॑ च॥ ॐ शान्तिः॒ शान्तिः॒ शान्तिः॑॥

%४.१.१
प॒रे॒यु॒वाꣳसं॑ प्र॒वतो॑ म॒हीरनु॑ ब॒हुभ्यः॒ पन्था॑मनपस्पशा॒नम्। 
वै॒व॒स्व॒तꣳ स॒ङ्गम॑नं॒ जना॑नां य॒मꣳ राजा॑नꣳ ह॒विषा॑ दुवस्यत। 
इ॒दं त्वा॒ वस्त्रं॑ प्रथ॒मन्वाग॒न्नपै॒तदू॑ह॒ यदि॒हाबि॑भः पु॒रा। 
इ॒ष्टा॒पू॒र्तमनु॒ सम्प॑श्य॒ दक्षि॑णां॒ यथा॑ ते द॒त्तं ब॑हु॒धा विब॑न्धुषु। 
इ॒मौ यु॑नज्मि ते व॒ह्नी असु॑नीथाय वो॒ढवे᳚। 
याभ्यां᳚ य॒मस्य॒ साद॑नꣳ सु॒कृतां॒ चापि॑ गच्छतात्। 
पू॒षा त्वे॒तश्च्या॑वयतु॒ प्रवि॒द्वानन॑ष्टपशु॒र्भुव॑नस्य गो॒पाः। 
स त्वै॒तेभ्यः॒ परि॑ददात्पि॒तृभ्यो॒ऽग्निर्दे॒वेभ्यः॑ सुवि॒दत्रे᳚भ्यः। 
पू॒षेमा आशा॒ अनु॑वेद॒ सर्वाः॒ सो अ॒स्माꣳ अभ॑यतमेन नेषत्। 
स्व॒स्ति॒दा अघृ॑णिः॒ सर्व॑वी॒रोऽप्र॑युच्छन्पु॒र ए॑तु॒ प्रवि॒द्वान्॥१॥

%४.१.२
आयु॑र्वि॒श्वायुः॒ परि॑पासति त्वा पू॒षा त्वा॑ पातु॒ प्रप॑थे पु॒रस्ता᳚त्। 
यत्राऽऽस॑ते सु॒कृतो॒ यत्र॒ ते य॒युस्तत्र॑ त्वा दे॒वः स॑वि॒ता द॑धातु। 
भुव॑नस्य पत इ॒दꣳ ह॒विः। 
अ॒ग्नये॑ रयि॒मते॒ स्वाहा᳚। 
पुरु॑षस्य सयाव॒र्यपेद॒घानि॑ मृज्महे। 
यथा॑ नो॒ अत्र॒ नाप॑रः पु॒रा ज॒रस॒ आय॑ति। 
पुरु॑षस्य सयावरि॒ वि ते᳚ प्रा॒णम॑सि स्रसम्। 
शरी॑रेण म॒हीमिहि॑ स्व॒धयेहि॑ पि॒तॄनुप॑ प्र॒जया॒ऽस्मानि॒हाव॑ह। 
मैवं॑ मा॒ꣴ॒ स्ता प्रि॑ये॒ऽहं दे॒वी स॒ती पि॑तृलो॒कं यदैषि॑। 
वि॒श्ववा॑रा॒ नभ॑सा॒ संव्य॑यन्त्यु॒भौ नो॑ लो॒कौ पय॑सा॒ऽभ्याव॑वृथ्स्व॥२॥

%४.१.३
इ॒यं नारी॑ पतिलो॒कं वृ॑णा॒ना निप॑द्यत॒ उप॑ त्वा मर्त्य॒ प्रेतम्᳚। 
विश्वं॑ पुरा॒णमनु॑ पा॒लय॑न्ती॒ तस्यै᳚ प्र॒जां द्रवि॑णं चे॒ह धे॑हि। 
उदी᳚र्ष्व नार्य॒भि जी॑वलो॒कमि॒तासु॑मे॒तमुप॑शेष॒ एहि॑। 
ह॒स्त॒ग्रा॒भस्य॑ दिधि॒षोस्त्वमे॒तत्पत्यु॑र्जनि॒त्वम॒भि सम्ब॑भूव। 
सु॒वर्ण॒ꣳ॒ हस्ता॑दा॒ददा॑ना मृ॒तस्य॑ श्रि॒यै ब्रह्म॑णे॒ तेज॑से॒ बला॑य। 
अत्रै॒व त्वमि॒ह व॒यꣳ सु॒शेवा॒ विश्वाः॒ स्पृधो॑ अ॒भिमा॑तीर्जयेम। 
धनु॒र्॒\mbox{}हस्ता॑दा॒ददा॑ना मृ॒तस्य॑ श्रि॒यै क्ष॒त्रायौज॑से॒ बला॑य। 
अत्रै॒व त्वमि॒ह व॒यꣳ सु॒शेवा॒ विश्वाः॒ स्पृधो॑ अ॒भिमा॑तीर्जयेम। 
मणि॒ꣳ॒ हस्ता॑दा॒ददा॑ना मृ॒तस्य॑ श्रि॒यै वि॒शे पुष्ट्यै॒ बला॑य। 
अत्रै॒व त्वमि॒ह व॒यꣳ सु॒शेवा॒ विश्वाः॒ स्पृधो॑ अ॒भिमा॑तीर्जयेम॥३॥

%४.१.४
इ॒मम॑ग्ने चम॒सं मा विजी᳚ह्वरः प्रि॒यो दे॒वाना॑मु॒त सो॒म्याना᳚म्। 
ए॒ष यश्च॑म॒सो दे॑व॒पान॒स्तस्मि॑न्दे॒वा अ॒मृता॑ मादयन्ताम्। 
अ॒ग्नेर्वर्म॒ परि॒ गोभि॑र्व्ययस्व॒ सं प्रोर्णु॑ष्व॒ मेद॑सा॒ पीव॑सा च। 
नेत्त्वा॑ धृ॒ष्णुर्\mbox{}हर॑सा॒ जर्\mbox{}हृ॑षाणो॒ दध॑द्विध॒क्ष्यन्पर्य॒ङ्खया॑तै। 
मैन॑मग्ने॒ विद॑हो॒ माऽभिशो॑चो॒ माऽस्य॒ त्वचं॑ चिक्षिपो॒ मा शरी॑रम्। 
य॒दा शृ॒तं क॒रवो॑ जातवे॒दोऽथे॑मेनं॒ प्रहि॑णुतात्पि॒तृभ्यः॑। 
शृ॒तं य॒दा क॒रसि॑ जातवे॒दोऽथे॑मेनं॒ परि॑दत्तात्पि॒तृभ्यः॑। 
य॒दा गच्छा॒त्यसु॑नीतिमे॒तामथा॑ दे॒वानां᳚ वश॒नीर्भ॑वाति। 
सूर्यं॑ ते॒ चक्षु॑र्गच्छतु॒ वात॑मा॒त्मा द्यां च॒ गच्छ॑ पृथि॒वीं च॒ धर्म॑णा। 
अ॒पो वा॑ गच्छ॒ यदि॒ तत्र॑ ते हि॒तमोष॑धीषु॒ प्रति॑तिष्ठा॒ शरी॑रैः। 
अ॒जो भा॒गस्तप॑सा॒ तं त॑पस्व॒ तं ते॑ शो॒चिस्त॑पतु॒ तं ते॑ अ॒र्चिः। 
यास्ते॑ शि॒वास्त॒नुवो॑ जातवेद॒स्ताभि॑र्वहे॒मꣳ सु॒कृतां॒ यत्र॑ लो॒काः। 
अ॒यं वै त्वम॒स्मादधि॒ त्वमे॒तद॒यं वै तद॑स्य॒ योनि॑रसि। 
वै॒श्वा॒न॒रः पु॒त्रः पि॒त्रे लो॑क॒कृज्जा॑तवेदो॒ वहे॑मꣳ सु॒कृतां॒ यत्र॑ लो॒काः॥४॥
\anuvakamend[वि॒द्वान॒भ्याव॑वृथ्स्वा॒भिमा॑तीर्जयेम॒ शरी॑रैश्च॒त्वारि॑ च]

%४.२.१
य ए॒तस्य॑ प॒थो गो॒प्तार॒स्तेभ्यः॒ स्वाहा॒ य ए॒तस्य॑ प॒थो र॑क्षि॒तार॒स्तेभ्यः॒ स्वाहा॒ य ए॒तस्य॑ प॒थो॑भिऽर॑क्षि॒तार॒स्तेभ्यः॒ स्वाहा᳚\-ऽ\-ऽ\-ख्या॒त्रे स्वाहा॑ऽपाख्या॒त्रे स्वाहा॑ऽभि॒लाल॑पते॒ स्वाहा॑ऽ\-प॒लाल॑\-पते॒ स्वाहा॒ऽग्नये॑ कर्म॒कृते॒ स्वाहा॒ यमत्र॒ नाधी॒मस्तस्मै॒ स्वाहा᳚। 
यस्त॑ इ॒ध्मं ज॒भर॑थ्सिष्विदा॒नो मू॒र्धानं॑ वात॒ तप॑ते त्वा॒या। 
दिवो॒ विश्व॑स्माथ्सीमघाय॒त उ॑रुष्यः। 
अ॒स्मात्त्वमधि॑ जा॒तो॑ऽसि॒ त्वद॒यं जा॑यतां॒ पुनः॑। 
अ॒ग्नये॑ वैश्वान॒राय॑ सुव॒र्गाय॑ लो॒काय॒ स्वाहा᳚॥५॥
\anuvakamend[य ए॒तस्य॒ त्वत्पञ्च॑]

%४.३.१
प्र के॒तुना॑ बृह॒ता भा᳚त्य॒ग्निरा॒विर्विश्वा॑नि वृष॒भो रो॑रवीति। 
दि॒वश्चि॒दन्ता॒दुप॒ मामु॒दान॑ड॒पामु॒पस्थे॑ महि॒षो व॑वर्ध। 
इ॒दं त॒ एकं॑ प॒र ऊ॑त॒ एकं॑ तृ॒तीये॑न॒ ज्योति॑षा॒ संवि॑शस्व। 
सं॒वेश॑नस्त॒नुवै॒ चारु॑रेधि प्रि॒यो दे॒वानां᳚ पर॒मे स॒धस्थे᳚। 
नाके॑ सुप॒र्णमुप॒ यत्पत॑न्तꣳ हृ॒दा वेन॑न्तो अ॒भ्यच॑क्षत त्वा। 
हिर॑ण्यपक्षं॒ वरु॑णस्य दू॒तं य॒मस्य॒ योनौ॑ शकु॒नं भु॑र॒ण्युम्। 
अति॑द्रव सारमे॒यौ श्वानौ॑ चतुर॒क्षौ श॒बलौ॑ सा॒धुना॑ प॒था। 
अथा॑ पि॒तॄन्थ्सु॑वि॒दत्रा॒ꣳ॒ अपी॑हि य॒मेन॒ ये स॑ध॒मादं॒ मद॑न्ति। 
यौ ते॒ श्वानौ॑ यमरक्षि॒तारौ॑ चतुर॒क्षौ प॑थि॒रक्षी॑ नृ॒चक्ष॑सा। 
ताभ्याꣳ॑ राज॒न्परि॑ देह्येनꣴ स्व॒स्ति चा᳚स्मा अनमी॒वं च॑ धेहि॥६॥

%४.३.२
उ॒रु॒ण॒साव॑सु॒तृपा॑वुलुम्ब॒लौ य॒मस्य॑ दू॒तौ च॑रतो॒ वशा॒ꣳ॒ अनु॑। 
ताव॒स्मभ्यं॑ दृ॒शये॒ सूर्या॑य॒ पुन॑र्दत्ता॒ वसु॑म॒द्येह भ॒द्रम्। 
सोम॒ एके᳚भ्यः पवते घृ॒तमेक॒ उपा॑सते। 
येभ्यो॒ मधु॑ प्र॒धाव॑ति॒ ताꣴश्चि॑दे॒वापि॑ गच्छतात्। 
ये युध्य॑न्ते प्र॒धने॑षु॒ शूरा॑सो॒ ये त॑नु॒त्यजः॑। 
ये वा॑ स॒हस्र॑दक्षिणा॒स्ताꣴश्चि॑दे॒वापि॑ गच्छतात्। 
तप॑सा॒ ये अ॑नाधृ॒ष्यास्तप॑सा॒ ये सुव॑र्ग॒ताः। 
तपो॒ ये च॑क्रि॒रे म॒हत्ताꣴश्चि॑दे॒वापि॑ गच्छतात्। 
अश्म॑न्वती रेवतीः॒ सꣳ र॑भध्व॒मुत्ति॑ष्ठत॒ प्रत॑रता सखायः। 
अत्रा॑ जहाम॒ ये अस॒न्नशे॑वाः  शि॒वान् व॒यम॒भि वाजा॒नुत्त॑रेम॥७॥

%४.३.३
यद्वै दे॒वस्य॑ सवि॒तुः प॒वित्रꣳ॑ स॒हस्र॑धारं॒  वित॑तम॒न्तरि॑क्षे। 
येनापु॑ना॒दिन्द्र॒मना᳚र्त॒मार्त्यै॒ तेना॒हं माꣳ स॒र्वत॑नुं पुनामि। 
या रा॒ष्ट्रात्प॒न्नादप॒ यन्ति॒ शाखा॑ अ॒भिमृ॑ता नृ॒पति॑मि॒च्छमा॑नाः। 
धा॒तुस्ताः सर्वाः॒ पव॑नेन पू॒ताः प्र॒जया॒स्मान्र॒य्या वर्च॑सा॒ सꣳसृ॑जाथ। 
उद्व॒यं तम॑स॒स्परि॒ पश्य॑न्तो॒ ज्योति॒रुत्त॑रम्। 
दे॒वं दे॑व॒त्रा सूर्य॒मग॑न्म॒ ज्योति॑रुत्त॒मम्। 
धा॒ता पु॑नातु सवि॒ता पु॑नातु। 
अ॒ग्नेस्तेज॑सा॒ सूर्य॑स्य॒ वर्च॑सा॥८॥
\anuvakamend[धे॒ह्युत्त॑रेमा॒ष्टौ च॑]

%४.४.१
यन्ते॑ अ॒ग्निमम॑न्थाम वृष॒भाये॑व॒ पक्त॑वे। 
इ॒मन्तꣳ श॑मयामसि क्षी॒रेण॑ चोद॒केन॑ च। 
यन्त्वम॑ग्ने स॒मद॑ह॒स्त्वमु॒ निर्वा॑पया॒ पुनः॑। 
क्या॒म्बूरत्र॑ जायतां पाकदू॒र्वा व्य॑ल्कशा। 
शीति॑के॒ शीति॑कावति॒ ह्लादु॑के॒ ह्लादु॑कावति। 
म॒ण्डू॒क्या॑ सुसङ्ग॒मये॒मꣴ स्व॑ग्निꣳ श॒मय॑। 
शं ते॑ धन्व॒न्या आपः॒ शमु॑ ते सन्त्वनू॒क्याः᳚। 
शं ते॑ समु॒द्रिया॒ आपः॒ शमु॑ ते सन्तु॒ वर्ष्याः᳚। 
शं ते॒ स्रव॑न्तीस्त॒नुवे॒ शमु॑ ते सन्तु॒ कूप्याः᳚। 
शन्ते॑ नीहा॒रो व॑र्\mbox{}षतु॒ शमु॒ पृष्वाऽव॑शीयताम्॥९॥

%४.४.२
अव॑ सृज॒ पुन॑रग्ने पि॒तृभ्यो॒ यस्त॒ आहु॑त॒श्चर॑ति स्व॒धाभिः॑। 
आयु॒र्वसा॑न॒ उप॑ यातु॒ शेष॒ꣳ॒ सङ्ग॑च्छतां त॒नुवा॑ जातवेदः। 
सङ्ग॑च्छस्व पि॒तृभिः॒ सꣴ स्व॒धाभिः॒ समि॑ष्टापू॒र्तेन॑ पर॒मे व्यो॑मन्। 
यत्र॒ भूम्यै॑ वृ॒णसे॒ तत्र॑ गच्छ॒ तत्र॑ त्वा दे॒वः स॑वि॒ता द॑धातु। 
यत्ते॑ कृ॒ष्णः  श॑कु॒न आ॑तु॒तोद॑ पिपी॒लः स॒र्प उ॒त वा॒ श्वाप॑दः। 
अ॒ग्निष्टद्विश्वा॑दनृ॒णं कृ॑णोतु॒ सोम॑श्च॒ यो ब्रा᳚ह्म॒णमा॑वि॒वेश॑। 
उत्ति॒ष्ठात॑स्त॒नुव॒ꣳ॒ सम्भ॑रस्व॒ मेह गात्र॒मव॑हा॒ मा शरी॑रम्। 
यत्र॒ भूम्यै॑ वृ॒णसे॒ तत्र॑ गच्छ॒ तत्र॑ त्वा दे॒वः स॑वि॒ता द॑धातु। 
इ॒दं त॒ एकं॑ प॒र ऊ॑त॒ एकं॑ तृ॒तीये॑न॒ ज्योति॑षा॒ संवि॑शस्व। 
सं॒वेश॑नस्त॒नुवै॒ चारु॑रेधि प्रि॒यो दे॒वानां᳚ पर॒मे स॒धस्थे᳚। 
उत्ति॑ष्ठ॒ प्रेहि॒ प्रद्र॒वौकः॑ कृणुष्व पर॒मे व्यो॑मन्। 
य॒मेन॒ त्वं य॒म्या॑ संविदा॒नोत्त॒मं नाक॒मधि॑ रोहे॒मम्। 
अश्म॑न्वती रेवती॒र्यद्वै दे॒वस्य॑ सवि॒तुः प॒वित्रं॒ या रा॒ष्ट्रात्प॒न्नादुद्व॒यं तम॑स॒स्परि॑ धा॒ता पु॑नातु। 
अ॒स्मात्त्वमधि॑ जा॒तो᳚ऽस्य॒यं त्वदधि॑जायताम्। 
अ॒ग्नये॑ वैश्वान॒राय॑ सुव॒र्गाय॑ लो॒काय॒ स्वाहा᳚॥१०॥
\anuvakamend[अव॑शीयताꣳ स॒धस्थे॒ पञ्च॑ च]

%४.५.१
आया॑तु दे॒वः सु॒मना॑भिरू॒तिभि॑र्य॒मो ह॑वे॒ह प्रय॑ताभिर॒क्ता। 
आसी॑दताꣳ सुप्र॒यते॑ह ब॒र्\mbox{}हिष्यूर्जा॑य जा॒त्यै मम॑ शत्रु॒हत्यै᳚। 
य॒मे इ॑व॒ यत॑माने॒ यदैतं॒ प्रवा᳚म्भर॒न्मानु॑षा देव॒यन्तः॑। 
आसी॑दत॒ꣴ॒ स्वमु॑ लो॒कं  विदा॑ने स्वास॒स्थे भ॑वत॒मिन्द॑वे नः। 
य॒माय॒ सोमꣳ॑ सुनुत य॒माय॑ जुहुता ह॒विः। 
य॒मꣳ ह॑ य॒ज्ञो ग॑च्छत्य॒ग्निदू॑तो॒ अर॑ङ्कृतः। 
य॒माय॑ घृ॒तव॑द्ध॒विर्जु॒होत॒ प्र च॑ तिष्ठत। 
स नो॑ दे॒वेष्वाय॑मद्दी॒र्घमायुः॒ प्र जी॒वसे᳚। 
य॒माय॒ मधु॑मत्तम॒ꣳ॒ राज्ञे॑ ह॒व्यं जु॑होतन। 
इ॒दं नम॒ ऋषि॑भ्यः पूर्व॒जेभ्यः॒ पूर्वे᳚भ्यः पथि॒कृद्भ्यः॑॥११॥

%४.५.२
योऽस्य॒ कौष्ठ्य॒ जग॑तः॒ पार्थि॑व॒स्यैक॑ इद्व॒शी। 
य॒मं भ॑ङ्ग्यश्र॒वो गा॑य॒ यो राजा॑नप॒रोध्यः॑। 
य॒मङ्गाय॑ भङ्ग्य॒श्रवो॒ यो राजा॑नप॒रोध्यः॑। 
येना॒ऽऽपो न॒द्यो॑ धन्वा॑नि॒ येन॒ द्यौः पृ॑थि॒वी दृ॒ढा। 
हि॒र॒ण्य॒क॒क्ष्यान् सु॒धुरान्॑ हिरण्या॒क्षान॑यः  श॒फान्। 
अश्वा॑न॒नश्य॑तो दा॒नं॒ य॒मो रा॑जाऽभि॒ तिष्ठ॑ति। 
य॒मो दा॑धार पृथि॒वीं य॒मो विश्व॑मि॒दं जग॑त्। 
य॒माय॒ सर्व॒मित्र॑स्थे॒ यत्प्रा॒णद्वा॒युर॑क्षि॒तम्। 
यथा॒ पञ्च॒ यथा॒ षड्य॒था पञ्च॑ द॒शर्\mbox{}ष॑यः। 
य॒मं यो वि॑द्या॒थ्स ब्रू॑याद्य॒थैक ऋषि॑र्विजान॒ते॥१२॥

%४.५.३
त्रिक॑द्रुकेभिः॒ पत॑ति॒ षडु॒र्वीरेक॒मिद्बृ॒हत्। 
गा॒य॒त्री त्रि॒ष्टुप्छन्दाꣳ॑सि॒ सर्वा॒ ता य॒म आहि॑ता। 
अह॑रह॒र्नय॑मानो॒ गामश्वं॒ पुरु॑षं॒ जग॑त्। 
वैव॑स्वतो॒ न तृ॑प्यति॒ पञ्च॑भि॒र्मान॑वैर्य॒मः। 
वैव॑स्वते॒ विवि॑च्यन्ते॒ यमे॒ राज॑नि ते ज॒नाः। 
ये चे॒ह स॒त्येनेच्छ॑न्ते॒ य उ॒ चानृ॑तवादि॒नः। 
ते रा॑जन्नि॒ह विवि॑च्यन्ते॒ऽथा य॑न्ति त्वा॒मुप॑। 
दे॒वाꣴश्च॒ ये न॑म॒स्यन्ति॒ ब्राह्म॑णाꣴश्चाप॒चित्य॑ति। 
यस्मि॑न्वृ॒क्षे सु॑पला॒शे दे॒वैः स॒म्पिब॑ते य॒मः। 
अत्रा॑ नो वि॒श्पतिः॑ पि॒ता पु॑रा॒णा अनु॑वेनति॥१३॥
\anuvakamend[प॒थि॒कृद्भ्यो॑ विजान॒तेऽनु॑ वेनति]

%४.६.१
वै॒श्वा॒न॒रे ह॒विरि॒दं जु॑होमि साह॒स्रमुथ्सꣳ॑ श॒तधा॑रमे॒तम्। 
तस्मि॑न्ने॒ष पि॒तरं॑ पिताम॒हं प्रपि॑तामहं बिभर॒त्पिन्व॑माने। 
द्र॒फ्सश्च॑स्कन्द पृथि॒वीमनु॒ द्यामि॒मं च॒ योनि॒मनु॒ यश्च॒ पूर्वः॑। 
तृ॒तीयं॒ योनि॒मनु॑ स॒ञ्चर॑न्तं द्र॒फ्सं जु॑हो॒म्यनु॑ स॒प्त होत्राः᳚। 
इ॒मꣳ स॑मु॒द्रꣳ श॒तधा॑र॒मुथ्सं॑व्य॒च्यमा॑नं॒ भुव॑नस्य॒ मध्ये᳚। 
घृ॒तं दुहा॑ना॒मदि॑तिं॒ जना॒याग्ने॒ मा हिꣳ॑सीः पर॒मे व्यो॑मन्। 
अपे॑त॒ वीत॒ वि च॑ सर्प॒तातो॒ येऽत्र॒ स्थ पु॑रा॒णा ये च॒ नूत॑नाः। 
अहो॑भिर॒द्भिर॒क्तुभि॒र्व्य॑क्तं य॒मो द॑दात्वव॒सान॑मस्मै। 
स॒वि॒तैतानि॒ शरी॑राणि पृथि॒व्यै मा॒तुरु॒पस्थ॒ आद॑धे। 
तेभि॑र्युज्यन्तामघ्नि॒याः॥१४॥

%४.६.२
शु॒नं वा॒हाः  शु॒नं ना॒राः  शु॒नं कृ॑षतु॒ लाङ्ग॑लम्। 
शु॒नं व॑र॒त्रा ब॑ध्यन्ताꣳ शु॒नमष्ट्रा॒मुदि॑ङ्गय॒ शुना॑सीरा शु॒नम॒स्मासु॑ धत्तम्। 
शुना॑सीरावि॒मां वाचं॒ यद्दि॒वि च॑क्र॒थुः पयः॑। 
तेने॒मामुप॑ सिञ्चतम्। 
सीते॒ वन्दा॑महे त्वा॒ऽर्वाची॑ सुभगे भव। 
यथा॑ नः सु॒भगा स॑सि॒ यथा॑ नः सु॒फला स॑सि। 
स॒वि॒तैतानि॒ शरी॑राणि पृथि॒व्यै मा॒तुरु॒पस्थ॒ आद॑धे। 
तेभि॑रदिते॒ शं भ॑व। 
विमु॑च्यध्वमघ्नि॒या दे॑व॒याना॒ अता॑रिष्म॒ तम॑सस्पा॒रम॒स्य। 
ज्योति॑रापाम॒ सुव॑रगन्म॥१५॥

%४.६.३
प्र वाता॒ वान्ति॑ प॒तय॑न्ति वि॒द्युत॒ उदोष॑धीर्जिहते॒ पिन्व॑ते॒ सुवः॑। 
इरा॒ विश्व॑स्मै॒ भुव॑नाय जायते॒ यत्प॒र्जन्यः॑ पृथि॒वीꣳ रेत॒साऽव॑ति। 
यथा॑ य॒माय॑ हा॒र्म्यमव॑प॒न्पञ्च॑ मान॒वाः। 
ए॒वं व॑पामि हा॒र्म्यं यथासा॑म जीवलो॒के भूर॑यः। 
चितः॑ स्थ परि॒चित॑ ऊर्ध्व॒चितः॑  श्रयध्वं पि॒तरो॑ दे॒वता᳚। 
प्र॒जाप॑तिर्वः सादयतु॒ तया॑ दे॒वत॑या। 
आप्या॑यस्व॒ सन्ते᳚॥१६॥%
\anuvakamend[अ॒घ्नि॒या अ॑गन्म स॒प्त च॑]

%४.७.१
उत्ते॑ तभ्नोमि पृथि॒वीं त्वत्परी॒मं लो॒कं नि॒दध॒न्मो अ॒हꣳ रि॑षम्। 
ए॒ताꣴ स्थूणां᳚ पि॒तरो॑ धारयन्तु॒ तेऽत्रा॑ य॒मः साद॑नात्ते मिनोतु। 
उप॑सर्प मा॒तरं॒ भूमि॑मे॒तामु॑रु॒व्यच॑सं पृथि॒वीꣳ सु॒शेवा᳚म्। 
ऊर्ण॑म्रदा युव॒तिर्दक्षि॑णावत्ये॒षा त्वा॑ पातु॒ निर्\mbox{}ऋ॑त्या उ॒पस्थे᳚। 
उछ्म॑ञ्चस्व पृथिवि॒ मा विबा॑धिथाः सूपाय॒नास्मै॑ भव सूपवञ्च॒ना। 
मा॒ता पु॒त्रं यथा॑ सि॒चाभ्ये॑नं भूमि वृणु। 
उ॒छ्मञ्च॑माना पृथि॒वी हि तिष्ठ॑सि स॒हस्रं॒ मित॒ उप॒ हि श्रय॑न्ताम्। 
ते गृ॒हासो॑ मधु॒श्चुतो॒ विश्वाहा᳚स्मै शर॒णाः स॒न्त्वत्र॑। 
एणी᳚र्धा॒ना हरि॑णी॒रर्जु॑नीः सन्तु धे॒नवः॑। 
तिल॑वथ्सा॒ ऊर्ज॑मस्मै॒ दुहा॑ना॒ विश्वाहा॑ स॒न्त्वनप॑स्फुरन्तीः॥१७॥

%४.७.२
ए॒षा ते॑ यम॒साद॑ने स्व॒धा निधी॑यते गृ॒हे। 
अक्षि॑ति॒र्नाम॑ ते असौ। 
इ॒दं पि॒तृभ्यः॒ प्रभ॑रेम ब॒र्\mbox{}हिर्दे॒वेभ्यो॒ जीव॑न्त॒ उत्त॑रं भरेम। 
तत्त्व॑मारो॒हासो॒ मेघ्यो॒ भवं॑ य॒मेन॒ त्वं य॒म्या॑ संविदा॒नः। 
मा त्वा॑ वृ॒क्षौ सम्बा॑धिष्टां॒ मा मा॒ता पृ॑थिवि॒ त्वम्। 
पि॒तॄन् हि यत्र॒ गच्छा॒स्येधा॑सं यम॒राज्ये᳚। 
मा त्वा॑ वृ॒क्षौ सम्बा॑धेथां॒ मा मा॒ता पृ॑थि॒वी म॒ही। 
वै॒व॒स्व॒तꣳ हि गच्छा॑सि यम॒राज्ये॒ विरा॑जसि। 
न॒ळं प्ल॒वमारो॑है॒तं न॒ळेन॑ प॒थोऽन्वि॑हि। 
स त्वं॑ न॒ळप्ल॑वो भू॒त्वा॒ सन्त॑र॒ प्रत॒रोत्त॑र॥१८॥

%४.७.३
स॒वि॒तैतानि॒ शरी॑राणि पृथि॒व्यै मा॒तुरु॒पस्थ॒ आद॑धे। 
तेभ्यः॑ पृथिवि॒ शं भ॑व। 
षड्ढो॑ता॒ सूर्यं॑ ते॒ चक्षु॑र्गच्छतु॒ वात॑मा॒त्मा द्यां च॒ गच्छ॑ पृथि॒वीं च॒ धर्म॑णा। 
अ॒पो वा॑ गच्छ॒ यदि॒ तत्र॑ ते हि॒तमोष॑धीषु॒ प्रति॑तिष्ठा॒ शरी॑रैः। 
परं॑ मृत्यो॒ अनु॒परे॑हि॒ पन्थां॒ यस्ते॒ स्व इत॑रो देव॒याना᳚त्। 
चक्षु॑ष्मते शृण्व॒ते ते᳚ ब्रवीमि॒ मा नः॑ प्र॒जाꣳ री॑रिषो॒ मोत वी॒रान्। 
शं वातः॒ शꣳ हि ते॒ घृणिः॒ शमु॑ ते स॒न्त्वोष॑धीः। 
कल्प॑न्तां मे॒ दिशः॑ श॒ग्माः। 
पृ॒थि॒व्यास्त्वा॑ लो॒के सा॑दयाम्य॒मुष्य॒ शर्मा॑सि पि॒तरो॑ दे॒वता᳚। 
प्र॒जाप॑तिस्त्वा सादयतु॒ तया॑ दे॒वत॑या। 
अ॒न्तरि॑क्षस्य त्वा दि॒वस्त्वा॑ दि॒शां त्वा॒ नाक॑स्य त्वा पृ॒ष्ठे ब्र॒ध्नस्य॑ त्वा वि॒ष्टपे॑ सादयाम्य॒मुष्य॒ शर्मा॑सि पि॒तरो॑ दे॒वता᳚। 
प्र॒जाप॑तिस्त्वा सादयतु॒ तया॑ दे॒वत॑या॥१९॥

%४.८.१
अ॒पू॒पवा᳚न्घृ॒तवाꣴ॑श्च॒रुरेह सी॑दतूत्तभ्नु॒वन् पृ॑थि॒वीं द्यामु॒तोपरि॑। 
यो॒नि॒कृतः॑ पथि॒कृतः॑ सपर्यत॒ ये दे॒वानां᳚ घृ॒तभा॑गा इ॒ह स्थ। 
ए॒षा ते॑ यम॒साद॑ने स्व॒धा निधी॑यते गृ॒हे॑ऽसौ। 
दशा᳚क्षरा॒ ताꣳ र॑क्षस्व॒ तां गो॑पायस्व॒ तां ते॒ परि॑ददामि॒ तस्यां᳚ त्वा॒ मा द॑भन्पि॒तरो॑ दे॒वता᳚। 
प्र॒जाप॑तिस्त्वा सादयतु॒ तया॑ दे॒वत॑या। 
अ॒पू॒पवा᳚ञ्छृ॒तवा᳚न् क्षी॒रवा॒न्दधि॑वा॒न्मधु॑माꣴश्च॒रुरेह सी॑दतूत्तभ्नु॒वन् पृ॑थि॒वीं द्यामु॒तोपरि॑। 
यो॒नि॒कृतः॑ पथि॒कृतः॑ सपर्यत॒ ये दे॒वानाꣳ॑ शृ॒तभा॑गाः क्षी॒रभा॑गा॒ दधि॑भागा॒ मधु॑भागा इ॒ह स्थ। 
ए॒षा ते॑ यम॒साद॑ने स्व॒धा निधी॑यते गृ॒हे॑ऽसौ। 
श॒ताक्ष॑रा स॒हस्रा᳚क्षरा॒ऽयुता᳚क्ष॒राऽच्यु॑ताक्षरा॒ ताꣳ र॑क्षस्व॒ तां गो॑पायस्व॒ तां ते॒ परि॑ददामि॒ तस्यां᳚ त्वा॒ मा द॑भन्पि॒तरो॑ दे॒वता᳚। 
प्र॒जाप॑तिस्त्वा सादयतु॒ तया॑ दे॒वत॑या॥२०॥
\anuvakamend[अन॑पस्फुरन्ती॒रुत्त॑र दे॒वत॑या॒ द्वे च॑]

%४.९.१
ए॒तास्ते᳚ स्व॒धा अ॒मृताः᳚ करोमि॒ यास्ते॑ धा॒नाः प॑रि॒किरा॒म्यत्र॑। 
तास्ते॑ य॒मः पि॒तृभिः॑ संविदा॒नोऽत्र॑ धे॒नूः का॑म॒दुघाः᳚ करोतु। 
त्वामर्जु॒नौष॑धीनां॒ पयो᳚ ब्र॒ह्माण॒ इद्वि॑दुः। 
तासां᳚ त्वा॒ मध्या॒दाद॑दे च॒रुभ्यो॒ अपि॑धातवे। 
दू॒र्वाणाꣴ॑ स्त॒म्बमाह॑रै॒तां प्रि॒यत॑मां॒ मम॑। 
इ॒मां दिशं॑ मनु॒ष्या॑णां॒ भूयि॒ष्ठानु॒ वि रो॑हतु। 
काशा॑नाꣴ स्त॒म्बमाह॑र॒ रक्ष॑सा॒मप॑हत्यै। 
य ए॒तस्यै॑ दि॒शः प॒राभ॑वन्नघा॒यवो॒ यथा॒ तेनाभ॑वा॒न्पुनः॑। 
द॒र्भाणाꣴ॑ स्त॒म्बमाह॑र पितृ॒णामोष॑धीं प्रि॒याम्। 
अन्वस्यै॒ मूलं॑ जीवा॒दनु॒ काण्ड॒मथो॒ फलम्᳚॥२१॥

%४.९.२
लो॒कं पृ॑ण॒ ता अ॑स्य॒ सूद॑दोहसः। 
शं वातः॒ शꣳ हि ते॒ घृणिः॒ शमु॑ ते स॒न्त्वोष॑धीः। 
कल्पन्तां ते॒ दिशः॒ सर्वाः᳚। 
इ॒दमे॒व मेतोऽप॑रा॒मार्ति॑माराम॒ काञ्च॒न। 
तथा॒ तद॒श्विभ्यां᳚ कृ॒तं मि॒त्रेण॒ वरु॑णेन च। 
व॒र॒णो वा॑रयादि॒दं दे॒वो वन॒स्पतिः॑। 
आर्त्यै॒ निर्\mbox{}ऋ॑त्यै॒ द्वेषा᳚च्च॒ वन॒स्पतिः॑। 
विधृ॑तिरसि॒ विधा॑रया॒स्मद॒घा द्वेषाꣳ॑सि श॒मि श॒मया॒स्मद॒घा द्वेषाꣳ॑सि य॒व य॒वया॒स्मद॒घा द्वेषाꣳ॑सि। 
पृ॒थि॒वीं ग॑च्छा॒न्तरि॑क्षं गच्छ॒ दिवं॑ गच्छ॒ दिशो॑ गच्छ॒ सुव॑र्गच्छ॒ सुव॑र्गच्छ॒ दिशो॑ गच्छ॒ दिवं॑ गच्छा॒न्तरि॑क्षं गच्छ पृथि॒वीं ग॑च्छा॒ऽऽपो वा॑ गच्छ॒ यदि॒ तत्र॑ ते हि॒तमोष॑धीषु॒ प्रति॑तिष्ठा॒ शरी॑रैः। 
अश्म॑न्वती रेवती॒र्यद्वै दे॒वस्य॑ सवि॒तुः प॒वित्रं॒ या रा॒ष्ट्रात्प॒न्नादुद्व॒यं तम॑स॒स्परि॑ धा॒ता पु॑नातु॥२२॥
\anuvakamend[फलं॑ पुनातु]

%४.१०.१
आ रो॑ह॒ताऽऽयु॑र्ज॒रसं॑ गृणा॒ना अ॑नुपू॒र्वं यत॑माना॒ यति॒ष्ट। 
इ॒ह त्वष्टा॑ सु॒जनि॑मा सु॒रत्नो॑ दी॒र्घमायुः॑ करतु जी॒वसे॑ वः। 
यथाऽहा᳚न्यनुपू॒र्वं भव॑न्ति॒ यथ॒र्तव॑ ऋ॒तुभि॒र्यन्ति॑ कॢ॒प्ताः। 
यथा॒ न पूर्व॒मप॑रो॒ जहा᳚त्ये॒वा धा॑त॒रायूꣳ॑षि कल्पयैषाम्। 
न हि॑ ते अग्ने त॒नुवै᳚ क्रू॒रं च॒कार॒ मर्त्यः॑। 
क॒पिर्ब॑भस्ति॒ तेज॑नं॒ पुन॑र्ज॒रायु॒ गौरि॑व। 
अप॑ नः॒ शोशु॑चद॒घमग्ने॑ शुशु॒ध्या र॒यिम्। 
अप॑ नः॒ शोशु॑चद॒घं मृ॒त्यवे॒ स्वाहा᳚। 
अ॒न॒ड्वाह॑म॒न्वार॑भामहे स्व॒स्तये᳚। 
स न॒ इन्द्र॑ इव दे॒वेभ्यो॒ वह्निः॑ स॒म्पार॑णो भव॥२३॥

%४.१०.२
इ॒मे जी॒वा वि॑ मृ॒तैराव॑वर्ति॒न्नभू᳚द्भ॒द्रा दे॒वहू॑तिं नो अ॒द्य। 
प्राञ्जो॑गामानृ॒तये॒ हसा॑य॒ द्राघी॑य॒ आयुः॑ प्रत॒रां दधा॑नाः। 
मृ॒त्योः प॒दं यो॒पय॑न्तो॒ यदैम॒ द्राघी॑य॒ आयुः॑ प्रत॒रां दधा॑नाः। 
आ॒प्याय॑मानाः प्र॒जया॒ धने॑न शु॒द्धाः पू॒ता भ॑वथ यज्ञियासः। 
इ॒मं जी॒वेभ्यः॑ परि॒धिं द॑धामि॒ मा नोऽनु॑गा॒दप॑रो॒ अर्ध॑मे॒तम्। 
श॒तं जी॑वन्तु श॒रदः॑ पुरू॒चीस्ति॒रो मृ॒त्युं द॑द्महे॒ पर्व॑तेन। 
इ॒मा नारी॑रविध॒वाः सु॒पत्नी॒राञ्ज॑नेन स॒र्पिषा॒ सम्मृ॑शन्ताम्। 
अ॒न॒श्रवो॑ अनमी॒वाः सु॒शेवा॒ आरो॑हन्तु॒ जन॑यो॒ योनि॒मग्रे᳚। 
यदाञ्ज॑नं त्रैककु॒दं जा॒तꣳ हि॒मव॑त॒स्परि॑। 
तेना॒मृत॑स्य॒ मूले॒नारा॑तीर्जम्भयामसि। 
यथा॒ त्वमु॑द्भि॒नथ्स्यो॑षधे पृथि॒व्या अधि॑। 
ए॒वमि॒म उद्भि॑न्दन्तु की॒र्त्या यश॑सा ब्रह्मवर्च॒सेन॑। 
अ॒जो᳚ऽस्यजा॒स्मद॒घा द्वेषाꣳ॑सि य॒वो॑ऽसि य॒वया॒स्मद॒घा द्वेषाꣳ॑सि॥२४॥
\anuvakamend[भ॒व॒ ज॒म्भ॒या॒म॒सि॒ त्रीणि॑ च]

%४.११.१
अप॑ नः॒ शोशु॑चद॒घमग्ने॑ शुशु॒ध्या र॒यिम्। 
अप॑ नः॒ शोशु॑चद॒घम्। 
सु॒क्षे॒त्रि॒या सु॑गातु॒या व॑सू॒या च॑ यजामहे। 
अप॑ नः॒ शोशु॑चद॒घम्। 
प्रयद्भन्दि॑ष्ठ एषां॒ प्रास्माका॑सश्च सू॒रयः॑। 
अप॑ नः॒ शोशु॑चद॒घम्। 
प्रयद॒ग्नेः सह॑स्वतो वि॒श्वतो॒ यन्ति॑ सू॒रयः॑। 
अप॑ नः॒ शोशु॑चद॒घम्। 
प्रयत्ते॑ अग्ने सू॒रयो॒ जाये॑महि॒ प्र ते॑ व॒यम्। 
अप॑ नः॒ शोशु॑चद॒घम्॥२५॥

%४.११.२
त्वꣳ हि वि॑श्वतोमुख वि॒श्वतः॑ परि॒भूरसि॑। 
अप॑ नः॒ शोशु॑चद॒घम्। 
द्विषो॑ नो विश्वतोमु॒खाऽति॑ ना॒वेव॑ पारय। 
अप॑ नः॒ शोशु॑चद॒घम्। 
स नः॒ सिन्धु॑मिव ना॒वयाति॑ पर्\mbox{}षा स्व॒स्तये᳚। 
अप॑ नः॒ शोशु॑चद॒घम्। 
आपः॑ प्रव॒णादि॑व य॒तीरपा॒स्मथ्स्य॑न्दताम॒घम्। 
अप॑ नः॒ शोशु॑चद॒घम्। 
उ॒द्व॒नादु॑द॒कानी॒वापा॒स्मथ्स्य॑न्दताम॒घम्। 
अप॑ नः॒ शोशु॑चद॒घम्। 
आ॒न॒न्दाय॑ प्रमो॒दाय॒ पुन॒रागा॒ꣴ॒ स्वान्गृ॒हान्। 
अप॑ नः॒ शोशु॑चद॒घम्। 
न वै तत्र॒ प्रमी॑यते॒ गौरश्वः॒ पुरु॑षः प॒शुः। 
यत्रे॒दं ब्रह्म॑ क्रि॒यते॑ परि॒धिर्जीव॑नाय॒कमप॑ नः॒ शोशु॑चद॒घम्॥२६॥
\anuvakamend[अ॒घम॒घं च॒त्वारि॑ च]

%४.१२.१
अप॑श्याम युव॒तिमा॒चर॑न्तीं मृ॒ताय॑ जी॒वां प॑रिणी॒यमा॑नाम्। 
अ॒न्धेन॒ या तम॑सा॒ प्रावृ॑ताऽसि॒ प्राची॒मवा॑ची॒मव॒यन्नरि॑ष्ट्यै। 
मयै॒तां मा॒ꣴ॒स्तां भ्रि॒यमा॑णा दे॒वी स॒ती पि॑तृलो॒कं यदैषि॑। 
वि॒श्ववा॑रा॒ नभ॑सा॒ संव्य॑यन्त्यु॒भौ नो॑ लो॒कौ पय॒साऽऽवृ॑णीहि। 
रयि॑ष्ठाम॒ग्निं मधु॑मन्तमू॒र्मिण॒मूर्जः॑ सन्तं त्वा॒ पय॒सोप॒ सꣳस॑देम। 
सꣳ र॒य्या समु॒ वर्च॑सा॒ सच॑स्वा नः स्व॒स्तये᳚। 
ये जी॒वा ये च॑ मृ॒ता ये जा॒ता ये च॒ जन्त्याः᳚। 
तेभ्यो॑ घृ॒तस्य॑ धारयितुं॒ मधु॑धारा व्युन्द॒ती। 
मा॒ता रु॒द्राणां᳚ दुहि॒ता वसू॑ना॒ꣴ॒ स्वसा॑ऽऽदि॒त्याना॑म॒मृत॑स्य॒ नाभिः॑। 
प्रणु॒वोचं॑ चिकि॒तुषे॒ जना॑य॒ मागामना॑गा॒मदि॑तिं वधिष्ट। 
पिब॑तूद॒कं तृणा᳚न्यत्तु। 
ओमुथ्सृ॒जत॥२७॥
\anuvakamend[व॒धि॒ष्ट॒ द्वे च॑]

%४.०.०
%प॒रे॒यु॒वाꣳसं॒ प्रवि॒द्वान्भुव॑नस्या॒भ्याव॑वृथ्स्वा॒जो भा॒गो॑ऽयं वै चतु॑श्चत्वारिꣳशत्। 
% य ए॒तस्य॒ त्वत्पञ्च॑। 
% प्रके॒तुने॒दन्ते॒ नाके॑ सुप॒र्णमपी॑हि॒ यौ ते॒ ये युध्य॑न्ते॒ तप॒साऽश्म॑न्वती रेवतीः॒ सꣳर॑भध्वꣳ स॒हस्र॑धारम॒ष्टाविꣳ॑शतिः। 
% यन्ते॒ यत्त॒ उत्ति॒ष्ठात॑ इ॒दन्त॒ उत्ति॑ष्ठ॒ प्रेह्यश्म॒न्॒ यद्वा उद्व॒यम॒यं पञ्च॑विꣳशतिः। 
% आया॑तु त्रि॒ꣳ॒शत्। 
% वै॒श्वा॒न॒रे तस्मि॑न्द्र॒फ्स इ॒ममपे॒ताहो॑भिर्युज्यन्तामघ्नि॒या अ॑दिते पा॒रं व॒ आप्या॑यस्व स॒प्तविꣳ॑शतिः। 
% उत्ते॑ गृ॒हेऽक्षि॑ति॒स्तेभ्यः॑ पृथिवि॒ षड्ढो॑ता॒ परं॑ मे श॒ग्माः पृ॑थि॒व्या अ॒न्तरि॑क्षस्य॒ द्वात्रिꣳ॑शत्। 
% अ॒पू॒पवा॑नसौ॒ दश॑ श॒त द॑श। 
% ए॒तास्ते॑ ते॒ दिशः॒ सर्वा॑ इ॒दमश्म॑न्विꣳश॒तिः। 
% आरो॑हत त॒नुवै᳚ क्रू॒रं च॒कार॒ पुन॑र्मृ॒त्यवे॒ मा नोनु॑गाद्दद्मह इ॒मा नारीः॒ परि॒ त्रयो॑विꣳशतिः। 
% अप॑नः सुक्षेत्रि॒या प्रयद्भन्दि॑ष्ठः॒ प्रयद॒ग्नेः प्रयत्ते॑ अग्ने॒ त्वꣳ हि द्विषः॒ सनः॒ सिन्धु॒मापः॑ प्रव॒णादु॑द्व॒नादा॑न॒न्दाय॒ न वै तत्र॒ यत्रे॒दं चतु॑र्विꣳशतिः। 
% अप॑श्या॒मावृ॑णीहि॒ द्वाद॑श द्वादश। 
% १२ प॒रे॒यु॒वाꣳस॒माया᳚त्वे॒तास्ते॑ स॒प्तविꣳ॑शतिः। 
% २७ प॒रे॒यु॒वाꣳस॒मोमुथ्सृ॒जत।

सन्त्वा॑ सिञ्चामि॒ यजुषा॑ प्र॒जामायु॒र्धनं॑ च॥ ॐ शान्तिः॒ शान्तिः॒ शान्तिः॑॥

सु॒म॒ङ्ग॒लीरि॒यं व॒धूरि॒माꣳ स॑मेत॒ पश्य॑त। 
सौभा᳚ग्यम॒स्यै द॒त्त्वायाथास्तं॒ वि परे॑तन। 
इ॒मां त्वमि॑न्द्र मीढ्वः सुपु॒त्राꣳ सु॒भगां᳚ कुरु। 
दशा᳚स्यां पु॒त्राना धे॑हि॒ पति॑मेकाद॒शं कृ॑धि॥ आ॒वह॑न्ती वितन्वा॒ना। 
कु॒र्वा॒णा चीर॑मा॒त्मनः॑। 
वासाꣳ॑सि॒ मम॒ गाव॑श्च। 
अ॒न्न॒पा॒ने च॑ सर्व॒दा। 
ततो॑ मे॒ श्रिय॒माव॑ह।

ॐ शान्तिः॒ शान्तिः॒ शान्तिः॑॥

\closesection