% !TeX program = XeLaTeX
% !TeX root = ../AraNyakabook-kindle.tex
\sect{प्रथमः प्रश्नः --- अरुणप्रश्नः}\setcounter{anuvakam}{0}
ॐ भ॒द्रं कर्णे॑भिः शृणु॒याम॑ देवाः। भ॒द्रं प॑श्येमा॒क्षभि॒र्यज॑त्राः। 
स्थि॒रैरङ्गैस्तुष्टु॒वा स॑स्त॒नूभि॑। व्यशे॑म दे॒वहि॑तं॒ यदायु॑। 
स्व॒स्ति न॒ इन्द्रो॑ वृ॒द्धश्र॑वाः। स्व॒स्ति न॑ पू॒षा वि॒श्ववे॑दाः। 
स्व॒स्ति न॒स्तार्क्ष्यो॒ अरि॑ष्टनेमिः। स्व॒स्ति नो॒ बृह॒स्पति॑र्दधातु॥
ॐ शान्ति॒ शान्ति॒ शान्ति॑॥

भ॒द्रं कर्णे॑भिः शृणु॒याम॑ देवाः। भ॒द्रं प॑श्येमा॒क्षभि॒र्यज॑त्राः। 
स्थि॒रैरङ्गैस्तुष्टु॒वा स॑स्त॒नूभि॑। व्यशे॑म दे॒वहि॑तं॒ यदायु॑। 
स्व॒स्ति न॒ इन्द्रो॑ वृ॒द्धश्र॑वाः। स्व॒स्ति न॑ पू॒षा वि॒श्ववे॑दाः। 
स्व॒स्ति न॒स्तार्क्ष्यो॒ अरि॑ष्टनेमिः। स्व॒स्ति नो॒ बृह॒स्पति॑र्दधातु। 
आप॑मापाम॒पः सर्वा। अ॒स्माद॒स्मादि॒तोऽमुत॑॥१॥ 

अ॒ग्निर्वा॒युश्च॒ सूर्य॑श्च। स॒ह स॑ञ्चस्क॒रर्द्धि॑या। 
वा॒य्वश्वा॑ रश्मि॒पत॑यः। मरीच्यात्मानो॒ अद्रु॑हः। 
दे॒वीर्भु॑वन॒सूव॑रीः। पु॒त्र॒व॒त्वाय॑ मे सुत। 
महानाम्नीर्म॑हामा॒नाः। म॒ह॒सो म॑हस॒ स्व॑। 
दे॒वीः प॑र्जन्य॒सूव॑रीः। पु॒त्र॒व॒त्वाय॑ मे सुत॥२॥

अ॒पाश्न्यु॑ष्णिम॒पा रक्ष॑। अ॒पाश्न्यु॑ष्णिम॒पारघम्। 
अपाघ्रा॒मप॑ चा॒वर्तिम्। अप॑दे॒वीरि॒तो हि॑त। 
वज्रं॑ दे॒वीरजी॑ताश्च। भुव॑नं देव॒सूव॑रीः। 
आ॒दि॒त्यानदि॑तिं दे॒वीम्। योनि॑नोर्ध्वमु॒दीष॑त। 
शि॒वा न॒ शन्त॑मा भवन्तु। दि॒व्या आप॒ ओष॑धयः। 
सु॒मृ॒डी॒का सर॑स्वति। मा ते॒ व्यो॑म स॒न्दृशि॑॥३॥\anuvakamend


स्मृति॑ प्र॒त्यक्ष॑मैति॒ह्यम्। अनु॑मानश्चतुष्ट॒यम्‌। 
ए॒तैरादि॑त्य\-मण्डलम्‌। सर्वै॑रेव॒ विधास्यते। 
सूर्यो॒ मरी॑चि॒माद॑त्ते। सर्वस्माद्भुव॑नाद॒धि। 
तस्याः पाकवि॑शेषे॒ण। स्मृ॒तं का॑ल\-वि॒शेष॑णम्‌। 
न॒दीव॒ प्रभ॑वात्का॒चित्‌। अ॒क्षय्यात्स्यन्द॒ते य॑था॥४॥

तां नद्योऽभि स॑माय॒न्ति। सो॒रुः सती॑ न नि॒व॑र्तते। 
ए॒वं ना॒नास॑मुत्था॒नाः। का॒लाः सं॑वत्स॒र श्रि॑ताः। 
 अणुशश्च म॑हश॒श्च। सर्वे॑ समव॒यन्त्रि॑तम्‌। 
सतै स॒र्वैः स॑मावि॒ष्टः। ऊ॒रुः स॑न्न नि॒वर्त॑ते। 
अधिसंवत्स॑रं वि॒द्यात्‌। तदेव॑ लक्ष॒णे॥५॥

अणुभिश्च म॑हद्भि॒श्च। स॒मारू॑ढः प्र॒दृश्य॑ते। 
संवत्सरः प्र॑त्यक्षे॒ण। ना॒धिस॑त्वः प्र॒दृश्य॑ते। 
प॒टरो॑ विक्लि॑धः पि॒ङ्गः। ए॒तद्व॑रुण॒लक्ष॑णम्‌। 
यत्रैत॑दुप॒दृश्य॑ते। स॒हस्रं॑ तत्र॒ नीय॑ते। 
एक हि शिरो ना॑ना मु॒खे। कृ॒त्स्नं त॑दृतु॒लक्ष॑णम्‌॥६॥

उभयतः सप्तेन्द्रिया॒णि। ज॒ल्पितं॑ त्वेव॒ दिह्य॑ते। 
शुक्लकृष्णे संव॑त्सर॒स्य। दक्षिणवाम॑योः पा॒र्श्वयोः। 
तस्यै॒षा भव॑ति। शु॒क्रं ते॑ अ॒न्यद्य॑ज॒तं ते॑ अ॒न्यत्‌। 
विषु॑रूपे॒ अह॑नी॒ द्यौरि॑वासि। विश्वा॒ हि मा॒या अव॑सि स्वधावः। 
भ॒द्रा ते॑ पूषन्नि॒ह रा॒तिर॒स्त्विति॑। नात्र॒ भुव॑नम्‌। न पू॒षा। 
न प॒शव॑। नाऽऽदित्यः संवत्सर एव प्रत्यक्षेण प्रियत॑मं वि॒द्यात्‌। 
एतद्वै संवत्सरस्य प्रियत॑म रू॒पम्‌। योऽस्य महानर्थ उत्पत्स्यमा॑नो भ॒वति। 
इदं पुण्यं कु॑रुष्वे॒ति। तमाहर॑णं द॒द्यात्‌॥७॥\anuvakamend


सा॒क॒ञ्जाना स॒प्तथ॑माहुरेक॒जम्‌। षडु॑द्य॒मा ऋष॑यो देव॒जा इति॑। 
तेषा॑मि॒ष्टानि॒ विहि॑तानि धाम॒शः। स्था॒त्रे रे॑जन्ते॒ विकृ॑तानि रूप॒शः। 
को नु॑ मर्या॒ अमि॑थितः। सखा॒ सखा॑यमब्रवीत्‌। 
जहा॑को अ॒स्मदी॑षते। यस्ति॒त्याज॑ सखि॒विद॒ सखा॑यम्‌। 
न तस्य॑ वा॒च्यपि॑ भा॒गो अ॑स्ति। यदी शृ॒णोत्य॒लक शृणोति॥८॥

न हि प्र॒वेद॑ सुकृ॒तस्य॒ पन्था॒मिति॑। ऋ॒तु\ur{}ऋ॑तुना नु॒द्यमा॑नः। 
विन॑नादा॒भिधा॑वः। षष्टिश्च त्रिश॑का व॒ल्गाः। 
शु॒क्लकृ॑ष्णौ च॒ षाष्टि॑कौ। सा॒रा॒ग॒व॒स्त्रैर्ज॒रद॑क्षः। 
व॒स॒न्तो वसु॑भिः स॒ह। सं॒व॒त्स॒रस्य॑ सवि॒तुः। 
प्रै॒ष॒कृत्प्र॑थ॒मः स्मृ॑तः। अ॒मूना॒दय॑तेत्य॒न्यान्‌॥९॥

अ॒मूश्च॑ परि॒रक्ष॑तः। ए॒ता वा॒चः प्र॑युज्य॒न्ते। 
यत्रैत॑दुप॒दृश्य॑ते। ए॒तदे॒व वि॑जानी॒यात्। 
प्र॒माणं॑ काल॒प॑र्यये। वि॒शे॒ष॒णं तु॑ वक्ष्या॒मः। 
ऋ॒तूनां तन्नि॒बोध॑त। शुक्लवासा॑ रुद्र॒गणः। 
ग्री॒ष्मेणा॑ऽऽवर्त॒ते स॑ह। नि॒जह॑न्‌ पृथि॑वी स॒र्वाम्‌॥१०॥

ज्यो॒तिषाऽप्रति॒ख्येन॑ सः। वि॒श्व॒रू॒पाणि॑ वासा॒सि। 
आ॒दि॒त्यानां नि॒बोध॑त। संवत्सरीणं॑ कर्म॒फलम्‌। 
व\ur{}षाभिर्द॑दता॒ सह। अदुःखो॑ दुःखच॑क्षुरि॒व। 
तद्मा॑ऽऽपीत इव॒ दृश्य॑ते। शीतेनाव्यथ॑यन्नि॒व। 
रु॒रुद॑क्ष इव॒ दृश्य॑ते। ह्लादयते ज्वल॑तश्चै॒व। 
शा॒म्यत॑श्चास्य॒ चक्षु॑षी। या वै प्रजा भ्र॑श्य॒न्ते। 
संवत्सरात्ता भ्र॑श्य॒न्ते। या॒ प्रति॑तिष्ठ॒न्ति। 
संवत्सरे ताः प्रति॑तिष्ठ॒न्ति। व॒\ar{}षाभ्य॑ इत्य॒र्थः॥११॥\anuvakamend


अक्षि॑दु॒खोत्थि॑तस्यै॒व। वि॒प्रस॑न्ने क॒नीनि॑के। 
आङ्क्ते चाद्ग॑णं ना॒स्ति। ऋ॒भूणां तन्नि॒बोध॑त। 
क॒न॒का॒भानि॑ वासा॒सि। अ॒हता॑नि नि॒बोध॑त। 
अन्नमश्नीत॑ मृज्मी॒त। अ॒हं वो॑ जीव॒नप्र॑दः। 
ए॒ता वा॒चः प्र॑युज्य॒न्ते। श॒रद्य॑त्रोप॒दृश्य॑ते॥१२॥

अभिधून्वन्तोऽभिघ्न॑न्त इ॒व। वा॒तव॑न्तो म॒रुद्ग॑णाः। 
अमुतो जेतुमिषुमु॑खमि॒व। सन्नद्धाः सह द॑दृशे॒ ह। 
अपध्वस्तैर्वस्तिव॑र्णैरि॒व। वि॒शि॒खास॑ कप॒र्दिनः। 
अक्रुद्धस्य योत्स्य॑मान॒स्य। क्रु॒द्धस्ये॑व॒ लोहि॑नी। 
हेमतश्चक्षु॑षी वि॒द्यात्‌। अ॒क्ष्णयो क्षिप॒णोरि॑व॥१३॥

दुर्भिक्षं देव॑लोके॒षु। म॒नूना॑मुद॒कं गृ॑हे। 
ए॒ता वा॒चः प्र॑वद॒न्तीः। वै॒द्युतो॑ यान्ति॒ शैशि॑रीः। 
ता अ॒ग्निः पव॑मना॒ अन्वैक्षत। इ॒ह जी॑वि॒कामप॑रिपश्यन्‌। 
तस्यै॒षा भव॑ति। इ॒हेह॑वः स्वत॒पसः। 
मरु॑त॒ सूर्य॑त्वचः। शर्म॑ स॒प्रथा॒ आवृ॑णे॥१४॥\anuvakamend


अति॑ता॒म्राणि॑ वासा॒सि। अ॒ष्टिव॑ज्रिश॒तघ्नि॑ च। 
विश्वे देवा विप्र॑हर॒न्ति। अ॒ग्निजि॑ह्वा अ॒सश्च॑त। 
नैव देवो॑ न म॒र्त्यः। न राजा व॑रुणो॒ विभुः। 
नाग्निर्नेन्द्रो न प॑वमा॒नः। मा॒तृक्क॑च्चन॒ विद्य॑ते। 
दि॒व्यस्यैका॒ धनु॑रार्त्निः। पृ॒थि॒व्यामप॑रा श्रि॒ता॥१५॥

तस्येन्द्रो वम्रि॑रूपे॒ण। ध॒नुर्ज्या॑मछि॒नत्स्व॑यम्। 
तदि॑न्द्र॒धनु॑\-रित्य॒ज्यम्। अ॒भ्रव॑र्णेषु॒ चक्ष॑ते। 
एतदेव शंयोर्बार्‌ह॑स्पत्य॒स्य। ए॒तद्रु॑द्रस्य॒ धनुः। 
रु॒द्रस्य॑ त्वेव॒ धनु॑रार्त्निः। शिर॒ उत्पि॑पेष। 
स प्र॑व॒र्ग्यो॑ऽभवत्‌। तस्मा॒द्यः सप्र॑व॒र्ग्येण॑ य॒ज्ञेन॒ यज॑ते। 
रु॒द्रस्य॒ स शिर॒ प्रति॑दधाति। नैन रु॒द्र आरु॑को भवति। य ए॒वं वेद॑॥१६॥\anuvakamend


अ॒त्यू॒र्ध्वा॒क्षोऽति॑रश्चात्‌। शिशि॑रः प्र॒दृश्य॑ते। 
नैव रूपं न॑ वासा॒सि। न चक्षु॑ प्रति॒दृश्य॑ते। 
अ॒न्योन्यं॒ तु न॑ हिस्रा॒तः। स॒तस्त॑द्देव॒लक्ष॑णम्‌। 
लोहितोऽक्ष्णि शा॑रशी॒र्ष्णिः। सू॒र्यस्यो॑दय॒नं प्र॑ति। 
त्वं करोषि॑ न्यञ्ज॒लिकाम्‌। त्वं॒ करो॑षि नि॒जानु॑काम्‌॥१७॥

निजानुका मे न्यञ्ज॒लिका। अमी वाचमुपास॑तामि॒ति। 
तस्मै सर्व ऋतवो॑ नम॒न्ते। मर्यादाकरत्वात्प्र॑\-पुरो॒धाम्। 
ब्राह्मण॑ आप्नो॒ति। य ए॑वं वे॒द। स खलु संवत्सर एतैः सेनानी॑भिः स॒ह। 
इन्द्राय सर्वान्कामान॑भिव॒हति। स द्र॒प्सः। तस्यै॒षा भव॑ति॥१८॥

अव॑द्र॒प्सो अशु॒मती॑मतिष्ठत्‌। इ॒या॒नः कृ॒ष्णो द॒शभि॑ स॒हस्रै। 
आव॒र्तमिन्द्र॒ शच्या॒ धम॑न्तम्‌। उप्स्नुहि तं नृमणामथ॑द्रामि॒ति। 
एतयैवेन्द्रः सलावृ॑क्या स॒ह। असुरान्‌ प॑रिवृ॒श्चति। 
पृथि॑व्य॒शुम॑ती। ताम॒न्वव॑स्थितः संवत्स॒रो दि॒वं च॑। 
नैवं विदुषाऽऽचार्यान्तेवा॒सिनौ। अन्योन्यस्मै द्रुह्या॒ताम्। यो द्रु॒ह्यति। 
भ्रश्यते स्व॑र्गाल्लो॒कात्‌। इत्यृतुम॑ण्डला॒नि। 
सूर्यमण्डलान्या\-ख्या॒यिकाः। अत ऊर्ध्व सनि॑र्व॒चनाः॥१९॥\anuvakamend


आरोगो भ्राजः पटर॑ पत॒ङ्गः। स्वर्णरो ज्योतिषिमा\sn{} विभा॒सः। 
ते अस्मै सर्वे दिवमा॑तप॒न्ति। ऊर्जं दुहाना अनपस्फुर॑न्त इ॒ति। 
कश्य॑पोऽष्ट॒मः। स महामेरुं न॑ जहा॒ति। 
तस्यै॒षा भव॑ति। यत्ते॒ शिल्पं॑ कश्यप रोच॒नाव॑त्‌। 
इ॒न्द्रि॒याव॑त्पुष्क॒लं चि॒त्रभा॑नु। यस्मि॒न्त्सूर्या॒ अर्पि॑ताः स॒प्त सा॒कम्‌॥२०॥

तस्मिन्‌ राजानमधिविश्रये॑ममि॒ति। ते अस्मै सर्वे कश्यपाज्ज्यो\-ति॑र्लभ॒न्ते। 
तान्त्सोमः कश्यपादधि॑निर्द्ध॒मति। भ्रस्ताकर्मकृ॑दिवै॒वम्‌। 
प्राणो जीवानीन्द्रिय॑जीवा॒नि। सप्त शी\ur{}ष॑ण्याः प्रा॒णाः। 
सूर्या इ॑त्याचा॒र्याः। अपश्यमहमेतान्त्सप्त सूर्यानि॒ति। 
पञ्चकर्णो॑ वात्स्या॒यनः। सप्तकर्ण॑श्च प्ला॒क्षिः॥२१॥

आनुश्रविक एव नौ कश्य॑प इ॒ति। उभौ॑ वेद॒यिते। 
न हि शेकुमिव महामे॑रुं ग॒न्तुम्। अपश्यमहमेत्सूर्यमण्डलं परिव॑र्तमा॒नम्। 
गा॒र्ग्यः प्रा॑णत्रा॒तः। गच्छन्त म॑हामे॒रुम्। एकं॑ चाज॒हतम्। 
भ्राजपटरपत॑ङ्गा नि॒हने। तिष्ठन्ना॑तप॒न्ति। तस्मा॑दि॒ह तप्त्रि॑तपाः॥२२॥

अ॒मुत्रे॒तरे। तस्मा॑दि॒हातप्त्रि॑तपाः। तेषा॑मेषा॒ भव॑ति। 
स॒प्त सूर्या॒ दिव॒मनु॒प्रवि॑ष्टाः। तान॒न्वेति॑ प॒थिभि॑र्दक्षि॒णावा\sn{}। 
ते अस्मै सर्वे घृतमा॑तप॒न्ति। ऊर्जं दुहाना अनपस्फुर॑न्त इ॒ति। 
सप्तर्त्विजः सूर्या इ॑त्याचा॒र्याः। तेषा॑मेषा॒ भव॑ति। स॒प्त दिशो॒ नाना॑सूर्याः॥२३॥

स॒प्त होता॑र ऋ॒त्विज॑। देवा आदित्या॑ ये स॒प्त। 
तेभिः सोमाभी रक्ष॑ण इ॒ति। तद॑प्याम्ना॒यः। 
दिग्भ्राज ऋतून्‌ करो॒ति। एत॑यैवा॒वृता सहस्रसूर्यताया इति वै॑शम्पा॒यनः। 
तस्यै॒षा भव॑ति। यद्द्याव॑ इन्द्र ते श॒त श॒तं भूमी। 
उ॒तस्युः। नत्वा॑ वज्रिन्त्स॒हस्र॒ सूर्या॥२४॥

अनु न जातमष्ट रोद॑सी इ॒ति। नानालिङ्गत्वादृतूनां नाना॑सूर्य॒त्वम्। 
अष्टौ तु व्यवसि॑ता इ॒ति। सूर्यमण्डलान्यष्टा॑त ऊ॒र्ध्वम्‌। 
तेषा॑मेषा॒ भव॑ति। चि॒त्रं दे॒वाना॒मुद॑गा॒दनी॑कम्। 
चक्षु॑र्मि॒त्रस्य॒ वरु॑णस्या॒ग्नेः। आऽप्रा॒ द्यावा॑पृथि॒वी अ॒न्तरि॑क्षम्। 
सूर्य आत्मा जगतस्तस्थु॑षश्चे॒ति॥२५॥\anuvakamend


क्वेदमभ्रं॑ निवि॒शते। क्वाय संवत्स॒रो मि॑थः। 
क्वाहः क्वेयं दे॑व रा॒त्री। क्व मासा ऋ॑तव॒ श्रिताः। 
अर्द्धमासा॑ मुहू॒र्ताः। निमेषास्तु॑टिभि॒ सह। 
क्वेमा आपो नि॑विश॒न्ते। य॒दीतो॑ यान्ति॒ सम्प्र॑ति। 
काला अप्सु नि॑विश॒न्ते। आ॒पः सूर्ये॑ स॒माहि॑ताः॥२६॥

अभ्राण्य॒पः प्र॑पद्य॒न्ते। वि॒द्युत्सूर्ये॑ स॒माहि॑ता। 
अनवर्णे इ॑मे भू॒मी। इ॒यं चा॑ऽसौ च॒ रोद॑सी। 
किस्विदत्रान्त॑रा भू॒तम्। ये॒नेमे वि॑धृते॒ उभे। 
वि॒ष्णुना॑ विधृ॑ते भू॒मी। इ॒ति व॑त्सस्य॒ वेद॑ना। 
इरा॑वती धेनु॒मती॒ हि भू॒तम्‌। सू॒य॒व॒सिनी॒ मनु॑षे दश॒स्ये॥२७॥

व्य॑ष्टभ्ना॒द्रोद॑सी॒ विष्ण॑वे॒ते। दा॒धर्थ॑ पृथि॒वीम॒भितो॑ म॒यूखै। 
किं तद्विष्णोर्ब॑ल\-मा॒हुः। का॒ दीप्ति॑ किं प॒राय॑णम्‌। 
एको॑ य॒द्धार॑यद्दे॒वः। रे॒जती॑ रोद॒सी उ॑भे। 
वाताद्विष्णोर्ब॑लमा॒हुः। अ॒क्षराद्दीप्ति॒रुच्य॑ते। 
त्रि॒पदा॒द्धार॑यद्दे॒वः। यद्विष्णो॑रेक॒मुत्त॑मम्‌॥२८॥

अ॒ग्नयो॑ वाय॑वश्चै॒व। ए॒तद॑स्य प॒राय॑णम्‌। 
पृच्छामि त्वा प॑रं मृ॒त्युम्‌। अ॒वमं॑ मध्य॒मञ्च॑तुम्‌। 
लो॒कं च॒ पुण्य॑पापा॒नाम्‌। ए॒तत्पृ॑च्छामि॒ सम्प्र॑ति। 
अ॒मुमा॑हुः प॑रं मृ॒त्युम्‌। प॒वमा॑नं तु॒ मध्य॑मम्‌। 
अ॒ग्निरे॒वाव॑मो मृ॒त्युः। च॒न्द्रमाश्चतु॒रुच्य॑ते॥२९॥

अ॒ना॒भो॒गाः प॑रं मृ॒त्युम्‌। पा॒पा सं॑यन्ति॒ सर्व॑दा। 
आभोगास्त्वेव॑ संय॒न्ति। य॒त्र पु॑ण्यकृ॒तो ज॑नाः। 
ततो॑ म॒ध्यम॑माय॒न्ति। च॒तुम॑ग्निं च॒ सम्प्र॑ति। 
पृच्छामि त्वा॑ पाप॒कृतः। य॒त्र या॑तय॒ते य॑मः। 
त्वं नस्तद्ब्रह्म॑न्‌ प्रब्रू॒हि। य॒दि वेत्थाऽस॒तो गृ॑हान्‌॥३०॥

क॒श्यपा॑दुदि॑ताः सू॒र्याः। पा॒पान्नि॑र्घ्नन्ति॒ सर्व॑दा। 
रोदस्योन्त॑र्दे\-शे॒षु। तत्र न्यस्यन्ते॑ वास॒वैः। 
तेऽशरीराः प्र॑पद्य॒न्ते। य॒था\-ऽपु॑ण्यस्य॒ कर्म॑णः। 
अपाण्य॒पाद॑केशा॒सः। त॒त्र ते॑ऽयोनि॒जा ज॑नाः। 
मृत्वा पुनर्मृत्युमा॑पद्य॒न्ते। अ॒द्यमा॑नाः स्व॒कर्म॑भिः॥३१॥

आशातिकाः क्रिम॑य इ॒व। ततः पूयन्ते॑ वास॒वैः। अपै॑तं मृ॒त्युं ज॑यति। 
य ए॒वं वेद॑। स खल्वैवं॑ विद्ब्रा॒ह्मणः। दी॒र्घश्रु॑त्तमो॒ भव॑ति। 
कश्य॑प॒स्याति॑थि॒ सिद्धग॑मन॒ सिद्धाग॑मनः। तस्यै॒षा भव॑ति। 
आयस्मिन्त्स॒प्त वा॑स॒वाः। रोह॑न्ति पू॒र्व्या॑ रुह॑॥३२॥

ऋषि॑र्‌ह दीर्घ॒श्रुत्त॑मः। इन्द्रस्य घर्मो अति॑थिरि॒ति। 
कश्यपः पश्य॑को भ॒वति। यत्सर्वं परिपश्यती॑ति सौ॒क्ष्म्यात्‌। 
अथाग्ने॑रष्टपु॑रुष॒स्य। तस्यै॒षा भव॑ति। 
अग्ने॒ नय॑ सु॒पथा॑ रा॒ये अ॒स्मान्‌। विश्वा॑नि देव व॒युना॑नि वि॒द्वान्‌। 
यु॒यो॒ध्य॑स्मज्जु॑हुरा॒णमेन॑। भूयिष्ठां ते नम उक्तिं वि॑धेमे॒ति॥३३॥\anuvakamend


अग्निश्च जात॑वेदा॒श्च। सहोजा अ॑जिरा॒प्रभुः। वैश्वानरो न॑र्यापा॒श्च। 
प॒ङ्क्तिरा॑धाश्च॒ सप्त॑मः। विसर्पेवाऽष्ट॑मोऽग्नी॒नाम्‌। 
एतेऽष्टौ वसवः, क्षि॑ता इ॒ति। यथर्त्वेवाग्नेरर्चिर्वर्ण॑विशे॒षाः। 
नीलार्चिश्च पीतकार्चिश्चे॒ति। अथ वायोरेकादशपुरुषस्यैका\-दश॑स्त्रीक॒स्य। 
प्रभ्राजमाना व्य॑वदा॒ताः॥३४॥


याश्च वासु॑किवै॒द्युताः। रजताः परु॑षाः श्या॒माः। कपिला अ॑तिलो॒हिताः। 
ऊर्ध्वा अवप॑तन्ता॒श्च। वैद्युत इ॑त्येका॒दश। नैनं वैद्युतो॑ हिन॒स्ति। 
य ए॑वं वे॒द। स होवाच व्यासः पा॑राश॒र्यः। 
विद्युद्वधमेवाहं मृत्युमैच्छमि॒ति। न त्वका॑म ह॒न्ति॥३५॥


य ए॑वं वे॒द। अथ ग॑न्धर्व॒गणाः। स्वान॒भ्राट्‌। 
अङ्घा॑रि॒र्बम्भा॑रिः। हस्त॒ सुह॑स्तः। कृशा॑नुर्वि॒श्वाव॑सुः। 
मूर्धन्वान्त्सूर्यव॒र्चाः। कृतिरित्येकादश ग॑न्धर्व॒गणाः। 
देवाश्च म॑हादे॒वाः। रश्मयश्च देवा॑ गर॒गिरः॥३६॥


नैनं गरो॑ हिन॒स्ति। य ए॑वं वे॒द। 
गौ॒री मि॑माय सलि॒लानि॒ तक्ष॑ती। एक॑पदी द्वि॒पदी॒ सा चतु॑ष्पदी। 
अ॒ष्टाप॑दी॒ नव॑पदी बभू॒वुषी। सहस्राक्षरा परमे व्यो॑मन्नि॒ति। 
वाचो॑ विशे॒षणम्‌। अथ निगद॑व्याख्या॒ताः। 
ताननुक्र॑मिष्या॒मः। व॒राहव॑ स्वत॒पसः॥३७॥

वि॒द्युन्म॑हसो॒ धूप॑यः। श्वापयो गृहमेधाश्चेत्ये॒ते। 
ये॒ चेमेऽशि॑मिवि॒\-द्विषः। पर्जन्याः सप्त पृथिवीमभिव॑र्‌ष॒न्ति। 
वृष्टि॑भिरि॒ति। एतयैव विभक्तिवि॑परी॒ताः। स॒प्तभि॒र्वा तै॑रुदी॒रिताः। 
अमूँल्लोकान\-भिव॑\ur{}ष॒न्ति। तेषा॑मेषा॒ भव॑ति। स॒मा॒नमे॒तदुद॑कम्‌॥३८॥

उ॒च्चैत्य॑व॒चाह॑भिः। भूमिं॑ प॒र्जन्या॒ जिन्व॑न्ति। दिवं जिन्वन्त्यग्न॑य इ॒ति। 
यदक्ष॑रं भू॒तकृ॑तम्‌। विश्वे॑ देवा उ॒पास॑ते। म॒हर्\mbox{}षि॑मस्य गो॒प्तारम्। 
ज॒मद॑ग्नि॒मकु॑र्वत। ज॒मद॑ग्नि॒\-राप्या॑यते। 
छन्दो॑भिश्चतुरुत्त॒रैः। राज्ञ॒ सोम॑स्य तृ॒प्तास॑॥३९॥

ब्रह्म॑णा वी॒र्या॑वता। शि॒वा न॑ प्र॒दिशो॒ दिश॑। 
तच्छं॒ योरावृ॑णीमहे। गा॒तुं य॒ज्ञाय॑। गा॒तुं य॒ज्ञप॑तये। 
दैवी स्व॒स्तिर॑स्तु नः। स्व॒स्तिर्मानु॑षेभ्यः। ऊ॒र्ध्वं जि॑गातु भेष॒जम्। 
शं नो॑ अस्तु द्वि॒पदे। शं चतु॑ष्पदे। 
सोमपा (३) असोमपा (३) इति निगद॑व्याख्या॒ताः॥४०॥\anuvakamend


स॒ह॒स्र॒वृदि॑यं भू॒मिः। प॒रं व्यो॑म स॒हस्र॑वृत्‌। अ॒श्विना॑ भुज्यू॑नास॒त्या। 
वि॒श्वस्य॑ जग॒तस्प॑ती। जाया भूमिः प॑तिर्व्यो॒म। मि॒थुन॑न्ता अ॒तुर्य॑थुः। 
पुत्रो बृहस्प॑ती रु॒द्रः। स॒रमा॑ इति॑ स्त्रीपु॒मम्‌। 
शु॒क्रं वा॑म॒न्यद्य॑ज॒तं वा॑म॒न्यत्‌। विषु॑रूपे॒ अह॑नी॒ द्यौरि॑व स्थः॥४१॥



विश्वा॒ हि मा॒या अव॑थः स्वधावन्तौ। 
भ॒द्रा वां पूषणावि॒ह रा॒तिर॑स्तु। वासात्यौ चि॒त्रौ जग॑तो नि॒धानौ। 
द्यावा॑भूमी च॒रथ॑ स॒ सखा॑यौ। ताव॒श्विना॑ रा॒सभाश्वा॒ हवं॑ मे। 
शु॒भ॒स्प॒ती॒ आ॒गत सू॒र्यया॑ स॒ह। त्युग्रो॑ह भु॒ज्युम॑श्विनोदमे॒घे। 
र॒यिं न कश्चि॑न्ममृ॒वां (२) अवा॑हाः। तमू॑हथुर्नौ॒भिरात्म॒न्वती॑भिः। 
अ॒न्त॒रि॒क्ष॒प्रुड्भि॒रपो॑दकाभिः॥४२॥


ति॒स्रः, क्षप॒स्त्रिरहा॑ति॒व्रज॑द्भिः। नास॑त्या भु॒ज्युमू॑हथुः पत॒ङ्गैः। 
स॒मु॒द्रस्य॒ धन्व॑न्ना॒र्द्रस्य॑ पा॒रे। त्रि॒भीरथै श॒तप॑द्भि॒ षड॑श्वैः। 
स॒वि॒तारं॒ वित॑न्वन्तम्‌। अनु॑बध्नाति शाम्ब॒रः। आपपू\ur{}षम्ब॑रश्चै॒व। 
स॒विता॑रेप॒सो॑ऽभवत्‌। त्य सुतृप्तं वि॑दित्वै॒व। ब॒हुसो॑म गि॒रं व॑शी॥४३॥

अन्वेति तुग्रो व॑क्रिया॒न्तम्‌। आयसूयान्त्सोम॑तृप्सु॒षु। स \linebreak सङ्ग्रामस्तमोद्योऽत्यो॒तः। 
वाचो गाः पि॑पाति॒ तत्‌। स तद्गोभिः स्तवाऽत्येत्य॒न्ये। र॒क्षसा॑नन्वि॒ताश्च॑ ये। 
अ॒न्वेति॒ परि॑वृत्या॒ऽस्तः। ए॒वमे॒तौ स्थो॑ अश्विना। 
ते ए॒ते द्यु॑ पृथि॒व्योः। अह॑रह॒र्गर्भं॑ दधाथे॥४४॥

तयो॑रे॒तौ व॒त्साव॑होरा॒त्रे। पृ॒थि॒व्या अह॑। दि॒वो रात्रि॑। 
ता अवि॑सृष्टौ। दम्प॑ती ए॒व भ॑वतः। तयो॑रे॒तौ व॒त्सौ। 
अ॒ग्निश्चा॑दि॒त्य॒श्च॑। रा॒त्रेर्व॒त्सः। श्वे॒त आ॑दि॒त्यः। अह्नो॒ऽग्निः॥४५॥

ता॒म्रो अ॑रु॒णः। ता अवि॑सृष्टौ। दम्प॑ती ए॒व भ॑वतः। 
तयो॑रे॒तौ व॒त्सौ। वृ॒त्रश्च॑ वैद्यु॒तश्च॑। अ॒ग्नेर्वृ॒त्रः। वै॒द्युत॑ आदि॒त्यस्य॑। 
ता अवि॑सृष्टौ। दम्प॑ती ए॒व भ॑वतः। तयो॑रे॒तौ व॒त्सौ॥४६॥


उ॒ष्मा च॑ नीहा॒रश्च॑। वृ॒त्रस्यो॒ष्मा। वै॒द्यु॒तस्य॑ नीहा॒रः। 
तौ तावे॒व प्रति॑पद्येते। सेय रात्री॑ ग॒र्भिणी॑ पु॒त्रेण॒ संव॑सति। 
तस्या॒ वा ए॒तदु॒ल्बणम्‌। यद्रात्रौ॑ र॒श्मय॑। 
यथा॒ गोर्ग॒र्भिण्या॑ उ॒ल्बणम्‌। ए॒वमे॒तस्या॑ उ॒ल्बणम्‌। 
प्रजयिष्णुः प्रजया च पशुभि॑श्च भ॒वति। 
य ए॑वं वे॒द। एतमुद्यन्तमपिय॑न्तं चे॒ति। 
आदित्यः पुण्य॑स्य व॒त्सः। अथ पवि॑त्राङ्गि॒रसः॥४७॥\anuvakamend


प॒वित्र॑वन्त॒ परि॒वाज॒मास॑ते। पि॒तैषां प्र॒त्नो अ॒भिर॑क्षति व्र॒तम्‌। 
म॒हः स॑मु॒द्रं वरु॑णस्ति॒रोद॑धे। धीरा॑ इच्छेकु॒र्धरु॑णेष्वा॒रभम्‌। 
प॒वित्रं॑ ते॒ वित॑तं॒ ब्रह्म॑ण॒स्पते। प्रभु॒र्गात्रा॑णि॒ पर्ये॑षिवि॒श्वत॑। 
अत॑प्ततनू॒र्न तदा॒मो अ॑श्नुते। शृ॒तास॒ इद्वह॑न्त॒स्तत्समा॑शत। 
ब्र॒ह्मा दे॒वानाम्‌। अस॑तः स॒द्ये तत॑क्षुः॥४८॥


ऋष॑यः स॒प्तात्रि॑श्च॒ यत्‌। सर्वेऽत्रयो अ॑गस्त्य॒श्च। 
नक्ष॑त्रै॒ शङ्कृ॑तोऽवसन्‌। अथ॑ सवितु॒ श्यावाश्व॒स्याऽवर्ति॑कामस्य। 
अ॒मी य ऋक्षा॒ निहि॑तास उ॒च्चा। नक्तं॒ ददृ॑श्रे॒ कुह॑चि॒द्दिवे॑युः। 
अद॑ब्धानि॒ वरु॑णस्य व्र॒तानि॑। वि॒चा॒कश॑च्च॒न्द्रमा॒ नक्ष॑त्रमेति। 
तत्स॑वि॒तुर्वरेण्यम्। भर्गो॑ दे॒वस्य॑ धीमहि॥४९॥


धियो॒ यो न॑ प्रचो॒दयात्‌। तत्स॑वि॒तुर्वृ॑णीमहे। 
व॒यं दे॒वस्य॒ भोज॑नम्‌। श्रेष्ठ सर्व॒धात॑मम्‌। 
तुरं॒ भग॑स्य धीमहि। अपा॑गूहत सविता॒ तृभी\sn। 
सर्वान्दि॒वो अन्ध॑सः। नक्तं॒ तान्य॑\-भवन्दृ॒शे। 
अस्थ्य॒स्थ्ना सम्भ॑विष्यामः। नाम॒ नामै॒व ना॒म मे॥५०॥


नपुस॑कं॒ पुमा॒स्त्र्य॑स्मि। स्थाव॑रोऽस्म्यथ॒ जङ्ग॑मः। 
य॒जेऽयक्षि॒ यष्टा॒हे च॑। मया॑ भू॒तान्य॑यक्षत। 
प॒शवो॑ मम॑ भूता॒नि। अनूबन्ध्योऽस्म्य॑हं वि॒भुः। 
स्त्रिय॑ स॒तीः। ता उ॑मे पु॒स आ॑हुः। 
पश्य॑दक्ष॒ण्वान्नविचे॑तद॒न्धः। क॒विर्यः पु॒त्र स इ॒मा चि॑केत॥५१॥


यस्ता वि॑जा॒नात्स॑वि॒तुः पि॒तास॑त्‌। अ॒न्धो मणिम॑विन्दत्‌। 
तम॑नङ्गुलि॒राव॑यत्‌। अ॒ग्री॒वः प्रत्य॑मुञ्चत्‌। 
तमजि॑ह्वा अ॒सश्च॑त। ऊर्ध्वमूलम॑वाक्छा॒खम्‌। 
वृ॒क्षं यो॑ वेद॒ सम्प्र॑ति। न स जातु जन॑ श्रद्द॒ध्यात्‌। 
मृ॒त्युर्मा॑ मार॒यादि॑तिः। हसित रुदि॑तं गी॒तम्‌॥५२॥


वीणा॑पणव॒लासि॑तम्‌। मृ॒तं जी॒वं च॑ यत्कि॒ञ्चित्। 
अ॒ङ्गानि॑ स्नेव॒ विद्धि॑ तत्‌। अतृ॑ष्य॒स्तृष्य॑ध्यायत्‌। 
अ॒स्माज्जा॒ता मे॑ मिथू॒ चर\snn। पुत्रो निर्‌ऋत्या॑ वैदे॒हः। 
अ॒चेता॑ यश्च॒ चेत॑नः। स॒ तं मणिम॑विन्दत्‌। 
सो॑ऽनङ्गुलि॒राव॑यत्‌। सो॒\aav{}ग्री॒वः प्रत्य॑मुञ्चत्‌॥५३॥


सोऽजि॑ह्वो अ॒सश्च॑त। नैतमृषिं विदित्वा नग॑रं प्र॒विशेत्‌। 
य॑दि प्र॒विशेत्‌। मि॒थौ चरि॑त्वा प्र॒विशेत्‌। 
तत्सम्भव॑स्य व्र॒तम्‌। आ॒तम॑ग्ने र॒थं ति॑ष्ठ। 
एकाश्वमेक॒योज॑नम्‌। एकचक्र॑मेक॒धुरम्‌। 
वा॒तध्रा॑जिग॒तिं वि॑भो। न॒ रि॒ष्यति॑ न व्य॒थते॥५४॥


ना॒स्याक्षो॑ यातु॒ सज्ज॑ति। यच्छ्वेतान्‌ रोहि॑ताश्चा॒ग्नेः। 
र॒थे यु॑क्त्वाऽधि॒\-तिष्ठ॑ति। एकया च दशभिश्च॑ स्वभू॒ते। 
द्वाभ्यामिष्टये विशत्या॒ च। तिसृभिश्च वहसे त्रिशता॒ च। 
नियुद्भिर्वायविह ता॑ विमु॒ञ्च॥५५॥\anuvakamend


आत॑नुष्व॒ प्रत॑नुष्व। उ॒द्धमाऽऽध॑म॒ सन्ध॑म। 
आदित्ये चन्द्र॑वर्णा॒नाम्‌। गर्भ॒माधे॑हि॒ यः पुमा\sn{}। 
इ॒तः सि॒क्त सूर्य॑गतम्‌। च॒न्द्रम॑से॒ रसं॑ कृधि। 
वारादं जन॑याग्रे॒ऽग्निम्‌। य एको॑ रुद्र॒ उच्य॑ते। 
अ॒स॒ङ्ख्या॒ताः स॑हस्रा॒णि। स्म॒र्यते॑ न च॒ दृश्य॑ते॥५६॥


ए॒वमे॒तं नि॑बोधत। आम॒न्द्रैरि॑न्द्र॒ हरि॑भिः। 
या॒हि म॒यूर॑रोमभिः। मा त्वा केचिन्नियेमुरि॑न्न पा॒शिनः। 
द॒ध॒न्वेव॒ ता इ॑हि। मा म॒न्द्रैरि॑न्द्र॒ हरि॑भिः। 
या॒मि म॒यूर॑रोमभिः। मा मा केचिन्नियेमुरि॑न्न पा॒शिनः। 
नि॒ध॒न्वेव॒ \mbox{तां (२)} इ॑मि। अणुभिश्च म॑हद्भि॒श्च॥५७॥


नि॒घृष्वै॑रस॒मायु॑तैः। कालैर्‌हरित्व॑माप॒न्नैः। 
इन्द्राऽऽया॑हि स॒हस्र॑युक्‌। अ॒ग्निर्वि॒भ्राष्टि॑वसनः। 
वा॒युः श्वेत॑सिकद्रु॒कः। सं॒व॒त्स॒रो वि॑षू॒वर्णै। 
नित्या॒स्तेऽनुच॑रास्त॒व। सुब्रह्मण्यो सुब्रह्मण्यो सु॑ब्रह्म॒ण्योम्‌। 
इन्द्राऽऽगच्छ हरिव आगच्छ मे॑धाति॒थेः। मेष वृषणश्व॑स्य मे॒ने॥५८॥


गौरावस्कन्दिन्नहल्या॑यै जा॒र। कौशिकब्राह्मण गौतम॑ब्रुवा॒ण। 
अ॒रु॒णाश्वा॑ इ॒हाग॑ताः। वस॑वः पृथिवि॒क्षित॑। 
अ॒ष्टौदि॒ग्वास॑सो॒\-ऽग्नय॑। अग्निश्च जातवेदाश्चेत्ये॒ते। 
ताम्राश्वास्ताम्र॒रथाः। ताम्रवर्णास्तथा॒\-ऽसिताः। 
दण्डहस्ता खाद॒ग्दतः। इतो रुद्रा परा॒ङ्गताः॥५९॥


उक्त स्थानं प्रमाणं च॑ पुर॒ इत। बृह॒स्पति॑श्च सवि॒ता च॑। 
वि॒श्वरू॑पैरि॒हाऽऽग॑ताम्‌। रथे॑नोदक॒वर्त्म॑ना। 
अ॒प्सुषा॑ इति॒ तद्द्व॑योः। उक्तो वेषो॑ वासा॒सि च। 
कालावयवानामित॑ प्रती॒ज्या। वासात्या॑ इत्य॒श्विनोः। 
कोऽन्तरिक्षे शब्दं क॑रोती॒ति। वासिष्टो रौहिणो मीमासां च॒क्रे। 
तस्यै॒षा भव॑ति। वा॒श्रेव॑ वि॒द्युदिति॑। 
ब्रह्म॑ण उ॒दर॑णमसि। ब्रह्म॑ण उदी॒रण॑मसि। 
ब्रह्म॑ण आ॒स्तर॑णमसि। ब्रह्म॑ण उप॒स्तर॑णमसि॥६०॥\anuvakamend


[अप॑क्रामत गर्भि॒ण्य॑]\\
अ॒ष्टयो॑नीम॒ष्टपु॑त्राम्‌। अ॒ष्टप॑त्नीमि॒मां महीम्‌। 
अ॒हं वेद॒ न मे॑ मृत्युः। न चामृ॑त्युर॒घाऽऽह॑रत्‌। 
अ॒ष्टयोन्य॒ष्टपु॑त्रम्‌। अ॒ष्टप॑दि॒दम॒न्तरि॑क्षम्‌। 
अ॒हं वेद॒ न मे॑ मृत्युः। न चामृ॑त्युर॒घाऽऽह॑रत्‌। 
अ॒ष्टयो॑नीम॒ष्टपु॑त्राम्‌। अ॒ष्टप॑त्नीम॒मूं दिवम्‌॥६१॥


अ॒हं वेद॒ न मे॑ मृत्युः। न चामृ॑त्युर॒घाऽऽह॑रत्‌। 
सु॒त्रामा॑णं म॒हीमू॒ षु। अदि॑ति॒र्द्यौरदि॑तिर॒न्तरि॑क्षम्। 
अदि॑तिर्मा॒ता स पि॒ता स पु॒त्रः। विश्वे॑ दे॒वा अदि॑ति॒ पञ्च॒जना। 
अदि॑तिर्जा॒तमदि॑ति॒र्जनि॑त्वम्‌। अ॒ष्टौ पु॒त्रासो॒ अदि॑तेः। 
ये जा॒तास्त॒न्व॑ परि॑। दे॒वां (२) उप॑प्रैत्स॒प्तभि॑॥६२॥


प॒रा॒ मा॒र्ता॒ण्डमास्य॑त्‌। स॒प्तभि॑ पु॒त्रैरदि॑तिः। 
उप॒प्रैत्पू॒र्व्यं॑ युगम्। प्र॒जायै॑ मृ॒त्यवे त॑त्‌। 
प॒रा॒ मा॒र्ता॒ण्डमाभ॑र॒दिति॑। ताननुक्र॑मिष्या॒मः। 
मि॒त्रश्च॒ वरु॑णश्च। धा॒ता चार्य॒मा च॑। 
अश॑श्च॒ भग॑श्च। इन्द्रश्च विवस्वाश्चेत्ये॒ते। 
हि॒र॒ण्य॒ग॒र्भो ह॒सः शु॑चि॒षत्‌। 
ब्रह्म॑जज्ञा॒नं तदित्प॒दमिति॑। ग॒र्भः प्रा॑जाप॒त्यः। 
अथ॒ पुरु॑षः स॒प्त पुरु॑षः॥६३॥\\\mbox{}
[य॒था॒स्था॒नं ग॑र्भि॒ण्य॑]\anuvakamend

योऽसौ॑ त॒पन्नु॒देति॑। स सर्वे॑षां भू॒तानां प्रा॒णाना॒दायो॒देति॑। 
मा मे प्र॒जाया॒ मा प॑शू॒नाम्‌। मा मम॑ प्रा॒णाना॒दायोद॑गाः। 
अ॒सौ योऽस्त॒मेति॑। स सर्वे॑षां भू॒तानां प्रा॒णाना॒दाया॒ऽस्तमेति॑। 
मा मे प्र॒जाया॒ मा प॑शू॒नाम्‌। मा मम॑ प्रा॒णाना॒दायाऽस्त॑ङ्गाः। 
अ॒सौ य आ॒पूर्य॑ति। स सर्वे॑षां भू॒तानां प्रा॒णैरा॒पूर्य॑ति॥६४॥


मा मे प्र॒जाया॒ मा प॑शू॒नाम्‌। मा मम॑ प्रा॒णैरा॒पूरि॑ष्ठाः। 
अ॒सौ यो॑ऽप॒क्षीय॑ति। स सर्वे॑षां भू॒तानां प्रा॒णैरप॑क्षीयति। 
मा मे प्र॒जाया॒ मा प॑शू॒नाम्‌। मा मम॑ प्रा॒णैरप॑क्षेष्ठाः। 
अ॒मूनि॒ नक्ष॑त्राणि। सर्वे॑षां भू॒तानां प्रा॒णैरप॑प्रसर्पन्ति॒ चोत्स॑र्पन्ति च। 
मा मे प्र॒जाया॒ मा प॑शू॒नाम्‌। मा मम॑ प्रा॒णैरप॑प्रसृपत॒ मोत्सृ॑पत॥६५॥


इ॒मे मासाश्चार्धमा॒साश्च॑। सर्वे॑षां भू॒तानां प्रा॒णैरप॑प्रसर्पन्ति॒ चोत्स॑र्पन्ति च। 
मा मे प्र॒जाया॒ मा प॑शू॒नाम्‌। मा मम॑ प्रा॒णैरप॑प्रसृपत॒ मोत्सृ॑पत। 
इ॒म ऋ॒तव॑। सर्वे॑षां भू॒तानां प्रा॒णैरप॑प्रसर्पन्ति॒ चोत्स॑र्पन्ति च। 
मा मे प्र॒जाया॒ मा प॑शू॒नाम्‌। मा मम॑ प्रा॒णैरप॑प्रसृपत॒ मोत्सृ॑पत। 
अ॒य सं॑वत्स॒रः। सर्वे॑षां भू॒तानां प्रा॒णैरप॑प्रसर्पति॒ चोत्स॑र्पति च॥६६॥


मा मे प्र॒जाया॒ मा प॑शू॒नाम्‌। मा मम॑ प्रा॒णैरप॑प्रसृप॒ मोत्सृ॑प। 
इ॒दमह॑। सर्वे॑षां भू॒तानां प्रा॒णैरप॑प्रसर्पति॒ चोत्स॑र्पति च। 
मा मे प्र॒जाया॒ मा प॑शू॒नाम्‌। मा मम॑ प्रा॒णैरप॑प्रसृप॒ मोत्सृ॑प। 
इ॒य रात्रि॑। सर्वे॑षां भू॒तानां प्रा॒णैरप॑प्रसर्पति॒ चोत्स॑र्पति च। 
मा मे प्र॒जाया॒ मा प॑शू॒नाम्‌। मा मम॑ प्रा॒णैरप॑प्रसृप॒ मोत्सृ॑प। 
ॐ भूर्भुव॒ स्व॑। एतद्वो मिथुनं मा नो मिथु॑न री॒ढ्वम्॥६७॥\anuvakamend


अथाऽऽदित्यस्याष्टपु॑रुष॒स्य। 
वसूनामादित्याना स्थाने स्वतेज॑सा भा॒नि। 
रुद्राणामादित्याना स्थाने स्वतेज॑सा भा॒नि। 
आदित्यानामादि\-त्याना स्थाने स्वतेज॑सा भा॒नि। 
सता सत्या॒नाम्‌। आदित्याना स्थाने स्वतेज॑सा भा॒नि। 
अभिधून्वता॑\-मभि॒घ्नताम्‌। वातव॑तां म॒रुताम्‌। 
आदित्याना स्थाने स्वतेज॑सा भा॒नि। 
ऋभूणामादित्याना स्थाने स्वतेज॑सा भा॒नि। 
विश्वेषां देवा॒नाम्‌। आदित्याना स्थाने स्वतेज॑सा भा॒नि। 
संवत्सर॑स्य स॒वितुः। आदित्यस्य स्थाने स्वतेज॑सा भा॒नि। 
ॐ भूर्भुव॒ स्व॑। रश्मयो वो मिथुनं मा नो मिथु॑न री॒ढ्वम्॥६८॥\anuvakamend


आरोगस्य स्थाने स्वतेज॑सा भा॒नि। भ्राजस्य स्थाने स्वतेज॑सा भा॒नि। 
पटरस्य स्थाने स्वतेज॑सा भा॒नि। पतङ्गस्य स्थाने स्वतेज॑सा भा॒नि। 
स्वर्णरस्य स्थाने स्वतेज॑सा भा॒नि। ज्योतिषीमतस्य स्थाने स्वतेज॑सा भा॒नि। 
विभासस्य स्थाने स्वतेज॑सा भा॒नि। कश्यपस्य स्थाने स्वतेज॑सा भा॒नि। 
ॐ भूर्भुव॒ स्व॑। आपो वो मिथुनं मा नो मिथु॑न री॒ढ्वम्॥६९॥\anuvakamend


अथ वायोरेकादशपुरुषस्यैकादश॑स्त्रीक॒स्य। 
प्रभ्राजमानाना रुद्राणा स्थाने स्वतेज॑सा भा॒नि। 
व्यवदाताना रुद्राणा स्थाने स्वतेज॑सा भा॒नि। 
वासुकिवैद्युताना रुद्राणा स्थाने स्वतेज॑सा भा॒नि। 
रजताना रुद्राणा स्थाने स्वतेज॑सा भा॒नि। 
परुषाणा रुद्राणा स्थाने स्वतेज॑सा भा॒नि। 
श्यामाना रुद्राणा स्थाने स्वतेज॑सा भा॒नि। 
कपिलाना रुद्राणा स्थाने स्वतेज॑सा भा॒नि। 
अतिलोहिताना रुद्राणा स्थाने स्वतेज॑सा भा॒नि। 
ऊर्ध्वाना रुद्राणा स्थाने स्वतेज॑सा भा॒नि॥७०॥


अवपतन्ताना रुद्राणा स्थाने स्वतेज॑सा भा॒नि। 
वैद्युताना रुद्राणा स्थाने स्वतेज॑सा भा॒नि। 
प्रभ्राजमानीना रुद्राणीना स्थाने स्वतेज॑सा भा॒नि। 
व्यवदातीना रुद्राणीना स्थाने स्वतेज॑सा भा॒नि। 
वासुकिवैद्युतीना रुद्राणीना स्थाने स्वतेज॑सा भा॒नि। 
रजताना रुद्राणीना स्थाने स्वतेज॑सा भा॒नि। 
परुषाणा रुद्राणीना स्थाने स्वतेज॑सा भा॒नि। 
श्यामाना रुद्राणीना स्थाने स्वतेज॑सा भा॒नि। 
कपिलाना रुद्राणीना स्थाने स्वतेज॑सा भा॒नि। 
अतिलोहितीना रुद्राणीना स्थाने स्वतेज॑सा भा॒नि। 
ऊर्ध्वाना रुद्राणीना स्थाने स्वतेज॑सा भा॒नि। 
अवपतन्तीना रुद्राणीना स्थाने स्वतेज॑सा भा॒नि। 
वैद्युतीना रुद्राणीना स्थाने स्वतेज॑सा भा॒नि। 
ॐ भूर्भुव॒ स्व॑। रूपाणि वो मिथुनं मा नो मिथु॑न री॒ढ्वम्॥७१॥\anuvakamend


अथाग्ने॑रष्टपु॑रुष॒स्य। 
अग्नेः पूर्वदिश्यस्य स्थाने स्वतेज॑सा भा॒नि। 
जातवेदस उपदिश्यस्य स्थाने स्वतेज॑सा भा॒नि। 
सहोजसो दक्षिणदिश्यस्य स्थाने स्वतेज॑सा भा॒नि। 
अजिराप्रभव उपदिश्यस्य स्थाने स्वतेज॑सा भा॒नि। 
वैश्वानरस्यापरदिश्यस्य स्थाने स्वतेज॑सा भा॒नि। 
नर्यापस उपदिश्यस्य स्थाने स्वतेज॑सा भा॒नि। 
पङ्क्तिराधस उदग्दिश्यस्य स्थाने स्वतेज॑सा भा॒नि। 
विसर्पिण उपदिश्यस्य स्थाने स्वतेज॑सा भा॒नि। 
ॐ भूर्भुव॒ स्व॑। दिशो वो मिथुनं मा नो मिथु॑न री॒ढ्वम्॥७२॥\anuvakamend


दक्षिणपूर्वस्यां दिशि विस॑र्पी न॒रकः। तस्मान्नः प॑रिपा॒हि। 
दक्षिणापरस्यां दिश्यविस॑र्पी न॒रकः। तस्मान्नः प॑रिपा॒हि। 
उत्तरपूर्वस्यां दिशि विषा॑दी न॒रकः। तस्मान्नः प॑रिपा॒हि। 
उत्तरापरस्यां दिश्यविषा॑दी न॒रकः। तस्मान्नः प॑रिपा॒हि। 
आ यस्मिन्त्सप्त वासवा इन्द्रियाणि शतक्रत॑वित्ये॒ते॥७३॥\anuvakamend


इ॒न्द्र॒घो॒षा वो॒ वसु॑भिः पु॒रस्ता॒दुप॑दधताम्‌। 
मनो॑जवसो वः पि॒तृभि॑र्दक्षिण॒त उप॑दधताम्‌। 
प्रचे॑ता वो रु॒द्रैः प॒श्चादुप॑दधताम्‌। 
वि॒श्वक॑र्मा व आदि॒त्यैरु॑त्तर॒त उप॑दधताम्‌। 
त्वष्टा॑ वो रू॒पैरु॒परि॑ष्टा॒दुप॑\-दधताम्‌। 
संज्ञानं वः प॑श्चादि॒ति। आ॒दि॒त्य सर्वो॒ऽग्निः पृ॑थि॒व्याम्‌। 
वा॒युर॒न्तरि॑क्षे। सूर्यो॑ दि॒वि। च॒न्द्रमा॑ दि॒क्षु। 
नक्ष॑त्राणि॒ स्वलो॒के। ए॒वा ह्ये॑व। ए॒वा ह्य॑ग्ने। 
ए॒वा हि वा॑यो। ए॒वा हीन्द्र। ए॒वा हि पू॑षन्‌। ए॒वा हि दे॑वाः॥७४॥\anuvakamend


आप॑मापाम॒पः सर्वा। अ॒स्माद॒स्मादि॒तोऽमुत॑।
अ॒ग्निर्वा॒युश्च॒ सूर्य॑श्च। स॒ह स॑ञ्चस्क॒रर्द्धि॑या। 
वा॒य्वश्वा॑ रश्मि॒पत॑यः। मरीच्यात्मानो॒ अद्रु॑हः। 
दे॒वीर्भु॑वन॒सूव॑रीः। पु॒त्र॒व॒त्वाय॑ मे सुत। 
महानाम्नीर्म॑हामा॒नाः। म॒ह॒सो म॑हस॒ स्व॑॥७५॥


दे॒वीः प॑र्जन्य॒सूव॑रीः। पु॒त्र॒व॒त्वाय॑ मे सुत। 
अ॒पाश्न्यु॑ष्णिम॒पा रक्ष॑। अ॒पाश्न्यु॑ष्णि\-म॒पारघम्। 
अपाघ्रा॒मप॑चा॒वर्तिम्। अप॑दे॒वीरि॒तो हि॑त। 
वज्रं॑ दे॒वीरजी॑ताश्च। भुव॑नं देव॒सूव॑रीः। 
आ॒दि॒त्यानदि॑तिं दे॒वीम्। योनि॑नोर्ध्वमु॒दीष॑त॥७६॥


भ॒द्रं कर्णे॑भिः शृणु॒याम॑ देवाः। भ॒द्रं प॑श्येमा॒क्षभि॒र्यज॑त्राः। 
स्थि॒रैरङ्गैस्तुष्टु॒वा स॑स्त॒नूभि॑। व्यशे॑म दे॒वहि॑तं॒ यदायु॑। 
स्व॒स्ति न॒ इन्द्रो॑ वृ॒द्धश्र॑वाः। स्व॒स्ति न॑ पू॒षा वि॒श्ववे॑दाः। 
स्व॒स्ति न॒स्तार्क्ष्यो॒ अरि॑ष्टनेमिः। स्व॒स्ति नो॒ बृह॒स्पति॑र्दधातु। 
के॒तवो॒ अरु॑णासश्च। ऋ॒ष॒यो वात॑रश॒नाः। 
प्र॒ति॒ष्ठा श॒तधा॑ हि। स॒माहि॑तासो सहस्र॒धाय॑सम्। 
शि॒वा न॒ शन्त॑मा भवन्तु। दि॒व्या आप॒ ओष॑धयः। 
सु॒मृ॒डी॒का सर॑स्वति। मा ते॒ व्यो॑म स॒न्दृशि॑॥७७॥\anuvakamend


यो॑ऽपां पुष्पं॒ वेद॑। पुष्प॑वान्‌ प्र॒जावान् पशु॒मान् भ॑वति। 
च॒न्द्रमा॒ वा अ॒पां पुष्पम्। पुष्प॑वान् प्र॒जावान् पशु॒मान्‌ भ॑वति। 
य ए॒वं वेद॑। यो॑ऽपामा॒यत॑नं॒ वेद॑। 
आ॒यत॑नवान्‌ भवति। अ॒ग्निर्वा अ॒पामा॒यत॑नम्। 
आ॒यत॑नवान्‌ भवति। योऽग्नेरा॒यत॑नं॒ वेद॑॥७८॥


आ॒यत॑नवान्‌ भवति। आपो॒ वा अ॒ग्नेरा॒यत॑नम्‌। 
आ॒यत॑नवान्‌ भवति। य ए॒वं वेद॑। 
यो॑ऽपामा॒यत॑नं॒ वेद॑। आ॒यत॑नवान्‌ भवति। 
वा॒युर्वा अ॒पामा॒यत॑नम्। आ॒यत॑नवान्‌ भवति। 
यो वा॒योरा॒यत॑नं॒ वेद॑। आ॒यत॑नवान्‌ भवति॥७९॥


आपो॒ वै वा॒योरा॒यत॑नम्। आ॒यत॑नवान्‌ भवति। 
य ए॒वं वेद॑। यो॑ऽपामा॒यत॑नं॒ वेद॑। 
आ॒यत॑नवान्‌ भवति। अ॒सौ वै तप॑न्न॒पामा॒यत॑नम्। 
आ॒यत॑नवान्‌ भवति। यो॑ऽमुष्य॒ तप॑त आ॒यत॑नं॒ वेद॑। 
आ॒यत॑नवान्‌ भवति। आपो॒ वा अ॒मुष्य॒ तप॑त आ॒यत॑नम्॥८०॥


आ॒यत॑नवान्‌ भवति। य ए॒वं वेद॑। 
यो॑ऽपामा॒यत॑नं॒ वेद॑। आ॒यत॑नवान्‌ भवति। 
च॒न्द्रमा॒ वा अ॒पामा॒यत॑नम्‌। आ॒यत॑नवान्‌ भवति। 
यश्च॒न्द्रम॑स आ॒यत॑नं॒ वेद॑। आ॒यत॑नवान्‌ भवति। 
आपो॒ वै च॒न्द्रम॑स आ॒यत॑नम्। आ॒यत॑नवान्‌ भवति॥८१॥


य ए॒वं वेद॑। यो॑ऽपामा॒यत॑नं॒ वेद॑। 
आ॒यत॑नवान्‌ भवति। नक्ष॑त्राणि॒ वा अ॒पामा॒यत॑नम्। 
आ॒यत॑नवान्‌ भवति। यो नक्ष॑त्राणामा॒यत॑नं॒ वेद॑। 
आ॒यत॑नवान्‌ भवति। आपो॒ वै नक्ष॑त्राणामा॒यत॑नम्। 
आ॒यत॑नवान्‌ भवति। य ए॒वं वेद॑॥८२॥


यो॑ऽपामा॒यत॑नं॒ वेद॑। आ॒यत॑नवान्‌ भवति।
प॒र्जन्यो॒ वा अ॒पामा॒यत॑नम्। आ॒यत॑नवान्‌ भवति। 
यः प॒र्जन्य॑स्य॒\aav\aav{}यत॑नं॒ वेद॑। आ॒यत॑नवान्‌ भवति। 
आपो॒ वै प॒र्जन्य॑स्य॒\aav\aav{}यत॑नम्। आ॒यत॑नवान्‌ भवति। 
य ए॒वं वेद॑। यो॑ऽपामा॒यत॑नं॒ वेद॑॥८३॥


आ॒यत॑नवान्‌ भवति। सं॒व॒त्स॒रो वा अ॒पामा॒यत॑नम्। 
आ॒यत॑नवान्‌ भवति। यः सं॑वत्स॒रस्य॒\aav\aav{}यत॑नं॒ वेद॑। 
आ॒यत॑नवान्‌ भवति। आपो॒ वै सं॑वत्स॒रस्य॒\aav\aav{}यत॑नम्‌। 
आ॒यत॑नवान्‌ भवति। य ए॒वं वेद॑। 
योऽप्सु नावं॒ प्रति॑ष्ठितां॒ वेद॑। प्रत्ये॒व ति॑ष्ठति॥८४॥


इ॒मे वै लो॒का अ॒प्सु प्रति॑ष्ठिताः। तदे॒षाऽभ्यनूक्ता। 
अ॒पा रस॒मुद॑यसन्न्‌। सूर्ये॑ शु॒क्र स॒माभृ॑तम्‌। 
अ॒पा रस॑स्य॒ यो रस॑। तं वो॑ गृह्णाम्युत्त॒ममिति॑। 
इ॒मे वै लो॒का अ॒पा रस॑। ते॑ऽमुष्मि॑न्नादि॒त्ये स॒माभृ॑ताः। 
जा॒नु॒द॒घ्नीमु॑त्तर\-वे॒दीं खा॒त्वा। अ॒पां पू॑रयि॒त्वा गु॑ल्फद॒घ्नम्‌॥८५॥


पुष्करपर्णैः पुष्करदण्डैः पुष्करैश्च॑ सस्ती॒र्य। तस्मि॑न्वि\-हा॒यसे। 
अ॒ग्निं प्र॒णीयो॑पसमा॒धाय॑। ब्र॒ह्म॒वा॒दिनो॑ वदन्ति। 
कस्मात्प्रणी॒ते\-ऽयम॒ग्निश्ची॒यते। साप्र॑णी॒तेऽयम॒प्सु ह्ययं॑ ची॒यते। 
अ॒सौ भुव॑ने॒प्यना॑हिताग्निरे॒ताः। तम॒भित॑ ए॒ता अ॒बीष्ट॑का॒ उप॑दधाति। 
अ॒ग्नि॒हो॒त्रे द॑र्शपूर्णमा॒सयो। प॒शु॒ब॒न्धे चा॑तुर्मा॒स्येषु॑॥८६॥


अथो॑ आहुः। सर्वे॑षु यज्ञक्र॒तुष्विति॑। 
ए॒तद्ध॑ स्म॒ वा आ॑हुः शण्डि॒लाः। कम॒ग्निं चि॑नुते। 
स॒त्रि॒यम॒ग्निं चि॑न्वा॒नः। सं॒व॒त्स॒रं प्र॒त्यक्षे॑ण। 
कम॒ग्निं चि॑नुते। सा॒वि॒त्रम॒ग्निं चि॑न्वा॒नः। 
अ॒मुमा॑दि॒त्यं प्र॒त्यक्षे॑ण। कम॒ग्निं चि॑नुते॥८७॥


ना॒चि॒के॒तम॒ग्निं चि॑न्वा॒नः। 
प्रा॒णान्प्र॒त्यक्षे॑ण। कम॒ग्निं चि॑नुते। चा॒तु॒\ar{}हो॒त्रि॒य\-म॒ग्निं चि॑न्वा॒नः। 
ब्रह्म॑ प्र॒त्यक्षे॑ण। कम॒ग्निं चि॑नुते। वै॒श्व॒सृ॒जम॒ग्निं चि॑न्वा॒नः। 
शरी॑रं प्र॒त्यक्षे॑ण। कम॒ग्निं चि॑नुते। उ॒पा॒नु॒वा॒क्य॑\-मा॒शुम॒ग्निं चि॑न्वा॒नः॥८८॥


इ॒माँल्लो॒कान्प्र॒त्यक्षे॑ण। कम॒ग्निं चि॑नुते। 
इ॒ममा॑रुणकेतुकम॒ग्निं चि॑न्वा॒न इति॑। य ए॒वासौ। 
इ॒तश्चा॒ऽमुत॑श्चाऽव्यतीपा॒ती। तमिति॑। 
योऽग्नेर्मि॑थू॒या वेद॑। मि॒थु॒न॒वान्भ॑वति। 
आपो॒ वा अ॒ग्नेर्मि॑थू॒याः। मि॒थु॒न॒वान्भ॑वति। य ए॒वं वेद॑॥८९॥\anuvakamend


आपो॒ वा इ॒दमा॑सन्त्सलि॒लमे॒व। स प्र॒जाप॑ति॒रेक॑ पुष्करप॒र्णे सम॑भवत्‌। 
तस्यान्त॒र्मन॑सि काम॒ सम॑वर्तत। इ॒द सृ॑जेय॒मिति॑। 
तस्मा॒द्यत्पुरु॑षो॒ मन॑साऽभि॒गच्छ॑ति। तद्वा॒चा व॑दति। 
तत्कर्म॑णा करोति। तदे॒षाऽभ्यनूक्ता। 
काम॒स्तदग्रे॒ सम॑वर्त॒ताधि॑। मन॑सो॒ रेत॑ प्रथ॒मं यदासीत्‌॥९०॥


स॒तो बन्धु॒मस॑ति॒ निर॑विन्दन्न्‌। हृ॒दि प्र॒तीष्या॑ क॒वयो॑ मनी॒षेति॑। 
उपै॑न॒न्तदुप॑नमति। यत्का॑मो॒ भव॑ति। य ए॒वं वेद॑। 
स तपो॑ऽतप्यत। स तप॑स्त॒प्त्वा। शरी॑रमधूनुत। तस्य॒ यन्मा॒समासीत्‌। 
ततो॑ऽरु॒णाः के॒तवो॒ वात॑रश॒ना ऋष॑य॒ उद॑तिष्ठन्न्॥९१॥


ये नखा। ते वै॑खान॒साः। ये वाला। ते वा॑लखि॒ल्याः। 
यो रस॑। सो॑ऽपाम्‌। अ॒न्त॒र॒तः कू॒र्मं भू॒त सर्प॑न्तम्‌। 
तम॑ब्रवीत्‌। मम॒ वैत्वङ्मा॒सा। सम॑भूत्‌॥९२॥


नेत्य॑ब्रवीत्। पूर्व॑मे॒वाहमि॒हास॒मिति॑। 
तत्पुरु॑षस्य पुरुष॒त्वम्‌। स स॒हस्र॑शी\ur{}षा॒ पुरु॑षः। 
स॒ह॒स्रा॒क्षः स॒हस्र॑पात्‌। भू॒त्वोद॑तिष्ठत्‌। 
तम॑ब्रवीत्‌। त्वं वै पूर्व सम॑भूः। 
त्वमि॒दं पूर्व॑ कुरु॒ष्वेति॑। स इ॒त आ॒दायाप॑॥९३॥


अ॒ञ्ज॒लिना॑ पु॒रस्ता॑दु॒पाद॑धात्‌। ए॒वाह्ये॒वेति॑। 
तत॑ आदि॒त्य उद॑तिष्ठत्‌। सा प्राची॒ दिक्‌। 
अथा॑ऽरु॒णः के॒तुर्द॑क्षिण॒त उ॒पाद॑धात्‌। 
ए॒वाह्यग्न॒ इति॑। ततो॒ वा अ॒ग्निरुद॑तिष्ठत्‌। 
सा द॑क्षि॒णा दिक्‌। अथा॑रु॒णः के॒तुः प॒श्चादु॒पाद॑धात्‌। 
ए॒वा हि वायो॒ इति॑॥९४॥


ततो॑ वा॒युरुद॑तिष्ठत्। सा प्र॒तीची॒ दिक्‌। 
अथा॑रु॒णः के॒तुरु॑त्तर॒त उ॒पाद॑धात्‌। ए॒वाहीन्द्रेति॑। 
ततो॒ वा इन्द्र॒ उद॑तिष्ठत्‌। सोदी॑ची॒ दिक्‌। 
अथा॑रु॒णः के॒तुर्मध्य॑ उ॒पाद॑धात्‌। ए॒वा हि पूष॒न्निति॑। 
ततो॒ वै पू॒षोद॑तिष्ठत्‌। सेयं दिक्‌॥९५॥


अथा॑रु॒णः के॒तुरु॒परि॑ष्टादु॒पाद॑धात्‌। ए॒वा हि देवा॒ इति॑। 
ततो॑ देवमनु॒ष्याः पि॒तर॑। ग॒न्ध॒र्वा॒प्स॒रस॒श्चोद॑तिष्ठन्न्‌। 
सोर्ध्वा दिक्‌। या वि॒प्रुषो॑ वि॒परा॑पतन्न्‌। 
ताभ्योऽसु॑रा॒ रक्षासि पिशा॒चाश्चोद॑तिष्ठन्न्‌। तस्मा॒त्ते परा॑भवन्न्‌। 
वि॒प्रुड्भ्यो॒ हि ते सम॑भवन्न्‌। तदे॒षाऽभ्यनूक्ता॥९६॥


आपो॑ ह॒ यद्बृ॑ह॒तीर्गर्भ॒माय\snn{}। दक्षं॒ दधा॑ना ज॒नय॑न्तीः स्वय॒म्भुम्‌। 
तत॑ इ॒मेध्यसृ॑ज्यन्त॒ सर्गा। अद्भ्यो॒ वा इ॒द सम॑भूत्‌। 
तस्मा॑दि॒द सर्वं॒ ब्रह्म॑ स्वय॒म्भ्विति॑। 
तस्मा॑दि॒द सर्व॒ शिथि॑लमि॒वाऽध्रुव॑मिवाभवत्‌। 
प्र॒जाप॑ति॒र्वाव तत्‌। आ॒त्मना॒ऽऽत्मानं॑ वि॒धाय॑। 
तदे॒वानु॒प्रावि॑शत्‌। तदे॒षाऽभ्यनूक्ता॥९७॥


वि॒धाय॑ लो॒कान्‌ वि॒धाय॑ भू॒तानि॑। वि॒धाय॒ सर्वा प्र॒दिशो॒ दिश॑श्च। 
प्र॒जाप॑तिः प्रथम॒जा ऋ॒तस्य॑। आ॒त्मना॒ऽऽत्मान॑म॒भि संवि॑वे॒शेति॑। 
सर्व॑मे॒वेदमा॒प्त्वा। सर्व॑मव॒रुद्ध्य॑। 
तदे॒वानु॒प्रवि॑शति। य ए॒वं वेद॑॥९८॥\anuvakamend


चतु॑ष्टय्य॒ आपो॑ गृह्णाति। च॒त्वारि॒ वा अ॒पा रू॒पाणि॑। 
मेघो॑ वि॒द्युत्‌। स्त॒न॒यि॒त्नुर्वृ॒ष्टिः। तान्ये॒वाव॑रुन्धे। 
आ॒तप॑ति॒ वर्ष्या॑ गृह्णाति। ताः पु॒रस्ता॒दुप॑दधाति। 
ए॒ता वै ब्र॑ह्मवर्च॒स्या आप॑। मु॒ख॒त ए॒व ब्र॑ह्मवर्च॒समव॑रुन्धे। 
तस्मान्मुख॒तो ब्र॑ह्मवर्च॒सित॑रः॥९९॥


कूप्या॑ गृह्णाति। ता द॑क्षिण॒त उप॑दधाति। 
ए॒ता वै ते॑ज॒स्विनी॒राप॑। तेज॑ ए॒वास्य॑ दक्षिण॒तो द॑धाति। 
तस्मा॒द्दक्षि॒णोऽर्ध॑स्तेज॒स्वित॑रः। स्था॒व॒रा गृ॑ह्णाति। 
ताः प॒श्चादुप॑दधाति। प्रति॑ष्ठिता॒ वै स्था॑व॒राः। 
प॒श्चादे॒व प्रति॑तिष्ठति। वह॑न्तीर्गृह्णाति॥१००॥


ता उ॑त्तर॒त उप॑दधाति। ओज॑सा॒ वा ए॒ता वह॑न्तीरि॒वोद्ग॑तीरि॒व आकूज॑तीरि॒व धाव॑न्तीः। 
ओज॑ ए॒वास्योत्तर॒तो द॑धाति। तस्मा॒दुत्त॒रोऽर्ध॑ ओज॒स्वित॑रः। 
स॒म्भा॒र्या गृ॑ह्णाति। ता मध्य॒ उप॑दधाति। 
इ॒यं वै स॑म्भा॒र्याः। अ॒स्यामे॒व प्रति॑तिष्ठति। 
प॒ल्व॒ल्या गृ॑ह्णाति। ता उ॒परि॑ष्टादु॒पाद॑धाति॥१०१॥


अ॒सौ वै प॑ल्व॒याः। अ॒मुष्या॑मे॒व प्रति॑तिष्ठति। 
दि॒क्षूप॑दधाति। दि॒क्षु वा आप॑। 
अन्नं॒ वा आप॑। अ॒द्भ्यो वा अन्नं॑ जायते। 
यदे॒वाद्भ्योऽन्नं॒ जाय॑ते। तदव॑रुन्धे। 
तं वा ए॒तम॑रु॒णाः के॒तवो॒ वात॑रश॒ना ऋष॑योऽचिन्वन्। 
तस्मा॑दारुणके॒तुक॑॥१०२॥


तदे॒षाऽभ्यनूक्ता। के॒तवो॒ अरु॑णासश्च। 
ऋ॒ष॒यो वात॑रश॒नाः। प्र॒ति॒ष्ठा श॒तधा॑ हि। 
स॒माहि॑तासो सहस्र॒धाय॑स॒मिति॑। श॒तश॑श्चै॒व स॒हस्र॑शश्च॒ प्रति॑तिष्ठति। 
य ए॒तम॒ग्निं चि॑नु॒ते। य उ॑चैनमे॒वं वेद॑॥१०३॥\anuvakamend


जा॒नु॒द॒घ्नीमु॑त्तरवे॒दीं खा॒त्वा। अ॒पां पू॑रयति। 
अ॒पा स॑र्व॒त्वाय॑। पु॒ष्क॒र॒प॒र्ण रु॒क्मं पुरु॑ष॒मित्युप॑दधाति। 
तपो॒ वै पु॑ष्करप॒र्णम्‌। स॒त्य रु॒क्मः। 
अ॒मृतं॒ पुरु॑षः। ए॒ताव॒द्वा वाऽस्ति। 
याव॑दे॒तत्। याव॑दे॒वास्ति॑॥१०४॥


तदव॑रुन्धे। कू॒र्ममुप॑दधाति। 
अ॒पामे॒व मेध॒मव॑रुन्धे। अथो स्व॒र्गस्य॑ लो॒कस्य॒ सम॑ष्ट्यै। 
आप॑मापाम॒पः सर्वा। अ॒स्माद॒स्मादि॒तोऽमुत॑। 
अ॒ग्निर्वा॒युश्च॒ सूर्य॑श्च। स॒ह स॑ञ्चस्क॒रर्द्धि॑या॒ इति॑। 
वा॒य्वश्वा॑ रश्मि॒पत॑यः। लो॒कं पृ॑णच्छि॒द्रं पृ॑ण॥१०५॥


यास्ति॒स्रः प॑रम॒जाः। इ॒न्द्र॒घो॒षा वो॒ वसु॑भिरे॒वाह्ये॒वेति॑। 
पञ्च॒चित॑य॒ उप॑दधाति। पाङ्क्तो॒ऽग्निः। 
यावा॑ने॒वाग्निः। तं चि॑नुते। 
लो॒कं पृ॑णया द्वि॒तीया॒मुप॑दधाति। पञ्च॑ पदा॒ वै वि॒राट्‌। 
तस्या॒ वा इ॒यं पाद॑। अ॒न्तरि॑क्षं॒ पाद॑। द्यौः पाद॑। 
दिश॒ पाद॑। प॒रोर॑जा॒ पाद॑। वि॒राज्ये॒व प्रति॑तिष्ठति। 
य ए॒तम॒ग्निं चि॑नु॒ते। य उ॑चैनमे॒वं वेद॑॥१०६॥\anuvakamend


अ॒ग्निं प्र॒णीयो॑पसमा॒धाय॑। तम॒भित ए॒ता अ॒बीष्टका॒ उप॑दधाति। 
अ॒ग्नि॒हो॒त्रे द॑र्शपूर्णमा॒सयो। प॒शु॒ब॒न्धे चा॑तुर्मा॒स्येषु॑। 
अथो॑ आहुः। सर्वेषु॑ यज्ञक्र॒तुष्विति॑। 
अथ॑ ह स्माहारु॒णः स्वा॑य॒म्भुव॑। सा॒वि॒त्रः सर्वो॒ऽग्निरित्यन॑नुषङ्गं मन्यामहे। 
नाना॒ वा ए॒तेषां वी॒र्या॑णि। कम॒ग्निं चि॑नुते॥१०७॥


स॒त्रि॒यम॒ग्निं चि॑न्वा॒नः। कम॒ग्निं चि॑नुते। 
सा॒वि॒त्रम॒ग्निं चि॑न्वा॒नः। कम॒ग्निं चि॑नुते। 
ना॒चि॒के॒तम॒ग्निं चि॑न्वा॒नः। कम॒ग्निं चि॑नुते। 
चा॒तु॒\ar{}हो॒त्रि॒य\-म॒ग्निं चि॑न्वा॒नः। कम॒ग्निं चि॑नुते। 
वै॒श्व॒सृ॒जम॒ग्निं चि॑न्वा॒नः। कम॒ग्निं चि॑नुते॥१०८॥


उ॒पा॒नु॒वा॒क्य॑मा॒शुम॒ग्निं चि॑न्वा॒नः। कम॒ग्निं चि॑नुते। 
इ॒ममा॑रुणकेतुक\-म॒ग्निं चि॑न्वा॒न इति॑। वृषा॒ वा अ॒ग्निः। 
वृषा॑णौ॒ सस्फा॑लयेत्‌। ह॒न्येतास्य य॒ज्ञः। 
तस्मा॒न्नानु॒षज्य॑। सोत्त॑रवे॒दिषु॑ क्र॒तुषु॑ चिन्वीत। 
उ॒त्त॒र॒वे॒द्या ह्य॑ग्निश्ची॒यते। प्र॒जाका॑मश्चिन्वीत॥१०९॥


प्रा॒जा॒प॒त्यो वा ए॒षोऽग्निः। प्रा॒जा॒प॒त्याः प्र॒जाः। 
प्र॒जावान्‌ भवति। य ए॒वं वेद॑। 
प॒शुका॑मश्चिन्वीत। सं॒ज्ञानं॒ वा ए॒तत्‌ प॑शू॒नाम्‌। 
यदाप॑। प॒शू॒नामे॒व सं॒ज्ञाने॒ऽग्निं चि॑नुते। 
प॒शु॒मान् भ॑वति। य ए॒वं वेद॑॥११०॥


वृष्टि॑कामश्चिन्वीत। आपो॒ वै वृष्टि॑। 
प॒र्जन्यो॒ व\ur{}षु॑को भवति। य ए॒वं वेद॑। 
आ॒म॒या॒वी चि॑न्वीत। आपो॒ वै भे॑ष॒जम्‌। 
भे॒ष॒जमे॒वास्मै॑ करोति। सर्व॒मायु॑रेति। 
अ॒भि॒चरश्चिन्वीत। वज्रो॒ वा आप॑॥१११॥


वज्र॑मे॒व भ्रातृ॑व्येभ्य॒ प्रह॑रति। स्तृ॒णु॒त ए॑नम्‌। 
तेज॑स्कामो॒ यश॑स्कामः। ब्र॒ह्म॒व॒र्च॒सका॑मः स्व॒र्गका॑मश्चिन्वीत। 
ए॒ताव॒द्वा वाऽस्ति। याव॑दे॒तत्‌। 
याव॑दे॒वास्ति॑। तदव॑रुन्धे। 
तस्यै॒तद्व्र॒तम्‌। व\ur{}ष॑ति॒ न धा॑वेत्‌॥११२॥


अ॒मृतं॒ वा आप॑। अ॒मृत॒स्यान॑न्तरित्यै। 
नाप्सु मूत्र॑पुरी॒षं कु॑र्यात्‌। न निष्ठी॑वेत्‌। 
न वि॒वस॑नः स्नायात्‌। गुह्यो॒ वा ए॒षोऽग्निः। 
ए॒तस्या॒ग्नेरन॑तिदाहाय। न पु॑ष्करप॒र्णानि॒ हिर॑ण्यं॒ वाऽधि॒तिष्ठेत्‌। 
ए॒तस्या॒ग्नेरन॑भ्यारोहाय। न कूर्म॒स्याश्नी॑यात्‌। 
नोद॒कस्या॒घातु॑का॒न्येन॑\-मोद॒कानि॑ भवन्ति। अ॒घातु॑का॒ आप॑। 
य ए॒तम॒ग्निं चि॑नु॒ते। य उ॑चैनमे॒वं वेद॑॥११३॥\anuvakamend


इ॒मानु॑कं॒ भु॑वना सीषधेम। इन्द्र॑श्च॒ विश्वे॑ च दे॒वाः। 
य॒ज्ञं च॑ नस्त॒न्वं च॑ प्र॒जां च॑। आ॒दि॒त्यैरिन्द्र॑ स॒ह सी॑षधातु। 
आ॒दि॒त्यैरिन्द्र॒ सग॑णो म॒रुद्भि॑। अ॒स्माकं॑ भूत्ववि॒ता त॒नूनाम्‌। 
आप्ल॑वस्व॒ प्रप्ल॑वस्व। आ॒ण्डीभ॑वज॒ मा मु॒हुः। 
सुखादीन्दु॑खनि॒धनाम्‌। प्रति॑मुञ्चस्व॒ स्वां पु॒रम्‌॥११४॥


मरी॑चयः स्वायम्भु॒वाः। ये श॑री॒राण्य॑कल्पयन्न्‌। 
ते ते॑ दे॒हं क॑ल्पयन्तु। मा च॑ ते॒ ख्यास्म॑ तीरिषत्‌। 
उत्ति॑ष्ठत॒ मा स्व॑प्त। अ॒ग्निमि॑च्छध्वं॒ भार॑ताः। 
राज्ञ॒ सोम॑स्य तृ॒प्तास॑। सूर्ये॑ण स॒युजो॑षसः। 
युवा॑ सु॒वासा। अ॒ष्टाच॑क्रा॒ नव॑द्वारा॥११५॥


दे॒वानां॒ पूर॑यो॒ध्या। तस्या हिरण्म॑यः को॒शः। 
स्व॒र्गो लो॒को ज्योति॒षाऽऽवृ॑तः। यो वै तां ब्रह्म॑णो वे॒द। 
अ॒मृते॑नाऽऽवृ॒तां पु॑रीम्‌। तस्मै ब्रह्म च॑ ब्रह्मा॒ च। 
आ॒युः कीर्तिं॑ प्र॒जां द॑दुः। वि॒भ्राज॑माना॒ हरि॑णीम्‌। 
य॒शसा॑ सम्प॒रीवृ॑ताम्‌। पुर हिरण्म॑यीं ब्र॒ह्मा॥११६॥


वि॒वेशा॑ऽप॒राजि॑ता। पराङेत्य॑ज्याम॒यी। 
पराङेत्य॑नाश॒की। इ॒ह चा॑मुत्र॑ चान्वे॒ति। 
वि॒द्वान्दे॑वासु॒रानु॑भ॒यान्‌। यत्कु॑मा॒री म॒न्द्रय॑ते। 
य॒द्यो॒षिद्यत्प॑ति॒व्रता। अरि॑ष्टं॒ यत्किं च॑ क्रि॒यते। 
अ॒ग्निस्तदनु॑वेधति। अ॒शृता॑सः शृ॑तास॒श्च॥११७॥


य॒ज्वानो॒ येऽप्य॑य॒ज्वन॑। स्व॑र्यन्तो॒ नापेक्षन्ते। 
इन्द्र॑म॒ग्निं च॑ ये वि॒दुः। सिक॑ता इव सं॒यन्ति॑। 
र॒श्मिभि॑ समु॒दीरि॑ताः। अ॒स्माल्लो॒काद॑मुष्मा॒च्च। 
ऋ॒षिभि॑रदात्पृ॒श्निभि॑। 
अपे॑त॒ वीत॒ वि च॑ सर्प॒तात॑। येऽत्र॒ स्थ पु॑रा॒णा ये च॒ नूत॑नाः। 
अहो॑भिर॒द्भिर॒क्तु\-भि॒र्व्य॑क्तम्‌॥११८॥


य॒मो द॑दात्वव॒सान॑मस्मै। नृ मु॑णन्तु नृपा॒त्वर्य॑। 
अ॒कृ॒ष्टा ये च॒ कृष्ट॑जाः। कु॒मारी॑षु क॒नीनी॑षु। 
जा॒रिणी॑षु च॒ ये हि॒ताः। रेत॑ पीता॒ आण्ड॑पीताः। 
अङ्गा॑रेषु च॒ ये हु॒ताः। उ॒भयान्‌ पुत्र॑पौत्र॒कान्‌। 
यु॒वे॒ऽहं य॒मराज॑गान्‌। श॒तमिन्नु श॒रद॑॥११९॥


अदो॒ यद्ब्रह्म॑ विल॒बम्‌। पि॒तृ॒णां च॑ य॒मस्य॑ च। 
वरु॑ण॒स्याश्वि॑नोर॒ग्नेः। म॒रुतां च वि॒हाय॑साम्‌। 
का॒म॒प्र॒यव॑णं मे अस्तु। स ह्ये॑वास्मि॑ स॒नात॑नः। 
इति नाको ब्रह्मिश्रवो॑ रायो॒ धनम्‌। पु॒त्रानापो॑ दे॒वीरि॒हाऽऽहि॑त॥१२०॥\anuvakamend


विशीर्ष्णीं॒ गृध्र॑शीर्ष्णीं च। अपेतो॑ निर्‌ऋ॒ति ह॑थः। 
परिबाध श्वे॑तकु॒क्षम्‌। नि॒जङ्घ शब॒लोद॑रम्‌। 
स॒ तान्‌ वा॒च्याय॑या स॒ह। अग्ने॒ नाश॑य स॒न्दृश॑। 
ई॒र्ष्या॒सू॒ये बु॑भु॒क्षाम्‌। म॒न्युं कृ॒त्यां च॑ दीधिरे। 
रथे॑न किशु॒काव॑ता। अग्ने॒ नाश॑य स॒न्दृश॑॥१२१॥\anuvakamend


प॒र्जन्या॑य॒ प्रगा॑यत। दि॒वस्पु॒त्राय॑ मी॒ढुषे। 
स नो॑ य॒वस॑मिच्छतु। इ॒दं वच॑ प॒र्जन्या॑य स्व॒राजे। 
हृ॒दो अ॒स्त्वन्त॑र॒न्तद्यु॑योत। म॒यो॒भूर्वातो॑ वि॒श्वकृ॑ष्टयः सन्त्व॒स्मे। 
सु॒पि॒प्प॒ला ओष॑धीर्दे॒वगो॑पाः। यो गर्भ॒मोष॑धीनाम्‌। 
गवां कृ॒णोत्यर्व॑ताम्‌। प॒र्जन्य॑ पुरु॒षीणाम्‌॥१२२॥\anuvakamend


पुन॑र्मामैत्विन्द्रि॒यम्‌। पुन॒रायु॒ पुन॒र्भग॑। 
पुन॒र्ब्राह्म॑णमैतु मा। पुन॒र्द्रवि॑णमैतु मा। 
यन्मे॒ऽद्य रेत॑ पृथि॒वीमस्का\sn{}। यदोष॑धीर॒प्यस॑र॒द्यदाप॑। 
इ॒दं तत्पुन॒राद॑दे। दी॒र्घा॒यु॒त्वाय॒ वर्च॑से। 
यन्मे॒ रेत॒ प्रसि॑च्यते। यन्म॒ आजा॑यते॒ पुन॑। 
तेन॑ माम॒मृतं॑ कुरु। तेन॑ सुप्र॒जसं॑ कुरु॥१२३॥\anuvakamend


अ॒द्भ्यस्तिरो॒ऽधाऽजा॑यत। तव॑ वैश्रव॒णः स॑दा। 
तिरो॑ऽधेहि सप॒त्नान्न॑। ये अपो॒ऽश्नन्ति॑ केच॒न। 
त्वा॒ष्ट्रीं मा॒यां वैश्रव॒णः। रथ सहस्र॒वन्धु॑रम्‌। 
पु॒रु॒श्च॒क्र सह॑स्राश्वम्‌। आस्था॒याया॑हि नो ब॒लिम्‌। 
यस्मै॑ भू॒तानि॑ ब॒लिमाव॑हन्ति। धनं॒ गावो॒ हस्ति॒ हिर॑ण्य॒मश्वा\sn{}॥१२४॥


असा॑म सुम॒तौ य॒ज्ञिय॑स्य। श्रियं॒ बिभ्र॒तोऽन्न॑मुखीं वि॒राजम्‌। 
सु॒द॒\ar{}शने च॑ क्रौ॒ञ्चे च॑। मै॒ना॒गे च॑ म॒हागि॑रौ। 
श॒तद्वा॒ट्टार॑गम॒न्ता। स॒हार्यं॒ नग॑रं॒ तव॑। 
इति मन्त्रा। कल्पो॑ऽत ऊ॒र्ध्वम्‌। यदि॒ बलि॒ हरेत्‌। 
हि॒र॒ण्य॒ना॒भये॑ वितु॒दये॑ कौबे॒राया॒यं ब॑लिः॥१२५॥


सर्वभूताधिपतये न॑म इ॒ति। अथ बलि हृत्वोप॑तिष्ठे॒त। 
क्ष॒त्रं क्ष॒त्रं वैश्रव॒णः। ब्राह्मणा॑ वय॒ स्मः। 
नम॑स्ते अस्तु॒ मा मा॑ हिसीः। अस्मात्प्रविश्यान्न॑मद्धी॒ति। 
अथ तमग्निमा॑दधी॒त। यस्मिन्नेतत्कर्म प्र॑युञ्जी॒त। 
ति॒रोऽधा॒ भूः। ति॒रोऽधा॒ भुव॑॥१२६॥


ति॒रोऽधा॒ स्व॑। ति॒रोऽधा॒ भूर्भुव॒ स्व॑। 
सर्वेषां लोकानामाधिपत्ये॑ सीदे॒ति। अथ तमग्नि॑मिन्धी॒त। 
यस्मिन्नेतत्कर्म प्र॑युञ्जी॒त। ति॒रोऽधा॒ भूः स्वाहा। 
ति॒रोऽधा॒ भुव॒ स्वाहा। ति॒रोऽधा॒ स्व॑ स्वाहा। 
ति॒रोऽधा॒ भूर्भुव॒ स्व॑ स्वाहा। 
यस्मिन्नस्य काले सर्वा आहुतीर्\mbox{}हुता॑ भवे॒युः॥१२७॥


अपि ब्राह्मण॑मुखी॒नाः। तस्मिन्नह्नः काले प्र॑युञ्जी॒त। 
पर॑ सु॒प्तज॑नाद्वे॒पि। मास्म प्रमाद्यन्त॑माध्या॒पयेत्‌। 
सर्वार्था सिद्ध्य॒न्ते। य ए॑वं वे॒द। 
क्षुध्यन्निद॑मजा॒नताम्‌। सर्वार्था न॑ सिद्ध्य॒न्ते। 
यस्ते॑ वि॒घातु॑को भ्रा॒ता। ममान्तर्‌हृ॑दये॒ श्रितः॥१२८॥


तस्मा॑ इ॒ममग्र॒पिण्डं॑ जुहोमि। स मेऽर्था॒न्मा विव॑धीत्‌। 
मयि॒ स्वाहा। रा॒जा॒धि॒रा॒जाय॑ प्रसह्यसा॒हिने। 
नमो॑ व॒यं वैश्रव॒णाय॑ कुर्महे। स मे॒ कामा॒न्काम॒कामा॑य॒ मह्यम्‌। 
का॒मे॒श्व॒रो वैश्रव॒णो द॑दातु। कु॒बे॒राय॑ वैश्रव॒णाय॑। 
म॒हा॒रा॒जाय॒ नम॑। के॒तवो॒ अरु॑णासश्च। 
ऋ॒ष॒यो वात॑रश॒नाः। प्र॒ति॒ष्ठा श॒तधा॑ हि। 
स॒माहि॑तासो सहस्र॒धाय॑सम्‌। शि॒वा न॒ शन्त॑मा भवन्तु। 
दि॒व्या आप॒ ओष॑धयः। सु॒मृ॒डी॒का सर॑स्वति। 
मा ते॒ व्यो॑म स॒न्दृशि॑॥१२९॥\anuvakamend


संवत्सरमेत॑द्व्रतं॒ चरेत्‌। द्वौ॑ वा मा॒सौ। 
नियमः स॑मासे॒न। तस्मिन्नियम॑विशे॒षाः। 
त्रिषवणमुदको॑पस्प॒र्शी। चतुर्थकालपान॑भक्त॒ स्यात्‌। 
अहरहर्वा भैक्ष॑मश्नी॒यात्‌। औदुम्बरीभिः समिद्भिरग्निं॑ परि॒चरेत्‌। 
पुनर्मामैत्त्विन्द्रियमि\-त्येतेनऽनु॑वाके॒न। उद्धृतपरिपूताभि\-रद्भिः कार्यं॑ कुर्वी॒त॥१३०॥


अ॑सञ्च॒यवान्‌। अग्नये वायवे॑ सूर्या॒य। 
ब्रह्मणे प्र॑जाप॒तये। चन्द्रमसे न॑क्षत्रे॒भ्यः। 
ऋतुभ्यः संव॑त्सरा॒य। वरुणायारुणायेति व्र॑तहो॒माः। 
प्र॒व॒र्ग्यव॑दादे॒शः। अरुणाः काण्डऋ॒षयः। 
अरण्ये॑ऽधीयी॒रन्न्‌। भद्रं कर्णेभिरिति द्वे॑ जपि॒त्वा॥१३१॥


महानाम्नीभिरुदक स॑स्प॒र्श्य। तमाचार्यो द॒द्यात्‌। 
शिवा नः शन्तमेत्योषधी॑राल॒भते। सुमृडीके॑ति भू॒मिम्‌। 
एवम॑पव॒र्गे। धे॑नुर्द॒क्षिणा। कसं वास॑श्च क्षौ॒मम्‌। 
अन्य॑द्वा शु॒क्लम्‌। य॑थाश॒क्ति वा। एवस्वाध्याय॑धर्मे॒ण। 
अरण्ये॑ऽधीयी॒त। तपस्वी पुण्यो भवति तपस्वी पु॑ण्यो भ॒वति॥१३२॥\anuvakamend


भ॒द्रं कर्णे॑भिः शृणु॒याम॑ देवाः। भ॒द्रं प॑श्येमा॒क्षभि॒र्यज॑त्राः। 
स्थि॒रैरङ्गैस्तुष्टु॒वा स॑स्त॒नूभि॑। व्यशे॑म दे॒वहि॑तं॒ यदायु॑। 
स्व॒स्ति न॒ इन्द्रो॑ वृ॒द्धश्र॑वाः। स्व॒स्ति न॑ पू॒षा वि॒श्ववे॑दाः। 
स्व॒स्ति न॒स्तार्क्ष्यो॒ अरि॑ष्टनेमिः। स्व॒स्ति नो॒ बृह॒स्पति॑र्दधातु॥\\

\centerline{॥ॐ शान्ति॒ शान्ति॒ शान्ति॑॥}
